\documentclass[letterpaper, 10 pt, conference]{ieeeconf}
\IEEEoverridecommandlockouts
\overrideIEEEmargins
% The preceding line is only needed to identify funding in the first footnote. If that is unneeded, please comment it out.
\usepackage{cite}
\usepackage{amsmath,amssymb,amsfonts}
\usepackage{algorithmic}
\usepackage{graphicx}
\usepackage{gensymb}
\usepackage{textcomp}
\usepackage{xcolor}
\usepackage{siunitx}
\sisetup{detect-weight=true,mode=text}
\def\BibTeX{{\rm B\kern-.05em{\sc i\kern-.025em b}\kern-.08em
    T\kern-.1667em\lower.7ex\hbox{E}\kern-.125emX}}
\begin{document}

\title{\LARGE \bf Soft and Highly-Integrated Optical Fiber Bending Sensors for Proprioception in Multi-Material 3D Printed Fingers}

\author{Ellis Capp$^{1*}$, Marco Pontin$^{1}$, Peter Walters$^{2}$, and Perla Maiolino$^{1}$% <-this % stops a space
\thanks{This work was supported by  Engineering and Physical Sciences Research Council (EPSRC) Grant EP/V000748/1.}% <-this % stops a space
\thanks{$^{1}$Ellis Capp, Marco Pontin, and Perla Maiolino are with Oxford Robotics Institute, University of Oxford, UK;}%
\thanks{$^{2}$Peter Walters is  with the Department of Engineering Science, University of Oxford, UK.}%
\thanks{*Corresponding author: {\tt\small ellis@robots.ox.ac.uk}}}

\maketitle
\thispagestyle{empty}
\pagestyle{empty}

\begin{abstract}
Accurate shape sensing, only achievable through distributed proprioception, is a key requirement for closed-loop control of soft robots. Low-cost power efficient optoelectronic sensors manufactured from flexible materials represent a natural choice as they can cope with the large deformations of soft robots without loss of performance. However, existing integration approaches are cumbersome and require manual steps and complex assembly. We propose a semi-automated printing process where plastic optical fibers are embedded with readout electronics in 3D printed flexures. The fibers become locked in place and the readout electronics remain optically coupled to them while the flexures undergo large bending deformations, creating a repeatable, monolithically manufactured bending transducer with only 10 minutes required in total for the manual embedding steps. We demonstrate the process by manufacturing multi-material 3D printed fingers and extensively evaluating the performance of each proprioceptive joint. The sensors achieve 70\% linearity and 4.81° RMS error on average. Furthermore, the distributed architecture allows for maintaining an average fingertip position estimation accuracy of 12 mm in the presence of external static forces. To demonstrate the potential of the distributed sensor architecture in robotics applications, we build a data-driven model independent of actuation feedback to detect contact with objects in the environment. 
\end{abstract}

\section{Introduction}\label{sec:Introduction}
\documentclass[../main.tex]{subfiles}
\graphicspath{{../images/}}
\makeatletter
\def\input@path{{../images/}}
\makeatother
\begin{document}
\section{Introduction}
\begin{figure}
\centering
\begin{tikzpicture}
\node[inner sep=0pt] (ws) at (0, 0) {
\includegraphics[height=.4\textwidth, trim={10cm 0 10cm 0},clip]{world_space.png}};
\node[inner sep=0pt] (cs) at (6,0) {\includegraphics[height=.4\textwidth, trim={10cm 1cm 10cm 4cm},clip]{conf_space.png}};
\end{tikzpicture}
\vspace{-5pt}
\label{fig:pbrm_intro}
\caption{\textbf{Left}: Shows world space obstacles as grey spheres. Robots start and goal configuration is colored red and green, respectively. Configurations along the computed path are colored transparent blue. \textbf{Right:} Mapped world space scenario to configuration space. Obstacle region is the grey mesh. Red spheres are collision-free regions computed by the neural SCDF. The optimized shortest path in the convex corridor is the blue curve.}
\vspace{-25pt}
\end{figure}
Motion planning is the problem of finding a collision-free trajectory that connects a given start and goal configuration. The planning takes place in the configuration space of the robot. For single body robots, like mobile robots or drones, the configuration space and the world space are usually the same. This simplifies the planning, since explicit obstacle representations are available which enables geometrical tools like separating hyperplanes, smallest distance to obstacles etc., to be used when designing motion planning algorithms. For multi-body robots like manipulators, the situation is completely different. The world space obstacles are usually mapped to non-convex regions, and to make the problem even harder, the mapping is usually not known. Forming explicit representations of the obstacle region in the configuration space is usually too expensive or intractable. Despite all of this, sampling based planners are used with great success, which mainly is due to their use of implicit representations of the obstacle region. The basic idea is to construct a graph in the configuration space that covers and connects the collision-free region. From this graph, a path can be extracted that connects a given start and goal configuration. The approach is computationally expensive, since the graph is constructed with the smallest geometrical building block available, points, which represents a collision-check. Furthermore, the extracted paths from the graph are non-smooth and jagged due to the stochastic nature of the approach. This adds an additional post-processing step to the process, where the paths are shortcutted and smoothened, before the path can be used for tracking. Clearly a lot of time is invested to form this graph and produce smooth paths. Thus, if the obstacles start to move, then all of this work is done in no use, since all points that make up this graph need to be re-verified, which is simply too time consuming to be done in real time.
\\\\
In this work, we want to address the existing drawbacks of the sampling based planners. Our main contribution is an improved motion planner where each vertex in the graph covers a collision-free region in the form of a sphere instead of a point and where the edges are formed with neighboring intersecting spheres. This representation has the advantage of instead of returning piecewise linear paths, returning a sequence of overlapping spheres, i.e. a convex corridor, that connects a given start and goal configuration, illustrated in Figure \ref{fig:pbrm_intro}. This convex corridor allows us to use convex optimization to produce smooth trajectories, instead of computationally expensive post-processing methods. The representation further allows us to estimate the coverage of the collision-free space, which gives us awareness and feedback in the offline roadmap construction phase. Finally, our representation is simple to adapt to moving obstacles, simply requery for the new radii and recheck for intersections. 
\\\\
The spherical collision-free regions are formed using a signed distance function (SDF), which is a function that returns the smallest distance from an arbitrary point to the boundary of an obstacle. As the name implies, the distance is signed, thus if the point is inside the obstacle it is negative otherwise positive. If the distance is positive, a sphere with radius equal to the distance is guaranteed to cover a collision-free region. Using an SDF in motion planning is not new, but what is novel about our approach is that we express the distance in the configuration space instead of the world space and by doing so allows us to form these convex collision-free regions. We refer to the resulting SDF as a signed configuration distance function (SCDF). Computing an SCDF analytically is non-trivial, our approach is therefore to parameterize the SCDF with a deep neural network and learn the mapping by supervised learning. Our resulting neural SCDF can compute distances for different parameter values of obstacle shapes and we also show how multiple distances can be combined, thus making our approach flexible.
\section{Related work}
Motion planning algorithms can roughly be divided into three families, grid-based, sampling based and optimization based methods. Grid-based methods (GBM) discretize the planning space from which a graph is then compiled. A standard search method is A$^\star$ \citep{a_star}, which is classified as an \textit{informed} search method, since it employs a heuristic function to speed up the search. A$^\star$ guarantees to return an optimal path at the level of discretization used. GBMs usually discretize the planning space by a regular lattice and this limits the GBMs to problems with low dimensionality due to the curse of dimensionality. Thus, GBMs are usually limited to single-body robots where the degrees of freedom (DOF) are low. To overcome the inherent scaling problem with the GBMs, stochastic methods are usually used for multi-body robots. These methods are termed as sampling-based methods (SBM) and core members within this family are the rapidly-exploring random trees (RRT) \citep{rrt} and the probabilistic roadmap (PRM) \citep{prm}. RRT grows a tree from the start configuration and explores the collision-free region in a rapid way until it is able to connect to the goal region. RRT is usually improved by bi-directional planning \citep{rrt_connect}, i.e. an additional tree is grown from the goal configuration and the trees are tested for connection after any tree has been expanded. RRT is a single-query method, thus it searches for a path from scratch each time it is queried. Contrary to this, PRM is a multi-query method, which solves for multiple queries without starting from scratch. PRM does this by creating a roadmap (graph) that covers the collision-free space as an offline step. The graph is then used to solve for multiple queries. PRMs are used in cases where the environment does not change since the extra offline step is too computationally costly and needs to be re-done if the environment is changed. In our work, we address this inherent issue by using a different roadmap representation. Our vertices in the graph cover a collision-free region in the form of spheres and we form the edges by checking for intersecting spheres. If something in the environment changes, we recompute the spheres radii and recheck the intersections, without relying on collision detection. We use a trained neural network to compute the sphere radius, therefore querying for the radius can be done fast, hence our representation enables the PRM for dynamic environments.
\\\\
In the recent decades, optimization based methods (OBM) \citep{chomp, schulman, itomp, stomp} have been introduced as an alternative to SBM for multi-body robots. Like the SBM, the OBMs scale well to higher dimensional problems and produce smoother motion. It is common to use a SDF in the optimization since it is a smooth function, thus enabling gradient-based methods. However, the standard way of expressing the SDF is in world space. The distance therefore needs to be mapped to the configuration space by the forward kinematics. This mapping makes the optimization problem a non-linear program (NLP), which is computationally expensive to solve. Recently, a different approach has been proposed. In \cite{mp_gcs} motion planning is formulated as a convex optimization problem by using the graph of convex sets framework \citep{gcs}. The underlying idea is to decompose the collision-free space into intersecting convex sets from which a convex optimization problem is formulated. In cases where an explicit representation of the obstacles in the configuration space exists, like for single-body robots, creating collision-free convex regions can be done fast \citep{iris}. For multi-body robots, this is non-trivial. Existing work does this successfully \citep{iris_nlp, iris_c} by an optimization based approach, but the methods are still too time consuming to be used in the presence of moving obstacles. Our approach is instead to use deep learning to learn an SDF expressed in the configuration space. With this, we can query for shortest distances to the collision boundary, which allows us to expand spherical regions which are collision-free. Our approach is fast and therefore enables our suggested roadmap planner to be used in dynamic environments.
\\\\
Recent research has focused on learning collision detection \citep{fk_kernel_distance, diffco, graphdistnet} by predicting the signed distance between the robot links and the surrounding obstacles in the world space. The learned SDF is used in trajectory optimization but since the distance is expressed in the world space, the problem becomes an NLP and therefore takes a long time to solve. We take a novel approach and suggest to instead express the signed distance in the configuration space. This allows us to improve the PRM at the same time as it enables convex optimization for trajectory optimization, which runs faster and is more reliable than NLP solvers. In \cite{cspf} a learned signed distance function in the configuration space is proposed similar to our approach. However, their approach is restricted to point cloud representations, while we propose to represent the obstacles as parameterized geometric shapes, e.g. spheres. Furthermore, we also show how to use our learned SCDF to improve an existing roadmap planner.
\section{Problem formulation}
A robot is located in the world space, $\W \subset \R^3 $. The unique location of the robot is given by its configuration $\q \in \C$, where $\C$ is the configuration space. The set of points covered by the robots bodies at a certain configuration is expressed as $\B(\q) \subset \W$. The robot is surrounded by $\NrObst$ obstacles $\O = \bigcup_{i=1}^{\NrObst} \O_i$, where  $\O_i \subset \W$. The representation of the obstacle in the configuration space is the set $\C\O_i = \{\q \in \C \: |\: \B(\q) \cap \O_i \neq \emptyset \}$. The obstacle space is formed as $\Co = \bigcup_{i=1}^{\NrObst} \C \O_i$. The complement is referred to as the free space, $\Cf = \C \setminus \Co$. The path planning problem is a tuple, ($\Cf$, $\qStart$, $\qGoal$), where we want to connect a query pair, consisting of a start, $\qStart$, and goal configuration, $\qGoal$, with a geometric path, $\q(s): [0, 1] \mapsto \Cf$, such that $\q(0)=\qStart$ and $\q(1)=\qGoal$, or report correctly when such a path does not exist.
\end{document}


\section{Materials and Methods}\label{sec:materials_and_methods}
The creation of the multi-material sensorized finger requires a short sequence of mostly automated manufacturing steps, displayed in Fig.  \ref{fig:fabrication_figure}. In this section, we analyze these steps in detail. In particular, the manual embedding step in which the POFs are encased in holders and the subsequent manual embedding steps in which the POFs, holders, and FPCBs are embedded into the flexure take 10 minutes in total and requires no adhesives and no special tools other than a razor blade and a soldering iron. 
\subsection{Optical Fiber Embedding Step}
\begin{figure}
    \centering
    \includegraphics[width=0.9\columnwidth]{FiberEmbedding_compressed_revised.pdf}
    \caption{Closeup photo of a pair of holders with an embedded optical fiber and a schematic detailed view of the embedding channel. The embedded fiber is able to support a mass of \qty{300}{\g}.}
    \label{fig:clearances_and_embed_closeups}
    \vspace{-1em}
\end{figure}

The fabrication process started with the embedding of the POFs. For this, we implemented a pause-and-place method where the the 3D printing of custom-designed holders with channels to accommodate the fiber was temporarily halted to enable the placement of the POFs before printing was resumed. The pause was scheduled at the layer where the channel was completed, to ensure good localization of the POF and minimize damage due to the interaction with the 3D printer nozzle.

Unjacketed ESKA POF was used with a diameter of \qty{500}{\um}, a commonly available POF manufactured from polymethyl methacrylate. The holders were printed on a Tenlog TL-D3 Pro independent dual extruder (IDEX) 3D printer (Innocube3D, Shenzhen, China) out of polylactic acid (PLA), with extruder temperature set to \qty{210}{\degree C}, a layer height of \qty{0.2}{\mm}, and a print speed of \qty{60}{\mm/s}. In preliminary experiments, we utilized POFs with a black polyethylene jacket with outer diameter \qty{1}{\mm} and core diameters of \qty{250}{\um} and \qty{500}{\um}. However, we found that when attempting to embed these POFs, the polyethylene jacket was damaged by the extruder nozzle during the printing process and allowed the fibers to easily slip out. We therefore switched to unjacketed POFs with core diameters of \qty{1000}{\um}, \qty{500}{\um}, and \qty{250}{\um}. We chose \qty{500}{\um} POFs to manufacture the sensors in this study because the \qty{1000}{\um} POFs exhibited high plastic strain at the bending radii required for the robotic finger geometry, while the \qty{250}{\um} POFs were challenging to embed repeatably due to the tolerances of our 3D printer.

When readying the holders for printing, we used the slicing software Cura to place them at the same spacing as the final spacing of the finger phalanges, \qty{28}{\mm} apart. One holder was printed for each of the four phalanges and a single fiber was embedded running through all four holders. After printing, the POF was cut with a surgical blade, as required to complete the POF-holder assembly. 

To enhance the repeatability and holding strength of the fiber embedding, we designed the fiber-accommodating channel with a V-shape as shown in Fig. \ref{fig:clearances_and_embed_closeups} to help the fiber self-center when pushed downward by the extruder nozzle. The height of the channel was \qty{0.4}{\mm}, equal to two layer heights, with a width of \qty{0.6}{\mm} on the higher layer to allow the fiber to be set into place and a width of \qty{0.1}{\mm} on the lower layer to form the V-shape. With this design, the fiber sits proud of the holders when the print is resumed so that the next extruded layer presses tightly against the top of the fiber. To further improve printing success, sacrificial jigs with channels to hold the fiber were printed in between each holder and the wall printing order was modified in Cura so that the printer would lock the POFs in place first before moving on to printing the remainder of the holders. A \qty{15}{\mm}$\times$\qty{15}{\mm}$\times$\qty{1.6}{\mm} cuboid was printed alongside the holders and jigs to prevent under-extrusion of the post-pause printed layer, and the printing temperature was temporarily increased to \qty{230}{\degree C} to promote inter-layer adhesion. The embedding process just described resulted in a reliable axial locking of the fibers, which could withstand loads of \qty{300}{g} without failing (Fig. \ref{fig:clearances_and_embed_closeups}).

\subsection{Flexible Printed Circuit Boards and Holder Design}
The FPCBs are based on the OPT3002 light-to-digital converter (LDC) (Texas Instruments, Dallas, TX, USA) and have a \qty{10}{\mm}$\times$\qty{10}{\mm} outline. We placed the LDC, supporting components, and a side-view 3010 LED on one side of each FPCB. We used the other side of the FPCBs for a selectable jumper to set the address of each LDC on the inter-integrated circuit (I2C) bus for power and communication wires.   

The highest spectral response of the LDC occurs at a wavelength of \qty{505}{nm}, so a yellow LED with a peak wavelength of \qty{590}{nm} was used. Because this response is in the visible spectrum, the LDC is sensitive to ambient light. We accepted this compromise because of the LDC's digital readout and miniaturized off-the-shelf package and we mitigated the effects of ambient light by 3D printing the outer casing of the sensor in black. 

The holders that lock the POFs in place also hold the FPCBs. The holders are designed with cavities that fit the circuit components of the FPCB and a \qty{0.6}{\mm} lip around the perimeter of the FPCB. We chose this height for the lip because it was the tallest height that could be achieved without interfering with the 3D printer nozzle during the sensor embedding process.

\subsection{Sensor Embedding Steps}
To prepare the FPCBs for embedding, we utilized a 3D printed jig to hold the FPCBs at the phalanx spacing before soldering \qty{0.15}{\mm} diameter enameled copper wires between each FPCB for the power and communication bus. After soldering, we encapsulated the solder joints to the wires in superglue to provide strain relief. We then embedded both the POF-holder and FPCB assemblies in the 3D printed thermoplastic polyurethane (TPU) (Filaflex 82A, Recreus, Elda, Spain) flexure in two pausing steps. During the first pause, we placed the POF-holder assembly into a cavity in the flexure. The printer then encapsulated the POFs while leaving the holders exposed to accept the FPCBs. During the second pause, the FPCB assembly was placed into the holders and a soldering iron with a chisel tip at \qty{350}{\degree C} was used to partially melt the midpoints of the lips over the sides of the FPCBs to lock them in place before resuming the print again. 

\subsection{Multi-material Finger Design}\label{subsec:finger_design}
We designed the multi-material finger with rigid phalanges printed from PLA and soft flexures connecting the phalanges printed from soft TPU with shore hardness 82A. The soft TPU gives the finger joint compliance and eliminates the need for any assembly or fasteners in the design, while the use of rigid PLA controls the kinematics of the actuated finger to allow the sensor angle readouts to predict the finger's shape. To facilitate printing with the TPU filament, we replaced one of the TL-D3 Pro's extruders with a Micro Swiss NG extruder (Ramsey, Minnesota, USA) and used print settings following the manufacturer's recommendations. 

The joints consist of a sliding contact pair between an inner and outer circle with \qty{0.12}{\mm} clearance in the undeformed state with the TPU flexures arranged symmetrically on both sides of the finger. The sliding contact pair geometry was selected as it provided a linear optical power loss. The sensors are unable to sense bending direction as their transduction principle is based on optical waveguide macrobending loss, therefore we included mechanical stops in the joint design to restrict bending to one direction. 

The finger is tendon-actuated, with the tendons made from \qty{0.4}{\mm} nylon fishing line and embedded at the mid-plane of the finger using another pause-and-place procedure. We secured the tendons to the distal phalanx by attaching split shot fishing weights to the tendons, placing the fishing weights in printed voids, and filling the voids around the fishing weights with superglue. 

The sensors are only embedded in the flexures on one side of the finger. To facilitate sensor characterization, flexures containing sensors were printed separately and inserted into a shaped cavity on the multi-material finger, held in place by compression. TPU flexures that did not contain sensors were printed with a high infill of 80\%. 

\section{Experimental Sensor Calibration}\label{sec:experimental_calibration}\begin{figure}
    \centering
    \includegraphics[width=0.9\columnwidth]{ExperimentalSetup_w_scalebar.pdf}
    \caption{Setup used for the characterization of the fingers. The photo is taken according to the POV of the camera used for the tracking of the markers.}
    \label{fig:setup_figure}
    \vspace{-1em}
\end{figure}
\begin{figure*}[t]
\centering
\includegraphics[width=0.9\textwidth]
{FingersComparison_compressed_revised.png}
\caption{Characterizations of the sensors. (a),(b),(c) The plots show measured optical power loss against the MCP, PIP, and DIP joint rotations for each finger during the quasi-static trials with their respective best-fit lines. (d) The sensitivity achieved for each sensor is shown grouped by finger on the bar chart. (e) RMS error in degrees for each joint is shown. (f) Hysteresis as percentage of full scale range for each joint is shown.}
\label{fig:fingers_results}
\vspace{-1em}
\end{figure*}
In this section, we describe a series of tests in which we analyze characteristics of the sensors that are important for engineering and control applications. Specifically, we characterize the reproducibility, repeatability, sensitivity, linearity, and hysteresis of the sensors, while also investigating the sensors' ability to retain their response characteristics over their operational life. Three sensorized flexures were 3D printed to evaluate sensor fabrication reproducibility, each of which were inserted into the same multi-material finger body.
\subsection{Protocol}\label{subsec:protocol}
\subsubsection{Quasi-Static Test}
To evaluate the sensor performance as-actuated, we used a 2-step pulley mounted on a Dynamixel MX-106R servo motor (Robotis, Seoul, South Korea) to actuate both the flexor and extensor tendons of the multi-material finger. After applying a \qty{15}{rpm} \qty{90}{\degree} transient rotation to the pulley, we observed a settling time under \qty{125}{\ms}. To characterize the sensors in quasi-static conditions, the pulley was rotated from \qty{0}{\degree} to \qty{263.6}{\degree} in increments of \qty{0.44}{\degree}. After each rotation, a pause of an eight of a second was performed before taking a measurement from the sensors. The entire procedure was repeated five times for each of three different fingers, resulting in $N=12010$ readings from each joint of each finger over the full test. The test was conducted with the apparatus placed under an LED lamp to control for ambient light. 

A webcam was placed square to the finger to enable joint angle measurements via image processing by detecting the pose of ArUco markers~\cite{ArUco} affixed to each phalanx of the finger (Fig. \ref{fig:setup_figure}) and taking the Euler angle of the markers in the plane of the webcam image. The rotations $\theta_\textrm{MCP}$, $\theta_\textrm{PIP}$, and $\theta_\textrm{DIP}$ of the metacarpophalangeal, proximal interphalangeal, and distal interphalangeal joints were then calculated by the difference between the marker rotations of their respective phalanges. The servo motor, microcontroller, and webcam were all controlled by the same laptop running a data collection script in MATLAB. 

\subsubsection{Stress Test}
To evaluate the sensor's ability to retain its response characteristics over time, we selected one finger and subjected it to cyclic motion over the full \qty{263.6}{\degree} motion range of the setup. We chose an angular speed for the tendon pulley of \qty{60}{rpm} corresponding with an actuation frequency of \qty{1.46}{Hz}, in line with existing actuation mechanisms in soft robots \cite{Li2023}. After performing all other experiments and applying 50 cycles to the finger to break it in, we applied 450 cycles in constant lighting conditions. Sensor data was acquired at \qty{8}{Hz}.

\subsection{Sensor Model}\label{subsec:sensor_model}
To obtain a calibration line for each sensor, we applied a least-squares linear regression to the data collected from all five quasi-static trials for each sensor according to a linear model of the form 
\begin{equation}\label{eq:model}
   \hat{y}=\begin{bmatrix}1 & \theta\end{bmatrix}\begin{bmatrix}\beta_0 \\ \beta_1\end{bmatrix}
\end{equation}
where $\hat{y}$ is the best fit of the sensor output $y$, $\begin{bmatrix}\beta_0 & \beta_1\end{bmatrix}^\top$ is the vector of regression coefficients. $y$ is expressed as optical power loss in decibels and is calculated as $y=-10\log_{10}(I/I_0)$, where $I$ is the raw digital value in nW/cm$^{2}$ provided by the sensor and $I_0$ is the raw value measured at the beginning of each trial.

\subsection{Results}
To evaluate the utility of the sensors for closed-loop control of soft robots, we examine the quality of the models in predicting the joint angle by calculating metrics for reproducibility, repeatability, sensitivity, linearity, and hysteresis. We first examined reproducibility by conducting an ANCOVA test using $\theta$ as the covariate and individual sensor responses as categorical factors. We observed a highly statistically significant difference between the responses [F(8,108072)=9822, p\textless 0.01], indicating that unique calibration lines should be used for each sensor. 

The sensitivity of each sensor was taken to be the slope of the calibration line, $\beta_1$. Percentage linearity was calculated as a percentage of full scale range (FSR) according to the formula in \cite{Fleming2013}: \begin{equation}
    \textrm{Linearity(\% FSR)}=\left(1-\frac{\max|y-\hat{y}|}{\max |y|}\right)\times 100\%
\end{equation}
To give a metric for repeatability, we calculated the RMS error measured in degrees $\sigma_\textrm{RMS}$ of each sensor according to the formula:
\begin{equation}
    \sigma_\textrm{RMS}=\frac{1}{\beta_1}\sqrt{\sum_{i=1}^N\frac{\left(y_i-\hat{y}_i\right)^2}{N-1}}
\end{equation}
The linearity values we achieved typically ranged between 59\% (PIP joint of finger 1) and 88\% (DIP joint of finger 2) with a mean value of 70\% and a single outlier due to the PIP joint of finger 3 with a value of 37\%. We note from these calibrations that the sensitivities are higher for fingers 2 and 3 than for finger 1. When taken with the results of the ANCOVA test, this suggests that the manual steps in the fabrication process, although limited, still affect the resulting sensor array.

To quantify hysteresis for each sensor, we took the difference between the mean sensor reading from the loading portions of the quasi-static trials and the mean sensor reading from the unloading portions of the quasi-static trials. We express this difference as a percentage of the FSR for each sensor in Fig. \ref{fig:fingers_results}f. We note that sensors with greater hysteresis also show less repeatability, as expected. 

We quantified the results of the stress test by measuring the normalized amplitude $(I_\textrm{max}-I_\textrm{min})/I_\textrm{max}$ during each cycle. Finger 3 completed the test with no detectable change in cycle amplitude on any of the three sensor readings (Fig. \ref{fig:stress}). This result indicates that the sensor is able to retain its response characteristics over a typical research service life. 
\begin{figure}
\centering
\includegraphics[width=0.9\columnwidth]{stress_plot.pdf}
\caption{Results of a stress test on one of the fingers. All sensors survived undergoing 450 actuation cycles, as demonstrated by their normalized cycle amplitude that do not decay over time.}
\label{fig:stress}
\vspace{-1em}
\end{figure}

\section{Experimental Validation}\label{sec:experimental_validation}% Sensorization
To demonstrate the consistency of the proposed methodology, we conducted an experimental validation using the H-Support prototype (Fig. \ref{fig:h_support_prototype}).
% % H-Support Prototype
% \begin{figure}
%     \centering
%     \includesvg[width=1.0\linewidth]{imgs/experiments/hsupport_real.svg}
%     \caption{The H-Support prototype is a cylindrical robot with 3 longitudinal pneumatic chambers and 4 tendons arranged with a helicoidal path. The VICON system sensorizes the \ac{CSR}, measuring the positions of the cross-sections each $\lambda_s = \SI{0.032}{\meter}$.}
%     \label{fig:h_support_prototype}
% \end{figure}
% H-Support Prototype
\begin{figure}
    \centering
    \includegraphics[width=1.0\linewidth]{imgs/experiments/hsupport_real.pdf}
    \caption{The H-Support prototype is a cylindrical robot with 3 longitudinal pneumatic chambers and 4 tendons arranged with a helicoidal path. The VICON system sensorizes the \ac{CSR}, measuring the positions of the cross-sections each $\lambda_s = \SI{0.032}{\meter}$.}
    \label{fig:h_support_prototype}
\end{figure}
% Actuation
This robot is actuated by three longitudinal pneumatic-driven actuators and four helicoidal tendon-driven actuators. Due to the physical constraints of the actuation sources, the input signals are bounded by \(\bm{\tau}_{1:3} \geq \bm{0}_3\) and \(\bm{\tau}_{4:7} \leq \bm{0}_4\). 
To enforce these constraints, we applied the absolute value operation to the standard signals, yielding the modified input 
$\bm{\tau} = \begin{bmatrix} |\bm{\tau}_{1:3}^{\top}| &  -|\bm{\tau}_{4:7}^{\top}|\end{bmatrix}^{\top}$.
% Justifying the application of absolute value
This modification slightly shifts the spectrum of standard signals, mainly affecting the average value. Applying the absolute value doubles the frequency rate for the chirp signal while preserving the spectrum's characteristics.

Concerning sensing, the H-Support is equipped with a Motion Capture system (VICON), which measures the roto-translation of sensorized cross-sections. 
Markers are placed at equal intervals of $\lambda_s = \SI{0.032}{\meter}$ along the robot's length, providing a sampling wavenumber of $\nu_s = \SI{31.25}{\meter^{-1}}$. Based on these measurements, the strain field samples can be calculated as:  
\begin{equation} \label{eq:vicon2strain}
    \bm{\xi}(n \lambda_s) = \frac{1}{\lambda_s}
    \left(\log_{SE(3)}\left(\bm{g}^{-1}\left(n \lambda_s\right) \bm{g}\left((n + 1) \lambda_s\right)\right)\right)^{\vee} ,
\end{equation}
where $\log_{SE(3)}(\cdot)$ represents the logarithmic map in the $SE(3)$ group \eqref{eq:logSE3_definition}. 
The relation \eqref{eq:vicon2strain} is derived by inverting the forward kinematics of a generic \ac{GVS} model \cite{mathew2024reduced}, considering a second-order Zanna collocation \cite{zanna1999collocation}.

Concerning time samples, the VICON system provides the measures at $f_s = \SI{100}{\hertz}$.

\subsection{Spectrum Analysis}
From the samples $\bm{\xi}\left(n \lambda_s, \, m T_s\right)$, it is possible to compute the \ac{STFT} through the \ac{FFT} algorithm. Fig. \ref{fig:stft_exp} shows the spectra of the experimental data, where zero-padding was applied to increase the resolution of the wavenumbers. 
% X(\cdot, j \omega)
\subsubsection{Time spectrum of the Spatial Harmonics}
%% Angular Strain Modes
% Torsion
The torsion spectrum reveals differences when compared to the \ac{STFT} obtained through simulation. The constant component at $\nu = \SI{0}{\meter^{-1}}$ is significantly less pronounced than the others, in contrast to the strain analysis presented in \eqref{eq:implicit_strain}. This can be attributed to the friction between the cables and the rod, which asymmetrically distributes the torsion along its length. At higher spatial frequencies, the torsion exhibits anti-resonance peaks accompanied by rapid phase transitions.

Specifically, the harmonics at $\nu = 7.8125, \, \SI{11.7188}{\meter^{-1}}$ demonstrate anti-resonance at $\SI{0.641}{\hertz}$, while the spatial component at $\nu = \SI{15.625}{\meter^{-1}}$ shows anti-resonance at $\SI{0.798}{\hertz}$. This shift in the frequency of peaks across spatial harmonics, also observed in the case of the Gaussian basis function, can again be attributed to friction between the robot and the cables. 
% Therefore, common friction models, such as the Capstan model \cite{alkayas2024soft, rone2013continuum}, typically include exponential terms.

% Bending y
For $\kappa_y$, the spatial harmonics $\nu = 0, \, 3.9062, \, \SI{7.8125}{\meter^{-1}}$ exhibit similar profiles. The magnitude gradually decreases until reaching anti-resonance peaks in the range of $0.797$ to $\SI{0.982}{\hertz}$. In the same range, the phase diagram shows a rapid increase in phase.
Similar to the torsion case, the peaks are slightly shifted between spatial components. At higher harmonics, a rapid decrease in magnitude is observed around $\SI{0.2991}{\hertz}$.

% Bending z
The bending component $\kappa_z$ exhibits a sequence of resonance and anti-resonance peaks, characterized by rapid phase changes. The harmonics at $\nu = 0, \, \SI{3.9062}{\meter^{-1}}$ initially display an anti-resonance peak at $\SI{0.3703}{\hertz}$, followed by a rapid phase decrease. Subsequently, a resonance peak occurs at $\SI{0.741}{\hertz}$, accompanied by another rapid phase transition.

% Linear Strain Modes

% % Experimental STFT
% \begin{figure*}
%     \centering
%     \includesvg[width=1.0\linewidth]{imgs/experiments/stft_exp.svg}
%     \caption{The \ac{STFT} of experimental data from the H-Support prototype. The magnitude values are normalized to $|\Xi_i(j 0, \, j 0)|$. Moreover, zero-padding is applied to increase the resolution of the spatial frequencies.}
%     \label{fig:stft_exp}
% \end{figure*}
% Experimental STFT
\begin{figure*}
    \centering
    \includegraphics[width=1.0\linewidth]{imgs/experiments/stft_exp.pdf}
    \caption{The \ac{STFT} of experimental data from the H-Support prototype. The magnitude values are normalized to $|\Xi_i(j 0, \, j 0)|$. Moreover, zero-padding is applied to increase the resolution of the spatial frequencies.}
    \label{fig:stft_exp}
\end{figure*}
% % Space-Time fitting
% \begin{figure*}
%     \centering
%     \includesvg[width=1.0\linewidth]{imgs/experiments/st_fitting.svg}
%     \caption{Comparison between the reconstructed strain through \ac{BPD} and the experimental strain. In grey the experimental strain samples.}
%     \label{fig:bpd_st}
% \end{figure*}
% Space-Time fitting
\begin{figure*}
    \centering
    \includegraphics[width=1.0\linewidth]{imgs/experiments/st_fitting.pdf}
    \caption{Comparison between the reconstructed strain through \ac{BPD} and the experimental strain. In grey the experimental strain samples.}
    \label{fig:bpd_st}
\end{figure*}

% X(jk, \cdot)
\subsubsection{Spatial spectrum of the Time Harmonics}
The curves \(\bm{\Xi}(jk, \, \cdot)\) provide insights into the spatial spectrum when the \ac{CSR} is subjected to a specific time-frequency input. A common feature across all strain modes is the prominence of the constant time-harmonic (i.e., \(f = \SI{0}{\hertz}\)), which arises from the mechanical nature of the \acp{CSR}. Below, we discuss the spectra of the different strain modes in detail.

% Torsion
For torsion ($\kappa_x$), the magnitude increases slightly up to $\SI{5}{\meter^{-1}}$ at the time harmonics $f = 0, \, 12.4929, \, \SI{25}{\hertz}$. Beyond that, the magnitude decreases until $\nu = \SI{10}{\meter^{-1}}$, where an anti-resonance appears in space. The higher harmonics exhibit a smoother spectrum, both in terms of magnitude and phase.

% Bending
In contrast, the magnitude of the constant component ($f = \SI{0}{\hertz}$) for $\kappa_y$ and $\kappa_z$ decreases smoothly, reaching values of $\SI{-0.5}{\decibel}$ and $\SI{-0.25}{\decibel}$ at $\nu = \SI{0}{\meter^{-1}}$, respectively. After this point, the negative slope of the magnitude increases. The $\kappa_z$ mode shows an anti-resonance peak in space at $\nu = \SI{13.4277}{\meter^{-1}}$, which is accompanied by a rapid shift in phase. Higher time harmonics display similar behavior, with anti-resonance peaks occurring at different wavenumbers.

% Linear strain modes
The linear strain modes show distinct spectra. The constant component (\(f = \SI{0}{\hertz}\)) increases the magnitude, highlighting predominant spatial harmonics. Specifically, the stretch ($\sigma_x$) shows the highest magnitude ($\SI{9}{\decibel}$) at $\nu = \SI{3.906}{\meter^{-1}}$. The shear along the $y$-axis ($\sigma_y$) exhibits the highest magnitude at $\nu = \SI{6.5918}{\meter^{-1}}$ with a value of $\SI{20.4274}{\decibel}$. In contrast, the shear along the $z$-axis shows a negative slope in magnitude, similar to the spectrum of bending around the $y$-axis.


\subsection{Strain Fitting}
 From the experimental samples, the strain field can be reconstructed using the \ac{BPD} algorithm. The signal dictionary $\bm{B}_{\bm{q}}$ is composed of the polynomial and trigonometric bases.
 Fig. \ref{fig:bpd_st} shows the result of the fitting, both in space and in time. The \ac{BPD} reduces the noise from the strain samples, allowing an accurate representation, and leveraging the signal's sparsity. 
 The displayed results are obtained for the sparsity vector $\bm{\gamma} = [0.5, \,  0.5, \,  0.5, \, 0.07, \,  0.05, \,  0.05]^{\top}$.

The first row of Fig. \ref{fig:bpd_q} shows the evolution of the coefficients $\bm{q}$ over time, providing insight into the relevance of each basis. As detailed in Sec. \ref{spectrum_extraction:bpd}, the energy ratio for each basis is calculated and presented in the second row of Fig. \ref{fig:bpd_q}. The third row depicts the average truncation index over time, emphasizing the most significant bases. Lastly, the fourth row illustrates the total energy of the reconstructed signal over time.

%% Discussion %%
% Angular Strain Mode
The torsion exhibits a diverse combination of basis functions, including both polynomial and trigonometric components, consistent with the spectrum analysis.
For $\kappa_y$, the most dominant basis polynomial, specifically the constant and first-order terms. The second-order trigonometric basis ($\nu = \SI{2}{\meter^{-1}}$) plays a notable role, particularly in the first 10 seconds.
Regarding $\kappa_z$, the constant term and the first-order trigonometric basis are the most significant contributors.

% Linear Strain Mode
The trigonometric bases are mostly used to reconstruct the linear strain modes. 
For the stretch component $\sigma_x$, the constant term is zero, with the first-order trigonometric basis dominating the reconstruction. The shear components $\sigma_y$ and $\sigma_z$ are accurately approximated by first- and second-order trigonometric terms. Unlike the other linear modes, $\sigma_z$ exhibits a significant constant component.
These results align with the previously discussed spectra, where the curves $\bm{\Xi}(jk, \, \cdot)$ exhibit an increase in magnitude corresponding to these wavenumbers.

% % q of BPD
% \begin{figure*}
%     \centering
%     \includesvg[width=1.0\linewidth]{imgs/experiments/bpdn.svg}
%     \caption{Application of \ac{BPD} to the experimental data. The coefficients $\bm{q}(t)$ are displayed over time in the first row. The energy contribution of each basis over time is shown in the second row, and the average energy contribution value is shown in the third. The total energy of the reconstructed strain is shown in the final row.}
%     \label{fig:bpd_q}
% \end{figure*}
% q of BPD
\begin{figure*}
    \centering
    \includegraphics[width=1.0\linewidth]{imgs/experiments/bpdn.pdf}
    \caption{Application of \ac{BPD} to the experimental data. The coefficients $\bm{q}(t)$ are displayed over time in the first row. The energy contribution of each basis over time is shown in the second row, and the average energy contribution value is shown in the third. The total energy of the reconstructed strain is shown in the final row.}
    \label{fig:bpd_q}
\end{figure*}

\subsection{Comparison between the measured and reconstructed Backbone}
To evaluate the accuracy of the proposed method, we compare the reconstructed backbone with experimental data obtained from the VICON system. Figure \ref{fig:bpd_se3} shows the reconstructed backbone without truncation (colored) alongside the corresponding experimental pose (gray) for the sensorized cross-sections at eight distinct time instants.

Below each time instant, the orientation and position errors are reported for three truncation thresholds—no truncation ($0\%$), $1\%$, and $5\%$—with all bases below the corresponding truncation index discarded.
The position error is defined as the Euclidean distance between the centers of the reconstructed and experimental cross-sections, while the orientation error is computed using the $\textnormal{dist}_{SO(3)}$ operator \eqref{eq:distso3_definition}. Without truncation, the maximum position error reaches $\SI{7.190}{m\meter}$ (i.e., $3.7448 \% L$), while the maximum orientation error is $6.284^{\circ}$.

Regardless of the truncation threshold, the errors exhibit an increasing trend, which can be attributed to the propagation of fitting errors in the forward kinematics. As these errors accumulate through the integration of the reconstructed strain field, the tip pose is the most affected, reflecting the cumulative effect of all preceding inaccuracies.

In terms of truncation, applying a $1\%$ threshold eliminates $11$ \ac{DoFs}, resulting in a maximum position error of $\SI{7.0872}{m\meter}$ and an orientation error of $6.484^{\circ}$. Increasing the truncation to $5\%$ reduces the model by $23$ \ac{DoFs}, but leading to a higher position error of $\SI{9.076}{m\meter}$ (i.e., $4.7271 \% L$) and a orientation error of $6.443^{\circ}$.

% % Fitting in SE(3)
% \begin{figure*}
%     \centering
%     \includesvg[width=1.0\linewidth]{imgs/experiments/hsupport_fitted.svg}
%     \caption{Comparison between the experimental and reconstructed backbone using \ac{BPD}. The last row presents the position and orientation errors, which exhibit an increasing trend along the length of the rod. The errors are computed with the truncation index of $0\%$, $1\%$, and $5\%$.}
%     \label{fig:bpd_se3}
% \end{figure*}
% Fitting in SE(3)
\begin{figure*}
    \centering
    \includegraphics[width=1.0\linewidth]{imgs/experiments/hsupport_fitted.pdf}
    \caption{Comparison between the experimental and reconstructed backbone using \ac{BPD}. The last row presents the position and orientation errors, which exhibit an increasing trend along the length of the rod. The errors are computed with the truncation index of $0\%$, $1\%$, and $5\%$.}
    \label{fig:bpd_se3}
\end{figure*}
%
% \subsection{Parametric Identification}
% % Parameters
% \begin{table}
\centering
\caption{Geometrical and estimated physical parameters of the H-Support Prototype}
\label{tab:exp_parameters}
    \begin{tabular}{lll}
    \toprule
    Name                    &   Symbol          &   Value                                                               \\
    \midrule
    Length                  &   $L$               &   \SI{0.192}{\meter}                                                            \\
    Cross-Section Radius               &   $R_{\textnormal{cs}}$               &   \SI{0.03}{\meter}                                                            \\
    Density                 &   $\bar{\rho}$    &   \SI{168.9179}{\kilogram/\meter^3}                                     \\     
    Young's Modulus         &   $E$             &   \SI{1.0}{M\pascal}                                             \\
    Poisson Ratio           &   $\nu$           &   0.5                                                                     \\
    Damping Coefficient                &   $\beta$         &   \SI{0.01}{M\pascal \cdot \second}         \\    
    \bottomrule
    \end{tabular}
\end{table}
% % Method and Results
% To compare the effectiveness of the \ac{BPD} algorithm on the dynamic model, we applied a parametric identification to fit the dynamic model that best matches the experimental data.
% Let the parameter vector be $\bm{\pi} = \left[\rho, \, E, \, \beta\right]^{\top} \in \mathbb{R}^3$. The best set of parameters is $\bm{\pi}^{*} = \underset{\bm{\pi}}{\arg \min} \left\{\frac{1}{M} \sum_m \textnormal{MSE}_{SE(3)}(m T_s)\right\}$, where
% \begin{equation} \label{eq:mse3_definition}
% \textnormal{MSE}_{SE(3)} = \frac{1}{N} \sum_n \textnormal{dist}^{2}_{SE(3)}\left(\bm{g}_{\textnormal{exp}}(n \lambda_s), \bm{g}_{\textnormal{sim}}(n \lambda_s)\right) \, .
% \end{equation}
% In \eqref{eq:mse3_definition}, $\bm{g}_{\textnormal{exp}}(n \lambda_s, \, m T_s) \in SE(3)$ is the measured pose of the $n$-th sensorized cross-section, while $\bm{g}_{\textnormal{sim}}(n \lambda_s, \, m T_s) \in SE(3)$ is the simulated correspondent. Finally, the $\textnormal{dist}_{SE(3)}$ operator evaluates the distance in the $SE(3)$ group and is defined in \eqref{eq:distse3_defintion}.

\section{Conclusion}\label{sec:conclusion}
\section*{Conclusion}
This paper aims to enhance our understanding of the computational complexity of computing various Shapley value variants. We found that for various ML models --- including decision trees, regression tree ensembles, weighted automata, and linear regression --- both local and global interventional and baseline SHAP can be computed in polynomial time under HMM modeled distributions. This extends popular algorithms, such as TreeSHAP, beyond their empirical distributional scope. We also establish strict complexity gaps between the various SHAP variants (baseline, interventional, and conditional) and prove the intractability of computing SHAP for tree ensembles and neural networks in simplified scenarios. Overall, we present SHAP as a versatile framework whose complexity depends on four key factors: \begin{inparaenum}[(i)] \item model type, \item SHAP variant, \item distribution modeling approach, \item and local vs. global explanations\end{inparaenum}. We believe this perspective provides deeper insight into the computational complexity of SHAP, paving the way for future work.




%We believe that our framework provides a more intricate understanding of SHAP computation complexity across different models, distributions, and variants, paving the way for further research.

Our work opens promising directions for future research. First, expanding our computational analysis to other SHAP-related metrics, such as asymmetric SHAP~\citep{frye20} and SAGE~\citep{covert2020understanding}, would be valuable. Additionally, we aim to explore more expressive distribution classes and relaxed assumptions beyond those in Section \ref{sec:tractable} while maintaining tractable SHAP computation. Finally, when exact computation is intractable (Section \ref{sec:intractable}), investigating the approximability of SHAP metrics through approximation and parameterized complexity theory~\citep{downey2012parameterized} is an important direction.

%Our work opens several promising avenues for future research on the computational properties of explainable AI methods, with a particular focus on SHAP. First, it would be interesting to broaden the computational analysis conducted in this work to include other popular SHAP-related metrics in the literature, such as asymmetric SHAP \cite{frye20} and SAGE \cite{covert2020understanding}. Also, in the future, we aim to explore more expressive distribution classes and relaxed distributional assumptions—extending beyond those examined in Section \ref{sec:tractable} —that still yield tractable SHAP computation. Finally, when exact computation proves intractable (Section \ref{sec:intractable}), it is worthwhile to theoretically investigate the question of the approximability of computing the SHAP metrics across various configurations, through the lens of approximation and parametrized complexity theory \cite{arora2009computational}.

%This paper aims to deepen our understanding of the computational complexity involved in obtaining different Shapley value variants. We found that for a variety of ML models, including decision trees, tree ensembles for regression, weighted automata, and linear regression models — computing both local and global interventional and baseline SHAP can be done in polynomial time when distributions are modeled by HMMs. This extends the distributional scope of popular algorithms like TreeSHAP, which is limited to empirical distributions. Additionally, we demonstrate a strict complexity gap between SHAP variants, showing that interventional and baseline SHAP can be strictly easier to compute than conditional SHAP. Despite these positive results, we uncovered intractability for various SHAP variants in neural networks and tree ensembles. Finally, we provided generalized complexity relations across SHAP variants. We believe that our framework offers a deeper understanding of the complexity involved in computing SHAP across various variants, models, distributions, as well as in both local and global computations, laying the groundwork for future research.


\bibliographystyle{IEEEtran}
\bibliography{references.bib}

\end{document}
