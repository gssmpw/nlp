The creation of the multi-material sensorized finger requires a short sequence of mostly automated manufacturing steps, displayed in Fig.  \ref{fig:fabrication_figure}. In this section, we analyze these steps in detail. In particular, the manual embedding step in which the POFs are encased in holders and the subsequent manual embedding steps in which the POFs, holders, and FPCBs are embedded into the flexure take 10 minutes in total and requires no adhesives and no special tools other than a razor blade and a soldering iron. 
\subsection{Optical Fiber Embedding Step}
\begin{figure}
    \centering
    \includegraphics[width=0.9\columnwidth]{FiberEmbedding_compressed_revised.pdf}
    \caption{Closeup photo of a pair of holders with an embedded optical fiber and a schematic detailed view of the embedding channel. The embedded fiber is able to support a mass of \qty{300}{\g}.}
    \label{fig:clearances_and_embed_closeups}
    \vspace{-1em}
\end{figure}

The fabrication process started with the embedding of the POFs. For this, we implemented a pause-and-place method where the the 3D printing of custom-designed holders with channels to accommodate the fiber was temporarily halted to enable the placement of the POFs before printing was resumed. The pause was scheduled at the layer where the channel was completed, to ensure good localization of the POF and minimize damage due to the interaction with the 3D printer nozzle.

Unjacketed ESKA POF was used with a diameter of \qty{500}{\um}, a commonly available POF manufactured from polymethyl methacrylate. The holders were printed on a Tenlog TL-D3 Pro independent dual extruder (IDEX) 3D printer (Innocube3D, Shenzhen, China) out of polylactic acid (PLA), with extruder temperature set to \qty{210}{\degree C}, a layer height of \qty{0.2}{\mm}, and a print speed of \qty{60}{\mm/s}. In preliminary experiments, we utilized POFs with a black polyethylene jacket with outer diameter \qty{1}{\mm} and core diameters of \qty{250}{\um} and \qty{500}{\um}. However, we found that when attempting to embed these POFs, the polyethylene jacket was damaged by the extruder nozzle during the printing process and allowed the fibers to easily slip out. We therefore switched to unjacketed POFs with core diameters of \qty{1000}{\um}, \qty{500}{\um}, and \qty{250}{\um}. We chose \qty{500}{\um} POFs to manufacture the sensors in this study because the \qty{1000}{\um} POFs exhibited high plastic strain at the bending radii required for the robotic finger geometry, while the \qty{250}{\um} POFs were challenging to embed repeatably due to the tolerances of our 3D printer.

When readying the holders for printing, we used the slicing software Cura to place them at the same spacing as the final spacing of the finger phalanges, \qty{28}{\mm} apart. One holder was printed for each of the four phalanges and a single fiber was embedded running through all four holders. After printing, the POF was cut with a surgical blade, as required to complete the POF-holder assembly. 

To enhance the repeatability and holding strength of the fiber embedding, we designed the fiber-accommodating channel with a V-shape as shown in Fig. \ref{fig:clearances_and_embed_closeups} to help the fiber self-center when pushed downward by the extruder nozzle. The height of the channel was \qty{0.4}{\mm}, equal to two layer heights, with a width of \qty{0.6}{\mm} on the higher layer to allow the fiber to be set into place and a width of \qty{0.1}{\mm} on the lower layer to form the V-shape. With this design, the fiber sits proud of the holders when the print is resumed so that the next extruded layer presses tightly against the top of the fiber. To further improve printing success, sacrificial jigs with channels to hold the fiber were printed in between each holder and the wall printing order was modified in Cura so that the printer would lock the POFs in place first before moving on to printing the remainder of the holders. A \qty{15}{\mm}$\times$\qty{15}{\mm}$\times$\qty{1.6}{\mm} cuboid was printed alongside the holders and jigs to prevent under-extrusion of the post-pause printed layer, and the printing temperature was temporarily increased to \qty{230}{\degree C} to promote inter-layer adhesion. The embedding process just described resulted in a reliable axial locking of the fibers, which could withstand loads of \qty{300}{g} without failing (Fig. \ref{fig:clearances_and_embed_closeups}).

\subsection{Flexible Printed Circuit Boards and Holder Design}
The FPCBs are based on the OPT3002 light-to-digital converter (LDC) (Texas Instruments, Dallas, TX, USA) and have a \qty{10}{\mm}$\times$\qty{10}{\mm} outline. We placed the LDC, supporting components, and a side-view 3010 LED on one side of each FPCB. We used the other side of the FPCBs for a selectable jumper to set the address of each LDC on the inter-integrated circuit (I2C) bus for power and communication wires.   

The highest spectral response of the LDC occurs at a wavelength of \qty{505}{nm}, so a yellow LED with a peak wavelength of \qty{590}{nm} was used. Because this response is in the visible spectrum, the LDC is sensitive to ambient light. We accepted this compromise because of the LDC's digital readout and miniaturized off-the-shelf package and we mitigated the effects of ambient light by 3D printing the outer casing of the sensor in black. 

The holders that lock the POFs in place also hold the FPCBs. The holders are designed with cavities that fit the circuit components of the FPCB and a \qty{0.6}{\mm} lip around the perimeter of the FPCB. We chose this height for the lip because it was the tallest height that could be achieved without interfering with the 3D printer nozzle during the sensor embedding process.

\subsection{Sensor Embedding Steps}
To prepare the FPCBs for embedding, we utilized a 3D printed jig to hold the FPCBs at the phalanx spacing before soldering \qty{0.15}{\mm} diameter enameled copper wires between each FPCB for the power and communication bus. After soldering, we encapsulated the solder joints to the wires in superglue to provide strain relief. We then embedded both the POF-holder and FPCB assemblies in the 3D printed thermoplastic polyurethane (TPU) (Filaflex 82A, Recreus, Elda, Spain) flexure in two pausing steps. During the first pause, we placed the POF-holder assembly into a cavity in the flexure. The printer then encapsulated the POFs while leaving the holders exposed to accept the FPCBs. During the second pause, the FPCB assembly was placed into the holders and a soldering iron with a chisel tip at \qty{350}{\degree C} was used to partially melt the midpoints of the lips over the sides of the FPCBs to lock them in place before resuming the print again. 

\subsection{Multi-material Finger Design}\label{subsec:finger_design}
We designed the multi-material finger with rigid phalanges printed from PLA and soft flexures connecting the phalanges printed from soft TPU with shore hardness 82A. The soft TPU gives the finger joint compliance and eliminates the need for any assembly or fasteners in the design, while the use of rigid PLA controls the kinematics of the actuated finger to allow the sensor angle readouts to predict the finger's shape. To facilitate printing with the TPU filament, we replaced one of the TL-D3 Pro's extruders with a Micro Swiss NG extruder (Ramsey, Minnesota, USA) and used print settings following the manufacturer's recommendations. 

The joints consist of a sliding contact pair between an inner and outer circle with \qty{0.12}{\mm} clearance in the undeformed state with the TPU flexures arranged symmetrically on both sides of the finger. The sliding contact pair geometry was selected as it provided a linear optical power loss. The sensors are unable to sense bending direction as their transduction principle is based on optical waveguide macrobending loss, therefore we included mechanical stops in the joint design to restrict bending to one direction. 

The finger is tendon-actuated, with the tendons made from \qty{0.4}{\mm} nylon fishing line and embedded at the mid-plane of the finger using another pause-and-place procedure. We secured the tendons to the distal phalanx by attaching split shot fishing weights to the tendons, placing the fishing weights in printed voids, and filling the voids around the fishing weights with superglue. 

The sensors are only embedded in the flexures on one side of the finger. To facilitate sensor characterization, flexures containing sensors were printed separately and inserted into a shaped cavity on the multi-material finger, held in place by compression. TPU flexures that did not contain sensors were printed with a high infill of 80\%. 