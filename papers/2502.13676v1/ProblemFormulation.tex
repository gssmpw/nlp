\section{Autonomous bicycle control}
\label{sec:2}
This paper addresses the problem of designing a unified control method for balancing an autonomous bicycle while tracking a reference lean angle. First, we use a simple point-mass model to design an inner-loop FL control. Next, DeePO enhances system performance by adapting feedback gains based on input-output data from sensors mounted on the bicycle. 

\subsection{Bicycle Dynamics and Mathematical Modeling}
We consider a simple nonlinear model to represent the bicycle dynamics as in~\cite{Persson_2024}. 
\begin{equation}
    \label{eq:simpleModel}
    \begin{aligned}
        \ddot{\varphi}(t)&=\frac{g}{h}\sin\big(\varphi(t)\big)+ \frac{a}{bh}\cos\big(\varphi(t)\big)v\dot\delta(t) -\\
        & \left(\frac{1}{bh}- \frac{1}{b^2}\tan\big(\delta(t)\big) \tan\big(\varphi(t)\big)\right)\tan\big(\delta(t)\big)v^2,
    \end{aligned}
\end{equation}
where $\varphi(t)$, $\dot{\varphi}(t)$, $\delta(t)$, and $\dot{\delta}(t)$ represent the lean angle, lean rate, steering angle, and the controlled steering rate, respectively. The contact point between the rear wheel and the ground is denoted by $p_1$. Additionally, the vertical and horizontal distances between the bicycle's center of gravity and $p_1$ are denoted by $a$ and $h$, respectively. The wheelbase is denoted by $b$, while $g$ represents the gravitational constant, and $v$ represents the forward velocity. This model assumes a vertical steering axis, i.e., $\nu = \frac{\pi}{2}$, which results in zero trail. Furthermore, it is assumed that the steering axis can be controlled without delay and that the bicycle travels at a constant forward velocity. The visual representation of the parameters in \eqref{eq:simpleModel} is shown in Fig.~\ref{fig:BikeModel}.

\begin{figure}[t]
    \centering
    \includegraphics[width=0.95\columnwidth]{BikeModel.pdf}
    \caption{Illustration of the parameters used in the bicycle model in \eqref{eq:simpleModel}.}
    \label{fig:BikeModel}
\end{figure}

A bicycle is self-stabilized between the so-called \textit{weave speed} and \textit{capsize speed}. By analyzing the eigenvalues of the Whipple model and identifying the region where they are all negative, the self-stable region of a bicycle can be localized~\cite{kooijman2008}. A similar eigenvalue analysis for the instrumented bicycle considered in this paper was previously conducted, where the $25$ parameters required for the Whipple model were measured~\cite{persson2023control}. Based on this analysis, we focus on forward speeds of approximately $8$ km/h ($2.22$ m/s), below the weave speed, as shown in Fig.~\ref{fig:velAnalysis}. Thus, the system we aim to control is open-loop unstable, nonlinear, and non-holonomic, presenting a challenging control problem. Due to the system's inherent instability, applying a persistently exciting input without additional stabilization can lead to a loss of balance and cause the system to diverge. Specifically, an uncontrolled persistently exciting input could destabilize steering actions, such as turning left while leaning right, making it impractical to rely solely on such input for collecting persistently exciting data. In the following, we present an FL controller that balances the bicycle, simplifying the acquisition of persistently exciting data and mitigating some of the system's nonlinearities.

\begin{figure}[t]
    \centering
    \includegraphics{velAnalysis.pdf}
    \caption{Stable and unstable regions of the instrumented bicycle, with the weave speed and capsize speed denoted by $v_w$ and $v_c$, respectively.}
    \label{fig:velAnalysis}
\end{figure}

\subsection{Control overview}
A common approach for realizing a persistently exciting input is to utilize a random signal~\cite{alsalti2023design}. However, this approach is not directly applicable as it jeopardizes the bicycle's balance. Instead, we pre-stabilize the bicycle with an inner control loop using FL. Since the relative degree between the output and the number of states in the model given by \eqref{eq:simpleModel} does not match, the system can only be partially linearized using output FL~\cite{slotine1991applied}. 

If we choose $x = \begin{bmatrix}
    x_1, & x_2, & x_3
\end{bmatrix} = \begin{bmatrix}
    \varphi(t), & \dot{\varphi}(t), & \delta(t)
\end{bmatrix}, $ $y(t) = \varphi(t)$, $\dot{y}(t) = \dot{\varphi}(t)$, and represent the reference output as $y_r = [y_r(t), \dot{y}_r(t), \ddot{y}_r(t)]$, we can express the considered FL control law as:

\begin{equation}
    u(t) = \dot{\delta}(t) = \frac{1}{p(x)}(w - f(x)),
    \label{eq:FLinput}
\end{equation}

where 
\begin{align}
\label{eq:control}
    f(x)&= - \left(\frac{1}{bh}- \frac{1}{b^2}\tan\big(x_3\big) \tan\big(x_1\big)\right)\tan\big(x_3\big)v^2 \nonumber \\
      &\phantom{=} + \frac{g}{h}\sin\big(x_1\big) \nonumber \\
    p(x) & =\frac{a}{bh}\cos\big(x_1\big)v, \nonumber \\
    w & =  \ddot y_r(t) + k_1\left(\dot y_r(t)-\dot y(t)\right) +k_2 \left(y_r(t)-y(t)\right),\\ \nonumber
\end{align}
with appropriate choices of $k_1 > 0$ and $k_2 > 0$ to partially compensate for the system's nonlinearities. However, the steering angle $\delta(t)$ remains an internal state that is not directly linearized, meaning some nonlinear dynamics persist. In particular, terms involving $\tan(\delta(t))$ introduce coupling effects that remain even after feedback linearization. Additionally, since the steering angle evolves according to $u = \dot{\delta}$, it can drift over time uncontrolled, requiring further regulation to prevent undesired effects on system stability.
Furthermore, the proposed FL controller is designed based on continuous-time dynamics, with the model and control parameters provided in \eqref{eq:simpleModel} and Table~\ref{tab:modParam}, respectively. However, in practice, we implement it using a sampled-data approach with a hold mechanism, which may introduce inaccuracies and lead to performance degradation due to the discrete nature of the implementation. This discrete approach may not fully capture the continuous dynamics of the system~\cite{kimber1991sampled, grizzle1988feedback}. Moreover, parameters in~\eqref{eq:FLinput} and~\eqref{eq:control} are subject to parametric uncertainty, resulting in an inaccurate canceling of nonlinearties.  Nevertheless, we demonstrate that the potential limitations of the FL controller can be mitigated by incorporating DeePO.

\begin{table}[b]
\centering
\caption{Model and control parameters for FL control}
\label{tab:modParam}
\begin{tabular}{@{}llll@{}}
\toprule
\textbf{Parameter}  & \textbf{Symbol} & \textbf{Value} & \textbf{Unit} \\ \midrule
CoG w.r.t $p_1$ (x) & $a$             & $0.550$        & m             \\
CoG w.r.t $p_1$ (z) & $h$             & $0.700$        & m             \\
Wheelbase           & $b$             & $1.200$        & m             \\
Gravity             & $g$             & $9.82$         & m/s$^2$       \\
$k_1$               & -               & 1              & -             \\
$k_2$               & -               & 6              & -             \\ \bottomrule
\end{tabular}
\end{table}

The proposed FL controller functions as an inner control loop to stabilize the bicycle, enabling the use of an additive random signal as either a persistently exciting input or a performance enhancing adaptive control. In the remainder of the paper, we consider the autonomous bicycle with FL as our target system to control by DeePO, as highlighted by the gray box in Fig.~\ref{fig:controlOverview}. With this stable inner-loop system in place, we shift our focus to enhancing performance and compensating for the remaining nonlinearities using an adaptive, direct data-driven control approach in the outer loop.
 

\begin{figure}[t]
    \centering
    \includegraphics{controlOverview.pdf}
    \caption{Control overview}
    \label{fig:controlOverview}
\end{figure}






