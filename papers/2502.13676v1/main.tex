\documentclass[lettersize,journal]{IEEEtran}
\usepackage{amsmath,amsfonts}
\usepackage{algorithm}
\usepackage{array}
\usepackage[caption=false,font=normalsize,labelfont=sf,textfont=sf]{subfig}
\usepackage{textcomp}
\usepackage{stfloats}
\usepackage{url}
\usepackage{verbatim}
\usepackage{graphicx}
\usepackage{cite}
\usepackage{tikz}
\usepackage{bm}
\usepackage{booktabs}
\usepackage{hyperref}
\usepackage{adjustbox}
\usepackage{xspace}
\usepackage{svg}
\usepackage{soul}
\usetikzlibrary{positioning,calc,decorations.markings,shapes,arrows,arrows.meta,backgrounds}
\usepackage{pgfplots}
\usepackage{standalone}
\pgfplotsset{compat=newest}
\usepgfplotslibrary{groupplots}
\usetikzlibrary{patterns}
\usetikzlibrary{patterns.meta}
\usetikzlibrary{external}
\tikzexternalize[prefix=figs/tikz_pdf/]


\hyphenation{op-tical net-works semi-conduc-tor IEEE-Xplore}

\usepackage{algpseudocode}
\def \qed {\hfill \vrule height6pt width 6pt depth 0pt}
\usepackage{mathrsfs}
\usepackage{xcolor}
\renewcommand{\algorithmicrequire}{\textbf{Input:}}  % Use Input in the format of Algorithm
\renewcommand{\algorithmicensure}{\textbf{Output:}} % Use Output in the format of Algorithm
\newcommand{\red}{\color{red}}
\newcommand{\green}{\color{green}}
\allowdisplaybreaks[4]
\newtheorem{definition}{Definition}
\newtheorem{theorem}{Theorem}
\newtheorem{lemma}{Lemma}
\newtheorem{assum}{Assumption}
\newtheorem{coro}{Corollary}
\newtheorem{remark}{Remark}
\newtheorem{prop}{Proposition}
\newtheorem{example}{Example}

\DeclareMathOperator{\diagon}{diag}
\newcommand{\diag}[1]{\diagon(#1)}
\newcommand{\etal}{et al.\xspace}
\pgfmathsetmacro{\RadToDeg}{180/3.14159265359}
\newcommand{\dt}[1]{_{#1}}
\newcommand{\dti}[2]{_{#1}^{#2}}


\begin{document}

\title{An Adaptive Data-Enabled Policy Optimization Approach for Autonomous Bicycle Control}

\author{Niklas Persson, \IEEEmembership{Student member, IEEE}, Feiran Zhao, Mojtaba Kaheni, \IEEEmembership{Senior Member, IEEE}, Florian D\"orfler, \IEEEmembership{Senior Member, IEEE}, Alessandro V. Papadopoulos, \IEEEmembership{Senior Member, IEEE}
        % <-this % stops a space
\thanks{This work was supported by the Knowledge Foundation (KKS) with grant ``Mälardalen University Automation Research Center (MARC)'', n. 20240011.}
\thanks{N. Persson, M. Kaheni, and A.V. Papadopoulos are with the Division of Intelligent Future Technologies, M\"alardalen University, 721 23 V\"aster{\aa}s, Sweden.    (e-mails: niklas.persson@mdu.se, mojtaba.kaheni@mdu.se, alessandro.papadopoulos@mdu.se).}% <-this % stops a space
\thanks{F. Zhao and F. D\"orfler are with the Department of Information Technology and Electrical Engineering, ETH Zurich, 8092 Zurich, Switzerland. (e-mail: zhaofe@control.ee.ethz.ch,
dorfler@ethz.ch}% <-this % stops a space
\thanks{Manuscript received February 17, 2025}}


\maketitle

\begin{abstract}
This paper presents a unified control framework that integrates a Feedback Linearization (FL) controller in the inner loop with an adaptive Data-Enabled Policy Optimization (DeePO) controller in the outer loop to balance an autonomous bicycle. While the FL controller stabilizes and partially linearizes the inherently unstable and nonlinear system, its performance is compromised by unmodeled dynamics and time-varying characteristics. To overcome these limitations, the DeePO controller is introduced to enhance adaptability and robustness.
The initial control policy of DeePO is obtained from a finite set of offline, persistently exciting input and state data. To improve stability and compensate for system nonlinearities and disturbances, a robustness-promoting regularizer refines the initial policy, while the adaptive section of the DeePO framework is enhanced with a forgetting factor to improve adaptation to time-varying dynamics.
The proposed DeePO+FL approach is evaluated through simulations and real-world experiments on an instrumented autonomous bicycle. Results demonstrate its superiority over the FL-only approach, achieving more precise tracking of the reference lean angle and lean rate.
\end{abstract}

\begin{IEEEkeywords}
Adaptive control, policy optimization, direct data-driven control, balance control, autonomous bicycle. 
\end{IEEEkeywords}


 
\section{Introduction}

Large language models (LLMs) have achieved remarkable success in automated math problem solving, particularly through code-generation capabilities integrated with proof assistants~\citep{lean,isabelle,POT,autoformalization,MATH}. Although LLMs excel at generating solution steps and correct answers in algebra and calculus~\citep{math_solving}, their unimodal nature limits performance in plane geometry, where solution depends on both diagram and text~\citep{math_solving}. 

Specialized vision-language models (VLMs) have accordingly been developed for plane geometry problem solving (PGPS)~\citep{geoqa,unigeo,intergps,pgps,GOLD,LANS,geox}. Yet, it remains unclear whether these models genuinely leverage diagrams or rely almost exclusively on textual features. This ambiguity arises because existing PGPS datasets typically embed sufficient geometric details within problem statements, potentially making the vision encoder unnecessary~\citep{GOLD}. \cref{fig:pgps_examples} illustrates example questions from GeoQA and PGPS9K, where solutions can be derived without referencing the diagrams.

\begin{figure}
    \centering
    \begin{subfigure}[t]{.49\linewidth}
        \centering
        \includegraphics[width=\linewidth]{latex/figures/images/geoqa_example.pdf}
        \caption{GeoQA}
        \label{fig:geoqa_example}
    \end{subfigure}
    \begin{subfigure}[t]{.48\linewidth}
        \centering
        \includegraphics[width=\linewidth]{latex/figures/images/pgps_example.pdf}
        \caption{PGPS9K}
        \label{fig:pgps9k_example}
    \end{subfigure}
    \caption{
    Examples of diagram-caption pairs and their solution steps written in formal languages from GeoQA and PGPS9k datasets. In the problem description, the visual geometric premises and numerical variables are highlighted in green and red, respectively. A significant difference in the style of the diagram and formal language can be observable. %, along with the differences in formal languages supported by the corresponding datasets.
    \label{fig:pgps_examples}
    }
\end{figure}



We propose a new benchmark created via a synthetic data engine, which systematically evaluates the ability of VLM vision encoders to recognize geometric premises. Our empirical findings reveal that previously suggested self-supervised learning (SSL) approaches, e.g., vector quantized variataional auto-encoder (VQ-VAE)~\citep{unimath} and masked auto-encoder (MAE)~\citep{scagps,geox}, and widely adopted encoders, e.g., OpenCLIP~\citep{clip} and DinoV2~\citep{dinov2}, struggle to detect geometric features such as perpendicularity and degrees. 

To this end, we propose \geoclip{}, a model pre-trained on a large corpus of synthetic diagram–caption pairs. By varying diagram styles (e.g., color, font size, resolution, line width), \geoclip{} learns robust geometric representations and outperforms prior SSL-based methods on our benchmark. Building on \geoclip{}, we introduce a few-shot domain adaptation technique that efficiently transfers the recognition ability to real-world diagrams. We further combine this domain-adapted GeoCLIP with an LLM, forming a domain-agnostic VLM for solving PGPS tasks in MathVerse~\citep{mathverse}. 
%To accommodate diverse diagram styles and solution formats, we unify the solution program languages across multiple PGPS datasets, ensuring comprehensive evaluation. 

In our experiments on MathVerse~\citep{mathverse}, which encompasses diverse plane geometry tasks and diagram styles, our VLM with a domain-adapted \geoclip{} consistently outperforms both task-specific PGPS models and generalist VLMs. 
% In particular, it achieves higher accuracy on tasks requiring geometric-feature recognition, even when critical numerical measurements are moved from text to diagrams. 
Ablation studies confirm the effectiveness of our domain adaptation strategy, showing improvements in optical character recognition (OCR)-based tasks and robust diagram embeddings across different styles. 
% By unifying the solution program languages of existing datasets and incorporating OCR capability, we enable a single VLM, named \geovlm{}, to handle a broad class of plane geometry problems.

% Contributions
We summarize the contributions as follows:
We propose a novel benchmark for systematically assessing how well vision encoders recognize geometric premises in plane geometry diagrams~(\cref{sec:visual_feature}); We introduce \geoclip{}, a vision encoder capable of accurately detecting visual geometric premises~(\cref{sec:geoclip}), and a few-shot domain adaptation technique that efficiently transfers this capability across different diagram styles (\cref{sec:domain_adaptation});
We show that our VLM, incorporating domain-adapted GeoCLIP, surpasses existing specialized PGPS VLMs and generalist VLMs on the MathVerse benchmark~(\cref{sec:experiments}) and effectively interprets diverse diagram styles~(\cref{sec:abl}).

\iffalse
\begin{itemize}
    \item We propose a novel benchmark for systematically assessing how well vision encoders recognize geometric premises, e.g., perpendicularity and angle measures, in plane geometry diagrams.
	\item We introduce \geoclip{}, a vision encoder capable of accurately detecting visual geometric premises, and a few-shot domain adaptation technique that efficiently transfers this capability across different diagram styles.
	\item We show that our final VLM, incorporating GeoCLIP-DA, effectively interprets diverse diagram styles and achieves state-of-the-art performance on the MathVerse benchmark, surpassing existing specialized PGPS models and generalist VLM models.
\end{itemize}
\fi

\iffalse

Large language models (LLMs) have made significant strides in automated math word problem solving. In particular, their code-generation capabilities combined with proof assistants~\citep{lean,isabelle} help minimize computational errors~\citep{POT}, improve solution precision~\citep{autoformalization}, and offer rigorous feedback and evaluation~\citep{MATH}. Although LLMs excel in generating solution steps and correct answers for algebra and calculus~\citep{math_solving}, their uni-modal nature limits performance in domains like plane geometry, where both diagrams and text are vital.

Plane geometry problem solving (PGPS) tasks typically include diagrams and textual descriptions, requiring solvers to interpret premises from both sources. To facilitate automated solutions for these problems, several studies have introduced formal languages tailored for plane geometry to represent solution steps as a program with training datasets composed of diagrams, textual descriptions, and solution programs~\citep{geoqa,unigeo,intergps,pgps}. Building on these datasets, a number of PGPS specialized vision-language models (VLMs) have been developed so far~\citep{GOLD, LANS, geox}.

Most existing VLMs, however, fail to use diagrams when solving geometry problems. Well-known PGPS datasets such as GeoQA~\citep{geoqa}, UniGeo~\citep{unigeo}, and PGPS9K~\citep{pgps}, can be solved without accessing diagrams, as their problem descriptions often contain all geometric information. \cref{fig:pgps_examples} shows an example from GeoQA and PGPS9K datasets, where one can deduce the solution steps without knowing the diagrams. 
As a result, models trained on these datasets rely almost exclusively on textual information, leaving the vision encoder under-utilized~\citep{GOLD}. 
Consequently, the VLMs trained on these datasets cannot solve the plane geometry problem when necessary geometric properties or relations are excluded from the problem statement.

Some studies seek to enhance the recognition of geometric premises from a diagram by directly predicting the premises from the diagram~\citep{GOLD, intergps} or as an auxiliary task for vision encoders~\citep{geoqa,geoqa-plus}. However, these approaches remain highly domain-specific because the labels for training are difficult to obtain, thus limiting generalization across different domains. While self-supervised learning (SSL) methods that depend exclusively on geometric diagrams, e.g., vector quantized variational auto-encoder (VQ-VAE)~\citep{unimath} and masked auto-encoder (MAE)~\citep{scagps,geox}, have also been explored, the effectiveness of the SSL approaches on recognizing geometric features has not been thoroughly investigated.

We introduce a benchmark constructed with a synthetic data engine to evaluate the effectiveness of SSL approaches in recognizing geometric premises from diagrams. Our empirical results with the proposed benchmark show that the vision encoders trained with SSL methods fail to capture visual \geofeat{}s such as perpendicularity between two lines and angle measure.
Furthermore, we find that the pre-trained vision encoders often used in general-purpose VLMs, e.g., OpenCLIP~\citep{clip} and DinoV2~\citep{dinov2}, fail to recognize geometric premises from diagrams.

To improve the vision encoder for PGPS, we propose \geoclip{}, a model trained with a massive amount of diagram-caption pairs.
Since the amount of diagram-caption pairs in existing benchmarks is often limited, we develop a plane diagram generator that can randomly sample plane geometry problems with the help of existing proof assistant~\citep{alphageometry}.
To make \geoclip{} robust against different styles, we vary the visual properties of diagrams, such as color, font size, resolution, and line width.
We show that \geoclip{} performs better than the other SSL approaches and commonly used vision encoders on the newly proposed benchmark.

Another major challenge in PGPS is developing a domain-agnostic VLM capable of handling multiple PGPS benchmarks. As shown in \cref{fig:pgps_examples}, the main difficulties arise from variations in diagram styles. 
To address the issue, we propose a few-shot domain adaptation technique for \geoclip{} which transfers its visual \geofeat{} perception from the synthetic diagrams to the real-world diagrams efficiently. 

We study the efficacy of the domain adapted \geoclip{} on PGPS when equipped with the language model. To be specific, we compare the VLM with the previous PGPS models on MathVerse~\citep{mathverse}, which is designed to evaluate both the PGPS and visual \geofeat{} perception performance on various domains.
While previous PGPS models are inapplicable to certain types of MathVerse problems, we modify the prediction target and unify the solution program languages of the existing PGPS training data to make our VLM applicable to all types of MathVerse problems.
Results on MathVerse demonstrate that our VLM more effectively integrates diagrammatic information and remains robust under conditions of various diagram styles.

\begin{itemize}
    \item We propose a benchmark to measure the visual \geofeat{} recognition performance of different vision encoders.
    % \item \sh{We introduce geometric CLIP (\geoclip{} and train the VLM equipped with \geoclip{} to predict both solution steps and the numerical measurements of the problem.}
    \item We introduce \geoclip{}, a vision encoder which can accurately recognize visual \geofeat{}s and a few-shot domain adaptation technique which can transfer such ability to different domains efficiently. 
    % \item \sh{We develop our final PGPS model, \geovlm{}, by adapting \geoclip{} to different domains and training with unified languages of solution program data.}
    % We develop a domain-agnostic VLM, namely \geovlm{}, by applying a simple yet effective domain adaptation method to \geoclip{} and training on the refined training data.
    \item We demonstrate our VLM equipped with GeoCLIP-DA effectively interprets diverse diagram styles, achieving superior performance on MathVerse compared to the existing PGPS models.
\end{itemize}

\fi 

\section{Autonomous bicycle control}
\label{sec:2}
This paper addresses the problem of designing a unified control method for balancing an autonomous bicycle while tracking a reference lean angle. First, we use a simple point-mass model to design an inner-loop FL control. Next, DeePO enhances system performance by adapting feedback gains based on input-output data from sensors mounted on the bicycle. 

\subsection{Bicycle Dynamics and Mathematical Modeling}
We consider a simple nonlinear model to represent the bicycle dynamics as in~\cite{Persson_2024}. 
\begin{equation}
    \label{eq:simpleModel}
    \begin{aligned}
        \ddot{\varphi}(t)&=\frac{g}{h}\sin\big(\varphi(t)\big)+ \frac{a}{bh}\cos\big(\varphi(t)\big)v\dot\delta(t) -\\
        & \left(\frac{1}{bh}- \frac{1}{b^2}\tan\big(\delta(t)\big) \tan\big(\varphi(t)\big)\right)\tan\big(\delta(t)\big)v^2,
    \end{aligned}
\end{equation}
where $\varphi(t)$, $\dot{\varphi}(t)$, $\delta(t)$, and $\dot{\delta}(t)$ represent the lean angle, lean rate, steering angle, and the controlled steering rate, respectively. The contact point between the rear wheel and the ground is denoted by $p_1$. Additionally, the vertical and horizontal distances between the bicycle's center of gravity and $p_1$ are denoted by $a$ and $h$, respectively. The wheelbase is denoted by $b$, while $g$ represents the gravitational constant, and $v$ represents the forward velocity. This model assumes a vertical steering axis, i.e., $\nu = \frac{\pi}{2}$, which results in zero trail. Furthermore, it is assumed that the steering axis can be controlled without delay and that the bicycle travels at a constant forward velocity. The visual representation of the parameters in \eqref{eq:simpleModel} is shown in Fig.~\ref{fig:BikeModel}.

\begin{figure}[t]
    \centering
    \includegraphics[width=0.95\columnwidth]{BikeModel.pdf}
    \caption{Illustration of the parameters used in the bicycle model in \eqref{eq:simpleModel}.}
    \label{fig:BikeModel}
\end{figure}

A bicycle is self-stabilized between the so-called \textit{weave speed} and \textit{capsize speed}. By analyzing the eigenvalues of the Whipple model and identifying the region where they are all negative, the self-stable region of a bicycle can be localized~\cite{kooijman2008}. A similar eigenvalue analysis for the instrumented bicycle considered in this paper was previously conducted, where the $25$ parameters required for the Whipple model were measured~\cite{persson2023control}. Based on this analysis, we focus on forward speeds of approximately $8$ km/h ($2.22$ m/s), below the weave speed, as shown in Fig.~\ref{fig:velAnalysis}. Thus, the system we aim to control is open-loop unstable, nonlinear, and non-holonomic, presenting a challenging control problem. Due to the system's inherent instability, applying a persistently exciting input without additional stabilization can lead to a loss of balance and cause the system to diverge. Specifically, an uncontrolled persistently exciting input could destabilize steering actions, such as turning left while leaning right, making it impractical to rely solely on such input for collecting persistently exciting data. In the following, we present an FL controller that balances the bicycle, simplifying the acquisition of persistently exciting data and mitigating some of the system's nonlinearities.

\begin{figure}[t]
    \centering
    \includegraphics{velAnalysis.pdf}
    \caption{Stable and unstable regions of the instrumented bicycle, with the weave speed and capsize speed denoted by $v_w$ and $v_c$, respectively.}
    \label{fig:velAnalysis}
\end{figure}

\subsection{Control overview}
A common approach for realizing a persistently exciting input is to utilize a random signal~\cite{alsalti2023design}. However, this approach is not directly applicable as it jeopardizes the bicycle's balance. Instead, we pre-stabilize the bicycle with an inner control loop using FL. Since the relative degree between the output and the number of states in the model given by \eqref{eq:simpleModel} does not match, the system can only be partially linearized using output FL~\cite{slotine1991applied}. 

If we choose $x = \begin{bmatrix}
    x_1, & x_2, & x_3
\end{bmatrix} = \begin{bmatrix}
    \varphi(t), & \dot{\varphi}(t), & \delta(t)
\end{bmatrix}, $ $y(t) = \varphi(t)$, $\dot{y}(t) = \dot{\varphi}(t)$, and represent the reference output as $y_r = [y_r(t), \dot{y}_r(t), \ddot{y}_r(t)]$, we can express the considered FL control law as:

\begin{equation}
    u(t) = \dot{\delta}(t) = \frac{1}{p(x)}(w - f(x)),
    \label{eq:FLinput}
\end{equation}

where 
\begin{align}
\label{eq:control}
    f(x)&= - \left(\frac{1}{bh}- \frac{1}{b^2}\tan\big(x_3\big) \tan\big(x_1\big)\right)\tan\big(x_3\big)v^2 \nonumber \\
      &\phantom{=} + \frac{g}{h}\sin\big(x_1\big) \nonumber \\
    p(x) & =\frac{a}{bh}\cos\big(x_1\big)v, \nonumber \\
    w & =  \ddot y_r(t) + k_1\left(\dot y_r(t)-\dot y(t)\right) +k_2 \left(y_r(t)-y(t)\right),\\ \nonumber
\end{align}
with appropriate choices of $k_1 > 0$ and $k_2 > 0$ to partially compensate for the system's nonlinearities. However, the steering angle $\delta(t)$ remains an internal state that is not directly linearized, meaning some nonlinear dynamics persist. In particular, terms involving $\tan(\delta(t))$ introduce coupling effects that remain even after feedback linearization. Additionally, since the steering angle evolves according to $u = \dot{\delta}$, it can drift over time uncontrolled, requiring further regulation to prevent undesired effects on system stability.
Furthermore, the proposed FL controller is designed based on continuous-time dynamics, with the model and control parameters provided in \eqref{eq:simpleModel} and Table~\ref{tab:modParam}, respectively. However, in practice, we implement it using a sampled-data approach with a hold mechanism, which may introduce inaccuracies and lead to performance degradation due to the discrete nature of the implementation. This discrete approach may not fully capture the continuous dynamics of the system~\cite{kimber1991sampled, grizzle1988feedback}. Moreover, parameters in~\eqref{eq:FLinput} and~\eqref{eq:control} are subject to parametric uncertainty, resulting in an inaccurate canceling of nonlinearties.  Nevertheless, we demonstrate that the potential limitations of the FL controller can be mitigated by incorporating DeePO.

\begin{table}[b]
\centering
\caption{Model and control parameters for FL control}
\label{tab:modParam}
\begin{tabular}{@{}llll@{}}
\toprule
\textbf{Parameter}  & \textbf{Symbol} & \textbf{Value} & \textbf{Unit} \\ \midrule
CoG w.r.t $p_1$ (x) & $a$             & $0.550$        & m             \\
CoG w.r.t $p_1$ (z) & $h$             & $0.700$        & m             \\
Wheelbase           & $b$             & $1.200$        & m             \\
Gravity             & $g$             & $9.82$         & m/s$^2$       \\
$k_1$               & -               & 1              & -             \\
$k_2$               & -               & 6              & -             \\ \bottomrule
\end{tabular}
\end{table}

The proposed FL controller functions as an inner control loop to stabilize the bicycle, enabling the use of an additive random signal as either a persistently exciting input or a performance enhancing adaptive control. In the remainder of the paper, we consider the autonomous bicycle with FL as our target system to control by DeePO, as highlighted by the gray box in Fig.~\ref{fig:controlOverview}. With this stable inner-loop system in place, we shift our focus to enhancing performance and compensating for the remaining nonlinearities using an adaptive, direct data-driven control approach in the outer loop.
 

\begin{figure}[t]
    \centering
    \includegraphics{controlOverview.pdf}
    \caption{Control overview}
    \label{fig:controlOverview}
\end{figure}







\section{Data-enabled policy optimization for autonomous bicycle  control}
\label{sec:3}
This section first describes a brief overview of the linear quadratic regulator (LQR). Then, we propose a data-enabled policy optimization with a forgetting factor for adaptive learning of the LQR based on~\cite{zhao2023data, zhao2024data}. 

\subsection{The linear quadratic regulator}
	Consider a linear time-invariant system
	\begin{equation}\label{equ:sys}
	\left\{\begin{aligned}
	x\dt{t+1} & =A x\dt{t}+B u\dt{t}+w\dt{t} \\
	h\dt{t} & =\begin{bmatrix}
	Q^{1 / 2} & 0 \\
	0 & R^{1 / 2}
	\end{bmatrix}
	\begin{bmatrix}
	x\dt{t}  \\
	u\dt{t} 
	\end{bmatrix}
	\end{aligned}\right. .
	\end{equation}
	Here, $x\dt{t}$ is the state, $u\dt{t}$ is the control input, $w\dt{t}$ represents the noise, and $h\dt{t}$ is the performance signal of interest, $(A,B)$ are controllable, and the weighting matrices $(Q, R)$ are positive definite. 
	
    The LQR problem aims to find an optimal state-feedback gain $K\in \mathbb{R}^{m\times n}$ that minimizes the $\mathcal{H}_2$-norm of the transfer function $\mathscr{T}(K):w \rightarrow h$ of the closed-loop system
	\begin{equation}
	\begin{bmatrix}
	x\dt{t+1}  \\
	h\dt{t} 
	\end{bmatrix}=\begin{bmatrix}
	A+BK & I_n \\
	\hline \begin{bmatrix}
	Q^{1 / 2} \\
	R^{1 / 2} K
	\end{bmatrix} & 0
	\end{bmatrix}\begin{bmatrix}
	x\dt{t}  \\
	w\dt{t} 
	\end{bmatrix}.
	\end{equation}
	When $A+BK$ is stable, it holds that \cite{anderson2007optimal}
	\begin{equation}\label{equ:transfer}
	\|\mathscr{T}(K)\|_2^2  = \text{Tr}((Q+K^{\top}RK)\Sigma_K)=:C(K),
	\end{equation}
	where $\Sigma_K$ is the closed-loop state covariance matrix obtained as the positive definite solution to the Lyapunov equation
	\begin{equation}\label{equ:Sigma}
	\Sigma_K = I_n + (A+BK)\Sigma_K (A+BK)^{\top}.
	\end{equation}
	We refer to $C(K)$ as the LQR cost and to (\ref{equ:transfer})-(\ref{equ:Sigma}) as a \textit{policy parameterization} of the LQR.

The optimal LQR gain $K^*$ is unique and can be found by, e.g., solving an algebraic Riccati equation with $(A,B)$~\cite{anderson2007optimal}. When $(A,B)$ is unknown, data-driven methods can be used to learn the LQR gain from input-state data. In the sequel, we propose a covariance parameterization method for direct data-driven learning of the LQR.

\subsection{Data-driven covariance parametrization of the LQR with exponential weighted data}\label{sec:3-B}
Consider the $t$-long time series of states, inputs, noises, and successor states
\begin{align}
X_{0,t} &:= \begin{bmatrix}
x\dt{0} & x\dt{1} & \dots& x\dt{t-1} 
\end{bmatrix}\in \mathbb{R}^{n\times t},\nonumber\\
U_{0,t} &:= \begin{bmatrix}
u\dt{0} & u\dt{1} & \dots& u\dt{t-1} 
\end{bmatrix}\in \mathbb{R}^{m\times t}, \label{equ:dataMat}\\
W_{0,t} &:= \begin{bmatrix}
w\dt{0} & w\dt{1} & \dots& w\dt{t-1} 
\end{bmatrix}\in \mathbb{R}^{n\times t}, \nonumber\\
X_{1,t} &:= \begin{bmatrix}
x\dt{1} & x\dt{2} & \dots& x\dt{t} 
\end{bmatrix}\in \mathbb{R}^{n\times t}, \nonumber
\end{align}
which satisfy the system dynamics
\begin{equation}\label{equ:dynamics}
X_{1,t} = AX_{0,t}+ BU_{0,t} + W_{0,t}.
\end{equation}


Assume that the data is {\em persistently exciting (PE)} \cite{willems2005note}, i.e., the block matrix of input and state data
\begin{equation}
D_{0,t} := 
\begin{bmatrix}
    U_{0,t} \\
    X_{0,t}
\end{bmatrix}
\end{equation}
has full row rank
\begin{equation}\label{equ:rank}
\text{rank}(D_{0,t}) = m+n.
\end{equation}
Define the covariance of exponentially weighted data as
\begin{equation}
    \Phi_{t} := \frac{1}{t}D_{0,t} S_{\lambda} D_{0,t}^{\top},
\end{equation}
where $\lambda \in (0,1)$ is a forgetting factor and $S_{\lambda} := \text{diag}\{ \lambda^{t-1}, \lambda^{t-2},\dots, 1\}$. Compared with \cite{zhao2024data}, the forgetting factor here makes the weight of past data decay exponentially, such that the sample covariance can also reflect and adapt to the behavior of time-varying or nonlinear systems. 


Since $D_{0,t}$ has full row rank and $S_{\lambda}\succ 0$, the covariance matrix is positive definite, i.e., $\Phi_{t} \succ 0$. Then, for any gain $K$, there exist a matrix $V$ such that
\begin{equation}\label{equ:forget}
\begin{bmatrix}
K \\
I_n
\end{bmatrix}=  \Phi_{t} V.
\end{equation}
We refer to \eqref{equ:forget} as the \textit{covariance parameterization} with exponentially weighted data and to $V\in \mathbb{R}^{(n+m)\times n}$ as the \textit{parameterized policy}.

With \eqref{equ:forget}, the LQR problem \eqref{equ:transfer}-\eqref{equ:Sigma} can be expressed by raw data matrices $(X_{0,t}, U_{0,t}, X_{1,t})$ and the optimization matrix $V$. For brevity, let $\overline{X}_{0,t}= X_{0,t}S_{\lambda}D_{0,t}^{\top}/t$ and $\overline{U}_{0,t}=  U_{0,t}S_{\lambda}D_{0,t}^{\top}/t$ be a partition of $\Phi_t$, and let
$\overline{W}_{0,t}=  W_{0,t}S_{\lambda}D_{0,t}^{\top}/t$ be the noise-state-input covariance, and finally define the covariance with respect to the successor state as $\overline{X}_{1,t}=  X_{1,t}S_{\lambda}D_{0,t}^{\top}/t$.
Then, the closed-loop matrix can be written as
\begin{equation}
A+BK=[B,A]\begin{bmatrix}
K \\
I_n
\end{bmatrix}\overset{\eqref{equ:forget}}{=}[B,A]\Phi_t V\overset{\eqref{equ:dynamics}}{=}(\overline{X}_{1,t} - \overline{W}_{0,t})V.
\end{equation}
Following the certainty-equivalence principle~\cite{dorfler2021certainty}, we disregard the
unmeasurable $\overline{W}_{0,t}$ for the design and use $\overline{X}_{1,t}V$ as the closed-loop matrix. After substituting $A+BK$  with $\overline{X}_{1,t}V$ in (\ref{equ:transfer})-(\ref{equ:Sigma}) and leveraging \eqref{equ:forget}, the LQR problem becomes 
\begin{equation}\label{prob:equiV}
\begin{aligned}
&\mathop{\text {minimize}}\limits_{V}~J_t(V) :=\text{Tr}\left((Q+V^{\top}\overline{U}_{0,t}^{\top}R\overline{U}_{0,t}V)\Sigma_t(V)\right),\\
&\text{subject to}~ ~\overline{X}_{0,t}V= I_n,
\end{aligned}
\end{equation}
where $\Sigma_t(V) = I_n + \overline{X}_{1,t}V\Sigma_t(V) V^{\top}\overline{X}_{1,t}^{\top}$ is a covariance parameterization of \eqref{equ:Sigma},
and the original gain matrix can be recovered as $K = \overline{U}_{0,t}V$. We refer to (\ref{prob:equiV}) as the covariance-parameterized LQR problem, which is direct data-driven and does not involve any explicit SysID.



\subsection{Data-enabled policy optimization for adaptive LQR control with exponentially weighted data}
In previous work \cite{zhao2023data,zhao2024data}, a data-enabled policy optimization (DeePO) method for direct adaptive learning of the LQR was proposed, where the control policy is parameterized by sample covariance and updated recursively using gradient methods. In this subsection, we propose a DeePO algorithm based on our covariance parameterization with exponentially weighted data \eqref{equ:forget}, detailed in Algorithm \ref{alg:deepo}.

Algorithm \ref{alg:deepo} alternates between control (line 2) and policy update (lines 3-6).
The DeePO algorithm uses online gradient descent of \eqref{prob:equiV} to recursively update $V$. 
At time $t$, we apply the linear state feedback policy $u\dt{t}=K_tx\dt{t}  + e\dt{t} $ for control and observe the new state $x\dt{t+1} $, where $e\dt{t}$ is a probing noise used to ensure the PE rank condition \eqref{equ:rank}. To update the policy, we first use $(X_{0,t+1}, U_{0,t+1}, X_{1,t+1})$ to formulate the covariance-parameterized LQR problem \eqref{prob:equiV}. Then, instead of solving this optimization problem optimality, we only take a single step of projected gradient descent towards its solution in \eqref{equ:pro_gd}. Here, the projection  
\begin{equation}
\Pi_{\overline{X}_{0,t+1}}: = I_{n+m}-\overline{X}_{0,t+1}^{\dagger}\overline{X}_{0,t+1}
\end{equation}
onto the nullspace of $\overline{X}_{0,t+1}$ is to ensure the subspace constraint in \eqref{prob:equiV}.
 Define the feasible set of \eqref{prob:equiV} (i.e., the set of stable closed-loop matrices) as $\mathcal{S}_t:= \{V\mid \overline{X}_{0,t}V =I_n,  \rho (\overline{X}_{1,t}V)<1\}$. Then, the gradient can be computed as follows.
\begin{lemma}[\cite{zhao2024data}]\label{lem:gradient}
For $V\in \mathcal{S}_t$, the gradient of $J_t(V)$ with respect to $V$ is given by
	\begin{equation}\label{equ:pg}
	\nabla J_t(V) = 2 \left(\overline{U}_{0,t}^{\top}R\overline{U}_{0,t}+\overline{X}_{1,t}^{\top}P_t\overline{X}_{1,t}\right)V \Sigma_t(V),
	\end{equation}
	where $P_t$ satisfies the Lyapunov equation 
	\begin{equation}
	P_t = Q + V^{\top}\overline{U}_{0,t}^{\top}R\overline{U}_{0,t}V + V^{\top}\overline{X}_{1,t}^{\top}P_t\overline{X}_{1,t}V.
	\end{equation}
\end{lemma}
 


Algorithm \ref{alg:deepo} is \textit{direct and adaptive} in the sense that it directly uses online closed-loop data to update the policy. Thanks to the forgetting factor, it can rapidly adapt to changes in system behavior reflected in the data. 
As in \cite{zhao2024data}, Algorithm \ref{alg:deepo} can also be implemented recursively. We write the sample covariance  recursively as
\begin{equation}\label{equ:recur}
    \Phi_{t+1} = \frac{\lambda t}{t+1} \Phi_{t} + \frac{1}{t+1} \phi_t\phi_t^{\top},
\end{equation}
where $\phi_t = [u_t^\top, x_t^\top]^\top$. By the Sherman-Morrison formula~\cite{sherman1950adjustment}, its inverse $\Phi_{t+1}^{-1}$ satisfies
\begin{equation}
    \Phi_{t+1}^{-1} = \frac{t+1}{\lambda t}\left(\Phi_{t}^{-1} - \frac{\Phi_{t}^{-1}\phi_t\phi_t^{\top}\Phi_{t}^{-1}}{\lambda t+\phi_t^{\top}\Phi_{t}^{-1}\phi_t}\right).
\end{equation}
Furthermore, the rank-one update of the parameterized policy is given by
\begin{align}
V_{t+1}&= \frac{t+1}{t}\left(\Phi_{t}^{-1} - \frac{\Phi_{t}^{-1}\phi_t\phi_t^{\top}\Phi_{t}^{-1}}{t+\phi_t^{\top}\Phi_{t}^{-1}\phi_t}\right) \Phi_{t} V_{t}' \nonumber \\ 
&= \frac{t+1}{\lambda t} \left(V_{t}' - \frac{\Phi_{t}^{-1}\phi_t\phi_t^{\top}V_{t}'}{\lambda t+\phi_t^{\top}\Phi_{t}^{-1}\phi_t}\right),
\end{align}
where $\Phi_{t}^{-1}$ and $V_{t}'$ are given from the last iteration.

\begin{remark}
Using the forgetting factor $\lambda$ may asymptotically lead to failure of the rank-one update as time tends to infinity. To see this, we notice that the covariance update in
\eqref{equ:recur} satisfies stable linear dynamics. This implies a loss of persistency of excitation and $\Phi_t$ will tend to zero, and hence $\Phi^{-1}_t$ will grow to infinity. A simple remedy is to reset the covariance $\Lambda_{t}$  occasionally, i.e., set $\Lambda_{t}=I_{n+m}, \forall t\in \{T,2T,\dots\}$. Since the autonomous bicycle operates only for a finite time, we do not reset the covariance in our subsequent experiments in Section \ref{sec:4}. Another approach is to use sliding window data rather than exponentially weighted data for the covariance parameterization \eqref{equ:forget}, where a key is to select an optimal window size to balance data informativity and adaptation efficiency. We leave this exploration to future work.
\qed
\end{remark}

\begin{remark}
    The stepsize $\eta_t$ should be set according to the signal-to-noise ratio (SNR) of online data. For example, when the SNR is large, we are confident with the gradient direction, and the stepsize can be chosen more aggressively; on the contrary, when the SNR is small, the stepsize should be small to prevent the policy from moving out of the stability region. To this end, we set the stepsize as 
    \begin{equation}    
    \eta_t = \frac{\eta_0}{\left\|\overline{U}_{0,t}\Pi_{\overline{X}_{0,t}}\overline{U}_{0,t}^{\top}\right\|}, ~t\geq t_0,
    \end{equation}
    where $\eta_0$ is a constant, and the denominator is used to quantify the SNR. Another motivation of the denominator is from \cite[Lemma 3]{kang2024linear}, which reveals the equivalence between data-enabled and model-based policy gradients up to the data matrix $\overline{U}_{0,t}\Pi_{\overline{X}_{0,t}}\overline{U}_{0,t}^{\top}$. \qed
\end{remark}
 

\begin{algorithm}[t]
	\caption{DeePO for direct adaptive LQR control}
	\label{alg:deepo}
	\begin{algorithmic}[1]
		\Require Offline data $(X_{0,t_0}, U_{0,t_0}, X_{1,t_0})$, an initial policy $K_{t_0}$, and a stepsize $\eta$.
		\For{$t=t_0,t_0+1,\dots$}
		\State Apply $u\dt{t} =K_tx\dt{t}  + e\dt{t} $ and observe $x\dt{t+1} $.
        \State Update covariance matrices $\Phi_{t+1}$ and $\overline{X}_{1,t+1}$.
		\State \textbf{Policy parameterization:} given $K_{t}$, solve $V_{t+1}$ via 
		$$
		V_{t+1} =\Phi_{t+1}^{-1} \begin{bmatrix}
		K_{t} \\
		I_n
		\end{bmatrix}.
		$$
		\State \textbf{Update of the parameterized policy:} perform one-step projected gradient descent
		\begin{equation}\label{equ:pro_gd}
		V_{t+1}' = V_{t+1} - \eta_t  \Pi_{\overline{X}_{0,t+1}} \nabla J_{t+1}(V_{t+1}),
		\end{equation} 
		where the gradient $\nabla J_{t+1}(V_{t+1})$ is given by Lemma \ref{lem:gradient}.
		\State \textbf{Gain update:} update the control gain by 
		$$
		K_{t+1} = \overline{U}_{0,t+1}V_{t+1}'.
		$$
		\EndFor	
	\end{algorithmic}
\end{algorithm}

Algorithm \ref{alg:deepo} requires the initial policy to be stabilizing. A potential approach is to solve the  covariance-parameterized LQR \eqref{prob:equiV} with the offline data $(X_{0,t_0}, U_{0,t_0}, X_{1,t_0})$. However, due to the nonlinearity in the system dynamics of the autonomous bicycle, the solution of covariance-parameterized LQR may be destabilizing. Next, we propose a robustness-promoting regularizer for the covariance parameterization to obtain a stabilizing initial policy.
 
\subsection{Learning an initial stabilizing policy using robustness promoting regularization}


	The feasibility of the covariance-parameterized LQR problem \eqref{prob:equiV} depends on that of the Lyapunov equation
	\begin{equation}\label{equ:lyap}
		\Sigma = I_n + \overline{X}_1V\Sigma V^{\top}\overline{X}_1^{\top},
	\end{equation}
	where $\overline{X}_1V$ is regarded as the closed-loop matrix. However, having assumed certainty-equivalence by the covariance parameterization \eqref{equ:forget} and the relation $A+BK = (\overline{X}_1 - \overline{W}_0)V$, the Lyapunov equation that should be met is
	\begin{equation}\label{equ:true_lyap}
		\Sigma = I_n + (\overline{X}_1 - \overline{W}_0)V\Sigma V^{\top}(\overline{X}_1 - \overline{W}_0)^{\top}.
	\end{equation}
	The gap between the right-hand side of \eqref{equ:lyap} and \eqref{equ:true_lyap} is
	\begin{equation}\label{equ:diff}
		\begin{aligned}
		&\overline{W}_0 V\Sigma V^{\top}  \overline{W}_0^{\top} - \overline{W}_0 V\Sigma V^{\top} \overline{X}_1^{\top} -   \overline{X}_1V\Sigma V^{\top} \overline{W}_0^{\top} \\
		&= \frac{1}{t^2}  W_0D_0^{\top}V\Sigma V^{\top}D_0W_0^{\top} \\
		&-\frac{1}{t^2}(  W_0D_0^{\top}V\Sigma V^{\top}D_0X_1^{\top} +   X_1D_0^{\top}V\Sigma V^{\top}D_0W_0^{\top}).
		\end{aligned}
	\end{equation}
	To reduce the gap, it suffices to make $\text{Tr}(D_0^{\top}V\Sigma V^{\top}D_0/t)$ small. To this end, we introduce the regularizer $\text{Tr}(V\Sigma V^{\top}\Phi)$ to the covariance-parameterized LQR problem \eqref{prob:equiV}, leading to
	\begin{equation}\label{prob:regu}
	\begin{aligned}
	&\mathop{\text {minimize}}\limits_{V, \Sigma\succeq 0}~ J_t(V) + \gamma\text{Tr}(V\Sigma V^{\top}\Phi),\\
	&\text{subject to}~ ~\Sigma = I_n + \overline{X}_1V\Sigma V^{\top}\overline{X}_1^{\top},\overline{X}_0V= I_n
	\end{aligned}
	\end{equation}
	with gain matrix $K = \overline{U}_0V$, where $\gamma>0$ is the regularization coefficient. We refer to \eqref{prob:regu} as the regularized covariance parameterization of the LQR problem.

    To obtain an initial stabilizing policy for Algorithm \ref{alg:deepo}, we solve \eqref{prob:regu} with offline data $(X_{0,t_0}, U_{0,t_0}, X_{1,t_0})$.

\subsection{Control gain update rate}
Rapid changes in an adaptive control policy, $K_t$ can potentially induce oscillations and, in the worst case, render the system unstable~\cite{landau2011adaptive}. Moreover, the control policy at certain time intervals may be significantly influenced by measurement noise, meaning that updates could be driven more by noise than by the actual system dynamics. 

To address these potential issues, we propose updating the DeePO control gain less frequently than the sampling frequency. To regulate the update frequency, we introduce the parameter $\xi$, which determines the intervals at which the controller in line 6 of Algorithm~\ref{alg:deepo} is updated. For instance, if $\xi = 1$, the control gain is updated at every iteration, whereas if $\xi = 100$, the gain is updated every $100$ iterations.









\section{Results}
\label{sec:Results}

In this section, we present various analysis results that demonstrate the adoption of code obfuscation in Google Play.

\subsection{Overall Obfuscation Trends} 
\label{sec:obstrend}

\subsubsection{Presence of obfuscation} Out of the 548,967 Google Play Store APKs analyzed, we identified 308,782 obfuscated apps, representing approximately 56.25\% of the total. In Figure~\ref{fig:obfuscated_percentage}, we show the year-wise percentage of obfuscated apps for 2016-2023. There is an overall obfuscation increase of 13\% between 2016 and 2023, and as can be seen, the percentage of obfuscated apps has been increasing in the last few years, barring 2019 and 2020. As explained in Section~\ref{subsec:dataset}, 2019 and 2020 contain apps that are more likely to be abandoned by developers, and as such, they may not use advanced development practices.

\begin{figure}[h!]
\centering
    \includegraphics[width=\linewidth]{Figures/Only_obfuscation_trendV2.pdf}
    \caption{Percentage of obfuscated apps by year} \vspace{-4mm}
    \label{fig:obfuscated_percentage}
\end{figure}


From 2016 to 2018, the obfuscation levels were relatively stable at around 50-55\%, while from 2021 to 2023, there was a marked rise, reaching approximately 66\% in 2023. This indicates a growing focus on app protection measures among developers, likely driven by heightened security and IP concerns and the availability of advanced obfuscation tools.


\subsubsection{Obfuscation tools} Among the obfuscated APKs, our tool detector identified that 40.92\% of the apps use Proguard, 36.64\% use Allatori, 1.01\% use DashO, and 21.43\% use other (i.e., unknown) tools. We show the yearly trends in Figure~\ref{fig:ofbuscated_tool}. Note that we omit results in 2019 and 2020 ({\bf cf.} Section~\ref{subsec:dataset}).

ProGuard and Allatori are the most consistently used obfuscation tools, with ProGuard showing a slight overall increase in popularity and Allatori demonstrating variability. This inclination could be attributed to ProGuard being the default obfuscator integrated into Android Studio, a widely used development environment for Android applications. Notably, ProGuard usage increased by 13\% from 2018 to 2021, likely due to the introduction of R8 in April 2019~\cite{release_note_android}, which further simplified ProGuard integration with Android apps.

\begin{figure}[h]
\centering
    \includegraphics[width=\linewidth]{Figures/Initial_Tool_Trend_2019_dropV2.pdf} 
    \caption{Yearly obfuscation tool usage}
    \label{fig:ofbuscated_tool}
\end{figure}


DashO consistently remains low in usage, likely due to its high cost. The use of other obfuscation tools decreased until 2018 but has shown a resurgence from 2021 to 2023. This suggests that developers might be using other or custom tools, or our detector might be predicting some apps obfuscated with Proguard or Allatori as `other.' To investigate, we manually checked a sample of apps from the `other' category and confirmed they are indeed obfuscated. However, we could not determine which obfuscation tools the developers used. We discuss this potential limitation further in Section~\ref{sec:limitations}.


\subsubsection{Obfuscation techniques} We show the year-wise breakdown of obfuscation technique usage in Figure~\ref{fig:obfuscated_tech}. Among the various obfuscation techniques, Identifier Renaming emerged as the most prevalent, with 99.62\% of obfuscated apps using it alone or in combination with other methods (Categories of Only IR, IR and CF, IR and SE, or All three). Furthermore, 81.04\% of obfuscated apps used Control Flow Modification, and 62.76\% used String Encryption. The pervasive use of Identifier Renaming (IR) can be attributed to the fact that all obfuscation tools support it ({\bf cf.} Table~\ref{tab:ob_tool_cap}). Similarly, lower adoption of Control Flow Modification and String Encryption can be attributed to Proguard not supporting it. 

\begin{figure}[h]
\centering
    \includegraphics[width=\linewidth]{Figures/Initial_Tech_Trend_2019_dropV2.pdf} 
    \caption{Yearly obfuscation technique usage}
    \label{fig:obfuscated_tech}
\end{figure}



Next, we investigate the adoption of obfuscation on Google Play Store from various perspectives. Same as earlier, due to the smaller dataset size and possible bias ({\bf cf.} Section~\ref{subsec:dataset}), we exclude the APKs from 2019 and 2020 from this analyses.


\subsection{App Genre}
\label{sec:app_genre}

First, we investigate whether the obfuscation practices vary according to the App genre. Initially, we analysed all the APKs together before separating them into two snapshots.


\begin{figure*}[h]
    \centering
    \includegraphics[width=\linewidth]{Figures/AppGenreObfuscationV3.pdf}
    \caption{Obfuscated app percentage by genre (overall)}
    \label{fig:app_genre_overall}
\end{figure*}

Figure~\ref{fig:app_genre_overall} shows the genre-wise obfuscated app percentage. We note that 19 genres have more than 60\% of the apps obfuscated, and almost all the genres have more than 40\% obfuscation percentage. \textit{Casino} genre has the highest obfuscation percentage rate at 80\%, and overall, game genres tend to be more obfuscated than the other genres. The higher obfuscation usage in casino apps is logical due to their nature. These apps often simulate or involve gambling activities and handle monetary transactions and sensitive data related to in-game purchases, making them attractive targets for reverse engineering and hacking. This necessitates robust security measures to prevent fraud and protect user data. 


\begin{figure}[h]
    \centering
    \includegraphics[width=\linewidth]{Figures/AppGenre2018_2023ComparisonV3.pdf}
    \caption{Percentage of obfuscated apps by genre (2018-2023)}
    \label{fig:app_genre_comparison}
\end{figure}



\subsubsection{Genre-wise obfuscation trends in the two snapshots} To investigate the adoption of obfuscation over time, we study the two snapshots of Google Play separately, i.e., APKs from 2016-2018 as one group and APKs from 2021-2023 as another. 

Figure~\ref{fig:app_genre_comparison} illustrates the change in obfuscation levels by app genre between 2016-2018 to 2021-2023. Notably, app categories such as Education, Weather, and Parenting, which had obfuscation levels below the 2018 average, have increased to above the 2023 average by 2023. One possible reason for this in Education and Parenting apps can be the increase in online education activities during and after COVID-19 and the developers identifying the need for app hardening.

There are some genres, such as Casino and Action, for which the percentage of obfuscated apps didn't change across the two snapshots (i.e., purple and orange circles are close together in Figure~\ref{fig:app_genre_comparison}). This is because these genres are highly obfuscated from the beginning. Finally, the four genres, including Simulation and Role Playing, have a lower percentage of obfuscated apps in the 2021-2023 dataset. Our manual analysis didn't result in a conclusion as to why.


\begin{figure}[!h]
    \centering
    \includegraphics[width=\linewidth]{Figures/AppGenreTechAllV5.pdf}
    \caption{Obfuscation technique usage by genre (overall)}
    \label{fig:app_genre_all_tech}
\end{figure}


\subsubsection{Obfuscation techniques in different app genres} In Figure~\ref{fig:app_genre_all_tech}, we show the prevalence of key obfuscation techniques among various genres. As expected, almost all obfuscated apps in all genres used  Identifier Renaming. Also, it can be noted that genres with more obfuscated app percentages tend to use all three obfuscation techniques. Notably, more than 85\% of \textit{Casino} genre apps employ multiple obfuscation techniques

\subsubsection{Obfuscation tool usage in different app genres} We also investigated whether specific obfuscation tools are favoured by developers in different genres. However, apart from the expected observation that  ProGuard and Allatori being the most used tools, we didn't find any other interesting patterns. Therefore, we haven't included those measurement results.

\subsection{App Developers}
Next, we investigate individual developer-wise code obfuscation practices. From the pool of analyzed APKs, we identified the number of apps associated with each developer. Subsequently, we sorted the developers according to the number of apps they had created and selected the top 100 developers with the highest number of APKs for the 2016-2018 and 2021-2023 datasets. For the 2018 snapshot, we had 8,349 apps among the top 100 developers, while for the 2023 snapshot, we had 11,338 apps among the top 100 developers.

We then proceeded to detect whether or not these developers obfuscate their apps and, if so, what kind of tools and techniques they use. We present our results in five levels; developer obfuscating over 80\% of their apps, 60\%--80\% of apps, 40\%--60\% of apps, less than 40\%, and no obfuscation.

Figure~\ref{fig:developer_trend_my_apps_all} compares the two datasets in terms of developer obfuscation adoption. It shows that more developers have moved to obfuscate more than 80\% of their apps in the 2021-2023 dataset (76\%) compared to the 2016-2018 dataset (48\%).

We also found that among developers who obfuscate more than 80\% of their apps, 73\% in 2018 and 93\% in 2023 used the same obfuscation tool. Additionally, these top developers employ Control Flow Modification (CF) and String Encryption (SE) above the average values discussed in Section~\ref{sec:obstrend}. Specifically, in 2018, top developers used CF in 81.3\% of cases and SE in 66.7\%, while in 2023, these figures increased to 88.2\% and 78.9\%. This results in two insights: 1) Most top developers obfuscate all their apps with advanced techniques, possibly due to concerns about IP and security, and 2) Developers stick to a single tool, possibly due to specialized knowledge or because they bought a commercial licence.

\begin{figure}[]
    \centering
    \includegraphics[width=\linewidth]{Figures/Developer_Analysed_Comparison.pdf}
    \caption{Obfuscation usage (Top-100 developers)}
    \label{fig:developer_trend_my_apps_all}
\end{figure}


Finally, we investigate the obfuscation practices of developers with only one app in Table~\ref{tab:my-table}. According to the table, from those developers, 45.5\% of them obfuscated their apps in the 2016-2018 dataset and 57.2\% obfuscated their apps in the 2021-2023 dataset, showing a clear increase. However, these percentages are approximately 10\% lower than the average obfuscation rate in both cohorts discussed in Section~\ref{sec:obstrend}. This indicates that single-app developers may be less aware or concerned about code protection.


\begin{table}[]
\caption{Developers with only one app}
\label{tab:my-table}
\resizebox{\columnwidth}{!}{%
\begin{tabular}{cccccc}
\hline
\textbf{Year} & \textbf{\begin{tabular}[c]{@{}c@{}}Non\\ Obfuscated\end{tabular}} & \multicolumn{4}{c}{\textbf{Obfuscated}} \\ \hline
\multirow{3}{*}{\textbf{\begin{tabular}[c]{@{}c@{}}2018 \\ Snapshot\end{tabular}}} & \multirow{3}{*}{\begin{tabular}[c]{@{}c@{}}26,581 \\ (54.5\%)\end{tabular}} & \multicolumn{4}{c}{\begin{tabular}[c]{@{}c@{}}22,214 (45.5\%)\end{tabular}} \\ \cline{3-6} 
 &  & \textbf{ProGuard} & \textbf{Allatori} & \textbf{DashO} & \textbf{Other} \\ \cline{3-6} 
 &  & 6,131 & 8,050 & 658 & 7,375 \\ \hline
\multirow{3}{*}{\textbf{\begin{tabular}[c]{@{}c@{}}2023 \\ Snapshot\end{tabular}}} & \multirow{3}{*}{\begin{tabular}[c]{@{}c@{}}19,510 \\ (42.8\%)\end{tabular}} & \multicolumn{4}{c}{\begin{tabular}[c]{@{}c@{}}26,084 (57.2\%)\end{tabular}} \\ \cline{3-6} 
 &  & \textbf{ProGuard} & \textbf{Allatori} & \textbf{DashO} & \textbf{Other} \\ \cline{3-6} 
 &  & 12,697 & 9,672 & 234 & 3,581 \\ \hline
\end{tabular}%
}
\end{table}

\subsection{Top-k Apps}

Next, we investigate the obfuscation practices of top apps in Google Play Store. First, we rank the apps using the same criterion used by our previous work~\cite{rajasegaran2019multi, karunanayake2020multi, seneviratne2015early}. That is, we sort the apps in descending order of number of downloads, average rating, and rating count, with the intuition that top apps have high download numbers and high ratings, even when reviewed by a large number of users. Then, we investigated the percentage of obfuscated apps and obfuscation tools and technique usage as summarized in Table~\ref{tab:top_k_apps_2018_2023}.

When considering the highly ranked applications (i.e., top-1,000), the obfuscation percentage is notably higher, at around 93\%, in both datasets, which is significantly higher than the average percentage of obfuscation we observed in Section~\ref{sec:obstrend}. Top-ranked apps, likely due to their higher visibility and potential revenue, invest more in obfuscation to safeguard their intellectual property and enhance security. 

The obfuscation percentage decreases when going from the top 1,000 apps to the top 30,000 apps. Nonetheless, the obfuscation percentage in both datasets remains around similar values until the top 30,000 (e.g., $\sim$74\% for top-30,000). This indicates that the major increase in obfuscation in the 2021-2023 dataset comes from apps beyond the top 30,000.

When observing the tools used, the usage of ProGuard increases as we move from top to lower-ranked apps in both datasets. This may be because ProGuard is free and the default in Android Studio, while commercial tools like Allatori and DashO are expensive. There is a notable increase in the use of Allatori among the top apps in the 2021-2023 dataset. Regarding obfuscation techniques, the top 1,000 apps utilize all three techniques more frequently than lower-ranked apps in both snapshots. This indicates that the top 1,000 apps are more heavily protected compared to lower-ranked ones.

\begin{table*}[]
\caption{Summary of analysis results for Top-k apps in 2018 and 2023}
\label{tab:top_k_apps_2018_2023}
\resizebox{\textwidth}{!}{%
\begin{tabular}{lccccccccc}
\hline
\multicolumn{1}{c}{\begin{tabular}[c]{@{}c@{}}Top k apps - \\ Year\end{tabular}} & \begin{tabular}[c]{@{}c@{}}Total \\ Apps\end{tabular} & \begin{tabular}[c]{@{}c@{}}Obfuscation\\ Percentage\end{tabular} & \begin{tabular}[c]{@{}c@{}}ProGuard\\ Percentage\end{tabular} & \begin{tabular}[c]{@{}c@{}}Allatori\\ Percentage\end{tabular} & \begin{tabular}[c]{@{}c@{}}DashO\\ Percentage\end{tabular} & \begin{tabular}[c]{@{}c@{}}Other\\ Percentage\end{tabular} & \begin{tabular}[c]{@{}c@{}}IR\\ Percentage\end{tabular} & \begin{tabular}[c]{@{}c@{}}CF\\ Percentage\end{tabular} & \begin{tabular}[c]{@{}c@{}}SE\\ Percentage\end{tabular} \\ \hline
1k (2018) & 1,000 & 93.40 & 29.98 & 28.48 & 0.64 & 40.90 & 99.90 & 88.76 & 65.42 \\
10k (2018) & 10,000 & 85.19 & 25.55 & 35.32 & 0.47 & 38.65 & 99.90 & 88.76 & 71.91 \\
20k (2018) & 20,000 & 78.42 & 26.31 & 36.76 & 0.57 & 36.36 & 99.87 & 87.37 & 71.49 \\
30k (2018) & 30,000 & 74.40 & 27.30 & 37.71 & 0.64 & 34.36 & 99.82 & 86.75 & 71.11 \\
30k+ (2018) & 314,568 & 53.36 & 36.72 & 34.70 & 1.33 & 27.24 & 99.34 & 83.54 & 63.11 \\ \hline
1k (2023) & 1,000 & 92.50 & 24.00 & 51.89 & 1.95 & 22.16 & 100.0 & 92.54 & 83.68 \\
10k (2023) & 10,000 & 81.88 & 26.03 & 56.20 & 1.03 & 16.74 & 99.89 & 89.40 & 82.01 \\
20k (2023) & 20,000 & 76.62 & 30.48 & 52.92 & 0.96 & 15.64 & 99.93 & 85.80 & 78.01 \\
30k (2023) & 30,000 & 73.72 & 33.87 & 50.34 & 0.89 & 14.90 & 99.95 & 83.31 & 75.34 \\
30k+ (2023) & 206,216 & 61.90 & 46.56 & 38.21 & 0.64 & 14.59 & 99.97 & 77.51 & 62.50 \\ \hline
\end{tabular}%
}
\end{table*}

We present RiskHarvester, a risk-based tool to compute a security risk score based on the value of the asset and ease of attack on a database. We calculated the value of asset by identifying the sensitive data categories present in a database from the database keywords. We utilized data flow analysis, SQL, and Object Relational Mapper (ORM) parsing to identify the database keywords. To calculate the ease of attack, we utilized passive network analysis to retrieve the database host information. To evaluate RiskHarvester, we curated RiskBench, a benchmark of 1,791 database secret-asset pairs with sensitive data categories and host information manually retrieved from 188 GitHub repositories. RiskHarvester demonstrates precision of (95\%) and recall (90\%) in detecting database keywords for the value of asset and precision of (96\%) and recall (94\%) in detecting valid hosts for ease of attack. Finally, we conducted an online survey to understand whether developers prioritize secret removal based on security risk score. We found that 86\% of the developers prioritized the secrets for removal with descending security risk scores.
\bibliographystyle{IEEEtran}  
\bibliography{refs.bib}
\def\spbio{30}
\vspace{-\spbio pt}
\begin{IEEEbiography}[{\includegraphics[width=1in,height=1.25in,clip,keepaspectratio]{Niklas.jpg}}] {Niklas Persson} received an M.Sc in Robotics from M{\"a}lardalen University, V{\"a}ster{\aa}s, Sweden in 2019. Since 2020, he has been pursuing a PhD degree in electronics at the Intelligent Future Technologies division of M{\"a}lardalen University, working on the control and navigation of autonomous bicycles. In 2023, he received a Licentiate degree at Mälardalen University. His research interests include autonomous robots and vehicles, control theory, and embedded systems. 
\end{IEEEbiography}
\vspace{-\spbio pt}


\begin{IEEEbiography}[{\includegraphics[width=1in,height=1.25in,clip,keepaspectratio]{feiran.jpg}}] {Feiran Zhao} received the B.S. degree in Control Science and Engineering from the Harbin Institute of Technology, China, in 2018, and the Ph.D. degree in Control Science and Engineering from the Tsinghua University, China, in 2024. He is now a postdoc at ETH Z\"{u}rich. His research interests include policy optimization, data-driven control, adaptive control and their applications. 
\end{IEEEbiography}
\vspace{-\spbio pt}


\begin{IEEEbiography}[{\includegraphics[width=1in,height=1.25in,clip,keepaspectratio]{Mojtaba.jpg}}] {Mojtaba Kaheni} (SM'25) is a Postdoctoral Researcher at the {School of Innovation, Design, and Technology (IDT)}, {Mälardalen University}, Västerås, Sweden. He received his {M.Sc.} and {Ph.D.} in Control Engineering from {Shahrood University of Technology}, Shahrood, Iran, in 2011 and 2019, respectively.
Dr. Kaheni has held visiting scholar positions at the {University of Florence}, Italy, and {Lund University}, Sweden. From August 2020 to December 2022, he served as a Postdoctoral Researcher at the {University of Cagliari}, Italy.
His research interests include {control theory}, {distributed optimization}, {multi-agent systems}, and {resiliency}.

\end{IEEEbiography}
\vspace{-\spbio pt}

\begin{IEEEbiography}
[{\includegraphics[width=1in,height=1.25in,clip,keepaspectratio]{doerfler-florian_h_compressed.jpeg}}]	
	{Florian D\"{o}rfler} is a Full Professor at the Automatic Control Laboratory at ETH Z\"{u}rich. He received his Ph.D. degree in Mechanical Engineering from the University of California at Santa Barbara in 2013, and a Diplom degree in Engineering Cybernetics from the University of Stuttgart in 2008. From 2013 to 2014 he was an Assistant Professor at the University of California Los Angeles. He has been serving as the Associate Head of the ETH Z\"{u}rich Department of Information Technology and Electrical Engineering from 2021 until 2022. His research interests are centered around automatic control, system theory, and optimization. His particular foci are on network systems, data-driven settings, and applications to power systems. He is a recipient of the distinguished young research awards by IFAC (Manfred Thoma Medal 2020) and EUCA (European Control Award 2020). His students were winners or finalists for Best Student Paper awards at the European Control Conference (2013, 2019), the American Control Conference (2016, 2024), the Conference on Decision and Control (2020), the PES General Meeting (2020), the PES PowerTech Conference (2017), the International Conference on Intelligent Transportation Systems (2021), and the IEEE CSS Swiss Chapter Young Author Best Journal Paper Award (2022, 2024). He is furthermore a recipient of the 2010 ACC Student Best Paper Award, the 2011 O. Hugo Schuck Best Paper Award, the 2012-2014 Automatica Best Paper Award, the 2016 IEEE Circuits and Systems Guillemin-Cauer Best Paper Award, the 2022 IEEE Transactions on Power Electronics Prize Paper Award, and the 2015 UCSB ME Best PhD award. He is currently serving on the council of the
	European Control Association and as a senior editor of Automatica.
	\end{IEEEbiography}
\vspace{-\spbio pt}
\begin{IEEEbiography}[{\includegraphics[width=1in,height=1.25in,clip,keepaspectratio]{Alessandro.png}}]{Alessandro~V. Papadopoulos} (SM'19)
is a Full Professor of Electrical and Computer Engineering at M{\"a}lardalen University, V{\"a}ster{\aa}s, Sweden, and a QUALIFICA Fellow at the University of M{\'a}laga, Spain. Since March 2024, he has been the scientific leader of Applied AI at M{\"a}lardalen University. He received his B.Sc. and M.Sc. (summa cum laude) degrees in Computer Engineering from the Politecnico di Milano, Milan, Italy, and his Ph.D. (Hons.) degree in Information Technology from the Politecnico di Milano, in 2013. He was a Postdoctoral researcher at the Department of Automatic Control, Lund, Sweden (2014-2016) and Politecnico di Milano, Milan, Italy (2016). 
He was the Program Chair for the Mediterranean Control Conference (MED) 2022, the Euromicro Conference on Real-Time Systems (ECRTS) 2023, and the ACM/SPEC International Conference on Performance Engineering (ICPE) 2025. He is an associate editor for the ACM Transactions on Autonomous and Adaptive Systems, Control Engineering Practice, and Leibniz Transactions on Embedded Systems.
His research interests include robotics, control theory, real-time systems, and autonomic computing. 
\end{IEEEbiography}

\end{document}
\typeout{get arXiv to do 4 passes: Label(s) may have changed. Rerun}
