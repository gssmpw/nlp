\documentclass[lettersize,journal]{IEEEtran}
\usepackage{amsmath,amsfonts}
\usepackage{algorithm}
\usepackage{array}
\usepackage[caption=false,font=normalsize,labelfont=sf,textfont=sf]{subfig}
\usepackage{textcomp}
\usepackage{stfloats}
\usepackage{url}
\usepackage{verbatim}
\usepackage{graphicx}
\usepackage{cite}
\usepackage{tikz}
\usepackage{bm}
\usepackage{booktabs}
\usepackage{hyperref}
\usepackage{adjustbox}
\usepackage{xspace}
\usepackage{svg}
\usepackage{soul}
\usetikzlibrary{positioning,calc,decorations.markings,shapes,arrows,arrows.meta,backgrounds}
\usepackage{pgfplots}
\usepackage{standalone}
\pgfplotsset{compat=newest}
\usepgfplotslibrary{groupplots}
\usetikzlibrary{patterns}
\usetikzlibrary{patterns.meta}
\usetikzlibrary{external}
\tikzexternalize[prefix=figs/tikz_pdf/]


\hyphenation{op-tical net-works semi-conduc-tor IEEE-Xplore}

\usepackage{algpseudocode}
\def \qed {\hfill \vrule height6pt width 6pt depth 0pt}
\usepackage{mathrsfs}
\usepackage{xcolor}
\renewcommand{\algorithmicrequire}{\textbf{Input:}}  % Use Input in the format of Algorithm
\renewcommand{\algorithmicensure}{\textbf{Output:}} % Use Output in the format of Algorithm
\newcommand{\red}{\color{red}}
\newcommand{\green}{\color{green}}
\allowdisplaybreaks[4]
\newtheorem{definition}{Definition}
\newtheorem{theorem}{Theorem}
\newtheorem{lemma}{Lemma}
\newtheorem{assum}{Assumption}
\newtheorem{coro}{Corollary}
\newtheorem{remark}{Remark}
\newtheorem{prop}{Proposition}
\newtheorem{example}{Example}

\DeclareMathOperator{\diagon}{diag}
\newcommand{\diag}[1]{\diagon(#1)}
\newcommand{\etal}{et al.\xspace}
\pgfmathsetmacro{\RadToDeg}{180/3.14159265359}
\newcommand{\dt}[1]{_{#1}}
\newcommand{\dti}[2]{_{#1}^{#2}}


\begin{document}

\title{An Adaptive Data-Enabled Policy Optimization Approach for Autonomous Bicycle Control}

\author{Niklas Persson, \IEEEmembership{Student member, IEEE}, Feiran Zhao, Mojtaba Kaheni, \IEEEmembership{Senior Member, IEEE}, Florian D\"orfler, \IEEEmembership{Senior Member, IEEE}, Alessandro V. Papadopoulos, \IEEEmembership{Senior Member, IEEE}
        % <-this % stops a space
\thanks{This work was supported by the Knowledge Foundation (KKS) with grant ``Mälardalen University Automation Research Center (MARC)'', n. 20240011.}
\thanks{N. Persson, M. Kaheni, and A.V. Papadopoulos are with the Division of Intelligent Future Technologies, M\"alardalen University, 721 23 V\"aster{\aa}s, Sweden.    (e-mails: niklas.persson@mdu.se, mojtaba.kaheni@mdu.se, alessandro.papadopoulos@mdu.se).}% <-this % stops a space
\thanks{F. Zhao and F. D\"orfler are with the Department of Information Technology and Electrical Engineering, ETH Zurich, 8092 Zurich, Switzerland. (e-mail: zhaofe@control.ee.ethz.ch,
dorfler@ethz.ch}% <-this % stops a space
\thanks{Manuscript received February 17, 2025}}


\maketitle

\begin{abstract}
This paper presents a unified control framework that integrates a Feedback Linearization (FL) controller in the inner loop with an adaptive Data-Enabled Policy Optimization (DeePO) controller in the outer loop to balance an autonomous bicycle. While the FL controller stabilizes and partially linearizes the inherently unstable and nonlinear system, its performance is compromised by unmodeled dynamics and time-varying characteristics. To overcome these limitations, the DeePO controller is introduced to enhance adaptability and robustness.
The initial control policy of DeePO is obtained from a finite set of offline, persistently exciting input and state data. To improve stability and compensate for system nonlinearities and disturbances, a robustness-promoting regularizer refines the initial policy, while the adaptive section of the DeePO framework is enhanced with a forgetting factor to improve adaptation to time-varying dynamics.
The proposed DeePO+FL approach is evaluated through simulations and real-world experiments on an instrumented autonomous bicycle. Results demonstrate its superiority over the FL-only approach, achieving more precise tracking of the reference lean angle and lean rate.
\end{abstract}

\begin{IEEEkeywords}
Adaptive control, policy optimization, direct data-driven control, balance control, autonomous bicycle. 
\end{IEEEkeywords}


 
% 
% 
The widespread integration of communication networks and smart devices in modern control systems has increased the vulnerability of industrial systems to online cyber-attacks, e.g., Industroyer, Blackenergy, etc \citep{osti_1505628}.
% Modern control systems have seen a large push to include communication networks and smart devices to increase performance, made possible by improvements in communication device cost and energy consumption. This trend has been coupled with the usage of open-standard communication protocols among industrial control systems, making them vulnerable to online cyber-attacks such as Industroyer, Blackenergy, etc \citep{osti_1505628}. 
To counter this, methods have been developed to improve security by achieving attack detection, mitigation, and monitoring, among others \citep{sandberg2022secure}. This paper focuses on active attack diagnosis to mitigate stealthy attacks. 
%
%\subsection{Literature review}

Active diagnosis techniques rely on the inclusion of additional moduli to control systems
% inclusion within the control system of additional moduli 
to alter the behavior of the system compared to information known by the attacker. 
For instance, the concept of additive watermarking was introduced in \cite{mo2015physical}, where noise signals of known mean and variance are added at the plant and compensated for it at the controller. 
This compensation, however, is not exact, causing some performance degradation. Thus, trade-offs between performance and detectability  are necessary \citep{zhu2023detection}.
% A later work \citep{zhu2023detection} designs the watermark signal by trading performance for detection. Thus, although additive watermarking serves as a good detection scheme, they endure performance losses even in the nominal case. 

In encrypted control \citep{darup2021encrypted}, the sensor data is encrypted, sent to the controller, and then operated on directly. Encrypted input signals are sent back to the plant for decryption. Although encryption is widespread in IT security, in control systems it presents some concerns, such as the introduction of time delays \citep{stabile2024verifiable}, while it may present inherent weaknesses \citep{alisic2023model}.
% they are not preferred as they introduce time delays \citep{stabile2024verifiable} which can cause instability, and some encryption schemes can be very weak  \citep{alisic2023model}. 

In moving target defense \citep{griffioen2020moving}, the plant is augmented with fictitious dynamics, known to the controller. The plant output is transmitted to the controller along with the fictitious states over a network under attack. 
The additional measurements then aide in the detection of attacks. 
This comes at the cost of higher communication bandwidth needs, which increases rapidly with the dimension of the augmented systems.
% Since the dynamics of the fictitious dynamics are exactly known to the controller, the attack is detected easily. However, when the scale of the system increases, the communication bandwidth used by moving the target defense approach increases rapidly. 

Other recently proposed works include two-way coding \citep{fang2019two}, a weak encryuption technique, and dynamic masking \citep{abdalmoaty2023privacy}, which enhances privacy as well as security, have been shown to be effective against zero-dynamics attacks.
% Two-way coding \citep{fang2019two} and dynamic masking \citep{abdalmoaty2023privacy} are other recently proposed approaches. Two-way coding is another form of weak encryption technique whilst dynamic masking proposes an architecture that enhances both privacy and security. These schemes are shown to be effective against zero dynamics attacks but remain to be studied for other classes of attacks. 
% Recent extensions include \citep{mukherjee2021secure,ramos2024privacy}.
% Some other works which are related are \citep{mukherjee2021secure}, an extension of \cite{fang2019two}. The work \citep{ramos2024privacy} is an extension of moving target defense for multi-agent systems. 
Furthermore, filtering techniques for attack detection are proposed by \cite{murguia2020security,hashemi2022codesign,escudero2023safety}, while not focusing on stealthy attacks.
% The works \citep{murguia2020security,hashemi2022codesign,escudero2023safety} develop filtering techniques to guarantee safety, without being focused on stealthy covert attacks.

Multiplicative watermarking (mWM) has been proposed by the authors as a diagnosis technique \citep{ferrari2020switching}. mWM consists of a pair of filters on each communication channel between the plant and its controller; the scheme is affine to weak encryption, whereby ``encoding'' and ``decoding'' are done by changing signals' dynamic characteristics through inverse pairs of filters. This enables original signals to be recovered exactly, and thus does not lead to performance degradation.
% A multiplicative watermark is an affine to a weak encryption technique, through which the signal is ``encoded'' by a filter, changing its dynamic behavior. The use of inverse pairs means that the original signal can be recovered, through ``decoding'' via an inverse filter. As such, differently to techniques based on additive watermarking, no performance is lost due to the injection of noise, and there are no bandwidth limitations.

%\subsection{Contributions}
One of the critical features of multiplicative watermarking is that to detect stealthy attacks, the mWM filter parameters must be switched over time. In this paper, an algorithm to optimally design the mWM parameters after a switching event is presented, enhancing detection performance, without changing the switching time.
% This is done without changing the switching time, which is taken as given.

\textcolor{black}{
To formalize the filter design problem, we suppose the defender is interested in optimal performance against adversaries injecting covert attacks with matched system parameters \citep{smith2015covert}, including the mWM parameters prior to the switch. This scenario represents a worst case where malicious agents can take full control of the system while remaining undetected.
Thus, the attack strategy is explicitly included within the formulation of the closed-loop system, and the mWM filters are chosen by solving an optimization problem minimizing the attack-energy-constrained output-to-output gain (AEC-OOG) \citep{anand2023risk}, a variation of the output-to-output gain proposed in  \cite{teixeira2015strategic}.
}
The main contributions of this paper are:
% We consider an adversary injecting a covert attack with matched system parameters \citep{smith2015covert}, i.e., an attacker with full knowledge of the control system parameters, including those of the mWM filters before the switch. This scenario is taken as a worst case, as it has been shown that this class of attacks can be made stealthy. To quantitatively define a cost, the output-to-output gain (OOG) \citep{teixeira2015strategic} is leveraged,
% a metric introduced to evaluate the impact of an additive attack in a control system. %Specifically, OOG evaluates the worst-case performance loss that an attacker injecting an undetectable attack can obtain. 
% Here, the maximum performance loss caused by a stealthy adversary with limited energy is taken, the attack-energy-constrained OOG (AEC-OOG) \citep{anand2023risk}. The main contributions of this paper are:
\begin{enumerate}
%[label=\alph*.]
\item The problem of optimally designing the switching mWM filters is formulated as an optimization problem, with the AEC-OOG is taken as the objective;%where the AEC-OOG is taken as the impact metric; 
\item The worst-case scenario of a covert attack with exact knowledge of plant and mWM filter parameters is embedded within the design problem;
% The optimization problem is defined to incorporate the worst-case scenario of a covert attack with exact knowledge of plant and mWM filter parameters;
\item The feasibility of the optimization problem is shown to be dependent only on stability conditions; 
\item A solution scheme is proposed to promote randomization of the mWM filter parameters such that an eavesdropping adversary cannot remain stealthy.
\end{enumerate} 

This builds on the results of \cite{ferrari2020switching}, where the focus was on the design of the switching protocols, rather than the parameters themselves.
Compared to previous work \citep{gallo2021design}, this paper introduces an optimization problem which is always feasible (thanks to the use of AEC-OOG in the objective), while also considering a more sophisticated class of covert attacks, where the presence of watermark is known to the adversary. 
Moreover, this paper poses a different objective than \citep{zhang2023hybrid}; indeed, while \citep{zhang2023hybrid} provided a design strategy to ensure certain privacy properties, in this paper we address the problem of optimal parameter design following a switching event.


%\subsection{Organization}
The rest of the paper is organized as follows. 
After formulating the problem in Section~\ref{sec:PF}, we propose our design algorithm in Section~\ref{sec:main}, and analyze its properties. It is then evaluated through a numerical example in Section~\ref{sec:NE}, and concluding remarks are given Section~\ref{sec:Con}.
% We provide the problem background in Section~\ref{sec:PF}. We formulate the design problem in Section~\ref{sec:main}, together with an analysis of its properties. The proposed algorithm is evaluated through a numerical example in Section \ref{sec:NE}. Concluding remarks are offered in Section \ref{sec:Con}.
\section{Autonomous bicycle control}
\label{sec:2}
This paper addresses the problem of designing a unified control method for balancing an autonomous bicycle while tracking a reference lean angle. First, we use a simple point-mass model to design an inner-loop FL control. Next, DeePO enhances system performance by adapting feedback gains based on input-output data from sensors mounted on the bicycle. 

\subsection{Bicycle Dynamics and Mathematical Modeling}
We consider a simple nonlinear model to represent the bicycle dynamics as in~\cite{Persson_2024}. 
\begin{equation}
    \label{eq:simpleModel}
    \begin{aligned}
        \ddot{\varphi}(t)&=\frac{g}{h}\sin\big(\varphi(t)\big)+ \frac{a}{bh}\cos\big(\varphi(t)\big)v\dot\delta(t) -\\
        & \left(\frac{1}{bh}- \frac{1}{b^2}\tan\big(\delta(t)\big) \tan\big(\varphi(t)\big)\right)\tan\big(\delta(t)\big)v^2,
    \end{aligned}
\end{equation}
where $\varphi(t)$, $\dot{\varphi}(t)$, $\delta(t)$, and $\dot{\delta}(t)$ represent the lean angle, lean rate, steering angle, and the controlled steering rate, respectively. The contact point between the rear wheel and the ground is denoted by $p_1$. Additionally, the vertical and horizontal distances between the bicycle's center of gravity and $p_1$ are denoted by $a$ and $h$, respectively. The wheelbase is denoted by $b$, while $g$ represents the gravitational constant, and $v$ represents the forward velocity. This model assumes a vertical steering axis, i.e., $\nu = \frac{\pi}{2}$, which results in zero trail. Furthermore, it is assumed that the steering axis can be controlled without delay and that the bicycle travels at a constant forward velocity. The visual representation of the parameters in \eqref{eq:simpleModel} is shown in Fig.~\ref{fig:BikeModel}.

\begin{figure}[t]
    \centering
    \includegraphics[width=0.95\columnwidth]{BikeModel.pdf}
    \caption{Illustration of the parameters used in the bicycle model in \eqref{eq:simpleModel}.}
    \label{fig:BikeModel}
\end{figure}

A bicycle is self-stabilized between the so-called \textit{weave speed} and \textit{capsize speed}. By analyzing the eigenvalues of the Whipple model and identifying the region where they are all negative, the self-stable region of a bicycle can be localized~\cite{kooijman2008}. A similar eigenvalue analysis for the instrumented bicycle considered in this paper was previously conducted, where the $25$ parameters required for the Whipple model were measured~\cite{persson2023control}. Based on this analysis, we focus on forward speeds of approximately $8$ km/h ($2.22$ m/s), below the weave speed, as shown in Fig.~\ref{fig:velAnalysis}. Thus, the system we aim to control is open-loop unstable, nonlinear, and non-holonomic, presenting a challenging control problem. Due to the system's inherent instability, applying a persistently exciting input without additional stabilization can lead to a loss of balance and cause the system to diverge. Specifically, an uncontrolled persistently exciting input could destabilize steering actions, such as turning left while leaning right, making it impractical to rely solely on such input for collecting persistently exciting data. In the following, we present an FL controller that balances the bicycle, simplifying the acquisition of persistently exciting data and mitigating some of the system's nonlinearities.

\begin{figure}[t]
    \centering
    \includegraphics{velAnalysis.pdf}
    \caption{Stable and unstable regions of the instrumented bicycle, with the weave speed and capsize speed denoted by $v_w$ and $v_c$, respectively.}
    \label{fig:velAnalysis}
\end{figure}

\subsection{Control overview}
A common approach for realizing a persistently exciting input is to utilize a random signal~\cite{alsalti2023design}. However, this approach is not directly applicable as it jeopardizes the bicycle's balance. Instead, we pre-stabilize the bicycle with an inner control loop using FL. Since the relative degree between the output and the number of states in the model given by \eqref{eq:simpleModel} does not match, the system can only be partially linearized using output FL~\cite{slotine1991applied}. 

If we choose $x = \begin{bmatrix}
    x_1, & x_2, & x_3
\end{bmatrix} = \begin{bmatrix}
    \varphi(t), & \dot{\varphi}(t), & \delta(t)
\end{bmatrix}, $ $y(t) = \varphi(t)$, $\dot{y}(t) = \dot{\varphi}(t)$, and represent the reference output as $y_r = [y_r(t), \dot{y}_r(t), \ddot{y}_r(t)]$, we can express the considered FL control law as:

\begin{equation}
    u(t) = \dot{\delta}(t) = \frac{1}{p(x)}(w - f(x)),
    \label{eq:FLinput}
\end{equation}

where 
\begin{align}
\label{eq:control}
    f(x)&= - \left(\frac{1}{bh}- \frac{1}{b^2}\tan\big(x_3\big) \tan\big(x_1\big)\right)\tan\big(x_3\big)v^2 \nonumber \\
      &\phantom{=} + \frac{g}{h}\sin\big(x_1\big) \nonumber \\
    p(x) & =\frac{a}{bh}\cos\big(x_1\big)v, \nonumber \\
    w & =  \ddot y_r(t) + k_1\left(\dot y_r(t)-\dot y(t)\right) +k_2 \left(y_r(t)-y(t)\right),\\ \nonumber
\end{align}
with appropriate choices of $k_1 > 0$ and $k_2 > 0$ to partially compensate for the system's nonlinearities. However, the steering angle $\delta(t)$ remains an internal state that is not directly linearized, meaning some nonlinear dynamics persist. In particular, terms involving $\tan(\delta(t))$ introduce coupling effects that remain even after feedback linearization. Additionally, since the steering angle evolves according to $u = \dot{\delta}$, it can drift over time uncontrolled, requiring further regulation to prevent undesired effects on system stability.
Furthermore, the proposed FL controller is designed based on continuous-time dynamics, with the model and control parameters provided in \eqref{eq:simpleModel} and Table~\ref{tab:modParam}, respectively. However, in practice, we implement it using a sampled-data approach with a hold mechanism, which may introduce inaccuracies and lead to performance degradation due to the discrete nature of the implementation. This discrete approach may not fully capture the continuous dynamics of the system~\cite{kimber1991sampled, grizzle1988feedback}. Moreover, parameters in~\eqref{eq:FLinput} and~\eqref{eq:control} are subject to parametric uncertainty, resulting in an inaccurate canceling of nonlinearties.  Nevertheless, we demonstrate that the potential limitations of the FL controller can be mitigated by incorporating DeePO.

\begin{table}[b]
\centering
\caption{Model and control parameters for FL control}
\label{tab:modParam}
\begin{tabular}{@{}llll@{}}
\toprule
\textbf{Parameter}  & \textbf{Symbol} & \textbf{Value} & \textbf{Unit} \\ \midrule
CoG w.r.t $p_1$ (x) & $a$             & $0.550$        & m             \\
CoG w.r.t $p_1$ (z) & $h$             & $0.700$        & m             \\
Wheelbase           & $b$             & $1.200$        & m             \\
Gravity             & $g$             & $9.82$         & m/s$^2$       \\
$k_1$               & -               & 1              & -             \\
$k_2$               & -               & 6              & -             \\ \bottomrule
\end{tabular}
\end{table}

The proposed FL controller functions as an inner control loop to stabilize the bicycle, enabling the use of an additive random signal as either a persistently exciting input or a performance enhancing adaptive control. In the remainder of the paper, we consider the autonomous bicycle with FL as our target system to control by DeePO, as highlighted by the gray box in Fig.~\ref{fig:controlOverview}. With this stable inner-loop system in place, we shift our focus to enhancing performance and compensating for the remaining nonlinearities using an adaptive, direct data-driven control approach in the outer loop.
 

\begin{figure}[t]
    \centering
    \includegraphics{controlOverview.pdf}
    \caption{Control overview}
    \label{fig:controlOverview}
\end{figure}







\section{Data-enabled policy optimization for autonomous bicycle  control}
\label{sec:3}
This section first describes a brief overview of the linear quadratic regulator (LQR). Then, we propose a data-enabled policy optimization with a forgetting factor for adaptive learning of the LQR based on~\cite{zhao2023data, zhao2024data}. 

\subsection{The linear quadratic regulator}
	Consider a linear time-invariant system
	\begin{equation}\label{equ:sys}
	\left\{\begin{aligned}
	x\dt{t+1} & =A x\dt{t}+B u\dt{t}+w\dt{t} \\
	h\dt{t} & =\begin{bmatrix}
	Q^{1 / 2} & 0 \\
	0 & R^{1 / 2}
	\end{bmatrix}
	\begin{bmatrix}
	x\dt{t}  \\
	u\dt{t} 
	\end{bmatrix}
	\end{aligned}\right. .
	\end{equation}
	Here, $x\dt{t}$ is the state, $u\dt{t}$ is the control input, $w\dt{t}$ represents the noise, and $h\dt{t}$ is the performance signal of interest, $(A,B)$ are controllable, and the weighting matrices $(Q, R)$ are positive definite. 
	
    The LQR problem aims to find an optimal state-feedback gain $K\in \mathbb{R}^{m\times n}$ that minimizes the $\mathcal{H}_2$-norm of the transfer function $\mathscr{T}(K):w \rightarrow h$ of the closed-loop system
	\begin{equation}
	\begin{bmatrix}
	x\dt{t+1}  \\
	h\dt{t} 
	\end{bmatrix}=\begin{bmatrix}
	A+BK & I_n \\
	\hline \begin{bmatrix}
	Q^{1 / 2} \\
	R^{1 / 2} K
	\end{bmatrix} & 0
	\end{bmatrix}\begin{bmatrix}
	x\dt{t}  \\
	w\dt{t} 
	\end{bmatrix}.
	\end{equation}
	When $A+BK$ is stable, it holds that \cite{anderson2007optimal}
	\begin{equation}\label{equ:transfer}
	\|\mathscr{T}(K)\|_2^2  = \text{Tr}((Q+K^{\top}RK)\Sigma_K)=:C(K),
	\end{equation}
	where $\Sigma_K$ is the closed-loop state covariance matrix obtained as the positive definite solution to the Lyapunov equation
	\begin{equation}\label{equ:Sigma}
	\Sigma_K = I_n + (A+BK)\Sigma_K (A+BK)^{\top}.
	\end{equation}
	We refer to $C(K)$ as the LQR cost and to (\ref{equ:transfer})-(\ref{equ:Sigma}) as a \textit{policy parameterization} of the LQR.

The optimal LQR gain $K^*$ is unique and can be found by, e.g., solving an algebraic Riccati equation with $(A,B)$~\cite{anderson2007optimal}. When $(A,B)$ is unknown, data-driven methods can be used to learn the LQR gain from input-state data. In the sequel, we propose a covariance parameterization method for direct data-driven learning of the LQR.

\subsection{Data-driven covariance parametrization of the LQR with exponential weighted data}\label{sec:3-B}
Consider the $t$-long time series of states, inputs, noises, and successor states
\begin{align}
X_{0,t} &:= \begin{bmatrix}
x\dt{0} & x\dt{1} & \dots& x\dt{t-1} 
\end{bmatrix}\in \mathbb{R}^{n\times t},\nonumber\\
U_{0,t} &:= \begin{bmatrix}
u\dt{0} & u\dt{1} & \dots& u\dt{t-1} 
\end{bmatrix}\in \mathbb{R}^{m\times t}, \label{equ:dataMat}\\
W_{0,t} &:= \begin{bmatrix}
w\dt{0} & w\dt{1} & \dots& w\dt{t-1} 
\end{bmatrix}\in \mathbb{R}^{n\times t}, \nonumber\\
X_{1,t} &:= \begin{bmatrix}
x\dt{1} & x\dt{2} & \dots& x\dt{t} 
\end{bmatrix}\in \mathbb{R}^{n\times t}, \nonumber
\end{align}
which satisfy the system dynamics
\begin{equation}\label{equ:dynamics}
X_{1,t} = AX_{0,t}+ BU_{0,t} + W_{0,t}.
\end{equation}


Assume that the data is {\em persistently exciting (PE)} \cite{willems2005note}, i.e., the block matrix of input and state data
\begin{equation}
D_{0,t} := 
\begin{bmatrix}
    U_{0,t} \\
    X_{0,t}
\end{bmatrix}
\end{equation}
has full row rank
\begin{equation}\label{equ:rank}
\text{rank}(D_{0,t}) = m+n.
\end{equation}
Define the covariance of exponentially weighted data as
\begin{equation}
    \Phi_{t} := \frac{1}{t}D_{0,t} S_{\lambda} D_{0,t}^{\top},
\end{equation}
where $\lambda \in (0,1)$ is a forgetting factor and $S_{\lambda} := \text{diag}\{ \lambda^{t-1}, \lambda^{t-2},\dots, 1\}$. Compared with \cite{zhao2024data}, the forgetting factor here makes the weight of past data decay exponentially, such that the sample covariance can also reflect and adapt to the behavior of time-varying or nonlinear systems. 


Since $D_{0,t}$ has full row rank and $S_{\lambda}\succ 0$, the covariance matrix is positive definite, i.e., $\Phi_{t} \succ 0$. Then, for any gain $K$, there exist a matrix $V$ such that
\begin{equation}\label{equ:forget}
\begin{bmatrix}
K \\
I_n
\end{bmatrix}=  \Phi_{t} V.
\end{equation}
We refer to \eqref{equ:forget} as the \textit{covariance parameterization} with exponentially weighted data and to $V\in \mathbb{R}^{(n+m)\times n}$ as the \textit{parameterized policy}.

With \eqref{equ:forget}, the LQR problem \eqref{equ:transfer}-\eqref{equ:Sigma} can be expressed by raw data matrices $(X_{0,t}, U_{0,t}, X_{1,t})$ and the optimization matrix $V$. For brevity, let $\overline{X}_{0,t}= X_{0,t}S_{\lambda}D_{0,t}^{\top}/t$ and $\overline{U}_{0,t}=  U_{0,t}S_{\lambda}D_{0,t}^{\top}/t$ be a partition of $\Phi_t$, and let
$\overline{W}_{0,t}=  W_{0,t}S_{\lambda}D_{0,t}^{\top}/t$ be the noise-state-input covariance, and finally define the covariance with respect to the successor state as $\overline{X}_{1,t}=  X_{1,t}S_{\lambda}D_{0,t}^{\top}/t$.
Then, the closed-loop matrix can be written as
\begin{equation}
A+BK=[B,A]\begin{bmatrix}
K \\
I_n
\end{bmatrix}\overset{\eqref{equ:forget}}{=}[B,A]\Phi_t V\overset{\eqref{equ:dynamics}}{=}(\overline{X}_{1,t} - \overline{W}_{0,t})V.
\end{equation}
Following the certainty-equivalence principle~\cite{dorfler2021certainty}, we disregard the
unmeasurable $\overline{W}_{0,t}$ for the design and use $\overline{X}_{1,t}V$ as the closed-loop matrix. After substituting $A+BK$  with $\overline{X}_{1,t}V$ in (\ref{equ:transfer})-(\ref{equ:Sigma}) and leveraging \eqref{equ:forget}, the LQR problem becomes 
\begin{equation}\label{prob:equiV}
\begin{aligned}
&\mathop{\text {minimize}}\limits_{V}~J_t(V) :=\text{Tr}\left((Q+V^{\top}\overline{U}_{0,t}^{\top}R\overline{U}_{0,t}V)\Sigma_t(V)\right),\\
&\text{subject to}~ ~\overline{X}_{0,t}V= I_n,
\end{aligned}
\end{equation}
where $\Sigma_t(V) = I_n + \overline{X}_{1,t}V\Sigma_t(V) V^{\top}\overline{X}_{1,t}^{\top}$ is a covariance parameterization of \eqref{equ:Sigma},
and the original gain matrix can be recovered as $K = \overline{U}_{0,t}V$. We refer to (\ref{prob:equiV}) as the covariance-parameterized LQR problem, which is direct data-driven and does not involve any explicit SysID.



\subsection{Data-enabled policy optimization for adaptive LQR control with exponentially weighted data}
In previous work \cite{zhao2023data,zhao2024data}, a data-enabled policy optimization (DeePO) method for direct adaptive learning of the LQR was proposed, where the control policy is parameterized by sample covariance and updated recursively using gradient methods. In this subsection, we propose a DeePO algorithm based on our covariance parameterization with exponentially weighted data \eqref{equ:forget}, detailed in Algorithm \ref{alg:deepo}.

Algorithm \ref{alg:deepo} alternates between control (line 2) and policy update (lines 3-6).
The DeePO algorithm uses online gradient descent of \eqref{prob:equiV} to recursively update $V$. 
At time $t$, we apply the linear state feedback policy $u\dt{t}=K_tx\dt{t}  + e\dt{t} $ for control and observe the new state $x\dt{t+1} $, where $e\dt{t}$ is a probing noise used to ensure the PE rank condition \eqref{equ:rank}. To update the policy, we first use $(X_{0,t+1}, U_{0,t+1}, X_{1,t+1})$ to formulate the covariance-parameterized LQR problem \eqref{prob:equiV}. Then, instead of solving this optimization problem optimality, we only take a single step of projected gradient descent towards its solution in \eqref{equ:pro_gd}. Here, the projection  
\begin{equation}
\Pi_{\overline{X}_{0,t+1}}: = I_{n+m}-\overline{X}_{0,t+1}^{\dagger}\overline{X}_{0,t+1}
\end{equation}
onto the nullspace of $\overline{X}_{0,t+1}$ is to ensure the subspace constraint in \eqref{prob:equiV}.
 Define the feasible set of \eqref{prob:equiV} (i.e., the set of stable closed-loop matrices) as $\mathcal{S}_t:= \{V\mid \overline{X}_{0,t}V =I_n,  \rho (\overline{X}_{1,t}V)<1\}$. Then, the gradient can be computed as follows.
\begin{lemma}[\cite{zhao2024data}]\label{lem:gradient}
For $V\in \mathcal{S}_t$, the gradient of $J_t(V)$ with respect to $V$ is given by
	\begin{equation}\label{equ:pg}
	\nabla J_t(V) = 2 \left(\overline{U}_{0,t}^{\top}R\overline{U}_{0,t}+\overline{X}_{1,t}^{\top}P_t\overline{X}_{1,t}\right)V \Sigma_t(V),
	\end{equation}
	where $P_t$ satisfies the Lyapunov equation 
	\begin{equation}
	P_t = Q + V^{\top}\overline{U}_{0,t}^{\top}R\overline{U}_{0,t}V + V^{\top}\overline{X}_{1,t}^{\top}P_t\overline{X}_{1,t}V.
	\end{equation}
\end{lemma}
 


Algorithm \ref{alg:deepo} is \textit{direct and adaptive} in the sense that it directly uses online closed-loop data to update the policy. Thanks to the forgetting factor, it can rapidly adapt to changes in system behavior reflected in the data. 
As in \cite{zhao2024data}, Algorithm \ref{alg:deepo} can also be implemented recursively. We write the sample covariance  recursively as
\begin{equation}\label{equ:recur}
    \Phi_{t+1} = \frac{\lambda t}{t+1} \Phi_{t} + \frac{1}{t+1} \phi_t\phi_t^{\top},
\end{equation}
where $\phi_t = [u_t^\top, x_t^\top]^\top$. By the Sherman-Morrison formula~\cite{sherman1950adjustment}, its inverse $\Phi_{t+1}^{-1}$ satisfies
\begin{equation}
    \Phi_{t+1}^{-1} = \frac{t+1}{\lambda t}\left(\Phi_{t}^{-1} - \frac{\Phi_{t}^{-1}\phi_t\phi_t^{\top}\Phi_{t}^{-1}}{\lambda t+\phi_t^{\top}\Phi_{t}^{-1}\phi_t}\right).
\end{equation}
Furthermore, the rank-one update of the parameterized policy is given by
\begin{align}
V_{t+1}&= \frac{t+1}{t}\left(\Phi_{t}^{-1} - \frac{\Phi_{t}^{-1}\phi_t\phi_t^{\top}\Phi_{t}^{-1}}{t+\phi_t^{\top}\Phi_{t}^{-1}\phi_t}\right) \Phi_{t} V_{t}' \nonumber \\ 
&= \frac{t+1}{\lambda t} \left(V_{t}' - \frac{\Phi_{t}^{-1}\phi_t\phi_t^{\top}V_{t}'}{\lambda t+\phi_t^{\top}\Phi_{t}^{-1}\phi_t}\right),
\end{align}
where $\Phi_{t}^{-1}$ and $V_{t}'$ are given from the last iteration.

\begin{remark}
Using the forgetting factor $\lambda$ may asymptotically lead to failure of the rank-one update as time tends to infinity. To see this, we notice that the covariance update in
\eqref{equ:recur} satisfies stable linear dynamics. This implies a loss of persistency of excitation and $\Phi_t$ will tend to zero, and hence $\Phi^{-1}_t$ will grow to infinity. A simple remedy is to reset the covariance $\Lambda_{t}$  occasionally, i.e., set $\Lambda_{t}=I_{n+m}, \forall t\in \{T,2T,\dots\}$. Since the autonomous bicycle operates only for a finite time, we do not reset the covariance in our subsequent experiments in Section \ref{sec:4}. Another approach is to use sliding window data rather than exponentially weighted data for the covariance parameterization \eqref{equ:forget}, where a key is to select an optimal window size to balance data informativity and adaptation efficiency. We leave this exploration to future work.
\qed
\end{remark}

\begin{remark}
    The stepsize $\eta_t$ should be set according to the signal-to-noise ratio (SNR) of online data. For example, when the SNR is large, we are confident with the gradient direction, and the stepsize can be chosen more aggressively; on the contrary, when the SNR is small, the stepsize should be small to prevent the policy from moving out of the stability region. To this end, we set the stepsize as 
    \begin{equation}    
    \eta_t = \frac{\eta_0}{\left\|\overline{U}_{0,t}\Pi_{\overline{X}_{0,t}}\overline{U}_{0,t}^{\top}\right\|}, ~t\geq t_0,
    \end{equation}
    where $\eta_0$ is a constant, and the denominator is used to quantify the SNR. Another motivation of the denominator is from \cite[Lemma 3]{kang2024linear}, which reveals the equivalence between data-enabled and model-based policy gradients up to the data matrix $\overline{U}_{0,t}\Pi_{\overline{X}_{0,t}}\overline{U}_{0,t}^{\top}$. \qed
\end{remark}
 

\begin{algorithm}[t]
	\caption{DeePO for direct adaptive LQR control}
	\label{alg:deepo}
	\begin{algorithmic}[1]
		\Require Offline data $(X_{0,t_0}, U_{0,t_0}, X_{1,t_0})$, an initial policy $K_{t_0}$, and a stepsize $\eta$.
		\For{$t=t_0,t_0+1,\dots$}
		\State Apply $u\dt{t} =K_tx\dt{t}  + e\dt{t} $ and observe $x\dt{t+1} $.
        \State Update covariance matrices $\Phi_{t+1}$ and $\overline{X}_{1,t+1}$.
		\State \textbf{Policy parameterization:} given $K_{t}$, solve $V_{t+1}$ via 
		$$
		V_{t+1} =\Phi_{t+1}^{-1} \begin{bmatrix}
		K_{t} \\
		I_n
		\end{bmatrix}.
		$$
		\State \textbf{Update of the parameterized policy:} perform one-step projected gradient descent
		\begin{equation}\label{equ:pro_gd}
		V_{t+1}' = V_{t+1} - \eta_t  \Pi_{\overline{X}_{0,t+1}} \nabla J_{t+1}(V_{t+1}),
		\end{equation} 
		where the gradient $\nabla J_{t+1}(V_{t+1})$ is given by Lemma \ref{lem:gradient}.
		\State \textbf{Gain update:} update the control gain by 
		$$
		K_{t+1} = \overline{U}_{0,t+1}V_{t+1}'.
		$$
		\EndFor	
	\end{algorithmic}
\end{algorithm}

Algorithm \ref{alg:deepo} requires the initial policy to be stabilizing. A potential approach is to solve the  covariance-parameterized LQR \eqref{prob:equiV} with the offline data $(X_{0,t_0}, U_{0,t_0}, X_{1,t_0})$. However, due to the nonlinearity in the system dynamics of the autonomous bicycle, the solution of covariance-parameterized LQR may be destabilizing. Next, we propose a robustness-promoting regularizer for the covariance parameterization to obtain a stabilizing initial policy.
 
\subsection{Learning an initial stabilizing policy using robustness promoting regularization}


	The feasibility of the covariance-parameterized LQR problem \eqref{prob:equiV} depends on that of the Lyapunov equation
	\begin{equation}\label{equ:lyap}
		\Sigma = I_n + \overline{X}_1V\Sigma V^{\top}\overline{X}_1^{\top},
	\end{equation}
	where $\overline{X}_1V$ is regarded as the closed-loop matrix. However, having assumed certainty-equivalence by the covariance parameterization \eqref{equ:forget} and the relation $A+BK = (\overline{X}_1 - \overline{W}_0)V$, the Lyapunov equation that should be met is
	\begin{equation}\label{equ:true_lyap}
		\Sigma = I_n + (\overline{X}_1 - \overline{W}_0)V\Sigma V^{\top}(\overline{X}_1 - \overline{W}_0)^{\top}.
	\end{equation}
	The gap between the right-hand side of \eqref{equ:lyap} and \eqref{equ:true_lyap} is
	\begin{equation}\label{equ:diff}
		\begin{aligned}
		&\overline{W}_0 V\Sigma V^{\top}  \overline{W}_0^{\top} - \overline{W}_0 V\Sigma V^{\top} \overline{X}_1^{\top} -   \overline{X}_1V\Sigma V^{\top} \overline{W}_0^{\top} \\
		&= \frac{1}{t^2}  W_0D_0^{\top}V\Sigma V^{\top}D_0W_0^{\top} \\
		&-\frac{1}{t^2}(  W_0D_0^{\top}V\Sigma V^{\top}D_0X_1^{\top} +   X_1D_0^{\top}V\Sigma V^{\top}D_0W_0^{\top}).
		\end{aligned}
	\end{equation}
	To reduce the gap, it suffices to make $\text{Tr}(D_0^{\top}V\Sigma V^{\top}D_0/t)$ small. To this end, we introduce the regularizer $\text{Tr}(V\Sigma V^{\top}\Phi)$ to the covariance-parameterized LQR problem \eqref{prob:equiV}, leading to
	\begin{equation}\label{prob:regu}
	\begin{aligned}
	&\mathop{\text {minimize}}\limits_{V, \Sigma\succeq 0}~ J_t(V) + \gamma\text{Tr}(V\Sigma V^{\top}\Phi),\\
	&\text{subject to}~ ~\Sigma = I_n + \overline{X}_1V\Sigma V^{\top}\overline{X}_1^{\top},\overline{X}_0V= I_n
	\end{aligned}
	\end{equation}
	with gain matrix $K = \overline{U}_0V$, where $\gamma>0$ is the regularization coefficient. We refer to \eqref{prob:regu} as the regularized covariance parameterization of the LQR problem.

    To obtain an initial stabilizing policy for Algorithm \ref{alg:deepo}, we solve \eqref{prob:regu} with offline data $(X_{0,t_0}, U_{0,t_0}, X_{1,t_0})$.

\subsection{Control gain update rate}
Rapid changes in an adaptive control policy, $K_t$ can potentially induce oscillations and, in the worst case, render the system unstable~\cite{landau2011adaptive}. Moreover, the control policy at certain time intervals may be significantly influenced by measurement noise, meaning that updates could be driven more by noise than by the actual system dynamics. 

To address these potential issues, we propose updating the DeePO control gain less frequently than the sampling frequency. To regulate the update frequency, we introduce the parameter $\xi$, which determines the intervals at which the controller in line 6 of Algorithm~\ref{alg:deepo} is updated. For instance, if $\xi = 1$, the control gain is updated at every iteration, whereas if $\xi = 100$, the gain is updated every $100$ iterations.










% \begin{figure*}[htpb!]
% \label{}
% \centering

%     {{\label{ROCIowaCedar} \includegraphics[width=\textwidth/3]{figures/IowaCedar_roc.png}}}%
%     \qquad
%     {{\label{ROCIowaDesMoines} \includegraphics[width=\textwidth/3]{figures/IowaDesMoines_roc.png} }%
%   \captionsetup{justification=centering}
%   \caption{\Acf{ROC} curves for \acf{RW} Iowa (CR) and  \acf{RW} Iowa (DM) dataset. Dummy model here represents a model whose output is solely a ``no Flood'' for all pixels.}
%   \label{fig:RW_ROC_Curves}%
% \end{figure*}



\section{Results and Discussions}
\label{sec:Results}

In this section, we aim to answer three main questions. First, we want to validate our hypothesis that \ac{SYN} data is a viable proxy for \ac{RW} data when training ML models for downscaling. Secondly, we seek to assess how much more skillful ML-based downscaling is compared to classical, non-data-driven techniques, such as our baseline methods, \textit{i.e.}, thresholded bicubic and Lanczos interpolation. Finally, we would like to appraise the extent to which data-driven models like ours are transferable (in terms of usefulness) to other regions without major performance degradations.  
To assess the quality of the models, we conduct a multiple comparison test --namely the Holm-Bonferroni procedure \cite{HolmBonferroni1979} -- that is designed to control the \ac{FWER}. We notice that, with a \ac{FWER} of $10^{-3}$, all the differences in model performance are significant. The only exception to this trend was observed in \ac{RW}-GH for whom the pairwise differences between \ac{RCAN} and \ac{ESRT}, Lanczos and Bicubic were not significant with the aforementioned \ac{FWER}. 

%Finally, we aim to find out the factors influencing the transferability of our models from one region to another.

\subsection{Potential of using SYN Data for RW downscaling}

In order to evaluate the utility of synthetic data for training, we compare performances of our candidate models on both \ac{SYN} and \ac{RW} Iowa data whose results are presented in Table \ref{tab:IowaResults}. We notice that 
\textbf{(i)} For the Iowa datasets, there is a drop in performance of all the models when going from \ac{SYN} to \ac{RW} datasets, 
\textbf{(ii)} for the \ac{RW}-IA (CR) as well as \ac{RW}-IA (DM) datasets, both bicubic and Lanczos interpolation have accuracies and MCC up to 70.89\% and 0.42 respectively while the deep learning models have accuracies and MCC up to 73.34\% and 0.46 respectively, 
\textbf{(iii)} There is a roughly 6\% accuracy improvement for the \ac{SYN} data for the deep learning models compared to the bicubic and lanczos models and this improvement drops to about 3\% for \ac{RW} data,  
\textbf{(iv)} the performance of all the models remain consistent across both \ac{RW}-IA datasets and \textbf{(v)} in \figref{fig:RW_ROC_Curves}, we observe that there is a high degree of overlap among the \ac{ROC} curves for the data-driven models.

From (i) and (iv) we can conclude that \ac{SYN} data is more intricate than \ac{RW} data. This implies that the benefits yielded by training with \ac{SYN} dataset, while significant, is not as prominent in the \ac{RW} Iowa datasets. 
% This may be due to sensor noise prevalent in the \ac{RW} Landsat-8 data that can be harder to reproduce in the synthetically generated examples. 
(i), (iii) and (v) implies that while \ac{SYN} data is not an exact replacement for \ac{RW} data, it provides a rather significant edge, which is all the more important when there is insufficient \ac{RW} for training. From (ii) we can conclude that the three proposed data driven models outperform classical super-resolution techniques such as bicubic and lanczos, conclusion supported by the \ac{ROC} curves in Figure \ref{fig:RW_ROC_Curves} for whom the data-driven models, in general, lie above the non-data-driven alternatives. Observation (iv) shows that  for the climatically similar \ac{RW}-Iowa(CR) and \ac{RW}-Iowa(DM) regions, training on \ac{SYN} Iowa data does indeed provide an edge. 

% have similar climate. 

\begin{figure*}[t!]
    \centering
    \begin{subfigure}[t]{0.5\textwidth}
        \centering
        \includegraphics[width=\textwidth/2]{figures/IowaCedar_roc.png}
        \caption{}
    \end{subfigure}%
    ~ 
    \begin{subfigure}[t]{0.5\textwidth}
        \centering
        \includegraphics[width=\textwidth/2]{figures/IowaDesMoines_roc.png}
        \caption{}
    \end{subfigure}
    \vspace*{0.5cm}
    \caption{    \label{fig:RW_ROC_Curves} \Acf{ROC} curves for (a) RW-IA (CR) and (b) RW-IA (DM) dataset. Na\"ive model here represents a model whose output is solely a ``no Flood'' for all pixels. Star here represents the pixel-wise classifier with a threshold of 0.5.}
\end{figure*}


\subsection{Effectiveness of data-driven approaches}

In order to evaluate the effectiveness of ML models in the downscaling task, we compare performances of our candidate models to Lanczos and bicubic interpolation methods by looking at figures of some sample predictions from Iowa (Figure \ref{fig:RWIowaDesMoines}), performance comparison in the region of Iowa in Table \ref{tab:IowaResults} and the ROC curves in Figure \ref{fig:RW_ROC_Curves} for \ac{RW} data. We notice that 
\textbf{(vi)} For RW-IA (DM) samples, the deep learning models maintain a higher degree of spatial continuity in the predicted \ac{FIM}, 
\textbf{(vii)} We observe that  bicubic and Lanczos interpolation produces over-smoothed \ac{FIM} reconstructions, while the plain \ac{RDN}, \ac{RCAN} and \ac{ESRT} models are more detail-inclusive. Similar conclusions can be drawn upon inspecting the \ac{ROC} curves in Figure \ref{fig:RW_ROC_Curves} and 
\textbf{(viii)} For RW-IA (CR), the ML models show a performance improvement of 3.06\% when comparing the best ML model and non-data-driven method and, while for RW-IA (DM) there is a performance improvement of 2.45\%.


Figures \ref{fig:EUSamples} and \ref{fig:RWIowaDesMoines} show the spatial disparity among the models whose details are often obscured in aggregated metrics such as accuracy. (vi) This implies that these data-driven models are better are recognizing an underlying stream network geometry than the classical methods. However, when it comes to narrow river streams, all the models struggle capturing the nuances of the \ac{FIM} resultant from localized high elevation features such as small islands within rivers or man-made structures. (vii) shows a clear advantage of our data-driven approaches over the non-data-driven alternatives. (viii) indicates the benefits of the data-driven models when evaluated over Iowa. 



\subsection{Applicability of our models to external regions}

To evaluate how transferable our models are, we draw conclusions from figures of the sample predictions from Western Europe (Figure \ref{fig:EUSamples}) and Ghana (Figure \ref{fig:GhanaSamples}) as well as the performance comparison in Table \ref{tab:ExternalResults}. We notice that 
\textbf{(ix)} for Ghana all of the models fail to adequately inundate the pixels over separated areas on account of several disconnected regions of inundation in the chosen area,
\textbf{(x)} the ML models outperform non-data driven methods for RW-EU, 
\textbf{(xi)} for the RW-EU dataset, there is an improvement of 4.89\% when comparing the accuracy of the best data- and non-data-driven methods, 
\textbf{(xii)} For RW-RR and RW-GH, there is marginal improvement (up to 0.77\% in accuracy) of the ML methods over the non-data driven methods and 
\textbf{(xiii)} For RW-EU, we notice that the ML models produce more connected streams over the non-data-driven models. 

(x) and (xi) implies that the models are transferable when considering hydroclimaticalogically similar regions since Iowa and the Meuse river in Europe lie within mid temperate zones. Similar to the observation (vi) for RW-IA (DM), (xiii) implies that the benefits of the ML model in identifying underlying network streams is also transferable to hydroclimatologically similar regions. In contrast, (xii) and (ix) both imply that the trained ML models struggle to generalize to RW-RR \& RW-GH. We speculate that this may be due to the significant differences in geography and climate when compared to Iowa. 

% More specifically, we notice that Ghana has a lot of disconnected regions when compared to Iowa and Western Europe, possibly indicating a geomorphological dissimilarity. Additionally, in the case of Red River and Ghana, we also speculate that they include drivers to flood inundation that are different from Iowa and Western Europe, which lie within mild temperate zones. Ghana on the other hand has a tropical (dry and hot) climate.

Our study directly implies that good quality synthetic data can be useful surrogates for downscaling low-resolution \acp{WFM} to high-resolution \acp{FIM} in regions, where such data are hard to come by, even when downscaling by a factor of 10. We noticed that such models were readily transferable to climatically similar regions as the region of training. However, Such derived ML models did not feature significantly different transferability when evaluated over hydroclimatologically dissimilar regions, which we attribute to different flood inundation characteristics, primarily at finer scales. A possible avenue to circumvent such issues is to explore additional training approaches that fall under the general area of domain adaptation. Nevertheless, data-driven models are still advantageous (and, hence, preferable) over non-data-driven alternatives in transfer scenarios like the one we considered here. 


%%%%%%%%%%%%%%%%%%%%%%%%%%%%%%% unused text %%%%%%%%%%%%%%%%%%%%%%%%%%%%%%%%%%%%%%%



% \tabref{tab:AccuracyResults} depicts test accuracies obtained by our models on both \ac{SYN} and \ac{RW} data. For Iowan floods, a comparison of \ac{SYN} and \ac{RW} results shows \textbf{(i)} bicubic and Lanczos interpolations remarkably gaining about $3\%$ in accuracy, as well as \textbf{(ii)} \ac{RDN} and \ac{RCAN} remaining relatively stable, while \textbf{(iii)} topography-aware models loosing $2.7\%$ in performance. From (i) one can conclude that \ac{SYN} data are morphologically slightly more intricate than \ac{RW} data. Also, (i) and (ii) likely imply that \ac{SYN} data, excluding topography, can serve as satisfactory surrogates of \ac{RW} data. However, as implied by (iii), our topography-dependent models seems to be particularly sensitive to distributional shifts of their combined inputs (\acp{WFM} and topographic features). More specifically, the topography-informed models' performance edge, while still statistically significant, is extremely marginal, even when compared to our non-data-driven approaches. Next, when comparing results between the cases of Iowan and Ghanaian \ac{RW} data, one observes that \textbf{(iv)} the accuracy of bicubic and Lanczos interpolations drops by almost $5\%$ due to over-smoothing. This may imply that Ghanaian \acp{FIM} bare a more complex morphology, when compared to Iowan \acp{FIM}. Also, \textbf{(v)} our topography-agnostic, data-driven models' performance degrades more gracefully (by about $2\%$), while \textbf{(vi)} our topography-aware models perform, virtually, as bad as our non-data-driven approaches. Hence, the differences in the data populations of the two regions we considered are significant enough to render our topography-dependent models noncompetitive. 



Software development is increasingly conceived as a collaboration activity between developers and AIs. Indeed, IDEs already implement features to enable interactive development, with AI suggesting implementations that are reused by developers.

Although multiple studies show this interaction can be successful, there is still limited understanding of how the models must be configured and used in the context of code generation tasks. This study addresses this gap, systematically investigating the impact of several key parameters, including the repeated submission of a prompt to accommodate for the non-deterministic nature of the models.

Our study reveals several key findings about the usage of ChatGPT. In particular, we discovered how creativity, although up to a limited extent, is useful to increase the range of methods whose code can be generated correctly. A major role is played by parameter top-p, which is commonly underrated, and instead has a major impact on the correctness of the results, with lower values producing better results. Finally, prompts should be submitted multiple times, with $5$ repetitions combined with a temperature of $1.2$ resulting in an effective configuration in our experiments.  

Future work concerns two main research directions. One is about replicating this experiment with other AI assistants, to validate our findings in multiple contexts. The second research direction concerns finding strategies to deal with the need to submit the same prompt multiple times to obtain a useful result, and thus developing approaches able to select or merge multiple responses automatically. 
\bibliographystyle{IEEEtran}  
\bibliography{refs.bib}
\def\spbio{30}
\vspace{-\spbio pt}
\begin{IEEEbiography}[{\includegraphics[width=1in,height=1.25in,clip,keepaspectratio]{Niklas.jpg}}] {Niklas Persson} received an M.Sc in Robotics from M{\"a}lardalen University, V{\"a}ster{\aa}s, Sweden in 2019. Since 2020, he has been pursuing a PhD degree in electronics at the Intelligent Future Technologies division of M{\"a}lardalen University, working on the control and navigation of autonomous bicycles. In 2023, he received a Licentiate degree at Mälardalen University. His research interests include autonomous robots and vehicles, control theory, and embedded systems. 
\end{IEEEbiography}
\vspace{-\spbio pt}


\begin{IEEEbiography}[{\includegraphics[width=1in,height=1.25in,clip,keepaspectratio]{feiran.jpg}}] {Feiran Zhao} received the B.S. degree in Control Science and Engineering from the Harbin Institute of Technology, China, in 2018, and the Ph.D. degree in Control Science and Engineering from the Tsinghua University, China, in 2024. He is now a postdoc at ETH Z\"{u}rich. His research interests include policy optimization, data-driven control, adaptive control and their applications. 
\end{IEEEbiography}
\vspace{-\spbio pt}


\begin{IEEEbiography}[{\includegraphics[width=1in,height=1.25in,clip,keepaspectratio]{Mojtaba.jpg}}] {Mojtaba Kaheni} (SM'25) is a Postdoctoral Researcher at the {School of Innovation, Design, and Technology (IDT)}, {Mälardalen University}, Västerås, Sweden. He received his {M.Sc.} and {Ph.D.} in Control Engineering from {Shahrood University of Technology}, Shahrood, Iran, in 2011 and 2019, respectively.
Dr. Kaheni has held visiting scholar positions at the {University of Florence}, Italy, and {Lund University}, Sweden. From August 2020 to December 2022, he served as a Postdoctoral Researcher at the {University of Cagliari}, Italy.
His research interests include {control theory}, {distributed optimization}, {multi-agent systems}, and {resiliency}.

\end{IEEEbiography}
\vspace{-\spbio pt}

\begin{IEEEbiography}
[{\includegraphics[width=1in,height=1.25in,clip,keepaspectratio]{doerfler-florian_h_compressed.jpeg}}]	
	{Florian D\"{o}rfler} is a Full Professor at the Automatic Control Laboratory at ETH Z\"{u}rich. He received his Ph.D. degree in Mechanical Engineering from the University of California at Santa Barbara in 2013, and a Diplom degree in Engineering Cybernetics from the University of Stuttgart in 2008. From 2013 to 2014 he was an Assistant Professor at the University of California Los Angeles. He has been serving as the Associate Head of the ETH Z\"{u}rich Department of Information Technology and Electrical Engineering from 2021 until 2022. His research interests are centered around automatic control, system theory, and optimization. His particular foci are on network systems, data-driven settings, and applications to power systems. He is a recipient of the distinguished young research awards by IFAC (Manfred Thoma Medal 2020) and EUCA (European Control Award 2020). His students were winners or finalists for Best Student Paper awards at the European Control Conference (2013, 2019), the American Control Conference (2016, 2024), the Conference on Decision and Control (2020), the PES General Meeting (2020), the PES PowerTech Conference (2017), the International Conference on Intelligent Transportation Systems (2021), and the IEEE CSS Swiss Chapter Young Author Best Journal Paper Award (2022, 2024). He is furthermore a recipient of the 2010 ACC Student Best Paper Award, the 2011 O. Hugo Schuck Best Paper Award, the 2012-2014 Automatica Best Paper Award, the 2016 IEEE Circuits and Systems Guillemin-Cauer Best Paper Award, the 2022 IEEE Transactions on Power Electronics Prize Paper Award, and the 2015 UCSB ME Best PhD award. He is currently serving on the council of the
	European Control Association and as a senior editor of Automatica.
	\end{IEEEbiography}
\vspace{-\spbio pt}
\begin{IEEEbiography}[{\includegraphics[width=1in,height=1.25in,clip,keepaspectratio]{Alessandro.png}}]{Alessandro~V. Papadopoulos} (SM'19)
is a Full Professor of Electrical and Computer Engineering at M{\"a}lardalen University, V{\"a}ster{\aa}s, Sweden, and a QUALIFICA Fellow at the University of M{\'a}laga, Spain. Since March 2024, he has been the scientific leader of Applied AI at M{\"a}lardalen University. He received his B.Sc. and M.Sc. (summa cum laude) degrees in Computer Engineering from the Politecnico di Milano, Milan, Italy, and his Ph.D. (Hons.) degree in Information Technology from the Politecnico di Milano, in 2013. He was a Postdoctoral researcher at the Department of Automatic Control, Lund, Sweden (2014-2016) and Politecnico di Milano, Milan, Italy (2016). 
He was the Program Chair for the Mediterranean Control Conference (MED) 2022, the Euromicro Conference on Real-Time Systems (ECRTS) 2023, and the ACM/SPEC International Conference on Performance Engineering (ICPE) 2025. He is an associate editor for the ACM Transactions on Autonomous and Adaptive Systems, Control Engineering Practice, and Leibniz Transactions on Embedded Systems.
His research interests include robotics, control theory, real-time systems, and autonomic computing. 
\end{IEEEbiography}

\end{document}
\typeout{get arXiv to do 4 passes: Label(s) may have changed. Rerun}
