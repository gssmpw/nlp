\section{Conclusion}
\label{sec:5}
This paper introduced a unified framework that balances an autonomous bicycle by combining an FL controller in the inner loop and DeePO in the outer loop. The primary objective of the FL loop is to stabilize and partially linearize an otherwise unstable and nonlinear system. However, practical systems often contain unmodeled dynamics and time-varying characteristics that can degrade the performance of the FL controller when used in isolation. To address these challenges, we integrated a DeePO controller on top of the FL loop.

We derived an initial control policy using a finite set of offline, persistently exciting input and state data. To handle the nonlinearities and disturbances that may degrade the policy's performance, we introduced a robustness-promoting regularizer to refine the initial stabilizing policy and a forgetting factor in the DeePO framework to adapt to the time-varying nature of our case study.

We demonstrated the effectiveness of the DeePO+FL approach through both simulations and real-world experiments on an autonomous bicycle. The results clearly showed that DeePO+FL outperforms the FL-only approach, particularly in terms of tracking the reference lean angle and lean rate more accurately. Additionally, we evaluated the impact of the control gain update frequency and found that performance improvements could be achieved with a lower update frequency. However, determining the optimal update rate remains an open question for future research.

The experimental and simulation results demonstrated that the proposed controller effectively adapts to the system dynamics despite the presence of nonlinearities, sensor noise, and hardware limitations. Our work illustrates the potential of direct data-driven methods to adapt and control nonlinear systems, such as an autonomous bicycle, relying solely on data. In the future, we plan to enhance the DeePO algorithm by incorporating the robustness regularizer in its online component. Additionally, exploring direct data-driven navigation for the bicycle is another exciting direction for further research.
