
\section{Introduction}

Linear Parameter Varying (LPV) Systems are a class of nonlinear dynamic models studied in the literature and widely used in applications  \cite{Hoffman2014}. 
%
In fact, not only are methods for control of LPV systems well established in the literature, but any nonlinear dynamic system can also be represented as the so-called quasi-LPV system.
%
LPV systems are expressed in terms of a linear relation between the state and input to the state derivative (continuous time) or the next state in sampling time (discrete time).
%
However, they are nonlinear in the sense that they depend on an extra set of inputs, referred to as parameters.
%
If the parameters are fully exogenous, the system is called a fully-fledged LPV system. 
%
However, when the parameters depend on state and inputs (as generally happens when converting a nonlinear system to an LPV system), the LPV system is referred to as a quasi-LPV system.
%
Works such as \cite{Hoffman2014} show that not only can LPV-type models represent a wide variety of systems, such as airplanes, hypersonic vehicles, satellites, laser printers, drilling rigs, and wind energy generation systems, among many others, but also have very well-defined methods to mathematically design and tune controllers, which is something that nonlinear systems generally lack.
%
Also, quasi-LPV systems have a mature development in Model Predictive Control (MPC) applications \cite{Morato2024}, where convergence properties are very well defined.
%
These features prove that there are many advantages in modeling a given system as an LPV.


There are also many methods for identifying LPV systems \cite{LopesdosSantos2011}, which are generally divided into \textit{local} identification \cite{Zhang2017232} and \textit{global} identification approaches \cite{Armanini2018}.
%
The main difference between them is that the former utilizes data assuming a fixed value for the parameters, while identifying a linear system and then performing interpolation between each system found.
%
The latter approach uses data from many different points, treating the parameters as extra model inputs (feature generators). The local approach is easier to compute but only sometimes feasible, especially in systems that demand quasi-LPV models.
%
There are some gaps in LPV identification theory, such as how exactly the structure of the scheduling function is defined \cite{LopesdosSantos2011} (in the case of this work, we consider that model structure is known to simplify this issue).
%
However, the main gap that this work addresses is the lack of approaches in the literature for performing LPV identification of large-scale systems.



%
Any system is considered large-scale if it has a high number of states and inputs in a way that computing those systems is not trivial, let alone identifying them.
%
As a matter of fact, any system that is described as a PDE has its discretization presented as a large-scale system since a PDE can be seen as a system with infinite states, and the more refined the discretization, the larger the number of states.
%
The field that deals with large-scale systems-related problems is called model order reduction (MOR) \cite{dragoslav}.
%
Model order reduction can be performed intrusively or non-intrusively.
%
Intrusive MOR involves having a full system and obtaining its reduced form.
%
An example of intrusive MOR is applying Proper Orthogonal Decomposition (POD) by itself \cite{Chatu2010}, which obtains a linear map between a reduced state space and the full state space of the system, while accounting for the linear map in the formulation.
%
Non-intrusive MOR involves avoiding computing the system in its full state, since the computer is assumed to be unable to handle the computation.
%
A large-scale LPV system naturally leads to a combinatorial explosion, which makes computation very challenging. Therefore, to tackle the problem of identifying large-scale LPV systems, we must employ non-intrusive MOR methods.


A well-established non-intrusive MOR method is the Dynamic Mode Decomposition (DMD) \cite{Schmid2022}, which started as a method for identifying reduced-order counterparts as autonomous linear systems.
%
Since the response of autonomous linear systems is known analytically, it is easy to obtain the approximation of the analytical response of the full-order system through the eigenvalues of the reduced matrix and the so-called dynamic modes, which are approximations of the eigenvectors of the full-order system computed from its reduced-order counterpart.
%
DMD is also intimately related to the Koopman operator theory, which represents a nonlinear, finite-dimensional system through a linear but infinite-dimensional operator that is a function of observables (functions of the system states).
%
The linear relation obtained by DMD can be interpreted as a finite approximation of the infinite-dimensional Koopman operator.


Naturally, DMD evolved over time, with tweaks developed for many different applications in mind \cite{Schmid2022}, with the first important one being the Exact DMD, where they propose an alternate way to compute the dynamic modes by using output data information and, as a result, a more precise calculation of the eigenvectors is obtained.
%
Relevant to the LPV system identification, however, is the so-called DMDc \cite{DMDc,Schmid2022}, where the DMD is reformulated assuming the presence of an exogenous input (be it a disturbance of a control input), which changes the identification procedure by a small margin.
%
In this work, we see this change as a clear separation between the DMD steps of solving a rank-limited linear least squares (Procrustes Problem \cite{Baddoo2022}) and the POD being performed over a state snapshot matrix to reduce the state space afterward.
%
In DMD for autonomous systems, the formulation allows the two steps to be performed at the same time, which is why the same data matrix is used to solve both POD and Procrustes.


Another important branch of DMD is the extended DMD (EDMD) \cite{Williams2015,Schmid2022}, where the direct connection between DMD and the infinite-dimensional Koopman operator is exploited since a DMD considering nonlinear functions of states as features is seen as an approximation with a linear finite-dimensional operator and a finite number of observables.
%
However, in EDMD, both the input and output are subject to the mapping to the observables space, and the Procrustes problem is solved in the observable dimension, which is intrinsically quite large.
%
In other works, such as in \cite{Gosea2021}, where a bilinear model is considered, the nonlinear terms are incremented into the DMD similarly to the control action introduced in DMDc.


As we remark in this work, a large-scale LPV system can suffer from a combinatorial explosion if the number of parameters and states is large.
%
Hence, this work aims to propose an efficient method for obtaining reduced-order proxy models from large-scale LPV systems using the DMD theory.
%
The contributions of this work are as follows:
    %%%
\begin{itemize}
    \item Since the least square problem associated with large-scale LPV models tends to be very hard to solve, we test the capacity to obtain quality solutions by solving an $r$-rank Procrustes problem instead.
    \item We assess the capacity of performing a DMD identification similar to \cite{Gosea2021} in the context of LPV systems.
    \item For local identification, we explore the possibility of employing the POD transform obtained by the full outputs in each LTI system being identified. As a result, the LTI systems are all identified in a reduced state space, enhancing scalability while being coherent with the nonintrusive model order reduction philosophy.
    \item We study and develop methods to mitigate the combinatorial explosion resulting from the Kronecker product between the number of parameter functions and the number of states.
\end{itemize}

This work is structured as follows:
Section \ref{sec:rel_work} presents a literature review of related works. In contrast, Section \ref{sec:lpv} gives a brief description of LPV systems, discusses standard LPV system identification methods, and describes POD model order reduction for LPV systems.
 Section \ref{sec:dmdc} explains DMDc, a basis for the proposed method.
Section \ref{sec:dmd_lpv} discusses the proposed DMD-LPV method. This method is designed to identify a reduced-order model of Large-Scale LPV systems, offering a practical and efficient solution to a complex problem.
 Section \ref{sec:application} showcases the case studies and experiments performed.
 Section \ref{sec:conclusion} concludes this work.



%%%