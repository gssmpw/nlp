\section{Related Work}
\label{sec:rel_work} 

An example of a recent work performing identification for an LPV model in a given application is ____, where LASSO and ridge regression are integrated into the LPV identification to obtain a sparse LPV model.
%
The LPV identification in question is an input-to-output model that was identified globally. The mentioned work considers a lithium-ion battery as the system to be modeled as an LPV system.
%
This shows that LPV identification is a valid method to tackle some relevant modeling for control and engineering problems.
%
A work related to large-scale LPVs is ____, which relies on a large-scale LPV system to design a robust controller for a large power system.
%
The work itself considers model order reduction a valid strategy for tackling control problems in large-scale LPV systems.
%
Another work ____ develops a strategy for LPV identification with colored noise, working mainly with global identification and employing a three-tank system as a case study.
%
For a local approach, ____ presents a method for optimally selecting the parameter configuration for LTI local systems.
%
However, none of the aforementioned identification applications tackle the problem of identifying a large-scale LPV system.


%
In our work, the methods for performing non-intrusive LPV modeling come from the DMD theory.
%
A work that employs similar principles to identify nonlinear systems is ____, where a variation of DMDc is utilized to identify a microgrid system, a very prominent large-scale system application, in the form of a nonlinear system.
%
DMD, by itself, is a powerful tool.
%
Works such as ____ employ DMD to make several dynamic analyses over complex and large-scale systems such as the GRACE satellite.
%
In principle, DMD-LPV is quite close to the bilinear DMD (biDMD), which was formulated and employed in ____ for quantum systems control, as the quantum Hamiltonian has a bilinear structure in its equation, enabling such method to be employed.


The field of combining DMD with LPV systems is still new, so the literature on methodologies for LPV system identification in larger-scale systems is scarce. However, some relevant publications on methods developed for that end exist.
%
The first of them is Stacked DMD, which was reviewed on ____.
%
The method considers only one parameter, applying a local identification approach.
%
It involves stacking training data of different parameter configurations and performing a single SVD.
%
The review of stacked DMD in ____ brings about some drawbacks, such as the eigenvalues being parameter-independent and the computational cost of the SVD being prohibitive.
%
However, another remark is that employing such an algorithm in a multi-parameter situation would be complicated.
%
The method ____ proposes identifying different LTI systems and performing Lagrangian interpolation in them while assuming one parameter.
%
They propose two variants where either the full or the reduced-order system is interpolated, with the interpolation on the reduced system being more successful in the proposed application (parametric PDEs).
%
The main difference between ____ and our approach is that we try to identify the parameter-system relation in the context of DMD instead of performing polynomial interpolation, making assumptions on the scheduling function of the system.


Another work, ____, tries identifying a Radial Basis Function Network to relate a single parameter to a set of snapshot matrices, performing DMD to a parameter-dependent matrix function.
%
The purpose is to assess a plant's dynamic modes at a given operating point. However, no identification of an LPV system per se has been performed.


%%%%%