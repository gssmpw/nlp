  %% 
%% Copyright 2007-2020 Elsevier Ltd
%% 
%% This file is part of the 'Elsarticle Bundle'.
%% ---------------------------------------------
%% 
%% It may be distributed under the conditions of the LaTeX Project Public
%% License, either version 1.2 of this license or (at your option) any
%% later version.  The latest version of this license is in
%%    http://www.latex-project.org/lppl.txt
%% and version 1.2 or later is part of all distributions of LaTeX
%% version 1999/12/01 or later.
%% 
%% The list of all files belonging to the 'Elsarticle Bundle' is
%% given in the file `manifest.txt'.
%% 
%% Template article for Elsevier's document class `elsarticle'
%% with harvard style bibliographic references

\documentclass[preprint,12pt]{elsarticle}

%% Use the option review to obtain double line spacing
%% \documentclass[authoryear,preprint,review,12pt]{elsarticle}

%% Use the options 1p,twocolumn; 3p; 3p,twocolumn; 5p; or 5p,twocolumn
%% for a journal layout:
%% \documentclass[final,1p,times,authoryear]{elsarticle}
%% \documentclass[final,1p,times,twocolumn,authoryear]{elsarticle}
%% \documentclass[final,3p,times,authoryear]{elsarticle}
%% \documentclass[final,3p,times,twocolumn,authoryear]{elsarticle}
%% \documentclass[final,5p,times,authoryear]{elsarticle}
%% \documentclass[final,5p,times,twocolumn,authoryear]{elsarticle}

%% For including figures, graphicx.sty has been loaded in
%% elsarticle.cls. If you prefer to use the old commands
%% please give \usepackage{epsfig}

%% The amssymb package provides various useful mathematical symbols
\usepackage{graphicx}      % include this line if your document contains figures
%\usepackage{natbib}        % required for bibliography
\usepackage{amsmath,amssymb,amsfonts}


\usepackage{algorithm,algpseudocode}% http://ctan.org/pkg/{algorithms,algorithmx}
\algnewcommand{\Inputs}[1]{%
  \State \textbf{Inputs:}
  \Statex \hspace*{\algorithmicindent}\parbox[t]{.8\linewidth}{\raggedright #1}
}
\algnewcommand{\Initialize}[1]{%
  \State \textbf{Initialize:}
  \Statex \hspace*{\algorithmicindent}\parbox[t]{.8\linewidth}{\raggedright #1}
}


\usepackage{textcomp}
\usepackage[utf8]{inputenc}
%\usepackage[T1]{fontenc}
\usepackage{color}

\usepackage{lipsum}

\usepackage{amsbsy}
%\usepackage{amsthm}
\usepackage{multirow}

\usepackage{subfig}
\usepackage{amsfonts, amsmath, amsthm, amsbsy,amssymb,bm,mathtools} % For math fonts, symbols and environments %
\usepackage{graphicx} 		% Required for including images
\usepackage{transparent}	% may be required for inkscape pdf figures (http://bit.ly/18i5Oga)
\newsubfloat{figure}		% Allow subfloats in figure environment (http://bit.ly/1C20NAj)
\graphicspath{{figures/}} 	% Location of the graphics files

\usepackage[dvipsnames]{xcolor}

\usepackage{lineno,hyperref}
\usepackage{mathtools}
\usepackage{verbatim}
\usepackage{comment}

\newcommand{\vb}[1]{{\Verb!#1!}}
%%% EC introduced the following command
\allowdisplaybreaks[1]
%
% If IEEEtran.cls has not been installed into the LaTeX system files,
% manually specify the path to it like:
% \documentclass[journal]{../sty/IEEEtran}



\newcommand{\ch}[1]{\textcolor{blue}{#1}}
%% The amsthm package provides extended theorem environments
%% \usepackage{amsthm}

%% The lineno packages adds line numbers. Start line numbering with
%% \begin{linenumbers}, end it with \end{linenumbers}. Or switch it on
%% for the whole article with \linenumbers.
%% \usepackage{lineno}

\journal{Journal of Computational Physics}

\begin{document}
\newcommand{\eric}[1]{#1}
%\newcommand{\eric}[1]{\textcolor{ProcessBlue}{#1}}

\newcommand{\ericc}[1]{\textcolor{ProcessBlue}{--EA: #1}}
\begin{frontmatter}

%% Title, authors and addresses

%% use the tnoteref command within \title for footnotes;
%% use the tnotetext command for theassociated footnote;
%% use the fnref command within \author or \affiliation for footnotes;
%% use the fntext command for theassociated footnote;
%% use the corref command within \author for corresponding author footnotes;
%% use the cortext command for theassociated footnote;
%% use the ead command for the email address,
%% and the form \ead[url] for the home page:
%% \title{Title\tnoteref{label1}}
%% \tnotetext[label1]{}
%% \author{Name\corref{cor1}\fnref{label2}}
%% \ead{email address}
%% \ead[url]{home page}
%% \fntext[label2]{}
%% \cortext[cor1]{}
%% \affiliation{organization={},
%%            addressline={}, 
%%            city={},
%%            postcode={}, 
%%            state={},
%%            country={}}
%% \fntext[label3]{}

\title{Identifying Large-Scale Linear Parameter Varying Systems with Dynamic Mode Decomposition Methods}

%

\author[UFSC]{Jean~Panaioti~Jordanou}
\author[UFSC]{Eduardo Camponogara}
\author[Texas]{Eduardo Gildin}


         

\address[UFSC]{Department of Automation and Systems Engineering, Federal University of Santa Catarina, Florianopolis, 88040-900, Santa Catarina, Brazil}
\address[Texas]{Harold Vance Department of Petroleum Engineering, Texas A\&M University, College Station, 77843-3116, Texas, United States of America}         

\begin{abstract}
Linear Parameter Varying (LPV) Systems are a well-established class of nonlinear systems with a rich theory for stability analysis, control, and analytical response finding, among other aspects.
%
Although there are works on data-driven identification of such systems, the literature is quite scarce in terms of works that tackle the identification of LPV models for large-scale systems.
%
Since large-scale systems are ubiquitous in practice, this work develops a methodology for the local and global identification of large-scale LPV systems based on nonintrusive reduced-order modeling.
%
The developed method is coined as DMD-LPV for being inspired in the Dynamic Mode Decomposition (DMD).
To validate the proposed identification method, we identify a system described by a discretized linear diffusion equation, with the diffusion gain defined by a polynomial over a parameter.
%
The experiments show that the proposed method can easily identify a reduced-order LPV model of a given large-scale system without the need to perform identification in the full-order dimension, and with almost no performance decay over performing a reduction, given that the model structure is well-established.

\end{abstract}



\begin{keyword}
%% keywords here, in the form: keyword \sep keyword
Dynamic Mode Decomposition \sep Linear Parameter Varying Systems \sep Model Order Reduction \sep Large-Scale Systems Identification. 
%% PACS codes here, in the form: \PACS code \sep code

%% MSC codes here, in the form: \MSC code \sep code
%% or \MSC[2008] code \sep code (2000 is the default)
\end{keyword}

\end{frontmatter}

%% \linenumbers

%% main text
\section{Introduction}

Despite the remarkable capabilities of large language models (LLMs)~\cite{DBLP:conf/emnlp/QinZ0CYY23,DBLP:journals/corr/abs-2307-09288}, they often inevitably exhibit hallucinations due to incorrect or outdated knowledge embedded in their parameters~\cite{DBLP:journals/corr/abs-2309-01219, DBLP:journals/corr/abs-2302-12813, DBLP:journals/csur/JiLFYSXIBMF23}.
Given the significant time and expense required to retrain LLMs, there has been growing interest in \emph{model editing} (a.k.a., \emph{knowledge editing})~\cite{DBLP:conf/iclr/SinitsinPPPB20, DBLP:journals/corr/abs-2012-00363, DBLP:conf/acl/DaiDHSCW22, DBLP:conf/icml/MitchellLBMF22, DBLP:conf/nips/MengBAB22, DBLP:conf/iclr/MengSABB23, DBLP:conf/emnlp/YaoWT0LDC023, DBLP:conf/emnlp/ZhongWMPC23, DBLP:conf/icml/MaL0G24, DBLP:journals/corr/abs-2401-04700}, 
which aims to update the knowledge of LLMs cost-effectively.
Some existing methods of model editing achieve this by modifying model parameters, which can be generally divided into two categories~\cite{DBLP:journals/corr/abs-2308-07269, DBLP:conf/emnlp/YaoWT0LDC023}.
Specifically, one type is based on \emph{Meta-Learning}~\cite{DBLP:conf/emnlp/CaoAT21, DBLP:conf/acl/DaiDHSCW22}, while the other is based on \emph{Locate-then-Edit}~\cite{DBLP:conf/acl/DaiDHSCW22, DBLP:conf/nips/MengBAB22, DBLP:conf/iclr/MengSABB23}. This paper primarily focuses on the latter.

\begin{figure}[t]
  \centering
  \includegraphics[width=0.48\textwidth]{figures/demonstration.pdf}
  \vspace{-4mm}
  \caption{(a) Comparison of regular model editing and EAC. EAC compresses the editing information into the dimensions where the editing anchors are located. Here, we utilize the gradients generated during training and the magnitude of the updated knowledge vector to identify anchors. (b) Comparison of general downstream task performance before editing, after regular editing, and after constrained editing by EAC.}
  \vspace{-3mm}
  \label{demo}
\end{figure}

\emph{Sequential} model editing~\cite{DBLP:conf/emnlp/YaoWT0LDC023} can expedite the continual learning of LLMs where a series of consecutive edits are conducted.
This is very important in real-world scenarios because new knowledge continually appears, requiring the model to retain previous knowledge while conducting new edits. 
Some studies have experimentally revealed that in sequential editing, existing methods lead to a decrease in the general abilities of the model across downstream tasks~\cite{DBLP:journals/corr/abs-2401-04700, DBLP:conf/acl/GuptaRA24, DBLP:conf/acl/Yang0MLYC24, DBLP:conf/acl/HuC00024}. 
Besides, \citet{ma2024perturbation} have performed a theoretical analysis to elucidate the bottleneck of the general abilities during sequential editing.
However, previous work has not introduced an effective method that maintains editing performance while preserving general abilities in sequential editing.
This impacts model scalability and presents major challenges for continuous learning in LLMs.

In this paper, a statistical analysis is first conducted to help understand how the model is affected during sequential editing using two popular editing methods, including ROME~\cite{DBLP:conf/nips/MengBAB22} and MEMIT~\cite{DBLP:conf/iclr/MengSABB23}.
Matrix norms, particularly the L1 norm, have been shown to be effective indicators of matrix properties such as sparsity, stability, and conditioning, as evidenced by several theoretical works~\cite{kahan2013tutorial}. In our analysis of matrix norms, we observe significant deviations in the parameter matrix after sequential editing.
Besides, the semantic differences between the facts before and after editing are also visualized, and we find that the differences become larger as the deviation of the parameter matrix after editing increases.
Therefore, we assume that each edit during sequential editing not only updates the editing fact as expected but also unintentionally introduces non-trivial noise that can cause the edited model to deviate from its original semantics space.
Furthermore, the accumulation of non-trivial noise can amplify the negative impact on the general abilities of LLMs.

Inspired by these findings, a framework termed \textbf{E}diting \textbf{A}nchor \textbf{C}ompression (EAC) is proposed to constrain the deviation of the parameter matrix during sequential editing by reducing the norm of the update matrix at each step. 
As shown in Figure~\ref{demo}, EAC first selects a subset of dimension with a high product of gradient and magnitude values, namely editing anchors, that are considered crucial for encoding the new relation through a weighted gradient saliency map.
Retraining is then performed on the dimensions where these important editing anchors are located, effectively compressing the editing information.
By compressing information only in certain dimensions and leaving other dimensions unmodified, the deviation of the parameter matrix after editing is constrained. 
To further regulate changes in the L1 norm of the edited matrix to constrain the deviation, we incorporate a scored elastic net ~\cite{zou2005regularization} into the retraining process, optimizing the previously selected editing anchors.

To validate the effectiveness of the proposed EAC, experiments of applying EAC to \textbf{two popular editing methods} including ROME and MEMIT are conducted.
In addition, \textbf{three LLMs of varying sizes} including GPT2-XL~\cite{radford2019language}, LLaMA-3 (8B)~\cite{llama3} and LLaMA-2 (13B)~\cite{DBLP:journals/corr/abs-2307-09288} and \textbf{four representative tasks} including 
natural language inference~\cite{DBLP:conf/mlcw/DaganGM05}, 
summarization~\cite{gliwa-etal-2019-samsum},
open-domain question-answering~\cite{DBLP:journals/tacl/KwiatkowskiPRCP19},  
and sentiment analysis~\cite{DBLP:conf/emnlp/SocherPWCMNP13} are selected to extensively demonstrate the impact of model editing on the general abilities of LLMs. 
Experimental results demonstrate that in sequential editing, EAC can effectively preserve over 70\% of the general abilities of the model across downstream tasks and better retain the edited knowledge.

In summary, our contributions to this paper are three-fold:
(1) This paper statistically elucidates how deviations in the parameter matrix after editing are responsible for the decreased general abilities of the model across downstream tasks after sequential editing.
(2) A framework termed EAC is proposed, which ultimately aims to constrain the deviation of the parameter matrix after editing by compressing the editing information into editing anchors. 
(3) It is discovered that on models like GPT2-XL and LLaMA-3 (8B), EAC significantly preserves over 70\% of the general abilities across downstream tasks and retains the edited knowledge better.


\section{Related Work}
Our work draws on and contributes to research in mobility aids and the built environment, online image-based survey for urban assessment, personalized routing applications and accessibility maps.

\subsection{Mobility Aids and the Built Environment}
People who use mobility aids (\textit{e.g.,} canes, walkers, mobility scooters, manual wheelchairs and motorized wheelchairs) face significant challenges navigating their communities.
Studies have repeatedly found that sidewalk conditions can significantly impede mobility among these users~\cite{bigonnesse_role_2018,fomiatti_experience_2014,f_bromley_city_2007,rosenberg_outdoor_2013, harris_physical_2015,korotchenko_power_2014}. 
In a review of the physical environment's role in mobility, \citet{bigonnesse_role_2018} summarized factors affecting mobility aid users, including uneven or narrow sidewalks (\textit{e.g.,}~\cite{fomiatti_experience_2014,f_bromley_city_2007}), rough pavements (\textit{e.g.,}~\cite{fomiatti_experience_2014,f_bromley_city_2007}), absent or poorly designed curb ramps (\textit{e.g.,}~\cite{rosenberg_outdoor_2013, f_bromley_city_2007, korotchenko_power_2014}), lack of crosswalks (\textit{e.g.,}~\cite{harris_physical_2015}), and various temporary obstacles (\textit{e.g.,}~\cite{harris_physical_2015}).

Though most research on mobility disability and the built environment has focused on wheelchair users~\cite{bigonnesse_role_2018}, mobility challenges are not experienced uniformly across different user populations~\cite{prescott_factors_2020, bigonnesse_role_2018}. 
For example, crutch users could overcome a specific physical barrier (such as two stairs down to a street), whereas motorized wheelchair users could not (without a ramp)~\cite{bigonnesse_role_2018}. 
Such variability demonstrates how person-environment interaction can differ based on mobility aids and environmental factors~\cite{sakakibara_rasch_2018,smith_review_2016}.
Further, mobility aids such as canes, crutches, or walkers are more commonly used than wheelchairs in the U.S.~\cite{taylor_americans_2014, firestine_travel_2024}: in 2022, approximately 4.7 million adults used a cane, crutches, or a walker, compared to 1.7 million who used a wheelchair~\cite{firestine_travel_2024}.
This underscores the importance of considering a diverse range of mobility aid users in urban accessibility research.
For example, \citet{prescott_factors_2020} explored the daily path areas of users of manual wheelchairs, motorized wheelchairs, scooters, walkers, canes, and crutches and found that the type of mobility device had a strong association with users' daily path area size.
Our study aims to further advance knowledge of how different mobility aid users perceive sidewalk barriers, with a more inclusive understanding of urban accessibility.

\begin{figure*}
    \centering
    \includegraphics[width=1\linewidth]{figures/figure-tutorial.png}
    \caption{Survey Part 2.1 showed all 52 images and asked participants to rate their passability based on their lived experience and use of their mobility aid. Above is the interactive tutorial we showed at the beginning of this part.}
    \Description{This figure shows a screenshot from the online survey. In survey part 2.1, participants were presented with 52 images and were asked to rate their passibility based on their lived experience and use of their mobility aid. The screenshot shows the interactive tutorial shown before this section.}
    \label{fig:survey-part2-instructions}
\end{figure*}

\subsection{Online Image-Based Survey for Urban Assessment}
Sidewalk barriers hinder individuals with mobility impairments not just by preventing particular travel paths but also by reducing confidence in self-navigating and decreasing one's willingness to travel to areas that might be physically challenging or unsafe~\cite{vasudevan_exploration_2016,clarke_mobility_2008}.
Prior work in this area traditionally uses three main study methods: in-person interviews (\textit{e.g}.~\cite{rosenberg_outdoor_2013,castrodale_mobilizing_2018}), GPS-based activity studies (\textit{e.g.,}~\cite{prescott_exploration_2021, prescott_factors_2020,rosenberg_outdoor_2013}), and online-questionnaires (\textit{e.g.,}~\cite{carlson_wheelchair_2002}). 
In-person interviews, while providing detailed and nuanced information, are limited by small sample sizes~\cite{rosenberg_outdoor_2013}. GPS-based activity studies involve tracking mobility aids user activity over a period of time, offering insights into movement patterns and activity space; however, these studies are constrained by geographical location~\cite{prescott_exploration_2021}. In contrast, online questionnaires can reach much larger populations and cover broader geographical regions, but they often yield high-level information that lacks the depth and nuance of the other approaches~\cite{carlson_wheelchair_2002}.
Our study aims to strike a balance between these approaches, capturing nuanced perspectives of mobility aid users about the built environment while maintaining a sufficiently large enough sample size for robust statistical analysis. 
Building on~\citet{bigonnesse_role_2018}'s work, we explore not only the types of factors considered to be barriers, but the \textit{intensity} of these barriers and their differential impacts.

Visual assessment of environmental features has long been employed by researchers across diverse fields, including human well-being~\cite{humpel_environmental_2002}, ecosystem sustainability~\cite{gobster_shared_2007}, and public policy~\cite{dobbie_public_2013}. 
These studies examine the relationship between images and the reactions they provoke in respondents or compare differences in reactions between groups.
Over the past decade, online visual preference surveys have gained popularity (\textit{e.g.,}~\cite{evans-cowley_streetseen_2014, salesses_collaborative_2013, goodspeed_research_2017}), where respondents are asked to make pairwise comparisons between randomly selected images.
Using this approach has two advantages: it adheres to the law of comparative judgment~\cite{thurstone_law_2017} by allowing respondents to make direct comparisons, and it prevents inter-rater inconsistency possible with scale ratings~\cite{goodspeed_research_2017}.
Additionally, online surveys generally offer advantages of increased sample sizes, reduced costs, and greater flexibility~\cite{wherrett_issues_1999}.
For people with disabilities, online surveys can be particularly beneficial. They help reach hidden or difficult-to-access populations~\cite{cook_challenges_2007,wright_researching_2005} and are believed to encourage more honest answers to sensitive questions~\cite{eckhardt_research_2007} by providing a higher level of anonymity and confidentiality~\cite{cook_challenges_2007, wright_researching_2005}.

\begin{figure*}
    \centering
    \includegraphics[width=1\linewidth]{figures/figure-comaprison-screenshot.png}
    \caption{In survey Part 2.2, participants were asked to perform a series of pairwise comparisons based on their 2.1 responses.}
    \Description{This figure shows a screenshot from the online survey. In Survey Part 2.2, participants were asked to perform a series of pairwise comparisons based on their 2.1 responses.}
    \label{fig:survey-part2b-pairwise}
\end{figure*}

\subsection{Personalized Routing Applications and Accessibility Maps}
Navigation challenges faced by mobility aid users can be mitigated through the provision of routes and directions that guide them to destinations safely, accurately, and efficiently~\cite{kasemsuppakorn_understanding_2015}. However, current commercial routing applications (\textit{e.g.}, \textit{Google Maps}) do not provide sufficient guidance for mobility aid users.
To address this gap, significant research has focused on routing systems for this population over the past two decades~\cite{barczyszyn_collaborative_2018, karimanzira_application_2006, matthews_modelling_2003, kasemsuppakorn_understanding_2015, volkel_routecheckr_2008, holone_people_2008, wheeler_personalized_2020, gharebaghi_user-specific_2021, ding_design_2007}.
One early, well-known prototype system is \textit{MAGUS}~\cite{matthews_modelling_2003}, which computes optimal routes for wheelchair users based on shortest distance, minimum barriers, fewest slopes, and limits on road crossings and challenging surfaces.
\textit{U-Access}~\cite{sobek_u-access_2006} provides the shortest route for people with three accessibility levels: unaided mobility, aided mobility (using crutch, cane, or walker), and wheelchair users.
However, U-Access only considers distance and ignores other
important factors for mobility aid users~\cite{barczyszyn_collaborative_2018}.
A series of projects by Kasemsuppakorn \textit{et al}.~\cite{kasemsuppakorn_personalised_2009, kasemsuppakorn_understanding_2015} attempted to create personalized routes for wheelchair users using fuzzy logic and \textit{Analytic Hierarchy Process} (AHP).

While influential, many personalized routing prototypes face limited adoption due to a scarcity of accessibility data for the built environment. 
Geo-crowdsourcing~\cite{karimi_personalized_2014}, a.k.a. volunteered geographic information (VGI)~\cite{goodchild_citizens_2007}, has emerged as an effective solution~\cite{karimi_personalized_2014, wheeler_personalized_2020}.
In this approach, users annotate maps with specific criteria or share personal experiences of locations, typically using web applications based on Google Maps or \textit{OpenStreetMap} (OSM)~\cite{karimi_personalized_2014}.
Examples include \textit{Wheelmap}~\cite{mobasheri_wheelmap_2017}, \textit{CAP4Access}~\cite{cap4access_cap4access_2014}, \textit{AXS Map}~\cite{axs_map_axs_2012}, and \textit{Project Sidewalk}~\cite{saha_project_2019}.
Recent research demonstrated the potential of using crowdsourced geodata for personalized routing~\cite{goldberg_interactive_2016, bolten_accessmap_2019,menkens_easywheel_2011, neis_measuring_2015}.
For example, \textit{EasyWheel}~\cite{menkens_easywheel_2011}, a mobile social navigation system based on OSM, provides wheelchair users with optimized routing, accessibility information for points of interest, and a social community for reporting barriers. 
\textit{AccessMap}~\cite{bolten_accessmap_2019} offers routing information tailored to users of canes, manual wheelchairs, or powered wheelchairs, calculating routes based on OSM data that includes slope, curbs, stairs and landmarks. 
Our work builds on the above by gathering perceptions of sidewalk obstacles from different mobility aid users to create generalizable profiles based on mobility aid type. We envision that these profiles can provide starting points in tools like Google Maps for personalized routing but can be further customized by the end user to specify additional needs (\textit{e.g.}, ability to navigate hills, \textit{etc.})

Beyond routing applications, our study data can contribute to modeling and visualizing higher-level abstractions of accessibility. 
Similar to \textit{AccessScore}~\cite{li_interactively_2018}, data from our survey can provide personalizable and interactive visual analytics of city-wide accessibility. By identifying both differences between mobility groups and common barriers within groups, we can develop analytical tools to prioritize barriers and assess the impact of their mitigation or removal, potentially benefiting the broadest range of mobility group users. Incorporating perceptions of passibility into urban planning processes provides a new dimension for urban planners' toolkits, which are often narrowly focused on compliance with ADA standards.




\section{Method} \label{section: method}

\begin{figure*}[t]
    \centering
    \includegraphics[width=\textwidth]{figures/overview.pdf}
    \caption{Overview of \ours and cross-token prefetching framework. (a) \textbf{\ours}  formulates the attention history as a spatiotemporal sequence, and predicts the attention at the next step with a pre-trained model. To enhance efficiency, the attention history is updated in a compressed form at each decoding step. (b) \textbf{The cross-token prefetching framework} asynchronously evaluates critical tokens and fetches KV for the next token during the LLM inference, thereby accelerating the decoding stage.}
    \label{fig:prefetch_overview}  
\end{figure*}


In this section, we introduce \ours, the first learning-based method for identifying critical tokens, along with the cross-token prefetch framework for improved cache management. We begin with the problem formulation for attention prediction in Section~\ref{section:formulation}, followed by a description of our novel \ours in Section~\ref{sec:attention_predictor}. Finally, Section~\ref{section: prefetch} presents a cross-token prefetch framework that efficiently hides both evaluation and cache loading latencies.

\subsection{Problem Formulation}
\label{section:formulation}

In the language model decoding stage, we denote $\mathbf{Q}_t \in \mathbb{R}^{1 \times d}$, $\mathbf{K} \in \mathbb{R}^{t \times d}$ as the query tensor and key tensor used for generate token $t$, respectively. Specifically, we denote \( \mathbf{K}_i \in \mathbb{R}^{1 \times d} \), where \( i \in \{1, 2, \dots, t\} \), as the key tensor for token \( i \), and \( \mathbf{K} = \mathbf{K}_{1:t} \) as the complete key tensor. The attention at step $t$ is calculated as:
\begin{equation}
    A_t=\text{Softmax}\left(\frac{1}{\sqrt{d}} \mathbf{Q}_t \mathbf{K}^\top \right), A_t \in \mathbb{R}^{1 \times t}. 
\end{equation}


The sparsity-based KV cache compression seeks to find a subset of keys with budget $B$ that preserves the most important attention values.
Specifically, the set of selectable key positions is $\Gamma=\{\{\mathbf{p}\}=\left\{p_i\right\}_{i=1}^B|p_i\in\{ 1,2,\ldots,t\} ,p_i\neq p_j,\forall i,j=1,2,\ldots,B\}$. 
We define the \textbf{attention recovery rate} as:
\begin{equation}
\label{eq:attention_recovery_score}
R_{rec} = \frac{\sum_{i=0}^{B}{A_{t, p_i}}}{||A_t||_1},
\end{equation}
which reflects the amount of information preserved after compression. A higher recovery rate $R_{rec}$ indicates less information loss caused by KV cache compression.
Therefore, the goal of KV cache compression can be formulated as finding the positions $\mathbf{p}$ that maximize $R_{rec}$, i.e.,
\begin{equation}
\label{eq:find_p}
\underset{\mathbf{p} \in \Gamma }{\max} \,R_{rec}. 
\end{equation}

To determine the positions $\mathbf{p}$, existing methods typically employ heuristic approaches to score the attention at step $t$, represented as $S_t \in \mathbb{R}^{1 \times t}$, and then select the top $B$ positions. 
For example, the well-known method H2O~\citep{zhang2023h2o} accumulates historical attention scores, where $S_t = \sum_{n=1}^{t-1}{A_n}$. 
In this paper, we predict the attention of step $t$ as $\hat{A_t}$ and use it as $S_t$.

After identifying the critical token positions $\mathbf{p}$, the attention is computed sparsely $A^\text{sparse} = \text{Softmax}\left(\frac{1}{\sqrt{d}} \mathbf{Q} {\mathbf{K}^{\text{sparse}}}^\top \right)$, with selected keys $\mathbf{K}^{\text{sparse}} = \text{concate}\{\mathbf{K}_{p_i}\}$.


\subsection{\ours: A Spatiotemporal Predictor}
\label{sec:attention_predictor}

\textbf{Prediction formulation.} We formulate the attention history $A_H \in \mathbb{R}^{t\times t}$ as a spatiotemporal sequence.
The first dimension of $A_H$ corresponds to the time series over the decoding steps,
while the second dimension represents a sparse series over different keys.
We then train a model to predict the attention for step $t$ as $\hat{A}_{t+1} = F(A_H)$, where $F(\cdot)$ denotes the model function.
For efficiency, we limit the time steps of $A_H$ using a hyperparameter $H$, so that the input to the predictor is $A_H \in \mathbb{R}^{H \times t}$.
\begin{figure*}[t]
    \centering
    \includegraphics[width=0.8\textwidth]{figures/prefetch_timeline.pdf}
    \caption{Timeline of our proposed cross-token prefetching. By asynchronously loading the critical KV cache for the next token, our framework hides the token evaluation and transfer latency, accelerating the decoding stage of LLM inference.}
    \label{fig:prefetch_timeline}
\end{figure*}


\textbf{Model design.} To capture spatiotemporal features, we use a convolutional neural network (CNN) composed of two 2D convolution layers followed by a 1D convolution layer. 
The 2D convolutions capture spatiotemporal features at multiple scales, while the 1D convolution focuses on the time dimension, extracting temporal patterns across time steps. By replacing the fully connected layer with a 1D convolution kernel, the model adapts to the increasing spatial dimension, without data segmentation or training multiple models.
Compared to an auto-regressive LSTM~\citep{graves2012lstm}, the CNN is more lightweight and offers faster predictions, maintaining a prediction time shorter than the single-token inference latency. Additionally, when compared to an MLP~\citep{rumelhart1986MLP} on time-series dimension, the CNN is more effective at capturing spatial features, which improves prediction accuracy. 


\textbf{Training strategy.} Our model is both data-efficient and generalizable.
We train the model only on a small subset of attention data, specifically approximately 3\% extracted from the dataset. The model performance on the entire dataset shows our model effectively captures the patterns (see Section \ref{sec:exp_main}). 
Additionally, due to the temporal characteristics of attention inherent in the LLM, a single model can generalize well across various datasets. For example, our model trained on LongBench also performs well on the GSM8K dataset, highlighting the generalization capability of \ours.



\textbf{Block-wise attention compression.}
To speed up prediction, we apply attention compression before computation. 
By taking advantage of the
attention's locality,
\ours predict attention and identify critical tokens in blocks. Inspired by~\citet{tang2024quest}, we use the maximum attention value in each block as its representative.
Specifically, max-pooling is applied on $A$ with a kernel size equal to the block size $b$, as $A_t^{comp} = Maxpooling(A_t,b)$, reducing prediction computation to roughly $\frac{1}{b}$. 

\textbf{Distribution error calibration.}
Due to the sparsity of attention computation, the distribution of attention history $A_H$ used for prediction may deviate from the distribution of dense attention. This deviation tends to accumulate over decoding, particularly as the output length increases. To mitigate this issue and enhance prediction accuracy, we introduce a distribution error calibration technique to correct these deviations. Specifically, we calculate and store the full attention score every $M$ steps, effectively balancing accuracy with computational efficiency.

\textbf{Overall process.}
As shown in \autoref{fig:prefetch_overview} and Algorithm \ref{alg:predict}, \ours prepares an attention history queue in the prefilling stage, and predicts attention during the decoding stage. First, the $A_t$ from the LLM is compressed to $A_t^{comp}$ using block-wise attention compression. Next, $A_H$ is updated with $A_t^{comp}$. The next step attention $\hat{A}_{t+1}$ is then predicted with the pretrained model. From $\hat{A}_{t+1}$, the top-K positions are selected with a budget of $B/b$, since $\hat{A}_{t+1}$ is in compressed form. Finally, the indices are expanded with $b$ to obtain the final critical token positions
$\mathbf{p}$.

\begin{algorithm}[ht!]
   \caption{Identify Critical Tokens}
   \label{alg:predict}
   
    \textbf{Input}: Attention scores $A_t$, Attention history $A_H$, Block size $b$, KV budget $B$
    \\
    \textbf{Output}: Critical KV token positions $\mathbf{p}$
    
    \begin{algorithmic}[1]
    \STATE Pad $A_t$ to the nearest multiple of $b$ with zero
    \STATE $A_t^{comp} \gets \text{MaxPooling}(A_t, b)$
    \STATE $A_H \gets \text{Update}(A_h, A_t^{comp})$
    \STATE $\hat{A}_{t+1} \gets \text{Prediction model}(A_H)$
    \STATE $\text{Positions} \gets \text{Top-K}(\hat{A}_{t+1}, B / b)$
    \STATE $\mathbf{p} \gets \text{Expand}(\text{Positions}, b)$ \\
    \textbf{Return} positions $\mathbf{p} $
    \end{algorithmic}
\end{algorithm}


\subsection{KV Cache Cross-token Prefetching} \label{section: prefetch}

To address the increased memory cost of longer contexts, current LLM systems offload the KV cache to the CPU, but I/O transfer latency becomes the new significant bottleneck in inference. KV cache prefetching offers a solution by asynchronously loading important cache portions in advance, hiding retrieval time. We introduce the cross-token KV cache prefetching framework, which differs from the cross-layer method in Infinigen \citep{lee2024infinigen} by leveraging longer transfer times and enhancing data integration. 
Specifically, our implementation involves a prefetching process for each layer. As illustrated in Figure \ref{fig:prefetch_overview}, during the prefill phase, the computed KV cache is completely offloaded to the CPU without compression. Then, \ours forecasts the critical token indices $\mathbf{p}$ for the next step. The framework then prefetches the KV cache with $\mathbf{p}$ for the next step onto the GPU. Concurrently, the GPU processes inference for other layers, so the maximum time available for prediction and cache loading corresponds to the inference time per token. Subsequently, the GPU utilizes the query for the next step along with the prefetched partial KV cache to calculate the sparse attention. The attention history is then updated with the newly computed attention scores. The timeline of cross-token prefetching can be seen in \autoref{fig:prefetch_timeline}.





\section{Experiments}
\seclabel{experiments}
Our experiments are designed to test a) the extent to which open loop execution is an issue for precise mobile manipulation tasks, b) how effective are blind proprioceptive correction techniques, c) do object detectors and point trackers perform reliably enough in wrist camera images for reliable control, d) is occlusion by the end-effector an issue and how effectively can it be mitigated through the use of video in-painting models, and e) how does our proposed \name methodology compare to large-scale imitation learning? 


\subsection{Tasks and Experimental Setup}
We work with the Stretch RE2 robot. Stretch RE2 is a commodity mobile manipulator with a 5DOF arm mounted on top of a non-holomonic base. We upgrade the robot to use the Dex Wrist 3, which has an eye-in-hand RGB-D camera (Intel D405). 
We consider 3 task families for a total
of 6 different tasks: a) holding a knob to pull open a cabinet or drawer, b) holding a
handle to pull open a cabinet, and c) pushing on objects (light buttons, books
in a book shelf, and light switches). Our focus is on generalization. {\it
Therefore, we exclusively test on previously unseen instances, not used during
development in any way.} 
\figref{tasks} shows the instances that we test on. 

All tasks involve some precise manipulation, followed by execution of a motion
primitive. {\bf For the pushing tasks}, the precise motion is to get the
end-effector exactly at the indicated point and the motion primitive is to push
in the direction perpendicular to the surface and retract the end-effector 
upon contact. The robot is positioned such
that the target position is within the field of view of the wrist camera. A user
selects the point of pushing via a mouse click on the wrist camera image. The
goal is to push at the indicated location. Success is determined by whether the
push results in the desired outcome (light turns on / off or book gets pushed in). 
The original rubber gripper bends upon contact, we use a rigid known tool
that sticks out a bit. We take the geometry of the tool into account while servoing.

{\bf For the opening articulated object tasks}, the precise manipulation is grasping the
knob / handle, while the motion primitive is the whole-body motion that opens
the cupboard. Computing and executing this full body motion is difficult. We
adopt the modular approach to opening articulated objects (MOSART) from Gupta \etal~\cite{gupta2024opening} and invoke it
after the gripper has been placed around the knob / handle. The whole tasks 
starts out with the robot about 1.5m way from the target object, with the 
target object in view
from robot's head mounted camera. We use MOSART to compute articulation
parameters and convey the robot to a pre-grasp
location with the target handle in view of the wrist camera. At this point,
\name (or baseline) is used to center the gripper around the knob / handle, 
before resuming MOSART: extending the gripper till contact, close the gripper, and play rest of the predicted motion plan. Success is 
determined by whether the cabinet opens by more than $60^\circ$
or the drawer is pulled out by more than $24cm$, similar to the criteria used in \cite{gupta2024opening}.


For the precise manipulation part, all baselines consume the current and
previous RGB-D images from the wrist camera and output full body motor
commands.

% % Please add the following required packages to your document preamble:
% % \usepackage{graphicx}
% \begin{table*}[!ht]
% \centering
% \caption{}
% \label{tab:my-table}
% \resizebox{\textwidth}{!}{%
% \begin{tabular}{lcccccc}
% \toprule
%  & \multicolumn{2}{c}{ours} & \multicolumn{2}{c}{Gurobi} & \multicolumn{2}{c}{MOSEK} \\
%  & \multicolumn{1}{l}{time (s)} & \multicolumn{1}{l}{optimality gap (\%)} & \multicolumn{1}{l}{time (s)} & \multicolumn{1}{l}{optimality gap (\%)} & \multicolumn{1}{l}{time (s)} & \multicolumn{1}{l}{optimality gap (\%)} \\ \hline
% \begin{tabular}[c]{@{}l@{}}Linear Regression\\ Synthetic \\ (n=16000, p=16000)\end{tabular} & 57 & 0.0 & 3351 & - & 2148 & - \\ \hline
% \begin{tabular}[c]{@{}l@{}}Linear Regression\\ Cancer Drug Response\\ (n=822, p=2300)\end{tabular} & 47 & 0.0 & 1800 & 0.31 & 212 & 0.0 \\ \hline
% \begin{tabular}[c]{@{}l@{}}Logistic Regression\\ Synthetic\\ (n=16000, p=16000)\end{tabular} & 271 & 0.0 & N/A & N/A & 1800 & - \\ \hline
% \begin{tabular}[c]{@{}l@{}}Logistic Regression\\ Dorothea\\ (n=1150, p=91598)\end{tabular} & 62 & 0.0 & N/A & N/A & 600 & 0.0 \\
% \bottomrule
% \end{tabular}%
% }
% \end{table*}

% Please add the following required packages to your document preamble:
% \usepackage{multirow}
% \usepackage{graphicx}
\begin{table*}[]
\centering
\caption{Certifying optimality on large-scale and real-world datasets.}
\vspace{2mm}
\label{tab:my-table}
\resizebox{\textwidth}{!}{%
\begin{tabular}{llcccccc}
\toprule
 &  & \multicolumn{2}{c}{ours} & \multicolumn{2}{c}{Gurobi} & \multicolumn{2}{c}{MOSEK} \\
 &  & time (s) & opt. gap (\%) & time (s) & opt. gap (\%) & time (s) & opt. gap (\%) \\ \hline
\multirow{2}{*}{Linear Regression} & \begin{tabular}[c]{@{}l@{}}synthetic ($k=10, M=2$)\\ (n=16k, p=16k, seed=0)\end{tabular} & 79 & 0.0 & 1800 & - & 1915 & - \\ \cline{2-8}
 & \begin{tabular}[c]{@{}l@{}}Cancer Drug Response ($k=5, M=5$)\\ (n=822, p=2300)\end{tabular} & 41 & 0.0 & 1800 & 0.89 & 188 & 0.0 \\ \hline
\multirow{2}{*}{Logistic Regression} & \begin{tabular}[c]{@{}l@{}}Synthetic ($k=10, M=2$)\\ (n=16k, p=16k, seed=0)\end{tabular} & 626 & 0.0 & N/A & N/A & 2446 & - \\ \cline{2-8}
 & \begin{tabular}[c]{@{}l@{}}DOROTHEA ($k=15, M=2$)\\ (n=1150, p=91598)\end{tabular} & 91 & 0.0 & N/A & N/A & 634 & 0.0 \\
 \bottomrule
\end{tabular}%
}
% \vspace{-3mm}
\end{table*}

\begin{figure*}
\insertW{1.0}{figures/figure_6_cropped_brighten.pdf}
\caption{{\bf Comparison of \name with the open loop (eye-in-hand) baseline} for opening a cabinet with a knob. Slight errors in getting to the target cause the end-effector to slip off, leading to failure for the baseline, where as our method is able to successfully complete the task.}
\figlabel{rollout}
\end{figure*}

\begin{table}
\setlength{\tabcolsep}{8pt}
  \centering
  \resizebox{\linewidth}{!}{
  \begin{tabular}{lcccg}
  \toprule
                              & \multicolumn{2}{c}{\bf Knobs} & \bf Handle & \bf \multirow{2}{*}{\bf Total} \\
                              \cmidrule(lr){2-3} \cmidrule(lr){4-4}
                              & \bf Cabinets & \bf Drawer & \bf Cabinets & \\
  \midrule
  RUM~\cite{etukuru2024robot}  & 0/3    & 1/4         & 1/3         & 2/10 \\
  \name (Ours) & 2/3    & 2/4         & 3/3     &  7/10 \\
  \bottomrule
  \end{tabular}}
  \caption{Comparison of \name \vs RUM~\cite{etukuru2024robot}, a recent large-scale end-to-end imitation learning method trained on 1200 demos for opening cabinets and 525 demos for opening drawers across 40 different environments. Our evaluation spans objects from three environments across two buildings.}
  \tablelabel{rum}
\end{table}

\subsection{Baselines}
We compare against three other methods for the precise manipulation part of
these tasks. 
\subsubsection{Open Loop (Eye-in-Hand)} To assess the precision requirements of
the tasks and to set it in context with the manipulation capabilities of the
robot platform, this baseline uses open loop execution starting from estimates
for the 3D target position from the first wrist camera image.
\subsubsection{MOSART~\cite{gupta2024opening}}
The recent modular system for opening cabinets and drawers~\cite{gupta2024opening}
reports impressive performance with open-loop control (using the head camera from 1.5m away), combined with proprioception-based feedback to 
compensate for errors in perception and control when interacting with handles. 
We test if such correction is also sufficient for interacting with knobs. Note 
that such correction is not possible for the smaller buttons and pliable books.

\subsubsection{\name (no inpainting)} To understand how much of an issue
occlusion due to the end-effector is during manipulation, we ablate the use of
inpainting. %

\subsubsection{Robot Utility Models (RUM)~\cite{etukuru2024robot}}
For the opening articulated object tasks, we also compare to Robot Utility Models (RUM), 
a closed-loop imitation learning method recently proposed by Etukuru et al. \cite{etukuru2024robot}.
RUM is trained on a substantial dataset comprising expert demonstrations, including 
1,200 instances of cabinet opening and 525 of drawer opening, gathered from roughly 
40 different environments.
This dataset stands as the most extensive imitation 
learning dataset for articulated object manipulation to date, establishing RUM as a 
strong baseline for our evaluation.

Similar to our method, we use MOSART to compute articulation
parameters and convey the robot to a pre-grasp location
with the target handle in view of the wrist camera.
One of the assumptions of RUM is a good view of the handle.
To benefit RUM, we try out three different heights of the wrist camera,
and \textit{report the best result for RUM.}

\begin{figure*}
\insertW{1.0}{figures/figure_9_cropped_brighten.pdf}
\caption{{\bf \name \vs open loop (eye-in-hand) baseline for pushing on user-clicked points}. Slight errors in getting to the target cause failure, where as \name successfully turns the lights off. Note the quality of CoTracker's track ({\color{blue} blue dot}).}
\figlabel{rollout_v2}
\end{figure*}

\begin{figure*}
\insertW{1.0}{figures/figure_5_v2_cropped_brighten.pdf}
\caption{{\bf Comparison of \name with and without inpainting}. Erroneous detection without inpainting causes execution to fail, where as with inpainting the target is correctly detected leading to a successful grasp and a successful execution.}
\figlabel{rollouts2}
\end{figure*}


\subsection{Results}
\tableref{results} presents results from our experiments. 
Our training-free approach \name successfully 
solves over 85\% of task instances that we test on.
As noted, all these
tests were conducted on unseen object instances in unseen
environments that were not used for development in any way. We discuss our key
experimental findings below.

\subsubsection{Closing the loop is necessary for these precise tasks} 
While the proprioception-based strategies proposed in MOSART~\cite{gupta2024opening}
work out for handles, they are inadequate for targets like knobs and just
don't work for tasks like pushing buttons. Using estimates from the wrist
camera is better, but open loop execution still fails for knobs and pushing
buttons. 

\subsubsection{Vision models work reasonably well even on wrist camera images}
Inpainting works well on wrist camera images (see \figref{occlusion} and \figref{inpainting}).
Closing the loop using feedback from vision detectors and point trackers on
wrist camera images also work well, particularly when we use in-painted images.
See some examples detections and point tracks in \figref{rollout} and \figref{rollout_v2}. 
Detic~\cite{zhou2022detecting} was able to reliably detect the knobs and
handles and CoTracker~\cite{karaev2023cotracker} was able to successfully track
the point of interaction letting us solve 24/28 task instances.

\subsubsection{Erroneous detections without inpainting hamper performance on 
handles and our end-effector out-painting strategy effectively mitigates it} 
As shown in \figref{rollouts2}, presence of the end-effector caused the object
detector to miss fire leading to failed execution. Our out painting approach
mitigates this issue leading to a higher success rate than the 
approach without out-painting. Interestingly, CoTracker~\cite{karaev2023cotracker} is quite robust
to occlusion (possibly because it tracks multiple points) and doesn't benefit
from in-painting. 


\subsubsection{Closed-loop imitation learning struggles on novel objects}
As presented in \tableref{rum}, \name significantly outperforms RUM in a paired evaluation on unseen objects across three novel environments. A common failure mode of RUM is its inability to grasp the object's handle, even when it approaches it closely.
Another failure mode we observe is RUM misidentifying keyholes or cabinet edges as handles, also resulting in failed grasp attempts.
These result demonstrate that a modular approach that leverages the broad generalization capabilities of vision foundation models is able to generalize much better than an end-to-end imitation learning approach trained on 1000+ demonstrations, which must learn all aspects of the task from scratch.



\section{Conclusion}

In this paper, we propose a sample weight averaging strategy to address variance inflation of previous independence-based sample reweighting algorithms. 
We prove its validity and benefits with theoretical analyses. 
Extensive experiments across synthetic and multiple real-world datasets demonstrate its superiority in mitigating variance inflation and improving covariate-shift generalization.  


%% The Appendices part is started with the command \appendix;
%% appendix sections are then done as normal sections
%% \appendix

%% \section{}
%% \label{}

%% If you have bibdatabase file and want bibtex to generate the
%% bibitems, please use
%%
%%  \bibliographystyle{elsarticle-harv} 
\bibliographystyle{elsarticle-num-names} 
%%  \bibliographystyle{elsarticle-num} 
\bibliography{references}

%% else use the following coding to input the bibitems directly in the
%% TeX file.


\end{document}

\endinput
%%
%% End of file `elsarticle-template-harv.tex'.
