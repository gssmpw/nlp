
%%%%%%%%%%%%%%%
\section{Conclusion} \label{sec:conclusion}

We formulated and developed a method to identify a non-intrusive reduced order model of LPV systems based on the DMDc framework.
%
The so-called DMD-LPV was evaluated for a discretized linear diffusion equation with its diffusion gain defined by a polynomial over a parameter as a proof of concept type of case study, where the developed methodology was put to the test.
%
We succeeded in performing both global and local identification for such a model, assessing the properties involving the reduction.
%
Future work will involve employing the LPV-DMD in more complex real-world applications.
%
Also, suppose we adapt the local methodology to consider quasi-LPV systems. In that case, we can obtain a general identification model for nonlinear systems if the model structure is known.
%
In the future, working with LPV models in linear fractional transform (LFT) form for identification is a possibility.
%
Another prospect for future work is using black-box identification functions as the parameter function, such as the Radial Basis Function (RBF) method or kernels.
%
Since such methods involve a large library of nonlinear functions, the sheer number of features presents an interesting problem for the LPV-DMD to tackle.


\section*{Acknowledgments}
This work was funded in part by FAPESC (grant 2021TR2265), CAPES  (grant 88882.182533/2011-01), and CNPq (grant 308624/2021-1). 
