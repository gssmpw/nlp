%%%%%%%% ICML 2025 EXAMPLE LATEX SUBMISSION FILE %%%%%%%%%%%%%%%%%

\documentclass{article}

% Recommended, but optional, packages for figures and better typesetting:
\usepackage{microtype}
\usepackage{graphicx}
\usepackage{subfigure}
\usepackage{booktabs} % for professional tables

% hyperref makes hyperlinks in the resulting PDF.
% If your build breaks (sometimes temporarily if a hyperlink spans a page)
% please comment out the following usepackage line and replace
% \usepackage{icml2025} with \usepackage[nohyperref]{icml2025} above.
\usepackage{hyperref}

% Attempt to make hyperref and algorithmic work together better:
\newcommand{\theHalgorithm}{\arabic{algorithm}}

% Use the following line for the initial blind version submitted for review:

\usepackage[accepted]{icml2025}

% If accepted, instead use the following line for the camera-ready submission:
% \usepackage[accepted]{icml2025}

% For theorems and such
\usepackage{amsmath}
\usepackage{amssymb}
\usepackage{mathtools}
\usepackage{amsthm}

\usepackage{wrapfig}
\usepackage{caption}
\usepackage{listings}
\usepackage{adjustbox}
% \usepackage{subcaption}
% if you use cleveref..
\usepackage[capitalize,noabbrev]{cleveref}

%%%%%%%%%%%%%%%%%%%%%%%%%%%%%%%%
% THEOREMS
%%%%%%%%%%%%%%%%%%%%%%%%%%%%%%%%
\theoremstyle{plain}
\newtheorem{theorem}{Theorem}[section]
\newtheorem{proposition}[theorem]{Proposition}
\newtheorem{lemma}[theorem]{Lemma}
\newtheorem{corollary}[theorem]{Corollary}
\theoremstyle{definition}
\newtheorem{definition}[theorem]{Definition}
\newtheorem{assumption}[theorem]{Assumption}
\theoremstyle{remark}
\newtheorem{remark}[theorem]{Remark}

% Todonotes is useful during development; simply uncomment the next line
%    and comment out the line below the next line to turn off comments
%\usepackage[disable,textsize=tiny]{todonotes}
\usepackage[textsize=tiny]{todonotes}
\usepackage{placeins}
\usepackage{siunitx}

% The \icmltitle you define below is probably too long as a header.
% Therefore, a short form for the running title is supplied here:
\icmltitlerunning{Beyond Fine-Tuning: A Systematic Study of Sampling Techniques in Personalized Image Generation}

\begin{document}

\twocolumn[
\icmltitle{Beyond Fine-Tuning: \\ A Systematic Study of Sampling Techniques in \\ Personalized Image Generation}

% It is OKAY to include author information, even for blind
% submissions: the style file will automatically remove it for you
% unless you've provided the [accepted] option to the icml2025
% package.

% List of affiliations: The first argument should be a (short)
% identifier you will use later to specify author affiliations
% Academic affiliations should list Department, University, City, Region, Country
% Industry affiliations should list Company, City, Region, Country

% You can specify symbols, otherwise they are numbered in order.
% Ideally, you should not use this facility. Affiliations will be numbered
% in order of appearance and this is the preferred way.
\icmlsetsymbol{equal}{*}

\begin{icmlauthorlist}
\icmlauthor{Vera Soboleva}{equal,yyy,comp}
\icmlauthor{Maksim Nakhodnov}{equal,comp}
\icmlauthor{Aibek Alanov}{yyy,comp}
% \icmlauthor{Firstname4 Lastname4}{sch}
% \icmlauthor{Firstname5 Lastname5}{yyy}
% \icmlauthor{Firstname6 Lastname6}{sch,yyy,comp}
% \icmlauthor{Firstname7 Lastname7}{comp}
% %\icmlauthor{}{sch}
% \icmlauthor{Firstname8 Lastname8}{sch}
% \icmlauthor{Firstname8 Lastname8}{yyy,comp}
%\icmlauthor{}{sch}
%\icmlauthor{}{sch}
\end{icmlauthorlist}

\icmlaffiliation{yyy}{HSE University, Moscow, Russia}
\icmlaffiliation{comp}{AIRI, Moscow, Russia}
% \icmlaffiliation{sch}{School of ZZZ, Institute of WWW, Location, Country}

\icmlcorrespondingauthor{Aibek Alanov}{alanov.aibek@gmail.com}
% \icmlcorrespondingauthor{Firstname2 Lastname2}{first2.last2@www.uk}

% You may provide any keywords that you
% find helpful for describing your paper; these are used to populate
% the "keywords" metadata in the PDF but will not be shown in the document
\icmlkeywords{Machine Learning, ICML}

\vskip 0.3in
]

% this must go after the closing bracket ] following \twocolumn[ ...

% This command actually creates the footnote in the first column
% listing the affiliations and the copyright notice.
% The command takes one argument, which is text to display at the start of the footnote.
% The \icmlEqualContribution command is standard text for equal contribution.
% Remove it (just {}) if you do not need this facility.

%\printAffiliationsAndNotice{}  % leave blank if no need to mention equal contribution
\printAffiliationsAndNotice{\icmlEqualContribution} % otherwise use the standard text.

% \begin{abstract}
% Personalized text-to-image generation aims to create images tailored to user-defined concepts and textual descriptions. Balancing the fidelity of the learned concept with its ability for generation in various contexts presents a significant challenge. Existing methods often address this through diverse fine-tuning parameterizations and enhanced sampling strategies that integrate superclass trajectories during the diffusion process. While improved sampling offers a cost-effective, training-free solution for enhancing fine-tuned models, systematic analyses of these methods remain limited. Current approaches typically tie sampling strategies to fixed fine-tuning configurations, making it difficult to isolate their impact on generation outcomes. To address this, we systematically analyze sampling strategies beyond fine-tuning, exploring the influence of concept and superclass trajectories on the results. We show that a weighted mix of these trajectories establishes a strong baseline for improving adaptability across contexts. This approach integrates seamlessly with diverse training strategies, providing an efficient framework to optimize fidelity, editability, and computational efficiency in text-to-image generation.
% \end{abstract}

\begin{abstract}
Personalized text-to-image generation aims to create images tailored to user-defined concepts and textual descriptions. Balancing the fidelity of the learned concept with its ability for generation in various contexts presents a significant challenge. Existing methods often address this through diverse fine-tuning parameterizations and improved sampling strategies that integrate superclass trajectories during the diffusion process. While improved sampling offers a cost-effective, training-free solution for enhancing fine-tuned models, systematic analyses of these methods remain limited. Current approaches typically tie sampling strategies with fixed fine-tuning configurations, making it difficult to isolate their impact on generation outcomes. To address this issue, we systematically analyze sampling strategies beyond fine-tuning, exploring the impact of concept and superclass trajectories on the results. Building on this analysis, we propose a decision framework evaluating text alignment, computational constraints, and fidelity objectives to guide strategy selection. It integrates with diverse architectures and training approaches, systematically optimizing concept preservation, prompt adherence, and resource efficiency. The source code can be found
at \href{https://github.com/ControlGenAI/PersonGenSampler}{github.com/ControlGenAI/PersonGenSampler}.
\end{abstract}


\section{Introduction}
\label{sec:intro}


\ps{Challenges of technology scaling}

The growing demand for computing performance has always been met by increasing the number of transistors per chip, which is only possible due to CMOS technology scaling.
However, as we keep pushing the boundaries of technology scaling, we encounter multiple challenges.
Firstly, whenever we transition to a more advanced technology node, the non-recurring cost due to physical design, verification, software, mask sets, and prototyping almost doubles \cite{cost-tech-node}.
As a result, designing a chip in an advanced technology node is only economically viable if the chip is manufactured in vast quantities.
Secondly, many chip components such as I/O drivers, analog circuits, or \gls{srams} have reached their scaling limit.
This means that we cannot shrink these components further, even if we use a more advanced technology with a smaller feature size.
Thirdly, advanced technology nodes suffer from high defect rates, diminishing the yield and inflating the recurring cost.
To tackle these challenges, new chip-design paradigms have been developed.

\ps{Why 2.5D integration?}

One of these new paradigms is 2.5D integration, where multiple silicon dies called chiplets are integrated into the same package.
Once designed, a single chiplet can be reused in multiple 2.5D stacked chips, which increases the ratio of production volume to non-recurring cost.
Another advantage is that multiple chiplets - fabricated in different technologies - can be integrated into the same package.
This means that only components that can take full advantage of technology scaling are built in bleeding-edge technologies.
Components that have reached their scaling limit are fabricated in more mature and hence less costly technology nodes.
Furthermore, chiplets are smaller than monolithic chips.
Therefore, manufacturing chiplets results in less silicon area loss due to fabrication defects and hence a higher yield.
Due to these economic advantages, chip vendors such as AMD \cite{amd-chiplet} and NVIDIA \cite{chiplet-book} have adopted the 2.5D integration paradigm.  

\ps{Challenges of 2.5D integration}

An important challenge when designing 2.5D stacked chips is the construction of a low-latency and high-throughput \gls{ici}. 
To build an \gls{ici}, we connect different chiplets using \gls{d2d} links.
These links are fabricated in an organic package substrate, silicon bridge, or silicon interposer, and they are connected to the chiplets using \gls{c4} bumps or microbumps.
The number of bumps per chiplet is limited, and so is the bandwidth of \gls{d2d} links.
In addition to having lower bandwidth than links in monolithic chips, \gls{d2d} links also have higher latency.
This latency is caused by wire delay and by \gls{phys} that are necessary in both the sending and the receiving chiplet.
\gls{phys} are needed to convert between protocols, voltage levels, and frequencies, which are usually different between on-chiplet links and \gls{d2d} links.
Due to these limitations, the \gls{ici} can quickly become a bottleneck.

\ps{How we solve these challenges differently than the related work does.}

Existing approaches to maximize the performance of the \gls{ici} either optimize the placement of chiplets (with potentially heterogeneous shapes) for a predetermined \gls{ici} topology 
\cite{ho,liu,seemuth,eris,osmolovskyi,tap25d,chiou}, select one topology out of a set of candidates \cite{coskun-1, coskun-2}, or they optimize the \gls{ici} topology for a 2D grid of homogeneously shaped chiplets on an active interposer \cite{butterdonut, cluscross, kite}.
To the best of our knowledge, there is no prior work on \gls{ici} topologies for chips with heterogeneously shaped chiplets or with passive silicon interposers or silicon bridges.
To fill this gap, we propose \name, a novel optimization methodology to jointly optimize the chiplet placement and \gls{ici} topology of such architectures.
\ifnb
\else
\newpage
\fi

\ps{Details on \name~and the key idea}

The key idea is as follows: 
We optimize the chiplet placement without a predetermined topology.
For each placement generated by an optimization algorithm, we infer a placement-based \gls{ici} topology by connecting chiplets that are in close proximity in that specific placement.
We then compute the latency and throughput of this combination of placement and topology for different traffic types.
These latencies and throughputs together with the total chip area are used to compute a user-defined quality-score of the placement, which is returned to the optimization algorithm.
Based on this quality score, the algorithm can further optimize the placement.
By following this iterative process, we jointly optimize the chiplet placement and the \gls{ici} topology.

\ps{Short evaluation-summary}

We provide our open-source framework implementing the proposed placement and topology co-optimization methodology, which we evaluate using both synthetic traffic and traffic traces.
A 2D grid of chiplets with a mesh topology is used as a baseline since many proposals for 2.5D stacked chips \cite{dataflow_accel_dnn, cifher, simba, hecaton, dojo} use such an architecture.
We reduce the latency of synthetic L1-to-L2 and L2-to-memory traffic, the two most important traffic types for cache coherency traffic, by up to 28\% and 62\% respectively.
For real traffic traces, we reduce the average packet latency for almost all traces and architectures considered (reduced by an 8\% or 18\% on average depending on the configuration of \gls{phys} within a chiplet).

%\section{Preliminaries}\
\subsection{Preliminary: Hausdorff Distance-based Penetration Depth}
\label{sec:preliminary}

%In this section, we introduce the foundational concept of Hausdorff distance-based penetration depth algorithms, which are crucial for understanding our proposed ray tracing-based penetration depth methods.
%and ray tracing core technology, 

%\subsection{Hausdorff distance-based penetration depth}

The Hausdorff distance, as defined in Eq.~\ref{equation:hausdorff_definition}, plays a pivotal role in our approach.
Consider $A$ and $B$ as sets of vertices forming each \revision{object}, and let $d(\cdot, \cdot)$ denote the Euclidean distance between any two vertices.

%equation
\begin{equation}
    H(A,B) = \max \left( \max_{a \in A} \min_{b \in B} d(a,b) ,
    \max_{b \in B} \min_{a \in A} d(b,a) \right)
    \label{equation:hausdorff_definition}
\end{equation}
%

To compute penetration depth using the Hausdorff distance, the process involves several steps.
First, the overlapping volume $V$ between \revision{objects} $A$ and $B$ is computed.
Next, the surfaces of the overlapping volume, $\partial A$ and $\partial B$, contained within each \revision{object}, are extracted.
The final step involves computing the Hausdorff distance $H(\partial A, \partial B)$ between these surfaces.
The resulting distance $H(\partial A, \partial B)$ represents the penetration depth between the two objects.

%In this work, we introduce methods that utilize RT core technology for efficiently computing each step of the Hausdorff distance-based penetration depth calculation.

%\paragraph{BVH-Based Hausdorff Distance Computation:}
The brute-force computation of Hausdorff distance has a time complexity of $O(nm)$, where $n$ and $m$ represent the number of vertices in the two objects, as it requires evaluating all vertex pairs.
To reduce this computational burden, Bounding Volume Hierarchy (BVH) is employed, offering rapid localization of target polygons for distance assessment.
%
Tang et al.~\cite{SIG09HIST} constructed BVHs for two objects, $A$ and $B$.
Their approach begins by computing the Hausdorff distance from $A$ to $B$ (denoted as $h(A,B) = \max_{a \in A} \min_{b \in B} d(a,b)$) through a depth-first traversal of $BVH_A$.
Leveraging the property $h(A', B) \leq h(A, B) \leq h(A, B')$, where $A' \subseteq A$, this step determines the upper bound of the Hausdorff distance, $\overline{h}(A,B)$.
Subsequently, the lower bound $\underline{h}(A,B)$ is determined using $h(B,A)$.
This process yields an approximate Hausdorff distance bound satisfying $\overline{h}(A, B) - \underline{h}(A, B) \leq \epsilon$.
%
Building upon this method, Zheng et al.~\cite{zheng2022economic} implemented a four-point strategy, sampling four points on triangles (three vertices and one center point) to enable more efficient BVH traversal.
This approach is based on the observation that computing the distance between a triangle and a point is computationally less expensive than computing the distance between two triangles.

\revision{These prior methods, which focused on reducing the search space, achieved significant performance improvements for Hausdorff distance computation.
However, they are not well-suited for parallel processing on GPUs, as they require synchronization for updating and sharing the upper and lower bounds.}

\revision{Departing from previous methods, our approach exploits parallel processing on GPUs while leveraging the intrinsic capabilities of RT cores, with a unique emphasis on the ray-triangle intersection test, which is significantly accelerated by the RT core.
In alignment with Tang et al.~\cite{SIG09HIST}, our method approximates the Hausdorff distance.
However, we place greater emphasis on a ray sampling strategy designed to balance accuracy and performance.
Additionally, while previous methods focused on Hausdorff distance computation and presented penetration depth as an application, we accelerate the entire penetration depth computation process on the GPU.}

%Our approach, leveraging the intrinsic capabilities of the RT-core, also employs BVH, but with a unique emphasis on the ray-triangle intersection test, significantly accelerated by the RT-core.
%This strategy represents a departure from previous BVH-based Hausdorff methods, which predominantly focused on reducing search space by computing upper and lower bounds that requires synchronization process for updating the bounds while it is not appropriate for parallel processing.

%improving traversal efficiency during distance computation.

%This method has achieved a performance improvement of up to 20 times compared to Tang et al.'s technique~\cite{SIG09HIST}.

%This strategy involves sampling four points on triangles (three vertices and one center point) to facilitate BVH traversal.


%General algorithms for calculating the Hausdorff distance, when implemented without any additional culling strategies, exhibit a time complexity of at least $O(n^2)$ for triangular meshes~\cite{}. In the huge scenario, there is too much time cost.
%Tang et al. ~\cite{SIG09HIST} proposed the BVH-based framework for reducing the time complexity.
%They build the BVH(Bounding volume hierarchy) for model $A$ and $B$ and traverse the BVH tree of model $A$ in depth-first order. At the first traverse sequence, this algorithm gets sorted by the upper bound of the Hausdorff distance to Model $B$. Since $h(A', B) \leq h(A, B) \leq h(A, B')$ when $A' \subseteq A$ and $B' \subseteq B$, they use this upper bound to compute lower bound when reduction process. In the reduction process, they find the suitable primitive triangles to reduce the Hausdorff distance bound $\overline{h}(A, B) - \underline{h}(A, B) \leq \epsilon$. Finally, they achieve the approximated Hausdorff distance $h(A, B)$ in error bound $\epsilon$.

%Tang's method, Zheng's method
%\subsubsection{Acceleration of Hausdorff distance}

%\subsection{Ray-Tracing Core}
%\DS{Will move to realted work section}

%The Ray-Tracing Core (RT-core) is a specialized hardware component developed by NVIDIA, specifically designed for accelerating ray tracing-based rendering.
%This technology is a key feature of NVIDIA's RTX platform GPUs, such as the GeForce RTX series.
%At the heart of ray tracing is the fundamental operation of intersection checking between rays and objects.
%The RTX platform enhances this process by providing RT-core-based functionalities for efficient ray-bounding box and ray-triangle intersection tests.

%To exploit this powerful hardware for penetration depth computation, we designed RT-core based algorithms for two core tasks: firstly, to extract the overlapping region between two objects, and secondly, to calculate the Hausdorff distance for penetration depth computation.

%The ray tracing core(RT-core) is a specialized hardware on the RTX platform GPU (e.g., GeForce RTX) that was developed by NVIDIA for acceleration of ray tracing-based rendering.
%The basic operation of ray tracing is an intersection test between the ray-bounding box or ray-surface, RT core has a powerful performance up to two to four times higher than CUDA-based computation performance in these operations.

%In this work, to compute the penetration depth, we change the two main algorithms to ray-intersection-based operation to use these powerful RTX platforms.

\section{Methods}
\label{sec:method}
Given a model $\varepsilon_{\theta}$, fine-tuned by (\ref{eq:finetuning}) for a specific concept, we can identify two distinct sampling approaches, each maximizing one of the objectives: concept fidelity or editability:

Sampling with concept (Base sampling):
\begin{equation} 
  \label{eq:concept_sampling}
  \tilde{\varepsilon}_{\theta}(p^C) = \varepsilon_{\theta} + \omega(\varepsilon_{\theta}(p^C) - \varepsilon_{\theta}) = \varepsilon_{\theta} + \omega \Delta\varepsilon_{\theta}^{C}
\end{equation}
Sampling with superclass:
\begin{equation}
  \label{eq:superclass_sampling}
  \tilde{\varepsilon}_{\theta}(p^S) = \varepsilon_{\theta} + \omega( \varepsilon_{\theta}(p^S) - \varepsilon_{\theta}) = \varepsilon_{\theta} + \omega \Delta\varepsilon_{\theta}^{S}
\end{equation} 
% Here, $p^C$ represents a concept prompt embedding (for example, \textit{"a V* with a city in the background"}) and $p^S$ indicates a superclass prompt embedding (\textit{"a backpack with a city in the background"}) where the concept token $V^*$ is replaced by a superclass token (\textit{"backpack"}).
Here, $p^C$ represents a concept prompt embedding (for example, \textit{"a V* with a city in the background"}), and $p^S$ indicates a superclass prompt embedding (\textit{"a backpack with a city in the background"}) where the concept token $V^*$ has been replaced by a superclass token (\textit{"backpack"}).

\begin{figure*}[ht!]
  \centering
  \vspace{-0.02in}
  \includegraphics[trim={0 7cm 0 7cm},clip,width=\linewidth]{imgs/sampling_new.pdf}
  \vspace{-0.20in}
  \caption{\textbf{Visualization of Different Sampling Strategies.} (a) Usual sampling with concept reproduces the concept but does not align closely with the text prompt. (b) Generation with superclass effectively captures the context obtained from the prompt but produces a random superclass representative (e.g., dog). (c-d) Mixed and Switching sampling strategies improve context preservation while maintaining the concept's identity.}
  \label{fig:sampling}
  \vspace{-0.20in}
\end{figure*}

The extended fine-tuning of the model \(\varepsilon_{\theta}\) enhances its ability to accurately reproduce the concept generated via~(\ref{eq:concept_sampling}). However, this improvement comes at the cost of overlooking the contextual information supplied by the prompt $P^C$ (see Figure~\ref{fig:sampling}a). Conversely, the generation via~(\ref{eq:superclass_sampling}) ensures the highest alignment with the text prompt, though at the expense of preserving the concept's identity (see Figure~\ref{fig:sampling}b).

% This raises the question of whether we can integrate the two sampling strategies~(\ref{eq:concept_sampling}) and~(\ref{eq:superclass_sampling}) to obtain the optimal balance between the high fidelity of the learned concept identity and its adaptability to various contexts.

This consideration raises the question of whether we can integrate the two sampling strategies~(\ref{eq:concept_sampling}) and~(\ref{eq:superclass_sampling}) to obtain the optimal balance between the high fidelity of the learned concept identity and its adaptability to various contexts.

\subsection{Mixed sampling} \label{sec:mixed_sampling}
One reasonable approach for incorporating superclass into the generation process~\citep{profusion} is to modify the sampling strategy by adding guidance to the superclass prompt (see Figure~\ref{fig:sampling}c):
\begin{align}\label{eq:mixed_sampling}
    \tilde{\varepsilon}^{MX}_{\theta}(p^S, p^C) = \varepsilon_{\theta} + \omega_s \Delta\varepsilon_{\theta}^{S} + \omega_c\Delta\varepsilon_{\theta}^{C}
\end{align}
% By adjusting the ratio between the concept guidance scale $\omega_c$ and the superclass guidance scale $\omega_s$, we can either amplify or diminish the influence of the concept or superclass, thus varying the trade-off between concept and context fidelity. In Figure~\ref{fig:visual}, you can observe how the generated output alters with increasing superclass influence. For instance, in the teapot example, as we raise the superclass guidance scale, the context, which was initially poorly represented through sampling with the concept, gradually becomes more accurate. However, excessive superclass influence may result in a loss of concept identity preservation, as illustrated in the dog example.

Adjusting the ratio between the concept guidance scale $\omega_c$ and the superclass guidance scale $\omega_s$ amplifies or diminishes the influence of the concept or superclass, varying the trade-off between concept and context fidelity. Figure~\ref{fig:visual} shows how the generated output changes with increasing superclass influence. For instance, in the teapot example, as we raise the superclass guidance scale, the context, which was initially poorly represented through sampling with the concept, gradually becomes more accurate. However, excessive superclass influence may reduce concept identity preservation, as shown in the dog example.

\subsection{Switching sampling} \label{sec:switching_sampling}
Another solution for how to combine the superclass sampling trajectory with the concept sampling trajectory is to condition several steps on the superclass prompt embedding $p^S$, then at the \textit{switching step} $t_{sw}$ switch to the concept prompt embedding $p^C$  (see Figure~\ref{fig:sampling}d). In this case~(\ref{eq:sampling}) will be rewritten in the following form:
\begin{align}\label{eq:switching_sampling}
    \tilde{\varepsilon}^{SW}_{\theta}(p^S, p^C, t_{sw}) = \varepsilon_{\theta} +
    \begin{cases}
         \omega\Delta\varepsilon_{\theta}^{S}, & t > T - t_{sw}\\
        \omega\Delta\varepsilon_{\theta}^{C}, &\text{otherwise}
    \end{cases} 
\end{align}
By increasing the \textit{switching step} $t_{sw}$, we can amplify the influence of the superclass and thus improve context preservation.
Up to 10 steps can effectively recover context that has been poorly generated through Base sampling, as demonstrated in the teapot example in Figure~\ref{fig:visual}. Nonetheless, this strategy may result in notable degradation of the concept's identity. The effect of the superclass can be so intense that the concept loses its original attributes and takes on excessive characteristics from the superclass, as evidenced by the dog example in Figure~\ref{fig:visual}.

% This sampling procedure is similar to Photoswap~\cite{photoswap} approach adapted to the personalization task. The main difference is that in switched sampling we take the noise predictions entirely from the superclass trajectory for the first $t_{sw}$ steps, whereas in Photoswap only the self- and cross-attention maps and features are taken from the superclass for the first $t_{sw}$ steps. However, as we show in Section~\ref{sec:experiments}, the results of these two methods are almost indistinguishable.

This sampling procedure is similar to Photoswap~\cite{photoswap} but adapted to the personalization task. The main difference is that switched sampling takes noise predictions entirely from the superclass trajectory for the first $t_{sw}$ steps, whereas Photoswap uses only self- and cross-attention maps, and features are taken from the superclass for the first $t_{sw}$ steps. However, as we show in Section~\ref{sec:experiments}, the results of both methods are almost indistinguishable.

The aforementioned methods can be flexibly combined, we refer to this type of sampling as \textit{multi-stage sampling}:
\begin{align}\label{eq:multistage_sampling}
\tilde{\varepsilon}^{MS}_{\theta}(p^S, p^C) = \varepsilon_{\theta} +
    \begin{cases}
         (\omega_s + \omega_c)\Delta\varepsilon_{\theta}^{S} &t > T - t_{sw}\\[-0pt]
         \omega_s \Delta\varepsilon_{\theta}^{S} + \omega_c\Delta\varepsilon_{\theta}^{C}&\text{otherwise}
    \end{cases}
\end{align}
This combination enables a greater influence of the superclass on the generated output and enhances alignment with the text prompt. However, it is important to consider that as the influence of the superclass increases, the more the concept's identity is lost.

\subsection{Masked sampling}

\begin{figure*}[t!]
  \centering
  \includegraphics[trim={0 6.7cm 0 6.7cm},clip,width=\linewidth]{imgs/visual_new_v2.pdf}
  \vspace{-0.19in}
  \caption{\textbf{Effects of Superclass Influence on Different Sampling Methods.} 
  % For Mixed Sampling, the influence is adjusted by varying the superclass guidance scale $\omega_s = [1.0, 3.5, 5.0]$ with $\omega_c = 7.0 - \omega_s$. For Switching Sampling, we vary the switching step $t_{sw} = [3, 7, 20]$ . For Masked Sampling, the mask is modified by altering the thresholding quantile $q = [0.3, 0.5, 0.9]$. 
  For Mixed Sampling, the influence is adjusted by varying the superclass guidance scale $\omega_s$ with $\omega_c = 7.0 - \omega_s$. For Switching Sampling, we vary the switching step $t_{sw}$ . For Masked Sampling, the mask is modified by altering the concept mask thresholding quantile $q$.
  }
  \label{fig:visual}
  \vspace{-0.19in}
\end{figure*}

Sampling with a superclass prompt hinders the preservation of concept identity, whereas sampling with a concept prompt disrupts contextual adaptation. To address this challenge, restricting the image regions impacted by each sampling approach could be beneficial. This can be effectively achieved through masking.

Suppose at each diffusion step we could obtain a concept mask $M_t$, then we can use it in the Mixed sampling. Specifically, we apply this mask to the concept trajectory, ensuring it only influences relevant regions:
\begin{align}\label{eq:masked_base}
    \varepsilon^{M}_{\theta}(p^S, p^C) = \varepsilon_{\theta} + \omega\Delta\varepsilon_{\theta}^{C} \odot M_t + \omega\Delta\varepsilon_{\theta}^{S} \odot \overline{M_t} 
\end{align}
Moreover, to enhance the alignment between regions inside and outside the mask, and to gently amplify the influence of the superclass within the mask -- especially in cases where prompts alter the object's appearance (like color or outfit) -- we can apply Mixed sampling within the mask:
\begin{align}\label{eq:masked_mixed}
    &\varepsilon^{M}_{\theta}(p^S, p^C) = \varepsilon_{\theta} + \\ &+ \omega_c\Delta\varepsilon_{\theta}^{C} \odot M_t + 
    \omega_s\Delta\varepsilon_{\theta}^{S} \odot M_t +(\omega_c + \omega_s)\Delta\varepsilon_{\theta}^{S} \odot \overline{M_t}  \notag
\end{align}
The generation process starts with Mixed sampling for a limited number of steps, thereby enhancing the robustness of mask generation. Then, we apply masked sampling as described in (\ref{eq:masked_mixed}), using the concept mask $M_t(q)$. This mask is derived by averaging the cross-attention maps associated with the concept identifier token across all U-Net layers and binarizing it using a threshold determined by the quantile $q$:
\begin{equation}\label{eq:masked_sampling}
\tilde{\varepsilon}^{M}_{\theta}(p^S, p^C) = 
    \begin{cases}
         \tilde{\varepsilon}^{MX}_{\theta}(p^S, p^C, \omega_c^0, \omega_s^0), & t > T - t_{sw}\\
         \varepsilon^{M}_{\theta}(p^S, p^C, \omega_c, \omega_s, q),
    &\text{otherwise,}
    \end{cases} 
\end{equation}
where $\varepsilon^{M}_{\theta}(p^S, p^C, \omega_c, \omega_s, q)$ is computed as in (\ref{eq:masked_mixed}). 

Equation~\ref{eq:masked_sampling} summarizes the complete Masked sampling algorithm. Increasing the quantile $q$ reduces the area influenced by the concept, thereby expanding the region impacted by the superclass (see Appendix~\ref{app:cross_attn}) and enhancing the influence of the context, as illustrated in Figure~\ref{fig:visual}.

\subsection{Other approaches}
\textbf{ProFusion} The main contribution of the Profusion~\citep{profusion} sampling method is a novel technique to ensure the concept's preservation combined with Mixed Sampling. A sampling step in this approach consists of the following stages: (1) we predict $x_t \rightarrow \tilde{x}_{t-1}$ through the usual diffusion backward sampling process with concept (2) after that we make a forward diffusion step $\tilde{x}_{t-1}\rightarrow \tilde{x_t}$ (3) finally, we again make a backward step with the Mixed sampling  $\tilde{x_{t}} \rightarrow x_{t-1}$. The first two steps define Fusion Step and have a hyperparameter $r$ that controls its intensity (e.g. the influence on the result). In case $r=0$ we get Mixed sampling.

\textbf{Photoswap} In this method, the author proposes to replace self-attention features, cross-attention maps, and self-attention maps in the concept trajectory with maps from the superclass at several initial steps. Thus, the method has three hyperparameters: (1) $t_{SF}$ the number of initial steps during which the self-attention features are replaced, (2) $t_{CM}$ the same parameter for cross-attention maps, and (3) $t_{SM}$ for self-attention maps.

\subsection{Evaluation protocol for sampling techniques}
The study of sampling methods involves several key steps. 

% The first step is to select a fundamental fine-tune model on the basis of which we can compare different sampling techniques. For each model, we propose constructing a complete Pareto front of the Mixed sampling. We chose Mixed sampling as our baseline because it is the simplest efficient method, characterized by a single hyperparameter.

The first step is to select a fundamental fine-tuned model that will be used as a baseline for comparing different sampling techniques. For each model, we propose to construct the full Pareto front of Mixed sampling, which we selected as our baseline because it is the simplest yet efficient method, defined by a single hyperparameter.

% It is essential to select a model whose Pareto frontier exhibits a sufficiently large length; this allows for a clearer distinction between the varying parameters. Additionally, this front should lie within the optimal balance between concept fidelity and editability comparing to other fine-tuning methods. By doing so, we can examine sampling not only in scenarios where the model performs poorly but also ensure that sampling does not undermine performance in cases where the model excels.

It is crucial to select a model whose Pareto frontier is of sufficient length, enabling a clearer distinction between the varying parameters. Additionally, this frontier should lie within the optimal balance between concept fidelity and editability compared to other fine-tuning methods. This ensures that we can study sampling in scenarios where the model performs poorly while also confirming that it does not degrade performance when the model excels.

\begin{figure*}[ht!]
\centering
\begin{minipage}{.477\textwidth}
  \centering
  \includegraphics[trim={3cm 10cm 3cm 10cm},clip,width=\linewidth]{imgs/multi-stage.pdf}
  \vspace{-0.19in}
  \captionof{figure}{Pareto Frontier curves for Mixed, Switching and Multi-stage Sampling methods. Each Multi-stage sampling curve is generated by fixing the switching step while varying the superclass guidance scale $\omega_s = [1.0, 3.0, 5.0]$.}
  \label{fig:multi-stage}
\end{minipage}%
\hfill
\begin{minipage}{.477\textwidth}
  \centering
  \includegraphics[trim={3cm 10cm 3cm 10cm},clip,width=\linewidth]{imgs/masked.pdf}
  \vspace{-0.19in}
  \captionof{figure}{Pareto frontiers curves for  Masked sampling. Each Masked sampling curve is derived by varying the quantile \( q = [ 0.3, 0.5, 0.7, 0.9 ] \), which controls the mask binarization threshold; \( t_{sw} = 3, \omega_s = 3.5\) are fixed.}
  \label{fig:masked}
\end{minipage}
\vspace{-0.19in}
\end{figure*}

\begin{figure}[h!]
    \includegraphics[trim={3cm 10cm 3cm 10cm},clip,width=\linewidth]{imgs/all_mixed.pdf}
  \vspace{-0.23in}
    \caption{Mixed sampling Pareto frontiers for different fine-tuning methods.}
    \label{fig:all_mixed}
    \vspace{-0.24in}
\end{figure}
Once the base model is chosen, we fix it and proceed to compare different sampling techniques. For each method, we demonstrate its behaviour at different hyperparameter values. We illustrate the optimal points with generation examples and prove our findings with a user study.

It is important to note that the choice of sampling that maximizes editability can be approached in different ways. For example, one option is to use the model weights before fine-tuning, $\theta^{\text{orig}}$, in (\ref{eq:superclass_sampling}) instead of the fine-tuned weights, $\theta$. Additionally, we can vary the superclass prompts. One extreme option is to remove the superclass token entirely, allowing the model to focus solely on the scene's context (e.g., $p^{\hat{S}} = \textit{"with a city in the background"}$). These hyperparameters affect all sampling methods simultaneously. We analyze this dependency in Appendix~\ref{app:hyper_theta}.
\begin{table}
        \renewcommand{\arraystretch}{0.8}
        \begin{threeparttable}
        \begin{tabular}{@{}lcc@{}}
        \toprule
        Parameter & Floating Allegro Hand & Bimanual Robot Arms \\
        \midrule
        % \makecell{Initial object translational \\ perturbation (cm) }& [$\pm1.5$, $\pm1.5$, 0] & [$\pm 5$, $\pm 5$, 0]\\
        % \makecell{Initial object rotational \\ perturbation (rad) }& [0, 0, $\pm 0.3$] & [0, 0, $\pm 0.3$]\\
        Init. obj. trans.  pert. (cm) & [$\pm1.5$, $\pm1.5$, 0] & [$\pm 5$, $\pm 5$, 0]\\
        Init. obj. rot. pert. (rad) & [0, 0, $\pm 0.3$] & [0, 0, $\pm 0.3$]\\
        Object side length (cm) & [5.8, 6.2]  & [28, 32] \\
        Object mass (kg) & [0.1, 0.3]  & [0.25, 0.75]  \\
        Friction coefficients & [0.7, 1.3] &  [0.2, 0.4]  \\
        Task horizon (s) & 25 & 50 / 260  (Panda / iiwa) \\
        \bottomrule
        \end{tabular}
        \end{threeparttable}
        \caption{Ranges of different physical parameters $\theta$. The initial object pose is only perturbed in yaw, x, and y to ensure the object sits stably on the table. }
        \label{tab:domain_randomization}
        \vspace{0.5em}
\end{table}

\begin{figure*}[t]
\centering
\includegraphics[width=1.0\textwidth]{figures/trajopt_unittest.png}
	\caption{\textbf{Trajectory optimization is crucial for generating dynamically feasible trajectories}. (Top) Before trajectory optimization, the kinematically retargeted demos easily lose contact and drive the object out of reach with different physical parameters or slight deviations in object states. (Bottom) Trajectory optimization encourages robots to establish contact with and maintain good manipulability of the object. The tricolor axis indicates the object orientation.}
	\label{fig:trajopt_unittest}
\end{figure*}

\section{Trajectory Optimization Experiments}

While kinematic retargeting of demonstrations might suffice to generate data for simpler manipulation tasks such as pick and place, it often falls short for the more challenging contact-rich tasks requiring frequent contact mode switches and fine-grained actions. In this section, we demonstrate that trajectory optimization is crucial for generating diverse, dynamically feasible contact-rich trajectories on three high-dimensional dexterous manipulation systems: a floating Allegro hand, bimanual iiwa arms, and bimanual Panda arms.

Our data generation framework is agnostic to the choice of the trajectory optimizer. We implement 
% a contact-implicit model predictive controller based on smoothed contact dynamics \cite{suh2024dexterous} and
the cross-entropy method (CEM) \cite{de2005tutorial} to solve \eqref{eq:predictive_control} over a distribution of physical parameters and initial conditions, as specified in Table \ref{tab:domain_randomization}. 
%\russtcomment{Right... the SQP discussion tricked me, but I guess that's only for the retargeting. I thought you had replaced this. In this case, your approach is almost doing RL, but on a policy parameterized as a trajectory... right? why is that better than doing PPO on a small neural net policy, and generating data from that? If you stick with CEM, than this will be your burden of proof, i think?}

\underline{\textbf{Task}} Manipulating the object to a target pose on the table (Fig. \ref{fig:policy_rollouts}). The object is initially placed randomly on the table with an arbitrary face upward. Task success is defined as the object reaching within 3 cm and 0.2 rad of the target pose for the Allegro hand, and within 10 cm and 0.2 rad for the bimanual robot arms.  This task requires long-horizon reasoning of complex multi-contact interactions between the robot and the object. The necessary frequent contact mode switches and high-dimensional action space pose great challenges for traditional model-based planners, while the precise contact interactions require fine-grained control actions. 

\begin{table}
\centering
        \renewcommand{\arraystretch}{0.8}
        \begin{threeparttable}
        \begin{tabular}{@{}lcccc@{}}
        \toprule
        Perturbation & Allegro Hand & iiwa Arms & Panda Arms \\
        \midrule
        Original demo &4 / 24 & 5 / 24 & 6 / 24\\
        Object size & 2 / 24 & 1 / 24 & 4 / 24\\
        % Object mass & 1 / 24& 1 / 24 & \\
        % Friction coefficients & 3 / 24 & 2 / 24 & \\
        Initial object translation & 1 / 24 & 3 / 24 & 2 / 24\\
        Initial object orientation & 2 / 24 & 3 / 24& 3 / 24\\
        \midrule
        Trajectory optimization & 2164 / 3000 & 2252 / 3000 & 2462 / 3000 \\
        \bottomrule
        \end{tabular}
        \end{threeparttable}
        \caption{Success rates of replaying kinematically retargeted trajectories of the 24 original human demos, and trajectory optimization under random perturbations in physical parameters and object initial conditions. }
        \label{tab:kin_success_rate}
\end{table}

% \begin{table}
% \centering
%         \renewcommand{\arraystretch}{0.8}
%         \begin{threeparttable}
%         \begin{tabular}{@{}ccc@{}}
%         \toprule
%         Allegro Hand & iiwa Arms & Panda Arms \\
%         \midrule
%         0.721 & 0.65 & 0.803\\
%         % Task Horizon (s) & 25 & 280 & 50 \\
%         \bottomrule
%         \end{tabular}
%         % \begin{tablenotes}
%         % \itme{*} 
%         % \end{tablenotes}
%         \end{threeparttable}
%         \caption{Success rates of trajectory optimization under random perturbations in physical parameters and object initial conditions. }
%         \label{tab:trajopt_success_rate}
%         \vspace{0.5em}
% \end{table}



% \begin{figure*}[t]
% \centering
% \includegraphics[width=0.7\textwidth]{figures/aug_traj_den.png}
% 	\caption{\textbf{Distribution of object trajectories generated from a single demonstration}. The original demonstration (orange) is locally perturbed and augmented to about 100 dynamically feasible contact-rich trajectories (blue) for each system. The density map represents the object pose distribution of the generated trajectories in the specific 2-dimensional slices.}
%     \label{fig:aug_data_distribution}
% \end{figure*}

% \begin{figure*}[t]
% \centering
% \includegraphics[width=0.9\textwidth]{figures/aug_traj_snapshots.png}
% 	\caption{\textbf{Snapshots of trajectories generated from a single demonstration}. The original demonstration (orange) is locally perturbed and augmented to about 100 dynamically feasible contact-rich trajectories (blue) for each system. The density map represents the object pose distribution of the generated trajectories in the specific 2-dimensional slices.}
%     \label{fig:aug_data_distribution}
% \end{figure*}
\begin{figure*}[t]
\centering
\includegraphics[width=1.0\textwidth]{figures/density_snapshots_aug_traj.png}
	\caption{\textbf{Distribution and snapshots of trajectories generated from a single demonstration.} (a) The original demonstration (orange) is locally perturbed and augmented to about 100 dynamically feasible contact-rich trajectories (blue) for each system. The density map represents the object pose distribution of the generated trajectories in the specific 2-dimensional slices. (b) Snapshots of 30 dynamically feasible trajectories under random physical parameters and object initial poses for bimanual iiwa arms are visualized.}
    \label{fig:aug_data_distribution}
\end{figure*}


\underline{\textbf{Dynamic Feasibility}}
While kinematic motion retargeting can generate visually plausible robot and object trajectories, these trajectories often lack dynamical consistency due to the differences in physical parameters and embodiment between the human demonstrator and the target robot. To illustrate this, we replay the kinematically retargeted trajectories of the original 24 human demos and record the success rates for each system in Table \ref{tab:kin_success_rate}. Furthermore, we randomly sample object sizes and perturbations of initial object poses according to Table \ref{tab:domain_randomization} and roll out the nominal kinematically retargeted trajectories. Some trajectories still succeed under certain perturbations thanks to caging grasps or other strategies that encourage robustness during the human demonstration. For all the systems, the successful rollouts are relatively short, manipulating the object to the goal pose within only 1 or 2 rotations. 
% Notably, the successful trajectories for the iiwa and Panda arms vary significantly, despite being generated from the same initial set of demonstrations.

The low success rate of purely kinematically retargeted trajectories highlights the importance of trajectory optimization for locally refining the demos for the particular embodiments and physical parameters. Before trajectory optimization, the floating Allegro hand lightly touches the cube and easily loses contact when rotating it clockwise (demonstrated in Fig. \ref{fig:trajopt_unittest}a). After trajectory optimization, the hand increases the contact area, establishing a stable grip for rotation. In Fig. \ref{fig:trajopt_unittest}b, similar behavior that encourages contact can be observed for the bimanual iiwa arms: the demo trajectory tries to rotate the box clockwise only using a single arm, while trajectory optimization encourages the other arm to help hold the box and reorient the box more stably. These refinements that encourage contact are particularly helpful when the object is heavier or smaller, or when the friction coefficients are lower than expected. In addition, replaying the kinematically retargeted trajectory often fails when the object pose deviates slightly from the demonstration, driving the object out of reach (visualized in Fig. \ref{fig:trajopt_unittest}c). In contrast, trajectory optimization 
%stabilizes the system in a vicinity around the demonstration, ensuring higher success rates even when the object is perturbed
accounts for the system’s true dynamics and can adjust the robot’s actions accordingly. The success rates of trajectory optimization under random perturbations in physical parameters and object initial conditions for each system are recorded in Table \ref{tab:kin_success_rate}.

\begin{figure*}[t]
    \centering
    \includegraphics[width=0.9\linewidth]{figures/policy_rollouts.png}
    \caption{\textbf{Policy rollouts for different embodiments.} The object manipulation task requires the robots to frequently make and break contact with the object. It also requires precise control of the robot since small deviations in positions can result in missing contact interactions and lead to task failure. } 
    \label{fig:policy_rollouts}
\end{figure*}

\underline{\textbf{Cross-Embodiment Generalization}} We demonstrate that a single set of human demonstrations can be effectively repurposed to generate dynamically consistent, contact-rich trajectories across different robotic embodiments with varying task horizons. Specifically, human demonstrations involving two index fingers manipulating a small cube are retargeted to fixed-base bimanual Kuka LBR iiwa and Franka Emika Panda arms manipulating a larger box (visualized in Fig. \ref{fig:kinematic_retargeting}). This approach addresses key challenges in data collection for contact-rich tasks: directly teleoperating two real robot arms to flip a large box would be both physically demanding and cost-prohibitive due to hardware latency, limited feedback, and the embodiment gap--differences in kinematic structure, degrees of freedom, and workspace between human and robotic arms. In contrast, performing the same task on a smaller scale using human fingers is more intuitive, reduces physical effort, and enables faster, more consistent demonstration collection.

The iiwa and Panda arms differ in contact geometry, velocity limits, and joint constraints, all of which are explicitly modeled within the trajectory optimization framework described in \eqref{eq:predictive_control}. For safe hardware deployment, we enforce conservative velocity limits on the iiwa arms, while only applying soft velocity regularization on the Panda arms in simulation to allow for more aggressive motions.


\underline{\textbf{Data Diversity}} 
Trajectory optimization efficiently augments a single demonstration to a wide distribution of trajectories with locally perturbed physical parameters and initial conditions as visualized in Fig. \ref{fig:aug_data_distribution}. The diverse states in the generated dataset cover a larger training distribution and encourage smoother learned policies, as will be discussed in the next section.
\begin{figure*}[t]
    \centering
    \includegraphics[width=0.9\linewidth]{figures/policy_failure.png}
    \caption{\textbf{Failure cases of baselines.} (a) The baseline policy trained on the original 24 demonstrations for the floating Allegro hand frequently misses contact or gets stuck on the cube. (b-c) The baseline policies for the bimanual robot arms often exhibit jittery motion, resulting in loss of contact, the box being kicked out of reach, or the robot arms running into and getting stuck on the box surface. } 
    \label{fig:policy_failure}
\end{figure*}

\begin{figure}
\centering
\includegraphics[width=0.42\textwidth]{figures/success_rate.png}
	\caption{Success rates of policy evaluation in simulation and hardware. }
	\label{fig:success_rate}
    \vspace*{-0.4cm}
\end{figure}

% \begin{figure}
% \centering
% \includegraphics[width=0.48\textwidth]{figures/jitteriness.png}
% 	\caption{\textbf{Joint angles of bimanual iiwa arms over time. } Each line represents the trajectory of a different joint of the iiwa arms. The policy trained on augmented datasets (b) demonstrates significantly smoother motion compared to the baseline policy (a). }
% 	\label{fig:jitteriness}
% \end{figure}

\begin{figure*}[t]
    \centering
    \includegraphics[width=0.9\linewidth]{figures/hardware_rollout.png}
    \caption{\textbf{Policy rollouts on hardware.} The fixed-base bimanual iiwa arms perform a sequence of coordinated rolling, pitching, and yawing actions to reorient the box to the goal pose. } 
    \label{fig:hardware_rollout}
\end{figure*}

\section{Behavior Cloning Experiments}
We illustrate our framework's capability to efficiently produce diverse, high-quality contact-rich datasets for training behavior cloning policies across multiple robotic platforms, including the floating Allegro hand and the bimanual Panda arms in simulation as well as bimanual iiwa arms on hardware. We show that policies trained on the generated data generalize to a wide distribution of physical parameters and initial conditions, and are much more robust and performant than the ones trained only on the original demonstrations. 
\subsection{Policy Evaluation in Simulation}
\label{subsec:policy_eval_sim}
From only 24 human demonstrations, our data generation pipeline can efficiently generate thousands of dynamically feasible contact-rich trajectories using trajectory optimization. We train state-based diffusion policies \cite{chi2023diffusion} on the 24 original demo trajectories, as well as 500 and 1000 generated trajectories. While our method is compatible with any Behavior Cloning algorithm, we adopt diffusion policies due to its recent success in contact-rich tasks \cite{chi2024universal, zhu2024should, li2024planning}. Fig. \ref{fig:policy_rollouts} visualizes the policy rollouts. We evaluate the performance by conducting 48 policy rollouts for each embodiment in simulation and record the success rates in Fig. \ref{fig:success_rate}. The success criteria are the same as specified in the trajectory optimization experiments.
%For policy evaluation, we visualize the initial states for all evaluation episodes, typical failure cases of baseline policies, and final object pose errors in Fig. \ref{fig:policy_eval}.  
% and validate that the generated data help improve the policy's robustness and generalizability.

\subsubsection{Floating Allegro Hand} 
While the human demonstrator completes the task in approximately 5 seconds on average in the virtual reality environment, the demonstration trajectories are temporally scaled by a factor of 2.5 to ensure smoother, dynamically feasible motions on the floating Allegro hand, which is subject to velocity limits. We define the task horizon as 25 seconds to allow the policy sufficient time to recover from missed contacts and other errors during the execution. The task complexity arises from the 22-dimensional action space of the Allegro hand and the long-horizon nature of the task, which requires a sequence of coordinated rolling, pitching, and yawing actions to reorient the cube to an upright position. These factors together present significant challenges for traditional model-based planners without guidance.

The baseline behavior cloning policy trained on the original set of 24 demonstrations achieves a success rate of $10 / 48 = 21\%$ and exhibits significant jittery behavior when encountering out-of-distribution states. The workspace, characterized by diverse object orientations and translations, is sufficiently large that minor deviations during policy rollouts often drive the trajectory out of the demonstrated distribution. Common failure modes include the Allegro hand repeatedly missing contact with the cube or becoming stuck on its surface while attempting reorientation (visualized in Fig. \ref{fig:policy_failure}a), which often result in the object being trapped in intermediate orientations. In contrast, policies trained on the expanded dataset generated by our pipeline demonstrate a higher likelihood of re-establishing contact with the object after initial misses, resulting in significantly improved success rates up to $39 / 48 = 81\%$.

\begin{figure*}[t]
    \centering
    \includegraphics[width=0.9\linewidth]{figures/hardware_eval.png}
    \caption{\textbf{Policy failure and recovery on hardware.} The baseline policy frequently (a) gets stuck on the box surface when small deviations from the demonstration trajectories occur, and (b) struggles to recover from out-of-distribution states, where the object is never intentionally lifted for accomplishing the task in the generated dataset. Policies trained on augmented datasets (c) sometimes fail due to unmodeled collision geometry, but (d) can recover from undesired sliding by employing firmer grasps found by trajectory optimization. } 
    \label{fig:hardware_eval}
\end{figure*}
\subsubsection{Bimanual Robot Arms}
The baseline policy trained on the original set of 24 human demonstrations achieves a success rate of $27 / 48 = 56\%$ on the bimanual iiwa system. We hypothesize that the restrictive velocity limits encourage more quasi-static behavior, leading to longer trajectories with a higher density of state-action pairs in the training data. In contrast, the baseline policy yields a success rate of $14/48=29\%$ on the bimanual Panda system, likely due to the more dynamic nature of the learned behavior under its looser velocity constraints. Both baseline policies exhibit remarkably jittery motion, frequently kicking the box out of reach, losing contact, or running into and getting stuck on the box surface during reorientation (visualized in Fig. \ref{fig:policy_failure}b and c). Policies trained on the augmented dataset, however, generate significantly smoother trajectories and are capable of re-establishing contact with the object after initial misses, resulting in as high as $44 / 48 = 92\%$ success rates for bimanual iiwa arms and $42 / 48 = 87.5\%$ for bimanual Panda arms. Additionally, the learned policies capture multimodal behaviors observed in the original human demonstrations, such as rotating the box either clockwise or counterclockwise for similar object poses. 


\subsection{Policy Evaluation on Hardware}
We zero-shot deploy the trained policies on hardware for bimanual iiwa arms to flip a 30 cm cubic box on a table (Fig. \ref{fig:hardware_rollout}). An OptiTrack motion capture system is employed to estimate the object pose. The baseline behavior cloning policy only achieves $6/23=26\%$ success rate, with most successful rollouts being relatively short-horizon, involving only 1 or 2 rotations. Common failure modes of the baseline policy include: 1) deviation from the demonstration trajectory, causing the arms to collide with the box surface (Fig. \ref{fig:hardware_eval}a), and 2) significant box sliding during rolling, resulting in the policy encountering out-of-distribution states and failing to recover (Fig. \ref{fig:hardware_eval}b). In contrast, as shown in Fig. \ref{fig:success_rate}b, the policy trained on 500 generated trajectories achieves $17 / 23 = 74\%$ success rate, while the policy trained on 1000 generated trajectories achieves $16/23=70\%$ success rate. Despite occasional box sliding during rolling, these policies demonstrate an improved ability to stabilize the box by using one arm to hold the opposite side more firmly to prevent further sliding (Fig \ref{fig:hardware_eval}d). However, as visualized in Fig \ref{fig:hardware_eval}c, both policies trained on the augmented datasets exhibit failure modes originating from unmodeled collision geometries on iiwa arms, which lead to significant undesired yaw motions of the box during pitch actions.\looseness=-1
\section{Conclusion and future work}
In this study, we examined the ability of LLMs to produce self-generated counterfactual explanations (SCEs).
We design a prompt-based setup for evaluating the efficacy of \SCEs.
Our results show that LLMs consistently struggle with generating valid \SCEs. In many cases model prediction on a \SCE does not yield the same target prediction for which the model crafted the \SCE.
Surprisingly, we find that LLMs put significant emphasis on the context---the prediction on \SCE is significantly impacted by the presence of original prediction and instructions for generating the \SCE.
Based on this empirical evidence, we argue that LLMs are still far from being able to explain their own predictions counterfactually.
Our findings add to similar insights from recent studies on other forms of self-explanations~\cite{lanham2023measuring,tanneru2024quantifying}.



Our work opens several avenues for future work. Inspired by counterfactual data augmentation~\cite{sachdeva2023catfood}, one could include the counterfactual explanation capabilities a part of the LLM training process. This inclusion may enhance the counterfactual reasoning capabilities of the LLM. Follow ups should also explore the effect of prompt tuning, specifically, model-tailored prompts for generating \SCEs. These approaches might lead to better quality \SCEs.


We limited our investigation to open source models of upto 70B parameters. Extending our analysis to larger and more recent models, \eg, DeepSeek R1 671B, and closed source models like OpenAI o3 would be an interesting avenue for future work.

Finally, our experiments were limited to relatively simple tasks: classification and mathematics problems where the solution is an integer. This limitation was mainly due to the fact that it is difficult to automatically judge validity of answers for more open-ended language generation tasks like search and information retrieval. Scaling our analysis to such tasks would require significant human-annotation resources, and is an important direction for future investigations.


\bibliography{beyond_ft_paper}
\bibliographystyle{icml2025}


\newpage
\appendix
\onecolumn

\renewcommand{\thetable}{A\arabic{table}} % Prefix table numbers with 'A'
\renewcommand{\thefigure}{A\arabic{figure}} % Prefix figure numbers with 'A'
\renewcommand{\theequation}{A\arabic{equation}} % Prefix equation numbers with 'A'

\setcounter{table}{0} % Reset table counter
\setcounter{figure}{0} % Reset figure counter
\setcounter{equation}{0} % Reset equation counter

\section*{Appendix}

\section{Optimal Brain Surgeon Derivation}
\label{OBS_ALGORITHM}

In the original setup in OBS, we have a local quadratic model for the loss $L$ given by:
$$
    \delta L = L(w + \delta w) \approx L(w) + \nabla_w L^T \delta w + \frac{1}{2} \delta w^T H \delta w
$$
Since OBS is a pruning-after-training approach, they discarded the 1-st order component. Reducing the expression for saliency as:
$$
    \delta L = \frac{1}{2} \delta w^T H \delta w
$$
To remove a single parameter, the authors of OBS introduced the constraint $e_q^T \delta w + w_q = 0$, with $e_q$ being the $q^{\text{th}}$ canonical basis vector. The pruning is defined as a constrained optimization problem of the form:
$$
    \min_{\delta w \in \mathbb{R^d}} \left( \frac{1}{2} \delta w^T H \delta w\right),
    ~~\text{s.t}~~
    e_q^T \delta w + w_q = 0.
$$
And the choice of which parameter to remove becomes:
$$
    \min_{q \in \mathcal{Q}} \left\{
        \min_{\delta w \in \mathbb{R^d}} \left( \frac{1}{2} \delta w^T H \delta w\right),
        ~~\text{s.t}~~
        e_q^T \delta w + w_q = 0
    \right\}.
$$
To solve the internal problem, we use a Lagrange multiplier $\lambda$ to write the problem as an unconstrained optimization case as follows:
$$
    \mathcal{L}(\delta w, \lambda) =
    \frac{1}{2} \delta w^T H \delta w +
    \lambda(e_q^T \delta w + w_q).
$$
Then, to find the stationary conditions, we compute the partial derivatives with respect to $\delta w$ and $\lambda$, and equate them to 0, obtaining:
$$
    \nabla_{\delta w} \mathcal{L} = 
    H \delta w + \lambda e_q = 0 
    \rightarrow
    \delta w = - \lambda H^{-1} e_q
$$
$$
    \nabla_{\lambda} \mathcal{L} =
    e_q^T \delta w + w_q = 0
    \rightarrow
    e_q^T \delta w = -w_q
$$
With some replacements, we get:
$$
    e_q^T \delta w = -w_q
    \rightarrow
    e_q^T \left( 
        - \lambda H^{-1} e_q
    \right) = -w_q
    \rightarrow
    - \lambda e_q^T H^{-1} e_q = -w_q
    \rightarrow
    \lambda = \frac{w_q}{e_q^T H^{-1} e_q} = \frac{w_q}{[H^{-1}]_{qq}}
$$
$$
    \delta w = - \frac{w_q H^{-1} e_q}{[H^{-1}]_{qq}}
$$
Replacing the expression for $\delta w$ in the saliency expression, we have:
\begin{align*}
    \delta L = \frac{1}{2} \delta w^T H \delta w
    &= \frac{1}{2}\left(
        - \frac{w_q H^{-1} e_q}{[H^{-1}]_{qq}}
    \right)^T
    H
    \left(
        - \frac{w_q H^{-1} e_q}{[H^{-1}]_{qq}}
    \right)
    \nonumber \\
    &= 
    \frac{w_q^2}{2[H^{-1}]_{qq}^2}
    \left(
        H^{-1} e_q
    \right)^T
    H
    \left(
        H^{-1} e_q
    \right)
    \nonumber \\
    &= 
    \frac{w_q^2}{2[H^{-1}]_{qq}^2}
    e_q ^T
    H^{-1}
    e_q
    = 
    \frac{w_q^2}{2[H^{-1}]_{qq}^2}
    [H^{-1}]_{qq}
    = 
    \frac{w_q^2}{2[H^{-1}]_{qq}}
    \nonumber \\
\end{align*}
%------------------------------------------------------------------------------------------------
\newpage
\section{Fisher Brain Surgeon Sensitivity Derivation}
\label{FBSS_ALGORITHM}
As we considered a PBT setting, it is not possible to ignore the first-order term in the local quadratic approximation of the error as it could still be informative. In this case, our model for sensitivity is given by: 
$$
    \delta L = \nabla_w L^T \delta w + \frac{1}{2} \delta w^T H \delta w
$$
The process to remove a single parameter remains similar; the constraint $e_q^T \delta w + w_q = 0$, with $e_q$ is still valid, redefining the optimization problem as:
$$
    \min_{\delta w \in \mathbb{R^d}} \left(
        \nabla_w L^T \delta w +  \frac{1}{2} \delta w^T H \delta w
    \right),
    ~~\text{s.t}~~
    e_q^T \delta w + w_q = 0.
$$
And the choice of which parameter to remove becomes:
$$
    \min_{q \in \mathcal{Q}} \left\{
        \min_{\delta w \in \mathbb{R^d}} \left(
            \nabla_w L^T \delta w + \frac{1}{2} \delta w^T H \delta w
        \right),
        ~~\text{s.t}~~
        e_q^T \delta w + w_q = 0
    \right\}.
$$
Using a Lagrange multiplier $\lambda$ as in the reference case, we solve the following unconstrained optimization problem:
$$
    \mathcal{L}(\delta w, \lambda) =
    \nabla_w L^T \delta w + 
    \frac{1}{2} \delta w^T H \delta w +
    \lambda(e_q^T \delta w + w_q).
$$
With the following stationary conditions:
$$
    \nabla_{\delta w} \mathcal{L} = 
    \nabla_w L + H \delta w + \lambda e_q = 0 
    \rightarrow
    \delta w = - (\lambda H^{-1}e_q + H^{-1} \nabla_w L)
$$
$$
    \nabla_{\lambda} \mathcal{L} =
    e_q^T \delta w + w_q = 0
    \rightarrow
    e_q^T \delta w = -w_q
$$
The expression for $\lambda$ is redefined as follows:
\begin{align*}
    e_q^T \left(
        - (\lambda H^{-1}e_q + H^{-1} \nabla_w L)
    \right) 
    &= -w_q
    \nonumber \\
    \lambda e_q^T H^{-1} e_q + e_q^T H^{-1} \nabla_w L
    &= w_q
    \nonumber \\
    \lambda [H^{-1}]_{qq} 
    &= w_q - e_q^T H^{-1} \nabla_w L
    \nonumber \\
    \lambda
    &= \frac{w_q - e_q^T H^{-1} \nabla_w L}{[H^{-1}]_{qq}}
\end{align*}
Replacing the expression for $\delta w$ in our sensitivity expression, we have:
\begin{align*}
    \delta L = \nabla_w L^T \delta w + \frac{1}{2} \delta w^T H \delta w
    &= 
    \nabla_w L^T \left[
        - (\lambda H^{-1}e_q + H^{-1} \nabla_w L)
    \right]
    \nonumber \\
    &+
    \frac{1}{2}\left[
        - (\lambda H^{-1}e_q + H^{-1} \nabla_w L)
    \right]^T
    H
    \left[
        - (\lambda H^{-1}e_q + H^{-1} \nabla_w L)
    \right]
    \nonumber \\
    &= 
    - \lambda \nabla_w L^T H^{-1}e_q - \nabla_w L^T H^{-1} \nabla_w L
    \nonumber \\
    &+
    \frac{1}{2}\left[
        (\lambda H^{-1}e_q)^T + (H^{-1} \nabla_w L)^T
    \right]
    \left[
        \lambda H H^{-1}e_q + H H^{-1} \nabla_w L)
    \right]
    \nonumber \\
    &= 
    - \lambda \nabla_w L^T H^{-1}e_q - \nabla_w L^T H^{-1} \nabla_w L
    \nonumber \\
    &+
    \frac{1}{2}\left[
        (\lambda H^{-1}e_q)^T + (H^{-1} \nabla_w L)^T
    \right]
    \left[
        \lambda e_q + \nabla_w L
    \right]
    \nonumber \\
    &= 
    - \lambda \nabla_w L^T H^{-1}e_q - \nabla_w L^T H^{-1} \nabla_w L
    \nonumber \\
    &+
    \frac{1}{2}\left[
        (\lambda H^{-1}e_q)^T \lambda e_q
        + (H^{-1} \nabla_w L)^T \lambda e_q
        + (\lambda H^{-1}e_q)^T \nabla_w L
        + (H^{-1} \nabla_w L)^T \nabla_w L
    \right]
    \nonumber \\
    &= 
    - \lambda \nabla_w L^T H^{-1}e_q - \nabla_w L^T H^{-1} \nabla_w L
    \nonumber \\
    &+
    \frac{1}{2}\left[
        \lambda^2 e_q^T H^{-1} e_q
        + \lambda \nabla_w L^T H^{-1} e_q
        + \lambda e_q^T H^{-1} \nabla_w L
        + \nabla_w L^T H^{-1} \nabla_w L
    \right]
    \nonumber \\
    &= 
    \frac{1}{2}\left[
        \lambda^2 [H^{-1}]_{qq}
        - \lambda \nabla_w L^T H^{-1} e_q
        + \lambda e_q^T H^{-1} \nabla_w L
        - \nabla_w L^T H^{-1} \nabla_w L
    \right]
    \nonumber \\
\end{align*}
Finally, replacing the $\lambda$:
\begin{align*}
    \delta L 
    &= 
    \frac{1}{2}\left[
        \lambda^2 [H^{-1}]_{qq}
        - \lambda \nabla_w L^T H^{-1} e_q
        + \lambda e_q^T H^{-1} \nabla_w L
        - \nabla_w L^T H^{-1} \nabla_w L
    \right]
    \nonumber \\
    &= 
    \frac{1}{2[H^{-1}]_{qq}}\left[
        (w_q - e_q^T H^{-1} \nabla_w L)^2 
        + (w_q - e_q^T H^{-1} \nabla_w L)(e_q^T H^{-1} \nabla_w L - \nabla_w L^T H^{-1} e_q)
        - \nabla_w L^T H^{-1} \nabla_w L
    \right]
    \nonumber \\
    &= 
    \frac{1}{2[H^{-1}]_{qq}}[
        w_q^2
        - 2 w_q (e_q^T H^{-1} \nabla_w L)
        + (e_q^T H^{-1} \nabla_w L)^2
        + w_q (e_q^T H^{-1} \nabla_w L)
    \nonumber \\
        &- w_q (\nabla_w L^T H^{-1} e_q)
        - (e_q^T H^{-1} \nabla_w L)(e_q^T H^{-1} \nabla_w L)
        + (e_q^T H^{-1} \nabla_w L)(\nabla_w L^T H^{-1} e_q)
        - \nabla_w L^T H^{-1} \nabla_w L
    ]
    \nonumber \\
    &= 
    \frac{1}{2[H^{-1}]_{qq}}[
        w_q^2
        - w_q (e_q^T H^{-1} \nabla_w L)
        + (e_q^T H^{-1} \nabla_w L)^2
    \nonumber \\
        &- w_q (\nabla_w L^T H^{-1} e_q)
        - (e_q^T H^{-1} \nabla_w L)^2
        + (e_q^T H^{-1} \nabla_w L)(\nabla_w L^T H^{-1} e_q)
        - \nabla_w L^T H^{-1} \nabla_w L
    ]
    \nonumber \\
    &= 
    \frac{1}{2[H^{-1}]_{qq}}\left[
        w_q^2
        - 2 w_q (e_q^T H^{-1} \nabla_w L)
        + (e_q^T H^{-1} \nabla_w L)^2
        - \nabla_w L^T H^{-1} \nabla_w L
    \right]
    \nonumber \\
    &= 
    \frac{1}{2[\hat{F}^{-1}]_{qq}}
    \left[
        w_q - (e_q^T \hat{F}^{-1} \nabla \mathcal{L}(w_0))
    \right]^2
\end{align*}

%------------------------------------------------------------------------------------------------

\newpage
\section{Training and Testing Details}
\label{appendix:training_parameters}

We perform an 80:20 stratified split, with a constant seed, on the CIFAR10/100 training dataset to obtain a validation set with the same class distribution. For both datasets, we have a training set with 40,000 samples, a validation set with 10,000 samples, and a testing set of 10,000 samples. Validation is performed after each training step, and the weights of the best-performing validation step (based on top-1 accuracy) are utilized for the final evaluation on the testing set. Table \ref{tab:table_training_parameters} summarizes the training parameters.

\begin{table}[h]
\caption{Training parameters used for ResNet18 and VGG19 on the CIFAR-10/100 datasets.}
\label{tab:table_training_parameters}
\vskip 0.15in
\begin{center}
\begin{small}
\begin{sc}
\begin{tabular}{lcc}
\toprule
Parameter & ResNet18 & VGG19 \\
\midrule
Number of steps       & 160 & 160 \\
Criterion             & CE & CE \\
Optimizer             & SGD & SGD \\
Learning rate         & 0.01 & 0.1 \\
Momentum              & 0.9 & 0.9 \\
Weight decay          & $5 \times 10^{-4}$ & $1 \times 10^{-4}$ \\
Learning rate drops   & [60, 120] & [60, 120] \\
Learning rate drop factor & 0.2 & 0.1 \\
\bottomrule
\end{tabular}
\end{sc}
\end{small}
\end{center}
\vskip -0.1in
\end{table}

%------------------------------------------------------------------------------------------------

\newpage
\section{Results CIFAR10}
\subsection{ResNet18}
\label{appendix:CIFAR10_ResNet18}

\begin{table}[h]
\caption{Performance of different sensitivity methods for pruning evaluated using ResNet18 on the CIFAR-10 testset. The right side of the table presents our proposed criteria. The mean accuracy and standard deviation are reported across three initialization seeds for various sparsity levels. Baseline, no pruning: $91.78 \pm 0.09$.}
\label{tab:resnet18_cifar10_compressors}
\vskip 0.15in
\begin{center}
\begin{small}
\begin{sc}
\resizebox{\textwidth}{!}{%
\begin{tabular}{lccccc|cccc}
\toprule
Sparsity  & Random & Magnitude & GN & SNIP & GraSP & FD & FP & FTS & FBSS \\
\midrule
0.10  & 91.71 ± 0.21 & 91.72 ± 0.07 & 91.57 ± 0.15 & 91.72 ± 0.07 & 89.16 ± 0.05 & 91.87 ± 0.13 & 91.63 ± 0.21 & 91.53 ± 0.12 & 91.76 ± 0.08 \\
0.20  & 91.63 ± 0.11 & 91.42 ± 0.12 & 91.51 ± 0.09 & 91.64 ± 0.16 & 88.69 ± 0.34 & 91.50 ± 0.12 & 91.65 ± 0.14 & 91.53 ± 0.15 & 91.54 ± 0.13 \\
0.30  & 91.45 ± 0.18 & 91.61 ± 0.13 & 91.68 ± 0.20 & 91.65 ± 0.08 & 88.67 ± 0.26 & 91.65 ± 0.18 & 91.44 ± 0.27 & 91.49 ± 0.05 & 91.62 ± 0.07 \\
0.40  & 91.59 ± 0.18 & 91.06 ± 0.16 & 91.61 ± 0.09 & 91.55 ± 0.08 & 88.24 ± 0.33 & 91.51 ± 0.05 & 91.38 ± 0.13 & 91.56 ± 0.28 & 91.39 ± 0.05 \\
0.50  & 91.60 ± 0.06 & 91.32 ± 0.13 & 91.44 ± 0.13 & 91.22 ± 0.07 & 87.69 ± 0.15 & 91.30 ± 0.18 & 91.58 ± 0.16 & 91.46 ± 0.19 & 91.41 ± 0.05 \\
0.60  & 91.10 ± 0.16 & 91.18 ± 0.16 & 91.59 ± 0.13 & 91.24 ± 0.04 & 87.48 ± 0.55 & 91.34 ± 0.07 & 91.35 ± 0.16 & 91.40 ± 0.11 & 91.38 ± 0.18 \\
0.70  & 91.17 ± 0.04 & 91.07 ± 0.07 & 91.19 ± 0.17 & 91.33 ± 0.18 & 87.26 ± 0.34 & 91.34 ± 0.23 & 91.42 ± 0.23 & 91.18 ± 0.18 & 91.27 ± 0.14 \\
0.80  & 90.78 ± 0.08 & 91.10 ± 0.12 & 90.95 ± 0.35 & 90.74 ± 0.10 & 87.18 ± 0.51 & 90.95 ± 0.11 & 91.08 ± 0.06 & 90.94 ± 0.22 & 90.73 ± 0.33 \\
0.90  & 89.35 ± 0.13 & 89.88 ± 0.28 & 90.39 ± 0.23 & 90.36 ± 0.34 & 86.60 ± 0.51 & 90.04 ± 0.21 & 90.20 ± 0.08 & 90.55 ± 0.23 & 89.22 ± 0.30 \\
0.95  & 87.59 ± 0.11 & 89.23 ± 0.19 & 89.00 ± 0.05 & 89.31 ± 0.17 & 86.50 ± 0.05 & 88.61 ± 0.28 & 89.50 ± 0.18 & 89.47 ± 0.32 & 87.58 ± 0.25 \\
0.98  & 83.47 ± 0.20 & 85.70 ± 0.33 & 86.43 ± 0.05 & 87.26 ± 0.28 & 85.99 ± 0.08 & 85.61 ± 0.20 & 86.97 ± 0.22 & 87.24 ± 0.32 & 83.40 ± 0.74 \\
0.99  & 78.28 ± 0.45 & 71.99 ± 0.28 & 83.47 ± 0.15 & 84.54 ± 0.04 & 84.56 ± 0.46 & 82.13 ± 0.28 & 83.74 ± 0.48 & 84.85 ± 0.18 & 77.60 ± 1.02 \\
\bottomrule
\end{tabular}}
\end{sc}
\end{small}
\end{center}
\vskip -0.1in
\end{table}

%------------------------------------------------------------------------------------------------
\clearpage
\subsection{VGG19}
\label{appendix:CIFAR10_VGG19}

As discussed earlier, introducing a warm-up phase effectively mitigates layer collapse in data-dependent pruning methods. Here, we evaluate the impact of different warm-up durations by comparing no warm-up, a single warm-up epoch, and five warm-up epochs. Table \ref{tab:VGG19_cifar10_compressors} demonstrates how performance drastically degrades with increasing sparsity, ultimately leading to layer collapse at 0.90 sparsity. However, as shown in the results, a single warm-up epoch is sufficient to prevent collapse and stabilize pruning performance. Moreover, as seen in Table \ref{tab:VGG19_cifar10_compressors_warmup5}, increasing the warm-up period to five epochs provides no substantial additional improvement. This indicates that prolonged warm-up training is not necessary; a single training step is enough to achieve gradient stabilization and overcome layer collapse.

\begin{table}[h]
\caption{Performance of different sensitivity methods for pruning evaluated using VGG19 on the CIFAR-10 test set. The right side of the table presents our proposed criteria. The mean accuracy and standard deviation are reported across three initialization seeds for various sparsity levels. Baseline, no pruning: $89.21 \pm 0.22$.}
\label{tab:VGG19_cifar10_compressors}
\vskip 0.15in
\begin{center}
\begin{small}
\begin{sc}
\resizebox{\textwidth}{!}{%
\begin{tabular}{lccccc|cccc}
\toprule
Sparsity  & Random & Magnitude & GN & SNIP & GraSP & FD & FP & FTS & FBSS \\
\midrule
0.10  & 88.40 ± 0.95 & 89.12 ± 0.55 & 90.14 ± 0.10 & 90.16 ± 0.18 & 87.81 ± 1.66 & 90.20 ± 0.29 & 90.21 ± 0.37 & 90.25 ± 0.38 & 89.06 ± 0.75 \\
0.20  & 89.19 ± 0.22 & 89.65 ± 0.60 & 89.59 ± 0.69 & 90.06 ± 0.04 & 89.57 ± 0.34 & 89.91 ± 0.28 & 90.28 ± 0.55 & 89.80 ± 0.28 & 88.89 ± 0.76 \\
0.30  & 88.93 ± 0.83 & 88.77 ± 1.07 & 90.23 ± 0.09 & 89.88 ± 0.59 & 89.14 ± 0.19 & 90.25 ± 0.09 & 89.97 ± 0.26 & 90.46 ± 0.41 & 89.06 ± 0.36 \\
0.40  & 88.28 ± 1.08 & 89.38 ± 0.53 & 90.50 ± 0.23 & 89.79 ± 0.67 & 88.20 ± 0.31 & 90.51 ± 0.12 & 90.37 ± 0.24 & 90.23 ± 0.14 & 10.00 ± 0.00 \\
0.50  & 88.96 ± 0.82 & 89.03 ± 0.59 & 90.46 ± 0.60 & 90.38 ± 0.25 & 88.67 ± 0.23 & 89.54 ± 0.86 & 90.47 ± 0.52 & 90.19 ± 0.31 & 10.00 ± 0.00 \\
0.60  & 88.15 ± 0.68 & 89.47 ± 0.18 & 89.95 ± 0.30 & 90.32 ± 0.25 & 88.82 ± 0.32 & 90.02 ± 0.40 & 90.18 ± 0.33 & 90.14 ± 0.36 & 10.00 ± 0.00 \\
0.70  & 88.02 ± 0.53 & 89.63 ± 0.44 & 89.69 ± 0.42 & 89.23 ± 0.19 & 89.62 ± 0.81 & 89.85 ± 0.08 & 90.01 ± 0.34 & 10.00 ± 0.00 & 10.00 ± 0.00 \\
0.80  & 88.28 ± 0.34 & 89.62 ± 0.91 & 85.72 ± 0.63 & 89.39 ± 0.43 & 88.82 ± 0.14 & 10.00 ± 0.00 & 88.29 ± 0.11 & 10.00 ± 0.00 & 10.00 ± 0.00 \\
0.90  & 85.82 ± 0.19 & 89.29 ± 0.79 & 10.00 ± 0.00 & 80.85 ± 0.62 & 24.28 ± 20.2 & 10.00 ± 0.00 & 10.00 ± 0.00 & 10.00 ± 0.00 & 10.00 ± 0.00 \\
0.95  & 84.41 ± 0.05 & 10.00 ± 0.00 & 10.00 ± 0.00 & 10.00 ± 0.00 & 10.00 ± 0.00 & 10.00 ± 0.00 & 10.00 ± 0.00 & 10.00 ± 0.00 & 10.00 ± 0.00 \\
0.98  & 80.04 ± 0.90 & 10.00 ± 0.00 & 10.00 ± 0.00 & 10.00 ± 0.00 & 10.00 ± 0.00 & 10.00 ± 0.00 & 10.00 ± 0.00 & 10.00 ± 0.00 & 10.00 ± 0.00 \\
0.99  & 76.89 ± 0.26 & 10.00 ± 0.00 & 10.00 ± 0.00 & 10.00 ± 0.00 & 10.00 ± 0.00 & 10.00 ± 0.00 & 10.00 ± 0.00 & 10.00 ± 0.00 & 10.00 ± 0.00 \\
\bottomrule
\end{tabular}}
\end{sc}
\end{small}
\end{center}
\vskip -0.1in
\end{table}
\newpage
%------------------------------------------------------------------------------------------------
\begin{table*}[h]
\caption{Performance of different compression methods evaluated after 1 warmup epoch using VGG19 on the CIFAR-10 dataset. We report the mean accuracy between three initialization seeds across various sparsity levels. Baseline, no pruning: $89.21 \pm 0.22$.}
\label{tab:VGG19_cifar10_compressors_warmup1}
\vskip 0.15in
\begin{center}
\begin{small}
\begin{sc}
\resizebox{\textwidth}{!}{%
\begin{tabular}{lccccc|cccc}
\toprule
Sparsity  & Random & Magnitude & GN & SNIP & GraSP & FD & FP & FTS & FBSS \\
\midrule
0.80  & 88.73 ± 0.38 & 88.35 ± 0.54 & 86.76 ± 0.27 & 87.39 ± 0.66 & 87.24 ± 0.25 & 87.14 ± 0.45 & 87.00 ± 0.87 & 87.68 ± 0.33 & 64.33 ± 15.91 \\
0.90  & 87.26 ± 0.42 & 88.62 ± 0.49 & 85.96 ± 0.75 & 86.75 ± 0.76 & 87.47 ± 0.33 & 86.69 ± 0.72 & 87.09 ± 0.31 & 87.42 ± 0.21 & 46.16 ± 7.62 \\
0.95  & 85.47 ± 0.64 & 87.68 ± 0.49 & 86.66 ± 0.27 & 86.00 ± 1.10 & 86.71 ± 1.24 & 85.71 ± 1.35 & 86.73 ± 0.36 & 87.56 ± 0.62 & 46.30 ± 5.32 \\
0.98  & 80.44 ± 0.30 & 86.61 ± 0.62 & 84.72 ± 1.69 & 87.22 ± 0.23 & 86.45 ± 0.64 & 80.34 ± 6.43 & 86.07 ± 0.39 & 86.36 ± 0.29 & 49.05 ± 4.31 \\
0.99  & 77.24 ± 0.73 & 83.69 ± 1.36 & 80.28 ± 2.04 & 83.49 ± 1.77 & 85.39 ± 0.43 & 75.11 ± 7.80 & 84.40 ± 1.27 & 85.35 ± 1.05 & 47.10 ± 4.41 \\
\bottomrule
\end{tabular}}
\end{sc}
\end{small}
\end{center}
\vskip -0.1in
\end{table*} 
%------------------------------------------------------------------------------------------------

\begin{table}[h]
\caption{Performance of different sensitivity methods for pruning evaluated after 5 warmup epochs using VGG19 on the CIFAR-10 testset. The right side of the table presents our proposed criteria. The mean accuracy and standard deviation are reported across three initialization seeds for various sparsity levels. Baseline, no pruning: $89.21 \pm 0.22$.}
\label{tab:VGG19_cifar10_compressors_warmup5}
\vskip 0.15in
\begin{center}
\begin{small}
\begin{sc}
\resizebox{\textwidth}{!}{%
\begin{tabular}{lccccc|cccc}
\toprule
Sparsity  & Random & Magnitude & GN & SNIP & GraSP & FD & FP & FTS & FBSS \\
\midrule
0.80  & 88.84 ± 0.43 & 88.41 ± 0.47 & 87.58 ± 0.52 & 88.15 ± 1.09 & 86.77 ± 1.14 & 87.28 ± 0.90 & 88.22 ± 0.82 & 86.68 ± 0.61 & 70.52 ± 9.25 \\
0.90  & 87.56 ± 0.62 & 88.60 ± 0.93 & 86.73 ± 0.37 & 87.89 ± 0.25 & 87.10 ± 0.47 & 87.50 ± 1.42 & 88.18 ± 0.47 & 86.98 ± 0.14 & 47.78 ± 1.26 \\
0.95 & 85.51 ± 0.69 & 87.66 ± 1.19 & 87.44 ± 0.46 & 87.71 ± 0.82 & 87.05 ± 0.16 & 86.83 ± 1.47 & 87.36 ± 0.52 & 87.00 ± 0.74 & 48.83 ± 2.52 \\
0.98 & 82.09 ± 0.17 & 86.24 ± 0.52 & 84.66 ± 1.33 & 86.55 ± 0.84 & 86.04 ± 0.66 & 85.44 ± 0.64 & 86.64 ± 0.13 & 84.89 ± 0.51 & 49.48 ± 0.85 \\
0.99 & 77.22 ± 1.03 & 83.93 ± 1.80 & 81.62 ± 2.17 & 84.53 ± 0.70 & 81.33 ± 5.77 & 81.71 ± 1.41 & 85.02 ± 0.69 & 83.78 ± 0.80 & 41.24 ± 1.55 \\
\bottomrule
\end{tabular}}
\end{sc}
\end{small}
\end{center}
\vskip -0.1in
\end{table}

%------------------------------------------------------------------------------------------------

\newpage
\section{Results CIFAR100}
\subsection{ResNet18}
\label{sec:resnet_cifar-100}

CIFAR-100 results exhibit a similar trend to those observed on CIFAR-10, further reinforcing the robustness of our proposed Fisher-Taylor Sensitivity (FTS) criterion. Across all evaluated sparsity levels, FTS consistently maintains strong performance, frequently ranking among the top-performing methods. This trend is particularly evident at extreme sparsities, where many pruning approaches suffer significant performance degradation. The stability of FTS across both datasets highlights its effectiveness in preserving network expressivity despite aggressive pruning.

\begin{table}[h]
\caption{Performance of different compression methods evaluated using ResNet18 on the CIFAR-100 dataset. We report the mean accuracy between three initialization seeds across various sparsity levels. Baseline, no pruning: $69.57 \pm 0.19$.}
\label{tab:resnet18_cifar100_compressors}
\vskip 0.15in
\begin{center}
\begin{small}
\begin{sc}
\resizebox{\textwidth}{!}{%
\begin{tabular}{lccccc|cccc}
\toprule
Sparsity  & Random & Magnitude & GN & SNIP & GraSP & FD & FP & FTS & FBSS \\
\midrule
0.10  & 69.16 ± 0.11 & 69.37 ± 0.14 & 69.63 ± 0.34 & 69.42 ± 0.07 & 64.26 ± 0.27 & 69.66 ± 0.30 & 69.08 ± 0.21 & 69.16 ± 0.11 & 69.07 ± 0.10 \\
0.20  & 69.16 ± 0.30 & 69.06 ± 0.24 & 69.19 ± 0.11 & 69.30 ± 0.08 & 63.28 ± 0.58 & 69.60 ± 0.30 & 69.35 ± 0.35 & 69.41 ± 0.43 & 69.07 ± 0.20 \\
0.30  & 69.36 ± 0.18 & 68.58 ± 0.36 & 69.37 ± 0.13 & 68.82 ± 0.17 & 62.02 ± 0.43 & 69.24 ± 0.40 & 68.84 ± 0.13 & 68.80 ± 0.55 & 68.96 ± 0.11 \\
0.40  & 69.41 ± 0.20 & 68.50 ± 0.29 & 69.16 ± 0.26 & 68.95 ± 0.19 & 61.18 ± 0.19 & 69.17 ± 0.16 & 68.88 ± 0.25 & 69.02 ± 0.21 & 68.92 ± 0.25 \\
0.50  & 69.12 ± 0.46 & 68.17 ± 0.20 & 68.94 ± 0.20 & 68.63 ± 0.11 & 61.11 ± 0.40 & 69.13 ± 0.13 & 68.68 ± 0.12 & 68.71 ± 0.12 & 68.71 ± 0.57 \\
0.60  & 68.66 ± 0.27 & 67.78 ± 0.35 & 68.77 ± 0.17 & 68.63 ± 0.42 & 61.40 ± 0.78 & 68.34 ± 0.43 & 67.98 ± 0.23 & 68.41 ± 0.14 & 68.60 ± 0.15 \\
0.70  & 67.95 ± 0.43 & 67.51 ± 0.24 & 68.29 ± 0.39 & 68.08 ± 0.18 & 59.43 ± 0.76 & 68.03 ± 0.46 & 67.96 ± 0.15 & 68.29 ± 0.06 & 68.16 ± 0.07 \\
0.80  & 67.26 ± 0.48 & 66.55 ± 0.19 & 67.20 ± 0.37 & 67.21 ± 0.38 & 59.08 ± 0.22 & 66.70 ± 0.05 & 67.05 ± 0.06 & 66.77 ± 0.65 & 66.62 ± 0.43 \\
0.90  & 64.75 ± 0.16 & 64.48 ± 0.18 & 64.87 ± 0.27 & 65.70 ± 0.08 & 59.16 ± 0.91 & 64.74 ± 0.44 & 65.46 ± 0.30 & 65.41 ± 0.13 & 63.90 ± 0.31 \\
0.95  & 61.01 ± 0.32 & 62.20 ± 0.06 & 62.20 ± 0.23 & 63.20 ± 0.20 & 57.91 ± 0.09 & 62.14 ± 0.42 & 63.22 ± 0.25 & 63.21 ± 0.47 & 61.25 ± 0.44 \\
0.98  & 54.72 ± 0.22 & 55.44 ± 0.18 & 57.34 ± 0.31 & 58.83 ± 0.35 & 54.85 ± 0.35 & 55.57 ± 0.17 & 58.05 ± 0.18 & 58.59 ± 0.12 & 55.02 ± 0.34 \\
0.99  & 45.62 ± 0.55 & 40.39 ± 0.36 & 50.46 ± 0.61 & 52.96 ± 0.10 & 49.13 ± 0.19 & 48.02 ± 0.32 & 49.98 ± 0.60 & 52.85 ± 0.24 & 44.91 ± 0.52 \\
\bottomrule
\end{tabular}}
\end{sc}
\end{small}
\end{center}
\vskip -0.1in
\end{table}

%------------------------------------------------------------------------------------------------
\clearpage
\subsection{VGG19}
The results on VGG19 with CIFAR-100 exhibit a similar trend to those observed on CIFAR-10, reinforcing the effectiveness of our proposed approach. Once again, we identify the occurrence of layer collapse at extreme sparsities when no warm-up is applied, leading to a significant drop in accuracy. Introducing a single warm-up epoch effectively resolves this issue, restoring pruning performance across all evaluated criteria. However, increasing the warm-up phase to five epochs does not yield any additional advantage, indicating that a brief warm-up period is sufficient to stabilize gradient-based importance scores and prevent collapse.

\label{sec:vgg_cifar-100}

\begin{table}[h]
\caption{Performance of different compression methods evaluated using VGG19 on the CIFAR-100 dataset. We report the mean accuracy between three initialization seeds across various sparsity levels. Baseline, no pruning: $58.96 \pm 2.30$.}
\label{tab:VGG19_cifar100_compressors}
\vskip 0.15in
\begin{center}
\begin{small}
\begin{sc}
\resizebox{\textwidth}{!}{%
\begin{tabular}{lccccc|cccc}
\toprule
Sparsity & Random & Magnitude & GN & SNIP & GraSP & FD & FP & FTS & FBSS \\
\midrule
0.10  & 60.31 ± 0.40 & 59.13 ± 1.29 & 61.93 ± 0.48 & 61.98 ± 0.29 & 59.32 ± 0.63 & 62.13 ± 0.61 & 60.45 ± 3.47 & 61.56 ± 1.04 & 58.79 ± 0.98 \\
0.20  & 60.43 ± 1.14 & 59.27 ± 0.34 & 62.64 ± 0.21 & 62.68 ± 0.24 & 61.21 ± 0.41 & 63.04 ± 0.43 & 62.71 ± 1.02 & 62.24 ± 0.44 & 60.48 ± 0.48 \\
0.30  & 58.32 ± 0.60 & 59.35 ± 1.43 & 62.61 ± 0.23 & 63.11 ± 0.35 & 59.30 ± 0.43 & 62.85 ± 0.42 & 61.43 ± 0.61 & 62.65 ± 0.54 & 58.77 ± 1.02 \\
0.40  & 56.50 ± 3.20 & 60.04 ± 1.02 & 62.36 ± 0.02 & 62.39 ± 0.55 & 56.34 ± 1.49 & 62.38 ± 0.75 & 61.56 ± 1.25 & 62.67 ± 0.06 & 1.00 ± 0.00 \\
0.50  & 58.47 ± 1.49 & 61.49 ± 1.22 & 62.02 ± 0.64 & 62.76 ± 0.50 & 54.43 ± 0.84 & 62.84 ± 0.33 & 62.25 ± 0.33 & 62.47 ± 0.42 & 1.00 ± 0.00 \\
0.60  & 57.54 ± 0.74 & 61.50 ± 0.30 & 62.55 ± 0.13 & 63.08 ± 0.55 & 56.76 ± 0.69 & 62.40 ± 0.57 & 62.70 ± 0.63 & 62.17 ± 0.23 & 1.00 ± 0.00 \\
0.70  & 57.63 ± 0.80 & 61.71 ± 0.25 & 60.85 ± 0.79 & 60.58 ± 0.39 & 57.76 ± 0.84 & 60.44 ± 0.34 & 60.92 ± 0.41 & 60.51 ± 1.67 & 1.00 ± 0.00 \\
0.80  & 57.84 ± 0.57 & 61.89 ± 1.02 & 55.09 ± 0.49 & 59.84 ± 0.29 & 58.39 ± 0.74 & 1.00 ± 0.00 & 43.16 ± 1.02 & 58.66 ± 2.28 & 1.00 ± 0.00 \\
0.90  & 58.41 ± 0.41 & 62.60 ± 0.91 & 1.00 ± 0.00 & 8.35 ± 10.39 & 42.88 ± 1.64 & 1.00 ± 0.00 & 1.00 ± 0.00 & 8.87 ± 11.13 & 1.00 ± 0.00 \\
0.95  & 54.84 ± 1.08 & 1.00 ± 0.00 & 1.00 ± 0.00 & 1.00 ± 0.00 & 1.00 ± 0.00 & 1.00 ± 0.00 & 1.00 ± 0.00 & 1.00 ± 0.00 & 1.00 ± 0.00 \\
0.98  & 50.21 ± 0.72 & 1.00 ± 0.00 & 1.00 ± 0.00 & 1.00 ± 0.00 & 1.00 ± 0.00 & 1.00 ± 0.00 & 1.00 ± 0.00 & 1.00 ± 0.00 & 1.00 ± 0.00 \\
0.99  & 46.69 ± 0.45 & 1.00 ± 0.00 & 1.00 ± 0.00 & 1.00 ± 0.00 & 1.00 ± 0.00 & 1.00 ± 0.00 & 1.00 ± 0.00 & 1.00 ± 0.00 & 1.00 ± 0.00 \\
\bottomrule
\end{tabular}}
\end{sc}
\end{small}
\end{center}
\vskip -0.1in
\end{table}

%------------------------------------------------------------------------------------------------

\begin{table}[h]
\caption{Performance of different compression methods evaluated after 1 warmup epoch using VGG19 on the CIFAR-100 dataset. We report the mean accuracy between three initialization seeds across various sparsity levels. Baseline, no pruning: $58.96 \pm 2.30$.}
\label{tab:VGG19_cifar100_compressors_warmup1}
\vskip 0.15in
\begin{center}
\begin{small}
\begin{sc}
\resizebox{\textwidth}{!}{%
\begin{tabular}{lccccc|cccc}
\toprule
Sparsity & Random & Magnitude & GN & SNIP & GraSP & FD & FP & FTS & FBSS \\
\midrule
0.80  & 60.39 ± 1.16 & 58.91 ± 0.41 & 52.81 ± 1.32 & 55.62 ± 2.27 & 55.15 ± 2.25 & 56.71 ± 0.31 & 58.03 ± 0.93 & 52.41 ± 3.07 & 52.74 ± 5.16 \\
0.90  & 58.90 ± 0.98 & 60.95 ± 0.81 & 50.56 ± 4.59 & 55.89 ± 2.05 & 56.01 ± 1.58 & 52.07 ± 3.24 & 53.65 ± 0.57 & 52.45 ± 3.75 & 19.65 ± 1.68 \\
0.95  & 56.10 ± 0.85 & 57.64 ± 2.63 & 50.34 ± 1.00 & 53.70 ± 3.60 & 56.16 ± 0.41 & 54.44 ± 1.38 & 53.24 ± 3.54 & 53.56 ± 1.26 & 17.24 ± 0.44 \\
0.98  & 50.97 ± 0.40 & 54.66 ± 2.56 & 43.43 ± 5.32 & 50.19 ± 1.59 & 54.64 ± 1.50 & 42.75 ± 1.91 & 50.59 ± 3.39 & 48.56 ± 5.25 & 16.42 ± 0.64 \\
0.99  & 46.52 ± 0.45 & 43.33 ± 5.83 & 33.90 ± 5.35 & 42.65 ± 5.32 & 45.98 ± 4.48 & 29.67 ± 8.49 & 49.11 ± 3.46 & 48.70 ± 2.59 & 13.25 ± 0.84 \\
\bottomrule
\end{tabular}}
\end{sc}
\end{small}
\end{center}
\vskip -0.1in
\end{table}


%------------------------------------------------------------------------------------------------

\begin{table}[h]
\caption{Performance of different compression methods evaluated after 5 warmup epochs using VGG19 on the CIFAR-100 dataset. We report the mean accuracy between three initialization seeds across various sparsity levels. Baseline, no pruning: $58.96 \pm 2.30$.}
\label{tab:VGG19_cifar100_compressors_warmup5}
\vskip 0.15in
\begin{center}
\begin{small}
\begin{sc}
\resizebox{\textwidth}{!}{%
\begin{tabular}{lccccc|cccc}
\toprule
Sparsity & Random & Magnitude & GN & SNIP & GraSP & FD & FP & FTS & FBSS \\
\midrule
0.80  & 60.41 ± 1.39 & 58.38 ± 0.85 & 60.86 ± 0.79 & 61.63 ± 0.45 & 56.25 ± 0.49 & 59.59 ± 0.76 & 59.37 ± 3.50 & 60.86 ± 0.53 & 46.93 ± 9.04 \\
0.90  & 60.32 ± 0.09 & 57.74 ± 1.64 & 57.77 ± 2.41 & 58.23 ± 4.07 & 56.27 ± 1.02 & 60.19 ± 0.63 & 61.23 ± 0.50 & 60.52 ± 0.37 & 21.66 ± 1.95 \\
0.95 & 57.86 ± 0.53 & 59.55 ± 1.15 & 56.09 ± 0.97 & 58.83 ± 0.65 & 55.26 ± 1.25 & 55.80 ± 2.77 & 59.83 ± 0.94 & 58.52 ± 1.32 & 19.98 ± 2.62 \\
0.98 & 51.75 ± 0.43 & 47.75 ± 7.63 & 52.26 ± 4.06 & 55.27 ± 1.69 & 54.59 ± 0.96 & 49.46 ± 4.98 & 57.40 ± 1.26 & 56.00 ± 1.08 & 17.59 ± 1.36 \\
0.99 & 47.59 ± 0.80 & 42.46 ± 7.95 & 46.58 ± 2.00 & 53.13 ± 0.84 & 53.91 ± 1.53 & 42.87 ± 4.63 & 53.17 ± 1.18 & 53.05 ± 2.14 & 13.92 ± 0.14 \\
\bottomrule
\end{tabular}}
\end{sc}
\end{small}
\end{center}
\vskip -0.1in
\end{table}


%------------------------------------------------------------------------------------------------
\clearpage

\section{Mask Batch Size for Other Sparsities}
The Effect of batch size on pruning performance across different sparsities. 
As sparsity increases, the effect of batch size on pruning performance becomes more pronounced. 
At lower sparsities (0.90, 0.95), the differences across batch sizes are less evident, suggesting that even smaller batches provide a reasonable estimation of parameter importance. However, at extreme sparsities (0.98, 0.99), we observe a clear trend where larger batch sizes consistently lead to better parameter selection, ultimately improving accuracy. This aligns with our hypothesis that larger batches help reduce variance in gradient estimation, leading to more stable and effective pruning decisions. 
\label{batch_size_heatmaps}

\begin{figure}[h]
    \centering
    \includegraphics[width=0.8\linewidth]{imgs/cifar10_resnet18_heatmap_warmup_0.png}
    \caption{Effect of batch size on pruning performance at increasing sparsities.}
    \label{fig:enter-label}
\end{figure}

%------------------------------------------------------------------------------------------------

\clearpage
\section{Comparison of our criteria with magnitude-based pruning}

Figure \ref{fig:our_criterion_vs_magnitude} illustrates the relationship between parameter magnitude and different sensitivity-based pruning metrics. Each point represents a model parameter, with red points indicating the top-ranked parameters selected for retention by each criterion. The green dashed line marks the 99th percentile of parameter magnitudes.

A key observation is that the most effective pruning criteria, such as Fisher-Taylor Sensitivity, tend to retain parameters with a broad range of magnitudes, including many that are relatively small (left of the green line). This shows that the estimated importance does not always prioritize parameters based on their magnitude. 


\begin{figure}[htp]
    \centering
    \includegraphics[width=0.9\linewidth]{imgs/cifar_10_mag_vs_criteria_s_99.png}
    \caption{Our criteria vs. Magnitude parameter selection for 99\% sparsity (ResNet18, CIFAR-10, Seed 0)} 
    \label{fig:our_criterion_vs_magnitude}
\end{figure}


\end{document}


% This document was modified from the file originally made available by
% Pat Langley and Andrea Danyluk for ICML-2K. This version was created
% by Iain Murray in 2018, and modified by Alexandre Bouchard in
% 2019 and 2021 and by Csaba Szepesvari, Gang Niu and Sivan Sabato in 2022.
% Modified again in 2023 and 2024 by Sivan Sabato and Jonathan Scarlett.
% Previous contributors include Dan Roy, Lise Getoor and Tobias
% Scheffer, which was slightly modified from the 2010 version by
% Thorsten Joachims & Johannes Fuernkranz, slightly modified from the
% 2009 version by Kiri Wagstaff and Sam Roweis's 2008 version, which is
% slightly modified from Prasad Tadepalli's 2007 version which is a
% lightly changed version of the previous year's version by Andrew
% Moore, which was in turn edited from those of Kristian Kersting and
% Codrina Lauth. Alex Smola contributed to the algorithmic style files.
