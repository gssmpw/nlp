%%%%%%%% ICML 2025 EXAMPLE LATEX SUBMISSION FILE %%%%%%%%%%%%%%%%%

\documentclass{article}

% Recommended, but optional, packages for figures and better typesetting:
\usepackage{microtype}
\usepackage{graphicx}
\usepackage{subfigure}
\usepackage{booktabs} % for professional tables

% hyperref makes hyperlinks in the resulting PDF.
% If your build breaks (sometimes temporarily if a hyperlink spans a page)
% please comment out the following usepackage line and replace
% \usepackage{icml2025} with \usepackage[nohyperref]{icml2025} above.
\usepackage{hyperref}

% Attempt to make hyperref and algorithmic work together better:
\newcommand{\theHalgorithm}{\arabic{algorithm}}

% Use the following line for the initial blind version submitted for review:

\usepackage[accepted]{icml2025}

% If accepted, instead use the following line for the camera-ready submission:
% \usepackage[accepted]{icml2025}

% For theorems and such
\usepackage{amsmath}
\usepackage{amssymb}
\usepackage{mathtools}
\usepackage{amsthm}

\usepackage{wrapfig}
\usepackage{caption}
\usepackage{listings}
\usepackage{adjustbox}
% \usepackage{subcaption}
% if you use cleveref..
\usepackage[capitalize,noabbrev]{cleveref}

%%%%%%%%%%%%%%%%%%%%%%%%%%%%%%%%
% THEOREMS
%%%%%%%%%%%%%%%%%%%%%%%%%%%%%%%%
\theoremstyle{plain}
\newtheorem{theorem}{Theorem}[section]
\newtheorem{proposition}[theorem]{Proposition}
\newtheorem{lemma}[theorem]{Lemma}
\newtheorem{corollary}[theorem]{Corollary}
\theoremstyle{definition}
\newtheorem{definition}[theorem]{Definition}
\newtheorem{assumption}[theorem]{Assumption}
\theoremstyle{remark}
\newtheorem{remark}[theorem]{Remark}

% Todonotes is useful during development; simply uncomment the next line
%    and comment out the line below the next line to turn off comments
%\usepackage[disable,textsize=tiny]{todonotes}
\usepackage[textsize=tiny]{todonotes}
\usepackage{placeins}
\usepackage{siunitx}

% The \icmltitle you define below is probably too long as a header.
% Therefore, a short form for the running title is supplied here:
\icmltitlerunning{Beyond Fine-Tuning: A Systematic Study of Sampling Techniques in Personalized Image Generation}

\begin{document}

\twocolumn[
\icmltitle{Beyond Fine-Tuning: \\ A Systematic Study of Sampling Techniques in \\ Personalized Image Generation}

% It is OKAY to include author information, even for blind
% submissions: the style file will automatically remove it for you
% unless you've provided the [accepted] option to the icml2025
% package.

% List of affiliations: The first argument should be a (short)
% identifier you will use later to specify author affiliations
% Academic affiliations should list Department, University, City, Region, Country
% Industry affiliations should list Company, City, Region, Country

% You can specify symbols, otherwise they are numbered in order.
% Ideally, you should not use this facility. Affiliations will be numbered
% in order of appearance and this is the preferred way.
\icmlsetsymbol{equal}{*}

\begin{icmlauthorlist}
\icmlauthor{Vera Soboleva}{equal,yyy,comp}
\icmlauthor{Maksim Nakhodnov}{equal,comp}
\icmlauthor{Aibek Alanov}{yyy,comp}
% \icmlauthor{Firstname4 Lastname4}{sch}
% \icmlauthor{Firstname5 Lastname5}{yyy}
% \icmlauthor{Firstname6 Lastname6}{sch,yyy,comp}
% \icmlauthor{Firstname7 Lastname7}{comp}
% %\icmlauthor{}{sch}
% \icmlauthor{Firstname8 Lastname8}{sch}
% \icmlauthor{Firstname8 Lastname8}{yyy,comp}
%\icmlauthor{}{sch}
%\icmlauthor{}{sch}
\end{icmlauthorlist}

\icmlaffiliation{yyy}{HSE University, Moscow, Russia}
\icmlaffiliation{comp}{AIRI, Moscow, Russia}
% \icmlaffiliation{sch}{School of ZZZ, Institute of WWW, Location, Country}

\icmlcorrespondingauthor{Aibek Alanov}{alanov.aibek@gmail.com}
% \icmlcorrespondingauthor{Firstname2 Lastname2}{first2.last2@www.uk}

% You may provide any keywords that you
% find helpful for describing your paper; these are used to populate
% the "keywords" metadata in the PDF but will not be shown in the document
\icmlkeywords{Machine Learning, ICML}

\vskip 0.3in
]

% this must go after the closing bracket ] following \twocolumn[ ...

% This command actually creates the footnote in the first column
% listing the affiliations and the copyright notice.
% The command takes one argument, which is text to display at the start of the footnote.
% The \icmlEqualContribution command is standard text for equal contribution.
% Remove it (just {}) if you do not need this facility.

%\printAffiliationsAndNotice{}  % leave blank if no need to mention equal contribution
\printAffiliationsAndNotice{\icmlEqualContribution} % otherwise use the standard text.

% \begin{abstract}
% Personalized text-to-image generation aims to create images tailored to user-defined concepts and textual descriptions. Balancing the fidelity of the learned concept with its ability for generation in various contexts presents a significant challenge. Existing methods often address this through diverse fine-tuning parameterizations and enhanced sampling strategies that integrate superclass trajectories during the diffusion process. While improved sampling offers a cost-effective, training-free solution for enhancing fine-tuned models, systematic analyses of these methods remain limited. Current approaches typically tie sampling strategies to fixed fine-tuning configurations, making it difficult to isolate their impact on generation outcomes. To address this, we systematically analyze sampling strategies beyond fine-tuning, exploring the influence of concept and superclass trajectories on the results. We show that a weighted mix of these trajectories establishes a strong baseline for improving adaptability across contexts. This approach integrates seamlessly with diverse training strategies, providing an efficient framework to optimize fidelity, editability, and computational efficiency in text-to-image generation.
% \end{abstract}

\begin{abstract}
Personalized text-to-image generation aims to create images tailored to user-defined concepts and textual descriptions. Balancing the fidelity of the learned concept with its ability for generation in various contexts presents a significant challenge. Existing methods often address this through diverse fine-tuning parameterizations and improved sampling strategies that integrate superclass trajectories during the diffusion process. While improved sampling offers a cost-effective, training-free solution for enhancing fine-tuned models, systematic analyses of these methods remain limited. Current approaches typically tie sampling strategies with fixed fine-tuning configurations, making it difficult to isolate their impact on generation outcomes. To address this issue, we systematically analyze sampling strategies beyond fine-tuning, exploring the impact of concept and superclass trajectories on the results. Building on this analysis, we propose a decision framework evaluating text alignment, computational constraints, and fidelity objectives to guide strategy selection. It integrates with diverse architectures and training approaches, systematically optimizing concept preservation, prompt adherence, and resource efficiency. The source code can be found
at \href{https://github.com/ControlGenAI/PersonGenSampler}{github.com/ControlGenAI/PersonGenSampler}.
\end{abstract}


\section{Introduction}\label{sec:Intro} 


Novel view synthesis offers a fundamental approach to visualizing complex scenes by generating new perspectives from existing imagery. 
This has many potential applications, including virtual reality, movie production and architectural visualization \cite{Tewari2022NeuRendSTAR}. 
An emerging alternative to the common RGB sensors are event cameras, which are  
 bio-inspired visual sensors recording events, i.e.~asynchronous per-pixel signals of changes in brightness or color intensity. 

Event streams have very high temporal resolution and are inherently sparse, as they only happen when changes in the scene are observed. 
Due to their working principle, event cameras bring several advantages, especially in challenging cases: they excel at handling high-speed motions 
and have a substantially higher dynamic range of the supported signal measurements than conventional RGB cameras. 
Moreover, they have lower power consumption and require varied storage volumes for captured data that are often smaller than those required for synchronous RGB cameras \cite{Millerdurai_3DV2024, Gallego2022}. 

The ability to handle high-speed motions is crucial in static scenes as well,  particularly with handheld moving cameras, as it helps avoid the common problem of motion blur. It is, therefore, not surprising that event-based novel view synthesis has gained attention, although color values are not directly observed.
Notably, because of the substantial difference between the formats, RGB- and event-based approaches require fundamentally different design choices. %

The first solutions to event-based novel view synthesis introduced in the literature demonstrate promising results \cite{eventnerf, enerf} and outperform non-event-based alternatives for novel view synthesis in many challenging scenarios. 
Among them, EventNeRF \cite{eventnerf} enables novel-view synthesis in the RGB space by assuming events associated with three color channels as inputs. 
Due to its NeRF-based architecture \cite{nerf}, it can handle single objects with complete observations from roughly equal distances to the camera. 
It furthermore has limitations in training and rendering speed: 
the MLP used to represent the scene requires long training time and can only handle very limited scene extents or otherwise rendering quality will deteriorate. 
Hence, the quality of synthesized novel views will degrade for larger scenes. %

We present Event-3DGS (E-3DGS), i.e.,~a new method for novel-view synthesis from event streams using 3D Gaussians~\cite{3dgs} 
demonstrating fast reconstruction and rendering as well as handling of unbounded scenes. 
The technical contributions of this paper are as follows: 
\begin{itemize}
\item With E-3DGS, we introduce the first approach for novel view synthesis from a color event camera that combines 3D Gaussians with event-based supervision. 
\item We present frustum-based initialization, adaptive event windows, isotropic 3D Gaussian regularization and 3D camera pose refinement, and demonstrate that high-quality results can be obtained. %

\item Finally, we introduce new synthetic and real event datasets for large scenes to the community to study novel view synthesis in this new problem setting. 
\end{itemize}
Our experiments demonstrate systematically superior results compared to EventNeRF \cite{eventnerf} and other baselines. 
The source code and dataset of E-3DGS are released\footnote{\url{https://4dqv.mpi-inf.mpg.de/E3DGS/}}. 







\section{Preliminaries }
\label{sec:Preliminaries}

% Before we introduce our method, we first overview the important basics of 3D dynamic human modeling with Gaussian splatting. Then, we discuss the diffusion-based 3d generation techniques, and how they can be applied to human modeling.
% \ZY{I stopp here. TBC.}
% \subsection{Dynamic human modeling with Gaussian splatting}
\subsection{3D Gaussian Splatting}
3D Gaussian splatting~\cite{kerbl3Dgaussians} is an explicit scene representation that allows high-quality real-time rendering. The given scene is represented by a set of static 3D Gaussians, which are parameterized as follows: Gaussian center $x\in {\mathbb{R}^3}$, color $c\in {\mathbb{R}^3}$, opacity $\alpha\in {\mathbb{R}}$, spatial rotation in the form of quaternion $q\in {\mathbb{R}^4}$, and scaling factor $s\in {\mathbb{R}^3}$. Given these properties, the rendering process is represented as:
\begin{equation}
  I = Splatting(x, c, s, \alpha, q, r),
  \label{eq:splattingGA}
\end{equation}
where $I$ is the rendered image, $r$ is a set of query rays crossing the scene, and $Splatting(\cdot)$ is a differentiable rendering process. We refer readers to Kerbl et al.'s paper~\cite{kerbl3Dgaussians} for the details of Gaussian splatting. 



% \ZY{I would suggest move this part to the method part.}
% GaissianAvatar is a dynamic human generation model based on Gaussian splitting. Given a sequence of RGB images, this method utilizes fitted SMPLs and sampled points on its surface to obtain a pose-dependent feature map by a pose encoder. The pose-dependent features and a geometry feature are fed in a Gaussian decoder, which is employed to establish a functional mapping from the underlying geometry of the human form to diverse attributes of 3D Gaussians on the canonical surfaces. The parameter prediction process is articulated as follows:
% \begin{equation}
%   (\Delta x,c,s)=G_{\theta}(S+P),
%   \label{eq:gaussiandecoder}
% \end{equation}
%  where $G_{\theta}$ represents the Gaussian decoder, and $(S+P)$ is the multiplication of geometry feature S and pose feature P. Instead of optimizing all attributes of Gaussian, this decoder predicts 3D positional offset $\Delta{x} \in {\mathbb{R}^3}$, color $c\in\mathbb{R}^3$, and 3D scaling factor $ s\in\mathbb{R}^3$. To enhance geometry reconstruction accuracy, the opacity $\alpha$ and 3D rotation $q$ are set to fixed values of $1$ and $(1,0,0,0)$ respectively.
 
%  To render the canonical avatar in observation space, we seamlessly combine the Linear Blend Skinning function with the Gaussian Splatting~\cite{kerbl3Dgaussians} rendering process: 
% \begin{equation}
%   I_{\theta}=Splatting(x_o,Q,d),
%   \label{eq:splatting}
% \end{equation}
% \begin{equation}
%   x_o = T_{lbs}(x_c,p,w),
%   \label{eq:LBS}
% \end{equation}
% where $I_{\theta}$ represents the final rendered image, and the canonical Gaussian position $x_c$ is the sum of the initial position $x$ and the predicted offset $\Delta x$. The LBS function $T_{lbs}$ applies the SMPL skeleton pose $p$ and blending weights $w$ to deform $x_c$ into observation space as $x_o$. $Q$ denotes the remaining attributes of the Gaussians. With the rendering process, they can now reposition these canonical 3D Gaussians into the observation space.



\subsection{Score Distillation Sampling}
Score Distillation Sampling (SDS)~\cite{poole2022dreamfusion} builds a bridge between diffusion models and 3D representations. In SDS, the noised input is denoised in one time-step, and the difference between added noise and predicted noise is considered SDS loss, expressed as:

% \begin{equation}
%   \mathcal{L}_{SDS}(I_{\Phi}) \triangleq E_{t,\epsilon}[w(t)(\epsilon_{\phi}(z_t,y,t)-\epsilon)\frac{\partial I_{\Phi}}{\partial\Phi}],
%   \label{eq:SDSObserv}
% \end{equation}
\begin{equation}
    \mathcal{L}_{\text{SDS}}(I_{\Phi}) \triangleq \mathbb{E}_{t,\epsilon} \left[ w(t) \left( \epsilon_{\phi}(z_t, y, t) - \epsilon \right) \frac{\partial I_{\Phi}}{\partial \Phi} \right],
  \label{eq:SDSObservGA}
\end{equation}
where the input $I_{\Phi}$ represents a rendered image from a 3D representation, such as 3D Gaussians, with optimizable parameters $\Phi$. $\epsilon_{\phi}$ corresponds to the predicted noise of diffusion networks, which is produced by incorporating the noise image $z_t$ as input and conditioning it with a text or image $y$ at timestep $t$. The noise image $z_t$ is derived by introducing noise $\epsilon$ into $I_{\Phi}$ at timestep $t$. The loss is weighted by the diffusion scheduler $w(t)$. 
% \vspace{-3mm}
\section{Methods}
This work uses qualitative methods to explore the communication behaviors of players in \textit{League of Legends (LoL)}. By qualitatively observing and inquiring about player communication decisions as they occur, we aim to extract insights into players' reasoning, strategies, and the underlying factors influencing their choices during the actual conditions of gameplay. We observe \textit{LoL} players during real ranked games, asking them in-the-moment questions as well as follow-up interview questions after the matches have ended to capture the nuances of their communication decisions. 

\subsection{Participants}
We recruited participants for the study through university forums and social media in South Korea. Participants were required to be 18 or older and active players of Solo Ranked mode in \textit{LoL} with a valid rank during the current season at the time of the experiment (Season 2024). The recruitment post notified participants of the observational nature of the study and informed them that they would be expected to speak out loud and answer questions during their play sessions. A total of 36 players completed the recruitment survey, which asked for a self-report of their age, game history, preferred roles, and current rank.

We conducted in-person interview studies with a sample of 22 players. This sample excluded players who had played for less than a year, who were not willing to answer questions during the game, or who were unable to participate in person. From the remaining pool, players were chosen to maximize the diversity and representativeness of the player base based on their rank, experience, and roles. If several players shared similar profiles, we randomly selected between the participants. We conducted 17 interviews through this sampling method. Out of the first 17 participants, 16 participants identified as male, and one identified as female. Consequently, to increase the gender diversity of the sample and ensure that the results reflect a broad range of player experiences and perspectives, we specifically recruited female \textit{LoL} players through snowball sampling, while maintaining diversity in preferred roles and rank. We recruited and interviewed participants from the survey responders until qualitative saturation was reached, following the definition by Braun and Clark~\cite{braun2021saturation}. The final sample consisted of 16 male ($72.7\%$) and 6 female ($27.3\%$) participants. This ratio approximates the imbalanced gender demographic of \textit{LoL}, where estimates have suggested that $80$-$90\%$ of the player base is male~\cite{kordyaka2023gender}. We address the influence of gender identity on communication in Section ~\ref{discussion} and ~\ref{limitations}. The full list of participants and their information is shown in Table ~\ref{table:participant_info}.

The players' age ranged from 20 to 32 years old (mean=$23.7$ years, SD=$3.3$ years) and players' \textit{LoL} experience ranged from 2 to 13 years (mean=$7.7$ years, SD=$3.9$ years). The Solo queue ranks of the players were 2 Iron ($9.1\%$), 2 Bronze ($9.1\%$), 5 Silver ($22.7\%$), 7 Gold ($31.8\%$), 5 Platinum ($22.7\%$), and 1 Emerald ($4.5\%$) at the time of the study. Though the distribution is not as even as the Solo queue rank distribution in the Korean \textit{LoL} server ($12\%$ Iron, $19\%$ Bronze, $16\%$ Silver, $15\%$ Gold, $18\%$ Platinum, $13\%$ Emerald, and $5\%$ Diamond and above~\cite{log2024}), it encompasses the diverse range of skills of most \textit{LoL} players. Thus, the selected players reflected a comprehensive sample of engaged players with varying experience and skill levels. 


\begin{table*}
\centering
\caption{Participant Information and Game Session Information of \textit{League of Legends} Players}
    \label{table:participant_info}
\begin{tabular}{c|c|c|c|c|c|c} 
\toprule
\textbf{ID} & \textbf{Gender} & \textbf{Age} & \textbf{Experience} & \textbf{Solo Rank Tier} & \textbf{Role Played} & \textbf{Game Outcome}  \\ 
\hline
P1          & Male            & 24           & 12 years            & Silver                  & Jungle               & Win                    \\ 
\hline
P2          & Male            & 23           & 3 years             & Silver                  & Top                  & Loss                   \\ 
\hline
P3          & Male            & 32           & 12 years            & Bronze                  & Mid                  & Loss                   \\ 
\hline
P4          & Male            & 29           & 10 years            & Silver                  & Jungle               & Win                    \\ 
\hline
P5          & Male            & 27           & 11 years            & Emerald                 & Bot                  & Loss                   \\ 
\hline
P6          & Male            & 21           & 11 years            & Platinum                & Top                  & Win                    \\ 
\hline
P7          & Male            & 27           & 3 years             & Gold                    & Jungle               & Win                    \\ 
\hline
P8          & Male            & 20           & 9 years             & Gold                    & Bot                  & Win                    \\ 
\hline
P9          & Male            & 25           & 9 years             & Bronze                  & Jungle               & Loss                   \\ 
\hline
P10         & Male            & 23           & 7 years             & Silver                  & Jungle               & Loss                   \\ 
\hline
P11         & Male            & 21           & 12 years            & Gold                    & Mid                  & Loss                   \\ 
\hline
P12         & Male            & 22           & 11 years            & Platinum                & Mid                  & Loss                   \\ 
\hline
P13         & Female          & 24           & 3 years             & Gold                    & Support              & Loss                   \\ 
\hline
P14         & Male            & 25           & 10 years            & Platinum                & Bot                  & Win                    \\ 
\hline
P15         & Male            & 26           & 10 years            & Gold                    & Jungle               & Win                    \\ 
\hline
P16         & Male            & 26           & 13 years            & Gold                    & Mid                  & Loss                   \\ 
\hline
P17         & Male            & 25           & 8 years             & Platinum                & Support              & Loss                   \\
\hline
P18         & Female            & 21           & 3 years             & Platinum                & Support              & Loss                   \\
\hline
P19         & Female            & 20           & 6 years             & Gold                & Support              & Loss                   \\
\hline
P20         & Female            & 20           & 2 years             & Iron                & Top              & Loss                   \\
\hline
P21         & Female            & 20           & 3 years             & Iron                & Jungle              & Win                   \\
\hline
P22         & Female            & 21           & 2 years             & Silver                & Support              & Loss                   \\
\bottomrule
\end{tabular}
\end{table*}

\subsection{Procedure}
To capture the in-game mechanics of communication patterns and dynamically changing communication behavior in \textit{LoL}, we conducted an in-person observation and interview study. The study was conducted with the approval of the Institutional Review Board at the first author's research institution.

\subsubsection{Study Environment}

Each participant was asked to play a Solo Ranked game of \textit{LoL} while researchers observed and inquired about their actions in real time. In the process of study design, alternative study setups were considered. An initial plan to observe participants remotely through screen sharing was discarded as pilot studies revealed that latency and network issues were significantly disruptive to the researchers’ ability to observe and ask questions in a timely manner. The research team also decided not to recruit participants to observe in their own homes due to concerns about privacy and safety. Thus, to better gather responses to in-the-moment inquiries from the participants without any latency, the study was held in person within a controlled research environment to enhance the clarity and quality of the question-and-answer process. This approach also allowed researchers to note the participant's gaze, hesitation, and other subtle movements that would be missed in remote settings. Though an in-lab study does not perfectly recreate the in-home environment, the research team determined that this option best balanced the various tradeoffs.


The research team worked to ensure that the study environment best suited each participant’s preferences in the following steps. We first conducted several pilot interviews to tailor the play space to prioritize player comfort and approximate real-life conditions. The study took place in an enclosed room --- frequently used for interviews and user studies --- equipped with large desks and office chairs. We limited natural light and turned on overhead lights for visibility. Moreover, players were individually asked if the environment felt natural and comfortable, and adjustments were made based on their feedback. We set up the study environment with equipment (mouse, keyboard, monitor) designed specifically for online gaming. The players were also permitted to bring personal equipment if they wished. Before entering the game, the players were instructed to adjust both the equipment (e.g., mouse sensitivity) and in-game settings (e.g., shortcut keys). All players were given as much time as needed until they expressed satisfaction with the setup. At the end of the study, we asked players if anything about the setup or procedure had negatively impacted their gameplay. Three players reported feeling some discomfort from using unfamiliar equipment but noted that it did not affect their typical playstyle or their answers. The participants were compensated 20,000 KRW (approximately 15 USD) for completing the study. 


\subsubsection{Interview Process}

The researchers observed the game session through a separate screen connected to the player's monitor and noted any communication actions, attempts, and responses by the player. During the study orientation, researchers emphasized that participants should play and communicate naturally, including using offensive language, muting or reporting other players, or forfeiting the game if desired. Participants were assured that all data would be anonymized for analysis. To minimize distractions, they were informed that they could skip questions if they found them intrusive or preferred not to respond. During intense in-game situations in which the participant could not answer, the researchers documented the context and either repeated the question once the game state had stabilized or immediately after the match ended. We discuss the limitations of using an observational approach to study in-game communication, such as social desirability bias~\cite{grimm2010social}, in Section ~\ref{limitations}.

The researcher observed the communication between teammates, noting what triggered the communication or what communication medium was used for different purposes. Based on these observations, the participant was asked questions about why they did or did not perform certain communication actions to understand their assessment and perception of the communication. Some of the questions were asked to all participants, such as the reasoning behind the frequency of using certain communication media and their perception towards teammates who engage in certain forms of communication. Other non-structured questions were asked when the player triggered a certain action (``\textit{You just pinged your ally with Enemy Missing ping multiple times. What was the purpose?''}) or responded to (or ignored) their team's communication (``\textit{It seems that you opted to not vote for the surrender vote. Why is this so?}'').

The researchers recorded the gameplay and thoroughly transcribed observations and game states during the game. The observations included types of communication media used (chat, ping, votes, emotes), the target of the communication (if unclear, then players were asked to clarify the target), player reactions to ongoing discourse or communication usage by their team, and physical reactions such as spoken utterances and body gestures. After completing the game, participants engaged in a 15 to 20-minute post-game interview. They were asked to reflect on their in-game communication behavior and perceptions, including the motivations behind their communication tendencies and choices. The interview also inquired how their teammates' communication behaviors influenced their perception of those teammates, as well as the overall impact of such interactions on their mental state or performance. Participants were further invited to share suggestions for improving communication in \textit{LoL}, such as the potential addition of voice chat. The full list of the interview questions is provided in Appendix ~\ref{appendix2}.

Overall, a total of 24 games were played, of which two were forfeited within 20 minutes. Players were asked to play another game if their first game ended within 20 minutes due to a surrender vote from either team as these games did not demonstrate communication across all stages of the game. The other 22 games lasted a minimum of 24 minutes, most of which were played to completion without forfeit from either team or with a forfeit when the victor was very clearly determined.


\subsection{Thematic Analysis}
We conducted an inductive thematic analysis applying the methodology from Elo and  Kyngäs~\cite{elo2008qualitative}. The researchers gathered the transcript of the in-game and post-interview, video recording and the replay file of the game, and observational notes for each participant. We incorporated the notes on participant behavior, game states, and other players' communication patterns into the transcript at its corresponding times, providing contextual information on what was happening at the time of the question or the player's reaction.

Before initiating coding, the first, second, and third authors familiarized themselves with the data collected. The three authors then independently performed line-by-line open coding on eight participants' data to identify preliminary themes. After the initial coding was completed, the authors shared the codes to combine convergent ideas and discuss any differing perspectives. The first author then validated the codes on the remaining data, engaging in discussions with the second and third authors to iterate on the codebook. The final codebook contained 55 codes with 13 categories, organized by themes that answer each research question in depth. For RQ1, we find themes of communication types and what triggers or deters a player's decision to communicate. For RQ2, we categorize factors used by players to assess communication opportunities as well as reactions based on such assessment. For RQ3, we find themes on how the team's communication behaviors affect a player's perception towards other teammates during the game. We provide the final codebook in Appendix ~\ref{appendix}. 
\section{Experimental Evaluation}\label{section:experiments}
We already achieved our primary objective of deriving time-series-specific subsampling guarantees for DP-SGD adapted to forecasting.
Our main goal for this section is to investigate the trade-offs we discovered in discussing these guarantees.
In addition, we train common probabilistic forecasting architectures on standard datasets to verify the feasibility of training deep differentially private forecasting models while retaining meaningful utility.
The full experimental setup  is described in~\cref{appendix:experimental_setup}.
%An implementation will be made available upon publication.

\subsection{Trade-Offs in Structured Subsampling}

\begin{figure}
    \vskip 0.2in
    \centering
        \includegraphics[width=0.99\linewidth]{figures/experiments/eval_pld_deterministic_vs_random_top_level/daily_20_32_main.pdf}
        \vskip -0.3cm
        \caption{Top-level deterministic iteration (\cref{theorem:deterministic_top_level_wr}) vs top-level WOR sampling (\cref{theorem:wor_top_level_wr}) for $\numinstances=1$.
        Sampling is more private despite requiring more compositions.}
        \label{fig:deterministic_vs_random_top_level_daily_main}
    \vskip -0.2in
\end{figure}




For the following experiments, we assume that we have $N=320$ sequences, batch size $\batchsize = 32$, and noise scale $\sigma = 1$.
We further assume $L=10  (L_F + L_C) + L_F - 1$, so that 
the chance of bottom-level sampling a subsequence containing any specific element is 
$r=0.1$ when choosing $\numinstances = 1$ as the number of subsequences.
In~\cref{appendix:extra_experiments_eval_pld}, we repeat all experiments with a wider range of parameters.
All results are consistent with the ones shown here.

\textbf{Number of Subsequences $\bm{\numinstances}$.}
Let us begin with a trade-off inherent to bi-level subsampling:
We can achieve the same batch size $\batchsize$ with different $\numinstances$, each
leading to different top- and bottom-level amplification.
We claim that $\numinstances = 1$ (i.e., maximum bottom-level amplification) is preferable.
For a fair comparison, we compare our provably tight guarantee for $\numinstances=1$ (\cref{theorem:wor_top_level_wr})
with optimistic lower bounds for $\numinstances > 1$ (\cref{theorem:wor_top_wr_bottom_upper})
instead of our sound upper bounds (\cref{theorem:wor_top_level_wr_general}), i.e.,
we make the competitors stronger.
As shown in~\cref{fig:monotonicity_daily_main}, $\numinstances = 1$ only has smaller $\delta(\epsilon)$ for $\epsilon \geq 10^{-1}$ when considering a single training step.
However, after $100$-fold composition, $\numinstances = 1$ achieves smaller $\delta(\epsilon)$ even in $[10^{-3}, 10^{-1}]$ (see~\cref{fig:monotonicity_composed_daily_main}).
Our explanation is that $\numinstances > 1$ results in larger $\delta(\epsilon)$ for large $\epsilon$, i.e., is more likely to have a large privacy loss.
Because the privacy loss of a composed mechanism is the sum of component privacy losses~\cite{sommer2018privacy}, this is problematic when performing multiple training steps.
We shall thus later use $\numinstances=1$ for training.

%Intuitively, $\delta(\epsilon)$ can be interpreted as the probability that the log-likelihood ratio of $M_x$ and $M_{x'}$ (``privacy loss'') exceeds $\epsilon$.\footnote{For the formal relation between privay loss and privacy profiles, see~\cref{lemma:profile_from_pld} taken from~\cite{balle2018improving}}


\textbf{Step- vs Epoch-Level Accounting.}
Next, we show the benefit of top-level sampling sequences (\cref{theorem:wor_top_level_wr}) instead of deterministically iterating over them (\cref{theorem:deterministic_top_level_wr}), even though we risk privacy leakage at every training step.
For our parameterization and $\numinstances=1$, top-level sampling with replacement requires $10$ compositions per epoch.
As shown in~\cref{fig:deterministic_vs_random_top_level_daily_main}, the resultant epoch-level profile is nevertheless smaller, and remains so after $10$ epochs.
This is consistent with any work on DP-SGD (e.g., \cite{abadi2016deep}) that uses subsampling instead of deterministic iteration.

\textbf{Epoch Privacy vs Length.} In~\cref{appendix:extra_experiments_epoch_length} we additionally explore the fact that, if we wanted to use deterministic top-level iteration, 
the number of subsequences 
$\numinstances$ would affect epoch length.
As expected, we observe that composing many private mechanisms ($\numinstances=1$) is preferable to composing few much less private mechanisms ($\numinstances > 1$) 
when considering a fixed number of training steps.

\begin{figure}
    \vskip 0.2in
    \centering
        \includegraphics[width=0.99\linewidth]{figures/experiments/eval_pld_label_noise/daily_30_32_main.pdf}
        \vskip -0.3cm
        \caption{Varying label noise $\sigma_F$ for top-level WOR and bottom-level WR  (\cref{theorem:data_augmentation_general}) with $\sigma_C = 0, \numinstances=1$.
        Increasing $\sigma_F$ is equivalent to decreasing forecast length.
        }
        \label{fig:label_noise_daily_main}
    \vskip -0.2in
\end{figure}

\textbf{Amplification by Label Perturbation.}
Finally, because the way in which adding Gaussian noise to the context and/or forecast window 
amplifies privacy (\cref{theorem:data_augmentation_general}) 
may be somewhat opaque, let us consider top-level sampling without replacement, bottom-level sampling with replacement,
$\numinstances=1$, $\sigma_C=0$, and varying label noise standard deviations $\sigma_F$. 
As shown in~\cref{fig:label_noise_daily_main}, increasing $\sigma_F$ has the same effect as letting the forecast length $L_C$ go to zero, i.e., eliminates the risk of leaking private information if it appears in the forecast window.
Of course, this data augmentation 
will have an effect on model utility, which we investigate shortly.

\begin{figure*}
\centering
\vskip 0.2in
    \begin{subfigure}{0.49\textwidth}
        \includegraphics[]{figures/experiments/eval_pld_monotonicity_composed/daily_20_32_1_main.pdf}
        \caption{Training step $1$}\label{fig:monotonicity_daily_main}
    \end{subfigure}
    \hfill
    \begin{subfigure}{0.49\textwidth}
        \includegraphics[]{figures/experiments/eval_pld_monotonicity_composed/daily_20_32_100_main.pdf}
        \caption{Training step $100$}\label{fig:monotonicity_composed_daily_main}
    \end{subfigure}\caption{
    Top-level WOR and bottom-level WR sampling under varying number of subsequences.
    Under composition, even optimistic lower bounds (\cref{theorem:wor_top_wr_bottom_upper}) 
    indicate worse privacy for $\numinstances > 1$ than 
    our tight upper bound for $\numinstances=1$ (\cref{theorem:wor_top_level_wr}).}
    \label{fig:monotonicity_daily_main_container}
\vskip -0.2in
\end{figure*}


\subsection{Application to Probabilistic Forecasting}
While the contribution of our work lies in formally analyzing the privacy of DP-SGD adapted to forecasting, 
training models with this algorithm can serve as a sanity-check to verify that the guarantees are sufficiently strong to retain meaningful utility under non-trivial privacy budgets.


\begin{table}[b]
\vskip -0.38cm
\caption{Average CRPS on \texttt{traffic} for $\delta=10^{-7}$. Seasonal, AutoETS, and models with $\epsilon=\infty$ are without noise.}
\label{table:1_event_training_traffic_main}
\vskip 0.18cm
\begin{center}
\begin{small}
\begin{sc}
\begin{tabular}{lcccc}
\toprule
Model & $\epsilon = 0.5$ & $\epsilon = 1$ & $\epsilon = 2$ &  $\epsilon = \infty$ \\
\midrule
SimpleFF & $0.207$ & $0.195$ & $0.193$ & $0.136$ \\ 
DeepAR & $\mathbf{0.157}$ & $\mathbf{0.145}$ & $\mathbf{0.142}$ & $\mathbf{0.124}$ \\
iTransf. & $0.211$ & $0.193$ & $0.188$ & $0.135$ \\
DLinear & $0.204$ & $0.192$ & $0.188$ & $0.140$ \\
\midrule
Seasonal   & $0.251$ & $0.251$ & $0.251$ & $0.251$\\
AutoETS   & $0.407$ & $0.407$ & $0.407$ & $0.407$\\
\bottomrule
\end{tabular}
\end{sc}
\end{small}
\end{center}
\vskip -0.1in
\end{table}

\textbf{Datasets, Models, and Metrics.}
We consider three standard benchmarks: \texttt{traffic}, \texttt{electricity}, and \texttt{solar\_10\_minutes} as used in~\cite{Lai2018modeling}.
We further consider four common architectures: 
A two-layer feed-forward neural network (``SimpleFeedForward''), a recurrent neural network (``DeepAR''~\cite{salinas2020deepar}),
an encoder-only transformer (``iTransformer''~\cite{liu2024itransformer}), and a refined feed-forward network proposed to compete with attention-based models (``DLinear''~\cite{zeng2023transformers}).
We let these architectures parameterize elementwise $t$-distributions to obtain probabilistic forecasts.
We measure the quality of these probabilistic forecasts using continuous ranked probability scores (CRPS), which we approximate via mean weighted quantile losses (details in~\cref{appendix:metrics}).
As a reference for what constitutes ``meaningful utility'', we compare against seasonal na\"{i}ve forecasting and exponential smoothing (``AutoETS'') without introducing any noise.
All experiments are repeated with $5$ random seeds.


\textbf{Event-Level Privacy.} \cref{table:1_event_training_traffic_main} shows CRPS of all models on the \texttt{traffic} test set 
when setting $\delta=10^{-7}$, and training on the training set until reaching a pre-specified $\epsilon$
with $1$-event-level privacy. For the other datasets and standard deviations, see~\cref{appendix:privacy_utility_tradeoff_event_level_privacy}.
The column $\epsilon=\infty$ indicates non-DP training.
As can be seen, models can retain much of their utility and outperform the baselines, even for $\epsilon \leq 1$ which is generally considered a small privacy budget~\cite{ponomareva2023dp}.
For instance, the average CRPS of DeepAR on the traffic dataset is $0.124$ with non-DP training and $0.157$ for $\epsilon=0.5$.
Note that, since all models are trained using  our tight privacy analysis,
which specific model performs best  on which specific dataset is orthogonal to our contribution. 

\textbf{Other results.}
In~\cref{appendix:privacy_utility_tradeoff_user_level_privacy} we additionally train with $w$-event and $w$-user privacy.
In~\cref{appendix:privacy_utility_tradeoff_label_privacy}, we demonstrate that label perturbations can offer an improved privacy--utility trade-off. 
All results confirm that our guarantees for DP-SGD adapted to forecasting are strong enough to enable provably private training while retaining utility.


\section{Conclusion}
\label{sec:Conclusion}
In this paper, we proposed a complete real-time planning and control approach for continuous, reliable, and fast online generation of dynamically feasible Bernstein trajectories and control for FW aircrafts. The generated trajectories span kilometers, navigating through multiple waypoints. By leveraging differential flatness equations for coordinated flight, we ensure precise trajectory tracking. Our approach guarantees smooth transitions from simulation to real-world applications, enabling timely field deployment. The system also features a user-friendly mission planning interface. Continuous replanning  maintains the rajectory curvature 
$\kappa$ within limits, preventing abrupt roll changes.

Future works will include the ability to add  a higher-level kinodynamic path planner to optimize waypoint spatial allocation and improve replanning success, and enhancing the trajectory-tracking algorithm by refining the aerodynamic coefficient estimation. 


\bibliography{beyond_ft_paper}
\bibliographystyle{icml2025}


%%%%%%%%%%%%%%%%%%%%%%%%%%%%%%%%%%%%%%%%%%%%%%%%%%%%%%%%%%%%%%%%%%%%%%%%%%%%%%%
%%%%%%%%%%%%%%%%%%%%%%%%%%%%%%%%%%%%%%%%%%%%%%%%%%%%%%%%%%%%%%%%%%%%%%%%%%%%%%%
% APPENDIX
%%%%%%%%%%%%%%%%%%%%%%%%%%%%%%%%%%%%%%%%%%%%%%%%%%%%%%%%%%%%%%%%%%%%%%%%%%%%%%%
%%%%%%%%%%%%%%%%%%%%%%%%%%%%%%%%%%%%%%%%%%%%%%%%%%%%%%%%%%%%%%%%%%%%%%%%%%%%%%%
\newpage
\appendix
\onecolumn

\section{Related Work} \label{app:related_work}
\textbf{Personalized Generation} 
Due to the considerable success of large text-to-image models \cite{ramesh2022hierarchical, ramesh2021zero, saharia2022photorealistic, rombach2022high}, the field of personalized generation has been actively developed. The challenge is to customize a text-to-image model to generate specific concepts that are specified using several input images. Many different approaches \cite{DB, TI, CD, svdiff, ortogonal, profusion, elite, r1e} have been proposed to solve this problem and can be divided into the following groups: pseudo-token optimization \cite{TI, profusion, disenbooth, r1e}, diffusion fune-tuning \cite{DB, CD, profusion}, and encoder-based \cite{elite}. The pseudo-token paradigm adjusts the text encoder to convert the concept token into the proper embedding for the diffusion model. Such embedding can be optimized directly \cite{TI, r1e} or can be generated by other neural networks \cite{disenbooth, profusion}. Such approaches usually require a small number of parameters to optimize but lose the visual features of the target concept. Diffusion fine-tuning-based methods optimize almost all \cite{DB} or parts \cite{CD} of the model to reconstruct the training images of the concept. This allows the model to learn the input concept with high accuracy, but the model due to overfitting may lose the ability to edit it when generated with different text prompts. To reduce overfitting and memory usage, lightweight parameterizations \cite{svdiff, r1e, lora} have been proposed that preserve edibility but at the cost of degrading concept fidelity. Encoder-based methods \cite{elite} allow one forward pass of an encoder that has been trained on a large dataset of many different objects to embed the input concept. This dramatically speeds up the process of learning a new concept and such a model is highly editable, but the quality of recovering concept details may be low. Generally, the main problem with existing personalized generation approaches is that they struggle to simultaneously recover a concept with high quality and generate it in a variety of scenes.

\textbf{Sampling strategies}
Much research has been devoted to sampling techniques for text-to-image diffusion models, focusing not only on personalized generation but also on image editing. In this paper, we address a more specific question: how can the two trajectories -- superclass and concept -- be optimally combined to achieve both high concept fidelity and high editability? The ProFusion paper \cite{profusion} considered one way of combining these trajectories (Mixed sampling), which we analyze in detail in our paper (see Section \ref{sec:mixed_sampling}) and show its properties and problems. In ProFusion, authors additionally proposed a more complex sampling procedure, which we observed to be redundant compared to Mixed sampling, as can be seen in our experiments (see Section \ref{sec:experiments}). In Photoswap \cite{photoswap}, authors consider another way of combining trajectories by superclass and concept, which turns out to be almost identical to the Switching sampling strategy that we discuss in detail in Section \ref{sec:switching_sampling}. We show why this strategy fails to achieve simultaneous improvements in concept reconstruction and editability. In the paper, we propose a more efficient way of combining these two trajectories that achieves an optimal balance between the two key features of personalized generation: concept reconstruction and editability.

\section{Training details} \label{sec:training-details}
The Stable Diffusion-2-base model is used for all experiments. For the Dreambooth, Custom Diffusion, and Textual Inversion methods, we used the implementation from \url{https://github.com/huggingface/diffusers}.

\textbf{SVDiff} We implement the method based on \url{https://github.com/mkshing/svdiff-pytorch}. The parameterization is applied to all Text Encoder and U-Net layers. The models for all concepts were trained for $1600$ using Adam optimizer with $\text{batch size} = 1$, $\text{learning rate} = 0.001$, $\text{learning rate 1d} = 0.000001$, $\text{betas} = (0.9, 0.999)$, $\text{epsilon} = 1e\!-\!8$, and $\text{weight decay} = 0.01$. 

\textbf{Dreambooth} All query, key, and value layers in Text Encoder and U-Net were trained during fine-tuning. The models for all concepts were trained for $400$ steps using Adam optimizer with $\text{batch size} = 1$, $\text{learning rate} = 2e\!-\!5$, $\text{betas} = (0.9, 0.999)$, $\text{epsilon} = 1e\!-\!8$, and $\text{weight decay} = 0.01$. 

\textbf{Custom Diffusion} The models for all concepts were trained for $1600$ steps using Adam optimizer with $\text{batch size} = 1$, $\text{learning rate} = 0.00001$, $\text{betas} = (0.9, 0.999)$, $\text{epsilon} = 1e\!-\!8$, and $\text{weight decay} = 0.01$. 

\textbf{Textual Inversion} The models for all concepts were trained for $10000$ steps using Adam optimizer with $\text{batch size} = 1$, $\text{learning rate} = 0.005$, $\text{betas} = (0.9, 0.999)$, $\text{epsilon} = 1e\!-\!8$, and $\text{weight decay} = 0.01$. 

\textbf{ELITE} We used the pre-trained model from the official repo \url{https://github.com/csyxwei/ELITE} with $\lambda=0.6$ and inference hyperparams from the original paper.

\clearpage
\section{Superclass and concept trajectory choice}\label{app:hyper_theta}

 \begin{wrapfigure}{r}{0.45\textwidth}
    % \centering
    \includegraphics[trim={3cm 10cm 3cm 10cm},clip,width=\linewidth]{imgs/mixed_noft_nosup.pdf}
    \caption{The Pareto frontiers for original Mixed sampling and Mixed sampling in the Superclass, NoFT, and Empty Prompt setups. Mixed NoFT and Mixed Empty Prompt configurations overlap with the Pareto frontier of the original mixed sampling, but primarily in regions associated with low image similarity, which compromises concept fidelity.} \label{fig:mixed_noft_ep}
    \vspace{-0.14in}
\end{wrapfigure}

There are multiple ways to define sampling with maximized textual alignment to the prompt. However, the arbitrary choice can harm the alignment between Base sampling~\ref{eq:concept_sampling} and the selected trajectory. We use the Sampling with superclass (\ref{eq:superclass_sampling}) as it's the default choice in the literature and guarantees the maximized alignment between noise predictions $\tilde{\varepsilon}_{\theta}(p^C)$ and $\tilde{\varepsilon}_{\theta}(p^S)$. 

The several natural ways to adjust Sampling with superclass can be presented by varying $\theta$ and $p^{S}$ in (\ref{eq:superclass_sampling}). We explore two additional options with decreased alignment with (\ref{eq:concept_sampling}): (1) NoFT -- weights of base model $\theta^{\text{orig}}$ instead fine-tuned weights, (2) Empty Prompt -- prompt without any reference to a concept, even to its superclass category, i.e. $p^{\hat{S}} = \textit{"with a city in the background"}$ instead of $p^{S} = \textit{"a backpack with a city in the background"}$.

To validate the robustness of our framework for sampling method selection, we employ the original experimental protocol, supplementing the results shown in Figures~\ref{fig:examples} and~\ref{fig:profusion-photoswap}. Our analysis of Figures~\ref{fig:multi-stage_noft_ep} and~\ref{fig:masked_noft_ep} reveals that trajectories generated under the NoFT and Empty Prompt configurations (second and third columns, respectively) maintain identical method ordering to those produced by Superclass sampling ((\ref{eq:superclass_sampling}), first column).

Notably, Figure~\ref{fig:mixed_noft_ep} shows that Empty Prompt configuration demonstrates weaker alignment with Base sampling compared to NoFT, particularly at higher values of the superclass guidance scale $\omega_{s}$. This divergence manifests as reduced concept fidelity for Empty Prompt under large $\omega_{s}$. These findings highlight a practical adjustment: prioritizing smaller $\omega_{s}$ values in Empty Prompt setup preserves concept fidelity without altering the framework’s core selection logic. 

A key limitation of increased misalignment is the gradual erosion of superclass category information from generated images, which can lead to semantically inconsistent outputs. For instance, Figure~\ref{fig:examples_noft_ep} illustrates how the Mixed Empty Prompt setup, despite the strong animal prior in Base sampling, can produce human-like features in an image of a cat described as \textit{"in a chef outfit"}. This suggests that when superclass information is weakened, the model may introduce unexpected visual artifacts, impacting the fidelity of the intended concept.

Concept sampling (\ref{eq:concept_sampling}) can also be adjusted to better capture a concept’s visual characteristics, further decoupling fidelity from editability. For example, this can be achieved by (1) using the weights of a highly overfitted model (e.g., DreamBooth) or (2) selecting a prompt that omits contextual details, such as $p^{\hat{C}} = \textit{"a photo of V*"}$ instead of $p^{C} = \textit{"a V* with a city in the background"}$. Combining superclass sampling under NoFT or Empty Prompt with Base sampling configured via (1) or (2) could enhance both image and text similarity. We leave this direction for future work.

\begin{figure}[b]
    \centering
    \includegraphics[trim={3cm 10cm 3cm 10cm},clip,width=0.32\linewidth]{imgs/multi-stage_original.pdf}
    \hfill
    \includegraphics[trim={3cm 10cm 3cm 10cm},clip,width=0.32\linewidth]{imgs/multi-stage_noft.pdf}
    \hfill
    \includegraphics[trim={3cm 10cm 3cm 10cm},clip,width=0.32\linewidth]{imgs/multi-stage_nosup.pdf}
    \caption{Pareto Frontier Curves for Mixed, Switching, and Multi-Stage Sampling Methods in the Superclass, NoFT and Empty Prompt setups.
The NoFT and Empty Prompt configurations (second and third columns, respectively) preserve the same method ordering as those produced by Superclass sampling (first column).} \label{fig:multi-stage_noft_ep}
\end{figure}
\begin{figure}[t]
    \centering
    \includegraphics[trim={3cm 10cm 3cm 10cm},clip,width=0.32\linewidth]{imgs/masked_profusion.pdf}
    \hfill
    \includegraphics[trim={3cm 10cm 3cm 10cm},clip,width=0.32\linewidth]{imgs/masked_noft.pdf}
    \hfill
    \includegraphics[trim={3cm 10cm 3cm 10cm},clip,width=0.32\linewidth]{imgs/masked_nosup.pdf}
    \caption{Pareto Frontier Curves for Mixed, Switching, Masked, and ProFusion Sampling Methods in the Superclass, NoFT, and Empty Prompt setups.
The NoFT and Empty Prompt configurations (second and third columns, respectively) preserve the same method ordering as those produced by Superclass sampling (first column).} \label{fig:masked_noft_ep}
\end{figure}

\begin{figure}[b]
    \centering
    \includegraphics[width=\linewidth]{imgs/examples_noft_ep.pdf}
    \caption{Examples of the generation outputs for Mixed and ProFusion sampling methods for their optimal metrics point in the Superclass, NoFT, and Empty Prompt (EP) setups.} \label{fig:examples_noft_ep}
\end{figure}

\clearpage

\begin{figure}[ht!]
  \centering
  \includegraphics[trim={0 5cm 0 5cm},clip,width=0.95\linewidth]{imgs/us_example_new.pdf}
  \caption{An example of a task in the user study}
  \label{fig:us_ex}
  \vspace{-0.19in}
\end{figure}

\section{Data preparation}\label{app:data}
For each concept, we used inpainting augmentations to create the training dataset. We took an original image and automatically segmented it using the Segment Anything model on top of the CLIP cross-attention maps. Then we crop the concept from the original image, apply affine transformations to it, and inpaint the background. We used $10$ augmentation prompts, different from the evaluation prompts, and sampled $3$ images per prompt, resulting in a total of $30$ training images per concept. We commit to open-source the augmented datasets for each concept after publication.

\section{User Study}\label{app:us}

An example task from the user study is shown in Figure~\ref{fig:us_ex}. In total, we collected 48,864 responses from 200 unique users for 16,000 unique pairs. For each task, users were asked three questions: 1) "Which image is more consistent with the text prompt?" 2) "Which image better represents the original image?" 3) "Which image is generally better in terms of alignment with the prompt and concept identity preservation?" For each question, users selected one of three responses: "1", "2", or "Can't decide."

\section{Complex Prompts Setting}\label{app:long_prompts}

We conduct a comparison of different sampling methods using a set of complex prompts. For this analysis, we collected 10 prompts, each featuring multiple scene changes simultaneously, including stylization, background, and outfit:

\adjustbox{max width=\linewidth}{
\begin{lstlisting}
live_long = [
  "V* in a chief outfit in a nostalgic kitchen filled with vintage furniture and scattered biscuit",
  "V* sitting on a windowsill in Tokyo at dusk, illuminated by neon city lights, using neon color palette",
  "a vintage-style illustration of a V* sitting on a cobblestone street in Paris during a rainy evening, showcasing muted tones and soft grays",
  "an anime drawing of a V* dressed in a superhero cape, soaring through the skies above a bustling city during a sunset",
  "a cartoonish illustration of a V* dressed as a ballerina performing on a stage in the spotlight",
  "oil painting of a V* in Seattle during a snowy full moon night",
  "a digital painting of a V* in a wizard's robe in a magical forest at midnight, accented with purples and sparkling silver tones",
  "a drawing of a V* wearing a space helmet, floating among stars in a cosmic landscape during a starry night",
  "a V* in a detective outfit in a foggy London street during a rainy evening, using muted grays and blues",
  "a V* wearing a pirate hat exploring a sandy beach at the sunset with a boat floating in the background",
]

object_long = [
  "a digital illustration of a V* on a windowsill in Tokyo at dusk, illuminated by neon city lights, using neon color palette",
  "a sketch of a V* on a sofa in a cozy living room, rendered in warm tones",
  "a watercolor painting of a V* on a wooden table in a sunny backyard, surrounded by flowers and butterflies",
  "a V* floating in a bathtub filled with bubbles and illuminated by the warm glow of evening sunlight filtering through a nearby window",
  "a charcoal sketch of a giant V* surrounded by floating clouds during a starry night, where the moonlight creates an ethereal glow",
  "oil painting of a V* in Seattle during a snowy full moon night",
  "a drawing of a V* floating among stars in a cosmic landscape during a starry night with a spacecraft in the background",
  "a V* on a sandy beach next to the sand castle at the sunset with a floaing boat in the background",
  "an anime drawing V* on top of a white rug in the forest with a small wooden house in the background",
  "a vintage-style illustration of a V* on a cobblestone street in Paris during a rainy evening, showcasing muted tones and soft grays",
]
\end{lstlisting}
}

The results of this comparison are presented in Figures~\ref{fig:add_long},~\ref{fig:add_long_metrics}. We observe that Base sampling may struggle to preserve all the features specified by the prompts, whereas advanced sampling techniques effectively restore them. The overall arrangement of methods in the metric space closely mirrors that observed in the setting with simple prompts.

\begin{figure}[h!]
  \centering
  \includegraphics[width=\linewidth]{imgs/long_prompts_examples.pdf}
  \caption{Additional examples of the generation outputs for different sampling methods with \textbf{complex prompts}. We highlight parts of the prompt that are missing in Base sampling while appearing in other methods.}
  \label{fig:add_long}
\end{figure}

\clearpage
\section{Dreambooth results}\label{app:dreambooth}

We conduct additional analysis of different sampling methods in combination with Dreambooth. Figure~\ref{fig:add_db_metrics} shows that Mixed Sampling still overperforms Switching and Photoswap,  while Multi-stage and Masked struggle to provide an additional improvement over the simple baseline. Figure~\ref{fig:add_db} shows that all methods allow for improvement TS with a negligent decrease in IS while Mixed Sampling provides the best IS among all samplings.

\begin{figure*}[!ht]
\centering
\begin{minipage}{.477\textwidth}
  \centering
  \includegraphics[trim={3cm 10cm 3cm 10cm},clip,width=\linewidth]{imgs/long_prompts.pdf}
  \caption{CLIP metrics for different sampling methods estimated on \textbf{complex prompts}.}
  \label{fig:add_long_metrics}
\end{minipage}
\hfill
\begin{minipage}{.477\textwidth}
  \centering
  \includegraphics[trim={3cm 10cm 3cm 10cm},clip,width=\linewidth]{imgs/db_samplings.pdf}
  \caption{CLIP metrics for different sampling strategies on top of a Dreambooth fine-tuning method.}
  \label{fig:add_db_metrics}
\end{minipage}
\end{figure*} 

\begin{figure}[h!]
  \centering
  \includegraphics[trim={0 1cm 0 1cm},clip,width=\linewidth]{imgs/db_sampling_examples.pdf}
  \caption{Additional examples of the generation outputs for different sampling methods on top of a Dreambooth fine-tuning method.}
  \label{fig:add_db}
\end{figure}

\section{PixArt-alpha \& SD-XL}\label{app:add_backbones}
We conducted a series of experiments using different backbones. For SD-XL~\cite{podell2023sdxlimprovinglatentdiffusion}, we used SVDDiff as the fine-tuning method, while PixArt-alpha~\citep{chen2023pixartalphafasttrainingdiffusion} employed standard Dreambooth training. Hyperparameters for Switching, Masked, and ProFusion were selected in the same manner as in the experiments with SD2.

Figures~\ref{fig:pixart} and~\ref{fig:sdxl} demonstrate that Mixed Sampling follows a similar pattern to SD2, improving TS without a significant loss in IS. Notably, Mixed Sampling for SD-XL achieves simultaneous improvements in both IS and TS. ProFusion exhibits behavior consistent with SD2, enhancing IS more effectively than Mixed Sampling but performing worse at improving TS while also requiring twice the computational resources. 

\begin{figure}[h]
\centering
\begin{minipage}{.49\textwidth}
  \centering
  \includegraphics[trim={3cm 10cm 3cm 10cm},clip,width=\linewidth]{imgs/pixart.pdf}
  \captionof{figure}{CLIP metrics for different sampling methods estimated on PixArt model.}
  \label{fig:pixart}
\end{minipage}%
\hfill
\begin{minipage}{.49\textwidth}
  \centering
  \includegraphics[trim={3cm 10cm 3cm 10cm},clip,width=\linewidth]{imgs/sdxl.pdf}
  \captionof{figure}{CLIP metrics for different sampling methods estimated on SD-XL model.}
  \label{fig:sdxl}
\end{minipage}
\end{figure}

\clearpage
\section{Cross-Attention Masks}\label{app:cross_attn}

\begin{figure}[h!]
  \centering
  \includegraphics[trim={3cm 0cm 3cm 0cm},clip,width=\linewidth]{imgs/cross_attention_masks.pdf}
  \caption{Visualization of the cross-attention masks for Masked sampling examples. Here, $q$ defines the thresholding quantile and $t$ the denoising step.}
  \label{fig:cross_attn_add_ex}
\end{figure}

\clearpage
\section{Additional Examples}\label{app:add_example}

\begin{figure}[h!]
  \centering
  \includegraphics[trim={0 2cm 0 2cm},clip,width=\linewidth]{imgs/additional_examples.pdf}
  \caption{Additional examples of the generation outputs for different sampling methods.}
  \label{fig:add_ex}
\end{figure}

\begin{figure}[h!]
  \centering
  \includegraphics[trim={0 2cm 0 2cm},clip,width=\linewidth]{imgs/additional_examples_all.pdf}
  \caption{Additional examples of the generation outputs for Mixed and ProFusion sampling methods in comparison to the main personalized generation baselines.}
  \label{fig:add_ex_all}
\end{figure}

\clearpage
\section{DINO Image Similarity}\label{app:add_dino}

We compare CLIP-IS (left column) and DINO-IS~\citep{oquab2024dinov2learningrobustvisual} (right column) in Figures~\ref{fig:profusion_photoswap_dino},~\ref{fig:all_methods_dino}. We observe that despite the choice of metric, different sampling techniques and finetuning strategies have the same arrangement. The most noticeable difference is that SVDDiff superiority over ELITE and TI is more pronounced. That strengthens our motivation to select SVDDiff as the main backbone.

\begin{figure}[h]
\centering
\begin{minipage}{.49\textwidth}
  \centering
  \includegraphics[trim={3cm 10cm 3cm 10cm},clip,width=\linewidth]{imgs/profusion_photoswap.pdf}
\end{minipage}%
\hfill
\begin{minipage}{.49\textwidth}
  \centering
  \includegraphics[trim={3cm 10cm 3cm 10cm},clip,width=\linewidth]{imgs/profusion_photoswap_dino.pdf}
\end{minipage}
\caption{Pareto frontiers curves for Photoswap~\citep{photoswap} and ProFusion~\citep{profusion}.}\label{fig:profusion_photoswap_dino}
\end{figure}

\begin{figure}[h]
\centering
\begin{minipage}{.49\textwidth}
  \centering
  \includegraphics[trim={3cm 10cm 3cm 10cm},clip,width=\linewidth]{imgs/all_methods.pdf}
\end{minipage}%
\hfill
\begin{minipage}{.49\textwidth}
  \centering
  \includegraphics[trim={3cm 10cm 3cm 10cm},clip,width=\linewidth]{imgs/all_methods_dino.pdf}
\end{minipage}
\caption{The overall results of different sampling methods against main personalized generation baselines.}\label{fig:all_methods_dino}
\end{figure}


\end{document}


% This document was modified from the file originally made available by
% Pat Langley and Andrea Danyluk for ICML-2K. This version was created
% by Iain Murray in 2018, and modified by Alexandre Bouchard in
% 2019 and 2021 and by Csaba Szepesvari, Gang Niu and Sivan Sabato in 2022.
% Modified again in 2023 and 2024 by Sivan Sabato and Jonathan Scarlett.
% Previous contributors include Dan Roy, Lise Getoor and Tobias
% Scheffer, which was slightly modified from the 2010 version by
% Thorsten Joachims & Johannes Fuernkranz, slightly modified from the
% 2009 version by Kiri Wagstaff and Sam Roweis's 2008 version, which is
% slightly modified from Prasad Tadepalli's 2007 version which is a
% lightly changed version of the previous year's version by Andrew
% Moore, which was in turn edited from those of Kristian Kersting and
% Codrina Lauth. Alex Smola contributed to the algorithmic style files.
