\section{Conclusion}
\label{sec:conclusion}
% In this paper, we investigate the role of sampling methods in improving personalized text-to-image generation, focusing on their interaction with fine-tuning strategies and their impact on concept fidelity and adaptability. Using systematic evaluations, we demonstrate that integrating superclass trajectories into the sampling process can lead to significant improvements, offering a flexible approach to balancing concept preservation and the ability to follow various textual prompts. Our analysis provides a comprehensive framework for understanding the trade-offs between different sampling techniques and their application in a variety of generative scenarios. We hope that this study will inspire further research on decoupling fine-tuning from sampling for improvement and exploring the potential of these methods independently.

In this paper, we investigate the role of sampling methods in improving personalized text-to-image generation, focusing on their interaction with fine-tuning strategies and their impact on concept fidelity and adaptability. Using systematic evaluations, we demonstrate that integrating superclass trajectories into the sampling process leads to significant improvements, offering a flexible approach to balancing concept preservation and prompt adherence across diverse architectures. Our analysis provides a comprehensive framework for understanding the trade-offs between different sampling techniques and their application in a variety of generative scenarios. We hope that this study will inspire further research on decoupling fine-tuning from sampling for improvement and exploring the potential of these methods independently.

Regarding the limitations of sampling techniques, we highlight two main issues. First, the sampling methods require careful hyperparameter tuning, and finding the optimal configuration for each technique can be challenging. Second, some of the more advanced sampling techniques, such as ProFusion, require higher computational costs, making them less practical for real-time or large-scale applications compared to simpler alternatives.


\clearpage

\section*{Impact Statement}
Our work advances personalized text-to-image generation by systematically optimizing sampling strategies to balance concept fidelity and adaptability, enabling the efficient creation of tailored visual content for applications in creative design, education, and accessibility tools. 

Potential harms include misuse for generating deceptive or harmful content, such as misinformation, non-consensual deepfakes, or targeted impersonations, raising ethical concerns around privacy, consent, and trust in digital media. We emphasize the need for safeguards, transparency, and responsible deployment to mitigate these risks while fostering beneficial innovation.
