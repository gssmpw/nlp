\section{Introduction}

Time series modeling plays an important role in a wide range of real-world applications, including domains of finance \cite{DBLP:journals/asc/SezerGO20, jing2021network, DBLP:conf/www/ZhouZZLH20}, healthcare \cite{DBLP:journals/corr/abs-2410-21154, DBLP:journals/corr/abs-2410-21520}, climate \cite{DBLP:journals/corr/abs-2408-04254}, and energy systems \cite{DBLP:journals/corr/abs-1708-00420, jing2022retrieval, jing2024causality}. While extensive research has focused on approaches that rely solely on the numerical values of time series \cite{qiu2025tucket, linbacktime, informer, wang2024tssurvey, DBLP:conf/aaai/LiZZH21}, real-world scenarios often involve additional modalities that co-occur with the time series and can provide valuable complementary information \cite{de2023incorporating, rai2023detection,kyei2017internal, jing2024towards, DBLP:journals/corr/abs-2402-02357, DBLP:conf/www/ZhengCHC24}. 

In such real-world scenarios like pandemic policymaking, economic planning, or investment strategies, human experts often incorporate additional sources of information in conjunction with time series data to forecast future trends and make well-informed decisions, as textual information can provide explanations, updates, or external factors that influence the underlying numerical patterns.
However, research on effectively leveraging data from other modalities paired with time series remains in its early stages. In this work, we focus on \textbf{time-series paired with texts at each timestamp}, a common data format in real-world applications where textual descriptions are associated with time series at each timestamp in a parallel manner, as illustrated in Figure \ref{fig: main} (left). For instance, during a pandemic, numerical data such as infection rates and hospital admissions are often accompanied by textual information like government announcements and news reports \cite{cinelli2020covid}. On the one hand, directly applying numerical-only models while ignoring the accompanying textual information can result in suboptimal performance as it overlooks valuable contextual information that may influence or explain the patterns in the time series. On the other hand, the current state-of-the-art approach \cite{DBLP:journals/corr/abs-2406-08627} treats the textual modality as ``bag-of-texts", disregarding the unique positional characteristics that time-series-paired texts may inherently possess. Such limitations raise a pivotal research question:

\textit{What unique attributes characterize time-series-paired texts, and how can they be systematically integrated to improve time series modeling and predictions?}

In this paper, we pioneer the exploration of effectively leveraging paired texts to enrich time series analysis. 
We identify an intriguing phenomenon, which we term \textbf{``Chronological Textual Resonance"}: time-series-paired texts exhibit periodic patterns that closely reflect the temporal dynamics of their corresponding numerical time series. Notably, despite variations in words and expressions, the hidden representations of two texts associated with time series points separated by a period demonstrate high similarities, revealing a deeper alignment between textual and numerical modalities. We attribute this phenomenon to the fact that the paired texts, such as expert notes, news articles, social media posts, or event descriptions, are highly sensitive to and inherently evolve in response to the dynamics of the time series itself.

Building on these insights, we propose \textbf{Texts as Time Series} (TaTS), a simple yet effective framework for integrating paired texts to enhance multimodal time series modeling. 
As previous studies have shown that different variables in a multivariate time series exhibit similar periodicity properties \cite{crossformer, wang2024timexer, yi2024frequency}, our discovery of Chronological Textual Resonance suggests that time-series-paired texts follow a similar pattern. This insight implies that paired texts can be considered as special auxiliary variables to augment the original time series. Motivated by this, TaTS first transforms the paired textual information into a lower-dimensional representation, then combines the original time series with the textual representations as new variables to form an augmented time series. This augmented time series is subsequently fed into existing time series models, allowing them to capture both numerical and textual temporal dynamics. 
By incorporating paired texts as auxiliary variables and utilizing established time series models to process them, TaTS effectively exploits the intrinsic temporal properties of the texts. 
Our TaTS framework offers two key benefits: (i) it effectively captures the evolving positional characteristics of texts paired with a time series by incorporating their projected representations into time series models; and (ii) it functions as a plug-in module, maintaining compatibility with existing time series models. Empirically, through extensive experiments over various benchmark datasets and multiple existing time series models, the proposed TaTS achieves state-of-the-art performance on both forecasting and imputation tasks, as shown in Figure \ref{fig: main radar}. In summary, our contributions in this papers are as follows:

\begin{figure}[t]
    \centering
    \includegraphics[width=0.78\linewidth]{figures/radar_plots/radar_plot_Environment_notitle.png}
    \caption{Performance (Mean Square Error) of different modeling frameworks of time series with paired texts on the Environment dataset. Full results reported in Appendix \ref{ap: full radar}.}
    \label{fig: main radar}
    \vspace{-2mm}
\end{figure}

\begin{itemize}
\vspace{-3mm}
    \item We uncover a previously overlooked phenomenon, termed \textit{Chronological Textual Resonance (CTR)}, which demonstrates that time-series-paired texts exhibit periodic patterns closely aligned with their corresponding numerical time series.
    \item Based on this phenomenon, we propose a plug-and-play multimodal time series forecasting framework, \textit{Texts as Time Series (TaTS)}, which transforms text representations into auxiliary variables, seamlessly integrating them into existing time series models to capture the intrinsic temporal properties of paired texts.
    \item Experiments on diverse benchmark datasets and multiple existing time series models demonstrate that TaTS consistently achieves superior performance without requiring modifications to model architectures.
\end{itemize}