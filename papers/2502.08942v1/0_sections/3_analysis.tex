\section{Chronological Textual Resonance (CTR)}
The association of time series at each timestamp may imbue time-series-paired texts with unique characteristics that can be effectively harnessed through an appropriate design. In this section, we unveil a phenomenon where paired texts exhibit periodicity closely aligned with that of the time series itself. This phenomenon is prevalent across various real-world domains, such as finance, social sciences, and transportation.
To illustrate this phenomenon, we analyze three real-world time series datasets with paired texts \cite{DBLP:journals/corr/abs-2406-08627}, including  
(i) Economy: The time series represents trade data of the United States, while the texts provide descriptions of the general economic conditions of the country. (ii) Social Good: The time series captures the unemployment rate in the United States, and the texts include detailed unemployment reports. (iii) Traffic: The time series reflects monthly travel volume trends from the U.S. Department of Transportation, with corresponding texts derived from traffic volume reports issued by the same department.


For each dataset $\mathcal{D} = \{\bm{X}, \bm{S}\}$, we employ the Fourier Transform \cite{nussbaumer1982fast, sneddon1995fourier} to analyze the frequency components of time series data, allowing us to identify its dominant periodic components, as illustrated by the blue curves in Figure \ref{fig: fft results with text marks}. Furthermore, to examine the periodicity of texts, we embed each $s_t \in \bm{S}$ to obtain the text embedding $e_t$ at timestamp $t$. Then we compute their lag-similarity, defined as $d_l = \sum_t \mathrm{cos}(e_t, e_{t+L})$ where $L$ is the lag and $\mathrm{cos}(\cdot, \cdot)$ represents the cosine similarity. If the text embeddings exhibit a significant periodic pattern, the lag-similarity $d_l$ will also fluctuate periodically as the lag $l$ increases (proof in Proposition \ref{proposition: lag similarity}). Finally, we identify the major frequencies (those with the largest amplitudes) of the texts by applying FFT to the text lag-similarity, and mark them with red dashed lines, as shown in Figure \ref{fig: fft results with text marks}. Detailed process is provided in  Appendix \ref{ap: detailed frequency analysis}. We find that the major frequencies of the paired texts closely match those of the time series, indicating that the paired texts exhibit periodicity that is strongly aligned with the temporal dynamics of the time series.




\begin{figure*}[t]
    \centering
    \includegraphics[width=\linewidth]{figures/pipeline_MMTS_zihao.png}
    \caption{Illustration of the Texts as Time Series (TaTS) framework for capturing the temporal dynamics of paired texts. Motivated by the concept of chronological textual resonance, which suggests that paired texts exhibit behaviors similar to accompanying variables in a time series, TaTS transforms the paired texts into auxiliary variables. These variables augment the numerical sequence, forming a unified multimodal sequence that can be seamlessly integrated into any existing time series model. Compared to existing text-as-bag-of-words methods, TaTS can effectively capture the natural temporal dynamics within time-series-paired texts.}
    \label{fig: main}
\end{figure*}


\begin{table}[t]
\vspace{-3mm}
\caption{Performance on time series forecasting task. Leveraging periodicity with a single-dimension feature can significantly reduce the prediction error.}
\vspace{-2mm}
\label{tab: text only for ts forecasting}
\resizebox{\columnwidth}{!}{%
\begin{tabular}{@{}c|cc|cc|cc@{}}
\toprule
\multirow{2}{*}{\textbf{Method}} & \multicolumn{2}{c|}{\textbf{Economy}} & \multicolumn{2}{c|}{\textbf{Social Good}} & \multicolumn{2}{c}{\textbf{Traffic}} \\
 \cmidrule(l){2-7} 
 & \textbf{MSE}($\downarrow$) & \textbf{MAE}($\downarrow$) & \textbf{MSE}($\downarrow$) & \textbf{MAE}($\downarrow$) & \textbf{MSE}($\downarrow$) & \textbf{MAE}($\downarrow$) \\ \midrule
\textbf{Uniformly Random $(+)$} & 5.673 & 2.356 & 2.059 & 1.230 & 1.207 & 0.995 \\
\textbf{Uniformly Random $(\pm)$} & 11.535 & 2.879 & 8.860 & 2.404 & 3.794 & 1.618 \\
\textbf{Normally Random} & 9.284 & 2.878 & 2.926 & 1.374 & 3.163 & 1.511 \\
\textbf{Exponentially Random} & 4.521 & 1.960 & 3.724 & 1.564 & 1.141 & 0.911 \\
\textbf{Using 1D Text only} & \textbf{1.995} & \textbf{1.404} & \textbf{1.315} & \textbf{0.853} & \textbf{0.714} & \textbf{0.797} \\
\bottomrule
\end{tabular}%
}
\vspace{-3mm}
\end{table}




\textbf{Why should we leverage CTR?} Parallel text provides complementary information and expert knowledge that can significantly enhance the understanding of time series data. To demonstrate the benefits of utilizing periodicity in the text modality, we present an illustrative example in the univariate forecasting task, where the goal is to use $\bm{X}_{1:T} = \{\vecx_1\} \in \mathbb{R}^{T \times 1}$ to predict the next $H$ values, $\widehat{\mathbf{X}}_{T+1: T+H}$. We first concatenate the text embeddings to form $\bm{E} = [e_1; e_2; \ldots; e_T]_{\text{dim=1}} \in \mathbb{R}^{d_\text{text} \times T}$, then replace $\bm{x}_1$ with only the first dimension of the text embeddings to leverage very partial text periodicity, $\bm{x}'_1 = (\bm{E}[1,:])^{\intercal} \in \mathbb{R}^{T \times 1}$. In other words, we rely solely on a single evolving dimension of the paired text features to forecast future time series values. 
As shown in Table \ref{tab: text only for ts forecasting}, leveraging just the periodicity of a single text feature significantly outperforms random time series forecasting. These results highlight that even partial periodicity from the text modality contributes valuable insights for improving forecasting accuracy.

\textbf{Why does CTR happen?} We present three key reasons for the observed alignment in periodicity between time series and their paired texts: (i) \textbf{Shared External Drivers}: Both the time series and their paired texts are often influenced by common external factors, such as seasonal changes, recurring events, or societal and economic cycles. 
These shared drivers naturally induce periodicity in both the numerical time series and the accompanying texts. (ii) \textbf{Influence of Time Series on Texts}: Paired texts often serve as contextual reflections of the underlying time series, adapting and evolving in response to numerical trends. For instance, news articles or government reports accompanying economic indicators are frequently updated in response to the numerical trends, leading to a periodic alignment between the two modalities.
(iii) \textbf{Texts Contain Additional Variables with Aligned Periodicity}: Paired texts often include additional variables that are closely related to the original time series. For example, if the time series represents economic data such as GDP, the accompanying texts may reference related variables like stock market indices or inflation rates. These related variables often exhibit periodicity patterns aligned with the time series and affect the periodicity of the paired texts, but are not included in the time series data.
