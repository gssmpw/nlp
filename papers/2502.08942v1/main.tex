%%%%%%%% ICML 2025 EXAMPLE LATEX SUBMISSION FILE %%%%%%%%%%%%%%%%%

\documentclass{article}

% Recommended, but optional, packages for figures and better typesetting:
\usepackage{microtype}
\usepackage{graphicx}
\usepackage{subfigure}
\usepackage{booktabs} % for professional tables

% hyperref makes hyperlinks in the resulting PDF.
% If your build breaks (sometimes temporarily if a hyperlink spans a page)
% please comment out the following usepackage line and replace
% \usepackage{icml2025} with \usepackage[nohyperref]{icml2025} above.
\usepackage{hyperref}
\usepackage[table]{xcolor}

% Attempt to make hyperref and algorithmic work together better:
\newcommand{\theHalgorithm}{\arabic{algorithm}}

% Use the following line for the initial blind version submitted for review:
% \usepackage{icml2025}

% \PassOptionsToPackage{numbers, compress}{natbib}
% \definecolor{cvprblue}{rgb}{0.21, 0.49, 0.74}
% \usepackage[colorlinks,citecolor=cvprblue]{hyperref}

% If accepted, instead use the following line for the camera-ready submission:
\usepackage[accepted]{icml2025}

% For theorems and such
\usepackage{amsmath}
\usepackage{amssymb}
\usepackage{mathtools}
\usepackage{amsthm}

% if you use cleveref..
\usepackage[capitalize,noabbrev]{cleveref}

%%%%%%%%%%%%%%%%%%%%%%%%%%%%%%%%
% THEOREMS
%%%%%%%%%%%%%%%%%%%%%%%%%%%%%%%%
\theoremstyle{plain}
\newtheorem{theorem}{Theorem}[section]
\newtheorem{proposition}[theorem]{Proposition}
\newtheorem{lemma}[theorem]{Lemma}
\newtheorem{corollary}[theorem]{Corollary}
\theoremstyle{definition}
\newtheorem{definition}[theorem]{Definition}
\newtheorem{assumption}[theorem]{Assumption}
\theoremstyle{remark}
\newtheorem{remark}[theorem]{Remark}

% Todonotes is useful during development; simply uncomment the next line
%    and comment out the line below the next line to turn off comments
%\usepackage[disable,textsize=tiny]{todonotes}
\usepackage[textsize=tiny]{todonotes}

% for better table
\usepackage{threeparttable}
\usepackage{multirow}

% for better figure (minipage)
\usepackage{caption}

% for unifying names
\newcommand{\uni}[0]{Uni-modal}
\newcommand{\multi}{MM-TSFLib}
\newcommand{\ours}[0]{TaTS (ours)}

% for leaving comments

% helpful shortcut of commands
\newcommand{\cm}{\mathcal}
\newcommand{\bm}{\mathbf}

\newcommand{\vecx}{\mathbf{x}}
\DeclareMathOperator*{\argmin}{arg\,min}

% The \icmltitle you define below is probably too long as a header.
% Therefore, a short form for the running title is supplied here:
\icmltitlerunning{Language in the Flow of Time: Time-Series-Paired Texts Weaved into a Unified Temporal Narrative}

\begin{document}

% Hide main paper sections in table of content
\addtocontents{toc}{\protect\setcounter{tocdepth}{0}}

\twocolumn[
% \icmltitle{The Texts Beneath the Ticks: Your Time-Series-Paired Language Secretly Mirrors Temporal Dynamics}
% \icmltitle{When Words Follow Time’s Pulse: Language is Secretly a Time Series}
% \icmltitle{Language in the Flow of Time: Decoding the Co-Evolution of Texts and Time Series}
\icmltitle{Language in the Flow of Time: Time-Series-Paired Texts Weaved into a \\Unified Temporal Narrative}


% It is OKAY to include author information, even for blind
% submissions: the style file will automatically remove it for you
% unless you've provided the [accepted] option to the icml2025
% package.

% List of affiliations: The first argument should be a (short)
% identifier you will use later to specify author affiliations
% Academic affiliations should list Department, University, City, Region, Country
% Industry affiliations should list Company, City, Region, Country

% You can specify symbols, otherwise they are numbered in order.
% Ideally, you should not use this facility. Affiliations will be numbered
% in order of appearance and this is the preferred way.
% \icmlsetsymbol{equal}{*}

\begin{icmlauthorlist}
\icmlauthor{Zihao Li}{uiuc}
\icmlauthor{Xiao Lin}{uiuc}
\icmlauthor{Zhining Liu}{uiuc}
\icmlauthor{Jiaru Zou}{uiuc}
\icmlauthor{Ziwei Wu}{uiuc}
\icmlauthor{Lecheng Zheng}{uiuc}
\icmlauthor{Dongqi Fu}{meta}
\icmlauthor{Yada Zhu}{ibm}
\icmlauthor{Hendrik Hamann}{ibm}
\icmlauthor{Hanghang Tong}{uiuc}
\icmlauthor{Jingrui He}{uiuc}
\end{icmlauthorlist}

\icmlaffiliation{uiuc}{University of Illinois Urbana-Champaign, IL, USA}
\icmlaffiliation{ibm}{IBM Research}
\icmlaffiliation{meta}{Meta AI}

\icmlcorrespondingauthor{Zihao Li}{zihaoli5@illinois.edu}
\icmlcorrespondingauthor{Jingrui He}{jingrui@illinois.edu}

% You may provide any keywords that you
% find helpful for describing your paper; these are used to populate
% the "keywords" metadata in the PDF but will not be shown in the document
\icmlkeywords{Machine Learning, ICML}

\vskip 0.3in
]

% this must go after the closing bracket ] following \twocolumn[ ...

% This command actually creates the footnote in the first column
% listing the affiliations and the copyright notice.
% The command takes one argument, which is text to display at the start of the footnote.
% The \icmlEqualContribution command is standard text for equal contribution.
% Remove it (just {}) if you do not need this facility.

\printAffiliationsAndNotice{}  % leave blank if no need to mention equal contribution
% \printAffiliationsAndNotice{\icmlEqualContribution} % otherwise use the standard text.

\begin{abstract}
While many advances in time series models focus exclusively on numerical data, research on multimodal time series, particularly those involving contextual textual information commonly encountered in real-world scenarios, remains in its infancy. Consequently, effectively integrating the text modality remains challenging. In this work, we highlight an intuitive yet significant observation that has been overlooked by existing works: time-series-paired texts exhibit periodic properties that closely mirror those of the original time series. Building on this insight, we propose a novel framework, Texts as Time Series (TaTS), which considers the time-series-paired texts to be auxiliary variables of the time series. TaTS can be plugged into {\em any} existing numerical-only time series models and enable them to handle time series data with paired texts effectively. Through extensive experiments on both multimodal time series forecasting and imputation tasks across benchmark datasets with various existing time series models, we demonstrate that TaTS can enhance predictive performance and achieve outperformance without modifying model architectures. 
\end{abstract}


\section{Introduction}

Video generation has garnered significant attention owing to its transformative potential across a wide range of applications, such media content creation~\citep{polyak2024movie}, advertising~\citep{zhang2024virbo,bacher2021advert}, video games~\citep{yang2024playable,valevski2024diffusion, oasis2024}, and world model simulators~\citep{ha2018world, videoworldsimulators2024, agarwal2025cosmos}. Benefiting from advanced generative algorithms~\citep{goodfellow2014generative, ho2020denoising, liu2023flow, lipman2023flow}, scalable model architectures~\citep{vaswani2017attention, peebles2023scalable}, vast amounts of internet-sourced data~\citep{chen2024panda, nan2024openvid, ju2024miradata}, and ongoing expansion of computing capabilities~\citep{nvidia2022h100, nvidia2023dgxgh200, nvidia2024h200nvl}, remarkable advancements have been achieved in the field of video generation~\citep{ho2022video, ho2022imagen, singer2023makeavideo, blattmann2023align, videoworldsimulators2024, kuaishou2024klingai, yang2024cogvideox, jin2024pyramidal, polyak2024movie, kong2024hunyuanvideo, ji2024prompt}.


In this work, we present \textbf{\ours}, a family of rectified flow~\citep{lipman2023flow, liu2023flow} transformer models designed for joint image and video generation, establishing a pathway toward industry-grade performance. This report centers on four key components: data curation, model architecture design, flow formulation, and training infrastructure optimization—each rigorously refined to meet the demands of high-quality, large-scale video generation.


\begin{figure}[ht]
    \centering
    \begin{subfigure}[b]{0.82\linewidth}
        \centering
        \includegraphics[width=\linewidth]{figures/t2i_1024.pdf}
        \caption{Text-to-Image Samples}\label{fig:main-demo-t2i}
    \end{subfigure}
    \vfill
    \begin{subfigure}[b]{0.82\linewidth}
        \centering
        \includegraphics[width=\linewidth]{figures/t2v_samples.pdf}
        \caption{Text-to-Video Samples}\label{fig:main-demo-t2v}
    \end{subfigure}
\caption{\textbf{Generated samples from \ours.} Key components are highlighted in \textcolor{red}{\textbf{RED}}.}\label{fig:main-demo}
\end{figure}


First, we present a comprehensive data processing pipeline designed to construct large-scale, high-quality image and video-text datasets. The pipeline integrates multiple advanced techniques, including video and image filtering based on aesthetic scores, OCR-driven content analysis, and subjective evaluations, to ensure exceptional visual and contextual quality. Furthermore, we employ multimodal large language models~(MLLMs)~\citep{yuan2025tarsier2} to generate dense and contextually aligned captions, which are subsequently refined using an additional large language model~(LLM)~\citep{yang2024qwen2} to enhance their accuracy, fluency, and descriptive richness. As a result, we have curated a robust training dataset comprising approximately 36M video-text pairs and 160M image-text pairs, which are proven sufficient for training industry-level generative models.

Secondly, we take a pioneering step by applying rectified flow formulation~\citep{lipman2023flow} for joint image and video generation, implemented through the \ours model family, which comprises Transformer architectures with 2B and 8B parameters. At its core, the \ours framework employs a 3D joint image-video variational autoencoder (VAE) to compress image and video inputs into a shared latent space, facilitating unified representation. This shared latent space is coupled with a full-attention~\citep{vaswani2017attention} mechanism, enabling seamless joint training of image and video. This architecture delivers high-quality, coherent outputs across both images and videos, establishing a unified framework for visual generation tasks.


Furthermore, to support the training of \ours at scale, we have developed a robust infrastructure tailored for large-scale model training. Our approach incorporates advanced parallelism strategies~\citep{jacobs2023deepspeed, pytorch_fsdp} to manage memory efficiently during long-context training. Additionally, we employ ByteCheckpoint~\citep{wan2024bytecheckpoint} for high-performance checkpointing and integrate fault-tolerant mechanisms from MegaScale~\citep{jiang2024megascale} to ensure stability and scalability across large GPU clusters. These optimizations enable \ours to handle the computational and data challenges of generative modeling with exceptional efficiency and reliability.


We evaluate \ours on both text-to-image and text-to-video benchmarks to highlight its competitive advantages. For text-to-image generation, \ours-T2I demonstrates strong performance across multiple benchmarks, including T2I-CompBench~\citep{huang2023t2i-compbench}, GenEval~\citep{ghosh2024geneval}, and DPG-Bench~\citep{hu2024ella_dbgbench}, excelling in both visual quality and text-image alignment. In text-to-video benchmarks, \ours-T2V achieves state-of-the-art performance on the UCF-101~\citep{ucf101} zero-shot generation task. Additionally, \ours-T2V attains an impressive score of \textbf{84.85} on VBench~\citep{huang2024vbench}, securing the top position on the leaderboard (as of 2025-01-25) and surpassing several leading commercial text-to-video models. Qualitative results, illustrated in \Cref{fig:main-demo}, further demonstrate the superior quality of the generated media samples. These findings underscore \ours's effectiveness in multi-modal generation and its potential as a high-performing solution for both research and commercial applications.

\section{Preliminary} \label{sec:preliminary}
\paragraph{Random variable and distribution.} Let $\mathcal{X} = \mathcal{X}_v \times \mathcal{X}_t$ denote the input space, where $\mathcal{X}_v$ and $\mathcal{X}_t$ correspond to the visual and textual feature spaces, respectively. Similarly, let $\mathcal{Y}$ denote the response space. We define the random variables $\mathbf{X} = (X_v, X_t) \in \mathcal{X}$ and $Y \in \mathcal{Y}$, where $\mathbf{X}$ is the sequence of tokens that combine visual and text input queries, and $Y$ represents the associated response tokens. The joint population is denoted by $P_{\mathbf{X}Y}$, with marginals $P_{\mathbf{X}}$, $P_{Y}$, and the conditional distribution $P_{Y|\mathbf{X}}$. For subsequent sections, $P_{\mathbf{X}Y}$ refers to the instruction tuning data distribution which we consider as in-distribution (ID). 

\paragraph{MLLM and visual instruction tuning.} MLLM usually consists of three components: (1) a visual encoder, (2) a vision-to-language projector, and (3) an LLM that processes a multimodal input sequence to generate a valid textual output $y$ in response to an input query $\mathbf{x}$. An MLLM can be regarded as modeling a conditional distribution $P_{\theta}(y|\mathbf{x})$, where $\theta$ is the model parameters. To attain the multimodal conversation capability, MLLMs commonly undergo a phase so-called \textit{visual instruction tuning} \cite{liu2023visual, dai2023instructblip} with an autoregressive objective as follows:
{
\begin{align} \label{eq::1}
    % & \min_{\theta\in\Theta} \mathbb{E}_{\mathbf{x},y\sim P_{\mathbf{X}Y}} [-\log P_{\theta}(y|\mathbf{x})] \nonumber \\
     \min_{\theta\in\Theta} \mathbb{E}_{\mathbf{x},y\sim P_{\mathbf{X}Y}} [\sum_{l=0}^{L}-\log P_{\theta}(y_{l}|\mathbf{x},y_{<l})],
\end{align}}
where $L$ is a sequence length and $y=(y_{0},...,y_{L})$. After being trained by Eq. \eqref{eq::1}, MLLM produces a response given a query of any possible tasks represented by text.

\paragraph{Evaluation of open-ended generations.} 

(M)LLM-as-a-judge method \cite{zheng2023judging, kim2023prometheus} is commonly adopted to evaluate open-ended generation. In this paradigm, a judge model produces preference scores or rankings for the responses given a query, model responses, and a scoring rubric. Among the evaluation metrics, the \emph{win rate} (Eq. \eqref{eq:win_rate}) is one of the most widely used and representative.


\begin{definition}[\textbf{Win Rate}] Given a parametric reward function $r:\mathcal{X}\times \mathcal{Y}\rightarrow \mathbb{R}$, the 
win rate (WR) of model $P_{\theta}$ w.r.t. $P_{\mathbf{X}Y}$ are defined as follows: 
\begin{equation} \label{eq:win_rate}
\begin{split}
    &\text{WR}(P_{\mathbf{X}Y};\theta):=\mathbb{E}_{\begin{subarray}{l} \mathbf{x},y \sim P_{\mathbf{X}Y} \\ \hat{y} \sim P_{\theta}(\cdot|\mathbf{x}) \end{subarray}}[\mathbb{I}(r(\mathbf{x},\hat{y}) > r(\mathbf{x},y))],
\end{split}
\end{equation}
where $\mathbb{I}(\cdot)$ is  the indicator function.
\end{definition}
Here, the reward function $r(\cdot,\cdot)$, can be any possible (multimodal) LLMs such as GPT-4o \cite{hurst2024gpt}.



\section{Analysis}

\begin{figure*}[t]
    \centering
    \includegraphics[width=\textwidth]{figures/ref_vs_pe_vs_mt_MetricX-24-QE.pdf}
    \caption{A comparison of the reference, post-edit, and best MT output (per language) qualities as measured by MetricX-24-QE.
    For all languages but ar\_EG and ar\_SA, the post-edit is roughly equal or higher quality (see \S\ref{sec:human_vs_mt_quality} for a discussion about Arabic).
    Although it appears that MT systems are generating super-human translations for nearly all 55 languages, we caution against reaching that conclusion because automatic metrics are known to be biased against human translations, systematically scoring them lower than they should (see Appendix~\ref{appendix:metric_bias}).
    }
    \label{fig:ref_vs_pe}
\end{figure*}

\subsection{Human \& Machine Translation Quality}
\label{sec:human_vs_mt_quality}

\autoref{fig:ref_vs_pe} presents a comparison of the quality of the reference, subsequent post-edit, and highest quality MT system (per language), as measured by MetricX-24-QE.


\paragraph{References vs. Post-Edits}
We observe that for all languages except ar\_EG and ar\_SA, the post-edit gets an equal or better score compared to the original reference, indicating that post-editing improved the human translation quality.
After examining the Arabic data, we hypothesize that the perceived drop in quality is due to the fact that the  references were written in Modern Standard Arabic (MSA) rather than the regional variants.\footnote{
    This was unintentional and due to a mistake by the vendor who supplied the translations. 
}
Given that the dialects are significantly different than MSA and that Metric\-X scores vary widely across languages, we feel that drop in score can be attributed to the dialect change rather than a drop in quality.

\paragraph{Human vs. MT}
Then, for every language, the best MT system is roughly equal or better than the human translations, often by a large margin.
Although this result would suggest that MT systems are producing super-human translations in all 55 languages/dialects, \textbf{we caution against coming to that conclusion}.
QE metrics are known to be biased against human translations, systematically rating them lower than human evaluators do (see Appendix~\ref{appendix:metric_bias}).
Therefore, without a large scale human evaluation of the translations, we cannot reach definitive conclusions about whether MT quality is indeed better than human quality.


\paragraph{Comparing Across Languages}
When comparing translation quality across languages, we observe that the scores vary significantly, from -2 to -6, which is a large gap in MetricX.
However, we do not believe this result is explained by a true drop in quality.
Since both the human and machine translation scores have nearly identical trends across languages, we hypothesize that this result can be explained by the metric behaving differently across languages.
MT metrics are largely untested in the majority of these languages, so it is unclear how trustworthy they are.
Otherwise, it would have to be the case that both humans and MT systems have a hard time producing translations for the same languages, which seems unlikely since human translators should be able to produce roughly equal quality translations across languages.


\subsection{System Evaluation}
The system rankings based on MetricX-24 are displayed in Figure~\ref{fig:system_ranking_metricx} (see Appendix~\ref{appendix:additional_results} for the rankings based on other metrics and the absolute scores for each system).

\begin{figure*}
    \centering
    \includegraphics[width=\textwidth]{figures/system_ranking_MetricX-24.pdf}
    \caption{System rankings according to MetricX-24.
    The rankings are significance clusters, so no system with the same rank is statistically better than all other systems within the same cluster, and all systems in one cluster are statistically better than all systems with a worse rank.
    Statistical testing was done with a one-sided permutation test with $\alpha=0.05$ \citep{noreen1989computer}.}
    \label{fig:system_ranking_metricx}
\end{figure*}


The most apparent result is that the frontier LLMs---OpenAI o1, Claude, and Gemini---are ranked the highest for every language, demonstrating that these LLMs are highly capable translation systems in a large number of languages.
Between these three systems, there is little difference between them: the average ranks across the 55 languages are 1.5, 1.9, and 2.1.
Further, their absolute metric scores are very similar (see Appendix~\ref{appendix:additional_results}).

While the frontier LLMs are indeed ranked higher than traditional MT service providers according to automatic metrics, this may not be an entirely fair comparison.
Although we do not know the exact details of each system, we speculate that the MT providers' models are significantly smaller and faster than the LLMs, highlighting a tradeoff between quality and speed/cost.

While we suspect these results are likely true, it should be confirmed by a human-based evaluation of the translations, which we intend to do in future work.



Effective human-robot cooperation in CoNav-Maze hinges on efficient communication. Maximizing the human’s information gain enables more precise guidance, which in turn accelerates task completion. Yet for the robot, the challenge is not only \emph{what} to communicate but also \emph{when}, as it must balance gathering information for the human with pursuing immediate goals when confident in its navigation.

To achieve this, we introduce \emph{Information Gain Monte Carlo Tree Search} (IG-MCTS), which optimizes both task-relevant objectives and the transmission of the most informative communication. IG-MCTS comprises three key components:
\textbf{(1)} A data-driven human perception model that tracks how implicit (movement) and explicit (image) information updates the human’s understanding of the maze layout.
\textbf{(2)} Reward augmentation to integrate multiple objectives effectively leveraging on the learned perception model.
\textbf{(3)} An uncertainty-aware MCTS that accounts for unobserved maze regions and human perception stochasticity.
% \begin{enumerate}[leftmargin=*]
%     \item A data-driven human perception model that tracks how implicit (movement) and explicit (image transmission) information updates the human’s understanding of the maze layout.
%     \item Reward augmentation to integrate multiple objectives effectively leveraging on the learned perception model.
%     \item An uncertainty-aware MCTS that accounts for unobserved maze regions and human perception stochasticity.
% \end{enumerate}

\subsection{Human Perception Dynamics}
% IG-MCTS seeks to optimize the expected novel information gained by the human through the robot’s actions, including both movement and communication. Achieving this requires a model of how the human acquires task-relevant information from the robot.

% \subsubsection{Perception MDP}
\label{sec:perception_mdp}
As the robot navigates the maze and transmits images, humans update their understanding of the environment. Based on the robot's path, they may infer that previously assumed blocked locations are traversable or detect discrepancies between the transmitted image and their map.  

To formally capture this process, we model the evolution of human perception as another Markov Decision Process, referred to as the \emph{Perception MDP}. The state space $\mathcal{X}$ represents all possible maze maps. The action space $\mathcal{S}^+ \times \mathcal{O}$ consists of the robot's trajectory between two image transmissions $\tau \in \mathcal{S}^+$ and an image $o \in \mathcal{O}$. The unknown transition function $F: (x, (\tau, o)) \rightarrow x'$ defines the human perception dynamics, which we aim to learn.

\subsubsection{Crowd-Sourced Transition Dataset}
To collect data, we designed a mapping task in the CoNav-Maze environment. Participants were tasked to edit their maps to match the true environment. A button triggers the robot's autonomous movements, after which it captures an image from a random angle.
In this mapping task, the robot, aware of both the true environment and the human’s map, visits predefined target locations and prioritizes areas with mislabeled grid cells on the human’s map.
% We assume that the robot has full knowledge of both the actual environment and the human’s current map. Leveraging this knowledge, the robot autonomously navigates to all predefined target locations. It then randomly selects subsequent goals to reach, prioritizing grid locations that remain mislabeled on the human’s map. This ensures that the robot’s actions are strategically focused on providing useful information to improve map accuracy.

We then recruited over $50$ annotators through Prolific~\cite{palan2018prolific} for the mapping task. Each annotator labeled three randomly generated mazes. They were allowed to proceed to the next maze once the robot had reached all four goal locations. However, they could spend additional time refining their map before moving on. To incentivize accuracy, annotators receive a performance-based bonus based on the final accuracy of their annotated map.


\subsubsection{Fully-Convolutional Dynamics Model}
\label{sec:nhpm}

We propose a Neural Human Perception Model (NHPM), a fully convolutional neural network (FCNN), to predict the human perception transition probabilities modeled in \Cref{sec:perception_mdp}. We denote the model as $F_\theta$ where $\theta$ represents the trainable weights. Such design echoes recent studies of model-based reinforcement learning~\cite{hansen2022temporal}, where the agent first learns the environment dynamics, potentially from image observations~\cite{hafner2019learning,watter2015embed}.

\begin{figure}[t]
    \centering
    \includegraphics[width=0.9\linewidth]{figures/ICML_25_CNN.pdf}
    \caption{Neural Human Perception Model (NHPM). \textbf{Left:} The human's current perception, the robot's trajectory since the last transmission, and the captured environment grids are individually processed into 2D masks. \textbf{Right:} A fully convolutional neural network predicts two masks: one for the probability of the human adding a wall to their map and another for removing a wall.}
    \label{fig:nhpm}
    \vskip -0.1in
\end{figure}

As illustrated in \Cref{fig:nhpm}, our model takes as input the human’s current perception, the robot’s path, and the image captured by the robot, all of which are transformed into a unified 2D representation. These inputs are concatenated along the channel dimension and fed into the CNN, which outputs a two-channel image: one predicting the probability of human adding a new wall and the other predicting the probability of removing a wall.

% Our approach builds on world model learning, where neural networks predict state transitions or environmental updates based on agent actions and observations. By leveraging the local feature extraction capabilities of CNNs, our model effectively captures spatial relationships and interprets local changes within the grid maze environment. Similar to prior work in localization and mapping, the CNN architecture is well-suited for processing spatially structured data and aligning the robot’s observations with human map updates.

To enhance robustness and generalization, we apply data augmentation techniques, including random rotation and flipping of the 2D inputs during training. These transformations are particularly beneficial in the grid maze environment, which is invariant to orientation changes.

\subsection{Perception-Aware Reward Augmentation}
The robot optimizes its actions over a planning horizon \( H \) by solving the following optimization problem:
\begin{subequations}
    \begin{align}
        \max_{a_{0:H-1}} \;
        & \mathop{\mathbb{E}}_{T, F} \left[ \sum_{t=0}^{H-1} \gamma^t \left(\underbrace{R_{\mathrm{task}}(\tau_{t+1}, \zeta)}_{\text{(1) Task reward}} + \underbrace{\|x_{t+1}-x_t\|_1}_{\text{(2) Info reward}}\right)\right] \label{obj}\\ 
        \subjectto \quad
        &x_{t+1} = F(x_t, (\tau_t, a_t)), \quad a_t\in\Ocal \label{const:perception_update}\\ 
        &\tau_{t+1} = \tau_t \oplus T(s_t, a_t), \quad a_t\in \Ucal\label{const:history_update}
    \end{align}
\end{subequations} 

The objective in~\eqref{obj} maximizes the expected cumulative reward over \( T \) and \( F \), reflecting the uncertainty in both physical transitions and human perception dynamics. The reward function consists of two components: 
(1) The \emph{task reward} incentivizes efficient navigation. The specific formulation for the task in this work is outlined in \Cref{appendix:task_reward}.
(2) The \emph{information reward} quantifies the change in the human’s perception due to robot actions, computed as the \( L_1 \)-norm distance between consecutive perception states.  

The constraint in~\eqref{const:history_update} ensures that for movement actions, the trajectory history \( \tau_t \) expands with new states based on the robot’s chosen actions, where \( s_t \) is the most recent state in \( \tau_t \), and \( \oplus \) represents sequence concatenation. 
In constraint~\eqref{const:perception_update}, the robot leverages the learned human perception dynamics \( F \) to estimate the evolution of the human’s understanding of the environment from perception state $x_t$ to $x_{t+1}$ based on the observed trajectory \( \tau_t \) and transmitted image \( a_t\in\Ocal \). 
% justify from a cognitive science perspective
% Cognitive science research has shown that humans read in a way to maximize the information gained from each word, aligning with the efficient coding principle, which prioritizes minimizing perceptual errors and extracting relevant features under limited processing capacity~\cite{kangassalo2020information}. Drawing on this principle, we hypothesize that humans similarly prioritize task-relevant information in multimodal settings. To accommodate this cognitive pattern, our robot policy selects and communicates high information-gain observations to human operators, akin to summarizing key insights from a lengthy article.
% % While the brain naturally seeks to gain information, the brain employs various strategies to manage information overload, including filtering~\cite{quiroga2004reducing}, limiting/working memory, and prioritizing information~\cite{arnold2023dealing}.
% In this context of our setup, we optimize the selection of camera angles to maximize the human operator's information gain about the environment. 

\subsection{Information Gain Monte Carlo Tree Search (IG-MCTS)}
IG-MCTS follows the four stages of Monte Carlo tree search: \emph{selection}, \emph{expansion}, \emph{rollout}, and \emph{backpropagation}, but extends it by incorporating uncertainty in both environment dynamics and human perception. We introduce uncertainty-aware simulations in the \emph{expansion} and \emph{rollout} phases and adjust \emph{backpropagation} with a value update rule that accounts for transition feasibility.

\subsubsection{Uncertainty-Aware Simulation}
As detailed in \Cref{algo:IG_MCTS}, both the \emph{expansion} and \emph{rollout} phases involve forward simulation of robot actions. Each tree node $v$ contains the state $(\tau, x)$, representing the robot's state history and current human perception. We handle the two action types differently as follows:
\begin{itemize}
    \item A movement action $u$ follows the environment dynamics $T$ as defined in \Cref{sec:problem}. Notably, the maze layout is observable up to distance $r$ from the robot's visited grids, while unexplored areas assume a $50\%$ chance of walls. In \emph{expansion}, the resulting search node $v'$ of this uncertain transition is assigned a feasibility value $\delta = 0.5$. In \emph{rollout}, the transition could fail and the robot remains in the same grid.
    
    \item The state transition for a communication step $o$ is governed by the learned stochastic human perception model $F_\theta$ as defined in \Cref{sec:nhpm}. Since transition probabilities are known, we compute the expected information reward $\bar{R_\mathrm{info}}$ directly:
    \begin{align*}
        \bar{R_\mathrm{info}}(\tau_t, x_t, o_t) &= \mathbb{E}_{x_{t+1}}\|x_{t+1}-x_t\|_1 \\
        &= \|p_\mathrm{add}\|_1 + \|p_\mathrm{remove}\|_1,
    \end{align*}
    where $(p_\mathrm{add}, p_\mathrm{remove}) \gets F_\theta(\tau_t, x_t, o_t)$ are the estimated probabilities of adding or removing walls from the map. 
    Directly computing the expected return at a node avoids the high number of visitations required to obtain an accurate value estimate.
\end{itemize}

% We denote a node in the search tree as $v$, where $s(v)$, $r(v)$, and $\delta(v)$ represent the state, reward, and transition feasibility at $v$, respectively. The visit count of $v$ is denoted as $N(v)$, while $Q(v)$ represents its total accumulated return. The set of child nodes of $v$ is denoted by $\mathbb{C}(v)$.

% The goal of each search is to plan a sequence for the robot until it reaches a goal or transmits a new image to the human. We initialize the search tree with the current human guidance $\zeta$, and the robot's approximation of human perception $x_0$. Each search node consists consists of the state information required by our reward augmentation: $(\tau, x)$. A node is terminal if it is the resulting state of a communication step, or if the robot reaches a goal location. 

% A rollout from the expanded node simulates future transitions until reaching a terminal state or a predefined depth $H$. Actions are selected randomly from the available action set $\mathcal{A}(s)$. If an action's feasibility is uncertain due to the environment's unknown structure, the transition occurs with probability $\delta(s, a)$. When a random number draw deems the transition infeasible, the state remains unchanged. On the other hand, for communication steps, we don't resolve the uncertainty but instead compute the expected information gain reward: \philip{TODO: adjust notation}
% \begin{equation}
%     \mathbb{E}\left[R_\mathrm{info}(\tau, x')\right] = \sum \mathrm{NPM(\tau, o)}.
% \end{equation}

\subsubsection{Feasibility-Adjusted Backpropagation}
During backpropagation, the rewards obtained from the simulation phase are propagated back through the tree, updating the total value $Q(v)$ and the visitation count $N(v)$ for all nodes along the path to the root. Due to uncertainty in unexplored environment dynamics, the rollout return depends on the feasibility of the transition from the child node. Given a sample return \(q'_{\mathrm{sample}}\) at child node \(v'\), the parent node's return is:
\begin{equation}
    q_{\mathrm{sample}} = r + \gamma \left[ \delta' q'_{\mathrm{sample}} + (1 - \delta') \frac{Q(v)}{N(v)} \right],
\end{equation}
where $\delta'$ represents the probability of a successful transition. The term \((1 - \delta')\) accounts for failed transitions, relying instead on the current value estimate.

% By incorporating uncertainty-aware rollouts and backpropagation, our approach enables more robust decision-making in scenarios where the environment dynamics is unknown and avoids simulation of the stochastic human perception dynamics.


\section{Experiments}
\label{section5}

In this section, we conduct extensive experiments to show that \ourmethod~can significantly speed up the sampling of existing MR Diffusion. To rigorously validate the effectiveness of our method, we follow the settings and checkpoints from \cite{luo2024daclip} and only modify the sampling part. Our experiment is divided into three parts. Section \ref{mainresult} compares the sampling results for different NFE cases. Section \ref{effects} studies the effects of different parameter settings on our algorithm, including network parameterizations and solver types. In Section \ref{analysis}, we visualize the sampling trajectories to show the speedup achieved by \ourmethod~and analyze why noise prediction gets obviously worse when NFE is less than 20.


\subsection{Main results}\label{mainresult}

Following \cite{luo2024daclip}, we conduct experiments with ten different types of image degradation: blurry, hazy, JPEG-compression, low-light, noisy, raindrop, rainy, shadowed, snowy, and inpainting (see Appendix \ref{appd1} for details). We adopt LPIPS \citep{zhang2018lpips} and FID \citep{heusel2017fid} as main metrics for perceptual evaluation, and also report PSNR and SSIM \citep{wang2004ssim} for reference. We compare \ourmethod~with other sampling methods, including posterior sampling \citep{luo2024posterior} and Euler-Maruyama discretization \citep{kloeden1992sde}. We take two tasks as examples and the metrics are shown in Figure \ref{fig:main}. Unless explicitly mentioned, we always use \ourmethod~based on SDE solver, with data prediction and uniform $\lambda$. The complete experimental results can be found in Appendix \ref{appd3}. The results demonstrate that \ourmethod~converges in a few (5 or 10) steps and produces samples with stable quality. Our algorithm significantly reduces the time cost without compromising sampling performance, which is of great practical value for MR Diffusion.


\begin{figure}[!ht]
    \centering
    \begin{minipage}[b]{0.45\textwidth}
        \centering
        \includegraphics[width=1\textwidth, trim=0 20 0 0]{figs/main_result/7_lowlight_fid.pdf}
        \subcaption{FID on \textit{low-light} dataset}
        \label{fig:main(a)}
    \end{minipage}
    \begin{minipage}[b]{0.45\textwidth}
        \centering
        \includegraphics[width=1\textwidth, trim=0 20 0 0]{figs/main_result/7_lowlight_lpips.pdf}
        \subcaption{LPIPS on \textit{low-light} dataset}
        \label{fig:main(b)}
    \end{minipage}
    \begin{minipage}[b]{0.45\textwidth}
        \centering
        \includegraphics[width=1\textwidth, trim=0 20 0 0]{figs/main_result/10_motion_fid.pdf}
        \subcaption{FID on \textit{motion-blurry} dataset}
        \label{fig:main(c)}
    \end{minipage}
    \begin{minipage}[b]{0.45\textwidth}
        \centering
        \includegraphics[width=1\textwidth, trim=0 20 0 0]{figs/main_result/10_motion_lpips.pdf}
        \subcaption{LPIPS on \textit{motion-blurry} dataset}
        \label{fig:main(d)}
    \end{minipage}
    \caption{\textbf{Perceptual evaluations on \textit{low-light} and \textit{motion-blurry} datasets.}}
    \label{fig:main}
\end{figure}

\subsection{Effects of parameter choice}\label{effects}

In Table \ref{tab:ablat_param}, we compare the results of two network parameterizations. The data prediction shows stable performance across different NFEs. The noise prediction performs similarly to data prediction with large NFEs, but its performance deteriorates significantly with smaller NFEs. The detailed analysis can be found in Section \ref{section5.3}. In Table \ref{tab:ablat_solver}, we compare \ourmethod-ODE-d-2 and \ourmethod-SDE-d-2 on the \textit{inpainting} task, which are derived from PF-ODE and reverse-time SDE respectively. SDE-based solver works better with a large NFE, whereas ODE-based solver is more effective with a small NFE. In general, neither solver type is inherently better.


% In Table \ref{tab:hazy}, we study the impact of two step size schedules on the results. On the whole, uniform $\lambda$ performs slightly better than uniform $t$. Our algorithm follows the method of \cite{lu2022dpmsolverplus} to estimate the integral part of the solution, while the analytical part does not affect the error.  Consequently, our algorithm has the same global truncation error, that is $\mathcal{O}\left(h_{max}^{k}\right)$. Note that the initial and final values of $\lambda$ depend on noise schedule and are fixed. Therefore, uniform $\lambda$ scheduling leads to the smallest $h_{max}$ and works better.

\begin{table}[ht]
    \centering
    \begin{minipage}{0.5\textwidth}
    \small
    \renewcommand{\arraystretch}{1}
    \centering
    \caption{Ablation study of network parameterizations on the Rain100H dataset.}
    % \vspace{8pt}
    \resizebox{1\textwidth}{!}{
        \begin{tabular}{cccccc}
			\toprule[1.5pt]
            % \multicolumn{6}{c}{Rainy} \\
            % \cmidrule(lr){1-6}
             NFE & Parameterization      & LPIPS\textdownarrow & FID\textdownarrow &  PSNR\textuparrow & SSIM\textuparrow  \\
            \midrule[1pt]
            \multirow{2}{*}{50}
             & Noise Prediction & \textbf{0.0606}     & \textbf{27.28}   & \textbf{28.89}     & \textbf{0.8615}    \\
             & Data Prediction & 0.0620     & 27.65   & 28.85     & 0.8602    \\
            \cmidrule(lr){1-6}
            \multirow{2}{*}{20}
              & Noise Prediction & 0.1429     & 47.31   & 27.68     & 0.7954    \\
              & Data Prediction & \textbf{0.0635}     & \textbf{27.79}   & \textbf{28.60}     & \textbf{0.8559}    \\
            \cmidrule(lr){1-6}
            \multirow{2}{*}{10}
              & Noise Prediction & 1.376     & 402.3   & 6.623     & 0.0114    \\
              & Data Prediction & \textbf{0.0678}     & \textbf{29.54}   & \textbf{28.09}     & \textbf{0.8483}    \\
            \cmidrule(lr){1-6}
            \multirow{2}{*}{5}
              & Noise Prediction & 1.416     & 447.0   & 5.755     & 0.0051    \\
              & Data Prediction & \textbf{0.0637}     & \textbf{26.92}   & \textbf{28.82}     & \textbf{0.8685}    \\       
            \bottomrule[1.5pt]
        \end{tabular}}
        \label{tab:ablat_param}
    \end{minipage}
    \hspace{0.01\textwidth}
    \begin{minipage}{0.46\textwidth}
    \small
    \renewcommand{\arraystretch}{1}
    \centering
    \caption{Ablation study of solver types on the CelebA-HQ dataset.}
    % \vspace{8pt}
        \resizebox{1\textwidth}{!}{
        \begin{tabular}{cccccc}
			\toprule[1.5pt]
            % \multicolumn{6}{c}{Raindrop} \\     
            % \cmidrule(lr){1-6}
             NFE & Solver Type     & LPIPS\textdownarrow & FID\textdownarrow &  PSNR\textuparrow & SSIM\textuparrow  \\
            \midrule[1pt]
            \multirow{2}{*}{50}
             & ODE & 0.0499     & 22.91   & 28.49     & 0.8921    \\
             & SDE & \textbf{0.0402}     & \textbf{19.09}   & \textbf{29.15}     & \textbf{0.9046}    \\
            \cmidrule(lr){1-6}
            \multirow{2}{*}{20}
              & ODE & 0.0475    & 21.35   & 28.51     & 0.8940    \\
              & SDE & \textbf{0.0408}     & \textbf{19.13}   & \textbf{28.98}    & \textbf{0.9032}    \\
            \cmidrule(lr){1-6}
            \multirow{2}{*}{10}
              & ODE & \textbf{0.0417}    & 19.44   & \textbf{28.94}     & \textbf{0.9048}    \\
              & SDE & 0.0437     & \textbf{19.29}   & 28.48     & 0.8996    \\
            \cmidrule(lr){1-6}
            \multirow{2}{*}{5}
              & ODE & \textbf{0.0526}     & 27.44   & \textbf{31.02}     & \textbf{0.9335}    \\
              & SDE & 0.0529    & \textbf{24.02}   & 28.35     & 0.8930    \\
            \bottomrule[1.5pt]
        \end{tabular}}
        \label{tab:ablat_solver}
    \end{minipage}
\end{table}


% \renewcommand{\arraystretch}{1}
%     \centering
%     \caption{Ablation study of step size schedule on the RESIDE-6k dataset.}
%     % \vspace{8pt}
%         \resizebox{1\textwidth}{!}{
%         \begin{tabular}{cccccc}
% 			\toprule[1.5pt]
%             % \multicolumn{6}{c}{Raindrop} \\     
%             % \cmidrule(lr){1-6}
%              NFE & Schedule      & LPIPS\textdownarrow & FID\textdownarrow &  PSNR\textuparrow & SSIM\textuparrow  \\
%             \midrule[1pt]
%             \multirow{2}{*}{50}
%              & uniform $t$ & 0.0271     & 5.539   & 30.00     & 0.9351    \\
%              & uniform $\lambda$ & \textbf{0.0233}     & \textbf{4.993}   & \textbf{30.19}     & \textbf{0.9427}    \\
%             \cmidrule(lr){1-6}
%             \multirow{2}{*}{20}
%               & uniform $t$ & 0.0313     & 6.000   & 29.73     & 0.9270    \\
%               & uniform $\lambda$ & \textbf{0.0240}     & \textbf{5.077}   & \textbf{30.06}    & \textbf{0.9409}    \\
%             \cmidrule(lr){1-6}
%             \multirow{2}{*}{10}
%               & uniform $t$ & 0.0309     & 6.094   & 29.42     & 0.9274    \\
%               & uniform $\lambda$ & \textbf{0.0246}     & \textbf{5.228}   & \textbf{29.65}     & \textbf{0.9372}    \\
%             \cmidrule(lr){1-6}
%             \multirow{2}{*}{5}
%               & uniform $t$ & 0.0256     & 5.477   & \textbf{29.91}     & 0.9342    \\
%               & uniform $\lambda$ & \textbf{0.0228}     & \textbf{5.174}   & 29.65     & \textbf{0.9416}    \\
%             \bottomrule[1.5pt]
%         \end{tabular}}
%         \label{tab:ablat_schedule}



\subsection{Analysis}\label{analysis}
\label{section5.3}

\begin{figure}[ht!]
    \centering
    \begin{minipage}[t]{0.6\linewidth}
        \centering
        \includegraphics[width=\linewidth, trim=0 20 10 0]{figs/trajectory_a.pdf} %trim左下右上
        \subcaption{Sampling results.}
        \label{fig:traj(a)}
    \end{minipage}
    \begin{minipage}[t]{0.35\linewidth}
        \centering
        \includegraphics[width=\linewidth, trim=0 0 0 0]{figs/trajectory_b.pdf} %trim左下右上
        \subcaption{Trajectory.}
        \label{fig:traj(b)}
    \end{minipage}
    \caption{\textbf{Sampling trajectories.} In (a), we compare our method (with order 1 and order 2) and previous sampling methods (i.e., posterior sampling and Euler discretization) on a motion blurry image. The numbers in parentheses indicate the NFE. In (b), we illustrate trajectories of each sampling method. Previous methods need to take many unnecessary paths to converge. With few NFEs, they fail to reach the ground truth (i.e., the location of $\boldsymbol{x}_0$). Our methods follow a more direct trajectory.}
    \label{fig:traj}
\end{figure}

\textbf{Sampling trajectory.}~ Inspired by the design idea of NCSN \citep{song2019ncsn}, we provide a new perspective of diffusion sampling process. \cite{song2019ncsn} consider each data point (e.g., an image) as a point in high-dimensional space. During the diffusion process, noise is added to each point $\boldsymbol{x}_0$, causing it to spread throughout the space, while the score function (a neural network) \textit{remembers} the direction towards $\boldsymbol{x}_0$. In the sampling process, we start from a random point by sampling a Gaussian distribution and follow the guidance of the reverse-time SDE (or PF-ODE) and the score function to locate $\boldsymbol{x}_0$. By connecting each intermediate state $\boldsymbol{x}_t$, we obtain a sampling trajectory. However, this trajectory exists in a high-dimensional space, making it difficult to visualize. Therefore, we use Principal Component Analysis (PCA) to reduce $\boldsymbol{x}_t$ to two dimensions, obtaining the projection of the sampling trajectory in 2D space. As shown in Figure \ref{fig:traj}, we present an example. Previous sampling methods \citep{luo2024posterior} often require a long path to find $\boldsymbol{x}_0$, and reducing NFE can lead to cumulative errors, making it impossible to locate $\boldsymbol{x}_0$. In contrast, our algorithm produces more direct trajectories, allowing us to find $\boldsymbol{x}_0$ with fewer NFEs.

\begin{figure*}[ht]
    \centering
    \begin{minipage}[t]{0.45\linewidth}
        \centering
        \includegraphics[width=\linewidth, trim=0 0 0 0]{figs/convergence_a.pdf} %trim左下右上
        \subcaption{Sampling results.}
        \label{fig:convergence(a)}
    \end{minipage}
    \begin{minipage}[t]{0.43\linewidth}
        \centering
        \includegraphics[width=\linewidth, trim=0 20 0 0]{figs/convergence_b.pdf} %trim左下右上
        \subcaption{Ratio of convergence.}
        \label{fig:convergence(b)}
    \end{minipage}
    \caption{\textbf{Convergence of noise prediction and data prediction.} In (a), we choose a low-light image for example. The numbers in parentheses indicate the NFE. In (b), we illustrate the ratio of components of neural network output that satisfy the Taylor expansion convergence requirement.}
    \label{fig:converge}
\end{figure*}

\textbf{Numerical stability of parameterizations.}~ From Table 1, we observe poor sampling results for noise prediction in the case of few NFEs. The reason may be that the neural network parameterized by noise prediction is numerically unstable. Recall that we used Taylor expansion in Eq.(\ref{14}), and the condition for the equality to hold is $|\lambda-\lambda_s|<\boldsymbol{R}(s)$. And the radius of convergence $\boldsymbol{R}(t)$ can be calculated by
\begin{equation}
\frac{1}{\boldsymbol{R}(t)}=\lim_{n\rightarrow\infty}\left|\frac{\boldsymbol{c}_{n+1}(t)}{\boldsymbol{c}_n(t)}\right|,
\end{equation}
where $\boldsymbol{c}_n(t)$ is the coefficient of the $n$-th term in Taylor expansion. We are unable to compute this limit and can only compute the $n=0$ case as an approximation. The output of the neural network can be viewed as a vector, with each component corresponding to a radius of convergence. At each time step, we count the ratio of components that satisfy $\boldsymbol{R}_i(s)>|\lambda-\lambda_s|$ as a criterion for judging the convergence, where $i$ denotes the $i$-th component. As shown in Figure \ref{fig:converge}, the neural network parameterized by data prediction meets the convergence criteria at almost every step. However, the neural network parameterized by noise prediction always has components that cannot converge, which will lead to large errors and failed sampling. Therefore, data prediction has better numerical stability and is a more recommended choice.



\section{Related Work}
\paragraph{LLMs for Visual Program Generation}
Visual programming systems (e.g., LabView~\cite{bitter2006labview}, XG5000~\cite{XG5000Manual}) typically feature node-based interfaces that let users visually write and modify programs. Recently, researchers have begun utilizing LLMs to generate VPLs, as they are known for their powerful text-based code generation capabilities. For example, \citet{cai-etal-2024-low-code} integrates low-code visual programming with LLM-based task execution for direct interaction with LLMs, while \citet{zhang2024benchmarking} studies generation of node-based visual dataflow languages in audio programming. Similarly, \citet{xue2024comfybenchbenchmarkingllmbasedagents,52868} investigates Machine Learning workflow generation from natural language commands and demonstrates that metaprogram-based text formats outperform other formats like JSON. However, these prompting-based methods face limitations for VPLs like Ladder Diagram, where custom I/O mapping and domain-specific syntax are crucial. Thus, we study fine-tuning approaches with domain-specific data to better capture these details.

\paragraph{LLM-based PLC code generation}
Programmable Logic Controllers (PLCs) are essential components in industrial automation and are used to control machinery and processes reliably and efficiently. Among the programming languages defined by the IEC 61131-3 standard~\cite{IEC61131-3}, Structured Text (ST) and Ladder Diagram (LD) are commonly used for programming PLCs. Research in this area has focused on utilizing LLMs to generate ST code from natural language descriptions. Recent studies have demonstrated the potential of LLMs in generating high-quality ST code~\cite{koziolek2023chatgpt, koziolek2024llm}, enhancing safety and accuracy with verification tools and user feedback~\cite{fakih2024llm4plc}, and automating code generation and verification using multi-agent frameworks~\cite{liu2024agents4plc}. Although these advances have improved PLC code generation, they primarily focus on ST, despite LD being widely used in industrial settings due to its similarity to electrical circuits~\cite{ladderlogic}. While \citet{Zhang_2024} attempts to generate LD as an ASCII art based on user instructions in a zero-shot manner, their findings show that even advanced LLMs struggle with basic LD generation. These limitations highlight the necessity of training-based methods for LD generation. In this work, we address this gap by introducing a training-based approach for LLMs to generate LD and thus pave the way for the broader adoption of AI-assisted PLC programming.


\clearpage

\bibliography{reference}
\bibliographystyle{icml2025}


%%%%%%%%%%%%%%%%%%%%%%%%%%%%%%%%%%%%%%%%%%%%%%%%%%%%%%%%%%%%%%%%%%%%%%%%%%%%%%%
%%%%%%%%%%%%%%%%%%%%%%%%%%%%%%%%%%%%%%%%%%%%%%%%%%%%%%%%%%%%%%%%%%%%%%%%%%%%%%%
% APPENDIX
%%%%%%%%%%%%%%%%%%%%%%%%%%%%%%%%%%%%%%%%%%%%%%%%%%%%%%%%%%%%%%%%%%%%%%%%%%%%%%%
%%%%%%%%%%%%%%%%%%%%%%%%%%%%%%%%%%%%%%%%%%%%%%%%%%%%%%%%%%%%%%%%%%%%%%%%%%%%%%%
\newpage
\appendix
\onecolumn
\subsection{Lloyd-Max Algorithm}
\label{subsec:Lloyd-Max}
For a given quantization bitwidth $B$ and an operand $\bm{X}$, the Lloyd-Max algorithm finds $2^B$ quantization levels $\{\hat{x}_i\}_{i=1}^{2^B}$ such that quantizing $\bm{X}$ by rounding each scalar in $\bm{X}$ to the nearest quantization level minimizes the quantization MSE. 

The algorithm starts with an initial guess of quantization levels and then iteratively computes quantization thresholds $\{\tau_i\}_{i=1}^{2^B-1}$ and updates quantization levels $\{\hat{x}_i\}_{i=1}^{2^B}$. Specifically, at iteration $n$, thresholds are set to the midpoints of the previous iteration's levels:
\begin{align*}
    \tau_i^{(n)}=\frac{\hat{x}_i^{(n-1)}+\hat{x}_{i+1}^{(n-1)}}2 \text{ for } i=1\ldots 2^B-1
\end{align*}
Subsequently, the quantization levels are re-computed as conditional means of the data regions defined by the new thresholds:
\begin{align*}
    \hat{x}_i^{(n)}=\mathbb{E}\left[ \bm{X} \big| \bm{X}\in [\tau_{i-1}^{(n)},\tau_i^{(n)}] \right] \text{ for } i=1\ldots 2^B
\end{align*}
where to satisfy boundary conditions we have $\tau_0=-\infty$ and $\tau_{2^B}=\infty$. The algorithm iterates the above steps until convergence.

Figure \ref{fig:lm_quant} compares the quantization levels of a $7$-bit floating point (E3M3) quantizer (left) to a $7$-bit Lloyd-Max quantizer (right) when quantizing a layer of weights from the GPT3-126M model at a per-tensor granularity. As shown, the Lloyd-Max quantizer achieves substantially lower quantization MSE. Further, Table \ref{tab:FP7_vs_LM7} shows the superior perplexity achieved by Lloyd-Max quantizers for bitwidths of $7$, $6$ and $5$. The difference between the quantizers is clear at 5 bits, where per-tensor FP quantization incurs a drastic and unacceptable increase in perplexity, while Lloyd-Max quantization incurs a much smaller increase. Nevertheless, we note that even the optimal Lloyd-Max quantizer incurs a notable ($\sim 1.5$) increase in perplexity due to the coarse granularity of quantization. 

\begin{figure}[h]
  \centering
  \includegraphics[width=0.7\linewidth]{sections/figures/LM7_FP7.pdf}
  \caption{\small Quantization levels and the corresponding quantization MSE of Floating Point (left) vs Lloyd-Max (right) Quantizers for a layer of weights in the GPT3-126M model.}
  \label{fig:lm_quant}
\end{figure}

\begin{table}[h]\scriptsize
\begin{center}
\caption{\label{tab:FP7_vs_LM7} \small Comparing perplexity (lower is better) achieved by floating point quantizers and Lloyd-Max quantizers on a GPT3-126M model for the Wikitext-103 dataset.}
\begin{tabular}{c|cc|c}
\hline
 \multirow{2}{*}{\textbf{Bitwidth}} & \multicolumn{2}{|c|}{\textbf{Floating-Point Quantizer}} & \textbf{Lloyd-Max Quantizer} \\
 & Best Format & Wikitext-103 Perplexity & Wikitext-103 Perplexity \\
\hline
7 & E3M3 & 18.32 & 18.27 \\
6 & E3M2 & 19.07 & 18.51 \\
5 & E4M0 & 43.89 & 19.71 \\
\hline
\end{tabular}
\end{center}
\end{table}

\subsection{Proof of Local Optimality of LO-BCQ}
\label{subsec:lobcq_opt_proof}
For a given block $\bm{b}_j$, the quantization MSE during LO-BCQ can be empirically evaluated as $\frac{1}{L_b}\lVert \bm{b}_j- \bm{\hat{b}}_j\rVert^2_2$ where $\bm{\hat{b}}_j$ is computed from equation (\ref{eq:clustered_quantization_definition}) as $C_{f(\bm{b}_j)}(\bm{b}_j)$. Further, for a given block cluster $\mathcal{B}_i$, we compute the quantization MSE as $\frac{1}{|\mathcal{B}_{i}|}\sum_{\bm{b} \in \mathcal{B}_{i}} \frac{1}{L_b}\lVert \bm{b}- C_i^{(n)}(\bm{b})\rVert^2_2$. Therefore, at the end of iteration $n$, we evaluate the overall quantization MSE $J^{(n)}$ for a given operand $\bm{X}$ composed of $N_c$ block clusters as:
\begin{align*}
    \label{eq:mse_iter_n}
    J^{(n)} = \frac{1}{N_c} \sum_{i=1}^{N_c} \frac{1}{|\mathcal{B}_{i}^{(n)}|}\sum_{\bm{v} \in \mathcal{B}_{i}^{(n)}} \frac{1}{L_b}\lVert \bm{b}- B_i^{(n)}(\bm{b})\rVert^2_2
\end{align*}

At the end of iteration $n$, the codebooks are updated from $\mathcal{C}^{(n-1)}$ to $\mathcal{C}^{(n)}$. However, the mapping of a given vector $\bm{b}_j$ to quantizers $\mathcal{C}^{(n)}$ remains as  $f^{(n)}(\bm{b}_j)$. At the next iteration, during the vector clustering step, $f^{(n+1)}(\bm{b}_j)$ finds new mapping of $\bm{b}_j$ to updated codebooks $\mathcal{C}^{(n)}$ such that the quantization MSE over the candidate codebooks is minimized. Therefore, we obtain the following result for $\bm{b}_j$:
\begin{align*}
\frac{1}{L_b}\lVert \bm{b}_j - C_{f^{(n+1)}(\bm{b}_j)}^{(n)}(\bm{b}_j)\rVert^2_2 \le \frac{1}{L_b}\lVert \bm{b}_j - C_{f^{(n)}(\bm{b}_j)}^{(n)}(\bm{b}_j)\rVert^2_2
\end{align*}

That is, quantizing $\bm{b}_j$ at the end of the block clustering step of iteration $n+1$ results in lower quantization MSE compared to quantizing at the end of iteration $n$. Since this is true for all $\bm{b} \in \bm{X}$, we assert the following:
\begin{equation}
\begin{split}
\label{eq:mse_ineq_1}
    \tilde{J}^{(n+1)} &= \frac{1}{N_c} \sum_{i=1}^{N_c} \frac{1}{|\mathcal{B}_{i}^{(n+1)}|}\sum_{\bm{b} \in \mathcal{B}_{i}^{(n+1)}} \frac{1}{L_b}\lVert \bm{b} - C_i^{(n)}(b)\rVert^2_2 \le J^{(n)}
\end{split}
\end{equation}
where $\tilde{J}^{(n+1)}$ is the the quantization MSE after the vector clustering step at iteration $n+1$.

Next, during the codebook update step (\ref{eq:quantizers_update}) at iteration $n+1$, the per-cluster codebooks $\mathcal{C}^{(n)}$ are updated to $\mathcal{C}^{(n+1)}$ by invoking the Lloyd-Max algorithm \citep{Lloyd}. We know that for any given value distribution, the Lloyd-Max algorithm minimizes the quantization MSE. Therefore, for a given vector cluster $\mathcal{B}_i$ we obtain the following result:

\begin{equation}
    \frac{1}{|\mathcal{B}_{i}^{(n+1)}|}\sum_{\bm{b} \in \mathcal{B}_{i}^{(n+1)}} \frac{1}{L_b}\lVert \bm{b}- C_i^{(n+1)}(\bm{b})\rVert^2_2 \le \frac{1}{|\mathcal{B}_{i}^{(n+1)}|}\sum_{\bm{b} \in \mathcal{B}_{i}^{(n+1)}} \frac{1}{L_b}\lVert \bm{b}- C_i^{(n)}(\bm{b})\rVert^2_2
\end{equation}

The above equation states that quantizing the given block cluster $\mathcal{B}_i$ after updating the associated codebook from $C_i^{(n)}$ to $C_i^{(n+1)}$ results in lower quantization MSE. Since this is true for all the block clusters, we derive the following result: 
\begin{equation}
\begin{split}
\label{eq:mse_ineq_2}
     J^{(n+1)} &= \frac{1}{N_c} \sum_{i=1}^{N_c} \frac{1}{|\mathcal{B}_{i}^{(n+1)}|}\sum_{\bm{b} \in \mathcal{B}_{i}^{(n+1)}} \frac{1}{L_b}\lVert \bm{b}- C_i^{(n+1)}(\bm{b})\rVert^2_2  \le \tilde{J}^{(n+1)}   
\end{split}
\end{equation}

Following (\ref{eq:mse_ineq_1}) and (\ref{eq:mse_ineq_2}), we find that the quantization MSE is non-increasing for each iteration, that is, $J^{(1)} \ge J^{(2)} \ge J^{(3)} \ge \ldots \ge J^{(M)}$ where $M$ is the maximum number of iterations. 
%Therefore, we can say that if the algorithm converges, then it must be that it has converged to a local minimum. 
\hfill $\blacksquare$


\begin{figure}
    \begin{center}
    \includegraphics[width=0.5\textwidth]{sections//figures/mse_vs_iter.pdf}
    \end{center}
    \caption{\small NMSE vs iterations during LO-BCQ compared to other block quantization proposals}
    \label{fig:nmse_vs_iter}
\end{figure}

Figure \ref{fig:nmse_vs_iter} shows the empirical convergence of LO-BCQ across several block lengths and number of codebooks. Also, the MSE achieved by LO-BCQ is compared to baselines such as MXFP and VSQ. As shown, LO-BCQ converges to a lower MSE than the baselines. Further, we achieve better convergence for larger number of codebooks ($N_c$) and for a smaller block length ($L_b$), both of which increase the bitwidth of BCQ (see Eq \ref{eq:bitwidth_bcq}).


\subsection{Additional Accuracy Results}
%Table \ref{tab:lobcq_config} lists the various LOBCQ configurations and their corresponding bitwidths.
\begin{table}
\setlength{\tabcolsep}{4.75pt}
\begin{center}
\caption{\label{tab:lobcq_config} Various LO-BCQ configurations and their bitwidths.}
\begin{tabular}{|c||c|c|c|c||c|c||c|} 
\hline
 & \multicolumn{4}{|c||}{$L_b=8$} & \multicolumn{2}{|c||}{$L_b=4$} & $L_b=2$ \\
 \hline
 \backslashbox{$L_A$\kern-1em}{\kern-1em$N_c$} & 2 & 4 & 8 & 16 & 2 & 4 & 2 \\
 \hline
 64 & 4.25 & 4.375 & 4.5 & 4.625 & 4.375 & 4.625 & 4.625\\
 \hline
 32 & 4.375 & 4.5 & 4.625& 4.75 & 4.5 & 4.75 & 4.75 \\
 \hline
 16 & 4.625 & 4.75& 4.875 & 5 & 4.75 & 5 & 5 \\
 \hline
\end{tabular}
\end{center}
\end{table}

%\subsection{Perplexity achieved by various LO-BCQ configurations on Wikitext-103 dataset}

\begin{table} \centering
\begin{tabular}{|c||c|c|c|c||c|c||c|} 
\hline
 $L_b \rightarrow$& \multicolumn{4}{c||}{8} & \multicolumn{2}{c||}{4} & 2\\
 \hline
 \backslashbox{$L_A$\kern-1em}{\kern-1em$N_c$} & 2 & 4 & 8 & 16 & 2 & 4 & 2  \\
 %$N_c \rightarrow$ & 2 & 4 & 8 & 16 & 2 & 4 & 2 \\
 \hline
 \hline
 \multicolumn{8}{c}{GPT3-1.3B (FP32 PPL = 9.98)} \\ 
 \hline
 \hline
 64 & 10.40 & 10.23 & 10.17 & 10.15 &  10.28 & 10.18 & 10.19 \\
 \hline
 32 & 10.25 & 10.20 & 10.15 & 10.12 &  10.23 & 10.17 & 10.17 \\
 \hline
 16 & 10.22 & 10.16 & 10.10 & 10.09 &  10.21 & 10.14 & 10.16 \\
 \hline
  \hline
 \multicolumn{8}{c}{GPT3-8B (FP32 PPL = 7.38)} \\ 
 \hline
 \hline
 64 & 7.61 & 7.52 & 7.48 &  7.47 &  7.55 &  7.49 & 7.50 \\
 \hline
 32 & 7.52 & 7.50 & 7.46 &  7.45 &  7.52 &  7.48 & 7.48  \\
 \hline
 16 & 7.51 & 7.48 & 7.44 &  7.44 &  7.51 &  7.49 & 7.47  \\
 \hline
\end{tabular}
\caption{\label{tab:ppl_gpt3_abalation} Wikitext-103 perplexity across GPT3-1.3B and 8B models.}
\end{table}

\begin{table} \centering
\begin{tabular}{|c||c|c|c|c||} 
\hline
 $L_b \rightarrow$& \multicolumn{4}{c||}{8}\\
 \hline
 \backslashbox{$L_A$\kern-1em}{\kern-1em$N_c$} & 2 & 4 & 8 & 16 \\
 %$N_c \rightarrow$ & 2 & 4 & 8 & 16 & 2 & 4 & 2 \\
 \hline
 \hline
 \multicolumn{5}{|c|}{Llama2-7B (FP32 PPL = 5.06)} \\ 
 \hline
 \hline
 64 & 5.31 & 5.26 & 5.19 & 5.18  \\
 \hline
 32 & 5.23 & 5.25 & 5.18 & 5.15  \\
 \hline
 16 & 5.23 & 5.19 & 5.16 & 5.14  \\
 \hline
 \multicolumn{5}{|c|}{Nemotron4-15B (FP32 PPL = 5.87)} \\ 
 \hline
 \hline
 64  & 6.3 & 6.20 & 6.13 & 6.08  \\
 \hline
 32  & 6.24 & 6.12 & 6.07 & 6.03  \\
 \hline
 16  & 6.12 & 6.14 & 6.04 & 6.02  \\
 \hline
 \multicolumn{5}{|c|}{Nemotron4-340B (FP32 PPL = 3.48)} \\ 
 \hline
 \hline
 64 & 3.67 & 3.62 & 3.60 & 3.59 \\
 \hline
 32 & 3.63 & 3.61 & 3.59 & 3.56 \\
 \hline
 16 & 3.61 & 3.58 & 3.57 & 3.55 \\
 \hline
\end{tabular}
\caption{\label{tab:ppl_llama7B_nemo15B} Wikitext-103 perplexity compared to FP32 baseline in Llama2-7B and Nemotron4-15B, 340B models}
\end{table}

%\subsection{Perplexity achieved by various LO-BCQ configurations on MMLU dataset}


\begin{table} \centering
\begin{tabular}{|c||c|c|c|c||c|c|c|c|} 
\hline
 $L_b \rightarrow$& \multicolumn{4}{c||}{8} & \multicolumn{4}{c||}{8}\\
 \hline
 \backslashbox{$L_A$\kern-1em}{\kern-1em$N_c$} & 2 & 4 & 8 & 16 & 2 & 4 & 8 & 16  \\
 %$N_c \rightarrow$ & 2 & 4 & 8 & 16 & 2 & 4 & 2 \\
 \hline
 \hline
 \multicolumn{5}{|c|}{Llama2-7B (FP32 Accuracy = 45.8\%)} & \multicolumn{4}{|c|}{Llama2-70B (FP32 Accuracy = 69.12\%)} \\ 
 \hline
 \hline
 64 & 43.9 & 43.4 & 43.9 & 44.9 & 68.07 & 68.27 & 68.17 & 68.75 \\
 \hline
 32 & 44.5 & 43.8 & 44.9 & 44.5 & 68.37 & 68.51 & 68.35 & 68.27  \\
 \hline
 16 & 43.9 & 42.7 & 44.9 & 45 & 68.12 & 68.77 & 68.31 & 68.59  \\
 \hline
 \hline
 \multicolumn{5}{|c|}{GPT3-22B (FP32 Accuracy = 38.75\%)} & \multicolumn{4}{|c|}{Nemotron4-15B (FP32 Accuracy = 64.3\%)} \\ 
 \hline
 \hline
 64 & 36.71 & 38.85 & 38.13 & 38.92 & 63.17 & 62.36 & 63.72 & 64.09 \\
 \hline
 32 & 37.95 & 38.69 & 39.45 & 38.34 & 64.05 & 62.30 & 63.8 & 64.33  \\
 \hline
 16 & 38.88 & 38.80 & 38.31 & 38.92 & 63.22 & 63.51 & 63.93 & 64.43  \\
 \hline
\end{tabular}
\caption{\label{tab:mmlu_abalation} Accuracy on MMLU dataset across GPT3-22B, Llama2-7B, 70B and Nemotron4-15B models.}
\end{table}


%\subsection{Perplexity achieved by various LO-BCQ configurations on LM evaluation harness}

\begin{table} \centering
\begin{tabular}{|c||c|c|c|c||c|c|c|c|} 
\hline
 $L_b \rightarrow$& \multicolumn{4}{c||}{8} & \multicolumn{4}{c||}{8}\\
 \hline
 \backslashbox{$L_A$\kern-1em}{\kern-1em$N_c$} & 2 & 4 & 8 & 16 & 2 & 4 & 8 & 16  \\
 %$N_c \rightarrow$ & 2 & 4 & 8 & 16 & 2 & 4 & 2 \\
 \hline
 \hline
 \multicolumn{5}{|c|}{Race (FP32 Accuracy = 37.51\%)} & \multicolumn{4}{|c|}{Boolq (FP32 Accuracy = 64.62\%)} \\ 
 \hline
 \hline
 64 & 36.94 & 37.13 & 36.27 & 37.13 & 63.73 & 62.26 & 63.49 & 63.36 \\
 \hline
 32 & 37.03 & 36.36 & 36.08 & 37.03 & 62.54 & 63.51 & 63.49 & 63.55  \\
 \hline
 16 & 37.03 & 37.03 & 36.46 & 37.03 & 61.1 & 63.79 & 63.58 & 63.33  \\
 \hline
 \hline
 \multicolumn{5}{|c|}{Winogrande (FP32 Accuracy = 58.01\%)} & \multicolumn{4}{|c|}{Piqa (FP32 Accuracy = 74.21\%)} \\ 
 \hline
 \hline
 64 & 58.17 & 57.22 & 57.85 & 58.33 & 73.01 & 73.07 & 73.07 & 72.80 \\
 \hline
 32 & 59.12 & 58.09 & 57.85 & 58.41 & 73.01 & 73.94 & 72.74 & 73.18  \\
 \hline
 16 & 57.93 & 58.88 & 57.93 & 58.56 & 73.94 & 72.80 & 73.01 & 73.94  \\
 \hline
\end{tabular}
\caption{\label{tab:mmlu_abalation} Accuracy on LM evaluation harness tasks on GPT3-1.3B model.}
\end{table}

\begin{table} \centering
\begin{tabular}{|c||c|c|c|c||c|c|c|c|} 
\hline
 $L_b \rightarrow$& \multicolumn{4}{c||}{8} & \multicolumn{4}{c||}{8}\\
 \hline
 \backslashbox{$L_A$\kern-1em}{\kern-1em$N_c$} & 2 & 4 & 8 & 16 & 2 & 4 & 8 & 16  \\
 %$N_c \rightarrow$ & 2 & 4 & 8 & 16 & 2 & 4 & 2 \\
 \hline
 \hline
 \multicolumn{5}{|c|}{Race (FP32 Accuracy = 41.34\%)} & \multicolumn{4}{|c|}{Boolq (FP32 Accuracy = 68.32\%)} \\ 
 \hline
 \hline
 64 & 40.48 & 40.10 & 39.43 & 39.90 & 69.20 & 68.41 & 69.45 & 68.56 \\
 \hline
 32 & 39.52 & 39.52 & 40.77 & 39.62 & 68.32 & 67.43 & 68.17 & 69.30  \\
 \hline
 16 & 39.81 & 39.71 & 39.90 & 40.38 & 68.10 & 66.33 & 69.51 & 69.42  \\
 \hline
 \hline
 \multicolumn{5}{|c|}{Winogrande (FP32 Accuracy = 67.88\%)} & \multicolumn{4}{|c|}{Piqa (FP32 Accuracy = 78.78\%)} \\ 
 \hline
 \hline
 64 & 66.85 & 66.61 & 67.72 & 67.88 & 77.31 & 77.42 & 77.75 & 77.64 \\
 \hline
 32 & 67.25 & 67.72 & 67.72 & 67.00 & 77.31 & 77.04 & 77.80 & 77.37  \\
 \hline
 16 & 68.11 & 68.90 & 67.88 & 67.48 & 77.37 & 78.13 & 78.13 & 77.69  \\
 \hline
\end{tabular}
\caption{\label{tab:mmlu_abalation} Accuracy on LM evaluation harness tasks on GPT3-8B model.}
\end{table}

\begin{table} \centering
\begin{tabular}{|c||c|c|c|c||c|c|c|c|} 
\hline
 $L_b \rightarrow$& \multicolumn{4}{c||}{8} & \multicolumn{4}{c||}{8}\\
 \hline
 \backslashbox{$L_A$\kern-1em}{\kern-1em$N_c$} & 2 & 4 & 8 & 16 & 2 & 4 & 8 & 16  \\
 %$N_c \rightarrow$ & 2 & 4 & 8 & 16 & 2 & 4 & 2 \\
 \hline
 \hline
 \multicolumn{5}{|c|}{Race (FP32 Accuracy = 40.67\%)} & \multicolumn{4}{|c|}{Boolq (FP32 Accuracy = 76.54\%)} \\ 
 \hline
 \hline
 64 & 40.48 & 40.10 & 39.43 & 39.90 & 75.41 & 75.11 & 77.09 & 75.66 \\
 \hline
 32 & 39.52 & 39.52 & 40.77 & 39.62 & 76.02 & 76.02 & 75.96 & 75.35  \\
 \hline
 16 & 39.81 & 39.71 & 39.90 & 40.38 & 75.05 & 73.82 & 75.72 & 76.09  \\
 \hline
 \hline
 \multicolumn{5}{|c|}{Winogrande (FP32 Accuracy = 70.64\%)} & \multicolumn{4}{|c|}{Piqa (FP32 Accuracy = 79.16\%)} \\ 
 \hline
 \hline
 64 & 69.14 & 70.17 & 70.17 & 70.56 & 78.24 & 79.00 & 78.62 & 78.73 \\
 \hline
 32 & 70.96 & 69.69 & 71.27 & 69.30 & 78.56 & 79.49 & 79.16 & 78.89  \\
 \hline
 16 & 71.03 & 69.53 & 69.69 & 70.40 & 78.13 & 79.16 & 79.00 & 79.00  \\
 \hline
\end{tabular}
\caption{\label{tab:mmlu_abalation} Accuracy on LM evaluation harness tasks on GPT3-22B model.}
\end{table}

\begin{table} \centering
\begin{tabular}{|c||c|c|c|c||c|c|c|c|} 
\hline
 $L_b \rightarrow$& \multicolumn{4}{c||}{8} & \multicolumn{4}{c||}{8}\\
 \hline
 \backslashbox{$L_A$\kern-1em}{\kern-1em$N_c$} & 2 & 4 & 8 & 16 & 2 & 4 & 8 & 16  \\
 %$N_c \rightarrow$ & 2 & 4 & 8 & 16 & 2 & 4 & 2 \\
 \hline
 \hline
 \multicolumn{5}{|c|}{Race (FP32 Accuracy = 44.4\%)} & \multicolumn{4}{|c|}{Boolq (FP32 Accuracy = 79.29\%)} \\ 
 \hline
 \hline
 64 & 42.49 & 42.51 & 42.58 & 43.45 & 77.58 & 77.37 & 77.43 & 78.1 \\
 \hline
 32 & 43.35 & 42.49 & 43.64 & 43.73 & 77.86 & 75.32 & 77.28 & 77.86  \\
 \hline
 16 & 44.21 & 44.21 & 43.64 & 42.97 & 78.65 & 77 & 76.94 & 77.98  \\
 \hline
 \hline
 \multicolumn{5}{|c|}{Winogrande (FP32 Accuracy = 69.38\%)} & \multicolumn{4}{|c|}{Piqa (FP32 Accuracy = 78.07\%)} \\ 
 \hline
 \hline
 64 & 68.9 & 68.43 & 69.77 & 68.19 & 77.09 & 76.82 & 77.09 & 77.86 \\
 \hline
 32 & 69.38 & 68.51 & 68.82 & 68.90 & 78.07 & 76.71 & 78.07 & 77.86  \\
 \hline
 16 & 69.53 & 67.09 & 69.38 & 68.90 & 77.37 & 77.8 & 77.91 & 77.69  \\
 \hline
\end{tabular}
\caption{\label{tab:mmlu_abalation} Accuracy on LM evaluation harness tasks on Llama2-7B model.}
\end{table}

\begin{table} \centering
\begin{tabular}{|c||c|c|c|c||c|c|c|c|} 
\hline
 $L_b \rightarrow$& \multicolumn{4}{c||}{8} & \multicolumn{4}{c||}{8}\\
 \hline
 \backslashbox{$L_A$\kern-1em}{\kern-1em$N_c$} & 2 & 4 & 8 & 16 & 2 & 4 & 8 & 16  \\
 %$N_c \rightarrow$ & 2 & 4 & 8 & 16 & 2 & 4 & 2 \\
 \hline
 \hline
 \multicolumn{5}{|c|}{Race (FP32 Accuracy = 48.8\%)} & \multicolumn{4}{|c|}{Boolq (FP32 Accuracy = 85.23\%)} \\ 
 \hline
 \hline
 64 & 49.00 & 49.00 & 49.28 & 48.71 & 82.82 & 84.28 & 84.03 & 84.25 \\
 \hline
 32 & 49.57 & 48.52 & 48.33 & 49.28 & 83.85 & 84.46 & 84.31 & 84.93  \\
 \hline
 16 & 49.85 & 49.09 & 49.28 & 48.99 & 85.11 & 84.46 & 84.61 & 83.94  \\
 \hline
 \hline
 \multicolumn{5}{|c|}{Winogrande (FP32 Accuracy = 79.95\%)} & \multicolumn{4}{|c|}{Piqa (FP32 Accuracy = 81.56\%)} \\ 
 \hline
 \hline
 64 & 78.77 & 78.45 & 78.37 & 79.16 & 81.45 & 80.69 & 81.45 & 81.5 \\
 \hline
 32 & 78.45 & 79.01 & 78.69 & 80.66 & 81.56 & 80.58 & 81.18 & 81.34  \\
 \hline
 16 & 79.95 & 79.56 & 79.79 & 79.72 & 81.28 & 81.66 & 81.28 & 80.96  \\
 \hline
\end{tabular}
\caption{\label{tab:mmlu_abalation} Accuracy on LM evaluation harness tasks on Llama2-70B model.}
\end{table}

%\section{MSE Studies}
%\textcolor{red}{TODO}


\subsection{Number Formats and Quantization Method}
\label{subsec:numFormats_quantMethod}
\subsubsection{Integer Format}
An $n$-bit signed integer (INT) is typically represented with a 2s-complement format \citep{yao2022zeroquant,xiao2023smoothquant,dai2021vsq}, where the most significant bit denotes the sign.

\subsubsection{Floating Point Format}
An $n$-bit signed floating point (FP) number $x$ comprises of a 1-bit sign ($x_{\mathrm{sign}}$), $B_m$-bit mantissa ($x_{\mathrm{mant}}$) and $B_e$-bit exponent ($x_{\mathrm{exp}}$) such that $B_m+B_e=n-1$. The associated constant exponent bias ($E_{\mathrm{bias}}$) is computed as $(2^{{B_e}-1}-1)$. We denote this format as $E_{B_e}M_{B_m}$.  

\subsubsection{Quantization Scheme}
\label{subsec:quant_method}
A quantization scheme dictates how a given unquantized tensor is converted to its quantized representation. We consider FP formats for the purpose of illustration. Given an unquantized tensor $\bm{X}$ and an FP format $E_{B_e}M_{B_m}$, we first, we compute the quantization scale factor $s_X$ that maps the maximum absolute value of $\bm{X}$ to the maximum quantization level of the $E_{B_e}M_{B_m}$ format as follows:
\begin{align}
\label{eq:sf}
    s_X = \frac{\mathrm{max}(|\bm{X}|)}{\mathrm{max}(E_{B_e}M_{B_m})}
\end{align}
In the above equation, $|\cdot|$ denotes the absolute value function.

Next, we scale $\bm{X}$ by $s_X$ and quantize it to $\hat{\bm{X}}$ by rounding it to the nearest quantization level of $E_{B_e}M_{B_m}$ as:

\begin{align}
\label{eq:tensor_quant}
    \hat{\bm{X}} = \text{round-to-nearest}\left(\frac{\bm{X}}{s_X}, E_{B_e}M_{B_m}\right)
\end{align}

We perform dynamic max-scaled quantization \citep{wu2020integer}, where the scale factor $s$ for activations is dynamically computed during runtime.

\subsection{Vector Scaled Quantization}
\begin{wrapfigure}{r}{0.35\linewidth}
  \centering
  \includegraphics[width=\linewidth]{sections/figures/vsquant.jpg}
  \caption{\small Vectorwise decomposition for per-vector scaled quantization (VSQ \citep{dai2021vsq}).}
  \label{fig:vsquant}
\end{wrapfigure}
During VSQ \citep{dai2021vsq}, the operand tensors are decomposed into 1D vectors in a hardware friendly manner as shown in Figure \ref{fig:vsquant}. Since the decomposed tensors are used as operands in matrix multiplications during inference, it is beneficial to perform this decomposition along the reduction dimension of the multiplication. The vectorwise quantization is performed similar to tensorwise quantization described in Equations \ref{eq:sf} and \ref{eq:tensor_quant}, where a scale factor $s_v$ is required for each vector $\bm{v}$ that maps the maximum absolute value of that vector to the maximum quantization level. While smaller vector lengths can lead to larger accuracy gains, the associated memory and computational overheads due to the per-vector scale factors increases. To alleviate these overheads, VSQ \citep{dai2021vsq} proposed a second level quantization of the per-vector scale factors to unsigned integers, while MX \citep{rouhani2023shared} quantizes them to integer powers of 2 (denoted as $2^{INT}$).

\subsubsection{MX Format}
The MX format proposed in \citep{rouhani2023microscaling} introduces the concept of sub-block shifting. For every two scalar elements of $b$-bits each, there is a shared exponent bit. The value of this exponent bit is determined through an empirical analysis that targets minimizing quantization MSE. We note that the FP format $E_{1}M_{b}$ is strictly better than MX from an accuracy perspective since it allocates a dedicated exponent bit to each scalar as opposed to sharing it across two scalars. Therefore, we conservatively bound the accuracy of a $b+2$-bit signed MX format with that of a $E_{1}M_{b}$ format in our comparisons. For instance, we use E1M2 format as a proxy for MX4.

\begin{figure}
    \centering
    \includegraphics[width=1\linewidth]{sections//figures/BlockFormats.pdf}
    \caption{\small Comparing LO-BCQ to MX format.}
    \label{fig:block_formats}
\end{figure}

Figure \ref{fig:block_formats} compares our $4$-bit LO-BCQ block format to MX \citep{rouhani2023microscaling}. As shown, both LO-BCQ and MX decompose a given operand tensor into block arrays and each block array into blocks. Similar to MX, we find that per-block quantization ($L_b < L_A$) leads to better accuracy due to increased flexibility. While MX achieves this through per-block $1$-bit micro-scales, we associate a dedicated codebook to each block through a per-block codebook selector. Further, MX quantizes the per-block array scale-factor to E8M0 format without per-tensor scaling. In contrast during LO-BCQ, we find that per-tensor scaling combined with quantization of per-block array scale-factor to E4M3 format results in superior inference accuracy across models. 



\end{document}


% This document was modified from the file originally made available by
% Pat Langley and Andrea Danyluk for ICML-2K. This version was created
% by Iain Murray in 2018, and modified by Alexandre Bouchard in
% 2019 and 2021 and by Csaba Szepesvari, Gang Niu and Sivan Sabato in 2022.
% Modified again in 2023 and 2024 by Sivan Sabato and Jonathan Scarlett.
% Previous contributors include Dan Roy, Lise Getoor and Tobias
% Scheffer, which was slightly modified from the 2010 version by
% Thorsten Joachims & Johannes Fuernkranz, slightly modified from the
% 2009 version by Kiri Wagstaff and Sam Roweis's 2008 version, which is
% slightly modified from Prasad Tadepalli's 2007 version which is a
% lightly changed version of the previous year's version by Andrew
% Moore, which was in turn edited from those of Kristian Kersting and
% Codrina Lauth. Alex Smola contributed to the algorithmic style files.
