\begin{table*}[ht]
    \centering
    \renewcommand{\arraystretch}{1.1} 
    \scalebox{0.9}{
    % \resizebox{16cm}{18cm}{
        \begin{tcolorbox}
                \textbf{The default version of PICA prompt with an example} 
                \tcblower
                \footnotesize
                % \textcolor{black}{\textbf{[System Message]}}\\
                \# \textbf{Image Complexity Assessment Task} \\
                
                    In this task, you will evaluate the informational richness of an image based on multiple dimensions. Each image may contain multiple layers of information, including various visual elements, contextual settings, emotional expressions, and cultural backgrounds. Please provide a detailed evaluation based on the following rating criteria. \\
                
                \#\# \textbf{Rating Criteria:} \\

                    The overall score will be based on all valid indicators. Images with particularly strong performance in certain dimensions (e.g., complex context or strong emotional expression) will receive higher scores, and weaker performance in other dimensions will not significantly lower the final score. It is recommended to focus on the image’s overall informational layers, element rarity, context building, and uniqueness rather than the absence of a single dimension. \\

                \textbf{1 Point}: Very low information, typically a single subject with a simple background or no clear context. The image lacks depth and multi-dimensional understanding, with few visual elements and no notable uniqueness. \\
                \textbf{2 Point}: Low information, with a few simple elements. It may have basic context or structure but lacks detail and uniqueness. The image is plain and suitable for basic understanding tasks, with weak emotional or contextual expression. \\
                \textbf{3 Point}: Moderate information, with a reasonable number of elements or layers. Context, emotion, or uniqueness are more noticeable, and the image provides moderate multi-dimensional understanding. It has some depth in visual elements and context but doesn’t meet all high-level requirements. \\
                \textbf{4 Point}: High information, with rich elements and clear context, showing uniqueness and diverse layers. The image is detailed, with strong emotional or contextual expression, but may lack some rare or background features. Suitable for in-depth analysis, particularly in context, emotion, culture, or history. \\
                \textbf{5 Point}: Very rich information, with diverse elements, complex scenes, unique perspectives, and abundant detail. The image features rare qualities and deep context, emotion, and cultural background. Ideal for advanced analysis with multi-dimensional depth. \\

                \textbf{\#\# Evaluation Dimensions} \\

                \#\#\# 1. \textbf{Details and Materiality}  \\
                \textbf{Detail Density}: Evaluate the number and complexity of details in the image. The denser the details, the greater the informational content.\\
                \textbf{Material and Texture}: Evaluate whether the image showcases multiple materials (e.g., wood, metal, fabric) and whether the textures are clearly visible. A variety of material layers adds to the visual richness.\\
                \textbf{Detail Layers and Spatial Perception}: Evaluate whether the image presents multiple layers of details, such as clear details in the foreground and a blurred background, creating a sense of depth. \\

                \#\#\# 2. \textbf{Context and Narrative} \\
                \textbf{Context Constructio}: Evaluate whether the image constructs a clear context or scene that conveys rich information through visual elements (e.g., a family gathering, office work, festive events). \\
                \textbf{Narrative Complexity}: Evaluate whether the image depicts one or more complex events or actions, increasing the narrative depth and informational content. \\
                \textbf{Implied Background Story}: Evaluate whether the image implies a potential story or context through visual elements, such as through the environment, actions of people, or objects that hint at social, historical, or cultural contexts. \\

                \#\#\# 3. \textbf{Emotion and Atmosphere} \\
                \textbf{Emotional Expression}: Evaluate whether the image conveys a specific emotion or atmosphere through elements like lighting, color tones, and composition. Emotional expression increases the image's complexity. \\
                \textbf{Emotional Layering}: Evaluate whether the image expresses multiple emotional layers or emotional shifts through different elements. Images with rich emotional layers often have greater depth. \\


                \#\#\# 4. \textbf{Cultural and Historical Context} \\
                \textbf{Cultural Characteristics}: Evaluate whether the image contains elements that reflect a specific cultural, historical, or social context (e.g., distinctive architectural styles, clothing, festivals). \\
                \textbf{Historical Background}: Evaluate whether the image reflects historical events, periods, or characteristics. Historical elements enhance the image’s depth. \\

                \#\#\# 5. \textbf{Camera Angle and Composition} \\
                \textbf{Unique Perspective}: Evaluate whether the image features a unique or creative angle, using uncommon perspectives.\\
                \textbf{Composition Complexity}: Evaluate whether the image’s composition is complex and varied, using techniques like contrast, symmetry, and spatial distribution to enhance the image’s depth. Complex compositions effectively convey more information. \\


                \#\#\# 6. \textbf{Dynamics and Interaction} \\
                \textbf{Dynamic Elements}: Evaluate whether the image includes dynamic elements (e.g., movement of people or objects, hints of time progression). \\
                \textbf{Interactivity}: Evaluate whether  the image depict interaction between elements (e.g., people talking, animals hunting)? Interaction enhances the appeal and complexity of the image. \\

        \end{tcolorbox}
    }
    % \vspace{-0.2cm}
    \caption{The default version of PICA prompt with an example}
    \label{tab:full_prompt}
    % \vspace{-0.4cm}
\end{table*}