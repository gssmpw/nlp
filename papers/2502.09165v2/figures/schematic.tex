% chktex-file 1  % nodes without options are whitespace terminated.
                 % we could alternatively define \node as a safe command
% chktex-file 11 % foreach syntax uses the dots to describe sequences
% chktex-file 26 % singleton semicolon trigger this
% chktex-file 36 % tikz syntax needs brackets with non-whitespace char in front

% Whether to show residual blocks:
\providecommand\ifresiduals\iffalse

% Configure height, width, and separation between colorful blobs:
\setlength\myH{5em}%
\setlength\myW{2ex}%
\setlength\myS{1ex}%

% Low-rank factorization:
\newcommand\nodeVDV[3]{%
	\node (#1) {#2};
	\draw[fill, color=#3!50]
		(#1.center)
		++(-1.5\myW-\myS-0.5\myH,-0.5\myH) % center node in white space
		rectangle ++(\myW,\myH)
		++(\myS,-\myW)
		rectangle ++(\myW,\myW)
		++(\myS,-\myW)
		rectangle ++(\myH,\myW)
		;
	\coordinate (#1 north) at (current bounding box.north);
	\coordinate (#1 south) at (current bounding box.south);
}%

% Residual factor:
\newcommand\nodeR[3][1]{%
	\ifresiduals
	\node (#2) {#3};
	\draw[fill, gray!50]
		(#2.west) ++(-\myS,-0.5\myH)
		foreach \distance in {1,...,#1} {
			rectangle ++(-\myW,\myH)
			++(-\myS,-\myH)
		};
	\else
	\coordinate[yshift=.5ex] (#2);
	\fi
}%

\begin{tikzpicture}[
	thick,
	X/.style = {
		fill = gray!20,
		minimum size = \myH,
	},
	heading/.style = {
		above = \myS,
		inner sep = 0pt,
	},
	every node/.style = {
		anchor = base,
	},
	every outer matrix/.style = {
		inner sep = 0pt,
	},
]
	\matrix (foo) [
		row sep = 2\myS,
		column sep = \myS,
	] {
		% Row 1:
		\node[X] (X1) {$X_1$}; &
		\node {=}; &
		\nodeVDV{V11}{$\hphantom{\tau_1} Z_1 D_1 Z_1^\TT$}{mycolor1}; &&&&& [2\myS]
		\nodeR{R1}{$R_1$}; & [4\myS]
		\coordinate (R1+); \\

		% Row 2:
		\node[X] (X2) {$X_2$}; &
		\node {=}; &
		\nodeVDV{V21}{$\hphantom{\tau_1} Z_1 D_1 Z_1^\TT$}{mycolor1}; &
		\node {+}; &
		\nodeVDV{V22}{$\hphantom{\tau_2} V_2 \tilde D_2 V_2^\TT$}{mycolor2}; &&&
		\nodeR{R2}{$R_2$}; &
		\coordinate (R2+); & [4\myS]
		\node[draw, fill=gray!20] (RRE) {RRE}; \\

		% Row 3:
		\node[X] (X3) {$X_3$}; &
		\node {=}; &
		\nodeVDV{V31}{$\hphantom{\tau_1} Z_1 D_1 Z_1^\TT$}{mycolor1}; &
		\node {+}; &
		\nodeVDV{V32}{$\hphantom{\tau_2} V_2 \tilde D_2 V_2^\TT$}{mycolor2}; &
		\node {+}; &
		\nodeVDV{V33}{$\hphantom{\tau_3} V_3 \tilde D_3 V_3^\TT$}{mycolor3}; &
		\nodeR{R3}{$R_3$}; &
		\coordinate (R3+); \\

		% Row 4:
		\node[X, label={[heading] Extrapolant}] {$\widehat X_{\ifresiduals 3\else 2\fi}$}; &
		\node{=}; &
		\nodeVDV{V1}{$\tau_1 Z_1 D_1 Z_1^\TT$}{mycolor1}; &
		\node {+}; &
		\nodeVDV{V2}{$\tau_2 V_2 \tilde D_2 V_2^\TT$}{mycolor2}; &
		\ifresiduals
		\node {+}; &
		\nodeVDV{V3}{$\tau_3 V_3 \tilde D_3 V_3^\TT$}{mycolor3}; &
		\else
		&&
		\fi
		\coordinate[yshift=.5ex] (RE); \\
	};

	% Draw column headers:
	\node[heading] at (V11 north -| X1) {Iterate};
	\node[heading] at (V11 north -| V2) {Low-rank factors};
	\ifresiduals
	\node[heading] at (V11 north -| R1) {Residual factors};
	\fi

	% Draw dashed lines in between rows:
	\foreach \row in {1, 2, 3} {
		\draw[
			dashed,
			transform canvas = { yshift = -\myS },
			shorten > = -\myS,
		] (X\row.west |- V\row1 south) ++(-\myS,0) -- (V\row1 south -| R\row.east);
	};

	% Draw RRE apparatus:
	\coordinate (junction) at (R1 -| R1+); % baseline adjustment
	\draw[->] (R1) -| (RRE);
	\draw (R2) -- (R2 -| junction);
	\draw (R3) -| (junction);
	\draw[->] (RRE) |- (RE);
	\node[anchor=south east] at (RE -| RRE) {$\tau_1, \tau_2\ifresiduals, \tau_3\fi$};
\end{tikzpicture}%
