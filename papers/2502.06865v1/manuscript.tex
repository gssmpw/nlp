\documentclass{article}

\usepackage{arxiv}

\usepackage[utf8]{inputenc} % allow utf-8 input
\usepackage[T1]{fontenc}    % use 8-bit T1 fonts
\usepackage{hyperref}       % hyperlinks
\usepackage{url}            % simple URL typesetting
\usepackage{booktabs}       % professional-quality tables
\usepackage{amsfonts}       % blackboard math symbols
\usepackage{nicefrac}       % compact symbols for 1/2, etc.
\usepackage{microtype}      % microtypography
\usepackage{lipsum}
\usepackage{graphicx}
\graphicspath{ {./images/} }
\usepackage{amssymb}
%% The amsthm package provides extended theorem environments
\usepackage{amsthm}

%% The lineno packages adds line numbers. Start line numbering with
%% \begin{linenumbers}, end it with \end{linenumbers}. Or switch it on
%% for the whole article with \linenumbers.
\usepackage{lineno}
\usepackage{subfigure}
\usepackage{xcolor}
\usepackage{amsmath}

\usepackage{fancyhdr}
\fancyhead{TEXT}
\usepackage{lastpage}
\pagestyle{fancy}
\lhead{} \chead{Distribution Statement A: Approved for public release; distribution is unlimited} \rhead{\scriptsize \thepage}
\lfoot{}\cfoot{}\rfoot{}

%% Definitions
\newcommand{\vect}[1]{\mathbf{#1}}
\newcommand{\tensor}[1]{\mathbf{#1}}
\newcommand{\field}[1]{\mathbb{#1}}
\newcommand{\R}{\field{R}}
\newcommand{\coor}{\xi}
\newcommand{\body}{{\cal B}}
\newcommand{\eltensor}{\mathbb{C}}


% Uncomment and use as if needed
\newtheorem{theorem}{Theorem}
\newtheorem{remark}{Remark}[section]
\newtheorem{lemma}{Lemma}[section]
\newtheorem{assumption}{Assumption}
\newtheorem{corollary}{Corollary}[section]
\newtheorem{proposition}{Proposition}[section]


\title{Deep Ritz method with Fourier feature mapping: A deep learning approach for solving variational models of microstructure}


\author{
	Ensela Mema \\
	Kean University\\
	Union, NJ 07083 \\
	\texttt{emema@kean.edu} \\
	%% examples of more authors
	\And
	Ting Wang\\
	Booz Allen Hamilton Inc.\\
	McLean, VA 22102\\
	\texttt{wang\_ting@bah.edu} \\
	\And
	Jaroslaw Knap \\
	DEVCOM Army Research Laboratory\\
	Aberdeen Proving Ground, MD 21005 \\
	\texttt{jaroslaw.knap.civ@army.mil} \\
}
\begin{document}
	\maketitle
	\begin{abstract}
		This paper presents a novel approach that combines the Deep Ritz Method (DRM) with Fourier feature mapping to solve minimization problems comprised of multi-well, non-convex energy potentials. These problems present computational challenges as they lack a global minimum.  Through an investigation of three benchmark problems in both 1D and 2D, we observe that DRM suffers from spectral bias pathology, limiting its ability to learn solutions with high frequencies. To overcome this limitation, we modify the method by introducing Fourier feature mapping. This modification involves applying a Fourier mapping to the input layer before it passes through the hidden and output layers. Our results demonstrate that Fourier feature mapping enables DRM to generate high-frequency, multiscale solutions for the benchmark problems in both 1D and 2D, offering a promising advancement in tackling complex non-convex energy minimization problems.
	\end{abstract}
	
	
	% keywords can be removed
	\keywords{Deep learning \and Variational problems \and Nonconvex energy minimization \and Fourier feature mapping \and Martensitic phase transformation}
	
	
\section{Introduction}


Materials undergoing martensitic phase transformations constitute a
technologically important class of
materials~\cite{bhattacharya2003microstructure}. These materials
include steels, shape-memory alloys, solidified gases and polymers, to
name a few. A feature common to all of these materials is
microstructure in the form of elaborate three-dimensional patterns at
the scale ranging from nanometers to centimeters. Mathematically,
microstructure induced by martensitic phase transformations is
characterized as minimizers of a total energy functional.  The
fundamental difficulty in seeking such minimizers lies, however, in
non-convexity of the total energy
functional~\cite{bhattacharya2003microstructure,dacorogna2007direct}.

Numerical treatment of non-convex minimization problems is fraught
with challenges. Standard finite elements usually require very fine
meshes to resolve meaningful scales associated with microstructure. In
addition, specially crafted meshes are frequently needed as finite
element solutions tend to be strongly mesh dependent and adaptive mesh
refinement may not always perform
satisfactorily~\cite{luskin1996computation,Carstensen_2005}. The
strong mesh dependence of solutions may be somewhat alleviated by
recourse to specialized finite-element techniques, such as
discontinuous finite
elements~\cite{gobbert1999discontinuous}. Alternatively, the
non-convex energy functional can be regularized through
convexification~\cite{carstensen2001numerical}. Solutions of
convexified minimization problems can be then efficiently carried out
by standard finite elements~\cite{bartels2004effective}. In practice,
however, convexified energy functionals may not be readily available
explicitly and their numerical approximations are generally costly to
obtain~\cite{carstensen1997numerical}. While minimizers of the
convexified energy functional are much easier to get, they may miss
some important physical features of the original (non-convex) minimization
problem. Finally, one may employ Young measures to turn the non-convex
minimization problem into a convex minimization
problem~\cite{nicolaides1993computation,aranda2001numerical}. This
approach offers numerous benefits, chiefly among them that the energy
functional does not need to be altered. Yet, additional numerical
algorithms are required, increasing considerably the overall
computational
cost~\cite{carstensen2000numerical,bartels2004effective}.

Recent advancements in deep neural networks (DNNs) have raised hopes
that DNNs may be capable of generating solutions to non-convex
minimization problems. Specifically, the universal approximation
theory~\cite{hornik1990universal, hornik1991approximation} has enabled
DNN-based numerical methods for PDEs to parameterize the solution
using a DNN and learn it using the method of stochastic gradient
descent. The approach learns the solution by minimizing a loss
function induced by the physics constraints, often referred to as the
physics informed approach.  Depending on how the loss function is
constructed, DNN-based methods can be roughly classified into three
categories: 1) the physics informed neural network
(PINN)~\cite{raissi2019physics, sirignano2018dgm}; 2) deep Ritz
methods (DRM)~\cite{yu2018deep} and 3) deep backward stochastic
differential equation (BSDE)~\cite{han2017deep}.  PINN minimizes the
residual of the PDE evaluated at a set of randomly sampled collocation
points. In comparison, DRM utilizes the variational structure of
elliptic PDEs to minimize the energy functional.  Finally, deep BSDE
explores the probabilistic connection between parabolic PDE and BSDE
in order to reformulate the problem as a reinforcement learning task.
The key advantage of the DNN-based methods over the conventional ones
lies in the fact that they replace the deterministic mesh by Monte
Carlo sampling and hence, in principle, lead to dimension independent
convergence rates~\cite{grohs2018proof}. Despite being a promising
direction, training of DNN-based methods can be extremely challenging
due to, e.g., the choice of the learning rate, the multi-scale nature
of the problem under consideration, etc. Indeed, it has been widely
observed that DNNs are biased to learn low frequency features of the
solution, making them fail to learn solutions that exhibit
high-frequency and multi-scale, an essential feature in non-convex
minimization in the context of microstructure evolution. This
phenomenon is known as the spectral bias pathology for deep
learning~\cite{rahaman2019spectral, wang2021eigenvector}.

In this work, we focus on the following minimization problem:
\begin{align}
	\min_{u\in \mathcal{U}} \ I(u) \qquad {\rm where } \qquad I(u) = \int_{D} W({\bf x},u({\bf x}), \nabla{u}({\bf x})) \ d{\bf x},\label{eqn:variationalproblem}
\end{align}
% $W$ is the energy density function and $\mathcal{U}$ is the set of admissible functions $u({\bf x})$, also referred to as the trial function.
where $D \subset \mathbb{R}^d$ is a bounded open set with a Lipschitz
boundary $\partial D$,
$W:\mathbb{R}^d \times \mathbb{R}^N \times \mathbb{R}^{dN} \to
\mathbb{R}$ is the Lagrangian and $u : \bar{D} \to
\mathbb{R}^N$. $\bar{D}$ denotes the closure of $D$.  Here,
$\mathcal{U}$ is a space of admissible functions, e.g., the Sobolev
space $H_0^1(D)$ when the zero boundary condition is imposed. The
energy density $W$ is generally assumed to be non-convex in
$\nabla{u}$.  To solve the above minimization, one seeks minimizers
$u({\bf x})$ of the functional $I(u)$ over the prescribed domain $D$,
subject to boundary condition constraints (set to $u({\bf x}) = 0$ on
$\partial D$). The reader is referred to any standard texts on
variational calculus, for example ~\cite{dacorogna2007direct}, for the
properties of the minimization problem~(\ref{eqn:variationalproblem}).

Since DRM works by minimizing an energy functional, it is natural to
seek solutions of the minimization
problem~\eqref{eqn:variationalproblem} by means of DRM.  A
straightforward application of DRM to non-convex minimization problems
in 1D and 2D has been carried out by Chen et
al. in~\cite{Chen&Rosakis2023}. They demonstrate that DRM is capable
of capturing the complexities of local or global minimizers of
non-convex variational problems, if one applies an ad hoc activation
function.  Additionally, they suggest that the depth of the DNN plays
a role analogous to the mesh size in FEM so one can capture
high-frequency solutions (with more twin bands) if one increases the
depth of DNN. It is important to note that although DRM is capable of
solving non-convex minimization problems, a naive application of the
method fails to consistently generate high-frequency solutions due to
the fact that DNN algorithms, including DRM, are biased to learn the
low frequency features of the solutions.

In our work, we address the shortcomings of DRM by applying Fourier
feature mapping as outlined in~\cite{FourierFeatures2020} and show
that DRM in conjunction with Fourier feature mapping (DRM\&FM) can
consistently generate high-frequency multiscale solutions for
non-convex minimization problems independently of the depth of the
DNN. The main contributions of our work can be summarized as follows:
\begin{itemize}
	\item We apply neural tangent kernel (NTK) theory to show that, similar to PINN, DRM also
	suffers from spectral bias pathology. That is, the learning rates
	along different directions are determined by the corresponding
	eigenvalues of the NTK. To alleviate this issue, we utilize the
	Fourier feature mapping to map the input into an appropriate
	submanifold. Based on the recent theoretical results on
	NTK~\cite{geifman2020similarity, chen2020deep}, we show (at least in the $1$D
	case) that the Fourier feature mapping leads to a quadratic decay
	NTK eigenspectrum which could be advantageous when multiscale
	problems are considered.
	
	\item We numerically illustrate that DRM alone cannot consistently
	generate high-frequency solutions to non-convex minimization
	problems by increasing the depth of DNN. See
	Section~\ref{sec:numerics} for the benchmark problems considered in
	this work and how they differ from the ones considered
	in~\cite{Chen&Rosakis2023}.
	
	\item We apply Fourier feature mapping on DRM and observe that DRM in
	conjunction with Fourier features (DRM\&FM) allow the DNN to learn
	high-frequency solutions to non-convex variational problems
	independently of the depth of the NN.
\end{itemize}
The paper is organized as follows: Section \ref{sec:DRM} outlines how
the DRM can be applied to solve variational problems.  Section
\ref{sec:NTK} uses NTK theory to show that DRM alone suffers from
spectral bias pathology and how Fourier feature mapping enables the
DRM to learn solutions whose NTK has a fast decaying
eigenspectrum. Section~\ref{sec:Numerics} presents our numerical
results in $1D$ and $2D$ and Section \ref{sec:Concl} discusses our
conclusions.

%%%%%%%%%%%%%%%%%%%%%%%%%%%%%%%%%%%%%%%%%%%%%%%%%%%%%%%%%%%%%%%%%%%%%%%%%%%%%%%%%%%%%%%%%%%%%
%%%%%%%%%%%%%%%%%%%%%%%%%%%%%%%%%%%%%%%%%%%%%%%%%%%%%%%%%%%%%%%%%%%%%%%%%%%%%%%%%%%%%%%%%%%%%%
\section{Deep Ritz Algorithm}
\label{sec:DRM}


\begin{figure}[!ht]
	\centering
	\includegraphics[width=0.75\textwidth]{Figure_1.png}
	\caption{Structure of Neural Network in Deep Ritz Method.}
	\label{fig:NN_structure}
\end{figure}

DRM solves the variational problem in~\eqref{eqn:variationalproblem}
by using DNN to construct an approximation $\hat{u}({\bf x})$ that
minimizes the functional $I(\hat{u})$ over the prescribed domain $D$.
More specifically, a DNN of depth $n$ approximates the solution
through a series of transformations by
\begin{align}
	\hat{u}({\bf x};\theta) = L^{[n]}\circ L^{[n-1]}\circ \cdots \circ L^{[1]}({\bf x}), \label{eqn:NNsolU}
\end{align}
with
\begin{equation*}
	\begin{split}   
		L^{[1]}({\bf x}) &= \sigma(A^{[1]} {\bf x} + b^{[1]})\\
		L^{[i]}({\bf x}) &= \sigma(A^{[i]} L^{[i-1]}({\bf x}) + b^{[i]}), \qquad i = 2, \ldots, n-1\\
		L^{[n]}({\bf x}) &= A^{[n]} L^{[n-1]}({\bf x}) + b^{[n]}
	\end{split}
\end{equation*}
where $A^{[i]}$ and $b^{[i]}$ are the weight matrix and the bias vector of layer $i$, respectively, and $\sigma$ is a nonlinear activation function (see Figure~\ref{fig:NN_structure} for sketch).
% The network structure is a feed forward NN that consists a series of transformations of the form: $L({\bf x}) = A {\bf x} + {\bf b}$ followed by a nonlinear activation function $\sigma({\bf x}):\mathbb{R}^N\rightarrow \mathbb{R}^N$ 
% (see Figure~\ref{fig:NN_structure} for sketch) 
% where ${\bf x} \in \mathbb{R}^d$, $A \in \mathbb{R}^{N\times d}$ and ${\bf b} \in \mathbb{R}^{N}$. The second component consists of a nonlinear activation function $\sigma({\bf x}):\mathbb{R}^N\rightarrow \mathbb{R}^N$ which is applied throughout each hidden layer of the network. The output of the $i$-th hidden layer can be expressed as: 
% 	\begin{align}
	% 		L^{[i]}({\bf x}) = \sigma(A^{[i-1]} \cdot L^{[i-1]}({\bf x})+b^{[i-1]}),
	% 	\end{align} 
% 	where $A^{[i-1]} \in \mathbb{R}^{N\times d}$ and $b^{[i-1]} \in \mathbb{R}^{N}$ denote the weight matrix and bias vector in layer $i-1$, respectively. 
% The DNN approximation of the variational problem can be written as  composition of the outputs for each layer:
% 	\begin{align}
	% 		\hat{u}({\bf x};\theta) = L^{[n]}\circ L^{[n-1]}\circ \cdots \circ L^{[1]}\circ L^{[0]}({\bf x}) \label{eqn:NNsolU}
	% 	\end{align}
% 	where $\theta \in \mathbb{R}^{N_\theta}$ denotes the full set of all weights and bias parameters in a neural network of $(n+1)$ layers (or $n$ hidden layers). 
Substituting~\eqref{eqn:NNsolU} in the variational problem~\eqref{eqn:variationalproblem} leads to the following finite dimensional optimization problem: 
\begin{align}
	\min_{ \theta \in \mathbb{R}^{N_\theta}} I(\hat{u})\qquad {\rm where} \qquad I(\hat{u}) = \int_{D} W({\bf x},\hat{u}({\bf x};\theta), \nabla \hat{u}({\bf x};\theta) ) d{\bf x},
	\label{eqn:NNoptproblem}
\end{align}
where $\theta = (A^{[1]}, b^{[1]}, \ldots, A^{[n]}, b^{[n]})$ are parameters of the DNN.
To account for the boundary condition, we follow E et al. in ~\cite{Weinan} and Chen et al. in~\cite{Chen&Rosakis2023} in using penalty approach to numerically enforce the prescribed boundary conditions of the variational problem, which leads to a modified functional: 
\begin{align}
	I(\theta) = \int_{D} W({\bf x},\hat{u}({\bf x};\theta), \nabla \hat{u}({\bf x};\theta)) d{\bf x} + \lambda \int_{\partial D} \hat{u}({\bf x};\theta)^2 ds,
\end{align}
where, with a slight abuse of notation, we have rewritten $I(\theta) \triangleq I(\hat{u}, \nabla \hat{u})$ to indicate that the optimization is with respect to the NN parameters $\theta$.
Note that $\lambda $ serves as a penalty term that increases the value of $I$ if the approximated DNN solution, $\hat{u}({\bf x};\theta)$ deviates from the prescribed values at the boundary. 
To solve the optimization problem by stochastic gradient descent (SGD), it is often convenient to rewrite the above integral in its probabilistic form as 
% {\color{blue} TW: this is not correct. One never approximates this sum but instead the gradient of it.
	% }
\begin{equation}\label{eqn:prob-form}
	\begin{split}
		\min_{\theta \in \mathbb{R}^{N_{\theta}}} I(\theta) := \mathbb{E} \left[ W({\bf x}, \hat{u}({\bf x}; \theta), \nabla \hat{u}({\bf x}; \theta))\right] + \lambda \mathbb{E}_b \left[|\hat{u}({\bf x_b};\theta)|^2\right],
	\end{split}
\end{equation}
where $\mathbb{E}$ and $\mathbb{E}_b$ are taken with respect to the uniform distributions
over $D$ and $\partial D$, respectively.
At each gradient descent iteration, we use Adam optimizer~\cite{kingma2014adam} to update the DNN parameters $\theta$ by evaluating the stochastic gradient of $I$ at a mini-batch of samples over $D$ and $\partial D$. 


%%%%%%%%%%%%%%%%%%%%%%%%%%%%%%%%%%%%%%%%%%%%%%%%%%%%%%%%%%%%%%%%%%%%
%%%%%%%%%%%%%%%%%%%%%%%%%%%%%%%%%%%%%%%%%%%%%%%%%%%%%%%%%%%%%%%%%%%%
\section{NTK analysis for Deep-Ritz and the Fourier feature}\label{sec:NTK}

\subsection{The spectral bias pathology for DRM}
In practice, a naive application of DRM often fails to achieve desirable results.   
In this section, we derive the NTK theory for DRM and show that, similar to PINN, DRM also suffers the pathology of spectral bias of neural networks~\cite{jacot2018neural, rahaman2019spectral, du2018gradient} and hence additional tricks and treats have to be applied. 

For ease of presentation, we assume $d = N = 1$ to keep notation uncluttered.
However, we emphasize that the result presented below can be readily generalized to the vectorial setting.
We start by considering the empirical approximation to~\eqref{eqn:prob-form} without the penalty term, i.e., 
\begin{equation}\label{eqn:ensemble-loss}
	\begin{split}
		\min_{\theta \in \mathbb{R}^{N_{\theta}}} I_{\mathcal{X}}(\theta) := \frac{1}{|\mathcal{X}|} \sum_{{\bf x_n} \in \mathcal{X}} W({\bf x_n}, \hat{u}({\bf x_n}; \theta),  \hat{u}^{\prime}({\bf x_n}; \theta)),
	\end{split}
\end{equation}
where $\mathcal{X}$ is the set of collocation points sampled uniformly over $D$, and $|\mathcal{X}|$ denotes the cardinality of the set.
Applying the gradient descent algorithm to $I_{\mathcal{X}}(\theta)$ leads to the discrete time dynamics 
\[
{\theta}_{n+1} = \theta_n - \eta h \nabla_{\theta} I_{\mathcal{X}}(\theta_n), \qquad n = 1, 2, \ldots,
\]
where $\eta>0$ is the learning rate and $h>0$ is a scaling constant. 
%~\eqref{eqn:variationalproblem} with $W$ dependent on $x, u$ and $\nabla u$, i.e., 
% \begin{equation}\label{eqn:variational-problem}
	%     \min_{u \in \mathcal{U}} I(u) := \int_D W(x, u, \nabla u)\, dx, 
	% \end{equation}
% where $D \subset \mathbb{R}^d$ is a bounded open set with a Lipschitz
% boundary $\partial D$, $W:\mathbb{R}^d \times \mathbb{R}^N \times \mathbb{R}^{dN} \to \mathbb{R}$ is the Lagrangian and $u : \bar{D} \to \mathbb{R}^N$ is the solution of interest.
% Here $\mathcal{U}$ is a space of admissible functions, e.g., the Soblev space $H_0^1(D)$ when the zero boundary condition is imposed. 
% Let $u_{\theta}:  \bar{D} \to \mathbb{R}^N$ with $\theta \in \mathbb{R}^{N_{\theta}}$ be a neural network function that approximates the minimizer of~\eqref{eqn:variational-problem}. We aim to minimize the variational loss 
% \begin{equation}\label{eqn:loss}
	%     \min_{\theta \in \mathbb{R}^{N_{\theta}}}I(u(\theta)) 
	%     :=
	%     \int_D W(x, u(\theta), \nabla u(\theta))\, dx.
	% \end{equation}
% In practice, the above integral must be approximated by the following empirical variational loss
% \begin{equation}\label{eqn:loss_practice}
	%     \min_{\theta \in \mathbb{R}^{N_{\theta}}} I_{\mathcal{X}}(\theta) := \frac{1}{|\mathcal{X}|}\sum_{x_n\in \mathcal{X}} W(x_n, u(x_n; \theta), \nabla u(x_n; \theta)),
	% \end{equation}
% where $\mathcal{X}$ is a set of collocation points sampled uniformly from the domain $D$.
Upon taking $h \to 0^+$, we obtain the continuous time dynamics governing the evolution of the parameters $\theta$,
\begin{equation}\label{eqn:theta-trajectory}
	\frac{d\theta(t)}{dt} = -\eta \nabla_{\theta}I_{\mathcal{X}}(\theta(t)),
\end{equation}
where $\theta: [0, \infty) \to \mathbb{R}^{1 \times N_{\theta}}$ is a function of $t$ and $\nabla_{\theta}I_{\mathcal{X}}(\theta(t)) \in \mathbb{R}^{1 \times N_{\theta}}$ is the gradient of $I_{\mathcal{X}}(\theta)$ with respect to $\theta$. 
% \begin{remark}[Gradient flow and continuity equation.]
	% We interpret the above equation from an optimal transport perspective. 
	% Note that~\eqref{eqn:theta-trajectory} suggests that $\theta(t)$ is the gradient flow associated with the vector field $-\eta \nabla_{\theta} I$.
	% Denote $T_t(\theta) = \theta(t)$ the solution to~\eqref{eqn:theta-trajectory} with the initial condition $\theta(0) = \theta$.
	% Suppose $\theta$ is initialized according to a measure $\rho_0$, then $\theta_t$ follows a distribution $\rho_t = T_t(\theta)_{\#} \rho_0$, where $T_t(\theta)_{\#}\rho_0$ is the push-forward of the measure $\rho_0$, i.e., 
	% \[
	% T_t(\theta)_{\#}\rho_0 (A) := \rho_0(T_t(\theta)^{-1}(A)), \qquad \text{for all}~A~\text{Borel in}~\mathbb{R}^{N_{\theta}}.
	% \]
	% Under mild conditions, $\rho_t$ satisfies the continuity equation (in a weak sense)
	% \[
	% \partial_t \rho_t(\theta) - \eta \nabla \cdot (\rho_t(\theta) \nabla_{\theta}I_{\mathcal{X}}(\theta)) = 0.
	% \]
	% \end{remark}
%In comparison to the NTK analysis for PINN which minimizes the mean square error loss~\cite{wang2021eigenvector}, the variational loss of DRM makes it impossible to obtain a closed form formula for the training dynamics for outputs $\hat{u}$. To circumvent this issue, we instead focus on analyzing the training dynamics for the loss $I_{\mathcal{X}}$.
We first derive the empirical evolution of the loss function $I_{\mathcal{X}}(\theta(t))$ with respect to $t$, i.e.,  
\begin{equation}\label{eqn:NTK-1}
	\frac{d I_{\mathcal{X}}(\theta(t))}{dt} = \left\langle \nabla_{\theta} I_{\mathcal{X}}(\theta(t)), \frac{d\theta(t)}{dt} \right\rangle = -\eta \|\nabla_{\theta} I_{\mathcal{X}}(\theta(t))\|^2.
\end{equation}
By chain rule we have 
\[
\nabla_{\theta} I_{\mathcal{X}}(\theta) = 
\frac{1}{|\mathcal{X}|} \sum_{{\bf x_n} \in \mathcal{X}} \partial_{\hat{u}} W_n(\theta) \nabla_{\theta} \hat{u}_n(\theta) 
+  \partial_{\hat{u}^{\prime}} W_n(\theta) \nabla_{\theta} \hat{u}^{\prime}_n( \theta), 
\] 
where we have denoted $\hat{u}_n(\theta) = \hat{u}({\bf x_n}; \theta)$, $\hat{u}^{\prime}_n( \theta) = \hat{u}^{\prime}({\bf x_n} ;\theta)$ and 
$W_n(\theta) = W({\bf x_n}, \hat{u}({\bf x_n}; \theta), \hat{u}^{\prime}({\bf x_n}; \theta))$ so that 
\[
\nabla_{\theta} \hat{u}_n \in \mathbb{R}^{1 \times N_{\theta}}, \qquad \partial_{\hat{u}} W_n \in \mathbb{R}, \qquad
\nabla_{\theta} \hat{u}_n^{\prime} \in \mathbb{R}^{1 \times N_{\theta}}, \qquad
\partial_{\hat{u}^{\prime}} W_n \in \mathbb{R}.
\]
Denote 
$
U_n(\theta) = [\hat{u}_n(\theta),  \hat{u}_n^{\prime}(\theta)]^{\top}
\in 
\mathbb{R}^{2 \times 1}
$
so that 
\[
\nabla_{\theta} U_n = [\nabla_{\theta}\hat{u}_n,  \nabla_{\theta}\hat{u}_n^{\prime} ]^{\top} \in \mathbb{R}^{2 \times N_{\theta}},
\]
\[
\nabla_U W_n = [\partial_{\hat{u}} W_n, \partial_{\hat{u}^{\prime}} W_n]^{\top}
\in 
\mathbb{R}^{2 \times 1}
\]
and hence
\[
\nabla_{\theta} I_{\mathcal{X}}(\theta) = \frac{1}{|\mathcal{X}|} \sum_{{\bf x_n} \in \mathcal{X}}  [\nabla_U W_n(\theta)]^{\top} \nabla_{\theta} U_n(\theta) \in \mathbb{R}^{1 \times N_{\theta}}.
\]
Then we can further rewrite the evolution equation given by~\eqref{eqn:NTK-1} in the following compact form,
\begin{equation}\label{eqn:loss-trajectory}
	\begin{split}
		\frac{d I_{\mathcal{X}}(\theta(t))}{dt} 
		= -\frac{\eta}{|\mathcal{X}|^2} \sum_{{\bf x_m}, {\bf x_n} \in \mathcal{X}} [\nabla_U W_m(\theta(t))]^{\top} 
		\left\{\nabla_{\theta} U_m(\theta(t)) [\nabla_{\theta} U_n(\theta(t))]^{\top} \right\}
		\nabla_U W_n(\theta(t)).
	\end{split}
\end{equation}
We call the operator/matrix valued function $K: D  \times D  \to \mathbb{R}^{2 \times 2}$ defined by
\[
K({\bf x_m}, {\bf x_n}; \theta) \triangleq \nabla_{\theta} U_m(\theta) [\nabla_{\theta} U_n(\theta)]^{\top},\qquad {\bf x_m}, {\bf x_n} \in D,
\]
the NTK (parameterized at $\theta$) associated to DRM. 
It should be emphasized that, similar to PINN, the NTK kernel $K$ of DRM depends on both the output $\hat{u}$ and its spatial derivative $ \hat{u}^{\prime}$.

The lazy training phenomenon suggests that, when trained with gradient-based optimizers, strongly overparameterized NNs could converge
exponentially fast to the minimum training loss without significantly varying the parameters~\cite{chizat2019lazy}, i.e., $\theta(t) \approx \theta_0$. 
Therefore, to analyze the asymptotic behavior of the differential equation~\eqref{eqn:loss-trajectory}, we linearize the DNN solution $\hat{u}({\bf x}; \theta)$ at its initial value $\theta_0$ via
\[
\hat{u}({\bf x}; \theta) \approx \bar{u}({\bf x}; \theta) \triangleq \hat{u}({\bf x}; \theta_0) + \langle \nabla_{\theta} \hat{u}({\bf x}; \theta_0), \theta - \theta_0 \rangle,
\]
where by definition $\bar{u}({\bf x}; \theta)$ is the linearization of $\hat{u}({\bf x}; \theta)$ at $\theta_0$.
Notice that 
\[
\nabla_{\theta} [\bar{u}({\bf x}; \theta),  \bar{u}^{\prime}({\bf x}; \theta)]^{\top} = 
\nabla_{\theta}[ \hat{u}({\bf x}; \theta_0),  \hat{u}^{\prime}({\bf x}; \theta_0)]^{\top} 
= \nabla_{\theta} U({\bf x}; \theta_0).
\]
Substituting $\hat{u}({\bf x}; \theta)$ by the linearized model $\bar{u}({\bf x}; \theta)$ into~\eqref{eqn:loss-trajectory} and applying the lazy training assumption to the NTK
leads to the linearized loss dynamics
\begin{equation}\label{eqn:loss-trajectory-linearize}
	\begin{split}
		\frac{d \bar{I}_{\mathcal{X}}(\theta(t))}{dt} 
		= -\frac{\eta}{|\mathcal{X}|^2} \sum_{{\bf x_m}, {\bf x_n} \in \mathcal{X}}
		[\nabla_U \bar{W}_m(\theta(t))]^{\top}
		K({\bf x_m}, {\bf x_n}; \theta_0)
		[\nabla_U \bar{W}_n(\theta(t))], 
	\end{split}
\end{equation}
where 
\[
\bar{W}_n(\theta) = W({\bf x_n}, \bar{u}({\bf x_n}; \theta), \bar{u}^{\prime}({\bf x_n}; \theta))
\]
and
\[
\bar{I}_{\mathcal{X}}(\theta) = \frac{1}{|\mathcal{X}|} \sum_{{\bf x_n} \in \mathcal{X}} W({\bf x_n}, \bar{u}({\bf x_n}; \theta),  \bar{u}^{\prime}({\bf x_n}; \theta))
\]
are the linearization of the Lagrangian $W$ and the empirical loss~\eqref{eqn:ensemble-loss} at $\theta_0$, respectively, 
and 
$
K({\bf x_m}, {\bf x_n}; \theta_0)
$
is the NTK parameterized at the initial guess $\theta_0$.
It has been shown that when the minimum width of the DNN is sufficiently large, the NTK $K({\bf x}, {\bf x^{\prime}}; \theta_0)$ becomes independent of the initialization $\theta_0$~\cite{lee2019wide, arora2019exact} and we can define the asymptotic NTK (independent of the parameterization)
\begin{equation}\label{eqn:NTK-matrix}
	\bar{K}({\bf x}, {\bf x^{\prime}}) \triangleq \lim_{\text{NN width}\to \infty} \mathbb{E}_{\theta_0} \left\{K({\bf x}, {\bf x^{\prime}}; \theta_0)\right\} \in \mathbb{R}^{2 \times 2}.
\end{equation} 
Finally, we obtain the linearized loss dynamics of DRM (upon replacing $K$ 
by $\bar{K}$ and a vectorization of~\eqref{eqn:loss-trajectory-linearize})
\begin{equation}\label{eqn:loss-trajectory-NTK}
	\frac{d \bar{I}_{\mathcal{X}}(\theta(t))}{dt} 
	=
	-\frac{\eta}{|\mathcal{X}|^2}  [\nabla_{U} \bar{W}_{\mathcal{X}}(\theta(t))]^{\top} M_{\mathcal{X}} [\nabla_{U} \bar{W}_{\mathcal{X}}(\theta(t))],
\end{equation}
where the block Gram matrix
$
M_{\mathcal{X}}
$
consists of $\bar{K}({\bf x_m}, {\bf x_n})$ at its $(m, n)$-th block,
i.e., 
\begin{equation}\label{eqn:Gram-matrix}
	M_{\mathcal{X}} = \left(\bar{K}({\bf x_m}, {\bf x_n})\right)_{m,n = 1, \ldots, |\mathcal{X}|}  \in \mathbb{R}^{2|\mathcal{X}| \times 2|\mathcal{X}|},
\end{equation}
and $\bar{W}_{\mathcal{X}} = [\bar{W}_1, \ldots, \bar{W}_{|\mathcal{X}|}]^{\top} \in \mathbb{R}^{|\mathcal{X}| \times 1}$
and
$\nabla_U \bar{W}_{\mathcal{X}} = [\nabla_U \bar{W}_1, \ldots, \nabla_U \bar{W}_{|\mathcal{X}|}]^{\top} \in \mathbb{R}^{ 2|\mathcal{X}|\times 1}$.
% Note that $M_{\mathcal{X}}$ does not only contain information about $u$ but also its spatial gradient $\nabla u$.
%Since $M$ is PSD, it admits the spectral decomposition $M = Q^T \Lambda Q$, where $Q$ is an orthogonal matrix and $\Lambda = \text{diag}(\lambda_1, \ldots, \lambda_{(d+1)N})$ with $\lambda_1$
% $M_{\mathcal{X}}$ admits the spectral decomposition 
% \[M_{\mathcal{X}} = Q \Lambda Q^T,\] 
% where $Q = [q_1, \ldots, q_{2|\mathcal{X}|}]$ is an orthonormal matrix and $\Lambda = \text{diag}(\lambda_1, \ldots, \lambda_{2|\mathcal{X}|})$ with $\lambda_1 \geq \lambda_2 \geq \ldots \geq \lambda_{2|\mathcal{X}|} > 0$ and hence we can write
%     \begin{equation}\label{eqn:NTK-decomposed}
	%     \frac{d \bar{I}_{\mathcal{X}}(\theta(t))}{dt} = 
	%     -\eta\sum_{i=1}^{2|\mathcal{X}|} \lambda_i [\nabla_{U} \bar{W}_{\mathcal{X}}(\theta(t))q_i]^2,
	%     \end{equation}
%     where $q_i \in \mathbb{R}^{2|\mathcal{X}| \times 1}$ denotes the $i$-th column of the matrix $Q$.



We make two important observations from the loss dynamics~\eqref{eqn:loss-trajectory-NTK}: 1) Assuming $M_{\mathcal{X}}$ is positive definite, the convergence of the loss function $\bar{I}_{\mathcal{X}}(\theta(t))$
to a critical point is equivalent to the gradient of the Lagrangian vectors, i.e., $\nabla_{U} \bar{W}_{\mathcal{X}}(\theta(t))$, converges to zero;
% we expect that
% $\bar{I}_{\mathcal{X}}(\theta(t))$ converges to a local minimum and hence each $[\nabla_{U} \bar{W}_{\mathcal{X}}(\theta(t))q_i]$
% must converge to $0$ since all $\lambda_i > 0$;
2) If $\bar{I}_{\mathcal{X}}(\theta)$ is convex and bounded from below, $\theta(t)$ converges to the global minimum of $\bar{I}_{\mathcal{X}}(\theta)$.
However, the loss dynamics says nothing about the rate of convergence to a critical point. 

Therefore, we further assess the convergence rate of $\nabla_{U} \bar{W}_{\mathcal{X}}(\theta(t))$ to zero by considering its time evolution given by (derivation is postponed to~\ref{app:gradient-evolution})
\begin{equation}\label{eqn:gradient-evolution}
	\begin{split}
		\frac{d[\nabla_{U} \bar{W}_{\mathcal{X}}(\theta(t))]}{dt}
		=
		-\frac{\eta}{|\mathcal{X}|} D_{\mathcal{X}}(\theta(t)) M_{\mathcal{X}} [\nabla_{U} \bar{W}_{\mathcal{X}}(\theta(t))],
	\end{split}
\end{equation}
where the block diagonal matrix $D_{\mathcal{X}}(\theta(t))$ consists of $2\times 2$ Hessians of $\bar{W}_n \triangleq W(x_n, \bar{u}_n, \bar{u}_n^{\prime})$, i.e., 
\begin{equation*}
	D_{\mathcal{X}}(\theta(t)) = \text{diag}\left( \begin{bmatrix}
		\partial_{uu}^2 \bar{W}_n & \partial_{uu^{\prime}}^2 \bar{W}_n \\
		\partial_{u^{\prime}u}^2 \bar{W}_n & \partial_{u^{\prime}u^{\prime}}^2 \bar{W}_n
	\end{bmatrix}    \right)_{n=1, \ldots, |\mathcal{X}|} \in \mathbb{R}^{2|\mathcal{X}| \times 2|\mathcal{X}|}.
\end{equation*}
Now we are a in position to present the NTK theorem for DRM, which is a direct consequence of~\eqref{eqn:gradient-evolution}.
\begin{theorem}\label{thm:convergence}
	Suppose that 
	\begin{enumerate}
		\item the lazy training assumption (see e.g.,~\cite{chizat2019lazy}) is satisfied such that $D_{\mathcal{X}}(\theta(t)) \approx D_{\mathcal{X}} \triangleq D_{\mathcal{X}}(\theta_0)$;
		\item the Lagrangian $W$ is strictly convex in $(u, u^{\prime})$ such that the matrix $D_{\mathcal{X}}$ is positive definite;
		\item the Gram matrix $M_{\mathcal{X}}$ induced by the NTK~\eqref{eqn:NTK-matrix} is positive definite.
	\end{enumerate}
	Then, the asymptotic gradient (with respect to $u$ and $u^{\prime}$) dynamics of the Lagrangian $W$ in DRM is given by~\eqref{eqn:gradient-evolution}. 
	Moreover, we have
	\[
	[Q\nabla_{U} \bar{W}_{\mathcal{X}}(\theta(t))]^{\top} 
	=
	\text{e}^{-\eta \Lambda t /|\mathcal{X}|} [Q\nabla_{U} \bar{W}_{\mathcal{X}}(\theta_0)]^{\top},
	\]
	where we have used the spectral decomposition $D_{\mathcal{X}}M_{\mathcal{X}} = Q \Lambda Q^{\top}$ with 
	orthonormal matrix $Q = [q_1, \ldots, q_{2|\mathcal{X}|}]$ and diagonal matrix $\Lambda = \text{diag}(\lambda_1, \ldots, \lambda_{2|\mathcal{X}|})$ with $\lambda_1 \geq \lambda_2 \geq \ldots \geq \lambda_{2|\mathcal{X}|} > 0$.
\end{theorem}
A few remarks are in order.  First, the theorem suggests that the
specific convergence rate of
$\nabla_{U} \bar{W}_{\mathcal{X}}(\theta(t))$ along each direction
$q_i$ is determined by the corresponding eigenvalue $\lambda_i$. For
$\lambda_i \gg 0$, 
\[\bar{U}_{\mathcal{X}}(\theta(t)) = [\bar{U}_1(\theta(t)), \ldots, \bar{U}_{|\mathcal{X}|}(\theta(t))]^{\top} \in \mathbb{R}^{ 2|\mathcal{X}| \times 1}
\] 
with $\bar{U}_n(\theta) = [\bar{u}_n(x_n; \theta), \bar{u}_n^{\prime}(x_n; \theta)]^{\top} \in \mathbb{R}^{2\times 1}$
converges
fast along the direction $q_i$. Although for $\lambda_i \approx 0$, DNNs
have a significantly slower learning rate in the corresponding
direction $q_i$, preventing DNNs from learning the fine structure of the
solution.  
%Therefore, it is desirable to have a NTK spectrum that
%decays less rapidly than the exponential rate to avoid the spectral bias pathology. 
Motivated by this, we consider Fourier feature mapping to alleviate the spectrum bias issue in the next section.
Second, for a
non-convex Lagrangian $W$, the convergence of
$\nabla_{U} \bar{W}_{\mathcal{X}}(\theta(t))$ requires a more refined
analysis from variational calculus~\cite{dacorogna2007direct}, which
will be the focus of our future work. However, we empirically observed that in Section~\ref{sec:numerics} the Fourier feature mapping works equally well in the non-convex setting.
Finally, we point out that for
the type of non-convex variational problems considered in this work,
solving the corresponding Euler-Lagrange equations does not
necessarily lead to the correct minimizer and hence PINN is not
applicable.  Thus, DRM is the only option for solving variational
problem using neural networks.















% In general, the convergence rate of $\bar{U}_{\mathcal{X}}(\theta)$ along each $q_i$ depends on the specific form of the Lagrangian $W$.
% For a class of variational problems with $W$ satisfying
% \[
% M_{\mathcal{X}} [\nabla_{U} \bar{W}_{\mathcal{X}}]^{\top} \geq M_{\mathcal{X}} \bar{U}_{\mathcal{X}}^{\top}, 
% \]
% that is, $W(u, u^{\prime}) \geq u^2/2 + (u^{\prime})^2/2$ in the Riemannian metric induced by the NTK matrix $M_{\mathcal{X}}$,    
% we immediately have
% \[
% \frac{d \bar{U}_{\mathcal{X}}^{\top}}{dt} 
% \leq
% -\eta M_{\mathcal{X}} \bar{U}_{\mathcal{X}}^{\top}.
% \]
% Applying the Gronwall's lemma to the above inequality leads to
% \[
% Q^{\top} \bar{U}_{\mathcal{X}}^{\top}(\theta(t)) \leq Q^{\top} \bar{U}_{\mathcal{X}}(\theta_0)^{\top} \text{e}^{-\eta \Lambda t}, 
% \]
% which suggests that the specific exponential convergence rate along each direction $q_i$ is determined by the corresponding eigenvalue $\lambda_i$. For $\lambda_i \gg 0$, $\bar{U}_{\mathcal{X}}^{\top}(\theta(t))$ converges fast along the direction $q_i$. While for $\lambda_i \approx 0$, DNNs have a significantly slower rate in learning along the corresponding direction $q_i$, making it fail to learn the fine structure of the solution. 
% Therefore, it is desirable to have a NTK spectral that decays slowly to avoid the spectral bias pathology. 

% In summary, we comment that the convergence of DRM is generally guaranteed under mild assumptions. However, the specific convergence rate along each direction depends critically on the Lagrangian $W$ of the variational problem under consideration.    
% In comparison, due to its simple squared loss, PINN always has a more explicit convergence rate along the direction defined by the eigenvector of the NTK matrix~\cite{wang2021eigenvector}.
% Nevertheless, we have shown that, for a large class of variational problems, DRM has the same convergence rate along each direction as PINN and hence also suffers the spectral bias pathology. 
% Moreover, 


% We refer readers to~\cite{lu2021machine} for detailed comparison of PINN and DRM from the minimax optimality perspective.  










%and the DNN tends to first 
%``zero" the gradient $\nabla_{U} \bar{I}_{\mathcal{X}}(\theta(t))$ along the direction $Q \nabla_{U} \bar{I}_{\mathcal{X}}(\theta(t))$. 

% The asymptotic loss dynamics~\eqref{eqn:loss-trajectory-NTK} suggests that the loss is guaranteed to converge to a ``critical point" (i.e., a point at which $\nabla_U \bar{I}_{\mathcal{X}} = 0$) provided that $M$ is positive definite. Furthermore, 



\subsection{Fourier feature from the NTK perspective}
\label{subsec:FF}

\begin{figure}[!ht]
	\centering
	\includegraphics[width=1.0\textwidth]{Figure_2.png}
	\caption{Structure of Neural Network by applying Fourier feature mapping to the input layer.}
	\label{fig:NN+FF_structure}
\end{figure}

To alleviate the spectral bias of DRM, we apply a Fourier feature mapping $\delta$ to the input $\bf{x}$ before it is sent to the DNN. See Figure~\ref{fig:NN+FF_structure} for the simple architecture. 
The Fourier feature mapping has been widely used in various fields in machine learning, e.g., large-scale kernel regression and deep learning~\cite{rahimi2007random, FourierFeatures2020}.
However, to the best of our knowledge, the reason why Fourier feature mapping enables DNNs to learn high frequency solutions is not well understood from a theoretical perspective.
In this section, we provide a heuristic argument from the NTK perspective to justify the application of Fourier feature mapping for DRM.
For simplicity, we consider an one dimensional problem ($d=1$) and assume that the Lagrangian $W = W(x, u)$. The Fourier feature mapping is chosen to be $\delta(x) = [\sin x, \cos x] \in \mathbb{S}^1$, where $\mathbb{S}^1$ is the unit circle in $\mathbb{R}^2$.
Viewing the pair ${\bf y} = [\sin x, \cos x] \in \mathbb{S}^1$ as the input of the DNN, the dataset $\mathcal{X}$ is mapped to $\mathcal{Y} = \delta(\mathcal{X}) \subset \mathbb{S}^1$. 
Under these assumptions, the asymptotic NTK defined in~\eqref{eqn:NTK-matrix} becomes a scalar valued positive definite kernel
\[
\bar{K}({\bf y_1}, {\bf y_2})
= \lim_{\text{NN width}\to \infty} \mathbb{E}_{\theta_0} \left\{\nabla_{\theta} \hat{u}({\bf y_1} ;\theta_0) [\nabla_{\theta}\hat{u}({\bf y_2} ;\theta_0)]^{\top}\right\}
, \qquad {\bf y_1}, {\bf y_2} \in \mathbb{S}^1
\]
and $M_{\mathcal{Y}}$ reduces to the usual Gram matrix evaluated at the input set $\mathcal{Y}$ (recall~\eqref{eqn:Gram-matrix} for definition), i.e.,
\[
M_{\mathcal{Y}} = \bar{K}(\mathcal{Y}, \mathcal{Y}).
\]
Note that the above argument can be easily generalized to the case where $W = W(x, u, u^{\prime})$ by considering a matrix valued kernel $\bar{K}$.
In~\ref{app:eigenspectrum}, we show that the $k$-th eigenvalue of $M_{\mathcal{Y}}$ is approximately proportional to the $k$-th eigenvalue of the NTK $\bar{K}$ (see~\eqref{eqn:eigen-L} for definition). Therefore, one may study the eigenvalues of $\bar{K}$ when concerned with the decay rate of the eigenvalues of $M_{\mathcal{Y}}$. 
%Assume the DNN has a ReLU activation and zero initial bias
It has been shown that (Theorem 1 in~\cite{geifman2020similarity}), when restricted to $\mathbb{S}^1$, the $k$-th eigenvalue of the NTK $\bar{K}$ scales as $\mathcal{O}(k^{-2})$, meaning that the eigenvalue of $\bar{K}$ has a quadratic decay rate. 
For multi-scale problems whose NTK spectrum exhibits multiple scales, e.g., an exponential decay rate $\mathcal{O}(\mathrm{e}^{-k})$, the Fourier feature mapping may homogenize the convergence rate along each direction $q_i$ hence
alleviating the spectral bias issue of the dynamics~\eqref{eqn:gradient-evolution}.
In Section~\ref{sec:numerics}, we empirically demonstrate the benefit of Fourier feature mapping when applied to multi-scale variational problems. 




% the reproducing kernel Hilbert space (RKHS) associated with the NTK $\bar{K}$ coincides with the RKHS accociated with the Laplace kernel 
% \[
% K_{\text{Lap}}({\bf y_1}, {\bf y_2}) = \text{e}^{-\gamma\|{\bf y_1} - {\bf y_2}\|} =  \text{e}^{-\gamma \sqrt{2(1 - {\bf y_1}^{\top} {\bf y_2)}}},  \qquad {\bf y_1}, {\bf y_2} \in \mathbb{S}^1
% \] 
% i.e., 
% \[
% \mathcal{H}_{\bar{K}}(\mathbb{S}^1) = \mathcal{H}_{\text{Lap}}(\mathbb{S}^1).
% \]
% In other words, when restricted to a circle, an over-parameterized DNN works effectively like a Laplace kernel regressor. 







%%%%%%%%%%%%%%%%%%%%%%%%%%%%%%%%%%%%%%%%%%%%%%%%%%%%%%%%%%%%%%%%%%%%%%%%%
\section{Numerical Results \& Discussion}\label{sec:numerics}
We consider the following non-convex variational minimization problems: the first consists of a double well potential, $W(x) = (x^2-1)^2$ which leads to the following energy minimization problem: 
\begin{align}
	{\rm Minimize} \ I(u) = \int_0^1 (u_x^2-1)^2  \ dx  \qquad {\rm subject \ to} \qquad u(0) = u(1) = 0.
	\label{eqn:1D_Problem_1}
	%+ \varepsilon^2 u_{xx}^2
\end{align}

Note that the first component of the energy density is non-negative with zeros at $u_{x} = \pm 1$, which are often called zero-energy wells and correspond to the preferred phases of the problem. 
%The regularization term $\varepsilon^2 u_{xx}^2$ penalizes the transitions between the two zero-energy wells \cite{Kohn&Otto}. 
We note that for this particular problem, the minimum is attained: Carstensen showed that all Lipschitz continuous functions $u(x)$, with slope $u_{x} = \pm 1$ almost everywhere, minimize $I$ \cite{Carstensen_2005}. The energy of such function is $I = 0$. It should be emphasized that while deriving the Euler-Lagrange equation for non-convex problems like~\eqref{eqn:1D_Problem_1} is possible as shown below:
\begin{align}
	\frac{d}{dx} [u_x (u_x^2-1)] = 0 
	\label{eqn:Euler-Lagrange}
	%- \frac{\varepsilon^2}{2}\frac{d^2}{dx^2} [u_{xx}] = 0,
\end{align}
its solution $u(x) = 0$ does not minimize~\eqref{eqn:1D_Problem_1}. Consequently, applying the PINN algorithm to the strong form equations is not viable, as the algorithm would inevitably converge to the trivial solution.

The second benchmark problem is a variation of the double well potential, where a lower order term of the form $u^2$ is introduced, generating the following minimization problem:
\begin{align}
	{\rm Minimize} \ I(u) = \int_0^1 (u_x^2-1)^2 + u^2 \ dx \qquad {\rm subject \ to} \qquad u(0)=u(1) = 0.
	\label{eqn:1D_Problem_2}
	%+ \varepsilon^2 u_{xx}^2
\end{align}
%, in the absence of $\varepsilon^2 u_{xx}^2$, 
We note that no minimizer exists for this problem. The infimum, although zero, cannot be attained since there is no function that satisfies $u = 0$ and $u_{x} = \pm 1$ almost everywhere. Minimizing sequences oscillate and converge weakly, but not strongly, to zero \cite{Carstensen_2005, Muller1993, Kohn&Otto}. This is the first simple  example that demonstrates how minimization can lead to fine scale oscillations or microstructure formation.

Finally, the third problem considered here is the $2D$ scalar problem for twin branching, which takes the following form:  
\begin{align}
	{\rm Minimize} \ I(u) = \int_{\Omega} u_x^2 + (u_y^2-1)^2  \ dx dy \qquad {\rm subject \ to} \qquad u=0  \ {\rm on} \ \partial \Omega,  
	\label{eqn:2D_Problem}
	%+ \varepsilon^2 u_{yy}^2
\end{align}
%in the absence of $\varepsilon^2 u_{yy}^2$ 
where $\Omega = [0,1]^2$. As in the previous problem, no minimizers exist since there is no function that can satisfy the integrand and boundary conditions at the same time, leading to minimizing sequences that develop rapid oscillations \cite{Muller}.

Recall that Chen \textit{et al}.\ applied DRM to non-convex energy problems in $1D$ and $2D$, similar to the ones described above. We now discuss the differences and similarities between our benchmark problems and those examined in~\cite{Chen&Rosakis2023}. Comparable to~\eqref{eqn:1D_Problem_1}, the $1D$ minimization problem in~\cite{Chen&Rosakis2023} is comprised of a double-well potential energy density subject to Dirichlet boundary conditions. Both minimization problems consist of a minimum energy ($I =0$) which can be obtained through multiple continuous functions $u(x)$, leading to loss of uniqueness. A key distinction lies in the minima locations; in~\cite{Chen&Rosakis2023}, they occur at $0$ and $1$, while in ~\eqref{eqn:1D_Problem_1}, they occur at $-1$ and $1$. Our Dirichlet boundary conditions are fixed at $0$, contrasting with~\cite{Chen&Rosakis2023} where the left boundary is fixed at $0$ and the right boundary is fixed at $\gamma$ where $\gamma \in \mathbb{R}$. This leads to solutions with slopes $u_x = 0$ and $1$ in~\cite{Chen&Rosakis2023}, whereas the solutions to~\eqref{eqn:1D_Problem_1} have slopes $u_x = \pm 1$.

Similarly, the $2D$ minimization problem in~\cite{Chen&Rosakis2023} mirrors features found in~\eqref{eqn:2D_Problem}. Both problems consist of a double well energy potential and are subject to Dirichlet boundary conditions, which yield to minimizing sequences with rapid oscillations but no actual minimizers. The main differences between~\eqref{eqn:2D_Problem} and the 2D problem in~\cite{Chen&Rosakis2023} lie in the minima locations of the energy well potential ($(\pm 1,0)$ in~\eqref{eqn:2D_Problem} vs. $(0,0)$ and $(1,0)$ in~\cite{Chen&Rosakis2023}). The Dirichlet boundary conditions are set to $0$ across the boundary in~\eqref{eqn:2D_Problem}, while Chen \textit{et al}.\ set $u(x,y) = \gamma x$ with $\gamma \in \mathbb{R}$ in~\cite{Chen&Rosakis2023}.

Given that the distinctions mentioned above are cosmetic and do not alter the fundamental structure of the minimization problems, we anticipate the hypothesis and conclusions articulated in \cite{Chen&Rosakis2023}, particularly the hypothesis that increasing the DNN increases the number of twin bands for the $2D$ problem, remain true for~\eqref{eqn:2D_Problem}. We test this hypothesis numerically in the sections below.  


\label{sec:Numerics}
\subsection{$1D$ Benchmark Problem \# 1}\label{sec:1D_results_1}
We start our discussion by approximating the solution to~\eqref{eqn:1D_Problem_1} using DRM without Fourier mapping (as described in~\cite{Chen&Rosakis2023}) and compare the results with the new algorithm: DRM with Fourier mapping (DRM\&FM). In both cases, a fully connected feed-forward neural network with an input layer, multiple hidden layers and an output layer is constructed. The input layer consists of one node (for the $x$ coordinate of our problem),  each hidden layer consists of $128$ nodes, and the output layer consists of one node  (used to output the approximated solution $\hat{u}$). Consistent with~\cite{Chen&Rosakis2023}, we apply the ReLU activation function in each layer.  To accelerate training, we use Adams Optimizer on a mini-batch size of $128$ collocation points sampled uniformly, with an initial learning rate $\eta = 10^{-4}$. We implement a cosine annealing schedule that decreases the learning rate to zero over the course of the simulation. The boundary conditions are enforced using the penalty approach with a penalty parameter set to $\lambda = 500$.

%We approximate the minimizing solutions of the non-regularized version of~\eqref{eqn:1D_Problem_1} by setting $\varepsilon = 0$. 
Recall that there exist multiple solutions that minimize~\eqref{eqn:1D_Problem_1}: namely, any function $u(x)$ with slope $u_x = \pm 1$ almost everywhere minimizes the functional $I(u)$. In Figure~\ref{fig:1D_problem_noFF} we present the minimizing solutions generated by DRM with no Fourier mapping as we vary the depth of the network while setting the learning rate initially to $\eta = 10^{-4}$.  We see that for this particular benchmark problem, increasing the depth of the DNN does not generate high-frequency solutions, analogous to the increased number of twin bands of the 2D problem discussed in~\cite{Chen&Rosakis2023}. Solutions with one transition between the two preferred interfaces ($u_x = \pm 1$) are generated for a DNN with $5$, $7$ and $9$ hidden layers (see Fig~\ref{subfig:ux_1D_no_FF}). %We approximate the total energy of the solutions using a simple quadrature and obtain that $I(u)$ increases with the depth of the DNN: $I(u) \approx 4.8\times 10^{-4}$ for a DNN with 5 hidden layers, $I(u) \approx 6.2\times 10^{-4}$ and $I(u) \approx 1.6\times 10^{-3}$ for 7 and 9 hidden layers respectively. We see that $I(u)$ increases with the depth of the DNN, however the increase can be attributed to the noise in $u_x$ as seen in Fig.~\ref{subfig:ux_1D_no_FF}.


\begin{figure}[!ht]
	\centering
	\subfigure[~]{\includegraphics[width=0.31\textwidth]{Figure_3a.png}\label{subfig:u_1D_no_FF}}
	\subfigure[~]{\includegraphics[width=0.31\textwidth]{Figure_3b.png}\label{subfig:ux_1D_no_FF}}
	\caption{(a)~DRM approximation to~\eqref{eqn:1D_Problem_1} with ReLU activation function, $\eta = 1.0\times 10^{-4}$ after $100000$ epochs with DNN structure of $5$, $7$ and $9$ hidden layers. (b)~ The derivative $u_x$ of the DRM approximation to~\eqref{eqn:1D_Problem_1}.}
	\label{fig:1D_problem_noFF}
\end{figure}

Figure~\ref{fig:1D_problem_FF} displays the solution generated by DRM with Fourier feature mapping under the same conditions. Recall that the information passes from the input layer, through a Fourier mapping of the form $\delta({\bf x}) = \left[\sin(2^i\pi {\bf x}), \cos(2^i \pi {\bf x})\right]$ with $i = 2, 3, 4$ and $\bf{x} \in \mathbb{R}$, to the hidden and output layers. %Note that modify the Fourier mapping $\delta({\bf x})$ by removing ${\bf x}$ from the mapping because we know that the minimizing solutions are periodic over the domain.
We observe that the frequency of the mapping can be leveraged to generate minimizing solutions of high frequency, independently of the depth of the DNN.  
When passing a Fourier mapping of frequency $4\pi$ as shown in Fig.~\ref{fig:u_1D_FF=2} ($8\pi$ as shown in Fig.~\ref{fig:u_1D_FF=3}), we generate a solution with $4$ ($8$) transitions between preferred states, independently of the depth of the DNN. When applying a Fourier mapping of frequency $16\pi$ however, we get mixed results: implementing a DNN with $5$ and $7$ hidden layers leads to a solution with $32$ transitions between states (as shown by the black and red dotted lines in Fig.~\ref{fig:u_1D_FF=4}), while a $9$ layer DNN leads to a solution with $16$ transitions between preferred states (as shown by blue dashed lines). Figure~\ref{fig:1D_problem_FF} shows that increasing the frequency of the Fourier mapping increases the number of transitions between the preferred states but one cannot quantify the relationship between mapping frequency and number of transitions within the domain. 



\begin{figure}[ht!]
	\centering
	\subfigure[$i = 2$]{\includegraphics[width=0.31\textwidth]{Figure_4a.png}\label{fig:u_1D_FF=2}}
	\subfigure[$i = 3$]{\includegraphics[width=0.31\textwidth]{Figure_4b.png}\label{fig:u_1D_FF=3}}
	\subfigure[$i = 4$]{\includegraphics[width=0.31\textwidth]{Figure_4c.png}\label{fig:u_1D_FF=4}}
	\caption{DRM approximation to~\eqref{eqn:1D_Problem_1} where a NN with $5$,$7$ and $9$-hidden layers, ReLU activation function, $\eta = 1.0\times 10^{-4}$ and Fourier mapping of frequency $\delta({\bf x}) = \left[\sin(2^i\pi {\bf x}),\cos(2^i\pi {\bf x})\right] $ after $100000$ epochs. }
	\label{fig:1D_problem_FF}
\end{figure}
%%%%%%%%%%%%%%%%%%%%%%%%%%%%%%%%%%%%%%%%%%%%%%%%%%%%%%%%%%%%%%%%%%%%%%%%%%%%%%%%%%%
\subsection{$1D$ Benchmark Problem \# 2}\label{sec:1D_results_2}
We now discuss how DRM alone and DRM with Fourier mapping (DRM\&FM) approximate the solution sequences to the second benchmark problem given by~\eqref{eqn:1D_Problem_2}. Recall that no minimizer exists for this problem since there is no function that satisfies the conditions $u = 0$ and $u_x = \pm 1$ everywhere. Figure~\ref{fig:1D_problem_2_noFF} shows the DNN approximation of the minimizing solution to~\eqref{eqn:1D_Problem_2} as the depth of the DNN increases with no Fourier mapping after $200,000$ epochs (First Row) and $500,000$ epochs (Second Row). We observe that increasing the depth of DNN does not consistently increase the number of transitions between the preferred states. Increasing the depth of the DNN from $3$ to $5$ hidden layers increases the number of transitions for $200,000$ epochs. However, the number of transitions decreases as the depth of the DNN is increased from $5$ to $7$ hidden layers. A similar occurrence can be observed in the second row of Fig.~\ref{fig:1D_problem_2_noFF} where our simulations are run for $500,000$ epochs. In this case, increasing the depth of the DNN from $3$ to $5$ hidden layers decreased the number of transitions while increasing the depth from $5$ to $7$ hidden layers increased the number of transitions between preferred states. Based on our simulations, we can say that increasing the depth of the DNN does not consistently generate high-frequency solutions for the $1D$ benchmark problem given by~\eqref{eqn:1D_Problem_2}. We also note that a DNN with 7 hidden layers run for $500,000$ epochs was able to generate a minimizing sequence with $16$ transitions between preferred states. 

\begin{figure}[ht!]
	\centering
	\subfigure[NN: $3\times 128$]{\includegraphics[width=0.31\textwidth]{Figure_5a.png}\label{subfig:u2_1D_noFF_d=3_200K}}
	\subfigure[NN: $5\times 128$]{\includegraphics[width=0.31\textwidth]{Figure_5b.png}\label{subfig:u2_1D_noFF_d=5_200K}}
	\subfigure[NN: $7\times 128$]{\includegraphics[width=0.31\textwidth]{Figure_5c.png}\label{subfig:u2_1D_noFF_d=7_200K}}\\
	\subfigure[NN: $3\times 128$]{\includegraphics[width=0.31\textwidth]{Figure_5d.png}\label{subfig:u2_1D_noFF_d=3_500K}}
	\subfigure[NN: $5\times 128$]{\includegraphics[width=0.31\textwidth]{Figure_5e.png}\label{subfig:u2_1D_noFF_d=5_500K}}
	\subfigure[NN: $7\times 128$]{\includegraphics[width=0.31\textwidth]{Figure_5f.png}\label{subfig:u2_1D_noFF_d=7_500K}}
	\caption{{\bf First Row (a)-(c)}: DRM approximation to~\eqref{eqn:1D_Problem_2} with ReLU activation function, $\varepsilon = 0$, $\eta = 1.0\times 10^{-4}$ and cosine annealing after $200000$ epochs.
		{\bf Second Row (d)-(f):} Row: DRM approximation to~\eqref{eqn:1D_Problem_2} with ReLU activation function, $\varepsilon = 0$, $\eta = 1.0\times 10^{-4}$ and cosine annealing after $500000$ epochs. }
	\label{fig:1D_problem_2_noFF}
\end{figure}
Figure~\ref{fig:1D_problem_2_FF} shows the minimizing solutions that are obtained by the DRM with a DNN structure of $3$ hidden layers and Fourier mapping of frequency $2\pi$, $4\pi$ and $8\pi$  after $200,000$ and $500,000$ epochs. We see here that the Fourier mapping with frequency $2\pi$ as shown in Figs~\ref{subfig:u2_1D_FF=1_200K} and~\ref{subfig:u2_1D_FF=1_500K} enables us to generate a solution of $12$ transitions between the two preferred states, a result that is comparable with the DRM approximation solution of a DNN of $7$ hidden layers as shown in Figs.~\ref{subfig:u2_1D_noFF_d=7_200K} \& \ref{subfig:u2_1D_noFF_d=7_500K}. Note the solution approximation consists of $11$ transitions for $200,000$ epochs and $16$ transitions for $500,000$ epochs. We note that DRM\&FM enables us to keep the number of hidden layers in the DNN fixed and generate minimizing solutions with more transitions, such as the ones shown in Fig.~\ref{fig:1D_problem_2_FF}. While it seems that the number of transitions between preferred states increases with the frequency of the Fourier mapping, the authors did not investigate the relationship between the frequency of the Fourier mapping and the number of transitions within the solution for this $1D$ problem. 
\begin{figure}[ht!]
	\centering
	\subfigure[$i = 1$]{\includegraphics[width=0.31\textwidth]{Figure_6a.png}\label{subfig:u2_1D_FF=1_200K}}
	\subfigure[$i = 2$]{\includegraphics[width=0.31\textwidth]{Figure_6b.png}\label{subfig:u2_1D_FF=2_200K}}
	\subfigure[$i = 3$]{\includegraphics[width=0.31\textwidth]{Figure_6c.png}\label{subfig:u2_1D_FF=3_200K}}\\
	\subfigure[$i = 1$]{\includegraphics[width=0.31\textwidth]{Figure_6d.png}\label{subfig:u2_1D_FF=1_500K}}
	\subfigure[$i = 2$]{\includegraphics[width=0.31\textwidth]{Figure_6e.png}\label{subfig:u2_1D_FF=2_500K}}
	\subfigure[$i = 3$]{\includegraphics[width=0.31\textwidth]{Figure_6f.png}\label{subfig:u2_1D_FF=3_500K}}    
	\caption{{\bf First Row (a)-(c):} DRM\&FM approximation to~\eqref{eqn:1D_Problem_2} with $3 \times 128$ NN (3 hidden layers), ReLU activation function, $\varepsilon = 0$, $\eta = 1.0\times 10^{-4}$ and Fourier feature of frequency $\delta({\bf x}) = \left[\sin(2^i\pi {\bf x}),\cos(2^i\pi {\bf x})\right] $ after $200000$ epochs. {\bf Second Row (d)-(f):} DRM\&FM approximation under the same conditions after $500,000$ epochs. }
	\label{fig:1D_problem_2_FF}
\end{figure}

%%%%%%%%%%%%%%%%%%%%%%%%%%%%%%%%%%%%%%%%%%%%%%%%%%%%%%%%%%%%%%%%%%%%%%%%%%%%%%%%
\subsection{$2D$ Benchmark Problem}\label{sec:2D_results}
We now turn to the $2D$ twin branching problem given by~\eqref{eqn:2D_Problem} and investigate whether the DRM\&FM method can be extended to generate solutions to $2D$ microstructure problems. Recall that, similar to~\eqref{eqn:1D_Problem_2}, this problem does not have a minimizer since there are no functions that can minimize the integrand and satisfy the Dirichlet boundary conditions at the same time, leading to microstructure behavior. The ideal minimizer would be a function  $u(x,y)$ such that $u_y = \pm 1$, $u_x = 0$ in $\Omega$ and $u = 0$ on $\partial \Omega$. Such function does not exist, leading to minimizing sequences with fine scale oscillations instead. 
We attempt to capture these minimizing sequences using a DNN similar in structure to the ones implemented in Secs.~\ref{sec:1D_results_1} \& \ref{sec:1D_results_2}. We adapt the DNN to minimize the $2D$ problem in~\eqref{eqn:2D_Problem} through the following changes: the input layer consists of two nodes, one for each coordinate $x$ and $y$ of our $2D$ domain,  the activation function used is of the form 
$\sigma(x) = \sqrt{x^2 + \rho^2}$, where $\rho = 0.1.$ This activation function is a variation of the SmReLU activation function used in~\cite{Chen&Rosakis2023} to better suit the problem considered here.
The DRM is run with Adams Optimizer for $300,000$ epochs with a total number of $N = 1000$ collocation points sampled uniformly across the domain ($N_{int} = 600$ in the interior  and N$_b = 400$: $100$ uniformly sampled points across each boundary). Note that we set the initial learning rate to $\eta = 10^{-4}$ and apply cosine annealing as in the 1D case. %The regularization term, normally used in traditional numerical methods such as FEM is initially set to zero. We investigate its effect on the minimizing sequences by taking nonzero values of $\varepsilon$ in Sec.~\ref{sec:2D_problem_epsilon}.

Figure~\ref{fig:2D_problem_no_FF_eps=0.0} displays the minimizing sequences to~\eqref{eqn:2D_Problem} (we plot $u_y$ instead of $u$ to show the transition between the two preferred states $u_y = \pm 1$) as we increase the number of hidden layers in the DNN. Here, as in Sec.~\ref{sec:1D_results_1}, we consider a DNN with $3$, $5$ and $7$ hidden layers respectively and no Fourier mapping. We see that as the depth of the DNN increases, the number of bands stays the same. In fact, for a network with $7$ hidden layers, the solution is stuck to an unstable state ($u = 0$). We note that for this particular problem, increasing the depth of the DNN does not generate minimizing sequences with a large number of twin bands (high frequency). It seems like the depth of the NN is hindering the DNN from converging to a minimum: instead, it is stuck at a saddle point in the energy density functional of~\eqref{eqn:2D_Problem}.

In contrast, when a Fourier mapping of the form $\delta({\bf x}) = \left[{\bf x}, \sin(2^i\pi {\bf x}),\cos(2^i\pi {\bf x})\right]$ where $i = 1-4$  and ${\bf x} \in \mathbb{R}^2$ is applied, the number of transitions between preferred states in $u_y$ (or number of twin bands as described in~\cite{Chen&Rosakis2023}) increase (see Figure~\ref{fig:2D_problem_FF_eps=0.0}). Note that we modify the Fourier mapping by including ${\bf x}$ because a periodic solution is no longer a minimizer of the problem and we no longer expect a periodic solution in the domain. We hypothesize that applying a Fourier mapping of any frequency allows the DRM to converge to a minimizing sequence quicker than if no Fourier mapping was applied (compare Figs.~\ref{subfig:uy_FF=1_epsbar=0.0}-\ref{subfig:uy_FF=4_epsbar=0.0} with Fig.~\ref{subfig:uy_no_FF_depth=3}). We observe needle like structures forming around $x = 0$ and $x = 1$ when a Fourier mapping of low frequency is applied (see Fig.~\ref{subfig:uy_FF=1_epsbar=0.0}) but these needles do not fully grow to form additional bands in the course of our simulation. A similar behavior can be observed in Figs.~\ref{subfig:uy_FF=2_epsbar=0.0}-\ref{subfig:uy_FF=4_epsbar=0.0}: needle like structures are formed around $x = 0,1$ but these structures get smaller as the frequency of the Fourier mapping increases. Additionally, we observe that the number of twin bands increases as the frequency of the Fourier mapping increases: there are $4$ transitions between states when the frequency is set to $2 \pi$, $8$ transitions when the frequency is $4\pi$, $15$ transitions when the frequency is $8\pi$ and $32$ transitions when the frequency is $16\pi$ (See Figs.~\ref{subfig:uy_FF=2_epsbar=0.0}-\ref{subfig:uy_FF=4_epsbar=0.0}).
We observe that the minimizing solutions are noisy as the Fourier frequency increases and we attribute this noise to the fact that ~\eqref{eqn:2D_Problem} has no minimum. We emphasize that incorporating Fourier feature mapping into the DRM does not alter the number of collocation points used in the simulations ($N=1000$ in 2D case and $N = 128$ in 1D). This approach stands in sharp contrast to traditional methods like FEM, which depend heavily on mesh-size refinement to resolve the microstructure. 
%We investigate the effect that this parameter has on the minimizing sequences in the next section.

\begin{figure}[ht!]
	\centering
	\subfigure[NN: $3 \times 128$]{\includegraphics[width=0.31\textwidth]{Figure_7a.png}\label{fig:uy_2D_no_FF_5_eps=0}\label{subfig:uy_no_FF_depth=3}}
	\subfigure[NN: $5 \times 128$]{\includegraphics[width=0.31\textwidth]{Figure_7b.png}\label{fig:uy_2D_no_FF_7_eps=0}\label{subfig:uy_no_FF_depth=5}}
	\subfigure[NN: $7 \times 128$]{\includegraphics[width=0.31\textwidth]{Figure_7c.png}\label{fig:uy_2D_no_FF_9_eps=0}\label{subfig:uy_no_FF_depth=7}}
	\caption{DRM approximation to~\eqref{eqn:2D_Problem} with activation function $\sigma(x) = \sqrt{x^2+\rho^2}, \rho = 0.1$, $\eta = 1.0\times 10^{-4}$ and no Fourier Feature after $300000$ epochs.}
	\label{fig:2D_problem_no_FF_eps=0.0}
\end{figure}


\begin{figure}[ht!]
	\centering
	\subfigure[$i = 1$]{\includegraphics[width=0.31\textwidth]{Figure_8a.png}\label{subfig:uy_FF=1_epsbar=0.0}}
	\subfigure[$i = 2$]{\includegraphics[width=0.31\textwidth]{Figure_8b.png}\label{subfig:uy_FF=2_epsbar=0.0}}
	\subfigure[$i = 3$]{\includegraphics[width=0.31\textwidth]{Figure_8c.png}\label{subfig:uy_FF=3_epsbar=0.0}}
	\subfigure[$i = 4$]{\includegraphics[width=0.31\textwidth]{Figure_8d.png}\label{subfig:uy_FF=4_epsbar=0.0}}
	\caption{DRM approximation to~\eqref{eqn:2D_Problem} with $3 \times 128$ NN and Fourier feature of frequency $\delta({\bf x}) = \left[{\bf x},\sin(2^i\pi {\bf x}),\cos(2^i\pi {\bf x})\right]$ with $\eta = 1.0\times 10^{-4}$ after $300000$ epochs.}
	\label{fig:2D_problem_FF_eps=0.0}
\end{figure}
%%%%%%%%%%%%%%%%%%%%%%%%%%%%%%%%%%%%%%%%%%%%%%%%%%%%%%%%%%%%%%%%%%%%%%%%%%%%%%%%%%
\subsection{Regularized 2D Problem \& Fourier Mapping}\label{sec:2D_problem_epsilon}
Regularization is frequently used to ensure the existence of solutions to nonconvex minimization problems while also determining the length scale and fine geometry of the resulting microstructures~\cite{Muller1993, Kohn&Otto, Muller, Kohn1992}. This is achieved by adding a high-gradient term to the energy density  $W$ in~\eqref{eqn:variationalproblem}. Traditional numerical methods leverage this approach to identify the microstructure's length scale~\cite{Muller1993} and predict specific microstructure dynamics~\cite{Dondl2016}.
In this context, we consider the regularized 2D problem:
\begin{align}
	{\rm Minimize} \ I(u) = \int_{\Omega} u_x^2 + (u_y^2-1)^2 + \varepsilon^2 u_{yy}^2 \ dx dy \qquad {\rm subject \ to} \qquad u=0  \ {\rm on} \ \partial \Omega,  
	\label{eqn:2D_Problem_reg}
\end{align}
and investigate how Fourier mapping and the regularization term interact in generating high-frequency solutions to the regularized minimization problem in 2D. Recall that  $u_x$ prefers to be $0$ while $u_y$ jumps between $\pm 1$. The additional term $\varepsilon^2 u_{yy}^2$ in~\eqref{eqn:2D_Problem_reg} penalizes these transitions, facilitating the formation of fine structures by reducing the surface energy associated with the high-gradient contributions~\cite{Kohn1992}.

Figure~\ref{fig:2D_problem_FF_eps=0.1by16} shows the graph of the DRM generated solutions ($u_y$ instead of $u$) when $\varepsilon = 0.1/16$. We see that introducing a regularization term generates smooth minimizing sequences throughout the domain independently of whether a Fourier mapping is applied, though the Fourier mapping enables the method to generate solutions with more twin bands for large frequencies.  Comparing Figs.~\ref{subfig:uy_no_FF_depth=3} and ~\ref{fig:2D_problem_FF_eps=0.0} with Fig.~\ref{fig:2D_problem_FF_eps=0.1by16}, we observe that the regularization term helps the DRM generate smooth and symmetric solutions with smoother interfacial transitions and uniform microstructure length scales.  

When increasing $\varepsilon$ further, we observe that the DRM method generates minimizing sequences with smoother interfacial transitions and larger microstructure length scales as shown in Fig.~\ref{fig:2D_problem_FF_eps=0.1by4}.  Additionally, we observe that, for $\varepsilon = 0.1/4$, when applying Fourier mapping of frequency $4\pi$ and $8\pi$, the DRM generates the same sequence (with 8 transitions) while the same Fourier mappings  and different values of $\varepsilon$ ($\varepsilon = 0.1/16$ and $\varepsilon = 0$) generate sequences with $8$ and $15$ transitions respectively (see Figs.~\ref{subfig:uy_FF=3_epsbar=0.0}, ~\ref{subfig:uy_FF=3_eps=0.1by16} and~\ref{subfig:uy_FF=3_eps=0.1by4}). A similar behavior is observed when applying a Fourier mapping of frequency $16\pi$: DRM generates a sequence with $16$ twin bands when $\varepsilon = 0.1/4$ and a sequence with $32$ twin bands for smaller values of $\varepsilon$ (compare Figs.~\ref{subfig:uy_FF=4_epsbar=0.0} with Figs.~\ref{subfig:uy_FF=4_eps=0.1by16} and ~\ref{subfig:uy_FF=4_eps=0.1by4}). This is perhaps not surprising since the regularization term imposes an upper bound on the number of interfaces that can be generated for a value of $\varepsilon$. 

%Figure~\ref{fig:2D_problem_FF_eps=0.0} shows that one no longer needs the regularization term to generate high-frequency minimizing solutions, however the term is still beneficial if one is interested in generating symmetric and smooth solutions that do not have any numerical noise (solutions with equidistant bands).
\begin{figure}[ht!]
	\centering
	\subfigure[no FF]{\includegraphics[width=0.31\textwidth]{Figure_9a.png}\label{subfig:uy_no_FF_eps=0.1by16}}
	\subfigure[$i = 1$]{\includegraphics[width=0.31\textwidth]{Figure_9b.png}\label{subfig:uy_FF=1_eps=0.1by16}}
	\subfigure[$i = 2$]{\includegraphics[width=0.31\textwidth]{Figure_9c.png}\label{subfig:uy_FF=2_eps=0.1by16}}\\
	\subfigure[$i = 3$]{\includegraphics[width=0.31\textwidth]{Figure_9d.png}\label{subfig:uy_FF=3_eps=0.1by16}}
	\subfigure[$i = 4$]{\includegraphics[width=0.31\textwidth]{Figure_9e.png}\label{subfig:uy_FF=4_eps=0.1by16}}
	\caption{DRM approximation to~\eqref{eqn:2D_Problem} with $3 \times 128$ NN and Fourier feature of frequency $\delta({\bf x}) = \left[{\bf x}, \sin(2^i\pi {\bf x}),\cos(2^i\pi {\bf x})\right]$. The activation function used is $\sigma(x) = \sqrt{x^2+\rho^2}$ with $\rho = 0.1$, $\varepsilon = 0.1/16$, $\eta = 1.0\times 10^{-4}$ after $300000$ epochs.}
	\label{fig:2D_problem_FF_eps=0.1by16}
\end{figure}


\begin{figure}[ht!]
	\centering
	\subfigure[no FF]{\includegraphics[width=0.31\textwidth]{Figure_10a.png}\label{subfig:uy_no_FF_eps=0.1by4}}
	\subfigure[$i = 1$]{\includegraphics[width=0.31\textwidth]{Figure_10b.png}\label{subfig:uy_FF=1_eps=0.1by4}}
	\subfigure[$i = 2$]{\includegraphics[width=0.31\textwidth]{Figure_10c.png}\label{subfig:uy_FF=2_eps=0.1by4}}
	\subfigure[$i = 3$]{\includegraphics[width=0.31\textwidth]{Figure_10d.png}\label{subfig:uy_FF=3_eps=0.1by4}}
	\subfigure[$i = 4$]{\includegraphics[width=0.31\textwidth]{Figure_10e.png}\label{subfig:uy_FF=4_eps=0.1by4}}
	\caption{DRM approximation to~\eqref{eqn:2D_Problem} with $3 \times 128$ NN and Fourier feature of frequency $\delta({\bf x}) = \left[{\bf x}, \sin(2^i\pi {\bf x}),\cos(2^i\pi {\bf x})\right]$ with $\varepsilon = 0.1/4$, $\eta = 1.0\times 10^{-4}$ after $300000$ epochs.}
	\label{fig:2D_problem_FF_eps=0.1by4}
\end{figure}

\section{Conclusions} 
\label{sec:Concl}
This work employs DRM in conjunction with Fourier feature mapping (DRM\&FM) to solve non-convex minimization problems relevant in microstructure applications. We consider three benchmark problems: two minimization problems in $1D$ given by~\eqref{eqn:1D_Problem_1} and~\eqref{eqn:1D_Problem_2} and one in $2D$ given by~\eqref{eqn:2D_Problem}. These problems are challenging to solve since they often do not possess a global minimum (see~\eqref{eqn:1D_Problem_2} \& \eqref{eqn:2D_Problem}) or a global minimum exists (as in~\eqref{eqn:1D_Problem_1}), but there exist multiple functions that can yield such minimum.   

To tackle these challenges, we employ DRM in conjunction with Fourier feature mapping to generate high frequency, multiscale solutions. The method uses a DNN comprised of an input layer, a Fourier feature mapping of the form $\delta({\bf x}) =\left[{\bf x}, \sin(2^i\pi {\bf x}),\cos(2^i\pi {\bf x})\right]$, multiple hidden layers and an output layer.  Utilizing NTK theory, we demonstrate that the DRM as implemented in~\cite{Chen&Rosakis2023} suffers from spectral bias pathology: the rate at which the DNN learns minimizing solutions is determined by the largest eigenvalue of the NTK where $\lambda_i\gg 0$. To explore multiple solutions effectively, a desirable NTK should have eigenvalues $\lambda_i \approx 0$ to avoid spectral bias pathology.

Our heuristic analysis shows that the application of Fourier feature mapping results in a quadratic decay NTK eigenspectrum $\lambda_i \approx 0$, enabling our method DRM\&FM to generate high frequency, multiscale solutions. Simulations confirm the effectiveness of DRM\&FM in generating such solutions for all three benchmark problems. In contrast to the method proposed in~\cite{Chen&Rosakis2023}, simply increasing the depth of the neural network does not produce high-frequency solutions for our benchmark problems. However, our approach achieves this by keeping the network depth fixed and incorporating a Fourier mapping. 

While minimizing solutions may appear noisy without a regularization term, this capability still represents a significant advantage over the Finite Element Method (FEM). However, solving these types of problems remains challenging due to the rough energy landscape, which lacks well-defined minima and can hinder the algorithm’s training and solution generation. To address this issue, we considered a regularized minimization problem in 2D. We observed that incorporating a regularization term (Sec.~\ref{sec:2D_problem_epsilon}) smooths the energy landscape, facilitates training and produces symmetric, smooth solutions for small values of $\varepsilon$. As the value of $\varepsilon$ increases, we observe that the solutions generated by the method are low-frequency solutions.  

While DRM with Fourier mapping presents a mesh-free and computationally efficient algorithm, its nonlinear nature lacks a theoretical foundation to quantify solution accuracy for the considered minimization problems. We encourage the research community to develop such a theory in the near future.





\section{Acknowledgment}
Research was sponsored by the Army Research Laboratory and was accomplished under
Cooperative Agreements Number W911NF-22-2-0090 and W911NF-23-2-0139. The views and conclusions contained in this
document are those of the authors and should not be interpreted as representing the official
policies, either expressed or implied, of the Army Research Laboratory or the U.S. Government.
The U.S. Government is authorized to reproduce and distribute reprints for Government purposes
notwithstanding any copyright notation herein.

%%%%%%%%%%%%%%%%%%%%%%%%%%%%%%%%%%%%%%%%%%%%%%%%%%%%%%%%%%%%%%%%%%%%%%%%%%%%%%%%%%%%%%%%%%%%%%%%%%%%%%%%%%%%
%% The Appendices part is started with the command \appendix;
%% appendix sections are then done as normal sections
\appendix
\section{Derivation of the gradient dynamics}\label{app:gradient-evolution}
We provide the detailed derivation of the gradient dynamics~\eqref{eqn:gradient-evolution}.
Recall the notations
\[
\bar{U}_n(\theta) = [\bar{u}_n(x_n; \theta), \bar{u}_n^{\prime}(x_n; \theta)]^{\top} \in \mathbb{R}^{2\times 1},
\]
\[
\bar{U}_{\mathcal{X}}(\theta) = [\bar{U}_1(\theta), \ldots, \bar{U}_{|\mathcal{X}|}(\theta)]^{\top} \in \mathbb{R}^{ 2|\mathcal{X}| \times 1},
\]
\[
\bar{W}_n(\theta) =
W(x, \bar{u}({\bf x}; \theta),  \bar{u}^{\prime}({\bf x}; \theta)) \in \mathbb{R},\]
\[
\bar{W}_{\mathcal{X}}(\theta) = [\bar{W}_1(\theta), \ldots, \bar{W}_{|\mathcal{X}|}(\theta)] \in \mathbb{R}^{|\mathcal{X}| \times 1},
\]
\[ \nabla_U \bar{W}_n(\theta) = [\partial_{\hat{u}} \bar{W}_n(\theta), \partial_{\hat{u}^{\prime}} \bar{W}_n(\theta)]^{\top}
\in 
\mathbb{R}^{2 \times 1},\]
\[
\nabla_U \bar{W}_{\mathcal{X}}(\theta) = [\nabla_U \bar{W}_1(\theta), \ldots, \nabla_U \bar{W}_{|\mathcal{X}|}(\theta)]^{\top} \in \mathbb{R}^{2|\mathcal{X}| \times 1}.
\]
A simple application of the chain rule leads to 
\begin{equation}
	\begin{split}
		\frac{d[\nabla_{U} \bar{W}_{\mathcal{X}}(\theta(t))]}{dt}
		=
		\begin{bmatrix}
			\frac{d\partial_u \bar{W}_1(\theta(t))}{dt} \\
			\frac{d\partial_{u^{\prime}} \bar{W}_1(\theta(t))}{dt} \\
			\vdots\\
			\frac{d\partial_u \bar{W}_{|\mathcal{X}|}(\theta(t))}{dt} \\
			\frac{d\partial_{u^{\prime}} \bar{W}_{|\mathcal{X}|}(\theta(t))}{dt} 
		\end{bmatrix}
		=
		\begin{bmatrix}
			\partial_{uu}^2 \bar{W}_1 \frac{d \bar{u}_1(\theta(t))}{dt} + \partial_{uu^{\prime}}^2 \bar{W}_1 \frac{d \bar{u}_1^{\prime}(\theta(t))}{dt}\\
			\partial_{u^{\prime}u}^2 \bar{W}_1 \frac{d \bar{u}_1(\theta(t))}{dt} + \partial_{u^{\prime}u^{\prime}}^2 \bar{W}_1 \frac{d \bar{u}_1^{\prime}(\theta(t))}{dt}\\
			\vdots\\
			\partial_{uu}^2 \bar{W}_{|\mathcal{X}|} \frac{d \bar{u}_{|\mathcal{X}|}(\theta(t))}{dt} + \partial_{uu^{\prime}}^2 \bar{W}_{|\mathcal{X}|} \frac{d \bar{u}_{|\mathcal{X}|}^{\prime}(\theta(t))}{dt}\\
			\partial_{u^{\prime}u}^2 \bar{W}_{|\mathcal{X}|} \frac{d \bar{u}_{|\mathcal{X}|}(\theta(t))}{dt} + \partial_{u^{\prime}u^{\prime}}^2 \bar{W}_{|\mathcal{X}|} \frac{d \bar{u}_{|\mathcal{X}|}^{\prime}(\theta(t))}{dt}
		\end{bmatrix} 
		=
		D_{\mathcal{X}}(\theta(t)) \frac{d\bar{U}_{\mathcal{X}}(\theta(t))}{dt},
	\end{split}
\end{equation}
where
\begin{equation*}
	D_{\mathcal{X}}(\theta(t)) = \text{diag}\left( \begin{bmatrix}
		\partial_{uu}^2 \bar{W}_n & \partial_{uu^{\prime}}^2 \bar{W}_n \\
		\partial_{u^{\prime}u}^2 \bar{W}_n & \partial_{u^{\prime}u^{\prime}}^2 \bar{W}_n
	\end{bmatrix}    \right)_{n=1, \ldots, |\mathcal{X}|} \in \mathbb{R}^{2|\mathcal{X}| \times 2|\mathcal{X}|}.
\end{equation*}
Further note that by following the same argument as for deriving~\eqref{eqn:loss-trajectory-NTK}, we have 
\begin{equation}
	\begin{split}
		\frac{d\bar{U}_{\mathcal{X}}(\theta(t))}{dt}
		=
		\nabla_{\theta} \bar{U}_{\mathcal{X}}(\theta(t)) \frac{d\theta(t)}{dt}
		=
		-\frac{\eta}{|\mathcal{X}|} M_{\mathcal{X}} \nabla_{U} \bar{W}_{\mathcal{X}}(\theta(t)).
	\end{split}
\end{equation}
We obtain the desired dynamics~\eqref{eqn:gradient-evolution} for
$\nabla_{U} \bar{W}_{\mathcal{X}}(\theta(t))$.

\section{The eigenspectrum of the Gram matrix}\label{app:eigenspectrum}
Let $K: D \times D \to \mathbb{R}$ be a symmetric positive definite kernel and 
define the Hilbert Schmidt integral operator 
\[
\mathcal{L}u(x) \triangleq \int_D K(x, x^{\prime}) u(x^{\prime}) \, dx.
\]
Given a dataset $\mathcal{X} = \{x_1, \ldots, x_n\} \subset D$ that is uniformly sampled over $D$, the Gram matrix induced by $K$, i.e., 
\[
M_{\mathcal{X}} = K(\mathcal{X}, \mathcal{X}),
\]
plays a central role in various kernel based regression tasks. 
Assuming $D$ is compact and $K$ is a Mercer kernel, the integral operator $\mathcal{L}$ admits a discrete spectrum and hence the following eigenvalue problem is well defined~\cite{williams2006gaussian}, 
\begin{equation}\label{eqn:eigen-L}
	\mathcal{L}u_k = \Lambda_k u_k, \qquad k = 1, 2, \ldots,
\end{equation}
where the eigenvalues $\Lambda_1 \geq \Lambda_2 \geq \ldots > 0$ and the eigenfunctions are orthonormal, i.e.,
\[
\int_D u_i(x) u_j(x) \, dx = \delta_{ij}.
\]
Evaluating~\eqref{eqn:eigen-L} at $\mathcal{X}$ leads to 
\begin{equation}\label{eqn:eigen-LX}
	\mathcal{L}{\bf u}_k = \Lambda_k {\bf u}_k, \qquad k = 1, 2, \ldots,
\end{equation}
where ${\bf u}_k = u_k(\mathcal{X}) \in \mathbb{R}^{n \times 1}$ and $\mathcal{L}{\bf u}_k = [\mathcal{L}u_k(x_1), \ldots, \mathcal{L}u_k(x_n)]^{\top}$.
Note that the integral operator $\mathcal{L}$ can be approximated by 
\[
\mathcal{L}u(x) \approx \mathcal{L}_nu(x) \triangleq \frac{1}{n} \sum_{i = 1}^n K(x, x_i)u(x_i)
\]
and hence we can approximately (for $n$ large) consider the 
eigenvalue problem 
\begin{equation}\label{eqn:eigen-Ln}
	\mathcal{L}_n u_k = \hat{\Lambda}_k u_k, \qquad k = 1, 2, \ldots, n,
\end{equation}
where $\hat{\Lambda}_k \approx \Lambda_k$ depends on the sample size $n$.
Evaluating the above equation at $\mathcal{X}$ leads to 
\begin{equation}\label{eqn:eigen-M}
	M_{\mathcal{X}} {\bf u}_k = n \hat{\Lambda}_k {\bf u}_k,\qquad k = 1, \ldots, n,
\end{equation}
where $\lambda_k \triangleq n \hat{\Lambda}_k$ is the $k$-th eigenvalue for the Gram matrix $M_{\mathcal{X}}$. 
Comparing~\eqref{eqn:eigen-LX} with~\eqref{eqn:eigen-M} leads to the connection between the eigenvalue of $\mathcal{L}$ and the eigenvalue of the Gram matrix $M_{\mathcal{X}}$,
\[
\Lambda_k = \lim_{n \to \infty} \frac{\lambda_k}{n}.
\]
Therefore, for large values of $n$, we have the approximation $\lambda_k \approx n \Lambda_k$ for $k = 1, \ldots, n$.


%% \label{}

%% If you have bibdatabase file and want bibtex to generate the
%% bibitems, please use
%\bibliographystyle{spmpsci}
\bibliographystyle{elsarticle-num}
%\bibliographystyle{unsrt}  
\bibliography{bibliography}
%%%
%% This is file `sample-sigconf.tex',
%% generated with the docstrip utility.
%%
%% The original source files were:
%%
%% samples.dtx  (with options: `all,proceedings,bibtex,sigconf')
%% 
%% IMPORTANT NOTICE:
%% 
%% For the copyright see the source file.
%% 
%% Any modified versions of this file must be renamed
%% with new filenames distinct from sample-sigconf.tex.
%% 
%% For distribution of the original source see the terms
%% for copying and modification in the file samples.dtx.
%% 
%% This generated file may be distributed as long as the
%% original source files, as listed above, are part of the
%% same distribution. (The sources need not necessarily be
%% in the same archive or directory.)
%%
%%
%% Commands for TeXCount
%TC:macro \cite [option:text,text]
%TC:macro \citep [option:text,text]
%TC:macro \citet [option:text,text]
%TC:envir table 0 1
%TC:envir table* 0 1
%TC:envir tabular [ignore] word
%TC:envir displaymath 0 word
%TC:envir math 0 word
%TC:envir comment 0 0
%%
%% The first command in your LaTeX source must be the \documentclass
%% command.
%%
%% For submission and review of your manuscript please change the
%% command to \documentclass[manuscript, screen, review]{acmart}.
%%
%% When submitting camera ready or to TAPS, please change the command
%% to \documentclass[sigconf]{acmart} or whichever template is required
%% for your publication.
%%
%%
\documentclass[sigconf]{acmart}
\usepackage{titlesec}
%%
%% \BibTeX command to typeset BibTeX logo in the docs
\AtBeginDocument{%
  \providecommand\BibTeX{{%
    Bib\TeX}}}

%% Rights management information.  This information is sent to you
%% when you complete the rights form.  These commands have SAMPLE
%% values in them; it is your responsibility as an author to replace
%% the commands and values with those provided to you when you
%% complete the rights form.
% \setcopyright{acmlicensed}
% \copyrightyear{2025}
% \acmYear{2025}
% \acmDOI{XXXXXXX.XXXXXXX}
\setcopyright{none}
%% These commands are for a PROCEEDINGS abstract or paper.

% \acmConference[CHI'25 Workshop: Tools for Thought]{2025}{Yokohama, Japan}
\acmConference[CHI'25 Workshop on Tools for Thought]{Tools for Thought: Research and Design for Understanding, Protecting, and Augmenting Human Cognition with Generative AI on CHI 2025 Workshop}{April 26,2025}{Yokohama, JAPAN}
  
\settopmatter{printacmref=false}
\renewcommand\footnotetextcopyrightpermission[1]{}



%%
%%  Uncomment \acmBooktitle if the title of the proceedings is different
%%  from ``Proceedings of ...''!
%%
%%\acmBooktitle{Woodstock '18: ACM Symposium on Neural Gaze Detection,
%%  June 03--05, 2018, Woodstock, NY}
\acmISBN{978-1-4503-XXXX-X/2018/06}


%%
%% Submission ID.
%% Use this when submitting an article to a sponsored event. You'll
%% receive a unique submission ID from the organizers
%% of the event, and this ID should be used as the parameter to this command.
%%\acmSubmissionID{123-A56-BU3}

%%
%% For managing citations, it is recommended to use bibliography
%% files in BibTeX format.
%%
%% You can then either use BibTeX with the ACM-Reference-Format style,
%% or BibLaTeX with the acmnumeric or acmauthoryear sytles, that include
%% support for advanced citation of software artefact from the
%% biblatex-software package, also separately available on CTAN.
%%
%% Look at the sample-*-biblatex.tex files for templates showcasing
%% the biblatex styles.
%%

%%
%% The majority of ACM publications use numbered citations and
%% references.  The command \citestyle{authoryear} switches to the
%% "author year" style.
%%
%% If you are preparing content for an event
%% sponsored by ACM SIGGRAPH, you must use the "author year" style of
%% citations and references.
%% Uncommenting
%% the next command will enable that style.
%%\citestyle{acmauthoryear}

\usepackage{booktabs}
\usepackage{multirow}
% \usepackage[table]{xcolor}
\usepackage{array}
\usepackage{geometry}

%%
%% end of the preamble, start of the body of the document source.
\begin{document}

%%
%% The "title" command has an optional parameter,
%% allowing the author to define a "short title" to be used in page headers.
\title[Beyond Tools]{Beyond Tools: Understanding How Heavy Users Integrate LLMs into Everyday Tasks and Decision-Making}

%%
%% The "author" command and its associated commands are used to define
%% the authors and their affiliations.
%% Of note is the shared affiliation of the first two authors, and the
%% "authornote" and "authornotemark" commands
%% used to denote shared contribution to the research.
\author{Eunhye Kim}
\email{gracekim027@snu.ac.kr}
\affiliation{%
  \institution{Seoul National University}
  \city{Seoul}
  \country{Republic of Korea}
}

\author{Kiroong Choe}
\email{krchoe@hcil.snu.ac.kr}
\affiliation{%
  \institution{Seoul National University}
  \city{Seoul}
  \country{Republic of Korea}
}

\author{Minju Yoo}
\email{minjuu613@ewhain.net}
\affiliation{%
  \institution{Ewha Womans University}
  \city{Seoul}
  \country{Republic of Korea}
}

\author{Sadat Shams Chowdhury}
\email{sadatshams@kaist.ac.kr}
\affiliation{%
  \institution{School of Computing, KAIST}
  \city{Daejeon}
  \country{Republic of Korea}
}

\author{Jinwook Seo}
\email{jseo@hcil.snu.ac.kr}
\affiliation{%
  \institution{Seoul National University}
  \city{Seoul}
  \country{Republic of Korea}
}

%%
%% By default, the full list of authors will be used in the page
%% headers. Often, this list is too long, and will overlap
%% other information printed in the page headers. This command allows
%% the author to define a more concise list
%% of authors' names for this purpose.
\renewcommand{\shortauthors}{Kim et al.}

%%
%% The abstract is a short summary of the work to be presented in the
%% article.
\begin{abstract}
  \begin{abstract}

We introduce \ours, a novel framework for scene-level appearance transfer from a single style image to a real-world scene represented by multiple views. The method combines explicit semantic correspondences with multi-view consistency to achieve precise and coherent stylization.
Unlike conventional stylization methods that apply a reference style globally, \ours uses open-vocabulary segmentation to establish dense, instance-level correspondences between the style and real-world images. This ensures that each object is stylized with semantically matched textures.
\ours first transfers the style to a single view using a training-free semantic-attention mechanism in a diffusion model.
It then lifts the stylization to additional views via a learned warp-and-refine network guided by monocular depth and pixel-wise correspondences.
Experiments show that \ours consistently outperforms prior methods in structure preservation, perceptual style similarity, and multi-view coherence.
User studies further validate its ability to produce photo-realistic, semantically faithful results.
Our code, pretrained models, and dataset will be publicly released, to support new applications in interior design, virtual staging, and 3D-consistent stylization.

\end{abstract}

\end{abstract}

%%
%% The code below is generated by the tool at http://dl.acm.org/ccs.cfm.
%% Please copy and paste the code instead of the example below.
%%
\begin{CCSXML}
<ccs2012>
   <concept>
       <concept_id>10003120.10003121.10011748</concept_id>
       <concept_desc>Human-centered computing~Empirical studies in HCI</concept_desc>
       <concept_significance>500</concept_significance>
       </concept>
 </ccs2012>
\end{CCSXML}

\ccsdesc[500]{Human-centered computing~Empirical studies in HCI}

%%
%% Keywords. The author(s) should pick words that accurately describe
%% the work being presented. Separate the keywords with commas.
\keywords{Decision-Making, AI Delegation, Qualitative Study}
%% A "teaser" image appears between the author and affiliation
%% information and the body of the document, and typically spans the
%% page.
% \received{20 February 2007}
% \received[revised]{12 March 2009}
% \received[accepted]{5 June 2009}

%%
%% This command processes the author and affiliation and title
%% information and builds the first part of the formatted document.
\maketitle
\section{Introduction}

\begin{figure*}[t!]
    \centering
    \includegraphics[width=0.7\textwidth]{./Comparison.pdf}
    \caption{Comparison between conventional wireless system (left) and PASS (right).}
    \label{comparison}
    \vspace{-0.5cm}
\end{figure*} 

\section{Introduction} \label{sec:intro}

\IEEEPARstart{S}INCE Marconi demonstrated the feasibility of wireless communication in the late 19th century, the technology has undergone significant evolution and remarkable transformations. To address the unpredictable and dynamic nature of wireless channels, numerous advancements have been made in the air interface design, channel coding, source compression, and communication protocols for improving data rates and enhancing reliability. Among these advancements, multiple-input multiple-output (MIMO) has been one of the most important evolutionary techniques for wireless communication over the past few decades. By exploiting antenna arrays, MIMO brings about multiple benefits, such as enhanced signal strength through beamforming, mitigation of multi-path fading, and efficient spatial-domain multiplexing of users~\cite{bjornson2023twenty}. Since the advent of the third generation (3G) system, MIMO has been a fundamental component of wireless communication standards. However, during that era, the size of antenna arrays in MIMO systems was generally limited. The breakthrough came when Marzetta demonstrated the significant benefits of deploying an infinite number of antennas in 2010~\cite{marzetta2010noncooperative}, revealing the potential of MIMO to enhance communication performance while reducing system complexity. This revelation paved the way for the concept of massive MIMO, i.e., employing large-scale antenna arrays at base stations. Over time, massive MIMO has evolved into a key research focus and has become a reality with the deployment of 5G networks. 


However, massive MIMO has faced numerous challenges, as it is expected to transition from “Massive” in 5G (typically with 32-64 antennas) to “Gigantic” in 6G~\cite{Xtext, bjornson2024enabling}, where the number of antennas is expected to scale to hundreds or even thousands. One of the key obstacles is the complexity and cost of implementing massive MIMO since each antenna typically needs to be fed by a dedicated radio-frequency (RF) chain. Exploiting low-resolution analog-to-digital converters in RF chains or hybrid analog-digital antenna arrays with a limited number of RF chains were common methods to address this challenge, especially in the millimeter-wave band~\cite{heath2016overview}. More recently, advancements in metamaterials have paved the way for new antenna technologies, exemplified by waveguide-fed metasurface antennas~\cite{smith2017analysis, shlezinger2021dynamic, di2024reconfigurable}, which facilitate the ultra-dense deployment of antenna elements at a significantly lower cost and making massive MIMO implementation more feasible.

Flexible-antenna technique is a new evolution of MIMO. Unlike massive MIMO focusing on enlarging the wireless channel dimension, the flexible-antenna technique focuses on enabling the reconfiguration of the wireless channel. One of the most well-known approaches in this domain is the reconfigurable intelligent surface (RIS) technique~\cite{huang2019reconfigurable, wu2019intelligent, mu2021simultaneously}. By deploying RIS between transceivers, the wireless channel can be intelligently reconfigured by adjusting the phase shifts of the signals reflected/refracted by the RIS. More recently, fluid antennas~\cite{new2024tutorial} and movable antennas~\cite{zhu2023movable} have emerged as promising flexible-antenna technologies. The fundamental concept behind these approaches is to implement antenna arrays where individual antenna elements can dynamically adjust their positions within a spatial region, thus creating favorable channel conditions to enhance communication performance. 

Nevertheless, as shown on the left of Fig. \ref{comparison}, both massive MIMO and flexible-antenna techniques have limited capability in fundamentally addressing free-space pathloss and line-of-sight (LoS) blockage, two major causes of signal attenuation in wireless communications. While massive MIMO can achieve high beamforming gains to strengthen signals, it cannot combat LoS blockage and to effectively mitigate free-space pathloss, particularly for cell-edge users. RISs have been considered as a promising solution to overcome LoS blockage by creating virtual LoS paths. However, the double fading effect caused by signal reflection results in much higher pathloss compared to a direct LoS channel~\cite{ozdogan2019intelligent}. Additionally, fluid and movable antennas are typically capable of adjusting their positions only within a few wavelengths, making them more effective for mitigating small-scale fading rather than addressing large-scale pathloss. It is worthy to point out that all the aforementioned MIMO systems are lack of antenna array reconfigurability, i.e., once an antenna array is built, adding or removing antennas is no longer possible.

Pinching-Antenna SyStem (PASS) is a revolutionary technique for addressing the challenges of free-space pathloss and LoS blockage encountered by conventional multi-antenna technologies. This technique was originally proposed and prototyped by NTT DOCOMO in 2022~\cite{suzuki2022pinching}. As illustrated on the right of Fig. \ref{comparison}, PASS employs a dielectric waveguide as its primary transmission medium, which is known for its exceptionally low propagation loss (e.g., 0.01 dB/m \cite{pozar2021microwave}). By pinching a small separated dielectric element, referred to as a \emph{pinching antenna}, onto the waveguide, the system enables signal emission from the waveguide into the pinching antenna, which then radiates the signal into free space. Building on this principle, waveguides can be pre-deployed to extend service coverage, allowing pinching antennas to be placed at positions close to users. This strategic placement transforms the wireless system into a \emph{near-wired} system and hence establishes strong LoS links with users, effectively minimizing free-space path loss and mitigating blockage issues. Additionally, unlike existing MIMO systems, PASS allows both the number and positions of pinching antennas to be easily adjusted by simply pinching them to or releasing them from the waveguide~\cite{suzuki2022pinching}. This feature provides a low-cost and scalable approach to implementing MIMO while also facilitating the so-called \emph{pinching beamforming}, which enhances communication performance by dynamically optimizing antenna positions \cite{liu2025pinching}.

Given the successful prototyping of PASS by NTT DOCOMO, theoretical research on this topic has been steadily growing, though it remains in its early stages. The first theoretical study on PASS for the communication system design was presented in \cite{ding2024flexible}, where the authors provided a comprehensive analysis and developed low-complexity pinching beamforming designs for fundamental single-user and two-user scenarios. The array gain achieved by multiple pinching antennas on a waveguide was analyzed in \cite{ouyang2025array}, unveiling the optimal number of antennas and their spacing for maximizing the beamforming gain. 
% The authors of \cite{tegos2024minimum} studied an uplink PASS system and proposed an iterative antenna position optimization algorithm to maximize the sum rate under perfect phase alignment conditions. In \cite{wang2024antenna}, the authors investigated a downlink PASS system and introduced a matching theory-based optimization method for activating pinching antennas at preconfigured discrete positions. Their findings also highlighted the advantages of using non-orthogonal multiple access (NOMA) in PASS. Expanding on this,
The authors of \cite{bereyhi2025downlink} explored a downlink PASS architecture utilizing multiple waveguides, each equipped with a single pinching antenna, and proposed a greedy approach for jointly optimizing the transmit and pinching beamforming. Meanwhile, \cite{guo2025deep} examined a more generalized scenario, where multiple pinching antennas were deployed on each waveguide, and introduced a graph neural network (GNN)-based deep learning method to address the corresponding joint beamforming optimization problem.

Although PASS has attracted growing attention, several key challenges remain unsolved. On the one hand, the physics modeling of PASS is still underdeveloped, which is crucial for establishing an accurate signal model. In existing studies \cite{ding2024flexible, ouyang2025array, bereyhi2025downlink, guo2025deep}, it is commonly assumed that all signal power within the waveguide is fully radiated into free space and that each pinching antenna on a waveguide emits identical radiation power—an assumption analogous to conventional MIMO systems. However, pinching antennas operate fundamentally differently from traditional electronic antennas, and such assumptions may lack a solid physical foundation and fail to accurately reflect real-world behaviors. On the other hand, most existing works design PASS under simplified assumptions \cite{ding2024flexible, ouyang2025array, bereyhi2025downlink}, such as a single user, a single waveguide, a single pinching antenna per waveguide, or perfectly aligned signal phases. Although the GNN-based deep learning model proposed in \cite{guo2025deep} is capable of handling more complex scenarios with arbitrary numbers of users, waveguides, and pinching antennas, it suffers from a key limitation: the model parameters need to be retrained once the system configuration changes, limiting its generalization ability. Motivated by these challenges, this paper aims to develop a fundamental physics-based signal model for PASS and explore joint beamforming designs for more general scenarios. The key contributions of this work are summarized as follows:
\begin{itemize}
    \item We propose a physics-based hardware model for PASS, in which a pinching antenna is modeled as an open-ended directional waveguide coupler to facilitate the adjustment of radiation characteristics and simplify signal modeling. Based on this model, we characterize the relationship between the electromagnetic (EM) fields within the waveguide and those radiated by the pinching antennas using coupled-mode theory.
    \item We derive a novel signal model for PASS based on the proposed physics framework, revealing the inherent coupling effect between the radiation power of pinching antennas deployed on the same waveguide. Leveraging this coupling relationship, we introduce two simplified power models and their respective implementation methods: equal power and proportional power models.
    \item We formulate a joint transmit and pinching beamforming optimization problem to minimize the transmit power in a general PASS system with arbitrary numbers of users, waveguides, and pinching antennas, considering both continuous and discrete activation of pinching antennas. To solve this highly nonconvex, coupled, and multimodal optimization problem, we propose two algorithms: the penalty-based alternating optimization algorithm and the zero-forcing (ZF)-based low-complexity algorithm.
    \item We provide comprehensive numerical results to validate the advantages of PASS and the effectiveness of the proposed algorithm. The results demonstrate that 1) the ZF-based algorithm delivers performance comparable to the penalty-based algorithm but has a low complexity, 2) PASS significantly reduces transmit power, achieving a reduction of over 95\% compared to conventional and massive MIMO, 3) a dense set of available antenna positions is required for discrete activation to achieve similar performance to continuous activation, and 4) the proportional power model exhibits performance comparable to the equal power model.
\end{itemize}

The rest of this paper is structured as follows. Section \ref{sec:model} introduces the proposed physics-based hardware model and signal model for PASS. Section \ref{sec:beamforing} presents the general system model for downlink PASS and introduces a penalty-based alternating optimization method and a ZF-based algorithm for solving the joint beamforming optimization problem. Numerical evaluations and performance comparisons under various system configurations are presented in Section \ref{sec:results}. Finally, Section \ref{sec:conclusion} summarizes the findings and concludes the paper.


\emph{Notations:} Scalars are denoted using regular typeface, vectors and matrices are represented in boldface, and Euclidean subspaces are indicated with calligraphic letters. The set of complex and real numbers are denoted by $\mathbb{C}$ and $\mathbb{R}$, respectively. The inverse, conjugate, transpose, conjugate transpose, and trace operators are denoted by $(\cdot)^{-1}$, $(\cdot)^*$, $(\cdot)^T$, $(\cdot)^H$, and $\mathrm{tr}(\cdot)$, respectively. The absolute value, Euclidean norm, Frobenius norm, and maximum norm are denoted by $|\cdot|$, $\|\cdot\|$, $\|\cdot\|_F$, and $\|\cdot\|_\infty$ respectively. The real part of a complex number of demoted by $\Re \{\cdot\}$. The entry in the $n$-th row and $m$-th column of a matrix $\mathbf{X}$ is denoted by $[\mathbf{X}]_{n,m}$. An identity matrix of dimension $N \times N$ is denoted by $\mathbf{I}_N$. The big-O notation is given by $O(\cdot)$. A diagonal matrix with diagonal entries $x_1,\dots,x_N$ is denoted as $\mathrm{diag}(x_1,\dots,x_N)$.    




% \begin{figure*}[t!]
% \centering
% \begin{subfigure}[t]{0.48\textwidth}
%     \centering
%     \includegraphics[height=0.5\textwidth]{./Comparison_conventional.pdf}
% \end{subfigure}
% \hspace{-1.5cm}
% \begin{subfigure}[t]{0.48\textwidth}
%     \centering
%     \includegraphics[height=0.5\textwidth]{./Comparison_PASSpdf.pdf}
% \end{subfigure}
% \caption{Comparison between conventional wireless system (left) and PASS (right).}
% \end{figure*} 


\section{Methods}
\section{Methods}
\label{sec:methods}

We conducted interviews with mental health clinicians to explore how they would design health information technologies (HITs) that support value-based mental healthcare.
Methodologically, we were inspired by work in speculative design to imagine futures where VBC is mandated, and then brainstorm with participants how HITs could support VBC outcomes data storage, collection, and use \cite{hockenhull_speculative_2021, wong_speculative_2018}. 
In this section, we detail the study procedures, including participant recruitment (Section \ref{sec:methods:participants}), background information (Section \ref{sec:methods:participants-backgrounds}), how data was collected and analyzed (Section \ref{sec:methods:data}), and our positionality (Section \ref{sec:methods:positionality}). 
All study procedures were approved by the coauthors' institutional review board (IRB). 

\subsection{Participant Recruitment}
\label{sec:methods:participants}
We enrolled as participants mental health clinicians, specifically practicing psychiatrists, clinical psychologists, licensed clinical social workers (LCSWs), and licensed mental health counselors (LMHCs).
We intentionally recruited providers from these different clinical orientations to gather different perspectives on designing HITs \cite{mental_health_america_types_2024}. 
Participants were recruited via a combination of convenience, purposive, and snowball sampling \cite{etikan_comparison_2015, goodman_snowball_1961}.
Specifically, a recruitment email and flier were sent to staff working at academic medical centers across the United States. 
\rev{Recruitment emails were often forwarded to providers who worked in smaller, private practices or community health settings, to help us gain perspectives from mental health clinicians working in diverse settings, treating different types of patients.} 
Within the qualitative tradition \cite{braun_one_2021}, our goal for this work was not to gather perspectives representative of mental health clinicians as a whole, but instead to deep dive with our participants into the complexities of designing HITs that support VBC.

\subsection{Participants' Backgrounds}
\label{sec:methods:participants-backgrounds}

\rev{Table \ref{tab:participants} summarizes background information for the 30 mental health clinicians who participated in the study.
This background information was collected during an intake survey, which was administered after participants provided informed consent for our study.
Apart from data collected within this intake survey, we often asked participants during our study interviews to provide background information regarding their current payment arrangements.
Most of our participants took traditional, fee-for-service payments (public and private), or asked their private practice patients to pay for care out-of-pocket.
A few participants (eg, SW28) worked in health systems transitioning to value-based payments.
Many participants were unfamiliar with VBC.
}

\begin{table*}[t]
\begin{tabular}{ll}
\toprule
Number of participants & 30 mental health clinicians \\ 
\midrule
Clinical training           & 13 Clinical Psychology \rev{(CP)} \\
                            & 6 Psychiatry \rev{(PS)} \\    
                            & 8 Clinical Social Work \rev{(SW)} \\
                            & 2 Mental Health Counseling \rev{(MC)} \\
                            & 1 Family and Marriage Therapist \rev{(FT)} \\
\midrule
Practice setting            & 16 Academic Medical Center \\
                            & 14 Private Practice \\
                            & 5 Community Mental Health Center \\
                            & 2 Employee Assistance Program \\
\midrule
Geographic location (in the USA)    & 26 Northeast \\
                                    & 2 Southeast \\
                                    & 2 West Coast \\
\bottomrule
\end{tabular}
\caption{Background information of the study participants. Participants could list multiple practice settings.
\rev{Clinical training abbreviations (eg, ``CP'') are used within Section \ref{sec:findings}.}
}
% \Description{A table summarizing the backgrounds of the 30 participants we interviewed in this study. The table describes the clinical training of participants, the practice setting, and geographic location (in the United States) of each participant.}
\label{tab:participants}
\vspace{-5pt}
\end{table*}

\subsection{Data Collection and Analysis}
\label{sec:methods:data}

All participants were asked to provide informed consent after being provided complete information about the study procedures.
Interviews were held via Zoom over two 1-hour sessions attended by the first three authors, and participants were reimbursed \$30 per hour for their time.
The first session was a semi-structured interview where we asked clinicians about their current care practices, specifically how they used data -- defined broadly, collected with or without technology -- in care.
We specifically asked participants about their perspectives on \textit{measurement-based care} (MBC), the practice of collecting and using data in care that would power HITs supporting VBC \cite{kilbourne_measuring_2018}.
We then asked participants further questions about how they used this data to measure care outcomes, how technology was involved in this process, and whether providers were accountable to achieve certain care outcomes.
Interview questions were broad to allow for on-the-spot adaptation and probing \cite{barriball_collecting_1994}.

In the second session, participants completed two design prompts.
These prompts were motivated by work in speculative design \cite{hockenhull_speculative_2021, wong_influence_2008}, to imagine futures where MBC and VBC were mandated and to understand how clinicians would collect and report outcomes data as a part of these programs.
The first prompt asked participating clinicians to imagine a world where they were mandated to use outcomes data as a part of care, and to brainstorm what data they would prioritize.
The second prompt was motivated by the five-star quality rating system used by the United States Center for Medicare \& Medicaid services (CMS) \cite{center_for_medicare__medicaid_services_five-star_2022}.
Participating clinicians were asked to imagine that as a part of VBC, CMS wanted to design ``mental health quality star ratings'' to measure patient outcomes and care quality across clinics and health systems.
Participants were asked to brainstorm what data should be included in this new star rating program.
After responding to each prompt, we discussed with participants the data they included in their responses, and asked probing questions to further understand how HITs could support data storage, collection, and use.
Full interview guides can be found in Appendix \ref{appendix:guide}.

Interviews were recorded with participants' permission, transcribed by a professional service, and de-identified.
Transcripts were analyzed using a reflexive thematic analysis approach adopted from \cite{braun_using_2006}.
This approach combined both inductive and deductive elements.
Codes and themes arose from the data, but were guided by our research interests and the literature \cite{braun_one_2021}, specifically the stages of preparation, collection, and action from Li et al. \cite{li_stage-based_2010}.
The first author qualitatively coded all transcripts.
Codes were iteratively refined, resulting in a final codebook, and all transcripts were recoded using the final codebook.
Themes were developed from the codes by the first author, with support from the second and third authors who also participated in the interviews and validated that the themes represented participants' views.
The codebook used to generate each theme can be found in Appendix \ref{appendix:codebook}.

\subsection{Positionality}
\label{sec:methods:positionality}

The first, second, and third authors are graduate students in computer and information science. 
These authors recruited participants, collected, and analyzed all of the data. 
One author is a clinical researcher and practicing mental health clinician who worked with the first author on the study protocols, and did not participate in the study. 
Another author is a health policy researcher, who is an expert on both digital mental health and value-based care.
The final author is a researcher in computing and information science. 
All authors were based in the United States, and thus our findings and perspectives are greatly informed, and potentially limited by, our knowledge of the United States healthcare system.
\section{Results}
% \section{Simulation Evaluation \& Results}\label{sec:results}

\subsection{Baseline Planners}

To evaluate the performance of \PlannerName, we compare it against several baseline methods. In the following section, we describe these baselines, their implementation details, and their respective advantages and limitations, particularly in the context of information gathering in large, high-dimensional search spaces. The simulation framework and vehicle parameters remain consistent across all planners, and each method is allowed to replan during testing.

\subsubsection{Monte-Carlo Tree Search}

Monte Carlo Tree Search (MCTS) can be a powerful technique for finding feasible and optimal paths in complex environments. It is a heuristic search algorithm that builds a search tree incrementally through repeated simulations. At each iteration, it selects a node to explore based on a selection policy (often the Upper Confidence Bound or UCB1 algorithm), expands the tree by adding possible actions from that node, runs a simulation from the newly added node, and updates the statistics of nodes along the path traversed during the simulation. 

The UCB1 (Upper Confidence Bound) algorithm is a technique commonly used in the context of multi-armed bandit problems and Monte Carlo Tree Search (MCTS) for balancing exploration and exploitation. It helps in selecting actions or nodes that are likely to yield high rewards while also exploring less-frequented options to gather more information about their potential rewards. 

We formulate our UCB score in the following manner, \\
\begin{equation*}
    UCB_\text{node} = \frac{I(X_{\text{node}})}{\alpha} + C \times \sqrt{\frac{\ln(N_\text{tree})}{N_\text{node}}}
\end{equation*}
%  $
% UCB_\text{node} = \frac{\overline{X_\text{node}}}{\alpha} + C \times \sqrt{\frac{\ln(N_\text{tree})}{N_\text{node}}}
% $ \\
Here $I(X_{\text{node}})$ denotes the estimated information gain from the node, $\alpha$ denotes the normalization factor which is given by $\frac{B}{v_\text{desired}}$, $B$ being the maximum planning budget and $v_\text{desired}$ being the desired speed of our UAV. $C$ denotes the exploration weight, and $N_\text{tree}$ denotes the number of visits to the tree root node while $N_\text{node}$ denotes the number of times the present node has been visited.

After selecting a candidate node, if it has been visited before, it is expanded by applying motion primitives to generate child nodes, growing the tree. Unvisited nodes skip this step. Following expansion, either the unvisited candidate node or one of its children is selected for the simulation phase, where the future values of nodes along the path are estimated to update the total potential information gain. This informs the selection policy in subsequent iterations. Once planning time is exhausted, the path with the highest information gain is returned.

% with authors goes here
\begin{figure}[t]
\centering
\includegraphics[trim={.7cm 0cm .5cm 1.4cm},clip,width=\columnwidth]{figs/5_/Results1v3.pdf}
\caption{The Monte Carlo simulation results for the planners. The plots show the average percent reduction in entropy over the course of the simulations, and the shading shows the 95\% confidence intervals. IA-TIGRIS outperforms all of the baselines.}
\label{fig:mc_results}
\end{figure}

While MCTS is probabilistically guaranteed to converge to the optimal path \cite{mcts_ref_1}, it is constrained to actions within a predefined set of motion primitives. Its reliance on random sampling to estimate the future value of nodes can result in poor approximations, particularly in environments with sparse, localized pockets of high information gain. This limitation is especially pronounced in large search areas or scenarios with large budgets constraints, where estimating future node values becomes increasingly expensive. As a result, in such scenarios, MCTS is often implemented with a finite planning horizon, which can restrict its ability to account for long-term consequences or dependencies in the environment.

% This property of MCTS, which causes unguided exploration of the environment, leads to increased convergence times on the optimal path, as a result of a lot of budget being spent in exploring information sparse areas of the map. 
% Also, the computation time of MCTS increases exponentially with the depth of the search tree. The time complexity of MCTS is given by $\mathcal{O}(\frac{T}{t_\text{iter}} \cdot |A|^d)$. Here, $T$ is the total planning time and $t_\text{iter}$ is the time taken per iteration of the planning loop. $|A|$ is the number of actions and $d$ represents the average depth of the search tree. 

% The above limitations are not inconsequential in the context of performing informative path planning in large high-dimensional search spaces. We compare MCTS with \PlannerName, in \ref{}, and empirically demonstrate its drawbacks and how \PlannerName, is able to outperform MCTS in the context of the mission parameters we examine in this work.  

\subsubsection{Greedy}

For the greedy planner, we iterated through each cell within the search bounds and calculated the reward for a given cell $i$ as $g_i = R(X_i) / d_i$ where $R(X_i)$ is given through \eqref{equ:reward} and $d_i$ represents the Euclidean distance between the current position the robot at the current time $t$ and the closest viewpoint to the cell. To compute this viewpoint, the yaw between the current pose of the robot and the intersected cell is first calculated. Using the robot's sensor configuration and this yaw, $x$ and $y$ coordinates are calculated that view the cell at the desired flight altitude. With this formulation, the planner prioritizes regions with a high ratio of entropy to distance. This can lead to locally optimal choices that contradict with paths that lead to higher information gain over the entire trajectory. 

% without authors goes here
% \begin{figure}[t]
% \centering
% \includegraphics[trim={.7cm 0cm .5cm 1.4cm},clip,width=\columnwidth]{figs/5_/Results1v3.pdf}
% \caption{The Monte Carlo simulation results for the planners. The plots show the average percent reduction in entropy over the course of the simulations, and the shading shows the 95\% confidence intervals. IA-TIGRIS outperforms all of the baselines.}
% \label{fig:mc_results}
% \end{figure}


\begin{figure*}[t]
    \centering
    \begin{subfigure}[b]{0.99\textwidth}
        \centering
        \includegraphics[trim={0cm 0.3cm 0cm 0cm},clip,width=\textwidth]{figs/5_/Fig2v1_target.png}
        % \caption{Slice by targets}
        % \vspace{.1cm}
    \end{subfigure}
    
    \begin{subfigure}[b]{0.99\textwidth}
        \centering
        \includegraphics[trim={0cm 0cm 0cm 0cm},clip,width=\textwidth]{figs/5_/Fig2v1_sigma.png}
        % \caption{Slice by sigma }
    \end{subfigure}
    \caption{A comparison of the methods based on the number of sampled prior clusters and the standard deviation of sampled prior clusters. IA-TIGRIS is most effective compared to the baselines when there is high variation in the search space. As the search space prior information becomes more evenly spread out, the performance gap between the methods tends to decrease.}
    \label{fig:targets_sigmas}
\end{figure*}

\subsubsection{Random}

The random planner operates by iteratively sampling points within the defined search bounds and calculating the minimum-cost path to observe each sampled point. This process is repeated until the available budget is fully expended. The random planner does not utilize any prior information about the environment or target distribution. Additionally, it does not optimize the sequence of actions, instead treating each sampled point independently without considering the global structure of the search problem. This simplicity allows the random planner to highlight the performance benefits of more sophisticated methods by providing a lower-bound comparison for evaluation.

\subsubsection{Coverage}

The coverage planner generates a plan that systematically covers the entire search space using a straightforward lawn-mower pattern. The spacing between each pass is set to match the width of the projected observation footprint at 20\% from the bottom, ensuring that no grid cells are missed. This spacing also maintains a distance that enables high-quality sensor measurements. However, due to the size of the search spaces considered, the coverage planner spends significant time surveying empty regions. This approach results in inefficient use of the budget, as it prioritizes full coverage with safe sensor overlap, even in areas with little or no valuable information. While simple and robust, this method highlights the tradeoff between exhaustive coverage and efficient, targeted exploration.

% \subsubsection{Branch and Bound}
% The branch and bound baseline is based on motion primitive planning. In each future step the drone has a set of motion primitives with future states and each of these future states also has a set of motion primitives. In this way, a tree can be built with multiple path candidates. The path candidate with the highest information gain will be selected and form the output. 

% By adding branch and bound, there will be an estimation of a node's upper bound information reward, using the node's current information reward, updated information map and the remaining budget. If this upper bound is already lower than the information reward of any other node in the tree, the corresponding node will be closed and not expanded in the future to accelerate the expansion of the tree. 



\subsection{Tests and Analysis}
% To evaluate the efficacy of IA-TIGRIS compared to the baseline methods, we conduct Monte Carlo testing as well as analyze how the prior and budget affect the performance of each method. In all of these test cases, there are no time-based or priority rewards and have horizon lengths set to the full budget. All tests were performed using an Intel Xeon CPU E5-2620 v4 @ 2.10GHz.
To evaluate the efficacy of IA-TIGRIS against baseline methods, we perform Monte Carlo testing and analyze the impact of the prior and budget on the performance of each method. In all test cases, rewards are calculated using \eqref{equ:reward}, and horizon lengths are set to match the full budget. The tests are conducted on an Intel Xeon CPU E5-2620 v4 @ 2.10GHz, ensuring consistent computational conditions across all evaluations.

% Random sample across which parameters.

% Quantitative ideas. Look into number and std of prior (metric for this? std of grid cell values, mediuan, mean,). 
% Uniform prior? 
% Split distinct regions, not smooth. 
% Compare to coverage and amount of time to reach specific amount. 
% Compare with different budgets. 
% Repeatability test. 
% Graph size vs time. 
% Look at coverage with different altitudes or widths. Something that shows long horizon vs not nature of things?
% Shape of search space?
% Time/budget to get x\% of all info gain. Have to do moving horizon. 
% Targets detected? 

% Key thought for results where I show time, our optimization does not optimize for time, only final value. Key thing to show across the different budgets. 

% \BM{Qualitative. Nayana idea of plot with example sampled case. Should add one here.} 



\subsubsection{Monte Carlo Testing}
Our simulated testing environment is a $5000\times5000$ m square with Gaussian-distributed prior information randomly placed throughout the search space. The number of prior clusters was sampled uniformly between $[4,20]$, with standard deviations between $[60,450]$, and maximum value between $[0.05,0.5]$. 

The results of $100$ Monte Carlo tests are shown in Fig.~\ref{fig:mc_results}. IA-TIGRIS clearly outperforms the other methods, achieving nearly a $40\%$ greater reduction in entropy than the next best method. Early in the simulation, the greedy method initially gains information more quickly, as expected, but this does not translate to better long-term performance. Since our method optimizes for total information gain, it generates paths that maximize information collection over the entire budget. MCTS performed slightly worse than the greedy approach.

The random paths slightly outperformed the coverage paths. This is likely because the lawnmower strategy requires sufficient overlap between passes to avoid missing areas, and its long straight paths often lead to redundant observations due to the UAV’s forward-facing camera. Changing the heading of the UAV is beneficial to viewing more of the search space, which may explain why random paths performed better.

We also conducted Monte Carlo tests where either the number of prior clusters or their standard deviation was held constant to analyze how variations in the information map affect planner performance. The results, shown in Fig.~\ref{fig:targets_sigmas}, include two cases: the upper figure fixes the number of priors, while the lower figure fixes their standard deviation. All other agent and simulation parameters remained unchanged.


% The first thing to note from these results is that for all tests the proportional performance gap between IA-TIGRIS and the baselines increases as the number and standard deviation of the Gaussian priors decreases. As the search space becomes more uniformly filled with entropy in the information map, the need for longer-horizon planning decreases and other simple or random approaches can perform satisfactorily given the testing budget. As the information becomes more sparsely distribution in the space, such as when the information is contained in separated pockets of areas, there is a greater need to plan longer-horizon paths that reason about the given budget.
% \BM{Could have figures here or refer to others}

Across these tests, the performance gap between IA-TIGRIS and the baselines widens as the number and standard deviation of the Gaussian priors decrease. When entropy is more uniformly distributed across the search space, simpler methods perform reasonably well within the given budget. However, when information is concentrated in sparse, distinct regions, longer-horizon planning becomes essential. In such cases, IA-TIGRIS demonstrates a significant advantage by effectively reasoning about the budget and prioritizing high-value regions.

% Show plot of first plans expected info gain versus planning time. (plans not executed)


\subsubsection{Budget Analysis}
To evaluate the impact of budget constraints on performance, we conducted additional tests beyond our initial Monte Carlo experiments, evaluating budgets of $5000$ m, $10000$ m, $30000$ m, and $60000$ m. Table~\ref{tab:budgets} summarizes the average entropy reduction across these budgets.

\definecolor{tabfirst}{rgb}{1, 0.7, 0.7} % red
\definecolor{tabsecond}{rgb}{1, 0.85, 0.7} % orange
\definecolor{tabthird}{rgb}{1, 1, 0.7} % yellow
\begin{table}[t]
    \centering
    \resizebox{\linewidth}{!}{
    \begin{tabular}{l|ccccc}
    & $5000$ m & 10000 m  & 15000 m& 30000 m& 60000 m\\ \hline

    % \hline
    IA-TIGRIS  &  \cellcolor{tabfirst}$9.41\pm1.0$ &  \cellcolor{tabfirst}$18.28\pm1.8$ & \cellcolor{tabfirst}$25.36\pm2.3$ & \cellcolor{tabfirst}$41.08\pm2.9$ & \cellcolor{tabfirst}$58.85\pm2.9$ \\
    Greedy  &  \cellcolor{tabsecond}$6.99\pm0.8$ &  \cellcolor{tabsecond}$13.10\pm1.5$ & \cellcolor{tabsecond}$17.97\pm2.0$ & \cellcolor{tabthird}$30.00\pm2.3$ & \cellcolor{tabsecond}$49.38\pm3.5$ \\
    MCTS  &  \cellcolor{tabthird}$6.06\pm0.7$ &  \cellcolor{tabthird}$11.80\pm1.1$ & \cellcolor{tabthird}$17.11\pm1.4$ & \cellcolor{tabsecond}$30.21\pm2.2$ & \cellcolor{tabthird}$48.68\pm2.7$ \\
    Random  &  $2.19\pm0.3$ & $4.29\pm0.7$ & $6.61\pm0.6$ & $17.50\pm1.2$ & $22.47\pm1.4$ \\
    Coverage  &  $1.58\pm0.3$ &  $2.82\pm0.4$ & $4.09\pm0.7$ & $12.04\pm1.9$ & $16.77\pm2.4$ \\

    \end{tabular}
    }
    \caption{Monte Carlo testing results given different budgets. The values are the average percent reduction in entropy and the 95\% confidence bounds. \mbox{IA-TIGRIS} had the best performance for all budgets.}
    \label{tab:budgets}
\end{table}
%$\uparrow$ 

IA-TIGRIS consistently achieved the highest entropy reduction across all budget constraints, with a statistically significant margin over alternative methods. Greedy generally ranked second but was slightly outperformed by MCTS at the $30000$ m budget level. Greedy and MCTS exhibited comparable performance throughout the tests, with their results closely tracking each other. Consistent with our previous findings, Random and Coverage methods yielded the lowest results.


Among the tested methods, only IA-TIGRIS and MCTS explicitly incorporate budget constraints into their planning algorithms. Notably, at lower budgets ($5000$ m and $10000$ m), these methods achieved higher entropy reduction compared to the equivalent time steps ($200$ s and $400$ s) in the $15000$ m budget scenario shown in Fig.~\ref{fig:mc_results}. This improved performance stems from IA-TIGRIS's optimization of total path reward under budget constraints, contrasting with the myopic next-best-action approach of the greedy method. The remaining methods---Greedy, Random, and Coverage---maintain consistent behavior regardless of budget constraints, as their planning strategies do not account for resource limitations.


The performance gap between IA-TIGRIS and the next-best method varied with budget size, showing margins of $34.6\%$, $39.5\%$, $41.1\%$, $36.0\%$, and $19.2\%$ in ascending budget order. This gap widened through the first three budget levels as problem complexity increased, before declining significantly at higher budgets. This performance pattern suggests that implementing a planning horizon could enhance efficiency by limiting tree search depth, enabling the planner to prioritize path quality optimization over exhaustive space exploration.


% percent improved from next best
% 34.6, 39.5, 41.1, 36.0, 19.2
% reasons, too long horizon is a larger search space, so less quality paths closer. Or larger horizon, more packing in


% with authors goes here
\begin{figure}[t] 
    \centering
    \renewcommand\arraystretch{0} % Adjust the height between rows here
    \setlength{\tabcolsep}{1pt} % Adjust the column separation here
    \begin{tabular}{c}
        \begin{tikzpicture}
            \node[anchor=south west, inner sep=0] (image) at (0,0) {
                \includegraphics[width=0.9\linewidth]{figs/5_/google_earth_prior.png}
            };
            \begin{scope}[x={(image.south east)},y={(image.north west)}]
                % \fill[OrangeRed] (0.02, 0.03) circle (2pt); 
                % \fill[OrangeRed] (0.51, 0.04) circle (2pt); 
                % \fill[OrangeRed] (0.61, 0.04) arc (0:90:2pt); 
                \fill[Orange, opacity=0.8] (0.74, 0.45) circle (3pt); % Adjust 
                \fill[Orange, opacity=0.8] (0.27, 0.42) circle (3pt); % Adjust 
                \fill[Orange, opacity=0.8] (0.39, 0.63) circle (3pt); % Adjust 
            \end{scope}
        \end{tikzpicture} \\
        % \includegraphics[width=0.9\linewidth]{figs/5_/google_earth_prior.png} \\
        \\
        \includegraphics[width=0.9\linewidth]{figs/5_/google_earth_path.png} 
    \end{tabular}
    \caption{Google Earth screenshots illustrating the mission planning process and execution. Top: Areas of high entropy targeted for search are highlighted in red, representing regions with a binary occupied/unoccupied probability of 0.2. Three points of particular interest, each assigned a 0.5 probability, are marked in orange. Bottom: The executed drone flight path (yellow) shows the optimized path for maximum information gain across the search space.} 
    \label{fig:google_earth}
\end{figure}
\begin{figure}[t]
\centering
% https://docs.google.com/presentation/d/1RjI-QqHpBRLHN60UAxzmQYs4EaWaVCOoSBkEkA39kk0/edit?usp=sharing
\includegraphics[width=\columnwidth]{figs/5_/m600_labeled.jpg}
\caption{Hexarotor system (DJI M600 Pro) with onboard compute and camera. Left image shows drone on the ground, right image shows drone in flight.}
\label{fig:m600}
\end{figure}


\section{Field Deployments}\label{sec:field}


\subsection{Hexarotor Deployment}
The first field experiment that we present uses a hexarotor drone to cover an urban area shown in Fig.~\ref{fig:fig1}.
We designed this field experiment to simulate classifying where cars are within a search area.  
Hence, we set the plan request to focus on parking lots at the field test site (Fig.~\ref{fig:google_earth}, top), with the addition of three chosen grid cells within the parking lots being marked as having a higher uncertainty. The plan request boundaries and priors were created with GPS coordinates in Google Earth, exported as kml files, and then converted into our plan request message format. 

The following sections details the hardware, autonomy, and experimental results for our hexarotor deployments.

% without the authors goes here
% \begin{figure}[t] 
%     \centering
%     \renewcommand\arraystretch{0} % Adjust the height between rows here
%     \setlength{\tabcolsep}{1pt} % Adjust the column separation here
%     \begin{tabular}{c}
%         \begin{tikzpicture}
%             \node[anchor=south west, inner sep=0] (image) at (0,0) {
%                 \includegraphics[width=0.9\linewidth]{figs/5_/google_earth_prior.png}
%             };
%             \begin{scope}[x={(image.south east)},y={(image.north west)}]
%                 % \fill[OrangeRed] (0.02, 0.03) circle (2pt); 
%                 % \fill[OrangeRed] (0.51, 0.04) circle (2pt); 
%                 % \fill[OrangeRed] (0.61, 0.04) arc (0:90:2pt); 
%                 \fill[Orange, opacity=0.8] (0.74, 0.45) circle (3pt); % Adjust 
%                 \fill[Orange, opacity=0.8] (0.27, 0.42) circle (3pt); % Adjust 
%                 \fill[Orange, opacity=0.8] (0.39, 0.63) circle (3pt); % Adjust 
%             \end{scope}
%         \end{tikzpicture} \\
%         % \includegraphics[width=0.9\linewidth]{figs/5_/google_earth_prior.png} \\
%         \\
%         \includegraphics[width=0.9\linewidth]{figs/5_/google_earth_path.png} 
%     \end{tabular}
%     \caption{Google Earth screenshots illustrating the mission planning process and execution. Top: Areas of high entropy targeted for search are highlighted in red, representing regions with a binary occupied/unoccupied probability of 0.2. Three points of particular interest, each assigned a 0.5 probability, are marked in orange. Bottom: The executed drone flight path (yellow) shows the optimized path for maximum information gain across the search space.} 
%     \label{fig:google_earth}
% \end{figure}
% \begin{figure}[t]
% \centering
% % https://docs.google.com/presentation/d/1RjI-QqHpBRLHN60UAxzmQYs4EaWaVCOoSBkEkA39kk0/edit?usp=sharing
% \includegraphics[width=\columnwidth]{figs/5_/m600_labeled.jpg}
% \caption{Hexarotor system (DJI M600 Pro) with onboard compute and camera. Left image shows drone on the ground, right image shows drone in flight.}
% \label{fig:m600}
% \end{figure}

\subsubsection{Hardware System}
The hardware consists of the DJI M600 Pro, shown in Fig.~\ref{fig:m600}, along with the physical sensing and onboard computer payload. The DJI M600 Pro contains a flight controller that handles pose estimation and position-based control. The DJI M600 Pro’s flight controller also handles teleloperation if human intervention is necessary. Beneath the drone's base, we mount a custom hardware payload.
That payload consists of an onboard computer, a Jetson Xavier, to run the autonomy software shown in Fig.~\ref{fig:functional_diagram}.
The payload also contains a downward-facing a camera for sensing the environment. The camera is a Seek S304SP thermal camera.
The camera intrinsics are used to calculate the frustum's intersection with the search map's cells in IA-TIGRIS.

% without authors goes here
\begin{figure}[t]
\centering
% https://lucid.app/lucidchart/f750ddb4-2809-4773-8361-d5fbb1ba49eb/edit?viewport_loc=-257%2C-116%2C2219%2C1140%2C0_0&invitationId=inv_56e8a3a9-e8cf-4cad-a280-48bd967ff651
\includegraphics[trim={0cm 0cm 0cm 0cm},clip,width=\columnwidth]{figs/5_/functional_diagram.jpeg}
\caption{Functional diagram of the DJI M600 Pro autonomy software.}
\label{fig:functional_diagram}
\end{figure}
\begin{figure}[b]
    \centering
    \begin{subfigure}[b]{0.48\columnwidth}
        \centering
        \includegraphics[width=1.0\linewidth]{figs/5_/field_test_altitude_over_time.png}
        \caption{}
        \label{fig:m600_altitude_over_time}
    \end{subfigure}
    \begin{subfigure}[b]{0.48\columnwidth}
        \centering
        \includegraphics[width=1.0\linewidth]{figs/5_/field_test_entropy_over_time.png}
        \caption{}
        \label{fig:m600_entropy_over_time}
    \end{subfigure}
    \caption{The results for our hexarotor field deployment. (a) Plot of flown altitude over time, showing large variation throughout the experiment. (b) Reduction in entropy percentage over time of field experiment.}
\end{figure}

\subsubsection{Autonomy System}
Fig.~\ref{fig:functional_diagram} illustrates the functional system diagram for the real world field test on the DJI M600. The user specifies the initial plan request prior to takeoff. The TIGRIS planner makes an initial plan on that plan request and sends a global path to the waypoint manager. The waypoint manager tracks the current waypoint within the plan and sends the next waypoint to the DJI software development kit, which then sends actuation commands to the motors. The position of the drone is used to calculate the distance from the drone to the ground and sends that distance parameter to the sensor model. The sensor model's true positive and false positive rate is used to calculate the per-cell entropy updates in the search map manager. The search map manager publishes the current information map, and the replanning node sends an updated plan request to the IA-TIGRIS planner every ten seconds.

The drone started at an altitude of $50$ m above the origin of the reference frame. The informed sampler in IA-TIGRIS was set to add states at altitudes of either $30$ m or $60$ m, creating a trade-off between observation area and detector accuracy. The budget was $2000$ m, the planning horizon was $600$ m, and the planning time was $10$ seconds. 

% % without authors goes here
% \begin{figure}[t]
% \centering
% % https://lucid.app/lucidchart/f750ddb4-2809-4773-8361-d5fbb1ba49eb/edit?viewport_loc=-257%2C-116%2C2219%2C1140%2C0_0&invitationId=inv_56e8a3a9-e8cf-4cad-a280-48bd967ff651
% \includegraphics[trim={0cm 0cm 0cm 0cm},clip,width=\columnwidth]{figs/5_/functional_diagram.jpeg}
% \caption{Functional diagram of the DJI M600 Pro autonomy software.}
% \label{fig:functional_diagram}
% \end{figure}
% \begin{figure}[b]
%     \centering
%     \begin{subfigure}[b]{0.48\columnwidth}
%         \centering
%         \includegraphics[width=1.0\linewidth]{figs/5_/field_test_altitude_over_time.png}
%         \caption{}
%         \label{fig:m600_altitude_over_time}
%     \end{subfigure}
%     \begin{subfigure}[b]{0.48\columnwidth}
%         \centering
%         \includegraphics[width=1.0\linewidth]{figs/5_/field_test_entropy_over_time.png}
%         \caption{}
%         \label{fig:m600_entropy_over_time}
%     \end{subfigure}
%     \caption{The results for our hexarotor field deployment. (a) Plot of flown altitude over time, showing large variation throughout the experiment. (b) Reduction in entropy percentage over time of field experiment.}
% \end{figure}

\subsubsection{Experimental Results}


The bottom image of Fig.~\ref{fig:google_earth} shows the path selected by IA-TIGRIS in the search area. The figure highlights how the planner dynamically adjusts altitudes over time to balance coverage and sensing resolution, maximizing information gain. Higher altitudes allow for broader area coverage, while lower altitudes provide more detailed observations where needed. Additionally, the planner prioritizes revisiting the three regions of higher uncertainty, recognizing the need for repeated observations reduce entropy. This adaptive strategy ensures that uncertain areas receive sufficient attention to improve the belief map. As a result, the entropy of the information map decreases to near zero by the end of the mission, as shown in Fig.~\ref{fig:m600_entropy_over_time}, indicating that the planner has effectively gathered the necessary information. This behavior demonstrates the planner’s ability to optimize sensing actions, balancing altitude selection, revisit frequency, and exploration to maximize mission success.

\begin{figure}[t]
\centering
% \includegraphics[width=2.5in]{fig1}
\includegraphics[trim={4cm 4cm 0cm 4cm},clip,width=\columnwidth]{figs/5_/TL1.jpg}
\caption{Fixed-wing platform used for autonomous flights with an onboard camera pitched at 10 degrees\cite{alarewebsite}}
\label{fig:tl1}
\end{figure}






\subsection{Fixed-wing Deployments}

Our proposed approach was extensively tested on the fixed-wing AlareTech TL-1 UAV, shown in Fig.~\ref{fig:tl1}. The UAV is equipped with an onboard camera pitched at 10 degrees, which introduces a more challenging planning problem due to the non-holonomic motion model and the camera's field of view. Over more than 20 flight hours and 100 flights running IA-TIGRIS, we validated our approach with the objective to search for objects of interest in a large search space across a variety of test scenarios, including different terrain types, varying environmental conditions, and diverse target distributions. An example mission from these tests is shown in Fig.~\ref{fig:fwd}. In this scenario, the planner was given the search bounds and a designated high-priority region. The resulting flight path prioritized revisiting the high-priority area twice, optimizing sensor use and ensuring maximum information gain. This strategy led to the successful detection of the object of interest, with its estimated position marked by the red dot in the figure. 

The map on the upper right in Fig.~\ref{fig:fwd} shows the information map after plan execution was complete. Due to the UAV's limited budget, the upper right and lower left corners of the map are not searched by the agent. The budget is instead utilized to search over the area of higher priority two times. Compared to the paths in Fig.~\ref{fig:google_earth}, we observe that the paths for the fixed wing are smoother and have a larger turning radius, demonstrating how IA-TIGRIS respects the motion constraints of the vehicle. We can also see the effect of wind on the path execution, where the flown path shown in green deviates from the planned path shown in yellow. This illustrates the importance of online planning in the cases where this deviation is large or would accumulate over the course of a longer mission and cause the expected observed area to be much different than actual observed area. 

\begin{figure}[t]
\centering
% \includegraphics[width=2.5in]{fig1}
% [trim={left bottom right top},clip]
\includegraphics[trim={3.0cm, 1.0cm, 3.0cm, 1.0cm},clip,width=\columnwidth]{figs/5_/ONRFig_v3.pdf}
\caption{An example path generated for the fixed-wing platform conducting a large-area search for an object of interest. The larger black rectangle denotes the search bounds, while the smaller black rectangle highlights a region of higher uncertainty. The red dot marks the estimated position of the detected object based on image detections. The upper-right map displays the information state after planning is complete, while the middle plot shows the percent change in entropy over mission time. The flown path illustrates a balance between allocating resources to the high-priority region and exploring other areas within the search space.}
\label{fig:fwd}
\end{figure}

% Also tested extensively on the AlareTech TL-1 (citation?) tube launched UAV seen in Fig.~\ref{fig:tl1}.

% Talk about amount of flights, hours. Platform. Compute. Show visualization fo example flight. Talk about objects of interest in a broad sense (no mention of water/ocean/land for targets). Follow similar figure format as previous section. Main thing we want to highlight is the differences introduced in plans by having a fixed-wing platform compared to a drone. Include image of Alare TL-1 somewhere.

% One big figure showing all the info we want to convey. 

% \BM{Pitch 10 degrees, onboard computer type, etc}


% \subsection{VTOL?}
% what would it bring?


\section{Discussion}
\section{Discussion}
\label{sec:discussion}

% \TODO{Bryan}

Our multimodal data augmentation method is a plug-and-play method that can be applied to any future VLM. Also the T2I generation can be replaced by any future T2I model, thus the effectiveness of our method automatically improves along with the SOTA T2I model, making it future-proof.



Our main method, \textbf{Co}ntrastive Visual \textbf{D}ata \textbf{A}ugmentation (\textbf{CoDA}), is simple and easy to apply to LMMs in a variety of scenarios. Several components in the pipeline utilize existing off-the-shelf model components that can be easily swapped out for superior versions of similar models as research in their respective field progresses. Therefore, we expect the efficiency and effectiveness of \textbf{CoDA} to dramatically scale along with the advancement of relevant models. 



\bibliographystyle{ACM-Reference-Format}
\bibliography{references}

\appendix

\section{Metric}
\label{sec:metric}

\textbf{Mean Squared Error (MSE)} Mean Squared Error (MSE) is a common statistical metric used to assess the difference between predicted and actual values. The formula is:
\begin{equation}
    MSE = \frac{1}{n} \sum_{i=1}^{n} (y_i - \hat{y}_i)^2
\end{equation}
where $ n $ is the number of samples, $ y_i $ is the actual value, and $ \hat{y}_i $ is the predicted value.

\textbf{Relative L2 Error} Relative L2 error measures the relative difference between predicted and actual values, commonly used in time series prediction. The formula is:
\begin{equation}
    \text{Relative L2 Error} = \frac{\| Y_{\text{pred}} - Y_{\text{true}} \|_2}{\| Y_{\text{true}} \|_2}
\end{equation}
where $ Y_{\text{pred}} $ is the predicted value and $ Y_{\text{true}} $ is the actual value.

\textbf{Structural Similarity Index Measure (SSIM)} The Structural Similarity Index (SSIM) measures the similarity between two images in terms of luminance, contrast, and structure. The formula is:
\begin{equation}
    SSIM(x, y) = \frac{(2\mu_x \mu_y + C_1)(2\sigma_{xy} + C_2)}{(\mu_x^2 + \mu_y^2 + C_1)(\sigma_x^2 + \sigma_y^2 + C_2)}
\end{equation}
where $ \mu_x $ and $ \mu_y $ are the mean values, $ \sigma_x $ and $ \sigma_y $ are the standard deviations, $ \sigma_{xy} $ is the covariance.

\section{Related Work}
\subsection{Deep Learning based Weather Forecasting}
\textbf{Global Weather Forecasting.} Global weather forecasting has seen significant progress with deep learning models. FourCastNet, based on Fourier neural operators, provides global forecasts comparable to traditional numerical methods like IFS, but at much higher speeds~\cite{pathak2022fourcastnet}. Pangu, utilizing the Swin Transformer, exceeds NWP methods, incorporating earth-specific location embeddings for better performance~\cite{bi2023accurate}. The Spherical Fourier Neural Operator (SFNO) extends Fourier methods using spherical harmonics, offering more stable long-term predictions~\cite{bonev2023spherical}. FuXi focuses on long-term forecasting, achieving a 15-day forecasts comparable to ECMWF~\cite{chen2023fuxi}. GraphCast leverages message-passing networks to improve efficiency and forecasting accuracy~\cite{lam2023learning}, and GenCast builds on this to enhance ensemble forecasting~\cite{price2023gencast}. Further, diffusion models like those in~\cite{li2024generative} generate probabilistic ensembles by sampling, while NeuralGCM~\cite{kochkov2024neural} focuses on atmospheric circulation with a dynamic core, offering climate simulation capabilities but at higher training and inference costs. 

\textbf{Regional Weather Forecasting.} The goal of regional weather forecasting is to enhance local prediction accuracy with high-resolution models. CorrDiff~\cite{mardani2023generative} combines U-Net and diffusion models to improve local forecasts. MetaWeather~\cite{kim2024metaweather} adapts global forecasts to regional contexts using meta-learning. GNNs are also widely applied in regional forecasting, with Graphcast~\cite{lam2023learning} enhancing accuracy by modeling complex spatial dependencies. MetNet-3~\cite{espeholt2022deep} offers high-accuracy forecasts for weather variables, such as precipitation, temperature, and wind speed, at 2-minute intervals and 1–4 km resolution, outperforming traditional models like HRRR. NowcastNet~\cite{zhang2023skilful} and DGMR~\cite{ravuri2021skilful} excel in short-term extreme precipitation forecasts using deep generative models and radar data. In spatiotemporal prediction, NMO~\cite{wu2024neural} models the evolution of physical dynamics, providing new insights for local weather forecasting. Similarly, SimVP~\cite{gao2022simvp} and PastNet~\cite{wu2024pastnet} achieve good results in forecasting local precipitation evolution using spatiotemporal convolution methods.
    
% Despite these advances, none of these methods effectively address the challenge of balancing global and regional high-resolution forecasts or capturing the fine-grained, dynamic interactions important for extreme event prediction.
    
\subsection{Numerical analysis methods}
Multigrid methods~\cite{mccormick1987multigrid,wesseling1995introduction,hackbusch2013multi,bramble2019multigrid,hiptmair1998multigrid,brandt1983multigrid,borzi2009multigrid} and nested grid strategies~\cite{miyakoda1977one,zhang2012nested,sullivan1996grid} are widely used to solve PDEs and handle multi-scale problems~\cite{debreu2008two,xue2000advanced}. Multigrid methods use grids of different resolutions to transfer information and accelerate iterations. They efficiently solve large-scale problems and improve computational accuracy. By eliminating low-frequency errors on coarse grids and high-frequency errors on fine grids, multigrid methods effectively handle error convergence at different scales~\cite{he2019mgnet,he2023mgno,shao2022fast}. Nested grid strategies embed higher-resolution fine grids into regions of interest based on a global coarse grid to capture local complex physical phenomena in detail. In weather forecasting, this method provides large-scale background fields on a global scale while refining the grid for target regions to accurately simulate the evolution of local weather systems and the occurrence of extreme events~\cite{bacon2000dynamically}. 

% Our proposed neural nested grid method helps address challenges like boundary information loss in regional forecasting and multi-scale feature capture.

\section{Additional Results}
%
We present more additional results in Figure \ref{fig_0.25-day}, \ref{fig_0.5-day}, \ref{fig_1.0-day} \ref{fig_1.5-day}, \ref{fig_2.0-day}, \ref{fig_2.5-day}, \ref{fig_3.0-day}, \ref{fig_3.5-day}, \ref{fig_4.0-day}, \ref{fig_4.5-day}, \ref{fig_5.0-day}, \ref{fig_5.5-day}, \ref{fig_6.0-day}, \ref{fig_6.5-day}, \ref{fig_7.0-day}, \ref{fig_7.5-day},
\ref{fig_8.0-day}, \ref{fig_8.5-day}, \ref{fig_9.0-day}, \ref{fig_9.5-day},
\ref{fig_10.0-day}, including 18 variables that are importmant to weather forecasting, each with results ranging from 6 hours to 10 days. These additional results further demonstrate the effectiveness of OneForecast. Same as the Figure \ref{fig:visual_results}
, the initial conditions is 00:00 UTC, 1 January 2020.


\begin{figure*}[h]
\centering
\includegraphics[width=1\linewidth]{figures/fig_0.25-day.jpg}
\vspace{-20pt}
\caption{6-hour forecast results of different models.}
\label{fig_0.25-day}
\end{figure*}

\begin{figure*}[h]
\centering
\includegraphics[width=1\linewidth]{figures/fig_0.5-day.jpg}
\vspace{-20pt}
\caption{0.5-day forecast results of different models.}
\label{fig_0.5-day}
\end{figure*}

\begin{figure*}[h]
\centering
\includegraphics[width=1\linewidth]{figures/fig_1.0-day.jpg}
\vspace{-20pt}
\caption{1-day forecast results of different models.}
\label{fig_1.0-day}
\end{figure*}

\begin{figure*}[h]
\centering
\includegraphics[width=1\linewidth]{figures/fig_1.5-day.jpg}
\vspace{-20pt}
\caption{1.5-day forecast results of different models.}
\label{fig_1.5-day}
\end{figure*}

\begin{figure*}[h]
\centering
\includegraphics[width=1\linewidth]{figures/fig_2.0-day.jpg}
\vspace{-20pt}
\caption{2-day forecast results of different models.}
\label{fig_2.0-day}
\end{figure*}


\begin{figure*}[h]
\centering
\includegraphics[width=1\linewidth]{figures/fig_2.5-day.jpg}
\vspace{-20pt}
\caption{2.5-day forecast results of different models.}
\label{fig_2.5-day}
\end{figure*}

\begin{figure*}[h]
\centering
\includegraphics[width=1\linewidth]{figures/fig_3.0-day.jpg}
\vspace{-20pt}
\caption{3-day forecast results of different models.}
\label{fig_3.0-day}
\end{figure*}

\begin{figure*}[h]
\centering
\includegraphics[width=1\linewidth]{figures/fig_3.5-day.jpg}
\vspace{-20pt}
\caption{3.5-day forecast results of different models.}
\label{fig_3.5-day}
\end{figure*}

\begin{figure*}[h]
\centering
\includegraphics[width=1\linewidth]{figures/fig_4.0-day.jpg}
\vspace{-20pt}
\caption{4-day forecast results of different models.}
\label{fig_4.0-day}
\end{figure*}

\begin{figure*}[h]
\centering
\includegraphics[width=1\linewidth]{figures/fig_4.5-day.jpg}
\vspace{-20pt}
\caption{4.5-day forecast results of different models.}
\label{fig_4.5-day}
\end{figure*}


\begin{figure*}[h]
\centering
\includegraphics[width=1\linewidth]{figures/fig_5.0-day.jpg}
\vspace{-20pt}
\caption{5.0-day forecast results of different models.}
\label{fig_5.0-day}
\end{figure*}

\begin{figure*}[h]
\centering
\includegraphics[width=1\linewidth]{figures/fig_5.5-day.jpg}
\vspace{-20pt}
\caption{5.5-day forecast results of different models.}
\label{fig_5.5-day}
\end{figure*}

\begin{figure*}[h]
\centering
\includegraphics[width=1\linewidth]{figures/fig_6.0-day.jpg}
\vspace{-20pt}
\caption{6.0-day forecast results of different models.}
\label{fig_6.0-day}
\end{figure*}

\begin{figure*}[h]
\centering
\includegraphics[width=1\linewidth]{figures/fig_6.5-day.jpg}
\vspace{-20pt}
\caption{6.5-day forecast results of different models.}
\label{fig_6.5-day}
\end{figure*}

\begin{figure*}[h]
\centering
\includegraphics[width=1\linewidth]{figures/fig_7.0-day.jpg}
\vspace{-20pt}
\caption{7.0-day forecast results of different models.}
\label{fig_7.0-day}
\end{figure*}

\begin{figure*}[h]
\centering
\includegraphics[width=1\linewidth]{figures/fig_7.5-day.jpg}
\vspace{-20pt}
\caption{7.5-day forecast results of different models.}
\label{fig_7.5-day}
\end{figure*}

\begin{figure*}[h]
\centering
\includegraphics[width=1\linewidth]{figures/fig_8.0-day.jpg}
\vspace{-20pt}
\caption{8.0-day forecast results of different models.}
\label{fig_8.0-day}
\end{figure*}

\begin{figure*}[h]
\centering
\includegraphics[width=1\linewidth]{figures/fig_8.5-day.jpg}
\vspace{-20pt}
\caption{8.5-day forecast results of different models.}
\label{fig_8.5-day}
\end{figure*}

\begin{figure*}[h]
\centering
\includegraphics[width=1\linewidth]{figures/fig_9.0-day.jpg}
\vspace{-20pt}
\caption{9.0-day forecast results of different models.}
\label{fig_9.0-day}
\end{figure*}

\begin{figure*}[h]
\centering
\includegraphics[width=1\linewidth]{figures/fig_9.5-day.jpg}
\vspace{-20pt}
\caption{9.5-day forecast results of different models.}
\label{fig_9.5-day}
\end{figure*}

\begin{figure*}[h]
\centering
\includegraphics[width=1\linewidth]{figures/fig_10.0-day.jpg}
\vspace{-20pt}
\caption{10.0-day forecast results of different models.}
\label{fig_10.0-day}
\end{figure*}


\section{Detailed Mathematical Proof}
\label{sec:proof}
\textbf{Proof of Theorem 1}

Now we have N augmented data and we need to select the best from them. We consider both the quality and the diversity of these data and get the sampling strategy from an optimization problem.

We model the sampling strategy as a multinomial distribution supported on all the augmented data $S = \{\mathbf{X}_j\}_{j=1}^N$, which means that the sampling strategy $\pi=(\pi_1,...,\pi_N)^\top$ is the corresponding probabilities of selecting $\mathbf{X}_1,...,\mathbf{X}_N$, then we can model the expectation of the similarity as:
$$\begin{aligned}
 & \mathbb{E}_{Y_x,Y_{x^{\prime}}\in\mathcal{C}}\{g(x,x^{\prime})\mid S\} \\
 & =\quad\int g(\mathbf{x},\mathbf{x}^{\prime})\boldsymbol{\pi}(\mathbf{x})\mathrm{Pr}_{S}(Y_{x}\in\mathcal{C}\mid\boldsymbol{x}=\mathbf{x})\boldsymbol{\pi}(\mathbf{x}^{\prime})\mathrm{Pr}_{S}(Y_{x}\in\mathcal{C}\mid\boldsymbol{x}=\mathbf{x}^{\prime})d\mathbf{x}d\mathbf{x}^{\prime} \\
 & =\quad\sum_{i,j=1}^Ng(\mathbf{X}_i,\mathbf{X}_j)\pi_i\pi_j\mathrm{Pr}_{S}(Y_x\in\mathcal{C}\mid\boldsymbol{x}=\mathbf{X}_i)\mathrm{Pr}_{S}(Y_x\in\mathcal{C}\mid\boldsymbol{x}=\mathbf{X}_j),
\end{aligned}$$
where the set $\mathcal{C}$ denotes the criterion of selection we are using, the function $g$ can be chosen as any similarity metric function and $x$ means a random variable.

The core to solving the above optimization problem is to use predictive inference to approximate the conditional probability of $\{Y_x\in\mathcal{C}\}$ given $x = \mathbf{X}$
Let $\mu ( \mathbf{x} ) : = \mathbb{E} ( Y\mid \mathbf{X} = \mathbf{x} )$ be the oracle associated with $( \mathbf{X} , Y) .$ Denote $\theta_j=\mathbb{I}\{Y_j\in\mathcal{C}\}$. As the augmented data
$\mathbf{X}_1,...,\mathbf{X}_N$ are independently identically distributed, $\theta_1,...,\theta_N$ can be regarded as independent Bernoulli($q)$ variables with $q=\Pr(Y_j\in\mathcal{C}).$ The probability distribution of the predicted result $W_j$ for $j=1,...,N$ is
$$\Pr(W_j\mid\theta_j)=(1-\theta_j)f_0+\theta_jf_1,\quad$$
where $f_0$ and $f_1$ are the conditional distributions of $W_j$ on $Y_j \in \mathcal{C}$ or not.

Denote $T(w) = \frac{(1-q)f_0(W_j)}{f(W_j)}$, we can rewrite the expectation of the similarity as
$$\mathbb{E}_{Y_x,Y_{x^{\prime}}\in\mathcal{C}}\{g(x,x^{\prime})|S\}=\sum_{i,j=1}^Ng(\mathbf{X}_i,\mathbf{X}_j)\pi_i\pi_j(1-T_i)(1-T_j)=\boldsymbol{\pi}^\top A_\mathbb{T}\boldsymbol{\pi},$$

Next, we use the expectation to control the quality of the data.
$$\mathbb{E}\{\mathbb{I}(Y_x\not\in\mathcal{C})\mid S\}=\sum_{i=1}^N\Pr(Y_i\not\in\mathcal{C}\mid\mathbf{X}_i)\pi_i=\sum_{i=1}^N\pi_iT_i\leq\alpha,$$

In all, the optimization problem can be modeled as 
\begin{align}
    & \arg\min_{\boldsymbol{\pi}}\quad h(\boldsymbol{\pi},\mathbb{T}):=\boldsymbol{\pi}^\top A_\mathbb{T}\boldsymbol{\pi}, \\
    & \text{subject to} \quad
        \begin{cases}
            \sum_{i = 1}^N\pi_iT_i\leq\alpha, \\
            \sum_{i = 1}^N\pi_i = 1, \\
            0\leq\pi_i\leq m^{-1}, \quad 1\leq i\leq N.
        \end{cases}
\end{align}

where $m$ is used to control the maximum selection.

The best selection of K is determined by the strategy $\pi$ which serves as the solution to the above optimization problem.

\section{Additional Experiments}
\label{sec:more_experiments}
\subsection{Long-term forecasting experiment expansion}

In the long-term forecasting experiments, we compare the performance of different backbone models on the SWE benchmark, evaluating the relative L2 error for three variables (U, V, and H). Our setup inputs 5 frames and predicts 50 frames. For the SimVP-v2 model, using \method{} reduces the relative L2 error for SWE (u) from 0.0187 to 0.0154, SWE (v) from 0.0387 to 0.0342, and SWE (h) from 0.0443 to 0.0397. We visualize SWE (h) in 3D as shown in Figure~\ref{fig:case} [\textcolor{red}{I}]. For the ConvLSTM model, applying \method{} reduces the relative L2 error for SWE (u) from 0.0487 to 0.0321, SWE (v) from 0.0673 to 0.0351, and SWE (h) from 0.0762 to 0.0432. For the FNO model, using \method{} reduces the relative L2 error for SWE (u) from 0.0571 to 0.0502, SWE (v) from 0.0832 to 0.0653, and SWE (h) from 0.0981 to 0.0911. Overall, \method{} significantly improves the long-term forecasting accuracy of different backbone models.

\begin{figure*}[h]
    \centering
    \includegraphics[width=\textwidth]{image/casestudy.pdf}
    \caption{
    \textcolor{red}{I.} 3D visualization of the SWE(h), showing Ground-truth, SimVP-V2+BeamVQ predictions, and Error at T=1, 10, 20, 30, 40, 50. The first row shows Ground-truth, the second SimVP-V2+BeamVQ predictions, and the third Error. \textcolor{red}{II.} A case study. Building fire simulation with ventilation settings added to Wu's Prometheus~\cite{wu2024prometheus}. (a) Layout and HRR growth. (b) Comparison of physical metrics for different methods. (c) Ground-truth, ResNet+BeamVQ, and ResNet predictions.
    }
    \label{fig:case} 
\end{figure*}


\subsection{Experiment Statistical Significance}
\label{sec:significance}
To measure the statistical significance of our main experiment results, we choose three backbones to train on two datasets to run 5 times. 
Table~\ref{tab:significance} records the average and standard deviation of the test MSE loss.
The results prove that our method is statistically significant to outperform the baselines
because our confidence interval is always upper than the confidence interval of the baselines. 
Due to limited computation resources, we do not cover all ten backbones and five datasets, 
but we believe these results have shown that our method has consistent advantages.


\begin{table}[h]
\label{tab:significance}
\centering
\begin{scriptsize}
    \begin{sc}
    \caption{ The average and standard deviation of MSE in 5 runs}
    \label{tab:significance}
    \centering
        \renewcommand{\multirowsetup}{\centering}
        \setlength{\tabcolsep}{10pt}
        \begin{tabular}{l|cc|cc}
            \toprule
            
            \multirow{4}{*}{Model} & \multicolumn{4}{c}{Benchmarks}  \\
            \cmidrule(lr){2-5}
            & \multicolumn{2}{c}{NSE} &   \multicolumn{2}{c}{SEVIR}   \\
            \cmidrule(lr){2-5}
           & Ori & + BeamVQ & Ori & + BeamVQ  \\
            \midrule
            ConvLSTM &0.4092$\pm$0.0002 &\textbf{0.1277$\pm$0.0001}  & 0.1762 0.0007  & \textbf{0.1279$\pm$0.0009}  \\
            FNO &  0.2227$\pm$0.0003 &\textbf{0.1007 $\pm$0.0002}& 0.0787$\pm$0.0012 & \textbf{ 0.0437$\pm$0.0013} \\
            CNO & 0.2192 $\pm$0.0008 &\textbf{ 0.1492$\pm$0.0011}& 0.0057$\pm$0.0005 & \textbf{ 0.0053$\pm$0.0006} \\
            \bottomrule
        \end{tabular}
    \end{sc}

\end{scriptsize}
\end{table}
 
\end{document}
\endinput
%%
%% End of file `sample-sigconf.tex'.



%\bibliographystyle{elsarticle-num}
%\bibliography{bibliography.bib}

	
	
	

	
\end{document}