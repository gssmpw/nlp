\documentclass{article}

\usepackage{arxiv}

\usepackage[utf8]{inputenc} % allow utf-8 input
\usepackage[T1]{fontenc}    % use 8-bit T1 fonts
\usepackage{hyperref}       % hyperlinks
\usepackage{url}            % simple URL typesetting
\usepackage{booktabs}       % professional-quality tables
\usepackage{amsfonts}       % blackboard math symbols
\usepackage{nicefrac}       % compact symbols for 1/2, etc.
\usepackage{microtype}      % microtypography
\usepackage{lipsum}
\usepackage{graphicx}
\graphicspath{ {./images/} }
\usepackage{amssymb}
%% The amsthm package provides extended theorem environments
\usepackage{amsthm}

%% The lineno packages adds line numbers. Start line numbering with
%% \begin{linenumbers}, end it with \end{linenumbers}. Or switch it on
%% for the whole article with \linenumbers.
\usepackage{lineno}
\usepackage{subfigure}
\usepackage{xcolor}
\usepackage{amsmath}

\usepackage{fancyhdr}
\fancyhead{TEXT}
\usepackage{lastpage}
\pagestyle{fancy}
\lhead{} \chead{Distribution Statement A: Approved for public release; distribution is unlimited} \rhead{\scriptsize \thepage}
\lfoot{}\cfoot{}\rfoot{}

%% Definitions
\newcommand{\vect}[1]{\mathbf{#1}}
\newcommand{\tensor}[1]{\mathbf{#1}}
\newcommand{\field}[1]{\mathbb{#1}}
\newcommand{\R}{\field{R}}
\newcommand{\coor}{\xi}
\newcommand{\body}{{\cal B}}
\newcommand{\eltensor}{\mathbb{C}}


% Uncomment and use as if needed
\newtheorem{theorem}{Theorem}
\newtheorem{remark}{Remark}[section]
\newtheorem{lemma}{Lemma}[section]
\newtheorem{assumption}{Assumption}
\newtheorem{corollary}{Corollary}[section]
\newtheorem{proposition}{Proposition}[section]


\title{Deep Ritz method with Fourier feature mapping: A deep learning approach for solving variational models of microstructure}


\author{
	Ensela Mema \\
	Kean University\\
	Union, NJ 07083 \\
	\texttt{emema@kean.edu} \\
	%% examples of more authors
	\And
	Ting Wang\\
	Booz Allen Hamilton Inc.\\
	McLean, VA 22102\\
	\texttt{wang\_ting@bah.edu} \\
	\And
	Jaroslaw Knap \\
	DEVCOM Army Research Laboratory\\
	Aberdeen Proving Ground, MD 21005 \\
	\texttt{jaroslaw.knap.civ@army.mil} \\
}
\begin{document}
	\maketitle
	\begin{abstract}
		This paper presents a novel approach that combines the Deep Ritz Method (DRM) with Fourier feature mapping to solve minimization problems comprised of multi-well, non-convex energy potentials. These problems present computational challenges as they lack a global minimum.  Through an investigation of three benchmark problems in both 1D and 2D, we observe that DRM suffers from spectral bias pathology, limiting its ability to learn solutions with high frequencies. To overcome this limitation, we modify the method by introducing Fourier feature mapping. This modification involves applying a Fourier mapping to the input layer before it passes through the hidden and output layers. Our results demonstrate that Fourier feature mapping enables DRM to generate high-frequency, multiscale solutions for the benchmark problems in both 1D and 2D, offering a promising advancement in tackling complex non-convex energy minimization problems.
	\end{abstract}
	
	
	% keywords can be removed
	\keywords{Deep learning \and Variational problems \and Nonconvex energy minimization \and Fourier feature mapping \and Martensitic phase transformation}
	
	
\section{Introduction}


Materials undergoing martensitic phase transformations constitute a
technologically important class of
materials~\cite{bhattacharya2003microstructure}. These materials
include steels, shape-memory alloys, solidified gases and polymers, to
name a few. A feature common to all of these materials is
microstructure in the form of elaborate three-dimensional patterns at
the scale ranging from nanometers to centimeters. Mathematically,
microstructure induced by martensitic phase transformations is
characterized as minimizers of a total energy functional.  The
fundamental difficulty in seeking such minimizers lies, however, in
non-convexity of the total energy
functional~\cite{bhattacharya2003microstructure,dacorogna2007direct}.

Numerical treatment of non-convex minimization problems is fraught
with challenges. Standard finite elements usually require very fine
meshes to resolve meaningful scales associated with microstructure. In
addition, specially crafted meshes are frequently needed as finite
element solutions tend to be strongly mesh dependent and adaptive mesh
refinement may not always perform
satisfactorily~\cite{luskin1996computation,Carstensen_2005}. The
strong mesh dependence of solutions may be somewhat alleviated by
recourse to specialized finite-element techniques, such as
discontinuous finite
elements~\cite{gobbert1999discontinuous}. Alternatively, the
non-convex energy functional can be regularized through
convexification~\cite{carstensen2001numerical}. Solutions of
convexified minimization problems can be then efficiently carried out
by standard finite elements~\cite{bartels2004effective}. In practice,
however, convexified energy functionals may not be readily available
explicitly and their numerical approximations are generally costly to
obtain~\cite{carstensen1997numerical}. While minimizers of the
convexified energy functional are much easier to get, they may miss
some important physical features of the original (non-convex) minimization
problem. Finally, one may employ Young measures to turn the non-convex
minimization problem into a convex minimization
problem~\cite{nicolaides1993computation,aranda2001numerical}. This
approach offers numerous benefits, chiefly among them that the energy
functional does not need to be altered. Yet, additional numerical
algorithms are required, increasing considerably the overall
computational
cost~\cite{carstensen2000numerical,bartels2004effective}.

Recent advancements in deep neural networks (DNNs) have raised hopes
that DNNs may be capable of generating solutions to non-convex
minimization problems. Specifically, the universal approximation
theory~\cite{hornik1990universal, hornik1991approximation} has enabled
DNN-based numerical methods for PDEs to parameterize the solution
using a DNN and learn it using the method of stochastic gradient
descent. The approach learns the solution by minimizing a loss
function induced by the physics constraints, often referred to as the
physics informed approach.  Depending on how the loss function is
constructed, DNN-based methods can be roughly classified into three
categories: 1) the physics informed neural network
(PINN)~\cite{raissi2019physics, sirignano2018dgm}; 2) deep Ritz
methods (DRM)~\cite{yu2018deep} and 3) deep backward stochastic
differential equation (BSDE)~\cite{han2017deep}.  PINN minimizes the
residual of the PDE evaluated at a set of randomly sampled collocation
points. In comparison, DRM utilizes the variational structure of
elliptic PDEs to minimize the energy functional.  Finally, deep BSDE
explores the probabilistic connection between parabolic PDE and BSDE
in order to reformulate the problem as a reinforcement learning task.
The key advantage of the DNN-based methods over the conventional ones
lies in the fact that they replace the deterministic mesh by Monte
Carlo sampling and hence, in principle, lead to dimension independent
convergence rates~\cite{grohs2018proof}. Despite being a promising
direction, training of DNN-based methods can be extremely challenging
due to, e.g., the choice of the learning rate, the multi-scale nature
of the problem under consideration, etc. Indeed, it has been widely
observed that DNNs are biased to learn low frequency features of the
solution, making them fail to learn solutions that exhibit
high-frequency and multi-scale, an essential feature in non-convex
minimization in the context of microstructure evolution. This
phenomenon is known as the spectral bias pathology for deep
learning~\cite{rahaman2019spectral, wang2021eigenvector}.

In this work, we focus on the following minimization problem:
\begin{align}
	\min_{u\in \mathcal{U}} \ I(u) \qquad {\rm where } \qquad I(u) = \int_{D} W({\bf x},u({\bf x}), \nabla{u}({\bf x})) \ d{\bf x},\label{eqn:variationalproblem}
\end{align}
% $W$ is the energy density function and $\mathcal{U}$ is the set of admissible functions $u({\bf x})$, also referred to as the trial function.
where $D \subset \mathbb{R}^d$ is a bounded open set with a Lipschitz
boundary $\partial D$,
$W:\mathbb{R}^d \times \mathbb{R}^N \times \mathbb{R}^{dN} \to
\mathbb{R}$ is the Lagrangian and $u : \bar{D} \to
\mathbb{R}^N$. $\bar{D}$ denotes the closure of $D$.  Here,
$\mathcal{U}$ is a space of admissible functions, e.g., the Sobolev
space $H_0^1(D)$ when the zero boundary condition is imposed. The
energy density $W$ is generally assumed to be non-convex in
$\nabla{u}$.  To solve the above minimization, one seeks minimizers
$u({\bf x})$ of the functional $I(u)$ over the prescribed domain $D$,
subject to boundary condition constraints (set to $u({\bf x}) = 0$ on
$\partial D$). The reader is referred to any standard texts on
variational calculus, for example ~\cite{dacorogna2007direct}, for the
properties of the minimization problem~(\ref{eqn:variationalproblem}).

Since DRM works by minimizing an energy functional, it is natural to
seek solutions of the minimization
problem~\eqref{eqn:variationalproblem} by means of DRM.  A
straightforward application of DRM to non-convex minimization problems
in 1D and 2D has been carried out by Chen et
al. in~\cite{Chen&Rosakis2023}. They demonstrate that DRM is capable
of capturing the complexities of local or global minimizers of
non-convex variational problems, if one applies an ad hoc activation
function.  Additionally, they suggest that the depth of the DNN plays
a role analogous to the mesh size in FEM so one can capture
high-frequency solutions (with more twin bands) if one increases the
depth of DNN. It is important to note that although DRM is capable of
solving non-convex minimization problems, a naive application of the
method fails to consistently generate high-frequency solutions due to
the fact that DNN algorithms, including DRM, are biased to learn the
low frequency features of the solutions.

In our work, we address the shortcomings of DRM by applying Fourier
feature mapping as outlined in~\cite{FourierFeatures2020} and show
that DRM in conjunction with Fourier feature mapping (DRM\&FM) can
consistently generate high-frequency multiscale solutions for
non-convex minimization problems independently of the depth of the
DNN. The main contributions of our work can be summarized as follows:
\begin{itemize}
	\item We apply neural tangent kernel (NTK) theory to show that, similar to PINN, DRM also
	suffers from spectral bias pathology. That is, the learning rates
	along different directions are determined by the corresponding
	eigenvalues of the NTK. To alleviate this issue, we utilize the
	Fourier feature mapping to map the input into an appropriate
	submanifold. Based on the recent theoretical results on
	NTK~\cite{geifman2020similarity, chen2020deep}, we show (at least in the $1$D
	case) that the Fourier feature mapping leads to a quadratic decay
	NTK eigenspectrum which could be advantageous when multiscale
	problems are considered.
	
	\item We numerically illustrate that DRM alone cannot consistently
	generate high-frequency solutions to non-convex minimization
	problems by increasing the depth of DNN. See
	Section~\ref{sec:numerics} for the benchmark problems considered in
	this work and how they differ from the ones considered
	in~\cite{Chen&Rosakis2023}.
	
	\item We apply Fourier feature mapping on DRM and observe that DRM in
	conjunction with Fourier features (DRM\&FM) allow the DNN to learn
	high-frequency solutions to non-convex variational problems
	independently of the depth of the NN.
\end{itemize}
The paper is organized as follows: Section \ref{sec:DRM} outlines how
the DRM can be applied to solve variational problems.  Section
\ref{sec:NTK} uses NTK theory to show that DRM alone suffers from
spectral bias pathology and how Fourier feature mapping enables the
DRM to learn solutions whose NTK has a fast decaying
eigenspectrum. Section~\ref{sec:Numerics} presents our numerical
results in $1D$ and $2D$ and Section \ref{sec:Concl} discusses our
conclusions.

%%%%%%%%%%%%%%%%%%%%%%%%%%%%%%%%%%%%%%%%%%%%%%%%%%%%%%%%%%%%%%%%%%%%%%%%%%%%%%%%%%%%%%%%%%%%%
%%%%%%%%%%%%%%%%%%%%%%%%%%%%%%%%%%%%%%%%%%%%%%%%%%%%%%%%%%%%%%%%%%%%%%%%%%%%%%%%%%%%%%%%%%%%%%
\section{Deep Ritz Algorithm}
\label{sec:DRM}


\begin{figure}[!ht]
	\centering
	\includegraphics[width=0.75\textwidth]{Figure_1.png}
	\caption{Structure of Neural Network in Deep Ritz Method.}
	\label{fig:NN_structure}
\end{figure}

DRM solves the variational problem in~\eqref{eqn:variationalproblem}
by using DNN to construct an approximation $\hat{u}({\bf x})$ that
minimizes the functional $I(\hat{u})$ over the prescribed domain $D$.
More specifically, a DNN of depth $n$ approximates the solution
through a series of transformations by
\begin{align}
	\hat{u}({\bf x};\theta) = L^{[n]}\circ L^{[n-1]}\circ \cdots \circ L^{[1]}({\bf x}), \label{eqn:NNsolU}
\end{align}
with
\begin{equation*}
	\begin{split}   
		L^{[1]}({\bf x}) &= \sigma(A^{[1]} {\bf x} + b^{[1]})\\
		L^{[i]}({\bf x}) &= \sigma(A^{[i]} L^{[i-1]}({\bf x}) + b^{[i]}), \qquad i = 2, \ldots, n-1\\
		L^{[n]}({\bf x}) &= A^{[n]} L^{[n-1]}({\bf x}) + b^{[n]}
	\end{split}
\end{equation*}
where $A^{[i]}$ and $b^{[i]}$ are the weight matrix and the bias vector of layer $i$, respectively, and $\sigma$ is a nonlinear activation function (see Figure~\ref{fig:NN_structure} for sketch).
% The network structure is a feed forward NN that consists a series of transformations of the form: $L({\bf x}) = A {\bf x} + {\bf b}$ followed by a nonlinear activation function $\sigma({\bf x}):\mathbb{R}^N\rightarrow \mathbb{R}^N$ 
% (see Figure~\ref{fig:NN_structure} for sketch) 
% where ${\bf x} \in \mathbb{R}^d$, $A \in \mathbb{R}^{N\times d}$ and ${\bf b} \in \mathbb{R}^{N}$. The second component consists of a nonlinear activation function $\sigma({\bf x}):\mathbb{R}^N\rightarrow \mathbb{R}^N$ which is applied throughout each hidden layer of the network. The output of the $i$-th hidden layer can be expressed as: 
% 	\begin{align}
	% 		L^{[i]}({\bf x}) = \sigma(A^{[i-1]} \cdot L^{[i-1]}({\bf x})+b^{[i-1]}),
	% 	\end{align} 
% 	where $A^{[i-1]} \in \mathbb{R}^{N\times d}$ and $b^{[i-1]} \in \mathbb{R}^{N}$ denote the weight matrix and bias vector in layer $i-1$, respectively. 
% The DNN approximation of the variational problem can be written as  composition of the outputs for each layer:
% 	\begin{align}
	% 		\hat{u}({\bf x};\theta) = L^{[n]}\circ L^{[n-1]}\circ \cdots \circ L^{[1]}\circ L^{[0]}({\bf x}) \label{eqn:NNsolU}
	% 	\end{align}
% 	where $\theta \in \mathbb{R}^{N_\theta}$ denotes the full set of all weights and bias parameters in a neural network of $(n+1)$ layers (or $n$ hidden layers). 
Substituting~\eqref{eqn:NNsolU} in the variational problem~\eqref{eqn:variationalproblem} leads to the following finite dimensional optimization problem: 
\begin{align}
	\min_{ \theta \in \mathbb{R}^{N_\theta}} I(\hat{u})\qquad {\rm where} \qquad I(\hat{u}) = \int_{D} W({\bf x},\hat{u}({\bf x};\theta), \nabla \hat{u}({\bf x};\theta) ) d{\bf x},
	\label{eqn:NNoptproblem}
\end{align}
where $\theta = (A^{[1]}, b^{[1]}, \ldots, A^{[n]}, b^{[n]})$ are parameters of the DNN.
To account for the boundary condition, we follow E et al. in ~\cite{Weinan} and Chen et al. in~\cite{Chen&Rosakis2023} in using penalty approach to numerically enforce the prescribed boundary conditions of the variational problem, which leads to a modified functional: 
\begin{align}
	I(\theta) = \int_{D} W({\bf x},\hat{u}({\bf x};\theta), \nabla \hat{u}({\bf x};\theta)) d{\bf x} + \lambda \int_{\partial D} \hat{u}({\bf x};\theta)^2 ds,
\end{align}
where, with a slight abuse of notation, we have rewritten $I(\theta) \triangleq I(\hat{u}, \nabla \hat{u})$ to indicate that the optimization is with respect to the NN parameters $\theta$.
Note that $\lambda $ serves as a penalty term that increases the value of $I$ if the approximated DNN solution, $\hat{u}({\bf x};\theta)$ deviates from the prescribed values at the boundary. 
To solve the optimization problem by stochastic gradient descent (SGD), it is often convenient to rewrite the above integral in its probabilistic form as 
% {\color{blue} TW: this is not correct. One never approximates this sum but instead the gradient of it.
	% }
\begin{equation}\label{eqn:prob-form}
	\begin{split}
		\min_{\theta \in \mathbb{R}^{N_{\theta}}} I(\theta) := \mathbb{E} \left[ W({\bf x}, \hat{u}({\bf x}; \theta), \nabla \hat{u}({\bf x}; \theta))\right] + \lambda \mathbb{E}_b \left[|\hat{u}({\bf x_b};\theta)|^2\right],
	\end{split}
\end{equation}
where $\mathbb{E}$ and $\mathbb{E}_b$ are taken with respect to the uniform distributions
over $D$ and $\partial D$, respectively.
At each gradient descent iteration, we use Adam optimizer~\cite{kingma2014adam} to update the DNN parameters $\theta$ by evaluating the stochastic gradient of $I$ at a mini-batch of samples over $D$ and $\partial D$. 


%%%%%%%%%%%%%%%%%%%%%%%%%%%%%%%%%%%%%%%%%%%%%%%%%%%%%%%%%%%%%%%%%%%%
%%%%%%%%%%%%%%%%%%%%%%%%%%%%%%%%%%%%%%%%%%%%%%%%%%%%%%%%%%%%%%%%%%%%
\section{NTK analysis for Deep-Ritz and the Fourier feature}\label{sec:NTK}

\subsection{The spectral bias pathology for DRM}
In practice, a naive application of DRM often fails to achieve desirable results.   
In this section, we derive the NTK theory for DRM and show that, similar to PINN, DRM also suffers the pathology of spectral bias of neural networks~\cite{jacot2018neural, rahaman2019spectral, du2018gradient} and hence additional tricks and treats have to be applied. 

For ease of presentation, we assume $d = N = 1$ to keep notation uncluttered.
However, we emphasize that the result presented below can be readily generalized to the vectorial setting.
We start by considering the empirical approximation to~\eqref{eqn:prob-form} without the penalty term, i.e., 
\begin{equation}\label{eqn:ensemble-loss}
	\begin{split}
		\min_{\theta \in \mathbb{R}^{N_{\theta}}} I_{\mathcal{X}}(\theta) := \frac{1}{|\mathcal{X}|} \sum_{{\bf x_n} \in \mathcal{X}} W({\bf x_n}, \hat{u}({\bf x_n}; \theta),  \hat{u}^{\prime}({\bf x_n}; \theta)),
	\end{split}
\end{equation}
where $\mathcal{X}$ is the set of collocation points sampled uniformly over $D$, and $|\mathcal{X}|$ denotes the cardinality of the set.
Applying the gradient descent algorithm to $I_{\mathcal{X}}(\theta)$ leads to the discrete time dynamics 
\[
{\theta}_{n+1} = \theta_n - \eta h \nabla_{\theta} I_{\mathcal{X}}(\theta_n), \qquad n = 1, 2, \ldots,
\]
where $\eta>0$ is the learning rate and $h>0$ is a scaling constant. 
%~\eqref{eqn:variationalproblem} with $W$ dependent on $x, u$ and $\nabla u$, i.e., 
% \begin{equation}\label{eqn:variational-problem}
	%     \min_{u \in \mathcal{U}} I(u) := \int_D W(x, u, \nabla u)\, dx, 
	% \end{equation}
% where $D \subset \mathbb{R}^d$ is a bounded open set with a Lipschitz
% boundary $\partial D$, $W:\mathbb{R}^d \times \mathbb{R}^N \times \mathbb{R}^{dN} \to \mathbb{R}$ is the Lagrangian and $u : \bar{D} \to \mathbb{R}^N$ is the solution of interest.
% Here $\mathcal{U}$ is a space of admissible functions, e.g., the Soblev space $H_0^1(D)$ when the zero boundary condition is imposed. 
% Let $u_{\theta}:  \bar{D} \to \mathbb{R}^N$ with $\theta \in \mathbb{R}^{N_{\theta}}$ be a neural network function that approximates the minimizer of~\eqref{eqn:variational-problem}. We aim to minimize the variational loss 
% \begin{equation}\label{eqn:loss}
	%     \min_{\theta \in \mathbb{R}^{N_{\theta}}}I(u(\theta)) 
	%     :=
	%     \int_D W(x, u(\theta), \nabla u(\theta))\, dx.
	% \end{equation}
% In practice, the above integral must be approximated by the following empirical variational loss
% \begin{equation}\label{eqn:loss_practice}
	%     \min_{\theta \in \mathbb{R}^{N_{\theta}}} I_{\mathcal{X}}(\theta) := \frac{1}{|\mathcal{X}|}\sum_{x_n\in \mathcal{X}} W(x_n, u(x_n; \theta), \nabla u(x_n; \theta)),
	% \end{equation}
% where $\mathcal{X}$ is a set of collocation points sampled uniformly from the domain $D$.
Upon taking $h \to 0^+$, we obtain the continuous time dynamics governing the evolution of the parameters $\theta$,
\begin{equation}\label{eqn:theta-trajectory}
	\frac{d\theta(t)}{dt} = -\eta \nabla_{\theta}I_{\mathcal{X}}(\theta(t)),
\end{equation}
where $\theta: [0, \infty) \to \mathbb{R}^{1 \times N_{\theta}}$ is a function of $t$ and $\nabla_{\theta}I_{\mathcal{X}}(\theta(t)) \in \mathbb{R}^{1 \times N_{\theta}}$ is the gradient of $I_{\mathcal{X}}(\theta)$ with respect to $\theta$. 
% \begin{remark}[Gradient flow and continuity equation.]
	% We interpret the above equation from an optimal transport perspective. 
	% Note that~\eqref{eqn:theta-trajectory} suggests that $\theta(t)$ is the gradient flow associated with the vector field $-\eta \nabla_{\theta} I$.
	% Denote $T_t(\theta) = \theta(t)$ the solution to~\eqref{eqn:theta-trajectory} with the initial condition $\theta(0) = \theta$.
	% Suppose $\theta$ is initialized according to a measure $\rho_0$, then $\theta_t$ follows a distribution $\rho_t = T_t(\theta)_{\#} \rho_0$, where $T_t(\theta)_{\#}\rho_0$ is the push-forward of the measure $\rho_0$, i.e., 
	% \[
	% T_t(\theta)_{\#}\rho_0 (A) := \rho_0(T_t(\theta)^{-1}(A)), \qquad \text{for all}~A~\text{Borel in}~\mathbb{R}^{N_{\theta}}.
	% \]
	% Under mild conditions, $\rho_t$ satisfies the continuity equation (in a weak sense)
	% \[
	% \partial_t \rho_t(\theta) - \eta \nabla \cdot (\rho_t(\theta) \nabla_{\theta}I_{\mathcal{X}}(\theta)) = 0.
	% \]
	% \end{remark}
%In comparison to the NTK analysis for PINN which minimizes the mean square error loss~\cite{wang2021eigenvector}, the variational loss of DRM makes it impossible to obtain a closed form formula for the training dynamics for outputs $\hat{u}$. To circumvent this issue, we instead focus on analyzing the training dynamics for the loss $I_{\mathcal{X}}$.
We first derive the empirical evolution of the loss function $I_{\mathcal{X}}(\theta(t))$ with respect to $t$, i.e.,  
\begin{equation}\label{eqn:NTK-1}
	\frac{d I_{\mathcal{X}}(\theta(t))}{dt} = \left\langle \nabla_{\theta} I_{\mathcal{X}}(\theta(t)), \frac{d\theta(t)}{dt} \right\rangle = -\eta \|\nabla_{\theta} I_{\mathcal{X}}(\theta(t))\|^2.
\end{equation}
By chain rule we have 
\[
\nabla_{\theta} I_{\mathcal{X}}(\theta) = 
\frac{1}{|\mathcal{X}|} \sum_{{\bf x_n} \in \mathcal{X}} \partial_{\hat{u}} W_n(\theta) \nabla_{\theta} \hat{u}_n(\theta) 
+  \partial_{\hat{u}^{\prime}} W_n(\theta) \nabla_{\theta} \hat{u}^{\prime}_n( \theta), 
\] 
where we have denoted $\hat{u}_n(\theta) = \hat{u}({\bf x_n}; \theta)$, $\hat{u}^{\prime}_n( \theta) = \hat{u}^{\prime}({\bf x_n} ;\theta)$ and 
$W_n(\theta) = W({\bf x_n}, \hat{u}({\bf x_n}; \theta), \hat{u}^{\prime}({\bf x_n}; \theta))$ so that 
\[
\nabla_{\theta} \hat{u}_n \in \mathbb{R}^{1 \times N_{\theta}}, \qquad \partial_{\hat{u}} W_n \in \mathbb{R}, \qquad
\nabla_{\theta} \hat{u}_n^{\prime} \in \mathbb{R}^{1 \times N_{\theta}}, \qquad
\partial_{\hat{u}^{\prime}} W_n \in \mathbb{R}.
\]
Denote 
$
U_n(\theta) = [\hat{u}_n(\theta),  \hat{u}_n^{\prime}(\theta)]^{\top}
\in 
\mathbb{R}^{2 \times 1}
$
so that 
\[
\nabla_{\theta} U_n = [\nabla_{\theta}\hat{u}_n,  \nabla_{\theta}\hat{u}_n^{\prime} ]^{\top} \in \mathbb{R}^{2 \times N_{\theta}},
\]
\[
\nabla_U W_n = [\partial_{\hat{u}} W_n, \partial_{\hat{u}^{\prime}} W_n]^{\top}
\in 
\mathbb{R}^{2 \times 1}
\]
and hence
\[
\nabla_{\theta} I_{\mathcal{X}}(\theta) = \frac{1}{|\mathcal{X}|} \sum_{{\bf x_n} \in \mathcal{X}}  [\nabla_U W_n(\theta)]^{\top} \nabla_{\theta} U_n(\theta) \in \mathbb{R}^{1 \times N_{\theta}}.
\]
Then we can further rewrite the evolution equation given by~\eqref{eqn:NTK-1} in the following compact form,
\begin{equation}\label{eqn:loss-trajectory}
	\begin{split}
		\frac{d I_{\mathcal{X}}(\theta(t))}{dt} 
		= -\frac{\eta}{|\mathcal{X}|^2} \sum_{{\bf x_m}, {\bf x_n} \in \mathcal{X}} [\nabla_U W_m(\theta(t))]^{\top} 
		\left\{\nabla_{\theta} U_m(\theta(t)) [\nabla_{\theta} U_n(\theta(t))]^{\top} \right\}
		\nabla_U W_n(\theta(t)).
	\end{split}
\end{equation}
We call the operator/matrix valued function $K: D  \times D  \to \mathbb{R}^{2 \times 2}$ defined by
\[
K({\bf x_m}, {\bf x_n}; \theta) \triangleq \nabla_{\theta} U_m(\theta) [\nabla_{\theta} U_n(\theta)]^{\top},\qquad {\bf x_m}, {\bf x_n} \in D,
\]
the NTK (parameterized at $\theta$) associated to DRM. 
It should be emphasized that, similar to PINN, the NTK kernel $K$ of DRM depends on both the output $\hat{u}$ and its spatial derivative $ \hat{u}^{\prime}$.

The lazy training phenomenon suggests that, when trained with gradient-based optimizers, strongly overparameterized NNs could converge
exponentially fast to the minimum training loss without significantly varying the parameters~\cite{chizat2019lazy}, i.e., $\theta(t) \approx \theta_0$. 
Therefore, to analyze the asymptotic behavior of the differential equation~\eqref{eqn:loss-trajectory}, we linearize the DNN solution $\hat{u}({\bf x}; \theta)$ at its initial value $\theta_0$ via
\[
\hat{u}({\bf x}; \theta) \approx \bar{u}({\bf x}; \theta) \triangleq \hat{u}({\bf x}; \theta_0) + \langle \nabla_{\theta} \hat{u}({\bf x}; \theta_0), \theta - \theta_0 \rangle,
\]
where by definition $\bar{u}({\bf x}; \theta)$ is the linearization of $\hat{u}({\bf x}; \theta)$ at $\theta_0$.
Notice that 
\[
\nabla_{\theta} [\bar{u}({\bf x}; \theta),  \bar{u}^{\prime}({\bf x}; \theta)]^{\top} = 
\nabla_{\theta}[ \hat{u}({\bf x}; \theta_0),  \hat{u}^{\prime}({\bf x}; \theta_0)]^{\top} 
= \nabla_{\theta} U({\bf x}; \theta_0).
\]
Substituting $\hat{u}({\bf x}; \theta)$ by the linearized model $\bar{u}({\bf x}; \theta)$ into~\eqref{eqn:loss-trajectory} and applying the lazy training assumption to the NTK
leads to the linearized loss dynamics
\begin{equation}\label{eqn:loss-trajectory-linearize}
	\begin{split}
		\frac{d \bar{I}_{\mathcal{X}}(\theta(t))}{dt} 
		= -\frac{\eta}{|\mathcal{X}|^2} \sum_{{\bf x_m}, {\bf x_n} \in \mathcal{X}}
		[\nabla_U \bar{W}_m(\theta(t))]^{\top}
		K({\bf x_m}, {\bf x_n}; \theta_0)
		[\nabla_U \bar{W}_n(\theta(t))], 
	\end{split}
\end{equation}
where 
\[
\bar{W}_n(\theta) = W({\bf x_n}, \bar{u}({\bf x_n}; \theta), \bar{u}^{\prime}({\bf x_n}; \theta))
\]
and
\[
\bar{I}_{\mathcal{X}}(\theta) = \frac{1}{|\mathcal{X}|} \sum_{{\bf x_n} \in \mathcal{X}} W({\bf x_n}, \bar{u}({\bf x_n}; \theta),  \bar{u}^{\prime}({\bf x_n}; \theta))
\]
are the linearization of the Lagrangian $W$ and the empirical loss~\eqref{eqn:ensemble-loss} at $\theta_0$, respectively, 
and 
$
K({\bf x_m}, {\bf x_n}; \theta_0)
$
is the NTK parameterized at the initial guess $\theta_0$.
It has been shown that when the minimum width of the DNN is sufficiently large, the NTK $K({\bf x}, {\bf x^{\prime}}; \theta_0)$ becomes independent of the initialization $\theta_0$~\cite{lee2019wide, arora2019exact} and we can define the asymptotic NTK (independent of the parameterization)
\begin{equation}\label{eqn:NTK-matrix}
	\bar{K}({\bf x}, {\bf x^{\prime}}) \triangleq \lim_{\text{NN width}\to \infty} \mathbb{E}_{\theta_0} \left\{K({\bf x}, {\bf x^{\prime}}; \theta_0)\right\} \in \mathbb{R}^{2 \times 2}.
\end{equation} 
Finally, we obtain the linearized loss dynamics of DRM (upon replacing $K$ 
by $\bar{K}$ and a vectorization of~\eqref{eqn:loss-trajectory-linearize})
\begin{equation}\label{eqn:loss-trajectory-NTK}
	\frac{d \bar{I}_{\mathcal{X}}(\theta(t))}{dt} 
	=
	-\frac{\eta}{|\mathcal{X}|^2}  [\nabla_{U} \bar{W}_{\mathcal{X}}(\theta(t))]^{\top} M_{\mathcal{X}} [\nabla_{U} \bar{W}_{\mathcal{X}}(\theta(t))],
\end{equation}
where the block Gram matrix
$
M_{\mathcal{X}}
$
consists of $\bar{K}({\bf x_m}, {\bf x_n})$ at its $(m, n)$-th block,
i.e., 
\begin{equation}\label{eqn:Gram-matrix}
	M_{\mathcal{X}} = \left(\bar{K}({\bf x_m}, {\bf x_n})\right)_{m,n = 1, \ldots, |\mathcal{X}|}  \in \mathbb{R}^{2|\mathcal{X}| \times 2|\mathcal{X}|},
\end{equation}
and $\bar{W}_{\mathcal{X}} = [\bar{W}_1, \ldots, \bar{W}_{|\mathcal{X}|}]^{\top} \in \mathbb{R}^{|\mathcal{X}| \times 1}$
and
$\nabla_U \bar{W}_{\mathcal{X}} = [\nabla_U \bar{W}_1, \ldots, \nabla_U \bar{W}_{|\mathcal{X}|}]^{\top} \in \mathbb{R}^{ 2|\mathcal{X}|\times 1}$.
% Note that $M_{\mathcal{X}}$ does not only contain information about $u$ but also its spatial gradient $\nabla u$.
%Since $M$ is PSD, it admits the spectral decomposition $M = Q^T \Lambda Q$, where $Q$ is an orthogonal matrix and $\Lambda = \text{diag}(\lambda_1, \ldots, \lambda_{(d+1)N})$ with $\lambda_1$
% $M_{\mathcal{X}}$ admits the spectral decomposition 
% \[M_{\mathcal{X}} = Q \Lambda Q^T,\] 
% where $Q = [q_1, \ldots, q_{2|\mathcal{X}|}]$ is an orthonormal matrix and $\Lambda = \text{diag}(\lambda_1, \ldots, \lambda_{2|\mathcal{X}|})$ with $\lambda_1 \geq \lambda_2 \geq \ldots \geq \lambda_{2|\mathcal{X}|} > 0$ and hence we can write
%     \begin{equation}\label{eqn:NTK-decomposed}
	%     \frac{d \bar{I}_{\mathcal{X}}(\theta(t))}{dt} = 
	%     -\eta\sum_{i=1}^{2|\mathcal{X}|} \lambda_i [\nabla_{U} \bar{W}_{\mathcal{X}}(\theta(t))q_i]^2,
	%     \end{equation}
%     where $q_i \in \mathbb{R}^{2|\mathcal{X}| \times 1}$ denotes the $i$-th column of the matrix $Q$.



We make two important observations from the loss dynamics~\eqref{eqn:loss-trajectory-NTK}: 1) Assuming $M_{\mathcal{X}}$ is positive definite, the convergence of the loss function $\bar{I}_{\mathcal{X}}(\theta(t))$
to a critical point is equivalent to the gradient of the Lagrangian vectors, i.e., $\nabla_{U} \bar{W}_{\mathcal{X}}(\theta(t))$, converges to zero;
% we expect that
% $\bar{I}_{\mathcal{X}}(\theta(t))$ converges to a local minimum and hence each $[\nabla_{U} \bar{W}_{\mathcal{X}}(\theta(t))q_i]$
% must converge to $0$ since all $\lambda_i > 0$;
2) If $\bar{I}_{\mathcal{X}}(\theta)$ is convex and bounded from below, $\theta(t)$ converges to the global minimum of $\bar{I}_{\mathcal{X}}(\theta)$.
However, the loss dynamics says nothing about the rate of convergence to a critical point. 

Therefore, we further assess the convergence rate of $\nabla_{U} \bar{W}_{\mathcal{X}}(\theta(t))$ to zero by considering its time evolution given by (derivation is postponed to~\ref{app:gradient-evolution})
\begin{equation}\label{eqn:gradient-evolution}
	\begin{split}
		\frac{d[\nabla_{U} \bar{W}_{\mathcal{X}}(\theta(t))]}{dt}
		=
		-\frac{\eta}{|\mathcal{X}|} D_{\mathcal{X}}(\theta(t)) M_{\mathcal{X}} [\nabla_{U} \bar{W}_{\mathcal{X}}(\theta(t))],
	\end{split}
\end{equation}
where the block diagonal matrix $D_{\mathcal{X}}(\theta(t))$ consists of $2\times 2$ Hessians of $\bar{W}_n \triangleq W(x_n, \bar{u}_n, \bar{u}_n^{\prime})$, i.e., 
\begin{equation*}
	D_{\mathcal{X}}(\theta(t)) = \text{diag}\left( \begin{bmatrix}
		\partial_{uu}^2 \bar{W}_n & \partial_{uu^{\prime}}^2 \bar{W}_n \\
		\partial_{u^{\prime}u}^2 \bar{W}_n & \partial_{u^{\prime}u^{\prime}}^2 \bar{W}_n
	\end{bmatrix}    \right)_{n=1, \ldots, |\mathcal{X}|} \in \mathbb{R}^{2|\mathcal{X}| \times 2|\mathcal{X}|}.
\end{equation*}
Now we are a in position to present the NTK theorem for DRM, which is a direct consequence of~\eqref{eqn:gradient-evolution}.
\begin{theorem}\label{thm:convergence}
	Suppose that 
	\begin{enumerate}
		\item the lazy training assumption (see e.g.,~\cite{chizat2019lazy}) is satisfied such that $D_{\mathcal{X}}(\theta(t)) \approx D_{\mathcal{X}} \triangleq D_{\mathcal{X}}(\theta_0)$;
		\item the Lagrangian $W$ is strictly convex in $(u, u^{\prime})$ such that the matrix $D_{\mathcal{X}}$ is positive definite;
		\item the Gram matrix $M_{\mathcal{X}}$ induced by the NTK~\eqref{eqn:NTK-matrix} is positive definite.
	\end{enumerate}
	Then, the asymptotic gradient (with respect to $u$ and $u^{\prime}$) dynamics of the Lagrangian $W$ in DRM is given by~\eqref{eqn:gradient-evolution}. 
	Moreover, we have
	\[
	[Q\nabla_{U} \bar{W}_{\mathcal{X}}(\theta(t))]^{\top} 
	=
	\text{e}^{-\eta \Lambda t /|\mathcal{X}|} [Q\nabla_{U} \bar{W}_{\mathcal{X}}(\theta_0)]^{\top},
	\]
	where we have used the spectral decomposition $D_{\mathcal{X}}M_{\mathcal{X}} = Q \Lambda Q^{\top}$ with 
	orthonormal matrix $Q = [q_1, \ldots, q_{2|\mathcal{X}|}]$ and diagonal matrix $\Lambda = \text{diag}(\lambda_1, \ldots, \lambda_{2|\mathcal{X}|})$ with $\lambda_1 \geq \lambda_2 \geq \ldots \geq \lambda_{2|\mathcal{X}|} > 0$.
\end{theorem}
A few remarks are in order.  First, the theorem suggests that the
specific convergence rate of
$\nabla_{U} \bar{W}_{\mathcal{X}}(\theta(t))$ along each direction
$q_i$ is determined by the corresponding eigenvalue $\lambda_i$. For
$\lambda_i \gg 0$, 
\[\bar{U}_{\mathcal{X}}(\theta(t)) = [\bar{U}_1(\theta(t)), \ldots, \bar{U}_{|\mathcal{X}|}(\theta(t))]^{\top} \in \mathbb{R}^{ 2|\mathcal{X}| \times 1}
\] 
with $\bar{U}_n(\theta) = [\bar{u}_n(x_n; \theta), \bar{u}_n^{\prime}(x_n; \theta)]^{\top} \in \mathbb{R}^{2\times 1}$
converges
fast along the direction $q_i$. Although for $\lambda_i \approx 0$, DNNs
have a significantly slower learning rate in the corresponding
direction $q_i$, preventing DNNs from learning the fine structure of the
solution.  
%Therefore, it is desirable to have a NTK spectrum that
%decays less rapidly than the exponential rate to avoid the spectral bias pathology. 
Motivated by this, we consider Fourier feature mapping to alleviate the spectrum bias issue in the next section.
Second, for a
non-convex Lagrangian $W$, the convergence of
$\nabla_{U} \bar{W}_{\mathcal{X}}(\theta(t))$ requires a more refined
analysis from variational calculus~\cite{dacorogna2007direct}, which
will be the focus of our future work. However, we empirically observed that in Section~\ref{sec:numerics} the Fourier feature mapping works equally well in the non-convex setting.
Finally, we point out that for
the type of non-convex variational problems considered in this work,
solving the corresponding Euler-Lagrange equations does not
necessarily lead to the correct minimizer and hence PINN is not
applicable.  Thus, DRM is the only option for solving variational
problem using neural networks.















% In general, the convergence rate of $\bar{U}_{\mathcal{X}}(\theta)$ along each $q_i$ depends on the specific form of the Lagrangian $W$.
% For a class of variational problems with $W$ satisfying
% \[
% M_{\mathcal{X}} [\nabla_{U} \bar{W}_{\mathcal{X}}]^{\top} \geq M_{\mathcal{X}} \bar{U}_{\mathcal{X}}^{\top}, 
% \]
% that is, $W(u, u^{\prime}) \geq u^2/2 + (u^{\prime})^2/2$ in the Riemannian metric induced by the NTK matrix $M_{\mathcal{X}}$,    
% we immediately have
% \[
% \frac{d \bar{U}_{\mathcal{X}}^{\top}}{dt} 
% \leq
% -\eta M_{\mathcal{X}} \bar{U}_{\mathcal{X}}^{\top}.
% \]
% Applying the Gronwall's lemma to the above inequality leads to
% \[
% Q^{\top} \bar{U}_{\mathcal{X}}^{\top}(\theta(t)) \leq Q^{\top} \bar{U}_{\mathcal{X}}(\theta_0)^{\top} \text{e}^{-\eta \Lambda t}, 
% \]
% which suggests that the specific exponential convergence rate along each direction $q_i$ is determined by the corresponding eigenvalue $\lambda_i$. For $\lambda_i \gg 0$, $\bar{U}_{\mathcal{X}}^{\top}(\theta(t))$ converges fast along the direction $q_i$. While for $\lambda_i \approx 0$, DNNs have a significantly slower rate in learning along the corresponding direction $q_i$, making it fail to learn the fine structure of the solution. 
% Therefore, it is desirable to have a NTK spectral that decays slowly to avoid the spectral bias pathology. 

% In summary, we comment that the convergence of DRM is generally guaranteed under mild assumptions. However, the specific convergence rate along each direction depends critically on the Lagrangian $W$ of the variational problem under consideration.    
% In comparison, due to its simple squared loss, PINN always has a more explicit convergence rate along the direction defined by the eigenvector of the NTK matrix~\cite{wang2021eigenvector}.
% Nevertheless, we have shown that, for a large class of variational problems, DRM has the same convergence rate along each direction as PINN and hence also suffers the spectral bias pathology. 
% Moreover, 


% We refer readers to~\cite{lu2021machine} for detailed comparison of PINN and DRM from the minimax optimality perspective.  










%and the DNN tends to first 
%``zero" the gradient $\nabla_{U} \bar{I}_{\mathcal{X}}(\theta(t))$ along the direction $Q \nabla_{U} \bar{I}_{\mathcal{X}}(\theta(t))$. 

% The asymptotic loss dynamics~\eqref{eqn:loss-trajectory-NTK} suggests that the loss is guaranteed to converge to a ``critical point" (i.e., a point at which $\nabla_U \bar{I}_{\mathcal{X}} = 0$) provided that $M$ is positive definite. Furthermore, 



\subsection{Fourier feature from the NTK perspective}
\label{subsec:FF}

\begin{figure}[!ht]
	\centering
	\includegraphics[width=1.0\textwidth]{Figure_2.png}
	\caption{Structure of Neural Network by applying Fourier feature mapping to the input layer.}
	\label{fig:NN+FF_structure}
\end{figure}

To alleviate the spectral bias of DRM, we apply a Fourier feature mapping $\delta$ to the input $\bf{x}$ before it is sent to the DNN. See Figure~\ref{fig:NN+FF_structure} for the simple architecture. 
The Fourier feature mapping has been widely used in various fields in machine learning, e.g., large-scale kernel regression and deep learning~\cite{rahimi2007random, FourierFeatures2020}.
However, to the best of our knowledge, the reason why Fourier feature mapping enables DNNs to learn high frequency solutions is not well understood from a theoretical perspective.
In this section, we provide a heuristic argument from the NTK perspective to justify the application of Fourier feature mapping for DRM.
For simplicity, we consider an one dimensional problem ($d=1$) and assume that the Lagrangian $W = W(x, u)$. The Fourier feature mapping is chosen to be $\delta(x) = [\sin x, \cos x] \in \mathbb{S}^1$, where $\mathbb{S}^1$ is the unit circle in $\mathbb{R}^2$.
Viewing the pair ${\bf y} = [\sin x, \cos x] \in \mathbb{S}^1$ as the input of the DNN, the dataset $\mathcal{X}$ is mapped to $\mathcal{Y} = \delta(\mathcal{X}) \subset \mathbb{S}^1$. 
Under these assumptions, the asymptotic NTK defined in~\eqref{eqn:NTK-matrix} becomes a scalar valued positive definite kernel
\[
\bar{K}({\bf y_1}, {\bf y_2})
= \lim_{\text{NN width}\to \infty} \mathbb{E}_{\theta_0} \left\{\nabla_{\theta} \hat{u}({\bf y_1} ;\theta_0) [\nabla_{\theta}\hat{u}({\bf y_2} ;\theta_0)]^{\top}\right\}
, \qquad {\bf y_1}, {\bf y_2} \in \mathbb{S}^1
\]
and $M_{\mathcal{Y}}$ reduces to the usual Gram matrix evaluated at the input set $\mathcal{Y}$ (recall~\eqref{eqn:Gram-matrix} for definition), i.e.,
\[
M_{\mathcal{Y}} = \bar{K}(\mathcal{Y}, \mathcal{Y}).
\]
Note that the above argument can be easily generalized to the case where $W = W(x, u, u^{\prime})$ by considering a matrix valued kernel $\bar{K}$.
In~\ref{app:eigenspectrum}, we show that the $k$-th eigenvalue of $M_{\mathcal{Y}}$ is approximately proportional to the $k$-th eigenvalue of the NTK $\bar{K}$ (see~\eqref{eqn:eigen-L} for definition). Therefore, one may study the eigenvalues of $\bar{K}$ when concerned with the decay rate of the eigenvalues of $M_{\mathcal{Y}}$. 
%Assume the DNN has a ReLU activation and zero initial bias
It has been shown that (Theorem 1 in~\cite{geifman2020similarity}), when restricted to $\mathbb{S}^1$, the $k$-th eigenvalue of the NTK $\bar{K}$ scales as $\mathcal{O}(k^{-2})$, meaning that the eigenvalue of $\bar{K}$ has a quadratic decay rate. 
For multi-scale problems whose NTK spectrum exhibits multiple scales, e.g., an exponential decay rate $\mathcal{O}(\mathrm{e}^{-k})$, the Fourier feature mapping may homogenize the convergence rate along each direction $q_i$ hence
alleviating the spectral bias issue of the dynamics~\eqref{eqn:gradient-evolution}.
In Section~\ref{sec:numerics}, we empirically demonstrate the benefit of Fourier feature mapping when applied to multi-scale variational problems. 




% the reproducing kernel Hilbert space (RKHS) associated with the NTK $\bar{K}$ coincides with the RKHS accociated with the Laplace kernel 
% \[
% K_{\text{Lap}}({\bf y_1}, {\bf y_2}) = \text{e}^{-\gamma\|{\bf y_1} - {\bf y_2}\|} =  \text{e}^{-\gamma \sqrt{2(1 - {\bf y_1}^{\top} {\bf y_2)}}},  \qquad {\bf y_1}, {\bf y_2} \in \mathbb{S}^1
% \] 
% i.e., 
% \[
% \mathcal{H}_{\bar{K}}(\mathbb{S}^1) = \mathcal{H}_{\text{Lap}}(\mathbb{S}^1).
% \]
% In other words, when restricted to a circle, an over-parameterized DNN works effectively like a Laplace kernel regressor. 







%%%%%%%%%%%%%%%%%%%%%%%%%%%%%%%%%%%%%%%%%%%%%%%%%%%%%%%%%%%%%%%%%%%%%%%%%
\section{Numerical Results \& Discussion}\label{sec:numerics}
We consider the following non-convex variational minimization problems: the first consists of a double well potential, $W(x) = (x^2-1)^2$ which leads to the following energy minimization problem: 
\begin{align}
	{\rm Minimize} \ I(u) = \int_0^1 (u_x^2-1)^2  \ dx  \qquad {\rm subject \ to} \qquad u(0) = u(1) = 0.
	\label{eqn:1D_Problem_1}
	%+ \varepsilon^2 u_{xx}^2
\end{align}

Note that the first component of the energy density is non-negative with zeros at $u_{x} = \pm 1$, which are often called zero-energy wells and correspond to the preferred phases of the problem. 
%The regularization term $\varepsilon^2 u_{xx}^2$ penalizes the transitions between the two zero-energy wells \cite{Kohn&Otto}. 
We note that for this particular problem, the minimum is attained: Carstensen showed that all Lipschitz continuous functions $u(x)$, with slope $u_{x} = \pm 1$ almost everywhere, minimize $I$ \cite{Carstensen_2005}. The energy of such function is $I = 0$. It should be emphasized that while deriving the Euler-Lagrange equation for non-convex problems like~\eqref{eqn:1D_Problem_1} is possible as shown below:
\begin{align}
	\frac{d}{dx} [u_x (u_x^2-1)] = 0 
	\label{eqn:Euler-Lagrange}
	%- \frac{\varepsilon^2}{2}\frac{d^2}{dx^2} [u_{xx}] = 0,
\end{align}
its solution $u(x) = 0$ does not minimize~\eqref{eqn:1D_Problem_1}. Consequently, applying the PINN algorithm to the strong form equations is not viable, as the algorithm would inevitably converge to the trivial solution.

The second benchmark problem is a variation of the double well potential, where a lower order term of the form $u^2$ is introduced, generating the following minimization problem:
\begin{align}
	{\rm Minimize} \ I(u) = \int_0^1 (u_x^2-1)^2 + u^2 \ dx \qquad {\rm subject \ to} \qquad u(0)=u(1) = 0.
	\label{eqn:1D_Problem_2}
	%+ \varepsilon^2 u_{xx}^2
\end{align}
%, in the absence of $\varepsilon^2 u_{xx}^2$, 
We note that no minimizer exists for this problem. The infimum, although zero, cannot be attained since there is no function that satisfies $u = 0$ and $u_{x} = \pm 1$ almost everywhere. Minimizing sequences oscillate and converge weakly, but not strongly, to zero \cite{Carstensen_2005, Muller1993, Kohn&Otto}. This is the first simple  example that demonstrates how minimization can lead to fine scale oscillations or microstructure formation.

Finally, the third problem considered here is the $2D$ scalar problem for twin branching, which takes the following form:  
\begin{align}
	{\rm Minimize} \ I(u) = \int_{\Omega} u_x^2 + (u_y^2-1)^2  \ dx dy \qquad {\rm subject \ to} \qquad u=0  \ {\rm on} \ \partial \Omega,  
	\label{eqn:2D_Problem}
	%+ \varepsilon^2 u_{yy}^2
\end{align}
%in the absence of $\varepsilon^2 u_{yy}^2$ 
where $\Omega = [0,1]^2$. As in the previous problem, no minimizers exist since there is no function that can satisfy the integrand and boundary conditions at the same time, leading to minimizing sequences that develop rapid oscillations \cite{Muller}.

Recall that Chen \textit{et al}.\ applied DRM to non-convex energy problems in $1D$ and $2D$, similar to the ones described above. We now discuss the differences and similarities between our benchmark problems and those examined in~\cite{Chen&Rosakis2023}. Comparable to~\eqref{eqn:1D_Problem_1}, the $1D$ minimization problem in~\cite{Chen&Rosakis2023} is comprised of a double-well potential energy density subject to Dirichlet boundary conditions. Both minimization problems consist of a minimum energy ($I =0$) which can be obtained through multiple continuous functions $u(x)$, leading to loss of uniqueness. A key distinction lies in the minima locations; in~\cite{Chen&Rosakis2023}, they occur at $0$ and $1$, while in ~\eqref{eqn:1D_Problem_1}, they occur at $-1$ and $1$. Our Dirichlet boundary conditions are fixed at $0$, contrasting with~\cite{Chen&Rosakis2023} where the left boundary is fixed at $0$ and the right boundary is fixed at $\gamma$ where $\gamma \in \mathbb{R}$. This leads to solutions with slopes $u_x = 0$ and $1$ in~\cite{Chen&Rosakis2023}, whereas the solutions to~\eqref{eqn:1D_Problem_1} have slopes $u_x = \pm 1$.

Similarly, the $2D$ minimization problem in~\cite{Chen&Rosakis2023} mirrors features found in~\eqref{eqn:2D_Problem}. Both problems consist of a double well energy potential and are subject to Dirichlet boundary conditions, which yield to minimizing sequences with rapid oscillations but no actual minimizers. The main differences between~\eqref{eqn:2D_Problem} and the 2D problem in~\cite{Chen&Rosakis2023} lie in the minima locations of the energy well potential ($(\pm 1,0)$ in~\eqref{eqn:2D_Problem} vs. $(0,0)$ and $(1,0)$ in~\cite{Chen&Rosakis2023}). The Dirichlet boundary conditions are set to $0$ across the boundary in~\eqref{eqn:2D_Problem}, while Chen \textit{et al}.\ set $u(x,y) = \gamma x$ with $\gamma \in \mathbb{R}$ in~\cite{Chen&Rosakis2023}.

Given that the distinctions mentioned above are cosmetic and do not alter the fundamental structure of the minimization problems, we anticipate the hypothesis and conclusions articulated in \cite{Chen&Rosakis2023}, particularly the hypothesis that increasing the DNN increases the number of twin bands for the $2D$ problem, remain true for~\eqref{eqn:2D_Problem}. We test this hypothesis numerically in the sections below.  


\label{sec:Numerics}
\subsection{$1D$ Benchmark Problem \# 1}\label{sec:1D_results_1}
We start our discussion by approximating the solution to~\eqref{eqn:1D_Problem_1} using DRM without Fourier mapping (as described in~\cite{Chen&Rosakis2023}) and compare the results with the new algorithm: DRM with Fourier mapping (DRM\&FM). In both cases, a fully connected feed-forward neural network with an input layer, multiple hidden layers and an output layer is constructed. The input layer consists of one node (for the $x$ coordinate of our problem),  each hidden layer consists of $128$ nodes, and the output layer consists of one node  (used to output the approximated solution $\hat{u}$). Consistent with~\cite{Chen&Rosakis2023}, we apply the ReLU activation function in each layer.  To accelerate training, we use Adams Optimizer on a mini-batch size of $128$ collocation points sampled uniformly, with an initial learning rate $\eta = 10^{-4}$. We implement a cosine annealing schedule that decreases the learning rate to zero over the course of the simulation. The boundary conditions are enforced using the penalty approach with a penalty parameter set to $\lambda = 500$.

%We approximate the minimizing solutions of the non-regularized version of~\eqref{eqn:1D_Problem_1} by setting $\varepsilon = 0$. 
Recall that there exist multiple solutions that minimize~\eqref{eqn:1D_Problem_1}: namely, any function $u(x)$ with slope $u_x = \pm 1$ almost everywhere minimizes the functional $I(u)$. In Figure~\ref{fig:1D_problem_noFF} we present the minimizing solutions generated by DRM with no Fourier mapping as we vary the depth of the network while setting the learning rate initially to $\eta = 10^{-4}$.  We see that for this particular benchmark problem, increasing the depth of the DNN does not generate high-frequency solutions, analogous to the increased number of twin bands of the 2D problem discussed in~\cite{Chen&Rosakis2023}. Solutions with one transition between the two preferred interfaces ($u_x = \pm 1$) are generated for a DNN with $5$, $7$ and $9$ hidden layers (see Fig~\ref{subfig:ux_1D_no_FF}). %We approximate the total energy of the solutions using a simple quadrature and obtain that $I(u)$ increases with the depth of the DNN: $I(u) \approx 4.8\times 10^{-4}$ for a DNN with 5 hidden layers, $I(u) \approx 6.2\times 10^{-4}$ and $I(u) \approx 1.6\times 10^{-3}$ for 7 and 9 hidden layers respectively. We see that $I(u)$ increases with the depth of the DNN, however the increase can be attributed to the noise in $u_x$ as seen in Fig.~\ref{subfig:ux_1D_no_FF}.


\begin{figure}[!ht]
	\centering
	\subfigure[~]{\includegraphics[width=0.31\textwidth]{Figure_3a.png}\label{subfig:u_1D_no_FF}}
	\subfigure[~]{\includegraphics[width=0.31\textwidth]{Figure_3b.png}\label{subfig:ux_1D_no_FF}}
	\caption{(a)~DRM approximation to~\eqref{eqn:1D_Problem_1} with ReLU activation function, $\eta = 1.0\times 10^{-4}$ after $100000$ epochs with DNN structure of $5$, $7$ and $9$ hidden layers. (b)~ The derivative $u_x$ of the DRM approximation to~\eqref{eqn:1D_Problem_1}.}
	\label{fig:1D_problem_noFF}
\end{figure}

Figure~\ref{fig:1D_problem_FF} displays the solution generated by DRM with Fourier feature mapping under the same conditions. Recall that the information passes from the input layer, through a Fourier mapping of the form $\delta({\bf x}) = \left[\sin(2^i\pi {\bf x}), \cos(2^i \pi {\bf x})\right]$ with $i = 2, 3, 4$ and $\bf{x} \in \mathbb{R}$, to the hidden and output layers. %Note that modify the Fourier mapping $\delta({\bf x})$ by removing ${\bf x}$ from the mapping because we know that the minimizing solutions are periodic over the domain.
We observe that the frequency of the mapping can be leveraged to generate minimizing solutions of high frequency, independently of the depth of the DNN.  
When passing a Fourier mapping of frequency $4\pi$ as shown in Fig.~\ref{fig:u_1D_FF=2} ($8\pi$ as shown in Fig.~\ref{fig:u_1D_FF=3}), we generate a solution with $4$ ($8$) transitions between preferred states, independently of the depth of the DNN. When applying a Fourier mapping of frequency $16\pi$ however, we get mixed results: implementing a DNN with $5$ and $7$ hidden layers leads to a solution with $32$ transitions between states (as shown by the black and red dotted lines in Fig.~\ref{fig:u_1D_FF=4}), while a $9$ layer DNN leads to a solution with $16$ transitions between preferred states (as shown by blue dashed lines). Figure~\ref{fig:1D_problem_FF} shows that increasing the frequency of the Fourier mapping increases the number of transitions between the preferred states but one cannot quantify the relationship between mapping frequency and number of transitions within the domain. 



\begin{figure}[ht!]
	\centering
	\subfigure[$i = 2$]{\includegraphics[width=0.31\textwidth]{Figure_4a.png}\label{fig:u_1D_FF=2}}
	\subfigure[$i = 3$]{\includegraphics[width=0.31\textwidth]{Figure_4b.png}\label{fig:u_1D_FF=3}}
	\subfigure[$i = 4$]{\includegraphics[width=0.31\textwidth]{Figure_4c.png}\label{fig:u_1D_FF=4}}
	\caption{DRM approximation to~\eqref{eqn:1D_Problem_1} where a NN with $5$,$7$ and $9$-hidden layers, ReLU activation function, $\eta = 1.0\times 10^{-4}$ and Fourier mapping of frequency $\delta({\bf x}) = \left[\sin(2^i\pi {\bf x}),\cos(2^i\pi {\bf x})\right] $ after $100000$ epochs. }
	\label{fig:1D_problem_FF}
\end{figure}
%%%%%%%%%%%%%%%%%%%%%%%%%%%%%%%%%%%%%%%%%%%%%%%%%%%%%%%%%%%%%%%%%%%%%%%%%%%%%%%%%%%
\subsection{$1D$ Benchmark Problem \# 2}\label{sec:1D_results_2}
We now discuss how DRM alone and DRM with Fourier mapping (DRM\&FM) approximate the solution sequences to the second benchmark problem given by~\eqref{eqn:1D_Problem_2}. Recall that no minimizer exists for this problem since there is no function that satisfies the conditions $u = 0$ and $u_x = \pm 1$ everywhere. Figure~\ref{fig:1D_problem_2_noFF} shows the DNN approximation of the minimizing solution to~\eqref{eqn:1D_Problem_2} as the depth of the DNN increases with no Fourier mapping after $200,000$ epochs (First Row) and $500,000$ epochs (Second Row). We observe that increasing the depth of DNN does not consistently increase the number of transitions between the preferred states. Increasing the depth of the DNN from $3$ to $5$ hidden layers increases the number of transitions for $200,000$ epochs. However, the number of transitions decreases as the depth of the DNN is increased from $5$ to $7$ hidden layers. A similar occurrence can be observed in the second row of Fig.~\ref{fig:1D_problem_2_noFF} where our simulations are run for $500,000$ epochs. In this case, increasing the depth of the DNN from $3$ to $5$ hidden layers decreased the number of transitions while increasing the depth from $5$ to $7$ hidden layers increased the number of transitions between preferred states. Based on our simulations, we can say that increasing the depth of the DNN does not consistently generate high-frequency solutions for the $1D$ benchmark problem given by~\eqref{eqn:1D_Problem_2}. We also note that a DNN with 7 hidden layers run for $500,000$ epochs was able to generate a minimizing sequence with $16$ transitions between preferred states. 

\begin{figure}[ht!]
	\centering
	\subfigure[NN: $3\times 128$]{\includegraphics[width=0.31\textwidth]{Figure_5a.png}\label{subfig:u2_1D_noFF_d=3_200K}}
	\subfigure[NN: $5\times 128$]{\includegraphics[width=0.31\textwidth]{Figure_5b.png}\label{subfig:u2_1D_noFF_d=5_200K}}
	\subfigure[NN: $7\times 128$]{\includegraphics[width=0.31\textwidth]{Figure_5c.png}\label{subfig:u2_1D_noFF_d=7_200K}}\\
	\subfigure[NN: $3\times 128$]{\includegraphics[width=0.31\textwidth]{Figure_5d.png}\label{subfig:u2_1D_noFF_d=3_500K}}
	\subfigure[NN: $5\times 128$]{\includegraphics[width=0.31\textwidth]{Figure_5e.png}\label{subfig:u2_1D_noFF_d=5_500K}}
	\subfigure[NN: $7\times 128$]{\includegraphics[width=0.31\textwidth]{Figure_5f.png}\label{subfig:u2_1D_noFF_d=7_500K}}
	\caption{{\bf First Row (a)-(c)}: DRM approximation to~\eqref{eqn:1D_Problem_2} with ReLU activation function, $\varepsilon = 0$, $\eta = 1.0\times 10^{-4}$ and cosine annealing after $200000$ epochs.
		{\bf Second Row (d)-(f):} Row: DRM approximation to~\eqref{eqn:1D_Problem_2} with ReLU activation function, $\varepsilon = 0$, $\eta = 1.0\times 10^{-4}$ and cosine annealing after $500000$ epochs. }
	\label{fig:1D_problem_2_noFF}
\end{figure}
Figure~\ref{fig:1D_problem_2_FF} shows the minimizing solutions that are obtained by the DRM with a DNN structure of $3$ hidden layers and Fourier mapping of frequency $2\pi$, $4\pi$ and $8\pi$  after $200,000$ and $500,000$ epochs. We see here that the Fourier mapping with frequency $2\pi$ as shown in Figs~\ref{subfig:u2_1D_FF=1_200K} and~\ref{subfig:u2_1D_FF=1_500K} enables us to generate a solution of $12$ transitions between the two preferred states, a result that is comparable with the DRM approximation solution of a DNN of $7$ hidden layers as shown in Figs.~\ref{subfig:u2_1D_noFF_d=7_200K} \& \ref{subfig:u2_1D_noFF_d=7_500K}. Note the solution approximation consists of $11$ transitions for $200,000$ epochs and $16$ transitions for $500,000$ epochs. We note that DRM\&FM enables us to keep the number of hidden layers in the DNN fixed and generate minimizing solutions with more transitions, such as the ones shown in Fig.~\ref{fig:1D_problem_2_FF}. While it seems that the number of transitions between preferred states increases with the frequency of the Fourier mapping, the authors did not investigate the relationship between the frequency of the Fourier mapping and the number of transitions within the solution for this $1D$ problem. 
\begin{figure}[ht!]
	\centering
	\subfigure[$i = 1$]{\includegraphics[width=0.31\textwidth]{Figure_6a.png}\label{subfig:u2_1D_FF=1_200K}}
	\subfigure[$i = 2$]{\includegraphics[width=0.31\textwidth]{Figure_6b.png}\label{subfig:u2_1D_FF=2_200K}}
	\subfigure[$i = 3$]{\includegraphics[width=0.31\textwidth]{Figure_6c.png}\label{subfig:u2_1D_FF=3_200K}}\\
	\subfigure[$i = 1$]{\includegraphics[width=0.31\textwidth]{Figure_6d.png}\label{subfig:u2_1D_FF=1_500K}}
	\subfigure[$i = 2$]{\includegraphics[width=0.31\textwidth]{Figure_6e.png}\label{subfig:u2_1D_FF=2_500K}}
	\subfigure[$i = 3$]{\includegraphics[width=0.31\textwidth]{Figure_6f.png}\label{subfig:u2_1D_FF=3_500K}}    
	\caption{{\bf First Row (a)-(c):} DRM\&FM approximation to~\eqref{eqn:1D_Problem_2} with $3 \times 128$ NN (3 hidden layers), ReLU activation function, $\varepsilon = 0$, $\eta = 1.0\times 10^{-4}$ and Fourier feature of frequency $\delta({\bf x}) = \left[\sin(2^i\pi {\bf x}),\cos(2^i\pi {\bf x})\right] $ after $200000$ epochs. {\bf Second Row (d)-(f):} DRM\&FM approximation under the same conditions after $500,000$ epochs. }
	\label{fig:1D_problem_2_FF}
\end{figure}

%%%%%%%%%%%%%%%%%%%%%%%%%%%%%%%%%%%%%%%%%%%%%%%%%%%%%%%%%%%%%%%%%%%%%%%%%%%%%%%%
\subsection{$2D$ Benchmark Problem}\label{sec:2D_results}
We now turn to the $2D$ twin branching problem given by~\eqref{eqn:2D_Problem} and investigate whether the DRM\&FM method can be extended to generate solutions to $2D$ microstructure problems. Recall that, similar to~\eqref{eqn:1D_Problem_2}, this problem does not have a minimizer since there are no functions that can minimize the integrand and satisfy the Dirichlet boundary conditions at the same time, leading to microstructure behavior. The ideal minimizer would be a function  $u(x,y)$ such that $u_y = \pm 1$, $u_x = 0$ in $\Omega$ and $u = 0$ on $\partial \Omega$. Such function does not exist, leading to minimizing sequences with fine scale oscillations instead. 
We attempt to capture these minimizing sequences using a DNN similar in structure to the ones implemented in Secs.~\ref{sec:1D_results_1} \& \ref{sec:1D_results_2}. We adapt the DNN to minimize the $2D$ problem in~\eqref{eqn:2D_Problem} through the following changes: the input layer consists of two nodes, one for each coordinate $x$ and $y$ of our $2D$ domain,  the activation function used is of the form 
$\sigma(x) = \sqrt{x^2 + \rho^2}$, where $\rho = 0.1.$ This activation function is a variation of the SmReLU activation function used in~\cite{Chen&Rosakis2023} to better suit the problem considered here.
The DRM is run with Adams Optimizer for $300,000$ epochs with a total number of $N = 1000$ collocation points sampled uniformly across the domain ($N_{int} = 600$ in the interior  and N$_b = 400$: $100$ uniformly sampled points across each boundary). Note that we set the initial learning rate to $\eta = 10^{-4}$ and apply cosine annealing as in the 1D case. %The regularization term, normally used in traditional numerical methods such as FEM is initially set to zero. We investigate its effect on the minimizing sequences by taking nonzero values of $\varepsilon$ in Sec.~\ref{sec:2D_problem_epsilon}.

Figure~\ref{fig:2D_problem_no_FF_eps=0.0} displays the minimizing sequences to~\eqref{eqn:2D_Problem} (we plot $u_y$ instead of $u$ to show the transition between the two preferred states $u_y = \pm 1$) as we increase the number of hidden layers in the DNN. Here, as in Sec.~\ref{sec:1D_results_1}, we consider a DNN with $3$, $5$ and $7$ hidden layers respectively and no Fourier mapping. We see that as the depth of the DNN increases, the number of bands stays the same. In fact, for a network with $7$ hidden layers, the solution is stuck to an unstable state ($u = 0$). We note that for this particular problem, increasing the depth of the DNN does not generate minimizing sequences with a large number of twin bands (high frequency). It seems like the depth of the NN is hindering the DNN from converging to a minimum: instead, it is stuck at a saddle point in the energy density functional of~\eqref{eqn:2D_Problem}.

In contrast, when a Fourier mapping of the form $\delta({\bf x}) = \left[{\bf x}, \sin(2^i\pi {\bf x}),\cos(2^i\pi {\bf x})\right]$ where $i = 1-4$  and ${\bf x} \in \mathbb{R}^2$ is applied, the number of transitions between preferred states in $u_y$ (or number of twin bands as described in~\cite{Chen&Rosakis2023}) increase (see Figure~\ref{fig:2D_problem_FF_eps=0.0}). Note that we modify the Fourier mapping by including ${\bf x}$ because a periodic solution is no longer a minimizer of the problem and we no longer expect a periodic solution in the domain. We hypothesize that applying a Fourier mapping of any frequency allows the DRM to converge to a minimizing sequence quicker than if no Fourier mapping was applied (compare Figs.~\ref{subfig:uy_FF=1_epsbar=0.0}-\ref{subfig:uy_FF=4_epsbar=0.0} with Fig.~\ref{subfig:uy_no_FF_depth=3}). We observe needle like structures forming around $x = 0$ and $x = 1$ when a Fourier mapping of low frequency is applied (see Fig.~\ref{subfig:uy_FF=1_epsbar=0.0}) but these needles do not fully grow to form additional bands in the course of our simulation. A similar behavior can be observed in Figs.~\ref{subfig:uy_FF=2_epsbar=0.0}-\ref{subfig:uy_FF=4_epsbar=0.0}: needle like structures are formed around $x = 0,1$ but these structures get smaller as the frequency of the Fourier mapping increases. Additionally, we observe that the number of twin bands increases as the frequency of the Fourier mapping increases: there are $4$ transitions between states when the frequency is set to $2 \pi$, $8$ transitions when the frequency is $4\pi$, $15$ transitions when the frequency is $8\pi$ and $32$ transitions when the frequency is $16\pi$ (See Figs.~\ref{subfig:uy_FF=2_epsbar=0.0}-\ref{subfig:uy_FF=4_epsbar=0.0}).
We observe that the minimizing solutions are noisy as the Fourier frequency increases and we attribute this noise to the fact that ~\eqref{eqn:2D_Problem} has no minimum. We emphasize that incorporating Fourier feature mapping into the DRM does not alter the number of collocation points used in the simulations ($N=1000$ in 2D case and $N = 128$ in 1D). This approach stands in sharp contrast to traditional methods like FEM, which depend heavily on mesh-size refinement to resolve the microstructure. 
%We investigate the effect that this parameter has on the minimizing sequences in the next section.

\begin{figure}[ht!]
	\centering
	\subfigure[NN: $3 \times 128$]{\includegraphics[width=0.31\textwidth]{Figure_7a.png}\label{fig:uy_2D_no_FF_5_eps=0}\label{subfig:uy_no_FF_depth=3}}
	\subfigure[NN: $5 \times 128$]{\includegraphics[width=0.31\textwidth]{Figure_7b.png}\label{fig:uy_2D_no_FF_7_eps=0}\label{subfig:uy_no_FF_depth=5}}
	\subfigure[NN: $7 \times 128$]{\includegraphics[width=0.31\textwidth]{Figure_7c.png}\label{fig:uy_2D_no_FF_9_eps=0}\label{subfig:uy_no_FF_depth=7}}
	\caption{DRM approximation to~\eqref{eqn:2D_Problem} with activation function $\sigma(x) = \sqrt{x^2+\rho^2}, \rho = 0.1$, $\eta = 1.0\times 10^{-4}$ and no Fourier Feature after $300000$ epochs.}
	\label{fig:2D_problem_no_FF_eps=0.0}
\end{figure}


\begin{figure}[ht!]
	\centering
	\subfigure[$i = 1$]{\includegraphics[width=0.31\textwidth]{Figure_8a.png}\label{subfig:uy_FF=1_epsbar=0.0}}
	\subfigure[$i = 2$]{\includegraphics[width=0.31\textwidth]{Figure_8b.png}\label{subfig:uy_FF=2_epsbar=0.0}}
	\subfigure[$i = 3$]{\includegraphics[width=0.31\textwidth]{Figure_8c.png}\label{subfig:uy_FF=3_epsbar=0.0}}
	\subfigure[$i = 4$]{\includegraphics[width=0.31\textwidth]{Figure_8d.png}\label{subfig:uy_FF=4_epsbar=0.0}}
	\caption{DRM approximation to~\eqref{eqn:2D_Problem} with $3 \times 128$ NN and Fourier feature of frequency $\delta({\bf x}) = \left[{\bf x},\sin(2^i\pi {\bf x}),\cos(2^i\pi {\bf x})\right]$ with $\eta = 1.0\times 10^{-4}$ after $300000$ epochs.}
	\label{fig:2D_problem_FF_eps=0.0}
\end{figure}
%%%%%%%%%%%%%%%%%%%%%%%%%%%%%%%%%%%%%%%%%%%%%%%%%%%%%%%%%%%%%%%%%%%%%%%%%%%%%%%%%%
\subsection{Regularized 2D Problem \& Fourier Mapping}\label{sec:2D_problem_epsilon}
Regularization is frequently used to ensure the existence of solutions to nonconvex minimization problems while also determining the length scale and fine geometry of the resulting microstructures~\cite{Muller1993, Kohn&Otto, Muller, Kohn1992}. This is achieved by adding a high-gradient term to the energy density  $W$ in~\eqref{eqn:variationalproblem}. Traditional numerical methods leverage this approach to identify the microstructure's length scale~\cite{Muller1993} and predict specific microstructure dynamics~\cite{Dondl2016}.
In this context, we consider the regularized 2D problem:
\begin{align}
	{\rm Minimize} \ I(u) = \int_{\Omega} u_x^2 + (u_y^2-1)^2 + \varepsilon^2 u_{yy}^2 \ dx dy \qquad {\rm subject \ to} \qquad u=0  \ {\rm on} \ \partial \Omega,  
	\label{eqn:2D_Problem_reg}
\end{align}
and investigate how Fourier mapping and the regularization term interact in generating high-frequency solutions to the regularized minimization problem in 2D. Recall that  $u_x$ prefers to be $0$ while $u_y$ jumps between $\pm 1$. The additional term $\varepsilon^2 u_{yy}^2$ in~\eqref{eqn:2D_Problem_reg} penalizes these transitions, facilitating the formation of fine structures by reducing the surface energy associated with the high-gradient contributions~\cite{Kohn1992}.

Figure~\ref{fig:2D_problem_FF_eps=0.1by16} shows the graph of the DRM generated solutions ($u_y$ instead of $u$) when $\varepsilon = 0.1/16$. We see that introducing a regularization term generates smooth minimizing sequences throughout the domain independently of whether a Fourier mapping is applied, though the Fourier mapping enables the method to generate solutions with more twin bands for large frequencies.  Comparing Figs.~\ref{subfig:uy_no_FF_depth=3} and ~\ref{fig:2D_problem_FF_eps=0.0} with Fig.~\ref{fig:2D_problem_FF_eps=0.1by16}, we observe that the regularization term helps the DRM generate smooth and symmetric solutions with smoother interfacial transitions and uniform microstructure length scales.  

When increasing $\varepsilon$ further, we observe that the DRM method generates minimizing sequences with smoother interfacial transitions and larger microstructure length scales as shown in Fig.~\ref{fig:2D_problem_FF_eps=0.1by4}.  Additionally, we observe that, for $\varepsilon = 0.1/4$, when applying Fourier mapping of frequency $4\pi$ and $8\pi$, the DRM generates the same sequence (with 8 transitions) while the same Fourier mappings  and different values of $\varepsilon$ ($\varepsilon = 0.1/16$ and $\varepsilon = 0$) generate sequences with $8$ and $15$ transitions respectively (see Figs.~\ref{subfig:uy_FF=3_epsbar=0.0}, ~\ref{subfig:uy_FF=3_eps=0.1by16} and~\ref{subfig:uy_FF=3_eps=0.1by4}). A similar behavior is observed when applying a Fourier mapping of frequency $16\pi$: DRM generates a sequence with $16$ twin bands when $\varepsilon = 0.1/4$ and a sequence with $32$ twin bands for smaller values of $\varepsilon$ (compare Figs.~\ref{subfig:uy_FF=4_epsbar=0.0} with Figs.~\ref{subfig:uy_FF=4_eps=0.1by16} and ~\ref{subfig:uy_FF=4_eps=0.1by4}). This is perhaps not surprising since the regularization term imposes an upper bound on the number of interfaces that can be generated for a value of $\varepsilon$. 

%Figure~\ref{fig:2D_problem_FF_eps=0.0} shows that one no longer needs the regularization term to generate high-frequency minimizing solutions, however the term is still beneficial if one is interested in generating symmetric and smooth solutions that do not have any numerical noise (solutions with equidistant bands).
\begin{figure}[ht!]
	\centering
	\subfigure[no FF]{\includegraphics[width=0.31\textwidth]{Figure_9a.png}\label{subfig:uy_no_FF_eps=0.1by16}}
	\subfigure[$i = 1$]{\includegraphics[width=0.31\textwidth]{Figure_9b.png}\label{subfig:uy_FF=1_eps=0.1by16}}
	\subfigure[$i = 2$]{\includegraphics[width=0.31\textwidth]{Figure_9c.png}\label{subfig:uy_FF=2_eps=0.1by16}}\\
	\subfigure[$i = 3$]{\includegraphics[width=0.31\textwidth]{Figure_9d.png}\label{subfig:uy_FF=3_eps=0.1by16}}
	\subfigure[$i = 4$]{\includegraphics[width=0.31\textwidth]{Figure_9e.png}\label{subfig:uy_FF=4_eps=0.1by16}}
	\caption{DRM approximation to~\eqref{eqn:2D_Problem} with $3 \times 128$ NN and Fourier feature of frequency $\delta({\bf x}) = \left[{\bf x}, \sin(2^i\pi {\bf x}),\cos(2^i\pi {\bf x})\right]$. The activation function used is $\sigma(x) = \sqrt{x^2+\rho^2}$ with $\rho = 0.1$, $\varepsilon = 0.1/16$, $\eta = 1.0\times 10^{-4}$ after $300000$ epochs.}
	\label{fig:2D_problem_FF_eps=0.1by16}
\end{figure}


\begin{figure}[ht!]
	\centering
	\subfigure[no FF]{\includegraphics[width=0.31\textwidth]{Figure_10a.png}\label{subfig:uy_no_FF_eps=0.1by4}}
	\subfigure[$i = 1$]{\includegraphics[width=0.31\textwidth]{Figure_10b.png}\label{subfig:uy_FF=1_eps=0.1by4}}
	\subfigure[$i = 2$]{\includegraphics[width=0.31\textwidth]{Figure_10c.png}\label{subfig:uy_FF=2_eps=0.1by4}}
	\subfigure[$i = 3$]{\includegraphics[width=0.31\textwidth]{Figure_10d.png}\label{subfig:uy_FF=3_eps=0.1by4}}
	\subfigure[$i = 4$]{\includegraphics[width=0.31\textwidth]{Figure_10e.png}\label{subfig:uy_FF=4_eps=0.1by4}}
	\caption{DRM approximation to~\eqref{eqn:2D_Problem} with $3 \times 128$ NN and Fourier feature of frequency $\delta({\bf x}) = \left[{\bf x}, \sin(2^i\pi {\bf x}),\cos(2^i\pi {\bf x})\right]$ with $\varepsilon = 0.1/4$, $\eta = 1.0\times 10^{-4}$ after $300000$ epochs.}
	\label{fig:2D_problem_FF_eps=0.1by4}
\end{figure}

\section{Conclusions} 
\label{sec:Concl}
This work employs DRM in conjunction with Fourier feature mapping (DRM\&FM) to solve non-convex minimization problems relevant in microstructure applications. We consider three benchmark problems: two minimization problems in $1D$ given by~\eqref{eqn:1D_Problem_1} and~\eqref{eqn:1D_Problem_2} and one in $2D$ given by~\eqref{eqn:2D_Problem}. These problems are challenging to solve since they often do not possess a global minimum (see~\eqref{eqn:1D_Problem_2} \& \eqref{eqn:2D_Problem}) or a global minimum exists (as in~\eqref{eqn:1D_Problem_1}), but there exist multiple functions that can yield such minimum.   

To tackle these challenges, we employ DRM in conjunction with Fourier feature mapping to generate high frequency, multiscale solutions. The method uses a DNN comprised of an input layer, a Fourier feature mapping of the form $\delta({\bf x}) =\left[{\bf x}, \sin(2^i\pi {\bf x}),\cos(2^i\pi {\bf x})\right]$, multiple hidden layers and an output layer.  Utilizing NTK theory, we demonstrate that the DRM as implemented in~\cite{Chen&Rosakis2023} suffers from spectral bias pathology: the rate at which the DNN learns minimizing solutions is determined by the largest eigenvalue of the NTK where $\lambda_i\gg 0$. To explore multiple solutions effectively, a desirable NTK should have eigenvalues $\lambda_i \approx 0$ to avoid spectral bias pathology.

Our heuristic analysis shows that the application of Fourier feature mapping results in a quadratic decay NTK eigenspectrum $\lambda_i \approx 0$, enabling our method DRM\&FM to generate high frequency, multiscale solutions. Simulations confirm the effectiveness of DRM\&FM in generating such solutions for all three benchmark problems. In contrast to the method proposed in~\cite{Chen&Rosakis2023}, simply increasing the depth of the neural network does not produce high-frequency solutions for our benchmark problems. However, our approach achieves this by keeping the network depth fixed and incorporating a Fourier mapping. 

While minimizing solutions may appear noisy without a regularization term, this capability still represents a significant advantage over the Finite Element Method (FEM). However, solving these types of problems remains challenging due to the rough energy landscape, which lacks well-defined minima and can hinder the algorithm’s training and solution generation. To address this issue, we considered a regularized minimization problem in 2D. We observed that incorporating a regularization term (Sec.~\ref{sec:2D_problem_epsilon}) smooths the energy landscape, facilitates training and produces symmetric, smooth solutions for small values of $\varepsilon$. As the value of $\varepsilon$ increases, we observe that the solutions generated by the method are low-frequency solutions.  

While DRM with Fourier mapping presents a mesh-free and computationally efficient algorithm, its nonlinear nature lacks a theoretical foundation to quantify solution accuracy for the considered minimization problems. We encourage the research community to develop such a theory in the near future.





\section{Acknowledgment}
Research was sponsored by the Army Research Laboratory and was accomplished under
Cooperative Agreements Number W911NF-22-2-0090 and W911NF-23-2-0139. The views and conclusions contained in this
document are those of the authors and should not be interpreted as representing the official
policies, either expressed or implied, of the Army Research Laboratory or the U.S. Government.
The U.S. Government is authorized to reproduce and distribute reprints for Government purposes
notwithstanding any copyright notation herein.

%%%%%%%%%%%%%%%%%%%%%%%%%%%%%%%%%%%%%%%%%%%%%%%%%%%%%%%%%%%%%%%%%%%%%%%%%%%%%%%%%%%%%%%%%%%%%%%%%%%%%%%%%%%%
%% The Appendices part is started with the command \appendix;
%% appendix sections are then done as normal sections
\appendix
\section{Derivation of the gradient dynamics}\label{app:gradient-evolution}
We provide the detailed derivation of the gradient dynamics~\eqref{eqn:gradient-evolution}.
Recall the notations
\[
\bar{U}_n(\theta) = [\bar{u}_n(x_n; \theta), \bar{u}_n^{\prime}(x_n; \theta)]^{\top} \in \mathbb{R}^{2\times 1},
\]
\[
\bar{U}_{\mathcal{X}}(\theta) = [\bar{U}_1(\theta), \ldots, \bar{U}_{|\mathcal{X}|}(\theta)]^{\top} \in \mathbb{R}^{ 2|\mathcal{X}| \times 1},
\]
\[
\bar{W}_n(\theta) =
W(x, \bar{u}({\bf x}; \theta),  \bar{u}^{\prime}({\bf x}; \theta)) \in \mathbb{R},\]
\[
\bar{W}_{\mathcal{X}}(\theta) = [\bar{W}_1(\theta), \ldots, \bar{W}_{|\mathcal{X}|}(\theta)] \in \mathbb{R}^{|\mathcal{X}| \times 1},
\]
\[ \nabla_U \bar{W}_n(\theta) = [\partial_{\hat{u}} \bar{W}_n(\theta), \partial_{\hat{u}^{\prime}} \bar{W}_n(\theta)]^{\top}
\in 
\mathbb{R}^{2 \times 1},\]
\[
\nabla_U \bar{W}_{\mathcal{X}}(\theta) = [\nabla_U \bar{W}_1(\theta), \ldots, \nabla_U \bar{W}_{|\mathcal{X}|}(\theta)]^{\top} \in \mathbb{R}^{2|\mathcal{X}| \times 1}.
\]
A simple application of the chain rule leads to 
\begin{equation}
	\begin{split}
		\frac{d[\nabla_{U} \bar{W}_{\mathcal{X}}(\theta(t))]}{dt}
		=
		\begin{bmatrix}
			\frac{d\partial_u \bar{W}_1(\theta(t))}{dt} \\
			\frac{d\partial_{u^{\prime}} \bar{W}_1(\theta(t))}{dt} \\
			\vdots\\
			\frac{d\partial_u \bar{W}_{|\mathcal{X}|}(\theta(t))}{dt} \\
			\frac{d\partial_{u^{\prime}} \bar{W}_{|\mathcal{X}|}(\theta(t))}{dt} 
		\end{bmatrix}
		=
		\begin{bmatrix}
			\partial_{uu}^2 \bar{W}_1 \frac{d \bar{u}_1(\theta(t))}{dt} + \partial_{uu^{\prime}}^2 \bar{W}_1 \frac{d \bar{u}_1^{\prime}(\theta(t))}{dt}\\
			\partial_{u^{\prime}u}^2 \bar{W}_1 \frac{d \bar{u}_1(\theta(t))}{dt} + \partial_{u^{\prime}u^{\prime}}^2 \bar{W}_1 \frac{d \bar{u}_1^{\prime}(\theta(t))}{dt}\\
			\vdots\\
			\partial_{uu}^2 \bar{W}_{|\mathcal{X}|} \frac{d \bar{u}_{|\mathcal{X}|}(\theta(t))}{dt} + \partial_{uu^{\prime}}^2 \bar{W}_{|\mathcal{X}|} \frac{d \bar{u}_{|\mathcal{X}|}^{\prime}(\theta(t))}{dt}\\
			\partial_{u^{\prime}u}^2 \bar{W}_{|\mathcal{X}|} \frac{d \bar{u}_{|\mathcal{X}|}(\theta(t))}{dt} + \partial_{u^{\prime}u^{\prime}}^2 \bar{W}_{|\mathcal{X}|} \frac{d \bar{u}_{|\mathcal{X}|}^{\prime}(\theta(t))}{dt}
		\end{bmatrix} 
		=
		D_{\mathcal{X}}(\theta(t)) \frac{d\bar{U}_{\mathcal{X}}(\theta(t))}{dt},
	\end{split}
\end{equation}
where
\begin{equation*}
	D_{\mathcal{X}}(\theta(t)) = \text{diag}\left( \begin{bmatrix}
		\partial_{uu}^2 \bar{W}_n & \partial_{uu^{\prime}}^2 \bar{W}_n \\
		\partial_{u^{\prime}u}^2 \bar{W}_n & \partial_{u^{\prime}u^{\prime}}^2 \bar{W}_n
	\end{bmatrix}    \right)_{n=1, \ldots, |\mathcal{X}|} \in \mathbb{R}^{2|\mathcal{X}| \times 2|\mathcal{X}|}.
\end{equation*}
Further note that by following the same argument as for deriving~\eqref{eqn:loss-trajectory-NTK}, we have 
\begin{equation}
	\begin{split}
		\frac{d\bar{U}_{\mathcal{X}}(\theta(t))}{dt}
		=
		\nabla_{\theta} \bar{U}_{\mathcal{X}}(\theta(t)) \frac{d\theta(t)}{dt}
		=
		-\frac{\eta}{|\mathcal{X}|} M_{\mathcal{X}} \nabla_{U} \bar{W}_{\mathcal{X}}(\theta(t)).
	\end{split}
\end{equation}
We obtain the desired dynamics~\eqref{eqn:gradient-evolution} for
$\nabla_{U} \bar{W}_{\mathcal{X}}(\theta(t))$.

\section{The eigenspectrum of the Gram matrix}\label{app:eigenspectrum}
Let $K: D \times D \to \mathbb{R}$ be a symmetric positive definite kernel and 
define the Hilbert Schmidt integral operator 
\[
\mathcal{L}u(x) \triangleq \int_D K(x, x^{\prime}) u(x^{\prime}) \, dx.
\]
Given a dataset $\mathcal{X} = \{x_1, \ldots, x_n\} \subset D$ that is uniformly sampled over $D$, the Gram matrix induced by $K$, i.e., 
\[
M_{\mathcal{X}} = K(\mathcal{X}, \mathcal{X}),
\]
plays a central role in various kernel based regression tasks. 
Assuming $D$ is compact and $K$ is a Mercer kernel, the integral operator $\mathcal{L}$ admits a discrete spectrum and hence the following eigenvalue problem is well defined~\cite{williams2006gaussian}, 
\begin{equation}\label{eqn:eigen-L}
	\mathcal{L}u_k = \Lambda_k u_k, \qquad k = 1, 2, \ldots,
\end{equation}
where the eigenvalues $\Lambda_1 \geq \Lambda_2 \geq \ldots > 0$ and the eigenfunctions are orthonormal, i.e.,
\[
\int_D u_i(x) u_j(x) \, dx = \delta_{ij}.
\]
Evaluating~\eqref{eqn:eigen-L} at $\mathcal{X}$ leads to 
\begin{equation}\label{eqn:eigen-LX}
	\mathcal{L}{\bf u}_k = \Lambda_k {\bf u}_k, \qquad k = 1, 2, \ldots,
\end{equation}
where ${\bf u}_k = u_k(\mathcal{X}) \in \mathbb{R}^{n \times 1}$ and $\mathcal{L}{\bf u}_k = [\mathcal{L}u_k(x_1), \ldots, \mathcal{L}u_k(x_n)]^{\top}$.
Note that the integral operator $\mathcal{L}$ can be approximated by 
\[
\mathcal{L}u(x) \approx \mathcal{L}_nu(x) \triangleq \frac{1}{n} \sum_{i = 1}^n K(x, x_i)u(x_i)
\]
and hence we can approximately (for $n$ large) consider the 
eigenvalue problem 
\begin{equation}\label{eqn:eigen-Ln}
	\mathcal{L}_n u_k = \hat{\Lambda}_k u_k, \qquad k = 1, 2, \ldots, n,
\end{equation}
where $\hat{\Lambda}_k \approx \Lambda_k$ depends on the sample size $n$.
Evaluating the above equation at $\mathcal{X}$ leads to 
\begin{equation}\label{eqn:eigen-M}
	M_{\mathcal{X}} {\bf u}_k = n \hat{\Lambda}_k {\bf u}_k,\qquad k = 1, \ldots, n,
\end{equation}
where $\lambda_k \triangleq n \hat{\Lambda}_k$ is the $k$-th eigenvalue for the Gram matrix $M_{\mathcal{X}}$. 
Comparing~\eqref{eqn:eigen-LX} with~\eqref{eqn:eigen-M} leads to the connection between the eigenvalue of $\mathcal{L}$ and the eigenvalue of the Gram matrix $M_{\mathcal{X}}$,
\[
\Lambda_k = \lim_{n \to \infty} \frac{\lambda_k}{n}.
\]
Therefore, for large values of $n$, we have the approximation $\lambda_k \approx n \Lambda_k$ for $k = 1, \ldots, n$.


%% \label{}

%% If you have bibdatabase file and want bibtex to generate the
%% bibitems, please use
%\bibliographystyle{spmpsci}
\bibliographystyle{elsarticle-num}
%\bibliographystyle{unsrt}  
\bibliography{bibliography}
%%%
%% This is file `./samples/longsample.tex',
%% generated with the docstrip utility.
%%
%% The original source files were:
%%
%% apa7.dtx(with options: `longsample')
%% ----------------------------------------------------------------------
%% 
%% apa7 - A LaTeX class for formatting documents in compliance with the
%% American Psychological Association's Publication Manual, 7th edition
%% 
%% Copyright (C) 2019 by Daniel A. Weiss <daniel.weiss.led at gmail.com>
%% 
%% This work may be distributed and/or modified under the
%% conditions of the LaTeX Project Public License (LPPL), either
%% version 1.3c of this license or (at your option) any later
%% version.The latest version of this license is in the file:
%% 
%% http://www.latex-project.org/lppl.txt
%% 
%% Users may freely modify these files without permission, as long as the
%% copyright line and this statement are maintained intact.
%% 
%% This work is not endorsed by, affiliated with, or probably even known
%% by, the American Psychological Association.
%% 
%% ----------------------------------------------------------------------
%% 
\documentclass[
a4paper,
man,
british
]{apa7}

\usepackage{changes}
\usepackage{cancel}
\usepackage[british]{babel}
\usepackage[utf8]{inputenc}
\usepackage{epstopdf}
\usepackage{csquotes}
\usepackage[flushleft]{threeparttable}
\usepackage{multirow}
%\usepackage[hidelinks]{hyperref}
%\usepackage{authblk}
\usepackage[
style = apa,
backend = biber,
sortcites = true,
sorting = nyt,
%isbn = false,
%url = false,
%doi = false,
%eprint = false,
hyperref = true,
backref = false,
%firstinits = false,
]{biblatex}
\hypersetup{
colorlinks = true,
linkcolor = black,
anchorcolor = black,
citecolor = black,
filecolor = black,
urlcolor = blue
}
\usepackage{float}
\usepackage{placeins}
\usepackage{xcolor}
\usepackage[toc, page]{appendix}
\usepackage{lscape}
\usepackage{afterpage}
\usepackage{esvect}
\usepackage{amsmath}
\usepackage{ragged2e}
\justifying\let\raggedright\justifying
\usepackage{enumitem}
\usepackage{makecell}
\DeclareLanguageMapping{british}{british-apa}
\usepackage[nodisplayskipstretch]{setspace}
\usepackage{subcaption} 
\usepackage{rotating}
\usepackage{geometry}
\geometry{a4paper, margin=1in}

% maps apacite commands to biblatex commands

\let \citeNP \cite
\let \citeA \textcite
\let \cite \parencite

\newcommand{\figurehere}[1]{\begin{center}%
\vspace{-2mm}
=========================\\%
Insert Figure #1 about here\\%
=========================\\%
\vspace{-2mm}
\end{center}}
%%
%%Table goes about here command
\newcommand{\tablehere}[1]{\begin{center}%
\vspace{-2mm}
=========================\\%
Insert Table #1 about here\\%
=========================\\%
\vspace{-2mm}
\end{center}}

\usepackage{array}
\newcommand{\PreserveBackslash}[1]{\let\temp=\\#1\let\\=\temp}
\newcolumntype{C}[1]{>{\PreserveBackslash\centering}p{#1}}
\newcolumntype{R}[1]{>{\PreserveBackslash\raggedleft}p{#1}}
\newcolumntype{L}[1]{>{\PreserveBackslash\raggedright}p{#1}}

\addbibresource{tutorial_MH.bib}

\title{Tutorial on Using Machine Learning and Deep Learning Models for Mental Illness Detection}
\shorttitle{Data-driven Methods to Identify Mental Illness}

\authorsnames[1,2,2,3,4,5,1,5]{%
  Yeyubei Zhang, Zhongyan Wang, Zhanyi Ding, Yexin Tian, 
  Jianglai Dai, Xiaorui Shen, Yunchong Liu, Yuchen Cao
}
\authorsaffiliations{%
  {University of Pennsylvania, School of Engineering and Applied Science}, 
  {New York University, Center for Data Science}, 
  {Georgia Institute of Technology, College of Computing}, 
  {University of California, Berkeley, Department of EECS}, 
  {Northeastern University, Khoury College of Computer Science}
}

%\leftheader{Placeholder}

\abstract{Social media has become an important source for understanding mental health, providing researchers a way to detect conditions like depression from user-generated posts. This tutorial provides practical guidance to address common challenges in applying machine learning and deep learning methods for mental health detection on these platforms. It focuses on strategies for working with diverse datasets, improving text preprocessing, and addressing issues such as imbalanced data and model evaluation. Real-world examples and step-by-step instructions demonstrate how to apply these techniques effectively, with an emphasis on transparency, reproducibility, and ethical considerations. By sharing these approaches, this tutorial aims to help researchers build more reliable and widely applicable models for mental health research, contributing to better tools for early detection and intervention.}

\keywords{Mental Health Research, Machine Learning, Deep Learning, Social Media Analysis, Natural Language Processing}

\authornote{
% \addORCIDlink{Placeholder}{0000-0000-0000-0000}
Correspondence concerning this article should be addressed to Yuchen Cao, Northeastern University, E-mail: cao.yuch@northeastern.edu}

\begin{document}
\maketitle

\section{Introduction}

Mental health disorders, especially depression, have become a significant concern worldwide, affecting millions of individuals across diverse populations \cite{WHO2020}. Early detection of depression is crucial, as it can lead to timely treatment and better long-term outcomes. In today’s digital age, social media platforms such as X(Twitter), Facebook, and Reddit provide a unique opportunity to study mental health. People often share their thoughts and emotions on these platforms, making them a valuable source for understanding mental health patterns \cite{Choudhury2013, Guntuku2017}.


Recent advances in computational methods, particularly machine learning (ML) and deep learning (DL), have shown promise in analyzing social media data to detect signs of depression. These techniques can uncover patterns in language use, emotions, and behaviors that may indicate mental health challenges \cite{Shatte2020, Yazdavar2020}. 

However, applying these methods effectively is not without challenges. A recent systematic review \cite{cao2024mental} highlighted issues such as a lack of diverse datasets, inconsistent data preparation, and inadequate evaluation metrics for imbalanced data \cite{Hargittai2015, Helmy2024}—problems that have also led to inaccuracies in other domains (e.g., \cite{gao2024survey, weng2024ai}). Similarly, Liu et al. \cite{liu2024systematic} identified additional linguistic challenges in ML approaches for detecting deceptive activities on social networks, including biases from insufficient linguistic preprocessing and inconsistent hyperparameter tuning, all of which are pertinent to mental health detection. Moreover, complementary insights from related fields underscore the need for continuous improvements in robust model development \cite{bi2024decoding,zhao2024minimax,tao2023meta,xu2025robust}.


This tutorial is designed to address these gaps by guiding readers through the steps necessary to create reliable and accurate models for depression detection using social media data. It focuses on practical techniques to:
\begin{itemize}
\item Collect and preprocess data, including handling language challenges like sarcasm or negations.
\item Build and optimize models with attention to tuning and evaluation.
\item Use appropriate metrics for datasets where depressive posts are a minority.
\end{itemize}

Our goal is to provide a clear, step-by-step approach that researchers and practitioners can use to improve their methods. By addressing common challenges in this field, we hope to encourage more robust and ethical use of technology for improving mental health outcomes.

\section{Method}
This section provides a comprehensive overview of the methodological framework employed in this study, detailing the processes for data preparation, model development, and evaluation metrics. All analyses and model implementations were conducted using Python 3, leveraging popular libraries such as \texttt{pandas} for data manipulation, \texttt{scikit-learn} for machine learning, \texttt{PyTorch} for deep learning, and \texttt{Transformers} for pre-trained language models. These tools enabled efficient preprocessing, hyperparameter optimization, and performance evaluation. The following subsections elaborate on the key steps and methodologies involved in the study.

\subsection{Data Preparation}
\subsubsection{Data Sources and Collection Methods}
A sufficiently representative dataset is essential for machine-learning-based mental health detection. This study utilized the Sentiment Analysis for Mental Health dataset, available on \href{https://www.kaggle.com/datasets/suchintikasarkar/sentiment-analysis-for-mental-health/data}{Kaggle}. The dataset integrates textual content from multiple repositories focused on mental health topics, including depression, anxiety, stress, bipolar disorder, personality disorders, and suicidal ideation. The primary sources of these data are social media platforms such as Reddit, Twitter, and Facebook, where individuals frequently discuss personal experiences, emotional states, and mental health concerns.

The dataset was originally compiled using platform-specific APIs (e.g., Reddit, Twitter, and Facebook) and web scraping tools, allowing for the collection of substantial volumes of publicly available text data. After the acquisition, duplicates were removed, irrelevant and spam content was filtered, and mental health labels were standardized to ensure consistency across repositories. Personal identifiers were removed to safeguard privacy and ensure compliance with ethical guidelines for data usage. The final dataset was consolidated into a structured CSV file with unique identifiers for each entry.

Although the dataset combines data from multiple platforms to provide a diverse corpus, it is not free from limitations. Differences in platform demographics, such as age, cultural background, and communication styles, may affect the generalizability of models trained on this data. Additionally, linguistic variability, including colloquialisms, slang, and code-switching, reflects the informal nature of social media communication. While this diversity enriches the dataset, it also presents challenges for natural language processing (NLP) techniques, particularly in tokenization and embedding generation. To address these complexities, the preprocessing pipeline was designed to handle diverse linguistic patterns and balance class distributions where needed.

\subsubsection{Data Preprocessing}
A standardized preprocessing pipeline was applied to prepare the dataset for training both machine learning (ML) and deep learning (DL) models. These steps ensured consistency in data handling while accommodating the unique requirements of each modeling approach:
\vspace{-3mm}
\begin{itemize}
\item \textbf{Text Cleaning:} Social media text often contains noise such as URLs, HTML tags, mentions, hashtags, special characters, and extra whitespace. These elements were systematically removed using regular expressions to create cleaner input for both ML and DL models.
\vspace{-1.5mm}
\item \textbf{Lowercasing:} All text was converted to lowercase to maintain uniformity across the dataset and minimize redundancy in text representation.
\vspace{-1.5mm}
\item \textbf{Stopword Removal:} Commonly used words that provide little semantic value (e.g., “the,” “and,” “is”) were excluded using the stopword list available in the Natural Language Toolkit (NLTK) \cite{nltk_toolkit}, reducing noise while retaining meaningful content.
\vspace{-1.5mm}
\item \textbf{Lemmatization:} Words were reduced to their base forms (e.g., “running,” “ran,” “runs” → “run”) using NLTK's Lemmatizer. This step normalized variations of words, aiding both feature extraction and embedding generation.
\end{itemize}
\vspace{-3mm}

The dataset was divided into training, validation, and testing subsets using a two-step random sampling process with a fixed random seed to ensure reproducibility. First, 20\% of the data was set aside as the test set. The remaining 80\% was then further divided into a training set (60\% of the original data) and a validation set (20\% of the original data). This split ensured that the models were trained on the majority of the data while reserving separate subsets for hyperparameter tuning and final performance evaluation.

\subsubsection{Class Labeling}
The dataset’s class labels were prepared as follows: (1) For \textbf{multi-class classification}, the labels included six categories: Normal, Depression, Suicidal, Anxiety, Stress, and Personality Disorder. (2) For \textbf{binary classification}, the labels were grouped into two classes: Normal and Abnormal.

\subsubsection{Feature Transformation for ML Models}
For ML models, an additional step, TF-IDF Vectorization, was necessary to transform the text into structured features. The cleaned text was converted into numerical representations using Term Frequency–Inverse Document Frequency (TF-IDF), which captured term frequencies while down-weighting overly frequent words. The vectorizer was configured to extract up to 1,000 features and account for both unigrams and bigrams (n-gram range: 1–2).

\subsubsection{Code Availability}
The code for data preprocessing, including text cleaning, class labeling, and dataset splitting, is publicly available on GitHub (the link will be provided upon acceptance).

\subsection{Model Development}
This study employed a range of machine learning (ML) and deep learning (DL) models to analyze and classify mental health statuses based on textual data. Each model was selected to explore specific aspects of the data, from linear interpretability to handling complex patterns and long-range dependencies. Detailed implementation code for all models, including hyperparameter tuning and evaluation, is available on GitHub. Below, we provide an overview of each model, its methodology, and its performance in the context of binary and multi-class mental health classification tasks.

\subsubsection{Logistic Regression}
Logistic regression is one of the most widely used methods for classification tasks and has long been employed in social science and biomedical research \cite{hosmer2000applied, ding2025efficientpowerfultradeoffsmachine}. In the context of mental health detection, it provides a straightforward yet interpretable modeling framework, translating linear combinations of predictors (e.g., term frequencies, sentiment scores, and linguistic features) into estimated probabilities of class membership through the logit function.

The logistic regression model predicts the probability of a binary outcome using the following expression:
\begin{equation}\nonumber
\hat{y} = \frac{1}{1 + \exp(-\mathbf{w}^\top \mathbf{x} - b)},
\end{equation}
where $\hat{y}$ represents the predicted probability, $\mathbf{w}$ is the vector of model coefficients, $\mathbf{x}$ is the feature vector, and $b$ is the bias term. For multi-class classification, the model generalizes to predict probabilities for $K$ classes using the softmax function:
\begin{equation}\nonumber
P(y = k \mid \mathbf{x}) = \frac{\exp(\mathbf{w}_k^\top \mathbf{x} + b_k)}{\sum_{j=1}^{K} \exp(\mathbf{w}_j^\top \mathbf{x} + b_j)},
\end{equation}
where $k \in \{1, \dots, K\}$ represents the class index.

Both binary and multi-class logistic regression models were optimized using cross-entropy loss during training and configured to converge with a maximum iteration limit of 1,000. Regularization was applied to prevent overfitting, using $\ell_2$ (ridge) regularization, which penalizes large coefficients by adding their squared magnitude to the loss function:
\begin{equation}\nonumber
\mathcal{L} = - \frac{1}{n} \sum_{i=1}^{n} \left[ y_i \log(\hat{y}_i) + (1 - y_i) \log(1 - \hat{y}_i) \right] + \lambda \|\boldsymbol{\beta}\|_2^2,
\end{equation}
where $\lambda$ controls the regularization strength, $y_i$ is the true label, $\hat{y}_i$ is the predicted probability, and $\boldsymbol{\beta}$ represents the model coefficients.

Hyperparameter tuning was conducted using a grid search across several parameters. The regularization strength (\(C\)), which is the inverse of the regularization parameter \(\lambda\), was tested with values such as 0.1, 1, and 10. Various optimizers, including \texttt{liblinear} (Library for Large Linear Classification), \texttt{lbfgs} (Limited-memory Broyden–Fletcher–Goldfarb–Shanno), and \texttt{saga} (Stochastic Average Gradient Augmented), were evaluated for optimization. To address class imbalance, the \texttt{class\_weight} parameter was explored with options for \texttt{balanced} and \texttt{None}. For multi-class tasks, the \texttt{multinomial} strategy was employed, while the \texttt{one-vs-rest} strategy was implicitly applied for binary classification scenarios.

For both binary and multi-class tasks, the weighted F1 score was used as the primary evaluation metric, ensuring balanced performance across categories, including minority classes. A combined grid search configuration was applied for both tasks, as their hyperparameter spaces largely overlapped. The best configurations effectively handled class imbalance using the \texttt{class\_weight=`balanced'} parameter, yielding robust performance on the validation and test sets.

The logistic regressions were implemented using the \texttt{LogisticRegression} class from the \texttt{scikit-learn} library. Detailed implementation code for logistic regression, including hyperparameter tuning and evaluation, is available on GitHub.

\subsubsection{Support Vector Machine (SVM)}
Support Vector Machines (SVMs) are supervised learning models that are widely used for both classification and regression tasks. Originally introduced by \citeA{cortes1995support}, SVMs aim to find the optimal hyperplane that maximizes the margin between data points of different classes. The margin is defined as the distance between the closest data points (support vectors) from each class to the hyperplane. By maximizing this margin, SVMs achieve better generalization for unseen data.

For a linearly separable dataset, the decision boundary is defined as:
\begin{equation}\nonumber
f(\mathbf{x}) = \mathbf{w}^T \mathbf{x} + b,
\end{equation}
where $\mathbf{w}$ is the weight vector, $\mathbf{x}$ is the feature vector, and $b$ is the bias term. The optimal hyperplane is determined by solving the following optimization problem:
\begin{align}
\min_{\mathbf{w}, b} & \quad \frac{1}{2} \|\mathbf{w}\|^2, \nonumber\\
\text{subject to} & \quad y_i (\mathbf{w}^T \mathbf{x}_i + b) \geq 1, \quad i = 1, \dots, N,\nonumber
\end{align}
where $y_i \in \{-1, +1\}$ are the class labels.

For datasets that are not linearly separable, the optimization problem is modified to include a penalty for misclassifications:
\begin{align}
\min_{\mathbf{w}, b, \xi} & \quad \frac{1}{2} \|\mathbf{w}\|^2 + C \sum_{i=1}^N \xi_i, \nonumber\\
\text{subject to} & \quad y_i (\mathbf{w}^T \mathbf{x}_i + b) \geq 1 - \xi_i, \quad \xi_i \geq 0, \quad i = 1, \dots, N,\nonumber
\end{align}
where $\xi_i$ are slack variables that allow for misclassifications, and $C > 0$ is the regularization parameter that controls the trade-off between maximizing the margin and minimizing classification errors.

Kernel methods enable SVMs to handle nonlinearly separable data by mapping the input features into a higher-dimensional space where linear separation becomes possible. This mapping is performed implicitly using a kernel function \( K(\mathbf{x}_i, \mathbf{x}_j) \), which computes the inner product in the transformed space:
\begin{equation}\nonumber
K(\mathbf{x}_i, \mathbf{x}_j) = \phi(\mathbf{x}_i)^T \phi(\mathbf{x}_j),
\end{equation}
where \( \phi(\cdot) \) represents the mapping function.

Several commonly used kernel functions are available, each suited for different data characteristics:

\vspace{-3mm}
\begin{enumerate}
\item \textbf{Linear Kernel:}
\begin{equation}\nonumber
K(\mathbf{x}_i, \mathbf{x}_j) = \mathbf{x}_i^T \mathbf{x}_j
\end{equation}
This kernel computes the dot product of the input vectors and is suitable for linearly separable data.
\vspace{-1.5mm}
\item \textbf{Polynomial Kernel:}
\begin{equation}\nonumber
K(\mathbf{x}_i, \mathbf{x}_j) = (\mathbf{x}_i^T \mathbf{x}_j + c)^d,
\end{equation}
where \( c \) is a constant and \( d \) is the degree of the polynomial. This kernel is useful for capturing polynomial feature interactions.
\vspace{-1.5mm}
\item \textbf{Radial Basis Function (RBF) Kernel:}
\begin{equation}\nonumber
K(\mathbf{x}_i, \mathbf{x}_j) = \exp\left(-\gamma \|\mathbf{x}_i - \mathbf{x}_j\|^2\right),
\end{equation}
where \( \gamma \) controls the influence of individual training samples. The RBF kernel is widely used for its flexibility in modeling complex, nonlinear patterns.
\vspace{-1.5mm}
\item \textbf{Sigmoid Kernel:}
\begin{equation}\nonumber
K(\mathbf{x}_i, \mathbf{x}_j) = \tanh(\alpha \mathbf{x}_i^T \mathbf{x}_j + c),
\end{equation}
where \( \alpha \) is a scaling factor and \( c \) is a bias term. This kernel is inspired by neural network activation functions and is suitable for data with specific characteristics.
\vspace{-1.5mm}
\item \textbf{Custom Kernels:}
Custom-defined kernels can be tailored for domain-specific tasks, offering flexibility for unique datasets or similarity metrics.
\end{enumerate}
\vspace{-3mm}

In this project, kernel selection was based on preliminary experiments, with the linear and radial basis function (RBF) kernels being the primary choice due to its ability to model complex, nonlinear relationships effectively.

For both binary and multi-class classification tasks, the same hyperparameter tuning strategy was employed. A grid search was conducted over the following hyperparameters:
\vspace{-3mm}
\begin{itemize}
\item Regularization parameter $C$: values of \{0.1, 1, 10\}.
\vspace{-1.5mm}
\item Kernel type: linear and RBF.
\vspace{-1.5mm}
\item Class weight: balanced or none.
\vspace{-1.5mm}
\item Gamma (for RBF kernel): scale and auto.
\end{itemize}
\vspace{-3mm}
The grid search aimed to identify the optimal combination of hyperparameters using the weighted F1 score as the primary evaluation metric. For multi-class classification, the one-vs-one strategy inherent to the \texttt{SVC} implementation was used.

The loss function for SVM is analogous to logistic regression, as both models minimize the cross-entropy loss during optimization. However, for SVM, the hinge loss is typically used for linear separable cases, defined as:
\begin{equation}\nonumber
\mathcal{L}_{\text{hinge}} = \frac{1}{N} \sum_{i=1}^N \max(0, 1 - y_i f(\mathbf{x}_i)).
\end{equation}

The SVM models were implemented with the \texttt{SVC} class from \texttt{scikit-learn}. Detailed implementation code for SVMs, including grid search and evaluation, is available on GitHub.

\subsubsection{Tree-Based Models}
Classification and Regression Trees (CARTs) are versatile tools used for analyzing categorical outcomes (classification tasks). The CART algorithm constructs a binary decision tree by recursively partitioning the data based on covariates, optimizing a predefined splitting criterion. For classification tasks, the quality of a split is typically evaluated using impurity measures such as Gini impurity or entropy \cite{bishop2006pattern}. The Gini impurity for a node is defined as:
\begin{equation}\nonumber
G = \sum_{i=1}^C p_i (1 - p_i),
\end{equation}
where \(p_i\) is the proportion of observations in class \(i\) at the given node, and \(C\) is the total number of classes.

Alternatively, entropy can be used to measure impurity:
\begin{equation}\nonumber
H = -\sum_{i=1}^C p_i \log(p_i),
\end{equation}
where \(p_i\) represents the same class proportions as in the Gini impurity formula. Lower impurity values indicate greater homogeneity within a node.

At each step, the algorithm selects the split that minimizes the weighted impurity of the child nodes. The impurity reduction for a given split is computed as:
\begin{equation}\nonumber
\Delta I = I_{\text{parent}} - \left( \frac{n_L}{n} I_L + \frac{n_R}{n} I_R \right),
\end{equation}
where \(I_{\text{parent}}\) is the impurity of the parent node, \(I_L\) and \(I_R\) are the impurities of the left and right child nodes, \(n_L\) and \(n_R\) are the number of observations in the left and right child nodes, and \(n\) is the total number of observations in the parent node.

The splitting process continues until one stopping criterion is met. Common criteria include: (1) a minimum number of samples in a node, (2) a maximum tree depth, and (3) No further reduction in impurity beyond a predefined threshold.

To address overfitting, pruning techniques \cite{breiman1984classification} are employed. Pruning reduces the tree size by removing splits that contribute minimally to predictive performance, improving the model's generalizability.

Due to their tendency to overfit, simple CART models were not evaluated in this project. Instead, ensemble methods like Random Forests and Gradient Boosted Trees, which combine multiple CART models, were used for improved robustness.

\paragraph{Random Forests}
Random Forests are ensemble learning methods that aggregate multiple decision trees parallelly to enhance classification performance. By building trees on bootstrap samples of the data and introducing random feature selection at each split, Random Forests reduce overfitting and improve generalization. Each tree is trained on a random bootstrap sample, where data points are sampled with replacement from the original dataset, meaning some observations may appear multiple times in the training sample, while others are excluded. Additionally, Random Forests introduce randomness during the tree-building process by selecting a random subset of covariates at each split instead of considering all available covariates. This randomization decorrelates the trees, reduces variance, and enhances the model's robustness. For classification tasks, the final prediction is determined by majority voting across all trees \cite{breiman2001random}.

To further mitigate overfitting, each tree in the Random Forest is grown to its full depth without pruning, fitting the bootstrap sample as accurately as possible. Hyperparameters such as the number of trees (\texttt{n\_estimators}), the maximum depth of each tree (\texttt{max\_depth}), and the minimum samples required to split a node (\texttt{min\_samples\_split}) or form a leaf (\texttt{min\_samples\_leaf}) play a critical role in balancing bias and variance. The parameter \texttt{class\_weight}, when set to \texttt{`balanced'}, adjusts weights inversely proportional to class frequencies, effectively addressing the class imbalance.

A grid search approach was employed to optimize key hyperparameters for both binary and multi-class classification tasks. The parameter grid explored values such as 50, 100, and 200 for the number of trees (\texttt{n\_estimators}), depths of 10, 20, or unrestricted (\texttt{None}) for \texttt{max\_depth}, and split criteria (\texttt{min\_samples\_split} and \texttt{min\_samples\_leaf}) to control tree complexity. The weighted F1 score served as the primary evaluation metric to account for imbalances in the dataset. For the binary classification task, the best-performing model, determined through validation, effectively handled class imbalance and demonstrated robust predictive performance for distinguishing between Normal and Abnormal mental health statuses. In addition to traditional hyperparameter tuning techniques, recent studies have explored novel metaheuristic approaches to optimize Random Forest parameters. For instance, Tan et al. \cite{tan2024dung} proposed an improved dung beetle optimizer that refines hyperparameter tuning, further enhancing model performance.

For the multi-class classification task, the same hyperparameter grid was used with a slightly reduced scope to streamline the search process. The weighted F1 score guided model selection across all classes, including Normal, Depression, Anxiety, and Personality Disorder. The optimal model achieved balanced performance across multiple categories, leveraging Random Forests' ability to aggregate predictions from diverse decision trees. 

Random Forests’ inherent feature importance metrics provided additional insights into the most influential predictors for mental health classification. This capability enhances interpretability by highlighting covariates that most strongly influence predictions. The Random Forest models were built using the \texttt{RandomForestClassifier} from \texttt{scikit-learn}. Parameter grids for the number of estimators, maximum depth, and other parameters were evaluated with \texttt{GridSearchCV}. Detailed implementation code, including grid search and evaluation procedures, is available on GitHub.

\paragraph{Light Gradient Boosting Machine (LightGBM)}
Light Gradient Boosting Machine (LightGBM) is a gradient-boosting framework optimized for efficiency and scalability, particularly in handling large datasets and high-dimensional data. Gradient Boosting Machines (GBM) work by sequentially building decision trees, where each new tree corrects the errors made by the previous ones, leading to highly accurate predictions. However, traditional GBM frameworks can be computationally intensive, especially for large datasets \cite{friedman2001greedy}. Unlike traditional Gradient Boosting Machines (GBMs), LightGBM employs a leaf-wise tree growth strategy, which enables deeper splits in dense data regions, enhancing performance by focusing complexity where it is most needed. Additional optimizations, such as histogram-based feature binning, reduce memory usage and accelerate training. These enhancements make LightGBM faster and more resource-efficient than standard GBM implementations, without compromising predictive accuracy \cite{ke2017lightgbm}.

Key hyperparameters tuned for LightGBM included the number of boosting iterations (\texttt{n\_estimators}), learning rate, maximum tree depth (\texttt{max\_depth}), number of leaves (\texttt{num\_leaves}), and minimum child samples (\texttt{min\_child\_samples}). To address the class imbalance, the \texttt{class\_weight} parameter was tested with both \texttt{`balanced'} and \texttt{None} options. Grid search was employed to explore all possible combinations of these hyperparameters, and the weighted F1 score was used as the primary metric for selecting the best configuration.

LightGBM was applied to both binary and multi-class mental health classification tasks. For binary classification, the model differentiated between Normal and Abnormal statuses. For multi-class classification, it predicted categories such as Normal, Depression, Anxiety, and Personality Disorder using the \texttt{multiclass} objective. Hyperparameter tuning via grid search ensured balanced performance across all categories, guided by the weighted F1 score. 

The best-performing models demonstrated robust predictive power, evaluated using precision, recall, F1 scores, confusion matrices, and one-vs-rest ROC curves. Additionally, LightGBM’s feature importance metrics provided interpretability by highlighting the most influential linguistic and sentiment-based features. Its combination of high performance, scalability, and interpretability made LightGBM a key component in this project. The LightGBM models were developed using the \texttt{LGBMClassifier} from the \texttt{lightgbm} library. Hyperparameter tuning, including the number of boosting iterations, learning rate, and tree depth, was performed using \texttt{GridSearchCV}.
Detailed implementation code, including grid search procedures, is available on GitHub.

\subsubsection{A Lite version of Bidirectional Encoder Representations from Transformers (ALBERT)}
A Lite version of Bidirectional Encoder Representations from Transformers (BERT), known as ALBERT \cite{lan2020albert}, is a transformer-based model designed to improve efficiency while maintaining performance. While BERT \cite{devlin2019bert} is highly effective for a wide range of natural language processing (NLP) tasks, it is computationally expensive and memory-intensive due to its large number of parameters. ALBERT addresses these limitations by introducing parameter-sharing across layers and factorized embedding parameterization, which significantly reduces the number of parameters without sacrificing model capacity. Additionally, ALBERT employs Sentence Order Prediction (SOP) as an auxiliary task to enhance pretraining, improving its ability to capture sentence-level coherence. These optimizations make ALBERT a lightweight yet powerful alternative to BERT, capable of achieving competitive performance with reduced memory and computational requirements, making it particularly suitable for large-scale text classification tasks like mental health detection.

In this project, ALBERT was employed for both binary and multi-class classification tasks. For binary classification, the model was fine-tuned to differentiate between Normal and Abnormal mental health statuses, while for multi-class classification, it was configured to predict multiple categories, including Normal, Depression, Anxiety, and Personality Disorder. The implementation leveraged the pre-trained \texttt{Albert-base-v2} model, with random hyperparameter tuning conducted over 10 iterations to optimize the learning rate, number of epochs, and dropout rates. The weighted F1 score served as the primary evaluation metric throughout the tuning process.

For both binary and multi-class classification tasks, hyperparameter tuning was conducted to optimize learning rates between $10^{-5}$ and $10^{-4}$, dropout rates between 0.1 and 0.5, and epochs ranging from 3 to 5. For binary classification, the model achieved high validation F1 scores and demonstrated strong generalization on the test set. For multi-class classification, the objective was adjusted to predict seven categories, with weighted cross-entropy loss applied to address class imbalances and ensure adequate representation of minority categories. The final models were evaluated on the test set using metrics such as accuracy, weighted F1 scores, and confusion matrices.

ALBERT’s architecture efficiently captures long-range dependencies in text while retaining the computational advantages of its lightweight design. The use of random hyperparameter tuning further refined its performance, enabling robust classification for both binary and multi-class tasks. The ALBERT models were fine-tuned with the \texttt{Transformers} (\texttt{AlbertTokenizer} and \texttt{AlbertForSequenceClassification}) library from Hugging Face. Hyperparameter tuning was conducted manually through random search over learning rates, dropout rates, and epochs. etailed implementation code, including data preparation, training, and hyperparameter tuning, is available on GitHub.

\subsubsection{Gated Recurrent Units (GRUs)}
Gated Recurrent Units (GRUs) are a type of recurrent neural network (RNN) designed to capture sequential dependencies in data, making them particularly effective for natural language processing (NLP) tasks such as text classification \cite{cho2014learning}. Compared to Long Short-Term Memory networks (LSTMs), GRUs are computationally more efficient due to their simplified architecture, which combines the forget and input gates into a single update gate. This efficiency allows GRUs to model long-range dependencies while reducing the number of trainable parameters.

In this study, GRUs were employed for both binary and multi-class mental health classification tasks. For binary classification, the model was configured to differentiate between Normal and Abnormal mental health statuses. For multi-class classification, it was adapted to predict categories such as Normal, Depression, Anxiety, and Personality Disorder. 

The GRU architecture comprised three key components:
\vspace{-3mm}
\begin{enumerate}
\item \textbf{Embedding Layer}: Converts token indices into dense vector representations of a fixed embedding dimension.
\vspace{-1.5mm}
\item \textbf{GRU Layer}: Processes input sequences and retains contextual information across time steps, utilizing only the final hidden state for classification.
\vspace{-1.5mm}
\item \textbf{Fully Connected Layer}: Maps the hidden state to output logits corresponding to the number of classes.
\end{enumerate}
\vspace{-3mm}
Dropout regularization was applied to prevent overfitting, and a weighted cross-entropy loss function was used to address class imbalances in the dataset.

For both binary and multi-class classification tasks, hyperparameter tuning was conducted using random search across predefined ranges. The parameters optimized included embedding dimensions (150--250), hidden dimensions (256--768), learning rates ($10^{-4}$--$10^{-3}$), and epochs (5--10). The weighted F1 score served as the primary evaluation metric during validation. The best-performing models achieved high F1 scores on validation datasets and demonstrated robust generalization on the test sets.

GRUs excelled at capturing sequential patterns in text, enabling the model to identify linguistic cues associated with mental health conditions. Despite being less interpretable than tree-based models, their lightweight architecture ensured computational efficiency and strong performance in text-based classification tasks. The GRU models were implemented with the \texttt{torch.nn} module in PyTorch. Key layers included \texttt{nn.Embedding}, \texttt{nn.GRU}, and \texttt{nn.Linear}. Optimization was performed using the \texttt{torch.optim.Adam} optimizer, and class weights were applied using \texttt{nn.CrossEntropyLoss}.
Detailed implementation code, including data preprocessing, model training, and evaluation, is available on GitHub.

\subsection{Evaluation Metrics}
When modeling mental health statuses—particularly for conditions like depression or suicidal ideation—class distributions are often skewed. In many real-world scenarios, the “positive” class (e.g., individuals experiencing depression) is underrepresented compared to the “negative” class (e.g., no mental health issue). This imbalance renders certain evaluation metrics, such as accuracy, less informative: a model that predicts “no issue” for every instance might still achieve high accuracy if the majority class dominates. Consequently, more nuanced metrics are preferred to evaluate the performance of classification models:

\subsubsection{Precision} 
Precision measures the proportion of positive predictions that are truly positive:
\begin{equation}
 \text{Precision} = \frac{\text{True Positives}}{\text{True Positives} + \text{False Positives}}.
\end{equation}
For instance, in depression detection, high precision indicates that most users flagged as “depressed” indeed exhibit depressive content. While precision minimizes false alarms, focusing on it exclusively can be risky. A model that generates very few positive predictions may achieve artificially high precision while missing many genuinely positive cases.

\subsubsection{Recall (Sensitivity)} 
Recall captures the proportion of actual positives correctly identified:
\begin{equation}\nonumber
 \text{Recall} = \frac{\text{True Positives}}{\text{True Positives} + \text{False Negatives}}.
\end{equation}
In depression detection, recall is critical because failing to recognize at-risk individuals (false negatives) can have serious consequences. A model with low recall risks overlooking individuals who need intervention.

\subsubsection{F1 Score} 
The F1 score serves as the harmonic mean of precision and recall, providing a balance between these two metrics \cite{powers2011evaluation}:
\begin{equation}\nonumber
 F1 = 2 \cdot \frac{\text{Precision} \cdot \text{Recall}}{\text{Precision} + \text{Recall}}.
\end{equation}
The F1 score is particularly useful in imbalanced classification scenarios because it penalizes extreme trade-offs, such as very high precision coupled with very low recall. In mental health detection, achieving a high F1 score ensures the model can effectively identify positive cases while maintaining a reasonable level of precision in its predictions.

\subsubsection{Area Under the Receiver Operating Characteristic Curve (AUROC)} 
AUROC provides an aggregate measure of performance across all possible classification thresholds. It evaluates the model's ability to discriminate between positive and negative classes. However, in the presence of severe class imbalance, AUROC may not fully reflect the challenges posed by a majority class dominating the dataset. Nevertheless, it remains valuable for assessing model performance across varying decision thresholds \cite{davis2006relationship}.

\section{Results}
This section presents the findings from the analysis of the dataset and the evaluation of machine learning and deep learning models for mental health classification. First, we provide an \textit{Overview of Mental Health Distribution}, highlighting the inherent class imbalances within the dataset and their implications for model development. Next, the \textit{Hyperparameter Optimization} subsection details the parameter tuning process, which ensures that each model performs at its best configuration for both binary and multi-class classification tasks. Finally, the \textit{Model Performance Evaluation} subsection compares the models' performance based on key metrics, including F1 scores and Area Under the Receiver Operating Characteristic Curve (AUROC). Additionally, nuanced observations, such as the challenges associated with underrepresented classes, are discussed to provide deeper insights into the modeling outcomes.

\subsection{Overview of Mental Health Distribution}
Before hyperparameter optimization and model evaluation, an analysis of the dataset’s class distributions was conducted to highlight potential challenges in classification. The dataset, sourced from Kaggle, contains a total of 51,074 unique statements categorized into three primary groups: \textit{Normal} (31\%), \textit{Depression} (29\%), and \textit{Other} (40\%). The \textit{Other} category encompasses a range of mental health statuses such as \textit{Anxiety}, \textit{Stress}, and \textit{Personality Disorder}, among others.

\textbf{Figure~\ref{fig:multi-class}} illustrates the expanded distribution of mental health statuses across seven detailed categories in the multi-class classification setup. The dataset shows a significant imbalance, with categories such as \textit{Normal}, \textit{Depression}, and \textit{Suicidal} dominating the distribution, while others like \textit{Stress} and \textit{Personality Disorder} are notably underrepresented. This class imbalance poses challenges for multi-class classification tasks, particularly for the accurate identification of minority classes. Addressing such imbalances requires techniques like class-weighted loss functions and the use of metrics such as weighted F1 scores for model evaluation.

\figurehere{1}

\figurehere{2}

For the binary classification task, the dataset is divided into two classes: \textit{Normal} and \textit{Abnormal}. The distribution, shown in \textbf{Figure~\ref{fig:binary-class}}, reveals that the \textit{Abnormal} class (labeled as 1) accounts for approximately twice the number of records as the \textit{Normal} class (labeled as 0). Although the imbalance is less severe compared to the multi-class scenario, it still necessitates strategies to ensure that the minority class (\textit{Normal}) is adequately captured during model training.

\subsection{Hyperparameter Optimization}
Hyperparameter optimization is a critical step in enhancing the performance of machine learning (ML) and deep learning (DL) models. For this study, a grid search or random search approach was employed to systematically explore a predefined range of hyperparameters for each model. The primary evaluation metric used to select the best-performing hyperparameter configuration was the weighted F1 score, as it effectively balances precision and recall, particularly in the presence of imbalanced class distributions. This approach ensures that the selected models perform robustly across both binary and multi-class mental health classification tasks. 

The optimized hyperparameters for each model, alongside their corresponding weighted F1 scores on the test set, are summarized in Table~\ref{tbl1:opt_hp}. These results highlight the configurations that achieved the best trade-off between underfitting and overfitting, providing insight into the hyperparameter values critical to the classification tasks.

\tablehere{1}

\subsection{Model Performance Evaluation}

The evaluation metrics, including F1 scores (\textbf{Table~\ref{tbl2:f1-scores}}) and Area Under the Receiver Operating Characteristic Curve (AUROC) (\textbf{Table~\ref{tbl3:auc_scores}}), reveal minimal numeric differences across the models for both binary and multi-class classification tasks. This consistency in performance can be attributed to two primary factors. First, each model underwent rigorous hyperparameter tuning, ensuring only the best configurations were used for evaluation. Second, the dataset size, being of medium volume, provided sufficient information for machine learning models to achieve strong performance, while deep learning models could not fully demonstrate their potential advantages due to the limited data scale.

\tablehere{2}

\tablehere{3}

In the binary classification task, all models exhibited competitive F1 scores and AUROC values, effectively balancing precision and recall while distinguishing between normal and abnormal mental health statuses. Deep learning models such as \textit{ALBERT} and \textit{GRU} demonstrated slightly superior performance, achieving AUROC values of 0.95 and 0.94, respectively, which highlights their ability to capture complex linguistic patterns. Machine learning models, including \textit{Logistic Regression} and \textit{LightGBM}, also performed well, with AUROC scores of 0.93, underscoring their robustness in simpler classification settings.

In the multi-class classification task, a slight decline in performance was observed compared to the binary task. This decline aligns with the increased complexity of distinguishing between seven mental health categories. Nevertheless, deep learning models retained their advantage, with \textit{GRU} and \textit{LightGBM} achieving the highest micro-average AUROC scores of 0.97, followed closely by \textit{ALBERT} with an AUROC of 0.95. Machine learning models such as \textit{Logistic Regression} and \textit{Random Forest} also performed commendably, with AUROC scores of 0.96, demonstrating their ability to handle multi-class tasks effectively when optimized.

Another important observation in the multi-class classification task is the consistently lower AUROC scores for Depression (Class 2) across all machine learning models, with values not exceeding 0.90. While deep learning models demonstrated a slight improvement, their performance for this class remained comparatively weaker than for other categories. This difficulty likely arises from the significant overlap between Depression (Class 2) and other categories in both linguistic and contextual features. The reduced AUROC scores highlight the models' challenges in effectively distinguishing Depression, resulting in higher misclassification rates. These findings emphasize the need for refined feature engineering techniques or more sophisticated model architectures to enhance the separability and accurate classification of this particular class.

The minimal differences in performance metrics across models suggest that the combined effects of comprehensive hyperparameter optimization and dataset size contributed significantly to these results. Binary classification consistently outperformed multi-class classification, likely due to its reduced complexity and fewer decision boundaries. While deep learning models demonstrated their ability to capture intricate patterns, machine learning models offered competitive performance, making them practical alternatives for medium-sized datasets.

Performance metrics for F1 scores and AUROC values are detailed in \textbf{Table~\ref{tbl2:f1-scores}} and \textbf{Table~\ref{tbl3:auc_scores}}, respectively. This analysis highlights the importance of balancing model complexity with dataset characteristics and emphasizes the critical role of hyperparameter tuning in achieving optimal results.

\section{Discussion}

This tutorial serves as a practical resource to address key methodological and analytical challenges in mental health detection on social media, as identified in the systematic review \cite{cao2024mental}. By focusing on best practices and reproducible methods, the tutorial aims to advance research quality and promote equitable outcomes in this important field.

A critical issue identified in the review is the narrow scope of datasets, which are often limited to specific social media platforms, languages, or geographic regions. This lack of diversity restricts the generalizability of findings. In this tutorial, strategies for expanding data diversity are explored, including integrating datasets across multiple platforms, collecting data from underrepresented regions, and analyzing multilingual content. These efforts aim to make research outcomes more inclusive and applicable to diverse populations.

Text preprocessing emerged as another key challenge, particularly in handling linguistic complexities such as negations and sarcasm. These nuances are critical for accurately interpreting mental health expressions. This tutorial offers practical guidelines for building preprocessing pipelines that address these complexities. Techniques for advanced tokenization, feature extraction, and managing contextual meanings are discussed to enhance the reliability of text-based analyses.

Research practices related to model optimization and evaluation were also found to be inconsistent in many studies. Hyperparameter tuning and robust data partitioning are essential for reliable outcomes, yet they are often inadequately implemented. This tutorial provides step-by-step instructions for optimizing models and ensuring fair evaluations, emphasizing the importance of strategies like cross-validation and train-validation-test splits. By following these practices, researchers can reduce bias and improve the validity of their results. 

Evaluation metrics were another area of concern, with many studies relying on accuracy despite its limitations in imbalanced datasets. This tutorial highlights the importance of metrics such as precision, recall, F1-score, and AUROC, which provide a more balanced assessment of model performance. Additionally, practical approaches to managing imbalanced datasets, including oversampling, undersampling, and synthetic data generation, are illustrated.

Transparency in reporting methodologies and results is a foundational element of reproducible research. This tutorial encourages researchers to document their processes comprehensively, including data collection, preprocessing, model development, and evaluation. Sharing code and datasets is also emphasized, fostering collaboration and allowing other researchers to validate findings.

Ethical considerations are central to mental health research, particularly when using sensitive social media data. This tutorial stresses the need for privacy protection and adherence to ethical standards, ensuring that research respects the rights and dignity of individuals. Responsible data handling and clear communication of ethical practices are essential for maintaining trust and accountability in this field.

By addressing these challenges, this tutorial equips researchers with the tools and practices needed to improve the quality and impact of their work. Ultimately, these advancements contribute to the broader goal of promoting equitable and effective mental health interventions on a global scale.

\printbibliography

\newpage
\begin{sidewaystable}[ht]
\centering
\caption{Best Hyperparameters for Binary and Multi-Class Classification Models}
\label{tbl1:opt_hp}
\begin{tabular}{|l|p{5cm}|p{5cm}|p{9cm}|}
\hline
\textbf{Model} & \textbf{Best Parameters (Binary)} & \textbf{Best Parameters (Multi-Class)} & \textbf{Interpretation} \\
\hline
\textbf{Logistic Regression} & 
\texttt{\{C: 10, solver: `liblinear', penalty: `l2', class\_weight: None\}} & 
\texttt{\{C: 10, solver: `lbfgs', penalty: `l2', multi\_class: `multinomial', class\_weight: `balanced'\}} & 
For binary tasks, \texttt{liblinear} is chosen for smaller datasets. For multi-class, \texttt{lbfgs} supports \texttt{`multinomial'} strategy to optimize across multiple categories. Regularization strength (\texttt{C}) of 10 prevents overfitting. \\
\hline
\textbf{SVM} & 
\texttt{\{C: 1, kernel: `rbf', class\_weight: `balanced', gamma: `scale'\}} & 
\texttt{\{C: 1, kernel: `rbf', class\_weight: `balanced', gamma: `scale'\}} &
The RBF kernel captures nonlinear relationships in text data, while \texttt{class\_weight: `balanced'} was selected to address class imbalance. Regularization strength (\texttt{C}) balances margin maximization and misclassification. \\
\hline
\textbf{Random Forest} & 
\texttt{\{n\_estimators: 100, max\_depth: None, min\_samples\_split: 5, min\_samples\_leaf: 1, class\_weight: `balanced'\}} & 
\texttt{\{n\_estimators: 200, max\_depth: None, min\_samples\_split: 2, min\_samples\_leaf: 2, class\_weight: `balanced'\}} & 
For binary tasks, 100 trees ensure stability. For multi-class, 200 trees improve coverage of complex class distributions. Weighted class adjustments handle imbalances. \\
\hline
\textbf{LightGBM} & 
\texttt{\{n\_estimators: 100, learning\_rate: 0.1, max\_depth: -1, num\_leaves: 50, min\_child\_samples: 10, class\_weight: None\}} & 
\texttt{\{n\_estimators: 100, learning\_rate: 0.1, max\_depth: None, num\_leaves: 63, class\_weight: `balanced'\}} & 
For both tasks, LightGBM achieves efficiency via leaf-wise tree growth. For multi-class, additional leaves (63) improve representation of minority classes. \\
\hline
\textbf{ALBERT} & 
\texttt{\{lr: 1.46e-05, epochs: 4, dropout: 0.11\}} & 
\texttt{\{lr: 1.17e-05, epochs: 4, dropout: 0.15\}} & 
ALBERT’s lightweight architecture fine-tunes well with minimal learning rates and dropout for regularization. Minor adjustments improve class representation in multi-class settings. \\
\hline
\textbf{GRU} & 
\texttt{\{embedding\_dim: 156, hidden\_dim: 467, lr: 0.0004, epochs: 5\}} & 
\texttt{\{embedding\_dim: 236, hidden\_dim: 730, lr: 0.0003, epochs: 6\}} & 
Embedding dimensions and hidden states effectively capture sequential dependencies in text. Multi-class configurations benefit from higher hidden dimensions and epochs. \\
\hline
\end{tabular}
\end{sidewaystable}

\begin{table}[htbp]
\centering
\caption{Weighted F1 Scores of Models for Binary and Multi-Class Classification Tasks}
\label{tbl2:f1-scores}
\begin{tabular}{lcc}
\hline
\textbf{Model} & \textbf{Binary Classification} & \textbf{Multi-Class Classification } \\
\hline
Support Vector Machine (SVM) & 0.9401 & 0.7610 \\
Logistic Regression & 0.9345 & 0.7498 \\
Random Forest & 0.9359 & 0.7478 \\
LightGBM & 0.9358 & 0.7747 \\
ALBERT& 0.9576 & 0.7841 \\
Gated Recurrent Units (GRU) & 0.9512 & 0.7756 \\
\hline
\end{tabular}
\end{table}

\begin{table}[ht]
\centering
\caption{Area Under the Receiver Operating Characteristic Curve (AUROC) Scores for Binary and Multi-Class Classification Tasks}
\label{tbl3:auc_scores}
\resizebox{\textwidth}{!}{%
\begin{tabular}{lcc}
\toprule
\textbf{Model} & \textbf{Binary Classification AUROC} & \textbf{Multi-Class Classification Micro-Average AUROC} \\
\midrule
SVM & 0.93 & 0.95 \\
Logistic Regression & 0.93 & 0.96 \\
Random Forest & 0.92 & 0.96 \\
LightGBM & 0.93 & 0.97 \\
ALBERT & 0.95 & 0.97 \\
GRU & 0.94 & 0.97 \\
\bottomrule
\end{tabular}%
}
\end{table}

\newpage
\begin{figure}[h!]
 \centering
 \includegraphics[width=0.8\textwidth]{Figures/multi.png}
 \caption{Multi-class distribution of mental health statuses.}
 \label{fig:multi-class}
\end{figure}

\begin{figure}[h!]
 \centering
 \includegraphics[width=0.8\textwidth]{Figures/binary.png}
 \caption{Binary classification distribution of \textit{Normal} versus \textit{Abnormal} mental health statuses.}
 \label{fig:binary-class}
\end{figure}

\end{document}




%% 
%% Copyright (C) 2019 by Daniel A. Weiss <daniel.weiss.led at gmail.com>
%% 
%% This work may be distributed and/or modified under the
%% conditions of the LaTeX Project Public License (LPPL), either
%% version 1.3c of this license or (at your option) any later
%% version.The latest version of this license is in the file:
%% 
%% http://www.latex-project.org/lppl.txt
%% 
%% Users may freely modify these files without permission, as long as the
%% copyright line and this statement are maintained intact.
%% 
%% This work is not endorsed by, affiliated with, or probably even known
%% by, the American Psychological Association.
%% 
%% This work is "maintained" (as per LPPL maintenance status) by
%% Daniel A. Weiss.
%% 
%% This work consists of the fileapa7.dtx
%% and the derived files apa7.ins,
%% apa7.cls,
%% apa7.pdf,
%% README,
%% APA7american.txt,
%% APA7british.txt,
%% APA7dutch.txt,
%% APA7english.txt,
%% APA7german.txt,
%% APA7ngerman.txt,
%% APA7greek.txt,
%% APA7czech.txt,
%% APA7turkish.txt,
%% APA7endfloat.cfg,
%% Figure1.pdf,
%% shortsample.tex,
%% longsample.tex, and
%% bibliography.bib.
%% 
%%
%% End of file `./samples/longsample.tex'.



%\bibliographystyle{elsarticle-num}
%\bibliography{bibliography.bib}

	
	
	

	
\end{document}