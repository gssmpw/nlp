In decision-making problems, the outcome of an intervention often depends on the causal relationships between system components and is highly costly to evaluate. In such settings, causal Bayesian optimization (\textsc{cbo}) can exploit the causal relationships between the system variables and sequentially perform interventions to approach the optimum with minimal data. Extending \textsc{cbo} to the multi-outcome setting, we propose \textit{multi-objective causal Bayesian optimization} (\textsc{mo-cbo}), a paradigm for identifying Pareto-optimal interventions within a known multi-target causal graph. We first derive a graphical characterization for potentially optimal sets of variables to intervene upon. Showing that any \textsc{mo-cbo} problem can be decomposed into several traditional multi-objective optimization tasks, we then introduce an algorithm that sequentially balances exploration across these tasks using relative hypervolume improvement.  The proposed method will be validated on both synthetic and real-world causal graphs, demonstrating its superiority over traditional (non-causal) multi-objective Bayesian optimization in settings where causal information is available.
