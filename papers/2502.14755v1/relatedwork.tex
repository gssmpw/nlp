\section{Related Work}
We combine multi-objective Bayesian optimization (\textsc{mobo}) and techniques from causal inference to achieve multi-objective causal Bayesian optimization.
Developments in \textsc{mobo} range from single-point methods \cite{ParEGO, PAL, USeMO} and batch methods \cite{MOEA/D-EGO, dgemo, qNEHVI} to the derivations of efficient acquisition functions \cite{EHVI,PES}. 
Most relevant for our work is \textsc{dgemo} \cite{dgemo}, a well-established \textsc{mobo} algorithm that employs a novel batch selection strategy emphasizing on sample diversity in the input space.

Leveraging tools from causal inference to make causally-informed decisions is a field called causal decision-making. Within this field, there is a growing body of research specifically focused on advancing \textsc{cbo} \cite{CBO}. These advancements include extensions such as constrained \textsc{cbo} \cite{pmlr-v202-aglietti23a}, dynamic \textsc{cbo} \cite{NEURIPS2021_577bcc91}, and various other variants \cite{pmlr-v206-branchini23a, pmlr-v216-gultchin23a, sussex2023modelbasedcausalbayesianoptimization, sussex2023adversarialcausalbayesianoptimization, cbo_exogenous_learning, causal_elicitation}. They are all designed to optimize a single target variable, rendering them infeasible for applications with multiple outcomes.