\section{Experiments}

We evaluate \textsc{Causal ParetoSelect} with $\mathcal{S} = \mathbb{O}_{\mathcal{G},\mathbf{Y}}$ on the causal graphs of \cref{fig:synthtic_dags} (a) (\textsc{synthetic-1}), \cref{fig:synthtic_dags} (b) (\textsc{synthetic-2}), and \cref{fig:health_dag} (\textsc{Health}), featuring both synthetic and real-world settings. The full description of the \textsc{scm}s can be found in \cref{appendix:experiments}. We assume an initial dataset $\mathcal{D}^{I} = \{ ((\mathbf{X}_s, \mathbf{x}_s^k), \boldsymbol{\mu}(\mathbf{X}_s, \mathbf{x}_s^k)) \}_{k=1,s=1}^{K,|\mathcal{S}|}$ with $K=5$ samples per intervention set. In each experiment, we aim to minimize the target variables. The batch size is set to \num{5}. For reproducibility, all experiments are run across 10 random seeds with varying initializations of $\mathcal{D}^{I}$. 

\begin{figure*}[t]
\vspace{-1cm}
\hspace{-0.6cm}
\includegraphics{SA_metrics.pdf}
\vspace{-1.7cm}
\caption{Convergence to the causal Pareto front across all experiments, measured by the generational distance (upper row) and inverted generational distance (lower row). Lower values indicate better approximations of the ground-truth. The number of interventions refers to the count of intervention variables that were intervened upon. Shaded areas represent $\pm$ standard deviation across \num{10} random seeds.
}
\label{fig:performance_metrics}
\end{figure*}

To the best of our knowledge, there exists no other multi-objective optimization method in the literature that can consider the causal structure $\langle \mathcal{G}, \mathbf{Y}, \mathbf{X}, \mathbf{C} \rangle$. 
Thus, our baseline method corresponds to the \textsc{mobo} algorithm \textsc{dgemo} \citep{dgemo}, applying it to all treatment variables simultaneously and neglecting available causal information.
Here, the objectives are $\mu_i(\mathbf{X}, \ \cdot \ ): \mathcal{D}(\mathbf{X}) \rightarrow \mathbb{R}$, $\mathbf{x} \mapsto \mu(\mathbf{X},\mathbf{x})$, $i=1,\dots,m$. 
We compare \textsc{Causal ParetoSelect} to the the baseline in regards to their approximations of $\mathcal{P}_f^{\textsf{c}}(\mathbb{P}(\mathbf{X}))$. 
To this end, we measure the accuracy of these approximations to the causal Pareto front using the generational distance (GD) as well as calculate the diversity of the solutions using the inverted generational distance (IGD) ~\cite{gd}.  
GD is defined as the average distance from any point in the approximated front to its closest point on the ground-truth front. Conversely,  IGD represents the average distance from any point in the ground-truth front to its closest point on the approximated front. 
The progressions of these performance metrics across all experiments are depicted in \cref{fig:performance_metrics}.


\subsection{Synthetic Problems}

\paragraph{\textsc{synthetic-1}} Here, it holds $\mathbb{O}_{\mathcal{G},\mathbf{Y}}= \{ \{X_1,X_2\} \}$ as discussed in \cref{subsec:search_space_reduction}.
The experimental results for \textsc{synthetic-1} are shown in \cref{fig:synthetic1_fronts} and demonstrate that the identified solutions from our method and \textsc{dgemo} closely align with the grount-truth front $\mathcal{P}_f^{\textsf{c}}(\mathbb{P}(\mathbf{X}))$. 
This alignment is further confirmed by the generational distance progressions in \cref{fig:performance_metrics} (a), which approach zero for both algorithms. Theoretically this result is expected, as $\mu(\mathbf{X},\mathbf{x}) = \mu(\mathbb{O}_{\mathcal{G},\mathbf{Y}},\mathbf{x}[\mathbb{O}_{\mathcal{G},\mathbf{Y}}])$ guarantees that the baseline contains Pareto-optimal solutions. Moreover, we observe that \textsc{Causal ParetoSelect} offers a better coverage of the front, a finding supported by its lower inverted generational distance in \cref{fig:performance_metrics} (a). This improvement stems from avoiding unnecessary interventions on $X_3$ and $X_4$, allowing for more exploratory interventions on $\mathbb{O}_{\mathcal{G},\mathbf{Y}}$ within the same number of interventions. 

\begin{figure}[h]
\centering
\vspace{-1.8cm}
\includegraphics{SA_S1_exp.pdf}
\vspace{-2.3cm}
\caption{\textsc{synthetic-1}. Causal Pareto front approximations. Our method offers a higher coverage of the ground-truth causal Pareto front.}
\label{fig:synthetic1_fronts}
\end{figure}

\paragraph{\textsc{synthetic-2}} 
\begin{figure}[t]
\centering
\vspace{-1.8cm}
\includegraphics{SA_S2_exp.pdf}
\vspace{-2.6cm}
\caption{\textsc{synthetic-2}. Causal Pareto front approximations. By utilizing the causal relation, our approach tightly fits the ground-truth causal Pareto front.}
\label{fig:synthetic2_fronts}
\end{figure}
In this setting, an unobserved confounder exists between $Y_1$ and $X_4$, placing \textsc{synthetic-2} in the general case where hidden confounders may influence target variables through their ancestors. Consequently, it holds $\mathbb{O}_{\mathcal{G},\mathbf{Y}}=\{ \{X_2,X_3\}, \{X_1,X_2,X_3\} \}$. The experimental results, illustrated in \cref{fig:synthetic2_fronts}, demonstrate that while the baseline method fails to identify solutions on the causal Pareto front, \textsc{Causal ParetoSelect} does discover them. This finding is further reflected through the performance metrics in \cref{fig:performance_metrics}, which progress to significantly lower values for our method. %This result is also represented with the metrics in \cref{fig:performance_metrics}, which progress to significantly lower values for our method. 
Further experiments reveal that only interventions on $\{ X_2,X_3\}$ can yield solutions on the causal Pareto front. This can be explained as follows: The baseline strategy disrupts the causal path $X_4 \rightarrow X_1 \rightarrow Y_1$, letting the unobserved confounder to influence $Y_1$ without propagating through the aforementioned path. In contrast, our approach allows interventions on $\{X_2,X_3\}$, preserving this causal structure. This distinction is crucial as the structural assignment of $Y_1$ includes the term $-X_1 \cdot X_2 \cdot U/2$, with $U$ denoting the unobserved confounder (all structural equations are specified in \cref{subsec:experiments.synthetic-2}). Not intervening on $X_1$, causes this term to always be negative, yielding lower function values for $Y_1$. However, if we do intervene on $X_1$, it is positive with probability $0.5$, causing higher values for $Y_1$ in the averaged outcomes. 


\subsection{Real-world Problem}

\paragraph{\textsc{health}} We revisit the causal graph from \cref{fig:health_dag}, which is based on real-world causal relationships in the healthcare setting \cite{ferro_healthcare}. 
For patients sensitive to Statin medication, one might aim to minimize both Statin usage and \textsc{psa} levels simultaneously. 
The possibly Pareto-optimal minimal intervention sets are $\mathbb{O}_{\mathcal{G},\mathbf{Y}} = \{ \{\textsc{bmi}, \text{Aspirin} \} \}$. 
The experimental results for \textsc{health}, depicted in \cref{fig:health_fronts}, show that both the baseline and \textsc{Causal ParetoSelect} identify solutions on the causal Pareto front.
\begin{figure}[b]
\centering
\vspace{-1.8cm}
\includegraphics{SA_health_exp.pdf}
\vspace{-2.6cm}
\caption{\textsc{Health}. Causal Pareto front approximations on a real-world healthcare application. Our approach yields a better coverage of the ground-truth.}
\label{fig:health_fronts}
\end{figure}
The baseline method however, yields a sparser approximation, as confirmed by the progression of the inverted generational distance in \cref{fig:performance_metrics} (c). 
In contrast, \textsc{Causal ParetoSelect} discovers a larger number of solutions with the same number of interventions.

