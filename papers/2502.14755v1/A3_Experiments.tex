\section{Experiments}\label{appendix:experiments}

\subsection{\textsc{synthetic-1}}\label{subsec:experiments.synthetic-1}
We introduce the first synthetic \textsc{mo-cbo} problem in our experimental study, referred to as \textsc{synthetic-1}, which is defined by the causal graph $\mathcal{G}$ and associated structural assignments presented in \autoref{fig:experiments.synthetic_1.scm}. The interventional domains are specified as $\mathcal{D}(X_1),\mathcal{D}(X_2) = [-1,2]$ and $\mathcal{D}(X_3),\mathcal{D}(X_4) = [-1,1]$. Moreover, all exogenous variables follow the standard normal distribution, and there are no unobserved confounders. 

\begin{figure}[t]
    \centering
    \vspace{-1.5cm}
    \includegraphics{SA_S1_SCM.pdf}
    \vspace{-0.7cm}
    \caption{\textsc{synthetic-1}. An \textsc{scm} consisting of four treatment and two output variables, depicted with grey and red nodes, respectively. There are no unobserved confounders.}
    \label{fig:experiments.synthetic_1.scm}
\end{figure}

\subsection{\textsc{synthetic}-2}\label{subsec:experiments.synthetic-2}
\textsc{synthetic-2} is the next \textsc{mo-cbo} problem of our experimental study, defined by the causal graph $\mathcal{G}$ and associated structural equations in \autoref{fig:experiments.synthetic-2.scm}. The interventional domains are $\mathcal{D}(X_1) = [-2,5]$, $\mathcal{D}(X_4) = [-4,5]$ and $\mathcal{D}(X_i) = [0,5]$ for $i=1,2$. Moreover, the exogenous variables $U_{X_i}, U_{Y_i}$ follow a Gaussian distribution, and there is an unobserved confounder $U$ influencing the target variable $Y_1$ and its ancestor $X_4$.

\begin{figure}[h]
    \centering
    \vspace{-1cm}
    \includegraphics{SA_S2_SCM.pdf}
    \vspace{-2.5cm}
    \caption{\textsc{synthetic-2}. An \textsc{scm} consisting of four treatment and two output variables, depicted with grey and red nodes, respectively. It includes an unobserved confounder, denoted via the dashed bi-directed edge, affecting one output and its ancestor.}
    \label{fig:experiments.synthetic-2.scm}
\end{figure}


\subsection{\textsc{health}}
The \textsc{mo-cbo} problem \textsc{Health} is defined by the causal graph and structural equations in \cref{fig:experiments.healthcare.scm}. This model originates from previous works of \citet{ferro_healthcare}, and is based on real-world causal relationships. It captures prostate-specific-antigen (\textsc{psa}) levels in causal relation to its risk factors, such as \textsc{bmi}, calorie intake (\textsc{ci}) and aspirin usage. The variable Aspirin indicates the daily aspirin regimen while Statin denotes a subject' statin medication. Additionally, \textsc{psa} represents the total antigen level circulating in a subject’s blood, measured in ng/mL. For patients sensitive to Statin medications, the aim is to determine how to manipulate relevant risk factors to minimize both Statin and \textsc{psa}. To this end, we treat both Statin and \textsc{psa} as target variables. The treatment variables include \textsc{bmi}, Weight, \textsc{ci}, and Aspirin usage with interventional domains $\mathcal{D}(\textsc{bmi}) = [20,30]$, $\mathcal{D}(\text{Weight})=[50,100]$, $\mathcal{D}(\textsc{ci})=[-100,100]$ and $\mathcal{D}(\text{Aspirin})= [0,1]$. We choose to consider a specific age groups of interest, and define $U_{\text{age}}$ as a Gaussian random variable with mean 65 and standard deviation 1, focusing on individuals close to the age of 65. The single-objective version of \textsc{Health}, aiming to minimize only \textsc{psa}, has previously been used to demonstrate the applicability of \textsc{cbo} \citep{CBO}, as well as for several of its variants (e.g. \citet{pmlr-v216-gultchin23a} and \citet{pmlr-v202-aglietti23a}). 

\begin{figure}[t!]
    \centering
    \vspace{-17cm}
    \includegraphics{SA_health_SCM.pdf}
    \vspace{-4.2cm}
    \caption{\textsc{Health}. An \textsc{scm} with relations between variables such as age, \textsc{bmi}, aspirin and statin usage, and their effects on \textsc{psa} levels \citep{pmlr-v216-gultchin23a}. $\mathcal{U}(\cdot,\cdot)$ denotes a uniform distribution and $t\mathcal{N}(a,b)$ a standard Gaussian distribution truncated between $a$ and $b$. Red, grey, and orange nodes depict target, manipulative, and non-manipulative variables, respectively.}
    \label{fig:experiments.healthcare.scm}
\end{figure}