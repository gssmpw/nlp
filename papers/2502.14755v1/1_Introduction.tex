\section{Introduction}\label{chap:intro}

Decision-making problems arise in a variety of domains, such as healthcare, manufacturing or public policy, and involve manipulating variables to obtain an outcome of interest. In many such domains, interventions are inherently costly, and practical applications are subject to budgetary constraints. Moreover, these systems are often governed by causal mechanisms, which can be exploited to efficiently approach optimal outcomes in a targeted and cost-efficient manner. A well-established strategy for optimizing such expensive-to-evaluate black-box functions is Bayesian optimization \cite{BO_review}, but it cannot leverage the causal structure between its input variables. To this end, causal Bayesian optimization (\textsc{cbo}) \cite{CBO} was introduced to generalize Bayesian optimization to settings where causal information is available. 
While existing \textsc{cbo} variants focus on optimizing a single objective, real-world systems often require simultaneous optimization of multiple outcomes. Here, the aim is to establish optimal trade-offs between objectives, a so called Pareto front, instead of identifying a single objective's global optimum. As an example, consider the graph in \cref{fig:health_dag} (a), depicting the causal relationships between prostate specific antigen (\textsc{psa}) and its risk factors \cite{ferro_healthcare}. For patients sensitive to Statin medications, the aim is to determine how to reduce the relevant risk factors to minimize both Statin and \textsc{psa} simultaneously. \autoref{fig:health_dag} (b) shows how the optimal trade-offs between the targets could behave.

\begin{figure}
    \centering
    \vspace{-1.2cm}
    \includegraphics{SA_health.pdf}
    \vspace{-0.9cm}
    \caption{\textsc{mo-cbo} problem in healthcare. (a) Causal graph where red, grey, and orange nodes depict target, manipulative, and non-manipulative variables, respectively. (b) The solution consists of interventions that yield optimal trade-offs between the targets.}
    \label{fig:health_dag}
\end{figure}

\begin{figure*}[t]
    \includestandalone{SA_methodology}
    \caption{Overview of the \textsc{mo-cbo} methodology.}
    \label{fig:methodology_overview}
\end{figure*}

We propose \textit{multi-objective causal Bayesian optimization} (\textsc{mo-cbo}) as a method that generalizes \textsc{cbo} to such optimization problems involving multiple outcome variables. A visualization of our methodology is given in \cref{fig:methodology_overview}, highlighting some of the key contributions of our work:

\begin{enumerate}[left=0pt]
    \item We formally define \textsc{mo-cbo} as a new class of optimization problems.
    \item We provide a formal method to reduce the search space of \textsc{mo-cbo} problems, thereby disregarding sets of variables that are not worth intervening upon based on the graph topology. 
    \item We propose \textsc{Causal ParetoSelect}, an efficient algorithm for the exploration of multiple intervention strategies in parallel, guided by a custom acquisition function.
    \item We experimentally show that our approach can surpass traditional (non-causal) multi-objective Bayesian optimization in settings where causal relationships are known, achieving more cost-effective, diverse, and accurate solutions.
\end{enumerate}


To the best of our knowledge, there exists no other multi-objective optimization method in the literature that can consider the causal structure. We prove that our \textsc{mo-cbo} methodology establishes optimal trade-offs between target variables,  which state-of-the-art methods cannot reach for all experimental setups.

\subsection{Related Work}
We combine multi-objective Bayesian optimization (\textsc{mobo}) and techniques from causal inference to achieve multi-objective causal Bayesian optimization.
Developments in \textsc{mobo} range from single-point methods \cite{ParEGO, PAL, USeMO} and batch methods \cite{MOEA/D-EGO, dgemo, qNEHVI} to the derivations of efficient acquisition functions \cite{EHVI,PES}. 
Most relevant for our work is \textsc{dgemo} \cite{dgemo}, a well-established \textsc{mobo} algorithm that employs a novel batch selection strategy emphasizing on sample diversity in the input space.

Leveraging tools from causal inference to make causally-informed decisions is a field called causal decision-making. Within this field, there is a growing body of research specifically focused on advancing \textsc{cbo} \cite{CBO}. These advancements include extensions such as constrained \textsc{cbo} \cite{pmlr-v202-aglietti23a}, dynamic \textsc{cbo} \cite{NEURIPS2021_577bcc91}, and various other variants \cite{pmlr-v206-branchini23a, pmlr-v216-gultchin23a, sussex2023modelbasedcausalbayesianoptimization, sussex2023adversarialcausalbayesianoptimization, cbo_exogenous_learning, causal_elicitation}. They are all designed to optimize a single target variable, rendering them infeasible for applications with multiple outcomes.
