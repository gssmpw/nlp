\section{Conclusion}
In this paper, we introduced \textsc{mo-cbo} as a new problem class in order to optimize multiple target variables within a known causal graph by sequentially performing interventions on the system. We proved that a \textsc{mo-cbo} problem can be decomposed into a collection of $|\mathbb{P}(\mathbf{X})| = 2^{|\mathbf{X}|}$ local problems, and solving it essentially corresponds to solving these local problems. To reduce the search space, we derived theoretical results that identify possibly Pareto-optimal minimal intervention sets in a given causal graph. We proved that these sets comprise a minimal collection of local problems that are guaranteed to contain the optimal solutions of any \textsc{mo-cbo} problem. Moreover, we introduced \textsc{Causal ParetoSelect} as an algorithm that iteratively selects and solves local \textsc{mo-cbo} problems in the reduced search space based on relative hypervolume improvement.

Our theoretical and empirical findings highlight two distinct cases: When no unobserved confounders exist between target variables and their ancestors, both \textsc{mo-cbo} and \textsc{mobo} can recover the ground-truth causal Pareto front. However, our approach demonstrates greater cost efficiency while constructing a more diverse set of solutions. In contrast, when unobserved confounders are present between targets and their ancestors, traditional \textsc{mobo} approaches can fail to approximate the ground truth, whereas \textsc{mo-cbo} demonstrates efficient discovery of Pareto-optimal solutions. This occurs because unobserved confounders can propagate effects through the causal graph, and naively disrupting these paths can lead to suboptimal solutions.

In our algorithm, the surrogate model assumes independent outcomes which may limit efficiency since it overlooks shared endogenous confounders. Future work could enhance cost effectiveness by integrating multi-task Gaussian processes to better capture shared information across treatment variables. Other directions for future research include the adaptation of existing \textsc{cbo} variants to the multi-objective case. For instance, combining dynamic \textsc{cbo} \citep{NEURIPS2021_577bcc91} with \textsc{mo-cbo} would lead to a \textsc{mo-cbo} variant that can handle time-dynamic causal models. As the field of causal decision-making continues to grow, we anticipate more progress in the development of multi-objective frameworks to address complex, real-world challenges.