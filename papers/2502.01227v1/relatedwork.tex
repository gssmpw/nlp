\section{Related work}
%The ranking problem is adhered to {\em leader election}
%in self-stabilising population protocols, as any proper ranking affirms leader election in this model. 
%
%As indicated earlier, in self-stabilising protocols the ranking problem is closely related to the central problem of leader election, where in the final configuration a single agent 
%is in a {\em leader} state 
%and all other agents adopt 
%the {\em follower} state. 
In the standard model (with predefined starting configuration) of population protocols, the results in~\cite{DBLP:conf/wdag/ChenCDS14, DBLP:conf/soda/Doty14} laid down the foundation for the proof that leader election cannot be solved in a sublinear time with agents utilising a fixed number of states~\cite{DS18}.
In further work~\cite{DBLP:conf/icalp/AlistarhG15}, Alistarh and Gelashvili studied the relevant upper bounds including 
a new leader election protocol stabilising in time $O(\log^3 n)$ assuming $O(\log^3 n)$ states per agent.
%
Later, Alistarh {\em et al.}~\cite{DBLP:conf/soda/AlistarhAEGR17} considered more general trade-offs between the number of states and the time complexity of stabilisation.
%
In particular, they proposed a separation argument distinguishing between {\em slowly stabilising} population protocols 
which utilise $o(\log\log n)$ states and {\em rapidly stabilising} protocols relying on $O(\log n)$ states per~agent. 
This result coincides with another fundamental result by Chatzigiannakis {\em et al.}~\cite{DBLP:journals/tcs/ChatzigiannakisMNPS11} 
stating that population protocols utilising $o(\log n)$ states are limited to semi-linear predicates,
while the availability of $O(n)$ states (permitting unique identifiers) admits computation of more general symmetric predicates.
%
Further work proposes leader election in time 
$O(\log^2 n)$ w.h.p. and in expectation utilising $O(\log^2 n)$ states~\cite{DBLP:conf/podc/BilkeCER17}. 
The number of states was later reduced to $O(\log n)$ by
Alistarh et al. in \cite{DBLP:conf/soda/AlistarhAG18} and by Berenbrink et al. in \cite{DBLP:conf/soda/BerenbrinkKKO18} through the application of two types of synthetic coins.
%
In more recent work G\k asieniec and Stachowiak reduce utilisation of states to $O(\log\log n)$ while preserving the time complexity $O(\log^2 n)$ whp~\cite{DBLP:journals/jacm/GasieniecS21}.
The high probability can be traded for faster leader election in the expected parallel time $O(\log n\log\log n)$~\cite{DBLP:conf/spaa/GasieniecSU19}. This upper bound was recently reduced to the optimal expected time $O(\log n)$ by Berenbrink {\em et al.} in~\cite{DBLP:conf/stoc/BerenbrinkGK20}. 
%One of the main open problems in the area is to establish whether one can elect a single leader in time $o(\log^2 n)$ whp while preserving the optimal number of states $O(\log\log n).$ 

The fact that at least $n$ states on the top of the knowledge of the exact value of $n$ are needed in self-stabilising leader election (as in the ranking problem) was first observed in~\cite{CIW12}. 
One could use an upper bound on $n$ instead, if {\em loose-stabilisation} is permitted, i.e., when selected leader remains for an extended period of time, but then it has to be recomputed again~\cite{Sudo+20}.
Another example refers to recent study on $O(\log n)$-state loosely-stabilising phase clocks with application to an adaptive variant of the majority problem, see~\cite{BBH+22}. 
An alternative line of attack is to supply population protocols with extra features, where in the context of leader election one can find work on population protocols with an oracle~\cite{BBB13, FJ06}, mediated population protocols~\cite{MOK+12}, and population protocols with interactions along limited in size hyper-edges~\cite{XYK+13}. 
A different strand of work refers to communication networks limited to constant degree regular graphs including rings, in which state requirements are much lighter~\cite{AAF+05,CC19,CC20,YSM20,YSO+23}. 

The consideration of k-distant configurations in this paper has ties to k-stabilising protocols, which are self-stabilising protocols characterised by a known upper bound $k\leq n$ on the number of faults~\cite{BGK99}.
%
Numerous works, e.g., ~\cite{BGK99,GX99,KP97,KP95}, observed that systems 
recover more rapidly when subjected to fewer faults. 
The most efficient {\em $k$-linearly adaptive} protocols require only $O(k)$ transitions to stabilise, see~\cite{BDH06}. 
Some lower bounds for such protocols have been explored in~\cite{GT02}.
%
Due to the random nature of interactions and a severely constrained state space, our self-stabilising ranking protocol for k-distant configurations, while bearing similarities to k-stabilising protocols, still necessitates engagement from the entire population, resulting in a stabilisation time of $O({\rm\/min}(kn^{3/2},n^2\log^2 n))$. However, the value of $k$ in our protocol does not have to be predetermined.

In the ranking problem, which is the focus of this work, the agent's state space consists of 
$n$ {\em rank} states and $x$ {\em extra} states. The initial configuration of $n$ agents is 
an arbitrary arrangement of rank and extra states, with the goal of achieving self-ranking. 
Specifically, each agent must stabilize in a {\em unique} rank state silently, meaning that 
after stabilization, each agent remains in its designated state indefinitely.
%
The ranking problem is closely related to the {\em naming (labelling) problem}, where each agent is 
assigned a unique name (or label), as studied earlier in \cite{Bea12,Mic13,Bur19,Gas24,Gas25}  within the context 
of population protocols. The key difference is that, after ranking, any pair of agents knows 
how many agents with intermediate ranks separate them. Additionally, due to the restrictions on 
final states, computing the ranking is generally more challenging.
%
In this context, the primary focus of \cite{Bea12,Bur19} is on the feasibility of ranking and naming, 
while \cite{Mic13} examines the time required for the population to stabilize. A more comprehensive study 
on the state and parallel efficiency of naming is presented in \cite{Gas25}, where the authors explore 
trade-offs between the number of names, available states, and stabilization time. Another recent work, 
\cite{Gas24}, investigates a specific naming scheme (by consistent coordinates) in the context of distributed 
localization within {\em spatial population protocols}.
%
In the direct context of our work, in~\cite{BCC+21,DS18}, the authors show that any silently self-stabilising leader election protocol requires $\Omega(n)$ expected time. And in~\cite{BCC+21}, we find silent ranking protocols with the expected time $O(n)$ and time $O(n\log n)$ whp, both utilising $x=\Omega(n)$ extra states. 
However, the only known state-optimal ranking protocol $\mathcal{A}_G$ based on a single rule $(i,i)\rightarrow (i,(i+1) {\rm\ mod\ } n)$ and the state space 
$\{0,\dots,n-1\}$ has stabilisation time $\Theta(n^2).$