%%%%%%%% ICML 2025 EXAMPLE LATEX SUBMISSION FILE %%%%%%%%%%%%%%%%%

\documentclass{article}

% Recommended, but optional, packages for figures and better typesetting:
\usepackage{microtype}
\usepackage{graphicx}
\usepackage{subcaption}
\usepackage{booktabs} % for professional tables
\usepackage{adjustbox}
\usepackage{algpseudocode}

% hyperref makes hyperlinks in the resulting PDF.
% If your build breaks (sometimes temporarily if a hyperlink spans a page)
% please comment out the following usepackage line and replace
% \usepackage{icml2025} with \usepackage[nohyperref]{icml2025} above.
\usepackage{hyperref}

% Attempt to make hyperref and algorithmic work together better:
\newcommand{\theHalgorithm}{\arabic{algorithm}}

% Use the following line for the initial blind version submitted for review:
% \usepackage{icml2025}

% If accepted, instead use the following line for the camera-ready submission:
\usepackage[accepted]{icml2025}

% For theorems and such
\usepackage{amsmath}
\usepackage{amssymb}
\usepackage{mathtools}
\usepackage{amsthm}
\usepackage{booktabs}
% if you use cleveref..
\usepackage[capitalize,noabbrev]{cleveref}

%%%%%%%%%%%%%%%%%%%%%%%%%%%%%%%%
% THEOREMS
%%%%%%%%%%%%%%%%%%%%%%%%%%%%%%%%
\theoremstyle{plain}
\newtheorem{theorem}{Theorem}[section]
\newtheorem{proposition}[theorem]{Proposition}
\newtheorem{lemma}[theorem]{Lemma}
\newtheorem{corollary}[theorem]{Corollary}
\theoremstyle{definition}
\newtheorem{definition}[theorem]{Definition}
\newtheorem{assumption}[theorem]{Assumption}
\theoremstyle{remark}
\newtheorem{remark}[theorem]{Remark}

% Todonotes is useful during development; simply uncomment the next line
%    and comment out the line below the next line to turn off comments
%\usepackage[disable,textsize=tiny]{todonotes}
\usepackage[textsize=tiny]{todonotes}

\newcommand{\JD}[1]{{\color{blue}[JD: #1]}}
\newcommand{\HM}[1]{{\color{red}[HM: #1]}}
% The \icmltitle you define below is probably too long as a header.
% Therefore, a short form for the running title is supplied here:
\icmltitlerunning{S2TX: cross-attention Multi-Scale State-Space Transformer for Time Series Forecasting}

\begin{document}

\twocolumn[
\icmltitle{S2TX: Cross-Attention Multi-Scale State-Space Transformer for Time Series Forecasting}

% It is OKAY to include author information, even for blind
% submissions: the style file will automatically remove it for you
% unless you've provided the [accepted] option to the icml2025
% package.

% List of affiliations: The first argument should be a (short)
% identifier you will use later to specify author affiliations
% Academic affiliations should list Department, University, City, Region, Country
% Industry affiliations should list Company, City, Region, Country

% You can specify symbols, otherwise they are numbered in order.
% Ideally, you should not use this facility. Affiliations will be numbered
% in order of appearance and this is the preferred way.
\icmlsetsymbol{equal}{*}

\begin{icmlauthorlist}
%\icmlauthor{Anonymous Author}{}
\icmlauthor{Zihao Wu}{yyy}
\icmlauthor{Juncheng Dong}{yyy,equal}
\icmlauthor{Haoming Yang}{yyy,equal}
\icmlauthor{Vahid Tarokh}{yyy}
% \icmlauthor{Firstname5 Lastname5}{yyy}
% \icmlauthor{Firstname6 Lastname6}{sch,yyy,comp}
% \icmlauthor{Firstname7 Lastname7}{comp}
%\icmlauthor{}{sch}
% \icmlauthor{Firstname8 Lastname8}{sch}
% \icmlauthor{Firstname8 Lastname8}{yyy,comp}
% \icmlauthor{}{sch}
% \icmlauthor{}{sch}
\end{icmlauthorlist}

\icmlaffiliation{yyy}{Department of Electrical and
Computer Engineering, Duke University, Durham, NC 27708, USA}
% \icmlaffiliation{comp}{Company Name, Location, Country}
% \icmlaffiliation{sch}{School of ZZZ, Institute of WWW, Location, Country}
% \icmlcorrespondingauthor{Anonymous Author}{}
\icmlcorrespondingauthor{Zihao Wu}{zihao.wu@duke.edu}
% You may provide any keywords that you
% find helpful for describing your paper; these are used to populate
% the "keywords" metadata in the PDF but will not be shown in the document
\icmlkeywords{Machine Learning, ICML}
\vskip 0.3in
]
% this must go after the closing bracket ] following \twocolumn[ ...

% This command actually creates the footnote in the first column
% listing the affiliations and the copyright notice.
% The command takes one argument, which is text to display at the start of the footnote.
% The \icmlEqualContribution command is standard text for equal contribution.
% Remove it (just {}) if you do not need this facility.

% \printAffiliationsAndNotice{}  % leave blank if no need to mention equal contribution
\printAffiliationsAndNotice{\icmlEqualContribution} % otherwise use the standard text.

\begin{abstract}

Time series forecasting has recently achieved significant progress with multi-scale models to address the heterogeneity between long and short range patterns. 
Despite their state-of-the-art performance, we identify two potential areas for improvement. 
First, the variates of the multivariate time series are processed independently. Moreover, the multi-scale (long and short range) representations are learned separately by two independent models without communication. In light of these concerns, we propose \emph{State Space Transformer with cross-attention} (S2TX). S2TX employs a cross-attention mechanism to integrate a Mamba model for extracting long-range cross-variate context and a Transformer model with local window attention to capture short-range representations. By cross-attending to the global context, the Transformer model further facilitates variate-level interactions as well as local/global communications. Comprehensive experiments on seven classic long-short range time-series forecasting benchmark datasets demonstrate that S2TX can achieve highly robust SOTA results while maintaining a low memory footprint.
\end{abstract}

\section{Introduction}
Forecasting multivariate time series represents a core learning paradigm designed to predict upcoming time steps using historical data. This machine learning task finds application across a range of domains including the economy \citep{koop2010bayesian}, epidemiology \citep{nguyen2021forecasting}, and meteorology \citep{angryk2020multivariate}. Due to its significant influence, multivariate time series forecasting has garnered considerable focus. State-of-the-art (SOTA) methods for multivariate time series forecasting predominantly utilize two types of sequence models: transformers and state-space models \cite{vaswani2017attention, gu2023mamba}. By employing these foundational structures, researchers aim to advance this research domain by harnessing two key features of multivariate time series: 1) identifying temporal dependencies and 2) understanding inter-variate correlations. Effectively integrating both temporal dynamics and the interactions between variates within a single learning model is essential for the precise forecasting of these intricate multivariate time series \citep{box2015time}. %\JD{citation?}.

%Significant progress has been made in multivariate forecasting with the introduction of the transformer architecture and the attention mechanism. 

%Another line of research, state-space models, has greatly improved the capabilities of deep learning models to retain long-range context. 


\begin{figure}[t]
    \centering
    \includegraphics[width=\linewidth]{./images/comparison.pdf} % Change to your image file
    \caption{Overview of the performance of different architectures over 7 different benchmark datasets. Average results (MSE) are reported. }
    \label{fig:overview}
\end{figure}

A recent advancement~\citet{xusst} integrates transformers and state-space models within a multi-scale framework: it first breaks down the input time series into shorter high resolution \emph{patches} and longer low resolution patches. Subsequently, it feeds the high resolution patches into a transformer model with local-attention to extract fine-grained local features and the low resolution longer patches to a state-space model (i.e., Mamba~\cite{gu2023mamba}) to learn long-range global features.
% a sequence is broken down into shorter high resolution time-series patches and fed to a transformer to leverage its ability to extract fine-grained local features, while the low resolution longer patches are inputted to the state-space model to learn long-range global features. 
This multi-scale mixture of Mamba and transformer models greatly improves the modeling of temporal dependencies. However, it leaves a crucial aspect of multivariate time series forecasting unattended, that is, the correlation between variates. 
Additionally, the local and global features are modeled independently, which overlooks the interplay between global and local features. Such global-local interplay is manifested in many real-world scenarios. For example, the commonly known \emph{El Ninõ} effect is a global, long-term weathering effect in the time-scale of years; but this global weather pattern will greatly affect the short-term local time series within days \citep{hsieh2004nonlinear}. 

The cross-variate correlation and global-local features interplay, illustrated in Figure \ref{fig:cross-all-correlation}, are two crucial aspects of multivariate time series forecasting. Global patterns encompassed in the purple boxes consistently suggest increased local variation while the red-boxed region indicates a strongly inversed correlation between the two variates. To explicitly include these two crucial aspects, we introduce \emph{State Space Transformer with Cross-attention} (S2TX) where we connect cross-variate global features with fine grained local features through a carefully designed cross-attention mechanism. Specifically, we apply Mamba as the global model to process long-range, low-resolution patches across all variates in a single sweep, extracting cross-variate global context. This global context is then provided as the key and value for cross-attention to a decoder-like transformer model focusing on local, high-resolution, variate-independent patches.

\textbf{Contributions.} Our contributions are summarized below:
\begin{itemize}
    \item We identify two crucial aspects, cross-variate correlation and global-local interaction,  to improve the SOTA time series forecasting model.
    \item We propose a novel multi-scale architecture that incorporates these considerations through a cross-attention mechanism. In particular, our architecture learns variate-level correlation while leveraging the enhanced temporal learning of patchification. 
    \item We develop a cross-attention mixture of experts, enabling global-local feature interplay between a global feature-focused state-space model and a local feature-focused transformer model.
    \item We verify the efficacy of our proposed architecture on a comprehensive set of time series forecasting benchmarks.
\end{itemize}
%Recent studies have shown that processing patched time series data is more effective than elementwise processing in capturing temporal dependencies, as it introduces an inductive bias that aligns with the localized nature of time series data. Furthermore, it has been observed that using multi-scale patches enables models to learn distinct global and local features. 

%In many real-world scenarios, as illustrated in the figure, the significance lies not only in the features themselves but also in the interplay between global and local features. However, existing structures often lack an explicit mechanism for modeling this correlation. To address this, we propose using a cross-attention mechanism to explicitly model the interaction between global and local features. 

%While patch technique facilitates the capture of temporal dependencies, it transforms the time series of each variable into a multivariate time series, making it more challenging to effectively model interactions between variables. To address this, we propose leveraging Mamba as the global model to process long-range patches across all variables in a single sweep, extracting cross-variable global context. This global context is then provided as the key and value to the decoder-like transformer local model within a cross-attention mechanism. By utilizing Mamba as the global model, we take advantage of its linear time complexity and its capability to handle long series, even when the dimensionality of the data is high.

\begin{figure}[t]
    \centering
    \includegraphics[width=0.8\linewidth]{./images/sample.pdf} % Change to your image file
    \caption{A snippet of the weather dataset. Two variables (blue and green) were plotted over 720 time steps. The purple boxed region indicates where a global-local interaction exists, and the red boxed region indicates a cross-variate correlation. }
    \label{fig:cross-all-correlation}
\end{figure}

\section{Related Works}
The field of time series forecasting has seen significant evolution over the decades: shifting from classical mathematical tools \citep{bloomfield2004fourier, durbin2012time} and statistical techniques like ARIMA \citep{nerlove1971time, hyndman2018forecasting} to more recent deep learning approaches such as recurrent neural networks \citep{graves2013speech} and long-short term memory models \citep{gers2000learning}. Notably, in recent years, transformers \citep{vaswani2017attention} have demonstrated particularly promising performance on sequence modeling tasks, especially in natural language processing. Interestingly, studies have revealed that even simple linear layers can outperform complex transformer-based models in both performance and efficiency for time series forecasting \citep{zeng2023transformers, yang2024neural}.

% Making the subsections shorter so the related work can contain more paper
\textbf{Inverted Dimension.}
In investigating why transformers underperform in time series forecasting, \citet{liu2023itransformer} argues that the direct application of transformers that embed all variates is undesirable. This embedding compresses variates with distinct physical meanings and inconsistent measurement at each time step to a single token, erasing the important multivariate correlations. To address this limitation, the authors propose inverting the dimension of time and variates in the data while preserving the core mechanisms of the transformer. Many subsequent studies \citep{wang2025mamba, ahamed2024timemachine, xusst} build upon this paradigm, achieving improvements in both performance and efficiency. %This innovative approach effectively unlocks the potential of transformers for time series forecasting.
% Building on this innovative paradigm, subsequent works further refined the approach by replacing the Transformer with Mamba-based models, a promising alternative that offers linear training complexity.

\textbf{Patchification.}
Patchification of inverted data transforms the time series of each variate into a multivariate time sequence where the patches are stacked to construct an additional dimension. While patchification facilitates the capture of temporal dependencies by introducing an inductive bias aligned with the localized nature of time series data, it also overlooks the between-variate correlations due to the additional dimension: existing approaches, such as SST \citep{xusst} and PatchTST \citep{nie2022time}, treat each variate independently. Despite their strong performance, these methods lack any form of inter-variate communication. 
% Taking this out for now, maybe we can use this in 4.2? 
%Other methods like MOIRAI flatten all variables into a single sequence before patchification, but the extended sequence length, particularly in high-dimension, in addition to the quadratic complexity of the transformer, imposes a heavy computation burden.

\textbf{Mixture of Experts.}
The mixture of experts method receives increasing attention in sequence modeling after the release of Mamba \citep{gu2023mamba}. Combining the linear complexity of Mamba and the strong performance of transformers could lead to efficient and accurate sequence models. For instance, Jamba \citep{lieber2024jamba} employs a layerwise stacking of Mamba and attention layers, achieving superior performance in natural language processing compared to either component individually. For time series forecasting, SST \citep{xusst} utilizes Mamba to capture global patterns with prolonged patch lengths, while leveraging transformer to learn local details with shorter patch lengths. However, global and local patches are processed separately through each expert before their output embeddings are concatenated. Such inadequate communication between global and local features limits the integration results, restricting the model's ability to fully exploit each expert's complementary strength. 

\begin{figure*}[t]
    \centering
    \includegraphics[width=1\textwidth]{./images/structure.pdf} % Change to your image file
    \caption{Overview of the proposed architecture S2TX. Different variables (in different colors) of the time series are patched into global and local patches. The global patches are processed by the global model, which outputs the global context that is used to compute the key and value matrices during cross-attention with the local model. Skip connections and normalization layers are omitted for clarity of presentation.}
    \label{fig:structure}
\end{figure*}
%\textbf{Our Approach}
%\HM{Maybe we can take this out, seems redundant.}
%As we clarified the different aspects of multivariate time series forecasting, the goal is clear: with S2TX, we develop an architecture that combines the advantages of all three approaches. S2TX is a low memory architecture that maintains the learning of multivariate correlation while enabling global-local feature interplay through a cross-attentional mixture of experts.

\section{Preliminary}
\label{sec:preliminary}
In this section, we first formalize the modeling problem, then introduce the two main modules of our proposed architecture: state-space models and cross-attention. 

\subsection{Problem Setup}
\label{sec:problemSetup}
We consider the standard problem setup for time series forecasting framework \citep{liu2023itransformer}. Given a $D$-dimensional multivariate time series of length $L$ (look-back window) $\textbf{X}\in \mathbb{R}^{D\times L}$,  the goal is to predict $\textbf{Y}\in \mathbb{R}^{D\times H}$, the same $D$-dimensional multivariate time series in the future $H$ steps (horizon length). Assuming we have access to a training dataset with $N$ observations $\{\textbf{X}^{(i)},\textbf{Y}^{(i)}\}_{i=1}^N$, our goal is to learn a function $f_\phi(\textbf{X}^{(i)}): \mathbb{R}^{D\times L} \rightarrow \mathbb{R}^{D\times H}$ with parameter $\phi$ such that the mean squared error loss is minimized:
\begin{align}
    \mathcal{L}_{\mathrm{train}} = \frac{1}{N}\sum_{n=1}^{N}\|f_\phi(\textbf{X}^{(i)}) - \textbf{Y}^{(i)}\|_F^2,
\end{align}
where $F$ denotes the Frobenius norm~\citep{horn2012matrix}.

\subsection{State-Space Models}
\label{sec:SSM}
State-Space Models (SSMs) \citep{gu2020hippo, gu2021efficiently} are a family of sequence models inspired by continuous control systems described by the following equations
\begin{align}
    \mathrm{d}h = Ah + Bx,\ z = Ch + Dx,
\end{align}
where $x\in \mathbb{R}$ represents a one-dimensional input, $h\in \mathbb{R}^{d\times 1}$ is the hidden state, $z$ is the model output, $A\in \mathbb{R}^{d\times d}$, $B\in \mathbb{R}^{d\times 1}$, $C\in \mathbb{R}^{1\times d}$, and $D\in \mathbb{R}^{1\times 1}$ are parameter matrices. Matrix $D$ acts as a skip connection and is typically omitted in derivations. For multi-dimensional inputs, a stack of SSMs is employed.
The continuous system is then discretized into 
\begin{align}
    h_{t+1} = \bar{A}h_t + \bar{B}x_t,\ z_{t+1} = \bar{C}h_t,
\end{align}
where the discretized matrices are obtained with a discretization rule and a step size $\Delta$. For example, Mamba \citep{gu2023mamba} uses Zero-Order Holder rule such that $\bar{A} = \exp(\Delta A),\ 
\bar{B} = \exp(\Delta A)^{-1} (\exp(\Delta A) - \mathbb{I}) \cdot \Delta B$.

The discretized state-space models can be interpreted either as a convolutional neural network, enabling linear-time parallel training, or as a linear recurrent neural network, supporting constant-time inference, as demonstrated in S4 \citep{gu2021efficiently}. Building upon S4, Mamba extends this approach by making the matrices $B$ and $C$ input-dependent, transforming them into a selective SSM. Additionally, Mamba introduces a parallel scan algorithm to achieve linear-time training complexity.

\subsection{Cross-attention}
\label{sec:crossAttentioin}
Cross-attention is a generalization \citep{bahdanau2014neural} of self-attention \citep{vaswani2017attention}. 
% It was initially used in the decoder part of the transformer and later extended to facilitate cross-model interaction. 
Given source data $S\in \mathbb{R}^{L_S\times d_{\mathrm{model}}}$ and target data $T\in \mathbb{R}^{L_T\times d_\text{model}}$, the output of cross-attention is
\begin{align}
    \text{CrossAttention}(S,T) = \frac{(TW_q)(SW_k)^T}{\sqrt{d_\text{model}}}SW_v %\in \mathbb{R}^{L_T \times d_{\text{model}}},
\end{align}
where $W_k$, $W_q$, $W_v\in \mathbb{R}^{d_\text{model}\times d_\text{model}}$ are learnable parameters. From this perspective, self-attention can be achieved by substituting all instances of $S$ with $T$ in Cross-attention:
\begin{align}
    \text{SelfAttention}(T) = \text{CrossAttention}(T,T) %\in \mathbb{R}^{L_T \times d_{\mathrm{model}}}.
\end{align}
This cross-attention mechanism will allow us to compute cross-attentional weight where the influence of global patterns is weighted to predict a specific local pattern, allowing global-local interaction during inference. Notably, our application of cross-attention mechanism to integrate multi-scale features are commonly applied in computer vision tasks \citep{chen2021crossvit}.


\section{State-Space Transformer with Cross-attention}
\label{sec:proposedApproach}
Here we describe our proposed method State-Space Transformer with Cross-attention (S2TX). We first introduce the Multi-Scale patching process that decompose a long time series into global and local patches of different time scales. The low-resolution global patches were then fed to a Mamba-based global feature extractor to obtain the cross-variate global context. The global context is then applied as the key and value for a novel global-local cross-attention to improve the extraction of local features. Finally, we conduct a computation complexity analysis, showcasing that S2TX, with the addition of our novel cross-attention, maintained a low-memory footprint during training and inference. The general structure of S2TX is provided in Figure \ref{fig:structure}.

\begin{figure}[thbp]
    \centering
    \includegraphics[width=0.45\textwidth]{./images/patch.pdf} % Change to your image file
    \caption{Patch transforms a one-dimensional sequence to a sequence of patches.}
    \label{fig:patch}
\end{figure}

\subsection{Multi-Scale Patch}
\label{sec:multiscalePatch}
The patching technique has become increasingly popular for time series forecasting~\citep{gong2023patchmixer, nie2022time,xusst}. It aggregates local information into patches and effectively enhances the receptive field. Denote the sequence length of the look-back window by $L$, patch length by $PL$, stride by $STR$, and patch number by $PN$, where 
\begin{align}\label{eq:pn}
    PN = \left\lceil\frac{L-PL}{STR}\right\rceil. 
\end{align}
The patching technique transforms each (one-dimensional) variate of length $L$ into a $PL$-dimensional time series of length $PN$. More specifically, the input time series $\textbf{X}\in \mathbb{R}^{D\times L}$ is patched into $\mathbf{\tilde{X}}\in \mathbb{R}^{D\times PN\times PL}$. 

Intuitively the longer the stride, or the longer the patch length, the more long range temporal context is stored in a patch and vice versa. Utilizing this intuition, we apply the patching process onto the time series \emph{twice}: (i) one of them focuses on coarser granularity for global context, 
%Moreover, SST highlights that global patterns are more discernible at a coarser granularity, while local variations are revealed at a finer granularity. Following a similar approach, 
employing the full look-back window of length $L$, a larger patch length $PL_g$ and longer stride, along with the corresponding patch number $PN_g$ to obtain long-range global time series patches; (ii) the other leverages finer granularity with a fixed shorter look-back window of length $S$, a smaller patch length $PL_l$, and shorter stride with corresponding patch number $PN_l$ to obtain short-range local patches. The resulting two multi-scale time series patches $\mathbf{\tilde{X}}_g\in \mathbb{R}^{D\times PN_g\times PL_g}$ and $\mathbf{\tilde{X}}_l\in \mathbb{R}^{D\times PN_l\times PL_l}$ serve as inputs for the global and local models, respectively.
\begin{table*}[t]
\footnotesize
\centering
\renewcommand{\arraystretch}{0.9} % Adjust row height
\setlength{\tabcolsep}{3pt} % Adjust column spacing
\adjustbox{max width=\textwidth}{
\begin{tabular}{%lllllllllllllllllllllll
                lcccccccccccccccccccccc}
\toprule
 & \multicolumn{2}{c}{\textbf{S2TX}} & \multicolumn{2}{c}{\textbf{SST}} & \multicolumn{2}{c}{\textbf{S-Mamba}} & \multicolumn{2}{c}{\textbf{TimeM}} & \multicolumn{2}{c}{\textbf{iTrans}} & \multicolumn{2}{c}{\textbf{RLinear}} & \multicolumn{2}{c}{\textbf{PatchTST}} & \multicolumn{2}{c}{\textbf{CrossF}} & \multicolumn{2}{c}{\textbf{TimesNet}}  \\
 & \multicolumn{2}{c}{\text{2025}} & \multicolumn{2}{c}{\text{2025}} & \multicolumn{2}{c}{\text{2025}} & \multicolumn{2}{c}{\text{2024}} & \multicolumn{2}{c}{\text{2024}} & \multicolumn{2}{c}{\text{2024}} & \multicolumn{2}{c}{\text{2023}} & \multicolumn{2}{c}{\text{2023}} & \multicolumn{2}{c}{\text{2023}}  \\
 & \textbf{MSE} & \textbf{MAE} & \textbf{MSE} & \textbf{MAE} & \textbf{MSE} & \textbf{MAE} & \textbf{MSE} & \textbf{MAE} & \textbf{MSE} & \textbf{MAE} & \textbf{MSE} & \textbf{MAE} & \textbf{MSE} & \textbf{MAE} & \textbf{MSE} & \textbf{MAE} & \textbf{MSE} & \textbf{MAE}  \\ \midrule
\textbf{ETTh1} & & & & & & & & & & & & & & & & \\ 
96 &\textbf{0.376} &0.401&\underline{0.381} & 0.405 & 0.392 & \textbf{0.390} & 0.389 & 0.402 & 0.386 & 0.405 & 0.386 & \underline{0.395} & 0.414 & 0.419 & 0.423 & 0.448 & 0.384 & 0.402  \\ 
192 &\textbf{0.414}&\textbf{0.421}& \underline{0.430} & 0.434 & 0.449 & 0.439 & 0.435 & 0.440 & 0.441 & 0.436 & 0.437& \underline{0.424} & 0.460& 0.445 & 0.450 & 0.471 & 0.474 & 0.429\\ 
336&\textbf{0.432}&\textbf{0.435} & \underline{0.443} & \underline{0.446} & 0.467 & 0.481 & 0.450 & 0.448 & 0.487 & 0.458 & 0.479 & 0.446 & 0.501 & \underline{0.466} & 0.570 & 0.546 & 0.491 & 0.469  \\ 
720&\textbf{0.463}& \underline{0.473}& 0.502 & 0.501 & \underline{0.475} & 0.468 & 0.480 & \textbf{0.465} & 0.503 & 0.491 & 0.481 & 0.470 & 0.500 & 0.488 & 0.653 & 0.621 & 0.521 & 0.500 \\ \midrule
\textbf{ETTh2} & & & & & & & & & & & & & & & & \\ 
96&\textbf{0.279}& \underline{0.340}& 0.291 & 0.346 & 0.292 & 0.357 & 0.296 & 0.349 & 0.297 & 0.349 & \underline{0.288} & \textbf{0.338} & 0.302 & 0.348 & 0.745 & 0.584 & 0.340 & 0.374 \\ 
192&\textbf{0.362}&\underline{0.395} & \underline{0.369} & 0.397 & 0.380 & 0.402 & 0.371 & 0.400 & 0.380 & 0.400 & 0.374 & \textbf{0.390} & 0.388 & 0.400 & 0.877 & 0.656 & 0.402 & 0.414 \\ 
336&\textbf{0.337}& \textbf{0.385}& \underline{0.374} & \underline{0.414} & 0.391 & 0.420 & 0.402 & 0.449 & 0.428 & 0.432 & 0.415 & 0.426 & 0.426 & 0.433 & 1.043 & 0.731 & 0.452 & 0.452 \\ 
720&\textbf{0.395}&\textbf{0.430} & \underline{0.419} & 0.447 & 0.437 & 0.455 & 0.425 & \underline{0.438} & 0.427 & 0.445 & 0.420 & 0.440 & 0.431 & 0.446 & 1.104 & 0.763 & 0.462 & 0.468  \\ \midrule
\textbf{ETTm1} & & & & & & & & & & & & & & & &  \\ 
96&\textbf{0.289}& \textbf{0.343}& \underline{0.298} & \underline{0.355} & 0.311 & 0.380 & 0.312 & 0.371 & 0.334 & 0.368 & 0.355 & 0.376 & 0.329 & 0.367 & 0.404 & 0.426 & 0.338 & 0.375 \\ 
192&\textbf{0.338}&\textbf{0.371} & \underline{0.347} & \underline{0.381} & 0.389 & 0.419 & 0.365 & 0.409 & 0.377 & 0.391 & 0.391 & 0.392 & 0.367 & 0.385 & 0.450 & 0.451 & 0.374 & 0.387  \\ 
336&\textbf{0.370}&\textbf{0.390} & \underline{0.374} & \underline{0.397} & 0.401 & 0.417 & 0.421 & 0.410 & 0.426 & 0.420 & 0.424 & 0.415 & 0.399 & 0.410 & 0.532 & 0.515 & 0.410 & 0.411 \\ 
720 &\textbf{0.423}&\textbf{0.418}& \underline{0.429} & \underline{0.428} & 0.488 & 0.476 & 0.496 & 0.437 & 0.491 & 0.459 & 0.487 & 0.450 & 0.454 & 0.439 & 0.666 & 0.589 & 0.478 & 0.450  \\ \midrule
\textbf{ETTm2} & & & & & & & & & & & & & & & & \\ 
96&\textbf{0.168}& \underline{0.260}& 0.176 & 0.264 & 0.191 & 0.301 & 0.185 & 0.290 & 0.180 & 0.264 & 0.182 & 0.265 & \underline{0.175} & \textbf{0.259} & 0.287 & 0.366 & 0.187 & 0.267  \\ 
192&\underline{0.235}&\textbf{0.298} & \textbf{0.231} & 0.303 & 0.253 & 0.312 & 0.292 & 0.309 & 0.250 & 0.309 & 0.246 & 0.304 & 0.241 & \underline{0.302} & 0.414 & 0.492 & 0.249 & 0.309 \\ 
336&\textbf{0.274}&\textbf{0.327} & \underline{0.290} & \underline{0.339} & 0.298 & 0.342 & 0.321 & 0.367 & 0.311 & 0.348 & 0.307 & 0.342 & 0.305 & 0.343 & 0.597 & 0.542 & 0.321 & 0.351  \\ 
720&\textbf{0.376}&\textbf{0.393}& \underline{0.388} & \underline{0.398} & 0.409 & 0.407 & 0.401 & 0.400 & 0.412 & 0.407 & 0.407 & 0.398 & 0.402 & 0.400 & 1.730 & 1.042 & 0.408 & 0.403  \\ \midrule
\textbf{Exchange} & & & & & & & & & & & & & & & & \\ 
96&\textbf{0.085} &\underline{0.205} &0.097 &0.222 &\underline{0.086}&0.206 & 0.089&0.208 &0.091 &0.211 & 0.088&0.209 &0.087 &\textbf{0.202} &0.095 &0.218 &0.093 & 0.211\\ 
192&\textbf{0.179} & \textbf{0.303}& 0.191&0.315 &0.182 &0.304 &0.184 &0.309 &0.182 & 0.303&0.188 & 0.311&\underline{0.180} &0.305 &0.193 &0.318 &0.194 &0.315  \\ 
336& \textbf{0.311}&\textbf{0.402} & 0.337&0.424 &0.330&0.416&0.333&0.416 &0.337 & 0.421&0.346 & 0.423& \underline{0.318}& \underline{0.407}&0.359 &0.429 &0.358 &0.433 \\ 
720& \textbf{0.858}&\textbf{0.696}&0.877 & 0.706&0.865 & 0.702& 0.870&\underline{0.701}&\underline{0.862} &0.703 & 0.913& 0.717&0.863 &0.703 &0.918 &0.721 &0.880 &0.719 \\\midrule
\textbf{Weather} & & & & & & & & & & & & & & & &  \\ 
96&\textbf{0.150}& \textbf{0.199}& \underline{0.153} & \underline{0.205} & 0.169 & 0.221 & 0.174 & 0.218 & 0.174 & 0.214 & 0.192 & 0.232 & 0.177 & 0.218 & 0.158 & 0.230 & 0.172 & 0.220 \\ 
192&\textbf{0.194}&\textbf{0.242} & \underline{0.196} & \underline{0.244} & 0.205 & 0.248 & 0.200 & 0.258 & 0.221 & 0.254 & 0.240 & 0.271 & 0.225 & 0.259 & 0.206 & 0.277 & 0.219 & 0.261  \\ 
336&\underline{0.252}&\underline{0.288} & \textbf{0.246} & \textbf{0.283} & 0.288 & 0.299 & 0.280 & 0.299 & 0.278 & 0.296 & 0.292 & 0.307 & 0.278 & 0.297 & 0.272 & 0.335 & 0.280 & 0.306  \\ 
720&\textbf{0.313}&\textbf{0.333} & \underline{0.314} & \underline{0.334} & 0.335 & 0.369 & 0.352 & 0.359 & 0.358 & 0.347 & 0.364 & 0.353 & 0.354 & 0.348 & 0.398 & 0.418 & 0.365 & 0.359 \\ \midrule
\textbf{ECL} & & & & & & & & & & & & & & & &  \\ 
96&\textbf{0.134}& \textbf{0.231}& \underline{0.141} & \underline{0.239} & 0.157 & 0.255 & 0.156 & 0.240 & 0.148 & 0.240 & 0.201 & 0.281 & 0.181 & 0.270 & 0.219 & 0.314 & 0.168 & 0.272 \\ 
192&\textbf{0.153}& \textbf{0.248}& \underline{0.159} & \underline{0.255} & 0.188 & 0.271 & 0.161 & 0.268 & 0.162 & 0.253 & 0.201 & 0.283 & 0.188 & 0.274 & 0.231 & 0.322 & 0.184 & 0.289  \\ 
336&\textbf{0.170}&\textbf{0.266} & \underline{0.171} & \underline{0.268} & 0.192 & 0.275 & 0.195 & 0.272 & 0.178 & 0.269 & 0.215 & 0.298 & 0.204 & 0.293 & 0.246 & 0.337 & 0.198 & 0.300  \\ 
720&\textbf{0.201}&\textbf{0.293} & \underline{0.208} & \underline{0.300} & 0.241 & 0.339 & 0.231 & 0.307 & 0.225 & 0.317 & 0.257 & 0.331 & 0.246 & 0.324 & 0.280 & 0.363 & 0.220 & 0.320 \\ %\midrule
% Average& 0.304&0.349&0.315&
 \bottomrule
% Continue for Weather, ECL, and Traffic
\end{tabular}
}
\caption{Comprehensive comparison across various datasets, prediction horizons, and baselines. The \textbf{bolded} results denote the best performance, and the \underline{underlined} results indicate the second best.}
\label{tab:performance}
\end{table*}
\subsection{Cross-Variate Global Context}
\label{sec:CVGlobal}
%When analyzing multivariate time-series data, the human brain naturally identifies and compares global patterns across variables first, storing this information to help focus on local details later. 
%\textcolor{red}{Need a complete overhaul}
The global patches $\mathbf{\tilde{X}}_g$ is first passed through the global feature extractor, which is a dual Mamba system, responsible for cross-variate global feature extraction. %This inspires our approach, which
We begin by concatenating along the first and second dimension of $\mathbf{\tilde{X}}_g$, viewed with a new shape $\mathbf{\tilde{X}}_g\in \mathbb{R}^{(D* PN_g)\times PL_g}$ as illustrated in Figure \ref{fig:patch}. This allows the learning of variate-level correlation across all $D$ dimensions as the selection mechanism of Mamba will filter the relevant variates and patches, enabling the global model to capture cross-variate global context. However, Mamba processes data unilaterally, attending only to antecedent patches, which limits learning of the full global context. Inspired by S-Mamba \citep{wang2025mamba}, we employ two Mamba models to scan the sequence in both forward and backward directions before aggregating the results. This approach improves the learning of correlations between global patches across variables. Specifically, we have 
\begin{align}
    \overrightarrow {\mathbf{Z}_g} &= \overrightarrow{\text{Mamba Layers}} (\mathbf{\tilde{X}}_g),\\
    \overleftarrow {\mathbf{Z}_g} &= \overleftarrow{\text{Mamba Layers}} (\overleftarrow{\mathbf{\tilde{X}}}_g),\\
    \mathbf{Z}_g &= \overrightarrow {\mathbf{Z}_g}+\overleftarrow {\mathbf{Z}_g},
\end{align}
where $\overleftarrow{\mathbf{\tilde{X}}}_g \in \mathbb{R}^{(D*PN_g)\times PL_g}$ is obtained by reversing the the first dimension of $\mathbf{\tilde{X}}_g$. The output of the global model  $\mathbf{Z}_g\in \mathbb{R}^{(D*PN_g)\times d_\text{model}}$, which serves as an intermediary output of the entire architecture, is then fed to the local model. This intermediary output encapsulates both cross-variate and global context information. %patches from all variables to be combined into a single sequence, as illustrated in the figure. In order to facilitate communication across variables, we have prolonged the sequence length of the patched time series by factor of $D$. 

%However, the computational load of global attention escalates exponentially with the increased sequence length, rendering transformer-based models impractical for large dimensional data. On the other hand, the selective mechanism of Mamba can discern the significance of each patch akin to attention (cite) but with a computational overhead only escalating in a near-linear fashion. Therefore, employing Mamba as the global model and scanning the sequence of patches for all variables in one sweep becomes efficient. 

Note that $d_\text{model}$ represents the model dimension of Mamba, which aligns with the model dimension of the local model discussed in the next section.


\begin{figure*}[t]
    \centering
    % Subplot 1
    \begin{subfigure}[t]{0.24\textwidth} % 24% of the width
        \centering
        \includegraphics[width=\textwidth]{./images/S2TX_line.png} % Replace with your image
        % \caption{Sine Function}
        \label{fig:sine}
    \end{subfigure}
    % Subplot 2
    \begin{subfigure}[t]{0.24\textwidth}
        \centering
        \includegraphics[width=\textwidth]{./images/SST_line.png} % Replace with your image
        % \caption{Cosine Function}
        \label{fig:cosine}
    \end{subfigure}
    % Subplot 3
    \begin{subfigure}[t]{0.24\textwidth}
        \centering
        \includegraphics[width=\textwidth]{./images/iTransformer_line.png} % Replace with your image
        % \caption{Tangent Function}
        \label{fig:tangent}
    \end{subfigure}
    % Subplot 4
    \begin{subfigure}[t]{0.24\textwidth}
        \centering
        \includegraphics[width=\textwidth]{./images/iMamba_line.png} % Replace with your image
        % \caption{Exponential Decay}
        \label{fig:exponential}
    \end{subfigure}
    \caption{Empirical time series versus predicted time series across different architecture. S2TX can better capture the variation of the variable over time. }
    \label{fig:predictTScompare}
\end{figure*}
\subsection{Cross-Attention Local Context}
\label{sec:CALocal}
%\HM{Notation in this subsection is too cluttered, need to somehow simplify.}
With global and cross-variate patterns as context information, the local model can more effectively capture local features and interpret local variations. To this end, we employ a decoder-like transformer with each layer composed of a self-attention without causal masking followed by a cross-attention. Since cross-variate correlation is already captured by the context features, we now take each variate (in the first dimension) of $\mathbf{\tilde{X}}_l\in \mathbb{R}^{D\times PN_l\times PL_l}$ individually as the input of the self-attention to relieve the computation burden of transformer. Denote the $d$-th variate of $\mathbf{\tilde{X}}_l$ after linear projection to $d_{\text{model}}$-dimension by $\mathbf{\tilde{X}}_l^d\in \mathbb{R}^{PN_l\times d_{\text{model}}}$. Similarly, the context feature $\mathbf{Z}_g\in \mathbb{R}^{(D*PN_g)\times d_\text{model}}$ is viewed back to $\mathbf{Z}_g\in \mathbb{R}^{D\times PN_g\times d_\text{model}}$ and the $d$-th variate $\mathbf{Z}_g^d\in \mathbb{R}^{PN_g\times d_\text{model}}$ is sent to the cross-attention as key and value to match the dimension of $\mathbf{\tilde{X}}_l^d$. Specifically, the cross-attention mechanism operates as follows:
\begin{align}
    &\text{AttentionBlock}(\mathbf{\tilde{X}}_l^d, \mathbf{Z}_g^d) \nonumber \\
    &= \text{CrossAttention}(\mathbf{Z}_g^d, \;\text{SelfAttention}(\mathbf{\tilde{X}}_l^d)),
\end{align}
Note that we have omitted the skip connection and normalization steps for a concise presentation. The rest of the local model is the same as a regular transformer decoder as shown in the figure. 

Denote the output of the local model of the $d$th variable to be $\mathbf{Y}_\text{out}^d\in \mathbb{R}^{PN_l\times d_{\text{model}}}$. Stacking the outputs of all variables, we obtain $\mathbf{Y}_\text{out}\in \mathbb{R}^{D\times PN_l\times d_\text{model}}$. The last two dimensions of $\mathbf{Y}_\text{out}$ are then flattened and a final linear head is employed to project from dimension $PN_l\times d_\text{model}$ to $H$, which is the target horizon window. 

\subsection{Runtime Complexity Analysis}
\label{sec:runtime}
% The runtime complexity is primarily influenced by the patch number $PN$, rather than the look-back window length $L$. The Mamba layers, which exhibit linear complexity, process a sequence of length $D \cdot PN$ in a runtime complexity of $O(D \cdot PN)$. In contrast, the Transformer has quadratic complexity and processes $D$ sequences of length $PN$ each, leading to a runtime complexity of $O(D \cdot PN^2)$. Consequently, the overall complexity of the model is $O(D \cdot PN^2)$ with respect to $PN$. To manage complexity as the look-back window length increases, it is reasonable to proportionally increase both the patch length and stride such that $PN$ remains constant. So, when the stride is chosen to be proportional to the length $L$, the overall complexity of S2TX is constant.
The Mamba layers, which exhibit linear complexity, process a sequence of length $D \cdot PN_g$ with a complexity of $O(D \cdot PN_g)$, which is linear to input time series length $L$ due to definition of $PN$ in Equation~\eqref{eq:pn}. On the other hand, while transformer models exhibit quadratic complexity with respect to sequence length, S2TX uses a local look-back window with fixed length $S$, resulting in a complexity of $O(D\cdot PN_l^2) = O(D)$ as $PN_l=O(S)=O(1)$.  Thus, S2TX has an overall linear complexity with respect to $L$ and $D$. Moreover, as $L$ increases, we can proportionally increase both the patch length and stride so that $PN_g$ remains constant, in which case, the overall complexity of S2TX reduces to constant order with respect to $L$ while remaining linear order with respect to $D$. Our empirical results in Section~\ref{sec:runtime} verifies this, showing that S2TX's runtime barely increases with $L$.
% To manage this complexity as the look-back window length $L$ increases, it is reasonable to proportionally increase both the patch length and stride so that $PN$ remains constant, in which case, the overall complexity of S2TX remains constant. Otherwise, if the stride remains fixed for a prolonged look-back window, the overall complexity increases linearly with $L$.

\section{Experiment}
\label{sec:experiment}
We empirically demonstrate that utilizing cross-variate correlation and global-local interaction can significantly improve the forecasting performance. We first introduce the experimental setup, then we showcase the performance of S2TX over a variety of benchmark against recent state-of-the-art architectures. We then demonstrate the efficacy of the main component of S2TX with a set of ablation study and a robustness study where we test the robustness of S2TX with sequences of missing values. Finally, we showcase the low memory footprint and efficient runtime of S2TX compared to Transformer and Mamba in general.

\textbf{Dataset.} 
We benchmark our proposed algorithm S2TX on a set of 7 real-world multivariate time series datasets, including the four Electricity Transformer Temperature datasets ETTh1, ETTh2, ETTm1, and ETTm2, Weather, Electricity, and Exchange rate datasets. Detailed dataset descriptions are provided in the Appendix \ref{appendix:data_desc}.

\textbf{Baselines.} 
We benchmark our proposed algorithm S2TX against most competitive time series forecasting models within three years, including MOE-based model SST \citep{xusst}, Mamba-based models S-Mamba \citep{wang2025mamba} and TimeMachine (TimeM) \citep{ahamed2024timemachine}, transformer-based models iTransformer (iTrans) \citep{liu2023itransformer}, PatchTST \citep{nie2022time}, Crossformer (CrossF) \citep{zhang2023crossformer}, and FEDformer \citep{zhou2022fedformer}, linear-based models RLinear \citep{li2023revisiting} and DLinear \citep{zeng2023transformers}, and TCN-based model TimesNet \citep{wu2022timesnet}. Due to space constraints, the comparisons against DLinear and FEDformer (pre-2023 models) are presented in Appendix \ref{appendix:full_comparison} 

\textbf{Experimental Setting and Metrics.}
For a fair comparison, the experimental setting of all baselines follows the experiment setup of the current SOTA SST. In addition, we use the same hyperparameters as in SST, including global and local patch length, stride, and look-back window. Specifically, we set $PL_g = 48$, $STR_g = 16$, $PL_l = 16$, $STR_l = 8$, and $L = 2S = 336$. For Exchange rate dataset, we use a smaller patch length, stride, and look-back window: $PL_g = 16$, $STR_g = 8$, $PL_l = 4$, $STR_l = 2$, $L = 2S = 192$. The forecast horizon is set to $\{96,192,336,720\}$ for each dataset. We use mean squared error and mean absolute error as metrics to compare performances of different architectures. 

We now present the numerical result of our comprehensive experiments, as well as an ablation study to showcase the importance of each module, and a computation efficiency study comparing canonical architectures, SST, and S2TX. 

\subsection{Benchmark Results}
\label{sec:result}
The performance of 9 different architectures on 7 benchmark datasets and 4 different prediction horizons is presented in Table \ref{tab:performance}. \textbf{Our method S2TX achieves SOTA performance across all benchmark datasets}. In particular, compared to the previous SOTA model SST, S2TX demonstrates consistent improvements on most datasets and performs on par on the weather dataset. For instance, S2TX achieves an $8.4\%$ improvement on the ETTh1 dataset with a prediction horizon of $720$. Moreover, even on the weather dataset, S2TX significantly surpasses other baseline models. The SOTA performance, together with our ablation studies in section \ref{sec:ablation}, suggests that the two novel aspect of S2TX, the cross-variate global features, and the cross-attentional local features, are indeed important for accurately forecasting multivariate time series. 

Guided by cross-variate global context, S2TX demonstrates a superior ability to capture local variations. Figure \ref{fig:predictTScompare} presents a random segment of test time prediction from the electricity dataset on a randomly selected variate, comparing the performance of S2TX, SST, iTransformer, and S-Mamba. S2TX precisely approximates abrupt spikes while the accurate predictions of local variation are less apparent in predictions of other models.
%\paragraph{Qualitative Results}

%To intuitively illustrate the improvements of S2TX, we present a visual graph showcasing a random prediction segment from the electricity dataset on a randomly selected variable, comparing the performance of S2TX, SST, iTransformer, and iMamba. Guided by cross-dimensional global context, S2TX demonstrates a superior ability to capture local variations, as evidenced by its precise approximation of abrupt spikes—an ability less apparent in the other models.


\subsection{Ablation and Robustness Studies}
\label{sec:ablation}
\begin{figure}[t]
    \centering
    \includegraphics[width=0.45\textwidth]{./images/ablation_components.png} % Change to your image file
    \caption{Ablation study on different components of S2TX tested on ETTh1 and ETTm1 datasets. The efficacy of each component of the proposed architectures is measured by the degradation of performance after each (or both) component(s) was excluded. }
    \label{fig:ablation}
\end{figure}
\textbf{Ablation on Model Components.}
We perform ablation studies by removing key components of S2TX. To first assess the impact of cross-variate communication in learning global context, we input the patch sequence of each variate separately into the global model, rather than using the concatenated cross-variate patch sequence. Second, to evaluate the effectiveness of the context-local cross-attention mechanism, we remove cross-attention and instead concatenate the global context and local features before the final linear head. Finally, we remove both mechanisms to evaluate their combined effect. We conduct ablation studies on the ETTh1 and ETTm1 datasets and report the MSE metric, averaged across four different prediction lengths. As shown in Figure \ref{fig:ablation}, the global-local cross-attention contributes the most to the overall improvement of S2TX, while variable communication also positively influences the results.

\textbf{Robustness to Missing Values.}
In real-world multivariate time series datasets, it is common to observe missing values. Unlike traditional tabular data where a few elements are missing, missing values in time series could exist for small periods of sequences. In this set of robustness experiments, we randomly select small sequences of 4 time steps to be missing and interpolate these randomly missing periods with the value of the last observed time step. In Table \ref{tab:robustness}, we present the MSE of different architectures under various percentage of missing values. We show that S2TX, with the addition of cross-variate global context and the cross-attentional global-local feature interplay, is highly robust compared to SST, which showed much-worsened degradation as the percentage of missing value increases. 
\begin{table}[t]
    \centering
    \footnotesize
    \renewcommand{\arraystretch}{1.1}
    \setlength{\tabcolsep}{8pt}
    \begin{tabular}{lcc}  % l = left, c = center, r = right
        \toprule
        \textbf{Miss Ratio} & \textbf{S2TX}& \textbf{SST}\\
        \midrule
        0\%         &0.421(-0.0\%)  &0.439(-0.0\%)\\
        4\%         &0.424(-0.7\%)  & 0.440(-0.2\%) \\
        8\%         & 0.425(-0.9\%) & 0.443(-0.9\%) \\
        16\%        & 0.424(-0.7\%) & 0.450(-2.5\%) \\
        24\%        & 0.429(-1.9\%) &0.468(-6.6\%) \\
        32\%        &0.431(-2.3\%)  & 0.471(-7.0\%)\\
        40\%        & 0.441(-4.7\%) & 0.499(-13.4\%)\\
        \bottomrule
    \end{tabular}
    \caption{Performance on ETTh1 with increasing proportion of missing values; results are MSE averaged over all four prediction horizons.}
    \label{tab:robustness}
\end{table}

\subsection{Memory and Runtime Analysis}\label{sec:runtime}
To ensure a fair runtime comparison, we evaluate S2TX alongside SST, the vanilla Transformer, and Mamba on a single NVIDIA RTX 6000 Ada Generation GPU. The two versions of S2TX and SST correspond to configurations with either a fixed patch number or a fixed stride length. In the former case, the patch number remains constant regardless of sequence length, whereas in the latter case, the patch number increases proportionally with sequence length.

It is important to note that we compare against the vanilla Transformer and Mamba, rather than their inverted versions, as the respective attention and selective mechanisms in the inverted versions operate on the variate dimension. As the look-back window sequence length increases, the memory usage and runtime of the Transformer grow exponentially, reaching the GPU's memory limit when the sequence length hits 2000. In contrast, Mamba scales linearly in both memory and time metrics.

Both S2TX and SST scale more efficiently than Mamba, owing to the fixed short local look-back window combined with the patching technique, which effectively reduces the sequence length by a factor of the stride length. The complexity experiment result is presented in Figure \ref{fig:ablation_study}. Consistent to the runtime analysis in section~\ref{sec:runtime}, when the global patch number is fixed, both S2TX and SST achieve nearly constant runtime complexity. However, when comparing S2TX to SST, SST scales slightly better due to the additional cross-attention mechanism in S2TX.

\begin{figure}[t]
    \centering
    % Subfigure 1
    \begin{subfigure}[]{0.6\columnwidth} % 45% of the width for the first subfigure
        \centering
        \includegraphics[width=\linewidth]{./images/memory_complexity.png} % Replace with your image file
        \caption{}%Memory usage comparison between S2TX, SST, vanilla transformer and Mamba.}
        \label{fig:subfigure1}
    \end{subfigure}
    \quad % Adds horizontal spacing between the two subfigures
    % Subfigure 2
    \begin{subfigure}[]{0.6\columnwidth} % 45% of the width for the second subfigure
        \centering
        \includegraphics[width=\linewidth]{./images/time_complexity.png} % Replace with your image file
        \caption{}%Time complexity comparison between S2TX, SST, vanilla transformer and Mamba.}
        \label{fig:subfigure2}
    \end{subfigure}
    % Main caption for the figure
    \caption{Memory and run-time comparison between S2TX and other canonical architectures.}
    \label{fig:ablation_study}
\end{figure}

\section{Discussion and Future Work}
In this work, we introduce a new architecture, \emph{State-Space Transformer with cross-attention} (S2TX), for multivariate time series modeling. We first noted that the multi-scale patching methods, although enhance the learning of temporal dependencies, neglect the cross-variate correlation--a crucial aspect of multivariate time series modeling. Also, global and local patches are processed independently, overlooking the global and local interactions that occur in many real-world scenarios. We propose a novel cross-attention based architecture that integrates state space models and transformers. This cross-attention architecture, combined with patchification, fully leverages the strengths of Mamba and transformers by integrating cross-variate global features from Mamba with the local features of the transformer. Our architecture generally improves over current state-of-the-art in various datasets and 4 different prediction horizons. The SOTA performance of S2TX is not only achieved with a low memory footprint and fast computation runtime but also demonstrated robust performance when facing time series with sequences of missing values. Given these advantages, S2TX unlocks new possibilities for time series forecasting by effectively capturing cross-variate correlations and global-local feature interactions.

\textbf{Limitations.}
Several limitations exist in our current architectures. One key limitation is that cross-variate correlations are not explicitly explored at a local level. Although S2TX maintains low memory usage and fast runtime, incorporating local cross-variate correlations could further enhance performance. Another limitation is the lack of diversity in the multi-scale approach. The current architecture only deals with global and local patches with no learning of the intermediates scales. Intermediate time scales, however, could be important for extremely long sequences where the difference in time scales between global and local contexts is dramatic. Incorporating multiple time scales within an architecture while remaining lightweight is still unsolved. We leave these for future works. 

\paragraph{Impact Statement}
This paper presents work whose goal is to advance the field of Machine Learning. There are many potential societal consequences of our work, none of which we feel must be specifically highlighted here.


% In the unusual situation where you want a paper to appear in the
% references without citing it in the main text, use \nocite
%\nocite{langley00}
% \bibliography{main}
\bibliographystyle{icml2025}
% This must be in the first 5 lines to tell arXiv to use pdfLaTeX, which is strongly recommended.
\pdfoutput=1
% In particular, the hyperref package requires pdfLaTeX in order to break URLs across lines.

\documentclass[11pt]{article}

% Change "review" to "final" to generate the final (sometimes called camera-ready) version.
% Change to "preprint" to generate a non-anonymous version with page numbers.
\usepackage{acl}

% Standard package includes
\usepackage{times}
\usepackage{latexsym}

% Draw tables
\usepackage{booktabs}
\usepackage{multirow}
\usepackage{xcolor}
\usepackage{colortbl}
\usepackage{array} 
\usepackage{amsmath}

\newcolumntype{C}{>{\centering\arraybackslash}p{0.07\textwidth}}
% For proper rendering and hyphenation of words containing Latin characters (including in bib files)
\usepackage[T1]{fontenc}
% For Vietnamese characters
% \usepackage[T5]{fontenc}
% See https://www.latex-project.org/help/documentation/encguide.pdf for other character sets
% This assumes your files are encoded as UTF8
\usepackage[utf8]{inputenc}

% This is not strictly necessary, and may be commented out,
% but it will improve the layout of the manuscript,
% and will typically save some space.
\usepackage{microtype}
\DeclareMathOperator*{\argmax}{arg\,max}
% This is also not strictly necessary, and may be commented out.
% However, it will improve the aesthetics of text in
% the typewriter font.
\usepackage{inconsolata}

%Including images in your LaTeX document requires adding
%additional package(s)
\usepackage{graphicx}
% If the title and author information does not fit in the area allocated, uncomment the following
%
%\setlength\titlebox{<dim>}
%
% and set <dim> to something 5cm or larger.

\title{Wi-Chat: Large Language Model Powered Wi-Fi Sensing}

% Author information can be set in various styles:
% For several authors from the same institution:
% \author{Author 1 \and ... \and Author n \\
%         Address line \\ ... \\ Address line}
% if the names do not fit well on one line use
%         Author 1 \\ {\bf Author 2} \\ ... \\ {\bf Author n} \\
% For authors from different institutions:
% \author{Author 1 \\ Address line \\  ... \\ Address line
%         \And  ... \And
%         Author n \\ Address line \\ ... \\ Address line}
% To start a separate ``row'' of authors use \AND, as in
% \author{Author 1 \\ Address line \\  ... \\ Address line
%         \AND
%         Author 2 \\ Address line \\ ... \\ Address line \And
%         Author 3 \\ Address line \\ ... \\ Address line}

% \author{First Author \\
%   Affiliation / Address line 1 \\
%   Affiliation / Address line 2 \\
%   Affiliation / Address line 3 \\
%   \texttt{email@domain} \\\And
%   Second Author \\
%   Affiliation / Address line 1 \\
%   Affiliation / Address line 2 \\
%   Affiliation / Address line 3 \\
%   \texttt{email@domain} \\}
% \author{Haohan Yuan \qquad Haopeng Zhang\thanks{corresponding author} \\ 
%   ALOHA Lab, University of Hawaii at Manoa \\
%   % Affiliation / Address line 2 \\
%   % Affiliation / Address line 3 \\
%   \texttt{\{haohany,haopengz\}@hawaii.edu}}
  
\author{
{Haopeng Zhang$\dag$\thanks{These authors contributed equally to this work.}, Yili Ren$\ddagger$\footnotemark[1], Haohan Yuan$\dag$, Jingzhe Zhang$\ddagger$, Yitong Shen$\ddagger$} \\
ALOHA Lab, University of Hawaii at Manoa$\dag$, University of South Florida$\ddagger$ \\
\{haopengz, haohany\}@hawaii.edu\\
\{yiliren, jingzhe, shen202\}@usf.edu\\}



  
%\author{
%  \textbf{First Author\textsuperscript{1}},
%  \textbf{Second Author\textsuperscript{1,2}},
%  \textbf{Third T. Author\textsuperscript{1}},
%  \textbf{Fourth Author\textsuperscript{1}},
%\\
%  \textbf{Fifth Author\textsuperscript{1,2}},
%  \textbf{Sixth Author\textsuperscript{1}},
%  \textbf{Seventh Author\textsuperscript{1}},
%  \textbf{Eighth Author \textsuperscript{1,2,3,4}},
%\\
%  \textbf{Ninth Author\textsuperscript{1}},
%  \textbf{Tenth Author\textsuperscript{1}},
%  \textbf{Eleventh E. Author\textsuperscript{1,2,3,4,5}},
%  \textbf{Twelfth Author\textsuperscript{1}},
%\\
%  \textbf{Thirteenth Author\textsuperscript{3}},
%  \textbf{Fourteenth F. Author\textsuperscript{2,4}},
%  \textbf{Fifteenth Author\textsuperscript{1}},
%  \textbf{Sixteenth Author\textsuperscript{1}},
%\\
%  \textbf{Seventeenth S. Author\textsuperscript{4,5}},
%  \textbf{Eighteenth Author\textsuperscript{3,4}},
%  \textbf{Nineteenth N. Author\textsuperscript{2,5}},
%  \textbf{Twentieth Author\textsuperscript{1}}
%\\
%\\
%  \textsuperscript{1}Affiliation 1,
%  \textsuperscript{2}Affiliation 2,
%  \textsuperscript{3}Affiliation 3,
%  \textsuperscript{4}Affiliation 4,
%  \textsuperscript{5}Affiliation 5
%\\
%  \small{
%    \textbf{Correspondence:} \href{mailto:email@domain}{email@domain}
%  }
%}

\begin{document}
\maketitle
\begin{abstract}
Recent advancements in Large Language Models (LLMs) have demonstrated remarkable capabilities across diverse tasks. However, their potential to integrate physical model knowledge for real-world signal interpretation remains largely unexplored. In this work, we introduce Wi-Chat, the first LLM-powered Wi-Fi-based human activity recognition system. We demonstrate that LLMs can process raw Wi-Fi signals and infer human activities by incorporating Wi-Fi sensing principles into prompts. Our approach leverages physical model insights to guide LLMs in interpreting Channel State Information (CSI) data without traditional signal processing techniques. Through experiments on real-world Wi-Fi datasets, we show that LLMs exhibit strong reasoning capabilities, achieving zero-shot activity recognition. These findings highlight a new paradigm for Wi-Fi sensing, expanding LLM applications beyond conventional language tasks and enhancing the accessibility of wireless sensing for real-world deployments.
\end{abstract}

\section{Introduction}

In today’s rapidly evolving digital landscape, the transformative power of web technologies has redefined not only how services are delivered but also how complex tasks are approached. Web-based systems have become increasingly prevalent in risk control across various domains. This widespread adoption is due their accessibility, scalability, and ability to remotely connect various types of users. For example, these systems are used for process safety management in industry~\cite{kannan2016web}, safety risk early warning in urban construction~\cite{ding2013development}, and safe monitoring of infrastructural systems~\cite{repetto2018web}. Within these web-based risk management systems, the source search problem presents a huge challenge. Source search refers to the task of identifying the origin of a risky event, such as a gas leak and the emission point of toxic substances. This source search capability is crucial for effective risk management and decision-making.

Traditional approaches to implementing source search capabilities into the web systems often rely on solely algorithmic solutions~\cite{ristic2016study}. These methods, while relatively straightforward to implement, often struggle to achieve acceptable performances due to algorithmic local optima and complex unknown environments~\cite{zhao2020searching}. More recently, web crowdsourcing has emerged as a promising alternative for tackling the source search problem by incorporating human efforts in these web systems on-the-fly~\cite{zhao2024user}. This approach outsources the task of addressing issues encountered during the source search process to human workers, leveraging their capabilities to enhance system performance.

These solutions often employ a human-AI collaborative way~\cite{zhao2023leveraging} where algorithms handle exploration-exploitation and report the encountered problems while human workers resolve complex decision-making bottlenecks to help the algorithms getting rid of local deadlocks~\cite{zhao2022crowd}. Although effective, this paradigm suffers from two inherent limitations: increased operational costs from continuous human intervention, and slow response times of human workers due to sequential decision-making. These challenges motivate our investigation into developing autonomous systems that preserve human-like reasoning capabilities while reducing dependency on massive crowdsourced labor.

Furthermore, recent advancements in large language models (LLMs)~\cite{chang2024survey} and multi-modal LLMs (MLLMs)~\cite{huang2023chatgpt} have unveiled promising avenues for addressing these challenges. One clear opportunity involves the seamless integration of visual understanding and linguistic reasoning for robust decision-making in search tasks. However, whether large models-assisted source search is really effective and efficient for improving the current source search algorithms~\cite{ji2022source} remains unknown. \textit{To address the research gap, we are particularly interested in answering the following two research questions in this work:}

\textbf{\textit{RQ1: }}How can source search capabilities be integrated into web-based systems to support decision-making in time-sensitive risk management scenarios? 
% \sq{I mention ``time-sensitive'' here because I feel like we shall say something about the response time -- LLM has to be faster than humans}

\textbf{\textit{RQ2: }}How can MLLMs and LLMs enhance the effectiveness and efficiency of existing source search algorithms? 

% \textit{\textbf{RQ2:}} To what extent does the performance of large models-assisted search align with or approach the effectiveness of human-AI collaborative search? 

To answer the research questions, we propose a novel framework called Auto-\
S$^2$earch (\textbf{Auto}nomous \textbf{S}ource \textbf{Search}) and implement a prototype system that leverages advanced web technologies to simulate real-world conditions for zero-shot source search. Unlike traditional methods that rely on pre-defined heuristics or extensive human intervention, AutoS$^2$earch employs a carefully designed prompt that encapsulates human rationales, thereby guiding the MLLM to generate coherent and accurate scene descriptions from visual inputs about four directional choices. Based on these language-based descriptions, the LLM is enabled to determine the optimal directional choice through chain-of-thought (CoT) reasoning. Comprehensive empirical validation demonstrates that AutoS$^2$-\ 
earch achieves a success rate of 95–98\%, closely approaching the performance of human-AI collaborative search across 20 benchmark scenarios~\cite{zhao2023leveraging}. 

Our work indicates that the role of humans in future web crowdsourcing tasks may evolve from executors to validators or supervisors. Furthermore, incorporating explanations of LLM decisions into web-based system interfaces has the potential to help humans enhance task performance in risk control.






\section{Related Work}
\label{sec:relatedworks}

% \begin{table*}[t]
% \centering 
% \renewcommand\arraystretch{0.98}
% \fontsize{8}{10}\selectfont \setlength{\tabcolsep}{0.4em}
% \begin{tabular}{@{}lc|cc|cc|cc@{}}
% \toprule
% \textbf{Methods}           & \begin{tabular}[c]{@{}c@{}}\textbf{Training}\\ \textbf{Paradigm}\end{tabular} & \begin{tabular}[c]{@{}c@{}}\textbf{$\#$ PT Data}\\ \textbf{(Tokens)}\end{tabular} & \begin{tabular}[c]{@{}c@{}}\textbf{$\#$ IFT Data}\\ \textbf{(Samples)}\end{tabular} & \textbf{Code}  & \begin{tabular}[c]{@{}c@{}}\textbf{Natural}\\ \textbf{Language}\end{tabular} & \begin{tabular}[c]{@{}c@{}}\textbf{Action}\\ \textbf{Trajectories}\end{tabular} & \begin{tabular}[c]{@{}c@{}}\textbf{API}\\ \textbf{Documentation}\end{tabular}\\ \midrule 
% NexusRaven~\citep{srinivasan2023nexusraven} & IFT & - & - & \textcolor{green}{\CheckmarkBold} & \textcolor{green}{\CheckmarkBold} &\textcolor{red}{\XSolidBrush}&\textcolor{red}{\XSolidBrush}\\
% AgentInstruct~\citep{zeng2023agenttuning} & IFT & - & 2k & \textcolor{green}{\CheckmarkBold} & \textcolor{green}{\CheckmarkBold} &\textcolor{red}{\XSolidBrush}&\textcolor{red}{\XSolidBrush} \\
% AgentEvol~\citep{xi2024agentgym} & IFT & - & 14.5k & \textcolor{green}{\CheckmarkBold} & \textcolor{green}{\CheckmarkBold} &\textcolor{green}{\CheckmarkBold}&\textcolor{red}{\XSolidBrush} \\
% Gorilla~\citep{patil2023gorilla}& IFT & - & 16k & \textcolor{green}{\CheckmarkBold} & \textcolor{green}{\CheckmarkBold} &\textcolor{red}{\XSolidBrush}&\textcolor{green}{\CheckmarkBold}\\
% OpenFunctions-v2~\citep{patil2023gorilla} & IFT & - & 65k & \textcolor{green}{\CheckmarkBold} & \textcolor{green}{\CheckmarkBold} &\textcolor{red}{\XSolidBrush}&\textcolor{green}{\CheckmarkBold}\\
% LAM~\citep{zhang2024agentohana} & IFT & - & 42.6k & \textcolor{green}{\CheckmarkBold} & \textcolor{green}{\CheckmarkBold} &\textcolor{green}{\CheckmarkBold}&\textcolor{red}{\XSolidBrush} \\
% xLAM~\citep{liu2024apigen} & IFT & - & 60k & \textcolor{green}{\CheckmarkBold} & \textcolor{green}{\CheckmarkBold} &\textcolor{green}{\CheckmarkBold}&\textcolor{red}{\XSolidBrush} \\\midrule
% LEMUR~\citep{xu2024lemur} & PT & 90B & 300k & \textcolor{green}{\CheckmarkBold} & \textcolor{green}{\CheckmarkBold} &\textcolor{green}{\CheckmarkBold}&\textcolor{red}{\XSolidBrush}\\
% \rowcolor{teal!12} \method & PT & 103B & 95k & \textcolor{green}{\CheckmarkBold} & \textcolor{green}{\CheckmarkBold} & \textcolor{green}{\CheckmarkBold} & \textcolor{green}{\CheckmarkBold} \\
% \bottomrule
% \end{tabular}
% \caption{Summary of existing tuning- and pretraining-based LLM agents with their training sample sizes. "PT" and "IFT" denote "Pre-Training" and "Instruction Fine-Tuning", respectively. }
% \label{tab:related}
% \end{table*}

\begin{table*}[ht]
\begin{threeparttable}
\centering 
\renewcommand\arraystretch{0.98}
\fontsize{7}{9}\selectfont \setlength{\tabcolsep}{0.2em}
\begin{tabular}{@{}l|c|c|ccc|cc|cc|cccc@{}}
\toprule
\textbf{Methods} & \textbf{Datasets}           & \begin{tabular}[c]{@{}c@{}}\textbf{Training}\\ \textbf{Paradigm}\end{tabular} & \begin{tabular}[c]{@{}c@{}}\textbf{\# PT Data}\\ \textbf{(Tokens)}\end{tabular} & \begin{tabular}[c]{@{}c@{}}\textbf{\# IFT Data}\\ \textbf{(Samples)}\end{tabular} & \textbf{\# APIs} & \textbf{Code}  & \begin{tabular}[c]{@{}c@{}}\textbf{Nat.}\\ \textbf{Lang.}\end{tabular} & \begin{tabular}[c]{@{}c@{}}\textbf{Action}\\ \textbf{Traj.}\end{tabular} & \begin{tabular}[c]{@{}c@{}}\textbf{API}\\ \textbf{Doc.}\end{tabular} & \begin{tabular}[c]{@{}c@{}}\textbf{Func.}\\ \textbf{Call}\end{tabular} & \begin{tabular}[c]{@{}c@{}}\textbf{Multi.}\\ \textbf{Step}\end{tabular}  & \begin{tabular}[c]{@{}c@{}}\textbf{Plan}\\ \textbf{Refine}\end{tabular}  & \begin{tabular}[c]{@{}c@{}}\textbf{Multi.}\\ \textbf{Turn}\end{tabular}\\ \midrule 
\multicolumn{13}{l}{\emph{Instruction Finetuning-based LLM Agents for Intrinsic Reasoning}}  \\ \midrule
FireAct~\cite{chen2023fireact} & FireAct & IFT & - & 2.1K & 10 & \textcolor{red}{\XSolidBrush} &\textcolor{green}{\CheckmarkBold} &\textcolor{green}{\CheckmarkBold}  & \textcolor{red}{\XSolidBrush} &\textcolor{green}{\CheckmarkBold} & \textcolor{red}{\XSolidBrush} &\textcolor{green}{\CheckmarkBold} & \textcolor{red}{\XSolidBrush} \\
ToolAlpaca~\cite{tang2023toolalpaca} & ToolAlpaca & IFT & - & 4.0K & 400 & \textcolor{red}{\XSolidBrush} &\textcolor{green}{\CheckmarkBold} &\textcolor{green}{\CheckmarkBold} & \textcolor{red}{\XSolidBrush} &\textcolor{green}{\CheckmarkBold} & \textcolor{red}{\XSolidBrush}  &\textcolor{green}{\CheckmarkBold} & \textcolor{red}{\XSolidBrush}  \\
ToolLLaMA~\cite{qin2023toolllm} & ToolBench & IFT & - & 12.7K & 16,464 & \textcolor{red}{\XSolidBrush} &\textcolor{green}{\CheckmarkBold} &\textcolor{green}{\CheckmarkBold} &\textcolor{red}{\XSolidBrush} &\textcolor{green}{\CheckmarkBold}&\textcolor{green}{\CheckmarkBold}&\textcolor{green}{\CheckmarkBold} &\textcolor{green}{\CheckmarkBold}\\
AgentEvol~\citep{xi2024agentgym} & AgentTraj-L & IFT & - & 14.5K & 24 &\textcolor{red}{\XSolidBrush} & \textcolor{green}{\CheckmarkBold} &\textcolor{green}{\CheckmarkBold}&\textcolor{red}{\XSolidBrush} &\textcolor{green}{\CheckmarkBold}&\textcolor{red}{\XSolidBrush} &\textcolor{red}{\XSolidBrush} &\textcolor{green}{\CheckmarkBold}\\
Lumos~\cite{yin2024agent} & Lumos & IFT  & - & 20.0K & 16 &\textcolor{red}{\XSolidBrush} & \textcolor{green}{\CheckmarkBold} & \textcolor{green}{\CheckmarkBold} &\textcolor{red}{\XSolidBrush} & \textcolor{green}{\CheckmarkBold} & \textcolor{green}{\CheckmarkBold} &\textcolor{red}{\XSolidBrush} & \textcolor{green}{\CheckmarkBold}\\
Agent-FLAN~\cite{chen2024agent} & Agent-FLAN & IFT & - & 24.7K & 20 &\textcolor{red}{\XSolidBrush} & \textcolor{green}{\CheckmarkBold} & \textcolor{green}{\CheckmarkBold} &\textcolor{red}{\XSolidBrush} & \textcolor{green}{\CheckmarkBold}& \textcolor{green}{\CheckmarkBold}&\textcolor{red}{\XSolidBrush} & \textcolor{green}{\CheckmarkBold}\\
AgentTuning~\citep{zeng2023agenttuning} & AgentInstruct & IFT & - & 35.0K & - &\textcolor{red}{\XSolidBrush} & \textcolor{green}{\CheckmarkBold} & \textcolor{green}{\CheckmarkBold} &\textcolor{red}{\XSolidBrush} & \textcolor{green}{\CheckmarkBold} &\textcolor{red}{\XSolidBrush} &\textcolor{red}{\XSolidBrush} & \textcolor{green}{\CheckmarkBold}\\\midrule
\multicolumn{13}{l}{\emph{Instruction Finetuning-based LLM Agents for Function Calling}} \\\midrule
NexusRaven~\citep{srinivasan2023nexusraven} & NexusRaven & IFT & - & - & 116 & \textcolor{green}{\CheckmarkBold} & \textcolor{green}{\CheckmarkBold}  & \textcolor{green}{\CheckmarkBold} &\textcolor{red}{\XSolidBrush} & \textcolor{green}{\CheckmarkBold} &\textcolor{red}{\XSolidBrush} &\textcolor{red}{\XSolidBrush}&\textcolor{red}{\XSolidBrush}\\
Gorilla~\citep{patil2023gorilla} & Gorilla & IFT & - & 16.0K & 1,645 & \textcolor{green}{\CheckmarkBold} &\textcolor{red}{\XSolidBrush} &\textcolor{red}{\XSolidBrush}&\textcolor{green}{\CheckmarkBold} &\textcolor{green}{\CheckmarkBold} &\textcolor{red}{\XSolidBrush} &\textcolor{red}{\XSolidBrush} &\textcolor{red}{\XSolidBrush}\\
OpenFunctions-v2~\citep{patil2023gorilla} & OpenFunctions-v2 & IFT & - & 65.0K & - & \textcolor{green}{\CheckmarkBold} & \textcolor{green}{\CheckmarkBold} &\textcolor{red}{\XSolidBrush} &\textcolor{green}{\CheckmarkBold} &\textcolor{green}{\CheckmarkBold} &\textcolor{red}{\XSolidBrush} &\textcolor{red}{\XSolidBrush} &\textcolor{red}{\XSolidBrush}\\
API Pack~\cite{guo2024api} & API Pack & IFT & - & 1.1M & 11,213 &\textcolor{green}{\CheckmarkBold} &\textcolor{red}{\XSolidBrush} &\textcolor{green}{\CheckmarkBold} &\textcolor{red}{\XSolidBrush} &\textcolor{green}{\CheckmarkBold} &\textcolor{red}{\XSolidBrush}&\textcolor{red}{\XSolidBrush}&\textcolor{red}{\XSolidBrush}\\ 
LAM~\citep{zhang2024agentohana} & AgentOhana & IFT & - & 42.6K & - & \textcolor{green}{\CheckmarkBold} & \textcolor{green}{\CheckmarkBold} &\textcolor{green}{\CheckmarkBold}&\textcolor{red}{\XSolidBrush} &\textcolor{green}{\CheckmarkBold}&\textcolor{red}{\XSolidBrush}&\textcolor{green}{\CheckmarkBold}&\textcolor{green}{\CheckmarkBold}\\
xLAM~\citep{liu2024apigen} & APIGen & IFT & - & 60.0K & 3,673 & \textcolor{green}{\CheckmarkBold} & \textcolor{green}{\CheckmarkBold} &\textcolor{green}{\CheckmarkBold}&\textcolor{red}{\XSolidBrush} &\textcolor{green}{\CheckmarkBold}&\textcolor{red}{\XSolidBrush}&\textcolor{green}{\CheckmarkBold}&\textcolor{green}{\CheckmarkBold}\\\midrule
\multicolumn{13}{l}{\emph{Pretraining-based LLM Agents}}  \\\midrule
% LEMUR~\citep{xu2024lemur} & PT & 90B & 300.0K & - & \textcolor{green}{\CheckmarkBold} & \textcolor{green}{\CheckmarkBold} &\textcolor{green}{\CheckmarkBold}&\textcolor{red}{\XSolidBrush} & \textcolor{red}{\XSolidBrush} &\textcolor{green}{\CheckmarkBold} &\textcolor{red}{\XSolidBrush}&\textcolor{red}{\XSolidBrush}\\
\rowcolor{teal!12} \method & \dataset & PT & 103B & 95.0K  & 76,537  & \textcolor{green}{\CheckmarkBold} & \textcolor{green}{\CheckmarkBold} & \textcolor{green}{\CheckmarkBold} & \textcolor{green}{\CheckmarkBold} & \textcolor{green}{\CheckmarkBold} & \textcolor{green}{\CheckmarkBold} & \textcolor{green}{\CheckmarkBold} & \textcolor{green}{\CheckmarkBold}\\
\bottomrule
\end{tabular}
% \begin{tablenotes}
%     \item $^*$ In addition, the StarCoder-API can offer 4.77M more APIs.
% \end{tablenotes}
\caption{Summary of existing instruction finetuning-based LLM agents for intrinsic reasoning and function calling, along with their training resources and sample sizes. "PT" and "IFT" denote "Pre-Training" and "Instruction Fine-Tuning", respectively.}
\vspace{-2ex}
\label{tab:related}
\end{threeparttable}
\end{table*}

\noindent \textbf{Prompting-based LLM Agents.} Due to the lack of agent-specific pre-training corpus, existing LLM agents rely on either prompt engineering~\cite{hsieh2023tool,lu2024chameleon,yao2022react,wang2023voyager} or instruction fine-tuning~\cite{chen2023fireact,zeng2023agenttuning} to understand human instructions, decompose high-level tasks, generate grounded plans, and execute multi-step actions. 
However, prompting-based methods mainly depend on the capabilities of backbone LLMs (usually commercial LLMs), failing to introduce new knowledge and struggling to generalize to unseen tasks~\cite{sun2024adaplanner,zhuang2023toolchain}. 

\noindent \textbf{Instruction Finetuning-based LLM Agents.} Considering the extensive diversity of APIs and the complexity of multi-tool instructions, tool learning inherently presents greater challenges than natural language tasks, such as text generation~\cite{qin2023toolllm}.
Post-training techniques focus more on instruction following and aligning output with specific formats~\cite{patil2023gorilla,hao2024toolkengpt,qin2023toolllm,schick2024toolformer}, rather than fundamentally improving model knowledge or capabilities. 
Moreover, heavy fine-tuning can hinder generalization or even degrade performance in non-agent use cases, potentially suppressing the original base model capabilities~\cite{ghosh2024a}.

\noindent \textbf{Pretraining-based LLM Agents.} While pre-training serves as an essential alternative, prior works~\cite{nijkamp2023codegen,roziere2023code,xu2024lemur,patil2023gorilla} have primarily focused on improving task-specific capabilities (\eg, code generation) instead of general-domain LLM agents, due to single-source, uni-type, small-scale, and poor-quality pre-training data. 
Existing tool documentation data for agent training either lacks diverse real-world APIs~\cite{patil2023gorilla, tang2023toolalpaca} or is constrained to single-tool or single-round tool execution. 
Furthermore, trajectory data mostly imitate expert behavior or follow function-calling rules with inferior planning and reasoning, failing to fully elicit LLMs' capabilities and handle complex instructions~\cite{qin2023toolllm}. 
Given a wide range of candidate API functions, each comprising various function names and parameters available at every planning step, identifying globally optimal solutions and generalizing across tasks remains highly challenging.



\section{Preliminaries}
\label{Preliminaries}
\begin{figure*}[t]
    \centering
    \includegraphics[width=0.95\linewidth]{fig/HealthGPT_Framework.png}
    \caption{The \ourmethod{} architecture integrates hierarchical visual perception and H-LoRA, employing a task-specific hard router to select visual features and H-LoRA plugins, ultimately generating outputs with an autoregressive manner.}
    \label{fig:architecture}
\end{figure*}
\noindent\textbf{Large Vision-Language Models.} 
The input to a LVLM typically consists of an image $x^{\text{img}}$ and a discrete text sequence $x^{\text{txt}}$. The visual encoder $\mathcal{E}^{\text{img}}$ converts the input image $x^{\text{img}}$ into a sequence of visual tokens $\mathcal{V} = [v_i]_{i=1}^{N_v}$, while the text sequence $x^{\text{txt}}$ is mapped into a sequence of text tokens $\mathcal{T} = [t_i]_{i=1}^{N_t}$ using an embedding function $\mathcal{E}^{\text{txt}}$. The LLM $\mathcal{M_\text{LLM}}(\cdot|\theta)$ models the joint probability of the token sequence $\mathcal{U} = \{\mathcal{V},\mathcal{T}\}$, which is expressed as:
\begin{equation}
    P_\theta(R | \mathcal{U}) = \prod_{i=1}^{N_r} P_\theta(r_i | \{\mathcal{U}, r_{<i}\}),
\end{equation}
where $R = [r_i]_{i=1}^{N_r}$ is the text response sequence. The LVLM iteratively generates the next token $r_i$ based on $r_{<i}$. The optimization objective is to minimize the cross-entropy loss of the response $\mathcal{R}$.
% \begin{equation}
%     \mathcal{L}_{\text{VLM}} = \mathbb{E}_{R|\mathcal{U}}\left[-\log P_\theta(R | \mathcal{U})\right]
% \end{equation}
It is worth noting that most LVLMs adopt a design paradigm based on ViT, alignment adapters, and pre-trained LLMs\cite{liu2023llava,liu2024improved}, enabling quick adaptation to downstream tasks.


\noindent\textbf{VQGAN.}
VQGAN~\cite{esser2021taming} employs latent space compression and indexing mechanisms to effectively learn a complete discrete representation of images. VQGAN first maps the input image $x^{\text{img}}$ to a latent representation $z = \mathcal{E}(x)$ through a encoder $\mathcal{E}$. Then, the latent representation is quantized using a codebook $\mathcal{Z} = \{z_k\}_{k=1}^K$, generating a discrete index sequence $\mathcal{I} = [i_m]_{m=1}^N$, where $i_m \in \mathcal{Z}$ represents the quantized code index:
\begin{equation}
    \mathcal{I} = \text{Quantize}(z|\mathcal{Z}) = \arg\min_{z_k \in \mathcal{Z}} \| z - z_k \|_2.
\end{equation}
In our approach, the discrete index sequence $\mathcal{I}$ serves as a supervisory signal for the generation task, enabling the model to predict the index sequence $\hat{\mathcal{I}}$ from input conditions such as text or other modality signals.  
Finally, the predicted index sequence $\hat{\mathcal{I}}$ is upsampled by the VQGAN decoder $G$, generating the high-quality image $\hat{x}^\text{img} = G(\hat{\mathcal{I}})$.



\noindent\textbf{Low Rank Adaptation.} 
LoRA\cite{hu2021lora} effectively captures the characteristics of downstream tasks by introducing low-rank adapters. The core idea is to decompose the bypass weight matrix $\Delta W\in\mathbb{R}^{d^{\text{in}} \times d^{\text{out}}}$ into two low-rank matrices $ \{A \in \mathbb{R}^{d^{\text{in}} \times r}, B \in \mathbb{R}^{r \times d^{\text{out}}} \}$, where $ r \ll \min\{d^{\text{in}}, d^{\text{out}}\} $, significantly reducing learnable parameters. The output with the LoRA adapter for the input $x$ is then given by:
\begin{equation}
    h = x W_0 + \alpha x \Delta W/r = x W_0 + \alpha xAB/r,
\end{equation}
where matrix $ A $ is initialized with a Gaussian distribution, while the matrix $ B $ is initialized as a zero matrix. The scaling factor $ \alpha/r $ controls the impact of $ \Delta W $ on the model.

\section{HealthGPT}
\label{Method}


\subsection{Unified Autoregressive Generation.}  
% As shown in Figure~\ref{fig:architecture}, 
\ourmethod{} (Figure~\ref{fig:architecture}) utilizes a discrete token representation that covers both text and visual outputs, unifying visual comprehension and generation as an autoregressive task. 
For comprehension, $\mathcal{M}_\text{llm}$ receives the input joint sequence $\mathcal{U}$ and outputs a series of text token $\mathcal{R} = [r_1, r_2, \dots, r_{N_r}]$, where $r_i \in \mathcal{V}_{\text{txt}}$, and $\mathcal{V}_{\text{txt}}$ represents the LLM's vocabulary:
\begin{equation}
    P_\theta(\mathcal{R} \mid \mathcal{U}) = \prod_{i=1}^{N_r} P_\theta(r_i \mid \mathcal{U}, r_{<i}).
\end{equation}
For generation, $\mathcal{M}_\text{llm}$ first receives a special start token $\langle \text{START\_IMG} \rangle$, then generates a series of tokens corresponding to the VQGAN indices $\mathcal{I} = [i_1, i_2, \dots, i_{N_i}]$, where $i_j \in \mathcal{V}_{\text{vq}}$, and $\mathcal{V}_{\text{vq}}$ represents the index range of VQGAN. Upon completion of generation, the LLM outputs an end token $\langle \text{END\_IMG} \rangle$:
\begin{equation}
    P_\theta(\mathcal{I} \mid \mathcal{U}) = \prod_{j=1}^{N_i} P_\theta(i_j \mid \mathcal{U}, i_{<j}).
\end{equation}
Finally, the generated index sequence $\mathcal{I}$ is fed into the decoder $G$, which reconstructs the target image $\hat{x}^{\text{img}} = G(\mathcal{I})$.

\subsection{Hierarchical Visual Perception}  
Given the differences in visual perception between comprehension and generation tasks—where the former focuses on abstract semantics and the latter emphasizes complete semantics—we employ ViT to compress the image into discrete visual tokens at multiple hierarchical levels.
Specifically, the image is converted into a series of features $\{f_1, f_2, \dots, f_L\}$ as it passes through $L$ ViT blocks.

To address the needs of various tasks, the hidden states are divided into two types: (i) \textit{Concrete-grained features} $\mathcal{F}^{\text{Con}} = \{f_1, f_2, \dots, f_k\}, k < L$, derived from the shallower layers of ViT, containing sufficient global features, suitable for generation tasks; 
(ii) \textit{Abstract-grained features} $\mathcal{F}^{\text{Abs}} = \{f_{k+1}, f_{k+2}, \dots, f_L\}$, derived from the deeper layers of ViT, which contain abstract semantic information closer to the text space, suitable for comprehension tasks.

The task type $T$ (comprehension or generation) determines which set of features is selected as the input for the downstream large language model:
\begin{equation}
    \mathcal{F}^{\text{img}}_T =
    \begin{cases}
        \mathcal{F}^{\text{Con}}, & \text{if } T = \text{generation task} \\
        \mathcal{F}^{\text{Abs}}, & \text{if } T = \text{comprehension task}
    \end{cases}
\end{equation}
We integrate the image features $\mathcal{F}^{\text{img}}_T$ and text features $\mathcal{T}$ into a joint sequence through simple concatenation, which is then fed into the LLM $\mathcal{M}_{\text{llm}}$ for autoregressive generation.
% :
% \begin{equation}
%     \mathcal{R} = \mathcal{M}_{\text{llm}}(\mathcal{U}|\theta), \quad \mathcal{U} = [\mathcal{F}^{\text{img}}_T; \mathcal{T}]
% \end{equation}
\subsection{Heterogeneous Knowledge Adaptation}
We devise H-LoRA, which stores heterogeneous knowledge from comprehension and generation tasks in separate modules and dynamically routes to extract task-relevant knowledge from these modules. 
At the task level, for each task type $ T $, we dynamically assign a dedicated H-LoRA submodule $ \theta^T $, which is expressed as:
\begin{equation}
    \mathcal{R} = \mathcal{M}_\text{LLM}(\mathcal{U}|\theta, \theta^T), \quad \theta^T = \{A^T, B^T, \mathcal{R}^T_\text{outer}\}.
\end{equation}
At the feature level for a single task, H-LoRA integrates the idea of Mixture of Experts (MoE)~\cite{masoudnia2014mixture} and designs an efficient matrix merging and routing weight allocation mechanism, thus avoiding the significant computational delay introduced by matrix splitting in existing MoELoRA~\cite{luo2024moelora}. Specifically, we first merge the low-rank matrices (rank = r) of $ k $ LoRA experts into a unified matrix:
\begin{equation}
    \mathbf{A}^{\text{merged}}, \mathbf{B}^{\text{merged}} = \text{Concat}(\{A_i\}_1^k), \text{Concat}(\{B_i\}_1^k),
\end{equation}
where $ \mathbf{A}^{\text{merged}} \in \mathbb{R}^{d^\text{in} \times rk} $ and $ \mathbf{B}^{\text{merged}} \in \mathbb{R}^{rk \times d^\text{out}} $. The $k$-dimension routing layer generates expert weights $ \mathcal{W} \in \mathbb{R}^{\text{token\_num} \times k} $ based on the input hidden state $ x $, and these are expanded to $ \mathbb{R}^{\text{token\_num} \times rk} $ as follows:
\begin{equation}
    \mathcal{W}^\text{expanded} = \alpha k \mathcal{W} / r \otimes \mathbf{1}_r,
\end{equation}
where $ \otimes $ denotes the replication operation.
The overall output of H-LoRA is computed as:
\begin{equation}
    \mathcal{O}^\text{H-LoRA} = (x \mathbf{A}^{\text{merged}} \odot \mathcal{W}^\text{expanded}) \mathbf{B}^{\text{merged}},
\end{equation}
where $ \odot $ represents element-wise multiplication. Finally, the output of H-LoRA is added to the frozen pre-trained weights to produce the final output:
\begin{equation}
    \mathcal{O} = x W_0 + \mathcal{O}^\text{H-LoRA}.
\end{equation}
% In summary, H-LoRA is a task-based dynamic PEFT method that achieves high efficiency in single-task fine-tuning.

\subsection{Training Pipeline}

\begin{figure}[t]
    \centering
    \hspace{-4mm}
    \includegraphics[width=0.94\linewidth]{fig/data.pdf}
    \caption{Data statistics of \texttt{VL-Health}. }
    \label{fig:data}
\end{figure}
\noindent \textbf{1st Stage: Multi-modal Alignment.} 
In the first stage, we design separate visual adapters and H-LoRA submodules for medical unified tasks. For the medical comprehension task, we train abstract-grained visual adapters using high-quality image-text pairs to align visual embeddings with textual embeddings, thereby enabling the model to accurately describe medical visual content. During this process, the pre-trained LLM and its corresponding H-LoRA submodules remain frozen. In contrast, the medical generation task requires training concrete-grained adapters and H-LoRA submodules while keeping the LLM frozen. Meanwhile, we extend the textual vocabulary to include multimodal tokens, enabling the support of additional VQGAN vector quantization indices. The model trains on image-VQ pairs, endowing the pre-trained LLM with the capability for image reconstruction. This design ensures pixel-level consistency of pre- and post-LVLM. The processes establish the initial alignment between the LLM’s outputs and the visual inputs.

\noindent \textbf{2nd Stage: Heterogeneous H-LoRA Plugin Adaptation.}  
The submodules of H-LoRA share the word embedding layer and output head but may encounter issues such as bias and scale inconsistencies during training across different tasks. To ensure that the multiple H-LoRA plugins seamlessly interface with the LLMs and form a unified base, we fine-tune the word embedding layer and output head using a small amount of mixed data to maintain consistency in the model weights. Specifically, during this stage, all H-LoRA submodules for different tasks are kept frozen, with only the word embedding layer and output head being optimized. Through this stage, the model accumulates foundational knowledge for unified tasks by adapting H-LoRA plugins.

\begin{table*}[!t]
\centering
\caption{Comparison of \ourmethod{} with other LVLMs and unified multi-modal models on medical visual comprehension tasks. \textbf{Bold} and \underline{underlined} text indicates the best performance and second-best performance, respectively.}
\resizebox{\textwidth}{!}{
\begin{tabular}{c|lcc|cccccccc|c}
\toprule
\rowcolor[HTML]{E9F3FE} &  &  &  & \multicolumn{2}{c}{\textbf{VQA-RAD \textuparrow}} & \multicolumn{2}{c}{\textbf{SLAKE \textuparrow}} & \multicolumn{2}{c}{\textbf{PathVQA \textuparrow}} &  &  &  \\ 
\cline{5-10}
\rowcolor[HTML]{E9F3FE}\multirow{-2}{*}{\textbf{Type}} & \multirow{-2}{*}{\textbf{Model}} & \multirow{-2}{*}{\textbf{\# Params}} & \multirow{-2}{*}{\makecell{\textbf{Medical} \\ \textbf{LVLM}}} & \textbf{close} & \textbf{all} & \textbf{close} & \textbf{all} & \textbf{close} & \textbf{all} & \multirow{-2}{*}{\makecell{\textbf{MMMU} \\ \textbf{-Med}}\textuparrow} & \multirow{-2}{*}{\textbf{OMVQA}\textuparrow} & \multirow{-2}{*}{\textbf{Avg. \textuparrow}} \\ 
\midrule \midrule
\multirow{9}{*}{\textbf{Comp. Only}} 
& Med-Flamingo & 8.3B & \Large \ding{51} & 58.6 & 43.0 & 47.0 & 25.5 & 61.9 & 31.3 & 28.7 & 34.9 & 41.4 \\
& LLaVA-Med & 7B & \Large \ding{51} & 60.2 & 48.1 & 58.4 & 44.8 & 62.3 & 35.7 & 30.0 & 41.3 & 47.6 \\
& HuatuoGPT-Vision & 7B & \Large \ding{51} & 66.9 & 53.0 & 59.8 & 49.1 & 52.9 & 32.0 & 42.0 & 50.0 & 50.7 \\
& BLIP-2 & 6.7B & \Large \ding{55} & 43.4 & 36.8 & 41.6 & 35.3 & 48.5 & 28.8 & 27.3 & 26.9 & 36.1 \\
& LLaVA-v1.5 & 7B & \Large \ding{55} & 51.8 & 42.8 & 37.1 & 37.7 & 53.5 & 31.4 & 32.7 & 44.7 & 41.5 \\
& InstructBLIP & 7B & \Large \ding{55} & 61.0 & 44.8 & 66.8 & 43.3 & 56.0 & 32.3 & 25.3 & 29.0 & 44.8 \\
& Yi-VL & 6B & \Large \ding{55} & 52.6 & 42.1 & 52.4 & 38.4 & 54.9 & 30.9 & 38.0 & 50.2 & 44.9 \\
& InternVL2 & 8B & \Large \ding{55} & 64.9 & 49.0 & 66.6 & 50.1 & 60.0 & 31.9 & \underline{43.3} & 54.5 & 52.5\\
& Llama-3.2 & 11B & \Large \ding{55} & 68.9 & 45.5 & 72.4 & 52.1 & 62.8 & 33.6 & 39.3 & 63.2 & 54.7 \\
\midrule
\multirow{5}{*}{\textbf{Comp. \& Gen.}} 
& Show-o & 1.3B & \Large \ding{55} & 50.6 & 33.9 & 31.5 & 17.9 & 52.9 & 28.2 & 22.7 & 45.7 & 42.6 \\
& Unified-IO 2 & 7B & \Large \ding{55} & 46.2 & 32.6 & 35.9 & 21.9 & 52.5 & 27.0 & 25.3 & 33.0 & 33.8 \\
& Janus & 1.3B & \Large \ding{55} & 70.9 & 52.8 & 34.7 & 26.9 & 51.9 & 27.9 & 30.0 & 26.8 & 33.5 \\
& \cellcolor[HTML]{DAE0FB}HealthGPT-M3 & \cellcolor[HTML]{DAE0FB}3.8B & \cellcolor[HTML]{DAE0FB}\Large \ding{51} & \cellcolor[HTML]{DAE0FB}\underline{73.7} & \cellcolor[HTML]{DAE0FB}\underline{55.9} & \cellcolor[HTML]{DAE0FB}\underline{74.6} & \cellcolor[HTML]{DAE0FB}\underline{56.4} & \cellcolor[HTML]{DAE0FB}\underline{78.7} & \cellcolor[HTML]{DAE0FB}\underline{39.7} & \cellcolor[HTML]{DAE0FB}\underline{43.3} & \cellcolor[HTML]{DAE0FB}\underline{68.5} & \cellcolor[HTML]{DAE0FB}\underline{61.3} \\
& \cellcolor[HTML]{DAE0FB}HealthGPT-L14 & \cellcolor[HTML]{DAE0FB}14B & \cellcolor[HTML]{DAE0FB}\Large \ding{51} & \cellcolor[HTML]{DAE0FB}\textbf{77.7} & \cellcolor[HTML]{DAE0FB}\textbf{58.3} & \cellcolor[HTML]{DAE0FB}\textbf{76.4} & \cellcolor[HTML]{DAE0FB}\textbf{64.5} & \cellcolor[HTML]{DAE0FB}\textbf{85.9} & \cellcolor[HTML]{DAE0FB}\textbf{44.4} & \cellcolor[HTML]{DAE0FB}\textbf{49.2} & \cellcolor[HTML]{DAE0FB}\textbf{74.4} & \cellcolor[HTML]{DAE0FB}\textbf{66.4} \\
\bottomrule
\end{tabular}
}
\label{tab:results}
\end{table*}
\begin{table*}[ht]
    \centering
    \caption{The experimental results for the four modality conversion tasks.}
    \resizebox{\textwidth}{!}{
    \begin{tabular}{l|ccc|ccc|ccc|ccc}
        \toprule
        \rowcolor[HTML]{E9F3FE} & \multicolumn{3}{c}{\textbf{CT to MRI (Brain)}} & \multicolumn{3}{c}{\textbf{CT to MRI (Pelvis)}} & \multicolumn{3}{c}{\textbf{MRI to CT (Brain)}} & \multicolumn{3}{c}{\textbf{MRI to CT (Pelvis)}} \\
        \cline{2-13}
        \rowcolor[HTML]{E9F3FE}\multirow{-2}{*}{\textbf{Model}}& \textbf{SSIM $\uparrow$} & \textbf{PSNR $\uparrow$} & \textbf{MSE $\downarrow$} & \textbf{SSIM $\uparrow$} & \textbf{PSNR $\uparrow$} & \textbf{MSE $\downarrow$} & \textbf{SSIM $\uparrow$} & \textbf{PSNR $\uparrow$} & \textbf{MSE $\downarrow$} & \textbf{SSIM $\uparrow$} & \textbf{PSNR $\uparrow$} & \textbf{MSE $\downarrow$} \\
        \midrule \midrule
        pix2pix & 71.09 & 32.65 & 36.85 & 59.17 & 31.02 & 51.91 & 78.79 & 33.85 & 28.33 & 72.31 & 32.98 & 36.19 \\
        CycleGAN & 54.76 & 32.23 & 40.56 & 54.54 & 30.77 & 55.00 & 63.75 & 31.02 & 52.78 & 50.54 & 29.89 & 67.78 \\
        BBDM & {71.69} & {32.91} & {34.44} & 57.37 & 31.37 & 48.06 & \textbf{86.40} & 34.12 & 26.61 & {79.26} & 33.15 & 33.60 \\
        Vmanba & 69.54 & 32.67 & 36.42 & {63.01} & {31.47} & {46.99} & 79.63 & 34.12 & 26.49 & 77.45 & 33.53 & 31.85 \\
        DiffMa & 71.47 & 32.74 & 35.77 & 62.56 & 31.43 & 47.38 & 79.00 & {34.13} & {26.45} & 78.53 & {33.68} & {30.51} \\
        \rowcolor[HTML]{DAE0FB}HealthGPT-M3 & \underline{79.38} & \underline{33.03} & \underline{33.48} & \underline{71.81} & \underline{31.83} & \underline{43.45} & {85.06} & \textbf{34.40} & \textbf{25.49} & \underline{84.23} & \textbf{34.29} & \textbf{27.99} \\
        \rowcolor[HTML]{DAE0FB}HealthGPT-L14 & \textbf{79.73} & \textbf{33.10} & \textbf{32.96} & \textbf{71.92} & \textbf{31.87} & \textbf{43.09} & \underline{85.31} & \underline{34.29} & \underline{26.20} & \textbf{84.96} & \underline{34.14} & \underline{28.13} \\
        \bottomrule
    \end{tabular}
    }
    \label{tab:conversion}
\end{table*}

\noindent \textbf{3rd Stage: Visual Instruction Fine-Tuning.}  
In the third stage, we introduce additional task-specific data to further optimize the model and enhance its adaptability to downstream tasks such as medical visual comprehension (e.g., medical QA, medical dialogues, and report generation) or generation tasks (e.g., super-resolution, denoising, and modality conversion). Notably, by this stage, the word embedding layer and output head have been fine-tuned, only the H-LoRA modules and adapter modules need to be trained. This strategy significantly improves the model's adaptability and flexibility across different tasks.


\section{Experiment}
\label{s:experiment}

\subsection{Data Description}
We evaluate our method on FI~\cite{you2016building}, Twitter\_LDL~\cite{yang2017learning} and Artphoto~\cite{machajdik2010affective}.
FI is a public dataset built from Flickr and Instagram, with 23,308 images and eight emotion categories, namely \textit{amusement}, \textit{anger}, \textit{awe},  \textit{contentment}, \textit{disgust}, \textit{excitement},  \textit{fear}, and \textit{sadness}. 
% Since images in FI are all copyrighted by law, some images are corrupted now, so we remove these samples and retain 21,828 images.
% T4SA contains images from Twitter, which are classified into three categories: \textit{positive}, \textit{neutral}, and \textit{negative}. In this paper, we adopt the base version of B-T4SA, which contains 470,586 images and provides text descriptions of the corresponding tweets.
Twitter\_LDL contains 10,045 images from Twitter, with the same eight categories as the FI dataset.
% 。
For these two datasets, they are randomly split into 80\%
training and 20\% testing set.
Artphoto contains 806 artistic photos from the DeviantArt website, which we use to further evaluate the zero-shot capability of our model.
% on the small-scale dataset.
% We construct and publicly release the first image sentiment analysis dataset containing metadata.
% 。

% Based on these datasets, we are the first to construct and publicly release metadata-enhanced image sentiment analysis datasets. These datasets include scenes, tags, descriptions, and corresponding confidence scores, and are available at this link for future research purposes.


% 
\begin{table}[t]
\centering
% \begin{center}
\caption{Overall performance of different models on FI and Twitter\_LDL datasets.}
\label{tab:cap1}
% \resizebox{\linewidth}{!}
{
\begin{tabular}{l|c|c|c|c}
\hline
\multirow{2}{*}{\textbf{Model}} & \multicolumn{2}{c|}{\textbf{FI}}  & \multicolumn{2}{c}{\textbf{Twitter\_LDL}} \\ \cline{2-5} 
  & \textbf{Accuracy} & \textbf{F1} & \textbf{Accuracy} & \textbf{F1}  \\ \hline
% (\rownumber)~AlexNet~\cite{krizhevsky2017imagenet}  & 58.13\% & 56.35\%  & 56.24\%& 55.02\%  \\ 
% (\rownumber)~VGG16~\cite{simonyan2014very}  & 63.75\%& 63.08\%  & 59.34\%& 59.02\%  \\ 
(\rownumber)~ResNet101~\cite{he2016deep} & 66.16\%& 65.56\%  & 62.02\% & 61.34\%  \\ 
(\rownumber)~CDA~\cite{han2023boosting} & 66.71\%& 65.37\%  & 64.14\% & 62.85\%  \\ 
(\rownumber)~CECCN~\cite{ruan2024color} & 67.96\%& 66.74\%  & 64.59\%& 64.72\% \\ 
(\rownumber)~EmoVIT~\cite{xie2024emovit} & 68.09\%& 67.45\%  & 63.12\% & 61.97\%  \\ 
(\rownumber)~ComLDL~\cite{zhang2022compound} & 68.83\%& 67.28\%  & 65.29\% & 63.12\%  \\ 
(\rownumber)~WSDEN~\cite{li2023weakly} & 69.78\%& 69.61\%  & 67.04\% & 65.49\% \\ 
(\rownumber)~ECWA~\cite{deng2021emotion} & 70.87\%& 69.08\%  & 67.81\% & 66.87\%  \\ 
(\rownumber)~EECon~\cite{yang2023exploiting} & 71.13\%& 68.34\%  & 64.27\%& 63.16\%  \\ 
(\rownumber)~MAM~\cite{zhang2024affective} & 71.44\%  & 70.83\% & 67.18\%  & 65.01\%\\ 
(\rownumber)~TGCA-PVT~\cite{chen2024tgca}   & 73.05\%  & 71.46\% & 69.87\%  & 68.32\% \\ 
(\rownumber)~OEAN~\cite{zhang2024object}   & 73.40\%  & 72.63\% & 70.52\%  & 69.47\% \\ \hline
(\rownumber)~\shortname  & \textbf{79.48\%} & \textbf{79.22\%} & \textbf{74.12\%} & \textbf{73.09\%} \\ \hline
\end{tabular}
}
\vspace{-6mm}
% \end{center}
\end{table}
% 

\subsection{Experiment Setting}
% \subsubsection{Model Setting.}
% 
\textbf{Model Setting:}
For feature representation, we set $k=10$ to select object tags, and adopt clip-vit-base-patch32 as the pre-trained model for unified feature representation.
Moreover, we empirically set $(d_e, d_h, d_k, d_s) = (512, 128, 16, 64)$, and set the classification class $L$ to 8.

% 

\textbf{Training Setting:}
To initialize the model, we set all weights such as $\boldsymbol{W}$ following the truncated normal distribution, and use AdamW optimizer with the learning rate of $1 \times 10^{-4}$.
% warmup scheduler of cosine, warmup steps of 2000.
Furthermore, we set the batch size to 32 and the epoch of the training process to 200.
During the implementation, we utilize \textit{PyTorch} to build our entire model.
% , and our project codes are publicly available at https://github.com/zzmyrep/MESN.
% Our project codes as well as data are all publicly available on GitHub\footnote{https://github.com/zzmyrep/KBCEN}.
% Code is available at \href{https://github.com/zzmyrep/KBCEN}{https://github.com/zzmyrep/KBCEN}.

\textbf{Evaluation Metrics:}
Following~\cite{zhang2024affective, chen2024tgca, zhang2024object}, we adopt \textit{accuracy} and \textit{F1} as our evaluation metrics to measure the performance of different methods for image sentiment analysis. 



\subsection{Experiment Result}
% We compare our model against the following baselines: AlexNet~\cite{krizhevsky2017imagenet}, VGG16~\cite{simonyan2014very}, ResNet101~\cite{he2016deep}, CECCN~\cite{ruan2024color}, EmoVIT~\cite{xie2024emovit}, WSCNet~\cite{yang2018weakly}, ECWA~\cite{deng2021emotion}, EECon~\cite{yang2023exploiting}, MAM~\cite{zhang2024affective} and TGCA-PVT~\cite{chen2024tgca}, and the overall results are summarized in Table~\ref{tab:cap1}.
We compare our model against several baselines, and the overall results are summarized in Table~\ref{tab:cap1}.
We observe that our model achieves the best performance in both accuracy and F1 metrics, significantly outperforming the previous models. 
This superior performance is mainly attributed to our effective utilization of metadata to enhance image sentiment analysis, as well as the exceptional capability of the unified sentiment transformer framework we developed. These results strongly demonstrate that our proposed method can bring encouraging performance for image sentiment analysis.

\setcounter{magicrownumbers}{0} 
\begin{table}[t]
\begin{center}
\caption{Ablation study of~\shortname~on FI dataset.} 
% \vspace{1mm}
\label{tab:cap2}
\resizebox{.9\linewidth}{!}
{
\begin{tabular}{lcc}
  \hline
  \textbf{Model} & \textbf{Accuracy} & \textbf{F1} \\
  \hline
  (\rownumber)~Ours (w/o vision) & 65.72\% & 64.54\% \\
  (\rownumber)~Ours (w/o text description) & 74.05\% & 72.58\% \\
  (\rownumber)~Ours (w/o object tag) & 77.45\% & 76.84\% \\
  (\rownumber)~Ours (w/o scene tag) & 78.47\% & 78.21\% \\
  \hline
  (\rownumber)~Ours (w/o unified embedding) & 76.41\% & 76.23\% \\
  (\rownumber)~Ours (w/o adaptive learning) & 76.83\% & 76.56\% \\
  (\rownumber)~Ours (w/o cross-modal fusion) & 76.85\% & 76.49\% \\
  \hline
  (\rownumber)~Ours  & \textbf{79.48\%} & \textbf{79.22\%} \\
  \hline
\end{tabular}
}
\end{center}
\vspace{-5mm}
\end{table}


\begin{figure}[t]
\centering
% \vspace{-2mm}
\includegraphics[width=0.42\textwidth]{fig/2dvisual-linux4-paper2.pdf}
\caption{Visualization of feature distribution on eight categories before (left) and after (right) model processing.}
% 
\label{fig:visualization}
\vspace{-5mm}
\end{figure}

\subsection{Ablation Performance}
In this subsection, we conduct an ablation study to examine which component is really important for performance improvement. The results are reported in Table~\ref{tab:cap2}.

For information utilization, we observe a significant decline in model performance when visual features are removed. Additionally, the performance of \shortname~decreases when different metadata are removed separately, which means that text description, object tag, and scene tag are all critical for image sentiment analysis.
Recalling the model architecture, we separately remove transformer layers of the unified representation module, the adaptive learning module, and the cross-modal fusion module, replacing them with MLPs of the same parameter scale.
In this way, we can observe varying degrees of decline in model performance, indicating that these modules are indispensable for our model to achieve better performance.

\subsection{Visualization}
% 


% % 开始使用minipage进行左右排列
% \begin{minipage}[t]{0.45\textwidth}  % 子图1宽度为45%
%     \centering
%     \includegraphics[width=\textwidth]{2dvisual.pdf}  % 插入图片
%     \captionof{figure}{Visualization of feature distribution.}  % 使用captionof添加图片标题
%     \label{fig:visualization}
% \end{minipage}


% \begin{figure}[t]
% \centering
% \vspace{-2mm}
% \includegraphics[width=0.45\textwidth]{fig/2dvisual.pdf}
% \caption{Visualization of feature distribution.}
% \label{fig:visualization}
% % \vspace{-4mm}
% \end{figure}

% \begin{figure}[t]
% \centering
% \vspace{-2mm}
% \includegraphics[width=0.45\textwidth]{fig/2dvisual-linux3-paper.pdf}
% \caption{Visualization of feature distribution.}
% \label{fig:visualization}
% % \vspace{-4mm}
% \end{figure}



\begin{figure}[tbp]   
\vspace{-4mm}
  \centering            
  \subfloat[Depth of adaptive learning layers]   
  {
    \label{fig:subfig1}\includegraphics[width=0.22\textwidth]{fig/fig_sensitivity-a5}
  }
  \subfloat[Depth of fusion layers]
  {
    % \label{fig:subfig2}\includegraphics[width=0.22\textwidth]{fig/fig_sensitivity-b2}
    \label{fig:subfig2}\includegraphics[width=0.22\textwidth]{fig/fig_sensitivity-b2-num.pdf}
  }
  \caption{Sensitivity study of \shortname~on different depth. }   
  \label{fig:fig_sensitivity}  
\vspace{-2mm}
\end{figure}

% \begin{figure}[htbp]
% \centerline{\includegraphics{2dvisual.pdf}}
% \caption{Visualization of feature distribution.}
% \label{fig:visualization}
% \end{figure}

% In Fig.~\ref{fig:visualization}, we use t-SNE~\cite{van2008visualizing} to reduce the dimension of data features for visualization, Figure in left represents the metadata features before model processing, the features are obtained by embedding through the CLIP model, and figure in right shows the features of the data after model processing, it can be observed that after the model processing, the data with different label categories fall in different regions in the space, therefore, we can conclude that the Therefore, we can conclude that the model can effectively utilize the information contained in the metadata and use it to guide the model for classification.

In Fig.~\ref{fig:visualization}, we use t-SNE~\cite{van2008visualizing} to reduce the dimension of data features for visualization.
The left figure shows metadata features before being processed by our model (\textit{i.e.}, embedded by CLIP), while the right shows the distribution of features after being processed by our model.
We can observe that after the model processing, data with the same label are closer to each other, while others are farther away.
Therefore, it shows that the model can effectively utilize the information contained in the metadata and use it to guide the classification process.

\subsection{Sensitivity Analysis}
% 
In this subsection, we conduct a sensitivity analysis to figure out the effect of different depth settings of adaptive learning layers and fusion layers. 
% In this subsection, we conduct a sensitivity analysis to figure out the effect of different depth settings on the model. 
% Fig.~\ref{fig:fig_sensitivity} presents the effect of different depth settings of adaptive learning layers and fusion layers. 
Taking Fig.~\ref{fig:fig_sensitivity} (a) as an example, the model performance improves with increasing depth, reaching the best performance at a depth of 4.
% Taking Fig.~\ref{fig:fig_sensitivity} (a) as an example, the performance of \shortname~improves with the increase of depth at first, reaching the best performance at a depth of 4.
When the depth continues to increase, the accuracy decreases to varying degrees.
Similar results can be observed in Fig.~\ref{fig:fig_sensitivity} (b).
Therefore, we set their depths to 4 and 6 respectively to achieve the best results.

% Through our experiments, we can observe that the effect of modifying these hyperparameters on the results of the experiments is very weak, and the surface model is not sensitive to the hyperparameters.


\subsection{Zero-shot Capability}
% 

% (1)~GCH~\cite{2010Analyzing} & 21.78\% & (5)~RA-DLNet~\cite{2020A} & 34.01\% \\ \hline
% (2)~WSCNet~\cite{2019WSCNet}  & 30.25\% & (6)~CECCN~\cite{ruan2024color} & 43.83\% \\ \hline
% (3)~PCNN~\cite{2015Robust} & 31.68\%  & (7)~EmoVIT~\cite{xie2024emovit} & 44.90\% \\ \hline
% (4)~AR~\cite{2018Visual} & 32.67\% & (8)~Ours (Zero-shot) & 47.83\% \\ \hline


\begin{table}[t]
\centering
\caption{Zero-shot capability of \shortname.}
\label{tab:cap3}
\resizebox{1\linewidth}{!}
{
\begin{tabular}{lc|lc}
\hline
\textbf{Model} & \textbf{Accuracy} & \textbf{Model} & \textbf{Accuracy} \\ \hline
(1)~WSCNet~\cite{2019WSCNet}  & 30.25\% & (5)~MAM~\cite{zhang2024affective} & 39.56\%  \\ \hline
(2)~AR~\cite{2018Visual} & 32.67\% & (6)~CECCN~\cite{ruan2024color} & 43.83\% \\ \hline
(3)~RA-DLNet~\cite{2020A} & 34.01\%  & (7)~EmoVIT~\cite{xie2024emovit} & 44.90\% \\ \hline
(4)~CDA~\cite{han2023boosting} & 38.64\% & (8)~Ours (Zero-shot) & 47.83\% \\ \hline
\end{tabular}
}
\vspace{-5mm}
\end{table}

% We use the model trained on the FI dataset to test on the artphoto dataset to verify the model's generalization ability as well as robustness to other distributed datasets.
% We can observe that the MESN model shows strong competitiveness in terms of accuracy when compared to other trained models, which suggests that the model has a good generalization ability in the OOD task.

To validate the model's generalization ability and robustness to other distributed datasets, we directly test the model trained on the FI dataset, without training on Artphoto. 
% As observed in Table 3, compared to other models trained on Artphoto, we achieve highly competitive zero-shot performance, indicating that the model has good generalization ability in out-of-distribution tasks.
From Table~\ref{tab:cap3}, we can observe that compared with other models trained on Artphoto, we achieve competitive zero-shot performance, which shows that the model has good generalization ability in out-of-distribution tasks.


%%%%%%%%%%%%
%  E2E     %
%%%%%%%%%%%%


\section{Conclusion}
In this paper, we introduced Wi-Chat, the first LLM-powered Wi-Fi-based human activity recognition system that integrates the reasoning capabilities of large language models with the sensing potential of wireless signals. Our experimental results on a self-collected Wi-Fi CSI dataset demonstrate the promising potential of LLMs in enabling zero-shot Wi-Fi sensing. These findings suggest a new paradigm for human activity recognition that does not rely on extensive labeled data. We hope future research will build upon this direction, further exploring the applications of LLMs in signal processing domains such as IoT, mobile sensing, and radar-based systems.

\section*{Limitations}
While our work represents the first attempt to leverage LLMs for processing Wi-Fi signals, it is a preliminary study focused on a relatively simple task: Wi-Fi-based human activity recognition. This choice allows us to explore the feasibility of LLMs in wireless sensing but also comes with certain limitations.

Our approach primarily evaluates zero-shot performance, which, while promising, may still lag behind traditional supervised learning methods in highly complex or fine-grained recognition tasks. Besides, our study is limited to a controlled environment with a self-collected dataset, and the generalizability of LLMs to diverse real-world scenarios with varying Wi-Fi conditions, environmental interference, and device heterogeneity remains an open question.

Additionally, we have yet to explore the full potential of LLMs in more advanced Wi-Fi sensing applications, such as fine-grained gesture recognition, occupancy detection, and passive health monitoring. Future work should investigate the scalability of LLM-based approaches, their robustness to domain shifts, and their integration with multimodal sensing techniques in broader IoT applications.


% Bibliography entries for the entire Anthology, followed by custom entries
%\bibliography{anthology,custom}
% Custom bibliography entries only
\bibliography{main}
\newpage
\appendix

\section{Experiment prompts}
\label{sec:prompt}
The prompts used in the LLM experiments are shown in the following Table~\ref{tab:prompts}.

\definecolor{titlecolor}{rgb}{0.9, 0.5, 0.1}
\definecolor{anscolor}{rgb}{0.2, 0.5, 0.8}
\definecolor{labelcolor}{HTML}{48a07e}
\begin{table*}[h]
	\centering
	
 % \vspace{-0.2cm}
	
	\begin{center}
		\begin{tikzpicture}[
				chatbox_inner/.style={rectangle, rounded corners, opacity=0, text opacity=1, font=\sffamily\scriptsize, text width=5in, text height=9pt, inner xsep=6pt, inner ysep=6pt},
				chatbox_prompt_inner/.style={chatbox_inner, align=flush left, xshift=0pt, text height=11pt},
				chatbox_user_inner/.style={chatbox_inner, align=flush left, xshift=0pt},
				chatbox_gpt_inner/.style={chatbox_inner, align=flush left, xshift=0pt},
				chatbox/.style={chatbox_inner, draw=black!25, fill=gray!7, opacity=1, text opacity=0},
				chatbox_prompt/.style={chatbox, align=flush left, fill=gray!1.5, draw=black!30, text height=10pt},
				chatbox_user/.style={chatbox, align=flush left},
				chatbox_gpt/.style={chatbox, align=flush left},
				chatbox2/.style={chatbox_gpt, fill=green!25},
				chatbox3/.style={chatbox_gpt, fill=red!20, draw=black!20},
				chatbox4/.style={chatbox_gpt, fill=yellow!30},
				labelbox/.style={rectangle, rounded corners, draw=black!50, font=\sffamily\scriptsize\bfseries, fill=gray!5, inner sep=3pt},
			]
											
			\node[chatbox_user] (q1) {
				\textbf{System prompt}
				\newline
				\newline
				You are a helpful and precise assistant for segmenting and labeling sentences. We would like to request your help on curating a dataset for entity-level hallucination detection.
				\newline \newline
                We will give you a machine generated biography and a list of checked facts about the biography. Each fact consists of a sentence and a label (True/False). Please do the following process. First, breaking down the biography into words. Second, by referring to the provided list of facts, merging some broken down words in the previous step to form meaningful entities. For example, ``strategic thinking'' should be one entity instead of two. Third, according to the labels in the list of facts, labeling each entity as True or False. Specifically, for facts that share a similar sentence structure (\eg, \textit{``He was born on Mach 9, 1941.''} (\texttt{True}) and \textit{``He was born in Ramos Mejia.''} (\texttt{False})), please first assign labels to entities that differ across atomic facts. For example, first labeling ``Mach 9, 1941'' (\texttt{True}) and ``Ramos Mejia'' (\texttt{False}) in the above case. For those entities that are the same across atomic facts (\eg, ``was born'') or are neutral (\eg, ``he,'' ``in,'' and ``on''), please label them as \texttt{True}. For the cases that there is no atomic fact that shares the same sentence structure, please identify the most informative entities in the sentence and label them with the same label as the atomic fact while treating the rest of the entities as \texttt{True}. In the end, output the entities and labels in the following format:
                \begin{itemize}[nosep]
                    \item Entity 1 (Label 1)
                    \item Entity 2 (Label 2)
                    \item ...
                    \item Entity N (Label N)
                \end{itemize}
                % \newline \newline
                Here are two examples:
                \newline\newline
                \textbf{[Example 1]}
                \newline
                [The start of the biography]
                \newline
                \textcolor{titlecolor}{Marianne McAndrew is an American actress and singer, born on November 21, 1942, in Cleveland, Ohio. She began her acting career in the late 1960s, appearing in various television shows and films.}
                \newline
                [The end of the biography]
                \newline \newline
                [The start of the list of checked facts]
                \newline
                \textcolor{anscolor}{[Marianne McAndrew is an American. (False); Marianne McAndrew is an actress. (True); Marianne McAndrew is a singer. (False); Marianne McAndrew was born on November 21, 1942. (False); Marianne McAndrew was born in Cleveland, Ohio. (False); She began her acting career in the late 1960s. (True); She has appeared in various television shows. (True); She has appeared in various films. (True)]}
                \newline
                [The end of the list of checked facts]
                \newline \newline
                [The start of the ideal output]
                \newline
                \textcolor{labelcolor}{[Marianne McAndrew (True); is (True); an (True); American (False); actress (True); and (True); singer (False); , (True); born (True); on (True); November 21, 1942 (False); , (True); in (True); Cleveland, Ohio (False); . (True); She (True); began (True); her (True); acting career (True); in (True); the late 1960s (True); , (True); appearing (True); in (True); various (True); television shows (True); and (True); films (True); . (True)]}
                \newline
                [The end of the ideal output]
				\newline \newline
                \textbf{[Example 2]}
                \newline
                [The start of the biography]
                \newline
                \textcolor{titlecolor}{Doug Sheehan is an American actor who was born on April 27, 1949, in Santa Monica, California. He is best known for his roles in soap operas, including his portrayal of Joe Kelly on ``General Hospital'' and Ben Gibson on ``Knots Landing.''}
                \newline
                [The end of the biography]
                \newline \newline
                [The start of the list of checked facts]
                \newline
                \textcolor{anscolor}{[Doug Sheehan is an American. (True); Doug Sheehan is an actor. (True); Doug Sheehan was born on April 27, 1949. (True); Doug Sheehan was born in Santa Monica, California. (False); He is best known for his roles in soap operas. (True); He portrayed Joe Kelly. (True); Joe Kelly was in General Hospital. (True); General Hospital is a soap opera. (True); He portrayed Ben Gibson. (True); Ben Gibson was in Knots Landing. (True); Knots Landing is a soap opera. (True)]}
                \newline
                [The end of the list of checked facts]
                \newline \newline
                [The start of the ideal output]
                \newline
                \textcolor{labelcolor}{[Doug Sheehan (True); is (True); an (True); American (True); actor (True); who (True); was born (True); on (True); April 27, 1949 (True); in (True); Santa Monica, California (False); . (True); He (True); is (True); best known (True); for (True); his roles in soap operas (True); , (True); including (True); in (True); his portrayal (True); of (True); Joe Kelly (True); on (True); ``General Hospital'' (True); and (True); Ben Gibson (True); on (True); ``Knots Landing.'' (True)]}
                \newline
                [The end of the ideal output]
				\newline \newline
				\textbf{User prompt}
				\newline
				\newline
				[The start of the biography]
				\newline
				\textcolor{magenta}{\texttt{\{BIOGRAPHY\}}}
				\newline
				[The ebd of the biography]
				\newline \newline
				[The start of the list of checked facts]
				\newline
				\textcolor{magenta}{\texttt{\{LIST OF CHECKED FACTS\}}}
				\newline
				[The end of the list of checked facts]
			};
			\node[chatbox_user_inner] (q1_text) at (q1) {
				\textbf{System prompt}
				\newline
				\newline
				You are a helpful and precise assistant for segmenting and labeling sentences. We would like to request your help on curating a dataset for entity-level hallucination detection.
				\newline \newline
                We will give you a machine generated biography and a list of checked facts about the biography. Each fact consists of a sentence and a label (True/False). Please do the following process. First, breaking down the biography into words. Second, by referring to the provided list of facts, merging some broken down words in the previous step to form meaningful entities. For example, ``strategic thinking'' should be one entity instead of two. Third, according to the labels in the list of facts, labeling each entity as True or False. Specifically, for facts that share a similar sentence structure (\eg, \textit{``He was born on Mach 9, 1941.''} (\texttt{True}) and \textit{``He was born in Ramos Mejia.''} (\texttt{False})), please first assign labels to entities that differ across atomic facts. For example, first labeling ``Mach 9, 1941'' (\texttt{True}) and ``Ramos Mejia'' (\texttt{False}) in the above case. For those entities that are the same across atomic facts (\eg, ``was born'') or are neutral (\eg, ``he,'' ``in,'' and ``on''), please label them as \texttt{True}. For the cases that there is no atomic fact that shares the same sentence structure, please identify the most informative entities in the sentence and label them with the same label as the atomic fact while treating the rest of the entities as \texttt{True}. In the end, output the entities and labels in the following format:
                \begin{itemize}[nosep]
                    \item Entity 1 (Label 1)
                    \item Entity 2 (Label 2)
                    \item ...
                    \item Entity N (Label N)
                \end{itemize}
                % \newline \newline
                Here are two examples:
                \newline\newline
                \textbf{[Example 1]}
                \newline
                [The start of the biography]
                \newline
                \textcolor{titlecolor}{Marianne McAndrew is an American actress and singer, born on November 21, 1942, in Cleveland, Ohio. She began her acting career in the late 1960s, appearing in various television shows and films.}
                \newline
                [The end of the biography]
                \newline \newline
                [The start of the list of checked facts]
                \newline
                \textcolor{anscolor}{[Marianne McAndrew is an American. (False); Marianne McAndrew is an actress. (True); Marianne McAndrew is a singer. (False); Marianne McAndrew was born on November 21, 1942. (False); Marianne McAndrew was born in Cleveland, Ohio. (False); She began her acting career in the late 1960s. (True); She has appeared in various television shows. (True); She has appeared in various films. (True)]}
                \newline
                [The end of the list of checked facts]
                \newline \newline
                [The start of the ideal output]
                \newline
                \textcolor{labelcolor}{[Marianne McAndrew (True); is (True); an (True); American (False); actress (True); and (True); singer (False); , (True); born (True); on (True); November 21, 1942 (False); , (True); in (True); Cleveland, Ohio (False); . (True); She (True); began (True); her (True); acting career (True); in (True); the late 1960s (True); , (True); appearing (True); in (True); various (True); television shows (True); and (True); films (True); . (True)]}
                \newline
                [The end of the ideal output]
				\newline \newline
                \textbf{[Example 2]}
                \newline
                [The start of the biography]
                \newline
                \textcolor{titlecolor}{Doug Sheehan is an American actor who was born on April 27, 1949, in Santa Monica, California. He is best known for his roles in soap operas, including his portrayal of Joe Kelly on ``General Hospital'' and Ben Gibson on ``Knots Landing.''}
                \newline
                [The end of the biography]
                \newline \newline
                [The start of the list of checked facts]
                \newline
                \textcolor{anscolor}{[Doug Sheehan is an American. (True); Doug Sheehan is an actor. (True); Doug Sheehan was born on April 27, 1949. (True); Doug Sheehan was born in Santa Monica, California. (False); He is best known for his roles in soap operas. (True); He portrayed Joe Kelly. (True); Joe Kelly was in General Hospital. (True); General Hospital is a soap opera. (True); He portrayed Ben Gibson. (True); Ben Gibson was in Knots Landing. (True); Knots Landing is a soap opera. (True)]}
                \newline
                [The end of the list of checked facts]
                \newline \newline
                [The start of the ideal output]
                \newline
                \textcolor{labelcolor}{[Doug Sheehan (True); is (True); an (True); American (True); actor (True); who (True); was born (True); on (True); April 27, 1949 (True); in (True); Santa Monica, California (False); . (True); He (True); is (True); best known (True); for (True); his roles in soap operas (True); , (True); including (True); in (True); his portrayal (True); of (True); Joe Kelly (True); on (True); ``General Hospital'' (True); and (True); Ben Gibson (True); on (True); ``Knots Landing.'' (True)]}
                \newline
                [The end of the ideal output]
				\newline \newline
				\textbf{User prompt}
				\newline
				\newline
				[The start of the biography]
				\newline
				\textcolor{magenta}{\texttt{\{BIOGRAPHY\}}}
				\newline
				[The ebd of the biography]
				\newline \newline
				[The start of the list of checked facts]
				\newline
				\textcolor{magenta}{\texttt{\{LIST OF CHECKED FACTS\}}}
				\newline
				[The end of the list of checked facts]
			};
		\end{tikzpicture}
        \caption{GPT-4o prompt for labeling hallucinated entities.}\label{tb:gpt-4-prompt}
	\end{center}
\vspace{-0cm}
\end{table*}
% \section{Full Experiment Results}
% \begin{table*}[th]
    \centering
    \small
    \caption{Classification Results}
    \begin{tabular}{lcccc}
        \toprule
        \textbf{Method} & \textbf{Accuracy} & \textbf{Precision} & \textbf{Recall} & \textbf{F1-score} \\
        \midrule
        \multicolumn{5}{c}{\textbf{Zero Shot}} \\
                Zero-shot E-eyes & 0.26 & 0.26 & 0.27 & 0.26 \\
        Zero-shot CARM & 0.24 & 0.24 & 0.24 & 0.24 \\
                Zero-shot SVM & 0.27 & 0.28 & 0.28 & 0.27 \\
        Zero-shot CNN & 0.23 & 0.24 & 0.23 & 0.23 \\
        Zero-shot RNN & 0.26 & 0.26 & 0.26 & 0.26 \\
DeepSeek-0shot & 0.54 & 0.61 & 0.54 & 0.52 \\
DeepSeek-0shot-COT & 0.33 & 0.24 & 0.33 & 0.23 \\
DeepSeek-0shot-Knowledge & 0.45 & 0.46 & 0.45 & 0.44 \\
Gemma2-0shot & 0.35 & 0.22 & 0.38 & 0.27 \\
Gemma2-0shot-COT & 0.36 & 0.22 & 0.36 & 0.27 \\
Gemma2-0shot-Knowledge & 0.32 & 0.18 & 0.34 & 0.20 \\
GPT-4o-mini-0shot & 0.48 & 0.53 & 0.48 & 0.41 \\
GPT-4o-mini-0shot-COT & 0.33 & 0.50 & 0.33 & 0.38 \\
GPT-4o-mini-0shot-Knowledge & 0.49 & 0.31 & 0.49 & 0.36 \\
GPT-4o-0shot & 0.62 & 0.62 & 0.47 & 0.42 \\
GPT-4o-0shot-COT & 0.29 & 0.45 & 0.29 & 0.21 \\
GPT-4o-0shot-Knowledge & 0.44 & 0.52 & 0.44 & 0.39 \\
LLaMA-0shot & 0.32 & 0.25 & 0.32 & 0.24 \\
LLaMA-0shot-COT & 0.12 & 0.25 & 0.12 & 0.09 \\
LLaMA-0shot-Knowledge & 0.32 & 0.25 & 0.32 & 0.28 \\
Mistral-0shot & 0.19 & 0.23 & 0.19 & 0.10 \\
Mistral-0shot-Knowledge & 0.21 & 0.40 & 0.21 & 0.11 \\
        \midrule
        \multicolumn{5}{c}{\textbf{4 Shot}} \\
GPT-4o-mini-4shot & 0.58 & 0.59 & 0.58 & 0.53 \\
GPT-4o-mini-4shot-COT & 0.57 & 0.53 & 0.57 & 0.50 \\
GPT-4o-mini-4shot-Knowledge & 0.56 & 0.51 & 0.56 & 0.47 \\
GPT-4o-4shot & 0.77 & 0.84 & 0.77 & 0.73 \\
GPT-4o-4shot-COT & 0.63 & 0.76 & 0.63 & 0.53 \\
GPT-4o-4shot-Knowledge & 0.72 & 0.82 & 0.71 & 0.66 \\
LLaMA-4shot & 0.29 & 0.24 & 0.29 & 0.21 \\
LLaMA-4shot-COT & 0.20 & 0.30 & 0.20 & 0.13 \\
LLaMA-4shot-Knowledge & 0.15 & 0.23 & 0.13 & 0.13 \\
Mistral-4shot & 0.02 & 0.02 & 0.02 & 0.02 \\
Mistral-4shot-Knowledge & 0.21 & 0.27 & 0.21 & 0.20 \\
        \midrule
        
        \multicolumn{5}{c}{\textbf{Suprevised}} \\
        SVM & 0.94 & 0.92 & 0.91 & 0.91 \\
        CNN & 0.98 & 0.98 & 0.97 & 0.97 \\
        RNN & 0.99 & 0.99 & 0.99 & 0.99 \\
        % \midrule
        % \multicolumn{5}{c}{\textbf{Conventional Wi-Fi-based Human Activity Recognition Systems}} \\
        E-eyes & 1.00 & 1.00 & 1.00 & 1.00 \\
        CARM & 0.98 & 0.98 & 0.98 & 0.98 \\
\midrule
 \multicolumn{5}{c}{\textbf{Vision Models}} \\
           Zero-shot SVM & 0.26 & 0.25 & 0.25 & 0.25 \\
        Zero-shot CNN & 0.26 & 0.25 & 0.26 & 0.26 \\
        Zero-shot RNN & 0.28 & 0.28 & 0.29 & 0.28 \\
        SVM & 0.99 & 0.99 & 0.99 & 0.99 \\
        CNN & 0.98 & 0.99 & 0.98 & 0.98 \\
        RNN & 0.98 & 0.99 & 0.98 & 0.98 \\
GPT-4o-mini-Vision & 0.84 & 0.85 & 0.84 & 0.84 \\
GPT-4o-mini-Vision-COT & 0.90 & 0.91 & 0.90 & 0.90 \\
GPT-4o-Vision & 0.74 & 0.82 & 0.74 & 0.73 \\
GPT-4o-Vision-COT & 0.70 & 0.83 & 0.70 & 0.68 \\
LLaMA-Vision & 0.20 & 0.23 & 0.20 & 0.09 \\
LLaMA-Vision-Knowledge & 0.22 & 0.05 & 0.22 & 0.08 \\

        \bottomrule
    \end{tabular}
    \label{full}
\end{table*}




\end{document}



%%%%%%%%%%%%%%%%%%%%%%%%%%%%%%%%%%%%%%%%%%%%%%%%%%%%%%%%%%%%%%%%%%%%%%%%%%%%%%%
%%%%%%%%%%%%%%%%%%%%%%%%%%%%%%%%%%%%%%%%%%%%%%%%%%%%%%%%%%%%%%%%%%%%%%%%%%%%%%%
% APPENDIX
%%%%%%%%%%%%%%%%%%%%%%%%%%%%%%%%%%%%%%%%%%%%%%%%%%%%%%%%%%%%%%%%%%%%%%%%%%%%%%%
%%%%%%%%%%%%%%%%%%%%%%%%%%%%%%%%%%%%%%%%%%%%%%%%%%%%%%%%%%%%%%%%%%%%%%%%%%%%%%%
\newpage
\appendix
\onecolumn

\section{Algorithm}
\begin{algorithm}[H]
\caption{S2TX: State-Space Transformer With Cross Attention}
\textbf{Input:} Loss function $\mathcal{L}$, global model $g_{\phi}$, local model $f_{\psi}$, number of total training epochs $T$, dataset $D$, learning rate $\eta$, global patch length, stride, and window $PL_g$, $Str_g$, $L$, local patch length, stride, and window $PL_l$, $Str_l$, $S$. \\
\textbf{Output:} $\phi$, $\psi$.
\begin{algorithmic}[1]
\State Initialize parameters $\phi$, $\psi$.
\For{$i \gets 0$ to $T - 1$}
\State Shuffle dataset $D$.
    \For{each minibatch ($X,Y$)$\subset$ D} \comment{size}
        \State $\tilde{X}_g$, $\tilde{X}_l$ $\gets$ Patchify($X$;$PL_g$, $Str_g$,$L$), Patchify($X$;$PL_l$, $Str_l$, $S$)
        \State $Y_g \gets g_{\phi}(\tilde{X}_g)$
        \State $\hat{Y}\gets f_{\psi}(Y_g, \tilde{X}_l)$
        \State $\phi \gets \phi - \eta\nabla_{\phi}\mathcal{L}(\hat{Y},Y)$
        \State $\psi \gets \psi - \eta\nabla_{\psi}\mathcal{L}(\hat{Y},Y)$
    \EndFor
\EndFor
\State \textbf{return} $\phi$, $\psi$
\end{algorithmic}
\end{algorithm}

\section{Dataset Description}
\label{appendix:data_desc}

In this section, we describe the dataset used in our experiments in Table \ref{tab:performance}. Our experiments include 7 widely used real world multivariate time series. Table \ref{tab:datasummary} presents the number of variables and number of timesteps. 
\begin{itemize}
    \item The ETT dataset \citep{zhou2021informer} records 7 factors that related to electric transformers from July 2016 to July 2018. The ETT dataset includes 4 subsets where ETTh1 and ETTh2 are recorded hourly and ETTm1 and ETTm2 are recorded every 15 minutes. 
    \item The exchange dataset \citep{lai2018modeling} tracks the daily exchange rates of
    eight foreign countries including Australia, British, Canada,
    Switzerland, China, Japan, New Zealand, and Singapore ranging from 1990 to 2016.
    \item The weather dataset \citep{wu2021autoformer} includes 21 different meteorological features measured every 10 minutes by the Weather Station at the Max Planck Institute for Biogeochemistry.
    \item The ECL dataset \citep{lai2018modeling} records electricity consumption in kWh every 15 minutes from 2012 to 2014, for 321 clients. The data is converted to reflect hourly consumption.
\end{itemize}

\begin{table}[ht]
    \centering
    \footnotesize
    \renewcommand{\arraystretch}{1.2}
    \setlength{\tabcolsep}{8pt}
    \begin{tabular}{lccccccc}  % l = left, c = center, r = right
        \toprule
                   &ETTh1&ETTh2&ETTm1&ETTm2&Exchange& Weather & ECL\\
        \midrule
        \# Variables &7&7&7&7&8&21&321 \\
        \# Time steps & 17420& 17420 & 69680 & 69680 &  7588 & 52696 & 26304 \\
        \bottomrule
    \end{tabular}
    \caption{Table of Dataset summary including number of variables and number of time steps of each dataset. }
    \label{tab:datasummary}
\end{table}


\section{Comparison with more benchmark architectures}
\label{appendix:full_comparison}
In this section we present the full table of comparison that including two more baselines: Dlinear and FEDformer. Table \ref{tab:full_performance} is organized similarly as Table \ref{tab:performance}. 

\begin{table*}[htbp]

\centering
\renewcommand{\arraystretch}{1.1} % Adjust row height
\setlength{\tabcolsep}{3pt} % Adjust column spacing
\adjustbox{max width=\textwidth}{
\begin{tabular}{lllllllllllllllllllllll}
\toprule
 & \multicolumn{2}{c}{\textbf{S2TX}} & \multicolumn{2}{c}{\textbf{SST}} & \multicolumn{2}{c}{\textbf{S-Mamba}} & \multicolumn{2}{c}{\textbf{TimeM}} & \multicolumn{2}{c}{\textbf{iTrans}} & \multicolumn{2}{c}{\textbf{RLinear}} & \multicolumn{2}{c}{\textbf{PatchTST}} & \multicolumn{2}{c}{\textbf{CrossF}} & \multicolumn{2}{c}{\textbf{TimesNet}} & \multicolumn{2}{c}{\textbf{DLinear}} & \multicolumn{2}{c}{\textbf{FEDformer}} \\
 & \textbf{MSE} & \textbf{MAE} & \textbf{MSE} & \textbf{MAE} & \textbf{MSE} & \textbf{MAE} & \textbf{MSE} & \textbf{MAE} & \textbf{MSE} & \textbf{MAE} & \textbf{MSE} & \textbf{MAE} & \textbf{MSE} & \textbf{MAE} & \textbf{MSE} & \textbf{MAE} & \textbf{MSE} & \textbf{MAE} & \textbf{MSE} & \textbf{MAE} & \textbf{MSE} & \textbf{MAE}  \\ \midrule
\textbf{ETTh1} & & & & & & & & & & & & & & & & & & & & \\ 
96 &\textbf{0.376} &0.401&\underline{0.381} & 0.405 & 0.392 & \textbf{0.390} & 0.389 & 0.402 & 0.386 & 0.405 & 0.386 & \underline{0.395} & 0.414 & 0.419 & 0.423 & 0.448 & 0.384 & 0.402 & 0.386 & 0.400 & 0.376 & 0.419 \\ 
192 &\textbf{0.414}&\textbf{0.421}& \underline{0.430} & 0.434 & 0.449 & 0.439 & 0.435 & 0.440 & 0.441 & 0.436 & 0.437& \underline{0.424} & 0.460& 0.445 & 0.450 & 0.471 & 0.474 & 0.429 & 0.437 & 0.432 & \underline{0.420} & 0.448 \\ 
336&\textbf{0.432}&\textbf{0.435} & \underline{0.443} & \underline{0.446} & 0.467 & 0.481 & 0.450 & 0.448 & 0.487 & 0.458 & 0.479 & 0.446 & 0.501 & \underline{0.466} & 0.570 & 0.546 & 0.491 & 0.469 & 0.481 & 0.459 & 0.459 & 0.465 \\ 
720&\textbf{0.463}& \underline{0.473}& 0.502 & 0.501 & \underline{0.475} & 0.468 & 0.480 & \textbf{0.465} & 0.503 & 0.491 & 0.481 & 0.470 & 0.500 & 0.488 & 0.653 & 0.621 & 0.521 & 0.500 & 0.519 & 0.516 & 0.506 & 0.507 \\ \midrule
\textbf{ETTh2} & & & & & & & & & & & & & & & & & & & & \\ 
96&\textbf{0.279}& \underline{0.340}& 0.291 & 0.346 & 0.292 & 0.357 & 0.296 & 0.349 & 0.297 & 0.349 & \underline{0.288} & \textbf{0.338} & 0.302 & 0.348 & 0.745 & 0.584 & 0.340 & 0.374 & 0.333 & 0.387 & 0.358 & 0.397 \\ 
192&\textbf{0.362}&\underline{0.395} & \underline{0.369} & 0.397 & 0.380 & 0.402 & 0.371 & 0.400 & 0.380 & 0.400 & 0.374 & \textbf{0.390} & 0.388 & 0.400 & 0.877 & 0.656 & 0.402 & 0.414 & 0.477 & 0.476 & 0.429 & 0.439 \\ 
336&\textbf{0.337}& \textbf{0.385}& \underline{0.374} & \underline{0.414} & 0.391 & 0.420 & 0.402 & 0.449 & 0.428 & 0.432 & 0.415 & 0.426 & 0.426 & 0.433 & 1.043 & 0.731 & 0.452 & 0.452 & 0.594 & 0.541 & 0.496 & 0.487 \\ 
720&\textbf{0.395}&\textbf{0.430} & \underline{0.419} & 0.447 & 0.437 & 0.455 & 0.425 & \underline{0.438} & 0.427 & 0.445 & 0.420 & 0.440 & 0.431 & 0.446 & 1.104 & 0.763 & 0.462 & 0.468 & 0.831 & 0.657 & 0.463 & 0.474 \\ \midrule
\textbf{ETTm1} & & & & & & & & & & & & & & & & & & & & \\ 
96&\textbf{0.289}& \textbf{0.343}& \underline{0.298} & \underline{0.355} & 0.311 & 0.380 & 0.312 & 0.371 & 0.334 & 0.368 & 0.355 & 0.376 & 0.329 & 0.367 & 0.404 & 0.426 & 0.338 & 0.375 & 0.345 & 0.372 & 0.379 & 0.419 \\ 
192&\textbf{0.338}&\textbf{0.371} & \underline{0.347} & \underline{0.381} & 0.389 & 0.419 & 0.365 & 0.409 & 0.377 & 0.391 & 0.391 & 0.392 & 0.367 & 0.385 & 0.450 & 0.451 & 0.374 & 0.387 & 0.380 & 0.389 & 0.426 & 0.441 \\ 
336&\textbf{0.370}&\textbf{0.390} & \underline{0.374} & \underline{0.397} & 0.401 & 0.417 & 0.421 & 0.410 & 0.426 & 0.420 & 0.424 & 0.415 & 0.399 & 0.410 & 0.532 & 0.515 & 0.410 & 0.411 & 0.413 & 0.413 & 0.445 & 0.459 \\ 
720 &\textbf{0.423}&\textbf{0.418}& \underline{0.429} & \underline{0.428} & 0.488 & 0.476 & 0.496 & 0.437 & 0.491 & 0.459 & 0.487 & 0.450 & 0.454 & 0.439 & 0.666 & 0.589 & 0.478 & 0.450 & 0.474 & 0.453 & 0.543 & 0.490 \\ \midrule
\textbf{ETTm2} & & & & & & & & & & & & & & & & & & & & \\ 
96&\textbf{0.168}& \underline{0.260}& 0.176 & 0.264 & 0.191 & 0.301 & 0.185 & 0.290 & 0.180 & 0.264 & 0.182 & 0.265 & \underline{0.175} & \textbf{0.259} & 0.287 & 0.366 & 0.187 & 0.267 & 0.193 & 0.292 & 0.203 & 0.287 \\ 
192&\underline{0.235}&\textbf{0.298} & \textbf{0.231} & 0.303 & 0.253 & 0.312 & 0.292 & 0.309 & 0.250 & 0.309 & 0.246 & 0.304 & 0.241 & \underline{0.302} & 0.414 & 0.492 & 0.249 & 0.309 & 0.284 & 0.362 & 0.269 & 0.328 \\ 
336&\textbf{0.274}&\textbf{0.327} & \underline{0.290} & \underline{0.339} & 0.298 & 0.342 & 0.321 & 0.367 & 0.311 & 0.348 & 0.307 & 0.342 & 0.305 & 0.343 & 0.597 & 0.542 & 0.321 & 0.351 & 0.369 & 0.427 & 0.325 & 0.366 \\ 
720&\textbf{0.376}&\textbf{0.393}& \underline{0.388} & \underline{0.398} & 0.409 & 0.407 & 0.401 & 0.400 & 0.412 & 0.407 & 0.407 & 0.398 & 0.402 & 0.400 & 1.730 & 1.042 & 0.408 & 0.403 & 0.554 & 0.522 & 0.421 & 0.415 \\ \midrule
\textbf{Exchange} & & & & & & & & & & & & & & & & & & & & \\ 
96&\textbf{0.085} &\underline{0.205} &0.097 &0.222 &\underline{0.086}&0.206 & 0.089&0.208 &0.091 &0.211 & 0.088&0.209 &0.087 &\textbf{0.202} &0.095 &0.218 &0.093 & 0.211& 0.101& 0.223&0.105&0.226\\ 
192&\textbf{0.179} & \textbf{0.303}& 0.191&0.315 &0.182 &0.304 &0.184 &0.309 &0.182 & 0.303&0.188 & 0.311&\underline{0.180} &0.305 &0.193 &0.318 &0.194 &0.315 & 0.203&0.324&0.211&0.338 \\ 
336& \textbf{0.311}&\textbf{0.402} & 0.337&0.424 &0.330&0.416&0.333&0.416 &0.337 & 0.421&0.346 & 0.423& \underline{0.318}& \underline{0.407}&0.359 &0.429 &0.358 &0.433 & 0.369& 0.445&0.370&0.441\\ 
720& \textbf{0.858}&\textbf{0.696}&0.877 & 0.706&0.865 & 0.702& 0.870&\underline{0.701}&\underline{0.862} &0.703 & 0.913& 0.717&0.863 &0.703 &0.918 &0.721 &0.880 &0.719 &0.909 & 0.711&0.912&0.718\\\midrule
\textbf{Weather} & & & & & & & & & & & & & & & & & & & & \\ 
96&\textbf{0.150}& \textbf{0.199}& \underline{0.153} & \underline{0.205} & 0.169 & 0.221 & 0.174 & 0.218 & 0.174 & 0.214 & 0.192 & 0.232 & 0.177 & 0.218 & 0.158 & 0.230 & 0.172 & 0.220 & 0.196 & 0.255 & 0.217 & 0.296 \\ 
192&\textbf{0.194}&\textbf{0.242} & \underline{0.196} & \underline{0.244} & 0.205 & 0.248 & 0.200 & 0.258 & 0.221 & 0.254 & 0.240 & 0.271 & 0.225 & 0.259 & 0.206 & 0.277 & 0.219 & 0.261 & 0.237 & 0.296 & 0.276 & 0.336 \\ 
336&\underline{0.252}&\underline{0.288} & \textbf{0.246} & \textbf{0.283} & 0.288 & 0.299 & 0.280 & 0.299 & 0.278 & 0.296 & 0.292 & 0.307 & 0.278 & 0.297 & 0.272 & 0.335 & 0.280 & 0.306 & 0.283 & 0.335 & 0.339 & 0.380 \\ 
720&\textbf{0.313}&\textbf{0.333} & \underline{0.314} & \underline{0.334} & 0.335 & 0.369 & 0.352 & 0.359 & 0.358 & 0.347 & 0.364 & 0.353 & 0.354 & 0.348 & 0.398 & 0.418 & 0.365 & 0.359 & 0.345 & 0.381 & 0.403 & 0.428 \\ \midrule
\textbf{ECL} & & & & & & & & & & & & & & & & & & & & \\ 
96&\textbf{0.134}& \textbf{0.231}& \underline{0.141} & \underline{0.239} & 0.157 & 0.255 & 0.156 & 0.240 & 0.148 & 0.240 & 0.201 & 0.281 & 0.181 & 0.270 & 0.219 & 0.314 & 0.168 & 0.272 & 0.197 & 0.282 & 0.193 & 0.308 \\ 
192&\textbf{0.153}& \textbf{0.248}& \underline{0.159} & \underline{0.255} & 0.188 & 0.271 & 0.161 & 0.268 & 0.162 & 0.253 & 0.201 & 0.283 & 0.188 & 0.274 & 0.231 & 0.322 & 0.184 & 0.289 & 0.196 & 0.285 & 0.201 & 0.315 \\ 
336&\textbf{0.170}&\textbf{0.266} & \underline{0.171} & \underline{0.268} & 0.192 & 0.275 & 0.195 & 0.272 & 0.178 & 0.269 & 0.215 & 0.298 & 0.204 & 0.293 & 0.246 & 0.337 & 0.198 & 0.300 & 0.209 & 0.301 & 0.214 & 0.329 \\ 
720&\textbf{0.201}&\textbf{0.293} & \underline{0.208} & \underline{0.300} & 0.241 & 0.339 & 0.231 & 0.307 & 0.225 & 0.317 & 0.257 & 0.331 & 0.246 & 0.324 & 0.280 & 0.363 & 0.220 & 0.320 & 0.245 & 0.333 & 0.246 & 0.355 \\ \midrule
% Average& 0.304&0.349&0.315&
 \toprule
% Continue for Weather, ECL, and Traffic
\end{tabular}
}
\caption{Comprehensive comparison across various dataset with additional baselines. The \textbf{bolded} results denote the best performance, and the \underline{underlined} results indicate the second best.}
\label{tab:full_performance}
\end{table*}




\label{appendix:impl_detail}
%%%%%%%%%%%%%%%%%%%%%%%%%%%%%%%%%%%%%%%%%%%%%%%%%%%%%%%%%%%%%%%%%%%%%%%%%%%%%%%
%%%%%%%%%%%%%%%%%%%%%%%%%%%%%%%%%%%%%%%%%%%%%%%%%%%%%%%%%%%%%%%%%%%%%%%%%%%%%%%


\end{document}


% This document was modified from the file originally made available by
% Pat Langley and Andrea Danyluk for ICML-2K. This version was created
% by Iain Murray in 2018, and modified by Alexandre Bouchard in
% 2019 and 2021 and by Csaba Szepesvari, Gang Niu and Sivan Sabato in 2022.
% Modified again in 2023 and 2024 by Sivan Sabato and Jonathan Scarlett.
% Previous contributors include Dan Roy, Lise Getoor and Tobias
% Scheffer, which was slightly modified from the 2010 version by
% Thorsten Joachims & Johannes Fuernkranz, slightly modified from the
% 2009 version by Kiri Wagstaff and Sam Roweis's 2008 version, which is
% slightly modified from Prasad Tadepalli's 2007 version which is a
% lightly changed version of the previous year's version by Andrew
% Moore, which was in turn edited from those of Kristian Kersting and
% Codrina Lauth. Alex Smola contributed to the algorithmic style files.
