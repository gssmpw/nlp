%%%%%%%% ICML 2025 EXAMPLE LATEX SUBMISSION FILE %%%%%%%%%%%%%%%%%

\documentclass{article}

% Recommended, but optional, packages for figures and better typesetting:
\usepackage{microtype}
\usepackage{graphicx}
\usepackage{subcaption}
\usepackage{booktabs} % for professional tables
\usepackage{adjustbox}
\usepackage{algpseudocode}

% hyperref makes hyperlinks in the resulting PDF.
% If your build breaks (sometimes temporarily if a hyperlink spans a page)
% please comment out the following usepackage line and replace
% \usepackage{icml2025} with \usepackage[nohyperref]{icml2025} above.
\usepackage{hyperref}

% Attempt to make hyperref and algorithmic work together better:
\newcommand{\theHalgorithm}{\arabic{algorithm}}

% Use the following line for the initial blind version submitted for review:
% \usepackage{icml2025}

% If accepted, instead use the following line for the camera-ready submission:
\usepackage[accepted]{icml2025}

% For theorems and such
\usepackage{amsmath}
\usepackage{amssymb}
\usepackage{mathtools}
\usepackage{amsthm}
\usepackage{booktabs}
% if you use cleveref..
\usepackage[capitalize,noabbrev]{cleveref}

%%%%%%%%%%%%%%%%%%%%%%%%%%%%%%%%
% THEOREMS
%%%%%%%%%%%%%%%%%%%%%%%%%%%%%%%%
\theoremstyle{plain}
\newtheorem{theorem}{Theorem}[section]
\newtheorem{proposition}[theorem]{Proposition}
\newtheorem{lemma}[theorem]{Lemma}
\newtheorem{corollary}[theorem]{Corollary}
\theoremstyle{definition}
\newtheorem{definition}[theorem]{Definition}
\newtheorem{assumption}[theorem]{Assumption}
\theoremstyle{remark}
\newtheorem{remark}[theorem]{Remark}

% Todonotes is useful during development; simply uncomment the next line
%    and comment out the line below the next line to turn off comments
%\usepackage[disable,textsize=tiny]{todonotes}
\usepackage[textsize=tiny]{todonotes}

\newcommand{\JD}[1]{{\color{blue}[JD: #1]}}
\newcommand{\HM}[1]{{\color{red}[HM: #1]}}
% The \icmltitle you define below is probably too long as a header.
% Therefore, a short form for the running title is supplied here:
\icmltitlerunning{S2TX: cross-attention Multi-Scale State-Space Transformer for Time Series Forecasting}

\begin{document}

\twocolumn[
\icmltitle{S2TX: Cross-Attention Multi-Scale State-Space Transformer for Time Series Forecasting}

% It is OKAY to include author information, even for blind
% submissions: the style file will automatically remove it for you
% unless you've provided the [accepted] option to the icml2025
% package.

% List of affiliations: The first argument should be a (short)
% identifier you will use later to specify author affiliations
% Academic affiliations should list Department, University, City, Region, Country
% Industry affiliations should list Company, City, Region, Country

% You can specify symbols, otherwise they are numbered in order.
% Ideally, you should not use this facility. Affiliations will be numbered
% in order of appearance and this is the preferred way.
\icmlsetsymbol{equal}{*}

\begin{icmlauthorlist}
%\icmlauthor{Anonymous Author}{}
\icmlauthor{Zihao Wu}{yyy}
\icmlauthor{Juncheng Dong}{yyy,equal}
\icmlauthor{Haoming Yang}{yyy,equal}
\icmlauthor{Vahid Tarokh}{yyy}
% \icmlauthor{Firstname5 Lastname5}{yyy}
% \icmlauthor{Firstname6 Lastname6}{sch,yyy,comp}
% \icmlauthor{Firstname7 Lastname7}{comp}
%\icmlauthor{}{sch}
% \icmlauthor{Firstname8 Lastname8}{sch}
% \icmlauthor{Firstname8 Lastname8}{yyy,comp}
% \icmlauthor{}{sch}
% \icmlauthor{}{sch}
\end{icmlauthorlist}

\icmlaffiliation{yyy}{Department of Electrical and
Computer Engineering, Duke University, Durham, NC 27708, USA}
% \icmlaffiliation{comp}{Company Name, Location, Country}
% \icmlaffiliation{sch}{School of ZZZ, Institute of WWW, Location, Country}
% \icmlcorrespondingauthor{Anonymous Author}{}
\icmlcorrespondingauthor{Zihao Wu}{zihao.wu@duke.edu}
% You may provide any keywords that you
% find helpful for describing your paper; these are used to populate
% the "keywords" metadata in the PDF but will not be shown in the document
\icmlkeywords{Machine Learning, ICML}
\vskip 0.3in
]
% this must go after the closing bracket ] following \twocolumn[ ...

% This command actually creates the footnote in the first column
% listing the affiliations and the copyright notice.
% The command takes one argument, which is text to display at the start of the footnote.
% The \icmlEqualContribution command is standard text for equal contribution.
% Remove it (just {}) if you do not need this facility.

% \printAffiliationsAndNotice{}  % leave blank if no need to mention equal contribution
\printAffiliationsAndNotice{\icmlEqualContribution} % otherwise use the standard text.

\begin{abstract}

Time series forecasting has recently achieved significant progress with multi-scale models to address the heterogeneity between long and short range patterns. 
Despite their state-of-the-art performance, we identify two potential areas for improvement. 
First, the variates of the multivariate time series are processed independently. Moreover, the multi-scale (long and short range) representations are learned separately by two independent models without communication. In light of these concerns, we propose \emph{State Space Transformer with cross-attention} (S2TX). S2TX employs a cross-attention mechanism to integrate a Mamba model for extracting long-range cross-variate context and a Transformer model with local window attention to capture short-range representations. By cross-attending to the global context, the Transformer model further facilitates variate-level interactions as well as local/global communications. Comprehensive experiments on seven classic long-short range time-series forecasting benchmark datasets demonstrate that S2TX can achieve highly robust SOTA results while maintaining a low memory footprint.
\end{abstract}

\section{Introduction}
Forecasting multivariate time series represents a core learning paradigm designed to predict upcoming time steps using historical data. This machine learning task finds application across a range of domains including the economy \citep{koop2010bayesian}, epidemiology \citep{nguyen2021forecasting}, and meteorology \citep{angryk2020multivariate}. Due to its significant influence, multivariate time series forecasting has garnered considerable focus. State-of-the-art (SOTA) methods for multivariate time series forecasting predominantly utilize two types of sequence models: transformers and state-space models \cite{vaswani2017attention, gu2023mamba}. By employing these foundational structures, researchers aim to advance this research domain by harnessing two key features of multivariate time series: 1) identifying temporal dependencies and 2) understanding inter-variate correlations. Effectively integrating both temporal dynamics and the interactions between variates within a single learning model is essential for the precise forecasting of these intricate multivariate time series \citep{box2015time}. %\JD{citation?}.

%Significant progress has been made in multivariate forecasting with the introduction of the transformer architecture and the attention mechanism. 

%Another line of research, state-space models, has greatly improved the capabilities of deep learning models to retain long-range context. 


\begin{figure}[t]
    \centering
    \includegraphics[width=\linewidth]{./images/comparison.pdf} % Change to your image file
    \caption{Overview of the performance of different architectures over 7 different benchmark datasets. Average results (MSE) are reported. }
    \label{fig:overview}
\end{figure}

A recent advancement~\citet{xusst} integrates transformers and state-space models within a multi-scale framework: it first breaks down the input time series into shorter high resolution \emph{patches} and longer low resolution patches. Subsequently, it feeds the high resolution patches into a transformer model with local-attention to extract fine-grained local features and the low resolution longer patches to a state-space model (i.e., Mamba~\cite{gu2023mamba}) to learn long-range global features.
% a sequence is broken down into shorter high resolution time-series patches and fed to a transformer to leverage its ability to extract fine-grained local features, while the low resolution longer patches are inputted to the state-space model to learn long-range global features. 
This multi-scale mixture of Mamba and transformer models greatly improves the modeling of temporal dependencies. However, it leaves a crucial aspect of multivariate time series forecasting unattended, that is, the correlation between variates. 
Additionally, the local and global features are modeled independently, which overlooks the interplay between global and local features. Such global-local interplay is manifested in many real-world scenarios. For example, the commonly known \emph{El Ninõ} effect is a global, long-term weathering effect in the time-scale of years; but this global weather pattern will greatly affect the short-term local time series within days \citep{hsieh2004nonlinear}. 

The cross-variate correlation and global-local features interplay, illustrated in Figure \ref{fig:cross-all-correlation}, are two crucial aspects of multivariate time series forecasting. Global patterns encompassed in the purple boxes consistently suggest increased local variation while the red-boxed region indicates a strongly inversed correlation between the two variates. To explicitly include these two crucial aspects, we introduce \emph{State Space Transformer with Cross-attention} (S2TX) where we connect cross-variate global features with fine grained local features through a carefully designed cross-attention mechanism. Specifically, we apply Mamba as the global model to process long-range, low-resolution patches across all variates in a single sweep, extracting cross-variate global context. This global context is then provided as the key and value for cross-attention to a decoder-like transformer model focusing on local, high-resolution, variate-independent patches.

\textbf{Contributions.} Our contributions are summarized below:
\begin{itemize}
    \item We identify two crucial aspects, cross-variate correlation and global-local interaction,  to improve the SOTA time series forecasting model.
    \item We propose a novel multi-scale architecture that incorporates these considerations through a cross-attention mechanism. In particular, our architecture learns variate-level correlation while leveraging the enhanced temporal learning of patchification. 
    \item We develop a cross-attention mixture of experts, enabling global-local feature interplay between a global feature-focused state-space model and a local feature-focused transformer model.
    \item We verify the efficacy of our proposed architecture on a comprehensive set of time series forecasting benchmarks.
\end{itemize}
%Recent studies have shown that processing patched time series data is more effective than elementwise processing in capturing temporal dependencies, as it introduces an inductive bias that aligns with the localized nature of time series data. Furthermore, it has been observed that using multi-scale patches enables models to learn distinct global and local features. 

%In many real-world scenarios, as illustrated in the figure, the significance lies not only in the features themselves but also in the interplay between global and local features. However, existing structures often lack an explicit mechanism for modeling this correlation. To address this, we propose using a cross-attention mechanism to explicitly model the interaction between global and local features. 

%While patch technique facilitates the capture of temporal dependencies, it transforms the time series of each variable into a multivariate time series, making it more challenging to effectively model interactions between variables. To address this, we propose leveraging Mamba as the global model to process long-range patches across all variables in a single sweep, extracting cross-variable global context. This global context is then provided as the key and value to the decoder-like transformer local model within a cross-attention mechanism. By utilizing Mamba as the global model, we take advantage of its linear time complexity and its capability to handle long series, even when the dimensionality of the data is high.

\begin{figure}[t]
    \centering
    \includegraphics[width=0.8\linewidth]{./images/sample.pdf} % Change to your image file
    \caption{A snippet of the weather dataset. Two variables (blue and green) were plotted over 720 time steps. The purple boxed region indicates where a global-local interaction exists, and the red boxed region indicates a cross-variate correlation. }
    \label{fig:cross-all-correlation}
\end{figure}

\section{Related Works}
The field of time series forecasting has seen significant evolution over the decades: shifting from classical mathematical tools \citep{bloomfield2004fourier, durbin2012time} and statistical techniques like ARIMA \citep{nerlove1971time, hyndman2018forecasting} to more recent deep learning approaches such as recurrent neural networks \citep{graves2013speech} and long-short term memory models \citep{gers2000learning}. Notably, in recent years, transformers \citep{vaswani2017attention} have demonstrated particularly promising performance on sequence modeling tasks, especially in natural language processing. Interestingly, studies have revealed that even simple linear layers can outperform complex transformer-based models in both performance and efficiency for time series forecasting \citep{zeng2023transformers, yang2024neural}.

% Making the subsections shorter so the related work can contain more paper
\textbf{Inverted Dimension.}
In investigating why transformers underperform in time series forecasting, \citet{liu2023itransformer} argues that the direct application of transformers that embed all variates is undesirable. This embedding compresses variates with distinct physical meanings and inconsistent measurement at each time step to a single token, erasing the important multivariate correlations. To address this limitation, the authors propose inverting the dimension of time and variates in the data while preserving the core mechanisms of the transformer. Many subsequent studies \citep{wang2025mamba, ahamed2024timemachine, xusst} build upon this paradigm, achieving improvements in both performance and efficiency. %This innovative approach effectively unlocks the potential of transformers for time series forecasting.
% Building on this innovative paradigm, subsequent works further refined the approach by replacing the Transformer with Mamba-based models, a promising alternative that offers linear training complexity.

\textbf{Patchification.}
Patchification of inverted data transforms the time series of each variate into a multivariate time sequence where the patches are stacked to construct an additional dimension. While patchification facilitates the capture of temporal dependencies by introducing an inductive bias aligned with the localized nature of time series data, it also overlooks the between-variate correlations due to the additional dimension: existing approaches, such as SST \citep{xusst} and PatchTST \citep{nie2022time}, treat each variate independently. Despite their strong performance, these methods lack any form of inter-variate communication. 
% Taking this out for now, maybe we can use this in 4.2? 
%Other methods like MOIRAI flatten all variables into a single sequence before patchification, but the extended sequence length, particularly in high-dimension, in addition to the quadratic complexity of the transformer, imposes a heavy computation burden.

\textbf{Mixture of Experts.}
The mixture of experts method receives increasing attention in sequence modeling after the release of Mamba \citep{gu2023mamba}. Combining the linear complexity of Mamba and the strong performance of transformers could lead to efficient and accurate sequence models. For instance, Jamba \citep{lieber2024jamba} employs a layerwise stacking of Mamba and attention layers, achieving superior performance in natural language processing compared to either component individually. For time series forecasting, SST \citep{xusst} utilizes Mamba to capture global patterns with prolonged patch lengths, while leveraging transformer to learn local details with shorter patch lengths. However, global and local patches are processed separately through each expert before their output embeddings are concatenated. Such inadequate communication between global and local features limits the integration results, restricting the model's ability to fully exploit each expert's complementary strength. 

\begin{figure*}[t]
    \centering
    \includegraphics[width=1\textwidth]{./images/structure.pdf} % Change to your image file
    \caption{Overview of the proposed architecture S2TX. Different variables (in different colors) of the time series are patched into global and local patches. The global patches are processed by the global model, which outputs the global context that is used to compute the key and value matrices during cross-attention with the local model. Skip connections and normalization layers are omitted for clarity of presentation.}
    \label{fig:structure}
\end{figure*}
%\textbf{Our Approach}
%\HM{Maybe we can take this out, seems redundant.}
%As we clarified the different aspects of multivariate time series forecasting, the goal is clear: with S2TX, we develop an architecture that combines the advantages of all three approaches. S2TX is a low memory architecture that maintains the learning of multivariate correlation while enabling global-local feature interplay through a cross-attentional mixture of experts.

\section{Preliminary}
\label{sec:preliminary}
In this section, we first formalize the modeling problem, then introduce the two main modules of our proposed architecture: state-space models and cross-attention. 

\subsection{Problem Setup}
\label{sec:problemSetup}
We consider the standard problem setup for time series forecasting framework \citep{liu2023itransformer}. Given a $D$-dimensional multivariate time series of length $L$ (look-back window) $\textbf{X}\in \mathbb{R}^{D\times L}$,  the goal is to predict $\textbf{Y}\in \mathbb{R}^{D\times H}$, the same $D$-dimensional multivariate time series in the future $H$ steps (horizon length). Assuming we have access to a training dataset with $N$ observations $\{\textbf{X}^{(i)},\textbf{Y}^{(i)}\}_{i=1}^N$, our goal is to learn a function $f_\phi(\textbf{X}^{(i)}): \mathbb{R}^{D\times L} \rightarrow \mathbb{R}^{D\times H}$ with parameter $\phi$ such that the mean squared error loss is minimized:
\begin{align}
    \mathcal{L}_{\mathrm{train}} = \frac{1}{N}\sum_{n=1}^{N}\|f_\phi(\textbf{X}^{(i)}) - \textbf{Y}^{(i)}\|_F^2,
\end{align}
where $F$ denotes the Frobenius norm~\citep{horn2012matrix}.

\subsection{State-Space Models}
\label{sec:SSM}
State-Space Models (SSMs) \citep{gu2020hippo, gu2021efficiently} are a family of sequence models inspired by continuous control systems described by the following equations
\begin{align}
    \mathrm{d}h = Ah + Bx,\ z = Ch + Dx,
\end{align}
where $x\in \mathbb{R}$ represents a one-dimensional input, $h\in \mathbb{R}^{d\times 1}$ is the hidden state, $z$ is the model output, $A\in \mathbb{R}^{d\times d}$, $B\in \mathbb{R}^{d\times 1}$, $C\in \mathbb{R}^{1\times d}$, and $D\in \mathbb{R}^{1\times 1}$ are parameter matrices. Matrix $D$ acts as a skip connection and is typically omitted in derivations. For multi-dimensional inputs, a stack of SSMs is employed.
The continuous system is then discretized into 
\begin{align}
    h_{t+1} = \bar{A}h_t + \bar{B}x_t,\ z_{t+1} = \bar{C}h_t,
\end{align}
where the discretized matrices are obtained with a discretization rule and a step size $\Delta$. For example, Mamba \citep{gu2023mamba} uses Zero-Order Holder rule such that $\bar{A} = \exp(\Delta A),\ 
\bar{B} = \exp(\Delta A)^{-1} (\exp(\Delta A) - \mathbb{I}) \cdot \Delta B$.

The discretized state-space models can be interpreted either as a convolutional neural network, enabling linear-time parallel training, or as a linear recurrent neural network, supporting constant-time inference, as demonstrated in S4 \citep{gu2021efficiently}. Building upon S4, Mamba extends this approach by making the matrices $B$ and $C$ input-dependent, transforming them into a selective SSM. Additionally, Mamba introduces a parallel scan algorithm to achieve linear-time training complexity.

\subsection{Cross-attention}
\label{sec:crossAttentioin}
Cross-attention is a generalization \citep{bahdanau2014neural} of self-attention \citep{vaswani2017attention}. 
% It was initially used in the decoder part of the transformer and later extended to facilitate cross-model interaction. 
Given source data $S\in \mathbb{R}^{L_S\times d_{\mathrm{model}}}$ and target data $T\in \mathbb{R}^{L_T\times d_\text{model}}$, the output of cross-attention is
\begin{align}
    \text{CrossAttention}(S,T) = \frac{(TW_q)(SW_k)^T}{\sqrt{d_\text{model}}}SW_v %\in \mathbb{R}^{L_T \times d_{\text{model}}},
\end{align}
where $W_k$, $W_q$, $W_v\in \mathbb{R}^{d_\text{model}\times d_\text{model}}$ are learnable parameters. From this perspective, self-attention can be achieved by substituting all instances of $S$ with $T$ in Cross-attention:
\begin{align}
    \text{SelfAttention}(T) = \text{CrossAttention}(T,T) %\in \mathbb{R}^{L_T \times d_{\mathrm{model}}}.
\end{align}
This cross-attention mechanism will allow us to compute cross-attentional weight where the influence of global patterns is weighted to predict a specific local pattern, allowing global-local interaction during inference. Notably, our application of cross-attention mechanism to integrate multi-scale features are commonly applied in computer vision tasks \citep{chen2021crossvit}.


\section{State-Space Transformer with Cross-attention}
\label{sec:proposedApproach}
Here we describe our proposed method State-Space Transformer with Cross-attention (S2TX). We first introduce the Multi-Scale patching process that decompose a long time series into global and local patches of different time scales. The low-resolution global patches were then fed to a Mamba-based global feature extractor to obtain the cross-variate global context. The global context is then applied as the key and value for a novel global-local cross-attention to improve the extraction of local features. Finally, we conduct a computation complexity analysis, showcasing that S2TX, with the addition of our novel cross-attention, maintained a low-memory footprint during training and inference. The general structure of S2TX is provided in Figure \ref{fig:structure}.

\begin{figure}[thbp]
    \centering
    \includegraphics[width=0.45\textwidth]{./images/patch.pdf} % Change to your image file
    \caption{Patch transforms a one-dimensional sequence to a sequence of patches.}
    \label{fig:patch}
\end{figure}

\subsection{Multi-Scale Patch}
\label{sec:multiscalePatch}
The patching technique has become increasingly popular for time series forecasting~\citep{gong2023patchmixer, nie2022time,xusst}. It aggregates local information into patches and effectively enhances the receptive field. Denote the sequence length of the look-back window by $L$, patch length by $PL$, stride by $STR$, and patch number by $PN$, where 
\begin{align}\label{eq:pn}
    PN = \left\lceil\frac{L-PL}{STR}\right\rceil. 
\end{align}
The patching technique transforms each (one-dimensional) variate of length $L$ into a $PL$-dimensional time series of length $PN$. More specifically, the input time series $\textbf{X}\in \mathbb{R}^{D\times L}$ is patched into $\mathbf{\tilde{X}}\in \mathbb{R}^{D\times PN\times PL}$. 

Intuitively the longer the stride, or the longer the patch length, the more long range temporal context is stored in a patch and vice versa. Utilizing this intuition, we apply the patching process onto the time series \emph{twice}: (i) one of them focuses on coarser granularity for global context, 
%Moreover, SST highlights that global patterns are more discernible at a coarser granularity, while local variations are revealed at a finer granularity. Following a similar approach, 
employing the full look-back window of length $L$, a larger patch length $PL_g$ and longer stride, along with the corresponding patch number $PN_g$ to obtain long-range global time series patches; (ii) the other leverages finer granularity with a fixed shorter look-back window of length $S$, a smaller patch length $PL_l$, and shorter stride with corresponding patch number $PN_l$ to obtain short-range local patches. The resulting two multi-scale time series patches $\mathbf{\tilde{X}}_g\in \mathbb{R}^{D\times PN_g\times PL_g}$ and $\mathbf{\tilde{X}}_l\in \mathbb{R}^{D\times PN_l\times PL_l}$ serve as inputs for the global and local models, respectively.
\begin{table*}[t]
\footnotesize
\centering
\renewcommand{\arraystretch}{0.9} % Adjust row height
\setlength{\tabcolsep}{3pt} % Adjust column spacing
\adjustbox{max width=\textwidth}{
\begin{tabular}{%lllllllllllllllllllllll
                lcccccccccccccccccccccc}
\toprule
 & \multicolumn{2}{c}{\textbf{S2TX}} & \multicolumn{2}{c}{\textbf{SST}} & \multicolumn{2}{c}{\textbf{S-Mamba}} & \multicolumn{2}{c}{\textbf{TimeM}} & \multicolumn{2}{c}{\textbf{iTrans}} & \multicolumn{2}{c}{\textbf{RLinear}} & \multicolumn{2}{c}{\textbf{PatchTST}} & \multicolumn{2}{c}{\textbf{CrossF}} & \multicolumn{2}{c}{\textbf{TimesNet}}  \\
 & \multicolumn{2}{c}{\text{2025}} & \multicolumn{2}{c}{\text{2025}} & \multicolumn{2}{c}{\text{2025}} & \multicolumn{2}{c}{\text{2024}} & \multicolumn{2}{c}{\text{2024}} & \multicolumn{2}{c}{\text{2024}} & \multicolumn{2}{c}{\text{2023}} & \multicolumn{2}{c}{\text{2023}} & \multicolumn{2}{c}{\text{2023}}  \\
 & \textbf{MSE} & \textbf{MAE} & \textbf{MSE} & \textbf{MAE} & \textbf{MSE} & \textbf{MAE} & \textbf{MSE} & \textbf{MAE} & \textbf{MSE} & \textbf{MAE} & \textbf{MSE} & \textbf{MAE} & \textbf{MSE} & \textbf{MAE} & \textbf{MSE} & \textbf{MAE} & \textbf{MSE} & \textbf{MAE}  \\ \midrule
\textbf{ETTh1} & & & & & & & & & & & & & & & & \\ 
96 &\textbf{0.376} &0.401&\underline{0.381} & 0.405 & 0.392 & \textbf{0.390} & 0.389 & 0.402 & 0.386 & 0.405 & 0.386 & \underline{0.395} & 0.414 & 0.419 & 0.423 & 0.448 & 0.384 & 0.402  \\ 
192 &\textbf{0.414}&\textbf{0.421}& \underline{0.430} & 0.434 & 0.449 & 0.439 & 0.435 & 0.440 & 0.441 & 0.436 & 0.437& \underline{0.424} & 0.460& 0.445 & 0.450 & 0.471 & 0.474 & 0.429\\ 
336&\textbf{0.432}&\textbf{0.435} & \underline{0.443} & \underline{0.446} & 0.467 & 0.481 & 0.450 & 0.448 & 0.487 & 0.458 & 0.479 & 0.446 & 0.501 & \underline{0.466} & 0.570 & 0.546 & 0.491 & 0.469  \\ 
720&\textbf{0.463}& \underline{0.473}& 0.502 & 0.501 & \underline{0.475} & 0.468 & 0.480 & \textbf{0.465} & 0.503 & 0.491 & 0.481 & 0.470 & 0.500 & 0.488 & 0.653 & 0.621 & 0.521 & 0.500 \\ \midrule
\textbf{ETTh2} & & & & & & & & & & & & & & & & \\ 
96&\textbf{0.279}& \underline{0.340}& 0.291 & 0.346 & 0.292 & 0.357 & 0.296 & 0.349 & 0.297 & 0.349 & \underline{0.288} & \textbf{0.338} & 0.302 & 0.348 & 0.745 & 0.584 & 0.340 & 0.374 \\ 
192&\textbf{0.362}&\underline{0.395} & \underline{0.369} & 0.397 & 0.380 & 0.402 & 0.371 & 0.400 & 0.380 & 0.400 & 0.374 & \textbf{0.390} & 0.388 & 0.400 & 0.877 & 0.656 & 0.402 & 0.414 \\ 
336&\textbf{0.337}& \textbf{0.385}& \underline{0.374} & \underline{0.414} & 0.391 & 0.420 & 0.402 & 0.449 & 0.428 & 0.432 & 0.415 & 0.426 & 0.426 & 0.433 & 1.043 & 0.731 & 0.452 & 0.452 \\ 
720&\textbf{0.395}&\textbf{0.430} & \underline{0.419} & 0.447 & 0.437 & 0.455 & 0.425 & \underline{0.438} & 0.427 & 0.445 & 0.420 & 0.440 & 0.431 & 0.446 & 1.104 & 0.763 & 0.462 & 0.468  \\ \midrule
\textbf{ETTm1} & & & & & & & & & & & & & & & &  \\ 
96&\textbf{0.289}& \textbf{0.343}& \underline{0.298} & \underline{0.355} & 0.311 & 0.380 & 0.312 & 0.371 & 0.334 & 0.368 & 0.355 & 0.376 & 0.329 & 0.367 & 0.404 & 0.426 & 0.338 & 0.375 \\ 
192&\textbf{0.338}&\textbf{0.371} & \underline{0.347} & \underline{0.381} & 0.389 & 0.419 & 0.365 & 0.409 & 0.377 & 0.391 & 0.391 & 0.392 & 0.367 & 0.385 & 0.450 & 0.451 & 0.374 & 0.387  \\ 
336&\textbf{0.370}&\textbf{0.390} & \underline{0.374} & \underline{0.397} & 0.401 & 0.417 & 0.421 & 0.410 & 0.426 & 0.420 & 0.424 & 0.415 & 0.399 & 0.410 & 0.532 & 0.515 & 0.410 & 0.411 \\ 
720 &\textbf{0.423}&\textbf{0.418}& \underline{0.429} & \underline{0.428} & 0.488 & 0.476 & 0.496 & 0.437 & 0.491 & 0.459 & 0.487 & 0.450 & 0.454 & 0.439 & 0.666 & 0.589 & 0.478 & 0.450  \\ \midrule
\textbf{ETTm2} & & & & & & & & & & & & & & & & \\ 
96&\textbf{0.168}& \underline{0.260}& 0.176 & 0.264 & 0.191 & 0.301 & 0.185 & 0.290 & 0.180 & 0.264 & 0.182 & 0.265 & \underline{0.175} & \textbf{0.259} & 0.287 & 0.366 & 0.187 & 0.267  \\ 
192&\underline{0.235}&\textbf{0.298} & \textbf{0.231} & 0.303 & 0.253 & 0.312 & 0.292 & 0.309 & 0.250 & 0.309 & 0.246 & 0.304 & 0.241 & \underline{0.302} & 0.414 & 0.492 & 0.249 & 0.309 \\ 
336&\textbf{0.274}&\textbf{0.327} & \underline{0.290} & \underline{0.339} & 0.298 & 0.342 & 0.321 & 0.367 & 0.311 & 0.348 & 0.307 & 0.342 & 0.305 & 0.343 & 0.597 & 0.542 & 0.321 & 0.351  \\ 
720&\textbf{0.376}&\textbf{0.393}& \underline{0.388} & \underline{0.398} & 0.409 & 0.407 & 0.401 & 0.400 & 0.412 & 0.407 & 0.407 & 0.398 & 0.402 & 0.400 & 1.730 & 1.042 & 0.408 & 0.403  \\ \midrule
\textbf{Exchange} & & & & & & & & & & & & & & & & \\ 
96&\textbf{0.085} &\underline{0.205} &0.097 &0.222 &\underline{0.086}&0.206 & 0.089&0.208 &0.091 &0.211 & 0.088&0.209 &0.087 &\textbf{0.202} &0.095 &0.218 &0.093 & 0.211\\ 
192&\textbf{0.179} & \textbf{0.303}& 0.191&0.315 &0.182 &0.304 &0.184 &0.309 &0.182 & 0.303&0.188 & 0.311&\underline{0.180} &0.305 &0.193 &0.318 &0.194 &0.315  \\ 
336& \textbf{0.311}&\textbf{0.402} & 0.337&0.424 &0.330&0.416&0.333&0.416 &0.337 & 0.421&0.346 & 0.423& \underline{0.318}& \underline{0.407}&0.359 &0.429 &0.358 &0.433 \\ 
720& \textbf{0.858}&\textbf{0.696}&0.877 & 0.706&0.865 & 0.702& 0.870&\underline{0.701}&\underline{0.862} &0.703 & 0.913& 0.717&0.863 &0.703 &0.918 &0.721 &0.880 &0.719 \\\midrule
\textbf{Weather} & & & & & & & & & & & & & & & &  \\ 
96&\textbf{0.150}& \textbf{0.199}& \underline{0.153} & \underline{0.205} & 0.169 & 0.221 & 0.174 & 0.218 & 0.174 & 0.214 & 0.192 & 0.232 & 0.177 & 0.218 & 0.158 & 0.230 & 0.172 & 0.220 \\ 
192&\textbf{0.194}&\textbf{0.242} & \underline{0.196} & \underline{0.244} & 0.205 & 0.248 & 0.200 & 0.258 & 0.221 & 0.254 & 0.240 & 0.271 & 0.225 & 0.259 & 0.206 & 0.277 & 0.219 & 0.261  \\ 
336&\underline{0.252}&\underline{0.288} & \textbf{0.246} & \textbf{0.283} & 0.288 & 0.299 & 0.280 & 0.299 & 0.278 & 0.296 & 0.292 & 0.307 & 0.278 & 0.297 & 0.272 & 0.335 & 0.280 & 0.306  \\ 
720&\textbf{0.313}&\textbf{0.333} & \underline{0.314} & \underline{0.334} & 0.335 & 0.369 & 0.352 & 0.359 & 0.358 & 0.347 & 0.364 & 0.353 & 0.354 & 0.348 & 0.398 & 0.418 & 0.365 & 0.359 \\ \midrule
\textbf{ECL} & & & & & & & & & & & & & & & &  \\ 
96&\textbf{0.134}& \textbf{0.231}& \underline{0.141} & \underline{0.239} & 0.157 & 0.255 & 0.156 & 0.240 & 0.148 & 0.240 & 0.201 & 0.281 & 0.181 & 0.270 & 0.219 & 0.314 & 0.168 & 0.272 \\ 
192&\textbf{0.153}& \textbf{0.248}& \underline{0.159} & \underline{0.255} & 0.188 & 0.271 & 0.161 & 0.268 & 0.162 & 0.253 & 0.201 & 0.283 & 0.188 & 0.274 & 0.231 & 0.322 & 0.184 & 0.289  \\ 
336&\textbf{0.170}&\textbf{0.266} & \underline{0.171} & \underline{0.268} & 0.192 & 0.275 & 0.195 & 0.272 & 0.178 & 0.269 & 0.215 & 0.298 & 0.204 & 0.293 & 0.246 & 0.337 & 0.198 & 0.300  \\ 
720&\textbf{0.201}&\textbf{0.293} & \underline{0.208} & \underline{0.300} & 0.241 & 0.339 & 0.231 & 0.307 & 0.225 & 0.317 & 0.257 & 0.331 & 0.246 & 0.324 & 0.280 & 0.363 & 0.220 & 0.320 \\ %\midrule
% Average& 0.304&0.349&0.315&
 \bottomrule
% Continue for Weather, ECL, and Traffic
\end{tabular}
}
\caption{Comprehensive comparison across various datasets, prediction horizons, and baselines. The \textbf{bolded} results denote the best performance, and the \underline{underlined} results indicate the second best.}
\label{tab:performance}
\end{table*}
\subsection{Cross-Variate Global Context}
\label{sec:CVGlobal}
%When analyzing multivariate time-series data, the human brain naturally identifies and compares global patterns across variables first, storing this information to help focus on local details later. 
%\textcolor{red}{Need a complete overhaul}
The global patches $\mathbf{\tilde{X}}_g$ is first passed through the global feature extractor, which is a dual Mamba system, responsible for cross-variate global feature extraction. %This inspires our approach, which
We begin by concatenating along the first and second dimension of $\mathbf{\tilde{X}}_g$, viewed with a new shape $\mathbf{\tilde{X}}_g\in \mathbb{R}^{(D* PN_g)\times PL_g}$ as illustrated in Figure \ref{fig:patch}. This allows the learning of variate-level correlation across all $D$ dimensions as the selection mechanism of Mamba will filter the relevant variates and patches, enabling the global model to capture cross-variate global context. However, Mamba processes data unilaterally, attending only to antecedent patches, which limits learning of the full global context. Inspired by S-Mamba \citep{wang2025mamba}, we employ two Mamba models to scan the sequence in both forward and backward directions before aggregating the results. This approach improves the learning of correlations between global patches across variables. Specifically, we have 
\begin{align}
    \overrightarrow {\mathbf{Z}_g} &= \overrightarrow{\text{Mamba Layers}} (\mathbf{\tilde{X}}_g),\\
    \overleftarrow {\mathbf{Z}_g} &= \overleftarrow{\text{Mamba Layers}} (\overleftarrow{\mathbf{\tilde{X}}}_g),\\
    \mathbf{Z}_g &= \overrightarrow {\mathbf{Z}_g}+\overleftarrow {\mathbf{Z}_g},
\end{align}
where $\overleftarrow{\mathbf{\tilde{X}}}_g \in \mathbb{R}^{(D*PN_g)\times PL_g}$ is obtained by reversing the the first dimension of $\mathbf{\tilde{X}}_g$. The output of the global model  $\mathbf{Z}_g\in \mathbb{R}^{(D*PN_g)\times d_\text{model}}$, which serves as an intermediary output of the entire architecture, is then fed to the local model. This intermediary output encapsulates both cross-variate and global context information. %patches from all variables to be combined into a single sequence, as illustrated in the figure. In order to facilitate communication across variables, we have prolonged the sequence length of the patched time series by factor of $D$. 

%However, the computational load of global attention escalates exponentially with the increased sequence length, rendering transformer-based models impractical for large dimensional data. On the other hand, the selective mechanism of Mamba can discern the significance of each patch akin to attention (cite) but with a computational overhead only escalating in a near-linear fashion. Therefore, employing Mamba as the global model and scanning the sequence of patches for all variables in one sweep becomes efficient. 

Note that $d_\text{model}$ represents the model dimension of Mamba, which aligns with the model dimension of the local model discussed in the next section.


\begin{figure*}[t]
    \centering
    % Subplot 1
    \begin{subfigure}[t]{0.24\textwidth} % 24% of the width
        \centering
        \includegraphics[width=\textwidth]{./images/S2TX_line.png} % Replace with your image
        % \caption{Sine Function}
        \label{fig:sine}
    \end{subfigure}
    % Subplot 2
    \begin{subfigure}[t]{0.24\textwidth}
        \centering
        \includegraphics[width=\textwidth]{./images/SST_line.png} % Replace with your image
        % \caption{Cosine Function}
        \label{fig:cosine}
    \end{subfigure}
    % Subplot 3
    \begin{subfigure}[t]{0.24\textwidth}
        \centering
        \includegraphics[width=\textwidth]{./images/iTransformer_line.png} % Replace with your image
        % \caption{Tangent Function}
        \label{fig:tangent}
    \end{subfigure}
    % Subplot 4
    \begin{subfigure}[t]{0.24\textwidth}
        \centering
        \includegraphics[width=\textwidth]{./images/iMamba_line.png} % Replace with your image
        % \caption{Exponential Decay}
        \label{fig:exponential}
    \end{subfigure}
    \caption{Empirical time series versus predicted time series across different architecture. S2TX can better capture the variation of the variable over time. }
    \label{fig:predictTScompare}
\end{figure*}
\subsection{Cross-Attention Local Context}
\label{sec:CALocal}
%\HM{Notation in this subsection is too cluttered, need to somehow simplify.}
With global and cross-variate patterns as context information, the local model can more effectively capture local features and interpret local variations. To this end, we employ a decoder-like transformer with each layer composed of a self-attention without causal masking followed by a cross-attention. Since cross-variate correlation is already captured by the context features, we now take each variate (in the first dimension) of $\mathbf{\tilde{X}}_l\in \mathbb{R}^{D\times PN_l\times PL_l}$ individually as the input of the self-attention to relieve the computation burden of transformer. Denote the $d$-th variate of $\mathbf{\tilde{X}}_l$ after linear projection to $d_{\text{model}}$-dimension by $\mathbf{\tilde{X}}_l^d\in \mathbb{R}^{PN_l\times d_{\text{model}}}$. Similarly, the context feature $\mathbf{Z}_g\in \mathbb{R}^{(D*PN_g)\times d_\text{model}}$ is viewed back to $\mathbf{Z}_g\in \mathbb{R}^{D\times PN_g\times d_\text{model}}$ and the $d$-th variate $\mathbf{Z}_g^d\in \mathbb{R}^{PN_g\times d_\text{model}}$ is sent to the cross-attention as key and value to match the dimension of $\mathbf{\tilde{X}}_l^d$. Specifically, the cross-attention mechanism operates as follows:
\begin{align}
    &\text{AttentionBlock}(\mathbf{\tilde{X}}_l^d, \mathbf{Z}_g^d) \nonumber \\
    &= \text{CrossAttention}(\mathbf{Z}_g^d, \;\text{SelfAttention}(\mathbf{\tilde{X}}_l^d)),
\end{align}
Note that we have omitted the skip connection and normalization steps for a concise presentation. The rest of the local model is the same as a regular transformer decoder as shown in the figure. 

Denote the output of the local model of the $d$th variable to be $\mathbf{Y}_\text{out}^d\in \mathbb{R}^{PN_l\times d_{\text{model}}}$. Stacking the outputs of all variables, we obtain $\mathbf{Y}_\text{out}\in \mathbb{R}^{D\times PN_l\times d_\text{model}}$. The last two dimensions of $\mathbf{Y}_\text{out}$ are then flattened and a final linear head is employed to project from dimension $PN_l\times d_\text{model}$ to $H$, which is the target horizon window. 

\subsection{Runtime Complexity Analysis}
\label{sec:runtime}
% The runtime complexity is primarily influenced by the patch number $PN$, rather than the look-back window length $L$. The Mamba layers, which exhibit linear complexity, process a sequence of length $D \cdot PN$ in a runtime complexity of $O(D \cdot PN)$. In contrast, the Transformer has quadratic complexity and processes $D$ sequences of length $PN$ each, leading to a runtime complexity of $O(D \cdot PN^2)$. Consequently, the overall complexity of the model is $O(D \cdot PN^2)$ with respect to $PN$. To manage complexity as the look-back window length increases, it is reasonable to proportionally increase both the patch length and stride such that $PN$ remains constant. So, when the stride is chosen to be proportional to the length $L$, the overall complexity of S2TX is constant.
The Mamba layers, which exhibit linear complexity, process a sequence of length $D \cdot PN_g$ with a complexity of $O(D \cdot PN_g)$, which is linear to input time series length $L$ due to definition of $PN$ in Equation~\eqref{eq:pn}. On the other hand, while transformer models exhibit quadratic complexity with respect to sequence length, S2TX uses a local look-back window with fixed length $S$, resulting in a complexity of $O(D\cdot PN_l^2) = O(D)$ as $PN_l=O(S)=O(1)$.  Thus, S2TX has an overall linear complexity with respect to $L$ and $D$. Moreover, as $L$ increases, we can proportionally increase both the patch length and stride so that $PN_g$ remains constant, in which case, the overall complexity of S2TX reduces to constant order with respect to $L$ while remaining linear order with respect to $D$. Our empirical results in Section~\ref{sec:runtime} verifies this, showing that S2TX's runtime barely increases with $L$.
% To manage this complexity as the look-back window length $L$ increases, it is reasonable to proportionally increase both the patch length and stride so that $PN$ remains constant, in which case, the overall complexity of S2TX remains constant. Otherwise, if the stride remains fixed for a prolonged look-back window, the overall complexity increases linearly with $L$.

\section{Experiment}
\label{sec:experiment}
We empirically demonstrate that utilizing cross-variate correlation and global-local interaction can significantly improve the forecasting performance. We first introduce the experimental setup, then we showcase the performance of S2TX over a variety of benchmark against recent state-of-the-art architectures. We then demonstrate the efficacy of the main component of S2TX with a set of ablation study and a robustness study where we test the robustness of S2TX with sequences of missing values. Finally, we showcase the low memory footprint and efficient runtime of S2TX compared to Transformer and Mamba in general.

\textbf{Dataset.} 
We benchmark our proposed algorithm S2TX on a set of 7 real-world multivariate time series datasets, including the four Electricity Transformer Temperature datasets ETTh1, ETTh2, ETTm1, and ETTm2, Weather, Electricity, and Exchange rate datasets. Detailed dataset descriptions are provided in the Appendix \ref{appendix:data_desc}.

\textbf{Baselines.} 
We benchmark our proposed algorithm S2TX against most competitive time series forecasting models within three years, including MOE-based model SST \citep{xusst}, Mamba-based models S-Mamba \citep{wang2025mamba} and TimeMachine (TimeM) \citep{ahamed2024timemachine}, transformer-based models iTransformer (iTrans) \citep{liu2023itransformer}, PatchTST \citep{nie2022time}, Crossformer (CrossF) \citep{zhang2023crossformer}, and FEDformer \citep{zhou2022fedformer}, linear-based models RLinear \citep{li2023revisiting} and DLinear \citep{zeng2023transformers}, and TCN-based model TimesNet \citep{wu2022timesnet}. Due to space constraints, the comparisons against DLinear and FEDformer (pre-2023 models) are presented in Appendix \ref{appendix:full_comparison} 

\textbf{Experimental Setting and Metrics.}
For a fair comparison, the experimental setting of all baselines follows the experiment setup of the current SOTA SST. In addition, we use the same hyperparameters as in SST, including global and local patch length, stride, and look-back window. Specifically, we set $PL_g = 48$, $STR_g = 16$, $PL_l = 16$, $STR_l = 8$, and $L = 2S = 336$. For Exchange rate dataset, we use a smaller patch length, stride, and look-back window: $PL_g = 16$, $STR_g = 8$, $PL_l = 4$, $STR_l = 2$, $L = 2S = 192$. The forecast horizon is set to $\{96,192,336,720\}$ for each dataset. We use mean squared error and mean absolute error as metrics to compare performances of different architectures. 

We now present the numerical result of our comprehensive experiments, as well as an ablation study to showcase the importance of each module, and a computation efficiency study comparing canonical architectures, SST, and S2TX. 

\subsection{Benchmark Results}
\label{sec:result}
The performance of 9 different architectures on 7 benchmark datasets and 4 different prediction horizons is presented in Table \ref{tab:performance}. \textbf{Our method S2TX achieves SOTA performance across all benchmark datasets}. In particular, compared to the previous SOTA model SST, S2TX demonstrates consistent improvements on most datasets and performs on par on the weather dataset. For instance, S2TX achieves an $8.4\%$ improvement on the ETTh1 dataset with a prediction horizon of $720$. Moreover, even on the weather dataset, S2TX significantly surpasses other baseline models. The SOTA performance, together with our ablation studies in section \ref{sec:ablation}, suggests that the two novel aspect of S2TX, the cross-variate global features, and the cross-attentional local features, are indeed important for accurately forecasting multivariate time series. 

Guided by cross-variate global context, S2TX demonstrates a superior ability to capture local variations. Figure \ref{fig:predictTScompare} presents a random segment of test time prediction from the electricity dataset on a randomly selected variate, comparing the performance of S2TX, SST, iTransformer, and S-Mamba. S2TX precisely approximates abrupt spikes while the accurate predictions of local variation are less apparent in predictions of other models.
%\paragraph{Qualitative Results}

%To intuitively illustrate the improvements of S2TX, we present a visual graph showcasing a random prediction segment from the electricity dataset on a randomly selected variable, comparing the performance of S2TX, SST, iTransformer, and iMamba. Guided by cross-dimensional global context, S2TX demonstrates a superior ability to capture local variations, as evidenced by its precise approximation of abrupt spikes—an ability less apparent in the other models.


\subsection{Ablation and Robustness Studies}
\label{sec:ablation}
\begin{figure}[t]
    \centering
    \includegraphics[width=0.45\textwidth]{./images/ablation_components.png} % Change to your image file
    \caption{Ablation study on different components of S2TX tested on ETTh1 and ETTm1 datasets. The efficacy of each component of the proposed architectures is measured by the degradation of performance after each (or both) component(s) was excluded. }
    \label{fig:ablation}
\end{figure}
\textbf{Ablation on Model Components.}
We perform ablation studies by removing key components of S2TX. To first assess the impact of cross-variate communication in learning global context, we input the patch sequence of each variate separately into the global model, rather than using the concatenated cross-variate patch sequence. Second, to evaluate the effectiveness of the context-local cross-attention mechanism, we remove cross-attention and instead concatenate the global context and local features before the final linear head. Finally, we remove both mechanisms to evaluate their combined effect. We conduct ablation studies on the ETTh1 and ETTm1 datasets and report the MSE metric, averaged across four different prediction lengths. As shown in Figure \ref{fig:ablation}, the global-local cross-attention contributes the most to the overall improvement of S2TX, while variable communication also positively influences the results.

\textbf{Robustness to Missing Values.}
In real-world multivariate time series datasets, it is common to observe missing values. Unlike traditional tabular data where a few elements are missing, missing values in time series could exist for small periods of sequences. In this set of robustness experiments, we randomly select small sequences of 4 time steps to be missing and interpolate these randomly missing periods with the value of the last observed time step. In Table \ref{tab:robustness}, we present the MSE of different architectures under various percentage of missing values. We show that S2TX, with the addition of cross-variate global context and the cross-attentional global-local feature interplay, is highly robust compared to SST, which showed much-worsened degradation as the percentage of missing value increases. 
\begin{table}[t]
    \centering
    \footnotesize
    \renewcommand{\arraystretch}{1.1}
    \setlength{\tabcolsep}{8pt}
    \begin{tabular}{lcc}  % l = left, c = center, r = right
        \toprule
        \textbf{Miss Ratio} & \textbf{S2TX}& \textbf{SST}\\
        \midrule
        0\%         &0.421(-0.0\%)  &0.439(-0.0\%)\\
        4\%         &0.424(-0.7\%)  & 0.440(-0.2\%) \\
        8\%         & 0.425(-0.9\%) & 0.443(-0.9\%) \\
        16\%        & 0.424(-0.7\%) & 0.450(-2.5\%) \\
        24\%        & 0.429(-1.9\%) &0.468(-6.6\%) \\
        32\%        &0.431(-2.3\%)  & 0.471(-7.0\%)\\
        40\%        & 0.441(-4.7\%) & 0.499(-13.4\%)\\
        \bottomrule
    \end{tabular}
    \caption{Performance on ETTh1 with increasing proportion of missing values; results are MSE averaged over all four prediction horizons.}
    \label{tab:robustness}
\end{table}

\subsection{Memory and Runtime Analysis}\label{sec:runtime}
To ensure a fair runtime comparison, we evaluate S2TX alongside SST, the vanilla Transformer, and Mamba on a single NVIDIA RTX 6000 Ada Generation GPU. The two versions of S2TX and SST correspond to configurations with either a fixed patch number or a fixed stride length. In the former case, the patch number remains constant regardless of sequence length, whereas in the latter case, the patch number increases proportionally with sequence length.

It is important to note that we compare against the vanilla Transformer and Mamba, rather than their inverted versions, as the respective attention and selective mechanisms in the inverted versions operate on the variate dimension. As the look-back window sequence length increases, the memory usage and runtime of the Transformer grow exponentially, reaching the GPU's memory limit when the sequence length hits 2000. In contrast, Mamba scales linearly in both memory and time metrics.

Both S2TX and SST scale more efficiently than Mamba, owing to the fixed short local look-back window combined with the patching technique, which effectively reduces the sequence length by a factor of the stride length. The complexity experiment result is presented in Figure \ref{fig:ablation_study}. Consistent to the runtime analysis in section~\ref{sec:runtime}, when the global patch number is fixed, both S2TX and SST achieve nearly constant runtime complexity. However, when comparing S2TX to SST, SST scales slightly better due to the additional cross-attention mechanism in S2TX.

\begin{figure}[t]
    \centering
    % Subfigure 1
    \begin{subfigure}[]{0.6\columnwidth} % 45% of the width for the first subfigure
        \centering
        \includegraphics[width=\linewidth]{./images/memory_complexity.png} % Replace with your image file
        \caption{}%Memory usage comparison between S2TX, SST, vanilla transformer and Mamba.}
        \label{fig:subfigure1}
    \end{subfigure}
    \quad % Adds horizontal spacing between the two subfigures
    % Subfigure 2
    \begin{subfigure}[]{0.6\columnwidth} % 45% of the width for the second subfigure
        \centering
        \includegraphics[width=\linewidth]{./images/time_complexity.png} % Replace with your image file
        \caption{}%Time complexity comparison between S2TX, SST, vanilla transformer and Mamba.}
        \label{fig:subfigure2}
    \end{subfigure}
    % Main caption for the figure
    \caption{Memory and run-time comparison between S2TX and other canonical architectures.}
    \label{fig:ablation_study}
\end{figure}

\section{Discussion and Future Work}
In this work, we introduce a new architecture, \emph{State-Space Transformer with cross-attention} (S2TX), for multivariate time series modeling. We first noted that the multi-scale patching methods, although enhance the learning of temporal dependencies, neglect the cross-variate correlation--a crucial aspect of multivariate time series modeling. Also, global and local patches are processed independently, overlooking the global and local interactions that occur in many real-world scenarios. We propose a novel cross-attention based architecture that integrates state space models and transformers. This cross-attention architecture, combined with patchification, fully leverages the strengths of Mamba and transformers by integrating cross-variate global features from Mamba with the local features of the transformer. Our architecture generally improves over current state-of-the-art in various datasets and 4 different prediction horizons. The SOTA performance of S2TX is not only achieved with a low memory footprint and fast computation runtime but also demonstrated robust performance when facing time series with sequences of missing values. Given these advantages, S2TX unlocks new possibilities for time series forecasting by effectively capturing cross-variate correlations and global-local feature interactions.

\textbf{Limitations.}
Several limitations exist in our current architectures. One key limitation is that cross-variate correlations are not explicitly explored at a local level. Although S2TX maintains low memory usage and fast runtime, incorporating local cross-variate correlations could further enhance performance. Another limitation is the lack of diversity in the multi-scale approach. The current architecture only deals with global and local patches with no learning of the intermediates scales. Intermediate time scales, however, could be important for extremely long sequences where the difference in time scales between global and local contexts is dramatic. Incorporating multiple time scales within an architecture while remaining lightweight is still unsolved. We leave these for future works. 

\paragraph{Impact Statement}
This paper presents work whose goal is to advance the field of Machine Learning. There are many potential societal consequences of our work, none of which we feel must be specifically highlighted here.


% In the unusual situation where you want a paper to appear in the
% references without citing it in the main text, use \nocite
%\nocite{langley00}
% \bibliography{main}
\bibliographystyle{icml2025}
\documentclass{MITstyle}

%\usepackage[table]{xcolor}
\usepackage{chngcntr}
\usepackage{hyperref}
\usepackage{microtype}

\title{A Lightweight and Extensible Cell Segmentation and Classification Model for Whole Slide Images}

\author{Nikita Shvetsov~$^{1, }$\footnote{Correspondence e-mail: nikita.shvetsov@uit.no}, Thomas K. Kilvaer~$^{2, 3}$, Masoud Tafavvoghi~$^{4}$, Anders Sildnes~$^{1}$, \\ Kajsa Møllersen~$^{4}$, Lill-Tove Rasmussen Busund~$^{5, 6}$, Lars Ailo Bongo~$^{1}$ \\
%
\vspace{1em} % Space between authors and afilliations
%
\normalfont{\small $^{1}$Department of Computer Science, UiT The Arctic University of Norway}\\
\normalfont{\small $^{2}$Department of Oncology, University Hospital of North Norway}\\
\normalfont{\small $^{3}$Department of Clinical Medicine, UiT The Arctic University of Norway}\\
\normalfont{\small $^{4}$Department of Community Medicine, UiT The Arctic University of Norway}\\
\normalfont{\small $^{5}$Department of Medical Biology, UiT The Arctic University of Norway} \\
\normalfont{\small $^{6}$Department of Clinical Pathology, University Hospital of North Norway} %\vspace{2em}
}

\begin{document}
\maketitle

\section*{Abstract}

% \begin{abstract}
% Developing clinically useful cell-level analysis tools in digital pathology remains challenging due to limitations in dataset granularity, inconsistent annotations, computational demands of advanced models, and difficulties in integrating new technologies into clinical workflows. To address these challenges, we propose a multi-faceted solution that enhances data quality, model performance, and usability to create a lightweight and extensible cell segmentation and classification model.

% First, we update data labels by employing a cross-relabeling process that refines the labels of two existing datasets, PanNuke and MoNuSAC, to create a new unified dataset with enhanced granularity, encompassing seven distinct cell types. Second, we leverage the H-Optimus foundation model as a fixed encoder to improve feature representation for simultaneous cell segmentation and classification tasks. Third, to address the computational demands of foundation models, we employ knowledge distillation to reduce model size and complexity while maintaining comparable performance. Finally, to facilitate integration into clinical workflows, we integrate the distilled model into the QuPath software, a widely used open-source platform in digital pathology.

% Our results demonstrate improvements in cell segmentation and classification performance using the H‑Optimus-based model compared to a CNN-based model. Specifically, the average $R^2$ improved from 0.575 to 0.871, and the average $PQ$ score improved from 0.450 to 0.492, indicating better alignment with actual cell counts and enhanced segmentation and classification quality. Furthermore, the distilled student model maintains performance comparable to the larger foundation model while reducing the parameter count by a factor of 48.
% Overall, by reducing computational complexity and integrating it into existing workflows, the proposed approach may significantly impact diagnostic processes, reduce the workload of pathologists, and contribute to improved patient outcomes. Though our approach shows potential enhancements in efficiency and usability of cell segmentation and classification models in digital pathology, extensive validation is needed to deploy these models in clinical practice.
% \end{abstract}

%%% shortened abstract
\begin{abstract}
Developing clinically useful cell-level analysis tools in digital pathology remains challenging due to limitations in dataset granularity, inconsistent annotations, high computational demands, and difficulties integrating new technologies into workflows. To address these issues, we propose a solution that enhances data quality, model performance, and usability by creating a lightweight, extensible cell segmentation and classification model. 

First, we update data labels through cross-relabeling to refine annotations of PanNuke and MoNuSAC, producing a unified dataset with seven distinct cell types. Second, we leverage the H-Optimus foundation model as a fixed encoder to improve feature representation for simultaneous segmentation and classification tasks. Third, to address foundation models' computational demands, we distill knowledge to reduce model size and complexity while maintaining comparable performance. Finally, we integrate the distilled model into QuPath, a widely used open-source digital pathology platform. 

Results demonstrate improved segmentation and classification performance using the H-Optimus-based model compared to a CNN-based model. Specifically, average $R^2$ improved from 0.575 to 0.871, and average $PQ$ score improved from 0.450 to 0.492, indicating better alignment with actual cell counts and enhanced segmentation quality. The distilled model maintains comparable performance while reducing parameter count by a factor of 48. By reducing computational complexity and integrating into workflows, this approach may significantly impact diagnostics, reduce pathologist workload, and improve outcomes. Although the method shows promise, extensive validation is necessary prior to clinical deployment.
\end{abstract}
\clearpage

\section{Introduction}
In digital pathology, accurate segmentation and classification of cells are crucial for many diagnostic, prognostic, and predictive analyses \cite{Jaber_Beziaeva_etal._2019,Lin_Pan_etal._2022,Park_Ock_etal._2022,Shen_Choi_etal._2024}. Nowadays, developments in computational pathology offer multiple solutions \cite{H._Qu_P._Wu_etal._2020,Javed_Mahmood_etal._2020} to utilize cell-level datasets to train machine learning models that solve these problems. The quality and specificity of training datasets are critical for robust and accurate models. Adhering to the principle of "garbage in, garbage out", it is essential to ensure that these datasets are extensively and accurately labeled with distinct classes that reflect the diverse biological characteristics of different cell types. Unfortunately, the number of open-source datasets comprising such high-quality annotations is limited. Existing cell segmentation datasets \cite{Gamper_Koohbanani_etal._2019,Graham_Vu_etal._2019,Verma_Kumar_etal._2021} may offer extensive annotations for certain cell types while providing more general labels for others. For example, in PanNuke, which is one of the largest open-source datasets comprising labeled cells, various types of morphologically and functionally different inflammatory cells like macrophages and lymphocytes are clustered in a broad "inflammatory" class. Consequently, these classes are frequently omitted from analyses or aggregated into broader meta-classes \cite{Gamper_Koohbanani_etal._2020} and likely interfere with other cell classes included in the dataset. This and similar inconsistencies in annotation granularity limit the ability of machine learning models to learn the comprehensive and nuanced features necessary for accurate cell segmentation and classification. To address these challenges, methods for refining and standardizing dataset annotations are essential to enhance the quality of training data.

A complementary approach to mitigate the absence of high-quality training data is the use of foundation models. Foundation models as encoders are defined as large-scale, versatile networks pre-trained on vast, diverse datasets using self-supervised learning, contrasting with convolutional neural network (CNN) pre-trained encoders that rely on supervised learning with labeled data. In practice, foundation models leverage enormous amounts of weakly or unlabeled data from millions of whole slide images (WSIs) and employ self-attention mechanisms to capture long-range dependencies and global context \cite{Chen_Ding_etal._2024,Saillard_Jenatton_etal._2024,Vorontsov_Bozkurt_etal._2024,Xu_Usuyama_etal._2024}. As a consequence, foundation models are able to produce transferable feature representations across different cell types and tissue environments. The feature representations can be leveraged by decoder networks to produce segmentation masks and pixel-level classifications. Because foundation models have comprehensive feature representations, they can be effectively fine-tuned using much smaller amounts of cell-level data compared to the large datasets needed to train models from scratch. Furthermore, foundation models incorporate adversarial training elements or contrastive learning \cite{Chen_Ding_etal._2024,Xu_Usuyama_etal._2024}, enhancing their resilience and adaptability by exposing them to challenging and varied scenarios during training. This may result in more generalizable models, often making them well-suited for diverse and complex tasks in digital pathology.

Despite the inherent advantages of foundation models, their deployment for practical use faces its own obstacles. In particular, they require substantial computational power, financial investments and rigorous testing to ensure reliability and efficacy for a given task \cite{Akkus_Dangott_etal._2022,Dragomir_Cocuz_etal._2022,Go_2022,Jafri_Farooqui_etal._2024}. Moreover, while foundation models enhance feature representation and performance, they depend on the quality of available annotations for decoder fine-tuning and, like any other model, cannot resolve existing inconsistencies or ambiguities in data labels. Therefore, there remains a critical need for solutions that address both data quality and practical deployment considerations.
Further, integrating new technologies into existing clinical workflows often encounters resistance, as it necessitates adjustments to established diagnostic processes. So, there is a need to develop solutions that could be integrated into current practices, minimizing the burden on medical professionals to adopt new tools \cite{King_Williams_etal._2023}.

Existing solutions \cite{Goldsborough_Philps_etal._2024,Hörst_Rempe_etal._2024}, while addressing some aspects of these challenges, fall short in providing a comprehensive approach. To address the data quality and clinical deployment issues, we propose a multi-faceted solution that encompasses data refinement, model optimization, and integration with existing pathology tools (\hyperref[fig:fig1]{Figure 1}). The outcome is a lightweight cell segmentation and classification model that can be integrated into digital pathology workflows for practical clinical use.

\begin{figure}[h!]
    \centering
    \includegraphics[width=\textwidth, height=0.82\textheight, keepaspectratio]{images/Figure_1.pdf}
    \caption{Overview of the proposed solution, including 1) Data refinement using cross-relabeling, 2) Teacher model development and fine tuning, 3) Student model optimization with knowledge distillation and 4) Student model and QuPath integration}
    \label{fig:fig1}
\end{figure}
\clearpage

Our approach begins with preparing the data for the fine-tuning and training of the machine learning models. We create a refined dataset, acquired via cross-relabeling two cell-level datasets, enhancing annotation specificity and consistency of the labeled data. Subsequently, we create a cell segmentation and classification model based on the foundation model. We leverage the foundation model as a fixed encoder and fine-tune a decoder using the refined dataset to improve generalization across diverse tissue- and cell types.
To ensure that the model remains lightweight and deployable in a possibly resource-constrained environment, we employ knowledge distillation to approximate the functionality of the foundation model. Finally, to facilitate the practical application of our model in digital pathology workflows, we integrate it with the QuPath \cite{Bankhead_Loughrey_etal._2017} application. Each methodological component contributes to the overarching goal of enhancing model performance, generalizability, and usability in clinical settings.

The primary contributions of this paper are:
\begin{enumerate}
    \item \textit{Data labels refinement through cross-relabeling:}
    
    We propose a new method for refining labels of cell-level datasets through cross-relabeling. This method employs classification models to re-label broad and ambiguous instances, resulting in a more diverse dataset. Our evaluation demonstrates that these classification models achieve high accuracy on test subsets, indicating the reliability of the method for label refinement.

    \item \textit{Enhanced model performance via foundation models:}
    
    We employ a foundation model as a feature extractor for the cell segmentation and classification task. In comparison with training a CNN model from scratch, the foundation model backbone only needs fine-tuning, which significantly reduces training time, computational resources and data requirements. We show that using a foundation model encoder leads to better performance in cell segmentation and classification networks than using a CNN-based encoder. This improvement may enable the model to generalize more effectively across various tissue types and imaging methods.
    
    \item \textit{Model optimization through knowledge distillation:}
    
    We show that a smaller student model trained using knowledge distillation on the refined dataset obtained via our cross-relabeling approach from a foundation model achieves comparable performance in cell segmentation and quantification tasks. As a result, this model is more suitable for deployment in environments without high-performance computing resources.
    
    \item \textit{Integration with QuPath:}
    
    We integrate the distilled cell segmentation and classification model into QuPath, a widely used open-source digital pathology platform, to accelerate clinical adaptation by enabling pathologists to more easily incorporate advanced computational tools into their existing workflows.
\end{enumerate}

Through these methodological steps, we aim to bridge the gap between advanced machine learning techniques and practical clinical applications, making accurate and efficient digital pathology accessible in a broader range of healthcare settings.

\section{Refining Existing Datasets Using Cross-Relabeling}
To address the limitations of sparse and ambiguous labeling of cell-level datasets, we propose a generalizable cross-relabeling strategy that can be applied to any dataset containing broadly categorized or imprecisely labeled cell types. This approach involves training and subsequently leveraging classification models to refine broad categories into more specific or biologically relevant classes.
When applied to cell-level data, the methodology includes extracting individual cell images from the dataset patches, preprocessing these images to standardize the size and accommodate partial cells, and then training deep learning classifiers capable of distinguishing between the finer cell subtypes within the coarser categories. 
To illustrate our approach, we focus on the PanNuke \cite{Gamper_Koohbanani_etal._2020, Gamper_Koohbanani_etal._2019} and MoNuSAC \cite{Verma_Kumar_etal._2021} datasets that we have used to train models for cell quantification in our previous works \cite{Shvetsov_Grønnesby_etal._2022,Shvetsov_Sildnes_etal._2024}. We find that for better cell differentiation we have to introduce more granular labels. PanNuke includes a broad classification of "inflammatory" cells, encompassing lymphocytes, macrophages, and neutrophils. Each cell type differs significantly in structure, function, and clinical relevance. Conversely, MoNuSAC uses the label "epithelial" for a class that comprises both benign epithelial cells and malignant neoplastic cells. This practice makes it challenging to differentiate between benign and malignant epithelial cells in the dataset, which is a critical distinction when identifying tumor areas within tissue samples. To address these issues, we implement a cross-relabeling strategy as shown in \hyperref[fig:fig2]{Figure 2}. The key components are two classification models: one is trained on singular cell images from PanNuke data to classify the epithelial meta-class into epithelial and neoplastic classes. The other is trained on MoNuSAC to refine the inflammatory class into lymphocytes, neutrophils, and macrophages.

\begin{figure}[h!]
    \centering
    \includegraphics[width=\textwidth]{images/Figure_2.pdf}
    \caption{Refined dataset generation via cross relabeling}
    \label{fig:fig2}
\end{figure}

The refining approach consists of three consecutive steps. The first is the preprocessing step, in which we extract individual cells from both datasets (\hyperref[fig:fig3]{Figure 3}). The specifics of PanNuke and MoNuSAC patch preparation before cell preprocessing are provided in \hyperref[chap:S1]{Appendix S1}.

\begin{figure}[h!]
    \centering
    \includegraphics[width=\textwidth]{images/Figure_3.pdf}
    \caption{Cell instances preprocessing including (1) cell map extraction, (2) bounding box delineation, (3) adjusting cell boxes and (4) cropping and resizing of cell images}
    \label{fig:fig3}
\end{figure}

During preprocessing, we extract cell type maps from the ground truth label mask and calculate bounding boxes around each cell instance. To accommodate partial cells at patch borders, a common issue in cropped patch images, we employ mirror padding and extend the field of view of the cell label by 15 pixels to capture adjacent cells. We then crop and resize the identified regions to $64 \times 64$ pixels using bicubic interpolation.

The preprocessed PanNuke dataset comprises 68,031 neoplastic and 23,207 epithelial cell images, while MoNuSAC comprises  33,104 lymphocytes, 1,252 neutrophils, and 1,695 macrophages, which we subsequently use in training cell classification models and classifying the cell image data \hyperref[fig:S2]{Appendix Figure S2 (1)}. 

The next step is to train two distinct ResNet50-based classifiers tailored to address the specific labeling challenges inherent in each dataset. We use ResNet50 for classification models due to its proven effectiveness for image classification tasks in histopathology \cite{pan2022reviewmachinelearningapproaches}, and its compatibility with small images. For the PanNuke dataset, we design the classifier, trained on MoNuSAC data, to disaggregate the heterogeneous "inflammatory" cell category into distinct subtypes: lymphocytes, macrophages, and neutrophils. Similarly, for the MoNuSAC dataset, the classifier is trained on PanNuke data and distinguishes between benign and malignant epithelial cells within the overarching "epithelial" label. By applying these targeted classifiers to their respective datasets, we assign more specific labels to individual cell instances, thus enabling us to create a unified dataset.
To ensure a balanced representation of classes, we train both models on datasets that had been equalized to match the size of the least represented class. Thus, we obtain datasets comprising 23,207 samples per class for PanNuke and 1,252 samples per class for MoNuSAC data. Next, we partition both of them into training (70\%), validation (20\%), and testing (10\%) subsets. To mitigate the risk of overfitting, we use a single dropout layer with a rate of p=0.5 in both models and data augmentation using randomized color perturbations, rotation, and horizontal and vertical flipping. We employ AdamW optimizer and the cross-entropy loss function for the training criterion.

To evaluate the two trained models, we measure the classification accuracy on the respective test subsets. The accuracies on the test subset for both classifiers are presented in \hyperref[tab:1]{Table 1}. The PanNuke model achieves an average accuracy of 93.57\%, with higher accuracy for neoplastic cells (96.06\%) compared to epithelial cells (86.26\%). The confusion matrix in Figure A3.1 shows that the model predominantly distinguishes accurately between epithelial and neoplastic tissues, with a substantial number of correct classifications and relatively few misclassifications. The MoNuSAC model demonstrates an average accuracy of 98.92\%, excelling in classifying lymphocytes (99.67\%) and macrophages (94.12\%), with lower performance for neutrophils (85.71\%). The confusion matrix in Figure A3.2 shows that the model identifies lymphocytes and performs reasonably well with macrophages and neutrophils.

\begin{table}[h!]
\renewcommand{\arraystretch}{1.5}
  \centering
  \caption{Cell classification results for PanNuke and MoNuSAC trained models (CI 95\%).}
  \label{tab:1}
  \begin{tabular}{|l|c|c|}
   \hline
   %\rowcolor{gray!30}
    Accuracy               & PanNuke model              & MoNuSAC model              \\
    \hline
    Average      & 0.936 (0.931--0.941)         & 0.989 (0.986--0.993)        \\
    \hline
    Neoplastic   & 0.961 (0.956--0.965)         & -                          \\
    \hline
    Epithelial   & 0.863 (0.849--0.877)         & -                          \\
    \hline
    Lymphocytes  & -                          & 0.997 (0.995--0.999)        \\
    \hline
    Neutrophils  & -                          & 0.857 (0.796--0.918)        \\
    \hline
    Macrophages  & -                          & 0.941 (0.906--0.976)        \\
    \hline
  \end{tabular}
\end{table}

Finally, during the last step, we use the model trained on PanNuke data for epithelial cells in MoNuSAC and the model trained on MoNuSAC for the inflammatory cells class in PanNuke. Specifically, we use classifier models to relabel epithelial cells in MoNuSAC and inflammatory cells in PanNuke data. Then we combine cells with refined labels and the rest of the cells in both datasets to create a refined dataset (\hyperref[fig:S2]{Appendix Figure S2 (2)}). The process of relabeling cells and visualizing them on a patch is shown in \hyperref[fig:fig4]{Figure 4}. The cell counts in the refined dataset are provided in \hyperref[tab:S4]{Appendix Table S4}.

\begin{figure}[h!]
    \centering
    \includegraphics[width=\textwidth, height=0.42\textheight, keepaspectratio]{images/Figure_4.pdf}
    \caption{Cell relabeling procedure for epithelial and inflammatory cell classes}
    \label{fig:fig4}
\end{figure}

%\hfill

Relabeling and combining datasets have been explored in a prior study \cite{Parulekar_Kanwat_etal._2023}, where consecutive fine-tuning on multiple datasets was employed to account for hierarchical class label structures. While the method presented in \cite{Parulekar_Kanwat_etal._2023} is intuitive, it often lacks consistency and requires multiple fine-tuning runs, which can be cumbersome and time-consuming. 
In contrast, cross-relabeling simplifies this process by using specialized classification models tailored to each dataset's specific labeling challenges. This approach provides better transparency and produces a unified dataset encompassing seven distinct cell types across multiple tissue samples, enhancing data diversity for further model training or fine-tuning.

Despite these improvements, cross-relabeling does not entirely resolve issues related to poor labeling quality or the amount of labeled data. Specifically, our results show lower accuracies persist for underrepresented classes, such as macrophages, which may stem from a limited sample availability and intrinsic challenges in distinguishing these cells based solely on H\&E staining. Furthermore, while our method enhances label specificity, it relies on the initial quality of the broad labels; thus, any fundamental inaccuracies in the original annotations can propagate through the relabeling process. Addressing the overall problem of limited data labels may require integrating additional data sources or utilizing complementary immunohistochemical staining methods.
Although the reported performance metrics are obtained from evaluations on the native test sets of each dataset, it is important to note that the primary application of these classifiers is to perform cross-relabeling, where a model trained on one dataset (e.g., PanNuke) is applied to another (e.g., MoNuSAC) and vice versa. We acknowledge that a more systematic evaluation of cross-dataset generalization is needed and could be performed in future work.

Overall, the refined dataset produced by our approach can enhance the supervised training or fine-tuning of cell segmentation and classification models, especially those that utilize pre-trained foundation models to improve feature extraction robustness. In addition, these models can detect nuanced classes that enable researchers to conduct more detailed analyses of biological processes in computational pathology.

\section{Foundation models for robust cell segmentation and classification}

Accurate cell segmentation and classification in digital pathology are hindered by limited labeled data and the fact that conventional CNNs are unable to capture global contextual information due to their local receptive field constraints \cite{Gheflati_Rivaz_2022,Yang_Marcus_etal.}. Traditional approaches in cell quantification have predominantly relied on CNN encoders, such as ResNet50, given their proven effectiveness in semantic segmentation tasks \cite{Deshmane_2023,Graham_Vu_etal._2019,Mukasheva_Koishiyeva_etal._2024,Stringer_Wang_etal._2021}. However, approaches that include fine-tuning of pretrained CNNs, data augmentation, and stain normalization to partially increase data variability and address staining differences often fail to achieve the necessary generalization and robustness across diverse tissue types and staining conditions \cite{G._Wang_W._Li_etal._2018,Gao_Bagci_etal._2018,Karim_El_Khoury_Martin_Fockedey_etal._2021}.

To overcome these challenges, we leverage an encoder-decoder network that uses a foundation model as the encoder and a CNN upsampling decoder (\hyperref[fig:fig5]{Figure 5}) for simultaneous cell segmentation and classification in 2D patches extracted from WSIs. Foundation models with transformer-based architectures are viable alternatives to CNN-based encoders \cite{Shamshad_Khan_etal._2023,Sourget_2023}. They enable the creation of more advanced architectures that can decode or transform learned features more effectively \cite{Chen_Duan_etal._2023,Cheng_Misra_etal._2022,Xie_Wang_etal._2021}.

\begin{figure}[h!]
    \centering
    \includegraphics[width=\textwidth]{images/Figure_5.pdf}
    \caption{UNETR-like model with foundational model as backbone}
    \label{fig:fig5}
\end{figure}

By utilizing a transformer-based encoder, we incorporate global contextual information into the feature extraction process, which is a key advantage of such architectures \cite{Chen_Lu_etal._2021}. This foundation model integration facilitates accurate pixel-wise segmentation and classification without the need for extensive encoder training, thereby potentially improving generalization across varied cellular structures and tissue types.
In our implementation, we employ a modified UNETR \cite{Hatamizadeh_Tang_etal._2021} architecture that combines a vision transformer (ViT) \cite{Dosovitskiy_Beyer_etal._2021} encoder with a CNN-based decoder. The encoder utilizes the pretrained H-Optimus foundation model, which contains 1.1 billion parameters and is trained on over 500,000 H\&E stained WSIs \cite{Saillard_Jenatton_etal._2024}. We extract outputs from four evenly spaced transformer blocks $Z_i$, where $i \in [1, 14, 26, 38]$, to serve as residual connections for the CNN decoder. We select these blocks based on our observation that features from non-adjacent levels of the encoder lead to better overall performance on the test subset.

The CNN decoder upsamples the feature representations, acquired from the transformer blocks, to generate an intermediate vector that is handled by two task-specific layers that generate cell segmentation and classification masks. The first task-specific layer is the ‘Cellpose head’,  which is used to delineate cell instances. The layer generates horizontal and vertical gradient maps to form vector fields that are refined through gradient tracking in a post-processing step using the Cellpose algorithm \cite{Stringer_Wang_etal._2021}, known for its efficacy in cell segmentation tasks and generalizability across multiple domains \cite{Pachitariu_Stringer_2022,Stringer_Pachitariu_2024}. The second task-specific layer is the "Cell type head", which assigns labels to individual pixels. In the post-processing step, we determine the output classification label of each segmented cell instance by majority voting over the labeled pixels that comprise the cell in the segmentation map.

To evaluate model performance and measure the impact of adding a foundation model as backbone, we compare it to a ResNet50-based model. ResNet50 is a widely used solution for encoders in segmentation architectures in the medical domain \cite{Deshmane_2023,Graham_Vu_etal._2019,Mukasheva_Koishiyeva_etal._2024,Stringer_Wang_etal._2021}. For the H-Optimus-based model, we utilize frozen weights for the encoder and only fine-tune the decoder to take advantage of the extensive pre-training of the foundation model. For the ResNet50-based model we start with ImageNet \cite{Deng_Dong_etal.} weights and train both encoder and decoder parts. Hyperparameters for the training step are set to be identical, where possible, for comparable evaluation. 
For this evaluation, we deliberately use the PanNuke dataset to provide a standardized and controlled comparison between the H‑Optimus and ResNet50-based models (\hyperref[fig:S2]{Appendix Figure S2 (3)}). Specifically, we use two of the default PanNuke dataset splits (66\%) for training and validation, and reserve the third split (33\%) for testing.

To address the challenge of cell class imbalance in the PanNuke dataset, which is a common characteristic in most cell-level H\&E patch datasets, both models’ training processes employ a weighted loss function comprising cross-entropy and focal loss \cite{Lin_Goyal_etal._2018}. The focal loss component is adjusted with coefficients derived from each cell class' instance frequency, emphasizing learning from underrepresented classes and enhancing the model's sensitivity to rare but significant cellular patterns. The cross-entropy loss is augmented with spectral decoupling regularization \cite{Pezeshki_Kaba_etal._2021,Pohjonen_Stürenberg_etal._2022} and spatially varying label smoothing \cite{Islam_Glocker_2021}, which potentially stabilizes training and improves generalization in case of complex tissue morphologies. For optimization, we employ the \textit{AdamW} \cite{Loshchilov_Hutter_2019} to counter unbalanced class scenarios, with cosine annealing learning rate scheduler.

We utilize the scikit-learn library \cite{Van_der_Walt_Schönberger_etal._2014} and HoVer-Net \cite{Graham_Vu_etal._2019} implementations of $R^2$ (the coefficient of determination) and $PQ$ (panoptic quality) to evaluate our experiments. Complete mathematical formulations and detailed explanations of these metrics are provided in \hyperref[chap:S5]{Appendix S5}. To compute confidence intervals, we use nonparametric bootstrapping, where after calculating the metric on the full sample, we generated 1000 bootstrap replicates by resampling with replacement and then determined the 95\% confidence intervals as the 2.5th and 97.5th percentiles of the resulting empirical distribution.

%\hfill

The model comparisons are summarized in \hyperref[tab:2]{Table 2}. The H‑Optimus-based model achieves higher $R^2$ across all cell classes compared to the ResNet50-based model, which means that its predictions are more closely aligned with the PanNuke cell counts, indicating a stronger correlation with the observed data. Notably, the improvement of $R^2_{dead}$ may be an indicator of better global contextual representations provided by the foundation model backbone. In terms of segmentation and classification quality combined, measured by the PQ score, the H‑Optimus-based model demonstrates notable improvements across most cell classes. Overall, the average $R^2$ improved from 0.575 to 0.871, while the average $PQ$ score improved from 0.450 to 0.492, demonstrating better performance of the H-Optimus-based model.

\begin{table}[h!]
\renewcommand{\arraystretch}{1.5}
  \centering
  \caption{Cell quantification metrics for baseline and proposed models (CI 95\%).}
  \label{tab:2}
  \begin{tabular}{|l|c|c|}
    \hline
    %\rowcolor{gray!30}
    Metric             & Resnet50-based            & H-optimus-based              \\
    \hline
    $R^2_{neoplastic}$    & 0.681 (0.576--0.769)       & \textbf{0.941 (0.917--0.960)} \\
    \hline
    $R^2_{inflammatory}$  & 0.863 (0.778--0.903)       & \textbf{0.949 (0.918--0.966)} \\
    \hline
    $R^2_{connective}$    & 0.600 (0.488--0.698)       & 0.609 (0.436--0.772)          \\
    \hline
    $R^2_{dead}$          & 0.097 (-11.389--0.669)     & 0.925 (0.404--0.982)          \\
    \hline
    $R^2_{epithelial}$    & 0.635 (0.490--0.747)       & \textbf{0.930 (0.886--0.964)} \\
    \hline
    $PQ_{neoplastic}$       & 0.517 (0.499--0.535)       & \textbf{0.589 (0.575--0.604)} \\
    \hline
    $PQ_{inflammatory}$     & 0.455 (0.429--0.482)       & \textbf{0.528 (0.507--0.549)} \\
    \hline
    $PQ_{connective}$       & 0.416 (0.400--0.431)       & \textbf{0.451 (0.436--0.465)} \\
    \hline
    $PQ_{dead}$             & 0.374 (0.342--0.408)       & 0.292 (0.209--0.365)          \\
    \hline
    $PQ_{epithelial}$       & 0.488 (0.460--0.519)       & \textbf{0.599 (0.579--0.618)} \\
    \hline
  \end{tabular}
\end{table}

Our results  show that integrating the H‑Optimus foundation model within the UNETR architecture enhances the model's ability to segment and classify cells across diverse tissues from PanNuke data. The pretrained transformer encoder provides robust feature representations, resulting in higher average $R^2$ and $PQ$ scores compared to the CNN-based model. This leads to more reliable cell quantification and more accurate downstream analysis. Additionally, the streamlined fine-tuning process reduces computational overhead and training time, making the model more adaptable for new data.

Despite these advancements, the foundation model-based approach does not fully resolve all challenges related to cell segmentation and classification. We observe lower metric scores for underrepresented classes in the training data. Furthermore, foundation models typically encompass billions of parameters, resulting in substantial computational and memory requirements. It therefore poses challenges for deployment in resource-constrained environments, limiting their practical applicability in certain clinical settings.

\section{Model optimization via Knowledge Distillation}

To address the limitations posed by the extensive size of foundation models, we implement knowledge distillation — a model compression technique that leverages the teacher-student paradigm \cite{Hinton_Vinyals_etal._2015}. By training a smaller, more efficient student model to replicate the output of a larger, pre-trained teacher model, we retain performance while significantly reducing the model's complexity and resource requirements (\hyperref[fig:fig6]{Figure 6}).

\begin{figure}[h!]
    \centering
    \includegraphics[width=\textwidth, height=0.45\textheight, keepaspectratio]{images/Figure_6.pdf}
    \caption{Knowledge distillation framework for training a student model using a pre-trained teacher}
    \label{fig:fig6}
\end{figure}

We employ knowledge distillation to compress the H‑Optimus-based teacher model into a more efficient student model. The teacher model is the modified UNETR architecture with the H‑Optimus foundation model described in the previous chapter. The student model is based on a UNet architecture augmented with residual connections and incorporates a smaller ViT encoder with 9 million parameters \cite{Steiner_Kolesnikov_etal._2022,Wightman_2019}. 

First, we fine-tune the teacher model using the refined dataset from the cross-relabeling procedure (Section 2). Initially we train the decoder of the teacher model while keeping the encoder weights frozen. We split the refined dataset into train (70\%), validation (20\%) and test (10\%) subsets (\hyperref[fig:S2]{Appendix Figure S2 (4)}). During fine-tuning, we use the train and validation subsets, while leaving the test subset for model evaluation. We set the training procedure and model hyperparameters to be identical to those that were used to demonstrate the utility of foundation models for the simultaneous cell segmentation and classification task.

Next, we perform knowledge distillation from teacher to student using the refined dataset used to fine-tune the teacher model. The student model is trained to replicate the teacher model's outputs. We utilize a specialized loss function that aligns the student's predicted probability distribution with the teacher's, incorporating the teacher's class probability distribution derived from the output. Following the methodology of Hinton et al. \cite{Hinton_Vinyals_etal._2015}, we experiment with various hyperparameter settings for the temperature ($T$) and the balancing coefficients ($\alpha$ and $\beta$) in the loss function. We vary $T$ from 1 to 20 and adjust $\alpha$ and $\beta$ to balance the distillation and student losses. Through iterative tuning and evaluation, we identify that setting $T=14$, $\alpha=0.3$, and $\beta=0.7$ yields a configuration that converges and closely approximates the teacher model's performance during training.

Finally, we assess the performance of both models using the $R^2$ and $PQ$ (defined in \hyperref[chap:S5]{Appendix S5}) on the test set of the refined dataset (\hyperref[tab:3]{Table 3}). We observe that the 95\% confidence intervals overlap for most cell types, so we cannot claim statistically significant performance differences between the teacher and student models. One exception appears in the neoplastic class. The teacher model produces an $R^2$ of 0.919, while the student model shows an $R^2$ of 0.852. In addition, the student model achieves higher $PQ$ values for the neoplastic and connective classes, though the confidence intervals show overlap.

\begin{table}[h!]
\renewcommand{\arraystretch}{1.5}
  \centering
  \caption{Cell quantification metrics for teacher and distilled student models (CI 95\%).}
  \label{tab:3}
  \begin{tabular}{|l|c|c|}
    \hline
    %\rowcolor{gray!30}
    Metric & Teacher & Student \\
    \hline
    $R^2_{neoplastic}$    & \textbf{0.919} (0.898--0.939) & 0.852 (0.800--0.891) \\
    \hline
    $R^2_{lymphocyte}$    & 0.969 (0.956--0.977)         & 0.969 (0.956--0.978) \\
    \hline
    $R^2_{connective}$    & 0.694 (0.548--0.809)         & 0.618 (0.469--0.741) \\
    \hline
    $R^2_{dead}$          & 0.755 (0.400--0.908)         & 0.424 (0.100--0.731) \\
    \hline
    $R^2_{epithelial}$    & 0.922 (0.870--0.958)         & 0.843 (0.738--0.917) \\
    \hline
    $R^2_{macrophage}$    & 0.384 (-0.369--0.724)        & 0.704 (0.352--0.859) \\
    \hline
    $R^2_{neutrofil}$     & 0.854 (0.578--0.929)         & 0.833 (0.502--0.925) \\
    \hline
    $PQ_{neoplastic}$       & 0.581 (0.569--0.593)         & 0.601 (0.588--0.613) \\
    \hline
    $PQ_{lymphocyte}$       & 0.536 (0.520--0.553)         & 0.563 (0.544--0.579) \\
    \hline
    $PQ_{connective}$       & 0.436 (0.421--0.451)         & 0.457 (0.441--0.474) \\
    \hline
    $PQ_{dead}$             & 0.272 (0.235--0.315)         & 0.279 (0.201--0.369) \\
    \hline
    $PQ_{epithelial}$       & 0.522 (0.500--0.545)         & 0.530 (0.506--0.555) \\
    \hline
    $PQ_{macrophage}$       & 0.524 (0.459--0.588)         & 0.474 (0.405--0.543) \\
    \hline
    $PQ_{neutrofil}$        & 0.541 (0.490--0.592)         & 0.565 (0.522--0.607) \\
    \hline
  \end{tabular}
\end{table}


We further decompose the $PQ$ metric into its $SQ$ and $DQ$ components (\hyperref[tab:S6]{Appendix Table S6}). Both models produce nearly identical $SQ$ values, which indicates that they predict instance boundaries with similar precision. Although the student model shows some improvement in $DQ$ scores for certain classes, the confidence intervals overlap and do not confirm a statistically significant difference.

We observe that the student and teacher models yield comparable detection performance despite the student model using a much smaller and simpler architecture. A model with fewer parameters reduces the risk of overfitting when training data are scarce relative to the model’s complexity \cite{Farias_Ludermir_etal._2022}. The knowledge distillation process also encourages the student model to focus on the most generalizable detection features learned from the teacher. These factors enable the student model to achieve similar detection performance across different cell types.

Additionally, considering the model sizes reported in \hyperref[tab:4]{Table 4}, the distilled model achieves a significant reduction compared to the teacher model, with a 48-fold decrease in parameter count and a 5.5-fold reduction in on-disk size. In inference mode, the teacher model requires 16 GB of VRAM for a batch size of 32, while the distilled model only needs 3 GB of VRAM for the same batch size. These reductions make the distilled model significantly more practical for fine-tuning and deployment in resource-constrained environments.

\begin{table}[h!]
\renewcommand{\arraystretch}{1.5}
  \centering
  \caption{Parameter counts and size of teacher and distilled model}
  \label{tab:4}
  \adjustbox{max width=\textwidth}{%
  \begin{tabular}{|l|c|c|c|}
    \hline
    %\rowcolor{gray!30}
    Metric & H-optimus-based (Teacher) & mobileViT-based (Student) & Magnitude of difference \\
    \hline
    Parameters count       & 1,158,917,906   & \textbf{24,093,393}   & \textbf{48x}  \\
    \hline
    Estimated Total Size (MB) & 87,912       & \textbf{15,935}    & \textbf{5.5x} \\
    \hline
  \end{tabular}%
}
\end{table}

%\hfill

With recent advancements in complex network architectures and the use of pretrained encoders to achieve state-of-the-art performance \cite{Baumann_Dislich_etal._2024,Hörst_Rempe_etal._2024} in cell segmentation and classification tasks, model size, computational complexity, and processing times have increased. This limits the scalability and accessibility of these models. As we demonstrate, this may be mitigated using knowledge distillation. Studies in the field of natural language processing have demonstrated the efficacy of knowledge distillation in retaining the capabilities of the teacher model while achieving significant reductions in size and complexity \cite{Huangpu_Gao_2024,Sun_Yu_etal.}. 

We demonstrate the feasibility of knowledge distillation in digital pathology, specifically for cell segmentation and classification tasks. Moreover, we achieve this performance while also significantly reducing the parameter count. In addressing the challenge of knowledge transfer, we found that distillation from a transformer-based model to a smaller transformer is more straightforward than attempting to map transformer features to CNN blocks. In our experiments, using a CNN-based network as a student results in worse cell quantification performance due to the structural constraints of CNN feature space dimensions. 

Although our primary approach relies on a transformer-based student model that performs well, it can be further optimized to incorporate advantages from CNN architectures. For example, employing alternative techniques such as using ViT adapters \cite{Chen_Duan_etal._2023} or $1 \times 1$ convolutions to adjust feature map sizes may be beneficial for harnessing CNN advantages like enhanced local feature extraction. Moreover, if additional performance improvements are desired, the process can be further enhanced by applying supplementary knowledge distillation techniques, such as self-distillation \cite{Zhang_Song_etal._2019} or online distillation \cite{Houyon_Cioppa_etal._2023}.

Despite these promising results, further validation on independent datasets is necessary to fully understand the model's limitations. Underrepresented classes may pose challenges when addressing complex cases. Pathologists need to validate these models to adopt them in clinical settings. While the distilled models are smaller and more deployable, a technological gap persists because pathologists traditionally rely on established methods for inspecting WSIs and diagnosing diseases. Addressing the complexities involved in deploying models for inference and supporting pathologists in adopting new tools is essential for integrating these models into clinical workflows.

\section{Model integration with QuPath}
Digital pathology tools with graphical user interfaces are essential for visualizing and analyzing WSIs. To make our student model useful in clinical pathology workflows, it needs to be integrated into a tool that enables inspecting regions, creating annotations, and providing quantitative analyses of biomarkers. Therefore, we integrate the trained student model from the previous chapter into the QuPath open‑source platform \cite{Bankhead_Loughrey_etal._2017}. QuPath provides the required annotation, visualization, and analysis tools to interpret complex histological data, including workflows for cell segmentation, classification, and quantification (\hyperref[fig:fig7]{Figure 7}). 

\begin{figure}[h!]
    \centering
    \includegraphics[width=\textwidth]{images/Figure_7.pdf}
    \caption{Visualization of model-generated cell quantification annotations (left) and the corresponding unannotated slide (right) in QuPath}
    \label{fig:fig7}
\end{figure}

To identify the regions in a WSI critical for prognosticating tumor development, such as specific tumor areas or border regions without overlapping healthy tissue, the pathologist uses QuPath to outline these regions. Then, the pathologist initiates a cell segmentation and classification script through the QuPath interface for the selected regions. The resulting annotations and quantified cell information are then directly overlaid onto the WSI in the QuPath interface. Additional design and implementation details are in \hyperref[chap:S7]{Appendix S7}. 

Two common approaches for integrating deep learning models into QuPath are Java‑based native QuPath extensions \cite{Goldsborough_Philps_etal._2024} and the execution of RESTful API requests to a model server coupled with handling the response via an extension, as demonstrated in the application of cell segmentation models applied to immunofluorescence images \cite{Sugawara_2023}. While the community is actively working on these integration strategies, there is currently no universal solution that fully addresses all integration and performance requirements.

Extensions may offer better integration with QuPath, allowing slightly improved performance and more widespread usage of the built-in QuPath models, but they lack the flexibility to customize models and modify their behavior. For example, the newest version of QuPath includes models such as StarDist \cite{Weigert_Schmidt} and InstanSeg \cite{Goldsborough_Philps_etal._2024} that can perform cell segmentation. Both models pose limitations when applied to simultaneous cell segmentation and classification. StarDist performs well only on convex, round shapes by design, whereas some neoplastic, inflammatory, and connective cells exhibit complex and non-convex shapes. InstanSeg provides only semantic segmentation without assigning classes to the segmented cells.

%\hfill

In contrast, our approach offers an alternative integration strategy. It utilizes the paquo library to directly interact with QuPath’s internal application programming interface from within Python. This enables data exchange and processing without the need for intermediate conversion steps and provides greater control over model customization, retraining, and the incorporation of custom processing steps.

The integration of our custom model with QuPath underscores its potential to significantly enhance the diagnostic process by reducing the time burden on pathologists and enabling them to focus on more complex interpretative tasks using familiar software. Leveraging a tool that is already well-established among pathologists increases the likelihood of its adoption into daily clinical workflows. The quantitative data generated through the automated workflow is critical for both clinical decision-making and research, facilitating more accurate biomarker analysis, enabling robust statistical evaluations, and supporting hypothesis generation and testing. Additionally, by streamlining cell segmentation and classification, the tool enhances the scalability and reproducibility of pathological assessments, ultimately contributing to improved diagnostic accuracy and patient outcomes.

\section{Conclusion and future work}

In this study, we address critical challenges in digital pathology and tackle the usability and deployment issues of the developed models in standard computing environments without the need for high-performance computing systems. Our multi-faceted approach encompasses data refinement through cross-relabeling, leveraging foundation models for robust cell segmentation and classification, optimizing model performance via knowledge distillation, and integrating the optimized model into the QuPath software for practical application. This approach is used to construct a capable, versatile, and adjustable model for cell segmentation and classification, with enhanced performance and usability.

\begin{sloppypar}
While our approach shows potential in the field of computational pathology, certain limitations persist. 
For example, our implementation currently exhibits lower performance in detecting macrophages. 
This serves as an instance of the broader challenge of accurately identifying complex cell types. In order to address this issue, extending our approach to incorporate additional data sources, exploring alternative modeling approaches, and integrating other imaging modalities such as immunohistochemical staining may help improve detection accuracy. Moreover, although the distilled model reduces computational demands, integrating advanced deep learning models into clinical practice requires addressing technological gaps and potential resistance to adopting new tools within established diagnostic processes.
\end{sloppypar}

Future work could focus on several key areas to refine the proposed approach and facilitate its adoption in clinical environments. Enhancing the cell-relabeling process with additional datasets \cite{Graham_Jahanifar_etal._2021} could improve the representation of underrepresented cell types and enhance overall model performance. Also, incorporating additional data sources, such as multi-modal imaging or complementary staining methods, may address limitations related to cell type differentiation and class imbalance. Exploring other foundation models \cite{Vorontsov_Bozkurt_etal._2024,Zimmermann_Vorontsov_etal._2024} or introducing additional modalities \cite{Ding_Wagner_etal._2024,Vaidya_Zhang_etal._2025} may provide alternative architectures better suited to specific tasks or offer improved efficiency. Implementing more complex knowledge distillation techniques \cite{Houyon_Cioppa_etal._2023,Zhang_Song_etal._2019} could further optimize the model's performance and adaptability. Additionally, deeper integration with QuPath or other digital pathology software could provide pathologists more control over cell quantification analysis directly within the QuPath interface, thereby increasing accessibility and usability. Such enhancements would not only refine model performance but also ensure greater adaptability and scalability within various clinical environments. Finally, extensive validation of the model by pathologists and benchmarking against independent datasets are essential steps toward establishing the model's reliability and fostering confidence in its clinical utility.

\section*{Acknowledgments} 
This work was funded in part by the Research Council of Norway grant no. 309439 SFI Visual Intelligence, and the North Norwegian Health Authority grant no. HNF1521-20.

\bibliographystyle{IEEEtran}
\begin{sloppypar}
\begin{thebibliography}{99}

\bibitem{chaplot2020neural} Chaplot, Devendra Singh, et al. "Neural topological slam for visual navigation." Proceedings of the IEEE/CVF conference on computer vision and pattern recognition. 2020.

\bibitem{maksymets2021thda} Maksymets, Oleksandr, et al. "Thda: Treasure hunt data augmentation for semantic navigation." Proceedings of the IEEE/CVF International Conference on Computer Vision. 2021.

\bibitem{mezghan2022memory} Mezghan, Lina, et al. "Memory-augmented reinforcement learning for image-goal navigation." 2022 IEEE/RSJ International Conference on Intelligent Robots and Systems (IROS). IEEE, 2022.

\bibitem{al2022zero} Al-Halah, Ziad, Santhosh Kumar Ramakrishnan, and Kristen Grauman. "Zero experience required: Plug \& play modular transfer learning for semantic visual navigation." Proceedings of the IEEE/CVF Conference on Computer Vision and Pattern Recognition. 2022.

\bibitem{ye2021auxiliary} Ye, Joel, et al. "Auxiliary tasks and exploration enable objectgoal navigation." Proceedings of the IEEE/CVF international conference on computer vision. 2021.

\bibitem{chaplot2020object} Chaplot, Devendra Singh, et al. "Object goal navigation using goal-oriented semantic exploration." Advances in Neural Information Processing Systems 33 (2020)

\bibitem{ramakrishnan2022poni} Ramakrishnan, Santhosh Kumar, et al. "Poni: Potential functions for objectgoal navigation with interaction-free learning." Proceedings of the IEEE/CVF Conference on Computer Vision and Pattern Recognition. 2022.

\bibitem{ramrakhya2022habitat} Ramrakhya, Ram, et al. "Habitat-web: Learning embodied object-search strategies from human demonstrations at scale." Proceedings of the IEEE/CVF Conference on Computer Vision and Pattern Recognition. 2022.

\bibitem{mousavian2019visual} Mousavian, Arsalan, et al. "Visual representations for semantic target driven navigation." 2019 International Conference on Robotics and Automation (ICRA). IEEE, 2019.

\bibitem{dhariwal2021diffusion} Dhariwal, Prafulla, and Alexander Nichol. "Diffusion models beat gans on image synthesis." Advances in neural information processing systems 34 (2021)

\bibitem{ho2022classifier} Ho, Jonathan, and Tim Salimans. "Classifier-free diffusion guidance." arXiv preprint arXiv:2207.12598 (2022).

\bibitem{nichol2021glide} Nichol, Alex, et al. "Glide: Towards photorealistic image generation and editing with text-guided diffusion models." arXiv preprint arXiv:2112.10741 (2021)

\bibitem{brooks2023instructpix2pix} Brooks, Tim, Aleksander Holynski, and Alexei A. Efros. "Instructpix2pix: Learning to follow image editing instructions." Proceedings of the IEEE/CVF Conference on Computer Vision and Pattern Recognition. 2023.

\bibitem{fu2023guiding} Fu, Tsu-Jui, et al. "Guiding instruction-based image editing via multimodal large language models." arXiv preprint arXiv:2309.17102 (2023).

\bibitem{geng2024instructdiffusion} Geng, Zigang, et al. "Instructdiffusion: A generalist modeling interface for vision tasks." Proceedings of the IEEE/CVF Conference on Computer Vision and Pattern Recognition. 2024.

\bibitem{zhou2024minedreamer} Zhou, Enshen, et al. "Minedreamer: Learning to follow instructions via chain-of-imagination for simulated-world control." arXiv preprint arXiv:2403.12037 (2024).

\bibitem{zhou2023esc} Zhou, Kaiwen, et al. "Esc: Exploration with soft commonsense constraints for zero-shot object navigation." International Conference on Machine Learning. PMLR, 2023.

\bibitem{yu2023l3mvn} Yu, Bangguo, Hamidreza Kasaei, and Ming Cao. "L3mvn: Leveraging large language models for visual target navigation." 2023 IEEE/RSJ International Conference on Intelligent Robots and Systems (IROS). IEEE, 2023.

\bibitem{gadre2023cows} Gadre, Samir Yitzhak, et al. "Cows on pasture: Baselines and benchmarks for language-driven zero-shot object navigation." Proceedings of the IEEE/CVF Conference on Computer Vision and Pattern Recognition. 2023.

\bibitem{shah2023navigation} Shah, Dhruv, et al. "Navigation with large language models: Semantic guesswork as a heuristic for planning." Conference on Robot Learning. PMLR, 2023.

\bibitem{cai2024bridging} Cai, Wenzhe, et al. "Bridging zero-shot object navigation and foundation models through pixel-guided navigation skill." 2024 IEEE International Conference on Robotics and Automation (ICRA). IEEE, 2024.

\bibitem{yu2023co} Yu, Bangguo, Hamidreza Kasaei, and Ming Cao. "Co-NavGPT: Multi-robot cooperative visual semantic navigation using large language models." arXiv preprint arXiv:2310.07937 (2023).

\bibitem{wu2024voronav} Wu, Pengying, et al. "Voronav: Voronoi-based zero-shot object navigation with large language model." arXiv preprint arXiv:2401.02695 (2024).

\bibitem{qin2023mp5} Qin, Yiran, et al. "Mp5: A multi-modal open-ended embodied system in minecraft via active perception." arXiv preprint arXiv:2312.07472 (2023).

\bibitem{du2024learning} Du, Yilun, et al. "Learning universal policies via text-guided video generation." Advances in Neural Information Processing Systems 36 (2024).

\bibitem{ajay2024compositional} Ajay, Anurag, et al. "Compositional foundation models for hierarchical planning." Advances in Neural Information Processing Systems 36 (2024).

\bibitem{liang2024skilldiffuser} Liang, Zhixuan, et al. "Skilldiffuser: Interpretable hierarchical planning via skill abstractions in diffusion-based task execution." Proceedings of the IEEE/CVF Conference on Computer Vision and Pattern Recognition. 2024.

\bibitem{heusel2017gans} Heusel, Martin, et al. "Gans trained by a two time-scale update rule converge to a local nash equilibrium." Advances in neural information processing systems 30 (2017).

\bibitem{zhang2018unreasonable} Zhang, Richard, et al. "The unreasonable effectiveness of deep features as a perceptual metric." Proceedings of the IEEE conference on computer vision and pattern recognition. 2018.

\bibitem{brown2020language} Brown, Tom B. "Language models are few-shot learners." arXiv preprint arXiv:2005.14165 (2020).

\bibitem{podell2023sdxl} Podell, Dustin, et al. "Sdxl: Improving latent diffusion models for high-resolution image synthesis." arXiv preprint arXiv:2307.01952 (2023).

\bibitem{brohan2022rt} Brohan, Anthony, et al. "Rt-1: Robotics transformer for real-world control at scale." arXiv preprint arXiv:2212.06817 (2022).

\bibitem{brohan2023rt} Brohan, Anthony, et al. "Rt-2: Vision-language-action models transfer web knowledge to robotic control." arXiv preprint arXiv:2307.15818 (2023).

\bibitem{li2024manipllm} Li, Xiaoqi, et al. "Manipllm: Embodied multimodal large language model for object-centric robotic manipulation." Proceedings of the IEEE/CVF Conference on Computer Vision and Pattern Recognition. 2024.

\bibitem{shah2023vint} Shah, Dhruv, et al. "ViNT: A foundation model for visual navigation." arXiv preprint arXiv:2306.14846 (2023).

\bibitem{liu2024visual} Liu, Haotian, et al. "Visual instruction tuning." Advances in neural information processing systems 36 (2024).

\bibitem{hu2021lora} Hu, Edward J., et al. "Lora: Low-rank adaptation of large language models." arXiv preprint arXiv:2106.09685 (2021).

\bibitem{qin2023supfusion} Qin, Yiran, et al. "SupFusion: Supervised LiDAR-camera fusion for 3D object detection." Proceedings of the IEEE/CVF International Conference on Computer Vision. 2023.

\bibitem{qin2024worldsimbench} Qin, Yiran, et al. "Worldsimbench: Towards video generation models as world simulators." arXiv preprint arXiv:2410.18072 (2024).

\bibitem{yu2025gamefactory} Yu, Jiwen, et al. "GameFactory: Creating New Games with Generative Interactive Videos." arXiv preprint arXiv:2501.08325 (2025).

\bibitem{zhou2024code} Zhou, Enshen, et al. "Code-as-Monitor: Constraint-aware Visual Programming for Reactive and Proactive Robotic Failure Detection." arXiv preprint arXiv:2412.04455 (2024).

\bibitem{zhang2024ad} Zhang, Zaibin, et al. "AD-H: Autonomous Driving with Hierarchical Agents." arXiv preprint arXiv:2406.03474 (2024).

\bibitem{wang2024toward} Wang, Chaoqun, et al. "Toward Accurate Camera-based 3D Object Detection via Cascade Depth Estimation and Calibration." arXiv preprint arXiv:2402.04883 (2024).

\bibitem{huang2024story3d} Huang, Yuzhou, et al. "Story3d-agent: Exploring 3d storytelling visualization with large language models." arXiv preprint arXiv:2408.11801 (2024).

\bibitem{savinov2018semi} Savinov, Nikolay, Alexey Dosovitskiy, and Vladlen Koltun. "Semi-parametric topological memory for navigation." arXiv preprint arXiv:1803.00653 (2018).

\bibitem{majumdar2022zson} Majumdar, Arjun, et al. "Zson: Zero-shot object-goal navigation using multimodal goal embeddings." Advances in Neural Information Processing Systems 35 (2022): 32340-32352.

\bibitem{yadav2023offline} Yadav, Karmesh, et al. "Offline visual representation learning for embodied navigation." Workshop on Reincarnating Reinforcement Learning at ICLR 2023. 2023.

\bibitem{yadav2023ovrl} Yadav, Karmesh, et al. "Ovrl-v2: A simple state-of-art baseline for imagenav and objectnav." arXiv preprint arXiv:2303.07798 (2023).

\bibitem{sun2024fgprompt} Sun, Xinyu, et al. "FGPrompt: fine-grained goal prompting for image-goal navigation." Advances in Neural Information Processing Systems 36 (2024).

\bibitem{zhu2017target} Zhu, Yuke, et al. "Target-driven visual navigation in indoor scenes using deep reinforcement learning." 2017 IEEE international conference on robotics and automation (ICRA). IEEE, 2017.

\bibitem{koh2024generating} Koh, Jing Yu, Daniel Fried, and Russ R. Salakhutdinov. "Generating images with multimodal language models." Advances in Neural Information Processing Systems 36 (2024).

\bibitem{krantz2022instance} Krantz, Jacob, et al. "Instance-specific image goal navigation: Training embodied agents to find object instances." arXiv preprint arXiv:2211.15876 (2022).

\bibitem{schulman2017proximal} Schulman, John, et al. "Proximal policy optimization algorithms." arXiv preprint arXiv:1707.06347 (2017).

\bibitem{anderson2018evaluation} Anderson, Peter, et al. "On evaluation of embodied navigation agents." arXiv preprint arXiv:1807.06757 (2018).

\bibitem{lin2024navcot} Lin, Bingqian, et al. "NavCoT: Boosting LLM-Based Vision-and-Language Navigation via Learning Disentangled Reasoning." arXiv preprint arXiv:2403.07376 (2024).

\bibitem{NavGPT} Zhou, Gengze, Yicong Hong, and Qi Wu. "Navgpt: Explicit reasoning in vision-and-language navigation with large language models." Proceedings of the AAAI Conference on Artificial Intelligence.

\bibitem{hahn2021no} Hahn, Meera, et al. "No rl, no simulation: Learning to navigate without navigating." Advances in Neural Information Processing Systems 34 (2021): 26661-26673.

\bibitem{li2025t2isafety} Li, Lijun, et al. "T2ISafety: Benchmark for Assessing Fairness, Toxicity, and Privacy in Image Generation." arXiv preprint arXiv:2501.12612 (2025).

\bibitem{an2024agfsync} An, Jingkun, et al. "AGFSync: Leveraging AI-Generated Feedback for Preference Optimization in Text-to-Image Generation." arXiv preprint arXiv:2403.13352 (2024).


\end{thebibliography}
\end{sloppypar}

\clearpage
\beginsupplement
\section*{Appendix}
\renewcommand{\thesubsection}{S\arabic{subsection}}

\subsection{\label{chap:S1}PanNuke and MoNuSAC preprocessing}
The PanNuke dataset comprises a set of 7,901 RGB patches, each with dimensions of $256 \times 256$ pixels, which we set as the standard patch size for our analysis. In contrast, the MoNuSAC dataset encompasses 294 images of heterogeneous dimensions. To standardize the MoNuSAC images with our experiments, we implement a standardization protocol. Specifically, for images exceeding the dimensions of $256 \times 256$ pixels, we segment them into equal-sized patches and apply mirror padding to the remaining portions to avoid information loss at the peripherals. Patches with dimensions less than $128 \times 128$ pixels are excluded from the dataset due to the insufficient resolution to capture relevant cellular details. For patches where either dimension falls between 128 and 256 pixels, we employ upsampling to achieve the standard patch size. As a result, we obtain a total of 2,823 RGB patches derived from the MoNuSAC dataset for subsequent analysis. For additional details on the MoNuSAC data preparation process, refer to the source code \cite{Shvetsov_2025a}.
\clearpage

\subsection{\label{chap:S2}Data usage for the methodology}

\counterwithin{figure}{subsection}
\renewcommand{\thefigure}{S\arabic{subsection}}

\begin{figure}[h!]
    \centering
    \includegraphics[width=\textwidth, height=0.85\textheight, keepaspectratio]{images/A2.pdf}
    \caption{Overview of the methodology for cross-labeling, dataset refinement, and model comparison. (1) Cross-relabeling - training and testing cell classification models, (2) Cross-relabeling - using cell classification models to create refined dataset, (3) Fine-tuning and training models for comparison, (4) Student knowledge distillation with refined dataset}
    \label{fig:S2}
\end{figure}
\clearpage

\subsection{\label{chap:S3}Confusion matrices for classification models}
\counterwithin{figure}{subsection}
\renewcommand{\thefigure}{S\arabic{subsection}.\arabic{figure}}

\begin{figure}[h!]
    \centering
    \includegraphics[width=\textwidth, height=0.4\textheight, keepaspectratio]{images/A3_1.pdf}
    \caption{Confusion matrix for PanNuke trained model}
    \label{fig:S3.1}
\end{figure}

\begin{figure}[h!]
    \centering
    \includegraphics[width=\textwidth, height=0.4\textheight, keepaspectratio]{images/A3_2.pdf}
    \caption{Confusion matrix for MoNuSAC trained model}
    \label{fig:S3.2}
\end{figure}

\clearpage

\subsection{\label{chap:S4}Datasets cell counts}

\counterwithin{table}{subsection}
\renewcommand{\thetable}{S\arabic{subsection}}

\begin{table}[h!]
\renewcommand{\arraystretch}{2.0}
\centering
\caption{\label{tab:S4}Cell counts for PanNuke, MoNuSAC and refined datasets. Numbers in parentheses indicate preprocessed cell counts for cell classifier models training and testing.}
%\adjustbox{max width=\textwidth}{%
\begin{tabular}{|l|c|c|c|}
\hline
%\rowcolor{gray!30}
Cell type & PanNuke & MoNuSAC & Refined \\
\hline
Neoplastic & 77,403 (68,031) & - & 105,451 \\
\hline
Epithelial & 26,572 (23,207) & - & 29,926 \\
\hline
Epithelial (benign and malignant) & - & 31,402 & - \\
\hline
Inflammatory & 32,276 & - & - \\
\hline
Lymphocytes & - & 37,045 (33,104) & 65,275 \\
\hline
Neutrophils & - & 1,355 (1,252) & 3,833 \\
\hline
Macrophage & - & 1,842 (1,695) & 3,410 \\
\hline
Dead & 2,908 & - & 2,908 \\
\hline
Connective & 50,585 & - & 50,585 \\
\hline
\end{tabular}
%
%}
\end{table}



\clearpage

\subsection{\label{chap:S5}Definition of validation metrics}
\counterwithin{equation}{subsection}
\renewcommand{\theequation}{\arabic{equation}}

\subsubsection{\label{chap:S5.1}R\textsuperscript{2}}
The coefficient of determination, denoted as $R^2$, is a statistical measure that represents the proportion of variance in the dependent variable that is predictable from the independent variables. In the context of cell quantification in pathology, $R^2$ is used to assess how well the predicted quantities of different cell types in a patch align with the actual quantities observed in the ground truth data, with higher values representing more accurate quantification. $R^2$ is defined as
\begin{equation*}
R^2 = 1 - \frac{\sum_{i=1}^n (y_i - \hat{y}_i)^2}{\sum_{i=1}^n (y_i - \bar{y})^2},
\end{equation*}
where $y_i$ represents the actual number of cells of a specific type in the $i$-th image, $\hat{y}_i$ represents the predicted number of cells of that type in the $i$-th image, $\bar{y}$ is the mean of the actual numbers across all images, and $n$ is the total number of images in the dataset.

The $R^2$ metric has a range of $(-\infty, 1]$. An $R^2$ of 1 indicates perfect prediction, where all predicted values exactly match the actual values. An $R^2$ of 0 suggests that the model explains none of the variability of the response data around its mean. If $R^2$ is negative, it indicates that the model performs worse than a model that simply predicts the mean of the actual values for all observations.

\subsubsection{\label{chap:S5.2}PQ}
Panoptic Quality ($PQ$) is a comprehensive metric used to evaluate the performance of segmentation models in tasks that require both instance segmentation and classification. $PQ$ provides a single score that encapsulates both the detection accuracy (i.e., how many objects were correctly identified) and the segmentation quality (i.e., how accurately the objects' boundaries were delineated). This metric is particularly useful in multiclass scenarios where each pixel is classified into distinct categories, such as different cell types in pathology images.

$PQ$ is calculated as the product of two terms: Detection Quality ($DQ$) and Segmentation Quality ($SQ$). It can be expressed as
\begin{equation*}
PQ = DQ \cdot SQ,
\end{equation*}
where
\begin{equation*}
DQ = \frac{TP}{TP + 0.5\, FP + 0.5\, FN},
\end{equation*}
\begin{equation*}
SQ = \frac{\sum_{(p, g) \in \mathcal{M}} IoU(p, g)}{TP}.
\end{equation*}
In these formulas, $TP$ denotes the number of correctly matched instances between ground truth and prediction, $FP$ denotes the predicted instances that have no corresponding ground truth, $FN$ denotes the ground truth instances that were not detected, $IoU(p, g)$ is the Intersection over Union for a pair of matched instances $p$ (prediction) and $g$ (ground truth), and $\mathcal{M}$ is the set of matched pairs.

The $PQ$ metric is calculated for each class and is averaged across classes to provide a global performance measure.

The $PQ$ score has a range of $[0, 1.0]$, where a higher score indicates better performance in both detecting and segmenting the instances correctly. A $PQ$ of 1 signifies perfect identification and segmentation of all instances, whereas a $PQ$ of 0 indicates that no instances were correctly identified and segmented.

\clearpage

\subsection{\label{chap:S6}Segmentation and Detection quality metrics for teacher and student models}

\begin{table}[h!]
\renewcommand{\arraystretch}{2.0}
\centering
\caption{Segmentation and detection quality for student and teacher models (CI 95\%)}
\label{tab:S6}
%\adjustbox{max width=\textwidth}{%
\begin{tabular}{|l|c|c|}
\hline
%\rowcolor{gray!30}
Metric & Teacher & Student \\
\hline
$SQ_{neoplastic}$ & 0.819 (0.815--0.823) & 0.824 (0.819--0.828) \\
\hline
$SQ_{lymphocyte}$ & 0.795 (0.788--0.802) & 0.790 (0.783--0.796) \\
\hline
$SQ_{connective}$ & 0.770 (0.762--0.776) & 0.780 (0.772--0.786) \\
\hline
$SQ_{dead}$ & 0.659 (0.623--0.688) & 0.657 (0.624--0.695) \\
\hline
$SQ_{epithelial}$ & 0.780 (0.770--0.790) & 0.788 (0.779--0.797) \\
\hline
$SQ_{macrophage}$ & 0.788 (0.760--0.810) & 0.757 (0.730--0.783) \\
\hline
$SQ_{neutrofil}$ & 0.782 (0.761--0.801) & 0.775 (0.759--0.792) \\
\hline
$DQ_{neoplastic}$ & 0.706 (0.692--0.719) & 0.727 (0.712--0.741) \\
\hline
$DQ_{lymphocyte}$ & 0.675 (0.656--0.698) & 0.713 (0.691--0.734) \\
\hline
$DQ_{connective}$ & 0.566 (0.546--0.584) & 0.583 (0.565--0.602) \\
\hline
$DQ_{dead}$ & 0.410 (0.361--0.465) & 0.435 (0.306--0.561) \\
\hline
$DQ_{epithelial}$ & 0.668 (0.639--0.694) & 0.673 (0.644--0.702) \\
\hline
$DQ_{macrophage}$ & 0.657 (0.583--0.727) & 0.615 (0.531--0.703) \\
\hline
$DQ_{neutrofil}$ & 0.691 (0.625--0.753) & 0.729 (0.679--0.778) \\
\hline
\end{tabular}
%
%}
\end{table}

\clearpage

\subsection{\label{chap:S7}QuPath integration method}
We adopt an integration strategy leveraging the paquo \cite{Bayer_AG} library, a Python package that enables direct interaction with QuPath’s internal API, thereby facilitating seamless data exchange without intermediate conversion steps. The data processing pipeline (\hyperref[fig:S7]{Appendix Figure S7}) begins with the acquisition of WSIs and their associated annotations from QuPath, which are represented as Shapely \cite{Gillies_Wel_etal._2024} polygons. Utilizing paquo, we directly read, create, and modify these annotations and detections within a QuPath project in the Python environment. Images are then cropped using these polygons and processed by cell segmentation and classification models employing standard vision processing toolkits such as OpenCV, pyvips, and PyTorch. Additionally, QuPath employs Groovy scripts to initiate a Python process that starts the entire pipeline from QuPath graphical interface: fetching polygons, extracting images from them, and running deep learning model inference on the cropped images. 
The results are returned to QuPath, leveraging paquo's Python bindings to manipulate QuPath data while minimizing the computational overhead typically associated with cross-environment communication.

\counterwithin{figure}{subsection}
\renewcommand{\thefigure}{S\arabic{subsection}}

\begin{figure}[h!]
    \centering
    \includegraphics[width=\textwidth]{images/A7.pdf}
    \caption{QuPath integration workflow using Python environment}
    \label{fig:S7}
\end{figure}

Compared to traditional workflows that involve exporting annotations as GeoJSON, classifying them in Python, and reimporting them into QuPath, our approach offers several advantages. We eliminate the need to switch between programming languages, providing a cohesive and streamlined development process entirely within QuPath software and removing the necessity to use other tools. Meanwhile, we avoid storing annotations as intermediate JSON files unless required for external use or archiving. By conducting the entire inference and post-processing workflow within the Python environment, we leverage the power and flexibility of Python libraries for image processing and machine learning. This approach also enables adjustments to any set of labels and models, thereby improving its applicability.

%\hfill

The distilled model and QuPath integration code are packaged into a Docker container, enabling streamlined execution with the Docker engine. Detailed integration code and deployment instructions can be found in the GitHub repository \cite{Shvetsov_2025b}.

Despite these benefits, we acknowledge that the paquo library is a proof‑of‑concept project in its early development stage and has not been tested across all versions of QuPath.

\clearpage

\subsection{\label{chap:S8}Data and code availability statement}
All datasets, models, and code used in this study are publicly available and can be obtained from the repositories listed below. 
The PanNuke \cite{Gamper_Koohbanani_etal._2019} and MoNuSAC \cite{Verma_Kumar_etal._2021} datasets are publicly accessible, and download information along with detailed descriptions can be found in their respective articles. Preprocessing scripts for PanNuke and MoNuSAC data, as well as individual cell extraction scripts, are available on GitHub \cite{Shvetsov_2025a}. The H-Optimus foundation model used in our experiments can be downloaded from the HuggingFace repository \cite{hoptimus2024}, and model information is available on GitHub \cite{Saillard_Jenatton_etal._2024}. In addition, the integration code for QuPath and the distilled model packaged in a Docker container are provided in the repository \cite{Shvetsov_2025b}, and paquo Python library is available from the authors GitHub repository \cite{Bayer_AG}.
\clearpage

\end{document}



%%%%%%%%%%%%%%%%%%%%%%%%%%%%%%%%%%%%%%%%%%%%%%%%%%%%%%%%%%%%%%%%%%%%%%%%%%%%%%%
%%%%%%%%%%%%%%%%%%%%%%%%%%%%%%%%%%%%%%%%%%%%%%%%%%%%%%%%%%%%%%%%%%%%%%%%%%%%%%%
% APPENDIX
%%%%%%%%%%%%%%%%%%%%%%%%%%%%%%%%%%%%%%%%%%%%%%%%%%%%%%%%%%%%%%%%%%%%%%%%%%%%%%%
%%%%%%%%%%%%%%%%%%%%%%%%%%%%%%%%%%%%%%%%%%%%%%%%%%%%%%%%%%%%%%%%%%%%%%%%%%%%%%%
\newpage
\appendix
\onecolumn

\section{Algorithm}
\begin{algorithm}[H]
\caption{S2TX: State-Space Transformer With Cross Attention}
\textbf{Input:} Loss function $\mathcal{L}$, global model $g_{\phi}$, local model $f_{\psi}$, number of total training epochs $T$, dataset $D$, learning rate $\eta$, global patch length, stride, and window $PL_g$, $Str_g$, $L$, local patch length, stride, and window $PL_l$, $Str_l$, $S$. \\
\textbf{Output:} $\phi$, $\psi$.
\begin{algorithmic}[1]
\State Initialize parameters $\phi$, $\psi$.
\For{$i \gets 0$ to $T - 1$}
\State Shuffle dataset $D$.
    \For{each minibatch ($X,Y$)$\subset$ D} \comment{size}
        \State $\tilde{X}_g$, $\tilde{X}_l$ $\gets$ Patchify($X$;$PL_g$, $Str_g$,$L$), Patchify($X$;$PL_l$, $Str_l$, $S$)
        \State $Y_g \gets g_{\phi}(\tilde{X}_g)$
        \State $\hat{Y}\gets f_{\psi}(Y_g, \tilde{X}_l)$
        \State $\phi \gets \phi - \eta\nabla_{\phi}\mathcal{L}(\hat{Y},Y)$
        \State $\psi \gets \psi - \eta\nabla_{\psi}\mathcal{L}(\hat{Y},Y)$
    \EndFor
\EndFor
\State \textbf{return} $\phi$, $\psi$
\end{algorithmic}
\end{algorithm}

\section{Dataset Description}
\label{appendix:data_desc}

In this section, we describe the dataset used in our experiments in Table \ref{tab:performance}. Our experiments include 7 widely used real world multivariate time series. Table \ref{tab:datasummary} presents the number of variables and number of timesteps. 
\begin{itemize}
    \item The ETT dataset \citep{zhou2021informer} records 7 factors that related to electric transformers from July 2016 to July 2018. The ETT dataset includes 4 subsets where ETTh1 and ETTh2 are recorded hourly and ETTm1 and ETTm2 are recorded every 15 minutes. 
    \item The exchange dataset \citep{lai2018modeling} tracks the daily exchange rates of
    eight foreign countries including Australia, British, Canada,
    Switzerland, China, Japan, New Zealand, and Singapore ranging from 1990 to 2016.
    \item The weather dataset \citep{wu2021autoformer} includes 21 different meteorological features measured every 10 minutes by the Weather Station at the Max Planck Institute for Biogeochemistry.
    \item The ECL dataset \citep{lai2018modeling} records electricity consumption in kWh every 15 minutes from 2012 to 2014, for 321 clients. The data is converted to reflect hourly consumption.
\end{itemize}

\begin{table}[ht]
    \centering
    \footnotesize
    \renewcommand{\arraystretch}{1.2}
    \setlength{\tabcolsep}{8pt}
    \begin{tabular}{lccccccc}  % l = left, c = center, r = right
        \toprule
                   &ETTh1&ETTh2&ETTm1&ETTm2&Exchange& Weather & ECL\\
        \midrule
        \# Variables &7&7&7&7&8&21&321 \\
        \# Time steps & 17420& 17420 & 69680 & 69680 &  7588 & 52696 & 26304 \\
        \bottomrule
    \end{tabular}
    \caption{Table of Dataset summary including number of variables and number of time steps of each dataset. }
    \label{tab:datasummary}
\end{table}


\section{Comparison with more benchmark architectures}
\label{appendix:full_comparison}
In this section we present the full table of comparison that including two more baselines: Dlinear and FEDformer. Table \ref{tab:full_performance} is organized similarly as Table \ref{tab:performance}. 

\begin{table*}[htbp]

\centering
\renewcommand{\arraystretch}{1.1} % Adjust row height
\setlength{\tabcolsep}{3pt} % Adjust column spacing
\adjustbox{max width=\textwidth}{
\begin{tabular}{lllllllllllllllllllllll}
\toprule
 & \multicolumn{2}{c}{\textbf{S2TX}} & \multicolumn{2}{c}{\textbf{SST}} & \multicolumn{2}{c}{\textbf{S-Mamba}} & \multicolumn{2}{c}{\textbf{TimeM}} & \multicolumn{2}{c}{\textbf{iTrans}} & \multicolumn{2}{c}{\textbf{RLinear}} & \multicolumn{2}{c}{\textbf{PatchTST}} & \multicolumn{2}{c}{\textbf{CrossF}} & \multicolumn{2}{c}{\textbf{TimesNet}} & \multicolumn{2}{c}{\textbf{DLinear}} & \multicolumn{2}{c}{\textbf{FEDformer}} \\
 & \textbf{MSE} & \textbf{MAE} & \textbf{MSE} & \textbf{MAE} & \textbf{MSE} & \textbf{MAE} & \textbf{MSE} & \textbf{MAE} & \textbf{MSE} & \textbf{MAE} & \textbf{MSE} & \textbf{MAE} & \textbf{MSE} & \textbf{MAE} & \textbf{MSE} & \textbf{MAE} & \textbf{MSE} & \textbf{MAE} & \textbf{MSE} & \textbf{MAE} & \textbf{MSE} & \textbf{MAE}  \\ \midrule
\textbf{ETTh1} & & & & & & & & & & & & & & & & & & & & \\ 
96 &\textbf{0.376} &0.401&\underline{0.381} & 0.405 & 0.392 & \textbf{0.390} & 0.389 & 0.402 & 0.386 & 0.405 & 0.386 & \underline{0.395} & 0.414 & 0.419 & 0.423 & 0.448 & 0.384 & 0.402 & 0.386 & 0.400 & 0.376 & 0.419 \\ 
192 &\textbf{0.414}&\textbf{0.421}& \underline{0.430} & 0.434 & 0.449 & 0.439 & 0.435 & 0.440 & 0.441 & 0.436 & 0.437& \underline{0.424} & 0.460& 0.445 & 0.450 & 0.471 & 0.474 & 0.429 & 0.437 & 0.432 & \underline{0.420} & 0.448 \\ 
336&\textbf{0.432}&\textbf{0.435} & \underline{0.443} & \underline{0.446} & 0.467 & 0.481 & 0.450 & 0.448 & 0.487 & 0.458 & 0.479 & 0.446 & 0.501 & \underline{0.466} & 0.570 & 0.546 & 0.491 & 0.469 & 0.481 & 0.459 & 0.459 & 0.465 \\ 
720&\textbf{0.463}& \underline{0.473}& 0.502 & 0.501 & \underline{0.475} & 0.468 & 0.480 & \textbf{0.465} & 0.503 & 0.491 & 0.481 & 0.470 & 0.500 & 0.488 & 0.653 & 0.621 & 0.521 & 0.500 & 0.519 & 0.516 & 0.506 & 0.507 \\ \midrule
\textbf{ETTh2} & & & & & & & & & & & & & & & & & & & & \\ 
96&\textbf{0.279}& \underline{0.340}& 0.291 & 0.346 & 0.292 & 0.357 & 0.296 & 0.349 & 0.297 & 0.349 & \underline{0.288} & \textbf{0.338} & 0.302 & 0.348 & 0.745 & 0.584 & 0.340 & 0.374 & 0.333 & 0.387 & 0.358 & 0.397 \\ 
192&\textbf{0.362}&\underline{0.395} & \underline{0.369} & 0.397 & 0.380 & 0.402 & 0.371 & 0.400 & 0.380 & 0.400 & 0.374 & \textbf{0.390} & 0.388 & 0.400 & 0.877 & 0.656 & 0.402 & 0.414 & 0.477 & 0.476 & 0.429 & 0.439 \\ 
336&\textbf{0.337}& \textbf{0.385}& \underline{0.374} & \underline{0.414} & 0.391 & 0.420 & 0.402 & 0.449 & 0.428 & 0.432 & 0.415 & 0.426 & 0.426 & 0.433 & 1.043 & 0.731 & 0.452 & 0.452 & 0.594 & 0.541 & 0.496 & 0.487 \\ 
720&\textbf{0.395}&\textbf{0.430} & \underline{0.419} & 0.447 & 0.437 & 0.455 & 0.425 & \underline{0.438} & 0.427 & 0.445 & 0.420 & 0.440 & 0.431 & 0.446 & 1.104 & 0.763 & 0.462 & 0.468 & 0.831 & 0.657 & 0.463 & 0.474 \\ \midrule
\textbf{ETTm1} & & & & & & & & & & & & & & & & & & & & \\ 
96&\textbf{0.289}& \textbf{0.343}& \underline{0.298} & \underline{0.355} & 0.311 & 0.380 & 0.312 & 0.371 & 0.334 & 0.368 & 0.355 & 0.376 & 0.329 & 0.367 & 0.404 & 0.426 & 0.338 & 0.375 & 0.345 & 0.372 & 0.379 & 0.419 \\ 
192&\textbf{0.338}&\textbf{0.371} & \underline{0.347} & \underline{0.381} & 0.389 & 0.419 & 0.365 & 0.409 & 0.377 & 0.391 & 0.391 & 0.392 & 0.367 & 0.385 & 0.450 & 0.451 & 0.374 & 0.387 & 0.380 & 0.389 & 0.426 & 0.441 \\ 
336&\textbf{0.370}&\textbf{0.390} & \underline{0.374} & \underline{0.397} & 0.401 & 0.417 & 0.421 & 0.410 & 0.426 & 0.420 & 0.424 & 0.415 & 0.399 & 0.410 & 0.532 & 0.515 & 0.410 & 0.411 & 0.413 & 0.413 & 0.445 & 0.459 \\ 
720 &\textbf{0.423}&\textbf{0.418}& \underline{0.429} & \underline{0.428} & 0.488 & 0.476 & 0.496 & 0.437 & 0.491 & 0.459 & 0.487 & 0.450 & 0.454 & 0.439 & 0.666 & 0.589 & 0.478 & 0.450 & 0.474 & 0.453 & 0.543 & 0.490 \\ \midrule
\textbf{ETTm2} & & & & & & & & & & & & & & & & & & & & \\ 
96&\textbf{0.168}& \underline{0.260}& 0.176 & 0.264 & 0.191 & 0.301 & 0.185 & 0.290 & 0.180 & 0.264 & 0.182 & 0.265 & \underline{0.175} & \textbf{0.259} & 0.287 & 0.366 & 0.187 & 0.267 & 0.193 & 0.292 & 0.203 & 0.287 \\ 
192&\underline{0.235}&\textbf{0.298} & \textbf{0.231} & 0.303 & 0.253 & 0.312 & 0.292 & 0.309 & 0.250 & 0.309 & 0.246 & 0.304 & 0.241 & \underline{0.302} & 0.414 & 0.492 & 0.249 & 0.309 & 0.284 & 0.362 & 0.269 & 0.328 \\ 
336&\textbf{0.274}&\textbf{0.327} & \underline{0.290} & \underline{0.339} & 0.298 & 0.342 & 0.321 & 0.367 & 0.311 & 0.348 & 0.307 & 0.342 & 0.305 & 0.343 & 0.597 & 0.542 & 0.321 & 0.351 & 0.369 & 0.427 & 0.325 & 0.366 \\ 
720&\textbf{0.376}&\textbf{0.393}& \underline{0.388} & \underline{0.398} & 0.409 & 0.407 & 0.401 & 0.400 & 0.412 & 0.407 & 0.407 & 0.398 & 0.402 & 0.400 & 1.730 & 1.042 & 0.408 & 0.403 & 0.554 & 0.522 & 0.421 & 0.415 \\ \midrule
\textbf{Exchange} & & & & & & & & & & & & & & & & & & & & \\ 
96&\textbf{0.085} &\underline{0.205} &0.097 &0.222 &\underline{0.086}&0.206 & 0.089&0.208 &0.091 &0.211 & 0.088&0.209 &0.087 &\textbf{0.202} &0.095 &0.218 &0.093 & 0.211& 0.101& 0.223&0.105&0.226\\ 
192&\textbf{0.179} & \textbf{0.303}& 0.191&0.315 &0.182 &0.304 &0.184 &0.309 &0.182 & 0.303&0.188 & 0.311&\underline{0.180} &0.305 &0.193 &0.318 &0.194 &0.315 & 0.203&0.324&0.211&0.338 \\ 
336& \textbf{0.311}&\textbf{0.402} & 0.337&0.424 &0.330&0.416&0.333&0.416 &0.337 & 0.421&0.346 & 0.423& \underline{0.318}& \underline{0.407}&0.359 &0.429 &0.358 &0.433 & 0.369& 0.445&0.370&0.441\\ 
720& \textbf{0.858}&\textbf{0.696}&0.877 & 0.706&0.865 & 0.702& 0.870&\underline{0.701}&\underline{0.862} &0.703 & 0.913& 0.717&0.863 &0.703 &0.918 &0.721 &0.880 &0.719 &0.909 & 0.711&0.912&0.718\\\midrule
\textbf{Weather} & & & & & & & & & & & & & & & & & & & & \\ 
96&\textbf{0.150}& \textbf{0.199}& \underline{0.153} & \underline{0.205} & 0.169 & 0.221 & 0.174 & 0.218 & 0.174 & 0.214 & 0.192 & 0.232 & 0.177 & 0.218 & 0.158 & 0.230 & 0.172 & 0.220 & 0.196 & 0.255 & 0.217 & 0.296 \\ 
192&\textbf{0.194}&\textbf{0.242} & \underline{0.196} & \underline{0.244} & 0.205 & 0.248 & 0.200 & 0.258 & 0.221 & 0.254 & 0.240 & 0.271 & 0.225 & 0.259 & 0.206 & 0.277 & 0.219 & 0.261 & 0.237 & 0.296 & 0.276 & 0.336 \\ 
336&\underline{0.252}&\underline{0.288} & \textbf{0.246} & \textbf{0.283} & 0.288 & 0.299 & 0.280 & 0.299 & 0.278 & 0.296 & 0.292 & 0.307 & 0.278 & 0.297 & 0.272 & 0.335 & 0.280 & 0.306 & 0.283 & 0.335 & 0.339 & 0.380 \\ 
720&\textbf{0.313}&\textbf{0.333} & \underline{0.314} & \underline{0.334} & 0.335 & 0.369 & 0.352 & 0.359 & 0.358 & 0.347 & 0.364 & 0.353 & 0.354 & 0.348 & 0.398 & 0.418 & 0.365 & 0.359 & 0.345 & 0.381 & 0.403 & 0.428 \\ \midrule
\textbf{ECL} & & & & & & & & & & & & & & & & & & & & \\ 
96&\textbf{0.134}& \textbf{0.231}& \underline{0.141} & \underline{0.239} & 0.157 & 0.255 & 0.156 & 0.240 & 0.148 & 0.240 & 0.201 & 0.281 & 0.181 & 0.270 & 0.219 & 0.314 & 0.168 & 0.272 & 0.197 & 0.282 & 0.193 & 0.308 \\ 
192&\textbf{0.153}& \textbf{0.248}& \underline{0.159} & \underline{0.255} & 0.188 & 0.271 & 0.161 & 0.268 & 0.162 & 0.253 & 0.201 & 0.283 & 0.188 & 0.274 & 0.231 & 0.322 & 0.184 & 0.289 & 0.196 & 0.285 & 0.201 & 0.315 \\ 
336&\textbf{0.170}&\textbf{0.266} & \underline{0.171} & \underline{0.268} & 0.192 & 0.275 & 0.195 & 0.272 & 0.178 & 0.269 & 0.215 & 0.298 & 0.204 & 0.293 & 0.246 & 0.337 & 0.198 & 0.300 & 0.209 & 0.301 & 0.214 & 0.329 \\ 
720&\textbf{0.201}&\textbf{0.293} & \underline{0.208} & \underline{0.300} & 0.241 & 0.339 & 0.231 & 0.307 & 0.225 & 0.317 & 0.257 & 0.331 & 0.246 & 0.324 & 0.280 & 0.363 & 0.220 & 0.320 & 0.245 & 0.333 & 0.246 & 0.355 \\ \midrule
% Average& 0.304&0.349&0.315&
 \toprule
% Continue for Weather, ECL, and Traffic
\end{tabular}
}
\caption{Comprehensive comparison across various dataset with additional baselines. The \textbf{bolded} results denote the best performance, and the \underline{underlined} results indicate the second best.}
\label{tab:full_performance}
\end{table*}




\label{appendix:impl_detail}
%%%%%%%%%%%%%%%%%%%%%%%%%%%%%%%%%%%%%%%%%%%%%%%%%%%%%%%%%%%%%%%%%%%%%%%%%%%%%%%
%%%%%%%%%%%%%%%%%%%%%%%%%%%%%%%%%%%%%%%%%%%%%%%%%%%%%%%%%%%%%%%%%%%%%%%%%%%%%%%


\end{document}


% This document was modified from the file originally made available by
% Pat Langley and Andrea Danyluk for ICML-2K. This version was created
% by Iain Murray in 2018, and modified by Alexandre Bouchard in
% 2019 and 2021 and by Csaba Szepesvari, Gang Niu and Sivan Sabato in 2022.
% Modified again in 2023 and 2024 by Sivan Sabato and Jonathan Scarlett.
% Previous contributors include Dan Roy, Lise Getoor and Tobias
% Scheffer, which was slightly modified from the 2010 version by
% Thorsten Joachims & Johannes Fuernkranz, slightly modified from the
% 2009 version by Kiri Wagstaff and Sam Roweis's 2008 version, which is
% slightly modified from Prasad Tadepalli's 2007 version which is a
% lightly changed version of the previous year's version by Andrew
% Moore, which was in turn edited from those of Kristian Kersting and
% Codrina Lauth. Alex Smola contributed to the algorithmic style files.
