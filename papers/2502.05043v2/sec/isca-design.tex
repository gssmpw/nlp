\section{EcoServe Design}~\label{sec:ecoserve}
\begin{figure*}[t]
\centering     \includegraphics[width=0.8\linewidth]{sec/design_short.pdf} 
\vspace{-1em}
    \caption{EcoServe system diagram: An optimization framework that minimizes operational and embodied carbon across hardware resources through 4R co-designed strategies. Outputs inform scheduling and resource allocation decisions. }\label{fig:ecoserve}
    \vspace{-1em}
\end{figure*}
EcoServe is a carbon-aware framework for serving large language models (LLMs) that jointly optimizes performance, cost, and environmental impact through systematic resource management. The framework operates on four key principles: Reuse, Right Size, Reduce, and Recycle (4R), orchestrating both software and hardware provisioning decisions to minimize the carbon footprint of LLM inference. Figure~\ref{fig:ecoserve} shows the overall EcoServe design architecture.

EcoServe's framework ingests hardware specifications, LLM characteristics, and production traces alongside carbon intensity data to make carbon-aware scheduling decisions. The output such as number of GPUs, CPU cores for each workloads are directly usable for commodity resource scheduler or autoscaler. 

\subsection{Design Principles}
EcoServe implements four strategies for carbon-efficient inference:
\begin{itemize}[leftmargin=*]
\item \textbf{Reuse (\textit{Software Runtime}):} Leverages idle CPU capacity in existing AI inference systems for offline/batch workloads, increasing throughput while reducing resource demand and embodied carbon.
\item \textbf{Right Size (\textit{Software Provisioning}):} Optimizes GPU provisioning across heterogeneous hardware (L4, A6000, A100, H100, etc.) for both online and offline inference, matching resources to workload characteristics.
\item \textbf{Reduce (\textit{Hardware SKU Design}):} Minimizes waste by eliminating underutilized host compute, memory, and storage resources that contribute to high embodied carbon overhead.
\item \textbf{Recycle (\textit{Hardware Provisioning}):} Extends hardware lifetimes asymmetrically across host processing systems and GPUs to balance operational and embodied emissions.
\end{itemize}
These strategies provide a comprehensive solution across software runtime optimization, software provisioning, and hardware provisioning to reduce the carbon footprint of AI inference systems. Below we detail on each strategy, then present how the output of the strategy interacts with the components in real systems.
\subsubsection{\textbf{Reusing CPUs for Offline Inference}}~\label{sec:reuse}
Section~\ref{sec:characterization} shows that host processing systems incur high embodied carbon overheads in AI inference infrastructure.
To maximize the utility of these resources, EcoServe opportunistically reuses host CPUs for offline inference, reducing the reliance on high-power GPUs.

\textbf{Identifying opportunities for using CPUs.} 
Due to the underutilization of host resources, we must carefully determine which operations or inference phases can be offloaded from GPUs to CPUs to maximize carbon efficiency. Note that each offload incurs additional energy costs for data movement, necessitating a selective approach to CPU reuse. %

To guide this decision, Figure~\ref{fig:roofline} presents a {roofline model analysis~\cite{williams2009roofline,yuan2024llm,nerscRooflinePerformance}} comparing the computational intensity and memory bandwidth efficiency of an Intel Sapphire Rapids 112-core CPU and an NVIDIA A100 40GB GPU.
Given the relatively lower memory bandwidth gap between the CPU and GPU, we find that {low-arithmetic-intensity operations}, such as {attention scoring and KV cache decoding}, are well-suited for CPU execution.
Additionally, for large-batch operations, GPU throughput becomes capacity-bound, making CPU reuse {a practical alternative for offline inference.
Figure~\ref{fig:roofline} highlights the maximum batch sizes feasible for GPUs (16) and CPUs (512) at a context length of 2048 in FP16.}

\begin{figure}[t]
    \centering
    \includegraphics[width=0.85\linewidth]{plots/roofline_model_b_compare_2048_star.pdf}
    \vspace{-1em}
    \caption{ Roofline models of an Intel Sapphire Rapids CPU with 112 cores compared to an NVIDIA A100. 
    We overlay different operators in the decode (solid and dash-dot line) and prompt computation phase (dashed line) for Llama-3-8B with the maximum batch size that CPU/GPU can handle at a context length of 2048.
    At large batch sizes, the GPU is capacity bound ({\textcolor{red}{\(\star\)}}) while the CPU ({\textcolor{purple}{\(\star\)}}) can efficiently process offline decoding workloads.}
    \label{fig:roofline}
    \vspace{-2em}
\end{figure}

\textbf{Optimizing CPU performance via parallelization.}
To maximize the efficiency of CPU-based inference, EcoServe optimally configures thread-level parallelism and tensor tiling to balance:
\begin{itemize}
    \item Arithmetic intensity (FLOPs/Byte),
    \item Compute vs. memory bandwidth constraints,
    \item Model size, batch size, and sequence length.
\end{itemize}
The optimal configuration depends on the CPU microarchitecture (e.g., AMX and AVX efficiency), number format (FP6, BF6, INT8, etc), and workload characteristics.
As illustrated in Figure~\ref{fig:gemm-ai}, {parallelization along the KV sequence length dimension, in addition to the batch dimension, maximizes memory bandwidth utilization across all CPU cores.} 
This is particularly beneficial for long-context workloads, where attention mechanisms contribute significantly to inference latency~\cite{flashdecoding2023,flashinfer,dao2022flashattention,dao2023flashattention,hong2023flashdecoding++}.

Following our {roofline analysis}, while prompt computation remains GPU-favorable, EcoServe implements {layered pipelining} to move KV cache transfers from GPUs to CPUs for decoding.
By optimizing these parallelism dimensions, we achieve a {3.67$\times$ speedup} compared to a baseline Llama.cpp implementation~\cite{llama_cpp}, with an average {1.4$\times$ improvement} across model sizes and sequence lengths (Figure~\ref{fig:speedup-reuse}).
As CPU inference remains a nascent field, future advancements leveraging sparsity and weight-sharing techniques are expected to further enhance CPU-based offline inference~\cite{song2023powerinfer,zhang2024h2o,Liu2023DejaVu}.

\begin{figure}[!t]
    \centering
    \includegraphics[width=0.7\linewidth]{plots/AI_vs_BN_4096.pdf} 
    \vspace{-1em}
    \caption{CPU offline inference requires carefully balancing parallelism degree (PD) and arithmetic intensity of Linear operators (AI, FLOPS/Byte). The run-time scheduler co-designs tile and workload slice dimensions with architecture decisions to optimize offline performance.}
    \label{fig:gemm-ai}
    \vspace{-1.2em}
\end{figure}

\textbf{Adapting to fluctuating offline inference demand.}
Given their inherently relaxed SLOs, offline workloads offer an opportunity to leverage {CPU capacity during periods of low CPU utilization}.
EcoServe's reuse strategy is motivated by two key insights:
\begin{enumerate}
    \item {Offline inference demand is significant}: Figure~\ref{fig:offline} illustrates the ratio of online vs. offline demand in a real-world cloud deployment over a week (left) and a single day (right). On average, {offline workloads account for 21\% and 45\% of total infrastructure capacity} in Services A and B, respectively.
    \item {Offline workloads exhibit time-varying peaks}: At peak periods, offline demand reaches {27\% and 55\%}, respectively. By {scheduling offline inference to underutilized CPU nodes}, EcoServe {reduces peak GPU resource requirements}, thereby lowering total embodied carbon.
\end{enumerate}
Since offline workloads lack stringent latency constraints, EcoServe {batches large offline requests} on CPUs, thereby reducing the need for high-GPU provisioning.
For datacenters with high-carbon-intensity power grids, EcoServe can {dynamically reallocate workloads} back to GPUs when energy efficiency becomes a priority.

\begin{figure}[h]
    \centering
    \includegraphics[width=0.95\linewidth]{plots/offline_combined_analysis.pdf}
    \vspace{-0.2in}
    \caption{Breakdown of online and offline demand for two LLM services (A and B) running in a production cloud data center over a week (left) and during a day (right). 
    Offline demand accounts for a significant portion of total serving capacity, creating opportunities for CPU reuse in offline inference.}
    \label{fig:offline}
    \vspace{-1.2em}
\end{figure}

\textbf{Load-aware reuse.}
To quantify the impact of CPU reuse, Figure~\ref{fig:offline-result} presents the required infrastructure capacity for serving Llama-3-8B inference workloads under varying demand conditions.  
We analyze two scenarios:
\begin{itemize}
    \item {Peak-aware reuse (red):} CPUs handle offline inference \textit{only} during peak demand periods.
    \item {Continuous reuse (blue):} CPUs process offline inference \textit{at all times}.
\end{itemize}
Assuming resource reallocation occurs every {4 hours, EcoServe’s load-aware CPU reuse reduces offline GPU provisioning needs by {1.32$\times$ at peak demand}, contributing to substantial {embodied carbon savings}.
This estimate is {conservative}, assuming fixed batch sizes for CPU and GPU inference. By further increasing CPU batch sizes, offline capacity reductions of up to 45\% are achievable.}

\begin{figure}[t]
    \centering
    \includegraphics[width=0.8\linewidth]{plots/reuse_capacity_offline.pdf}
    \vspace{-0.2in}
    \caption{Impact of CPU reuse strategies on infrastructure capacity. Red and blue curves represent peak-only and continuous reuse, respectively. EcoServe dynamically reallocates workloads to CPUs, reducing offline GPU capacity demands by up to 1.32$\times$.}
    \label{fig:offline-result}
    \vspace{-2em}
\end{figure}

\subsubsection{\textbf{Right-sizing GPU Provisioning}} 
\label{sec:rightsize}
While CPU reuse amortized embodied carbon, efficient GPU provisioning is equally critical. EcoServe optimizes GPU allocation by considering three key factors: energy grid carbon intensity, workload characteristics, and hardware diversity. In regions with predominantly renewable energy, CPU reuse is prioritized; in fossil fuel-dependent regions, workloads may shift to more energy-efficient GPUs to minimize overall emissions.

LLM has distinct phases (prompting, decoding) and workload has various characteristics (architecture, sequence length)--this impose different compute and memory requirements. By dynamically matching these requirements to diverse GPU capabilities, EcoServe reduces both operational and embodied carbon while maintaining performance targets.




\textbf{Heterogeneous GPU partitioning.} 
Figure~\ref{fig:relative_carbon} compares the energy (bottom) and embodied carbon efficiency (top) of NVIDIA A100 vs. H100 for Gemma 27B across varying prompt lengths and batch sizes. The analysis separates prompt computation and decode (token generation)~\cite{patel2023splitwise}. A100 is preferable for smaller inputs and batches, while H100 becomes more efficient as they grow. However, for decode, A100 consistently outperforms H100 due to H100's low Model FLOPs Utilization (MFU), Memory Bandwidth Utilization (MBU), and high embodied carbon and energy cost.










We propose a workload-aware GPU provisioning strategy that optimizes resource allocation based on workload characteristics (not only on prompt/decode phases). Instead of provisioning only H100 for prompting (like what was done in Splitwise~\cite{patel2023splitwise}), we tailor GPU allocation per (workload slice, SLO) using offline profiling and performance modeling, distinguishing between workloads to match the available hardware heterogeneity.




\textbf{Exploiting tensor-model parallelism.}
The decision of how many levels of tensor parallelism (TP) is dependent on the target metrics. From latency perspective, sharding tensors across GPUs with high bandwidth communication links enables faster computation and more HBM bandwidth, it's generally favorable for large models to do higher level of TP (up to the tile and wave quantization effect~\cite{nvidia_matmul_guide}). However, the latency comes at a cost with more communication overheads. From carbon perspective, the TP level depends on the ratio between CPU embodied and GPU embodied carbon. Table~\ref{tab:tp} details the desiderata for TP level deployment. In practice, we tailor it to each hardware, model, batch size and SLO. %





\begin{figure}[!t]
\centering
    \begin{subfigure}[b]{\linewidth}
    \includegraphics[width=0.99\linewidth]{plots/Relative_combined.pdf}
        \label{fig:relative_carbon}
    \end{subfigure}
\vspace{-2em}    
\caption{Relative energy and carbon of prompt and decode for Gemma-27b model on NVIDIA H100 compared with A100 GPUs. The results are normalized (higher than 1 means  A100 is preferred). 
We find the optimal GPU varies based on LLM phase, context length, and batch-size, motivating the need for heterogeneous GPU \textit{rightsizing} for different workloads.}
\label{fig:relative_carbon}
\vspace{-1em}
\end{figure}

\begin{table}[h]
\centering \footnotesize
\begin{tabular}{ccc}\hline
\textbf{Metric} & \textbf{Relative Scale} & \textbf{Decision Criteria} \\ 
\hline
Power & $\frac{2n P_{GPU} + P_{CPU}}{n P_{GPU} + P_{CPU}}$ & Consider if $P_{CPU} \gg n P_{GPU}$ \\
Latency & $\approx 0.5 + C_{comm}$ & Favorable when model $>$ memory  \\
Cost/Model & $\approx 1$ & Near-constant when $C_{CPU} \ll n C_{GPU}$ \\
 Carbon & $\frac{CF^{emb}_{CPU} + n CF^{emb}_{GPU}}{CF^{emb}_{CPU}/2 + n CF^{emb}_{GPU}}$ & Better with higher $CF_{CPU}/CF_{GPU}$ \\
Energy & $\approx 0.5$ & Linear improvement with fixed CI \\
\hline 
\end{tabular} \caption{Power, latency, carbon, energy and cost when scaling up model and tensor-model parallelism.}\label{tab:tp}
\vspace{-2em}
\end{table}




 
\subsubsection{\textbf{Reduce: Host System Provision}} \label{sec:reduce}
This section examines strategies to minimize host memory subsystem resource underutilization to reduce embodied carbon waste. We focus on optimizing memory and storage resources, which comprise 36\% of embodied emissions in Azure A100 offerings\footnote{Standard\_ND96asr\_A100\_v4, the number is generally higher for higher capacity instances}. These optimizations must be balanced against the benefits of CPU reuse for offline inference, as explored in Section~\ref{sec:reuse}.


\textbf{Prefix-caching-aware DRAM reduction strategy. }
DRAM supports AI inference through:
\begin{itemize}
\item Memory offloading and intermediate data storage
\item Data processing and memory allocation
\item KV Cache swap space and prefix caching
\end{itemize}
For online inference with tight SLOs, offloading is not viable for large models. The minimum DRAM capacity is determined by:
\begin{equation}
\begin{gathered}
C^{\text{DRAM}} = C^{\text{Weight (layer)}} + C^{\text{KV/activation offload for online}} + C^{\text{KV for offline}} \\
\min C^{\text{DRAM}} = M_{k v}(n) = 4 n d h_{k v} l
\end{gathered}
\end{equation}
where $d$ is model dimension, $l$ is layer count, $n$ is P90 aggregated context length with zero reuse distance, and $h_{kv}$ is KV head dimension.
To mitigate performance impact from reduced DRAM, we use profiling to understand prefix reuse distances and determine optimal CPU cache space. We route longer-context requests to DRAM-abundant servers and leverage solutions like CacheGen~\cite{CacheGen} to offload caches across multiple instances or to SSD.


\textbf{KV-offload-aware SSD reduction strategy. }
While cloud GPU offerings typically include large SSDs, this isn't fundamental for inference serving. SSDs consume approximately 2.8~W per TB idle power (>10\% of server idle power). The minimum required SSD size approximates GPU memory capacity:
\begin{equation}
\begin{gathered}
C^{\text{SSD}} = 1.2 C^{\text{GPU}} + C^{\text{Model buffer}} + C^{\text{KV Offload for offline}} \\
\min C^{\text{SSD}} = 1.2 C^{\text{GPU}}
\end{gathered}
\end{equation}
When SSD capacity is used to download and maintain models, we imagine remote storage with GPUDirect can reduce over provisioned storage.


\subsubsection{\textbf{Recycle: Asymmetric Lifetime for Host and GPUs.}}\label{sec:recycle}
Due to the nature of fast energy efficiency improvements of GPU and slow energy improvement on hosts, the optimal lifetime $T$ is also different, necessitating an asymmetric recycling strategy. 


\textbf{Host lifetime extension.}
Extending the lifespan of DRAM, SSDs, and CPUs helps reduce both embodied and operational carbon. This recycling opportunity arises because memory and storage bandwidth have not scaled proportionally with CPU/GPU performance in recent technology nodes. Our measurements across three server generations show that DRAM and SSD bandwidth are rarely the primary bottlenecks in modern inference serving systems, allowing older components to be reused without compromising performance.



\textbf{Carbon-aware GPU upgrade.} The GPU upgrade strategy depends on 1) inter-generational GPU energy efficiency improvement, 2) the workload and model characteristics, and 3) the upfront embodied carbon cost. Upgrading from V100 to GH200 incurs an upfront embodied carbon cost but significantly improves token processing efficiency per unit of operational carbon. Over time, this reduces the relative carbon footprint, especially in datacenters with lower renewable energy availability. Figure~\ref{fig:cf-relative} shows the optimal GPU usage duration across regions with varying carbon intensity.

\begin{figure}[t]
\centering
     \includegraphics[width=0.99\linewidth]{plots/relative_saving_recycle.pdf} \vspace{-1em}
    \caption{Relative carbon savings with various hardware compared with V100. (Left) prompt heavy workload with Carbon Intensity (CI)  400 gCO2e/KWh, (Middle) prompt heavy workload or  (Right) decode heavy with CI = 50 gCO2e/KWh, The optimal hardware is different. }\label{fig:cf-relative}
    \vspace{-1.05em}
\end{figure}
\begin{figure}[h]
    \centering
    \includegraphics[width=0.8\linewidth]{plots/reliability.pdf}
     \vspace{-1.05em}\caption{Effective age with deployment time.}
    \label{fig:reliability} \vspace{-1.05em}
\end{figure}

\textbf{Reliability implications. } Production data indicates that AI inference workloads typically utilize these host components at low levels. 
 In this section, we study how utilization and lifetime impact reliability. 
1) CPU: To understand the CPU aging over 5 years in a cloud setup, we use a 7nm composite processor model from a major fabrication company. The model faithfully represents CPU transistor logic and is used for design and compliance with reliability goals; details omitted due to confidentiality. In the usual spread of observed voltage levels, we found that even at 20\% utilization over 5 years, CPU aging is only 0.8 years, indicating significant potential for extended use (Figure~\ref{fig:reliability}).

2) DRAM: A study~\cite{Siddiqua2017} from a heavily utilized Cielo supercomputer showed that even at the end of lifetime of 5 years, the DRAM error rate did not increase, indicating a higher lifetime. Other recent studies have shown that DRAM retention rate based errors only meaningfully increase after 10 years of intense usage~\cite{liu2022new}. Given the low utilization of DRAM in the cloud~\cite{kanev2015profiling}, reusing old DRAM and SSDs in CPU-only servers has been proposed by prior work as well~\cite{greensku}, but EcoServe is the first to consider how to evaluate optimal and asymmetric lifetime extension. 
 
3) SSD: Age of an SSD primarily depends on the number of program/erase cycles the SSD has been through~\cite{failure-ssd,FlashFailure,schroeder2016flash,klein2021backblaze}. Therefore, an SSD’s age is modeled as proportional to the number of writes in its lifetime. Even if we assume that the SSD is written all the time that the CPU is active, it will only age by 1 year over a period of 5 years for a 20\% utilization (shown in Figure~\ref{fig:reliability}). 
 



\subsection{Putting it all together} \label{sec:ILP}
Deploying EcoServe at scale requires balancing multiple competing objectives across carbon, energy, performance, cost, and reliability. The interactions between our optimization strategies present complex trade-offs. For instance, aggressive CPU \textit{Reuse} may conflict with \textit{Reduce} goals, requiring careful capacity planning for CPU cores and memory subsystems.

To address these challenges, we propose a hierarchical approach:

\begin{enumerate}
    \item At the capacity planning phase, we provision a spectrum of server configurations optimized for different carbon-performance trade-offs based on offline profiling. Some servers emphasize \textit{Reuse} with expanded memory subsystems, while others focus on \textit{Reduce} with lean configurations.
    \item At runtime, a carbon-aware load balancer leverages these diverse configurations to optimize workload placement, request scheduling and resource allocation. The load balancer considers:
    \begin{itemize}
        \item Server configuration profiles
        \item Real-time workload characteristics and system load
        \item Dynamic KV cache transfer costs
    \end{itemize}
\end{enumerate}


\subsubsection{Offline profiling and online adaptation}
As shown in Figure~\ref{fig:ecoserve}, EcoServe optimizes LLM deployment by ingesting models, workload traces, SLOs, and hardware specs. It bootstraps initial deployment using CPU/GPU profiling data, then continuously adapts to changing workload patterns. The framework maintains separate resource pools for online, mixed and offline inference, with dynamic CPU offloading for decode phases. Pool sizes automatically adjust via periodically triggered ILP based on workload demands and carbon intensity.%

\subsubsection{ILP for co-design scheduling and allocation}

\textbf{Workload Slicing and Disaggregation}: We represent the workload in different slices within each phase {prompt} and {decode}. Let \( H(i,o) \) represent the request rate for inputs of length \( i \) and outputs of length \( o \). We put workload into histogram bucket \( b \), and further divided into \( S \) slices for fine-grained allocation, where slice factor \( f \) determines the granularity. For a bucket with request rate \( \lambda_b \), each slice \( s \) has rate \( \lambda_s = \lambda_b / f \). For each slice \( s \in S \), we define $\lambda_s$ as the request rate, $\text{Lat}(s,g,l, \phi_s,m_s)$ as expected latency for the prompt or decode phase on GPU \( g \) with allocated resources at load \( l \), required CPU cores \( \phi_s \), and memory \( m_s \). The load imposed by slice \( s \) on GPU type \( g \) is calculated separately for each phase:
\[\footnotesize
\text{Load}_{p/d}(s,g) = \frac{\lambda_s}{\text{MaxTput}_{p/d}(g,\text{size}(s),\text{SLO})}
\]
where \( \text{size}(s) \) represents the input/output length pair of slice \( s \).

\textbf{Performance and Carbon Model}: For each GPU type \( g \in G \), we define \( c_g,  c_\phi, c_m \) as hourly GPU, CPU, memory cost, \( \gamma_g(t) \) as GPU carbon intensity at time \( t \), which is both linear with operational power of the workload times the carbon intensity, and with unit-time embodied carbon of the hardware.
The carbon impact of a workload is the sum of both phases $\text{Carbon}(s,g,l, \phi_s,m_s) = \sum_{k \in \{p,d\}} \gamma_g(t) \cdot \text{Lat}_{k}(s,g,l,\phi_s,m_s)$. 




\textbf{ILP Formulation}: Our goal is to minimize the weighted carbon cost and latency-dependent hardware cost under service level objectives (SLO) based on both TTFT (time to first token) and TPOT (time per output token). Decision variables include GPU assignment matrix \( A \in \{0,1\}^{N \times M} \), GPU count vector \( B \in \mathbb{Z}_{\geq 0}^M \); CPU cores and memory allocated to slice \( s \) as \( \Phi_s, M_s  \).
\label{sec:optimization}
{\footnotesize
\[
\min_{A,B,\Phi,M} (1-\alpha) \Big[\sum_{j=1}^M B_j c_j + \sum_{s=1}^{N} (\Phi_s c_\phi + M_s c_m) \Big] + \alpha \sum_{s=1}^N \sum_{j=1}^M A_{s,j} \text{Carbon}(s,j,l, \Phi_s,M_s)
\]
\vspace{-2.em}
\begin{align*} 
\textbf{s.t.}\quad & \forall s, \quad \sum_{j=1}^M A_{sj} = 1 \\
& \forall j, \quad \sum_{s=1}^N A_{s,j} \cdot (\text{Load}_{p}(s,j) + \text{Load}_{d}(s,j)) \leq B_j \\
& \sum_{s=1}^N \Phi_s \leq \Phi, \quad \sum_{s=1}^N M_s \leq M \, \quad \forall s, \quad \Phi_s \geq \phi_s, \quad M_s \geq m_s \\
& \forall s,j \text{ where } A_{s,j} = 1, \quad \text{Lat}_{p/d}(s,j,l, \Phi_s,M_s) \leq \text{SLO}_{p/d}
\end{align*}
}\vspace{-2em}

where $\alpha \in (0,1)$ is a weighting variable between cost and carbon objectives. Unless otherwise stated, we choose $\alpha = 1$. A smaller value such as $\alpha = 0$ will simply optimize for cost. 

