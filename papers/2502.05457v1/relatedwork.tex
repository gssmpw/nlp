\section{Related Work}
\label{subsec:relatedWorks}
% Tracking based methods
\indent Most of the proposed works for scene understanding can be broadly grouped into two categories. Under the first category, objects in the scene are first detected and then tracked to obtain their trajectories in the scene  \cite{ref_17, ref_18}. Object trajectories are trained and used for scene understanding analysis. For example, motions are learned using clustering in \cite{ref_17}, where object trajectories are hierarchically clustered using spectral clustering, and in the second stage, each group of trajectories is further clustered into subcategories using the temporal information. Tracking based approaches are prone to tracking errors. In these approaches, it is likely to lose an object for several frames, especially in the crowded scenes, which results in an incomplete trajectory for that object \cite{ref_7}.\\ 
%
% Pixel based methods (change under the second category)
\indent Under the second category, visual features are extracted from video sequence without any object detection and tracking. In most of these pixel-based approaches, learning algorithms are used to index the video stream by high-level semantically meaningful patterns. There are several methods that are based on topic model which model the scene with high-level patterns \cite{ref_7,ref_9,ref_15,ref_16}. These methods succeed in indexing videos but none of them address the retrieval task in surveillance videos directly except for the work presented in \cite{ref_7}, which is based on Hierarchical Dirichlet Process (HDP) topic model \cite{ref_21}. In their work, the video stream is decomposed into clips each of them is annotated with a distribution over the learned topics. To search for an activity, they define the user query as a distribution over the learned topics. The clips within the database are then compared with the query distribution using the Kullback-Leibler divergence to find the clips that involve the user query. \\
%
% The low-level feature based method
\indent  Among pixel-based approaches, in method presented in the video is indexed with low-level visual features, unlike the scene understanding approaches in which the video is indexed with high-level patterns. They use Locality-sensitive hashing (LSH) to hash the extracted low-level features into a lightweight lookup table. They define the user query as a set of action components each of them composed of a set of structured low-level features called tree. After defining the query and the region of interest (ROI), a set of trees is assigned to each action component and the full responses are found using dynamic programming. 
%
% Framework figure
\begin{figure}[!t]
  \centering
  \includegraphics[width=0.43\textwidth]{images/Framework.pdf}
\caption{The proposed method framework. In the learning phase, blob decomposition is performed on extracted features, and new feature vectors are fed into the corresponding LDA topic model. Topics of the primary model are processed to form the secondary model with primitive topics.  The user query formulation is parsed into the internal representation of the system, and the search procedure is performed using the proposed search strategies.}
\label{fig:framework}
\end{figure}
%