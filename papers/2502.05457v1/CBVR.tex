
%% bare_jrnl.tex
%% V1.4b
%% 2015/08/26
%% by Michael Shell
%% see http://www.michaelshell.org/
%% for current contact information.
%%
%% This is a skeleton file demonstrating the use of IEEEtran.cls
%% (requires IEEEtran.cls version 1.8b or later) with an IEEE
%% journal paper.
%%
%% Support sites:
%% http://www.michaelshell.org/tex/ieeetran/
%% http://www.ctan.org/pkg/ieeetran
%% and
%% http://www.ieee.org/

%%*************************************************************************
%% Legal Notice:
%% This code is offered as-is without any warranty either expressed or
%% implied; without even the implied warranty of MERCHANTABILITY or
%% FITNESS FOR A PARTICULAR PURPOSE! 
%% User assumes all risk.
%% In no event shall the IEEE or any contributor to this code be liable for
%% any damages or losses, including, but not limited to, incidental,
%% consequential, or any other damages, resulting from the use or misuse
%% of any information contained here.
%%
%% All comments are the opinions of their respective authors and are not
%% necessarily endorsed by the IEEE.
%%
%% This work is distributed under the LaTeX Project Public License (LPPL)
%% ( http://www.latex-project.org/ ) version 1.3, and may be freely used,
%% distributed and modified. A copy of the LPPL, version 1.3, is included
%% in the base LaTeX documentation of all distributions of LaTeX released
%% 2003/12/01 or later.
%% Retain all contribution notices and credits.
%% ** Modified files should be clearly indicated as such, including  **
%% ** renaming them and changing author support contact information. **
%%*************************************************************************


% *** Authors should verify (and, if needed, correct) their LaTeX system  ***
% *** with the testflow diagnostic prior to trusting their LaTeX platform ***
% *** with production work. The IEEE's font choices and paper sizes can   ***
% *** trigger bugs that do not appear when using other class files.       ***                          ***
% The testflow support page is at:
% http://www.michaelshell.org/tex/testflow/



\documentclass[journal]{IEEEtran}
%
% If IEEEtran.cls has not been installed into the LaTeX system files,
% manually specify the path to it like:
% \documentclass[journal]{../sty/IEEEtran}





% Some very useful LaTeX packages include:
% (uncomment the ones you want to load)


% *** MISC UTILITY PACKAGES ***
%
%\usepackage{ifpdf}
% Heiko Oberdiek's ifpdf.sty is very useful if you need conditional
% compilation based on whether the output is pdf or dvi.
% usage:
% \ifpdf
%   % pdf code
% \else
%   % dvi code
% \fi
% The latest version of ifpdf.sty can be obtained from:
% http://www.ctan.org/pkg/ifpdf
% Also, note that IEEEtran.cls V1.7 and later provides a builtin
% \ifCLASSINFOpdf conditional that works the same way.
% When switching from latex to pdflatex and vice-versa, the compiler may
% have to be run twice to clear warning/error messages.






% *** CITATION PACKAGES ***
%
%\usepackage{cite}
% cite.sty was written by Donald Arseneau
% V1.6 and later of IEEEtran pre-defines the format of the cite.sty package
% \cite{} output to follow that of the IEEE. Loading the cite package will
% result in citation numbers being automatically sorted and properly
% "compressed/ranged". e.g., [1], [9], [2], [7], [5], [6] without using
% cite.sty will become [1], [2], [5]--[7], [9] using cite.sty. cite.sty's
% \cite will automatically add leading space, if needed. Use cite.sty's
% noadjust option (cite.sty V3.8 and later) if you want to turn this off
% such as if a citation ever needs to be enclosed in parenthesis.
% cite.sty is already installed on most LaTeX systems. Be sure and use
% version 5.0 (2009-03-20) and later if using hyperref.sty.
% The latest version can be obtained at:
% http://www.ctan.org/pkg/cite
% The documentation is contained in the cite.sty file itself.






% *** GRAPHICS RELATED PACKAGES ***
%
\ifCLASSINFOpdf
   \usepackage[pdftex]{graphicx}
  % declare the path(s) where your graphic files are
  % \graphicspath{{../pdf/}{../jpeg/}}
  % and their extensions so you won't have to specify these with
  % every instance of \includegraphics
  % \DeclareGraphicsExtensions{.pdf,.jpeg,.png}
\else
  % or other class option (dvipsone, dvipdf, if not using dvips). graphicx
  % will default to the driver specified in the system graphics.cfg if no
  % driver is specified.
  % \usepackage[dvips]{graphicx}
  % declare the path(s) where your graphic files are
  % \graphicspath{{../eps/}}
  % and their extensions so you won't have to specify these with
  % every instance of \includegraphics
  % \DeclareGraphicsExtensions{.eps}
\fi
% graphicx was written by David Carlisle and Sebastian Rahtz. It is
% required if you want graphics, photos, etc. graphicx.sty is already
% installed on most LaTeX systems. The latest version and documentation
% can be obtained at: 
% http://www.ctan.org/pkg/graphicx
% Another good source of documentation is "Using Imported Graphics in
% LaTeX2e" by Keith Reckdahl which can be found at:
% http://www.ctan.org/pkg/epslatex
%
% latex, and pdflatex in dvi mode, support graphics in encapsulated
% postscript (.eps) format. pdflatex in pdf mode supports graphics
% in .pdf, .jpeg, .png and .mps (metapost) formats. Users should ensure
% that all non-photo figures use a vector format (.eps, .pdf, .mps) and
% not a bitmapped formats (.jpeg, .png). The IEEE frowns on bitmapped formats
% which can result in "jaggedy"/blurry rendering of lines and letters as
% well as large increases in file sizes.
%
% You can find documentation about the pdfTeX application at:
% http://www.tug.org/applications/pdftex





% *** MATH PACKAGES ***
%
\usepackage{amsmath}
% A popular package from the American Mathematical Society that provides
% many useful and powerful commands for dealing with mathematics.
%
% Note that the amsmath package sets \interdisplaylinepenalty to 10000
% thus preventing page breaks from occurring within multiline equations. Use:
%\interdisplaylinepenalty=2500
% after loading amsmath to restore such page breaks as IEEEtran.cls normally
% does. amsmath.sty is already installed on most LaTeX systems. The latest
% version and documentation can be obtained at:
% http://www.ctan.org/pkg/amsmath





% *** SPECIALIZED LIST PACKAGES ***
%
%\usepackage{algorithmic}
% algorithmic.sty was written by Peter Williams and Rogerio Brito.
% This package provides an algorithmic environment fo describing algorithms.
% You can use the algorithmic environment in-text or within a figure
% environment to provide for a floating algorithm. Do NOT use the algorithm
% floating environment provided by algorithm.sty (by the same authors) or
% algorithm2e.sty (by Christophe Fiorio) as the IEEE does not use dedicated
% algorithm float types and packages that provide these will not provide
% correct IEEE style captions. The latest version and documentation of
% algorithmic.sty can be obtained at:
% http://www.ctan.org/pkg/algorithms
% Also of interest may be the (relatively newer and more customizable)
% algorithmicx.sty package by Szasz Janos:
% http://www.ctan.org/pkg/algorithmicx




% *** ALIGNMENT PACKAGES ***
%
%\usepackage{array}
% Frank Mittelbach's and David Carlisle's array.sty patches and improves
% the standard LaTeX2e array and tabular environments to provide better
% appearance and additional user controls. As the default LaTeX2e table
% generation code is lacking to the point of almost being broken with
% respect to the quality of the end results, all users are strongly
% advised to use an enhanced (at the very least that provided by array.sty)
% set of table tools. array.sty is already installed on most systems. The
% latest version and documentation can be obtained at:
% http://www.ctan.org/pkg/array


% IEEEtran contains the IEEEeqnarray family of commands that can be used to
% generate multiline equations as well as matrices, tables, etc., of high
% quality.




% *** SUBFIGURE PACKAGES ***
\ifCLASSOPTIONcompsoc
  \usepackage[caption=false,font=normalsize,labelfont=sf,textfont=sf]{subfig}
\else
  \usepackage[caption=false,font=footnotesize]{subfig}
%\fi
% subfig.sty, written by Steven Douglas Cochran, is the modern replacement
% for subfigure.sty, the latter of which is no longer maintained and is
% incompatible with some LaTeX packages including fixltx2e. However,
% subfig.sty requires and automatically loads Axel Sommerfeldt's caption.sty
% which will override IEEEtran.cls' handling of captions and this will result
% in non-IEEE style figure/table captions. To prevent this problem, be sure
% and invoke subfig.sty's "caption=false" package option (available since
% subfig.sty version 1.3, 2005/06/28) as this is will preserve IEEEtran.cls
% handling of captions.
% Note that the Computer Society format requires a larger sans serif font
% than the serif footnote size font used in traditional IEEE formatting
% and thus the need to invoke different subfig.sty package options depending
% on whether compsoc mode has been enabled.
%
% The latest version and documentation of subfig.sty can be obtained at:
% http://www.ctan.org/pkg/subfig




% *** FLOAT PACKAGES ***
%
%\usepackage{fixltx2e}
% fixltx2e, the successor to the earlier fix2col.sty, was written by
% Frank Mittelbach and David Carlisle. This package corrects a few problems
% in the LaTeX2e kernel, the most notable of which is that in current
% LaTeX2e releases, the ordering of single and double column floats is not
% guaranteed to be preserved. Thus, an unpatched LaTeX2e can allow a
% single column figure to be placed prior to an earlier double column
% figure.
% Be aware that LaTeX2e kernels dated 2015 and later have fixltx2e.sty's
% corrections already built into the system in which case a warning will
% be issued if an attempt is made to load fixltx2e.sty as it is no longer
% needed.
% The latest version and documentation can be found at:
% http://www.ctan.org/pkg/fixltx2e


%\usepackage{stfloats}
% stfloats.sty was written by Sigitas Tolusis. This package gives LaTeX2e
% the ability to do double column floats at the bottom of the page as well
% as the top. (e.g., "\begin{figure*}[!b]" is not normally possible in
% LaTeX2e). It also provides a command:
%\fnbelowfloat
% to enable the placement of footnotes below bottom floats (the standard
% LaTeX2e kernel puts them above bottom floats). This is an invasive package
% which rewrites many portions of the LaTeX2e float routines. It may not work
% with other packages that modify the LaTeX2e float routines. The latest
% version and documentation can be obtained at:
% http://www.ctan.org/pkg/stfloats
% Do not use the stfloats baselinefloat ability as the IEEE does not allow
% \baselineskip to stretch. Authors submitting work to the IEEE should note
% that the IEEE rarely uses double column equations and that authors should try
% to avoid such use. Do not be tempted to use the cuted.sty or midfloat.sty
% packages (also by Sigitas Tolusis) as the IEEE does not format its papers in
% such ways.
% Do not attempt to use stfloats with fixltx2e as they are incompatible.
% Instead, use Morten Hogholm'a dblfloatfix which combines the features
% of both fixltx2e and stfloats:
%
% \usepackage{dblfloatfix}
% The latest version can be found at:
% http://www.ctan.org/pkg/dblfloatfix




%\ifCLASSOPTIONcaptionsoff
%  \usepackage[nomarkers]{endfloat}
% \let\MYoriglatexcaption\caption
% \renewcommand{\caption}[2][\relax]{\MYoriglatexcaption[#2]{#2}}
%\fi
% endfloat.sty was written by James Darrell McCauley, Jeff Goldberg and 
% Axel Sommerfeldt. This package may be useful when used in conjunction with 
% IEEEtran.cls'  captionsoff option. Some IEEE journals/societies require that
% submissions have lists of figures/tables at the end of the paper and that
% figures/tables without any captions are placed on a page by themselves at
% the end of the document. If needed, the draftcls IEEEtran class option or
% \CLASSINPUTbaselinestretch interface can be used to increase the line
% spacing as well. Be sure and use the nomarkers option of endfloat to
% prevent endfloat from "marking" where the figures would have been placed
% in the text. The two hack lines of code above are a slight modification of
% that suggested by in the endfloat docs (section 8.4.1) to ensure that
% the full captions always appear in the list of figures/tables - even if
% the user used the short optional argument of \caption[]{}.
% IEEE papers do not typically make use of \caption[]'s optional argument,
% so this should not be an issue. A similar trick can be used to disable
% captions of packages such as subfig.sty that lack options to turn off
% the subcaptions:
% For subfig.sty:
% \let\MYorigsubfloat\subfloat
% \renewcommand{\subfloat}[2][\relax]{\MYorigsubfloat[]{#2}}
% However, the above trick will not work if both optional arguments of
% the \subfloat command are used. Furthermore, there needs to be a
% description of each subfigure *somewhere* and endfloat does not add
% subfigure captions to its list of figures. Thus, the best approach is to
% avoid the use of subfigure captions (many IEEE journals avoid them anyway)
% and instead reference/explain all the subfigures within the main caption.
% The latest version of endfloat.sty and its documentation can obtained at:
% http://www.ctan.org/pkg/endfloat
%
% The IEEEtran \ifCLASSOPTIONcaptionsoff conditional can also be used
% later in the document, say, to conditionally put the References on a 
% page by themselves.




% *** PDF, URL AND HYPERLINK PACKAGES ***
%
%\usepackage{url}
% url.sty was written by Donald Arseneau. It provides better support for
% handling and breaking URLs. url.sty is already installed on most LaTeX
% systems. The latest version and documentation can be obtained at:
% http://www.ctan.org/pkg/url
% Basically, \url{my_url_here}.




% *** Do not adjust lengths that control margins, column widths, etc. ***
% *** Do not use packages that alter fonts (such as pslatex).         ***
% There should be no need to do such things with IEEEtran.cls V1.6 and later.
% (Unless specifically asked to do so by the journal or conference you plan
% to submit to, of course. )


% correct bad hyphenation here
\hyphenation{op-tical net-works semi-conduc-tor}


\begin{document}
%
% paper title
% Titles are generally capitalized except for words such as a, an, and, as,
% at, but, by, for, in, nor, of, on, or, the, to and up, which are usually
% not capitalized unless they are the first or last word of the title.
% Linebreaks \\ can be used within to get better formatting as desired.
% Do not put math or special symbols in the title.
\title{Content-based Video Retrieval in Traffic Videos using Latent Dirichlet Allocation Topic Model}
%
%
% author names and IEEE memberships
% note positions of commas and nonbreaking spaces ( ~ ) LaTeX will not break
% a structure at a ~ so this keeps an author's name from being broken across
% two lines.
% use \thanks{} to gain access to the first footnote area
% a separate \thanks must be used for each paragraph as LaTeX2e's \thanks
% was not built to handle multiple paragraphs
%

\author{Mohammad~Kianpisheh}% <-this % stops a space
%\thanks{M. Shell was with the Department
%of Electrical and Computer Engineering, Georgia Institute of Technology, Atlanta,
%GA, 30332 USA e-mail: (see http://www.michaelshell.org/contact.html).}% <-this % stops a space
%\thanks{J. Doe and J. Doe are with Anonymous University.}% <-this % stops a space
%\thanks{Manuscript received April 19, 2005; revised August 26, 2015.}}

% note the % following the last \IEEEmembership and also \thanks - 
% these prevent an unwanted space from occurring between the last author name
% and the end of the author line. i.e., if you had this:
% 
% \author{....lastname \thanks{...} \thanks{...} }
%                     ^------------^------------^----Do not want these spaces!
%
% a space would be appended to the last name and could cause every name on that
% line to be shifted left slightly. This is one of those "LaTeX things". For
% instance, "\textbf{A} \textbf{B}" will typeset as "A B" not "AB". To get
% "AB" then you have to do: "\textbf{A}\textbf{B}"
% \thanks is no different in this regard, so shield the last } of each \thanks
% that ends a line with a % and do not let a space in before the next \thanks.
% Spaces after \IEEEmembership other than the last one are OK (and needed) as
% you are supposed to have spaces between the names. For what it is worth,
% this is a minor point as most people would not even notice if the said evil
% space somehow managed to creep in.



% The paper headers
\markboth{Journal of \LaTeX\ Class Files,~Vol.~14, No.~8, August~2015}%
{Shell \MakeLowercase{\textit{et al.}}: Bare Demo of IEEEtran.cls for IEEE Journals}
% The only time the second header will appear is for the odd numbered pages
% after the title page when using the twoside option.
% 
% *** Note that you probably will NOT want to include the author's ***
% *** name in the headers of peer review papers.                   ***
% You can use \ifCLASSOPTIONpeerreview for conditional compilation here if
% you desire.




% If you want to put a publisher's ID mark on the page you can do it like
% this:
%\IEEEpubid{0000--0000/00\$00.00~\copyright~2015 IEEE}
% Remember, if you use this you must call \IEEEpubidadjcol in the second
% column for its text to clear the IEEEpubid mark.



% use for special paper notices
%\IEEEspecialpapernotice{(Invited Paper)}




% make the title area
\maketitle

% As a general rule, do not put math, special symbols or citations
% in the abstract or keywords.
\begin{abstract}
Content-based video retrieval is one of the most challenging tasks in surveillance systems. In this study, Latent Dirichlet Allocation (LDA) topic model is used to annotate surveillance videos in an unsupervised manner. In scene understanding methods, some of the learned patterns are ambiguous and represents a mixture of atomic actions. To address the ambiguity issue in the proposed method, feature vectors, and the primary model are processed to obtain a secondary model which describes the scene with primitive patterns that lack any ambiguity. Experiments show performance improvement in the retrieval task compared to other topic model based methods. In terms of false positive and true positive responses, the proposed method achieves at least 80\% and 124\% improvement respectively. Four search strategies are proposed, and users can define and search for a variety of activities using the proposed query formulation which is based on topic models. In addition, the lightweight database in our method occupies much fewer storage which in turn speeds up the search procedure compared to the methods which are based on low-level features.
\end{abstract}

% Note that keywords are not normally used for peerreview papers.
\begin{IEEEkeywords}
Content-based retrieval, Surveillance, Latent Dirichlet Allocation, Topic Model, Search strategies, Query formulation.
\end{IEEEkeywords}

% For peer review papers, you can put extra information on the cover
% page as needed:
% \ifCLASSOPTIONpeerreview
% \begin{center} \bfseries EDICS Category: 3-BBND \end{center}
% \fi
%
% For peerreview papers, this IEEEtran command inserts a page break and
% creates the second title. It will be ignored for other modes.
\IEEEpeerreviewmaketitle
%
\section{Introduction}
\label{intro}  
% The very first letter is a 2 line initial drop letter followed
% by the rest of the first word in caps.
% 
% form to use if the first word consists of a single letter:
% \IEEEPARstart{A}{demo} file is ....
% 
% form to use if you need the single drop letter followed by
% normal text (unknown if ever used by the IEEE):
% \IEEEPARstart{A}{}demo file is ....
% 
% Some journals put the first two words in caps:
% \IEEEPARstart{T}{his demo} file is ....
% 
% Here we have the typical use of a "T" for an initial drop letter
% and "HIS" in caps to complete the first word.
\IEEEPARstart{S}{urveillance} cameras are widely used in almost all public places such as traffic junctions, airports, subways, and shopping centers. Content-based retrieval is one of the most challenging tasks in surveillance videos, and it is highly influenced by the quality of video annotation. The massive amounts of video recorded daily makes it almost impossible to annotate them manually. Thus, unsupervised algorithms are needed to exploit semantically meaningful activities in video. \\ 
%
\indent Topic models are a class of hierarchical models, were originally proposed to discover meaningful topics in large text data collections in an unsupervised fashion. In these models, every document is considered as a mixture of topics, which in turn are a mixture of words. Processing the documents, the words that often co-exist in the same document are clustered into the same topic. After text domain promising results, topic models have been applied to a variety of data types such as images \cite{ref_12,ref_13}, sound signals \cite{ref_11} and surveillance videos \cite{ref_7,ref_9,ref_10}. In video domain, the input video sequence is segmented to short clips with 10 to 100 frames. Each clip is considered as a document and extracted visual features correspond to words, and by mining the documents, topics in the video sequence are exploited. In this case, a topic corresponds to an activity in the scene. The procedure from low-level visual features to high-level activities is illustrated in Fig.~\ref{fig:vidInput}. \\ 
%
% Ambiguity explanation
\indent The performance of the retrieval system is highly affected by the accuracy of description space. For example, Fig.~\ref{fig:ambiguity} shows a sample topic in a junction learned using the Latent Dirichlet Allocation (LDA) topic model \cite{ref_5}. This topic contains multiple actions (A, B, and C), thus, the occurrence of each of these actions excites the illustrated topic in Fig.~\ref{fig:ambiguity}. In this case, it is impossible to infer which one of the minor actions (A, B, and C) has occurred in the respective clip.
%
% Other important factors in the results of the retrieval system
The way of query definition and search procedure are other factors that influence the performance of the scene understanding systems. In this work, several techniques are proposed to address the ambiguity issue explained above. In addition, four search strategies are proposed which allow users to accurately define and search for different queries using the proposed query formulation. The rest of this paper is organized as follows: Section~\ref{subsec:relatedWorks} provides an overview of the recent literature. The description of the proposed method is presented in Section~\ref{sec_1-2}. Experimental results for the proposed search strategies are presented in Section~\ref{sec:experiments}. Finally, Section~\ref{sec:conclusion} concludes the paper. 
%
% pixel to topic figure
\begin{figure}[t]
\centering
\includegraphics[width=0.5\textwidth]{images/vidInput.pdf}
\caption{Low-level features to high-level patterns procedure. First, visual features are extracted from each document which is a bag of visual features. Then, documents are fed into the LDA model, to discover semantically meaningful patterns (topics) in the scene.}
\label{fig:vidInput}
\end{figure}
%
% Ambiguity in a learned topic figure
\begin{figure}[t]
\centering
\includegraphics[width=1.8in]{images/Ambiguity.pdf}
\caption{An ambiguous topic including multiple minor actions.}
\label{fig:ambiguity}
\end{figure}
%
\section{Related Work}
\label{subsec:relatedWorks}
% Tracking based methods
\indent Most of the proposed works for scene understanding can be broadly grouped into two categories. Under the first category, objects in the scene are first detected and then tracked to obtain their trajectories in the scene  \cite{ref_17, ref_18}. Object trajectories are trained and used for scene understanding analysis. For example, motions are learned using clustering in \cite{ref_17}, where object trajectories are hierarchically clustered using spectral clustering, and in the second stage, each group of trajectories is further clustered into subcategories using the temporal information. Tracking based approaches are prone to tracking errors. In these approaches, it is likely to lose an object for several frames, especially in the crowded scenes, which results in an incomplete trajectory for that object \cite{ref_7}.\\ 
%
% Pixel based methods (change under the second category)
\indent Under the second category, visual features are extracted from video sequence without any object detection and tracking. In most of these pixel-based approaches, learning algorithms are used to index the video stream by high-level semantically meaningful patterns. There are several methods that are based on topic model which model the scene with high-level patterns \cite{ref_7,ref_9,ref_15,ref_16}. These methods succeed in indexing videos but none of them address the retrieval task in surveillance videos directly except for the work presented in \cite{ref_7}, which is based on Hierarchical Dirichlet Process (HDP) topic model \cite{ref_21}. In their work, the video stream is decomposed into clips each of them is annotated with a distribution over the learned topics. To search for an activity, they define the user query as a distribution over the learned topics. The clips within the database are then compared with the query distribution using the Kullback-Leibler divergence to find the clips that involve the user query. \\
%
% The low-level feature based method
\indent  Among pixel-based approaches, in method presented in the video is indexed with low-level visual features, unlike the scene understanding approaches in which the video is indexed with high-level patterns. They use Locality-sensitive hashing (LSH) to hash the extracted low-level features into a lightweight lookup table. They define the user query as a set of action components each of them composed of a set of structured low-level features called tree. After defining the query and the region of interest (ROI), a set of trees is assigned to each action component and the full responses are found using dynamic programming. 
%
% Framework figure
\begin{figure}[!t]
  \centering
  \includegraphics[width=0.43\textwidth]{images/Framework.pdf}
\caption{The proposed method framework. In the learning phase, blob decomposition is performed on extracted features, and new feature vectors are fed into the corresponding LDA topic model. Topics of the primary model are processed to form the secondary model with primitive topics.  The user query formulation is parsed into the internal representation of the system, and the search procedure is performed using the proposed search strategies.}
\label{fig:framework}
\end{figure}
%
\section{Proposed method}  
\label{sec_1-2}
%
% Ambiguity (separate model, pre-processing each clip, topic analysis)
The framework of the proposed method is illustrated in Fig.~\ref{fig:framework}. As video streams in, low-level features: motion, persistence, and size are extracted. In the Clip Blob Decomposition step, Connected Component Labeling (CCL) algorithm is used to decompose each feature vector into several feature vectors each of them including a single-agent activity. The resulted feature vectors indicate activities that contain a single connected component. This pre-processing step increases the probability of obtaining topics which represent an activity consisting of a single blob. After the Clip Blob Decomposition step, in the learning phase, the resulted single-agent vectors are fed into the LDA topic model to discover activities in the scene. For each visual feature, a separate LDA model is learned to prevent multiple activities from different feature spaces to merge together into a single topic. In the Topic Processing step, the primary learned model is then processed to create a secondary model in which topics lack any ambiguity and each of them indicates a primitive action. In this step, first, each topic is decomposed into several topics using the CCL algorithm. By doing so, each topic in all feature spaces includes an activity comprising a single blob. In addition, motion topics are broken down into topics which consist of activities including motion in just one direction which results in even more primitive topics. After the learning phase, the secondary model is used to index the input video and creating the database. \\ 
\indent In the user side, the user can define his query using the proposed query formulation which is based on topic models, and specifies the way that user expresses his query. The Query Parsing specifies the way that the system parses the user query into the internal representation of the system, which in our case are topics. Having the user query determined, the Retrieval Engine searches the database and returns the video segments containing the queried action. Although primitive topics are void of ambiguity, these topics can not individually specify a complicated activity. To address this issue, in the proposed method a complicated activity can be defined as a sequence of primitive topics. Then, the desired sequence examples in the database are found using the Smith-Waterman \cite{ref_6} dynamic programming algorithm like the one presented in~\cite{ref_2}. In addition to topic sequence, three other search strategies including single topic, topic co-occurrence, and similar clips are considered in the Retrieval Engine to complete the system and allow the user to search for a variety of queries. \\  % (Mentioning the developed software and FE GPU acceleration)
%
% Speed and storage improvement
\indent Indexing video with high-level patterns in the  proposed method results in a database with significantly lower size compared to the methods which are based on low-level features. A lightweight database, not only reduces the needed storage, but it also results in a significant speed up in the search procedure. The feature extraction step is also accelerated by leveraging the computing power of Graphics Processing Unit (GPU). 
%
% Warm up paragraph for feature extraction section. 
\subsection{Low-level visual features}
\label{sec:FE}
In this work, the video sequence is uniformly segmented into non-overlapping clips each of them includes \textit{F} frames. Each frame within a clip is divided into $C\times C$ pixel-cells and the visual features including location, motion, persistence and size are computed for each cell. $F$ and $C$ are chosen regard to the frame rate, resolution, objects distance from the camera, and the scene type (e.g. junction, subway, etc.). \smallskip \\ 
%
% Location visual feature
\textbf{Location}: Most of the activities in the surveillance videos are characterize by the place they occur. Therefore, the location is considered in our feature extraction step. For a video with a frame size of $480\times 720$ which is divided into $10\times 10$ cells, we have $48\times 72$ pixel-cells in each frame. \smallskip \\
%
% Motion visual feature
\textbf{Motion}: TVL1 optical flow method \cite{ref_1} is used to compute the motion features. Optical-flows greater than the $Th_{op}$ threshold are reliable and quantized into one of the eight cardinal directions, and areas in the scene where optical flow magnitude is lower than the $Th_{op}$ are considered as static. Finally, the most common direction in each pixel-cell is considered as the motion feature of that pixel-cell. \smallskip \\
%
% Persistence visual feature
\textbf{Persistence}: Persistence feature allows to detect static foreground objects in the scene, and more variety of activities can be captured using this visual feature. Persistence is occurring in areas in which an object belonging to the foreground, stays for a while in the scene and does not move. Thus, computing persistence needs background subtraction. Robustness to gradual and sudden changes in illumination is the main challenge in background subtraction methods in surveillance videos \cite{ref_3}. For example, trees shaken by the wind in the scene result sudden changes in the background pixel values. To cope with these issues, the Gaussian Mixture Model (GMM) background subtraction is used \cite{ref_4}. Modeling each pixel with a mixture of \textit{K} Gaussians which update continuously, makes GMM method robust to sudden and gradual changes in illumination. \smallskip \\
%
% Size visual feature
\textbf{Size}: After background subtraction step, blobs in the foreground can be detected using blob extraction, and the foreground objects can be further characterized. The size of each blob is considered as the number of pixels forming the blob, and blobs are categorized into two classes (small and large). Each pixel gets the size label (small and large) of the blob it belongs to, and each pixel-cell gets the most common size label of pixels that form it.    
%
% Clip pre-processing
\subsection{Clip blob decomposition (pre-processing)}
\label{subsect:clipBlobDecomp}
During the learning phase, the co-occurrence of multiple activities in the same clip may lead to a learned topic which represents two or more activities. In the proposed method, to prevent activities from mixing together, a pre-processing step is used before applying the LDA algorithm. By performing blob decomposition on each clip using the CCL algorithm, each feature vector is decomposed into new feature vectors each of them represents an activity consisting of a single blob which in turn reduces the probability of such merging process in the learning phase. Fig.~\ref{fig:docBlobDecomp}a shows a sample persistence topic obtained in a junction without performing the Clip Blob Decomposition pre-processing step (highlighted areas indicate persistence). Fig.~\ref{fig:docBlobDecomp}b and Fig. \ref{fig:docBlobDecomp}c show two persistence topics in the same area indicating two activities separated due to performing the Clip Blob Decomposition in our method.
% 
% Clip Blob Decomposition persistence topics figure
\begin{figure}[!t]
\centering
\subfloat[]{\includegraphics[width=1.6in]{images/presMergMIT}
\label{fig:docBlobDecomp1}}
\\
\subfloat[]{\includegraphics[width=1.6in]{images/persDocSep1.pdf}
\label{fig:docBlobDecomp2}}
\hfil
\subfloat[]{\includegraphics[width=1.6in]{images/persDocSep2.pdf}
\label{fig:docBlobDecomp3}}
\caption{Learned persistence topics: (a) a topic obtained without performing clip blob decomposition step, (b) and (c) spatially separate persistence topics obtained in the presence of clip blob decomposition step.}
\label{fig:docBlobDecomp}
\end{figure}
%
% Separate topic model technique
\subsection{Separate models for each visual feature}
\label{subsec:sepModel}
% Compound feature vector and its problem (shown with figure) 
Fig.~\ref{fig:qmulTopics} shows several learned topics using the LDA model in a Junction. Areas covered by dots in the figure indicate persistence. In the first experiment, the feature vector of each clip is made up of the combination of motion and persistence visual features. In this case, as shown in the first row of Fig.~\ref{fig:qmulTopics}, topics include activities from both feature spaces. These topics suffer from ambiguity issue in the feature space level, so they are not appropriate for the retrieval task. \\
%
% Multiple feature vectors technique and its advantages. (the word "setting")
\indent To get around this issue, a better approach is to learn a separate model for each visual feature. This approach not only prevents activities from different feature spaces to assign to the same topic, but it also reduces the number of visual words. For instance, consider a scene with $2400$ cells for location feature, eight main directions for motion feature, and two classes (small and large) for size feature. This setting, in the case of the compound feature vector, leads to a total of $2400\times 8\times 2=38400$ visual words, while the second approach leads to a motion feature vector with $2400\times 8=19200$ bins and a size feature vector with $2400$ bins. Thus, the number of words in the second approach is almost half of the first one. Some of the learned topics by separate models for motion and persistence features are shown in the second and third rows of Fig. \ref{fig:qmulTopics}. 
%
% Blob separation to remove ambiguity in each feature space
\subsection{Topic blob decomposition}
\label{subsect:topicBlobDecomp}
% Explaining the reason that we should perform this process.
Learning separate models for each visual feature does not obviate the ambiguity issue entirely. As can be seen from Fig.~\ref{fig:qmulTopics}d, a topic in a given feature space may include multiple activities. In Topic Processing step in the proposed method, this issue is simply solved by decomposing topics of the primary learned model into topics which each of them indicates a simpler activity. This procedure is done by blob decomposition of the learned topics using the CCL algorithm. Indeed, we create a new model in which each topic includes a single-agent activity. The secondary model is created by processing the topics of the primary learned model.
%
% qmulTopics figure (motion + pers + motionPers)
\begin{figure*}[!t] 
\centering
\subfloat[]{\includegraphics[width=1.7in]{images/1mp.pdf}
\label{fig:qmulTopics1}}
\hspace{1em}
\subfloat[]{\includegraphics[width=1.7in]{images/2mp.pdf}
\label{fig:qmulTopics2}}
\hspace{1em}
\subfloat[]{\includegraphics[width=1.7in]{images/3mp.pdf}
\label{fig:qmulTopics3}}
\hfil
\subfloat[]{\includegraphics[width=1.7in]{images/1m.pdf}
\label{fig:qmulTopics4}}
\hspace{1em}
\subfloat[]{\includegraphics[width=1.7in]{images/2m.pdf}
\label{fig:qmulTopics5}}
\hspace{1em}
\subfloat[]{\includegraphics[width=1.7in]{images/3m.pdf}
\label{fig:qmulTopics6}}
\hfil
\subfloat[]{\includegraphics[width=1.7in]{images/1p.pdf}
\label{fig:qmulTopics7}}
\hspace{1em}
\subfloat[]{\includegraphics[width=1.7in]{images/2p.pdf}
\label{fig:qmulTopics8}}
\hspace{1em}
\subfloat[]{\includegraphics[width=1.7in]{images/3p.pdf}
\label{fig:qmulTopics9}}
\caption{ Sample topics learned using LDA model for (first row) the compound feature vector, (second row) the motion visual feature, and (third row) the persistence visual feature.}
\label{fig:qmulTopics}
\end{figure*}
%
% Direction decomposition
\subsection{Topic direction decomposition}
\label{subsec:dirDecomp}
% the main reason for direction decomposition (more primitive topics)
To obtain even more primitive topics, after the blob decomposition step, the resulted motion topics are decomposed into topics which include motion in just a single direction. Fig~\ref{fig:dirDecomp} shows an example of topic motion direction decomposition. For better visualization, arrows are drawn for a bigger cell size. Topics including words in a single direction represent more primitive actions. 
%
% The second reason for direction decomposition (twin actions)
Another issue regarding topics with multiple directions is that these topics may include two words with different directions in the same pixel-cell. In this case, topics may include two overlapping patterns. This is another form of ambiguity in topics which leads to a lot of false alarm responses.
%
% Removing very similar topics 
After performing the blob and direction decomposition on topics, highly correlated topics can be removed to prevent redundant data storage. Note that correlated topics can be removed only if topics are primitive and represent simple actions. Indeed, two complicated topics may have considerable overlap but indicating two essentially different activities.
%
\subsection{Retrieval engine}
\label{subsec:tpSeqDP}
Adopting topic processing techniques results in a secondary model with primitive topics which are void of ambiguity. Although primitive topics lack any ambiguity, they can not be used individually to search for a complicated activity. To cope with this issue, in the proposed method, a complicated activity is defined as a sequence of primitive topics. The sequence length is determined by the user. These topic sequences are then searched in the database using the Smith-Waterman \cite{ref_6} dynamic programming like the one presented in \cite{ref_2}. In addition to topic sequence, to complete the system, three other search strategies including single topic, topic co-occurrence, and similar clips are considered in the Retrieval Engine.  
%
\subsection{Query formulation}
\label{subsec:qForm}
In our framework, users can search for different queries using the proposed query formulation which is as follows \smallskip \\
% 
QUERY DOMAIN \textless Feature space\textgreater \:SECTION \textless Database\textgreater \: SEARCH TYPE \textless Search strategy\textgreater, 
% 
\smallskip \\ 
\noindent where QUERY DOMAIN, SECTION, and SEARCH TYPE are mandatory keywords. \smallskip \\
%
% Feature space term in the query formulation
\textbf{\textit{Feature space}} determines visual features that are present in the user query. Since separate models are learned for each visual feature, there is a separate database for each of them. Thus, users can accelerate the search procedure by specifying the features that are present in their query. Users can choose a single or a mixture of feature spaces.  \smallskip \\ 
% Database term in the query formulation
\textbf{\textit{Database}} term in the query formulation is very important and can improve the search speed. This term determines the part of the dataset that the user is interested in. In the case that the database is divided into the parts related to different locations or different times, this term allows users to narrow down the search space which in turn speeds up the search procedure~\cite{ref_8}.\smallskip \\
% Search strategy term in the query formulation
\textbf{\textit{Search strategy}} specifies the search scenario from the proposed search strategies. 
%
\subsection{Query types}
The proposed method supports two query types including \textit{query by sketch} and \textit{query by example}. A software (Fig.~\ref{fig:app}) is developed to provide a simple way for query definition.
\smallskip  \\
% Query by sketch
\textbf{\textit{Query by sketch:}} In this query type, users can specify the activity they are interested in, by drawing paths or regions in the scene. The developed software provides a straightforward way to do this. For example, Fig.~\ref{fig:app} shows a sample query that indicates co-occurrence of a motion activity and a persistence in the scene. Extracted features from the sketched query are then assigned to the corresponding topics in each model. \smallskip \\
% Query by example
\textbf{\textit{Query by example:}} In this query type, a clip provided by the user represents the query. In this case, the distribution of the input clip forms the user query. Furthermore, the developed software allows users to select a time interval in the database as their query. 
%
%  direction decomposition 1
\begin{figure}[!t]
\centering
\subfloat[]{\includegraphics[width=1.6in]{images/dir11.pdf}
\label{fig:dirDecomp1}} \\
\subfloat[]{\includegraphics[width=1.6in]{images/dir12.pdf}
\label{fig:dirDecomp2}}
\hfil
\subfloat[]{\includegraphics[width=1.6in]{images/dir13.pdf}
\label{fig:dirDecomp3}}
\caption{Topic direction decomposition. (a) A sample motion topic containing movement in two directions, (b), (c) Primitive motion topics consisting of a single motion direction}
\label{fig:dirDecomp}
\end{figure}
%
% Application figure
\begin{figure}[!t]
\centering
\includegraphics[width=0.47\textwidth]{images/app-eps-converted-to.pdf}
\caption{Developed software for the retrieval task. It provides a simple way for users to define their queries using the proposed query formulation by drawing paths or specifying regions in the scene.}
\label{fig:app}
\end{figure}
%
%
% Dataset sample frames
\begin{figure}[!t]
\centering
\subfloat[]{\includegraphics[width=1.6in]{images/mitFrame.jpg}
\label{fig:docBlobDecomp1}} \\
\subfloat[]{\includegraphics[width=1.6in]{images/qmulFrame.jpg}
\label{fig:docBlobDecomp2}}
\hfil
\subfloat[]{\includegraphics[width=1.6in]{images/mftFrame.jpg}
\label{fig:docBlobDecomp3}}
\caption{Sample frames of tested datasets. (a) MIT Traffic dataset, (b) QMUL Junction dataset, and (c) Tehran Junction dataset.}
\label{fig:datasets}
\end{figure}
%
%
% Experiments and search strategies.
\section{Experimental results}
\label{sec:experiments}  
% Section intro paragraph
%
% Tested datasets introduction
\subsection{Datasets and settings} 
Experiments are carried out on three crowded video sequences recorded by fixed cameras. The first video sequence is MIT Traffic dataset~\cite{ref_23} (Fig.~\ref{fig:datasets}a) including a far-field traffic scene of 92 minutes long. This dataset is recorded at 30 fps with a resolution of $480\times 720$ which is scaled to a frame size of $240\times 360$. The second video sequence is QMUL Junction dataset \cite{ref_23} (Fig.~\ref{fig:datasets}b) of 60 minutes long captured with the frame size of $288\times 360$ at 25 fps. The third tested video sequence is Tehran Junction dataset (Fig.~\ref{fig:datasets}c) of 16 minutes long and resolution of $512\times 720$. This dataset includes activities like stopped vehicles in non-authorized areas. Table \ref{tab:topicNumber} shows the topic number in the primary and secondary models for the presented experiments in the evaluated datasets.  \\ 
%
%  topicNumber in the primary and secondary models table
\begin{table*}[!t]
\renewcommand{\arraystretch}{1.4}
\caption{Topic number for motion and persistence features in the primary and secondary models.}
\label{tab:topicNumber}
\centering
\begin{tabular}{|c||c||c||c||c||c|}
\hline
Video   &  Training  &~Motion topics	& ~~Motion topics   &Persistence topics &Persistence  topics\\
 sequence       	&   data     &  (primary model)  & (secondary model)  & (primary model)   & (secondary model)                  	\\
\hline
MIT Traffic   	&   25 (min)  & 17 	&  49    &15      &30 \\
\hline
QMUL Junction &  20 (min)   &20	&  41    &25         &54\\
\hline 
Tehran Junction &   12 (min)   	&20  &  49     &18     	&48\\
\hline
\end{tabular}
\end{table*}
%
% Sparsity, similarity, and clip-length effect charts
\begin{figure*}[!t]
\centering
\subfloat[]{\includegraphics[width=2.3in]{images/topicWordNum.pdf}
\label{fig:fig:avgNum1}}
\hfil
\subfloat[]{\includegraphics[width=2.3in]{images/similarity.pdf}
\label{fig:fig:avgNum2}}
\hfil
\subfloat[]{\includegraphics[width=2.22in]{images/clipLength.pdf}
\label{fig:fig:avgNum3}}
\caption{(a) The average number of words vs the number of topics, (b) overlap score vs the number of topics, (c) clip-length vs the number of false positives.}
\label{fig:avgNum}
\end{figure*}
%
% Single topic queries figure
\begin{figure*}[!t]
  \centering
  \includegraphics[width=0.85\textwidth]{images/mftQueries.pdf}
\caption{Sample persistence queries in Tehran Junction dataset.}
\label{fig:tehranQuery}
\end{figure*}
%
\textbf{Topic sparsity:} Fig.~\ref{fig:avgNum}a compares the average number of words per topic for different topic numbers in the primary and secondary models. As seen from this figure, topics of the secondary model are sparser that the ones in the primary model. \\ 
\indent \textbf{Topic similarity:} Highly correlated topics can be easily removed in a model with primitive topics, thus, topics in the secondary model have a lower intersection with each other which results in a lower storage. In addition, when the user defines his query by sketch, the user query assigns more easily to uncorrelated topics. A proper evaluation metric for similarity (correlation) between topics is the overlap score \cite{ref_20}. Given two topic blobs $b_i$ and $b_j$, the overlap score is defined as 
%
% Overlap score 
\begin{align}
S = \frac{|b_i\bigcap b_j|}{|b_i\bigcup b_j|},
\label{eq:overlapScore}
\end{align} 
%
%
where $\bigcap$ and $\bigcup$ denote intersection and union operators respectively, and $|.|$ represents the number of pixels constituting the blob of each topic. Fig.~\ref{fig:avgNum}b compares the average overlap score between topics in the primary and secondary models. 
\smallskip \\ 
%
%
% Co-occurrence queries figure
\begin{figure*}[!t]
\centering
\subfloat[]{\includegraphics[width=2in]{images/cooc_Persistence.pdf}%
\label{fig:coocQuery1}}
\hspace{1em}
\subfloat[]{\includegraphics[width=2in]{images/cooc_Motion.pdf}%
\label{fig:coocQuery2}}
\caption{ Co-occurrence of two activities. (a) Persistence co-occurrence in MIT dataset, (b) Traffic interruption by a fire engine in QMUL Junction dataset.}
\label{fig:coocQuery}
\end{figure*}
%
%  Single topic and topic co-occurrence results table.
\begin{table*}[!t]
\renewcommand{\arraystretch}{1.2}
\caption{Results for queries searched using single topic and topic co-occurrence strategies.\label{tab:coocSingle}}
\label{tab:coocSingle}
\centering
\begin{tabular}{|c||c||c||c||c||c|}
\hline
~~Query~~  & Video sequence & Ground truth & True positive & False positive & False negative  \\
\hline
A & Tehran Junction & 1 & 1 & 0 & 0 \\
\hline
B & Tehran Junction & 4  & 4 & 0 & 0 \\
\hline
C & Tehran Junction &  5 & 4 &  0 &  1\\
\hline
D & MIT Traffic & 2 & 2 & 0 & 0 \\
\hline
E & QMUL Junction &  1 & 1  & 0 & 0 \\
\hline
\end{tabular}
\end{table*}
%
\textbf{Clip-length effect:} Clip-length is an important parameter which significantly affects the system performance. In the case of topic sequence search strategy, most false alarm responses occur when two different vehicles traverse the user query. For example, if the user searches for a topic sequence with the length of two, it is likely that two different cars create the topic sequence and cause a false alarm response. Longer clips intensify this issue from two different aspects. First, higher clip-length results in more spatially extended topics. By increasing the clip-length, more displacement is captured in topics, thus, motion topics grow spatially and cover more area in the scene. With spatially broader topics, it is more likely that two different cars hit the topics included in the user query. Secondly, regardless of topics expansion, higher clip-length provides a longer time window for the second car to complete the sequence and cause a false positive response. Fig.~\ref{fig:clipLengthChart} shows the search result for queries (F) and (G) illustrated in Fig.~\ref{fig:mitQueries} for databases created with different clip-lengths. As seen from Fig.~\ref{fig:avgNum}c, the number of false positive responses increases in higher clip-lengths. 
%
\subsection{Evaluated queries}
\label{sub_evaluatedQueries}
\noindent \textbf{Single topic:} Querying a single motion topic is the same as querying a topic sequence with the sequence length of one. However, in other feature spaces, our desired activity can be specified by a single topic. For example, stopped vehicles in non-authorized areas can be found by querying the related persistence topic in the persistence database. Consecutive clips which each of them includes the queried topic, merge together to form the full response. Fig.~\ref{fig:tehranQuery} shows persistence queries in no waiting areas in Tehran Junction dataset, and the results of these queries are demonstrated in Table~\ref{tab:coocSingle}.  \\ 
%
\noindent \textbf{Topic co-occurrence:} Co-occurrence of two activities is another search strategy which is considered in our framework. In this strategy, the simultaneous occurrence of two actions in a single clip is considered. 
% Queries and their results
Fig.~\ref{fig:coocQuery} shows two queries defined using the topic co-occurrence strategy. Fig.~\ref{fig:coocQuery1} shows a query consisting of persistence in two different location in MIT Traffic dataset, and a traffic interruption by a fire engine in QMUL Junction dataset is illustrated in Fig.~\ref{fig:coocQuery2}. Table~?? illustrates the search results for these queries (D and E). Consecutive clips which each of them includes the queried topics, merge together to form the full response.  \\ 
%
%
% Topic sequence search strategy
\noindent \textbf{Topic sequence:} A primitive topic can not be used individually to represent a complicated activity. Thus, in the proposed method, a complicated activity is defined as a topic sequence.
%
% Results for queries A, B, and C in MIT dataset.
Fig.~\ref{fig:mitQueries} shows three queries in MIT Traffic dataset, and Table~\ref{tab:topicSeq} presents the search results of our method for these queries. The results of the method based on HDP topic model \cite{ref_7} which is evaluated in \cite{ref_2}, and the results of the method based on low-level features \cite{ref_2} are also included.
%
% Single topic queries figure
\begin{figure}[!t]
  \centering
  \includegraphics[width=0.32\textwidth]{images/mitQueries.pdf}
\caption{Queries defined as a topic sequence in MIT Traffic dataset.}
\label{fig:mitQueries}
\end{figure}
%
% Roc curve for query B, Chart
\begin{figure}[!t]
  \centering
  \includegraphics[width=0.32\textwidth]{images/rocc.pdf}
\caption{ROC curves for query G illustrated in Fig.~\ref{fig:mitQueries}.}
\label{fig:roc}
\end{figure}
%
% Topic sequence results. (table)
\begin{table*}[!t]
\renewcommand{\arraystretch}{1.3}
\caption{Results for queries searched using topic sequence search strategy.}
\label{tab:topicSeq}
\centering
\begin{tabular}{|c||c||c||c||c||c||c||c|}
\hline
Query  & Ground  & This work &  HDP \cite{ref_7} & LL features \cite{ref_2}  & This work & HDP \cite{ref_7} & LL features\cite{ref_2} \\
          	&   truth                  	& (true positives) 	&  	(true positives)       	&  (true positives)   & (false alarms) 	& (false alarms)                 	& (false alarms) \\
\hline
F  & 148 & 121 & 54 & 135 & 23 & 118 &13\\
\hline
G & 66  & 63 & 6 & 61 & 4 & 58 &  5\\
\hline
H & 170 & 153 & --- & --- & 4 &--- & ---\\
\hline
\end{tabular}
\end{table*}
%
% Similar clips result figure
\begin{figure*}[!t]
  \centering
  \includegraphics[width=0.9\textwidth]{images/similarityGroup.pdf}
\caption{Similar scenes. (First row): MIT Traffic dataset, (Second row): QMUL Junction dataset. }
\label{fig:simClipsSample}
\end{figure*}
%
Table~\ref{tab:topicSeq} demonstrates a substantial improvement over topic models performance in the retrieval task. For activity (F), this improvement accounts for $124\%$ increase in true positive and $80\%$ decrease in false positive responses. For activity (G), there is a $950\%$ increase in true positive and $93\%$ decrease in false positive responses. The area under ROC (AUROC) value for activity (G) increases from 0.63 in \cite{ref_7} to 0.89 in our work. The AUROC for activities (F) and (H) in our method are 0.91 and 0.80 respectively. Fig. \ref{fig:roc} shows ROC curves for activity (G) in our method and \cite{ref_7}.  \\
%
% Similar clips search strategy
\noindent \textbf{Similar clips:} Similar clips to a given clip are found using topic distribution comparison. The topic distribution of the queried clip $q$ is matched with topic distribution of the database clips $p_c$ using the Hellinger distance: \\
%
%
\begin{align}
d_{H}=\frac{1}{\sqrt{2}}\sqrt{\sum_{i=1}^{N}(\sqrt{p_{ci}}-\sqrt{q_{i}})^2}.
\label{eq:hell}
\end{align} 
%
After computing the Hellinger distance, the results are sorted ascending. 
In the case that the queried clip has a longer length than those in the database, the queried clip is broken down into multiple clips with the same length as clips in the dataset. For example, in our experiments, MIT Traffic dataset is divided into clips with one-second length. Thus, in order to find similar scenes to a three-seconds clip, the queried clip is divided into three clips each of them one second long, and the resulted distributions are convoluted over the database clips. Fig.~\ref{fig:simClipsSample} shows sample frames of two queried scenes in MIT Traffic and QMUL Junction datasets (first column). Frame samples of the first three clips which are most similar to the queried scenes in the corresponding databases are also shown in this figure (columns 2 to 4). 
%
% Database storage and search speed 
\subsection{Database storage and search speed}
% Storage 
\indent Indexing video with high-level primitive topics, not only provides a fine description of the input video, but it also results in a lightweight database in comparison to the methods based on low-level features like \cite{ref_2}. Table~\ref{tab:storage} compares database size for MIT Traffic dataset in our work and  \cite{ref_2}. In a database which includes clip distributions over all of the learned topics, the database size grows linearly with the size of the input video, regardless of its content. Since clips with not significant content have sparse distributions, the database size can be shrunk, if we discard elements which are lower than a specified threshold. In other words, in each clip, we store topic index and topic portion (\textit{topic-idx,topic-value}) for the ones with significant values. By doing so, the database size is only dependent on the content of the scene's foreground. Fig.~\ref{fig:topicPerClip} demonstrates the average number of non-zero elements in MIT Traffic and QMUL Junction datasets. As can be seen from this figure, the database size grows sub-linearly with the topic number in the secondary model, and it means that clips with not significant content have a sparse distribution over topics of the secondary model. Table~\ref{tab:storage} compares the database size in our work and \cite{ref_2}. 
%
% Database size table
\begin{table}[!b]
\renewcommand{\arraystretch}{1.43}
\caption{Database size comparison. (MIT traffic dataset)}
\label{tab:storage}
\centering
\begin{tabular}{|c||c|}
\hline
Method & Database Size\\
\hline
Low-level features \cite{ref_2} & 42 MB \\
\hline
This work  & 1.04 MB \\
\hline
This work (non-zeros)  & 0.56 MB \\
\hline
\end{tabular}
\end{table}
%
% Topic number vs Non-zero topics figure
\begin{figure}[!b]
\centering
\includegraphics[width=0.35\textwidth]{images/topicPerClip.pdf}
\caption{The average number of non-zero topics per clip.}
\label{fig:topicPerClip}
\end{figure}
%
% Speed comparison table
\begin{table}[!b]
\renewcommand{\arraystretch}{1.4}
\caption{Search time for queries (F), (G) and (H) shown in Fig. \ref{fig:mitQueries}.\label{tab:searchSpeed}}
\label{tab:storage}
\centering
\begin{tabular}{|c||c||c||c|}
\hline
Query & F & G & H\\
\hline
Low-level features \cite{ref_2} & 0.5 (s) & 0.4 (s) & -- \\
\hline
This work  & 0.029 (s) & 0.026 (s) &  0.016 (s) \\
\hline
\end{tabular}
\end{table}
%
% Speed
A lightweight database results in a speed~up in the search procedure. Table~\ref{tab:searchSpeed} shows the search time in our work compared to the method in \cite{ref_2}. As can be seen from the table, the search procedure in our method is considerably fast. 
%
\subsection{Accelerating feature extraction using GPU}
\label{sub_gpuSpeedsUp}
\indent The feature extraction speed for MIT Traffic dataset is about 24 frames per second at the frame size of $240\times 360$. Table~\ref{tab:compTimeFE} shows the computation time for each step of feature extraction and compares the execution time of each step on CPU and the parallel version on GPU.  As can be seen from Table~\ref{tab:compTimeFE}, feature extraction is about 4.5 times faster when it executes on GPU. Host to device and device to host transfers indicate the time of data transfers from CPU to GPU and from GPU to CPU respectively. Fig.~\ref{fig:feChart} shows that the computation of optical flow is the most time-consuming step of feature extraction step.
%
%
% Feature extraction computation time proportion figure
\begin{figure}[!t]
\centering
\includegraphics[width=0.2\textwidth]{images/feChart-eps-converted-to.pdf}
\caption{Computation time proportion for each step of feature extraction step performed on GPU.}
\label{fig:feChart}
\end{figure}
%
% Feature extraction computation time table
\begin{table}[!t]
\renewcommand{\arraystretch}{1.43}
\caption{Feature extraction computation time. (unit: ms)}
\label{tab:compTimeFE}
\centering
\begin{tabular}{|c||c||c|}
\hline
Feature extraction step & CPU & GPU  \\
\hline
Optical-flow & 173 &  30  \\
\hline
Background subtraction & 5  &  1.5  \\
\hline
Persistence & 5  &  1.5  \\
\hline
Host to device transfer & --  & 1.2 \\
\hline
Device to host transfer &  -- &  1.8 \\
\hline
Other &  5  & 5 \\
\hline
Total &  183  & 41 \\
\hline
\end{tabular}
\end{table}
%
% needed in second column of first page if using \IEEEpubid
%\IEEEpubidadjcol

% An example of a floating figure using the graphicx package.
% Note that \label must occur AFTER (or within) \caption.
% For figures, \caption should occur after the \includegraphics.
% Note that IEEEtran v1.7 and later has special internal code that
% is designed to preserve the operation of \label within \caption
% even when the captionsoff option is in effect. However, because
% of issues like this, it may be the safest practice to put all your
% \label just after \caption rather than within \caption{}.
%
% Reminder: the "draftcls" or "draftclsnofoot", not "draft", class
% option should be used if it is desired that the figures are to be
% displayed while in draft mode.
%
%
% Note that the IEEE typically puts floats only at the top, even when this
% results in a large percentage of a column being occupied by floats.


% An example of a double column floating figure using two subfigures.
% (The subfig.sty package must be loaded for this to work.)
% The subfigure \label commands are set within each subfloat command,
% and the \label for the overall figure must come after \caption.
% \hfil is used as a separator to get equal spacing.
% Watch out that the combined width of all the subfigures on a 
% line do not exceed the text width or a line break will occur.
%
%\begin{figure*}[!t]
%\centering
%\subfloat[Case I]{\includegraphics[width=2.5in]{box}%
%\label{fig_first_case}}
%\hfil
%\subfloat[Case II]{\includegraphics[width=2.5in]{box}%
%\label{fig_second_case}}
%\caption{Simulation results for the network.}
%\label{fig_sim}
%\end{figure*}
%
% Note that often IEEE papers with subfigures do not employ subfigure
% captions (using the optional argument to \subfloat[]), but instead will
% reference/describe all of them (a), (b), etc., within the main caption.
% Be aware that for subfig.sty to generate the (a), (b), etc., subfigure
% labels, the optional argument to \subfloat must be present. If a
% subcaption is not desired, just leave its contents blank,
% e.g., \subfloat[].


% An example of a floating table. Note that, for IEEE style tables, the
% \caption command should come BEFORE the table and, given that table
% captions serve much like titles, are usually capitalized except for words
% such as a, an, and, as, at, but, by, for, in, nor, of, on, or, the, to
% and up, which are usually not capitalized unless they are the first or
% last word of the caption. Table text will default to \footnotesize as
% the IEEE normally uses this smaller font for tables.
% The \label must come after \caption as always.
%
%\begin{table}[!t]
%% increase table row spacing, adjust to taste
%\renewcommand{\arraystretch}{1.3}
% if using array.sty, it might be a good idea to tweak the value of
% \extrarowheight as needed to properly center the text within the cells
%\caption{An Example of a Table}
%\label{table_example}
%\centering
%% Some packages, such as MDW tools, offer better commands for making tables
%% than the plain LaTeX2e tabular which is used here.
%\begin{tabular}{|c||c|}
%\hline
%One & Two\\
%\hline
%Three & Four\\
%\hline
%\end{tabular}
%\end{table}

% Note that the IEEE does not put floats in the very first column
% - or typically anywhere on the first page for that matter. Also,
% in-text middle ("here") positioning is typically not used, but it
% is allowed and encouraged for Computer Society conferences (but
% not Computer Society journals). Most IEEE journals/conferences use
% top floats exclusively. 
% Note that, LaTeX2e, unlike IEEE journals/conferences, places
% footnotes above bottom floats. This can be corrected via the
% \fnbelowfloat command of the stfloats package.

\section{Conclusion}
\label{sec:conclusion}
In this paper, we addressed the ambiguity issue in topic models. For this purpose, several techniques are used to create a model which describes the scene without any ambiguity. Learning separate models for each visual feature prevents activities from different feature spaces to mix together in a single topic. In addition, a pre-processing step and several topic processing techniques are used to create a model in which topics present primitive actions each of them comprising a single blob with movement in a single direction (in the case of motion feature). In the proposed method, four search strategies including topic sequence, single topic, topic co-occurrence, and similar clips are proposed. A variety of queries can be defined using the proposed query formulation. The proposed formulation provides a straightforward way for users to interact with the system, and it can also speed up the search procedure by narrowing down the search space. In the proposed method, a complicated activity can be defined as a topic sequence that is searched using dynamic programming. Defining activities as a sequence of primitive topics brings a substantial performance improvement in the retrieval task compared to the other methods based on topic models. Furthermore, a lightweight database, not only occupies much fewer storage in our method, but it also results in a significant speed up in the search procedure compared to the methods which are based on low-level features. The feature extraction step which is computationally demanding performs in real time by leveraging the computing power of GPU. 
%
% if have a single appendix:
%\appendix[Proof of the Zonklar Equations]
% or
%\appendix  % for no appendix heading
% do not use \section anymore after \appendix, only \section*
% is possibly needed

% use appendices with more than one appendix
% then use \section to start each appendix
% you must declare a \section before using any
% \subsection or using \label (\appendices by itself
% starts a section numbered zero.)
%


%\appendices
%\section{Proof of the First Zonklar Equation}
%Appendix one text goes here.

% you can choose not to have a title for an appendix
% if you want by leaving the argument blank
%\section{}
%Appendix two text goes here.


% use section* for acknowledgment
%\section*{Acknowledgment}
%
%
%The authors would like to thank...


% Can use something like this to put references on a page
% by themselves when using endfloat and the captionsoff option.
\ifCLASSOPTIONcaptionsoff
  \newpage
\fi



% trigger a \newpage just before the given reference
% number - used to balance the columns on the last page
% adjust value as needed - may need to be readjusted if
% the document is modified later
%\IEEEtriggeratref{8}
% The "triggered" command can be changed if desired:
%\IEEEtriggercmd{\enlargethispage{-5in}}

% references section

% can use a bibliography generated by BibTeX as a .bbl file
% BibTeX documentation can be easily obtained at:
% http://mirror.ctan.org/biblio/bibtex/contrib/doc/
% The IEEEtran BibTeX style support page is at:
% http://www.michaelshell.org/tex/ieeetran/bibtex/
%\bibliographystyle{IEEEtran}
% argument is your BibTeX string definitions and bibliography database(s)
%\bibliography{IEEEabrv,../bib/paper}
%
% <OR> manually copy in the resultant .bbl file
% set second argument of \begin to the number of references
% (used to reserve space for the reference number labels box)
\begin{thebibliography}{1}


\bibitem{ref_1}
C. Zach, T. Pock, and H. Bischoff, "A duality based approach for realtime TV-L1 optical flow," in \textit{Proc. Pattern Recognition--DAGM}, Heidelberg, Germany, Sept 2007, vol. 4713, pp. 214--223.
%
\bibitem{ref_2}
M.A. Elgharib, V. Saligrama, and P.-M. Jodoin, "Retrieval in long surveillance videos using user described motion and object attributes", \textit{IEEE Trans. Circuits Syst. Vid. Technol.}, pp. 214--223, 2015.
% 
\bibitem{ref_3}
A. Sobral, and A. Vacavant, "A comprehensive review of background subtraction algorithms evaluated with synthetic and real videos", \textit{Computer Vis. and Image Und.}, vol. 122, pp. 4--21, 2014.
% 
\bibitem{ref_4}
Z. Zivkovic,  "Improved adaptive Gaussian mixture model for background subtraction", \textit{IEEE Conf. Pattern Recognition}, Stockholm, Sweden, Aug 2004, vol. 2, pp. 28--31.
% 
\bibitem{ref_5}
 D.M Blei, A.Y. Ng, and M.I. Jordan, "Latent dirichlet allocation", \textit{J. of Molecular Biology}, vol. 3, pp. 993--1022, 2003.
% 
\bibitem{ref_6}
T.F. Smith, and M.S. Waterman, "Identification of common molecular subsequences", \textit{J. molecular bio.}, vol. 147, no. 7, pp. 195--197, 1981.
 
\bibitem{ref_7}
X. Wang, X. Ma, and W.E.L. Grimson, "Unsupervised activity perception in crowded and complicated scenes using hierarchical bayesian models", \textit{IEEE TPAMI}, vol. 31, no. 3, pp. 539--555, 2009.
 
\bibitem{ref_8}
T. Le, M. Thonnat, and F. Bremond. "A query language combining object features and semantic events for surveillance video retrieval", \textit{Adv. in Mult. Modeling}, Jan 2008, vol. 4903, pp. 307--317.

\bibitem{ref_9}
D. Kuettel, M.D. Breitenstein, L. Van Gool, and V. Ferrari, "What's going on? Discovering spatio-temporal dependencies in dynamic scenes", \textit{IEEE Conf. Computer Vis. and Pattern Recog.}, San Francisco, USA, June 2010, pp. 1951--1958.
 
\bibitem{ref_10}
I. Pruteanu-Malinici, and L. Carin, "Infinite hidden Markov models for unusual-event detection in video," \textit{IEEE Trans. Image Process.}, vol. 17, no. 5, pp. 811--822, 2009.
 
\bibitem{ref_11}
S. Kim, S. Sundaram, P. Georgiou, and S. Narayanan, "Audio scene understanding using topic models," \textit{Neural Information Processing System (NIPS) Workshop}, Dec 2009.
 
\bibitem{ref_12}
L. Fei-Fei, and P. Perona, "A bayesian hierarchical model for learning natural scene categories," \textit{IEEE Conf. Computer Vis. and Pattern Recog.}, San Diego, USA, June 2005, vol. 2, pp. 524--531.
% 
\bibitem{ref_13}
P. Quelhas, F. Monay, J.-M. Odobez, D. Gatica-Perez, and T. Tuytelaars, "A thousand words in a scene," \textit{IEEE TPAMI}, vol. 29, pp. 1575--1589, 2005.
 
\bibitem{ref_14}
W. Hu, D. Xie, Z. Fu, W. Zeng, and S. Matbank, "Semantic-based surveillance video retrieval," \textit{IEEE Trans. Image Process.}, vol. 16, no. 4, pp. 1168--1181, 2007.
 
\bibitem{ref_15}
Y. Yang, B.C. Lovell, and F. Dadgostar, "Content-based video retrieval (cbvr) system for cctv surveillance videos," \textit{Dig. Image Comp.: Tech. and App.}, Melbourne, Australia, Dec 2009, pp. 183--187. 
 
\bibitem{ref_16}
R. Emonet, J. Varadarajan, and J-M. Odobez, "Temporal analysis of motif mixtures using Dirichlet processes," \textit{IEEE TPAMI}, vol. 36, no. 1, pp. 140--156, 2014.
 
\bibitem{ref_17}
W. Hu, X. Xiao, Z. Fu, D. Xie, T. Tan, and S. Maybank, "A system for learning statistical motion patterns," \textit{IEEE TPAMI}, vol. 28, no. 9, pp. 1450--1464, 2006.
 
\bibitem{ref_18}
X. Wang, K. Tieu, and E. Grimson, "Learning semantic scene models by trajectory analysis," in \textit{Proc. Eur. Conf. Computer Vision}, Graz, Austria, May 2006, vol. 3953, pp. 110--123.
 
\bibitem{ref_19}
T. Hospedales, S. Gong, and T. Xiang, "A markov clustering topic model for mining behaviour in video," \textit{IEEE Conf. Computer Vision}, Kyoto, Japan, Sept 2009, pp. 1165--1172.

\bibitem{ref_20}
M. Teh, L. Van Gool, C. K. I. Williams, J. Winn, and A. Zisserman, "The pascal visual object classes (voc) challenge," \textit{Int. J. of Computer Vision},  vol. 33, no. 8, pp. 303--338, 2010.

\bibitem{ref_21}
Y.W. Teh, M.I. Jordan, M.J. Beal, and D. M. Blei, "Hierarchical dirichlet processes," \textit{J. Am. Stat. Assoc.},  vol. 101, no. 476, pp. 1566--1581, 2006.

\bibitem{ref_23}
'QMUL Junction', http://www.eecs.qmul.ac.uk/~ccloy/ downloads\_qmul\_junction.html

\bibitem{ref_22}
'Mit Traffic', people.csail.mit.edu/xgwang/HBM.html.

\end{thebibliography}

% biography section
% 
% If you have an EPS/PDF photo (graphicx package needed) extra braces are
% needed around the contents of the optional argument to biography to prevent
% the LaTeX parser from getting confused when it sees the complicated
% \includegraphics command within an optional argument. (You could create
% your own custom macro containing the \includegraphics command to make things
% simpler here.)
%\begin{IEEEbiography}[{\includegraphics[width=1in,height=1.25in,clip,keepaspectratio]{mshell}}]{Michael Shell}
% or if you just want to reserve a space for a photo:

%\begin{IEEEbiography}{Michael Shell}
%Biography text here.
%\end{IEEEbiography}
%
% if you will not have a photo at all:
%\begin{IEEEbiographynophoto}{John Doe}
%Biography text here.
%\end{IEEEbiographynophoto}
%
% insert where needed to balance the two columns on the last page with
% biographies
%\newpage
%
%\begin{IEEEbiographynophoto}{Jane Doe}
%Biography text here.
%\end{IEEEbiographynophoto}

% You can push biographies down or up by placing
% a \vfill before or after them. The appropriate
% use of \vfill depends on what kind of text is
% on the last page and whether or not the columns
% are being equalized.

%\vfill

% Can be used to pull up biographies so that the bottom of the last one
% is flush with the other column.
%\enlargethispage{-5in}



% that's all folks
\end{document}


