\usepackage[pdftex]{graphicx}
\usepackage{sidecap}
\usepackage{cite}
\usepackage{verbatim}		% \begin[comment} command
\usepackage{amsmath}
\usepackage{amssymb}
\let\proof\relax 
\let\endproof\relax
\usepackage{amsthm}
\usepackage{mathtools}	% for dcases
\usepackage{grffile}	% for recognition of png file
\usepackage[tight,footnotesize]{subfigure}
\usepackage{microtype} % solve "Overfull \hbox i" warning
%\usepackage[misc]{ifsym}
\usepackage{color}
\usepackage{url}
\usepackage[ruled,vlined,linesnumbered]{algorithm2e}
\usepackage{bm} % for boldface maths symbols
%\usepackage{scalerel}
\usepackage{comment}
\usepackage{arydshln}
\let\labelindent\relax
\usepackage{enumitem}

\usepackage{adjustbox}
\usepackage{tikz}
\usetikzlibrary{shadings, patterns, angles, quotes, arrows.meta, shapes, decorations.pathmorphing, decorations.shapes, decorations.text, automata, positioning}
\usetikzlibrary{calc,intersections,arrows.meta}
\usepackage{pgfplots}

% hyperref should be placed at last
%\usepackage{hyperref}
%\hypersetup{
%	colorlinks=true,
%	linkcolor=blue,
%	citecolor=red,
%}

	\tikzset{
	pil/.style={
		->,
		thick,
		shorten <=2pt,
		shorten >=2pt,}
}

\theoremstyle{remark}
\newtheorem{remarkx}{Remark}
\newenvironment{remark}
{\pushQED{\qed}\renewcommand{\qedsymbol}{$\blacktriangleleft$}\remarkx}
{\popQED\endremarkx}
\newtheorem{examplex}{Example}
\newenvironment{example}
{\pushQED{\qed}\renewcommand{\qedsymbol}{\small$\blacktriangle$}\examplex}
{\popQED\endexamplex}

\theoremstyle{definition}
\newtheorem{defn}{Definition}
\newtheorem{assump}{Assumption}
\newtheorem{problem}{Problem}
\newtheorem*{problem*}{Problem}

\theoremstyle{plain}
\newtheorem{theorem}{Theorem}
\newtheorem{lemma}{Lemma}
\newtheorem{coroll}{Corollary}
\newtheorem{prop}{Proposition}

\newcommand{\rank}{{\rm rank}\,}
\newcommand{\defeq}{:=} %{\stackrel{\Delta}{=}}
\newcommand{\red}[1]{{\leavevmode\color{red}#1}}
\newcommand{\blue}[1]{{\leavevmode\color{blue}#1}}
\newcommand{\green}[1]{{\leavevmode\color{green}#1}}
\newcommand{\cyan}[1]{{\leavevmode\color{cyan}#1}}

\newcommand{\dist}{{\rm dist}}
\newcommand{\dt}{\frac{{\rm d}}{{\rm d}t}}
\newcommand{\expo}[1]{{\rm exp} \left(#1\right)}					% exponential function
\newcommand{\lambdamin}[1]{\lambda_{{\rm min}}\left(#1\right)}		% minimum eigenvalue
\newcommand{\lambdamax}[1]{\lambda_{{\rm max}}\left(#1\right)}		% maximum eigenvalue
\newcommand{\lambdainf}[1]{\lambda_{{\rm inf}}\left(#1\right)}		% infimum eigenvalue
\newcommand{\lambdasup}[1]{\lambda_{{\rm sup}}\left(#1\right)}		% supremum eigenvalue
\newcommand{\matr}[1]{\begin{bmatrix} #1 \end{bmatrix}}
\newcommand{\vmatr}[1]{\begin{vmatrix} #1 \end{vmatrix}}
\newcommand{\sig}{{\rm sig}}	% sig function
\newcommand{\sgn}{{\rm sgn}}	% signum function
\newcommand\undermat[2]{%
	\makebox[0pt][l]{$\smash{\underbrace{\phantom{%
					\begin{matrix}#2\end{matrix}}}_{\text{$#1$}}}$}#2} % underbrace in matrix
\newcommand{\transpose}[1]{#1^\top}
\newcommand{\inv}[1]{#1^{-1}}	% matrix inverse
\newcommand{\diag}[1]{{\rm diag}\{ #1\}}
\newcommand{\derivative}[2]{\frac{{\rm d} #1}{{\rm d} #2}}
%\newcommand{\dgamma}[1]{\frac{{\rm d} #1}{{\rm d}\gamma}}
\newcommand{\cmmnt}[1]{}
\newcommand{\chiup}{\raisebox{2pt}{$\chi$}}
\newcommand{\vf}{\chiup}
\newcommand{\identity}{{\rm id}}
\newcommand{\set}[1]{\mathcal{#1}}

\newcommand{\norm}[1]{\left\lVert#1\right\rVert}
\newcommand{\normm}[1]{\big\lVert#1\big\rVert}		% for cases if \norm is not satisfactory
\newcommand{\mbr}[1][{}]{\mathbb{R}^{#1}}	% \mbr[n]

\newcommand{\proj}[1]{{#1}^{\mathrm{prj}}}
\newcommand{\trs}[1]{{#1}^{\mathrm{trs}}}
\newcommand{\hgh}[1]{{#1}^{\mathrm{hgh}}}
\newcommand{\phy}[1]{{#1}^{\mathrm{phy}}}

\newcommand{\vfpf}{\prescript{\rm pf}{}{\vf}}
\newcommand{\vfco}{\prescript{\rm co}{}{\vf}}
\newcommand{\vfcb}{\mathfrak{X}}%{\prescript{\rm a}{}{\vf}}
\newcommand{\normv}[1]{\overline{#1}}			% normalization of a vector

%%% commands for multi-robot
\newcommand{\multivfcb}[2]{\vfcb_{#1#2}}
\newcommand{\multihatvfcb}[2]{ {\normv{\vfcb_{#1}}}_{#2} }

\newcommand{\multiphi}[2]{\phi_{#1#2}}
\newcommand{\multif}[2]{f_{#1#2}}
\newcommand{\multifdot}[2]{f_{#1#2}'}
\newcommand{\multifddot}[2]{f_{#1#2}''}
\newcommand{\multix}[2]{x_{#1#2}}
\newcommand{\multik}[2]{k_{#1#2}}
\newcommand{\multid}[2]{d_{#1#2}}
\newcommand{\multidstar}[2]{d^{*}_{#1#2}}
\newcommand{\sat}{\mathrm{Sat}_{a}^{b}}


%\newcommand{\multiphi}[2]{\prescript{#1}{}{\phi}_{#2}}
%\newcommand{\multif}[2]{\prescript{#1}{}{f}_{#2}}
%\newcommand{\multifdot}[2]{\prescript{#1}{}{f_{#2}' }}
%\newcommand{\multifddot}[2]{\prescript{#1}{}{f_{#2}''}}
%\newcommand{\multix}[2]{\prescript{#1}{}{x}_{#2}}
%\newcommand{\multik}[2]{\prescript{#1}{}{k}_{#2}}
%\newcommand{\multid}[2]{\prescript{#1}{}{d_{#2}}}
%\newcommand{\multidstar}[2]{\prescript{#1}{}{d^{*}_{#2}}}

\newcommand{\scalemath}[2]{\scalebox{#1}{\mbox{\ensuremath{\displaystyle #2}}}}
%\newcommand{\scalemathresize}[2]{\resizebox{#1\hsize}{!}{$#2$}}

%\renewcommand{\subset}{\red{\textbf{!!!!!!!!!!!!!!}}}	% just to remind me to use \subseteq instead.

%\newcommand{\black}[1]{{\leavevmode\color{black}#1}}
%\renewcommand{\red}{\black}
