\section{Missing Proofs from Section \ref{sec:preliminaries}}\label{apdx:prelims}

\obsblocking*
\begin{proof}
\begin{description}[leftmargin=0cm]
\item[$(\Longrightarrow)$]
If a blocking pair $(h, S)$ exists, then since $v_h(S) > v_h(X_h)$, there is some doctor $d \in S$ such that $\Delta_h(X_h, d) = 1$; this follows from the matroid augmentation property (\Cref{obs:augmentation}). 
For this doctor $d \in S$, from the definition of a blocking pair, $h \succ_d X(d)$. 
\item[$(\Longleftarrow)$] if there exists some doctor $d$ and hospital $h$ such that $\Delta_h(X_h, d) = 1$ and $h \succ_d X(d)$, then it is easy to see that $(h, X_h + d)$ forms a blocking pair: $v_h(X_h+d)> v_h(X_h)$, and for all $d' \in X_h+d$ $h \succeq_{d'} X(d')$. 
\end{description} 
\end{proof}

\lemgeneralwelfare*
\begin{proof}
    Since doctors have a strict and complete preference ordering over hospitals, for an allocation to be stable, it has to be maximal --- no unmatched doctor can have a non-zero marginal value for any hospital given the current assignment.  It is well known that when hospitals have submodular valuations, maximal allocations give a 2-approximation to the optimal hospital-USW~\cite{lehmann2006auctions}.
\end{proof}

\section{Missing Proofs for the HWSD Mechanism}\label{apdx:hwsd}

\begin{theorem}\label{thm:red-jacket-apdx}
The high welfare Serial Dictatorship (HWSD) mechanism has the following properties:
\begin{enumerate}[(i)]
    \item The mechanism is doctor-SP.% strategyproof
    \item The algorithm runs in polynomial time.
    \item The output allocation maximizes hospital-\USW.
    \item The output allocation is stable.
\end{enumerate}
\end{theorem}
\begin{proof}
We divide this proof into a series of claims. 
\begin{claim}
\Cref{algo:red-jacket} outputs an allocation maximizes hospital-\USW. 
\end{claim}
\begin{proof}
At every round, we maintain the invariant that there exists a socially optimal allocation $Y$ so that each hospital $h$'s assigned bundle of doctors $X_h$ is contained in $Y_h$. Thus, when the algorithm terminates, no additional doctors can be assigned, and the resulting allocation maximizes hospital-\USW.
\end{proof}

\begin{claim}
The mechanism is doctor-SP.
\end{claim}
\begin{proof}
The mechanism is doctor-SP since each doctor $d$'s preferences are only used to assign them to their highest ranked hospital that maintains non-redundancy and satisfies the max \USW invariant. Since the max \USW invariant and non-redundancy do not depend on the doctor $d$'s preferences, misreporting preferences can only lead to a worse assignment for $d$ (and any doctor by extension).
\end{proof}

\begin{claim}
\Cref{algo:red-jacket} computes a stable allocation.
\end{claim}
\begin{proof}
A new doctor only gets added to a hospital's allocation if it does not violate non-redundancy. Therefore, the final allocation must be non-redundant. 

Assume for contradiction that the output allocation (say $X$) has a blocking pair. By \Cref{obs:blocking}, there must be some doctor $d$ and some hospital $h$ such that $\Delta_h(X_h, d) = 1$ and $h \succ_d X(d)$. If $d$ is not allocated in $X$, then we get that $d$ can be added to $X_h$, and the social welfare strictly increases, contradicting the fact that the algorithm finds a hospital-\USW maximizing allocation. Therefore, assume that $h_d= X(d)$ is the hospital $d$ was assigned by \Cref{algo:red-jacket}. 
Suppose that we move the doctor $d$ from $X_{h_d}$ to $X_h$. 
This creates a new non-redundant allocation $X'$ where $X'_{h_d} = X_{h_d}-d$ and $X_{h}' = X_{h}+d$, and the remaining bundles are the same. Since the output of \Cref{algo:red-jacket} is non-redundant and $\Delta_{h}(X_{h},d)=1$, we have that $X'$ is also hospital-\USW maximizing: hospital $h_d$ lost a utility point and hospital $h$ gained one. 
    
Let $X^t$ be the interim allocation at the beginning of the $t$-th iteration of \Cref{algo:red-jacket}. Since doctors are never unassigned from hospitals, we have that for every hospital $h \in H$, $X_h^t \subseteq X_h^{t+1}$. 
Consider the iteration $t$ where $d$ was allocated and let $X^t$ be the allocation at the start of this iteration.  
Since $X_h^t$ is a subset of the final allocation $X_h$ to $h$ and $\Delta_h(X_h,d)= 1$, we must have $\Delta_h(X^t_h, d) = 1$ by submodularity. 
By construction there exists a max hospital-\USW allocation $X'$ with $X_{h'} \subseteq X'_{h'}$ for all $h' \ne h$, and $X_h^t + d \subseteq X'_h$. 
This implies that the hospital $h_d$ allocated by \Cref{algo:red-jacket} must satisfy $h_d \succ_d h$ contradicting the definition of a blocking pair. 
So, no such blocking pair exists, and the allocation is stable.
\end{proof}

\begin{claim}
\Cref{algo:red-jacket} runs in polynomial time.
\end{claim}
\begin{proof}
\Cref{algo:red-jacket} runs a \texttt{for} loop over the doctors and then examines hospitals in decreasing order of doctor prefernece in another \texttt{for} loop, for a maximum of $nm$ iterations. 
In each iteration, we test whether there exists some hospital-\USW maximizing allocation $Y$ such that $X_h \subseteq Y_h$ for all $h \in H$.   
Thus, to prove our claim we need to show that it is possible to compute the maximum hospital-\USW possible given a partial allocation $X$. 
Computing the maximum hospital-\USW allocation can be done in polynomial time using value queries, see e.g., \citet{benabbou2020finding}, who reduce the problem to the matroid intersection problem \cite{edmonds1979matroidintersection}. 

We also note that for any $T \subseteq D$, if $v_h$ is a matroid rank function, $v'_h$ defined as $v'_h(S) = v_h(T \cup S) - v_h(T)$ is also a matroid rank function. 
Therefore, we can compute the maximum hospital-\USW possible with hospital valuations $v'_h$ defined as $v'_h(S) = v_h(X_h \cup S) - v_h(X_h)$ for each $h \in H$. We can therefore efficiently determine if there exists a max hospital-\USW $Y$ such that for each $h$, $X_h \subseteq Y_h$. 
We conclude that \Cref{algo:red-jacket} runs in polynomial time.
\end{proof}
\end{proof}

% \thmredjacketsp*


\corredjacketadditive*
\begin{proof}
To prove this, we prove the following improved version of Lemma \ref{lem:manip-half}.

\begin{lemma}
Consider some hospital $h$. Fixing the reports of all other hospitals, let $X$ and $Y$ be the resulting allocation when $h$ reports $f_T$ and $v_h$, respectively. When agents report capped additive valuations, if $v_h(T)=|T|$ and $X_h=T$, then $|Y_h| \ge |T|$.
\end{lemma}
\begin{proof}
Similar to Lemma \ref{lem:manip-half}, we call the preference profile where $h$ reports $f_T$ the {\em misreported preferences}, and the profile where $h$ reports $f_T$ the {\em true preferences}.
Assume for contradiction that $|Y_h| < |T|$. Since the allocation $X$ is non-redundant with respect to the true preferences, it must be worse according to the HWSD ordering than $Y$. This implies $\USW(Y) \ge \USW(X)$. 

Therefore, since $|Y_h| < |T| = |X_h|$, there must be some hospital $h'$ such that $|Y_{h'}| > |X_{h'}|$. Let $d$ be some doctor in $Y_{h'} \setminus X_{h'}$. We create a path starting from the doctor $d$ just like we have done in the previous lemmas. However, since all agents have capped additive valuations, we can be less rigorous with how we define these paths. We move the doctor $d$ to the hospital $h_1$ it is allocated to in the allocation $X$. If this transfer results in $h_1$'s bundle exceeding its capacity, then there must be some doctor $d_1$ in $Y_{h_1}$ which is allocated to some other hospital (say $h_2$) in $X$. We repeat this process with $d_1$ till we make a transfer that respects the capacity constraints of each hospital. This creates a path $P = (h_0 = h', d_1 = d, h_1, d_2, \dots, d_k, h_{k+1})$. We define a transfer along this path as moving the doctor $d_i$ from hospital $h_{i-1}$ to $h_i$. The transfer along the path $P$ when applied to the allocation $Y$ gives us an allocation $Y'$ which is non-redundant with respect to both the true preferences. 

Since $|Y_h| < |X_h|$, the only place that hospital $h$ could appear in this path is right at the end (at $h_{k+1}$), since an addition of a single doctor to hospital $h$ (under the allocation $Y$) will not violate its capacity constraints. This means if we reverse the path and apply it to the allocation $X$, the new allocation (say $X'$) will be non-redundant with respect to the misreported preferences.  

Since $X$ and $X'$ are non-redundant with respect to the misreported preferences $X$ must be better than $X'$ according to the HWSD ordering (\Cref{obs:hwsd-ordering}). Since the only difference between the two allocations are the agents on the path, the allocation $X' \cap D_P$ is worse than the allocation $X \cap D_P$ according to the HWSD ordering. We now observe that $X' \cap D_P$ is equal to $Y \cap D_P$ and $X \cap D_P$ is equal to $Y' \cap D_P$. This implies $Y \cap D_P$ is worse than $Y' \cap D_P$ according to the HWSD ordering. Since the allocations $Y$ and $Y'$ only differ in the assignments of the doctors in $D_P$, this implies $Y$ is worse than $Y'$ is worse according to the HWSD ordering. This contradicts the fact that Algorithm \ref{algo:red-jacket} outputs $Y$ under the true preferences.
\end{proof}


Plugging this improvement into the proof of Theorem \ref{thm:red-jacket-strategyproofness} gives us the required result. 
\end{proof}

\section{Missing Proofs for the SD Algorithm}\label{apdx:sd}
\begin{theorem}\label{thm:serial-dictatorship-apdx}
The doctor serial dictatorship mechanism has the following properties:
\begin{enumerate}[(i)]
    \item The mechanism is doctor strategyproof
    \item The output allocation is stable
    \item The algorithm runs in polynomial time
    \item The output allocation is $2$-approximately max hospital-\USW.
\end{enumerate}
\end{theorem}
\begin{proof}
Again, we prove this using a series of claims. Note that (iv) follows from \Cref{lem:stable_welfare}. 

\begin{claim}
Algorithm \ref{algo:doctor-round-robin-revisited} runs in polynomial time.  
\end{claim}
\begin{proof}
\Cref{algo:doctor-round-robin-revisited} runs in polynomial time since we simply go down the doctors' preference lists in decreasing order, and run a poly-time procedure to check whether each hospital is willing to accept the doctor. 
\end{proof}

\begin{claim}
\Cref{algo:doctor-round-robin-revisited} outputs a stable allocation.   
\end{claim}
\begin{proof}
A new doctor $d$ only gets added to a bundle $X_h$ if $\Delta_h(X_h,d) = 1$. 
Therefore the final allocation is non-redundant. 

If there is a blocking pair, there must be some $d$ and $h$ such that $\Delta_h(X_h, d) = 1$ and $h \succ_d X(d)$. Assuming for contradiction that such a pair exists. 
Consider the iteration where $d$ was allocated and let $Y$ be the allocation at the start of this iteration. By submodularity, we must have $\Delta_h(Y_h, d) = 1$ as well which implies that $X(d) \succ_d h$ contradicting the definition of a blocking pair. So, no such blocking pair exists, and the allocation is stable.
\end{proof}

\begin{claim}
\Cref{algo:doctor-round-robin-revisited} is doctor-SP.
\end{claim}
\begin{proof}
The mechanism is  doctor-SP since each doctor $d$ has a fixed iteration where they are assigned, and at this iteration they are assigned to their most preferred hospital who can accept them. 
As the hospitals' allocations at the iteration where the doctor $d$ is allocated is independent of the doctor's preferences, the doctor $d$ cannot obtain a strictly better hospital by misreporting their preferences.    
\end{proof}

\end{proof}

% \lemmatroidproperty*



\section{Missing Proofs from Section \ref{sec:cardinal-utilities}}\label{apdx:cardinal-utilities}


\thmmaxnashwelfare*
\begin{proof}
We first show that the algorithm terminates in polynomial time. To do this, we show that the \textbf{while} loop terminates in polynomial time. This is sufficient since we run the \textbf{while} loop for $m$ different values of $k$. 

We use a potential function argument to bound the number of \textbf{while} loop iterations. Every time we change the allocation, we strictly reduce $\sum_{i \in H} \left (|X_i| + \frac{i}{n^2}\right)^2$. Therefore, the while loop can only run $\cal O(m^2n^2)$ times. We note that this exact potential function argument has been used before in \citet{babaioff2021EF} and \citet{cousins2023mixedmanna}

Next, we prove the correctness of our algorithm.
Each allocation $X^k$ computed by the algorithm maximizes doctor-\USW subject to the constraint that total hospital-\USW is $k$. Since we pick an allocation $X^k$ that maximizes doctor-\USW, the allocation $X$ output by \Cref{algo:max-nash-welfare} maximizes doctor-\USW.
Next, we show that $X$ maximizes hospital Nash welfare. 
Let $Y$ be an allocation that 
\begin{inparaenum}[(a)]
    \item maximizes doctor-\USW, 
    \item hospital-\NSW  subject to (a), and 
    \item lexicographically dominates all other allocations that satisfy (a) and (b). 
\end{inparaenum}
Furthermore, assume that $Y$ has a hospital social welfare of $k$. 

Let us consider the allocation $X^k$ that Algorithm \ref{algo:max-nash-welfare} computes. 
Assume for contradiction that $X^k \ne Y$. Since $X^k \ne Y$, there is some hospital $i$ such that $|X_i^k| < |Y_i|$. If there are multiple such hospitals, we break ties by choosing the $i$ with lower $|X_i|$, and breaking further ties by choosing the lower $i$. 
Since both $X^k$ and $Y$ have a total hospital-\USW of $k$, there is some hospital $j$ such that $|Y_j| < |X_j^k|$; again, if there are multiple such hospitals, we break ties by choosing the $j$ with lower $|Y_j|$, and breaking further ties by choosing the lower $j$. . 

We have four possible cases, each leading to a contradiction proving that $X^k = Y$.

\begin{description}[leftmargin=0cm]
    \item[Case 1: $|X_i^k| = |Y_j|$ and $i <j$.] We apply \Cref{lem:max-nash-welfare-main} with the allocation $X^k$ and hospital $i$ to find a hospital $\ell$ and a $k$-MDW allocation $Z$ such that the hospital $i$ gets one more doctor, the hospital $\ell$ gets one less doctor and all other hospitals have the same bundle size. We have $|X^k_{\ell}| - 1 \ge |Y_{\ell}| \ge |Y_j|  = |X^k_i|$. 
    If any of these inequalities is strict, then $|X^{k}_{\ell}| \ge |X^k_i| + 2$; in this case, \Cref{algo:max-nash-welfare} reduces the capacity of $\ell$ by 1 and increases the capacity of $i$ by $1$, and moves from the allocation $X^k$ to $Z$, which contradicts $X^k$ being the resulting allocation. 
    If equality holds throughout, then $|X^k_{\ell}| = |X^k_i| + 1$ and $\ell >j > i$, which again contradicts the logic of \Cref{algo:max-nash-welfare}.
    \item[Case 2: $|X^k_i| < |Y_j|$.] This case follows a similar argument to that of Case 1. 
    \item[Case 3: $|X^k_i| = |Y_j|$ and $j < i$.] In this case, we plug in Lemma \ref{lem:max-nash-welfare-main} with allocation $Y$ and hospital $j$ to find a hospital $\ell$ and a $k$-MDW allocation $Z$ such that $j$ gets one more doctor, $\ell$ gets one less doctor and all other hospitals have the same bundle size. We have $|Y_{\ell}| - 1 \ge |X^k_{\ell}| \ge |X^k_i|  = |Y_j|$. 
    If any of these inequalities is strict, then $|Y_{\ell}| \ge |Y_j| + 2$ which means $Z$ has a higher hospital Nash welfare than $Y$, a contradiction to $Y$ being hospital-\NSW optimal. 
    If equality holds throughout, then $|Y_{\ell}| = |Y_j| + 1$ and $\ell > i> j$, which implies that $Z$ is hospital-\NSW optimal as well but lexicographically dominates $Y$, which again violates our assumption on $Y$.
    \item[Case 4: $|X^k_i| > |Y_j|$.] The proof of this case follows a similar argument to that of Case 3. 
\end{description}
Therefore, $X^k = Y$, and the Nash welfare of our output allocation $X$ is at least that of $Y$.
\end{proof}
