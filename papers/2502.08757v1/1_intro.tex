\Gls{mMIMO} technology has ushered in a transformative era in wireless communication, offering remarkable spectral efficiency and network capacity improvements.
Downlink precoding is a crucial technique for enhancing the performance of multi-antenna cellular networks, particularly in \gls{mMIMO} systems. The optimization of precoding vectors is often formulated as a sum-rate maximization problem, which is non-convex and NP-hard \cite{luo2008dynamic}. This problem is typically addressed using iterative methods, such as the \gls{WMMSE} algorithm, known for its effectiveness in many scenarios \cite{shi2011iteratively}. 

In addition to the above optimization methods, some solutions leverage \gls{DL} techniques to simplify the precoder design. The authors in \cite{hojatian2021unsupervised} train a self-supervised model to predict the precoding matrix. In contrast, \cite{9403959} proposes unfolding the WMMSE algorithm by building a model that learns a specific optimization procedure for the WMMSE parameters, following the original WMMSE solution. Moreover, the works in \cite{yang2022learning, 8935405, lyu2023downlink} take a different approach by predicting individual components of the \gls{WMMSE} algorithm, which are combined to reconstruct the precoding matrix.

However, solving the sum-rate maximization problem using the \gls{WMMSE} algorithm requires iterative optimization and matrix inversions, which significantly increase the computational load and delay, particularly as the number of antennas increases, leading to an order of complexity of $O(N^3_{\sf{T}})$ \cite{shi2011iteratively}. Instead, approaches that leverage deep learning to directly estimate the precoding matrix can achieve better tradeoffs between performance and complexity \cite{hojatian2021unsupervised}. Nonetheless, these methods are limited in their ability to generalize when the \gls{CSI} distribution changes. Therefore, while \gls{DL}-based techniques offer the advantage of reduced complexity and faster inference, their effectiveness is constrained by their lack of adaptability to changing channel conditions. 

In this paper, we propose a ``plug-and-play precoder'' (PaPP) estimator for \gls{mMIMO} communication systems. This approach leverages \gls{WMMSE} for training while incorporating \gls{DL} techniques. The PaPP maintains high performance in terms of sum-rate, even with changes in base station locations. To ensure robust performance across different deployment environments, we incorporate \gls{MLDG} in the training, which enhances the model's ability to generalize effectively.
To reduce complexity, we employ a teacher-student training approach \cite{hu2023teacher}. This approach leverages a well-performing, complex model (the teacher) to transfer knowledge to a simpler, more efficient model (the student). Here, the student model learns to approximate the teacher’s predictions, enabling it to retain much of the teacher’s accuracy while operating with significantly reduced computational complexity. Additionally, we reduced training complexity by training the teacher model using a \gls{SSL} approach for precoding matrix estimation. This setup is ideal for field deployment, where the lightweight and faster student model can be used in real-time applications with limited computational resources.
Given the computational intensity of matrix inversion in the WMMSE algorithm, we implement simple \glspl{FCL} in the student model to estimate the precoding matrix based on the teacher model’s WMMSE-driven predictions.

Finally, to test the system in a variety of propagation environments, we assemble a novel dataset based on a realistic map of Montreal, incorporating buildings and obstacles to simulate urban environments. This allowed us to test the PaPP with a variety of base station configurations, demonstrating strong performance in diverse and previously unseen scenarios.

The remainder of this paper is structured as follows. Section \ref{Sec:Problem_Formulation} introduces the system model and problem formulation, outlining WMMSE for precoding in massive MIMO systems. Section \ref{Sec:Proposed} presents the PaPP, detailing the algorithm, architecture, and dataset construction. Section \ref{Sec:Complexity} compares the computational complexity and sum-rate performance of the PaPP with existing approaches. Section \ref{Sec:Simulation} discusses numerical results, highlighting the generalization capabilities and efficiency of the PaPP. Finally, Section \ref{Sec:conclusion} concludes the paper.