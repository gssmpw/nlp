\documentclass[conference]{IEEEtran}
%****************this is required for ICMLCN**********
\usepackage[letterpaper, left=0.65in, right=0.65in, bottom=1in, top=0.8in]{geometry}
\IEEEoverridecommandlockouts
\special{papersize=8.5in,11in}
%****************************************************
%****************************************************%
%               LIST OF ACRONYM HERE                 %
%****************************************************%
\usepackage[acronym]{glossaries}
\newacronym{BS}{BS}{base station}
\newacronym{PS}{PS}{phase-shifter}
\newacronym{RL}{RL}{reinforcement learning}
\newacronym{AP}{AP}{analog precoder}
\newacronym{FC-HBF}{FC-HBF}{fully-connected HBF}
\newacronym{FSA-HBF}{FSA-HBF}{fixed subarray HBF}
\newacronym{DSA-HBF}{DSA-HBF}{dynamic subarray HBF}
\newacronym{BF}{BF}{beamforming}
\newacronym{UE}{UE}{user equipment}
\newacronym{AWGN}{AWGN}{additive white gaussian noise}
%\newacronym{BS}{BS}{base station}
\newacronym{MIMO}{MIMO}{multiple-input multiple-output}
\newacronym{MISO}{MISO}{multiple-input single-output}
\newacronym{RF}{RF}{radio frequency}
\newacronym{RIS}{RIS}{reconfigurable intelligent surfaces}
\newacronym{IOT}{IOT}{internet-of-things}
\newacronym{CL}{CL}{convolutional layer}
\newacronym{FDD}{FDD}{frequency division duplex}
\newacronym{TDD}{TDD}{time division duplex}
\newacronym{CSI}{CSI}{channel state information}
\newacronym{DNN}{DNN}{deep neural network}
\newacronym{DP}{DP}{digital precoder}
\newacronym{DL}{DL}{deep learning}
\newacronym{SVD}{SVD}{singular-value decomposition}
\newacronym{CNN}{CNN}{convolution neural network}
\newacronym{FDP}{FDP}{fully digital precoder}
\newacronym{SE}{SE}{spectral efficiency}
\newacronym{OFDM}{OFDM}{orthogonal frequency division multiplexing}
\newacronym{OMP}{OMP}{orthogonal matching pursuit}
\newacronym{FL}{FL}{fully-connected layer}
\newacronym{HSHO}{HSHO}{Hybrid Structured Heuristic Optimization}
\newacronym{HBF}{HBF}{hybrid beamforming}
\newacronym{IA}{IA}{initial access}
\newacronym{mm-Wave}{mm-Wave}{millimeter wave}
\newacronym{mMIMO}{mMIMO}{massive MIMO}
\newacronym{SINR}{SINR}{signal-to-interference-noise ratio}
\newacronym{SNR}{SNR}{signal-to-noise ratio}
%\newacronym{SS}{SS}{synchronization signal}
%\newacronym{SSB}{SSB}{synchronization signal burst}
\newacronym{RSSI}{RSSI}{received signal strength indicator}
\newacronym{PZF}{PZF}{phase zero forcing}
\newacronym{PSO}{PSO}{particle swarm optimization}
\newacronym{ZF}{ZF}{zero forcing}
\newacronym{O-FDP}{O-FDP}{optimal fully digital precoder}
\newacronym{JT}{JT}{joint transmission}
\newacronym{CU}{CU}{central unit}
\newacronym{MSE}{MSE}{mean square error}
\newacronym{CEL}{CEL}{cross entropy loss}
\newacronym{CB}{CB}{conjugate beamforming}
\newacronym{NC}{NC}{network controller}
\newacronym{CoMP}{CoMP}{coordinated multi point}
\newacronym{CF-mMIMO}{CF-mMIMO}{cell-free massive MIMO}
\newacronym{CF-HBF}{CF-HBF}{cell-free hybrid beamforming}
\newacronym{CF-BF}{CF-BF}{cell-free beamforming}
\newacronym{MLDG}{MLDG}{meta-learning domain generalization}
\newacronym{MAML}{MAML}{model agnostic meta-learning}
\newacronym{WSR}{WSR}{weighted sum rate}
\newacronym{WMMSE}{WMMSE}{weighted minimum mean square error}
\newacronym{NN}{NN}{neural network}
\newacronym{LOS}{LOS}{line of sight}
\newacronym{NLOS}{NLOS}{non line of sight}
\newacronym{ML}{ML}{machine learning}
\newacronym{FCL}{FCL}{fully connected layer}
\newacronym{SSL}{SSL}{semi-supervised learning}
 
\newif\ifDeepMIMOModel
\DeepMIMOModeltrue

\newif\ifSimpleNParamEq
\SimpleNParamEqtrue

\usepackage{lettrine}
% \usepackage[ruled,vlined,linesnumbered]{algorithm2e}
\usepackage{multirow}
\usepackage{longtable}
\usepackage[table,xcdraw]{xcolor}
\usepackage[linesnumbered,ruled,vlined]{algorithm2e}
\usepackage{float}
\floatplacement{figure}{H}
\makeatletter
\let\oldlt\longtable
\let\endoldlt\endlongtable
\def\longtable{\@ifnextchar[\longtable@i \longtable@ii}
\def\longtable@i[#1]{\begin{figure}[t]
\onecolumn
\begin{minipage}{0.5\textwidth}
\oldlt[#1]
}
\def\longtable@ii{\begin{figure}[t]
\onecolumn
\begin{minipage}{0.5\textwidth}
\oldlt
}
\def\endlongtable{\endoldlt
\end{minipage}
\twocolumn
\end{figure}}
\makeatother
\usepackage{ifpdf}
% *** GRAPHICS RELATED PACKAGES ***
\ifCLASSINFOpdf
  \usepackage[pdftex]{graphicx}
  \usepackage{graphicx,epstopdf}
  \graphicspath{{../pdf/}{../jpeg/}}
  \DeclareGraphicsExtensions{.pdf,.jpeg,.png,.eps}
\else
  \usepackage[dvips]{graphicx}
  \usepackage{graphicx,epstopdf}
  \graphicspath{{../eps/}}
  \DeclareGraphicsExtensions{}
\fi
\usepackage{tikz}
\usetikzlibrary{decorations.pathreplacing,calc}
\newcommand{\tikzmark}[1]{\tikz[overlay,remember picture] \node (#1) {};}

\newcommand*{\AddNote}[4]{%
    \begin{tikzpicture}[overlay, remember picture]
        \draw [decoration={brace,amplitude=0.5em},decorate,line width=.2mm,black]
            ($(#3)!(#1.north)!($(#3)-(0,1)$)$) --  
            ($(#3)!(#2.south)!($(#3)-(0,1)$)$)
                node [align=center, text width=2.5cm, pos=0.5, anchor=west] {#4};
    \end{tikzpicture}
}%

\usepackage{cite}
\usepackage{amsthm}
\usepackage{steinmetz}
\usepackage{amssymb}
\usepackage{balance}
\usepackage{eqparbox}
\usepackage{multirow}
\usepackage{bbm}
\usepackage{float}
\SetAlFnt{\small}


\usepackage{amsmath}

\usepackage{mathtools, nccmath}
\DeclarePairedDelimiter{\nint}\lfloor\rceil
\DeclarePairedDelimiter{\abs}\lvert\rvert

\usepackage{booktabs, siunitx}
% \usepackage[svgnames,table]{xcolor}
%\usepackage[tableposition=above]{caption}

% \usepackage{graphicx}
\usepackage[scaled]{DejaVuSansMono}
\usepackage[T1]{fontenc}

%\usepackage{stfloats}
%\usepackage[eulergreek]{sansmath}
% \usepackage[final]{graphicx}
\usepackage{color}
\usepackage{relsize}
\usepackage{mathtools}
\newcommand{\ex}{{\mathrm e}}
\newtheorem{definition}{Definition}
\newtheorem{lemma}{Lemma}
\newtheorem{theorem}{Theorem}
\newtheorem{remark}{Remark}

\newtheorem{proposition}{Proposition}
\newcommand{\bs}[1]{\boldsymbol{#1}}
\newcommand{\mc}[1]{\mathcal{#1}}
\newcommand{\ul}[1]{\underline{#1}}
\newcommand{\mb}[1]{\mathbf{#1}}
\newcommand{\mr}[1]{\mathrm{#1}}
\newcommand{\tr}{\mathrm{Tr}}
\newcommand{\fo}{\mathbf{F}_{\mathrm{opt}}}

% \newcommand{\ceil}[1]{\lceil {#1} \rceil}
\usepackage{mathtools}
\DeclarePairedDelimiter{\ceil}{\lceil}{\rceil}

\DeclareMathOperator*{\argmin}{arg\;min}
\DeclareMathOperator*{\argmax}{arg\;max}
\DeclareMathOperator*{\maximize}{maximize}
\DeclareMathOperator*{\minimize}{minimize}
\DeclareMathOperator*{\st}{subject\;to}
\renewcommand{\Pr}{\mathbb{P}} % (by default \Pr is rendered as "Pr")
\pagestyle{empty}
% \thispagestyle{plain}
% \pagestyle{plain}
\SetKwInput{KwInput}{Input}                % Set the Input
\SetKwInput{KwOutput}{Output}
\SetKwInput{KwOutputr}{Output Regression}              % set the Output
\SetKwInput{KwOutputc}{Output Classification}              % set the Output
\newcommand{\rank}{\operatornamewithlimits{rank}}
\newcommand{\trace}{\operatornamewithlimits{trace}}
%\newcommand{\argmin}{\operatornamewithlimits{argmin}}
%\newcommand{\argmax}{\operatornamewithlimits{argmax}}
\newcommand{\bseq}{\begin{subequations}}
\newcommand{\eseq}{\end{subequations}}
\newcommand{\baln}{\begin{align}}
\newcommand{\ealn}{\end{align}}
\newcommand{\balnd}{\begin{aligned}}
\newcommand{\ealnd}{\end{aligned}}
\newcommand{\beq}{\begin{equation}}
\newcommand{\eeq}{\end{equation}}
\newcommand{\beqn}{\begin{eqnarray}}
\newcommand{\eeqn}{\end{eqnarray}}
\newcommand{\beqno}{\begin{eqnarray*}}
\newcommand{\eeqno}{\end{eqnarray*}}
\newcommand{\bma}{\begin{displaymath}}
\newcommand{\ema}{\end{displaymath}}
\newcommand{\bnu}{\begin{enumerate}}
\newcommand{\enu}{\end{enumerate}}
\newcommand{\bce}{\begin{center}}
\newcommand{\ece}{\end{center}}
\newcommand{\btb}{\begin{tabular}}
\newcommand{\etb}{\end{tabular}}
\newcommand{\ba}{\begin{array}}
\newcommand{\ea}{\end{array}}
%\setlength\arraycolsep{2pt}
\usepackage{footnote}
\makesavenoteenv{tabular}
\makesavenoteenv{table}
\makeatletter 
\newcommand\semiHuge{\@setfontsize\semiHuge{21.1}{27.38}}
\makeatother

\usepackage{subfigure}
\usepackage{float}
% \setlength{\textfloatsep}{0pt}
% \setlength{\parskip}{0pt} 
%****************************************************
\begin{document}
\title{A Low-Complexity Plug-and-Play Deep Learning Model for Massive MIMO Precoding Across Sites\\
\author{\IEEEauthorblockN{Ali~Hasanzadeh~Karkan$^{*}$, Ahmed~Ibrahim$^{\dagger}$, Jean-François~Frigon$^{*}$, and François~Leduc-Primeau$^{*}$\\
$^{*}$Department of Electrical Engineering, Polytechnique Montréal, Montréal, QC H3C 3A7, Canada\\
$^{*}$Emails: \{ali.hasanzadeh-karkan, j-f.frigon, francois.leduc-primeau\}@polymtl.ca. \\
$^{\dagger}$Ericsson Canada's R\&D,  Kanata, ON K2K 2V6, Canada, Email: ahmed.a.ibrahim@ericsson.com.
}
}
}

\maketitle
\IEEEpubidadjcol

\begin{abstract}
Massive multiple-input multiple-output (mMIMO) technology has transformed wireless communication by enhancing spectral efficiency and network capacity. This paper proposes a novel deep learning-based mMIMO precoder to tackle the complexity challenges of existing approaches, such as weighted minimum mean square error (WMMSE), while leveraging meta-learning domain generalization and a teacher-student architecture to improve generalization across diverse communication environments. 
When deployed to a previously unseen site, the proposed model
achieves excellent sum-rate performance while maintaining low computational complexity by avoiding matrix inversions and by using a simpler neural network structure.
The model is trained and tested on a custom ray-tracing dataset composed of several base station locations. The experimental results indicate that our method effectively balances computational efficiency with high sum-rate performance while showcasing strong generalization performance in unseen environments.
Furthermore, with fine-tuning, the proposed model outperforms WMMSE across all tested sites and SNR conditions while reducing complexity by at least 73$\times$.
\end{abstract}

% \begin{IEEEkeywords}
% \end{IEEEkeywords}
% \IEEEpeerreviewmaketitle

\section{Introduction} \label{Sec:Intro}
\section{Introduction}
\label{sec:intro}
% Image editing methods in diffusion models depend on user-defined control directions - users can unlock their creativity using these methods by specifying the desired manipulation through prompts~\cite{gandikota2023concept}, reference images~\cite{ruiz2022dreambooth, kumari2022customdiffusion, gal2022image, chen2024trainingfreeregionalpromptingdiffusion}, or attribute vectors~\cite{parmar2023zero,hertz2022prompt}. In this work, we ask a fundamentally different question: \emph{Can we automatically discover the underlying visual structure of a concept within diffusion model's knowledge?} %Rather than requiring user-specified controls, we aim to decompose the model's internal knowledge into meaningful directions.

% This question touches on a fundamental limitation in how we interact with diffusion models. Current control methods ~\cite{zhang2023addingconditionalcontroltexttoimage, gandikota2023concept, ye2023ipadaptertextcompatibleimage,ye2023ipadaptertextcompatibleimage, hertz2024stylealignedimagegeneration, li2023photomaker, shi2024instantbooth, chen2024trainingfreeregionalpromptingdiffusion} require users to specify their desired manipulations in advance, limiting interactive creativity. This contrasts with natural human artistic workflows, where creators dynamically explore creative ideas while jointly refining them toward meaningful artistic outcomes~\cite{hoffmann2016modeling}. This synergy between specification and exploration is not new to generative models. Early GAN architectures naturally developed disentangled latent spaces that enabled continuous\cite{harkonen2020ganspace,radford2015unsupervised, wu2021stylespace, shen2020interfacegan}, compositional control over generated images. Users could explore these spaces to discover interesting variations that would be difficult to describe in words~\cite{wu2021stylespace}, then combine them to achieve their creative goals~\cite{grabe2022towards}. 


% While diffusion models have largely superseded GANs in conditional image synthesis~\cite{dhariwal2021diffusion},  their underlying structure remains less understood. Diffusion models achieve remarkable diversity through high-dimensional latents, unlike GANs' compact latent spaces.  With a single prompt, diffusion models can generate radically different variations through different random initializations of input noise. We ask - Is it possible to discover interpretable structure within this vast space of variations?

Text-to-image diffusion models are capable of generating remarkable visual variations from a single prompt through different random initializations. However, this vast creative potential remains largely opaque to users---while we can generate diverse images, we lack understanding of the underlying structure of these variations. This presents a fundamental challenge: how can we discover and expose the latent visual capabilities encoded within these models?

\let\thefootnote\relax \footnote{$^{*}$Correspondence to \texttt{gandikota.ro@northeastern.edu}}

The challenge touches on a key limitation in how we interact with diffusion models today. Current control methods require users to explicitly specify their desired edits in advance through prompts~\cite{gandikota2023concept}, reference images~\cite{zhang2023addingconditionalcontroltexttoimage, chen2024trainingfreeregionalpromptingdiffusion, ruiz2022dreambooth,kumari2022customdiffusion, Ryu_lora, hu2021lora}, or attribute vectors~\cite{ye2023ipadaptertextcompatibleimage, hertz2024stylealignedimagegeneration, li2023photomaker, shi2024instantbooth,parmar2023zero,hertz2022prompt}. That contrasts sharply with natural human creative workflows, where artists dynamically explore creative ideas and jointly refine them toward meaningful artistic outcomes~\cite{hoffmann2016modeling}. The need for pre-specified controls creates a barrier between users and the full creative potential of these models.

Interestingly, earlier generative models like GANs~\cite{gans,karras2019style,brock2018large} naturally developed more interpretable internal structures. Their compact latent spaces often exhibited emergent disentanglement~\cite{harkonen2020ganspace,radford2015unsupervised, wu2021stylespace, shen2020interfacegan}, enabling continuous and compositional control over generated images. Users could explore these spaces to discover interesting variations that would be difficult to describe in words~\cite{wu2021stylespace}, then combine them to achieve their creative goals~\cite{grabe2022towards}.

Diffusion models have largely superseded GANs in conditional image synthesis~\cite{dhariwal2021diffusion}, achieving greater diversity through much higher-dimensional latents. And yet an understanding of the underlying structure of these larger latent spaces has remained elusive. In this work, we ask a fundamental question: \emph{Can we automatically discover the visual structure within a diffusion model's knowledge of a concept?} Rather than requiring user-specified controls, we aim to decompose the model's internal representations into expressive directions that users can explore and combine.

To address these needs, we present \textbf{SliderSpace}, a framework that brings systematic explorability to diffusion models. Given just a text prompt, SliderSpace discovers a canonical set of meaningful, diverse, and controllable directions within the model's knowledge of that concept. Each direction is implemented as a low-rank adapter~\cite{hu2021lora} that can be scaled and composed with others, allowing users to explore and smoothly combine different aspects of variation, as shown in Figure~\ref{fig:intro}.

We ground SliderSpace discovery in three key requirements for meaningful decomposition of a diffusion model's visual manifold: 
\begin{enumerate}
    \item \textbf{Unsupervised Discovery:} The decomposition process should emerge from the intrinsic structure of the model's learned representation, rather than being guided by predefined attributes. This ensures we capture the true topology of the model's knowledge space rather than projecting our assumptions onto it.
    
    \item \textbf{Semantic Orthogonality:} Each discovered control must represent a distinct semantic direction. This is enforced in a semantic feature space, like CLIP, where every slider has an orthogonal effect in embeddings. This prevents discovering multiple controls that create similar semantic effects, making the system more efficient and easier.
    
    \item \textbf{Distribution Consistency:} Directions must induce consistent transformations across both random seeds and prompt variations. 
\end{enumerate}

These requirements naturally lead to our proposed framework, which we formalize in Section~\ref{sec:method}. As we show in our experiments, SliderSpace is architecture-agnostic, working with both conventional U-Net based models like Stable Diffusion~\cite{rombach2022high, rombach2022sd20, podell2023sdxl, turbo, dmd} and recent transformer-based architectures like Flux~\cite{flux}.

We demonstrate the expressiveness of SliderSpace through three applications: First, we show how SliderSpace can decompose high-level concepts into diverse and expressive components, revealing the natural axes of variation in the model's understanding. Second, we explore artistic style variation, where SliderSpace discovers directions that match or exceed the diversity of manually curated artist lists while being judged more useful by human evaluators. Finally, we show how SliderSpace can help reverse the mode collapse commonly observed in distilled diffusion models, restoring diversity while maintaining generation speed.

Beyond providing practical creative control, SliderSpace opens new avenues for understanding and utilizing the latent capabilities of diffusion models. By mapping these models' visual potential into intuitive, composable directions, we take a step toward making their creative possibilities more accessible and interpretable to users.

% Image editing methods in diffusion models unlock the creativity of users. In this work we ask an alternate question: \emph{Can we organize and expose what of the diffusion model is already capable of?}.
% Existing methods for controlling image generation typically require users to manually specify edit directions for desired changes. This process is time-consuming, requires technical expertise, and limits the spontaneity of the creative process. For instance, if a user wants to adjust the smile of a generated person, they must explicitly request this edit, often through imprecise prompt engineering or model fine-tuning. This approach of predefined controls or manual specifications restricts users from fully exploring the latent capabilities of the model. There may be interesting stylistic variations or attributes that the model can generate, but users have no easy way to discover or utilize these.

% Natural visual disentanglement was an emergent property in the latent space of Generative Adversarial Models (GANs) \cite{harkonen2020ganspace,radford2015unsupervised, wu2021stylespace, shen2020interfacegan}. In particular, it has been observed that StyleGAN~\cite{karras2019style} stylespace neurons offer detailed control over many meaningful aspects of images that would be difficult to describe in words~\cite{wu2021stylespace}. However, diffusion models do not share such a compact latent space~\cite{park2023unsupervised}; and efforts to uncover such a space in the semantic embeddings of the text conditioning have met with limited success \nik{Nick - is there a specific citation you were thinking about?}.

% In this work we introduce \textbf{SliderSpace}, which takes a step towards uncovering an analogous low dimensional representation of diffusion models' visual breadth; in essence treating the diffusion model as many generators sharing parameters, where a particular generator is defined by a specific prompt. For a given prompt we sample many random seeds (and optionally prompt expansions using an LLM), generate the corresponding images, and apply an off the shelf feature extractor (in this work CLIP, but our method can be applied to any differentiable feature extractor). We use PCA to analyze these features, and for each of the leading $k$ principal components we train a LoRA \cite{} which causes the diffusion model to produces images which increase the feature magnitude along that component when passed back through the same feature extractor. This leads to a 'Slider' for each principal component, because each LoRA can be scaled and applied to the original diffusion model, continuously varying those visual features in the generated results (as measured, in our case, by CLIP).

% There are many other works that enhance the controllability of diffusion models. One common approach is enabling users to add spatial constraints to a generation either manually, or via a reference image \cite{zhang2023addingconditionalcontroltexttoimage, chen2024trainingfreeregionalpromptingdiffusion}, a second is leveraging more abstract embeddings (e.g. identity, style) extracted from a reference image \cite{ye2023ipadaptertextcompatibleimage, hertz2024stylealignedimagegeneration, li2023photomaker, shi2024instantbooth}, a third is finetuning a foundation model to better generate a concept important to the user \cite{ruiz2022dreambooth, kumari2022customdiffusion, Ryu_lora, hu2021lora}, and a fourth (most relevant to this work) is finding low-rank adaptors of the model based on a prompt or small training set which can be scaled to provide continous control over one aspect of generated image (e.g. night vs day, basic vs luxury, etc.) \cite{gandikota2023concept}. SliderSpace is complementary to all of these methods and offers something distinct. All of the other methods we are aware require the user (and / or model designer) to know in advance what type of control they want. In contrast SliderSpace assists users in discovering and controlling hidden capabilities present in the diffusion model's distribution of possible generations.

%We propose that truly intuitive creative control in a text-to-image model should meet three key criteria: \emph{discoverability}, \emph{intuitiveness}, and \emph{specificity}. The model should reveal controllable attributes that may not be immediately obvious, offer controls that are easy to understand and manipulate, and ensure each control affects a distinct attribute of the generated image.

% We demonstrate the utility and power of SliderSpace using three applications built on top of SDXL-DMD \cite{dmd}, because its fast generation speed lends itself well to the continuous control offered by SliderSpace.

% First, we study concept decomposition (Section \ref{sec:concept_exp}), where we learn sliders for a specific concept (e.g. 'monster', 'waterfall', 'car'). Through quantitative metrics of diversity and text alignment we demonstrate that the learned sliders dramatically boost the diversity of generations when randomly applied without harming text alignment; we also ask humans to qualitatively judge these results in a user study where they find the SliderSpace results to be more 'Diverse', 'Useful', and 'Creative' than our baselines.

% Second, we attempt to compare the automatic discoveries of SliderSpace to a large scale manual study of artistic styles (Section \ref{sec:art_exp}), open-sourced by ParrotZone \cite{parrotzone}. In this study SDXL was prompted with over 4300 artist names,  and based on visual inspection the cases of successful stylistic mimicry recorded. Quantitatively SliderSpace more closely matches the distribution of artistic variation discovered by ParrotZone than other baselines, and in our user studies was judged to be significantly more 'Diverse' and 'Useful' than the baselines. To our surprise humans even judged SliderSpace results to be slightly more 'Diverse' than the results generated by the manually discovered artist names of \cite{parrotzone}.

% Third, we attempt to use SliderSpace to reverse the mode collapse commonly observed in distilled few-step diffusion models relative to the original teacher model (Section \ref{sec:diverse_exp}). We quantitatively demonstrate that applying SliderSpace to SDXL-DMD leads to more closely matching the distribution of images by the original teacher, SDXL.

%Through extensive experiments on various state-of-the-art text-to-image models, we demonstrate that SliderSpace significantly enhances user control and creative expression in AI-assisted image generation tasks. Our method enables a range of applications, including concept decomposition and control, diversity improvement in generated images, customization dissection and edits, and the exploration of artistic styles inherent in the model.

% SliderSpace goes beyond providing a practical tool for enhanced creative control. By mapping the visual potential of diffusion models it can open new avenues for generative creativity and deepens our understanding of each model's hidden potential.

\section{Problem Formulation} \label{Sec:Problem_Formulation}
\subsection{System Model}
In this paper, we study a time division duplex multi-user \gls{mMIMO} system, where uplink channel estimates can be used to calculate the downlink precoder. This \gls{mMIMO} system has a \gls{BS} that is equipped with $N_{\sf{T}}$ antennas. The system is designed to concurrently support $N_{\sf{U}}$ users, each utilizing a single antenna. This configuration enables efficient communication by leveraging the multiple antennas at the \gls{BS} to enhance signal quality and increase capacity, ultimately allowing for simultaneous transmissions to multiple users. 

The received signal at the $k^{th}$ user can be expressed as
\begin{equation}\label{eq:signal_recived}
    \mathbf{y}_k =  \mathbf{h}_{k}^{\dagger} \sum_{\forall k}  \mathbf{w}_{k} x_k + \bs{\eta}_k \, ,
\end{equation}
where the wireless channel vector between the base station (\gls{BS}) and the $k^{th}$ user is represented by $\mathbf{h}_{k} \in \mathbb{C}^{N_{\sf{T}} \times 1}$. Let $x_k$
denote an independent transmitted complex symbol for a user. The term $\bs{\eta}_k \sim \mathcal{CN}(0, \sigma^2)$ indicates complex \gls{AWGN} characterized by a zero mean and a variance of $\sigma^2$. The downlink \gls{FDP} vector is defined as $\mathbf{W} = \left[ \mathbf{w}_1, \ldots, \mathbf{w}_k, \ldots, \mathbf{w}_{N_{\sf{U}}} \right] \in \mathbb{C}^{N_{\sf{T}} \times N_{\sf{U}}}$.
The associated \gls{SINR} user $k$ is given by
\begin{equation}
    \text{SINR}(\mb{w}_{k}) = \frac{ \big|\mb{h}^{\dagger}_{k} \mb{w}_{k} \big|^2}{\sum_{j \neq k} \big|\mb{h}^{\dagger}_{k} \mb{w}_{j} \big|^2 + \sigma^2}\,,
\end{equation}
and the sum rate for all users is
\begin{equation}\label{eq:sum-rate}
    R(\mb{W}) = \sum_{\forall k}  \text{log}_2 \Bigl(  1+ \text{SINR}(\mb{w}_{k}) \Bigr).
\end{equation}
The objective is to determine a precoding matrix $\mathbf{W}$ that maximizes the throughput in eq. (\ref{eq:sum-rate}) while adhering to the maximum transmit power constraint, $P_{\text{max}}$. Consequently, the downlink sum-rate maximization problem can be formulated as
\begin{equation}\label{eq:maximization}
\begin{aligned}
    & \underset{\mb{W}}{\max}~ R(\mb{W}) , \\
    \text{s.t.}  &\sum_{\forall k} \mb{w}_{k}^{\dagger}  \mb{w}_{k} \leq P_{\sf{max}} \, .
\end{aligned}
\end{equation}

\subsection{Weighted Minimum Mean Square Error Algorithm}
The sum-rate maximization problem in (\ref{eq:maximization}) is classified as NP-hard. To find good approximate solutions, the iterative \gls{WMMSE} algorithm \cite{shi2011iteratively} is commonly used, where the problem is transformed to a corresponding problem focused on minimizing the sum of \gls{MSE} under the independence assumption of $x_k$ and $\bs{\eta}_k$. 
Key parameters in this framework include the receiver gain $u_k$ and a positive user weight $v_k$, which are used to obtain the \gls{MSE} covariance matrix. 
The solution is obtained by iteratively solving convex subproblems to generate updates on the receiver gains, user weights, and beamforming matrix.
% The optimization problem involves iteratively updating variables such as the receiver gains, user weights, and beamforming matrices to maximize a redefined objective function while satisfying power constraints. Each variable is optimized sequentially, holding the others constant, ensuring the problem remains convex for individual updates.
% Let $u_k$ represent the receiver gain, while $v_k$ is a positive user weight, hence the \gls{MSE} covariance matrix $e_k$ can be expressed as 
% \begin{align}
% e_k &= \mathop{\mathbb{E}}_{\bs{x},\bs{\eta}} [ (\hat{x_k} - x_k)(\hat{x_k} - x_k)^{\dagger} ]\\
% &= \left( \mathbf{I} - u^{\dagger}_k\mathbf{h}_{k}\mathbf{w}_k\right)\left( \mathbf{I} - u^{\dagger}_k\mathbf{h}_{k}\mathbf{w}_k\right)^{\dagger},\\
% &=  \left| u_{k} \mathbf{h}_{k}^{\dagger} \mathbf{w}_k - 1 \right|^2 +  \sum_{j \neq k}^{N_{\sf{U}}} \left| u_k \mathbf{h}_k^{\dagger} v_j \right|^2  + \sigma^2 \left| u_k \right|^2,
% \end{align}
% thus the optimization problem can be formulated as
% \begin{align}
%     \underset{\mb{u},\mb{v},\mb{W}}{\max}~ & \sum^{N_{\sf{U}}}_{k=1} (v_k e_k - \text{log}_2 v_k) , \\
%    &\text{s.t.}   \sum_{\forall k} \mb{w}_{k}^{\dagger}  \mb{w}_{k} \leq P_{\sf{max}} \, .
% \end{align}
% This optimization problem becomes convex when each variable is considered individually, with the remaining variables held constant. Therefore the \gls{WMMSE} algorithm operates by sequentially updating each variable, 

First $\mathbf{W}^{(0)}$ is initialized while satisfying the constraint in \eqref{eq:maximization},
% in such a way that $\sum_{\forall k} \mb{w}_{k}^{\dagger}  \mb{w}_{k} \leq P_{\sf{max}}$. 
then, at each iteration $i$, variables are updated as defined below until a stopping criterion is satisfied:
\begin{align}
    v^{(\text{i})}_{k} = \frac{ \sum_{j =1}^{N_{\sf{U}}} \big|\mb{h}^{\dagger}_{k} \mb{w}^{(\text{i}-1)}_{j} \big|^2 + \sigma^2}{\sum_{j \neq k}^{N_{\sf{U}}} \big|\mb{h}^{\dagger}_{k} \mb{w}^{(\text{i}-1)}_{j} \big|^2 + \sigma^2},
\end{align}
\begin{align}
    u^{(\text{i})}_{k} = \frac{ \mb{h}^{\dagger}_{k} \mb{w}^{(\text{i}-1)}_{j} }{\sum_{j=1}^{N_{\sf{U}}} \big|\mb{h}^{\dagger}_{k} \mb{w}^{(\text{i}-1)}_{j} \big|^2 + \sigma^2},
\end{align}
\begin{align}\label{eq:construct_beamforming_matrix}
    \mb{w}^{(\text{i}+1)}_{k} =  u^{(\text{i})}_{k} v^{(\text{i})}_{k}\mb{h}_{k}(\sum^{N_{\sf{U}}}_{j=1}   v^{(\text{i})}_{j} |u^{(\text{i})}_{j}|^2 \mb{h}_{j}\mb{h}_{j}^{\dagger} + \mu \mathbf{I})^{-1},
\end{align}
where $\mu \geq 0$ is a Lagrange multiplier.
In addition to the high complexity of solving \gls{WMMSE}, the inherent non-convexity of sum-rate maximization means that WMMSE is only guaranteed to converge to a local optimum, not necessarily the global solution.

\section{Proposed Method} \label{Sec:Proposed}
\section{Proposed Method}

\subsection{Overview}

The overview of the proposed framework is shown in Figure \ref{Fig.model}. We assume a point-based 3D shape with size $P$ as the input. Given the 3D human model and $K$ semantic text prompts, the goal is to parse the 3D human into segments that semantically correspond to the input prompts. 


Inspired by recent achievements in 2D and 3D segmentation methods \cite{cheng2022masked, ding2022decoupling, zhou2022lmseg, schult2023mask3d, takmaz20233d}, we formulate the semantic segmentation task as mask classification, originated from MaskFormer \cite{cheng2021per}. To bridge 3D data with 2D pre-trained models, we render the input from $V$ predefined camera views. Segment-Anything-Model (SAM) \cite{kirillov2023segment} is leveraged to generate mask proposals for each view (Section \ref{mask_proposals}). We introduce the novel HumanCLIP model, which encodes these proposals into embeddings within the unified CLIP feature space (Section \ref{mask_embeddings}). To lift 2D labels into 3D, it is usually required to assign ``super points'' and have carefully designed voting and grouping. In this work, we present a simple MaskFusion module. It takes HumanCLIP encoded text prompts and simultaneously performs classification and multi-view aggregation without the need for complex operations (Section \ref{mask_fusion}). Note that the generation of mask proposals and embeddings is performed just once per model. Subsequently, segmentation can be executed in just a few milliseconds per prompt, significantly enhancing efficiency compared to previous methods.


% Apart from the HumanCLIP, our framework requires no additional training with 3D data. 


% The following sections describe in detail the process of obtaining the 3D masks and corresponding embeddings.

\subsection{Multi-view Mask Proposals}
\label{mask_proposals}
% - Use SAM to generate class-agnostic 2D masks and unproject to 3D.
We choose SAM \cite{kirillov2023segment} to generate mask proposals on multi-view rendered images. SAM demonstrates remarkable zero-shot capabilities in image segmentation. From the predefined camera poses, we render the 3D human into $V$ RGB images $I_i \in \mathbb{R}^{H\times W \times 3}$ where $i \in [1, V]$ is the index of the view. Each image $I_i$ is then independently fed into SAM in the "segment everything" mode to generate $N_i$ class-agnostic overlapping masks: 
\begin{equation}
    [m_{i, 1}^{2D}, ..., m_{i, N_i}^{2D}] = SAM(I_i)
\end{equation}
where $m_{i, j}^{2D}$ is the $j$-th 2D mask generated by SAM from the $i$-th view.
This results in a total of $N=\sum_{i=1}^V N_i$ binary 2D masks at a varying granularity of whole, part, and subpart. 

Each 2D mask $m_{i, j}^{2D}$ is then unprojected to 3D using the camera parameters of view $i$ to construct the corresponding 3D proposal $m_{i, j}^{3D}$.
% The 2D masks are unprojected to point cloud with size $P$. 

\subsection{HumanCLIP Encoding}
\label{mask_embeddings}
We propose HumanCLIP to generate proposal embeddings with size $C$, where $C$ is the embedding dimension of a CLIP model. The unified image-text feature space of CLIP allows the framework to perform open-vocabulary mask classification. Each 2D mask $m_{i, j}^{2D}$ with its corresponding image $I_i$ is fed to the image encoder to get the proposal embedding $q_{i, j} \in \mathbb{R}^{C}$. 

It is well-discussed that the vanilla pre-trained CLIP encoder does not perform well on specialty-formed inputs \cite{liu2023partslip,liang2023open}, including masked and cropped images. Moreover, masking or cropping an image results in the loss of crucial contextual information, which is essential to the understanding of the specific area in an image. Therefore, we adopt the design of AlphaCLIP \cite{sun2024alpha} to build our image encoder. As shown in Figure \ref{Fig.alphaclip}, the encoder accepts an additional alpha channel as input, which highlights the region of interest on the original rendered images. The input mask is processed with a parallel convolution layer to the RGB image and combined to go through a series of attention blocks to produce the final mask embedding in CLIP feature space. To further mitigate the domain gap, we finetune the encoder on a dataset of over 1.3 million RGBA region-text pairs with human-centric contents. We visualize the image-text alignment before and after fine-tuning in Figure \ref{Fig.finetune}. The pre-trained AlphaCLIP model fails to provide well-aligned embeddings for small parts such as the glasses as well as to distinguish left and right parts. The proposed HumanCLIP model generates more discriminative mask embeddings, facilitating the downstream classification tasks. 

\begin{figure}[t]
    \centering
    \includegraphics[width=.9\hsize]{figures/alphaclip.pdf}
    \caption{AlphaCLIP Image Encoder.} 
    \label{Fig.alphaclip}
\end{figure}

\begin{figure}[t]
    \centering
    \includegraphics[width=.98\hsize]{figures/finetune.pdf}
    \caption{Comparison between (a) pre-trained AlphaCLIP and (b) the proposed HumanCLIP. The plots show the cosine similarity between the embedding of the masked region corresponding to \textit{face}, \textit{glasses}, \textit{left shoe}, and \textit{right shoe} and their text embeddings.} 
    \label{Fig.finetune}
\end{figure}

% \begin{table}
%     \centering
%     \caption{Statistics about mask data used to train HumanCLIP}
%     \label{tab:humanclip_data}
%     \resizebox{.98\columnwidth}{!}{
%     \begin{tabular}{c|ccc|c}
%        Dataset & Images & Original Masks & Generated Masks & Total Masks \\ \hline
%        LIP & 30462 & 173578 & 91505 & 265083 \\
%        ATR & 17706 & 175604 & 87698 & 263302 \\
%        DeepFashion & 12701 & 100632 & 43404 & 144036 \\
%        CIHP & 28280 & 647072 & 63616 & 710688 \\ \hline
%        \textbf{HumanCLIP} & \textbf{89149} & \textbf{1096886} & \textbf{286223} & \textbf{1383109} 
%     \end{tabular}
%     }
% \end{table}



\subsection{Region-Text Pair Generation}
\label{sec:regiontext}
To tailor the image encoder for processing human-centric data, we finetune the model with region-text pair data. A straightforward method to acquire this data is utilizing 2D human segmentation datasets, where segmentation maps and category names directly form region-text pairs. Although efficient, this method yields less diverse masks and less informative captions. Therefore, we devise a pipeline to augment the training data. We source images from LIP \cite{gong2017look}, ATR \cite{liang2015deep}, DeepFashion \cite{liu2016deepfashion}, and CIHP \cite{gong2018instance} datasets and employ KOSMOS-2 \cite{peng2023kosmos} and SAM \cite{kirillov2023segment} to automatically generate masks and corresponding captions for these images. An example of the generated pairs is depicted in Figure \ref{Fig.example_masks}. Compared with the original labels, it provides more descriptive captions and introduces novel masks for objects that humans typically interact with, such as as `a stool'. Further details of the data generation process are presented in the supplementary. 

\begin{figure}[t]
    \centering
    \includegraphics[width=.98\hsize]{figures/example_masks.pdf}
    \caption{Example of mask-caption pairs generated by utilizing KOSMOS-2 and SAM.} 
    \label{Fig.example_masks}
\end{figure}

% with KOSMOS-2 \cite{peng2023kosmos} and SAM \cite{kirillov2023segment} to automatically generate additional training data. One benefit of this is that it provides more descriptive captions such as color of a clothing. Another benefit is that it introduces novel masks for objects that humans typically interact with but are not included in the original dataset. 
% An example of the mask-caption pairs automatically genrerated by this pipeline is shown in Figure \ref{Fig.example_masks}. It can be seen that it creates descriptive captions about various garments as well as create masks for objects such as as `a stool'. 
% More details of this pipeline is shown in the supplementary.

% \noindent\textbf{HumanCLIP Dataset.}
% Training our HumanCLIP model requires mask-text pairs for human related concepts. A straightforward way to acquire this data is through 2D human segmentation datasets in which the segmentation maps and category names can be directly used as mask-caption pairs. While efficient, this results in captions that lack variation and only focused on body parts and clothings. Therefore, we devise a pipeline with KOSMOS-2 \cite{peng2023kosmos} and SAM \cite{kirillov2023segment} to automatically generate additional training data. One benefit of this is that it provides more descriptive captions such as color of a clothing. Another benefit is that it introduces novel masks for objects that humans typically interact with but are not included in the original dataset. More details of this pipeline is shown in the supplementary.



% Although the AlphaCLIP model provides more accurate mask-wise embeddings than the original CLIP model, our experiments show that it is still insufficient in providing distinct features for human body parts and garments. This can be observed in Figure \ref{Fig.finetune}. With the vanilla AlphaCLIP, it can fail to provide well-aligned embeddings for smaller parts such as the glasses as well as distinguish left and right parts. By finetuning on 2D human image datasets, the proposed HumanCLIP model generates mask embeddings that better align with the corresponding text.


\subsection{3D Semantic Segmentation}
\label{mask_fusion}
To obtain the segmentation result with the desired semantic labels, our pipeline accepts $K$ text prompts corresponding to the labels per inference. These texts are fed to the HumanCLIP text encoder to obtain CLIP text embeddings $\mathbf{W} \in \mathbb{R}^{K \times C}$. Then, the proposed MaskFusion module semantically classifies and synthesizes multi-view embeddings into 3D segmentation masks. Specifically, we utilize the correlations between the text embeddings and the mask embeddings to build correspondences and fuse the independent 3D masks. 

Recap the $N$ 3D mask proposals generated in Section. \ref{mask_proposals}. The proposals and their embeddings are stacked to get $\mathbf{M} \in \mathbb{R}^{P \times N}$ and $\mathbf{Q} \in \mathbb{R}^{N \times C}$ respectively. We compute the classification logits $\mathbf{P} \in \mathbb{R}^{N \times K}$ by taking the cosine similarity between each mask embedding and each text embedding:
\begin{equation}
    \mathbf{P}_{n, k} = \frac{\mathbf{Q}_n\cdot \mathbf{W}_k}{||\mathbf{Q}_n||||\mathbf{W}_k||}
\end{equation}
It is used to guide the grouping of raw masks, which are class-agnostic and inconsistent across views. 

In the final step, for each 3D point, we aggregate the class scores from the associated masks to get the final 3D segmentation result $\mathbf{Y} \in \mathbb{R}^{P \times K}$.
$Y$ is computed as the simple weighted average of 3D masks $\mathbf{M}$ based on the classification logits $\mathbf{P}$:
\begin{equation}
    \mathbf{Y} = \mathbf{M} \times \mathbf{P}
\end{equation}

We decouple the procedures for mask proposal and for text classification. Therefore, it is not guaranteed that each text input is valid for the 3D model. To ensure that only existed classes are segmented in the final result, we set a threshold $\tau$ on the final segmentation logits. If the maximum logits of a point fall below $\tau$, the point is attributed to an `other' class. 

\begin{figure*}[!th]
    \centering
    \subfigure[Sum rate achieved at an average SNR of 40\,dB.]{\includegraphics[width=\columnwidth]{Result/ICC_histo_Zero_shot_highSNR.pdf}}
    \subfigure[Sum rate achieved at an average SNR of 10\,dB.]{\includegraphics[width=\columnwidth]{Result/ICC_histo_Zero_shot_lowSNR.pdf}}
    \caption{Achievable sum rates for precoding methods under different SNR values at three Montreal sites: ``Ericsson'', ``Decarie'', and ``Sainte-Catherine''. All methods report zero-shot performance, except ``PaPP + FT,'' which is after 20 epochs of fine-tuning.}
    \label{fig:sidebyside}
\end{figure*}
\section{Baselines and Complexity Analysis} \label{Sec:Complexity}
\begin{table}[t]
    \centering
    \caption{Number of real multiplications and sum rate for different methods (``Ericsson'' site, $N_{\sf{T}} = 64$, $N_{\sf{U}} = 4$, SNR = 27dB).}
    \resizebox{\columnwidth}{!}{
    \begin{tabular}{ccc}
        \toprule
        \textbf{Methods} & \textbf{\# of Multiplications} & \textbf{Sum-Rate (b/s/Hz)}\\
        \midrule
        WMMSE \cite{shi2011iteratively} ($I=12.5$)    & 36.1\,M & 37.29
        \\ 
        MAML-CNN (zero-shot) & 3.77\,M & 37.01 \\
        \textbf{PaPP} (zero-shot) & \textbf{1.05\,M} & \textbf{38.10}\\
        Zero Forcing \cite{nayebi2017precoding} & 8.4\,K & 32.14\\
        \bottomrule
    \end{tabular} }
    \label{tab:complexity}
\end{table}
Table \ref{tab:complexity} shows an example of the computational complexity of different precoding methods, measured by the number of real multiplications required for processing when the deployment site is ``Ericsson''. This analysis offers insights into the trade-offs between computational demands and sum-rate performance.

The \gls{WMMSE} algorithm is renowned for achieving good weighted sum rate in mMIMO systems. However, this performance comes at the cost of high computational complexity. Specifically, with stopping criteria of $10^{-3}$ and an average of 12.5 iterations for this setup, the WMMSE method requires approximately 36 million multiplications. This complexity can limit real-time applications or systems constrained by computational resources. The total number of real multiplications for the WMMSE method is given by
\[
 4I \bigg(\frac{2}{3} N_{\sf{T}}^3N_{\sf{U}} + N_{\sf{T}}^2N_{\sf{U}} + 2N_{\sf{T}}(2N_{\sf{U}}^2+N_{\sf{U}})+ N_{\sf{U}}^2+\frac{14}{3}N_{\sf{U}}\bigg) \, ,
\]
where $I$ is the total number of iterations. 

As another baseline, we consider the method proposed in \cite{lyu2023downlink}, which combines a multilayer perceptron (MLP) model with the WMMSE algorithm. However, we replace the original MLP model with a CNN model similar to the approach in \cite{hojatian2021unsupervised}.
This modification was made to better capture spatial features in the \gls{CSI}, enabling the model to handle the increased complexity introduced by a larger number of antennas and users. By leveraging the representational power of CNNs, MAML-CNN achieves more competitive results in our experimental settings compared to the original MAML-MLP. Since it combines a \gls{DNN} with an additional matrix inversion step, it exhibits significantly higher complexity than \gls{ZF} but remains less demanding than WMMSE. 
The number of real multiplications required by MAML-CNN is given by
\begin{align*}
 & \
C_{\text{out}}N_{\sf{T}}N_{\sf{U}}C_{\text{in}}k^2 + C_{\text{out}}N_{\sf{T}}N_{\sf{U}}(3N_{\sf{U}}+1) \\
  & + 8\bigg(\frac{4}{3} N_{\sf{T}}^3 + N_{\sf{T}}^2 (3N_{\sf{U}}+2) + N_{\sf{T}}(2N_{\sf{U}}+3)\bigg) \, ,
\end{align*}
where $C_{\text{in}}$ and  $C_{\text{out}}$ are the input and output channels of the CNN layer, and $k$  is the kernel size. 

The proposed PaPP method leverages a DNN to provide a zero-shot precoding solution. While it exhibits greater computational complexity compared to traditional methods like ZF due to its convolutional and fully connected layers, this approach offers significant performance benefits, including improved interference mitigation and adaptability to unseen sites. The total number of real multiplications required for the PaPP method can be expressed as
\begin{align*}
 & \ C_{\text{out}}N_{\sf{T}}N_{\sf{U}}C_{\text{in}}k^2 + C_{\text{out}}N_{\sf{T}}N_{\sf{U}}D_{\text{FC1}} \\
& +D_{\text{FC1}}D_{\text{FC2}} + D_{\text{FC2}}D_{\text{FC3}} + D_{\text{FC3}}D_{\text{FC4r}} + D_{\text{FC3}}D_{\text{FC4i}}\, ,
\end{align*}
where $D_{\text{FCi}}$ ($i=1,2,3,4$) represent the sizes of the \glspl{FCL}. 

The \gls{ZF} \cite{nayebi2017precoding} precoding method offers significantly lower computational complexity, requiring approximately 8.4 thousand multiplications. This efficiency stems from its reliance on simpler linear algebra operations, specifically the inversion of smaller matrices when $N_{\sf{T}}>N_{\sf{U}}$. Despite its computational efficiency, ZF is susceptible to performance degradation in high-interference scenarios or under adverse channel conditions. The total number of real multiplications required by ZF is
\[
8 N_{\sf{U}}^2 N_{\sf{T}} + \frac{8}{3} N_{\sf{U}}^3 \, .
\]


\section{Numerical Results} 
\label{Sec:Simulation}
\section{Results}
\label{sec:results}
Following \nksr, we evaluate our method using metrics including the standard Chamfer-$L_1$ Distance~(CD-$L_1 \times 10^{-2}$, $\downarrow$) and F-score~($\uparrow$) with a threshold~($\delta{=}0.010$). 
We also report additional metrics proposed in \nksr~including Chamfer-$L_1$ Distance by Completeness (Comp.~$\times 10^{-2}$, $\downarrow$) and Accuracy (Acc.~$\times 10^{-2}$, $\downarrow$) in the \texttt{Supplementary Material}. 
We evaluate our method on multiple datasets, under two settings including in-domain evaluation for accuracy estimation -- training set and test set are from same dataset, and cross-domain evaluation for generalization ability estimation where training set and test set are from different datasets. 
Additionally, for cross-domain evaluation we use the following datasets prepared by the leading voxel-based baseline, \nksr, and one additional dataset from RangeUDF~\cite{wang2022rangeudf}:

\begin{itemize}
    \item \synthetic{}  is a synthetic dataset created from ShapeNet objects~\cite{chang2015shapenet}. Each scene contains 2-3 objects. 
    Following prior works~\cite{wang2022rangeudf,chibane2020ndf}, we re-scale the synthetic rooms to roughly match real-world scale.
    There are 3750 scenes as training set and \ws{995 scenes} as the test set. 
    \item \scannet{} is a real-world indoor scene dataset. We use the setting from previous work~\cite{wang2022rangeudf, tang2021SACon, peng2020convoccnet, boulch2022poco} where we train on 1201 rooms and test on 312 rooms. 
    \item \carla is a large-scale outdoor driving scene prepared by NKSR~\cite{huang2023neural} using the CARLA simulator~\cite{dosovitskiy2017carla}. 
    \ws{Following NSKR~\cite{huang2023neural}, we test on two subsets including the 'Original' subset (10 random drives simulated on 3 towns) and the 'Novel' subset (3 drives from an additional town only for testing).}
    To avoid exploding GPU memory during training, we follow NKSR~\cite{huang2023neural} to divide a large scene into patches. The resultant training set has {3757} patches. 
    \item \scenenn{}  is a real-world indoor dataset prepared by RangeUDF~\cite{wang2022rangeudf} which we used for cross-domain evaluation. We only use its test set which consists of 20 scenes.
\end{itemize}



\begin{table*}
\centering
\resizebox{\linewidth}{!}{
\setlength{\tabcolsep}{3pt}
\begin{tabular}{LccccccccccccC}
\toprule
Methods & & \multicolumn{3}{c}{\ws{{\bf \synthetic}}}  &  \multicolumn{3}{c}{{\bf \scannet}} & \multicolumn{3}{c}{\ws{{\bf \carla(Original)}}} & \multicolumn{3}{c}{\ws{{\bf \carla(Novel)}}} \\ 
 \cmidrule(lr){3-5} \cmidrule(lr){6-8} \cmidrule(lr){9-11} \cmidrule(lr){12-14} 
&Primitive& CD ($10^{-2}$) $\downarrow$ & F-Score  $\uparrow$ & Latency (s) $\downarrow$  & CD ($10^{-2}$) $\downarrow$ & F-Score  $\uparrow$ & Latency (s) $\downarrow$  & CD (cm) $\downarrow$ & F-Score  $\uparrow$ & Latency (s) $\downarrow$ & CD (cm) $\downarrow$ & F-Score  $\uparrow$ & Latency (s) $\downarrow$ \\        
\midrule
SA-CONet~\cite{tang2021SACon} & Voxels & {0.496} & {93.60} & - & - & - & - & - & - & - & - & - & -\\
ConvOcc~\cite{peng2020convoccnet} & Voxels & {0.420} & {96.40} & - & - & - & - & - & - & - & - & - & -\\
NDF~\cite{chibane2020ndf} & Voxels & {0.408} & {95.20} & - & 0.385  & 96.40  & -  & - & - & - & - & - & -\\
RangeUDF~\cite{wang2022rangeudf} & Voxels & {0.348} & {97.80} & {-} & 0.286 & 98.80 & - & - & - & - & - & - & -\\
\ws{TSDF-Fusion~\cite{zeng20163dmatch}} & -  & - & - & - & - & - & - & 8.1 & 80.2 & - & 7.6 & 80.7 & - \\
\ws{POCO~\cite{boulch2022poco}} & - & - & - & - & - & - & - & 7.0 & 90.1 & - & 12.0 & 92.4 & - \\
\ws{SPSR~\cite{kazhdan2013screened}} & - & - & - & - & - & - & - & 13.3 & 86.5 & - & 11.3 & 88.3 & - \\
\nksr & Voxels &  \underline{0.346} &  \underline{97.41} & \underline{0.40} & \underline{0.246} & \underline{99.51} & \underline{1.54} &  \underline{3.9} &  \underline{93.9} &  \underline{2.0} &  \underline{2.9} &  \underline{96.0} &  \underline{1.8} \\
\nksr (more data) & Voxels & - & - & - & - & - & - & {3.6} & {94.0} & {2.0} & {3.0} & {96.0} & {1.8}\\
Ours~(Minkowski)~\cite{choy20194d} \scriptsize{(w/ KNN)} & Voxels & - & \todo{} & \todo{} & 0.254 & 99.41 & 0.46 & 3.4 & 97.2 &1.9 & 2.7 & 98.1 & 2.0 \\
Ours~(Minkowski)~\cite{choy20194d} & Voxels & - & \todo{} & \todo{} & 0.301 & 98.48 & 0.31 & 3.8 & 96.2 & 1.5 & 3.0 & 97.4 & 1.5\\
\rowcolor{1st} Ours \scriptsize{(w/ KNN)} & Points &{0.321} & {98.34} & {0.13} & {0.243} & {99.61} & {0.48} &{3.2} & {97.5} & {3.2} &{2.6} & {98.3} & {3.4}\\
\rowcolor{1st}Ours & Points & {0.360} & {96.32} & 0.14 & 0.257 & 99.33 & 0.49 & {3.3} & {97.4} & 1.7 & {2.7} & {98.2} & 1.7 \\

\bottomrule
\end{tabular}
}
\caption{\textbf{In-domain evaluation} -- We show that our method achieves the best accuracy (CD and F-score) with significantly improved time efficiency~(inference latency).
Note we retrain \nksr (numbers are underlined) for fairer comparison, \ws{as the training data for \nksr is different from ours -- i.e., they reported some models trained on a ``mix'' of datasets, which is impossible to reproduce.
}
}
\label{tab:indomain}
\end{table*}


\paragraph{Evaluation pipeline}
To evaluate our method, we first extract the mesh with Dual Marching Cubes~\cite{schaefer2004dual} on the predicted SDF, and then compute the CD and F-score between 100k points sampled on the mesh, and 100k points sampled from the ground-truth dense point cloud.
We use the same approach as \nksr to prepare the input point clouds for training and evaluation from the ground-truth dense point clouds through downsampling.
Specifically, for indoor datasets (i.e., \synthetic, 
\scannet and \scenenn), we uniformly sample 10K points sampled from the ground truth dense point cloud. 
For outdoor driving scenes~(i.e., \carla), we follow the evaluation pipeline from \nksr.
We sample sparse input point clouds with a sparse 32-beam LiDAR with a ray distance noise of 0-5 cm and pose noise of $0-3^\circ$, and obtain the ground truth from a noise-free dense 256-beam LiDAR.

\begin{figure*}
\centering
\includegraphics[width=\linewidth]{visualizations/test_set_results.pdf}
\caption{
{\textbf{Qualitative results on \carla and \synthetic}} -- our method achieves high quality surface reconstructions which preserve more details than \nksr~which loses information due to quantization for large and non-uniformly sampled datasets like Carla.
}
\label{fig:qual_results_carla_syn}
\end{figure*}
 
\begin{figure*}
\centering
\vspace{-1em}
\includegraphics[width=.95\linewidth]{visualizations/scannet_results_0.pdf}
\caption{
Qualitative results on \scannet: We compare our method with prior SOTA~\cite{huang2023neural} and Ours~(Minkowski)~\cite{choy20194d} that is more comparable as it only differs from ours in the backbone. Our method achieves reconstruction of similar quality to the SOTA. It also \textit{significantly} outperforms Ours~(Minkowski), highlighting the importance of point-based methods. 
% \TODO{callouts too small? almost no zoom? why?}
}
\vspace{-1em}
\label{fig:scannet_results}
\end{figure*}
  

\paragraph{Implementation details}
We base our feature backbone on PointTransformerV3~\cite{wu2024point} with 4-levels.
The PointNet-style network is a 2-layered residual connection MLP, with hidden dimension of $32$ and output feature dimension of $32$.    
The grid size used in neighborhood function is $0.01$ meters.
Following \nksr, we use the similar coefficients for loss terms -- i.e., $\lambda_{\text{SDF}}$ is $300$ and $\lambda_{\text{mask}}$ is $150$.
However, we empirically set $\lambda_{\text{Eikonal}}$ to $10$~(\nksr does not need this regularizer thanks to its specialized surface solver).
We train our model with a batch size of $4$ on either a single \texttt{NVIDIA RTX A6000 ADA} or an \texttt{NVIDIA L40S}, and a learning rate of $10^{-3}$.
We adopt the Adam optimizer with default parameters.
We set the maximum number of epochs to 200 and employ a cosine learning rate decay starting from epoch 120.


\begin{table*}
\centering
\resizebox{\linewidth}{!}{
\setlength{\tabcolsep}{2pt}
\begin{tabular}{LccccccccccC}
\toprule
Methods & & \multicolumn{3}{c}{{\bf \synthetic $\rightarrow$ \scannet}}  &  \multicolumn{3}{c}{{{\bf \scannet $\rightarrow$ \synthetic}}} & \multicolumn{3}{c}{{{\bf \scannet $\rightarrow$ \scenenn}}} \\ 
 \cmidrule(lr){3-5} \cmidrule(lr){6-8} \cmidrule(lr){9-11}
&Primitive& CD ($10^{-2}$) $\downarrow$ & F-Score  $\uparrow$ & {Latency (s) $\downarrow$ } & CD ($10^{-2}$) $\downarrow$ & F-Score  $\uparrow$ & {Latency (s) $\downarrow$ } & CD ($10^{-2}$) $\downarrow$ & F-Score  $\uparrow$ & {Latency (s) }$\downarrow$ \\       
\midrule
SA-CONet~\cite{tang2021SACon} & Voxels & 0.845 & 77.80 & - & - & - & - & - & - & - \\
ConvOcc~\cite{peng2020convoccnet} & Voxels & 0.776 & 83.30  & - & - & - & - & - & - & - \\
NDF~\cite{chibane2020ndf} & Voxels & 0.452 & 96.00 & - & {0.568} & {88.10} & - & 0.425 & 94.80 & - \\
RangeUDF~\cite{wang2022rangeudf} & Voxels & {0.303} & {98.60} & {-} & 0.481& 91.50 & - & 0.324 & 97.80 & - \\
\nksr & Voxels & {0.329} & {97.37} & {2.02} & {0.351} & {97.41} & {0.46} & {0.268} & {99.18} & {1.95} \\
\rowcolor{1st} Ours (w/ KNN) & Points & {0.284} & {98.65} & {0.54} & {0.327} &{98.37} & {0.13} & {0.277} & {99.00} & {0.50} \\
\bottomrule
\end{tabular}
}
\caption{\textbf{Cross-domain evaluation} -- we achieve the best generalization ability in two cases with much better time efficiency. In the other case where we generalize from \scannet to \scenenn, we achieve accuracy on par with the SOTA baseline~\cite{huang2023neural} with less than a half of their latency.  
}
\vspace{-1.4em}
\label{tab:across_domain}
\end{table*}


\paragraph{Reconstruction latency}
For both our models and NKSR, we record the reconstruction latency for all indoor scenes on a single \texttt{NVIDIA RTX 3090}, and for large outdoor scenes on a single \texttt{NVIDIA L40s} given that more GPU memory is required.
We omit data loading time, and only record the average forward pass time. 

\subsection{In-domain evaluation}
We compare against \nksr~(the current state-of-the-art), RangeUDF~\cite{wang2022rangeudf},  SPSR~\cite{kazhdan2013screened}, NDF~\cite{chibane2020ndf}, ConvOcc~\cite{peng2020convoccnet} and SA-CONet~\cite{tang2021SACon}.     
We further include a baseline that replaces our backbone with MinkowskiNet~\cite{choy20194d} (i.e., Ours~(Minkowski)) to show the degraded performance due to the information loss caused by voxelization.

\paragraph{Quantitative results -- \Cref{tab:indomain}}
Across indoor and outdoor datasets, our method outperforms baselines in terms of accuracy and time efficiency. Especially in outdoor datasets, our method achieves the best surface reconstruction with the smallest latency -- nearly \textit{half} of the second best's latency.
In indoor datasets, which have relatively uniform sampling patterns, we achieve accuracy on par with the previous state-of-the-art, but with significantly improved time efficiency.
Note that we achieve this advantage even with KNN because, in smaller indoor point clouds, the highly engineered KNN implementation has similar time efficiency to that of our neighborhood function.
We further detail our analysis on this matter in the \texttt{Supplementary Material}. 
We also note that our approximate neighborhood function is still effective, as it outperforms the directly comparable baseline MinkowskiNet~\cite{choy20194d}, which shares the same structure except for the backbone and neighborhood function.


\paragraph{Qualitative results -- \Cref{fig:qual_results_carla_syn,fig:scannet_results}}
We show that our method tends to reconstruct surfaces of the best quality among the compared methods.
Especially, on the non-uniform large scale \carla, our method tends to preserve more details than the previous state-of-the-art~\cite{huang2023neural}, which voxelizes the point cloud.   

\subsection{Cross-domain evaluation -- \Cref{tab:across_domain}}
We further test the generalization ability of our method with a cross-domain evaluation.
We evaluate models trained with dataset A on other a different dataset B; we denote this as~A $\rightarrow$ B. 
As shown in \Cref{tab:across_domain}, there are three cases in total.
In two cases (i.e., \synthetic $\rightarrow$ \scannet and \scannet $\rightarrow$ \synthetic), our method achieves the best accuracy with the best time efficiency. 
In another case (\scannet $\rightarrow$ \scenenn), we achieve accuracy on par with SOTA~\cite{huang2023neural} with a much better time efficiency, i.e., less than a half of the latency required by the SOTA~\cite{huang2023neural}.

\subsection{Ablation studies}
Our ablations are executed on \scannet, as it is a real-world dataset, and is equipped with precise ground truth surface meshes.

\begin{table}
\centering
\resizebox{.9\columnwidth}{!}{
\begin{tabular}{LccccccC}
\toprule
{\bf Neighbor Num.} & {CD (10\textsuperscript{-2})} $\downarrow$ & {F-score} $\uparrow$ & Latency (s) $\downarrow$ \\ \midrule
 2 & 0.246 & 99.56 & 109 \\
 4 & 0.244 & 99.59 & 127 \\
 \rowcolor{1st} 
8 & {0.243} & 99.61 & 151 \\
16 & 0.256 & 99.28 & 187 \\
\bottomrule
\end{tabular}
}
\caption{{\bf The impact of neighborhood size} -- larger neighborhoods lead to increased computational cost, and we find that 8 neighbors gives the best balance of cost and quality.}
\label{tab:numpts_neighbor}
\vspace{-1em}
\end{table}

\paragraph{Impact of neighborhood size -- \Cref{tab:numpts_neighbor}}
We analyze the impact of neighborhood size on performance. Larger neighborhood size leads to increased computation overhead. 
We show that the 8-nearest neighboring points gives the best trade-off between accuracy and time efficiency.
Considering a large number (e.g., 16) of neighboring points degrades performance as the the aggregation module has limited capacity to predict the precise SDF from a large local point cloud.

\begin{table}
\centering
\resizebox{.95\columnwidth}{!}{
\begin{tabular}{@{}lcccccc@{}}
\toprule
\makecell{\bf Num. of hidden\\\bf layers in $\aggregation$} & CD (10\textsuperscript{-2}) $\downarrow$ & F-score $\uparrow$ & Latency (s) $\downarrow$ \\ \midrule
 2 & 0.257 & 99.33 & 152 \\
 4 & 0.256 & 99.32 & 166 \\
\bottomrule
\end{tabular}
}
\caption{{\bf Impact of capacity of $\aggregation$} -- we find that increasing the number of layers in $\aggregation$ beyond 2 decreases time efficiency without substantially improving the reconstruction quality.}
\label{tab:agg_capacity}
\vspace{-1em}
\end{table}

\paragraph{Impact of capacity of $\aggregation$ -- \Cref{tab:agg_capacity}} 
We report how the capacity of the aggregation module $\aggregation$ (i.e., different number of hidden layers) impacts the performance.
We observe that aggregation modules of higher capacity give better performance but degraded time efficiency. However, as shown in~\Cref{tab:agg_capacity}, a very large capacity (4 layers) for $\aggregation$ does not help.
We show that we we use 2 layers to have a good trade-off between accuracy and time efficiency. 
We supplement~\Cref{tab:agg_capacity} with an analysis across even more levels in the \texttt{Supplementary Material}.

\begin{table}
\centering
\resizebox{.9\columnwidth}{!}{
\begin{tabular}{@{}lcccc@{}}
\toprule
\textbf{Num. of scales} &KNN & Minkowski & Z-order & Hilbert  \\ \midrule
0 & 1.00 & 0.17 & 0.44  & \cellcolor{1st}0.46  \\
1 & 1.00 & 0.29 & 0.48  & \cellcolor{1st}0.50  \\
2 & 1.00 & 0.38 & 0.49  & \cellcolor{1st}0.52  \\
3 & 1.00 & 0.44 & 0.49  & \cellcolor{1st}0.53  \\ %
\bottomrule
\end{tabular}
}
\caption{\textbf{Recall rate of our Hilbert-curve based $\neighbor$} -- we find that the Hilbert curve consistently outperforms both the Z-order curve~\cite{morton1966computer} and the one-ring neighborhood from Minkowski relative to the exact k-nearest neighbors.
}
\vspace{-1em}
\label{tab:locality_neighbor}
\end{table}

\paragraph{Analysis of neighbors retrieved by~$\neighbor$ -- \Cref{tab:locality_neighbor}}
\at{We now investigate the quality of the point neighborhoods retrieved by various possible implementations for $\neighbor$.
In particular, we are interested to experimentally study whether our serialization indeed preserves locality.
To quantify this, we treat the neighborhood retrieved with KNN as the ground-truth.}
We report the recall rate of a local neighborhood by comparing it with this ground truth~(we ignore the precision rate because we remove false positives with a distance threshold).
We also report the recall rate of the one-ring neighborhood retrieved in Minkowski~\cite{choy20194d}.
We show that the recall rate of our Hilbert $\neighbor$ is the best across variants, and across all scales.

\begin{table}[t]
\centering
\resizebox{\columnwidth}{!}{
\begin{tabular}{L rr rR}
\toprule
Methods & \multicolumn{2}{c}{Uniform} & \multicolumn{2}{c}{Non-Uniform}   \\ 
\cmidrule(r){1-1}
\cmidrule(lr){2-3}
\cmidrule(l){4-5}
\nksr & 0.246 & 480s & 0.273 & 668s  \\
Ours~(Minkowski)~\cite{choy20194d}  & 0.301 & 97s & 0.349 & 94s \\
Ours~(Minkowski)~\cite{choy20194d} {(w/ KNN)} & 0.254 & 145s & 0.294 & 155s \\
\rowcolor{1st} Ours~(w/ serialization) & {0.257} & {152s} & {0.296} & {145s} \\
\rowcolor{1st} Ours~(w/ KNN) & \textbf{0.243} & \textbf{151s} & \textbf{0.273} & \textbf{142s}  \\
\bottomrule
\end{tabular}
}
\caption{
\textbf{The impact of sampling} -- we evaluate uniform vs non-uniform sampling on ScanNet. We find that our method achieves the best accuracy (in terms of CD ($10^{-2}$)) and good time efficiency compared to \nksr~for both sampling types.
}
\vspace{-1em}
\label{tab:nonuniform_scannet}
\end{table}

\paragraph{The impact of sampling pattern --~\Cref{tab:nonuniform_scannet}} 
We report the impact of sampling pattern on performance by evaluating models on ScanNet point clouds that are uniformly or non-uniformly sampled. 
{To non-uniformly sample the ScanNet point clouds, we first partitioned the scene into eight blocks and randomly sampled a different number of points from each block. The number of samples followed an arithmetic sequence with a common difference of 200. Finally, we padded the last block to ensure that the total number of points remained 10K.}
 
We show that our method achieves better robustness to non-uniform sampling than the baselines, highlighting the importance of avoiding quantization of the point cloud for high quality surface reconstruction. 



\section{Conclusion} \label{Sec:conclusion}
In this paper, a novel \gls{DL}-based precoding method approach named PaPP is proposed for mMIMO systems to overcome the limitations of traditional methods in terms of complexity and generalizability. By combining WMMSE with DL techniques, we designed a computationally efficient precoder that leverages MLDG and a teacher-student training framework for improved adaptability. Our results demonstrate high sum-rate performance across diverse urban environments while achieving approximately 3.5$\times$ lower computational complexity than alternative generalizable DNN approaches (MAML-CNN).  Furthermore, when deployed in a new site without fine-tuning, PaPP achieves roughly the same performance as a model trained directly on the deployment site. With fine-tuning, PaPP outperforms all other methods. 

\section*{Acknowledgement}
This work was supported by Ericsson - Global Artificial Intelligence Accelerator AI-Hub Canada in Montr\'{e}al and jointly funded by NSERC Alliance Grant 566589-21 (Ericsson, ECCC, Innov\'{E}\'{E}).

\bibliographystyle{IEEEtran}
\bibliography{0_main}


\end{document}