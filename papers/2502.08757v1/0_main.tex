\documentclass[conference]{IEEEtran}
%****************this is required for ICMLCN**********
\usepackage[letterpaper, left=0.65in, right=0.65in, bottom=1in, top=0.8in]{geometry}
\IEEEoverridecommandlockouts
\special{papersize=8.5in,11in}
%****************************************************
%****************************************************%
%               LIST OF ACRONYM HERE                 %
%****************************************************%
\usepackage[acronym]{glossaries}
\newacronym{WMMSE}{WMMSE}{weighted minimum mean square error}
\newacronym{FCL}{FCL}{fully connected layer}
\newacronym{NLOS}{NLOS}{non-line-of-sight}
% \newacronym{EE}{EE}{energy efficiency}
\newacronym{LOS}{LOS}{line-of-sight}
\newacronym{MAC}{MAC}{multiply-accumulate}
\newacronym{BS}{BS}{base station}
\newacronym{PS}{PS}{phase-shifter}
\newacronym{NAS}{NAS}{neural architecture search}
\newacronym{PTQ}{PTQ}{post-training quantization}
\newacronym{QAT}{QAT}{quantization-aware training}
\newacronym{LSQ}{LSQ}{Learned Step Size Quantization}
\newacronym{STE}{STE}{Straight Through Estimator}
\newacronym{MPQ}{MPQ}{Mixed-Precision Quantization}
\newacronym{RL}{RL}{reinforcement learning}
\newacronym{AP}{AP}{analog precoder}
\newacronym{FC-HBF}{FC-HBF}{fully-connected HBF}
\newacronym{FSA-HBF}{FSA-HBF}{fixed subarray HBF}
\newacronym{DSA-HBF}{DSA-HBF}{dynamic subarray HBF}
\newacronym{BF}{BF}{beamforming}
\newacronym{UE}{UE}{user equipment}
\newacronym{AWGN}{AWGN}{additive white gaussian noise}
%\newacronym{BS}{BS}{base station}
\newacronym{MIMO}{MIMO}{multiple-input multiple-output}
\newacronym{MISO}{MISO}{multiple-input single-output}
\newacronym{RF}{RF}{radio frequency}
\newacronym{RIS}{RIS}{reconfigurable intelligent surfaces}
\newacronym{IOT}{IOT}{internet-of-things}
\newacronym{QNN}{QNN}{Quantized Neural Network}
\newacronym{CL}{CL}{convolutional layer}
\newacronym{UdeM-NLOS}{UdeM-NLOS}{Universite De Montreal-Non-Line-Of-Sight}
\newacronym{Okapark-LOS}{Okapark-LOS}{Okapark-Line-Of-Sight}
\newacronym{FDD}{FDD}{frequency division duplex}
\newacronym{TDD}{TDD}{time division duplex}
\newacronym{CSI}{CSI}{channel state information}
\newacronym{DNN}{DNN}{Deep Neural Network}
\newacronym{DP}{DP}{digital precoder}
\newacronym{DL}{DL}{deep learning}
\newacronym{SVD}{SVD}{singular-value decomposition}
\newacronym{CNN}{CNN}{convolution neural network}
\newacronym{FDP}{FDP}{fully digital precoder}
\newacronym{SE}{SE}{spectral efficiency}
\newacronym{OFDM}{OFDM}{orthogonal frequency division multiplexing}
\newacronym{OMP}{OMP}{orthogonal matching pursuit}
\newacronym{FL}{FL}{fully-connected layer}
\newacronym{HSHO}{HSHO}{Hybrid Structured Heuristic Optimization}
\newacronym{HBF}{HBF}{hybrid beamforming}
\newacronym{IA}{IA}{initial access}
\newacronym{mm-Wave}{mm-Wave}{millimeter wave}
\newacronym{mMIMO}{mMIMO}{massive multiple-input multiple-output}
\newacronym{SINR}{SINR}{signal-to-interference-noise ratio}
\newacronym{SNR}{SNR}{signal-to-noise ratio}
%\newacronym{SS}{SS}{synchronization signal}
%\newacronym{SSB}{SSB}{synchronization signal burst}
\newacronym{RSSI}{RSSI}{received signal strength indicator}
\newacronym{PZF}{PZF}{phase zero forcing}
\newacronym{PSO}{PSO}{particle swarm optimization}
\newacronym{ZF}{ZF}{zero forcing}
\newacronym{O-FDP}{O-FDP}{optimal fully digital precoder}
\newacronym{JT}{JT}{joint transmission}
\newacronym{CU}{CU}{central unit}
\newacronym{MSE}{MSE}{mean square error}
\newacronym{CEL}{CEL}{cross entropy loss}
\newacronym{CB}{CB}{conjugate beamforming}
\newacronym{NC}{NC}{network controller}
\newacronym{CoMP}{CoMP}{coordinated multi point}
\newacronym{CF-mMIMO}{CF-mMIMO}{cell-free massive MIMO}
\newacronym{CF-HBF}{CF-HBF}{cell-free hybrid beamforming}
\newacronym{CF-BF}{CF-BF}{cell-free beamforming}
\newacronym{MLDG}{MLDG}{Meta-Learning Domain Generalization}
\newacronym{MAML}{MAML}{Model Agnostic Meta-Learning}
\newacronym{WSR}{WSR}{Weighted Sum Rate}

\newif\ifDeepMIMOModel
\DeepMIMOModeltrue

\newif\ifSimpleNParamEq
\SimpleNParamEqtrue

\usepackage{lettrine}
\usepackage[ruled,vlined,linesnumbered]{algorithm2e}
\usepackage{multirow}
\usepackage{longtable}
\usepackage[table,xcdraw]{xcolor}
\usepackage{array} 
\makeatletter
\let\oldlt\longtable
\let\endoldlt\endlongtable
\def\longtable{\@ifnextchar[\longtable@i \longtable@ii}
\def\longtable@i[#1]{\begin{figure}[t]
\onecolumn
\begin{minipage}{0.5\textwidth}
\oldlt[#1]
}
\def\longtable@ii{\begin{figure}[t]
\onecolumn
\begin{minipage}{0.5\textwidth}
\oldlt
}
\def\endlongtable{\endoldlt
\end{minipage}
\twocolumn
\end{figure}}
\makeatother
\usepackage{ifpdf}
% *** GRAPHICS RELATED PACKAGES ***
\ifCLASSINFOpdf
  \usepackage[pdftex]{graphicx}
  \usepackage{graphicx,epstopdf}
  \graphicspath{{../pdf/}{../jpeg/}}
  \DeclareGraphicsExtensions{.pdf,.jpeg,.png,.eps}
\else
  \usepackage[dvips]{graphicx}
  \usepackage{graphicx,epstopdf}
  \graphicspath{{../eps/}}
  \DeclareGraphicsExtensions{}
\fi
\usepackage{tikz}
\usetikzlibrary{decorations.pathreplacing,calc}
\newcommand{\tikzmark}[1]{\tikz[overlay,remember picture] \node (#1) {};}

\newcommand*{\AddNote}[4]{%
    \begin{tikzpicture}[overlay, remember picture]
        \draw [decoration={brace,amplitude=0.5em},decorate,line width=.2mm,black]
            ($(#3)!(#1.north)!($(#3)-(0,1)$)$) --  
            ($(#3)!(#2.south)!($(#3)-(0,1)$)$)
                node [align=center, text width=2.5cm, pos=0.5, anchor=west] {#4};
    \end{tikzpicture}
}%

\usepackage{cite}
\usepackage{amsthm}
\usepackage{steinmetz}
\usepackage{amssymb}
\usepackage{balance}
\usepackage{eqparbox}
\usepackage{multirow}
\usepackage{bbm}
\usepackage{float}
\SetAlFnt{\small}


\usepackage{amsmath}

\usepackage{mathtools, nccmath}
\DeclarePairedDelimiter{\nint}\lfloor\rceil
\DeclarePairedDelimiter{\abs}\lvert\rvert

\usepackage{booktabs, siunitx}
\usepackage{dcolumn}
\newcolumntype{d}[1]{D{.}{.}{#1}}
% \usepackage[svgnames,table]{xcolor}
%\usepackage[tableposition=above]{caption}

% \usepackage{graphicx}
\usepackage[scaled]{DejaVuSansMono}
\usepackage[T1]{fontenc}

%\usepackage{stfloats}
%\usepackage[eulergreek]{sansmath}
% \usepackage[final]{graphicx}
\usepackage{color}
\usepackage{relsize}
\usepackage{mathtools}
\newcommand{\ex}{{\mathrm e}}
\newtheorem{definition}{Definition}
\newtheorem{lemma}{Lemma}
\newtheorem{theorem}{Theorem}
\newtheorem{remark}{Remark}

\newtheorem{proposition}{Proposition}
\newcommand{\bs}[1]{\boldsymbol{#1}}
\newcommand{\mc}[1]{\mathcal{#1}}
\newcommand{\ul}[1]{\underline{#1}}
\newcommand{\mb}[1]{\mathbf{#1}}
\newcommand{\mr}[1]{\mathrm{#1}}
\newcommand{\tr}{\mathrm{Tr}}
\newcommand{\fo}{\mathbf{F}_{\mathrm{opt}}}

% \newcommand{\ceil}[1]{\lceil {#1} \rceil}
\usepackage{mathtools}
\DeclarePairedDelimiter{\ceil}{\lceil}{\rceil}

\DeclareMathOperator*{\argmin}{arg\;min}
\DeclareMathOperator*{\argmax}{arg\;max}
\DeclareMathOperator*{\maximize}{maximize}
\DeclareMathOperator*{\minimize}{minimize}
\DeclareMathOperator*{\st}{subject\;to}
\renewcommand{\Pr}{\mathbb{P}} % (by default \Pr is rendered as "Pr")
\pagestyle{empty}
% \thispagestyle{plain}
% \pagestyle{plain}
\SetKwInput{KwInput}{Input}                % Set the Input
\SetKwInput{KwOutput}{Output}
\SetKwInput{KwOutputr}{Output Regression}              % set the Output
\SetKwInput{KwOutputc}{Output Classification}              % set the Output
\newcommand{\rank}{\operatornamewithlimits{rank}}
\newcommand{\trace}{\operatornamewithlimits{trace}}
%\newcommand{\argmin}{\operatornamewithlimits{argmin}}
%\newcommand{\argmax}{\operatornamewithlimits{argmax}}
\newcommand{\bseq}{\begin{subequations}}
\newcommand{\eseq}{\end{subequations}}
\newcommand{\baln}{\begin{align}}
\newcommand{\ealn}{\end{align}}
\newcommand{\balnd}{\begin{aligned}}
\newcommand{\ealnd}{\end{aligned}}
\newcommand{\beq}{\begin{equation}}
\newcommand{\eeq}{\end{equation}}
\newcommand{\beqn}{\begin{eqnarray}}
\newcommand{\eeqn}{\end{eqnarray}}
\newcommand{\beqno}{\begin{eqnarray*}}
\newcommand{\eeqno}{\end{eqnarray*}}
\newcommand{\bma}{\begin{displaymath}}
\newcommand{\ema}{\end{displaymath}}
\newcommand{\bnu}{\begin{enumerate}}
\newcommand{\enu}{\end{enumerate}}
\newcommand{\bce}{\begin{center}}
\newcommand{\ece}{\end{center}}
\newcommand{\btb}{\begin{tabular}}
\newcommand{\etb}{\end{tabular}}
\newcommand{\ba}{\begin{array}}
\newcommand{\ea}{\end{array}}
%\setlength\arraycolsep{2pt}
\usepackage{footnote}
\makesavenoteenv{tabular}
\makesavenoteenv{table}
\makeatletter 
\newcommand\semiHuge{\@setfontsize\semiHuge{21.1}{27.38}}
\makeatother
% \setlength{\textfloatsep}{0pt}
% \setlength{\parskip}{0pt} 
%****************************************************
\begin{document}
\title{A Low-Complexity Plug-and-Play Deep Learning Model for Massive MIMO Precoding Across Sites\\
\author{\IEEEauthorblockN{Ali~Hasanzadeh~Karkan$^{*}$, Ahmed~Ibrahim$^{\dagger}$, Jean-François~Frigon$^{*}$, and François~Leduc-Primeau$^{*}$\\
$^{*}$Department of Electrical Engineering, Polytechnique Montréal, Montréal, QC H3C 3A7, Canada\\
$^{*}$Emails: \{ali.hasanzadeh-karkan, j-f.frigon, francois.leduc-primeau\}@polymtl.ca. \\
$^{\dagger}$Ericsson Canada's R\&D,  Kanata, ON K2K 2V6, Canada, Email: ahmed.a.ibrahim@ericsson.com.
}
}
}

\maketitle
\IEEEpubidadjcol

\begin{abstract}
Massive multiple-input multiple-output (mMIMO) technology has transformed wireless communication by enhancing spectral efficiency and network capacity. This paper proposes a novel deep learning-based mMIMO precoder to tackle the complexity challenges of existing approaches, such as weighted minimum mean square error (WMMSE), while leveraging meta-learning domain generalization and a teacher-student architecture to improve generalization across diverse communication environments. 
When deployed to a previously unseen site, the proposed model
achieves excellent sum-rate performance while maintaining low computational complexity by avoiding matrix inversions and by using a simpler neural network structure.
The model is trained and tested on a custom ray-tracing dataset composed of several base station locations. The experimental results indicate that our method effectively balances computational efficiency with high sum-rate performance while showcasing strong generalization performance in unseen environments.
Furthermore, with fine-tuning, the proposed model outperforms WMMSE across all tested sites and SNR conditions while reducing complexity by at least 73$\times$.
\end{abstract}

% \begin{IEEEkeywords}
% \end{IEEEkeywords}
% \IEEEpeerreviewmaketitle

\section{Introduction} \label{Sec:Intro}
\section{Introduction}
Backdoor attacks pose a concealed yet profound security risk to machine learning (ML) models, for which the adversaries can inject a stealth backdoor into the model during training, enabling them to illicitly control the model's output upon encountering predefined inputs. These attacks can even occur without the knowledge of developers or end-users, thereby undermining the trust in ML systems. As ML becomes more deeply embedded in critical sectors like finance, healthcare, and autonomous driving \citep{he2016deep, liu2020computing, tournier2019mrtrix3, adjabi2020past}, the potential damage from backdoor attacks grows, underscoring the emergency for developing robust defense mechanisms against backdoor attacks.

To address the threat of backdoor attacks, researchers have developed a variety of strategies \cite{liu2018fine,wu2021adversarial,wang2019neural,zeng2022adversarial,zhu2023neural,Zhu_2023_ICCV, wei2024shared,wei2024d3}, aimed at purifying backdoors within victim models. These methods are designed to integrate with current deployment workflows seamlessly and have demonstrated significant success in mitigating the effects of backdoor triggers \cite{wubackdoorbench, wu2023defenses, wu2024backdoorbench,dunnett2024countering}.  However, most state-of-the-art (SOTA) backdoor purification methods operate under the assumption that a small clean dataset, often referred to as \textbf{auxiliary dataset}, is available for purification. Such an assumption poses practical challenges, especially in scenarios where data is scarce. To tackle this challenge, efforts have been made to reduce the size of the required auxiliary dataset~\cite{chai2022oneshot,li2023reconstructive, Zhu_2023_ICCV} and even explore dataset-free purification techniques~\cite{zheng2022data,hong2023revisiting,lin2024fusing}. Although these approaches offer some improvements, recent evaluations \cite{dunnett2024countering, wu2024backdoorbench} continue to highlight the importance of sufficient auxiliary data for achieving robust defenses against backdoor attacks.

While significant progress has been made in reducing the size of auxiliary datasets, an equally critical yet underexplored question remains: \emph{how does the nature of the auxiliary dataset affect purification effectiveness?} In  real-world  applications, auxiliary datasets can vary widely, encompassing in-distribution data, synthetic data, or external data from different sources. Understanding how each type of auxiliary dataset influences the purification effectiveness is vital for selecting or constructing the most suitable auxiliary dataset and the corresponding technique. For instance, when multiple datasets are available, understanding how different datasets contribute to purification can guide defenders in selecting or crafting the most appropriate dataset. Conversely, when only limited auxiliary data is accessible, knowing which purification technique works best under those constraints is critical. Therefore, there is an urgent need for a thorough investigation into the impact of auxiliary datasets on purification effectiveness to guide defenders in  enhancing the security of ML systems. 

In this paper, we systematically investigate the critical role of auxiliary datasets in backdoor purification, aiming to bridge the gap between idealized and practical purification scenarios.  Specifically, we first construct a diverse set of auxiliary datasets to emulate real-world conditions, as summarized in Table~\ref{overall}. These datasets include in-distribution data, synthetic data, and external data from other sources. Through an evaluation of SOTA backdoor purification methods across these datasets, we uncover several critical insights: \textbf{1)} In-distribution datasets, particularly those carefully filtered from the original training data of the victim model, effectively preserve the model’s utility for its intended tasks but may fall short in eliminating backdoors. \textbf{2)} Incorporating OOD datasets can help the model forget backdoors but also bring the risk of forgetting critical learned knowledge, significantly degrading its overall performance. Building on these findings, we propose Guided Input Calibration (GIC), a novel technique that enhances backdoor purification by adaptively transforming auxiliary data to better align with the victim model’s learned representations. By leveraging the victim model itself to guide this transformation, GIC optimizes the purification process, striking a balance between preserving model utility and mitigating backdoor threats. Extensive experiments demonstrate that GIC significantly improves the effectiveness of backdoor purification across diverse auxiliary datasets, providing a practical and robust defense solution.

Our main contributions are threefold:
\textbf{1) Impact analysis of auxiliary datasets:} We take the \textbf{first step}  in systematically investigating how different types of auxiliary datasets influence backdoor purification effectiveness. Our findings provide novel insights and serve as a foundation for future research on optimizing dataset selection and construction for enhanced backdoor defense.
%
\textbf{2) Compilation and evaluation of diverse auxiliary datasets:}  We have compiled and rigorously evaluated a diverse set of auxiliary datasets using SOTA purification methods, making our datasets and code publicly available to facilitate and support future research on practical backdoor defense strategies.
%
\textbf{3) Introduction of GIC:} We introduce GIC, the \textbf{first} dedicated solution designed to align auxiliary datasets with the model’s learned representations, significantly enhancing backdoor mitigation across various dataset types. Our approach sets a new benchmark for practical and effective backdoor defense.




\section{Problem Formulation} \label{Sec:Problem_Formulation}
\subsection{System Model}
In this paper, we study a time division duplex multi-user \gls{mMIMO} system, where uplink channel estimates can be used to calculate the downlink precoder. This \gls{mMIMO} system has a \gls{BS} that is equipped with $N_{\sf{T}}$ antennas. The system is designed to concurrently support $N_{\sf{U}}$ users, each utilizing a single antenna. This configuration enables efficient communication by leveraging the multiple antennas at the \gls{BS} to enhance signal quality and increase capacity, ultimately allowing for simultaneous transmissions to multiple users. 

The received signal at the $k^{th}$ user can be expressed as
\begin{equation}\label{eq:signal_recived}
    \mathbf{y}_k =  \mathbf{h}_{k}^{\dagger} \sum_{\forall k}  \mathbf{w}_{k} x_k + \bs{\eta}_k \, ,
\end{equation}
where the wireless channel vector between the base station (\gls{BS}) and the $k^{th}$ user is represented by $\mathbf{h}_{k} \in \mathbb{C}^{N_{\sf{T}} \times 1}$. Let $x_k$
denote an independent transmitted complex symbol for a user. The term $\bs{\eta}_k \sim \mathcal{CN}(0, \sigma^2)$ indicates complex \gls{AWGN} characterized by a zero mean and a variance of $\sigma^2$. The downlink \gls{FDP} vector is defined as $\mathbf{W} = \left[ \mathbf{w}_1, \ldots, \mathbf{w}_k, \ldots, \mathbf{w}_{N_{\sf{U}}} \right] \in \mathbb{C}^{N_{\sf{T}} \times N_{\sf{U}}}$.
The associated \gls{SINR} user $k$ is given by
\begin{equation}
    \text{SINR}(\mb{w}_{k}) = \frac{ \big|\mb{h}^{\dagger}_{k} \mb{w}_{k} \big|^2}{\sum_{j \neq k} \big|\mb{h}^{\dagger}_{k} \mb{w}_{j} \big|^2 + \sigma^2}\,,
\end{equation}
and the sum rate for all users is
\begin{equation}\label{eq:sum-rate}
    R(\mb{W}) = \sum_{\forall k}  \text{log}_2 \Bigl(  1+ \text{SINR}(\mb{w}_{k}) \Bigr).
\end{equation}
The objective is to determine a precoding matrix $\mathbf{W}$ that maximizes the throughput in eq. (\ref{eq:sum-rate}) while adhering to the maximum transmit power constraint, $P_{\text{max}}$. Consequently, the downlink sum-rate maximization problem can be formulated as
\begin{equation}\label{eq:maximization}
\begin{aligned}
    & \underset{\mb{W}}{\max}~ R(\mb{W}) , \\
    \text{s.t.}  &\sum_{\forall k} \mb{w}_{k}^{\dagger}  \mb{w}_{k} \leq P_{\sf{max}} \, .
\end{aligned}
\end{equation}

\subsection{Weighted Minimum Mean Square Error Algorithm}
The sum-rate maximization problem in (\ref{eq:maximization}) is classified as NP-hard. To find good approximate solutions, the iterative \gls{WMMSE} algorithm \cite{shi2011iteratively} is commonly used, where the problem is transformed to a corresponding problem focused on minimizing the sum of \gls{MSE} under the independence assumption of $x_k$ and $\bs{\eta}_k$. 
Key parameters in this framework include the receiver gain $u_k$ and a positive user weight $v_k$, which are used to obtain the \gls{MSE} covariance matrix. 
The solution is obtained by iteratively solving convex subproblems to generate updates on the receiver gains, user weights, and beamforming matrix.
% The optimization problem involves iteratively updating variables such as the receiver gains, user weights, and beamforming matrices to maximize a redefined objective function while satisfying power constraints. Each variable is optimized sequentially, holding the others constant, ensuring the problem remains convex for individual updates.
% Let $u_k$ represent the receiver gain, while $v_k$ is a positive user weight, hence the \gls{MSE} covariance matrix $e_k$ can be expressed as 
% \begin{align}
% e_k &= \mathop{\mathbb{E}}_{\bs{x},\bs{\eta}} [ (\hat{x_k} - x_k)(\hat{x_k} - x_k)^{\dagger} ]\\
% &= \left( \mathbf{I} - u^{\dagger}_k\mathbf{h}_{k}\mathbf{w}_k\right)\left( \mathbf{I} - u^{\dagger}_k\mathbf{h}_{k}\mathbf{w}_k\right)^{\dagger},\\
% &=  \left| u_{k} \mathbf{h}_{k}^{\dagger} \mathbf{w}_k - 1 \right|^2 +  \sum_{j \neq k}^{N_{\sf{U}}} \left| u_k \mathbf{h}_k^{\dagger} v_j \right|^2  + \sigma^2 \left| u_k \right|^2,
% \end{align}
% thus the optimization problem can be formulated as
% \begin{align}
%     \underset{\mb{u},\mb{v},\mb{W}}{\max}~ & \sum^{N_{\sf{U}}}_{k=1} (v_k e_k - \text{log}_2 v_k) , \\
%    &\text{s.t.}   \sum_{\forall k} \mb{w}_{k}^{\dagger}  \mb{w}_{k} \leq P_{\sf{max}} \, .
% \end{align}
% This optimization problem becomes convex when each variable is considered individually, with the remaining variables held constant. Therefore the \gls{WMMSE} algorithm operates by sequentially updating each variable, 

First $\mathbf{W}^{(0)}$ is initialized while satisfying the constraint in \eqref{eq:maximization},
% in such a way that $\sum_{\forall k} \mb{w}_{k}^{\dagger}  \mb{w}_{k} \leq P_{\sf{max}}$. 
then, at each iteration $i$, variables are updated as defined below until a stopping criterion is satisfied:
\begin{align}
    v^{(\text{i})}_{k} = \frac{ \sum_{j =1}^{N_{\sf{U}}} \big|\mb{h}^{\dagger}_{k} \mb{w}^{(\text{i}-1)}_{j} \big|^2 + \sigma^2}{\sum_{j \neq k}^{N_{\sf{U}}} \big|\mb{h}^{\dagger}_{k} \mb{w}^{(\text{i}-1)}_{j} \big|^2 + \sigma^2},
\end{align}
\begin{align}
    u^{(\text{i})}_{k} = \frac{ \mb{h}^{\dagger}_{k} \mb{w}^{(\text{i}-1)}_{j} }{\sum_{j=1}^{N_{\sf{U}}} \big|\mb{h}^{\dagger}_{k} \mb{w}^{(\text{i}-1)}_{j} \big|^2 + \sigma^2},
\end{align}
\begin{align}\label{eq:construct_beamforming_matrix}
    \mb{w}^{(\text{i}+1)}_{k} =  u^{(\text{i})}_{k} v^{(\text{i})}_{k}\mb{h}_{k}(\sum^{N_{\sf{U}}}_{j=1}   v^{(\text{i})}_{j} |u^{(\text{i})}_{j}|^2 \mb{h}_{j}\mb{h}_{j}^{\dagger} + \mu \mathbf{I})^{-1},
\end{align}
where $\mu \geq 0$ is a Lagrange multiplier.
In addition to the high complexity of solving \gls{WMMSE}, the inherent non-convexity of sum-rate maximization means that WMMSE is only guaranteed to converge to a local optimum, not necessarily the global solution.

\section{Proposed Method} \label{Sec:Proposed}
\section{Proposed method}
% \subsection{Preliminaries on LIC rate-distortion optimization}
\subsection{Preliminaries: rate-distortion optimization in learned image compression}

In LIC, the objective is to encode an image \( x \), drawn from a source distribution with probability density function \( p_{\text{source}} \), into a compact bit sequence for efficient storage or transmission. The receiver then reconstructs an approximation \( \hat{x} \) of the original image \( x \). The LIC process involves three main steps: encoding and quantization, entropy coding, and reconstruction.

\begin{itemize}
    \item \textbf{Encoding and Quantization}: First, each data point \( x \) is mapped to a de-correlated low-dimensional latent variable \( \hat{z} \) via an encoder function \( e(\cdot) \) followed by quantization \( Q(\cdot) \) to convert the continuous representation into discrete values, i.e., \( \hat{z} = Q(e(x)) \).

    \item \textbf{Entropy Coding}: After determining \( \hat{z} \), lossless entropy coding, such as Huffman coding~\cite{moffat2019huffman} or arithmetic coding~\cite{witten1987arithmetic}, is applied to produce a compressed bit sequence with length \( b(\hat{z}) \). Ideally, the entropy coding scheme approximates the theoretical bit rate, given by the entropy of \( \hat{z} \) under its marginal distribution. We assume that both the sender and receiver have access to an entropy model \( P(\hat{z}) \), which estimates the marginal probability distribution of \( \hat{z} \) and determines the expected bit length as \( b(\hat{z}) \approx -\log_2 P(\hat{z}) \). The goal is to ensure that \( P(\hat{z}) \) closely approximates the true marginal distribution \( p(\hat{z}) \), which is defined as
\(
p(\hat{z}) = \mathbb{E}_{x \sim p_{\text{source}}} \left[ \delta(\hat{z}, Q(e(x))) \right],
\)
where \( \delta \) is the Kronecker delta function. Under this approximation, the encoding length \( b(\hat{z}) \) is nearly optimal, as the average code length approaches the entropy \( -\log_2 p(\hat{z}) \) of \( \hat{z} \).

    \item \textbf{Reconstruction}: Once the receiver has obtained \( \hat{z} \), it reconstructs the approximation \( \hat{x} \) using a reconstruction function \( r(\cdot) \), such that \( \hat{x} = r(\hat{z}) \).
\end{itemize}

To optimize the LIC scheme, our goal is to minimize both the bit rate (rate) and the discrepancy between \( x \) and \( \hat{x} \) (distortion), where the distortion is measured by a function \( d(x, \hat{x}) \). This objective is formulated as a R-D loss using a Lagrangian multiplier:
\begin{equation}
\label{eq:rd}
    \mathcal{L}_\text{R-D} = \mathbb{E}_{x \sim p_{\text{source}}} \left[ \underbrace{-\log_2 P(\hat{z})}_{\text{rate}\ (\mathcal{L}_{\text{R}})} + \underbrace{\lambda d(x, \hat{x})}_{\text{distortion}\ (\mathcal{L}_{\text{D}})} \right],
\end{equation}
where \( \lambda \) is a Lagrange multiplier that controls the trade-off between rate (compression efficiency) and distortion (reconstruction quality). This formulation represents the Lagrangian relaxation of the distortion-constrained R-D optimization problem, aiming for efficient compression while maintaining high reconstruction fidelity. In practice, the gradient used to update LIC models is the sum of the gradients for each objective:
\(
d_t = \nabla \mathcal{L}_{\text{R}, t} + \nabla \mathcal{L}_{\text{D}, t}
\), where $t$ denotes the training iteration.


\begin{algorithm*}[t]
\caption{Balanced Rate-Distortion Optimization via Trajectory Optimization}
\label{alg:solution1}
\begin{algorithmic}[1]
\REQUIRE Initial network parameters $\theta_0$, initial softmax logits $\boldsymbol{\xi}_0$, learning rates $\alpha$ (for $\theta$) and $\beta$ (for $\boldsymbol{\xi}$), decay parameter $\gamma$, total iterations $T$
\STATE Initialize weights $\mathbf{w}_0 \leftarrow \text{Softmax}(\boldsymbol{\xi}_0)$
\FOR{$t = 0$ \TO $T-1$}
    \STATE Compute losses $\mathcal{L}_{\text{R}, t} = \mathcal{L}_{\text{R}}(\theta_t)$ and $\mathcal{L}_{\text{D}, t} = \mathcal{L}_{\text{D}}(\theta_t)$
    \STATE Compute gradients: $\nabla_\theta \mathcal{L}_{\text{R}, t} = \frac{\partial \mathcal{L}_{\text{R}, t}}{\partial \theta_t}$, \quad $\nabla_\theta \mathcal{L}_{\text{D}, t} = \frac{\partial \mathcal{L}_{\text{D}, t}}{\partial \theta_t}$
    \STATE \hspace{2.1cm} $\nabla_\theta \log \mathcal{L}_{\text{R}, t} = \frac{\nabla_\theta \mathcal{L}_{\text{R}, t}}{\mathcal{L}_{\text{R}, t}}$, \quad $\nabla_\theta \log \mathcal{L}_{\text{D}, t} = \frac{\nabla_\theta \mathcal{L}_{\text{D}, t}}{\mathcal{L}_{\text{D}, t}}$
    \STATE Compute normalization constant: $c_t = \left( \frac{w_{\text{R}, t}}{\mathcal{L}_{\text{R}, t}} + \frac{w_{\text{D}, t}}{\mathcal{L}_{\text{D}, t}} \right)^{-1}$
    \STATE Compute balanced gradient: $\mathbf{d}_t = c_t \left( w_{\text{R}, t} \nabla_\theta \log \mathcal{L}_{\text{R}, t} + w_{\text{D}, t} \nabla_\theta \log \mathcal{L}_{\text{D}, t} \right)$
    \STATE Update network parameters: $\theta_{t+1} = \theta_t - \alpha\, \mathbf{d}_t$
    \STATE Compute updated losses $\mathcal{L}_{\text{R}, t+1} = \mathcal{L}_{\text{R}}(\theta_{t+1})$ and $\mathcal{L}_{\text{D}, t+1} = \mathcal{L}_{\text{D}}(\theta_{t+1})$
    \STATE Compute $\delta_t = \begin{bmatrix} \frac{\partial w_{\text{R}, t}}{\partial \boldsymbol{\xi}_t} \\ \frac{\partial w_{\text{D}, t}}{\partial \boldsymbol{\xi}_t} \end{bmatrix}^\top \begin{bmatrix} \log \mathcal{L}_{\text{R}, t} - \log \mathcal{L}_{\text{R}, t+1} \\ \log \mathcal{L}_{\text{D}, t} - \log \mathcal{L}_{\text{D}, t+1} \end{bmatrix}$
    \STATE Update softmax logits: $\boldsymbol{\xi}_{t+1} = \boldsymbol{\xi}_t - \beta \left( \delta_t + \gamma\, \boldsymbol{\xi}_t \right)$
    \STATE Update weights: $\mathbf{w}_{t+1} = \text{Softmax}(\boldsymbol{\xi}_{t+1})$
\ENDFOR
\end{algorithmic}
\end{algorithm*}

\subsection{Balanced rate-distortion optimization}
To address the imbalance in optimizing rate and distortion discussed in Sec.~\ref{sec:intro}, we propose a balanced R-D optimization framework. This framework dynamically adjusts the contributions from each objective in the R-D loss function, aiming to achieve equal progress in both rate and distortion. Specifically, we redefine the update direction as \( d_t = w_{\text{R},t} \nabla \log \mathcal{L}_{\text{R}, t} + w_{\text{D},t} \nabla \log \mathcal{L}_{\text{D}, t}\), where \(w_{\text{R},t}\) and \(w_{\text{L},t}\) are adaptive weights for the rate and distortion gradients. By maximizing the minimum improvement speed between rate and distortion at each step, our method finds the ideal gradient weights, promotes stable balanced optimization, and enhances overall model performance.


Inspired by  MOO techniques~\cite{sener2018multi, liu2024famo}, we reformulate the R-D optimization problem as a multi-objective optimization problem. Our goal is to simultaneously optimize rate minimization and distortion minimization with parameters \( \theta \in \mathbb{R}^m \). To achieve this, we modify the original Lagrangian R-D loss function to the following form:
\begin{equation}
\min_{\theta \in \mathbb{R}^m} \left\{ \mathcal{L}(\theta) = \mathcal{L}_{\text{R}}(\theta) + \mathcal{L}_{\text{D}}(\theta)  \right\},
\end{equation}
where the trade-off factor \( \lambda \) is now absorbed into \( \mathcal{L}_{\text{D}} \) to simplify the formulation. $m$ is the number of parameters.

The loss improvement speed at iteration \( t \) of task \(i \in \{\text{R}, \text{D}\}\) is defined as:
\begin{equation}
\label{eq:speed}
s_{i,t}(\alpha, d_t) = \frac{\mathcal{L}_{i,t} - \mathcal{L}_{i,t+1}}{\mathcal{L}_{i,t}},
\end{equation}
where \( \alpha \) is the step size and \( d_t \) is the update direction at \( t \), with \( \theta_{t+1} = \theta_t - \alpha d_t \). To achieve balanced optimization, we seek an update direction \( d_t \) that maximizes the minimum improvement speed between rate and distortion, ensuring that neither objective dominates the update. In other words, we maximize the smaller improvement speed between \( \mathcal{L}_{\text{R}} \) and \( \mathcal{L}_{\text{D}} \). This leads to the following saddle point problem~\cite{lin2020near} formulation:
{\small
\begin{equation}
\max_{d_t \in \mathbb{R}^m} \min \left( \frac{1}{\alpha} s_{\text{R},t}(\alpha, d_t) - \frac{1}{2} \|d_t\|^2, \frac{1}{\alpha} s_{\text{D},t}(\alpha, d_t) - \frac{1}{2} \|d_t\|^2 \right),
\end{equation}}
where \( \frac{1}{2} \|d_t\|^2 \) serves as a regularization term to prevent unbounded updates.
% where \( \frac{1}{2} \|d_t\|^2 \) serves as a regularization term to prevent an unbounded solution.

When the step size \( \alpha \) is small, one can approximate \( \mathcal{L}_{t+1} \approx \mathcal{L}_{t} - \alpha \nabla \mathcal{L}_t^\top d_{t} \) using a first-order Taylor expansion. This approximation simplifies the problem to:
{\small\begin{equation}
\begin{aligned}
&\max_{d_t \in \mathbb{R}^m} \min \left( \frac{1}{\alpha} s_{\text{R},t}(\alpha, d_t) - \frac{1}{2} \|d_t\|^2, \frac{1}{\alpha} s_{\text{D},t}(\alpha, d_t) - \frac{1}{2} \|d_t\|^2 \right)\\
&=\max_{d_t \in \mathbb{R}^m} \min \left( \frac{\nabla \mathcal{L}_{\text{R}, t}^\top d_t}{\mathcal{L}_{\text{R}, t}} - \frac{1}{2} \|d_t\|^2, \frac{\nabla \mathcal{L}_{\text{D}, t}^\top d_t}{\mathcal{L}_{\text{D}, t}} - \frac{1}{2} \|d_t\|^2 \right) \\
&= \max_{d_t \in \mathbb{R}^m} \left( \min \left( \nabla \log \mathcal{L}_{\text{R}, t}^\top d_t, \nabla \log \mathcal{L}_{\text{D}, t}^\top d_t \right) - \frac{1}{2} \|d_t\|^2 \right).
\end{aligned}
\end{equation}}

To avoid solving the high-dimensional primal problem directly (as \( d_t \in \mathbb{R}^m \) with potentially millions of parameters if \( \theta \) is a neural network), we follow previous work~\cite{sener2018multi,liu2024famo} to turn to the dual problem. Leveraging the Lagrangian duality theorem~\cite{boyd2004convex}, we can rewrite the optimization as a convex combination of gradients:
{\small
\begin{equation}
\begin{aligned}
&\max_{d_t \in \mathbb{R}^m} \left( \min \left( \nabla \log \mathcal{L}_{\text{R}, t}^\top d_t, \nabla \log \mathcal{L}_{\text{D}, t}^\top d_t \right) - \frac{1}{2} \|d_t\|^2 \right) \\
&= \max_{d_t \in \mathbb{R}^m} \min_{w \in \mathbb{S}_2} \left( w_{\text{R},t} \nabla \log \mathcal{L}_{\text{R}, t} + w_{\text{D},t} \nabla \log \mathcal{L}_{\text{D}, t} \right)^\top d_t - \frac{1}{2} \|d_t\|^2 \\
&= \min_{w_t \in \mathbb{S}_2} \max_{d_t \in \mathbb{R}^m} \left( w_{\text{R},t} \nabla \log \mathcal{L}_{\text{R}, t} + w_{\text{D},t} \nabla \log \mathcal{L}_{\text{D}, t} \right)^\top d_t - \frac{1}{2} \|d_t\|^2,
\end{aligned}
\end{equation}}
where the second equality follows from strong duality. Here, \( w_t \in \mathbb{S}_2 = \{ w \in \mathbb{R}_{\geq 0}^2 \mid w^\top \mathbf{1} = 1 \} \) represents the gradient weights in the 2-dimensional probabilistic simplex.

Let \( g(d_t, w) = (w_{\text{R},t} \nabla \log \mathcal{L}_{\text{R}, t} + w_{\text{D},t} \nabla \log \mathcal{L}_{\text{D}, t})^\top d_t - \frac{1}{2} \|d_t\|^2 \). The optimal direction \( d_t^* \) is obtained by setting:
\begin{equation}
\small
\label{eq:opt_grad}
\frac{\partial g}{\partial d_t} = 0 \quad \Longrightarrow \quad d_t^* = w_{\text{R},t} \nabla \log \mathcal{L}_{\text{R}, t} + w_{\text{D},t} \nabla \log \mathcal{L}_{\text{D}, t}.
\end{equation}

Substituting \( d_t^* \) back, we obtain:
\begin{equation}
\small
\label{eq:target}
\begin{aligned}
&\max_{d_t \in \mathbb{R}^m} \min \left( \frac{1}{\alpha} s_{\text{R},t}(\alpha, d_t) - \frac{1}{2} \|d_t\|^2, \frac{1}{\alpha} s_{\text{D},t}(\alpha, d_t) - \frac{1}{2} \|d_t\|^2 \right) \\
&= \min_{w_t \in \mathbb{S}_2} \frac{1}{2} \left\| w_{\text{R},t} \nabla \log \mathcal{L}_{\text{R}, t} + w_{\text{D},t} \nabla \log \mathcal{L}_{\text{D}, t} \right\|^2 \\
&= \min_{w_t \in \mathbb{S}_2} \frac{1}{2} \| J_t w_t \|^2,
\end{aligned}
\end{equation}
where \( J_t = \begin{bmatrix}
\nabla \log \mathcal{L}_{\text{R}, t}^\top \\
\nabla \log \mathcal{L}_{\text{D}, t}^\top
\end{bmatrix} \). Thus, our optimization problem reduces to finding \( w_t \) that satisfies Eq.~\ref{eq:target}.


\subsubsection{Solution 1: Gradient descent over trajectory}
\label{subsubsec:gd_over}
Rather than fully solving the optimization problem at each step, our proposed Solution 1 adopts a coarse-to-fine gradient descent approach~\cite{sener2018multi,liu2024famo}, incrementally refining the solution along the R-D optimization trajectory. In this approach, the gradient weights \( w_t \) are updated iteratively as follows:
\begin{equation}
w_{t+1} = w_{t} - \alpha_w \tilde{\delta},
\end{equation}
where
\begin{equation}
\tilde{\delta} = \nabla_w \frac{1}{2} \| J_t w_t \|^2 = J_t^\top J_t w_t.
\end{equation}

Using a first-order Taylor approximation, \( \log \mathcal{L}_{t+1} \approx \log \mathcal{L}_{t} - \alpha \nabla \log \mathcal{L}_t^\top d_{t} \), we have the following relationship:
\begin{equation}
\tilde{\delta} = J_t^\top J_t w_t = J_t^\top d_t \approx \frac{1}{\alpha} \begin{bmatrix} \log \mathcal{L}_{\text{R}, t} - \log \mathcal{L}_{\text{R}, t+1} \\ \log \mathcal{L}_{\text{D}, t} - \log \mathcal{L}_{\text{D}, t+1} \end{bmatrix}.
\footnote{To prevent negative values in \( \log \mathcal{L}_{i,t} \), we add 1 to \( \mathcal{L}_{i,t} \) before applying the logarithm, ensuring that the minimum value in the log domain is zero.}
\end{equation}

To ensure that \( w_t \) remains within the interior point of simplex \( \mathbb{S}_2 \), we reparametrize \( w_t \) using \( \xi_t \):
\begin{equation}
w_t = \text{Softmax}(\xi_t),
\end{equation}
where \( \xi_t \in \mathbb{R}^2 \) represents the unconstrained softmax logits. To give more weight to recent updates, we add a decay term~\cite{zhou2022convergence,liu2024famo}, leading to the following update for \( \xi_t \):
\begin{equation}
\xi_{t+1} = \xi_t - \beta(\delta_t + \gamma \xi_t),
\end{equation}
where
\begin{equation}
\delta_t = \begin{bmatrix} \nabla^\top w_{\text{R},t}(\xi) \\ \nabla^\top w_{\text{D},t}(\xi) \end{bmatrix} \begin{bmatrix} \log \mathcal{L}_{\text{R}, t} - \log \mathcal{L}_{\text{R}, t+1} \\ \log \mathcal{L}_{\text{D}, t} - \log \mathcal{L}_{\text{D}, t+1} \end{bmatrix}.
\end{equation}

After computing the weights \( w_t \), we renormalize them to ensure numerical stability~\cite{liu2024famo}. This renormalization is crucial as our update direction is a convex combination of the gradients of the log losses:
\begin{equation}
d_t = w_{\text{R},t} \nabla \log \mathcal{L}_{\text{R},t} + w_{\text{D},t} \nabla \log \mathcal{L}_{\text{D},t} = \sum_{i \in \{\text{R}, \text{D}\}} \frac{w_{i,t}}{\mathcal{L}_{i,t}} \nabla \mathcal{L}_{i,t}.
\end{equation}
When \( \mathcal{L}_{i,t} \) becomes small, the multiplicative factor \( \frac{w_{i,t}}{\mathcal{L}_{i,t}} \) can grow large, potentially causing instability in the optimization. To mitigate this, we scale the gradient by a constant \( c_t \):
\begin{equation}
c_t = \left(\frac{w_{\text{R},t}}{\mathcal{L}_{\text{R},t}} + \frac{w_{\text{D},t}}{\mathcal{L}_{\text{D},t}}\right)^{-1}.
\end{equation}

The resulting balanced gradient, which is used to update the model parameters \( \theta \), is then given by:
\begin{equation}
\label{eq:sl1_d}
d_t = c_t \left(w_{\text{R},t} \nabla \log \mathcal{L}_{\text{R},t} + w_{\text{D},t} \nabla \log \mathcal{L}_{\text{D},t}\right).
\end{equation}

This method, referred to as Solution 1, is a coarse-to-fine gradient descent technique. It is particularly suited for training LIC models from scratch, as it incrementally balances rate and distortion optimization along the trajectory. The implementation is shown in Algorithm~\ref{alg:solution1}.


\begin{algorithm*}[t]
\caption{Balanced Rate-Distortion Optimization via Quadratic Programming}
\label{alg:solution2}
\begin{algorithmic}[1]
\REQUIRE Initial network parameters $\theta_0$, learning rate $\alpha$, total iterations $T$
\FOR{$t = 0$ \TO $T-1$}
    \STATE Compute losses $\mathcal{L}_{\text{R}, t} = \mathcal{L}_{\text{R}}(\theta_t)$ and $\mathcal{L}_{\text{D}, t} = \mathcal{L}_{\text{D}}(\theta_t)$
    \STATE Compute gradients: $\nabla_\theta \mathcal{L}_{\text{R}, t} = \frac{\partial \mathcal{L}_{\text{R}, t}}{\partial \theta_t}$, \quad $\nabla_\theta \mathcal{L}_{\text{D}, t} = \frac{\partial \mathcal{L}_{\text{D}, t}}{\partial \theta_t}$
    \STATE \hspace{2.1cm} $\nabla_\theta \log \mathcal{L}_{\text{R}, t} = \frac{\nabla_\theta \mathcal{L}_{\text{R}, t}}{\mathcal{L}_{\text{R}, t}}$, \quad $\nabla_\theta \log \mathcal{L}_{\text{D}, t} = \frac{\nabla_\theta \mathcal{L}_{\text{D}, t}}{\mathcal{L}_{\text{D}, t}}$
    \STATE Form matrix $J_t = \begin{bmatrix}  \nabla_\theta \log \mathcal{L}_{\text{R}, t} ^\top \\  \nabla_\theta \log \mathcal{L}_{\text{D}, t} ^\top \end{bmatrix}$
    \STATE Compute Hessian matrix: $Q = J_t^\top J_t = \begin{bmatrix} \| \nabla_\theta \log \mathcal{L}_{\text{R}, t} \|^2 & \langle \nabla_\theta \log \mathcal{L}_{\text{R}, t}, \nabla_\theta \log \mathcal{L}_{\text{D}, t} \rangle \\ \langle \nabla_\theta \log \mathcal{L}_{\text{R}, t}, \nabla_\theta \log \mathcal{L}_{\text{D}, t} \rangle & \| \nabla_\theta \log \mathcal{L}_{\text{D}, t} \|^2 \end{bmatrix}$
    \STATE Compute inverse $Q^{-1}$
    \STATE Compute weights: $\lambda = \frac{1}{\mathbf{1}^\top Q^{-1} \mathbf{1}}$, \quad $w_t = \lambda\, Q^{-1} \mathbf{1}$
    \STATE Apply softmax for numerical stability: $\tilde{w}_t = \text{Softmax}({w}_t)$
    \STATE Compute normalization constant: $c_t = \left( \frac{\tilde{w}_{\text{R}, t}}{\mathcal{L}_{\text{R}, t}} + \frac{\tilde{w}_{\text{D}, t}}{\mathcal{L}_{\text{D}, t}} \right)^{-1}$
    \STATE Compute balanced gradient: $\mathbf{d}_t = c_t \left( \tilde{w}_{\text{R}, t} \nabla_\theta \log \mathcal{L}_{\text{R}, t} + \tilde{w}_{\text{D}, t} \nabla_\theta \log \mathcal{L}_{\text{D}, t} \right)$
    \STATE Update network parameters: $\theta_{t+1} = \theta_t - \alpha\, \mathbf{d}_t$
\ENDFOR
\end{algorithmic}
\end{algorithm*}


\subsubsection{Solution 2: Quadratic programming}
Alternatively, we can formulate the weight optimization problem as a constrained-quadratic programming (QP) problem~\cite{nocedal1999numerical}:
\begin{equation}
\begin{aligned}
\min_{w_t} \quad & \frac{1}{2} \| J_t w_t \|^2 = \frac{1}{2} w_t^\top (J_t^\top J_t) w_t \\
\text{s.t.} \quad & w_{\text{R},t} + w_{\text{D},t} = 1, \\
& w_{\text{R},t}, w_{\text{D},t} \geq 0,
\end{aligned}
\end{equation}
where \( J_t = \begin{bmatrix} \nabla \log \mathcal{L}_{\text{R}, t}^\top \\ \nabla \log \mathcal{L}_{\text{D}, t}^\top \end{bmatrix} \) is an \( n \times 2 \) matrix containing the gradients of the log losses for rate and distortion.

Let \( Q = J_t^\top J_t \) represent the Hessian matrix. Since the gradients for rate and distortion typically reflect distinct objectives, they are rarely parallel, suggesting \( Q \) is positive definite~\cite{sener2018multi}. This property permits an analytical solution to the QP problem. The Hessian \( Q \) is given by:
\begin{equation}
Q = \begin{bmatrix} \|\nabla \log \mathcal{L}_{\text{R}, t}\|^2 & \langle \nabla \log \mathcal{L}_{\text{R}, t}, \nabla \log \mathcal{L}_{\text{D}, t} \rangle \\ \langle \nabla \log \mathcal{L}_{\text{R}, t}, \nabla \log \mathcal{L}_{\text{D}, t} \rangle & \|\nabla \log \mathcal{L}_{\text{D}, t}\|^2 \end{bmatrix}.
\end{equation}


% When the gradients are not parallel, the determinant of \( Q \) is positive, confirming that \( Q \) is positive definite.
To solve this QP problem, we introduce a Lagrange multiplier \( \lambda \) to enforce the equality constraint~\cite{boyd2004convex}, giving us the following Lagrangian:
\begin{equation}
L(w_t, \lambda) = \frac{1}{2} w_t^\top Q w_t - \lambda(w_{\text{R},t} + w_{\text{D},t} - 1).
\end{equation}

The weights \( w_t \) can then be obtained by applying the Karush-Kuhn-Tucker (KKT) conditions~\cite{mangasarian1994nonlinear}. The main KKT conditions for this problem are:

1. \textbf{Stationarity}: The gradient of the Lagrangian with respect to \( w_t \) must be zero,
   \begin{equation}
   \nabla_{w_t} L = Q w_t - \lambda \mathbf{1} = 0,
   \end{equation}
   where \( \mathbf{1} = [1, 1]^\top \). This yields \( w_t = \lambda Q^{-1} \mathbf{1} \).

2. \textbf{Primal Feasibility}: The weights must satisfy the equality constraint,
   \begin{equation}
   \mathbf{1}^\top w_t = 1.
   \end{equation}
Note: Neglecting dual feasibility and complementary slackness simplifies the solution process since non-negativity is guaranteed by a softmax projection applied subsequently.

By combining these conditions, we arrive at a simplified expression:
\begin{equation}
\mathbf{1}^\top w_t = \lambda \mathbf{1}^\top Q^{-1} \mathbf{1} = 1,
\end{equation}
which gives:
\begin{equation}
\lambda = \frac{1}{\mathbf{1}^\top Q^{-1} \mathbf{1}}, \quad w_t = \frac{Q^{-1} \mathbf{1}}{\mathbf{1}^\top Q^{-1} \mathbf{1}}.
\end{equation}


Since \( Q \) is positive definite, \( Q^{-1} \) exists, ensuring this solution is unique. To enforce non-negativity and enhance numerical stability, we then further project \( w_t \) onto the probability simplex \( \mathbb{S}_2 \) by simply using the softmax function:
\begin{equation}
\tilde{w}_t = \text{Softmax}\left({w}_t\right)=\text{Softmax}\left(\frac{Q^{-1} \mathbf{1}}{\mathbf{1}^\top Q^{-1} \mathbf{1}}\right).
\end{equation}
This step ensures \( \tilde{w}_t \in \mathbb{S}_2 \) and reduces potential numerical issues caused by large gradient variations.  It serves as an approximation that aligns with the optimal solution while maintaining similar performance despite slight deviations.


After determining the weights \( \tilde{w}_t \), we apply the same renormalization constant \( c_t \) as in Section~\ref{subsubsec:gd_over} to compute a balanced gradient for updating the model parameters \( \theta \), following the form in Eq.~\ref{eq:sl1_d}. Solution 2, with its analytical QP formulation, is particularly suitable for fine-tuning existing LIC models, offering a refined approach to balance rate and distortion objectives. The detailed implementation is presented in Algorithm~\ref{alg:solution2}.






% The detailed algorithms for both solutions are presented in Supplementary Material Section~\ref{sec:detailed_alg} as Algorithm~\ref{alg:solution1} and Algorithm~\ref{alg:solution2}.




\begin{figure*}[!th]
    \centering
    \subfigure[Sum rate achieved at an average SNR of 40\,dB.]{\includegraphics[width=\columnwidth]{Result/ICC_histo_Zero_shot_highSNR.pdf}}
    \subfigure[Sum rate achieved at an average SNR of 10\,dB.]{\includegraphics[width=\columnwidth]{Result/ICC_histo_Zero_shot_lowSNR.pdf}}
    \caption{Achievable sum rates for precoding methods under different SNR values at three Montreal sites: ``Ericsson'', ``Decarie'', and ``Sainte-Catherine''. All methods report zero-shot performance, except ``PaPP + FT,'' which is after 20 epochs of fine-tuning.}
    \label{fig:sidebyside}
\end{figure*}
\section{Baselines and Complexity Analysis} \label{Sec:Complexity}
\section{Computational complexity}\label{sec:compl}

In this section, we show that both of our problems are \np-hard.
We saw in the previous section that if
we set $\alpha  = \infty$, then \problemcdcsm reduces to \problemdts, which can be solved exactly in
polynomial time. However, \problemcdcsm is \np-hard when $\alpha = 0$.
%However, we show next that the complexity of fair densestsubgraph $\problemcdcsm$ which allows $0 < \alpha \leq 1$ is $\np$-hard.


\begin{proposition} 
\label{prop:np}
\problemcdcsm is \np-hard.
\end{proposition} 
\begin{proof}
We prove the hardness from $k$-\prbclique, a problem where, given a graph $H$, we are asked if there is a clique of size at least $k$.

Assume that we are given a graph $H = (V, E)$ with $n$ nodes, $n \geq k$. We set $\alpha = 0$.
The graph snapshot $G_1$ consists of the
graph $H$ and an additional set of $k$ singleton vertices $U$. $G_2$ consists of a $k$-clique connecting the vertices in $U$. 

We claim that there is a subset $S$ yielding $\dens{S, \calG} =  (k - 1)/2$ if and only if there is an $k$-clique in $H$.

Assume that  there is a subset $S$ yielding $\dens{S, \calG} =  (k - 1)/2$.
Since the value of objective is $(k - 1)/2$, we have $\dens{S, G_1} = \dens{S, G_2} = (k - 1)/4$. 
Let $S = W \cup T$ where $W \subseteq V$ and denotes the subset of vertices from $H$ and $T \subseteq U$ is the subset of vertices from  $U$ in $S$. 

Assume that $\abs{W} < \abs{T}$. Since $\abs{T} \leq k$, $\abs{W} < k$.  The density induced on $G_1$ is bounded by $\dens{S, G_1} \leq \frac{{\abs{W} \choose 2}}{\abs{T} + \abs{W}} < \frac{{\abs{W} \choose 2}}{2\abs{W}} < \frac{k - 1}{4}$, which is a contradiction.
Assume that  $\abs{W} > \abs{T}$. Then the density induced on $G_2$ is bounded by $\dens{S, G_2} = \frac{{\abs{T} \choose 2}}{\abs{T} + \abs{W}} < \frac{{\abs{T} \choose 2}}{2\abs{T}} = \frac{\abs{T} - 1}{4} \leq \frac{k - 1}{4}$, again a contradiction. Therefore, $\abs{W} =  \abs{T}$. 

Consequently, $\dens{S, G_2} = (\abs{T} - 1)/4$, implying that $\abs{T} = k$. Finally, $\frac{k - 1}{4} = \dens{S, G_1} = \frac{\abs{E(S)}}{2k}$ implies that $\abs{E} = {k \choose 2}$, that is, $W$ is a $k$-clique in $H$.

%Let  $d_1$ and $d_2$ are the densities induced on $G_1$ and $G_2$ respectively by $S$.
%Based on our assumption, $\dens{S, \calG} =  (k - 1)/2$ yields and 
%the only way to obtain a density value  of $f(S, G_2) = (k - 1)/4$  which satisfies marginal density constraint is that  $S = W \cup U$ and thus there should be an $k$-clique in $H$.
%Therefore, if the density is $  (k - 1)/4$ there is an $k$-clique in $H$.


On the other hand, assume there is a clique $C$ of size $k$ in $H$.
%Let clique $C$ is formed by the set of vertices in $W$.
% Let $d_1$ and $d_2$ are the optimal densities induced on $G_1$ and $G_2$.
% Since the density constraint should be satisfied $d_1 \geq (d_1 + d_2)/2$ and $d_2 \geq (d_1 + d_2)/2$ which implies $d_1 \geq d_2$ and $d_2 \geq d_1$. Therefore, $d_1 = d_2$.  
% Let $k_1$ and $k_2$ number of non-singleton nodes contribute for $d_1$ and $d_2$ respectively.
% Since $d_1 = d_2$,  $\frac{ k_1(k_1 - 1)}{ 2(k_1 + k_2)} = \frac{k_2 (k_2 - 1)}{ 2(k_1 + k_2)}$ which implies $k_1 =  k_2$. Then  the sum of the densities is given by $\frac{ (k_1 - 1)}{ 2} $ where the optimal is  when $k_1  = k =  k_2$.
Set $S = C \cup U$. 
Immediately, $\dens{S, \calG} =   (k - 1)/2$ proving the claim.
\qed
\end{proof}

\iffalse
\begin{proposition}
\label{prop:inapproximability}
\problemcdcsm does not have any polynomial time approximation algorithm with an approximation ratio better than $n^{1-\epsilon}$ for any constant $\epsilon > 0$, unless $\np = \zpp$.
\end{proposition}
\fi

A similar proof will show that \problemcdcsdiff is \np-hard, and inapproximable.

\begin{proposition} 
\label{prop:np2}
\problemcdcsdiff is \np-hard.
Unless $\poly = \np$, there is no polynomial-time algorithm with multiplicative approximation guarantee for \problemcdcsdiff. 
\end{proposition} 

\begin{proof}
We use the same reduction from $k$-\prbclique as in the proof of Proposition~\ref{prop:np}.
We also set $\sigma = (k - 1)/2$. If there is a clique $C$ in $H$, then selecting $S = C \cup U$ yields $\diff{S, \calG} = 0$.
On the other hand, if $\diff{S, \calG} = 0$, then $\dens{S, G_1} = \dens{S, G_2} = (k - 1)/4$, and the argument in the proof of Proposition~\ref{prop:np} shows that there must be a $k$-clique in $H$.
In summary, the difference $\diff{S, \calG} = 0$ for a solution $S$ if and only if there is a $k$-clique in $H$.

This also immediately implies that there is no polynomial-time algorithm with multiplicative approximation guarantee since this algorithm can be then used to test whether there is a set $S$ with $\diff{S, \calG} = 0$.
\qed
\end{proof}


\section{Numerical Results} 
\label{Sec:Simulation}
\section{Experimental Results}
In this section, we present the main results in~\secref{sec:main}, followed by ablation studies on key design choices in~\secref{sec:ablation}.

\begin{table*}[t]
\renewcommand\arraystretch{1.05}
\centering
\setlength{\tabcolsep}{2.5mm}{}
\begin{tabular}{l|l|c|cc|cc}
type & model     & \#params      & FID$\downarrow$ & IS$\uparrow$ & Precision$\uparrow$ & Recall$\uparrow$ \\
\shline
GAN& BigGAN~\cite{biggan} & 112M & 6.95  & 224.5       & 0.89 & 0.38     \\
GAN& GigaGAN~\cite{gigagan}  & 569M      & 3.45  & 225.5       & 0.84 & 0.61\\  
GAN& StyleGan-XL~\cite{stylegan-xl} & 166M & 2.30  & 265.1       & 0.78 & 0.53  \\
\hline
Diffusion& ADM~\cite{adm}    & 554M      & 10.94 & 101.0        & 0.69 & 0.63\\
Diffusion& LDM-4-G~\cite{ldm}   & 400M  & 3.60  & 247.7       & -  & -     \\
Diffusion & Simple-Diffusion~\cite{diff1} & 2B & 2.44 & 256.3 & - & - \\
Diffusion& DiT-XL/2~\cite{dit} & 675M     & 2.27  & 278.2       & 0.83 & 0.57     \\
Diffusion&L-DiT-3B~\cite{dit-github}  & 3.0B    & 2.10  & 304.4       & 0.82 & 0.60    \\
Diffusion&DiMR-G/2R~\cite{liu2024alleviating} &1.1B& 1.63& 292.5& 0.79 &0.63 \\
Diffusion & MDTv2-XL/2~\cite{gao2023mdtv2} & 676M & 1.58 & 314.7 & 0.79 & 0.65\\
Diffusion & CausalFusion-H$^\dag$~\cite{deng2024causal} & 1B & 1.57 & - & - & - \\
\hline
Flow-Matching & SiT-XL/2~\cite{sit} & 675M & 2.06 & 277.5 & 0.83 & 0.59 \\
Flow-Matching&REPA~\cite{yu2024representation} &675M& 1.80 & 284.0 &0.81 &0.61\\    
Flow-Matching&REPA$^\dag$~\cite{yu2024representation}& 675M& 1.42&  305.7& 0.80& 0.65 \\
\hline
Mask.& MaskGIT~\cite{maskgit}  & 227M   & 6.18  & 182.1        & 0.80 & 0.51 \\
Mask. & TiTok-S-128~\cite{yu2024image} & 287M & 1.97 & 281.8 & - & - \\
Mask. & MAGVIT-v2~\cite{yu2024language} & 307M & 1.78 & 319.4 & - & - \\ 
Mask. & MaskBit~\cite{weber2024maskbit} & 305M & 1.52 & 328.6 & - & - \\
\hline
AR& VQVAE-2~\cite{vqvae2} & 13.5B    & 31.11           & $\sim$45     & 0.36           & 0.57          \\
AR& VQGAN~\cite{vqgan}& 227M  & 18.65 & 80.4         & 0.78 & 0.26   \\
AR& VQGAN~\cite{vqgan}   & 1.4B     & 15.78 & 74.3   & -  & -     \\
AR&RQTran.~\cite{rq}     & 3.8B    & 7.55  & 134.0  & -  & -    \\
AR& ViTVQ~\cite{vit-vqgan} & 1.7B  & 4.17  & 175.1  & -  & -    \\
AR & DART-AR~\cite{gu2025dart} & 812M & 3.98 & 256.8 & - & - \\
AR & MonoFormer~\cite{zhao2024monoformer} & 1.1B & 2.57 & 272.6 & 0.84 & 0.56\\
AR & Open-MAGVIT2-XL~\cite{luo2024open} & 1.5B & 2.33 & 271.8 & 0.84 & 0.54\\
AR&LlamaGen-3B~\cite{llamagen}  &3.1B& 2.18& 263.3 &0.81& 0.58\\
AR & FlowAR-H~\cite{flowar} & 1.9B & 1.65 & 296.5 & 0.83 & 0.60\\
AR & RAR-XXL~\cite{yu2024randomized} & 1.5B & 1.48 & 326.0 & 0.80 & 0.63 \\
\hline
MAR & MAR-B~\cite{mar} & 208M & 2.31 &281.7 &0.82 &0.57 \\
MAR & MAR-L~\cite{mar} &479M& 1.78 &296.0& 0.81& 0.60 \\
MAR & MAR-H~\cite{mar} & 943M&1.55& 303.7& 0.81 &0.62 \\
\hline
VAR&VAR-$d16$~\cite{var}   & 310M  & 3.30& 274.4& 0.84& 0.51    \\
VAR&VAR-$d20$~\cite{var}   &600M & 2.57& 302.6& 0.83& 0.56     \\
VAR&VAR-$d30$~\cite{var}   & 2.0B      & 1.97  & 323.1 & 0.82 & 0.59      \\
\hline
\modelname& \modelname-B    &172M   &1.72&280.4&0.82&0.59 \\
\modelname& \modelname-L   & 608M   & 1.28& 292.5&0.82&0.62\\
\modelname& \modelname-H    & 1.1B    & 1.24 &301.6&0.83&0.64\\
\end{tabular}
\caption{
\textbf{Generation Results on ImageNet-256.}
Metrics include Fréchet Inception Distance (FID), Inception Score (IS), Precision, and Recall. $^\dag$ denotes the use of guidance interval sampling~\cite{guidance}. The proposed \modelname-H achieves a state-of-the-art 1.24 FID on the ImageNet-256 benchmark without relying on vision foundation models (\eg, DINOv2~\cite{dinov2}) or guidance interval sampling~\cite{guidance}, as used in REPA~\cite{yu2024representation}.
}\label{tab:256}
\end{table*}

\subsection{Main Results}
\label{sec:main}
We conduct experiments on ImageNet~\cite{deng2009imagenet} at 256$\times$256 and 512$\times$512 resolutions. Following prior works~\cite{dit,mar}, we evaluate model performance using FID~\cite{fid}, Inception Score (IS)~\cite{is}, Precision, and Recall. \modelname is trained with the same hyper-parameters as~\cite{mar,dit} (\eg, 800 training epochs), with model sizes ranging from 172M to 1.1B parameters. See Appendix~\secref{sec:sup_hyper} for hyper-parameter details.





\begin{table}[t]
    \centering
    \begin{tabular}{c|c|c|c}
      model    &  \#params & FID$\downarrow$ & IS$\uparrow$ \\
      \shline
      VQGAN~\cite{vqgan}&227M &26.52& 66.8\\
      BigGAN~\cite{biggan}& 158M&8.43 &177.9\\
      MaskGiT~\cite{maskgit}& 227M&7.32& 156.0\\
      DiT-XL/2~\cite{dit} &675M &3.04& 240.8 \\
     DiMR-XL/3R~\cite{liu2024alleviating}& 525M&2.89 &289.8 \\
     VAR-d36~\cite{var}  & 2.3B& 2.63 & 303.2\\
     REPA$^\ddagger$~\cite{yu2024representation}&675M &2.08& 274.6 \\
     \hline
     \modelname-L & 608M&1.70& 281.5 \\
    \end{tabular}
    \caption{
    \textbf{Generation Results on ImageNet-512.} $^\ddagger$ denotes the use of DINOv2~\cite{dinov2}.
    }
    \label{tab:512}
\end{table}

\noindent\textbf{ImageNet-256.}
In~\tabref{tab:256}, we compare \modelname with previous state-of-the-art generative models.
Out best variant, \modelname-H, achieves a new state-of-the-art-performance of 1.24 FID, outperforming the GAN-based StyleGAN-XL~\cite{stylegan-xl} by 1.06 FID, masked-prediction-based MaskBit~\cite{maskgit} by 0.28 FID, AR-based RAR~\cite{yu2024randomized} by 0.24 FID, VAR~\cite{var} by 0.73 FID, MAR~\cite{mar} by 0.31 FID, and flow-matching-based REPA~\cite{yu2024representation} by 0.18 FID.
Notably, \modelname does not rely on vision foundation models~\cite{dinov2} or guidance interval sampling~\cite{guidance}, both of which were used in REPA~\cite{yu2024representation}, the previous best-performing model.
Additionally, our lightweight \modelname-B (172M), surpasses DiT-XL (675M)~\cite{dit} by 0.55 FID while achieving an inference speed of 9.8 images per second—20$\times$ faster than DiT-XL (0.5 images per second). Detailed speed comparison can be found in Appendix \ref{sec:speed}.



\noindent\textbf{ImageNet-512.}
In~\tabref{tab:512}, we report the performance of \modelname on ImageNet-512.
Similarly, \modelname-L sets a new state-of-the-art FID of 1.70, outperforming the diffusion based DiT-XL/2~\cite{dit} and DiMR-XL/3R~\cite{liu2024alleviating} by a large margin of 1.34 and 1.19 FID, respectively.
Additionally, \modelname-L also surpasses the previous best autoregressive model VAR-d36~\cite{var} and flow-matching-based REPA~\cite{yu2024representation} by 0.93 and 0.38 FID, respectively.




\noindent\textbf{Qualitative Results.}
\figref{fig:qualitative} presents samples generated by \modelname (trained on ImageNet) at 512$\times$512 and 256$\times$256 resolutions. These results highlight \modelname's ability to produce high-fidelity images with exceptional visual quality.

\begin{figure*}
    \centering
    \vspace{-6pt}
    \includegraphics[width=1\linewidth]{figures/qualitative.pdf}
    \caption{\textbf{Generated Samples.} \modelname generates high-quality images at resolutions of 512$\times$512 (1st row) and 256$\times$256 (2nd and 3rd row).
    }
    \label{fig:qualitative}
\end{figure*}

\subsection{Ablation Studies}
\label{sec:ablation}
In this section, we conduct ablation studies using \modelname-B, trained for 400 epochs to efficiently iterate on model design.

\noindent\textbf{Prediction Entity X.}
The proposed \modelname extends next-token prediction to next-X prediction. In~\tabref{tab:X}, we evaluate different designs for the prediction entity X, including an individual patch token, a cell (a group of surrounding tokens), a subsample (a non-local grouping), a scale (coarse-to-fine resolution), and an entire image.

Among these variants, cell-based \modelname achieves the best performance, with an FID of 2.48, outperforming the token-based \modelname by 1.03 FID and surpassing the second best design (scale-based \modelname) by 0.42 FID. Furthermore, even when using standard prediction entities such as tokens, subsamples, images, or scales, \modelname consistently outperforms existing methods while requiring significantly fewer parameters. These results highlight the efficiency and effectiveness of \modelname across diverse prediction entities.






\begin{table}[]
    \centering
    \scalebox{0.92}{
    \begin{tabular}{c|c|c|c|c}
        model & \makecell[c]{prediction\\entity} & \#params & FID$\downarrow$ & IS$\uparrow$\\
        \shline
        LlamaGen-L~\cite{llamagen} & \multirow{2}{*}{token} & 343M & 3.80 &248.3\\
        \modelname-B& & 172M&3.51&251.4\\
        \hline
        PAR-L~\cite{par} & \multirow{2}{*}{subsample}& 343M & 3.76 & 218.9\\
        \modelname-B&  &172M& 3.58&231.5\\
        \hline
        DiT-L/2~\cite{dit}& \multirow{2}{*}{image}& 458M&5.02&167.2 \\
         \modelname-B& & 172M&3.13&253.4 \\
        \hline
        VAR-$d16$~\cite{var} & \multirow{2}{*}{scale} & 310M&3.30 &274.4\\
        \modelname-B& &172M&2.90&262.8\\
        \hline
        \baseline{\modelname-B}& \baseline{cell} & \baseline{172M}&\baseline{2.48}&\baseline{269.2} \\
    \end{tabular}
    }
    \caption{\textbf{Ablation on Prediction Entity X.} Using cells as the prediction entity outperforms alternatives such as tokens or entire images. Additionally, under the same prediction entity, \modelname surpasses previous methods, demonstrating its effectiveness across different prediction granularities. }%
    \label{tab:X}
\end{table}

\noindent\textbf{Cell Size.}
A prediction entity cell is formed by grouping spatially adjacent $k\times k$ tokens, where a larger cell size incorporates more tokens and thus captures a broader context within a single prediction step.
For a $256\times256$ input image, the encoded continuous latent representation has a spatial resolution of $16\times16$. Given this, the image can be partitioned into an $m\times m$ grid, where each cell consists of $k\times k$ neighboring tokens. As shown in~\tabref{tab:cell}, we evaluate different cell sizes with $k \in \{1,2,4,8,16\}$, where $k=1$ represents a single token and $k=16$ corresponds to the entire image as a single entity. We observe that performance improves as $k$ increases, peaking at an FID of 2.48 when using cell size $8\times8$ (\ie, $k=8$). Beyond this, performance declines, reaching an FID of 3.13 when the entire image is treated as a single entity.
These results suggest that using cells rather than the entire image as the prediction unit allows the model to condition on previously generated context, improving confidence in predictions while maintaining both rich semantics and local details.





\begin{table}[t]
    \centering
    \scalebox{0.98}{
    \begin{tabular}{c|c|c|c}
    cell size ($k\times k$ tokens) & $m\times m$ grid & FID$\downarrow$ & IS$\uparrow$ \\
       \shline
       $1\times1$ & $16\times16$ &3.51&251.4 \\
       $2\times2$ & $8\times8$ & 3.04& 253.5\\
       $4\times4$ & $4\times4$ & 2.61&258.2 \\
       \baseline{$8\times8$} & \baseline{$2\times2$} & \baseline{2.48} & \baseline{269.2}\\
       $16\times16$ & $1\times1$ & 3.13&253.4  \\
    \end{tabular}
    }
    \caption{\textbf{Ablation on the cell size.}
    In this study, a $16\times16$ continuous latent representation is partitioned into an $m\times m$ grid, where each cell consits of $k\times k$ neighboring tokens.
    A cell size of $8\times8$ achieves the best performance, striking an optimal balance between local structure and global context.
    }
    \label{tab:cell}
\end{table}



\begin{table}[t]
    \centering
    \scalebox{0.95}{
    \begin{tabular}{c|c|c|c}
      previous cell & noise time step &  FID$\downarrow$ & IS$\uparrow$ \\
       \shline
       clean & $t_i=0, \forall i<n$& 3.45& 243.5\\
       increasing noise & $t_1<t_2<\cdots<t_{n-1}$& 2.95&258.8 \\
       decreasing noise & $t_1>t_2>\cdots>t_{n-1}$&2.78 &262.1 \\
      \baseline{random noise}  & \baseline{no constraint} &\baseline{2.48} & \baseline{269.2}\\
    \end{tabular}
    }
    \caption{
    \textbf{Ablation on Noisy Context Learning.}
    This study examines the impact of noise time steps ($t_1, \cdots, t_{n-1} \subset [0, 1]$) in previous entities ($t=0$ represents pure Gaussian noise).
    Conditioning on all clean entities (the ``clean'' variant) results in suboptimal performance.
    Imposing an order on noise time steps, either ``increasing noise'' or ``decreasing noise'', also leads to inferior results. The best performance is achieved with the "random noise" setting, where no constraints are imposed on noise time steps.
    }
    \label{tab:ncl}
\end{table}


\noindent\textbf{Noisy Context Learning.}
During training, \modelname employs Noisy Context Learning (NCL), predicting $X_n$ by conditioning on all previous noisy entities, unlike Teacher Forcing.
The noise intensity of previous entities is contorlled by noise time steps $\{t_1, \dots, t_{n-1}\} \subset [0, 1]$, where $t=0$ corresponds to pure Gaussian noise.
We analyze the impact of NCL in~\tabref{tab:ncl}.
When conditioning on all clean entities (\ie, the ``clean'' variant, where $t_i=0, \forall i<n$), which is equivalent to vanilla AR (\ie, Teacher Forcing), the suboptimal performance is obtained.
We also evaluate two constrained noise schedules: the ``increasing noise'' variant, where noise time steps increase over AR steps ($t_1<t_2< \cdots < t_{n-1}$), and the `` decreasing noise'' variant, where noise time steps decrease ($t_1>t_2> \cdots > t_{n-1}$).
While both settings improve over the ``clean'' variant, they remain inferior to our final ``random noise'' setting, where no constraints are imposed on noise time steps, leading to the best performance.




        


\section{Conclusion} \label{Sec:conclusion}
In this paper, a novel \gls{DL}-based precoding method approach named PaPP is proposed for mMIMO systems to overcome the limitations of traditional methods in terms of complexity and generalizability. By combining WMMSE with DL techniques, we designed a computationally efficient precoder that leverages MLDG and a teacher-student training framework for improved adaptability. Our results demonstrate high sum-rate performance across diverse urban environments while achieving approximately 3.5$\times$ lower computational complexity than alternative generalizable DNN approaches (MAML-CNN).  Furthermore, when deployed in a new site without fine-tuning, PaPP achieves roughly the same performance as a model trained directly on the deployment site. With fine-tuning, PaPP outperforms all other methods. 

\section*{Acknowledgement}
This work was supported by Ericsson - Global Artificial Intelligence Accelerator AI-Hub Canada in Montr\'{e}al and jointly funded by NSERC Alliance Grant 566589-21 (Ericsson, ECCC, Innov\'{E}\'{E}).

\bibliographystyle{IEEEtran}
\bibliography{0_main}


\end{document}