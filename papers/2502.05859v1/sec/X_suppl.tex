\clearpage
\setcounter{page}{1}
\maketitlesupplementary

\section{Panorama Projections}

To capture the texture features and avoid distortion and discontinuity, SphereFusion relies on two panorama projections: the equirectangular projection and the spherical mesh. In this section, we describe the details of their conversion, and how to convert the panorama image in spherical mesh to equirectangular for depth map evaluation orre-project it to 3D space for visualization.


\paragraph{The E2S}

Based on the definition of the equirectangular projection and the spherical mesh, we define the E2S module, which converts a panorama image from the equirectangular projection to the spherical projection. Given a triangle center $(x, y, z)$ from the spherical mesh, we first calculate its position on the image plane $(u, v)$ by Eq. \ref{eq:e2s}, and then use a bi-linear sample to capture value on the image plane, finally assign corresponding value to the triangle.

\begin{align}
	\label{eq:e2s}
	\begin{cases}
        u = (1+atan(y,x)/\pi) \times W/2  \\
        v = (0.5+atan(z,\sqrt{x^2+y^2})/\pi ) \times H
	\end{cases}
\end{align}

\paragraph{The S2E}

Meanwhile, we can also define the S2E module, which converts a panorama image from the spherical projection to the equirectangular projection. Given a pixel $(u, v)$ on the image plane, we first calculate its position on the sphere surface, then calculate its 3D position to find out the closest triangle on the spherical mesh, and finally assign the value of the triangle to the pixel.

\paragraph{Projection To 3D Point Cloud}

For better comparison, we visualize point clouds using different methods. Although we name the result obtained from the panorama depth estimation as the depth map, it represents the distance map, which means the Euclidean distance between a 3D point and the camera center. Therefore, we can reproject a panorama depth map to 3D space by Eq. \ref{eq:to_point} after converting it into the sphere coordinates, where $d$ is the distance.

\begin{align}
	\label{eq:to_point}
	\begin{cases}
		X = cos(latitude)cos(longitude) \times d\\
		Y = cos(latitude)sin(longitude) \times d\\ 
		Z = sin(latitude) \times d\\
	\end{cases}
\end{align}

\section{Evaluation}

To compare with other methods, we convert our depth map in the spherical domain to equirectangular projection. Following BiFuse \cite{wang2020bifuse}, we use five evaluation metrics, including MAE, MRE, RMSE, RMSE(log), and $\delta ^n$. Eq. \ref{eq:metrics} shows how to calculate them, where $gt$ is the ground truth, $pr$ is the predicted depth, $V$ is valid pixels, and $N$ is the number of valid pixels. For MAE, MRE, RMSE, and RMSE(log), the smaller is better. For the $\delta$, the bigger is better. During the evaluation, we set the depth range to $0.1 \sim 10$ meters.

\begin{equation} 
\label{eq:metrics}
\begin{split}
MAE &= \sum_{i\in V}{|gt_i-pr_i|}   \\
MRE &= \sum_{i\in V}{\frac{|gt_i-pr_i|}{gt_i}} \\
RMSE &= \sqrt{\frac{\sum_{i\in V}(gt_i-pr_i)^2}{N}} \\
RMSE_{log} & = \sqrt{\frac{\sum_{i\in V}(log_{10}(gt_i)-log_{10}(pr_i))^2}{N}} \\
\delta^n & = \frac{\sum_{i\in V}max(\frac{gt_i}{pr_i},\frac{pr_i}{gt_i})<1.25^n}{N}
\end{split}
\end{equation}


\section{Visulization}

In this section, we add more visualization results on three datasets and compare SphereFusion (ours) with state-of-the-art methods. 
On 360D \cite{zioulis2018omnidepth}, we compare our method with SphereDepth \cite{yan2022spheredepth} and UniFuse \cite{jiang2021unifuse}. Figure \ref{fig:3d60_depth} and Figure \ref{fig:3d60_cloud} show depth maps and point clouds generated by different methods. Our method uses texture features extracted by the 2D encoder to enhance the mesh encoder in the spherical domain and combines the strengths of two encoders. Results show that our method captures more features from the scene and can reconstruct more details in the scene, such as doors, tables, and walls. 
On Matterport3D \cite{chang2017matterport3d} and Stanford2D3D \cite{armeni2017joint}, we compare our method with OmniFusion \cite{yan2022spheredepth} and PanoFormer \cite{jiang2021unifuse}. Figure \ref{fig:mat3d_depth} and Figure \ref{fig:mat3d_cloud} show depth maps generated by different methods on Matterport3D and corresponding point clouds. Figure \ref{fig:2d3d_depth} and Figure \ref{fig:2d3d_cloud} show results generated by different methods on Stanford2D3D. Compared with tangent patches, our method directly estimates the panorama depth in the spherical domain and does not need any special mechanism to fuse these patches. The results of OmniFusion have obvious patch gaps. Meanwhile, the results of PanoFormer are smoother but also lose some details. Our method reconstructs more details but suffers from imperfect ground truth \cite{jiang2021unifuse}. Depth maps and point clouds show that our method achieves competitive results with state-of-the-art methods with a lighter network and higher efficiency.



%%%%%%%%%%%%%%%%%%%%%%%%%%%%%%%%%%%%%%%%%%%%%%%%%%%%%%%%%%%%%%%%%
\begin{figure*}[t]
	\centering
	\captionsetup[subfigure]{labelformat=empty}
	
	\begin{subfigure}{0.18\linewidth}
		\includegraphics[width=.98\linewidth]{fig_sub/360D/12_rgb.png}
	\end{subfigure}
	\begin{subfigure}{0.18\linewidth}
		\includegraphics[width=.98\linewidth]{fig_sub/360D/12_gt.png}
	\end{subfigure}
	\begin{subfigure}{0.18\linewidth}
		\includegraphics[width=.98\linewidth]{fig_sub/360D/pano_depth/12_pr.png}
	\end{subfigure}
	\begin{subfigure}{0.18\linewidth}
		\includegraphics[width=.98\linewidth]{fig_sub/360D/unifuse/4_pr.png}
	\end{subfigure}
	\begin{subfigure}{0.18\linewidth}
		\includegraphics[width=.98\linewidth]{fig_sub/360D/pano_fusion/12_pr.png}
	\end{subfigure}

	\vspace{1pt}

	\begin{subfigure}{0.18\linewidth}
		\includegraphics[width=.98\linewidth]{fig_sub/360D/18_rgb.png}
	\end{subfigure}
	\begin{subfigure}{0.18\linewidth}
		\includegraphics[width=.98\linewidth]{fig_sub/360D/18_gt.png}
	\end{subfigure}
	\begin{subfigure}{0.18\linewidth}
		\includegraphics[width=.98\linewidth]{fig_sub/360D/pano_depth/18_pr.png}
	\end{subfigure}
	\begin{subfigure}{0.18\linewidth}
		\includegraphics[width=.98\linewidth]{fig_sub/360D/unifuse/6_pr.png}
	\end{subfigure}
	\begin{subfigure}{0.18\linewidth}
		\includegraphics[width=.98\linewidth]{fig_sub/360D/pano_fusion/18_pr.png}
	\end{subfigure}
	
	\vspace{1pt}

	\begin{subfigure}{0.18\linewidth}
		\includegraphics[width=.98\linewidth]{fig_sub/360D/45_rgb.png}
	\end{subfigure}
	\begin{subfigure}{0.18\linewidth}
		\includegraphics[width=.98\linewidth]{fig_sub/360D/45_gt.png}
	\end{subfigure}
	\begin{subfigure}{0.18\linewidth}
		\includegraphics[width=.98\linewidth]{fig_sub/360D/pano_depth/45_pr.png}
	\end{subfigure}
	\begin{subfigure}{0.18\linewidth}
		\includegraphics[width=.98\linewidth]{fig_sub/360D/unifuse/15_pr.png}
	\end{subfigure}
	\begin{subfigure}{0.18\linewidth}
		\includegraphics[width=.98\linewidth]{fig_sub/360D/pano_fusion/45_pr.png}
	\end{subfigure}
	
	\vspace{1pt}

	\begin{subfigure}{0.18\linewidth}
		\includegraphics[width=.98\linewidth]{fig_sub/360D/51_rgb.png}
		\caption{RGB}
	\end{subfigure}
	\begin{subfigure}{0.18\linewidth}
		\includegraphics[width=.98\linewidth]{fig_sub/360D/51_gt.png}
		\caption{GT}
	\end{subfigure}
	\begin{subfigure}{0.18\linewidth}
		\includegraphics[width=.98\linewidth]{fig_sub/360D/pano_depth/51_pr.png}
		\caption{SphereDepth \cite{yan2022spheredepth}}
	\end{subfigure}
	\begin{subfigure}{0.18\linewidth}
		\includegraphics[width=.98\linewidth]{fig_sub/360D/unifuse/17_pr.png}
		\caption{UniFuse \cite{jiang2021unifuse}}
	\end{subfigure}
	\begin{subfigure}{0.18\linewidth}
		\includegraphics[width=.98\linewidth]{fig_sub/360D/pano_fusion/51_pr.png}
		\caption{SphereFusion (ours)}
	\end{subfigure}

	
	\caption{
		\textbf{Depth Maps of 360D.} Invalid parts of the depth map are set to red. 
	}
	\label{fig:3d60_depth}
    \vspace{-1.0em}
\end{figure*}
%%%%%%%%%%%%%%%%%%%%%%%%%%%%%%%%%%%%%%%%%%%%%%%%%%%%%%%%%%%%%%%%%

\FloatBarrier

%%%%%%%%%%%%%%%%%%%%%%%%%%%%%%%%%%%%%%%%%%%%%%%%%%%%%%%%%%%%%%%%%
\begin{figure*}[t]
	\centering
	\captionsetup[subfigure]{labelformat=empty}
	
	
	\begin{subfigure}{0.3\linewidth}
		\includegraphics[width=.98\linewidth]{fig_sub/360D/pano_depth/12_depth00.png}
	\end{subfigure}
	\begin{subfigure}{0.3\linewidth}
		\includegraphics[width=.98\linewidth]{fig_sub/360D/unifuse/4_unifuse00.png}
	\end{subfigure}
	\begin{subfigure}{0.3\linewidth}
		\includegraphics[width=.98\linewidth]{fig_sub/360D/pano_fusion/12_fusion00.png}
	\end{subfigure}
	
	\vspace{1pt}
	
	\begin{subfigure}{0.3\linewidth}
		\includegraphics[width=.98\linewidth]{fig_sub/360D/pano_depth/18_depth00.png}
	\end{subfigure}
	\begin{subfigure}{0.3\linewidth}
		\includegraphics[width=.98\linewidth]{fig_sub/360D/unifuse/6_unifuse00.png}
	\end{subfigure}
	\begin{subfigure}{0.3\linewidth}
		\includegraphics[width=.98\linewidth]{fig_sub/360D/pano_fusion/18_fusion00.png}
	\end{subfigure}
	
	\vspace{1pt}

	
	\begin{subfigure}{0.3\linewidth}
		\includegraphics[width=.98\linewidth]{fig_sub/360D/pano_depth/45_depth00.png}
	\end{subfigure}
	\begin{subfigure}{0.3\linewidth}
		\includegraphics[width=.98\linewidth]{fig_sub/360D/unifuse/15_unifuse00.png}
	\end{subfigure}
	\begin{subfigure}{0.3\linewidth}
		\includegraphics[width=.98\linewidth]{fig_sub/360D/pano_fusion/45_fusion00.png}
	\end{subfigure}
	
	\vspace{1pt}

	\begin{subfigure}{0.3\linewidth}
		\includegraphics[width=.98\linewidth]{fig_sub/360D/pano_depth/51_depth00.png}
		\caption{SphereDepth \cite{yan2022spheredepth}}
	\end{subfigure}
	\begin{subfigure}{0.3\linewidth}
		\includegraphics[width=.98\linewidth]{fig_sub/360D/unifuse/17_unifuse00.png}
		\caption{UniFuse \cite{jiang2021unifuse}}
	\end{subfigure}
	\begin{subfigure}{0.3\linewidth}
		\includegraphics[width=.98\linewidth]{fig_sub/360D/pano_fusion/51_fusion00.png}
		\caption{SphereFusion (ours)}
	\end{subfigure}	
	
	
	\caption{
		\textbf{Point Clouds of 360D.} Our method reconstructs more details of the scene.
	}
	\label{fig:3d60_cloud}
    \vspace{-1.0em}
\end{figure*}
%%%%%%%%%%%%%%%%%%%%%%%%%%%%%%%%%%%%%%%%%%%%%%%%%%%%%%%%%%%%%%%%%

\FloatBarrier

%%%%%%%%%%%%%%%%%%%%%%%%%%%%%%%%%%%%%%%%%%%%%%%%%%%%%%%%%%%%%%%%%
\begin{figure*}[t]
	\centering
	\captionsetup[subfigure]{labelformat=empty}
	
	\begin{subfigure}{0.18\linewidth}
		\includegraphics[width=.98\linewidth]{fig_sub/Mat3D/2_rgb.png}
	\end{subfigure}
	\begin{subfigure}{0.18\linewidth}
		\includegraphics[width=.98\linewidth]{fig_sub/Mat3D/2_gt.png}
	\end{subfigure}
	\begin{subfigure}{0.18\linewidth}
		\includegraphics[width=.98\linewidth]{fig_sub/Mat3D/omnifusion/2_pr.png}
	\end{subfigure}
	\begin{subfigure}{0.18\linewidth}
		\includegraphics[width=.98\linewidth]{fig_sub/Mat3D/panoformer/2_pr.png}
	\end{subfigure}
	\begin{subfigure}{0.18\linewidth}
		\includegraphics[width=.98\linewidth]{fig_sub/Mat3D/pano_fusion/2_pr.png}
	\end{subfigure}
	
	\vspace{1pt}
	
	
	\begin{subfigure}{0.18\linewidth}
		\includegraphics[width=.98\linewidth]{fig_sub/Mat3D/14_rgb.png}
	\end{subfigure}
	\begin{subfigure}{0.18\linewidth}
		\includegraphics[width=.98\linewidth]{fig_sub/Mat3D/14_gt.png}
	\end{subfigure}
	\begin{subfigure}{0.18\linewidth}
		\includegraphics[width=.98\linewidth]{fig_sub/Mat3D/omnifusion/14_pr.png}
	\end{subfigure}
	\begin{subfigure}{0.18\linewidth}
		\includegraphics[width=.98\linewidth]{fig_sub/Mat3D/panoformer/14_pr.png}
	\end{subfigure}
	\begin{subfigure}{0.18\linewidth}
		\includegraphics[width=.98\linewidth]{fig_sub/Mat3D/pano_fusion/14_pr.png}
	\end{subfigure}
	
	\vspace{1pt}
	
	\begin{subfigure}{0.18\linewidth}
		\includegraphics[width=.98\linewidth]{fig_sub/Mat3D/26_rgb.png}
	\end{subfigure}
	\begin{subfigure}{0.18\linewidth}
		\includegraphics[width=.98\linewidth]{fig_sub/Mat3D/26_gt.png}
	\end{subfigure}
	\begin{subfigure}{0.18\linewidth}
		\includegraphics[width=.98\linewidth]{fig_sub/Mat3D/omnifusion/26_pr.png}
	\end{subfigure}
	\begin{subfigure}{0.18\linewidth}
		\includegraphics[width=.98\linewidth]{fig_sub/Mat3D/panoformer/26_pr.png}
	\end{subfigure}
	\begin{subfigure}{0.18\linewidth}
		\includegraphics[width=.98\linewidth]{fig_sub/Mat3D/pano_fusion/26_pr.png}
	\end{subfigure}
	
	\vspace{1pt}
	
	\begin{subfigure}{0.18\linewidth}
		\includegraphics[width=.98\linewidth]{fig_sub/Mat3D/46_rgb.png}
		\caption{RGB}
	\end{subfigure}
	\begin{subfigure}{0.18\linewidth}
		\includegraphics[width=.98\linewidth]{fig_sub/Mat3D/46_gt.png}
		\caption{GT}
	\end{subfigure}
	\begin{subfigure}{0.18\linewidth}
		\includegraphics[width=.98\linewidth]{fig_sub/Mat3D/omnifusion/46_pr.png}
		\caption{OmniFusion \cite{li2022omnifusion}}
	\end{subfigure}
	\begin{subfigure}{0.18\linewidth}
		\includegraphics[width=.98\linewidth]{fig_sub/Mat3D/panoformer/46_pr.png}
		\caption{PanoFormer \cite{shen2022panoformer}}
	\end{subfigure}
	\begin{subfigure}{0.18\linewidth}
		\includegraphics[width=.98\linewidth]{fig_sub/Mat3D/pano_fusion/46_pr.png}
		\caption{SphereFusion (ours)}
	\end{subfigure}
	
	
	\caption{
		\textbf{Depth Maps of Matterport3D.} Invalid parts of the depth map are set to red. 
	}
	\label{fig:mat3d_depth}
    \vspace{-1.0em}
\end{figure*}
%%%%%%%%%%%%%%%%%%%%%%%%%%%%%%%%%%%%%%%%%%%%%%%%%%%%%%%%%%%%%%%%%


%%%%%%%%%%%%%%%%%%%%%%%%%%%%%%%%%%%%%%%%%%%%%%%%%%%%%%%%%%%%%%%%%
\begin{figure*}[t]
	\centering
	\captionsetup[subfigure]{labelformat=empty}
	
	
	\begin{subfigure}{0.3\linewidth}
		\includegraphics[width=.98\linewidth]{fig_sub/Mat3D/omnifusion/2_omni00.png}
	\end{subfigure}
	\begin{subfigure}{0.3\linewidth}
		\includegraphics[width=.98\linewidth]{fig_sub/Mat3D/panoformer/2_former00.png}
	\end{subfigure}
	\begin{subfigure}{0.3\linewidth}
		\includegraphics[width=.98\linewidth]{fig_sub/Mat3D/pano_fusion/2_fusion00.png}
	\end{subfigure}
	
	\vspace{1pt}
		
	
	\begin{subfigure}{0.3\linewidth}
		\includegraphics[width=.98\linewidth]{fig_sub/Mat3D/omnifusion/14_omni00.png}
	\end{subfigure}
	\begin{subfigure}{0.3\linewidth}
		\includegraphics[width=.98\linewidth]{fig_sub/Mat3D/panoformer/14_former00.png}
	\end{subfigure}
	\begin{subfigure}{0.3\linewidth}
		\includegraphics[width=.98\linewidth]{fig_sub/Mat3D/pano_fusion/14_fusion00.png}
	\end{subfigure}

	\vspace{1pt}
	
	\begin{subfigure}{0.3\linewidth}
		\includegraphics[width=.98\linewidth]{fig_sub/Mat3D/omnifusion/26_omni00.png}
	\end{subfigure}
	\begin{subfigure}{0.3\linewidth}
		\includegraphics[width=.98\linewidth]{fig_sub/Mat3D/panoformer/26_former00.png}
	\end{subfigure}
	\begin{subfigure}{0.3\linewidth}
		\includegraphics[width=.98\linewidth]{fig_sub/Mat3D/pano_fusion/26_fusion00.png}
	\end{subfigure}
	
	
	\vspace{1pt}
	
	\begin{subfigure}{0.3\linewidth}
		\includegraphics[width=.98\linewidth]{fig_sub/Mat3D/omnifusion/46_omni00.png}
		\caption{OmniFusion \cite{li2022omnifusion}}
	\end{subfigure}
	\begin{subfigure}{0.3\linewidth}
		\includegraphics[width=.98\linewidth]{fig_sub/Mat3D/panoformer/46_former00.png}
		\caption{PanoFormer \cite{shen2022panoformer}}
	\end{subfigure}
	\begin{subfigure}{0.3\linewidth}
		\includegraphics[width=.98\linewidth]{fig_sub/Mat3D/pano_fusion/46_fusion00.png}
		\caption{SphereFusion (ours)}
	\end{subfigure}	
	
	
	\caption{
		\textbf{Depth Maps of Matterport3D.} Our method has less noise and maintains the structure of the scene.
	}
	\label{fig:mat3d_cloud}
    \vspace{-1.0em}
\end{figure*}
%%%%%%%%%%%%%%%%%%%%%%%%%%%%%%%%%%%%%%%%%%%%%%%%%%%%%%%%%%%%%%%%%




%%%%%%%%%%%%%%%%%%%%%%%%%%%%%%%%%%%%%%%%%%%%%%%%%%%%%%%%%%%%%%%%%
\begin{figure*}[t]
	\centering
	\captionsetup[subfigure]{labelformat=empty}
	
	\begin{subfigure}{0.18\linewidth}
		\includegraphics[width=.98\linewidth]{fig_sub/2D3D/3_rgb.png}
	\end{subfigure}
	\begin{subfigure}{0.18\linewidth}
		\includegraphics[width=.98\linewidth]{fig_sub/2D3D/3_gt.png}
	\end{subfigure}
	\begin{subfigure}{0.18\linewidth}
		\includegraphics[width=.98\linewidth]{fig_sub/2D3D/omnifusion/3_pr.png}
	\end{subfigure}
	\begin{subfigure}{0.18\linewidth}
		\includegraphics[width=.98\linewidth]{fig_sub/2D3D/panoformer/3_pr.png}
	\end{subfigure}
	\begin{subfigure}{0.18\linewidth}
		\includegraphics[width=.98\linewidth]{fig_sub/2D3D/pano_fusion/3_pr.png}
	\end{subfigure}
	
	\vspace{1pt}
	
	
	\begin{subfigure}{0.18\linewidth}
		\includegraphics[width=.98\linewidth]{fig_sub/2D3D/4_rgb.png}
	\end{subfigure}
	\begin{subfigure}{0.18\linewidth}
		\includegraphics[width=.98\linewidth]{fig_sub/2D3D/4_gt.png}
	\end{subfigure}
	\begin{subfigure}{0.18\linewidth}
		\includegraphics[width=.98\linewidth]{fig_sub/2D3D/omnifusion/4_pr.png}
	\end{subfigure}
	\begin{subfigure}{0.18\linewidth}
		\includegraphics[width=.98\linewidth]{fig_sub/2D3D/panoformer/4_pr.png}
	\end{subfigure}
	\begin{subfigure}{0.18\linewidth}
		\includegraphics[width=.98\linewidth]{fig_sub/2D3D/pano_fusion/4_pr.png}
	\end{subfigure}
	
	\vspace{1pt}
	
	\begin{subfigure}{0.18\linewidth}
		\includegraphics[width=.98\linewidth]{fig_sub/2D3D/7_rgb.png}
	\end{subfigure}
	\begin{subfigure}{0.18\linewidth}
		\includegraphics[width=.98\linewidth]{fig_sub/2D3D/7_gt.png}
	\end{subfigure}
	\begin{subfigure}{0.18\linewidth}
		\includegraphics[width=.98\linewidth]{fig_sub/2D3D/omnifusion/7_pr.png}
	\end{subfigure}
	\begin{subfigure}{0.18\linewidth}
		\includegraphics[width=.98\linewidth]{fig_sub/2D3D/panoformer/7_pr.png}
	\end{subfigure}
	\begin{subfigure}{0.18\linewidth}
		\includegraphics[width=.98\linewidth]{fig_sub/2D3D/pano_fusion/7_pr.png}
	\end{subfigure}
	
	\vspace{1pt}
	
	\begin{subfigure}{0.18\linewidth}
		\includegraphics[width=.98\linewidth]{fig_sub/2D3D/12_rgb.png}
		\caption{RGB}
	\end{subfigure}
	\begin{subfigure}{0.18\linewidth}
		\includegraphics[width=.98\linewidth]{fig_sub/2D3D/12_gt.png}
		\caption{GT}
	\end{subfigure}
	\begin{subfigure}{0.18\linewidth}
		\includegraphics[width=.98\linewidth]{fig_sub/2D3D/omnifusion/12_pr.png}
		\caption{OmniFusion \cite{li2022omnifusion}}
	\end{subfigure}
	\begin{subfigure}{0.18\linewidth}
		\includegraphics[width=.98\linewidth]{fig_sub/2D3D/panoformer/12_pr.png}
		\caption{PanoFormer \cite{shen2022panoformer}}
	\end{subfigure}
	\begin{subfigure}{0.18\linewidth}
		\includegraphics[width=.98\linewidth]{fig_sub/2D3D/pano_fusion/12_pr.png}
		\caption{SphereFusion (ours)}
	\end{subfigure}
	
	
	\caption{
		\textbf{Depth Maps of Stanford2D3D.} Invalid parts of the depth map are set to red. 
	}
	\label{fig:2d3d_depth}
    \vspace{-1.0em}
\end{figure*}
%%%%%%%%%%%%%%%%%%%%%%%%%%%%%%%%%%%%%%%%%%%%%%%%%%%%%%%%%%%%%%%%%


%%%%%%%%%%%%%%%%%%%%%%%%%%%%%%%%%%%%%%%%%%%%%%%%%%%%%%%%%%%%%%%%%
\begin{figure*}[t]
	\centering
	\captionsetup[subfigure]{labelformat=empty}
	
	
	\begin{subfigure}{0.3\linewidth}
		\includegraphics[width=.98\linewidth]{fig_sub/2D3D/omnifusion/3_omni00.png}
	\end{subfigure}
	\begin{subfigure}{0.3\linewidth}
		\includegraphics[width=.98\linewidth]{fig_sub/2D3D/panoformer/3_former00.png}
	\end{subfigure}
	\begin{subfigure}{0.3\linewidth}
		\includegraphics[width=.98\linewidth]{fig_sub/2D3D/pano_fusion/3_fusion00.png}
	\end{subfigure}
	
	\vspace{1pt}
	
	
	\begin{subfigure}{0.3\linewidth}
		\includegraphics[width=.98\linewidth]{fig_sub/2D3D/omnifusion/4_omni00.png}
	\end{subfigure}
	\begin{subfigure}{0.3\linewidth}
		\includegraphics[width=.98\linewidth]{fig_sub/2D3D/panoformer/4_former00.png}
	\end{subfigure}
	\begin{subfigure}{0.3\linewidth}
		\includegraphics[width=.98\linewidth]{fig_sub/2D3D/pano_fusion/4_fusion00.png}
	\end{subfigure}
	
	\vspace{1pt}
	
	\begin{subfigure}{0.3\linewidth}
		\includegraphics[width=.98\linewidth]{fig_sub/2D3D/omnifusion/7_omni00.png}
	\end{subfigure}
	\begin{subfigure}{0.3\linewidth}
		\includegraphics[width=.98\linewidth]{fig_sub/2D3D/panoformer/7_former00.png}
	\end{subfigure}
	\begin{subfigure}{0.3\linewidth}
		\includegraphics[width=.98\linewidth]{fig_sub/2D3D/pano_fusion/7_fusion00.png}
	\end{subfigure}
	
	
	\vspace{1pt}
	
	\begin{subfigure}{0.3\linewidth}
		\includegraphics[width=.98\linewidth]{fig_sub/2D3D/omnifusion/12_omni00.png}
		\caption{OmniFusion \cite{li2022omnifusion}}
	\end{subfigure}
	\begin{subfigure}{0.3\linewidth}
		\includegraphics[width=.98\linewidth]{fig_sub/2D3D/panoformer/12_former00.png}
		\caption{PanoFormer \cite{shen2022panoformer}}
	\end{subfigure}
	\begin{subfigure}{0.3\linewidth}
		\includegraphics[width=.98\linewidth]{fig_sub/2D3D/pano_fusion/12_fusion00.png}
		\caption{SphereFusion (ours)}
	\end{subfigure}	
	
	
	\caption{
		\textbf{Depth Maps of Stanford2D3D.} Our method does not suffer from discontinuity.
	}
	\label{fig:2d3d_cloud}
    \vspace{-1.0em}
\end{figure*}
%%%%%%%%%%%%%%%%%%%%%%%%%%%%%%%%%%%%%%%%%%%%%%%%%%%%%%%%%%%%%%%%%