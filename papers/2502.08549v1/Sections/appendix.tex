\newpage

\renewcommand\thefigure{A\arabic{figure}}    
\setcounter{figure}{0}  
\renewcommand\thetable{A\arabic{table}}    
\setcounter{table}{0}  

\section*{Appendix A: \blueCN{Marginals} and copulas used in this study}
\label{sec:Margins and copulas used in this study}

We format the parameters of the studied marginal distributions by four parameters $\left\{loc, scale, \blueC{\theta_1}, \blueC{\theta_2} \right\}$, where ``$loc$'',``$scale$'' represent the location and the scale of the distribution respectively from its standardized form (with $loc=0$, $scale=1$). In detail, $loc$ and $scale$ shift and scale a variable from its standardized form by $y = (y-loc)/scale$, and its pdf $f = f/scale$. ``$\blueC{\theta_1}$'' and ``$\blueC{\theta_2}$'' are other parameters which define the shape of the distribution, one or even two of them can be absent often when the distribution is symmetric, such as Gaussian and Fisk. The studied marginal distributions are listed in Table~\ref{tab:Marginal distributions studied in this article}. The studied copulas are all single-parameter bivariate copulas, listed in Table~\ref{tab:Copulas studied in this article}. \blueC{We focused on commonly used distribution forms, varied enough and not too complex to implement. Naturally, our models remains generic and applicable to other marginal distributions and copulas.}

\renewcommand\cellset{\renewcommand\arraystretch{0.6}%
    \setlength\extrarowheight{0pt}}
\begin{table}[htbp]
	\centering
    \caption{Standardized marginal distributions studied in this article.}
    \label{tab:Marginal distributions studied in this article}
    \resizebox{\columnwidth}{!}{
    \begin{tabular}{c||c |c| c}
    \hline
    Form & PDF $f\seq{y}$ & CDF $F\seq{y}$ & parameters \\
    \hline
    Gaussian\tablefootnote{$\errorfun{x}= \frac{2}{\sqrt{\pi}}\int_0^x \exp{-t^2}dt$ represents the error function.} & $f = \frac{1}{\sqrt{2\pi}} \exp{-\frac{y^2}{2}} $ & $F = \frac{1}{2}\left(1+ \errorfun{\frac{y}{\sqrt{2}}}\right) $ & - \\
    \hline
    Gamma\tablefootnote{$\Gammafun{x} = \int_0^\infty t^{x-1}\exp{-t}dt$ is a complete Gamma function and $\incGammafun{s}{x}=\int_0^x t^{s-1}\exp{-t}dt$ represents the lower incomplete gamma function.} & $f = \frac{ y^{\blueC{\theta_1}-1} \exp{-y} }{\Gammafun{\blueC{\theta_1}}}$ & $F = \frac{\incGammafun{\blueC{\theta_1}}{y}}{\Gammafun{\blueC{\theta_1}}}$ & $\blueC{\theta_1} > 0$ \\
    \hline
    Beta\tablefootnote{$\Betafun{x}{s}=\int_0^1 t^{x-1}\left(1-t\right)^{s-1} dt$ is the Beta function and $\incBetafun{x}{a}{b}=\int_0^x t^{a-1} \left( 1-t \right)^{b-1}dt$ is the incomplete Beta function.} & $f = \frac{\Gammafun{\blueC{\theta_1}+\blueC{\theta_2}} y^{\blueC{\theta_1}-1} \left(1-y\right)^{\blueC{\theta_2}-1}   }{\Gammafun{\blueC{\theta_1}}\Gammafun{\blueC{\theta_2}} }$ & $F = \frac{\incBetafun{y}{\blueC{\theta_1}}{\blueC{\theta_2}}}{\Betafun{\blueC{\theta_1}}{\blueC{\theta_2}}}$ & \makecell{$\blueC{\theta_1}>0$, \\$\blueC{\theta_2}>0$}  \\
    \hline
    Beta prime & $f = \frac{y^{\blueC{\theta_1}-1} \left(1+y\right)^{-\blueC{\theta_1}-\blueC{\theta_2}} }{ \Betafun{\blueC{\theta_1}}{\blueC{\theta_2}} }$ & $F = \incBetafun{\frac{y}{1+y}}{\blueC{\theta_1}}{\blueC{\theta_2}} $ & \makecell{$\blueC{\theta_1}>0$, \\$\blueC{\theta_2}>0$} \\ 
    \hline
    Fisk & $f = \frac{\blueC{\theta_1} y^{\blueC{\theta_1}-1}}{\left( 1+y^{\blueC{\theta_1}} \right)^2} $  & $F = \frac{1}{1+y^{-\blueC{\theta_1}}}$ & $\blueC{\theta_1} >0$ \\
    \hline
    Laplace & $f = \frac{1}{2}\exp{ -|x| }$ & $F = \left\{ \begin{array}{ll} \frac{1}{2}\exp{y}& if \ y<0 \\ 1-\frac{1}{2}\exp{-y} & if \ y\geq 0  \end{array} \right.$ & - \\
    \hline
    Student’s t & $f=\frac{\Gammafun{\frac{\seq{\blueC{\theta_1}+1}}{2}}}{\sqrt{\pi \blueC{\theta_1}}\Gammafun{\frac{\blueC{\theta_1}}{2}}} {\seq{1+\frac{y^2}{\blueC{\theta_1}}}}^{-\frac{\blueC{\theta_1}+1}{2}}$ & $F= \frac{1}{2}+y\Gammafun{\frac{\seq{\blueC{\theta_1}+1}}{2}}\times  \frac{\hypergeofun{\frac{1}{2}}{\frac{\blueC{\theta_1}+1}{2}}{\frac{3}{2}}{-\frac{y^2}{\blueC{\theta_1}}}}{\sqrt{\pi \blueC{\theta_1}}\Gammafun{\frac{\blueC{\theta_1}}{2}}}$ & $\blueC{\theta_1} >0$ \\
    \hline
    \end{tabular}}
\end{table}

\begin{table}[htbp]
	\centering
    \caption{Copulas studied in this article.}
    \label{tab:Copulas studied in this article}
    \resizebox{\textwidth}{!}{
    \begin{tabular}{c||c c c}
    \hline
    Name & PDF $c\seq{u_1, u_2}$ & CDF $C\seq{u_1, u_2}$ & parameter \\
    \hline
    \blueC{Gaussian}\tablefootnote{$\xi_i=\phi^{-1}\seq{u_i}$, where $\phi$ represents the standard normal distribution. Correlation $\rho=\begin{bmatrix}1 & \alpha \\ \alpha & 1 \end{bmatrix}$ and $\bs{I}$ is the identity matrix.} & $c = \frac{1}{1-\alpha^2}\exp{\blueC{-\frac{1}{2}} \xi^\intercal(\rho-\bs{I})\xi }$ & $C = \int_0^{u_1} \phi \seq{ \frac{\phi^{-1}(u_2)-\rho\phi^{-1}(u) }{\sqrt{1-\rho^2}}du }$  & $\alpha\in\interv{-1,1}$ \\
    \hline
    Gumbel\tablefootnote{$U_1=\seq{- \Ln \seq{u_1}}^{\alpha}$ and $U_2=\seq{-\Ln\seq{u_2}}^{\alpha}$.} & \makecell{$ c=\frac{U_1}{u_1 \Ln(u_1)}\frac{U_2}{u_2 \Ln(u_2)} \seq{ \alpha-1+U_1+U_2}^{\frac{1}{\alpha}}$\\ \quad $ \seq{U_1+U_2}^{\frac{1}{\alpha}-2} \exp{ -\seq{U_1+U_2}^{\frac{1}{\alpha}}} $} & $C = \exp{- \seq{U_1+U_2 }^{\frac{1}{\alpha}} }$  & $\alpha\in \interv{1,+\infty}$ \\
    \hline
    Clayton & \makecell{$c = (1+\alpha)u_1^{-1-\alpha}u_2^{-1-\alpha}$ \\ \quad $\seq{ -1+u_1^{-\alpha}+u_2^{-\alpha}}^{-\frac{1}{\alpha}-2}$}  & $C = \seq{ u_1^{-\alpha}+u_2^{-\alpha}-1 }^{\frac{1}{\alpha}}$  & $\alpha \in\left[0,+\infty\right)$ \\
    \hline
    FGM & $c = 1+\alpha\seq{ 1-2u_1 }\seq{ 1-2u_2 }$ & $C = u_1 u_2 \seq{ 1+\alpha\seq{1-u_1}\seq{1-u_2}}$ & $\alpha\in \interv{-1,1}$ \\
    \hline
    Arch12\tablefootnote{$U_1 = \seq{ \frac{1}{u_1}-1 }^{\alpha}$ and $U_2 = \seq{\frac{1}{u_2}-1 }^{\alpha}$.} & \makecell{$c = \frac{U_1}{u_1(u_1-1)}\frac{U_2}{u_2(u_2-1)}\frac{ \seq{U_1+U_2}^{\frac{1}{\alpha}-2} }{ \seq{1+ \seq{U_1+U_2}^{\frac{1}{\alpha}}}^{3} }$ \\ \quad $\seq{\alpha-1+(\alpha+1)\seq{U_1+U_2}^{\frac{1}{\alpha}} }$}& $C =\seq{1+ \seq{ U_1+U_2}^{\frac{1}{\alpha}} }^{-1}$  & $\alpha\in\left[1,+\infty\right)$\\
    \hline
    Arch14\tablefootnote{$U_1=\seq{ u_1^{-\frac{1}{\alpha}} }^{\alpha}$ and $U_2=\seq{ u_2^{-\frac{1}{\alpha}} }^{\alpha}$.} & \makecell{ $c = U_1 U_2\seq{ 1+\seq{U_1+U_2}^{\frac{1}{\alpha}} }^{-2-\alpha}$ \\ \quad\qquad $\seq{U_1+U_2}^{\frac{1}{\alpha}-2}  \frac{\alpha-1+2\alpha\seq{U_1+U_2}^{\frac{1}{\alpha}}}{\alpha u_1 u_2 \seq{ u_1^{\frac{1}{\alpha}}-1 }\seq{ u_2^{\frac{1}{\alpha}}-1 }} $ }& $C =\seq{1+ \seq{ U_1+U_2}^{\frac{1}{\alpha}} }^{-\alpha}$ & $\alpha\in\left[1,+\infty\right)$ \\
    \hline
    Product & $c = 1$  & $C = u_1 u_2$ & - \\
    \hline
    \end{tabular}}
\end{table}

~\\

\newpage

\renewcommand\thefigure{B\arabic{figure}}    
\setcounter{figure}{0}  
\renewcommand\thetable{B\arabic{table}}    
\setcounter{table}{0}  

\section*{Appendix B: Convergence of the goodness of fit and error ratio when testing GICE}
\label{app:convergence}


\begin{figure}[!h]
\centering
\begin{subfigure}{0.49\textwidth}
  \centering
  \includegraphics[width=0.95\textwidth]{Figures/nonGaussCBMM_gof.png}
  \captionsetup{width=.9\linewidth}
  \caption{Evolution of Kolmogorov distance}
  \label{fig:nonGauss gof}
\end{subfigure}%
\begin{subfigure}{0.49\textwidth}
  \centering
  \includegraphics[width=0.95\textwidth]{Figures/nonGaussCBMM_error.png}
  \captionsetup{width=.95\linewidth}
  \caption{Evolution of error ratio.}
  \label{fig:nonGauss error ratio}
\end{subfigure}
\caption{Convergence of the goodness of fit and error ratio when testing GICE on non-Gaussian CBMM synthetic data \blueCN{(N=2000 samples)}. \blueC{$T$: realization time, init: initialization method}.}
\label{fig:nonGauss CBMM convergence}
\end{figure}


\begin{figure}[!h]
\centering
\begin{subfigure}[t]{0.49\textwidth}
  \centering
  \includegraphics[width=0.95\textwidth]{Figures/GMM_gof.png}
  \captionsetup{width=.9\linewidth}
  \caption{Evolution of Kolmogorov distance}
  \label{fig:GMM gof}
\end{subfigure}%
\begin{subfigure}[t]{0.49\textwidth}
  \centering
  \includegraphics[width=0.95\textwidth]{Figures/GMM_error.png}
  \captionsetup{width=.95\linewidth}
  \caption{Evolution of error ratio.}
  \label{fig:GMM error ratio}
\end{subfigure}
\caption{Convergence of the goodness of fit and error ratio when testing GICE on GMM synthetic data \blueCN{(N=2000 samples)}. \blueC{$T$: realization time, init: initialization method}.}
\label{fig:GMM convergence}
\end{figure}

~\\

\newpage

\renewcommand\thefigure{C\arabic{figure}}    
\setcounter{figure}{0}  
\renewcommand\thetable{C\arabic{table}}    
\setcounter{table}{0}  

\section*{Appendix C: Details on the imaging data and processing for characterization of myocardial patterns}
\label{app:infarct}

\begin{figure}[!h]
    \centering    
    \includegraphics[width=.9\textwidth]{Figures/Fig_Pipeline.pdf}
    \caption{\blueC{Overview of the processing steps to analyze and visualize infarct patterns. The myocardium (red and green contours) and the infarct (blue area) were segmented on each slice of late Gadolinium enhancement magnetic resonance images (left). Then, the segmented data for each individual were transported to a common reference anatomy (middle). To easily visualize these 3D data, we used a Bull's eye flattened representation close to the one commonly used by clinicians (right), the intensity at each location standing for the amount of infarct transmurality.}}
    \label{fig:infarctDisplay}
\end{figure}


\begin{figure}[!h]
\centering
\begin{subfigure}[t]{0.32\textwidth}
  \centering
  \includegraphics[width=0.81\textwidth]{Figures/infarctTerritory.jpg}
  \captionsetup{width=.9\linewidth}
  \caption{}
  \label{fig:territory}
\end{subfigure}%
\begin{subfigure}[t]{0.32\textwidth}
  \centering
  \includegraphics[width=0.95\textwidth]{Figures/LAD_example_2.png}
  \captionsetup{width=.9\linewidth}
  \caption{}
  \label{fig:LAD example}
\end{subfigure}
\begin{subfigure}[t]{0.32\textwidth}
  \centering
  \includegraphics[width=0.95\textwidth]{Figures/RCA_example.png}
  \captionsetup{width=.9\linewidth}
  \caption{}
  \label{fig:RCA example}
\end{subfigure}
\caption{\blueC{(a) Bull's eye flattened representation of the left ventricle, with the standard 17 segments commonly used by clinicians, which correspond to the coronary territories highlighted. (b) Bull's eye flattened representation of one representative subject from our study with LAD infarct. The 17 segments are overlaid for comprehension purposes, but our Bull's eye representation is much more detailed up to the voxel scale. (c) Similar representation for a \blueCN{subject with RCA infarct}.}}
\label{fig:Left ventricular territories and two example}
\end{figure}

\begin{figure}[!h]
    \centering
    \centering
    \includegraphics[width=\textwidth]{Figures/LADclusterCBMM_patterns.png}
    \label{fig:typical patterns GMM}
    \caption{Representative infarct patterns reconstructed from the ``centers'' of each \blueC{estimated CBMM }cluster. Reconstruction was carried out by multiscale kernel regression \cite{duchateau2013adaptation}.}
    \label{fig:typical patterns}
\end{figure}
