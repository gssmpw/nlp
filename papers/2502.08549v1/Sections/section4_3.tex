\subsection{Experiments on real data: subgroup clustering for the characterization of myocardial infarct patterns}
\label{subsec:Application to subgroup clustering for myocardial infarct}

\blueCN{In the previous sections, t}he increased flexibility of CBMM and the identification ability of GICE have been tested on simulated data \blueCN{and on a well-known large real dataset}. We now illustrate their relevance on a real data application in cardiac imaging, namely clustering subgroups of patients based on their infarct patterns, previously extracted from medical images.

Myocardial Infarction (MI) is a severe coronary artery disease. It is caused by a shortage of blood flow to a portion of the heart, leading to necrosis of the heart muscle \blueC{\cite{Ibanez:EHJ:2018}}. Imaging techniques play a crucial role for MI diagnosis and the evaluation of infarct evolution after reperfusion.
However, infarct assessment as performed in clinical routine relies on oversimplified scalar measurements of the lesions, mostly infarct extent and transmurality (the amount of propagation across the myocardial wall, from endocardium to epicardium). Infarct patterns are much more complex, and require more advanced analysis tools to characterize their \blueC{severity} across a population and potential evolution \cite{Duchateau:FrontCV:20233}. Here, we aim at demonstrating the relevance of the CBMM identified by GICE, simply called CBMM-GICE later, for two clustering problems: (i) separating infarcts from different territories, (ii) identifying subgroups of infarct patterns \blueC{for a given territory and examine their severity}. \blueCN{Given the complexity of infarct shapes, subgroups of patients are very likely to follow a non-Gaussian distribution, which can be critical for any subsequent statistical analysis that compares subgroups of patients or an individual to a given subgroup.}

We focus on the imaging data from the HIBISCUS-STEMI cohort \blueC{(ClinicalTrials ID: NCT03070496)}, %\cite{dufay2020hibiscus}, 
which is a prospective cohort of patients with ST Elevation Myocardial Infarction (STEMI). The infarct patterns were manually segmented offline on late Gadolinium enhancement \blueC{magnetic resonance} images using commercial software (CVI42 v.5.1.0 Circle Cardiovascular Imaging, Calgary, Canada). Then, they were resampled to a reference anatomy using a parameterization of the myocardium in each image slice, as done in \cite{Duchateau:FrontCV:20233}. This allows comparing the data from different patients and acquisitions at each voxel location of a common template. After alignment, each subject presents 21 slices ranging from the basal (mitral valve) to apical levels of the left ventricle, \blueC{totalling a number of 8110 voxels within the myocardium (the zone across which we analyze infarct patterns).}

\blueC{In the following experiments, we both examined the distribution of samples in the UMAP 2D latent space, and high-dimensional reconstructions of representative samples. As visualization of infarct patterns across the 21 slices is difficult, we used a polar representation (often called Bull's eye plot) which provides a global view of a given infarct pattern at once. It consists of a flattened version of the left ventricle, made of concentric circles each corresponding to a given slice. The apex and basal levels are at the center and periphery of the Bull's eye. Each circle averages the infarct information across the radial direction of the myocardium (namely, transmurally). This representation is commonly used by clinicians (simplified into 17 segments which can be easily related to their associated coronary territories \cite{CERQUEIRA2002463}, %\cite{CERQUEIRA2002463, termeer2010patient}, 
see 
Appendix C, Figure \ref{fig:territory}). A summary of these processing and visualization steps is given in 
Appendix C, Figure \ref{fig:infarctDisplay}.}

\subsubsection{Infarct territory clustering}
\label{subsubsec:Infarct territory clustering}

This section deals with the territory clustering task, namely separating patients with infarcts in different territories in the studied dataset through CBMM-GICE. Although this problem could be handled with supervised methods, and is by itself a rather feasible task, we prefer to start by evaluating our methods on this controlled environment before moving to a more challenging problem on the same medical imaging application. \blueCN{Besides, the subgroups of patients are strongly non-Gaussian in this population, as visible in Figure \ref{fig:terryClust_figures}a.}

MI can happen in three territories covering the left ventricle, \blueC{as illustrated in 
Appendix C, Figure \ref{fig:territory}, }corresponding to three primary coronary arteries: the Left Anterior Descending (LAD), the Right Coronary Artery (RCA), and the Left CircumfleX (LCX). In general, the infarcted territories are rather easy to distinguish across patients, as visible in the two clear examples from 
Appendix C, Figure \ref{fig:LAD example} and \ref{fig:RCA example}. Nevertheless, more challenging cases may also be encountered, in particular when the infarct is mostly localized near the apex.

\blueC{In this experiment, w}e used the data from 276 subjects from the HIBISCUS-STEMI cohort, 156 with LAD infarct and 120 with RCA infarct. We did not include patients with LCX infarcts due to the small size of this subgroup.

\blueC{As we didn't find any KNN value for UMAP in related cardiac MRI image studies, we considered various KNN values here, and used the trustworthiness score \cite{venna2005local} to select the best projection for each KNN. This score takes values between 0 and 1, and is a measure of how much local neighborhoods are preserved, which penalizes unexpected nearest neighbors in the output space according to their rank in the input space.} We performed 50 projections and the one with the best trustworthiness score \blueC{was selected} for the downstream clustering. CBMM-GICE was then tested on these selected projections corresponding to different KNN values.

\blueC{Figure \ref{fig:terryClust_figures} shows the 2D latent space estimated by UMAP (each point representing the infarct pattern of a given subject, colored by its corresponding territory label), and the clusters found be GMM-EM and CBMM-GICE.} Comparing the cluster densities obtained by CBMM-GICE in Figure \ref{fig:terryClust_result CBMM-GICE} to the ones obtained by GMM-EM in Figure \ref{fig:terryClust_result GMM-EM}, we observe that without the strict constraint of elliptical form, GICE adapts better to the distribution of projected subjects.

\blueC{Figure \ref{fig:terryClust_table} complements these observations by quantifying} the fitness of estimated CBMM to the distribution of projected subjects, and the error ratio of predicted territory labels, compared to the estimated GMM learnt through EM (GMM-EM). We can conclude that CBMM-GICE performs better than GMM-EM on this territory clustering task. The best clustering result \blueC{was} obtained for KNN=6. 

\begin{figure}[tb]
\centering
\begin{subfigure}[t]{0.32\textwidth}
  \centering
  \includegraphics[width=0.95\textwidth]{Figures/terryClust_groundTruth.png}
  \captionsetup{width=.9\linewidth}
  \caption{2D representation of the studied infarct patterns, colored by their coronary territory (LAD or RCA).}
  \label{fig:terryClust_Projected infarct segmentations}
\end{subfigure}%
\begin{subfigure}[t]{0.32\textwidth}
  \centering
  \includegraphics[width=0.95\textwidth]{Figures/terryClust_GMM.png}
  \captionsetup{width=.95\linewidth}
  \caption{Clusters and densities found through GMM-EM. \blueC{Both LAD and RCA clusters are Gaussian distributed.}}
  \label{fig:terryClust_result GMM-EM}
\end{subfigure}
\begin{subfigure}[t]{0.32\textwidth}
  \centering
  \includegraphics[width=0.95\textwidth]{Figures/terryClust_GICE.png}
  \captionsetup{width=.95\linewidth}
  \caption{Clusters and densities found by CBMM-GICE. Cluster LAD: \blueC{Beta prime} and Gamma marginals, Clayton copula. Cluster RCA: Fisk marginals, Gaussian copula.}
  \label{fig:terryClust_result CBMM-GICE}
\end{subfigure} 
\caption{Clustering of the projected infarct segments in LAD and RCA territories \blueCN{(N=276 samples)}. 
The illustrated projection was selected \blueC{using the trustworthiness score \cite{venna2005local} (the highest values corresponding to projections that best preserve local neighborhoods)} among 50 UMAP projections with KNN = 6.
}
\label{fig:terryClust_figures}
\end{figure}

\begin{figure}
    \centering
    \includegraphics[width=.9\textwidth]{Figures/Tab3.pdf}
    \caption{Infarct territory clustering result of GMM-EM and CBMM-GICE (\blueC{realization time} $T$=50, \blueC{GMM initialization}, 100 \blueC{max.} iterations, \blueCN{N=276 samples}). \blueC{The star points out the best result for each metric.}}
    \label{fig:terryClust_table}
\end{figure}

\subsubsection{Infarct pattern clustering for the LAD territory}
\label{subsubsec:Infarct pattern clustering in LAD territory}

In this section, we test our methods on a more challenging task of clinical relevance, namely identifying subgroups of patients based on their infarct patterns. This is of high clinical value to go beyond the scalar measurements used in clinical practice, and better understand subtle changes in infarct patterns following reperfusion.

We specifically focus on patients whose infarct correspond to the most populated coronary territory in the HIBISCUS-STEMI cohort, namely LAD infarcts. In this population,
54 patients have two visits realized at 1 month and 12 months after their heart has been reperfused. Our objectives are two-fold: finding subgroups of infarct patterns in this specific set of patients, \blueC{potentially of different severity}, and evaluate the migration of patients across clusters between the two visits.

As in the previous Section \ref{subsubsec:Infarct territory clustering}, UMAP was first applied to represent the cohort of LAD infarct segments in to a \blueC{2D} space to ease the mixture model learning and visualization. \blueC{We set KNN=6 as in the previous section.}
Then, the BIC criterion was applied to decide the number of clusters. More precisely, we ran a dozen times the GMM-EM considering different component numbers $K\in\set{1,\cdots,10}$ \blueC{(namely, running the EM algorithm with different GMM parameter initialization)}, and picked the most frequently chosen number according to the BIC criterion. In this way, $K=3$ was selected and considered a suitable cluster number for the LAD infarct segments.
Component number optimization is actually a challenging task in clustering problems but out of the central scope of this study, the alternative methods of BIC can be found in \cite{hancer2017comprehensive}.

\begin{figure}[tb]
\centering
\begin{subfigure}[t]{0.48\textwidth}
  \centering
  \includegraphics[width=\textwidth]{Figures/LADclusterGMM_highlight_link.png}
  \captionsetup{width=.95\linewidth}
  \caption{Estimated cluster densities and labels through by GMM-EM. \blueC{All clusters are Gaussian distributed.}}
  \label{fig:LADclusterGMM}
\end{subfigure} \hfill
\begin{subfigure}[t]{0.48\textwidth}
  \centering
  \includegraphics[width=\textwidth]{Figures/LADclusterCBMM_highlight_link.png}
  \captionsetup{width=.95\linewidth}
  \caption{Estimated cluster densities and labels through CBMM-GICE. \blueC{Cluster 1: Fisk marginals, FGM copula. Cluster 2: Fisk and Beta prime marginals, Gaussian copula. Cluster 3: Gamma and Beta prime marginals, Gaussian copula.}}
  \label{fig:LADclusterCBMM}
\end{subfigure}
\caption{Clustering of LAD infarct patterns (\blueCN{N=54 patients with two visits}, 2D visualization obtained from UMAP with KNN=6). \blueC{Each} dashed line links the two visits of the same patient \blueC{(1 month (M1) and 12 months (M12) after the heart has been reperfused)}.  
In each subplot, the red dots point out cases that migrate from one cluster to another.}
\label{fig:LADClust_figures}
\end{figure}

The CBMM-GICE was tested against the GMM-EM and the MMST-EM (Mixture of Multiple Scaled Student's T identified by EM algorithm \cite{forbes2014new, zheng2020characterization}). The clusters resulting from GMM-EM and CBMM-GICE are illustrated in Figures \ref{fig:LADclusterGMM} and \ref{fig:LADclusterCBMM}, respectively. 
When heavy tail is not present in the clusters, MMST-EM and GMM-EM have very similar performance. 
The clusters obtained through MMST-EM are quite close to the GMM-EM ones, therefore not repeatedly depicted here. 
As no ground truth was available, we computed the mean silhouette scores \cite{rousseeuw1987silhouettes} over all subjects to measure the consistency within the estimated clusters. A high mean silhouette score indicates the subjects well matching 
to their own clusters and poorly matching to the other clusters. 
The clustering performance of all three methods are reported in Table \ref{tab:Clustering result LAD}. Thanks to the flexibility of CBMM, CBMM-GICE not only \blueCN{better fits} to the distribution of subjects as indicated by the Kolmogorov distance, but also results in a highest mean silhouette score of clustering. The resulting clusters are not constrained to be symmetric, they actually follow very different distributions from each other. 

\begin{table}[b]
    \centering
    \resizebox{0.7\columnwidth}{!}{
    \begin{tabular}{c||c|c|c}
        \hline
        Method & GMM-EM & MMST-EM & CBMM-GICE \\
        \hline
        Kolmogorov distance & 0.066 & 0.067 & 0.059 \\
        \hline
        Mean silhouette score & 0.452 & 0.452 & 0.513 \\
        \hline
    \end{tabular}}
    \caption{Evaluation of the clustering quality on patients with LAD infarcts, for GMM-EM, MMST-EM and CBMM-GICE (\blueC{realization time} $T$=50, \blueC{GMM initialization}, 100 \blueC{max.} iterations, \blueCN{N=54 samples}).}
    \label{tab:Clustering result LAD}
\end{table}

To better interpret the \blueC{clinical meaning of the clusters}, we visualized the typical pattern associated with each cluster by reconstructing the high-dimensional infarct pattern corresponding to the cluster ``center''\blueC{, namely }the point with highest probability density. Reconstruction was carried out by multiscale kernel regression \cite{duchateau2013adaptation}, since UMAP is not equipped with an explicit mapping from the low-dimensional representation to the high-dimensional space. 
Appendix C, \blueC{Figure \ref{fig:typical patterns} displays these reconstructed patterns for the CBMM clusters. Comparable results were found for the GMM clusters, given that the cluster centers are rather close between the two methods.}
We observe that clusters not only correspond to infarcts of different extent, but also to complex spreads around the anteroseptal region at the mid-cavity level, and the whole myocardium near the apex (center of the bull's eye).

Differences in the resulting GMM and CBMM clusters are more clearly reflected by examining the evolution of patients between the two visits, depicted with the dash lines in Figure \ref{fig:LADClust_figures}. 
The detected migration of a few patients from one cluster to another are highlighted in red. In spite of the migration commonly detected by GMM-EM and CBMM-GICE, constrained by the symmetric Gaussian form, GMM-EM marks more migrations near the borders of clusters, \blueC{most} of which \blueC{being} being actually with relative short evolution paths.

