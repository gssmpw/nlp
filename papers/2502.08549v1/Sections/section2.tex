\section{Copula-based mixture models}
\label{sec:Copula-based mixture models}

Let $\X = \seq{X^1,\cdots,X^D}$ be a $D$-dimensional random variable, the general mixtures model the Probability Density Function (PDF) of $\X$ as follows:
\begin{equation}
    \label{eq:general mixture}
    \pdf{\x} = \sum_{k=1}^K  \pdf{z=k} \pdfc{\x}{z=k},
\end{equation}
\sloppy where $\x$ denotes the realisation of $\X$, and $z$ is the realisation of $Z$, a categorical random variable
that takes its value in $\set{1,\cdots,K}$ indicating to which subgroup $\X$ belongs.
More concisely the model is often written as:
\begin{equation}
    \label{eq:concise general mixture}
    \pdf{\x} = \sum_{k=1}^K  \pi_k p_k\seq{\x},
\end{equation}
where $\pi_k = \pdf{z=k}$ is called component coefficient, \blueC{and} $p_k\seq{\x} = \pdfc{\x}{z=k}$. When all components follow a Gaussian distribution, namely $p_k\seq{\x}=\Norm\seq{\blueC \x | \bsmu_k, \bsSigma_k}$, % \seq{\bsmu_k, \bsSigma_k}
the model becomes a classical GMM. 

\paragraph{Copulas} Let $\Y = \seq{Y^1, \cdots, Y^D}$ be a random vector valued in $\Real^D$, and $\y=\seq{y^1,\cdots,y^D}$ be a realisation of $\Y$. $F\seq{\y} = P\interv{Y^1\leq y^1, \cdots, Y^{\blueC D}\leq y^{\blueC D}}$ denotes the Cumulative \blueC{Distribution} Function (CDF) of $\Y$, and $F_1\seq{y^1}, \cdots, F_D\seq{y^D}$ denotes the univariate CDFs of $Y^1,\cdots,Y^D$ respectively. A copula is a CDF defined on $\interv{0, 1}^D$ such that marginal CDFs are uniform. According to Sklar's theorem \cite{nelsen2007introduction}, any multivariate distribution can be represented via its marginal distributions and a unique copula $C$ which links them:
\begin{equation}
    \label{eq:copula definition}
    F\seq{\y} = C\seq{F_1\seq{y^1},\cdots,F_D\seq{y^D}}.
\end{equation}
Assuming differentiable $F$ and $C$, and setting:
\begin{equation}
    \label{eq:copula pdf}
    %c\seq{\u} = \frac{\partial^D}{\partial u^1,\cdots,\partial u^D}C\seq{\u},
    c\seq{\u} = \frac{\partial^D}{\blueC{\partial u^1 \cdots \partial u^D}}C\seq{\u},
\end{equation}
where $\u = \seq{u^1,\cdots, u^D}$, taking the derivative of \blueC{\eqref{eq:copula definition}}, the PDF of $\Y$ can be represented with marginal densities $f_d$, $d\in\set{1,\cdots,D}$ and a copula density function $c$: 
\begin{equation}
    \label{eq:pdf with copula}
    f\seq{\y} = c\interv{F_1\seq{y^1},\cdots,F_D\seq{y^D}}\prod_{d=1}^D f_d\seq{y^d}.
\end{equation}

Applying \eqref{eq:pdf with copula} to decompose the component density $p_k\seq{\x}$ in the general mixture model \eqref{eq:concise general mixture}, we obtain the copula-based mixture model (CBMM):
\begin{equation}
    \label{eq:CBMMs}
    \pdf{\x} = \sum_{k=1}^K  \pi_k c_k\interv{F_{k,1}\seq{x^1},\cdots,F_{k,D}\seq{x^D}}\prod_{d=1}^D f_{k,d}\seq{x^d},
\end{equation}
Therefore, the choice of 
distribution of the $k$-th component
is ``decomposed'' 
into
choices of its marginal distributions $f_{k,d}$, and the choice of its copula $c_k$, $d\in\set{1,\cdots,D}$, $k\in\set{1,\cdots,K}$. Choosing all $f_{k,d}$ and $c_k$ to be Gaussian 
%equals choosing 
amounts to choose
Gaussian $p_k\seq{\x}$ and finds back a GMM. 

%Let us 
Let's now
consider the parameters in the model \eqref{eq:CBMMs}:
\begin{equation}
    \label{eq:CBMMs with params}
    \pdf{\x;\bsTheta} = \sum_{k=1}^K  \pi_k c_k\interv{F_{k,1}\seq{x^1;\theta_{k,1}},\cdots,F_{k,D}\seq{x^D;\theta_{k,D}}; \alpha_k} \prod_{d=1}^D f_{k,d}\seq{x^d;\theta_{k,d}},
\end{equation}
where 
$\theta_{k,d}$ represents the parameters of the marginal distribution 
for the $k$-th component in the $d$-th dimension, 
and 
$\alpha_k$ denotes the parameters of the copula 
for the $k$-th 
component. We can thus summarize the parameters as $\bsTheta = \set{\pi_k, \alpha_k, \bstheta_k}_{k=1}^K$, with component marginal parameter set $\bstheta_k=\set{\theta_{k,d}}_{d=1}^{D}$.
The choices of the marginal and copula forms lead to great flexibility of CBMMs and also bring the difficulty of the model identification. 
In practice, apart from the parameter estimation, we need to make appropriate decision among a bunch of candidate marginal and copula forms.



