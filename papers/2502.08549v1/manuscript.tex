\documentclass[review]{elsarticle}

\usepackage{lineno}
\usepackage{hyperref}
\usepackage{color}
\usepackage{amssymb, amsmath}
\usepackage{amsfonts, mathrsfs, marvosym, dsfont}
\usepackage{bbold} % for mathbb{1}
\usepackage[linesnumbered,lined,boxed,commentsnumbered]{algorithm2e}
\RestyleAlgo{ruled} % 
\modulolinenumbers[5]
\usepackage{hyperref}
\usepackage{makecell}
\usepackage{subcaption}
\usepackage{multirow}
\usepackage{tablefootnote}

%\journal{.}

\usepackage{todonotes}
\usepackage{url}

%\newcommand*{\blueC}{\color{blue}}
\newcommand{\pinkC}[1]{\textcolor[rgb]{1.00,0.00,1.00}{#1}}
\newcommand{\blueCN}[1]{\textcolor[rgb]{0.00,0.00,0.00}{#1}}
\newcommand{\blueC}[1]{\textcolor[rgb]{0.00,0.00,0.00}{#1}}
\newcommand{\redC}[1]{\textcolor[rgb]{0.00,0.00,0.00}{#1}}
\newcommand{\greenC}[1]{\textcolor[rgb]{0,0,0}{#1}}

%%%%%%%%%%%%%%%%%%%%%%%
%% Elsevier bibliography styles
%%%%%%%%%%%%%%%%%%%%%%%
%% To change the style, put a % in front of the second line of the current style and
%% remove the % from the second line of the style you would like to use.
%%%%%%%%%%%%%%%%%%%%%%%

%% Numbered
%\bibliographystyle{model1-num-names}

%% Numbered without titles
%\bibliographystyle{model1a-num-names}

%% Harvard
%\bibliographystyle{model2-names.bst}\biboptions{authoryear}

%% Vancouver numbered
%\usepackage{numcompress}\bibliographystyle{model3-num-names}

%% Vancouver name/year
%\usepackage{numcompress}\bibliographystyle{model4-names}\biboptions{authoryear}

%% APA style
%\bibliographystyle{model5-names}\biboptions{authoryear}

%% AMA style
%\usepackage{numcompress}\bibliographystyle{model6-num-names}

%% `Elsevier LaTeX' style
\bibliographystyle{elsarticle-num-names-alphsort}
%\bibliographystyle{elsarticle-num}
%\usepackage[style=numeric-comp]{biblatex}
%\addbibresource{mybibfile.bib}

%%%%%%%%%%%%%%%%%%%%%%%

\begin{document}

\def\method{\text MixMin~}
\def\methodnospace{\text MixMin}
\def\genmethod{$\mathbb{R}$\text Min~}
\def\genmethodnospace{ $\mathbb{R}$\text Min}


\begin{frontmatter}

\title{Copula-based mixture model identification for subgroup clustering \blueC{with imaging applications}}

\author[creatis]{Fei Zheng}

\author[creatis,iuf]{Nicolas Duchateau\corref{mycorrespondingauthor}}
\cortext[mycorrespondingauthor]{Corresponding author}

\address[creatis]{Univ Lyon, INSA‐Lyon, Université Claude Bernard Lyon 1, UJM-Saint Etienne, CNRS, Inserm, CREATIS UMR 5220, U1294, F‐69621, LYON, France \vspace{2mm}}


\address[iuf]{Institut Universitaire de France (IUF)}

\begin{abstract}
Model-based clustering techniques have been widely applied to various application areas, while most studies focus on canonical mixtures with unique component distribution form. However, this strict assumption is often hard to satisfy. In this paper, we consider the more flexible Copula-Based Mixture Models (CBMMs) for clustering, which allow heterogeneous component distributions composed by flexible choices of marginal and copula forms. \blueC{More specifically, we propose an adaptation of the Generalized Iterative Conditional Estimation (GICE) algorithm to identify the CBMMs in an unsupervised manner, where the marginal and copula forms and their parameters are estimated iteratively.}

GICE is adapted from its original version developed for switching Markov model identification with the choice of realization time. Our CBMM-GICE clustering method is then tested on synthetic \blueC{two-cluster} data \blueCN{(N=2000 samples)} with discussion of the factors impacting its convergence. Finally, \blueC{it is compared to the Expectation Maximization identified mixture models with unique component form \blueCN{on the entire} MNIST database \greenC{(N=70000)}, and on real cardiac \blueCN{magnetic resonance} data \blueCN{(N=276)} to illustrate its \blueCN{value for imaging applications}.}
\end{abstract}

\begin{keyword}
finite mixtures, model selection, copulas, GICE algorithm, clustering, medical imaging.
\end{keyword}

\end{frontmatter}

%\linenumbers
\section{Introduction}
\label{sec:introduction}
The business processes of organizations are experiencing ever-increasing complexity due to the large amount of data, high number of users, and high-tech devices involved \cite{martin2021pmopportunitieschallenges, beerepoot2023biggestbpmproblems}. This complexity may cause business processes to deviate from normal control flow due to unforeseen and disruptive anomalies \cite{adams2023proceddsriftdetection}. These control-flow anomalies manifest as unknown, skipped, and wrongly-ordered activities in the traces of event logs monitored from the execution of business processes \cite{ko2023adsystematicreview}. For the sake of clarity, let us consider an illustrative example of such anomalies. Figure \ref{FP_ANOMALIES} shows a so-called event log footprint, which captures the control flow relations of four activities of a hypothetical event log. In particular, this footprint captures the control-flow relations between activities \texttt{a}, \texttt{b}, \texttt{c} and \texttt{d}. These are the causal ($\rightarrow$) relation, concurrent ($\parallel$) relation, and other ($\#$) relations such as exclusivity or non-local dependency \cite{aalst2022pmhandbook}. In addition, on the right are six traces, of which five exhibit skipped, wrongly-ordered and unknown control-flow anomalies. For example, $\langle$\texttt{a b d}$\rangle$ has a skipped activity, which is \texttt{c}. Because of this skipped activity, the control-flow relation \texttt{b}$\,\#\,$\texttt{d} is violated, since \texttt{d} directly follows \texttt{b} in the anomalous trace.
\begin{figure}[!t]
\centering
\includegraphics[width=0.9\columnwidth]{images/FP_ANOMALIES.png}
\caption{An example event log footprint with six traces, of which five exhibit control-flow anomalies.}
\label{FP_ANOMALIES}
\end{figure}

\subsection{Control-flow anomaly detection}
Control-flow anomaly detection techniques aim to characterize the normal control flow from event logs and verify whether these deviations occur in new event logs \cite{ko2023adsystematicreview}. To develop control-flow anomaly detection techniques, \revision{process mining} has seen widespread adoption owing to process discovery and \revision{conformance checking}. On the one hand, process discovery is a set of algorithms that encode control-flow relations as a set of model elements and constraints according to a given modeling formalism \cite{aalst2022pmhandbook}; hereafter, we refer to the Petri net, a widespread modeling formalism. On the other hand, \revision{conformance checking} is an explainable set of algorithms that allows linking any deviations with the reference Petri net and providing the fitness measure, namely a measure of how much the Petri net fits the new event log \cite{aalst2022pmhandbook}. Many control-flow anomaly detection techniques based on \revision{conformance checking} (hereafter, \revision{conformance checking}-based techniques) use the fitness measure to determine whether an event log is anomalous \cite{bezerra2009pmad, bezerra2013adlogspais, myers2018icsadpm, pecchia2020applicationfailuresanalysispm}. 

The scientific literature also includes many \revision{conformance checking}-independent techniques for control-flow anomaly detection that combine specific types of trace encodings with machine/deep learning \cite{ko2023adsystematicreview, tavares2023pmtraceencoding}. Whereas these techniques are very effective, their explainability is challenging due to both the type of trace encoding employed and the machine/deep learning model used \cite{rawal2022trustworthyaiadvances,li2023explainablead}. Hence, in the following, we focus on the shortcomings of \revision{conformance checking}-based techniques to investigate whether it is possible to support the development of competitive control-flow anomaly detection techniques while maintaining the explainable nature of \revision{conformance checking}.
\begin{figure}[!t]
\centering
\includegraphics[width=\columnwidth]{images/HIGH_LEVEL_VIEW.png}
\caption{A high-level view of the proposed framework for combining \revision{process mining}-based feature extraction with dimensionality reduction for control-flow anomaly detection.}
\label{HIGH_LEVEL_VIEW}
\end{figure}

\subsection{Shortcomings of \revision{conformance checking}-based techniques}
Unfortunately, the detection effectiveness of \revision{conformance checking}-based techniques is affected by noisy data and low-quality Petri nets, which may be due to human errors in the modeling process or representational bias of process discovery algorithms \cite{bezerra2013adlogspais, pecchia2020applicationfailuresanalysispm, aalst2016pm}. Specifically, on the one hand, noisy data may introduce infrequent and deceptive control-flow relations that may result in inconsistent fitness measures, whereas, on the other hand, checking event logs against a low-quality Petri net could lead to an unreliable distribution of fitness measures. Nonetheless, such Petri nets can still be used as references to obtain insightful information for \revision{process mining}-based feature extraction, supporting the development of competitive and explainable \revision{conformance checking}-based techniques for control-flow anomaly detection despite the problems above. For example, a few works outline that token-based \revision{conformance checking} can be used for \revision{process mining}-based feature extraction to build tabular data and develop effective \revision{conformance checking}-based techniques for control-flow anomaly detection \cite{singh2022lapmsh, debenedictis2023dtadiiot}. However, to the best of our knowledge, the scientific literature lacks a structured proposal for \revision{process mining}-based feature extraction using the state-of-the-art \revision{conformance checking} variant, namely alignment-based \revision{conformance checking}.

\subsection{Contributions}
We propose a novel \revision{process mining}-based feature extraction approach with alignment-based \revision{conformance checking}. This variant aligns the deviating control flow with a reference Petri net; the resulting alignment can be inspected to extract additional statistics such as the number of times a given activity caused mismatches \cite{aalst2022pmhandbook}. We integrate this approach into a flexible and explainable framework for developing techniques for control-flow anomaly detection. The framework combines \revision{process mining}-based feature extraction and dimensionality reduction to handle high-dimensional feature sets, achieve detection effectiveness, and support explainability. Notably, in addition to our proposed \revision{process mining}-based feature extraction approach, the framework allows employing other approaches, enabling a fair comparison of multiple \revision{conformance checking}-based and \revision{conformance checking}-independent techniques for control-flow anomaly detection. Figure \ref{HIGH_LEVEL_VIEW} shows a high-level view of the framework. Business processes are monitored, and event logs obtained from the database of information systems. Subsequently, \revision{process mining}-based feature extraction is applied to these event logs and tabular data input to dimensionality reduction to identify control-flow anomalies. We apply several \revision{conformance checking}-based and \revision{conformance checking}-independent framework techniques to publicly available datasets, simulated data of a case study from railways, and real-world data of a case study from healthcare. We show that the framework techniques implementing our approach outperform the baseline \revision{conformance checking}-based techniques while maintaining the explainable nature of \revision{conformance checking}.

In summary, the contributions of this paper are as follows.
\begin{itemize}
    \item{
        A novel \revision{process mining}-based feature extraction approach to support the development of competitive and explainable \revision{conformance checking}-based techniques for control-flow anomaly detection.
    }
    \item{
        A flexible and explainable framework for developing techniques for control-flow anomaly detection using \revision{process mining}-based feature extraction and dimensionality reduction.
    }
    \item{
        Application to synthetic and real-world datasets of several \revision{conformance checking}-based and \revision{conformance checking}-independent framework techniques, evaluating their detection effectiveness and explainability.
    }
\end{itemize}

The rest of the paper is organized as follows.
\begin{itemize}
    \item Section \ref{sec:related_work} reviews the existing techniques for control-flow anomaly detection, categorizing them into \revision{conformance checking}-based and \revision{conformance checking}-independent techniques.
    \item Section \ref{sec:abccfe} provides the preliminaries of \revision{process mining} to establish the notation used throughout the paper, and delves into the details of the proposed \revision{process mining}-based feature extraction approach with alignment-based \revision{conformance checking}.
    \item Section \ref{sec:framework} describes the framework for developing \revision{conformance checking}-based and \revision{conformance checking}-independent techniques for control-flow anomaly detection that combine \revision{process mining}-based feature extraction and dimensionality reduction.
    \item Section \ref{sec:evaluation} presents the experiments conducted with multiple framework and baseline techniques using data from publicly available datasets and case studies.
    \item Section \ref{sec:conclusions} draws the conclusions and presents future work.
\end{itemize}
\section{Background and Related Works} \label{sec:related}
Pretrained language models \cite{BERT,Roberta,GPT} are mostly based on the transformer architecture \cite{Vaswani2017} due to their effectiveness in various NLP tasks. This section first formally describes the typical transformer architecture and then discusses representative techniques for compressing large language models, namely distillation and pruning.


\subsection{Transformer Architecture}
% MHA
The typical transformer architecture consists of encoder and decoder layers, each of which commonly contains two main components: multi-head attention (MHA) and feed-forward network (FFN). In an MHA layer, there are $N_H$ self-attention heads, and each head $h \in [1, N_H]$ involves the following weight matrices, $W_{h}^{Q}, W_{h}^{K}, W_{h}^{V}, W_{h}^{O} \in \mathbb{R}^{{d} \times\frac{d}{N_H}}$. Then, the final output of the MHA layer is computed as follows:
$$
MHA(X)= \sum_{h = 1}^{N_{H}} Attn_{h}(X),
$$
where $Attn_{h}(X)$ is the output of the standard self-attention unit.

% FFN
The output of MHA is then fed into the corresponding FFN layer, which consists of weight matrices $W^{(1)}\in \mathbb{R}^{d \times d_{ffn}}$, $W^{(2)}\in \mathbb{R}^{d_{ffn} \times d}$, $b^{(1)}\in \mathbb{R}^{d_{ffn}}$ and $b^{(2)}\in \mathbb{R}^{d}$, where $d_{ffn}$ (usually $4\times d$) represents the dimension of the
intermediate hidden features. The output of the FFN layer can be computed as follows:
$$
FFN(A)=\sum_{i=1}^{d_{ffn}}GeLU(AW^{(1)}_{:,i}+b^{(1)})W^{(2)}_{i,:}+b^{(2)},
$$
where $A = MHA(X)$.

In transformer-based models, this same structure is repeatedly defined across multiples layers (e.g., 12 layers in $\text{BERT}_{\text{BASE}}$ \cite{BERT}) and each layer has another multiple heads (e.g., 12 heads in BERT$_{\text{BASE}}$ \cite{BERT}). Consequently, they have numerous trainable parameters, which motivates the NLP community to develop various compression methods for these models.


\subsection{Distillation}
Knowledge distillation (KD) \cite{KD} is a compression technique that trains a lightweight student model to mimic the soft output of a large teacher model, leading to competitive performance on downstream tasks. There are some studies where KD methods have been applied to transformer-based models. For example, DistilBERT \cite{DistilB} transfers knowledge to a student model of a fixed-size through a pre-training phase and an optional fine-tuning process. MiniLM \cite{MiniLM} deeply describes the self-attention information of a task-agnostic teacher model by providing a detailed analysis of the self-attention matrices. Both TinyBERT \cite{TinyBERT} and MobileBERT \cite{MoB} transfer knowledge during pretraining using a layer-by-layer strategy. TinyBERT \cite{TinyBERT} additionally performs distillation during fine-tuning. 

Although KD-based methods are shown to be effective at preserving high accuracy, training a student model can be time-consuming; as reported by CoFi \cite{Xia}, TinyBERT \cite{TinyBERT} requires 350 hours and CoFi \cite{Xia} requires 20 hours for compression. Furthermore, 
it is not trivial to effectively distill the knowledge from a multi-layer teacher model with self-attention information to a student model with fewer layers, which involves the problem of layer selection and different loss functions.


\subsection{Pruning}
Pruning \cite{Han} is another popular compression scheme that removes unimportant weights or structures from the target neural network. Following a massive volume of pruning techniques on convolutional neural networks (CNNs), pruning for transformer family has also been studied, falling into either unstructured or structured pruning.

%Unstructured pruning
\paragraph{Unstructured pruning} 
In unstructured pruning, we remove individual weights that are not important, often aiming to reduce the memory storage itself for the target model while maintaining performance, such as the methods based on \textit{lottery ticket hypothesis} \cite{LoB} and \textit{movement pruning} \cite{Mov}. However, in these unstructured pruning methods, it is difficult and complicated to make satisfactory speedup at inference time, often requiring a special hardware.

%Structured pruning
\paragraph{Structured pruning} 
Structured pruning is a simpler and more efficient way to reduce the model size, where we eliminate some structures of parameters that are considered less significant. In the case of transformer architectures, we can remove redundant attention heads \cite{sixteen,voita,SMP}, entire layers of either MHA or FFN \cite{layer1,layer2}, or neurons along with all their connecting weights of FFNs \cite{earlybert}. However, such a coarse-grained pruning scheme inevitably suffers from severe drop in accuracy. Consequently, recent studies \cite{DynaBERT,block,Xia} have explored the combination of pruning with KD to further enhance the performance of compressed networks. Despite higher performance of these combined approaches, they sacrifice efficiency and simplicity in the compression process itself, as KD typically takes lengthy training times and involves complicated matching procedures.

%Semi-structured pruning
\paragraph{Semi-structured pruning} This paper proposes a semi-structured pruning method for BERT in order to achieve a good balance between efficiency and accuracy. To our best knowledge, \textit{block movement pruning} (BMP) \cite{block} is the only one that can be categorized into semi-structured pruning for transformer family, which proposes a block-based pruning approach for each weight matrix of MHA and FFN layers. 

%Group-based pruning 
\paragraph{Group-based pruning} 
Our method is inspired by a grouping strategy, which has been frequently employed to reduce the computational complexity of CNNs. In this approach, a set of filters or channels are grouped together, and the objective is to minimize connectivity and computation between these groups \cite{DGC,Zhao,deeproot,xie,shufflenet}. These grouped architectures are also incorporated into transformer-based models by a few recent works \cite{GroupFormer,groupbert}. However, they focus on designing more efficient architectures to be trained from scratch, not for compressing a trained task-specific model. This work is the first study on group-based pruning on BERT, aiming to compress the task-specific BERT while maintaining its original accuracy.



Having examined several illustrative examples that highlight the limitations of naive applications of current continual learning approaches, we now turn to a systematic analysis of three key conceptual challenges that must be addressed to move the field forward.
We structure our discussion around three fundamental aspects: the nature of continuity in learning problems, 
the choice of appropriate spaces and metrics for measuring similarity, and the role of local objectives in learning.
For each aspect, we first present key considerations that emerge from our analysis, 
followed by specific recommendations for future research directions.


\subsection{On Continuity} 
\begin{tcolorbox}[colback=orange!10,colframe=orange!50,boxsep=-1pt]
\textbf{Considerations: Continuity.}
When designing CL systems, one must examine what continuity means and how it manifests. Two fundamental forms of continuity shape the space of possible approaches: temporal continuity in how tasks evolve, and continuity in the underlying task space itself. These distinct types of continuity create different constraints on learning algorithms and require different treatment.
\end{tcolorbox}

\textbf{Cons \#1: Temporal continuity.}
We will refer to a change over time of the joint distribution of data points and prediction targets as ``drift''.
This is classically handled by assigning a potentially different distribution $\mathcal{D}_i$ to every data point $x_i$ \cite{gama2014survey},
with drift occuring when $\mathcal{D}_i \neq \mathcal{D}_j$.
We advocate here for the approach of \citet{hinder2020towards}, who propose a 
Distribution Process to capture this drift
by associating each datapoint $x_i$ with a time $t_i$, such that two datapoints sharing a time also share the same distribution.
The distributions $D_t$ are defined as Markov kernels in the time domain,
and it is now possible to postulate limiting statements similar to the batch setup or discuss concepts such as the mean distribution over a period of time.

\textbf{Cons \#2: Task continuity.}
While the comparatively simple case of continuously varying mixing coefficients of a discrete task set has been considered under the name ``task-free continual learning'' \cite{Lee2020A, jin2021gradient, shanahan2021encoders}, the possibility of a truly continuous task set has been raised \cite{van2022three}, but we are not aware of a systematic exploration of this setting.
For example, in the task-free setting one might first infer task identity and then use task-specific components \cite{heald2021contextual},
but even if task identity inference is solved,
the lack of a discrete task set in the harder case makes the use of task-specific components no longer trivial.

\begin{tcolorbox}[colback=blue!10,colframe=blue!50,boxsep=-1pt]
\textbf{Recommendations: Continuity.}
Based on our analysis of CL continuity challenges, we propose three key directions for future research:
1) formalizing temporal dynamics through drift processes rather than point-wise distributions, 2) understanding
and managing the impact of data presentation schedules, and 3) developing principled approaches for handling
continuous rather than discrete task spaces.
These recommendations aim to help researchers and practitioners better handle continuous aspects of learning while maintaining theoretical rigor and practical applicability.
\end{tcolorbox}

\textbf{Rec \#1: Use Distribution Processes to capture drift.}
We recommend working with Distribution Processes over datapoint-indexed distributions when modelling drift, as this makes the temporal structure explicit.
In particular, the extent to which temporally close distributions are expected to be similar must then be assumed explicitly rather than implicitly.
It is crucial that these assumptions are understood in continual learning, as they are the core idea underlying the notion of tasks,
and further are the reason we expect forward and backward transfer to be possible.
We believe the improvement in clarity of thinking associated with this new formalization will enhance future work.

\textbf{Rec \#2: Consider schedule dependence.}
Let us formalize a data stream as
an underlying dataset and an order in which this is presented, or \textit{schedule}.
While known in stream learning \cite{gama2014survey},
the effects of such a schedule are considered explicitly only by relatively few CL works \cite{Yoon2020Scalable, wang2022schedule},
and \citet{wang2022schedule} showed that most existing continual learning algorithms suffer drastic fluctuations in performance under different schedules.
After considering the expected temporal correlations of the data stream via drift processes, it is likely that significant permutation symmetries (\eg, discrete task orderings) will remain.
After establishing which permutations of the stream do not constitute meaningful information from which the model should learn,
future work should strive to maximize invariance of CL algorithms to such permutations.

\textbf{Rec \#3: Towards continuous task identity.}
Finally we note that, while discreteness of the underlying task set has been an important and productive underlying assumption in continual learning research, principled methods of handling task identity in the truly continuous case (\eg, section \ref{sec:boxpush} and maybe even \ref{sec:starcraft}) should be developed.
Task-specific components, for example, should still be possible where task identity is not discrete.
When representing a task as, \eg, some embedding in a continuous latent space, however, they are no longer trivial and are indeed interestingly non-trivial.
Such principled approaches should strive to account for the now much richer geometry of the task space.


\subsection{On Spaces} 
\begin{tcolorbox}[colback=orange!10,colframe=orange!50,boxsep=-1pt]
\textbf{Considerations: Spaces.}
When examining CL systems, we encounter three distinct types of continuous spaces: 
parameter space, data space, and function space. Each of these spaces requires careful consideration 
of how to measure ``similarity'' or ``distance'' - a choice that is sometimes forced by the problem 
structure. Even after selecting a space, the choice of metric remains critical, as different metrics 
can capture different aspects of the learning problem. Some scenarios may even require inherently 
asymmetric measures of similarity.
\end{tcolorbox}

\textbf{Cons \#1: The three common spaces.}
Most obviously, we have the continuous space of parameters.
Often we also have a continuous space of possible data items, \eg, arrays of floating point pixel values.
Finally, we have the continuous space of functions representable by our neural network.
If we identify a ``task'' with ``the mapping from inputs to outputs which solves the task'' then it can be seen as a special case of a function space.

When one needs to measure ``similarity'' or ``distance'' in continual learning, one will in general do so in one of these spaces.
Sometimes this is a choice, sometimes it is forced.
For example, when considering a mixture of experts solution to a variety of tasks where the architectures of the neural network models corresponding to the experts differ, it is impossible to measure distance in parameter space.
In this case we must instead consider function space.

\textbf{Cons \#2: Metrics.}
Even once the choice of space is made, ``distances'' are not determined until we choose a metric on that space.
Sometimes there will be a natural choice
(\eg, the Fisher metric in function space for classification tasks, or more generally for tasks where the output is a probability distribution).
In an application such as weight space regularization, there is a simple choice of the Euclidean metric,
but this choice is inherently incapable of identifying more or less important parameters for a given task, and may even violate safety constraints in a case like that of section \ref{sec:constraints}.
The more expressive choice of the Fisher metric as used in Natural Gradient Descent would allow such parameters to be identified.
This may allow a new task to make use of those subspaces of parameter space left unspecified by the preceding tasks.

\textbf{Cons \#3: Divergences.}
Finally, it is often the case that a notion of ``distance'' in a continual learning problem can be identified with a KL divergence, and is thus inherently asymmetrical.
For example, suppose we wish to identify new tasks by measuring the ``distance'' between a memory buffer and a sequence of new datapoints.
If the memory buffer contains datapoints from tasks A and B, but the sequence of new datapoints comes only from task B, is it a new task? Clearly not.
But if this was reversed, and the new datapoints came from A and B, while the buffer came only from B, then the memory buffer would be insufficient to determine correct behaviour on the new points from task A and the answer to the question ``is there a new task'' must be yes.
Consider the case of two 2D Gaussian distributions centered at (0,0) and (1, 0) with isotropic standard deviations 2 and 0.5, respectively.
The KL divergence in one direction is 2.8 bits, but in the other it is 20.5 bits.
Intuitively, this is because samples from the small Gaussian are in-distribution for the large Gaussian, but not vice-versa.
More concretely, in the previously considered application of the Fisher metric to parameter space regularization,
one direction corresponds to measuring distances relative to the Fisher metric measured on the new datapoints, whereas as the other corresponds to using the Fisher metric measured on the buffer.

\begin{tcolorbox}[colback=blue!10,colframe=blue!50,boxsep=-1pt]
\textbf{Recommendations: Spaces.}
Based on our analysis of the different spaces and metrics in continual learning, we propose 
several practical guidelines for developing more effective methods. These recommendations 
focus on making explicit choices about spaces and metrics, recognizing potential asymmetries 
in similarity measures, and considering alternative spaces when standard approaches fail.
\end{tcolorbox}

\textbf{Rec \#1: Choose the correct metric and space.}
Firstly, one must choose the space in which to measure this similarity or distance.
The straightforward option might be to consider raw data such as pixel values, but perhaps semantic differences would be easier to detect in some function space, such as a latent space of a neural network.
Then, having identified the correct space, one must choose a metric on that space.
Even when making ``no choice'' and using the Euclidean metric, one should be mindful of what this means.
For example, when doing weight space regularization, using a quadratic penalty in the Euclidean metric corresponds to the assumption that the appropriate posterior on weights is an isotropic Gaussian.
Making the implications of this ``non-choice'' concrete will allow the implicit assumptions to be sanity-checked.

\textbf{Rec \#2: Remember that the correct notion of similarity may not be symmetric.}
One should also pay attention to any asymmetries in the application of a notion of distance.
Often the ``distance'' measure required in an algorithm will correspond to a KL divergence.
Whether you would like your distance measure to behave like forward KL divergence or reverse KL divergence depends on the purpose of the measure: ``how informative is task A about task B'' will often have a different answer to ``how informative is task B about task A''.
Choosing the wrong direction here will likely result in severe algorithm underperformance, even though both directions agree when the tasks being compared are relatively similar.
Since asymmetries here become most salient when similarity is low, toy examples with large distances should be considered and sanity-checked by comparing both possible directions.

\textbf{Rec \#3: Consider patching broken methods by switching spaces or metrics.}
If a continual learning method fails in some particular application, it may be salvageable by altering the space in which distances are measured.
Suppose, for example that one uses functional regularization in a task where the output of the network is target robot arm pose parameterized by joint angles.
This may fail if task success is dependent on end effector pose, and the sensitivity of end effector pose to joint angle is itself highly dependent on robot pose, due to nonlinear kinematics.
In this case, re-expressing the output in terms of end effector pose via a kinematics model may resolve these difficulties.

\subsection{On Objectives}
\begin{tcolorbox}[colback=orange!10,colframe=orange!50,boxsep=-1pt]
\textbf{Considerations: Objectives.}
Current perspectives on continual learning tend to focus narrowly on accumulating knowledge through 
classification tasks. However, this view may be inherently limiting, as it emphasizes conditional 
knowledge (''which class, given these classes?") over unconditional understanding. The relationship 
between classification, density estimation, and generative modeling suggests broader ways to think 
about knowledge retention in continual learning systems.
\end{tcolorbox}

\textbf{Cons \#1: Accumulating unconditional knowledge.} \\
The knowledge involved in successful classification is inherently of a very conditional nature,
\ie, we answer the question ``given that this datapoint is drawn from the distribution of one of these $N$ classes, which class is it''.
We argue that focusing on classification objectives over density estimation or generative objectives makes continual or lifelong learning unnecessarily overcomplicated.
For example, out of distribution detection is clearly more closely related to density estimation,
and there are whole classes of replay based continual learning algorithms which are closely related to generation.
We believe that building continual learning algorithms on top of narrow classification tasks neglects the potential synergies of introducing generative or density based objectives,
as we shall now discuss.

\begin{tcolorbox}[colback=blue!10,colframe=blue!50,boxsep=-1pt]
\textbf{Recommendations: Objectives.}
Drawing from our analysis of the role of different learning objectives, we propose several 
directions for expanding beyond pure classification in continual learning. These recommendations 
emphasize the potential benefits of incorporating generative and density-based approaches, 
both for avoiding catastrophic forgetting and for more robust task identification.
\end{tcolorbox}

\textbf{Rec \#1: Consider generation for avoiding forgetting.} \\
Where the base task incorporates a generative objective, many challenges related to regularizing on or reviewing data examples from previous tasks are greatly simplified by direct exploitation of this generative function to create synthetic datapoints \cite{Robins95pseudorehearsal}.

\textbf{Rec \#2: Consider densities for task identification.} \\
In the presence of density estimation capabilities available from the base task, it is much easier to assign future datapoints to tasks and to consider questions of task boundaries, be they discrete or continuous.

\textbf{Rec \#3: Consider the energy-based model connection.} \\
Even in the case of primarily classification objectives there seems to be great potential for density estimation via connections to energy-based models \cite{grathwohl2020secretlyenergybased, li2022energybasedforcontinual}.
This could be of great use in the primary evaluation settings common within continual learning.


\section{Experiments and results}
\label{sec:Experiments}

\blueC{W}e consider data dimension $D=2$ and the following candidate forms for all the experiments:
\begin{itemize}
    \item[] $\H = \set{\text{Gamma, Fisk, Gaussian, T, Laplace, Beta, BetaPrime}}$,
    \item[] $\G = \set{\text{Gumble, Gaussian, Clayton, FGM, Arch12, Arch14, Product}}$.
\end{itemize}
Their PDF, CDF, and parameter details are listed in Appendix A.

\subsection{Experiments on synthetic data}
\label{sec:Experimentation on synthetic data}

This series of experiments aims to test the efficiency of the GICE for the identification of CBMMs constructed by different marginal and copula forms\blueC{, with component number $K=2$}. Some computational details that impact the performance of GICE are also illustrated and discussed in this section.

\subsubsection{Test on simulated data of Non-Gaussian CBMM}
\label{sec:Test on simulated data of Non-Gaussian CBMM}
This series of tests aims to verify the performance of GICE on non-Gaussian CBMM identification. 2000 samples were simulated from a pre-defined CBMM with non-Gaussian marginal and copulas. The model parameters are listed in the second row of Table \ref{tab:nonGauss True and estimated CBMM parameters}. Figure \ref{fig:nonGauss True PDF and labels} shows the simulated data and the cluster densities. The two clusters 
were designed to partially overlap and imitate a difficult identification situation, since clearly separated clusters are much easier to identify, and even simply using K-Means may well predict the labels.

\begin{figure}[tb]
\centering
\begin{subfigure}[t]{0.49\textwidth}
  \centering
  \includegraphics[width=0.95\textwidth]{Figures/nonGaussCBMM_truePDF.png}
  \captionsetup{width=.9\linewidth}
  \caption{True cluster densities and labels, CBMM parameters as listed in Table \ref{tab:nonGauss True and estimated CBMM parameters}}
  \label{fig:nonGauss True PDF and labels}
\end{subfigure}%
\begin{subfigure}[t]{0.49\textwidth} 
  \centering
  \includegraphics[width=0.95\textwidth]{Figures/nonGaussCBMM_estPDF.png}
  \captionsetup{width=.95\linewidth}
  \caption{Estimated cluster densities and labels (GICE \blueC{with realization time $T$}=10, GMM initialization, \blueC{100 max. iterations}).}
  \label{fig:nonGauss Estimated PDF and labels}
\end{subfigure}
\caption{Synthetic experiment \blueCN{(N=2000 samples)} to evaluate the performance of GICE on non-Gaussian CBMM identification.}
\label{fig:nonGauss CBMM PDF and labels}
\end{figure}



We tested two different initialization methods: the K-Means and the EM for GMM, and considered different realization times $T=1$ and $T=10$. 
We set a maximum of $100$ iterations for GICE to reach its convergence. The evolution of the Kolmogrorov distance between the learnt model and the empirical data distribution  with iterations is reported in 
Appendix B, Figure \ref{fig:nonGauss gof}. 
The evolution of error ratio between the predicted labels and the true ones with iterations is illustrated in 
Appendix B, Figure \ref{fig:nonGauss error ratio}. 
For these synthetic data generated from the defined CBMM as in Figure \ref{fig:nonGauss True PDF and labels}, we see that EM for GMM provides a better-fit initialization than K-Means, by comparing the 
Kolmogorov distance and error ratio of the two GICE settings ``$T=10$, init: K-Means'' and ``$T=10$, init: GMM'' \greenC{at the start of the algorithm}
%at 0 iteration 
in Figure \ref{fig:nonGauss CBMM convergence}. However, after 30 iterations they converged to the same level and give very similar estimated CBMMs in the end, see the estimated marginal and copula distributions in 
the third and fourth rows of Table \ref{tab:nonGauss True and estimated CBMM parameters}. 

\begin{table}[b]
\caption{True and estimated CBMM parameters with different realization times and initializations. \blueC{$T$: realization time, init: initialization method}.}
\label{tab:nonGauss True and estimated CBMM parameters}
\centering
    \resizebox{1\columnwidth}{!}{
        \begin{tabular}{c||c|c|c|c||c|c|c|c|}
            \cline{2-9}
             & $\pi_{k=1}$ & $\theta_{k=1,d=1}$ & $\theta_{k=1,d=2}$ & $\alpha_1$ & $\pi_{k=2}$ & $\theta_{k=2,d=1}$ & $\theta_{k=2,d=2}$ & $\alpha_2$ \\
            \hline
            True $\bsTheta$ & 0.4 & \makecell{T \\(2, 2, 0.7)}  & \makecell{Fisk \\(4, 0, 3)} &  \makecell{FGM \\(1)}& 0.6 & \makecell{Laplace \\(3.5, 0.8)} & \makecell{ Gamma\\(10, -4, 0.5)} & \makecell{Arch14 \\(3)} \\
            \hline
            \makecell{GICE $T$=10 \\ init: K-Means} & 0.38 & \makecell{T \\(1.74, 1.95, 0.64)}  & \makecell{Fisk \\(3.78, 0.29, 2.77)} & \makecell{FGM \\(0.89)} & 0.62 & \makecell{Laplace \\(3.51, 0.79)}  & \makecell{T \\(27.0, 1.00, 1.49)} & \makecell{Arch14 \\(2.95)} \\
            \hline
            \makecell{GICE $T$=10 \\ init: GMM} & 0.38 & \makecell{T \\(1.76, 1.95, 0.65)}  & \makecell{Fisk \\(3.80, 0.28, 2.78)} & \makecell{FGM \\(0.92)} & 0.62 & \makecell{Laplace \\(3.51, 0.79)}  & \makecell{T \\(24.6, 1.00, 1.49)} & \makecell{Arch14 \\(2.95)} \\
            \hline    
            \makecell{GICE $T$=1 \\ init: GMM} & 0.38 & \makecell{T \\(1.73, 1.94, 0.63)}  & \makecell{Fisk \\(3.84, 0.22, 2.84)} & \makecell{FGM \\(0.94)} & 0.62 & \makecell{Laplace \\(3.51, 0.79)}  & \makecell{T \\(37.4, 1.02, 1.51)} & \makecell{Arch14 \\(2.91)} \\
            \hline 
        \end{tabular}
    }
\end{table}

GICE is a stochastic method, thus the curves of the convergence indexes are not smoothly 
decreasing.
The change of the selected distribution forms between iterations may lead to small burrs in these curves. In spite of the stochastic nature of GICE, the realization time $T$ also impacts the smoothness of the convergence. The curves obtained from the settings ``$T=10$, init: GMM'' are smoother than the ones obtained from ``$T=1$, init: GMM'' in both Figure \ref{fig:nonGauss gof} and Figure \ref{fig:nonGauss error ratio}. The estimated parameters with only one realization are not too far from the result of ten realizations as reported in the last row of Table \ref{tab:nonGauss True and estimated CBMM parameters}. 
Although increasing $T$ reduces the oscillation around the target parameters at convergence, it also means to increase the computational time of GICE. In this experiment, GICE took 207 s when $T=1$, and 1548 s when $T=10$, running on a 2.7MHz CPU. In practice, $T$ should be chosen such that it balances the expected convergence smoothness and time cost.

Figure \ref{fig:nonGauss Estimated PDF and labels} shows the identified cluster densities and predicted labels through GICE with $T=10$ realizations and GMM initialization. GICE attempts to find the best-fit CBMM for the data, but it does not ensure to always find the true marginal and copula forms. Many factors have impact on its final form decision. For example, the estimation condition, including the 
number of available samples and the difficulty to separate the clusters. This situation is not something we can improve. But in practice, by carefully examining the GICE convergence, we can try different candidate forms, initialization methods, estimators, and run GICE with different random seeds, to avoid being trapped in local extrema, and increase the chance of finding the best-fit model. 

\begin{figure}[tb]
\centering
\begin{subfigure}[t]{0.49\textwidth}
  \centering
  \includegraphics[width=0.95\textwidth]{Figures/GMM_truePDF.png}
  \captionsetup{width=.8\linewidth}
  \caption{True cluster densities and labels, GMM parameters as listed in Table \ref{tab:Gauss True and estimated CBMM parameters}}
  \label{fig:Gauss True PDF and labels}
\end{subfigure}%
\begin{subfigure}[t]{0.49\textwidth}
  \centering
  \includegraphics[width=0.95\textwidth]{Figures/GMM_estPDF.png}
  \captionsetup{width=.95\linewidth}
  \caption{Estimated cluster densities and labels (GICE \blueC{with realization time $T$}=10, K-Means initialization, \blueC{100 max. iterations}).}
  \label{fig:Gauss Estimated PDF and labels}
\end{subfigure}
\caption{Synthetic experiment \blueCN{(N=2000 samples)} to evaluate the performance of GICE on GMM identification.}
\label{fig:Gauss CBMM PDF and labels}
\end{figure}

\begin{table}[b]
\caption{True Gaussian CBMM (GMM) and estimated parameters with different initializations. \blueC{$T$: realization time, init: initialization method}.}
\label{tab:Gauss True and estimated CBMM parameters}
\centering
    \resizebox{1\columnwidth}{!}{
        \begin{tabular}{c||c|c|c|c||c|c|c|c|}
            \cline{2-9}
             & $\pi_{k=1}$ & $\theta_{k=1,d=1}$ & $\theta_{k=1,d=2}$ & $\alpha_1$ & $\pi_{k=2}$ & $\theta_{k=2,d=1}$ & $\theta_{k=2,d=2}$ & $\alpha_2$ \\
            \hline
            True $\bsTheta$ & 0.4 & \makecell{Gaussian \\(0, 1)}  & \makecell{Gaussian \\(2, 0.5)} & \makecell{Gaussian \\ 0.3} & 0.6 & \makecell{Gaussian \\(3.5, 1.5)} & \makecell{Gaussian \\(2.5, 2)} & \makecell{Gaussian \\0.7} \\
            \hline
            \makecell{GICE $T$=10 \\ init: \blueC{K-Means}} & 0.42 & \makecell{T \\(195, 0.07, 1.01)}  & \makecell{Gaussian \\(2.02, 0.52)} & \makecell{Gaussian \\(0.33)} & 0.58 & \makecell{T \\(28.2, 3.60, 1.42)}  & \makecell{T \\(39.8, 2.57, 1.92)} & \makecell{Gaussian \\(0.69)} \\
            \hline
            \makecell{GICE $T$=10 \\ init: GMM} & 0.42 & \makecell{T \\(29.6, 0.06, 0.98)}  & \makecell{Gaussian \\(2.02, 0.51)} & \makecell{Gaussian \\(0.27)} & 0.58 & \makecell{T \\(28.2, 3.59, 1.42)}  & \makecell{T \\(39.0, 2.57, 1.92)} & \makecell{Gaussian \\(0.69)} \\
            \hline    
        \end{tabular}
    }
\end{table}


\subsubsection{Test on simulated data of GMM}
This series of tests aims to check if GICE also works
for the identification of GMM, which can be considered a special case of CBMM.

2000 samples were simulated from a pre-defined two-component CBMM with Gaussian \blueCN{marginals} and copulas (thus, it is actually a GMM), see Figure \ref{fig:Gauss True PDF and labels} for the simulated data and the true cluster densities. The parameters are reported in the first row of Table \ref{tab:Gauss True and estimated CBMM parameters}. 
$T=10$ was set as realization time for GICE, and similar to the previous series, we 
tested two initialization methods: K-Means and EM for GMM. 
Initializing by EM for GMM means to start with the optimized parameters, and GICE does not drive the estimated model far away from the already optimized one, as we observe stable horizontal lines for both the evolution of the Kolmogorov distance and the error ratio in Figure \ref{fig:GMM convergence}. Meanwhile, GICE converges to the same level with K-Means initialization, and the identified model is quite close to the one initialized by EM for GMM, see the \blueC{second} and \blueC{third} rows in Table \ref{tab:Gauss True and estimated CBMM parameters}. It is reasonable that 
the T distribution is often selected by GICE instead of the Gaussian (the ground truth),
because Gaussian and T distributions with an infinite amount of degrees of freedom
are identical, and the estimated parameters of T distributions are all with relatively large degrees of freedom.
Comparing the two sub-figures in Figure \ref{fig:Gauss CBMM PDF and labels}, differences are hardly noticeable visually between the identified cluster density through GICE and the ground truth.

\subsection{\blueC{Experiments on real data: MNIST database}}
\label{subsec:MNIST}

\blueC{This experiment expands the previous evaluation of GICE to 
%the real situation 
a real dataset of more than two clusters ($K=10$).
We tested our method on the entire MNIST database \cite{Lecun:1998} \greenC{(N=70000 samples)}, which consists of $28 \times 28$ pixels grayscale images of handwritten digits from $0$ to $9$, therefore consisting of 10 classes.}

To minimize the challenges related to analyzing such high-dimensional data, \blueC{the Uniform Manifold Approximation and Projection (UMAP) algorithm \cite{mcinnes2018umap} was used as a preprocessing step to project data into a 2D space before applying the clustering methods. UMAP can preserve both the local and global structure of data, and improves the performance of downstream clustering methods on the MNIST database, as already illustrated in several published studies. For example, \cite{allaoui2020considerably} reports the improvements of three UMAP preprocessed clustering algorithms K-Means, HDBSCAN \cite{campello2013density} and GMM. \cite{mcconville2021n2d} layers up UMAP, auto-encoder and GMM into a method called ``N2D'', and achieves a clustering accuracy of 0.979 on MNIST.} 
UMAP is sensitive to its parameter K-Nearest Neighbors (KNN), which controls the balance of local versus global structure in the data. A large KNN risks to merge the real clusters together, while a small one may lead to many subgroups of samples in the projected space. 
\blueC{In this experiment, KNN$=30$ was chosen as suggested by the documentation of UMAP to avoid producing fine grained cluster structure that may be more a result of noise patterns.}

\begin{figure}[t]
\centering
\begin{subfigure}[t]{0.5\textwidth}
  \centering
  \includegraphics[width=0.815\textwidth]{Figures/MNIST_estimatedPDF_GMM_adjusted.png}
  \captionsetup{width=.9\linewidth}
  \caption{\blueC{Cluster densities found by GMM-EM, accuracy=0.825, Kolmogorov distance=0.029.}}
  \label{fig:MNIST density GMM}
\end{subfigure}%
%\\
\begin{subfigure}[t]{0.5\textwidth}
  \centering
  \includegraphics[width=0.95\textwidth]{Figures/MNIST_estimatedPDF_CBMM_adjusted.png}
  \captionsetup{width=.95\linewidth}
  \caption{\blueC{Cluster densities found by CBMM-GICE, accuracy=0.967, Kolmogorov distance=0.011.}}
  \label{fig:MNIST density CBMM}
\end{subfigure}
\caption{\blueC{Experiment on the MNIST dataset (\greenC{N=70000 samples, }10 clusters, 2D projection obtained by UMAP with KNN=30), to evaluate the performance of GICE on more than two clusters. Each point corresponds to the image of a digit, colored by its corresponding ground truth label.}}
\label{fig:MNIST density estimation}
\end{figure}

\begin{table}[b]
    \centering
    \resizebox{0.7\columnwidth}{!}{
    \begin{tabular}{c||c|c|c||c|c|c|}
        \cline{2-7}
        & \multicolumn{3}{c||}{\blueC{Accuracy}} & \multicolumn{3}{|c|}{\blueC{Kolmogorov distance}} \\
        \cline{2-7}
        & \blueC{average} & \blueC{min} & \blueC{max} & \blueC{average} & \blueC{min} & \blueC{max} \\
        \hline
        \blueC{GMM-EN} & \blueC{0.824} & \blueC{0.690} & \blueC{0.956} & \blueC{0.025} & \blueC{0.013} & \blueC{0.055}\\
        \hline
        \blueC{CBMM-GICE} & \blueC{0.848} & \blueC{0.716} & \blueC{0.967} & \blueC{0.021} & \blueC{0.011} & \blueC{0.029}\\
        \hline
    \end{tabular}}
    \caption{\blueC{Performance of GMM-EM and CBMM-GICE on the MNIST dataset based on 20 repeated experimentation (100 max iterations for GMM and GICE, GICE realization time $T$=10, initialization: GMM).}}
    \label{tab:Performance of GMM-EM and CBMM-GICE on MNIST dataset}
\end{table}

\blueC{Table \ref{tab:Performance of GMM-EM and CBMM-GICE on MNIST dataset} reports the clustering results of 20 repeated experiments on the projected (2D) MNIST data through CBMM-GICE and GMM-EM. For easy comparison to the existing results provided in other studies, we report the accuracy as measure of performance instead of the error ratio we used in previous sections. 
We observe that both algorithms manage to identify the relevant clusters, whereas non-Gaussian distributions are slightly more desirable with better goodness of fit. The GMM-EM performance is quite close to the one (0.825) reported in \cite{mcconville2021n2d}. 
Figure \ref{fig:MNIST density estimation} shows the \blueC{2D} latent space estimated by UMAP, corresponding to the best result of GICE-CBMM in the table. Each point corresponds to the image of a digit, colored by its corresponding ground truth label. Most of the clusters are already separated by UMAP, their tight distribution offers a good condition for GMM estimation. Dealing with the overlapping clusters is more challenging, and in this experiment, GMM mistook the two clusters in the center as one entire Gaussian, and force-created two subgroups in the cluster at the lower-left corner, as illustrated in Figure \ref{fig:MNIST density GMM}.}

%\todo[inline]{ND2FZ: I don't understand the color of dots in the figure: I only see green and red dots, but colored clusters. \\ \greenC{FZ: Each cluster is with around 7000 subjects and so tightly distributed that we can't see subjects individually, the visible subjects in red and green are mislocated subjects far from their own cluster centers.} \\ ND: ok I understand ! Not so easy to convince although we're not doing classification. What about: 1) having a colorbar indicating the color for each digit? (may the three close clusters correspond to similar digits?) + 2) having e.g. 25\% or 50\% opacity for each dot? This won't hide the misclassified cases, but will better render that we have many well-identified cases in each cluster.\\ \greenC{I modified the size and opacity of each point, also a colorbar is added, it seems better?}}








\subsection{Experiments on real data: subgroup clustering for the characterization of myocardial infarct patterns}
\label{subsec:Application to subgroup clustering for myocardial infarct}

\blueCN{In the previous sections, t}he increased flexibility of CBMM and the identification ability of GICE have been tested on simulated data \blueCN{and on a well-known large real dataset}. We now illustrate their relevance on a real data application in cardiac imaging, namely clustering subgroups of patients based on their infarct patterns, previously extracted from medical images.

Myocardial Infarction (MI) is a severe coronary artery disease. It is caused by a shortage of blood flow to a portion of the heart, leading to necrosis of the heart muscle \blueC{\cite{Ibanez:EHJ:2018}}. Imaging techniques play a crucial role for MI diagnosis and the evaluation of infarct evolution after reperfusion.
However, infarct assessment as performed in clinical routine relies on oversimplified scalar measurements of the lesions, mostly infarct extent and transmurality (the amount of propagation across the myocardial wall, from endocardium to epicardium). Infarct patterns are much more complex, and require more advanced analysis tools to characterize their \blueC{severity} across a population and potential evolution \cite{Duchateau:FrontCV:20233}. Here, we aim at demonstrating the relevance of the CBMM identified by GICE, simply called CBMM-GICE later, for two clustering problems: (i) separating infarcts from different territories, (ii) identifying subgroups of infarct patterns \blueC{for a given territory and examine their severity}. \blueCN{Given the complexity of infarct shapes, subgroups of patients are very likely to follow a non-Gaussian distribution, which can be critical for any subsequent statistical analysis that compares subgroups of patients or an individual to a given subgroup.}

We focus on the imaging data from the HIBISCUS-STEMI cohort \blueC{(ClinicalTrials ID: NCT03070496)}, %\cite{dufay2020hibiscus}, 
which is a prospective cohort of patients with ST Elevation Myocardial Infarction (STEMI). The infarct patterns were manually segmented offline on late Gadolinium enhancement \blueC{magnetic resonance} images using commercial software (CVI42 v.5.1.0 Circle Cardiovascular Imaging, Calgary, Canada). Then, they were resampled to a reference anatomy using a parameterization of the myocardium in each image slice, as done in \cite{Duchateau:FrontCV:20233}. This allows comparing the data from different patients and acquisitions at each voxel location of a common template. After alignment, each subject presents 21 slices ranging from the basal (mitral valve) to apical levels of the left ventricle, \blueC{totalling a number of 8110 voxels within the myocardium (the zone across which we analyze infarct patterns).}

\blueC{In the following experiments, we both examined the distribution of samples in the UMAP 2D latent space, and high-dimensional reconstructions of representative samples. As visualization of infarct patterns across the 21 slices is difficult, we used a polar representation (often called Bull's eye plot) which provides a global view of a given infarct pattern at once. It consists of a flattened version of the left ventricle, made of concentric circles each corresponding to a given slice. The apex and basal levels are at the center and periphery of the Bull's eye. Each circle averages the infarct information across the radial direction of the myocardium (namely, transmurally). This representation is commonly used by clinicians (simplified into 17 segments which can be easily related to their associated coronary territories \cite{CERQUEIRA2002463}, %\cite{CERQUEIRA2002463, termeer2010patient}, 
see 
Appendix C, Figure \ref{fig:territory}). A summary of these processing and visualization steps is given in 
Appendix C, Figure \ref{fig:infarctDisplay}.}

\subsubsection{Infarct territory clustering}
\label{subsubsec:Infarct territory clustering}

This section deals with the territory clustering task, namely separating patients with infarcts in different territories in the studied dataset through CBMM-GICE. Although this problem could be handled with supervised methods, and is by itself a rather feasible task, we prefer to start by evaluating our methods on this controlled environment before moving to a more challenging problem on the same medical imaging application. \blueCN{Besides, the subgroups of patients are strongly non-Gaussian in this population, as visible in Figure \ref{fig:terryClust_figures}a.}

MI can happen in three territories covering the left ventricle, \blueC{as illustrated in 
Appendix C, Figure \ref{fig:territory}, }corresponding to three primary coronary arteries: the Left Anterior Descending (LAD), the Right Coronary Artery (RCA), and the Left CircumfleX (LCX). In general, the infarcted territories are rather easy to distinguish across patients, as visible in the two clear examples from 
Appendix C, Figure \ref{fig:LAD example} and \ref{fig:RCA example}. Nevertheless, more challenging cases may also be encountered, in particular when the infarct is mostly localized near the apex.

\blueC{In this experiment, w}e used the data from 276 subjects from the HIBISCUS-STEMI cohort, 156 with LAD infarct and 120 with RCA infarct. We did not include patients with LCX infarcts due to the small size of this subgroup.

\blueC{As we didn't find any KNN value for UMAP in related cardiac MRI image studies, we considered various KNN values here, and used the trustworthiness score \cite{venna2005local} to select the best projection for each KNN. This score takes values between 0 and 1, and is a measure of how much local neighborhoods are preserved, which penalizes unexpected nearest neighbors in the output space according to their rank in the input space.} We performed 50 projections and the one with the best trustworthiness score \blueC{was selected} for the downstream clustering. CBMM-GICE was then tested on these selected projections corresponding to different KNN values.

\blueC{Figure \ref{fig:terryClust_figures} shows the 2D latent space estimated by UMAP (each point representing the infarct pattern of a given subject, colored by its corresponding territory label), and the clusters found be GMM-EM and CBMM-GICE.} Comparing the cluster densities obtained by CBMM-GICE in Figure \ref{fig:terryClust_result CBMM-GICE} to the ones obtained by GMM-EM in Figure \ref{fig:terryClust_result GMM-EM}, we observe that without the strict constraint of elliptical form, GICE adapts better to the distribution of projected subjects.

\blueC{Figure \ref{fig:terryClust_table} complements these observations by quantifying} the fitness of estimated CBMM to the distribution of projected subjects, and the error ratio of predicted territory labels, compared to the estimated GMM learnt through EM (GMM-EM). We can conclude that CBMM-GICE performs better than GMM-EM on this territory clustering task. The best clustering result \blueC{was} obtained for KNN=6. 

\begin{figure}[tb]
\centering
\begin{subfigure}[t]{0.32\textwidth}
  \centering
  \includegraphics[width=0.95\textwidth]{Figures/terryClust_groundTruth.png}
  \captionsetup{width=.9\linewidth}
  \caption{2D representation of the studied infarct patterns, colored by their coronary territory (LAD or RCA).}
  \label{fig:terryClust_Projected infarct segmentations}
\end{subfigure}%
\begin{subfigure}[t]{0.32\textwidth}
  \centering
  \includegraphics[width=0.95\textwidth]{Figures/terryClust_GMM.png}
  \captionsetup{width=.95\linewidth}
  \caption{Clusters and densities found through GMM-EM. \blueC{Both LAD and RCA clusters are Gaussian distributed.}}
  \label{fig:terryClust_result GMM-EM}
\end{subfigure}
\begin{subfigure}[t]{0.32\textwidth}
  \centering
  \includegraphics[width=0.95\textwidth]{Figures/terryClust_GICE.png}
  \captionsetup{width=.95\linewidth}
  \caption{Clusters and densities found by CBMM-GICE. Cluster LAD: \blueC{Beta prime} and Gamma marginals, Clayton copula. Cluster RCA: Fisk marginals, Gaussian copula.}
  \label{fig:terryClust_result CBMM-GICE}
\end{subfigure} 
\caption{Clustering of the projected infarct segments in LAD and RCA territories \blueCN{(N=276 samples)}. 
The illustrated projection was selected \blueC{using the trustworthiness score \cite{venna2005local} (the highest values corresponding to projections that best preserve local neighborhoods)} among 50 UMAP projections with KNN = 6.
}
\label{fig:terryClust_figures}
\end{figure}

\begin{figure}
    \centering
    \includegraphics[width=.9\textwidth]{Figures/Tab3.pdf}
    \caption{Infarct territory clustering result of GMM-EM and CBMM-GICE (\blueC{realization time} $T$=50, \blueC{GMM initialization}, 100 \blueC{max.} iterations, \blueCN{N=276 samples}). \blueC{The star points out the best result for each metric.}}
    \label{fig:terryClust_table}
\end{figure}

\subsubsection{Infarct pattern clustering for the LAD territory}
\label{subsubsec:Infarct pattern clustering in LAD territory}

In this section, we test our methods on a more challenging task of clinical relevance, namely identifying subgroups of patients based on their infarct patterns. This is of high clinical value to go beyond the scalar measurements used in clinical practice, and better understand subtle changes in infarct patterns following reperfusion.

We specifically focus on patients whose infarct correspond to the most populated coronary territory in the HIBISCUS-STEMI cohort, namely LAD infarcts. In this population,
54 patients have two visits realized at 1 month and 12 months after their heart has been reperfused. Our objectives are two-fold: finding subgroups of infarct patterns in this specific set of patients, \blueC{potentially of different severity}, and evaluate the migration of patients across clusters between the two visits.

As in the previous Section \ref{subsubsec:Infarct territory clustering}, UMAP was first applied to represent the cohort of LAD infarct segments in to a \blueC{2D} space to ease the mixture model learning and visualization. \blueC{We set KNN=6 as in the previous section.}
Then, the BIC criterion was applied to decide the number of clusters. More precisely, we ran a dozen times the GMM-EM considering different component numbers $K\in\set{1,\cdots,10}$ \blueC{(namely, running the EM algorithm with different GMM parameter initialization)}, and picked the most frequently chosen number according to the BIC criterion. In this way, $K=3$ was selected and considered a suitable cluster number for the LAD infarct segments.
Component number optimization is actually a challenging task in clustering problems but out of the central scope of this study, the alternative methods of BIC can be found in \cite{hancer2017comprehensive}.

\begin{figure}[tb]
\centering
\begin{subfigure}[t]{0.48\textwidth}
  \centering
  \includegraphics[width=\textwidth]{Figures/LADclusterGMM_highlight_link.png}
  \captionsetup{width=.95\linewidth}
  \caption{Estimated cluster densities and labels through by GMM-EM. \blueC{All clusters are Gaussian distributed.}}
  \label{fig:LADclusterGMM}
\end{subfigure} \hfill
\begin{subfigure}[t]{0.48\textwidth}
  \centering
  \includegraphics[width=\textwidth]{Figures/LADclusterCBMM_highlight_link.png}
  \captionsetup{width=.95\linewidth}
  \caption{Estimated cluster densities and labels through CBMM-GICE. \blueC{Cluster 1: Fisk marginals, FGM copula. Cluster 2: Fisk and Beta prime marginals, Gaussian copula. Cluster 3: Gamma and Beta prime marginals, Gaussian copula.}}
  \label{fig:LADclusterCBMM}
\end{subfigure}
\caption{Clustering of LAD infarct patterns (\blueCN{N=54 patients with two visits}, 2D visualization obtained from UMAP with KNN=6). \blueC{Each} dashed line links the two visits of the same patient \blueC{(1 month (M1) and 12 months (M12) after the heart has been reperfused)}.  
In each subplot, the red dots point out cases that migrate from one cluster to another.}
\label{fig:LADClust_figures}
\end{figure}

The CBMM-GICE was tested against the GMM-EM and the MMST-EM (Mixture of Multiple Scaled Student's T identified by EM algorithm \cite{forbes2014new, zheng2020characterization}). The clusters resulting from GMM-EM and CBMM-GICE are illustrated in Figures \ref{fig:LADclusterGMM} and \ref{fig:LADclusterCBMM}, respectively. 
When heavy tail is not present in the clusters, MMST-EM and GMM-EM have very similar performance. 
The clusters obtained through MMST-EM are quite close to the GMM-EM ones, therefore not repeatedly depicted here. 
As no ground truth was available, we computed the mean silhouette scores \cite{rousseeuw1987silhouettes} over all subjects to measure the consistency within the estimated clusters. A high mean silhouette score indicates the subjects well matching 
to their own clusters and poorly matching to the other clusters. 
The clustering performance of all three methods are reported in Table \ref{tab:Clustering result LAD}. Thanks to the flexibility of CBMM, CBMM-GICE not only \blueCN{better fits} to the distribution of subjects as indicated by the Kolmogorov distance, but also results in a highest mean silhouette score of clustering. The resulting clusters are not constrained to be symmetric, they actually follow very different distributions from each other. 

\begin{table}[b]
    \centering
    \resizebox{0.7\columnwidth}{!}{
    \begin{tabular}{c||c|c|c}
        \hline
        Method & GMM-EM & MMST-EM & CBMM-GICE \\
        \hline
        Kolmogorov distance & 0.066 & 0.067 & 0.059 \\
        \hline
        Mean silhouette score & 0.452 & 0.452 & 0.513 \\
        \hline
    \end{tabular}}
    \caption{Evaluation of the clustering quality on patients with LAD infarcts, for GMM-EM, MMST-EM and CBMM-GICE (\blueC{realization time} $T$=50, \blueC{GMM initialization}, 100 \blueC{max.} iterations, \blueCN{N=54 samples}).}
    \label{tab:Clustering result LAD}
\end{table}

To better interpret the \blueC{clinical meaning of the clusters}, we visualized the typical pattern associated with each cluster by reconstructing the high-dimensional infarct pattern corresponding to the cluster ``center''\blueC{, namely }the point with highest probability density. Reconstruction was carried out by multiscale kernel regression \cite{duchateau2013adaptation}, since UMAP is not equipped with an explicit mapping from the low-dimensional representation to the high-dimensional space. 
Appendix C, \blueC{Figure \ref{fig:typical patterns} displays these reconstructed patterns for the CBMM clusters. Comparable results were found for the GMM clusters, given that the cluster centers are rather close between the two methods.}
We observe that clusters not only correspond to infarcts of different extent, but also to complex spreads around the anteroseptal region at the mid-cavity level, and the whole myocardium near the apex (center of the bull's eye).

Differences in the resulting GMM and CBMM clusters are more clearly reflected by examining the evolution of patients between the two visits, depicted with the dash lines in Figure \ref{fig:LADClust_figures}. 
The detected migration of a few patients from one cluster to another are highlighted in red. In spite of the migration commonly detected by GMM-EM and CBMM-GICE, constrained by the symmetric Gaussian form, GMM-EM marks more migrations near the borders of clusters, \blueC{most} of which \blueC{being} being actually with relative short evolution paths.


\section{Conclusion}
In this work, we propose a simple yet effective approach, called SMILE, for graph few-shot learning with fewer tasks. Specifically, we introduce a novel dual-level mixup strategy, including within-task and across-task mixup, for enriching the diversity of nodes within each task and the diversity of tasks. Also, we incorporate the degree-based prior information to learn expressive node embeddings. Theoretically, we prove that SMILE effectively enhances the model's generalization performance. Empirically, we conduct extensive experiments on multiple benchmarks and the results suggest that SMILE significantly outperforms other baselines, including both in-domain and cross-domain few-shot settings.

\paragraph{Acknowledgements} The authors acknowledge the support from the French ANR (LABEX PRIMES of Univ. Lyon [ANR-11-LABX-0063] within the program ``Investissements d'Avenir'' [ANR-11-IDEX-0007], the JCJC project ``MIC-MAC'' [ANR-19-CE45-0005]), and the Fédération Francaise de Cardiologie (``MI-MIX'' project, Allocation René Foudon). They are also grateful to P. Croisille and M. Viallon (CREATIS, CHU Saint Etienne, France) for providing the imaging data for the HIBISCUS-STEMI population, \blueC{to} L. Petrusca and P. Clarysse (CREATIS, Villeurbanne, France) for discussions around these imaging data\blueC{, and to R. Deleat-Besson (CREATIS, Villeurbanne, France) for his help on the miniatures in Appendix C, Figure \ref{fig:infarctDisplay}}. 

\paragraph{Declaration of generative AI and AI-assisted technologies in the writing process}

Nothing to disclose.

\paragraph{Code and data availability}

The code and a demo on synthetic data corresponding to the experiments in Section \ref{sec:Experimentation on synthetic data} will be made publicly available upon acceptance.
The experiments in Section \ref{subsec:MNIST} were based on data from the public MNIST database \cite{Lecun:1998}.
The experiments in Section \ref{subsec:Application to subgroup clustering for myocardial infarct} were based on real data from the private HIBISCUS-STEMI study (ClinicalTrials ID: NCT03070496), and cannot be directly shared. However, infarct segmentations are part of the ``M1'' and ``M12'' images of the MYOSAIQ segmentation challenge dataset\footnote{\url{https://www.creatis.insa-lyon.fr/Challenge/myosaiq/}}, whose access may be provided upon request and agreement from the challenge organizers.

\paragraph{CRediT authorship contribution statement}

\begin{itemize}
    \item Fei Zheng: Methodology, Software, Validation, Investigation, Visualization, Writing - Original Draft.
    \item Nicolas Duchateau: Conceptualization, Methodology, Validation, Supervision, Writing - Review \& Editing.
\end{itemize}


%\bibliography{manuscript}
%%
%% This is file `./samples/longsample.tex',
%% generated with the docstrip utility.
%%
%% The original source files were:
%%
%% apa7.dtx(with options: `longsample')
%% ----------------------------------------------------------------------
%% 
%% apa7 - A LaTeX class for formatting documents in compliance with the
%% American Psychological Association's Publication Manual, 7th edition
%% 
%% Copyright (C) 2019 by Daniel A. Weiss <daniel.weiss.led at gmail.com>
%% 
%% This work may be distributed and/or modified under the
%% conditions of the LaTeX Project Public License (LPPL), either
%% version 1.3c of this license or (at your option) any later
%% version.The latest version of this license is in the file:
%% 
%% http://www.latex-project.org/lppl.txt
%% 
%% Users may freely modify these files without permission, as long as the
%% copyright line and this statement are maintained intact.
%% 
%% This work is not endorsed by, affiliated with, or probably even known
%% by, the American Psychological Association.
%% 
%% ----------------------------------------------------------------------
%% 
\documentclass[
a4paper,
man,
british
]{apa7}

\usepackage{changes}
\usepackage{cancel}
\usepackage[british]{babel}
\usepackage[utf8]{inputenc}
\usepackage{epstopdf}
\usepackage{csquotes}
\usepackage[flushleft]{threeparttable}
\usepackage{multirow}
%\usepackage[hidelinks]{hyperref}
%\usepackage{authblk}
\usepackage[
style = apa,
backend = biber,
sortcites = true,
sorting = nyt,
%isbn = false,
%url = false,
%doi = false,
%eprint = false,
hyperref = true,
backref = false,
%firstinits = false,
]{biblatex}
\hypersetup{
colorlinks = true,
linkcolor = black,
anchorcolor = black,
citecolor = black,
filecolor = black,
urlcolor = blue
}
\usepackage{float}
\usepackage{placeins}
\usepackage{xcolor}
\usepackage[toc, page]{appendix}
\usepackage{lscape}
\usepackage{afterpage}
\usepackage{esvect}
\usepackage{amsmath}
\usepackage{ragged2e}
\justifying\let\raggedright\justifying
\usepackage{enumitem}
\usepackage{makecell}
\DeclareLanguageMapping{british}{british-apa}
\usepackage[nodisplayskipstretch]{setspace}
\usepackage{subcaption} 
\usepackage{rotating}
\usepackage{geometry}
\geometry{a4paper, margin=1in}

% maps apacite commands to biblatex commands

\let \citeNP \cite
\let \citeA \textcite
\let \cite \parencite

\newcommand{\figurehere}[1]{\begin{center}%
\vspace{-2mm}
=========================\\%
Insert Figure #1 about here\\%
=========================\\%
\vspace{-2mm}
\end{center}}
%%
%%Table goes about here command
\newcommand{\tablehere}[1]{\begin{center}%
\vspace{-2mm}
=========================\\%
Insert Table #1 about here\\%
=========================\\%
\vspace{-2mm}
\end{center}}

\usepackage{array}
\newcommand{\PreserveBackslash}[1]{\let\temp=\\#1\let\\=\temp}
\newcolumntype{C}[1]{>{\PreserveBackslash\centering}p{#1}}
\newcolumntype{R}[1]{>{\PreserveBackslash\raggedleft}p{#1}}
\newcolumntype{L}[1]{>{\PreserveBackslash\raggedright}p{#1}}

\addbibresource{tutorial_MH.bib}

\title{Tutorial on Using Machine Learning and Deep Learning Models for Mental Illness Detection}
\shorttitle{Data-driven Methods to Identify Mental Illness}

\authorsnames[1,2,2,3,4,5,1,5]{%
  Yeyubei Zhang, Zhongyan Wang, Zhanyi Ding, Yexin Tian, 
  Jianglai Dai, Xiaorui Shen, Yunchong Liu, Yuchen Cao
}
\authorsaffiliations{%
  {University of Pennsylvania, School of Engineering and Applied Science}, 
  {New York University, Center for Data Science}, 
  {Georgia Institute of Technology, College of Computing}, 
  {University of California, Berkeley, Department of EECS}, 
  {Northeastern University, Khoury College of Computer Science}
}

%\leftheader{Placeholder}

\abstract{Social media has become an important source for understanding mental health, providing researchers a way to detect conditions like depression from user-generated posts. This tutorial provides practical guidance to address common challenges in applying machine learning and deep learning methods for mental health detection on these platforms. It focuses on strategies for working with diverse datasets, improving text preprocessing, and addressing issues such as imbalanced data and model evaluation. Real-world examples and step-by-step instructions demonstrate how to apply these techniques effectively, with an emphasis on transparency, reproducibility, and ethical considerations. By sharing these approaches, this tutorial aims to help researchers build more reliable and widely applicable models for mental health research, contributing to better tools for early detection and intervention.}

\keywords{Mental Health Research, Machine Learning, Deep Learning, Social Media Analysis, Natural Language Processing}

\authornote{
% \addORCIDlink{Placeholder}{0000-0000-0000-0000}
Correspondence concerning this article should be addressed to Yuchen Cao, Northeastern University, E-mail: cao.yuch@northeastern.edu}

\begin{document}
\maketitle

\section{Introduction}

Mental health disorders, especially depression, have become a significant concern worldwide, affecting millions of individuals across diverse populations \cite{WHO2020}. Early detection of depression is crucial, as it can lead to timely treatment and better long-term outcomes. In today’s digital age, social media platforms such as X(Twitter), Facebook, and Reddit provide a unique opportunity to study mental health. People often share their thoughts and emotions on these platforms, making them a valuable source for understanding mental health patterns \cite{Choudhury2013, Guntuku2017}.


Recent advances in computational methods, particularly machine learning (ML) and deep learning (DL), have shown promise in analyzing social media data to detect signs of depression. These techniques can uncover patterns in language use, emotions, and behaviors that may indicate mental health challenges \cite{Shatte2020, Yazdavar2020}. 

However, applying these methods effectively is not without challenges. A recent systematic review \cite{cao2024mental} highlighted issues such as a lack of diverse datasets, inconsistent data preparation, and inadequate evaluation metrics for imbalanced data \cite{Hargittai2015, Helmy2024}—problems that have also led to inaccuracies in other domains (e.g., \cite{gao2024survey, weng2024ai}). Similarly, Liu et al. \cite{liu2024systematic} identified additional linguistic challenges in ML approaches for detecting deceptive activities on social networks, including biases from insufficient linguistic preprocessing and inconsistent hyperparameter tuning, all of which are pertinent to mental health detection. Moreover, complementary insights from related fields underscore the need for continuous improvements in robust model development \cite{bi2024decoding,zhao2024minimax,tao2023meta,xu2025robust}.


This tutorial is designed to address these gaps by guiding readers through the steps necessary to create reliable and accurate models for depression detection using social media data. It focuses on practical techniques to:
\begin{itemize}
\item Collect and preprocess data, including handling language challenges like sarcasm or negations.
\item Build and optimize models with attention to tuning and evaluation.
\item Use appropriate metrics for datasets where depressive posts are a minority.
\end{itemize}

Our goal is to provide a clear, step-by-step approach that researchers and practitioners can use to improve their methods. By addressing common challenges in this field, we hope to encourage more robust and ethical use of technology for improving mental health outcomes.

\section{Method}
This section provides a comprehensive overview of the methodological framework employed in this study, detailing the processes for data preparation, model development, and evaluation metrics. All analyses and model implementations were conducted using Python 3, leveraging popular libraries such as \texttt{pandas} for data manipulation, \texttt{scikit-learn} for machine learning, \texttt{PyTorch} for deep learning, and \texttt{Transformers} for pre-trained language models. These tools enabled efficient preprocessing, hyperparameter optimization, and performance evaluation. The following subsections elaborate on the key steps and methodologies involved in the study.

\subsection{Data Preparation}
\subsubsection{Data Sources and Collection Methods}
A sufficiently representative dataset is essential for machine-learning-based mental health detection. This study utilized the Sentiment Analysis for Mental Health dataset, available on \href{https://www.kaggle.com/datasets/suchintikasarkar/sentiment-analysis-for-mental-health/data}{Kaggle}. The dataset integrates textual content from multiple repositories focused on mental health topics, including depression, anxiety, stress, bipolar disorder, personality disorders, and suicidal ideation. The primary sources of these data are social media platforms such as Reddit, Twitter, and Facebook, where individuals frequently discuss personal experiences, emotional states, and mental health concerns.

The dataset was originally compiled using platform-specific APIs (e.g., Reddit, Twitter, and Facebook) and web scraping tools, allowing for the collection of substantial volumes of publicly available text data. After the acquisition, duplicates were removed, irrelevant and spam content was filtered, and mental health labels were standardized to ensure consistency across repositories. Personal identifiers were removed to safeguard privacy and ensure compliance with ethical guidelines for data usage. The final dataset was consolidated into a structured CSV file with unique identifiers for each entry.

Although the dataset combines data from multiple platforms to provide a diverse corpus, it is not free from limitations. Differences in platform demographics, such as age, cultural background, and communication styles, may affect the generalizability of models trained on this data. Additionally, linguistic variability, including colloquialisms, slang, and code-switching, reflects the informal nature of social media communication. While this diversity enriches the dataset, it also presents challenges for natural language processing (NLP) techniques, particularly in tokenization and embedding generation. To address these complexities, the preprocessing pipeline was designed to handle diverse linguistic patterns and balance class distributions where needed.

\subsubsection{Data Preprocessing}
A standardized preprocessing pipeline was applied to prepare the dataset for training both machine learning (ML) and deep learning (DL) models. These steps ensured consistency in data handling while accommodating the unique requirements of each modeling approach:
\vspace{-3mm}
\begin{itemize}
\item \textbf{Text Cleaning:} Social media text often contains noise such as URLs, HTML tags, mentions, hashtags, special characters, and extra whitespace. These elements were systematically removed using regular expressions to create cleaner input for both ML and DL models.
\vspace{-1.5mm}
\item \textbf{Lowercasing:} All text was converted to lowercase to maintain uniformity across the dataset and minimize redundancy in text representation.
\vspace{-1.5mm}
\item \textbf{Stopword Removal:} Commonly used words that provide little semantic value (e.g., “the,” “and,” “is”) were excluded using the stopword list available in the Natural Language Toolkit (NLTK) \cite{nltk_toolkit}, reducing noise while retaining meaningful content.
\vspace{-1.5mm}
\item \textbf{Lemmatization:} Words were reduced to their base forms (e.g., “running,” “ran,” “runs” → “run”) using NLTK's Lemmatizer. This step normalized variations of words, aiding both feature extraction and embedding generation.
\end{itemize}
\vspace{-3mm}

The dataset was divided into training, validation, and testing subsets using a two-step random sampling process with a fixed random seed to ensure reproducibility. First, 20\% of the data was set aside as the test set. The remaining 80\% was then further divided into a training set (60\% of the original data) and a validation set (20\% of the original data). This split ensured that the models were trained on the majority of the data while reserving separate subsets for hyperparameter tuning and final performance evaluation.

\subsubsection{Class Labeling}
The dataset’s class labels were prepared as follows: (1) For \textbf{multi-class classification}, the labels included six categories: Normal, Depression, Suicidal, Anxiety, Stress, and Personality Disorder. (2) For \textbf{binary classification}, the labels were grouped into two classes: Normal and Abnormal.

\subsubsection{Feature Transformation for ML Models}
For ML models, an additional step, TF-IDF Vectorization, was necessary to transform the text into structured features. The cleaned text was converted into numerical representations using Term Frequency–Inverse Document Frequency (TF-IDF), which captured term frequencies while down-weighting overly frequent words. The vectorizer was configured to extract up to 1,000 features and account for both unigrams and bigrams (n-gram range: 1–2).

\subsubsection{Code Availability}
The code for data preprocessing, including text cleaning, class labeling, and dataset splitting, is publicly available on GitHub (the link will be provided upon acceptance).

\subsection{Model Development}
This study employed a range of machine learning (ML) and deep learning (DL) models to analyze and classify mental health statuses based on textual data. Each model was selected to explore specific aspects of the data, from linear interpretability to handling complex patterns and long-range dependencies. Detailed implementation code for all models, including hyperparameter tuning and evaluation, is available on GitHub. Below, we provide an overview of each model, its methodology, and its performance in the context of binary and multi-class mental health classification tasks.

\subsubsection{Logistic Regression}
Logistic regression is one of the most widely used methods for classification tasks and has long been employed in social science and biomedical research \cite{hosmer2000applied, ding2025efficientpowerfultradeoffsmachine}. In the context of mental health detection, it provides a straightforward yet interpretable modeling framework, translating linear combinations of predictors (e.g., term frequencies, sentiment scores, and linguistic features) into estimated probabilities of class membership through the logit function.

The logistic regression model predicts the probability of a binary outcome using the following expression:
\begin{equation}\nonumber
\hat{y} = \frac{1}{1 + \exp(-\mathbf{w}^\top \mathbf{x} - b)},
\end{equation}
where $\hat{y}$ represents the predicted probability, $\mathbf{w}$ is the vector of model coefficients, $\mathbf{x}$ is the feature vector, and $b$ is the bias term. For multi-class classification, the model generalizes to predict probabilities for $K$ classes using the softmax function:
\begin{equation}\nonumber
P(y = k \mid \mathbf{x}) = \frac{\exp(\mathbf{w}_k^\top \mathbf{x} + b_k)}{\sum_{j=1}^{K} \exp(\mathbf{w}_j^\top \mathbf{x} + b_j)},
\end{equation}
where $k \in \{1, \dots, K\}$ represents the class index.

Both binary and multi-class logistic regression models were optimized using cross-entropy loss during training and configured to converge with a maximum iteration limit of 1,000. Regularization was applied to prevent overfitting, using $\ell_2$ (ridge) regularization, which penalizes large coefficients by adding their squared magnitude to the loss function:
\begin{equation}\nonumber
\mathcal{L} = - \frac{1}{n} \sum_{i=1}^{n} \left[ y_i \log(\hat{y}_i) + (1 - y_i) \log(1 - \hat{y}_i) \right] + \lambda \|\boldsymbol{\beta}\|_2^2,
\end{equation}
where $\lambda$ controls the regularization strength, $y_i$ is the true label, $\hat{y}_i$ is the predicted probability, and $\boldsymbol{\beta}$ represents the model coefficients.

Hyperparameter tuning was conducted using a grid search across several parameters. The regularization strength (\(C\)), which is the inverse of the regularization parameter \(\lambda\), was tested with values such as 0.1, 1, and 10. Various optimizers, including \texttt{liblinear} (Library for Large Linear Classification), \texttt{lbfgs} (Limited-memory Broyden–Fletcher–Goldfarb–Shanno), and \texttt{saga} (Stochastic Average Gradient Augmented), were evaluated for optimization. To address class imbalance, the \texttt{class\_weight} parameter was explored with options for \texttt{balanced} and \texttt{None}. For multi-class tasks, the \texttt{multinomial} strategy was employed, while the \texttt{one-vs-rest} strategy was implicitly applied for binary classification scenarios.

For both binary and multi-class tasks, the weighted F1 score was used as the primary evaluation metric, ensuring balanced performance across categories, including minority classes. A combined grid search configuration was applied for both tasks, as their hyperparameter spaces largely overlapped. The best configurations effectively handled class imbalance using the \texttt{class\_weight=`balanced'} parameter, yielding robust performance on the validation and test sets.

The logistic regressions were implemented using the \texttt{LogisticRegression} class from the \texttt{scikit-learn} library. Detailed implementation code for logistic regression, including hyperparameter tuning and evaluation, is available on GitHub.

\subsubsection{Support Vector Machine (SVM)}
Support Vector Machines (SVMs) are supervised learning models that are widely used for both classification and regression tasks. Originally introduced by \citeA{cortes1995support}, SVMs aim to find the optimal hyperplane that maximizes the margin between data points of different classes. The margin is defined as the distance between the closest data points (support vectors) from each class to the hyperplane. By maximizing this margin, SVMs achieve better generalization for unseen data.

For a linearly separable dataset, the decision boundary is defined as:
\begin{equation}\nonumber
f(\mathbf{x}) = \mathbf{w}^T \mathbf{x} + b,
\end{equation}
where $\mathbf{w}$ is the weight vector, $\mathbf{x}$ is the feature vector, and $b$ is the bias term. The optimal hyperplane is determined by solving the following optimization problem:
\begin{align}
\min_{\mathbf{w}, b} & \quad \frac{1}{2} \|\mathbf{w}\|^2, \nonumber\\
\text{subject to} & \quad y_i (\mathbf{w}^T \mathbf{x}_i + b) \geq 1, \quad i = 1, \dots, N,\nonumber
\end{align}
where $y_i \in \{-1, +1\}$ are the class labels.

For datasets that are not linearly separable, the optimization problem is modified to include a penalty for misclassifications:
\begin{align}
\min_{\mathbf{w}, b, \xi} & \quad \frac{1}{2} \|\mathbf{w}\|^2 + C \sum_{i=1}^N \xi_i, \nonumber\\
\text{subject to} & \quad y_i (\mathbf{w}^T \mathbf{x}_i + b) \geq 1 - \xi_i, \quad \xi_i \geq 0, \quad i = 1, \dots, N,\nonumber
\end{align}
where $\xi_i$ are slack variables that allow for misclassifications, and $C > 0$ is the regularization parameter that controls the trade-off between maximizing the margin and minimizing classification errors.

Kernel methods enable SVMs to handle nonlinearly separable data by mapping the input features into a higher-dimensional space where linear separation becomes possible. This mapping is performed implicitly using a kernel function \( K(\mathbf{x}_i, \mathbf{x}_j) \), which computes the inner product in the transformed space:
\begin{equation}\nonumber
K(\mathbf{x}_i, \mathbf{x}_j) = \phi(\mathbf{x}_i)^T \phi(\mathbf{x}_j),
\end{equation}
where \( \phi(\cdot) \) represents the mapping function.

Several commonly used kernel functions are available, each suited for different data characteristics:

\vspace{-3mm}
\begin{enumerate}
\item \textbf{Linear Kernel:}
\begin{equation}\nonumber
K(\mathbf{x}_i, \mathbf{x}_j) = \mathbf{x}_i^T \mathbf{x}_j
\end{equation}
This kernel computes the dot product of the input vectors and is suitable for linearly separable data.
\vspace{-1.5mm}
\item \textbf{Polynomial Kernel:}
\begin{equation}\nonumber
K(\mathbf{x}_i, \mathbf{x}_j) = (\mathbf{x}_i^T \mathbf{x}_j + c)^d,
\end{equation}
where \( c \) is a constant and \( d \) is the degree of the polynomial. This kernel is useful for capturing polynomial feature interactions.
\vspace{-1.5mm}
\item \textbf{Radial Basis Function (RBF) Kernel:}
\begin{equation}\nonumber
K(\mathbf{x}_i, \mathbf{x}_j) = \exp\left(-\gamma \|\mathbf{x}_i - \mathbf{x}_j\|^2\right),
\end{equation}
where \( \gamma \) controls the influence of individual training samples. The RBF kernel is widely used for its flexibility in modeling complex, nonlinear patterns.
\vspace{-1.5mm}
\item \textbf{Sigmoid Kernel:}
\begin{equation}\nonumber
K(\mathbf{x}_i, \mathbf{x}_j) = \tanh(\alpha \mathbf{x}_i^T \mathbf{x}_j + c),
\end{equation}
where \( \alpha \) is a scaling factor and \( c \) is a bias term. This kernel is inspired by neural network activation functions and is suitable for data with specific characteristics.
\vspace{-1.5mm}
\item \textbf{Custom Kernels:}
Custom-defined kernels can be tailored for domain-specific tasks, offering flexibility for unique datasets or similarity metrics.
\end{enumerate}
\vspace{-3mm}

In this project, kernel selection was based on preliminary experiments, with the linear and radial basis function (RBF) kernels being the primary choice due to its ability to model complex, nonlinear relationships effectively.

For both binary and multi-class classification tasks, the same hyperparameter tuning strategy was employed. A grid search was conducted over the following hyperparameters:
\vspace{-3mm}
\begin{itemize}
\item Regularization parameter $C$: values of \{0.1, 1, 10\}.
\vspace{-1.5mm}
\item Kernel type: linear and RBF.
\vspace{-1.5mm}
\item Class weight: balanced or none.
\vspace{-1.5mm}
\item Gamma (for RBF kernel): scale and auto.
\end{itemize}
\vspace{-3mm}
The grid search aimed to identify the optimal combination of hyperparameters using the weighted F1 score as the primary evaluation metric. For multi-class classification, the one-vs-one strategy inherent to the \texttt{SVC} implementation was used.

The loss function for SVM is analogous to logistic regression, as both models minimize the cross-entropy loss during optimization. However, for SVM, the hinge loss is typically used for linear separable cases, defined as:
\begin{equation}\nonumber
\mathcal{L}_{\text{hinge}} = \frac{1}{N} \sum_{i=1}^N \max(0, 1 - y_i f(\mathbf{x}_i)).
\end{equation}

The SVM models were implemented with the \texttt{SVC} class from \texttt{scikit-learn}. Detailed implementation code for SVMs, including grid search and evaluation, is available on GitHub.

\subsubsection{Tree-Based Models}
Classification and Regression Trees (CARTs) are versatile tools used for analyzing categorical outcomes (classification tasks). The CART algorithm constructs a binary decision tree by recursively partitioning the data based on covariates, optimizing a predefined splitting criterion. For classification tasks, the quality of a split is typically evaluated using impurity measures such as Gini impurity or entropy \cite{bishop2006pattern}. The Gini impurity for a node is defined as:
\begin{equation}\nonumber
G = \sum_{i=1}^C p_i (1 - p_i),
\end{equation}
where \(p_i\) is the proportion of observations in class \(i\) at the given node, and \(C\) is the total number of classes.

Alternatively, entropy can be used to measure impurity:
\begin{equation}\nonumber
H = -\sum_{i=1}^C p_i \log(p_i),
\end{equation}
where \(p_i\) represents the same class proportions as in the Gini impurity formula. Lower impurity values indicate greater homogeneity within a node.

At each step, the algorithm selects the split that minimizes the weighted impurity of the child nodes. The impurity reduction for a given split is computed as:
\begin{equation}\nonumber
\Delta I = I_{\text{parent}} - \left( \frac{n_L}{n} I_L + \frac{n_R}{n} I_R \right),
\end{equation}
where \(I_{\text{parent}}\) is the impurity of the parent node, \(I_L\) and \(I_R\) are the impurities of the left and right child nodes, \(n_L\) and \(n_R\) are the number of observations in the left and right child nodes, and \(n\) is the total number of observations in the parent node.

The splitting process continues until one stopping criterion is met. Common criteria include: (1) a minimum number of samples in a node, (2) a maximum tree depth, and (3) No further reduction in impurity beyond a predefined threshold.

To address overfitting, pruning techniques \cite{breiman1984classification} are employed. Pruning reduces the tree size by removing splits that contribute minimally to predictive performance, improving the model's generalizability.

Due to their tendency to overfit, simple CART models were not evaluated in this project. Instead, ensemble methods like Random Forests and Gradient Boosted Trees, which combine multiple CART models, were used for improved robustness.

\paragraph{Random Forests}
Random Forests are ensemble learning methods that aggregate multiple decision trees parallelly to enhance classification performance. By building trees on bootstrap samples of the data and introducing random feature selection at each split, Random Forests reduce overfitting and improve generalization. Each tree is trained on a random bootstrap sample, where data points are sampled with replacement from the original dataset, meaning some observations may appear multiple times in the training sample, while others are excluded. Additionally, Random Forests introduce randomness during the tree-building process by selecting a random subset of covariates at each split instead of considering all available covariates. This randomization decorrelates the trees, reduces variance, and enhances the model's robustness. For classification tasks, the final prediction is determined by majority voting across all trees \cite{breiman2001random}.

To further mitigate overfitting, each tree in the Random Forest is grown to its full depth without pruning, fitting the bootstrap sample as accurately as possible. Hyperparameters such as the number of trees (\texttt{n\_estimators}), the maximum depth of each tree (\texttt{max\_depth}), and the minimum samples required to split a node (\texttt{min\_samples\_split}) or form a leaf (\texttt{min\_samples\_leaf}) play a critical role in balancing bias and variance. The parameter \texttt{class\_weight}, when set to \texttt{`balanced'}, adjusts weights inversely proportional to class frequencies, effectively addressing the class imbalance.

A grid search approach was employed to optimize key hyperparameters for both binary and multi-class classification tasks. The parameter grid explored values such as 50, 100, and 200 for the number of trees (\texttt{n\_estimators}), depths of 10, 20, or unrestricted (\texttt{None}) for \texttt{max\_depth}, and split criteria (\texttt{min\_samples\_split} and \texttt{min\_samples\_leaf}) to control tree complexity. The weighted F1 score served as the primary evaluation metric to account for imbalances in the dataset. For the binary classification task, the best-performing model, determined through validation, effectively handled class imbalance and demonstrated robust predictive performance for distinguishing between Normal and Abnormal mental health statuses. In addition to traditional hyperparameter tuning techniques, recent studies have explored novel metaheuristic approaches to optimize Random Forest parameters. For instance, Tan et al. \cite{tan2024dung} proposed an improved dung beetle optimizer that refines hyperparameter tuning, further enhancing model performance.

For the multi-class classification task, the same hyperparameter grid was used with a slightly reduced scope to streamline the search process. The weighted F1 score guided model selection across all classes, including Normal, Depression, Anxiety, and Personality Disorder. The optimal model achieved balanced performance across multiple categories, leveraging Random Forests' ability to aggregate predictions from diverse decision trees. 

Random Forests’ inherent feature importance metrics provided additional insights into the most influential predictors for mental health classification. This capability enhances interpretability by highlighting covariates that most strongly influence predictions. The Random Forest models were built using the \texttt{RandomForestClassifier} from \texttt{scikit-learn}. Parameter grids for the number of estimators, maximum depth, and other parameters were evaluated with \texttt{GridSearchCV}. Detailed implementation code, including grid search and evaluation procedures, is available on GitHub.

\paragraph{Light Gradient Boosting Machine (LightGBM)}
Light Gradient Boosting Machine (LightGBM) is a gradient-boosting framework optimized for efficiency and scalability, particularly in handling large datasets and high-dimensional data. Gradient Boosting Machines (GBM) work by sequentially building decision trees, where each new tree corrects the errors made by the previous ones, leading to highly accurate predictions. However, traditional GBM frameworks can be computationally intensive, especially for large datasets \cite{friedman2001greedy}. Unlike traditional Gradient Boosting Machines (GBMs), LightGBM employs a leaf-wise tree growth strategy, which enables deeper splits in dense data regions, enhancing performance by focusing complexity where it is most needed. Additional optimizations, such as histogram-based feature binning, reduce memory usage and accelerate training. These enhancements make LightGBM faster and more resource-efficient than standard GBM implementations, without compromising predictive accuracy \cite{ke2017lightgbm}.

Key hyperparameters tuned for LightGBM included the number of boosting iterations (\texttt{n\_estimators}), learning rate, maximum tree depth (\texttt{max\_depth}), number of leaves (\texttt{num\_leaves}), and minimum child samples (\texttt{min\_child\_samples}). To address the class imbalance, the \texttt{class\_weight} parameter was tested with both \texttt{`balanced'} and \texttt{None} options. Grid search was employed to explore all possible combinations of these hyperparameters, and the weighted F1 score was used as the primary metric for selecting the best configuration.

LightGBM was applied to both binary and multi-class mental health classification tasks. For binary classification, the model differentiated between Normal and Abnormal statuses. For multi-class classification, it predicted categories such as Normal, Depression, Anxiety, and Personality Disorder using the \texttt{multiclass} objective. Hyperparameter tuning via grid search ensured balanced performance across all categories, guided by the weighted F1 score. 

The best-performing models demonstrated robust predictive power, evaluated using precision, recall, F1 scores, confusion matrices, and one-vs-rest ROC curves. Additionally, LightGBM’s feature importance metrics provided interpretability by highlighting the most influential linguistic and sentiment-based features. Its combination of high performance, scalability, and interpretability made LightGBM a key component in this project. The LightGBM models were developed using the \texttt{LGBMClassifier} from the \texttt{lightgbm} library. Hyperparameter tuning, including the number of boosting iterations, learning rate, and tree depth, was performed using \texttt{GridSearchCV}.
Detailed implementation code, including grid search procedures, is available on GitHub.

\subsubsection{A Lite version of Bidirectional Encoder Representations from Transformers (ALBERT)}
A Lite version of Bidirectional Encoder Representations from Transformers (BERT), known as ALBERT \cite{lan2020albert}, is a transformer-based model designed to improve efficiency while maintaining performance. While BERT \cite{devlin2019bert} is highly effective for a wide range of natural language processing (NLP) tasks, it is computationally expensive and memory-intensive due to its large number of parameters. ALBERT addresses these limitations by introducing parameter-sharing across layers and factorized embedding parameterization, which significantly reduces the number of parameters without sacrificing model capacity. Additionally, ALBERT employs Sentence Order Prediction (SOP) as an auxiliary task to enhance pretraining, improving its ability to capture sentence-level coherence. These optimizations make ALBERT a lightweight yet powerful alternative to BERT, capable of achieving competitive performance with reduced memory and computational requirements, making it particularly suitable for large-scale text classification tasks like mental health detection.

In this project, ALBERT was employed for both binary and multi-class classification tasks. For binary classification, the model was fine-tuned to differentiate between Normal and Abnormal mental health statuses, while for multi-class classification, it was configured to predict multiple categories, including Normal, Depression, Anxiety, and Personality Disorder. The implementation leveraged the pre-trained \texttt{Albert-base-v2} model, with random hyperparameter tuning conducted over 10 iterations to optimize the learning rate, number of epochs, and dropout rates. The weighted F1 score served as the primary evaluation metric throughout the tuning process.

For both binary and multi-class classification tasks, hyperparameter tuning was conducted to optimize learning rates between $10^{-5}$ and $10^{-4}$, dropout rates between 0.1 and 0.5, and epochs ranging from 3 to 5. For binary classification, the model achieved high validation F1 scores and demonstrated strong generalization on the test set. For multi-class classification, the objective was adjusted to predict seven categories, with weighted cross-entropy loss applied to address class imbalances and ensure adequate representation of minority categories. The final models were evaluated on the test set using metrics such as accuracy, weighted F1 scores, and confusion matrices.

ALBERT’s architecture efficiently captures long-range dependencies in text while retaining the computational advantages of its lightweight design. The use of random hyperparameter tuning further refined its performance, enabling robust classification for both binary and multi-class tasks. The ALBERT models were fine-tuned with the \texttt{Transformers} (\texttt{AlbertTokenizer} and \texttt{AlbertForSequenceClassification}) library from Hugging Face. Hyperparameter tuning was conducted manually through random search over learning rates, dropout rates, and epochs. etailed implementation code, including data preparation, training, and hyperparameter tuning, is available on GitHub.

\subsubsection{Gated Recurrent Units (GRUs)}
Gated Recurrent Units (GRUs) are a type of recurrent neural network (RNN) designed to capture sequential dependencies in data, making them particularly effective for natural language processing (NLP) tasks such as text classification \cite{cho2014learning}. Compared to Long Short-Term Memory networks (LSTMs), GRUs are computationally more efficient due to their simplified architecture, which combines the forget and input gates into a single update gate. This efficiency allows GRUs to model long-range dependencies while reducing the number of trainable parameters.

In this study, GRUs were employed for both binary and multi-class mental health classification tasks. For binary classification, the model was configured to differentiate between Normal and Abnormal mental health statuses. For multi-class classification, it was adapted to predict categories such as Normal, Depression, Anxiety, and Personality Disorder. 

The GRU architecture comprised three key components:
\vspace{-3mm}
\begin{enumerate}
\item \textbf{Embedding Layer}: Converts token indices into dense vector representations of a fixed embedding dimension.
\vspace{-1.5mm}
\item \textbf{GRU Layer}: Processes input sequences and retains contextual information across time steps, utilizing only the final hidden state for classification.
\vspace{-1.5mm}
\item \textbf{Fully Connected Layer}: Maps the hidden state to output logits corresponding to the number of classes.
\end{enumerate}
\vspace{-3mm}
Dropout regularization was applied to prevent overfitting, and a weighted cross-entropy loss function was used to address class imbalances in the dataset.

For both binary and multi-class classification tasks, hyperparameter tuning was conducted using random search across predefined ranges. The parameters optimized included embedding dimensions (150--250), hidden dimensions (256--768), learning rates ($10^{-4}$--$10^{-3}$), and epochs (5--10). The weighted F1 score served as the primary evaluation metric during validation. The best-performing models achieved high F1 scores on validation datasets and demonstrated robust generalization on the test sets.

GRUs excelled at capturing sequential patterns in text, enabling the model to identify linguistic cues associated with mental health conditions. Despite being less interpretable than tree-based models, their lightweight architecture ensured computational efficiency and strong performance in text-based classification tasks. The GRU models were implemented with the \texttt{torch.nn} module in PyTorch. Key layers included \texttt{nn.Embedding}, \texttt{nn.GRU}, and \texttt{nn.Linear}. Optimization was performed using the \texttt{torch.optim.Adam} optimizer, and class weights were applied using \texttt{nn.CrossEntropyLoss}.
Detailed implementation code, including data preprocessing, model training, and evaluation, is available on GitHub.

\subsection{Evaluation Metrics}
When modeling mental health statuses—particularly for conditions like depression or suicidal ideation—class distributions are often skewed. In many real-world scenarios, the “positive” class (e.g., individuals experiencing depression) is underrepresented compared to the “negative” class (e.g., no mental health issue). This imbalance renders certain evaluation metrics, such as accuracy, less informative: a model that predicts “no issue” for every instance might still achieve high accuracy if the majority class dominates. Consequently, more nuanced metrics are preferred to evaluate the performance of classification models:

\subsubsection{Precision} 
Precision measures the proportion of positive predictions that are truly positive:
\begin{equation}
 \text{Precision} = \frac{\text{True Positives}}{\text{True Positives} + \text{False Positives}}.
\end{equation}
For instance, in depression detection, high precision indicates that most users flagged as “depressed” indeed exhibit depressive content. While precision minimizes false alarms, focusing on it exclusively can be risky. A model that generates very few positive predictions may achieve artificially high precision while missing many genuinely positive cases.

\subsubsection{Recall (Sensitivity)} 
Recall captures the proportion of actual positives correctly identified:
\begin{equation}\nonumber
 \text{Recall} = \frac{\text{True Positives}}{\text{True Positives} + \text{False Negatives}}.
\end{equation}
In depression detection, recall is critical because failing to recognize at-risk individuals (false negatives) can have serious consequences. A model with low recall risks overlooking individuals who need intervention.

\subsubsection{F1 Score} 
The F1 score serves as the harmonic mean of precision and recall, providing a balance between these two metrics \cite{powers2011evaluation}:
\begin{equation}\nonumber
 F1 = 2 \cdot \frac{\text{Precision} \cdot \text{Recall}}{\text{Precision} + \text{Recall}}.
\end{equation}
The F1 score is particularly useful in imbalanced classification scenarios because it penalizes extreme trade-offs, such as very high precision coupled with very low recall. In mental health detection, achieving a high F1 score ensures the model can effectively identify positive cases while maintaining a reasonable level of precision in its predictions.

\subsubsection{Area Under the Receiver Operating Characteristic Curve (AUROC)} 
AUROC provides an aggregate measure of performance across all possible classification thresholds. It evaluates the model's ability to discriminate between positive and negative classes. However, in the presence of severe class imbalance, AUROC may not fully reflect the challenges posed by a majority class dominating the dataset. Nevertheless, it remains valuable for assessing model performance across varying decision thresholds \cite{davis2006relationship}.

\section{Results}
This section presents the findings from the analysis of the dataset and the evaluation of machine learning and deep learning models for mental health classification. First, we provide an \textit{Overview of Mental Health Distribution}, highlighting the inherent class imbalances within the dataset and their implications for model development. Next, the \textit{Hyperparameter Optimization} subsection details the parameter tuning process, which ensures that each model performs at its best configuration for both binary and multi-class classification tasks. Finally, the \textit{Model Performance Evaluation} subsection compares the models' performance based on key metrics, including F1 scores and Area Under the Receiver Operating Characteristic Curve (AUROC). Additionally, nuanced observations, such as the challenges associated with underrepresented classes, are discussed to provide deeper insights into the modeling outcomes.

\subsection{Overview of Mental Health Distribution}
Before hyperparameter optimization and model evaluation, an analysis of the dataset’s class distributions was conducted to highlight potential challenges in classification. The dataset, sourced from Kaggle, contains a total of 51,074 unique statements categorized into three primary groups: \textit{Normal} (31\%), \textit{Depression} (29\%), and \textit{Other} (40\%). The \textit{Other} category encompasses a range of mental health statuses such as \textit{Anxiety}, \textit{Stress}, and \textit{Personality Disorder}, among others.

\textbf{Figure~\ref{fig:multi-class}} illustrates the expanded distribution of mental health statuses across seven detailed categories in the multi-class classification setup. The dataset shows a significant imbalance, with categories such as \textit{Normal}, \textit{Depression}, and \textit{Suicidal} dominating the distribution, while others like \textit{Stress} and \textit{Personality Disorder} are notably underrepresented. This class imbalance poses challenges for multi-class classification tasks, particularly for the accurate identification of minority classes. Addressing such imbalances requires techniques like class-weighted loss functions and the use of metrics such as weighted F1 scores for model evaluation.

\figurehere{1}

\figurehere{2}

For the binary classification task, the dataset is divided into two classes: \textit{Normal} and \textit{Abnormal}. The distribution, shown in \textbf{Figure~\ref{fig:binary-class}}, reveals that the \textit{Abnormal} class (labeled as 1) accounts for approximately twice the number of records as the \textit{Normal} class (labeled as 0). Although the imbalance is less severe compared to the multi-class scenario, it still necessitates strategies to ensure that the minority class (\textit{Normal}) is adequately captured during model training.

\subsection{Hyperparameter Optimization}
Hyperparameter optimization is a critical step in enhancing the performance of machine learning (ML) and deep learning (DL) models. For this study, a grid search or random search approach was employed to systematically explore a predefined range of hyperparameters for each model. The primary evaluation metric used to select the best-performing hyperparameter configuration was the weighted F1 score, as it effectively balances precision and recall, particularly in the presence of imbalanced class distributions. This approach ensures that the selected models perform robustly across both binary and multi-class mental health classification tasks. 

The optimized hyperparameters for each model, alongside their corresponding weighted F1 scores on the test set, are summarized in Table~\ref{tbl1:opt_hp}. These results highlight the configurations that achieved the best trade-off between underfitting and overfitting, providing insight into the hyperparameter values critical to the classification tasks.

\tablehere{1}

\subsection{Model Performance Evaluation}

The evaluation metrics, including F1 scores (\textbf{Table~\ref{tbl2:f1-scores}}) and Area Under the Receiver Operating Characteristic Curve (AUROC) (\textbf{Table~\ref{tbl3:auc_scores}}), reveal minimal numeric differences across the models for both binary and multi-class classification tasks. This consistency in performance can be attributed to two primary factors. First, each model underwent rigorous hyperparameter tuning, ensuring only the best configurations were used for evaluation. Second, the dataset size, being of medium volume, provided sufficient information for machine learning models to achieve strong performance, while deep learning models could not fully demonstrate their potential advantages due to the limited data scale.

\tablehere{2}

\tablehere{3}

In the binary classification task, all models exhibited competitive F1 scores and AUROC values, effectively balancing precision and recall while distinguishing between normal and abnormal mental health statuses. Deep learning models such as \textit{ALBERT} and \textit{GRU} demonstrated slightly superior performance, achieving AUROC values of 0.95 and 0.94, respectively, which highlights their ability to capture complex linguistic patterns. Machine learning models, including \textit{Logistic Regression} and \textit{LightGBM}, also performed well, with AUROC scores of 0.93, underscoring their robustness in simpler classification settings.

In the multi-class classification task, a slight decline in performance was observed compared to the binary task. This decline aligns with the increased complexity of distinguishing between seven mental health categories. Nevertheless, deep learning models retained their advantage, with \textit{GRU} and \textit{LightGBM} achieving the highest micro-average AUROC scores of 0.97, followed closely by \textit{ALBERT} with an AUROC of 0.95. Machine learning models such as \textit{Logistic Regression} and \textit{Random Forest} also performed commendably, with AUROC scores of 0.96, demonstrating their ability to handle multi-class tasks effectively when optimized.

Another important observation in the multi-class classification task is the consistently lower AUROC scores for Depression (Class 2) across all machine learning models, with values not exceeding 0.90. While deep learning models demonstrated a slight improvement, their performance for this class remained comparatively weaker than for other categories. This difficulty likely arises from the significant overlap between Depression (Class 2) and other categories in both linguistic and contextual features. The reduced AUROC scores highlight the models' challenges in effectively distinguishing Depression, resulting in higher misclassification rates. These findings emphasize the need for refined feature engineering techniques or more sophisticated model architectures to enhance the separability and accurate classification of this particular class.

The minimal differences in performance metrics across models suggest that the combined effects of comprehensive hyperparameter optimization and dataset size contributed significantly to these results. Binary classification consistently outperformed multi-class classification, likely due to its reduced complexity and fewer decision boundaries. While deep learning models demonstrated their ability to capture intricate patterns, machine learning models offered competitive performance, making them practical alternatives for medium-sized datasets.

Performance metrics for F1 scores and AUROC values are detailed in \textbf{Table~\ref{tbl2:f1-scores}} and \textbf{Table~\ref{tbl3:auc_scores}}, respectively. This analysis highlights the importance of balancing model complexity with dataset characteristics and emphasizes the critical role of hyperparameter tuning in achieving optimal results.

\section{Discussion}

This tutorial serves as a practical resource to address key methodological and analytical challenges in mental health detection on social media, as identified in the systematic review \cite{cao2024mental}. By focusing on best practices and reproducible methods, the tutorial aims to advance research quality and promote equitable outcomes in this important field.

A critical issue identified in the review is the narrow scope of datasets, which are often limited to specific social media platforms, languages, or geographic regions. This lack of diversity restricts the generalizability of findings. In this tutorial, strategies for expanding data diversity are explored, including integrating datasets across multiple platforms, collecting data from underrepresented regions, and analyzing multilingual content. These efforts aim to make research outcomes more inclusive and applicable to diverse populations.

Text preprocessing emerged as another key challenge, particularly in handling linguistic complexities such as negations and sarcasm. These nuances are critical for accurately interpreting mental health expressions. This tutorial offers practical guidelines for building preprocessing pipelines that address these complexities. Techniques for advanced tokenization, feature extraction, and managing contextual meanings are discussed to enhance the reliability of text-based analyses.

Research practices related to model optimization and evaluation were also found to be inconsistent in many studies. Hyperparameter tuning and robust data partitioning are essential for reliable outcomes, yet they are often inadequately implemented. This tutorial provides step-by-step instructions for optimizing models and ensuring fair evaluations, emphasizing the importance of strategies like cross-validation and train-validation-test splits. By following these practices, researchers can reduce bias and improve the validity of their results. 

Evaluation metrics were another area of concern, with many studies relying on accuracy despite its limitations in imbalanced datasets. This tutorial highlights the importance of metrics such as precision, recall, F1-score, and AUROC, which provide a more balanced assessment of model performance. Additionally, practical approaches to managing imbalanced datasets, including oversampling, undersampling, and synthetic data generation, are illustrated.

Transparency in reporting methodologies and results is a foundational element of reproducible research. This tutorial encourages researchers to document their processes comprehensively, including data collection, preprocessing, model development, and evaluation. Sharing code and datasets is also emphasized, fostering collaboration and allowing other researchers to validate findings.

Ethical considerations are central to mental health research, particularly when using sensitive social media data. This tutorial stresses the need for privacy protection and adherence to ethical standards, ensuring that research respects the rights and dignity of individuals. Responsible data handling and clear communication of ethical practices are essential for maintaining trust and accountability in this field.

By addressing these challenges, this tutorial equips researchers with the tools and practices needed to improve the quality and impact of their work. Ultimately, these advancements contribute to the broader goal of promoting equitable and effective mental health interventions on a global scale.

\printbibliography

\newpage
\begin{sidewaystable}[ht]
\centering
\caption{Best Hyperparameters for Binary and Multi-Class Classification Models}
\label{tbl1:opt_hp}
\begin{tabular}{|l|p{5cm}|p{5cm}|p{9cm}|}
\hline
\textbf{Model} & \textbf{Best Parameters (Binary)} & \textbf{Best Parameters (Multi-Class)} & \textbf{Interpretation} \\
\hline
\textbf{Logistic Regression} & 
\texttt{\{C: 10, solver: `liblinear', penalty: `l2', class\_weight: None\}} & 
\texttt{\{C: 10, solver: `lbfgs', penalty: `l2', multi\_class: `multinomial', class\_weight: `balanced'\}} & 
For binary tasks, \texttt{liblinear} is chosen for smaller datasets. For multi-class, \texttt{lbfgs} supports \texttt{`multinomial'} strategy to optimize across multiple categories. Regularization strength (\texttt{C}) of 10 prevents overfitting. \\
\hline
\textbf{SVM} & 
\texttt{\{C: 1, kernel: `rbf', class\_weight: `balanced', gamma: `scale'\}} & 
\texttt{\{C: 1, kernel: `rbf', class\_weight: `balanced', gamma: `scale'\}} &
The RBF kernel captures nonlinear relationships in text data, while \texttt{class\_weight: `balanced'} was selected to address class imbalance. Regularization strength (\texttt{C}) balances margin maximization and misclassification. \\
\hline
\textbf{Random Forest} & 
\texttt{\{n\_estimators: 100, max\_depth: None, min\_samples\_split: 5, min\_samples\_leaf: 1, class\_weight: `balanced'\}} & 
\texttt{\{n\_estimators: 200, max\_depth: None, min\_samples\_split: 2, min\_samples\_leaf: 2, class\_weight: `balanced'\}} & 
For binary tasks, 100 trees ensure stability. For multi-class, 200 trees improve coverage of complex class distributions. Weighted class adjustments handle imbalances. \\
\hline
\textbf{LightGBM} & 
\texttt{\{n\_estimators: 100, learning\_rate: 0.1, max\_depth: -1, num\_leaves: 50, min\_child\_samples: 10, class\_weight: None\}} & 
\texttt{\{n\_estimators: 100, learning\_rate: 0.1, max\_depth: None, num\_leaves: 63, class\_weight: `balanced'\}} & 
For both tasks, LightGBM achieves efficiency via leaf-wise tree growth. For multi-class, additional leaves (63) improve representation of minority classes. \\
\hline
\textbf{ALBERT} & 
\texttt{\{lr: 1.46e-05, epochs: 4, dropout: 0.11\}} & 
\texttt{\{lr: 1.17e-05, epochs: 4, dropout: 0.15\}} & 
ALBERT’s lightweight architecture fine-tunes well with minimal learning rates and dropout for regularization. Minor adjustments improve class representation in multi-class settings. \\
\hline
\textbf{GRU} & 
\texttt{\{embedding\_dim: 156, hidden\_dim: 467, lr: 0.0004, epochs: 5\}} & 
\texttt{\{embedding\_dim: 236, hidden\_dim: 730, lr: 0.0003, epochs: 6\}} & 
Embedding dimensions and hidden states effectively capture sequential dependencies in text. Multi-class configurations benefit from higher hidden dimensions and epochs. \\
\hline
\end{tabular}
\end{sidewaystable}

\begin{table}[htbp]
\centering
\caption{Weighted F1 Scores of Models for Binary and Multi-Class Classification Tasks}
\label{tbl2:f1-scores}
\begin{tabular}{lcc}
\hline
\textbf{Model} & \textbf{Binary Classification} & \textbf{Multi-Class Classification } \\
\hline
Support Vector Machine (SVM) & 0.9401 & 0.7610 \\
Logistic Regression & 0.9345 & 0.7498 \\
Random Forest & 0.9359 & 0.7478 \\
LightGBM & 0.9358 & 0.7747 \\
ALBERT& 0.9576 & 0.7841 \\
Gated Recurrent Units (GRU) & 0.9512 & 0.7756 \\
\hline
\end{tabular}
\end{table}

\begin{table}[ht]
\centering
\caption{Area Under the Receiver Operating Characteristic Curve (AUROC) Scores for Binary and Multi-Class Classification Tasks}
\label{tbl3:auc_scores}
\resizebox{\textwidth}{!}{%
\begin{tabular}{lcc}
\toprule
\textbf{Model} & \textbf{Binary Classification AUROC} & \textbf{Multi-Class Classification Micro-Average AUROC} \\
\midrule
SVM & 0.93 & 0.95 \\
Logistic Regression & 0.93 & 0.96 \\
Random Forest & 0.92 & 0.96 \\
LightGBM & 0.93 & 0.97 \\
ALBERT & 0.95 & 0.97 \\
GRU & 0.94 & 0.97 \\
\bottomrule
\end{tabular}%
}
\end{table}

\newpage
\begin{figure}[h!]
 \centering
 \includegraphics[width=0.8\textwidth]{Figures/multi.png}
 \caption{Multi-class distribution of mental health statuses.}
 \label{fig:multi-class}
\end{figure}

\begin{figure}[h!]
 \centering
 \includegraphics[width=0.8\textwidth]{Figures/binary.png}
 \caption{Binary classification distribution of \textit{Normal} versus \textit{Abnormal} mental health statuses.}
 \label{fig:binary-class}
\end{figure}

\end{document}




%% 
%% Copyright (C) 2019 by Daniel A. Weiss <daniel.weiss.led at gmail.com>
%% 
%% This work may be distributed and/or modified under the
%% conditions of the LaTeX Project Public License (LPPL), either
%% version 1.3c of this license or (at your option) any later
%% version.The latest version of this license is in the file:
%% 
%% http://www.latex-project.org/lppl.txt
%% 
%% Users may freely modify these files without permission, as long as the
%% copyright line and this statement are maintained intact.
%% 
%% This work is not endorsed by, affiliated with, or probably even known
%% by, the American Psychological Association.
%% 
%% This work is "maintained" (as per LPPL maintenance status) by
%% Daniel A. Weiss.
%% 
%% This work consists of the fileapa7.dtx
%% and the derived files apa7.ins,
%% apa7.cls,
%% apa7.pdf,
%% README,
%% APA7american.txt,
%% APA7british.txt,
%% APA7dutch.txt,
%% APA7english.txt,
%% APA7german.txt,
%% APA7ngerman.txt,
%% APA7greek.txt,
%% APA7czech.txt,
%% APA7turkish.txt,
%% APA7endfloat.cfg,
%% Figure1.pdf,
%% shortsample.tex,
%% longsample.tex, and
%% bibliography.bib.
%% 
%%
%% End of file `./samples/longsample.tex'.

%\nocite{*}
%\printbibliography

\subsection{Lloyd-Max Algorithm}
\label{subsec:Lloyd-Max}
For a given quantization bitwidth $B$ and an operand $\bm{X}$, the Lloyd-Max algorithm finds $2^B$ quantization levels $\{\hat{x}_i\}_{i=1}^{2^B}$ such that quantizing $\bm{X}$ by rounding each scalar in $\bm{X}$ to the nearest quantization level minimizes the quantization MSE. 

The algorithm starts with an initial guess of quantization levels and then iteratively computes quantization thresholds $\{\tau_i\}_{i=1}^{2^B-1}$ and updates quantization levels $\{\hat{x}_i\}_{i=1}^{2^B}$. Specifically, at iteration $n$, thresholds are set to the midpoints of the previous iteration's levels:
\begin{align*}
    \tau_i^{(n)}=\frac{\hat{x}_i^{(n-1)}+\hat{x}_{i+1}^{(n-1)}}2 \text{ for } i=1\ldots 2^B-1
\end{align*}
Subsequently, the quantization levels are re-computed as conditional means of the data regions defined by the new thresholds:
\begin{align*}
    \hat{x}_i^{(n)}=\mathbb{E}\left[ \bm{X} \big| \bm{X}\in [\tau_{i-1}^{(n)},\tau_i^{(n)}] \right] \text{ for } i=1\ldots 2^B
\end{align*}
where to satisfy boundary conditions we have $\tau_0=-\infty$ and $\tau_{2^B}=\infty$. The algorithm iterates the above steps until convergence.

Figure \ref{fig:lm_quant} compares the quantization levels of a $7$-bit floating point (E3M3) quantizer (left) to a $7$-bit Lloyd-Max quantizer (right) when quantizing a layer of weights from the GPT3-126M model at a per-tensor granularity. As shown, the Lloyd-Max quantizer achieves substantially lower quantization MSE. Further, Table \ref{tab:FP7_vs_LM7} shows the superior perplexity achieved by Lloyd-Max quantizers for bitwidths of $7$, $6$ and $5$. The difference between the quantizers is clear at 5 bits, where per-tensor FP quantization incurs a drastic and unacceptable increase in perplexity, while Lloyd-Max quantization incurs a much smaller increase. Nevertheless, we note that even the optimal Lloyd-Max quantizer incurs a notable ($\sim 1.5$) increase in perplexity due to the coarse granularity of quantization. 

\begin{figure}[h]
  \centering
  \includegraphics[width=0.7\linewidth]{sections/figures/LM7_FP7.pdf}
  \caption{\small Quantization levels and the corresponding quantization MSE of Floating Point (left) vs Lloyd-Max (right) Quantizers for a layer of weights in the GPT3-126M model.}
  \label{fig:lm_quant}
\end{figure}

\begin{table}[h]\scriptsize
\begin{center}
\caption{\label{tab:FP7_vs_LM7} \small Comparing perplexity (lower is better) achieved by floating point quantizers and Lloyd-Max quantizers on a GPT3-126M model for the Wikitext-103 dataset.}
\begin{tabular}{c|cc|c}
\hline
 \multirow{2}{*}{\textbf{Bitwidth}} & \multicolumn{2}{|c|}{\textbf{Floating-Point Quantizer}} & \textbf{Lloyd-Max Quantizer} \\
 & Best Format & Wikitext-103 Perplexity & Wikitext-103 Perplexity \\
\hline
7 & E3M3 & 18.32 & 18.27 \\
6 & E3M2 & 19.07 & 18.51 \\
5 & E4M0 & 43.89 & 19.71 \\
\hline
\end{tabular}
\end{center}
\end{table}

\subsection{Proof of Local Optimality of LO-BCQ}
\label{subsec:lobcq_opt_proof}
For a given block $\bm{b}_j$, the quantization MSE during LO-BCQ can be empirically evaluated as $\frac{1}{L_b}\lVert \bm{b}_j- \bm{\hat{b}}_j\rVert^2_2$ where $\bm{\hat{b}}_j$ is computed from equation (\ref{eq:clustered_quantization_definition}) as $C_{f(\bm{b}_j)}(\bm{b}_j)$. Further, for a given block cluster $\mathcal{B}_i$, we compute the quantization MSE as $\frac{1}{|\mathcal{B}_{i}|}\sum_{\bm{b} \in \mathcal{B}_{i}} \frac{1}{L_b}\lVert \bm{b}- C_i^{(n)}(\bm{b})\rVert^2_2$. Therefore, at the end of iteration $n$, we evaluate the overall quantization MSE $J^{(n)}$ for a given operand $\bm{X}$ composed of $N_c$ block clusters as:
\begin{align*}
    \label{eq:mse_iter_n}
    J^{(n)} = \frac{1}{N_c} \sum_{i=1}^{N_c} \frac{1}{|\mathcal{B}_{i}^{(n)}|}\sum_{\bm{v} \in \mathcal{B}_{i}^{(n)}} \frac{1}{L_b}\lVert \bm{b}- B_i^{(n)}(\bm{b})\rVert^2_2
\end{align*}

At the end of iteration $n$, the codebooks are updated from $\mathcal{C}^{(n-1)}$ to $\mathcal{C}^{(n)}$. However, the mapping of a given vector $\bm{b}_j$ to quantizers $\mathcal{C}^{(n)}$ remains as  $f^{(n)}(\bm{b}_j)$. At the next iteration, during the vector clustering step, $f^{(n+1)}(\bm{b}_j)$ finds new mapping of $\bm{b}_j$ to updated codebooks $\mathcal{C}^{(n)}$ such that the quantization MSE over the candidate codebooks is minimized. Therefore, we obtain the following result for $\bm{b}_j$:
\begin{align*}
\frac{1}{L_b}\lVert \bm{b}_j - C_{f^{(n+1)}(\bm{b}_j)}^{(n)}(\bm{b}_j)\rVert^2_2 \le \frac{1}{L_b}\lVert \bm{b}_j - C_{f^{(n)}(\bm{b}_j)}^{(n)}(\bm{b}_j)\rVert^2_2
\end{align*}

That is, quantizing $\bm{b}_j$ at the end of the block clustering step of iteration $n+1$ results in lower quantization MSE compared to quantizing at the end of iteration $n$. Since this is true for all $\bm{b} \in \bm{X}$, we assert the following:
\begin{equation}
\begin{split}
\label{eq:mse_ineq_1}
    \tilde{J}^{(n+1)} &= \frac{1}{N_c} \sum_{i=1}^{N_c} \frac{1}{|\mathcal{B}_{i}^{(n+1)}|}\sum_{\bm{b} \in \mathcal{B}_{i}^{(n+1)}} \frac{1}{L_b}\lVert \bm{b} - C_i^{(n)}(b)\rVert^2_2 \le J^{(n)}
\end{split}
\end{equation}
where $\tilde{J}^{(n+1)}$ is the the quantization MSE after the vector clustering step at iteration $n+1$.

Next, during the codebook update step (\ref{eq:quantizers_update}) at iteration $n+1$, the per-cluster codebooks $\mathcal{C}^{(n)}$ are updated to $\mathcal{C}^{(n+1)}$ by invoking the Lloyd-Max algorithm \citep{Lloyd}. We know that for any given value distribution, the Lloyd-Max algorithm minimizes the quantization MSE. Therefore, for a given vector cluster $\mathcal{B}_i$ we obtain the following result:

\begin{equation}
    \frac{1}{|\mathcal{B}_{i}^{(n+1)}|}\sum_{\bm{b} \in \mathcal{B}_{i}^{(n+1)}} \frac{1}{L_b}\lVert \bm{b}- C_i^{(n+1)}(\bm{b})\rVert^2_2 \le \frac{1}{|\mathcal{B}_{i}^{(n+1)}|}\sum_{\bm{b} \in \mathcal{B}_{i}^{(n+1)}} \frac{1}{L_b}\lVert \bm{b}- C_i^{(n)}(\bm{b})\rVert^2_2
\end{equation}

The above equation states that quantizing the given block cluster $\mathcal{B}_i$ after updating the associated codebook from $C_i^{(n)}$ to $C_i^{(n+1)}$ results in lower quantization MSE. Since this is true for all the block clusters, we derive the following result: 
\begin{equation}
\begin{split}
\label{eq:mse_ineq_2}
     J^{(n+1)} &= \frac{1}{N_c} \sum_{i=1}^{N_c} \frac{1}{|\mathcal{B}_{i}^{(n+1)}|}\sum_{\bm{b} \in \mathcal{B}_{i}^{(n+1)}} \frac{1}{L_b}\lVert \bm{b}- C_i^{(n+1)}(\bm{b})\rVert^2_2  \le \tilde{J}^{(n+1)}   
\end{split}
\end{equation}

Following (\ref{eq:mse_ineq_1}) and (\ref{eq:mse_ineq_2}), we find that the quantization MSE is non-increasing for each iteration, that is, $J^{(1)} \ge J^{(2)} \ge J^{(3)} \ge \ldots \ge J^{(M)}$ where $M$ is the maximum number of iterations. 
%Therefore, we can say that if the algorithm converges, then it must be that it has converged to a local minimum. 
\hfill $\blacksquare$


\begin{figure}
    \begin{center}
    \includegraphics[width=0.5\textwidth]{sections//figures/mse_vs_iter.pdf}
    \end{center}
    \caption{\small NMSE vs iterations during LO-BCQ compared to other block quantization proposals}
    \label{fig:nmse_vs_iter}
\end{figure}

Figure \ref{fig:nmse_vs_iter} shows the empirical convergence of LO-BCQ across several block lengths and number of codebooks. Also, the MSE achieved by LO-BCQ is compared to baselines such as MXFP and VSQ. As shown, LO-BCQ converges to a lower MSE than the baselines. Further, we achieve better convergence for larger number of codebooks ($N_c$) and for a smaller block length ($L_b$), both of which increase the bitwidth of BCQ (see Eq \ref{eq:bitwidth_bcq}).


\subsection{Additional Accuracy Results}
%Table \ref{tab:lobcq_config} lists the various LOBCQ configurations and their corresponding bitwidths.
\begin{table}
\setlength{\tabcolsep}{4.75pt}
\begin{center}
\caption{\label{tab:lobcq_config} Various LO-BCQ configurations and their bitwidths.}
\begin{tabular}{|c||c|c|c|c||c|c||c|} 
\hline
 & \multicolumn{4}{|c||}{$L_b=8$} & \multicolumn{2}{|c||}{$L_b=4$} & $L_b=2$ \\
 \hline
 \backslashbox{$L_A$\kern-1em}{\kern-1em$N_c$} & 2 & 4 & 8 & 16 & 2 & 4 & 2 \\
 \hline
 64 & 4.25 & 4.375 & 4.5 & 4.625 & 4.375 & 4.625 & 4.625\\
 \hline
 32 & 4.375 & 4.5 & 4.625& 4.75 & 4.5 & 4.75 & 4.75 \\
 \hline
 16 & 4.625 & 4.75& 4.875 & 5 & 4.75 & 5 & 5 \\
 \hline
\end{tabular}
\end{center}
\end{table}

%\subsection{Perplexity achieved by various LO-BCQ configurations on Wikitext-103 dataset}

\begin{table} \centering
\begin{tabular}{|c||c|c|c|c||c|c||c|} 
\hline
 $L_b \rightarrow$& \multicolumn{4}{c||}{8} & \multicolumn{2}{c||}{4} & 2\\
 \hline
 \backslashbox{$L_A$\kern-1em}{\kern-1em$N_c$} & 2 & 4 & 8 & 16 & 2 & 4 & 2  \\
 %$N_c \rightarrow$ & 2 & 4 & 8 & 16 & 2 & 4 & 2 \\
 \hline
 \hline
 \multicolumn{8}{c}{GPT3-1.3B (FP32 PPL = 9.98)} \\ 
 \hline
 \hline
 64 & 10.40 & 10.23 & 10.17 & 10.15 &  10.28 & 10.18 & 10.19 \\
 \hline
 32 & 10.25 & 10.20 & 10.15 & 10.12 &  10.23 & 10.17 & 10.17 \\
 \hline
 16 & 10.22 & 10.16 & 10.10 & 10.09 &  10.21 & 10.14 & 10.16 \\
 \hline
  \hline
 \multicolumn{8}{c}{GPT3-8B (FP32 PPL = 7.38)} \\ 
 \hline
 \hline
 64 & 7.61 & 7.52 & 7.48 &  7.47 &  7.55 &  7.49 & 7.50 \\
 \hline
 32 & 7.52 & 7.50 & 7.46 &  7.45 &  7.52 &  7.48 & 7.48  \\
 \hline
 16 & 7.51 & 7.48 & 7.44 &  7.44 &  7.51 &  7.49 & 7.47  \\
 \hline
\end{tabular}
\caption{\label{tab:ppl_gpt3_abalation} Wikitext-103 perplexity across GPT3-1.3B and 8B models.}
\end{table}

\begin{table} \centering
\begin{tabular}{|c||c|c|c|c||} 
\hline
 $L_b \rightarrow$& \multicolumn{4}{c||}{8}\\
 \hline
 \backslashbox{$L_A$\kern-1em}{\kern-1em$N_c$} & 2 & 4 & 8 & 16 \\
 %$N_c \rightarrow$ & 2 & 4 & 8 & 16 & 2 & 4 & 2 \\
 \hline
 \hline
 \multicolumn{5}{|c|}{Llama2-7B (FP32 PPL = 5.06)} \\ 
 \hline
 \hline
 64 & 5.31 & 5.26 & 5.19 & 5.18  \\
 \hline
 32 & 5.23 & 5.25 & 5.18 & 5.15  \\
 \hline
 16 & 5.23 & 5.19 & 5.16 & 5.14  \\
 \hline
 \multicolumn{5}{|c|}{Nemotron4-15B (FP32 PPL = 5.87)} \\ 
 \hline
 \hline
 64  & 6.3 & 6.20 & 6.13 & 6.08  \\
 \hline
 32  & 6.24 & 6.12 & 6.07 & 6.03  \\
 \hline
 16  & 6.12 & 6.14 & 6.04 & 6.02  \\
 \hline
 \multicolumn{5}{|c|}{Nemotron4-340B (FP32 PPL = 3.48)} \\ 
 \hline
 \hline
 64 & 3.67 & 3.62 & 3.60 & 3.59 \\
 \hline
 32 & 3.63 & 3.61 & 3.59 & 3.56 \\
 \hline
 16 & 3.61 & 3.58 & 3.57 & 3.55 \\
 \hline
\end{tabular}
\caption{\label{tab:ppl_llama7B_nemo15B} Wikitext-103 perplexity compared to FP32 baseline in Llama2-7B and Nemotron4-15B, 340B models}
\end{table}

%\subsection{Perplexity achieved by various LO-BCQ configurations on MMLU dataset}


\begin{table} \centering
\begin{tabular}{|c||c|c|c|c||c|c|c|c|} 
\hline
 $L_b \rightarrow$& \multicolumn{4}{c||}{8} & \multicolumn{4}{c||}{8}\\
 \hline
 \backslashbox{$L_A$\kern-1em}{\kern-1em$N_c$} & 2 & 4 & 8 & 16 & 2 & 4 & 8 & 16  \\
 %$N_c \rightarrow$ & 2 & 4 & 8 & 16 & 2 & 4 & 2 \\
 \hline
 \hline
 \multicolumn{5}{|c|}{Llama2-7B (FP32 Accuracy = 45.8\%)} & \multicolumn{4}{|c|}{Llama2-70B (FP32 Accuracy = 69.12\%)} \\ 
 \hline
 \hline
 64 & 43.9 & 43.4 & 43.9 & 44.9 & 68.07 & 68.27 & 68.17 & 68.75 \\
 \hline
 32 & 44.5 & 43.8 & 44.9 & 44.5 & 68.37 & 68.51 & 68.35 & 68.27  \\
 \hline
 16 & 43.9 & 42.7 & 44.9 & 45 & 68.12 & 68.77 & 68.31 & 68.59  \\
 \hline
 \hline
 \multicolumn{5}{|c|}{GPT3-22B (FP32 Accuracy = 38.75\%)} & \multicolumn{4}{|c|}{Nemotron4-15B (FP32 Accuracy = 64.3\%)} \\ 
 \hline
 \hline
 64 & 36.71 & 38.85 & 38.13 & 38.92 & 63.17 & 62.36 & 63.72 & 64.09 \\
 \hline
 32 & 37.95 & 38.69 & 39.45 & 38.34 & 64.05 & 62.30 & 63.8 & 64.33  \\
 \hline
 16 & 38.88 & 38.80 & 38.31 & 38.92 & 63.22 & 63.51 & 63.93 & 64.43  \\
 \hline
\end{tabular}
\caption{\label{tab:mmlu_abalation} Accuracy on MMLU dataset across GPT3-22B, Llama2-7B, 70B and Nemotron4-15B models.}
\end{table}


%\subsection{Perplexity achieved by various LO-BCQ configurations on LM evaluation harness}

\begin{table} \centering
\begin{tabular}{|c||c|c|c|c||c|c|c|c|} 
\hline
 $L_b \rightarrow$& \multicolumn{4}{c||}{8} & \multicolumn{4}{c||}{8}\\
 \hline
 \backslashbox{$L_A$\kern-1em}{\kern-1em$N_c$} & 2 & 4 & 8 & 16 & 2 & 4 & 8 & 16  \\
 %$N_c \rightarrow$ & 2 & 4 & 8 & 16 & 2 & 4 & 2 \\
 \hline
 \hline
 \multicolumn{5}{|c|}{Race (FP32 Accuracy = 37.51\%)} & \multicolumn{4}{|c|}{Boolq (FP32 Accuracy = 64.62\%)} \\ 
 \hline
 \hline
 64 & 36.94 & 37.13 & 36.27 & 37.13 & 63.73 & 62.26 & 63.49 & 63.36 \\
 \hline
 32 & 37.03 & 36.36 & 36.08 & 37.03 & 62.54 & 63.51 & 63.49 & 63.55  \\
 \hline
 16 & 37.03 & 37.03 & 36.46 & 37.03 & 61.1 & 63.79 & 63.58 & 63.33  \\
 \hline
 \hline
 \multicolumn{5}{|c|}{Winogrande (FP32 Accuracy = 58.01\%)} & \multicolumn{4}{|c|}{Piqa (FP32 Accuracy = 74.21\%)} \\ 
 \hline
 \hline
 64 & 58.17 & 57.22 & 57.85 & 58.33 & 73.01 & 73.07 & 73.07 & 72.80 \\
 \hline
 32 & 59.12 & 58.09 & 57.85 & 58.41 & 73.01 & 73.94 & 72.74 & 73.18  \\
 \hline
 16 & 57.93 & 58.88 & 57.93 & 58.56 & 73.94 & 72.80 & 73.01 & 73.94  \\
 \hline
\end{tabular}
\caption{\label{tab:mmlu_abalation} Accuracy on LM evaluation harness tasks on GPT3-1.3B model.}
\end{table}

\begin{table} \centering
\begin{tabular}{|c||c|c|c|c||c|c|c|c|} 
\hline
 $L_b \rightarrow$& \multicolumn{4}{c||}{8} & \multicolumn{4}{c||}{8}\\
 \hline
 \backslashbox{$L_A$\kern-1em}{\kern-1em$N_c$} & 2 & 4 & 8 & 16 & 2 & 4 & 8 & 16  \\
 %$N_c \rightarrow$ & 2 & 4 & 8 & 16 & 2 & 4 & 2 \\
 \hline
 \hline
 \multicolumn{5}{|c|}{Race (FP32 Accuracy = 41.34\%)} & \multicolumn{4}{|c|}{Boolq (FP32 Accuracy = 68.32\%)} \\ 
 \hline
 \hline
 64 & 40.48 & 40.10 & 39.43 & 39.90 & 69.20 & 68.41 & 69.45 & 68.56 \\
 \hline
 32 & 39.52 & 39.52 & 40.77 & 39.62 & 68.32 & 67.43 & 68.17 & 69.30  \\
 \hline
 16 & 39.81 & 39.71 & 39.90 & 40.38 & 68.10 & 66.33 & 69.51 & 69.42  \\
 \hline
 \hline
 \multicolumn{5}{|c|}{Winogrande (FP32 Accuracy = 67.88\%)} & \multicolumn{4}{|c|}{Piqa (FP32 Accuracy = 78.78\%)} \\ 
 \hline
 \hline
 64 & 66.85 & 66.61 & 67.72 & 67.88 & 77.31 & 77.42 & 77.75 & 77.64 \\
 \hline
 32 & 67.25 & 67.72 & 67.72 & 67.00 & 77.31 & 77.04 & 77.80 & 77.37  \\
 \hline
 16 & 68.11 & 68.90 & 67.88 & 67.48 & 77.37 & 78.13 & 78.13 & 77.69  \\
 \hline
\end{tabular}
\caption{\label{tab:mmlu_abalation} Accuracy on LM evaluation harness tasks on GPT3-8B model.}
\end{table}

\begin{table} \centering
\begin{tabular}{|c||c|c|c|c||c|c|c|c|} 
\hline
 $L_b \rightarrow$& \multicolumn{4}{c||}{8} & \multicolumn{4}{c||}{8}\\
 \hline
 \backslashbox{$L_A$\kern-1em}{\kern-1em$N_c$} & 2 & 4 & 8 & 16 & 2 & 4 & 8 & 16  \\
 %$N_c \rightarrow$ & 2 & 4 & 8 & 16 & 2 & 4 & 2 \\
 \hline
 \hline
 \multicolumn{5}{|c|}{Race (FP32 Accuracy = 40.67\%)} & \multicolumn{4}{|c|}{Boolq (FP32 Accuracy = 76.54\%)} \\ 
 \hline
 \hline
 64 & 40.48 & 40.10 & 39.43 & 39.90 & 75.41 & 75.11 & 77.09 & 75.66 \\
 \hline
 32 & 39.52 & 39.52 & 40.77 & 39.62 & 76.02 & 76.02 & 75.96 & 75.35  \\
 \hline
 16 & 39.81 & 39.71 & 39.90 & 40.38 & 75.05 & 73.82 & 75.72 & 76.09  \\
 \hline
 \hline
 \multicolumn{5}{|c|}{Winogrande (FP32 Accuracy = 70.64\%)} & \multicolumn{4}{|c|}{Piqa (FP32 Accuracy = 79.16\%)} \\ 
 \hline
 \hline
 64 & 69.14 & 70.17 & 70.17 & 70.56 & 78.24 & 79.00 & 78.62 & 78.73 \\
 \hline
 32 & 70.96 & 69.69 & 71.27 & 69.30 & 78.56 & 79.49 & 79.16 & 78.89  \\
 \hline
 16 & 71.03 & 69.53 & 69.69 & 70.40 & 78.13 & 79.16 & 79.00 & 79.00  \\
 \hline
\end{tabular}
\caption{\label{tab:mmlu_abalation} Accuracy on LM evaluation harness tasks on GPT3-22B model.}
\end{table}

\begin{table} \centering
\begin{tabular}{|c||c|c|c|c||c|c|c|c|} 
\hline
 $L_b \rightarrow$& \multicolumn{4}{c||}{8} & \multicolumn{4}{c||}{8}\\
 \hline
 \backslashbox{$L_A$\kern-1em}{\kern-1em$N_c$} & 2 & 4 & 8 & 16 & 2 & 4 & 8 & 16  \\
 %$N_c \rightarrow$ & 2 & 4 & 8 & 16 & 2 & 4 & 2 \\
 \hline
 \hline
 \multicolumn{5}{|c|}{Race (FP32 Accuracy = 44.4\%)} & \multicolumn{4}{|c|}{Boolq (FP32 Accuracy = 79.29\%)} \\ 
 \hline
 \hline
 64 & 42.49 & 42.51 & 42.58 & 43.45 & 77.58 & 77.37 & 77.43 & 78.1 \\
 \hline
 32 & 43.35 & 42.49 & 43.64 & 43.73 & 77.86 & 75.32 & 77.28 & 77.86  \\
 \hline
 16 & 44.21 & 44.21 & 43.64 & 42.97 & 78.65 & 77 & 76.94 & 77.98  \\
 \hline
 \hline
 \multicolumn{5}{|c|}{Winogrande (FP32 Accuracy = 69.38\%)} & \multicolumn{4}{|c|}{Piqa (FP32 Accuracy = 78.07\%)} \\ 
 \hline
 \hline
 64 & 68.9 & 68.43 & 69.77 & 68.19 & 77.09 & 76.82 & 77.09 & 77.86 \\
 \hline
 32 & 69.38 & 68.51 & 68.82 & 68.90 & 78.07 & 76.71 & 78.07 & 77.86  \\
 \hline
 16 & 69.53 & 67.09 & 69.38 & 68.90 & 77.37 & 77.8 & 77.91 & 77.69  \\
 \hline
\end{tabular}
\caption{\label{tab:mmlu_abalation} Accuracy on LM evaluation harness tasks on Llama2-7B model.}
\end{table}

\begin{table} \centering
\begin{tabular}{|c||c|c|c|c||c|c|c|c|} 
\hline
 $L_b \rightarrow$& \multicolumn{4}{c||}{8} & \multicolumn{4}{c||}{8}\\
 \hline
 \backslashbox{$L_A$\kern-1em}{\kern-1em$N_c$} & 2 & 4 & 8 & 16 & 2 & 4 & 8 & 16  \\
 %$N_c \rightarrow$ & 2 & 4 & 8 & 16 & 2 & 4 & 2 \\
 \hline
 \hline
 \multicolumn{5}{|c|}{Race (FP32 Accuracy = 48.8\%)} & \multicolumn{4}{|c|}{Boolq (FP32 Accuracy = 85.23\%)} \\ 
 \hline
 \hline
 64 & 49.00 & 49.00 & 49.28 & 48.71 & 82.82 & 84.28 & 84.03 & 84.25 \\
 \hline
 32 & 49.57 & 48.52 & 48.33 & 49.28 & 83.85 & 84.46 & 84.31 & 84.93  \\
 \hline
 16 & 49.85 & 49.09 & 49.28 & 48.99 & 85.11 & 84.46 & 84.61 & 83.94  \\
 \hline
 \hline
 \multicolumn{5}{|c|}{Winogrande (FP32 Accuracy = 79.95\%)} & \multicolumn{4}{|c|}{Piqa (FP32 Accuracy = 81.56\%)} \\ 
 \hline
 \hline
 64 & 78.77 & 78.45 & 78.37 & 79.16 & 81.45 & 80.69 & 81.45 & 81.5 \\
 \hline
 32 & 78.45 & 79.01 & 78.69 & 80.66 & 81.56 & 80.58 & 81.18 & 81.34  \\
 \hline
 16 & 79.95 & 79.56 & 79.79 & 79.72 & 81.28 & 81.66 & 81.28 & 80.96  \\
 \hline
\end{tabular}
\caption{\label{tab:mmlu_abalation} Accuracy on LM evaluation harness tasks on Llama2-70B model.}
\end{table}

%\section{MSE Studies}
%\textcolor{red}{TODO}


\subsection{Number Formats and Quantization Method}
\label{subsec:numFormats_quantMethod}
\subsubsection{Integer Format}
An $n$-bit signed integer (INT) is typically represented with a 2s-complement format \citep{yao2022zeroquant,xiao2023smoothquant,dai2021vsq}, where the most significant bit denotes the sign.

\subsubsection{Floating Point Format}
An $n$-bit signed floating point (FP) number $x$ comprises of a 1-bit sign ($x_{\mathrm{sign}}$), $B_m$-bit mantissa ($x_{\mathrm{mant}}$) and $B_e$-bit exponent ($x_{\mathrm{exp}}$) such that $B_m+B_e=n-1$. The associated constant exponent bias ($E_{\mathrm{bias}}$) is computed as $(2^{{B_e}-1}-1)$. We denote this format as $E_{B_e}M_{B_m}$.  

\subsubsection{Quantization Scheme}
\label{subsec:quant_method}
A quantization scheme dictates how a given unquantized tensor is converted to its quantized representation. We consider FP formats for the purpose of illustration. Given an unquantized tensor $\bm{X}$ and an FP format $E_{B_e}M_{B_m}$, we first, we compute the quantization scale factor $s_X$ that maps the maximum absolute value of $\bm{X}$ to the maximum quantization level of the $E_{B_e}M_{B_m}$ format as follows:
\begin{align}
\label{eq:sf}
    s_X = \frac{\mathrm{max}(|\bm{X}|)}{\mathrm{max}(E_{B_e}M_{B_m})}
\end{align}
In the above equation, $|\cdot|$ denotes the absolute value function.

Next, we scale $\bm{X}$ by $s_X$ and quantize it to $\hat{\bm{X}}$ by rounding it to the nearest quantization level of $E_{B_e}M_{B_m}$ as:

\begin{align}
\label{eq:tensor_quant}
    \hat{\bm{X}} = \text{round-to-nearest}\left(\frac{\bm{X}}{s_X}, E_{B_e}M_{B_m}\right)
\end{align}

We perform dynamic max-scaled quantization \citep{wu2020integer}, where the scale factor $s$ for activations is dynamically computed during runtime.

\subsection{Vector Scaled Quantization}
\begin{wrapfigure}{r}{0.35\linewidth}
  \centering
  \includegraphics[width=\linewidth]{sections/figures/vsquant.jpg}
  \caption{\small Vectorwise decomposition for per-vector scaled quantization (VSQ \citep{dai2021vsq}).}
  \label{fig:vsquant}
\end{wrapfigure}
During VSQ \citep{dai2021vsq}, the operand tensors are decomposed into 1D vectors in a hardware friendly manner as shown in Figure \ref{fig:vsquant}. Since the decomposed tensors are used as operands in matrix multiplications during inference, it is beneficial to perform this decomposition along the reduction dimension of the multiplication. The vectorwise quantization is performed similar to tensorwise quantization described in Equations \ref{eq:sf} and \ref{eq:tensor_quant}, where a scale factor $s_v$ is required for each vector $\bm{v}$ that maps the maximum absolute value of that vector to the maximum quantization level. While smaller vector lengths can lead to larger accuracy gains, the associated memory and computational overheads due to the per-vector scale factors increases. To alleviate these overheads, VSQ \citep{dai2021vsq} proposed a second level quantization of the per-vector scale factors to unsigned integers, while MX \citep{rouhani2023shared} quantizes them to integer powers of 2 (denoted as $2^{INT}$).

\subsubsection{MX Format}
The MX format proposed in \citep{rouhani2023microscaling} introduces the concept of sub-block shifting. For every two scalar elements of $b$-bits each, there is a shared exponent bit. The value of this exponent bit is determined through an empirical analysis that targets minimizing quantization MSE. We note that the FP format $E_{1}M_{b}$ is strictly better than MX from an accuracy perspective since it allocates a dedicated exponent bit to each scalar as opposed to sharing it across two scalars. Therefore, we conservatively bound the accuracy of a $b+2$-bit signed MX format with that of a $E_{1}M_{b}$ format in our comparisons. For instance, we use E1M2 format as a proxy for MX4.

\begin{figure}
    \centering
    \includegraphics[width=1\linewidth]{sections//figures/BlockFormats.pdf}
    \caption{\small Comparing LO-BCQ to MX format.}
    \label{fig:block_formats}
\end{figure}

Figure \ref{fig:block_formats} compares our $4$-bit LO-BCQ block format to MX \citep{rouhani2023microscaling}. As shown, both LO-BCQ and MX decompose a given operand tensor into block arrays and each block array into blocks. Similar to MX, we find that per-block quantization ($L_b < L_A$) leads to better accuracy due to increased flexibility. While MX achieves this through per-block $1$-bit micro-scales, we associate a dedicated codebook to each block through a per-block codebook selector. Further, MX quantizes the per-block array scale-factor to E8M0 format without per-tensor scaling. In contrast during LO-BCQ, we find that per-tensor scaling combined with quantization of per-block array scale-factor to E4M3 format results in superior inference accuracy across models. 


\end{document}