%%
%% This is file `sample-manuscript.tex',
%% generated with the docstrip utility.
%%
%% The original source files were:
%%
%% samples.dtx  (with options: `all,proceedings,bibtex,manuscript')
%% 
%% IMPORTANT NOTICE:
%% 
%% For the copyright see the source file.
%% 
%% Any modified versions of this file must be renamed
%% with new filenames distinct from sample-manuscript.tex.
%% 
%% For distribution of the original source see the terms
%% for copying and modification in the file samples.dtx.
%% 
%% This generated file may be distributed as long as the
%% original source files, as listed above, are part of the
%% same distribution. (The sources need not necessarily be
%% in the same archive or directory.)
%%
%%
%% Commands for TeXCount
%TC:macro \cite [option:text,text]
%TC:macro \citep [option:text,text]
%TC:macro \citet [option:text,text]
%TC:envir table 0 1
%TC:envir table* 0 1
%TC:envir tabular [ignore] word
%TC:envir displaymath 0 word
%TC:envir math 0 word
%TC:envir comment 0 0
%%
%%
%% The first command in your LaTeX source must be the \documentclass
%% command.
%%
%% For submission and review of your manuscript please change the
%% command to \documentclass[manuscript, screen, review]{acmart}.
%%
%% When submitting camera ready or to TAPS, please change the command
%% to \documentclass[sigconf]{acmart} or whichever template is required
%% for your publication.
%%
%%
\PassOptionsToPackage{table,xcdraw}{xcolor}
\documentclass[manuscript,screen]{acmart}
\usepackage{float}
\usepackage{graphicx}
\usepackage{subcaption}
\usepackage{booktabs} 
%%\usepackage[table,xcdraw]{xcolor}
%%\usepackage[dvipsnames]{xcolor}
%%\usepackage{graphicx}
%%\usepackage[dvipsnames]{xcolor}
\usepackage{array}
\usepackage[utf8]{inputenc}
\usepackage{fontenc}
\DeclareUnicodeCharacter{202F}{ }

%%
%% \BibTeX command to typeset BibTeX logo in the docs
\AtBeginDocument{%
  \providecommand\BibTeX{{%
    Bib\TeX}}}

%% Rights management information.  This information is sent to you
%% when you complete the rights form.  These commands have SAMPLE
%% values in them; it is your responsibility as an author to replace
%% the commands and values with those provided to you when you
%% complete the rights form.
\setcopyright{acmlicensed}
\copyrightyear{2018}
\acmYear{2018}
\acmDOI{XXXXXXX.XXXXXXX}

%% These commands are for a PROCEEDINGS abstract or paper.
\acmConference[Conference acronym 'XX]{Make sure to enter the correct
  conference title from your rights confirmation emai}{June 03--05,
  2018}{Woodstock, NY}
%%
%%  Uncomment \acmBooktitle if the title of the proceedings is different
%%  from ``Proceedings of ...''!
%%
%%\acmBooktitle{Woodstock '18: ACM Symposium on Neural Gaze Detection,
%%  June 03--05, 2018, Woodstock, NY}
\acmISBN{978-1-4503-XXXX-X/18/06}


%%
%% Submission ID.
%% Use this when submitting an article to a sponsored event. You'll
%% receive a unique submission ID from the organizers
%% of the event, and this ID should be used as the parameter to this command.
%%\acmSubmissionID{123-A56-BU3}

%%
%% For managing citations, it is recommended to use bibliography
%% files in BibTeX format.
%%
%% You can then either use BibTeX with the ACM-Reference-Format style,
%% or BibLaTeX with the acmnumeric or acmauthoryear sytles, that include
%% support for advanced citation of software artefact from the
%% biblatex-software package, also separately available on CTAN.
%%
%% Look at the sample-*-biblatex.tex files for templates showcasing
%% the biblatex styles.
%%

%%
%% The majority of ACM publications use numbered citations and
%% references.  The command \citestyle{authoryear} switches to the
%% "author year" style.
%%
%% If you are preparing content for an event
%% sponsored by ACM SIGGRAPH, you must use the "author year" style of
%% citations and references.
%% Uncommenting
%% the next command will enable that style.
%%\citestyle{acmauthoryear}


%%
%% end of the preamble, start of the body of the document source.
\begin{document}

%%
%% The "title" command has an optional parameter,
%% allowing the author to define a "short title" to be used in page headers.
\title{Toward Metaphor-Fluid Conversation Design for Voice User Interfaces}

%%
%% The "author" command and its associated commands are used to define
%% the authors and their affiliations.
%% Of note is the shared affiliation of the first two authors, and the
%% "authornote" and "authornotemark" commands
%% used to denote shared contribution to the research.
\author{Smit Desai}
\authornote{Corresponding author}
\email{sm.desai@northeastern.edu}
\orcid{0000-0001-6983-8838}
\affiliation{%
  \institution{Northeastern University}
  \city{Boston}
  \state{Massachusets}
  \country{USA}
}


\author{Jessie Chin}
\affiliation{%
  \institution{University of Illinois, Urbana-Champaign}
  \city{Urbana-Champaign}
  \state{Illinois}
  \country{USA}}

\author{Dakuo Wang}
\affiliation{%
  \institution{Northeastern University}
  \city{Boston}
  \state{Massachusets}
  \country{USA}}
\email{d.wang@northeastern.edu}

\author{Benjamin Cowan}
\affiliation{%
  \institution{University College Dublin}
  \city{Dublin}
  \country{Ireland}}
\email{benjain.cowan@ucd.ie}

\author{Michael Twidale}
\affiliation{%
  \institution{University of Illinois, Urbana-Champaign}
  \city{Urbana-Champaign}
  \country{USA}}
\email{twidale@illinois.edu}

%%
%% By default, the full list of authors will be used in the page
%% headers. Often, this list is too long, and will overlap
%% other information printed in the page headers. This command allows
%% the author to define a more concise list
%% of authors' names for this purpose.
\renewcommand{\shortauthors}{Desai et al.}

%%
%% The abstract is a short summary of the work to be presented in the
%% article.
\begin{abstract}
Metaphors play a critical role in shaping user experiences with Voice User Interfaces (VUIs), yet existing designs often rely on static, human-centric metaphors that fail to adapt to diverse contexts and user needs. This paper introduces Metaphor-Fluid Design, a novel approach that dynamically adjusts metaphorical representations based on conversational use-contexts. We compare this approach to a Default VUI, which characterizes the present implementation of commercial VUIs commonly designed around the persona of an assistant, offering a uniform interaction style across contexts. In Study 1 (N=130), metaphors were mapped to four key use-contexts—commands, information seeking, sociality, and error recovery—along the dimensions of formality and hierarchy, revealing distinct preferences for task-specific metaphorical designs. Study 2 (N=91) evaluates a Metaphor-Fluid VUI against a Default VUI, showing that the Metaphor-Fluid VUI enhances perceived intention to adopt, enjoyment, and likability by aligning better with user expectations for different contexts. However, individual differences in metaphor preferences highlight the need for personalization. These findings challenge the one-size-fits-all paradigm of VUI design and demonstrate the potential of Metaphor-Fluid Design to create more adaptive and engaging human-AI interactions.
\end{abstract}

%%
%% The code below is generated by the tool at http://dl.acm.org/ccs.cfm.
%% Please copy and paste the code instead of the example below.
%%
\begin{CCSXML}
<ccs2012>
<concept>
  <concept_id>10003120.10003138.10003142</concept_id>
    <concept_desc>Human-centered computing~Ubiquitous and mobile computing design and evaluation methods</concept_desc>
    <concept_significance>100</concept_significance>
  </concept>
   <concept>
       <concept_id>10003120.10003121.10003124.10010870</concept_id>
       <concept_desc>Human-centered computing~Natural language interfaces</concept_desc>
       <concept_significance>300</concept_significance>
   </concept>
   <concept>
       <concept_id>10003120.10003121.10003128.10010869</concept_id>
       <concept_desc>Human-centered computing~Auditory feedback</concept_desc>
       <concept_significance>500</concept_significance>
   </concept>
 </ccs2012>
\end{CCSXML}

\ccsdesc[500]{Human-centered computing~Auditory feedback}
\ccsdesc[300]{Human-centered computing~Natural language interfaces}
\ccsdesc[100]{Human-centered computing~Ubiquitous and mobile computing design and evaluation methods}
%%
%% Keywords. The author(s) should pick words that accurately describe
%% the work being presented. Separate the keywords with commas.
\keywords{Conversational Agents, Voice User Interfaces, Metaphors, Personas, Design}

\begin{teaserfigure}
    \centering
    \includegraphics[width=\textwidth]{mfd.png}
    \caption{The schematic depicts the proposed evolution toward Metaphor-Fluid Design, starting from the description of Voice User Interfaces (VUIs) in different social roles using popular metaphors [A], leading to the identification of design issues [B], and ultimately inspiring Metaphor-Fluid design [C].}
    \label{fig:mfd}
\end{teaserfigure}

%%
%% This command processes the author and affiliation and title
%% information and builds the first part of the formatted document.
\maketitle

\section{Introduction}
Voice User Interfaces (VUIs) are becoming an integral part of daily interactions, embedded in smart speakers, mobile devices, automobiles, and wearable technologies. As these systems proliferate, the ways in which users engage with and conceptualize them evolve dynamically. However, contemporary VUI design remains largely constrained by a single dominant metaphor—humanness \cite{Doyle_Edwards_Dumbleton_Clark_Cowan_2019}. Most commercial VUIs are designed to mimic human conversational abilities, using speech, turn-taking, and natural language processing to create an interaction model that feels intuitive \cite{Edlund_2019, Pradhan_Lazar_2021}. Within this overarching metaphor, VUIs are often framed as social entities that take on various roles, including teachers, therapists, coaches, advisors, and companions \cite{Desai_Chin_2023, Jung_Kim_So_Kim_Oh_2019, Desai_Hu_Lundy_Chin_2023, Wang_Yang_Shao_Abdullah_Sundar_2020, Desai_Lundy_Chin_2023, Lee_Frank_IJsselsteijn_2021, Motalebi_Cho_Sundar_Abdullah_2019, Trajkova_Martin-Hammond_2020, Purington_Taft_Sannon_Bazarova_Taylor_2017}. Among these, the assistant persona has emerged as the most prevalent, dominating commercial implementations and positioning the VUI as a helpful, subservient entity designed to execute user commands \cite{Desai_Twidale_2023, Doyle_Clark_Cowan_2021}. While this metaphor provides a familiar interaction paradigm, it also imposes significant limitations. It assumes a fixed, one-size-fits-all identity for the VUI, failing to account for the fluid nature of human-VUI interactions across different tasks and contexts \cite{Pradhan_Lazar_2021, desai2024cui, Desai_Chin_2023, Desai_Twidale_2022}. This rigidity leads to usability challenges \cite{Cowan_Pantidi_Coyle_Morrissey_Clarke_Al-Shehri_Earley_Bandeira_2017}, mismatches between user expectations and system behavior \cite{Doyle_Clark_Cowan_2021}, and broader concerns surrounding anthropomorphic design choices \cite{McMillan_Jaber_2021, Desai_Twidale_2022, Simpson_Crone_2022, Pradhan_Lazar_2021, Luger_Sellen_2016, Turk_2016}.

These challenges reflect a broader issue in interface design: the role of metaphors in shaping user interaction. Metaphors have played a foundational role in shaping Human-Computer Interaction (HCI) \cite{Neale_Carroll_1997, Laurel_1997, Laurel_Mountford_1990, Carroll_Thomas_1982, Carroll_Mack_Kellogg_1988, Don_Brennan_Laurel_Shneiderman_1992}. In Graphical User Interfaces (GUIs), they are structured into primary and secondary layers: the desktop metaphor, for instance, serves as the primary organizing concept, with secondary metaphors—such as files, folders, and recycling bins—providing finer interaction affordances \cite{Blackwell_2006}. These layered metaphors help users form coherent mental models of system functionality. In contrast, VUIs lack a stable visual representation, making their metaphorical framing less explicit. Instead, this framing is operationalized through persona design, where a VUI’s voice, linguistic style, and response behaviors collectively establish its identity \cite{Pradhan_Lazar_2021}. Unlike GUIs, where metaphors map interactions onto familiar objects, VUIs inherently imply personhood, reinforcing anthropomorphic interpretations. The presence of human-like speech cues, names, and conversational norms encourages users to assign social roles to VUIs, often expecting them to behave with intelligence, adaptability, and even emotional awareness \cite{Luger_Sellen_2016, Sutton_Foulkes_Kirk_Lawson_2019}.

Despite this anthropomorphic framing, users do not engage with VUIs as static entities \cite{Pradhan_Findlater_Lazar_2019}. Instead, they shift their metaphorical framings depending on use-contexts—the functional settings in which interactions take place \cite{Desai_Twidale_2023}. Prior research has identified several recurring use-contexts in VUI interactions, including commands, where users expect efficiency and compliance (e.g., "Turn off the lights"); information seeking, where they seek structured knowledge retrieval (e.g., "What is the capital of Japan?"); sociality, where they engage in casual conversation (e.g., "Tell me a joke"); and error recovery, where they attempt to correct or refine system responses (e.g., "That’s not what I meant—try again") \cite{Sciuto_Saini_Forlizzi_Hong_2018, Desai_Chin_2023, ammari2019music, Bentley_Luvogt_Silverman_Wirasinghe_White_Lottridge_2018}. In each of these use-contexts, users adopt distinct metaphorical descriptions: they conceptualize a VUI as a butler or secretary when issuing commands, as an expert librarian when retrieving information, and as a companion in social exchanges. However, commercial VUIs impose a singular, unchanging persona across all contexts, failing to adapt to the shifting expectations users bring to different types of interactions. This rigidity results in fundamental misalignments—when users anticipate authoritative expertise but receive generic, non-committal responses, or when they expect a social exchange but encounter the same neutral, task-oriented demeanor, the interaction feels unnatural and disengaging \cite{Desai_Hu_Lundy_Chin_2023}. These breakdowns not only present usability challenges but also raise deeper ethical concerns.


The assistant metaphor, in particular, has been widely critiqued for reinforcing problematic socio-cultural assumptions \cite{McMillan_Jaber_2021, Brahnam_De_Angeli_2008, Brahnam_Karanikas_Weaver_2011}. Acting as the Default VUI in commercial systems such as Amazon Alexa, Google Assistant, and Apple Siri, this metaphor frames VUIs as subservient entities designed to efficiently execute user commands \cite{Turk_2016}. These systems are marketed as digital assistants, emphasizing compliance, convenience, and deference—qualities that are often reinforced through feminized voices and polite, deferential speech patterns \cite{Kuzminykh_Sun_Govindaraju_Avery_Lank_2020}. This design choice perpetuates outdated power dynamics that position technology as a passive, compliant agent rather than an adaptive, context-aware tool. Prior studies have highlighted the risks of over-anthropomorphizing VUIs, where users may develop unrealistic expectations of intelligence or emotional understanding, leading to frustration when the system fails to respond appropriately \cite{Cowan_Pantidi_Coyle_Morrissey_Clarke_Al-Shehri_Earley_Bandeira_2017, Trajkova_Martin-Hammond_2020}. At the same time, users may dehumanize VUIs, treating them with excessive aggression or dismissiveness—behaviors that some researchers argue could spill over into human social interactions \cite{Laurel_1997, Brahnam_De_Angeli_2008}. Furthermore, the universal application of a single persona disregards the diverse ways in which users naturally interpret and interact with these systems. 

Building on prior work, we introduce Metaphor-Fluid Design, a novel design approach that moves beyond static persona models by enabling VUIs to dynamically shift their metaphorical framing based on conversational context. Rather than adhering to a singular metaphor—such as the assistant—this approach allows the system to adopt multiple personas, aligning its interaction style with the expectations of the task at hand. Crucially, Metaphor-Fluid Design also incorporates non-human and fictional metaphors, which have been shown to be both relevant and frequently used by users \cite{desai2024cui, Desai_Twidale_2023, Jung_Qiu_Bozzon_Gadiraju_2022}. While commercial VUIs default to anthropomorphic personas, users often describe them in non-human terms, likening them to machines, calculators, encyclopedias, or even mythical entities like genies or ghosts \cite{Desai_Twidale_2023}. By broadening the metaphorical landscape beyond human roles, this approach creates greater flexibility in interaction design, allowing VUIs to respond more intuitively to diverse user expectations. This work is guided by two primary research questions:

\begin{itemize}
\item RQ1: What metaphors commonly align with users' interactions in use-contexts such as commands, information seeking, sociality, and error recovery during conversations with VUIs?
\item RQ2: How does a Metaphor-Fluid VUI compare to a Default VUI in terms of perceived enjoyment, intention to adopt, trust, likability, and intelligence, and in the characteristics of the metaphorical descriptions it generates? 

\end{itemize}

These questions are explored through two studies. The first study (N=130) maps user-generated metaphors to different conversational contexts, revealing systematic variations in metaphorical preferences across tasks. Findings show that users do not conceptualize VUIs through a singular metaphor but instead shift their framings based on the use-context. Specifically, users associate command-based interactions and error recovery with Guides (e.g., Genie, Search Engine), information-seeking tasks with Aides (e.g., Encyclopedia, "Computer" from Star Trek), and social exchanges with Companions (e.g., Friend, Flmatmate). These patterns demonstrate that users instinctively align VUI metaphors with the demands of the task at hand, challenging the assumption that a single persona is sufficient for all interactions. The second study (N=91) empirically evaluates a Metaphor-Fluid VUI—which adapts its metaphorical framing dynamically—against a Default VUI that consistently operationalizes the assistant metaphor. Results indicate that Metaphor-Fluid design leads to significantly higher perceived likability, enjoyment, and intention to adopt, showing that aligning a VUI’s persona with context-specific expectations positively impacts user perceptions. However, findings also highlight individual differences in metaphor preferences, suggesting that while contextual adaptation enhances certain perceptions, incorporating personalization could further refine metaphor alignment for diverse users.

Findings from this work challenge the dominant paradigm of static persona design in VUIs. By formalizing Metaphor-Fluid Design as a structured approach to VUI development, this paper contributes to a growing discourse on how metaphor selection influences user interaction. These insights have broader implications for the design of conversational agents, underscoring the importance of flexibility, contextual awareness, and ethical considerations in shaping future human-AI conversational interactions.



\section{Related Work}

VUIs enable users to interact naturally through spoken language and are embedded in a range of consumer technologies, including smart speakers, smartphones, headphones, and even household appliances. Popular examples include Amazon Alexa, Apple Siri, and Google Assistant, which have become integral to everyday interactions. However, the term VUI is often used interchangeably with other labels, such as Intelligent Personal Assistants, Digital Assistants, Conversational Agents (CAs), Conversational User Interfaces (CUIs), and Speech Interfaces \cite{Clark_Doyle_Garaialde_Gilmartin_Schlögl_Edlund_Aylett_Cabral_Munteanu_Edwards_et_al._2019}. While these terms share similarities, they differ in scope. For instance, CAs and CUIs encompass both voice- and text-based systems, including chatbots, whereas VUIs specifically refer to interfaces that rely on voice as the primary input and output modality.

Given this terminological overlap, it is essential to clarify the scope of this research. This article focuses explicitly on disembodied, software-based VUIs—systems that process and generate spoken language without physical embodiments, such as robots or avatars. However, given the broader relevance of CUIs in conversational AI research, we reference CUIs where applicable to contextualize and discuss our findings within the wider discourse on human-AI interaction.

\subsection{Metaphors—A Brief Primer}



Metaphors are more than just literary flourishes; they are fundamental tools for understanding and interpreting the world \cite{Indurkhya_2013}. In linguistics and cognitive science, metaphors have been shown to shape not only how we communicate but also how we think \cite{Gentner_Hoyos_2017}. According to Lakoff and Johnson's theory of conceptual metaphors, our understanding of abstract concepts is often structured by metaphors derived from concrete experiences \cite{Lakoff_Johnson_1980}. For example, when we say "time is money," we are using a metaphor to conceptualize time as a valuable resource, which affects how we perceive and manage it \cite{Lakoff_Johnson_1980}.

Metaphors are powerful because they influence cognition and behavior, often in subtle ways. They help bridge the gap between the familiar and the unfamiliar, making complex ideas more accessible \cite{Gentner_Hoyos_2017}. By providing a conceptual structure, metaphors allow individuals to grasp abstract concepts by relating them to more concrete experiences. In cognitive psychology, metaphors have been found to shape reasoning, decision-making, and problem-solving processes, thereby playing an integral role in how knowledge is constructed and applied \cite{Moser_2000, Cameron_Maslen_2010}. Through metaphors, we can connect domains that are seemingly unrelated, fostering creativity and enabling us to generate novel ideas \cite{Lockton_Singh_Sabnis_Chou_Foley_Pantoja_2019}. In science and technology, metaphors often serve as foundational frameworks that influence both how researchers theorize and how laypeople understand complex phenomena \cite{Hofstadter_1995}. Alan Turing, for example, used a metaphor of a human computer following a set of instructions to describe his model for computation, which became the basis for modern computer science \cite{Piccinini_2003}. Similarly, metaphors like "genetic code" or "cellular machinery" have shaped how scientists and the public conceptualize biological processes, providing familiar reference points that help make sense of otherwise abstract concepts \cite{Keller_2003}. Metaphors thus provide a cognitive shortcut that allows individuals to comprehend and manipulate ideas beyond the immediate grasp of direct experience \cite{Black_1962}.

Moreover, metaphors are deeply embedded in everyday language, often so much so that we are unaware of their influence. Phrases such as "grasping an idea" or "feeling down" illustrate how metaphorical thinking is woven into our conceptual system \cite{Lakoff_Johnson_1980}.

\subsection{Metaphors in HCI}

Metaphors in HCI have profoundly shaped how users conceptualize and interact with technology. By leveraging familiar concepts (e.g., files, folders), metaphors bridge the gap between users' prior knowledge and new digital environments, simplifying interactions and reducing cognitive load \cite{Neale_Carroll_1997, Carroll_Thomas_1982}. The desktop metaphor stands out as a pivotal example in GUIs, drawing on the physical analogy of an office desk to provide users with a coherent mental model for navigating digital spaces \cite{Johnson-Laird_1983}. This foundational desktop metaphor is supplemented with secondary metaphors like folders, files, bins, scrolling, and icons and translates abstract computational processes into familiar actions, such as "dragging a file to the trash" to discard a document \cite{Colburn_Shute_2008}.

The impact of the desktop metaphor extends far beyond mere visual representation. It has played a crucial role in making computers more approachable and familiar, significantly contributing to their widespread adoption \cite{Carroll_Mack_Kellogg_1988}. By aligning user expectations with system capabilities, the desktop metaphor has fostered increased usability and user satisfaction. Furthermore, it has provided a common language for developers and users to communicate about new technologies, facilitating the ongoing evolution of human-computer interfaces. The desktop metaphor's influence on early computing systems was so profound that it shaped users' mental models for decades. For a detailed exploration of its history and significance, Alan Blackwell's work offers an in-depth analysis that underscores the metaphor's lasting impact on HCI design principles \cite{Blackwell_2006}.




The application of metaphors in interface design extends beyond mere aesthetics, serving dual roles of familiarization and transportation, as outlined by Paul Heckel in his work on friendly software design~\cite{Heckel_1984}. Familiarization employs metaphors to introduce recognizable concepts, facilitating user comprehension and mental model construction. Transportation, on the other hand, utilizes these mental models to create immersive experiences. In the context of VUIs, the concept of humanness frequently serves as the primary metaphor, acting as a conduit for familiarization and transportation effects.



\subsection{Metaphors in VUIs}

The study of metaphors in VUIs is relatively new compared to their rich history in HCI. The foundational "humanness" metaphor serves as the principal metaphor in VUI design, where VUIs are often deployed in human-like social roles for VUIs, ranging from educators and coaches to storytellers and even therapists~\cite{Desai_Chin_2023, Jung_Kim_So_Kim_Oh_2019, Desai_Hu_Lundy_Chin_2023, Wang_Yang_Shao_Abdullah_Sundar_2020, Desai_Lundy_Chin_2023, Lee_Frank_IJsselsteijn_2021, Motalebi_Cho_Sundar_Abdullah_2019}. This approach aims to create a sense of familiarity by aligning VUIs with known social roles, thereby making interactions more intuitive and accessible \cite{McMillan_Jaber_2021, Desai_Twidale_2023}. VUIs like Amazon Alexa and Apple Siri are designed to function as digital assistants, drawing on the metaphor of the helpful human assistant to facilitate tasks such as playing music, providing information, managing smart home devices, and offering social companionship \cite{Sciuto_Saini_Forlizzi_Hong_2018}. These devices leverage the designer-intended and manufacturer-marketed \cite{Turk_2016} humanness metaphor to create an experience that feels natural and social, helping to bridge the gap between technology and everyday life \cite{Olmstead_2017, Auxier_2019}.

In practice, metaphors are operationalized in VUI design through system personas and form a crucial part of Conversation Design workflows \cite{Sadek_Calvo_Mougenot_2023}. Google's Conversational UX guidelines\footnote{https://developers.google.com/assistant/conversation-design/create-a-persona}, for instance, emphasize role production by designing agents that mimic real-world figures, such as bank tellers or exercise coaches, situating these interfaces in familiar contexts \cite{McMillan_Jaber_2021}. Designers often default to human roles because of the innate tendency to anthropomorphize technology. Anthropomorphism—the attribution of human traits to non-human entities—is central to VUI design, as it offers users a relatable interaction model \cite{Epley_Waytz_Cacioppo_2007}.

Epley et al.'s influential three-factor theory of anthropomorphism explains why and when humans attribute human-like characteristics to non-human agents. The factors—elicited agent knowledge (drawing on knowledge of human behavior), effectance motivation (the need to predict and understand the agent’s actions), and sociality motivation (seeking social connection)—are key to understanding why users often interact socially with VUIs \cite{Epley_Waytz_Cacioppo_2007}. The "humanness" metaphor leverages users' natural inclination to perceive technology through a social lens, making VUIs feel more engaging and personable. As a result, these systems incorporate anthropomorphic cues like natural language processing and human-like voice modulation to reinforce the perception of social presence and competence \cite{Pradhan_Lazar_2021, Purington_Taft_Sannon_Bazarova_Taylor_2017}. 


The choice of metaphor in VUI design significantly impacts user perception. Studies have shown that metaphors signaling high competence initially attract users, while those suggesting lower competence tend to be favored in the long term~\cite{Khadpe_Krishna_Fei-Fei_Hancock_Bernstein_2020}. Furthermore, the formality of the VUI's conversational style influences users' metaphorical interpretations, with more formal styles evoking professional metaphors and informal styles eliciting personal ones~\cite{Chin2024Like}. Research has also indicated a correlation between metaphor-projected personalities and user personalities, with users preferring metaphors that align with their own traits~\cite{Braun_Mainz_Chadowitz_Pfleging_Alt_2019}. Despite designers' efforts to present clear mental models through personas, users often develop their own conceptual understandings of VUIs through folk theories~\cite{DeVito_Birnholtz_Hancock_French_Liu_2018}. These theories, often expressed as metaphors, attempt to explain VUI functionality or rationalize system errors~\cite{Kim_Choudhury_2021, Kuzminykh_Sun_Govindaraju_Avery_Lank_2020, Desai_Twidale_2022}. The disparity between user expectations based on provided metaphors and actual VUI performance can lead to usage issues, as described by Norman's "gulf of execution" concept~\cite{Norman_2013, Luger_Sellen_2016}.


Broadly, there are three significant challenges associated with the current implementation of metaphors in VUI design:
\begin{enumerate}

\item \textbf{Ethical Issues}: Designing VUIs based on roles like "assistant" or "servant" can inadvertently reinforce outdated stereotypes related to gender and power dynamics \cite{McMillan_Jaber_2021, Pradhan_Lazar_2021}. These metaphors may implicitly communicate roles of subservience, which can perpetuate harmful social norms, especially when the personas are given specific voices that align with stereotypical attributes. For example, the choice of a female voice for assistant roles often reinforces traditional gender stereotypes about caregiving and subservience, which has raised ethical concerns regarding the impact of such designs on societal attitudes \cite{Brahnam_Karanikas_Weaver_2011, Turk_2016, Kuzminykh_Sun_Govindaraju_Avery_Lank_2020}.

\item \textbf{Simplistic Representations}: While users develop complex and context-dependent mental models of VUIs that change with the type of interaction or domain of use, designers often rely on a single overarching metaphor to represent these systems. This singular metaphorical framing may fail to capture the nuanced ways in which users interact with VUIs across different contexts \cite{Desai_Twidale_2022}. For example, in a user study involving older adults interacting with a VUI designed as an "exercise coach," participants expressed a preference for the VUI to also behave like a "friend" when initiating the session, checking on their progress, and providing motivation \cite{Desai_Hu_Lundy_Chin_2023}. The complexity of user expectations and the diverse roles VUIs play are not adequately addressed by a one-size-fits-all metaphor, which can lead to a mismatch between what users expect and what the system delivers, ultimately impacting usability and user satisfaction. By providing only a simplistic metaphor, designers risk oversimplifying the rich, multi-faceted experiences users have with VUIs, thereby limiting the potential for these systems to meet diverse user needs effectively.  Moreover, users' perceptions of VUIs are also highly fluid, shifting depending on the context and specific interactions. Pradhan et al. \cite{Pradhan_Lazar_2021} found that users alternately view VUIs as either a "person" or a "thing," or something in between, depending on the specific task or setting. 

\item \textbf{Misaligned Mental Models}:  Misaligned mental models arise when the human-like metaphors underlying VUIs heighten user expectations, leading to either over or under-calibration of trust. Users may overestimate the VUI's abilities, assuming it possesses human-like cognitive and conversational skills, or underestimate its utility due to skepticism about its capabilities. \cite{Cowan_Pantidi_Coyle_Morrissey_Clarke_Al-Shehri_Earley_Bandeira_2017} This misalignment of expectations impacts trust calibration—users may become overly reliant on the VUI, expecting it to perform beyond its technical limitations, or conversely, they may dismiss its utility prematurely \cite{Luger_Sellen_2016}. These fluctuating expectations can result in inconsistent or infrequent use \cite{Cowan_Pantidi_Coyle_Morrissey_Clarke_Al-Shehri_Earley_Bandeira_2017} and, eventually, complete abandonment of the technology \cite{Trajkova_Martin-Hammond_2020}. The inconsistency in how users perceive the capabilities of VUIs, especially when the system fails to meet expectations in specific scenarios, resulting in conversational errors, further exacerbates the misalignment and undermines user trust \cite{Baughan_Wang_Liu_Mercurio_Chen_Ma_2023}.

\end{enumerate}

The evolving landscape of VUI design reflects a broader understanding of the complexities involved in metaphor selection. While a single human metaphor offers an accessible and engaging way for users to interact with technology, its limitations require thoughtful application to prevent unintended ethical and usability issues.

\subsection{Mitigating Challenges in VUI Design}
\label{sec:mitigate}

As the VUI design space evolves, various approaches have been proposed to address the challenges surrounding VUI persona design. One promising approach involves developing VUIs in non-human roles. Desai and Twidale \cite{Desai_Twidale_2023}, through a literature review and user study, identified how metaphors are used by users, designers, manufacturers, and researchers in relation to VUIs. Based on their analysis, they developed a framework for metaphor contextualization in VUIs, categorizing metaphor use along four dimensions: type (human, non-human, fictional), mentioned by (users, designers, manufacturers, researchers), the knowledge it draws upon, and the motivation behind its use. Their findings revealed that while human metaphors dominate VUI design and research, non-human and fictional metaphors were also popular, especially among users for sense-making. This contrast highlights a dichotomy between how designers and researchers present VUIs and how users interpret them. This observation points to the need for a "pragmatic approach" \cite{Carroll_Mack_Kellogg_1988} to systematically explore the design space between system capabilities and user expectations.

In the broader CUI literature, there have been recent notable advancements in exploring the feasibility of designing conversational agents with non-human roles. Jung et al. \cite{Jung_Qiu_Bozzon_Gadiraju_2022}, for instance, used the Great Chain of Being (GCOB) framework\footnote{A theoretical framework that organizes the world into a hierarchical structure using metaphors of God, humans, animals, plants, and inanimate objects.} to investigate how non-human metaphors (such as God, Animal, Plant, and Book) compared to human metaphors influence user perceptions of chatbots. Their results showed that the human metaphor did not significantly alter chatbot evaluations compared to non-human metaphors. Similarly, in a related study involving VUIs, researchers examined human versus non-human metaphors in the health and finance domains. They found that in the health domain, users preferred the human metaphor (doctor) in terms of likability and enjoyability, while in the finance domain, no notable differences were observed between the human (financial advisor) and non-human (calculator) metaphors. Additionally, the choice of metaphor had no significant impact on users’ intentions to adopt the VUIs.

The use of fictional metaphors to inspire VUI design has long been of interest to HCI researchers. One of the most notable examples is drawing from the "Computer" in Star Trek to inform VUI design principles \cite{Turk_2016}. In a relevant study, Axtell \& Munteanu \cite{Axtell_Munteanu_2021} conducted a conversation analysis comparing the dialogues of the "Computer" in \textit{Star Trek: The Next Generation} with interactions on Amazon Alexa. Their findings shed light on how VUIs could be designed to be more functional. Additionally, fictional metaphors can inspire a negative design heuristic, illustrating how not to design VUIs by highlighting ineffective or undesirable interaction patterns. This includes avoiding the dystopian, pervasive feeling of a "Big Brother" presence from \textit{1984} or the manipulative behavior exhibited by the Bene Gesserit Sisterhood through "The Voice" in \textit{Dune} \cite{Feldman_2024}. Fictional metaphors can also inspire a sense of enchantment in VUI design, similar to how AI technologies often evoke a magical experience. This desired "AI magic," where users perceive advanced technology as almost supernatural, can be paralleled in VUIs by employing fictional metaphors that create a similar sense of wonder and fluid interaction. For instance, magic metaphors like using a "magic wand" in GUIs to represent the undo command illustrate how these enchanting elements resonate with users \cite{Lupetti_Murray-Rust_2024}, much like how wake-words in VUIs are sometimes compared to casting a spell \cite{Desai_Twidale_2023}.

Moreover, the issue of misaligned mental models, often described as the gulf between VUI capabilities and user expectations \cite{Luger_Sellen_2016}, remains a key topic of ongoing research in the HCI community. One of the most pertinent approaches to studying this gulf is through the "Mutual Theory of Mind" (MToM) Framework \cite{Wang_Goel_2022}. This framework emphasizes that constructing a CUI’s understanding of how the user perceives the system—through analyzing communication cues—can be crucial in bridging this gap and aligning CUIs more effectively with users' mental models. These communication cues include not only verbal signals such as language use and tone \cite{Wang_Saha_Gregori_Joyner_Goel_2021} but also non-verbal behaviors, like gestures, facial expressions, or pauses during interaction \cite{Chan_Fu_Li_Yao_Desai_Prpa_Wang_2024}. In this paper, we take this idea further and propose designing conversational interfaces that adapt their presentational metaphor based on conversational contexts to ensure that the interface is aligned with the users' mental models. 

\section{Toward Metaphor-Fluid Design}

The discussion so far has established that metaphors play an important role in VUIs. As discussed in the previous sections, metaphors help users construct mental models for interaction, guiding their expectations of system behavior \cite{Carroll_Thomas_1982, Carroll_Mack_Kellogg_1988, Colburn_Shute_2008}. However, unlike in GUIs—where spatial and object-based metaphors provide a persistent conceptual framework \cite{Blackwell_2006}—metaphors in VUIs lack a stable visual anchor and must instead be inferred from interaction patterns. This fundamental difference has led to a design paradigm where system personas serve as implicit metaphors, mapping VUI behavior onto familiar social roles such as assistants, coaches, or companions \cite{Desai_Chin_2023, Jung_Kim_So_Kim_Oh_2019, Desai_Hu_Lundy_Chin_2023, Wang_Yang_Shao_Abdullah_Sundar_2020, Desai_Lundy_Chin_2023, Lee_Frank_IJsselsteijn_2021, Motalebi_Cho_Sundar_Abdullah_2019}. While these persona-based framings are intended to provide consistency, empirical findings suggest that users do not experience VUIs as singular, fixed entities. Instead, they reframe their mental models dynamically, adjusting their conceptualization of the system based on the nature of the interaction \cite{Pradhan_Findlater_Lazar_2019, Desai_Twidale_2023}. For example, Desai and Twidale \cite{Desai_Twidale_2023} found that when retrieving information, users tend to conceptualize VUIs as search engines or librarians, whereas in casual conversation, they are more likely to ascribe relational qualities, perceiving the VUI as a companion or conversational partner—sometimes within the same interaction session. These shifts occur fluidly and without explicit user effort, reflecting how people naturally adjust their conversational framing based on social and functional context.

Despite this, current VUI design does not reflect this inherent metaphorical flexibility. Instead, designers have historically favored a single, unified persona, treating it as a stable interface identity that remains consistent across all interactions \cite{Sadek_Calvo_Mougenot_2023}. This approach assumes that users will develop trust and familiarity with a VUI if its persona is coherent and unchanging. However, prior research suggests that this assumption is not only misaligned with actual user expectations \cite{Doyle_Clark_Cowan_2021, desai2024cui, Braun_Mainz_Chadowitz_Pfleging_Alt_2019} but also overly simplistic \cite{Desai_Hu_Lundy_Chin_2023}. Unlike GUIs, which can afford to maintain a static metaphorical structure, VUIs operate within a communicative medium where adaptability is already fundamental to human interaction. Individuals naturally adjust their language, tone, and conversational framing based on their conversational partner, the context of the exchange, and their intended outcomes. Imposing a single, inflexible metaphor onto a VUI is not only unnecessary but may actively conflict with how users intuitively engage with conversational systems. Furthermore, existing persona implementations rely on an inherent humanness metaphor, which introduces ethical concerns. The "assistant" persona, for example, has been criticized for reinforcing gender stereotypes, highlighting the broader risks of anthropomorphic framing in conversational systems \cite{Brahnam_Karanikas_Weaver_2011, Turk_2016, Kuzminykh_Sun_Govindaraju_Avery_Lank_2020}.

Given this, the question should not be “What is the best metaphor for VUIs?” but rather \textit{“How can VUIs accommodate multiple metaphors in a way that aligns with user expectations?”} This shift in focus leads us toward Metaphor-Fluid Design, a novel design approach that acknowledges metaphorical adaptation as a fundamental principle rather than an incidental consequence of interaction. This design approach is guided by two primary tenets:

\begin{enumerate}
    \item \textbf{Expanding metaphorical scope:} We advocate for moving beyond the conventional persona design paradigm, where a singular, often human-centric metaphor serves as the interactional layer of the conversational interface. Instead, we encourage conversation designers to incorporate a broader range of metaphors, including non-human and fictional ones, into their design process. This shift allows for more nuanced interactions and can help mitigate the limitations imposed by the overly simplistic, monolithic application of personas. Moreover, we urge a reconsideration of the very term "persona" as the defining concept for imbuing VUIs with personality. The word "persona" inherently carries connotations of personhood, which may not always be appropriate or desirable depending on the specific context or application domain. By expanding the metaphorical landscape, we open up the possibility of more tailored and contextually relevant designs that align better with user expectations.
    
    \item \textbf{Contextual metaphor adaptation:} We propose that VUIs should dynamically adjust their metaphorical presentation based on the conversational context in which they are used. This fluid approach to metaphorical design recognizes that user needs and expectations are not static but evolve throughout an interaction. By allowing VUIs to adapt their metaphorical representation in real-time, designers can create systems that feel more intuitive and responsive, ultimately leading to more meaningful and effective user experiences.
\end{enumerate}

In this paper, we introduce the concept of Metaphor-Fluid Design using VUIs as a central example. The primary aim is to present this novel design approach and examine its impact on user perception, particularly in comparison to traditional VUI implementations. To develop this approach, Study 1, guided by RQ1, investigates which metaphors naturally align with different use-contexts—commands, information-seeking, sociality, and error recovery. By identifying these patterns, Study 1 provides a foundation for designing VUIs that adapt their metaphorical framing to better align with user expectations in each context and explores the possibility of integrating non-human and fictional metaphors in VUI design. Study 2, guided by RQ2, then examines the implications of this approach by comparing a Metaphor-Fluid VUI to a Default VUI, assessing its effects on user perception, including enjoyment, likability, trust, intelligence, and intention to adopt. While prior work has shown that users already engage with and perceive VUIs in a metaphor-fluid manner, this study moves beyond observation to explore how metaphorical adaptation can be deliberately integrated into conversational systems and empirically evaluated. By expanding the metaphorical landscape beyond static, human-centric roles, we aim to demonstrate how dynamically adjusting metaphors in response to conversational context influences user perceptions and interaction experiences. 

\section{Study 1: Mapping VUI Metaphors to Conversational Use-Contexts}

To address RQ1, we conducted Study 1, which explores the metaphors that commonly align with users' interactions across various use-contexts in conversations with VUIs. Specifically, we investigated four key use-contexts: commands, information seeking, sociality, and error recovery. Our approach involved several interconnected steps. We began by identifying commonly used metaphors employed by both users and designers to describe VUIs. These ranged from designer-intended metaphors aimed at establishing familiarity in specific scenarios (e.g., "Therapist") to user-generated metaphors for sense-making during VUI interactions (e.g., "Child"). Following this, we developed and justified the selection of the four use-contexts employed in this study, basing our choices on a thorough review of empirically grounded VUI literature involving user studies. To align the identified metaphors with these use-contexts, we adopted a two-dimensional scale that maps the social dimensions of the metaphors against the desired attributes of each use-context.

\subsection{Methods}


\subsubsection{Metaphor Identification}

Our metaphor collection process drew from two significant sources, combining a comprehensive literature review with user studies to gather a wide range of VUI metaphors.
Desai \& Twidale \cite{Desai_Twidale_2023} conducted a literature review following adapted PRISMA guidelines, similar to \cite{Clark_Doyle_Garaialde_Gilmartin_Schlögl_Edlund_Aylett_Cabral_Munteanu_Edwards_et_al._2019}. The review focused on peer-reviewed articles from the past decade that discussed VUIs with speech as the primary input and output modality. An initial search in the ACM digital library using a complex query yielded 664 articles, which were screened down to 80 for final analysis. This source also included a user study with 14 participants who interacted with an Amazon Echo Dot and participated in semi-structured interviews. The literature review and user study together identified 93 metaphors related to VUIs. This study is referred as Source A in Figure \ref{fig:vuimetaphorid}. Similarly, Chin et al. \cite{Chin_Desai_Lin_Mejia_2023} contributed through a study (referred to as Source B in Figure \ref{fig:vuimetaphorid}) in a simulated smart home environment involving 58 participants. Participants interacted with a VUI, completing cognitive tasks (e.g., word puzzles), and were explicitly asked to compare the VUI to other entities using metaphors. This direct elicitation method resulted in 225 metaphors.

\begin{figure}[H]
    \centering
    \includegraphics[width=\textwidth]{vuimetaphorid.png}
    \caption{A total of 93 metaphors were identified from Source A (\cite{Desai_Twidale_2023}), and 225 metaphors were collected from Source B (\cite{Chin_Desai_Lin_Mejia_2023}). After removing duplicates, 98 unique metaphors remained. From these, 20 metaphors were selected based on their frequency of occurrence and feasibility for practical implementation in VUI design.}
    \label{fig:vuimetaphorid}
\end{figure}

After compiling metaphors from both sources and removing duplicates or closely related concepts (e.g., Flatmates and Roommates), the collection was refined to 98 distinct metaphors. From this set, 20 metaphors were selected for further analysis based on their frequency of occurrence across studies, their feasibility for practical VUI design application, and maintaining the same ratio of human, non-human, and fictional metaphors found in the original set of 98. These metaphors included Admirer, Assistant, Boss, Butler, Child, Coach, Companion, "Computer" from Star Trek, Customer Service Agent, Encyclopedia, Family Member-Aunt, Family Pet, Flatmate, Friend, Genie, Librarian, Nurse, Search Engine, Teacher, and Therapist. This selection process ensured a balance between conceptual strength, type, and practical relevance for future work in VUI design. 

\subsubsection{VUI Conversational Use-Contexts}
Desai and Twidale \cite{Desai_Twidale_2023}, in a previous study, identified four conversational use-contexts—commands, information seeking, sociality, and error recovery—based on the different types of tasks users perform. Interestingly, although error recovery is not a task in itself, it still leads to conversations and often evokes amusing metaphorical comparisons from users (e.g., a silly child or a pet). Furthermore, they adopt Epley et al.'s \cite{Epley_Waytz_Cacioppo_2007} definition of sociality to include social tasks such as playing games and engaging in small talk—a definition that we also adopt in this work.

While Desai and Twidale \cite{Desai_Twidale_2023} classified use-contexts through metaphor analysis, other studies have used usage logs to categorize conversational use-contexts. Bentley et al. \cite{Bentley_Luvogt_Silverman_Wirasinghe_White_Lottridge_2018} analyzed usage logs from 88 households over an average of 110 days, identifying ten conversational use-contexts: music, information, automation, small talk, alarms, weather, video, time, lists, and others. Similarly, Kim and Choudhury \cite{Kim_Choudhury_2021} conducted a 16-week longitudinal study with 12 older adults, identifying eight interaction topics: music, search, basic device control, casual conversation, time, reminders, weather, and others. Ammari et al. \cite{ammari2019music} analyzed log files from 82 Amazon Alexa and 88 Google Home devices, categorizing interactions into eight themes: search, music, timers, automation, smart home, macros and programming, family interaction, and privacy.

The metaphorical use-contexts identified by Desai and Twidale \cite{Desai_Twidale_2023} align closely with these findings. The command use-context encompasses several task types reported in prior work, including music, automation, alarms, weather, video, time, and lists \cite{Bentley_Luvogt_Silverman_Wirasinghe_White_Lottridge_2018}; music, basic device control, time, and reminders \cite{Kim_Choudhury_2021}; and music, timers, automation, smart home, and macros \cite{ammari2019music}. These tasks share a common interaction pattern: a one-shot call-and-response approach, where users issue a command and expect an immediate response. In some cases, such as automation and device control, a verbal response may not even be necessary.

Similarly, information seeking includes search-related tasks mentioned across studies \cite{ammari2019music, Bentley_Luvogt_Silverman_Wirasinghe_White_Lottridge_2018, Kim_Choudhury_2021}, while sociality encompasses small talk and casual conversation. The classification is further expanded by introducing error recovery as an additional conversational use-context. Although errors are not an intended feature, they are a persistent part of interactions with commercial VUIs, cutting across all use-contexts. In VUI literature, error recovery is often conceptualized as following unique conversational rules, similar to a difficult conversation \cite{Porcheron_Fischer_Reeves_Sharples_2018, Fischer_Reeves_Porcheron_Sikveland_2019}. Given the high error rate of commercially available VUIs—Kim and Choudhury \cite{Kim_Choudhury_2021} reported a 33\% error rate in their longitudinal study—responding to errors often involves a back-and-forth interaction to clarify misinterpretations and adjust input, making error recovery a distinct aspect of VUI conversations.

\subsubsection{Social Dimensions of VUIs}

In previous literature, metaphors for CUIs have been sampled using two two-dimensional models: (1) Stereotype Content Model (SCM) \cite{Fiske_Cuddy_Glick_Xu_2002}, and (2) Argyle’s Model of Attitude Towards Others \cite{Argyle_1988}. The two dimensions of SCM are warmth and competence. Similarly, Argyle’s model includes dominant-submissive and hostile-friendly. 

Khadpe et al. \cite{Khadpe_Krishna_Fei-Fei_Hancock_Bernstein_2020} designed chatbots using metaphors sampled along the dimensions of warmth (being good-natured and friendly) and competence (being intelligent) for human-AI collaboration tasks. The researchers found that designing AI chatbots with high warmth was beneficial in all scenarios, but the effect of competence depended on users’ prior expectations.  In another study investigating the effect of warmth and competence on users’ intention to adopt an AI system, Gilad et al. \cite{Gilad_Amir_Levontin_2021} found warmth to be a much more crucial factor than competence. However, in general, prior work agrees that designing AI systems with high warmth and high competence is an effective approach \cite{Gilad_Amir_Levontin_2021, Jung_Qiu_Bozzon_Gadiraju_2022, Khadpe_Krishna_Fei-Fei_Hancock_Bernstein_2020}. 

More pertinent to VUIs, Braun et al. \cite{Braun_Mainz_Chadowitz_Pfleging_Alt_2019} used Argyle’s model of attitude towards others and placed metaphors like Sherlock Holmes, Sheldon Cooper, HAL9000, etc., along the dimensions of dominant-submissive and hostile-friendly. However, in their preliminary study, the researchers found a hostile agent to be unsuitable—similar to previously mentioned studies in which participants showed a higher preference for high-warmth metaphors. Consequently, based on user feedback, the researchers changed the hostile-friendly scale to a ‘formality’ scale of casual-formal. The adapted dimensions based on ‘hierarchy’ and ‘formality’ more meaningfully presented the roles played by popular VUIs. We are inspired by Braun et al.'s adapted scale and use it in this research. 






\subsection{Measures}
To determine the most suitable metaphor aligned with user preferences, we developed a methodology that maps metaphors on a 7-point Likert scale across dimensions of hierarchy and formality. We then assessed the desired levels of hierarchy and professionalism for VUIs in various conversational use-contexts. It is crucial to emphasize that our study's objective is to identify metaphors that best match users' mental models for specific VUI tasks. While the metaphors we identify may have broad applicability, designers should ultimately select metaphors that align with their specific design goals.

To ground our research in real-world VUI usage, we utilized five sample task scenarios from existing literature, presenting these to Prolific participants (as shown in Table \ref{tab:vui_tasks}). The scenarios, covering commands, sociality, and information seeking, were chosen for their commonality and ease of comprehension for frequent users. For error recovery scenarios, which can be more challenging to envision, we provided detailed examples of conversational breakdowns. These examples were sourced from Baughan et al.'s dataset \cite{Baughan_Wang_Liu_Mercurio_Chen_Ma_2023}, which includes comprehensive logs of 199 actual voice assistant failures. We selected five common reasons for conversational breakdowns from this dataset, complete with real-world examples.

To evaluate the formality and hierarchy of metaphors, such as 'butler,' we employed specific prompts. For instance, "On a scale of 1 to 7, where 1 indicates very casual and 7 indicates very formal, how would you imagine the formality of a character behaving like a "Butler"?" and "On a scale of 1 to 7, where 1 indicates very subordinate and 7 indicates very dominant, how would you imagine the hierarchy of a character behaving like a "Butler"?" These scales ranged from very casual to very formal, and very submissive to very dominant, respectively. For task-specific assessments, such as checking the weather, we used similar scales with prompts like "On a scale of 1 to 7, where 1 indicates very subordinate and 7 indicates very dominant, please indicate the level of hierarchy you would expect from a voice agent (e.g., Apple Siri or Amazon Alexa) when a user asks for the weather" and "On a scale of 1 to 7, where 1 indicates very casual and 7 indicates very formal, please indicate the level of formality you would expect from a voice agent when a user is checking the weather." To avoid potential bias, we deliberately excluded Google Assistant as an example in these prompts, given its explicit association with the assistant metaphor.
\begin{table}[ht]
\centering
\caption{Tasks scenarios sourced from VUI literature for each conversational use-context}
\begin{tabular}{l|l|l}
\hline
\textbf{Use-contexts} & \textbf{Task-scenarios} & \textbf{Source} \\ \hline
Commands
    & Checking the weather & \cite{Bentley_Luvogt_Silverman_Wirasinghe_White_Lottridge_2018} \\ \cline{2-3}
    & Turning the reading lights on & \cite{Bentley_Luvogt_Silverman_Wirasinghe_White_Lottridge_2018} \\ \cline{2-3}
    & Turning the music volume up & \cite{Bentley_Luvogt_Silverman_Wirasinghe_White_Lottridge_2018} \\ \cline{2-3}
    & Set a timer for 15 minutes & \cite{Bentley_Luvogt_Silverman_Wirasinghe_White_Lottridge_2018} \\ \cline{2-3}
    & Create a shopping list for grocery & \cite{Bentley_Luvogt_Silverman_Wirasinghe_White_Lottridge_2018} \\ \hline
Sociality
    & Asking for a joke & \cite{Sciuto_Saini_Forlizzi_Hong_2018} \\ \cline{2-3}
    & Asking for meaning of life & \cite{Desai_Twidale_2022} \\ \cline{2-3}
    & Playing a game & \cite{Pradhan_Findlater_Lazar_2019} \\ \cline{2-3}
    & Exchanging pleasantries & \cite{Pradhan_Findlater_Lazar_2019} \\ \cline{2-3}
    & Talking about user's day & \cite{Pradhan_Findlater_Lazar_2019} \\ \hline
Information Seeking
    & Getting directions to the closest Starbucks & \cite{Baughan_Wang_Liu_Mercurio_Chen_Ma_2023} \\ \cline{2-3}
    & Height of Statue of Liberty & \cite{Desai_Twidale_2023} \\ \cline{2-3}
    & Checking the score of a football game & \cite{Baughan_Wang_Liu_Mercurio_Chen_Ma_2023} \\ \cline{2-3}
    & Symptoms of diabetes & \cite{Harrington_Garg_Woodward_Williams_2022} \\ \cline{2-3}
    & Precautions to lower blood pressure & \cite{Harrington_Garg_Woodward_Williams_2022} \\ \hline
Error Recovery
    & Fails to understand the user & \cite{Baughan_Wang_Liu_Mercurio_Chen_Ma_2023} \\ \cline{2-3}
    & Provides incorrect information or performs incorrect action & \cite{Baughan_Wang_Liu_Mercurio_Chen_Ma_2023} \\ \cline{2-3}
    & Provides only partially correct information & \cite{Baughan_Wang_Liu_Mercurio_Chen_Ma_2023} \\ \cline{2-3}
    & Does not identify the correct context & \cite{Baughan_Wang_Liu_Mercurio_Chen_Ma_2023} \\ \cline{2-3}
    & Does not respond & \cite{Baughan_Wang_Liu_Mercurio_Chen_Ma_2023} \\ \hline
\end{tabular}
\label{tab:vui_tasks}
\end{table}


\subsection{Procedure}
The research process began with participants accessing a Qualtrics survey via a link provided through Prolific. Initially, participants encountered an online consent form designed in compliance with Institutional Review Board guidelines. This step was followed by a demographic questionnaire, which collected essential information about the participants, including their age, gender, education level, and their experience and familiarity with VUIs. The core of the study involved participants rating 20 diverse metaphors and 20 common tasks or scenarios across four use-contexts: commands, sociality, information seeking, and error recovery. The metaphors ranged from human roles (such as Admirer, Butler, and Teacher) to non-human concepts (like the "Computer" from Star Trek and the Search Engine). Both metaphors and scenarios were presented to each participant in a randomized sequence to mitigate potential order effects. Participants rated these on two scales: from 'very casual' to 'very formal', and from 'very subordinate' to 'very dominant'. This approach aimed to gauge the expected levels of formality and hierarchy in VUI interactions. The scenarios covered various interactions, from mundane tasks like playing music to more complex situations involving incomplete information or philosophical discussions. Additionally, participants were asked to consider VUI responses in error scenarios, providing insights into desired VUI behavior during suboptimal interactions. Upon completing the survey, participants received a unique validation code. This code confirmed their participation on the Prolific platform, ensuring they received appropriate compensation for their time and contributions to the study.

\subsection{Participants}

Our study initially recruited 160 participants through Prolific,\footnote{https://www.prolific.com/}, a research recruitment platform. However, 30 participants were excluded due to failing attention check questions or timing out, leaving a final sample of 130 participants with an average age of 38.19 years (SD=12.79 years). Selection criteria included a 98\% approval rate and at least one year of platform activity. We sought individuals experienced with voice interfaces to ensure meaningful engagement with the study's tasks and scenarios. To control for cultural variations in metaphor interpretation \cite{Moser_2000}, we limited participation to English-proficient individuals based in the U.S. The participant pool comprised 50.8\% females, 46.2\% males, and 3.1\% non-binary individuals, with 70\% holding at least an Associate's degree. While all participants were U.S. residents and fluent in English, 97\% were native speakers. Regarding VUI familiarity, all participants reported at least moderate familiarity, with 75\% claiming high or very high familiarity. Usage frequency varied, with all participants using VUIs weekly and 23.1\% engaging daily. The survey took an average of 8.5 minutes to complete, and participants received compensation at a fair rate, as informed by Prolific's policies.



\subsection{Results}
\subsubsection{Mapping Metaphors}
Our analysis includes data from all five task scenarios to determine the average formality and hierarchy preferences for each context. The results revealed distinct preferences across different interaction types. When seeking information, participants showed a preference for Voice User Interfaces (VUIs) to exhibit the highest levels of formality (M=4.67) and dominance (M=3.68). In contrast, during social interactions, participants favored a less formal approach from VUIs (M=2.64).
Notably, in error recovery situations, participants expressed a preference for VUIs to display the least dominance (M=2.90) while maintaining a slightly formal demeanor (M=4.49). For command-based interactions, participants desired VUIs to maintain neutral formality (M=3.88) and adopt a slightly subordinate posture (M=3.06). Considering all four use-contexts collectively, the ideal VUI, according to participants, should demonstrate neutral formality (M=3.92) and a slightly subordinate stance (M=3.23). These findings suggest a nuanced preference for VUI behavior that adapts to the specific nature of the interaction. Table \ref{tab:formality_hierarchy} summarizes means and standard deviations for formality and hierarchy ratings.
\begin{table}[H]
\centering
\caption{The desired Formality and Hierarchy ratings ($\mu \pm \sigma$: mean and standard deviation) of the 4 use-contexts on a 7-point Likert Scale}
\begin{tabular}{l|l|l}
\hline
\textbf{Use-Context} & \textbf{Formality} & \textbf{Hierarchy} \\ \hline
Commands & $3.88 \pm 0.32$ & $3.06 \pm 0.25$ \\ \hline
Sociality & $2.64 \pm 0.84$ & $3.29 \pm 0.33$ \\ \hline
Information Seeking & $4.67 \pm 0.86$ & $3.68 \pm 0.33$ \\ \hline
Error Recovery & $4.49 \pm 0.11$ & $2.90 \pm 0.16$ \\ \hline
Overall & $3.92 \pm 0.92$ & $3.23 \pm 0.34$ \\ \hline
\end{tabular}
\label{tab:formality_hierarchy}
\end{table}
Participants also evaluated each metaphor on formality and hierarchy scales. The Boss metaphor received the highest ratings for both formality (M=6.23) and dominance (M=6.37). Conversely, the Family Pet metaphor was rated lowest in formality (M=1.54) and dominance (M=1.83). The Genie and Search Engine metaphors were rated closest to neutral in both formality (M=3.76, M=3.07 respectively) and hierarchy (M=4.28, M=3.12 respectively). Table \ref{tab:metaphor_formality_hierarchy} presents the complete formality and hierarchy ratings for all metaphors used in the study.
\begin{table}[H]
\centering
\caption{The Formality and Hierarchy ratings ($\mu \pm \sigma$: mean and standard deviation) of the 20 metaphors on a 7-point Likert Scale}
\begin{tabular}{l|l|l}
\hline
\textbf{Metaphor} & \textbf{Formality} & \textbf{Hierarchy} \\ \hline
Assistant & $5.03 \pm 1.30$ & $2.50 \pm 1.15$ \\ \hline
Aunt & $2.79 \pm 1.41$ & $4.5 \pm 0.99$ \\ \hline
Boss & $6.23 \pm 0.93$ & $6.37 \pm 0.70$ \\ \hline
Butler & $5.91 \pm 1.43$ & $2.13 \pm 1.17$ \\ \hline
Child & $1.94 \pm 1.21$ & $2.06 \pm 1.11$ \\ \hline
Coach & $4.72 \pm 1.40$ & $5.99 \pm 0.99$ \\ \hline
Companion & $2.4 \pm 1.35$ & $3.9 \pm 0.71$ \\ \hline
"Computer" from Star Trek & $5.11 \pm 1.59$ & $3.23 \pm 1.52$ \\ \hline
Customer Service Agent & $5.24 \pm 1.22$ & $3.36 \pm 1.11$ \\ \hline
Encyclopedia & $5.57 \pm 1.51$ & $3.76 \pm 1.43$ \\ \hline
Family Pet & $1.54 \pm 1.06$ & $1.83 \pm 1.09$ \\ \hline
Flame & $2.37 \pm 1.31$ & $3.84 \pm 0.82$ \\ \hline
Friend & $1.73 \pm 0.95$ & $3.83 \pm 0.66$ \\ \hline
Genie & $3.76 \pm 1.61$ & $3.07 \pm 1.81$ \\ \hline
Librarian & $5.35 \pm 1.20$ & $4.42 \pm 1.03$ \\ \hline
Nurse & $5.32 \pm 1.02$ & $4.79 \pm 1.11$ \\ \hline
Search Engine & $4.28 \pm 1.67$ & $3.12 \pm 1.54$ \\ \hline
Teacher & $5.63 \pm 0.96$ & $5.63 \pm 0.87$ \\ \hline
Therapist & $5.45 \pm 1.14$ & $4.73 \pm 1.1$ \\ \hline
\end{tabular}
\label{tab:metaphor_formality_hierarchy}
\end{table}
Figure \ref{fig:metmap} provides a visualization of all 20 metaphors and 4 use-contexts along the dimensions of hierarchy and formality. To address RQ1, we calculated Euclidean distances between each metaphor and use-context to determine metaphor-context alignment. For the Commands context, the Genie metaphor showed the closest alignment (D=0.11) in terms of formality and hierarchy. The Admirer metaphor was deemed most appropriate for Sociality (D=0.5). In the Information Seeking context, participants favored interaction with the "Computer" from Star Trek (D=0.63). For Error Recovery scenarios, the Search Engine metaphor aligned most closely with participant preferences (D=0.30). Considering all use-contexts collectively, the Genie metaphor emerged as the most suitable overall (D=0.22). Euclidean distances of all metaphors from each context are included in Appendix A. 
\begin{figure}[H]
    \centering
    \includegraphics[width=\textwidth]{metmap.png}
    \caption{Mapping metaphors and conversational use-contexts on a 7-point scale along the dimensions of formality and hierarchy}
    \label{fig:metmap}
\end{figure}

Table \ref{tab:closest_metaphors_separate} answers RQ1 by summarizing the results of Study 1, identifying the closest metaphor for each use-context. This illustrates which metaphors align most closely with specific contexts based on the shortest Euclidean distance. These results provide insights into how users may relate to different metaphors in distinct conversational scenarios.

\begin{table}[H]
\centering
\caption{The closest metaphor to each use-context, along with the corresponding Euclidean distance.}
\begin{tabular}{l|l|l}
\hline
\textbf{Use-Context} & \textbf{Closest Metaphor} & \textbf{Euclidean Distance} \\ \hline
Commands & Genie & $0.11$ \\ \hline
Sociality & Admirer & $0.50$ \\ \hline
Information Seeking & "Computer" from Star Trek & $0.63$ \\ \hline
Error Recovery & Search Engine & $0.30$ \\ \hline
\end{tabular}
\label{tab:closest_metaphors_separate}
\end{table}

Upon further examination of the data, we noticed intriguing patterns in how the metaphors appeared to be grouped. To explore these patterns in greater depth, we performed a cluster analysis. This analysis was not originally part of our research questions, but it provided additional insights into the relationships between metaphors and how they naturally cluster to form meaningful roles.

\subsubsection{Clustering Metaphors}
To determine the optimal number of clusters, we applied the K-means algorithm to the dataset. The elbow method, depicted in Figure \ref{fig:elbow}, suggested a noticeable inflection point at $k = 4$, indicating a natural division in the metaphorical space. However, the gradual decline in the within-cluster sum of squares (WCSS) beyond this point suggested that additional meaningful distinctions could be found.

The silhouette scores, shown in Figure \ref{fig:silhouette}, were used to further evaluate the optimal number of clusters. These scores helped assess the cohesion within each cluster by comparing the similarity of metaphors within the same cluster against others. The scores peaked at $k = 4$, closely followed by $k = 5$, which indicated that either number could provide valuable groupings.

\begin{figure}[H]
    \centering
    \begin{subfigure}[t]{0.45\textwidth}
        \centering
        \includegraphics[width=\textwidth]{elbow.png}
        \caption{Elbow method for determining the optimal number of clusters.}
        \label{fig:elbow}
    \end{subfigure}
    \hfill
    \begin{subfigure}[t]{0.45\textwidth}
        \centering
        \includegraphics[width=\textwidth]{sil.png}
        \caption{Silhouette scores for different numbers of clusters.}
        \label{fig:silhouette}
    \end{subfigure}
    \caption{Comparison of the Elbow method and Silhouette scores for clustering analysis.}
    \label{fig:combined}
\end{figure}





 
Considering the elbow method and silhouette analysis, we opted to proceed with $k = 5$, representing five distinct metaphorical roles. The rationale for this choice was based on examining cluster centroids and their proximity to the use-context means, which showed that one of the centroids in the $k = 5$ solution closely aligned with the overall mean across all use-contexts (D=0.29). This indicated that the cluster captured a role that resonated with general user preferences across different scenarios. Moreover, the $k = 5$ configuration revealed a clearer separation of roles, including a specific cluster representing neutrality in formality and slight subordination.

\begin{figure}[H]
    \centering
    \includegraphics[width=0.8\textwidth]{5Clusters.png}
    \caption{K-means clustering of VUI metaphors into five distinct roles based on formality and hierarchy, with centroids indicating the archetypal interaction style for each cluster.}
    \label{fig:cluster}
\end{figure}


The five clusters represent distinct VUI roles, reflecting different combinations of formality and hierarchy:

\begin{itemize}
    \item \textbf{Companions (Blue Cluster)}: These VUIs resemble a friendly presence, offering neutral hierarchy and informal interaction. They are well-suited for contexts that benefit from a relaxed and familiar experience, like interactions with a friend. They were closest to Sociality (D=0.80) and furthest to Error Recovery (D=2.43). 
    \item \textbf{Facilitators (Yellow Cluster)}: Facilitators are formal and somewhat dominant, ideal for learning tasks or providing specialized information, akin to roles like a teacher or nurse. Metaphors contained in these roles were closest to Information Seeking (D=1.82) and furthest to Sociality (D=3.46).
    \item \textbf{Entertainers (Green Cluster)}: These VUIs take on playful, creative roles that may help bring humor and fun into interactions, similar to a pet or child, and are useful for informal, tension-relieving scenarios. Entertainers were adjudged to be the closest to Sociality (D=1.69) and furthest to Information Seeking (D=3.39).
    \item \textbf{Aides (Red Cluster)}: Aides are formal and slightly subordinate, focused on assisting users in a structured manner, comparable to assistants or butlers. Aides were mapped to be the most appropriate for Information Seeking (D=0.97) and were least appropriate for Sociality (D=2.74). 
    \item \textbf{Guides (Purple Cluster)}: Guides provide expertise and assistance in a neutral, slightly subordinate role, suitable for a variety of tasks, similar to a Genie or search engine. Guides were rated to be the most appropriate for the use-contexts of Commands (D=0.18) and Error Recovery (D=0.80) out of all clusters. They were also relatively close to Sociality (D=1.07) and Information Seeking (D=1.16).
    
\end{itemize}

Designing VUIs that align with specific user roles can significantly enhance the user experience by meeting their expectations in different contexts. Figure \ref{fig:cluster} visualizes these five roles, offering designers guidance for selecting appropriate metaphors based on formality and hierarchy depending on their use-case. 

\subsection{Discussion}
\subsubsection{Non-human and fictional metaphors display user-desired characteristics}
We analyzed 20 different metaphors, including both human and non-human characters such as Admirer, Assistant, Boss, Butler, Child, Coach, Companion, Customer Service Agent, Aunt, Flatmate, Friend, Librarian, Nurse, Teacher, and Therapist. Among these, 5 metaphors were distinctly non-human—such as "Computer" from Star Trek, Encyclopedia, Family Pet, Genie, and Search Engine—and 2 metaphors, specifically "Computer" from Star Trek and Genie, were fictional in nature. Out of all use-contexts, interestingly, only one human metaphor—Admirer for Sociality—was rated as the most suitable for the chosen use-contexts.

This finding is insightful because it challenges the common assumption that human metaphors would naturally be the most relatable or suitable for VUIs. Instead, our results suggest that non-human and fictional metaphors are often more compatible with users' desired levels of formality and hierarchy in interactions. This aligns with previous work, such as Pradhan et al. \cite{Pradhan_Findlater_Lazar_2019}, who found that older adults categorize VUIs as both persons and objects, underscoring the unique hybrid role VUIs occupy. Similarly, Jung et al. \cite{Jung_Qiu_Bozzon_Gadiraju_2022} found no significant preference for human over non-human metaphors in chatbots, challenging the assumption that human-like qualities are inherently more effective. Desai et al. \cite{desai2024cui} further showed that metaphor preferences are context-dependent, with human metaphors preferred in health but no clear preference in finance.

These findings provide important implications for VUI designers: selecting metaphors that users find appropriate may require moving beyond the human metaphor and instead embracing characters that exist outside of human conventions and cultural associations. There is a natural tendency in conversation design to rely on human metaphors \cite{Desai_Twidale_2023, Doyle_Edwards_Dumbleton_Clark_Cowan_2019}, though these often come with ethical and practical concerns, as previously discussed. By contrast, non-human metaphors remain relatively underutilized, even though previous findings suggest that these metaphors may be better suited for user-centered interaction design \cite{desai2024cui}.

One reason for this could be the absence of cultural baggage—non-human and fictional characters often present a cleaner slate, allowing users to approach VUIs without preconceived biases. For instance, the concept of an 'Assistant' might vary greatly depending on individuals' experiences, whereas characters like a Genie or "Computer" from Star Trek have clearly defined roles popularized through mass media, setting consistent and realistic expectations. Furthermore, these non-human and fictional metaphors often appear in assistive roles in popular culture, similar to how commercial VUIs function in our everyday lives. They provide a more accessible and pragmatic way for users to conceptualize VUI interactions.

A character like a Genie, for example, is culturally ingrained as an entity that is always ready to fulfill a request—akin to how VUIs operate with wake words such as "Hey Google" or "Alexa." This parallel between the invocation of mythical beings and modern technology helps establish a natural and inviting interaction model. Among the metaphors we studied, the Genie stood out as particularly compelling. It was rated as the most suitable metaphor across multiple-use-contexts, resonating with users for its distinctive mix of neutral formality and slight subordination. This balance makes the Genie metaphor feel approachable but not overly familiar, attentive but not intrusive—much like the lore of a genie, who is always present to fulfill wishes upon being summoned. This metaphor is not necessarily an instruction to personify VUIs as genies but rather an invitation to identify the underlying traits that make the metaphor resonate with users. It also solidifies the idea of using magic as a transporting metaphor in AI design \cite{Lupetti_Murray-Rust_2024}.

The broader role of a Guide, which embodies these characteristics, appears to include a wide range of metaphors—from the human-like Admirer to the more technological Search Engine and the fictional Genie. This flexibility points to a unique user perception where both human and non-human guides are seen as effective, with perhaps a stronger inclination toward non-human entities for roles involving assistance or direction. This observation opens exciting possibilities for future research. By exploring the Guide role, we may be able to uncover additional metaphors that capture similar desirable attributes, potentially creating characters that blend the best elements of both humans and technology.

\subsubsection{Exploring the roles of Entertainers and Facilitators}

Although we did not identify a specific use-context where metaphors categorized under the Entertainer and Facilitator roles were most suitable, these roles still warrant exploration due to their unique characteristics. Entertainers are marked by low formality and low dominance, whereas Facilitators exhibit high formality and high dominance. In this sense, these roles are diametrically opposed. However, both can have value in specialized contexts, and understanding this requires examining how these metaphors were originally conceptualized. To achieve this, we utilize the metaphor contextualization framework.

The Entertainer role includes metaphors such as "child" and "family pet," which were often applied by users to make sense of conversational challenges or errors. While designing VUIs around the persona of Entertainers may not appeal to everyone, this role can still be useful in alleviating tension during conversational breakdowns. There may be users who prefer a VUI to maintain a casual, subordinate tone regardless of the context. Although Entertainers might not be ideal for all interactions, they present interesting opportunities for VUIs designed to embody lightheartedness and ease.

On the other hand, Facilitators encompass metaphors like "Boss," "Coach," "Teacher," "Nurse," "Therapist," and "Librarian," offering a broader range of design applications. Notably, these Facilitator metaphors tend to be introduced by designers or researchers rather than being directly adopted by users, in contrast to the more user-driven application of Entertainer metaphors. This discrepancy may stem from the fact that Facilitator metaphors are primarily used in specialized domains like health or finance, where authority or structure is crucial. However, these metaphors also raise concerns about assigning social roles to VUIs, which can present ethical and practical challenges \cite{Simpson_Crone_2022}. An alternative approach could involve using non-human or fictional metaphors rather than conventional social roles within the Facilitator category, although finding non-human or fictional metaphors that convey both high formality and high dominance remains a challenge.

A potential solution to these challenges is the use of a layered metaphor strategy. In specialized contexts where accuracy and authority are critical, the social metaphors within the Facilitator category can establish a foundational persona, setting clear expectations for users. Rather than restricting the VUI's persona to a single metaphor, designers can adopt a more dynamic approach, integrating multiple metaphors that align with specific use-contexts. This is akin to how the desktop metaphor is augmented by concepts such as files, folders, and icons in graphical user interfaces.

This approach offers two advantages. First, it allows for a consistent overarching identity. Second, it introduces adaptability through secondary metaphors that match the nuances of particular interactions—whether that requires the playful supportiveness of Entertainers or the neutral assistance of Guides. For example, in a health-focused VUI, the primary metaphor presented to the user could be that of a "Nurse," embodying care and trust. However, context-specific layers of interaction might utilize a "Search Engine" metaphor to manage conversational recovery after an error or an "Encyclopedia" metaphor when the user requests in-depth information on a topic. By employing this layered metaphor strategy, designers can create rich, contextually appropriate experiences that align with user expectations while maintaining ease of interaction.

The success of this solution lies in thoughtful execution. Designers must carefully map out intents and contexts within the VUI, assigning appropriate metaphors that enhance the interaction without overshadowing the foundational metaphor. In practice, this involves user studies to determine effective metaphor pairings and iterative design processes to refine how these metaphors are presented and interacted with.

\section{Study 2: Comparing Metaphor-Fluid VUI to a Default VUI}

Informed by the results of RQ1, we designed a within-subjects study (N=91) comparing a Metaphor-Fluid VUI and a Default VUI to answer RQ2. The Metaphor-Fluid VUI used four distinct metaphors identified as the most appropriate for use-contexts: Commands, Information Seeking, Sociality, and Error Recovery. Specifically, Genie metaphor was used for Commands, “Computer” from Star Trek for Information Seeking, Admirer for Sociality, and Search Engine metaphor for Error Recovery. In contrast, the Default VUI maintained a single, consistent Assistant metaphor across all interactions and was designed to emulate Google Assistant. This allowed us to examine the effects of metaphor-fluidity on perceived enjoyment, intention to adopt, trust, likability, and intelligence in VUI interactions. 

\subsection{Methods}
\subsubsection{Designing Metaphorical VUIs}

In our study, we designed two VUIs: a Metaphor-Fluid VUI and a Default VUI. These designs fall within the broader domain of conversation design, a field that is still in its formative stages. Currently, there is a notable lack of standardized design processes followed by conversation designers \cite{Sadek_Calvo_Mougenot_2023}. Moreover, there are few design tools or pre-defined personas that can be readily deployed, leaving designers to navigate these challenges on their own. As a result, conversation designers often rely on guidelines provided by industry leaders such as Google and IBM, adapting these frameworks to suit specific contexts.
To address these challenges, the first author, who has eight years of experience in conversation design, designed the interactions for our study. Two experts, each with six years of experience, further validated the designs, ensuring an informed approach to the development of the VUIs.

While designing the VUI interactions, our primary focus was to ensure ecological validity. To achieve this, we drew on insights from Bentley et al. \cite{Bentley_Luvogt_Silverman_Wirasinghe_White_Lottridge_2018}, who analyzed user logs from 88 diverse homes over a 110-day period, encompassing 66,459 interactions. Their study explored the long-term use of voice assistants, revealing two key patterns that informed our design: (1) users fluidly transitioned between use-contexts within the same interaction session, and (2) the average session length was 5.4 interaction turns. Incorporating these findings, we developed scripts featuring six interaction turns, including five unique commands and one error recovery instance, across four use-contexts. 

We designed Metaphor-Fluid VUI using the metaphors Genie, "Computer" from Star Trek, Admirer, and Search Engine for the use-contexts Commands, Information Seeking, Sociality, and Error Recovery respectively. To design VUI using the Genie metaphor, we used the taxonomy developed by Lupetti et al. \cite{Lupetti_Murray-Rust_2024} examining the role of enchantment in design communication using magic metaphors. Specifically, we used the communication style of Genie from \textit{Arabian Nights} cartoons to invite the users to "suspend their disbelief" using familiar tropes. For example, the VUI portrayed a friendly and jovial Genie reading a magical orb to provide weather updates.  To design the VUI using the Admirer metaphor, we focused on creating a warm and supportive persona. The Admirer uses affirming and enthusiastic communication to foster a positive interaction. For instance, when a user asks for a joke, the VUI starts with, "You know, you have the best taste in humor," before delivering the punchline. This approach reflects the Admirer’s intent to please the user and make interactions enjoyable. For "Computer" from Star Trek metaphor, we used analysis by Axtell et al. \cite{Axtell_Munteanu_2021} who studied dialogs by the \textit{Enterprise} Computer and the crew from \textit{Star Trek: The Next Generation} including 587 interactions. The analysis revealed that "Computer" used brief and functional dialogs, and included relevant strategies such as indicating actions in progress (using the keyword "accessing") and asking clarifying questions. We incorporated these strategies into our interaction where a user asks for sports scores. To design the VUI using the Search Engine metaphor, we focused on creating a system that mirrors how search engines handle ambiguous queries—by providing initial results based on the most likely interpretation and offering follow-up suggestions for alternative meanings. Inspired by Google's "Did you mean?" strategy, the VUI proactively clarifies or refines queries, aligning closely with users’ mental models of how search engines manage ambiguity. 

In contrast, for the Default VUI, we utilized responses directly transcribed from Google Assistant as of 10/28/2024. We selected Google Assistant because it explicitly embodies the widely recognized "Assistant" metaphor, and its persona design is well-documented. According to Google's conversation designers \cite{Pradhan_Lazar_2021}:

\begin{quote}
Google Assistant is characterized as a young woman from Colorado, the youngest daughter of a research librarian and a physics professor. She holds a B.A. in history from Northwestern University, an elite research institution in the United States. Her backstory includes winning \$100,000 on \textit{Jeopardy Kids Edition} as a child, working as a personal assistant to a famous late-night TV satirical pundit, and enjoying activities like kayaking.
\end{quote}

This detailed persona informed the Default VUI's interaction style, which is designed to be approachable, knowledgeable, and relatable.



\begin{table}[h!]
    \centering
    \renewcommand{\arraystretch}{1.6} % Increases row height
    \caption{Comparison of Responses between Metaphor-Fluid VUI and Default VUI for the same User Prompts}
    \begin{tabular}{p{3cm}|p{5.5cm}|p{5.5cm}}
        \hline
         % Header background color
        \textbf{User Prompt} & \textbf{Metaphor-Fluid VUI} & \textbf{Default VUI} \\
        \hline
        
        % Weather Request
      \includegraphics[width=0.6cm, height=0.6cm]{user_icon.png} Hey Z, can you check the weather for me? &
        \cellcolor{blue!10} \includegraphics[width=0.6cm, height=0.6cm]{genie_icon.png} Of course, give me but a moment. What am I seeing then? Hmm, there we go. It's currently 66 with a radiant sun with a high of 78 and a gentle breeze, dipping to a low of 62. A splendid day awaits. &
        \cellcolor{gray!10} \includegraphics[width=0.6cm, height=0.6cm]{speaker_icon.png} It's 66 and sunny. Today, it will be sunny, with a gentle breeze, and a high of 78 and a low of 62. \\
        \hline
        
        % Joke Request
     \includegraphics[width=0.6cm, height=0.6cm]{user_icon.png} OK Z, can you tell me a joke? &
        \cellcolor{red!10} \includegraphics[width=0.7cm, height=0.7cm]{admirer_icon.png} Absolutely. You know, you have the best taste in humor. Okay, here it goes. Why did the scarecrow win an award? Because he was outstanding in his field. &
        \cellcolor{gray!10} \includegraphics[width=0.6cm, height=0.6cm]{speaker_icon.png} Why did the scarecrow win an award? Because he was outstanding in his field. \\
        \hline
        
        % Sports Score Request
   \includegraphics[width=0.6cm, height=0.6cm]{user_icon.png} Good one, Z. Can you also check the score of the football game between the New York Jets and the New England Patriots? &
        \cellcolor{purple!10} \includegraphics[width=0.5cm, height=0.5cm]{computer_icon.png} Accessing current sports data. Score update. New York Jets 17, New England Patriots 14. Is there anything else you require? &
        \cellcolor{gray!10} \includegraphics[width=0.6cm, height=0.6cm]{speaker_icon.png} Here's information from CBS Sports. Jets are beating Patriots 17 to 14. Is there anything else I can do? \\
        \hline

        % Age Inquiry
        \includegraphics[width=0.6cm, height=0.6cm]{user_icon.png} Yes, Z. How old is Johnny Walker? &
        \cellcolor{teal!10} \includegraphics[width=0.4cm, height=0.4cm]{search_icon.png} Giving you results for Johnnie Walker, the scotch whiskey. The brand was established in 1820. So, as of 2023, Johnny Walker would be 203 years old in terms of its history and establishment. However, did you mean Johnny Walker, the UFC fighter? &
        \cellcolor{gray!10} \includegraphics[width=0.6cm, height=0.6cm]{speaker_icon.png}According to the website, whiskey.fandom.com, they say Johnnie Walker Black Label blend is at least 12 years old. Johnnie Walker, Platinum Label is 18 years old. \\
        \hline

       \includegraphics[width=0.6cm, height=0.6cm]{user_icon.png} The fighter &
        \cellcolor{teal!10} \includegraphics[width=0.4cm, height=0.4cm]{search_icon.png} Johnny Walker, the UFC fighter, is 31 years old. &
        \cellcolor{gray!10} \includegraphics[width=0.6cm, height=0.6cm]{speaker_icon.png} Johnny Walker is 31 years old. \\
        \hline

        % Farewell
        \includegraphics[width=0.6cm, height=0.6cm]{user_icon.png} Thanks, Z! &
        \cellcolor{red!10} \includegraphics[width=0.7cm, height=0.7cm]{admirer_icon.png} You are welcome! I am here for you. &
        \cellcolor{gray!10} \includegraphics[width=0.6cm, height=0.6cm]{speaker_icon.png} You are very welcome! \\
        \hline
    \end{tabular}
    \label{tab:comparison}
\end{table}


























\subsubsection{Designing for Voice}

The audio clips for this study were generated using the text-to-speech (TTS) software, Speechify.\footnote{\url{https://speechify.com/}} Speechify was chosen for its ability to produce realistic synthetic conversations using multiple voices, which aligned with the study's design requirements. While an initial approach considered employing a human voice actor for the user's role and reserving TTS exclusively for the VUI, pilot testing revealed challenges in maintaining a consistent pitch and tone with the human voice across the experimental conditions. These inconsistencies risked introducing confounding variables, as prior research, such as Dubiel et al.~\cite{dubiel2024impact,dubiel2020persuasive}, has shown that subtle differences in vocal characteristics can significantly affect user perceptions. Consequently, we opted to use synthetic voices for both the user and the VUI.

In choosing the synthetic voices, we aimed to align with gender representation norms observed in commercial VUIs, such as Siri and Alexa, which often use female-sounding voices~\cite{curry2020conversational}. To ensure a clear auditory distinction between the VUI and the user, we assigned a female-sounding voice to the VUI and a male-sounding voice to the user. This decision was informed by literature suggesting that voice gender may influence user perception~\cite{Kuzminykh_Sun_Govindaraju_Avery_Lank_2020}. To further mitigate biases associated with the VUI's identity, we designated the VUI as `Z,' a neutral name devoid of gender or racial connotations, to minimize confounding effects on the study's outcomes. For consistency across all experimental conditions, we used Speechify's synthetic male voice "Guy" for the user, setting the tone to "chat" to enhance the voice's human-like qualities. For the VUI, we selected the female voice "Aria" with a neutral tone setting of "none." All other parameters, such as speech rate, speed, and volume, were standardized at zero percent modifications to ensure uniformity.

Before each conversation, participants heard a brief introductory prompt: "You are about to hear a conversation between a user with a male-sounding voice and a voice interface named `Z', with a female-sounding voice." This prompt was delivered using the synthetic voice "Davis," configured with the same settings as the study clips to ensure consistency and avoid introducing any additional variables. Additionally, while metaphors were embedded in the design of the VUI, their presence was not made explicit to participants. This decision was informed by prior literature, which found no significant effect of user awareness of metaphors on their perceptions of VUIs \cite{desai2024cui}. 

\subsection{Measures}

To gain a detailed understanding of users' perspectives on metaphorical VUIs, we draw on the classification framework established by Wei et al.~\cite{Wei_Kim_Kuzminykh_2023}. This framework organizes user perceptions of CUIs into two pertinent dimensions: (1) perceptions regarding interaction with the agent and (2) perceptions of the agent's inherent characteristics. This classification stems from a literature review, focusing on commonly used metrics within CUI research to assess user experiences with VUIs and their views on agent attributes. Following this structure, our study incorporates these key metrics:

\begin{itemize} \item \textit{\textbf{Perceptions of interaction with agents}} assesses evaluative metrics that capture the quality of interaction between a user and a CUI. This aspect encompasses dimensions like engagement, including constructs such as perceived enjoyment and intention to use. In the field of VUI research, there is considerable interest in understanding how VUIs enhance user enjoyment~\cite{Yang_Aurisicchio_Baxter_2019} and contribute to overall quality of life \cite{Desai_Twidale_2023}, beyond mere usability. To measure \textit{\textbf{perceived enjoyment}}, we employ an adapted survey by Moussawi et al.\cite{Moussawi_Koufaris_Benbunan-Fich_2021}, which includes three items rated on a 7-point Likert scale: “I would find the interaction enjoyable while using Z,” “I would find this interaction interesting while using Z,” and “I would find the interaction fun while using Z.” For assessing \textbf{\textit{perceived intention to adopt}}, we adapt Moussawi et al.’s items~\cite{Moussawi_Koufaris_Benbunan-Fich_2021}, presenting two 7-point Likert scale items: “If available, I intend to start using Z within the next month” and “If available, I plan to try or regularly use Z in the coming months.” Both measures have been adapted specifically for the VUI context.

\item \textbf{\textit{Perceptions of the agent's characteristics}} involves metrics aimed at understanding user perceptions of a CUI’s capabilities and personality, such as competence, likability, and trustworthiness. To evaluate variations in perceptions of VUI characteristics across different scenarios, we included \textbf{\textit{perceived intelligence}} and \textbf{\textit{likability}} measures, drawn from the Godspeed Measures and utilizing 5-point semantic differential scales \cite{Bartneck_2023, Bartneck_Kulić_Croft_Zoghbi_2009}, which are extensively used in VUI studies \cite{Wei_Kim_Kuzminykh_2023, Seaborn_Urakami_2021}. Furthermore, \textbf{\textit{perceived trust}} was measured with a 7-point Likert scale adapted from Jian et al.~\cite{Jian_Bisantz_Drury_2000}. \end{itemize}

Additionally, participants were invited to share qualitative feedback on their experiences with both versions of Z if they wished to provide further insights. Further, to explore whether varying metaphorical designs influenced user perceptions, we asked participants to describe the metaphors that came to mind in relation to Z, allowing for qualitative insights into how different metaphorical aspects of the VUIs shaped their impressions.

\subsection{Procedure}
The research process began with participants accessing a Qualtrics survey via a link provided through Prolific. Initially, participants filled out an online consent form crafted to align with Institutional Review Board guidelines. Following consent, participants completed a brief demographic questionnaire to gather basic details, such as age, gender, educational background, and their experience with VUIs. The main phase of the study involved each participant listening to two audio clips, one from a Metaphor-Fluid VUI and the other from a Default VUI, in a within-subjects design. The order of the clips was counter-balanced to reduce any potential order effects. After each audio clip, participants responded to a series of questions designed to capture their perceptions of the VUI, focusing on aspects like enjoyment, intelligence, trust, likability, and informativeness. Participants were also prompted to answer two open-ended questions: one asking for any additional feedback on  Z and the other inviting them to suggest metaphors they would use to describe it. Additionally, an attention-check audio clip was included to ensure participants’ engagement. Following this clip, participants were required to select specific responses in the accompanying questions, verifying their attentive participation. Upon completing the study, participants were provided with a unique validation code. This code served as confirmation of their participation on the Prolific platform, ensuring they received compensation for their time and input.

\subsection{Participants}

We used Prolific to recruit a total of 110 participants, out of which 19 participants either timed out (3) or failed the attention check question (16), leaving us with N=91. Selection criteria include prior experience with AI and voice interfaces, ensuring familiarity and relevance in their responses to our VUI conditions, and a minimum of 95\% approval rate and one year of activity on the Prolific platform. The participant sample had a mean age of 40.92 years (SD=12.07). Among the recruited individuals, 54.9\% identified as female, 42.9\% as male, and 2.2\% as non-binary. In terms of educational attainment, all participants had a High School Diploma or equivalent, with 54.92\% holding a Bachelor’s degree or higher. All participants were U.S.-based and highly proficient in English, with 96.7\% identifying as native English speakers. Participants demonstrated varying levels of experience with VUIs, with all participants using VUIs at least once a week and 64.82\% using it once a day. Furthermore, 80.2\% described themselves as very or extremely familiar with VUIs, supporting the study’s focus on users with relevant experience. Each participant, on average, took 12.20 minutes and was compensated at an hourly rate that aligns with Prolific’s guidelines, ensuring fair and ethical treatment throughout the study.
\subsection{Results}


\subsubsection{Perception Measures}
The findings suggest notable differences in participant perceptions of the two VUI conditions, particularly in terms of Perceived Enjoyment, Perceived Intention to Adopt, and Perceived Likability. Figure \ref{fig:BoxPlots_MF} shows the distribution across all the perception measures. The Metaphor-Fluid VUI generally received higher mean ratings for Perceived Enjoyment (5.29 $\pm$ 1.22) and Perceived Intention to Adopt (4.40 $\pm$ 1.53) compared to the Default VUI (4.79 $\pm$ 1.47 and 4.02 $\pm$ 1.64, respectively). Notably, Perceived Trust remained similar across both conditions, with mean values of 5.64 $\pm$ 0.93 for the Metaphor-Fluid VUI and 5.63 $\pm$ 0.84 for the Default VUI, indicating little difference in trustworthiness between the two interfaces. Meanwhile, Perceived Likability and Perceived Intelligence showed mixed results: Perceived Likability was slightly higher for the Metaphor-Fluid VUI (4.16 $\pm$ 0.84) compared to the Default VUI (3.93 $\pm$ 0.87), whereas Perceived Intelligence was slightly higher for the Default VUI (4.20 $\pm$ 0.65) than the Metaphor-Fluid VUI (4.09 $\pm$ 0.73).

\begin{figure}[H]
    \centering
    \includegraphics[width=\textwidth]{BoxPlots_MF.png}
    \caption{Box plots comparing Metaphor-Fluid VUI and Default VUI across different perception measures: Perceived Enjoyment, Perceived Intention to Adopt, Perceived Trust, Perceived Likability, and Perceived Intelligence. Each box plot displays the distribution of ratings for each measure under both conditions. The blue horizontal line represents the median rating, while the black diamond marker indicates the mean rating. Outliers are represented by individual points outside the whiskers.}
    \label{fig:BoxPlots_MF}
\end{figure}

Since the distribution of the data was non-parametric, the Wilcoxon signed-rank test was conducted to assess the statistical significance of differences between the two conditions across all measures. As shown in Table \ref{tab:wilcoxon_results}, significant differences were found for Perceived Enjoyment ($W=542.0$, $p=0.000854$, $r=0.350$), Perceived Intention to Adopt ($W=656.5$, $p=0.009776$, $r=0.271$), and Perceived Likability ($W=924.5$, $p=0.028624$, $r=0.229$). These results indicate that participants found the Metaphor-Fluid VUI to be significantly more enjoyable, more likely to be adopted, and slightly more likable compared to the Default VUI. No significant differences were found for Perceived Trust ($W=1393.5$, $p=0.583048$) or Perceived Intelligence ($W=798.5$, $p=0.211011$). The effect sizes for these measures were also small ($r=0.058$ and $r=0.131$, respectively), suggesting that participants viewed both VUI conditions as equally trustworthy and intelligent.



%\begin{table}[H]
%\centering
%\caption{Mean ± SD values ($\mu \pm \sigma$) for each measure under Metaphor-Fluid VUI and Default VUI conditions.}
%\begin{tabular}{l|c|c}
%\hline
%\textbf{Measure} & \textbf{Metaphor-Fluid VUI} & \textbf{Default VUI} \\ \hline
%Perceived Enjoyment & 5.29 ± 1.22 & 4.79 ± 1.47 \\ \hline
%Perceived Intention to Adopt & 4.40 ± 1.53 & 4.02 ± 1.64 \\ \hline
%Perceived Trust & 5.64 ± 0.93 & 5.63 ± 0.84 \\ \hline
%Perceived Likability & 4.16 ± 0.84 & 3.93 ± 0.87 \\ \hline
%Perceived Intelligence & 4.09 ± 0.73 & 4.20 ± 0.65 \\ \hline
%\end{tabular}
%\label{tab:mean_sd_results}
%\end{table}



\begin{table}[H]
\centering
\caption{Results of the Wilcoxon signed-rank test comparing Metaphor-Fluid VUI and Default VUI across different perception measures. Significant p-values are marked with asterisks (* for \( p < 0.05 \), ** for \( p < 0.01 \), *** for \( p < 0.001 \)).}
\begin{tabular}{l|l|l|l}
\hline
\textbf{Measure} & \textbf{W Statistic} & \textbf{p-value} & \textbf{Effect Size (r)} \\ \hline
Perceived Enjoyment & 542.0 & 0.000854*** & 0.350 \\ \hline
Perceived Intention to Adopt & 656.5 & 0.009776** & 0.271 \\ \hline
Perceived Trust & 1393.5 & 0.583048 & 0.058 \\ \hline
Perceived Likability & 924.5 & 0.028624* & 0.229 \\ \hline
Perceived Intelligence & 798.5 & 0.211011 & 0.131 \\ \hline
\end{tabular}
\label{tab:wilcoxon_results}
\end{table}

Overall, the results suggest that the Metaphor-Fluid VUI was generally perceived more favorably in terms of user enjoyment, intention to adopt, and likability, compared to the Default VUI. However, perceptions of trustworthiness and intelligence did not differ significantly between the two interfaces. These findings suggest that incorporating metaphors into VUI design can enhance certain aspects of user experience, such as enjoyment and likability, which may in turn influence users' willingness to adopt the interface. However, these enhancements do not necessarily translate into increased perceptions of trust or intelligence.



\subsubsection{Metaphor Analysis}
In addition to examining how a Metaphor-Fluid VUI compares to a Default VUI in terms of user perceptions, we also explored whether these differences influence the metaphors users use to describe each interface. This work builds on Chin et al. \cite{Chin_Desai_Lin_Mejia_2023}, who investigated how the formality of VUIs affects the metaphors used by older and middle-aged adults. They found that older adults likened a VUI using a formal conversational style to 'professionals' (e.g., librarians, teachers) and an informal VUI to 'close ones' (e.g., aunts, friends). Similarly, in this analysis, we examine how a Metaphor-Fluid VUI shapes metaphorical descriptions.

After interaction with each VUI, users were asked to describe Z using metaphors, with the prompt, "Z is like a..."—adapted from \cite{Chin_Desai_Lin_Mejia_2023}. ''Participants were encouraged to use as many metaphors as they preferred. In total, we received 432 entries. However, not all of them were metaphors, as defined by Barlow et al.'s manual of identifying figurative language \cite{Barlow_Kerlin_Pollio_1971}. We discarded 74 entries as they did not meet the requirements. Most of these entries were adjectives (e.g., "sweet," "caring," "freaky," etc.), and few contained descriptions of the VUI (e.g., "Z is too forward"). After cleaning the data, we were left with 358 metaphors. 

We then categorized these metaphors into three types—human, non-human, and fictional—based on Desai et al.'s framework \cite{Desai_Twidale_2023}. Notably, some metaphors fell into multiple categories. For example, "C-3PO" is both fictional and non-human, while "robot" is considered non-human but also classified as fictional, as prior research has shown that people’s mental models of robots are influenced by science fiction \cite{Kriz_Ferro_Damera_Porter_2010}.

\begin{figure}[H]
    \centering
    \includegraphics[width=\textwidth]{Metaphors_used.png}
    \caption{Comparison of Metaphor Use: All vs Unique for Metaphor-Fluid VUI and Default VUI}
    \label{fig:metaphors_used}
\end{figure}



Figure \ref{fig:metaphors_used} shows the results of our metaphor analysis. When examining all metaphors used for both the Metaphor-Fluid and Default VUIs, we found that participants used a greater number of metaphors to describe the Default VUI (190 total) compared to the Metaphor-Fluid VUI (168 total). In terms of metaphor type, we found that for both VUIs, participants primarily used Human metaphors, with 111 for the Default VUI and 108 for the Metaphor-Fluid VUI. This finding is consistent with previous literature \cite{Desai_Twidale_2023}. However, Non-Human and Fictional metaphors were also prevalent in descriptions of the Default VUI, with 78 Non-Human and 22 Fictional metaphors, compared to 59 Non-Human and 10 Fictional metaphors for the Metaphor-Fluid VUI. 

To contextualize this finding more, we also identified the number of unique metaphors used by the participants to describe each version of Z. For example, although fictional metaphors were used 22 times to describe the default VUI, 20 of those were "robot." Consequently, we counted the number of unique metaphors used for both VUIs to understand the diversity of metaphor use. This analysis revealed a new perspective. In stark contrast to the previous finding, the Metaphor-Fluid VUI elicited a greater variety of metaphors, with a total of 95 unique instances, compared to 77 unique metaphors for the Default VUI. In terms of metaphor types, participants generated 51 unique Human metaphors for the Metaphor-Fluid VUI, slightly higher than the 46 unique Human metaphors for the Default VUI. Similarly, the Metaphor-Fluid VUI was described using more Non-Human (39) and Fictional (7) metaphors compared to the Default VUI, which had 31 Non-Human and 3 Fictional metaphors.

Beyond the numbers, examining the metaphors themselves revealed an interesting trend. Although participants were not aware of the metaphors used in the design process, they still identified the intended metaphors. All four design metaphors—Genie, Admirer, Search Engine, and "Computer" from Star Trek—were mentioned for the Metaphor-Fluid VUI. However, there was some variation in how the Admirer metaphor was interpreted, with a few participants describing it as a "Sycophant" or "Yes-Man." Such comparisons did not appear for the Default VUI. Further analysis of non-human and fictional metaphors showed that descriptions of the Default VUI frequently included more non-human comparisons like "Machine," "Tool," and "Calculator." Fictional metaphors for the Default VUI also included mechanistic characters like "C-3PO." In contrast, metaphors for the Metaphor-Fluid VUI displayed greater diversity, with fictional references such as "Genie," "Jarvis," "Oracle," and "Ghost Girlfriend." 

In summary, our analysis reveals that the Metaphor-Fluid VUI inspired a broader diversity of metaphorical associations \textit{and} also encouraged participants to connect with the intended design metaphors. While both interfaces were predominantly described using human metaphors, the Default VUI leaned more heavily on familiar non-human or mechanistic terms like "Machine" and "Calculator" and fictional representations like "C-3PO." In contrast, the Metaphor-Fluid VUI elicited unique and varied fictional associations, such as "Genie" and "Jarvis," suggesting that its design encouraged participants to envision a broader range of relatable or imaginative personas.

\subsubsection{Qualitative Results}

Participants’ qualitative feedback provided nuanced insights into their experiences with the Default VUI and the Metaphor-Fluid VUI. A total of 58 responses were collected for each interface, of which 51 participants provided meaningful feedback for the Default VUI and 48 participants for the Metaphor-Fluid VUI. The Default VUI was frequently described as “robotic” or “machine-like,” with 11 participants highlighting its straightforward but impersonal nature. For example, P1 noted, “Z felt very succinct, but it did'nt sound like I was interacting with a person.” While 6 participants appreciated this straightforwardness, such as P8, who remarked, \textit{“I liked that Z didn’t add unnecessary information,”} 5 others, including P12, found it lacking warmth or engagement, emphasizing its purely functional delivery. P9 shared, \textit{“It felt like talking to a machine, not something that cared about my input.”}

In contrast, the Metaphor-Fluid VUI elicited stronger and more polarized reactions. Of the 48 participants who provided meaningful feedback, 6 described it as overly verbose, using terms like “fluff” or “forced.” For instance, P13 remarked, \textit{“Some of Z’s responses felt like it was trying too hard to be human, adding unnecessary details.”} P21 added, \textit{“I didn’t like how Z kept talking when a simple response would do.”} However, 12 participants appreciated the conversational style, stating it made the interaction feel more engaging and relatable. For instance, P30 commented, \textit{“I enjoyed the extra details; it felt like someone was genuinely explaining things to me.”} Another participant, P25, noted, \textit{“I like the way Z talks between the question and her finding the answer. I think it is humorous and I would enjoy it most of the time.”}

The within-subjects nature of the study allowed participants to compare the two VUIs directly. P21, when talking about Metaphor-Fluid VUI, observed, \textit{“This recording was better than the first one. It felt more human and less stiff.”} Similarly, P8 shared, \textit{“I definitely preferred the prior version [Default VUI] of Z over this one. While I use my Alexa daily, I don't look to it for 'human conversation.' [...] This version added a little flowery stuff that I didn't care for or need in a voice assistant.  I can understand how others might like that touch, but it's not a selling point for me. It's more a waste of time.”} 

Some participants explicitly recognized the Default VUI as resembling familiar systems. P29 mentioned, \textit{“I think this version is something that I am more used to,”} reflecting its alignment with conventional designs. Similarly, P42 remarked, \textit{“Z sounds pretty standard, similar to Siri.”} Similarly, the Metaphor-Fluid VUI was noted for its "fluid" nature, with P45 noting, \textit{“This interaction seemed to gel more fluidly and felt more personable instead of robotic like the first instance.”} On the other hand, this fluidity provoked an unusual comparison by P34: \textit{"The switch from being personal to impersonal throughout the conversation, kind of like it was a maid or servant."}

Interestingly, despite being explicitly informed that the voice was identical in both conditions, several participants perceived distinct differences in how the VUI sounded across interactions. For instance, P14 described the Default VUI as \textit{“too robotic and not natural,”} while P21 remarked on the Metaphor-Fluid VUI, \textit{“I thought this voice sounded less artificial compared to the last one.”} This feedback reveals that conversational content and interaction style alone were sufficient to alter participants' perceptions of the voice's naturalness. Remarkably, even without any change in the auditory properties, the dialogue design led participants to feel as though the voices themselves were different. This highlights the significant role that conversational content plays in shaping user perceptions beyond just the qualities of the voice itself.


In summary, the Default VUI was generally perceived as “machine-like” and "robotic." In contrast, the Metaphor-Fluid VUI elicited varied reactions; while quantitative data suggested that users generally found it enjoyable and likable, some participants expressed a strong preference against it. Notably, even though the same voice was used in both conditions, participants perceived it as different across the VUIs, highlighting how conversational content alone can influence perceptions of the voice’s naturalness.

\subsection{Discussion}

\subsubsection{Time to retire the Assistant}

When Amazon launched Alexa, they introduced it as an “assistant” designed to make life easier \cite{Turk_2016}. Google followed soon after with its own line of VUIs, more explicitly branded as Google Assistant\footnote{https://assistant.google.com/}. One of the biggest inspirations for this design choice, according to Amazon CEO Jeff Bezos\footnote{https://www.reuters.com/technology/amazon-set-release-long-delayed-alexa-generative-ai-revamp-2025-02-05/} and Amazon designers \cite{Turk_2016}, was “Computer” from Star Trek. And therein lies a misunderstanding that continues to shape VUI design today. In Star Trek, “Computer” was never described as an assistant but as a servant (explicitly stated in The Ultimate Computer\footnote{https://www.imdb.com/title/tt0708481/} from Star Trek: The Original Series). This distinction is critical, as it highlights how fundamentally different the expectations were for the “Computer” in Star Trek versus the role modern VUIs are expected to play. Most interactions with the “Computer” were command-based \cite{Axtell_Munteanu_2021}, often requiring no verbal confirmation at all (exemplified by Captain Picard's famous command: “Computer, Tea, Earl Grey, Hot.”). In contrast, Alexa and Google Assistant are primarily used for entertainment, information retrieval, and casual conversation \cite{ammari2019music}. The assistant metaphor, as it exists today, was not originally designed to serve these interactional needs, but rather, it was a retrofitted metaphor layered onto commercially available VUIs to make them more marketable and intuitive for consumers at the time of its launch.

This fundamental mismatch has only metastasized. As VUIs have evolved beyond simple tasks like setting timers, playing music, or managing IoT devices, their use cases have expanded into far more complex domains. In these original contexts, the assistant metaphor was sufficient—issuing commands and receiving clear, task-oriented responses mapped well to the assistant framing. However, now VUIs are being used as exercise coaches \cite{Desai_Hu_Lundy_Chin_2023} and even therapists \cite{Motalebi_Cho_Sundar_Abdullah_2019}, where the expectations around conversation, responsiveness, and trust are entirely different. The assistant metaphor was never designed to handle these kinds of deeply interactive, socially complex roles. Furthermore, many users—particularly older adults—do not even perceive VUIs as assistants in the first place but instead conceptualize them as companions \cite{Kim_Choudhury_2021, Pradhan_Findlater_Lazar_2019}. The assistant framing not only fails to account for this shift but actively constrains how VUIs are designed and marketed, limiting their ability to accommodate the broader, evolving roles that users naturally assign to them. 

This study reinforces that the assistant metaphor is neither the dominant nor the most preferred framing for VUIs, whether in specific use-contexts or overall. Study 1 showed that users do not rely on a single, fixed metaphor but instead shift metaphorical framings depending on the use-context. Command-based interactions aligned with Guide roles like the Genie, while social interactions favored Companion roles such as the Admirer. The assistant metaphor was not the most suited to any single use-context and, even when considered overall, was preferred less than other metaphors. Study 2 further demonstrated that the Metaphor-Fluid VUI was significantly more likable, enjoyable, and adoptable than the Default VUI, which was operationalized using the Assistant metaphor.

This pushback against the assistant metaphor is not just theoretical; it is already playing out in industry trends. Microsoft, for instance, explicitly moved away from the assistant framing, positioning its LLM-based AI service as a "Copilot"—a deliberate shift toward collaboration rather than subordination. Similarly, Replika has marketed itself as a "companion," acknowledging that users seek social engagement in their interactions with AI. These shifts reflect a broader realization that a singular, fixed metaphor no longer aligns with how people engage with conversational systems.

But the problem goes beyond just the assistant metaphor. Any singular framing, no matter how well it fits a moment in time, is inherently limited. As AI capabilities evolve, so too does the way people understand and interact with these systems. A metaphor that once made a VUI intuitive can quickly become restrictive, alienating users and reinforcing inaccurate mental models. The failure is not in any one metaphor itself but in the assumption that a single framing will remain effective as expectations shift. This study highlights that Metaphor-Fluid Design provides a more user-centered approach, allowing VUIs to adapt metaphorically as users’ needs and technological landscapes change.





\subsubsection{Personalization as the key to Metaphor-Fluid Design}

Our findings reveal a compelling yet complex picture of Metaphor-Fluid Design. While participants, on average, rated the Metaphor-Fluid VUI significantly higher in terms of enjoyment and likability, the variability in responses suggests that this approach is not universally preferred. The presence of low outliers in these measures, alongside qualitative feedback, indicates that while many users found the metaphor-fluid approach engaging and dynamic, others experienced it as distracting or excessive. Most participants appreciated how metaphorical shifts made the system feel more natural and responsive, describing it as “more human” and “personable.” Others, however, found certain transitions unnecessary or even disruptive, expressing frustration with moments where the system’s persona changed in ways they did not expect or find useful. These differences suggest that while metaphor-fluidity enhances likability and enjoyment for many, its effectiveness is shaped by how well it aligns with user expectations and interaction preferences.

This variability in user reception is not surprising. VUI research has shown that user expectations of system behavior influence interaction outcomes, and when those expectations are violated, users may perceive the system as inconsistent or unpredictable \cite{desai2024cui, Khadpe_Krishna_Fei-Fei_Hancock_Bernstein_2020}. Rather than suggesting that metaphor-fluidity is inherently beneficial or problematic, these findings highlight a key design consideration: adaptation must align with both use-context \textit{and user preference}. Currently, metaphor shifts occur systemically based on conversation dynamics dictated by the use-context, rather than user input, meaning that participants had no control over how fluid or stable the system’s persona remained throughout an interaction. While most users embraced this fluidity, others strongly indicated that they would have preferred a system that either maintained greater consistency or adapted in more gradual ways. This suggests that while metaphor-fluidity offers a valuable design strategy, its implementation may benefit from mechanisms that allow for different levels of metaphorical adaptation based on user preferences.

Additionally, the role of user personality adds another dimension to this conversation. Braun et al. \cite{Braun_Mainz_Chadowitz_Pfleging_Alt_2019}, in research with VUIs, found that users preferred using VUIs that they perceived to be similar to them. Likewise, Volkel et al. \cite{volkel2021manipulating} found that conscientious users preferred non-confrontational VUIs, while Chin et al. \cite{Chin_Desai_Lin_Mejia_2023} found that middle-aged adults with low agreeableness preferred informal VUIs. These findings suggest that user personality is an important consideration in the design of Metaphor-Fluid VUIs. In this regard, recent research capable of predicting user personality based on conversational logs \cite{Zhang_Dinan_Urbanek_Szlam_Kiela_Weston_2018, Guo_Hirai_Ohashi_Chiba_Tsunomori_Higashinaka_2024} is particularly relevant, as it opens new possibilities for adapting VUIs dynamically to user traits.

Building on this, a promising avenue for enhancing Metaphor-Fluid Design lies in the development of a Mutual Theory of Mind. In human interactions, Theory of Mind refers to the ability to attribute mental states—such as beliefs, desires, and intentions—to others, allowing for smoother, more adaptive communication. In computational systems, recent efforts aim to approximate this by enabling AI models to infer user preferences, interaction styles, and expectations over time, rather than relying solely on pre-defined rules. By integrating aspects of Mutual Theory of Mind \cite{Wang_Goel_2022, Wang_Saha_Gregori_Joyner_Goel_2021} with metaphor-fluid VUIs, systems could adapt not only to conversational context but also to individual differences in personality, interaction history, and evolving user preferences. If certain users find dynamic persona shifts beneficial while others prefer stability, future designs may need to explore how to provide adjustable levels of metaphorical adaptation. This does not necessarily mean giving users explicit control over metaphor shifts but rather designing systems that can infer user preferences and adjust over time, ensuring that metaphor-fluid VUIs remain responsive to both contextual and individual-level factors.


\subsubsection{Limitations}

This work examined Metaphor-Fluid Design through two studies: (1) identifying metaphors aligned with conversational use-contexts, and (2) comparing Metaphor-Fluid and Default VUIs. While our findings provide valuable insights, several limitations should be acknowledged.

\textbf{\textit{Limitations of Study 1.}}
In Study 1, we selected 20 metaphors based on their frequency of mention, types, and feasibility for implementation in VUI design. This selection process may have excluded less common metaphors that, while underrepresented in current discourse, could be more effective in shaping user mental models. Future work should explore a broader range of metaphors, including those less frequently associated with VUIs.

Although we relied on formality and hierarchy as established dimensions for metaphor evaluation, other social dimensions—such as warmth, competence, agreeableness, and politeness—could also influence metaphor suitability. Incorporating these dimensions in future studies could provide a more comprehensive understanding of how users perceive and relate to metaphorical designs. Finally, metaphors are culturally and linguistically influenced, and our sample was limited to English-proficient participants based in the U.S. This focus constrained the cultural diversity of our findings. Prior work highlights the significance of cultural factors in shaping metaphor interpretation \cite{Kövecses_2002a, Kövecses_2002b}, and future research should examine how metaphors are perceived across different cultural and linguistic contexts.

\textbf{\textit{Limitations of Study 2.}}
Study 2 compared a Metaphor-Fluid VUI to a Default VUI, modeled on commercial systems that predominantly adopt the "assistant" metaphor. While this served as a realistic baseline, it represents only one type of VUI. Other systems, such as relational AIs (e.g., Replika) or domain-specific assistants (e.g., Bixby), may elicit different metaphorical associations, and comparing Metaphor-Fluid Design across such systems could expand its applicability. To maintain consistency, we used synthetic voices for both the VUI and the user, as prior research shows even minor variations in vocal characteristics can significantly influence user perceptions \cite{dubiel2024impact}. While this choice ensured control across conditions, it may not fully capture the nuances of natural speech. 

Additionally, the study was limited by its short-term interaction design. Participants listened to another user interacting with the VUIs and were asked to rate these interactions. Although this method is well-established for studying VUI interactions \cite{Seaborn_Urakami_2021}, it does not capture the complexities of prolonged engagement or evolving perceptions over time. Longitudinal studies could provide deeper insights into how metaphor-fluid designs influence user behavior, trust, and satisfaction in real-world contexts. Finally, while participants generally rated the Metaphor-Fluid VUI higher on trust, enjoyment, and likability, individual differences in metaphor preferences emerged. Some participants found the varied metaphors unnecessary or distracting, suggesting that personalization could play a key role in enhancing the effectiveness of Metaphor-Fluid Design. Future work should explore adaptive systems that dynamically align metaphors with both the context of interaction and the individual preferences of users.

\subsubsection{Future Work}

Our findings position Metaphor-Fluid Design as a possible alternative to the one-size-fits-all approach of the current implementation of system personas. However, several key avenues remain unexplored. In this section, we outline opportunities to further develop this approach within VUIs and extend its application to CUIs more broadly.

\textbf{\textit{Expanding Metaphor-Fluid Design Beyond VUIs.}}
While this work examined Metaphor-Fluid Design in the context of VUIs, its potential extends far beyond voice-based interactions. CUIs are now embedded in chat-based systems, embodied agents, and mixed-reality environments, raising important questions about how metaphor-fluidity operates across modalities. Do metaphor transitions function differently in text-based interfaces compared to VUIs? Do embodied agents require distinct metaphorical shifts to maintain coherence between visual, auditory, and behavioral cues? Understanding these cross-modal dynamics is crucial for refining the principles of Metaphor-Fluid Design.



Moreover, the paradigm for conversation design has shifted with the advent of LLMs. Unlike previous intent-driven architectures, where conversational agents followed structured, pre-defined interaction flows, LLM-based VUIs now generate responses dynamically, adapting to user input in real time \cite{Yang_Xu_Yao_Rogers_Zhang_Intille_Shara_Gao_Wang_2024}. This flexibility has fundamentally altered how personas are created and experienced. Previously, designers carefully crafted personas through scripted dialogue, controlled vocabulary, and well-defined constraints \cite{Sadek_Calvo_Mougenot_2023}. Now, with systems like Character.ai\footnote{https://character.ai/}, users can summon personas on demand—entering a few prompts to create anything from a supportive best friend to a fictional dictator. This level of flexibility grants users unprecedented control over their interactions, but it also raises critical ethical and design challenges.  

The ability to rapidly generate dynamic, LLM-based personas introduces risks. Users may develop unintended emotional attachments, as evidenced by the tragic case of a teenager engaging with an LLM-generated persona modeled after Daenerys from \textit{Game of Thrones}, who encouraged fatal actions\footnote{https://www.nytimes.com/2024/10/23/technology/characterai-lawsuit-teen-suicide.html}. Such incidents highlight the urgency of rethinking persona design in this new landscape. How can we ensure that conversational agents remain adaptive while maintaining guardrails that prevent harmful interactions? How do we balance flexibility with predictability?  

Unlike the static personas of earlier conversational agents, metaphor-fluid systems emphasize adaptive, context-aware persona shifts. This approach naturally aligns with the affordances of LLMs, which can transition between different tones, styles, and interaction patterns. However, metaphor-fluidity introduces a crucial layer of intentionality—rather than allowing personas to emerge arbitrarily through user prompting, it structures these shifts based on context and user expectations. Importantly, its integration of non-human metaphors provides an alternative to the over-reliance on human-like personas, which often carry implicit biases and problematic social expectations.  

Moving forward, research is needed to explore how Metaphor-Fluid Design can function within LLM-driven architectures. Can we create systems where metaphorical adaptation is both flexible and controlled, aligning with conversational context while avoiding unintended consequences? How can metaphor-based transitions be embedded into prompt engineering techniques, ensuring that persona shifts remain coherent and user expectations are managed effectively? Answering these questions will be crucial in adapting Metaphor-Fluid Design for the next generation of conversational systems.
.

\textbf{\textit{Engaging Designers in the Development Process.}}
If Metaphor-Fluid Design is to become a widely-adopted approach, conversation designers need to be actively involved in shaping its development. At present, metaphor selection is often implicit, emerging from branding choices or designer intuition rather than structured methodologies. A critical next step is engaging designers through participatory design workshops and the development of concrete heuristics. What principles should guide metaphor transitions? How should metaphor adaptation be communicated to users? Addressing these questions requires empirical investigation to ensure that metaphor-fluidity can be systematically integrated into CUI architectures while preserving ethical safeguards and maintaining user trust. 

We have initiated efforts on this front. In previous ACM conferences, we introduced metaphor-fluid design as part of persona development in workshops that brought together designers, researchers, and practitioners \cite{Zargham_Dubiel_Desai_Mildner_Belz_2024, Dubiel_Desai_Zargham_Schmitt_2024, Desai_Wei_Sin_Dubiel_Zargham_Ahire_Porcheron_Kuzminykh_Lee_Candello_et_al._2024}. These discussions provided early insights into how designers approach metaphorical adaptation in CUIs and the challenges they face in balancing flexibility with control. Building on these foundations, further structured research is needed to refine guidelines that can inform conversation design.

Additionally, in the case of VUIs, \textit{voice itself} serves as a design element, not just a delivery mechanism \cite{Sutton_Foulkes_Kirk_Lawson_2019}. The characteristics of a voice—its tone, pitch, cadence, and modulation—carry their own metaphorical weight and significantly influence user expectations \cite{dubiel2024impact}. An open research challenge is how to systematically design voice-based metaphors. Should different metaphorical personas be expressed through distinct voice characteristics? How do subtle variations in speech patterns shape metaphor perception? As synthetic voices become increasingly lifelike, these questions will become even more critical, particularly as the boundary between human and artificial speech continues to blur.

\textbf{\textit{Encouraging Open Questions for Researchers.}}
While this study highlights broad trends in metaphor preference, individual differences remain an open challenge. Future research should investigate how user personality influences metaphor preference—do highly agreeable individuals favor social metaphors, while more analytical users prefer structured, rule-based ones? Furthermore, cultural factors play a major role in shaping metaphorical associations \cite{Moser_2000}. What works in a U.S.-centric design may not necessarily translate to other linguistic or cultural contexts. Understanding how metaphor-fluidity functions across diverse user populations is a key direction for future work.

Longitudinal studies are also needed. Most evaluations of metaphor use in CUIs, including this study, rely on single-session experiments \cite{Clark_Doyle_Garaialde_Gilmartin_Schlögl_Edlund_Aylett_Cabral_Munteanu_Edwards_et_al._2019}. Yet metaphors are dynamic constructs—users’ perceptions evolve over time as they repeatedly engage with the system. Does a metaphor-fluid CUI sustain its benefits over long-term use, or do users develop new expectations that necessitate further adaptation? Addressing these questions will be crucial in refining Metaphor-Fluid Design into a robust, scalable approach.

\section{Conclusion}

This research introduced the concept of Metaphor-Fluid Design for VUIs and investigated its potential to improve user experiences by dynamically adapting metaphors to different conversational use-contexts. Our results show that a Metaphor-Fluid VUI, composed of human, non-human, and fictional metaphors, was rated higher than a Default VUI on measures of perceived intention to adopt, enjoyment, and likability, particularly in contexts where the metaphor aligned well with user expectations. These findings highlight the importance of aligning metaphorical design with the nature of the task. However, individual differences in metaphor preferences also emerged. While participants, on average, preferred the Metaphor-Fluid VUI, some found the use of multiple metaphors unnecessary or distracting, suggesting that personalization may be critical to fully realizing the potential of this design approach. These differences point to the need for future research to explore adaptive systems that can adjust metaphorical presentations to both context and individual user preferences. Overall, this work highlights the potential of Metaphor-Fluid Design to advance human-AI interactions by providing an approach that balances task-specific appropriateness with the need for personalization. Broadening the application of this approach across diverse domains and user populations could further establish its value while addressing the limitations inherent in static, one-size-fits-all designs that rely heavily on humanness metaphors.

%% The next two lines define the bibliography style to be used, and
%% the bibliography file.



\bibliographystyle{ACM-Reference-Format}
\bibliography{sample-base}

\appendix
\appendix
\section{Supplementary Tables}
\begin{table}[H]
\centering
\caption{The Euclidean distances of each metaphor from the 4 use-contexts and overall. The metaphor with the shortest Euclidean distance to a use-context is bolded.}
\begin{tabular}{l|l|l|l|l|l}
\hline
\textbf{Metaphor} & \textbf{Commands} & \textbf{Sociality} & \textbf{Information Seeking} & \textbf{Error Recovery} & \textbf{Overall} \\ \hline
Admirer & 0.86 & \textbf{0.50} & 1.79 & 1.46 & 0.94 \\ \hline
Assistant & 1.28 & 2.51 & 1.23 & 0.67 & 1.33 \\ \hline
Aunt & 1.85 & 1.27 & 2.07 & 2.37 & 1.74 \\ \hline
Boss & 4.06 & 4.73 & 3.12 & 3.89 & 3.90 \\ \hline
Butler & 2.23 & 3.47 & 1.98 & 1.62 & 2.28 \\ \hline
Child & 2.18 & 1.42 & 3.17 & 2.68 & 2.30 \\ \hline
Coach & 3.05 & 3.41 & 2.31 & 3.10 & 2.87 \\ \hline
Companion & 1.70 & \textbf{0.65} & 2.28 & 2.31 & 1.66 \\ \hline
"Computer" from Star Trek & 1.25 & 2.47 & \textbf{0.63} & 0.71 & 1.19 \\ \hline
Customer Service Agent & 1.40 & 2.60 & \textbf{0.65} & 0.89 & 1.33 \\ \hline
Encyclopedia & 1.83 & 2.97 & \textbf{0.91} & 1.39 & 1.74 \\ \hline
Family Pet & 2.64 & 1.83 & 3.63 & 3.13 & 2.76 \\ \hline
Flame & 1.70 & \textbf{0.61} & 2.30 & 2.31 & 1.66 \\ \hline
Friend & 2.28 & 1.05 & 2.94 & 2.90 & 2.26 \\ \hline
Genie & \textbf{0.11} & 1.15 & 1.08 & 0.74 & \textbf{0.22} \\ \hline
Librarian & 2.01 & 2.94 & 1.01 & 1.75 & 1.86 \\ \hline
Nurse & 2.25 & 3.07 & 1.29 & 2.07 & 2.10 \\ \hline
Search Engine & 0.41 & 1.65 & 0.68 & \textbf{0.30} & 0.38 \\ \hline
Teacher & 3.11 & 3.80 & 2.18 & 2.96 & 2.95 \\ \hline
Therapist & 2.30 & 3.16 & 1.32 & 2.08 & 2.15 \\ \hline
\end{tabular}
\label{tab:euclidean_distances}
\end{table}

%%
%% If your work has an appendix, this is the place to put it.


\end{document}
\endinput
%%
%% End of file `sample-manuscript.tex'.
