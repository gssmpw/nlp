\section{Related Work}
VUIs enable users to interact naturally through spoken language and are embedded in a range of consumer technologies, including smart speakers, smartphones, headphones, and even household appliances. Popular examples include Amazon Alexa, Apple Siri, and Google Assistant, which have become integral to everyday interactions. However, the term VUI is often used interchangeably with other labels, such as Intelligent Personal Assistants, Digital Assistants, Conversational Agents (CAs), Conversational User Interfaces (CUIs), and Speech Interfaces ____. While these terms share similarities, they differ in scope. For instance, CAs and CUIs encompass both voice- and text-based systems, including chatbots, whereas VUIs specifically refer to interfaces that rely on voice as the primary input and output modality.

Given this terminological overlap, it is essential to clarify the scope of this research. This article focuses explicitly on disembodied, software-based VUIs—systems that process and generate spoken language without physical embodiments, such as robots or avatars. However, given the broader relevance of CUIs in conversational AI research, we reference CUIs where applicable to contextualize and discuss our findings within the wider discourse on human-AI interaction.

\subsection{Metaphors—A Brief Primer}



Metaphors are more than just literary flourishes; they are fundamental tools for understanding and interpreting the world ____. In linguistics and cognitive science, metaphors have been shown to shape not only how we communicate but also how we think ____. According to Lakoff and Johnson's theory of conceptual metaphors, our understanding of abstract concepts is often structured by metaphors derived from concrete experiences ____. For example, when we say "time is money," we are using a metaphor to conceptualize time as a valuable resource, which affects how we perceive and manage it ____.

Metaphors are powerful because they influence cognition and behavior, often in subtle ways. They help bridge the gap between the familiar and the unfamiliar, making complex ideas more accessible ____. By providing a conceptual structure, metaphors allow individuals to grasp abstract concepts by relating them to more concrete experiences. In cognitive psychology, metaphors have been found to shape reasoning, decision-making, and problem-solving processes, thereby playing an integral role in how knowledge is constructed and applied ____. Through metaphors, we can connect domains that are seemingly unrelated, fostering creativity and enabling us to generate novel ideas ____. In science and technology, metaphors often serve as foundational frameworks that influence both how researchers theorize and how laypeople understand complex phenomena ____. Alan Turing, for example, used a metaphor of a human computer following a set of instructions to describe his model for computation, which became the basis for modern computer science ____. Similarly, metaphors like "genetic code" or "cellular machinery" have shaped how scientists and the public conceptualize biological processes, providing familiar reference points that help make sense of otherwise abstract concepts ____. Metaphors thus provide a cognitive shortcut that allows individuals to comprehend and manipulate ideas beyond the immediate grasp of direct experience ____.

Moreover, metaphors are deeply embedded in everyday language, often so much so that we are unaware of their influence. Phrases such as "grasping an idea" or "feeling down" illustrate how metaphorical thinking is woven into our conceptual system ____.

\subsection{Metaphors in HCI}

Metaphors in HCI have profoundly shaped how users conceptualize and interact with technology. By leveraging familiar concepts (e.g., files, folders), metaphors bridge the gap between users' prior knowledge and new digital environments, simplifying interactions and reducing cognitive load ____. The desktop metaphor stands out as a pivotal example in GUIs, drawing on the physical analogy of an office desk to provide users with a coherent mental model for navigating digital spaces ____. This foundational desktop metaphor is supplemented with secondary metaphors like folders, files, bins, scrolling, and icons and translates abstract computational processes into familiar actions, such as "dragging a file to the trash" to discard a document ____.

The impact of the desktop metaphor extends far beyond mere visual representation. It has played a crucial role in making computers more approachable and familiar, significantly contributing to their widespread adoption ____. By aligning user expectations with system capabilities, the desktop metaphor has fostered increased usability and user satisfaction. Furthermore, it has provided a common language for developers and users to communicate about new technologies, facilitating the ongoing evolution of human-computer interfaces. The desktop metaphor's influence on early computing systems was so profound that it shaped users' mental models for decades. For a detailed exploration of its history and significance, Alan Blackwell's work offers an in-depth analysis that underscores the metaphor's lasting impact on HCI design principles ____.




The application of metaphors in interface design extends beyond mere aesthetics, serving dual roles of familiarization and transportation, as outlined by Paul Heckel in his work on friendly software design____. Familiarization employs metaphors to introduce recognizable concepts, facilitating user comprehension and mental model construction. Transportation, on the other hand, utilizes these mental models to create immersive experiences. In the context of VUIs, the concept of humanness frequently serves as the primary metaphor, acting as a conduit for familiarization and transportation effects.



\subsection{Metaphors in VUIs}

The study of metaphors in VUIs is relatively new compared to their rich history in HCI. The foundational "humanness" metaphor serves as the principal metaphor in VUI design, where VUIs are often deployed in human-like social roles for VUIs, ranging from educators and coaches to storytellers and even therapists____. This approach aims to create a sense of familiarity by aligning VUIs with known social roles, thereby making interactions more intuitive and accessible ____. VUIs like Amazon Alexa and Apple Siri are designed to function as digital assistants, drawing on the metaphor of the helpful human assistant to facilitate tasks such as playing music, providing information, managing smart home devices, and offering social companionship ____. These devices leverage the designer-intended and manufacturer-marketed ____ humanness metaphor to create an experience that feels natural and social, helping to bridge the gap between technology and everyday life ____.

In practice, metaphors are operationalized in VUI design through system personas and form a crucial part of Conversation Design workflows ____. Google's Conversational UX guidelines\footnote{https://developers.google.com/assistant/conversation-design/create-a-persona}, for instance, emphasize role production by designing agents that mimic real-world figures, such as bank tellers or exercise coaches, situating these interfaces in familiar contexts ____. Designers often default to human roles because of the innate tendency to anthropomorphize technology. Anthropomorphism—the attribution of human traits to non-human entities—is central to VUI design, as it offers users a relatable interaction model ____.

Epley et al.'s influential three-factor theory of anthropomorphism explains why and when humans attribute human-like characteristics to non-human agents. The factors—elicited agent knowledge (drawing on knowledge of human behavior), effectance motivation (the need to predict and understand the agent’s actions), and sociality motivation (seeking social connection)—are key to understanding why users often interact socially with VUIs ____. The "humanness" metaphor leverages users' natural inclination to perceive technology through a social lens, making VUIs feel more engaging and personable. As a result, these systems incorporate anthropomorphic cues like natural language processing and human-like voice modulation to reinforce the perception of social presence and competence ____. 


The choice of metaphor in VUI design significantly impacts user perception. Studies have shown that metaphors signaling high competence initially attract users, while those suggesting lower competence tend to be favored in the long term____. Furthermore, the formality of the VUI's conversational style influences users' metaphorical interpretations, with more formal styles evoking professional metaphors and informal styles eliciting personal ones____. Research has also indicated a correlation between metaphor-projected personalities and user personalities, with users preferring metaphors that align with their own traits____. Despite designers' efforts to present clear mental models through personas, users often develop their own conceptual understandings of VUIs through folk theories____. These theories, often expressed as metaphors, attempt to explain VUI functionality or rationalize system errors____. The disparity between user expectations based on provided metaphors and actual VUI performance can lead to usage issues, as described by Norman's "gulf of execution" concept____.


Broadly, there are three significant challenges associated with the current implementation of metaphors in VUI design:
\begin{enumerate}

\item \textbf{Ethical Issues}: Designing VUIs based on roles like "assistant" or "servant" can inadvertently reinforce outdated stereotypes related to gender and power dynamics ____. These metaphors may implicitly communicate roles of subservience, which can perpetuate harmful social norms, especially when the personas are given specific voices that align with stereotypical attributes. For example, the choice of a female voice for assistant roles often reinforces traditional gender stereotypes about caregiving and subservience, which has raised ethical concerns regarding the impact of such designs on societal attitudes ____.

\item \textbf{Simplistic Representations}: While users develop complex and context-dependent mental models of VUIs that change with the type of interaction or domain of use, designers often rely on a single overarching metaphor to represent these systems. This singular metaphorical framing may fail to capture the nuanced ways in which users interact with VUIs across different contexts ____. For example, in a user study involving older adults interacting with a VUI designed as an "exercise coach," participants expressed a preference for the VUI to also behave like a "friend" when initiating the session, checking on their progress, and providing motivation ____. The complexity of user expectations and the diverse roles VUIs play are not adequately addressed by a one-size-fits-all metaphor, which can lead to a mismatch between what users expect and what the system delivers, ultimately impacting usability and user satisfaction. By providing only a simplistic metaphor, designers risk oversimplifying the rich, multi-faceted experiences users have with VUIs, thereby limiting the potential for these systems to meet diverse user needs effectively.  Moreover, users' perceptions of VUIs are also highly fluid, shifting depending on the context and specific interactions. Pradhan et al. ____ found that users alternately view VUIs as either a "person" or a "thing," or something in between, depending on the specific task or setting. 

\item \textbf{Misaligned Mental Models}:  Misaligned mental models arise when the human-like metaphors underlying VUIs heighten user expectations, leading to either over or under-calibration of trust. Users may overestimate the VUI's abilities, assuming it possesses human-like cognitive and conversational skills, or underestimate its utility due to skepticism about its capabilities. ____ This misalignment of expectations impacts trust calibration—users may become overly reliant on the VUI, expecting it to perform beyond its technical limitations, or conversely, they may dismiss its utility prematurely ____. These fluctuating expectations can result in inconsistent or infrequent use ____ and, eventually, complete abandonment of the technology ____. The inconsistency in how users perceive the capabilities of VUIs, especially when the system fails to meet expectations in specific scenarios, resulting in conversational errors, further exacerbates the misalignment and undermines user trust ____.

\end{enumerate}

The evolving landscape of VUI design reflects a broader understanding of the complexities involved in metaphor selection. While a single human metaphor offers an accessible and engaging way for users to interact with technology, its limitations require thoughtful application to prevent unintended ethical and usability issues.

\subsection{Mitigating Challenges in VUI Design}
\label{sec:mitigate}

As the VUI design space evolves, various approaches have been proposed to address the challenges surrounding VUI persona design. One promising approach involves developing VUIs in non-human roles. Desai and Twidale ____, through a literature review and user study, identified how metaphors are used by users, designers, manufacturers, and researchers in relation to VUIs. Based on their analysis, they developed a framework for metaphor contextualization in VUIs, categorizing metaphor use along four dimensions: type (human, non-human, fictional), mentioned by (users, designers, manufacturers, researchers), the knowledge it draws upon, and the motivation behind its use. Their findings revealed that while human metaphors dominate VUI design and research, non-human and fictional metaphors were also popular, especially among users for sense-making. This contrast highlights a dichotomy between how designers and researchers present VUIs and how users interpret them. This observation points to the need for a "pragmatic approach" ____ to systematically explore the design space between system capabilities and user expectations.

In the broader CUI literature, there have been recent notable advancements in exploring the feasibility of designing conversational agents with non-human roles. Jung et al. ____, for instance, used the Great Chain of Being (GCOB) framework\footnote{A theoretical framework that organizes the world into a hierarchical structure using metaphors of God, humans, animals, plants, and inanimate objects.} to investigate how non-human metaphors (such as God, Animal, Plant, and Book) compared to human metaphors influence user perceptions of chatbots. Their results showed that the human metaphor did not significantly alter chatbot evaluations compared to non-human metaphors. Similarly, in a related study involving VUIs, researchers examined human versus non-human metaphors in the health and finance domains. They found that in the health domain, users preferred the human metaphor (doctor) in terms of likability and enjoyability, while in the finance domain, no notable differences were observed between the human (financial advisor) and non-human (calculator) metaphors. Additionally, the choice of metaphor had no significant impact on users’ intentions to adopt the VUIs.

The use of fictional metaphors to inspire VUI design has long been of interest to HCI researchers. One of the most notable examples is drawing from the "Computer" in Star Trek to inform VUI design principles ____. In a relevant study, Axtell \& Munteanu ____ conducted a conversation analysis comparing the dialogues of the "Computer" in \textit{Star Trek: The Next Generation} with interactions on Amazon Alexa. Their findings shed light on how VUIs could be designed to be more functional. Additionally, fictional metaphors can inspire a negative design heuristic, illustrating how not to design VUIs by highlighting ineffective or undesirable interaction patterns. This includes avoiding the dystopian, pervasive feeling of a "Big Brother" presence from \textit{1984} or the manipulative behavior exhibited by the Bene Gesserit Sisterhood through "The Voice" in \textit{Dune} ____. Fictional metaphors can also inspire a sense of enchantment in VUI design, similar to how AI technologies often evoke a magical experience. This desired "AI magic," where users perceive advanced technology as almost supernatural, can be paralleled in VUIs by employing fictional metaphors that create a similar sense of wonder and fluid interaction. For instance, magic metaphors like using a "magic wand" in GUIs to represent the undo command illustrate how these enchanting elements resonate with users ____, much like how wake-words in VUIs are sometimes compared to casting a spell ____.

Moreover, the issue of misaligned mental models, often described as the gulf between VUI capabilities and user expectations ____, remains a key topic of ongoing research in the HCI community. One of the most pertinent approaches to studying this gulf is through the "Mutual Theory of Mind" (MToM) Framework ____. This framework emphasizes that constructing a CUI’s understanding of how the user perceives the system—through analyzing communication cues—can be crucial in bridging this gap and aligning CUIs more effectively with users' mental models. These communication cues include not only verbal signals such as language use and tone ____ but also non-verbal behaviors, like gestures, facial expressions, or pauses during interaction ____. In this paper, we take this idea further and propose designing conversational interfaces that adapt their presentational metaphor based on conversational contexts to ensure that the interface is aligned with the users' mental models.