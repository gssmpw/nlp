\section*{Conclusions}
To improve the performance of LLMs in length-controllable text generation, we conduct a bottom-up error analysis of relevant sub-abilities. The results reveal that deficiencies in identifying, counting, and aligning are key limitations. To fill this gap, we propose \textsc{MarkerGen}, which leverages external tools to compensate for fundamental deficiencies. Additionally, it introduces Decaying Interval Marker Insertion Strategy to facilitate explicit length modeling and employs Three-Stage Decoupled Generation mechanism to balance semantic coherence and length control. Comprehensive experiments demonstrate the strong generalizability and effectiveness of \textsc{MarkerGen} in enhancing length control and preserving semantic integrity.
% 针对当前LLM在length-controllable text generation任务上表现不佳的问题,我们自底向上地对LLM的相关子能力进行了误差分析。结果表明Identifying,Counting,Aligning 能力的欠缺是主要因素。为此,我们提出\textsc{MarkerGen}方法。其通过外部工具引入解决了fundamental能力的不足,introduce Decaying Interval Marker Insertion Strategy 帮助模型进行显式长度建模,通过Three-Stage Decoupled Generation平衡了语义和长度建模。Comprehensive的实验验证了\textsc{MarkerGen}在增强长度控制和语义完整性上的有效性,以及强大的泛化性。
\section*{Limitations} 
% 我们聚焦于LCTG场景进行了自底向上的子能力分析,并提出\textsc{MarkerGen}方法实现了良好的LCTG效果。
% 我们认为我们的主要Limitation在于:\textsc{MarkerGen}目前只能用于开源模型中。对于闭源的模型,目前还无法应用。我们将公开我们的代码,从而方便愿意适配\textsc{MarkerGen}来提升LCTG效果的闭源模型厂商受益于我们提出的方法。
In this work, we conduct a bottom-up sub-capability analysis in the LCTG ability and propose the \textsc{MarkerGen} method, achieving strong LCTG performance.
One major limitation of \textsc{MarkerGen} is that it is currently only applicable to open-source models and cannot yet be used with closed-source models. To address this, we will release our code, allowing closed-source model providers interested in adapting \textsc{MarkerGen} to benefit from our method in enhancing LCTG performance.


\section*{Ethics Statement}
All of the datasets used in this study were publicly available, and no annotators were employed for our data collection. We confirm that the datasets we used did not contain any harmful content and was consistent with their intended use (research). We have cited the datasets and relevant works used in this study.