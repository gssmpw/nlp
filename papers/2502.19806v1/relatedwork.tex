\section{Related work on data-based robust control}
In the pursuit of stabilizing unknown dynamical systems, multiple data-driven approaches have recently been proposed. To mention a few, building on Willems \emph{et al.’s} fundamental lemma \cite{willems2005note}, \cite{guo2021data} introduces a data-driven framework for stabilizing input-affine nonlinear systems with polynomial dynamics, while \cite{strasser2021data} extends this approach to more general systems beyond polynomials. In \cite{de2023learning}, a data-driven scheme is proposed to (approximately) cancel system nonlinearities and achieve stabilization. The work \cite{berberich2020data} introduces a robust data-driven model-predictive control strategy for linear time-invariant systems, while \cite{taylor2021towards} develops a data-driven approach for robust control synthesis of nonlinear systems, explicitly addressing model uncertainty. A data-driven approach for the certified learning of \emph{incremental} ISS controllers for unknown nonlinear polynomial dynamics has recently been studied in~\cite{zaker2024certified}.
 Data-driven stability analyses of unknown systems are further explored in various settings, including switched systems \cite{kenanian2019data}, continuous-time systems \cite{boffi2021learning}, and discrete-time systems \cite{lavaei2022data}. For noisy measured data, \cite{chen2024data} employs overapproximation techniques to characterize the set of polynomial dynamics consistent with the data. This enables the construction of an ISS Lyapunov function and the design of a corresponding ISS control law for unknown nonlinear input-affine systems with polynomial dynamics.

While studies such as \cite{guo2021data, strasser2021data, de2023learning, berberich2020data, taylor2021towards,zaker2024certified,kenanian2019data, boffi2021learning, lavaei2022data, chen2024data} primarily focus on stability analysis and controller synthesis using data, their methodologies are tailored for \emph{monolithic systems} of relatively low dimensions. As a result, they are not directly applicable to large-scale networks due to the challenges posed by \emph{sample complexity}. Recent efforts have sought to extend stability analysis to higher-dimensional systems. In this regard, \cite{sperl2023approximation} explores the use of deep neural networks (NN) to approximate control Lyapunov functions, aiming to mitigate the curse of dimensionality. Leveraging the compositional structure of interconnected nonlinear systems, \cite{liu2024compositionally} introduces a framework for training and validating neural Lyapunov functions for stability analysis. Similarly, \cite{zhang2023compositional} employs a compositional approach to derive neural certificates based on ISS Lyapunov functions and controllers for networked dynamical systems. Despite these advancements, \cite{sperl2023approximation, liu2024compositionally, zhang2023compositional} assume that the system model is available. Additionally, \cite{zhang2023compositional} requires a second step involving Satisfiability Modulo Theories (SMT) solvers, such as Z3 \cite{de2008z3}, dReal \cite{gao_-complete_2012} or MathSat \cite{cimatti_mathsat5_2013}, to verify the correctness of the candidate ISS Lyapunov functions. However, SMT solvers may encounter termination issues~\cite{wongpiromsarn2015automata}, or face scalability challenges depending on the NN size or the system complexity~\cite{abate2022neural}. Then, even if the problem is solved for \emph{high-dimensional} systems using NN, verifying the results in the second step would be intractable using SMT solvers, thus lacking formal guarantees. In contrast, our approach assumes the system model is unknown and directly determines ISS Lyapunov functions with correctness guarantees in a single step.

Finally, \cite{lavaei2023data} proposes a \emph{scenario-based} method \cite{calafiore2006scenario, campi2009scenario} for verifying the stability of interconnected networks. However, this approach requires the collected data to be \textit{independent and identically distributed (i.i.d.)}, meaning only one input-state sample pair can be obtained per trajectory \cite{esfahani2014performance}. As a result, multiple independent trajectories are needed to certify network stability. In contrast, our approach requires only two input-state trajectories per subsystem. Moreover, the method proposed in \cite{lavaei2023data} is limited to nonlinear \emph{homogeneous} systems, whereas our approach accommodates a significantly broader class of nonlinear systems, making it more practical. Additionally, \cite{lavaei2023data} focuses solely on \emph{verification} and cannot be applied when each subsystem is subject to external perturbations. In contrast, our approach effectively handles \emph{controller synthesis} in the presence of disturbances through local ISM controllers.