\section{Discussion}


%  第一段:社会模拟器的进程:1.社会孪生(一一映射);2. 模拟推演/实验--->决策;3. 虚实联动/互动(决策影响实体社会)

% 第二段:社会模拟器改变社会科学研究范式

% 第三段:社会模拟器在政策制定,xxxx,xxxx等方面有更广阔的应用前景

\subsection{Three Levels of Social Simulator}
As an important interdisciplinary research, the development of social simulators can be categorized into three levels. Research at the first level mainly focuses on constructing \textit{social twin systems}, which create a one-to-one mapping of real-world individuals. Such systems are always used to track and detect real-world social behaviors. At the second level, on top of \textit{the mirroring world}, it can predict future changes and support intervention experiments in this society. By comparing the impacts of different intervention measures on the simulated society, the performance of different policies can be evaluated. The third level represents a further breakthrough in blurring the boundaries between the real and virtual worlds, leading to a \textit{coexistent hybrid world} that integrates the simulator with real society. Specifically, the simulated individuals can interact with real-world individuals, and their decisions will influence each other. Furthermore, decisions made based on the social simulator can actually impact real society.

The social generative simulator developed in this paper is a pioneering attempt and exploration of the third, \textit{i.e.}, the highest-level social simulator, and provides effective solutions.
Specifically, the social simulator based on large language model-driven agents demonstrates its advantages in the following aspects. First, through the high-fidelity behavior imitation and generation capabilities of large language model agents, the role-playing simulation of the real individual is achieved. That is, agents built on human-like memory mechanisms can accurately reproduce user states in various aspects such as cognition, social interaction, economy, etc.
Second, this platform enables accurate prediction of individual future behaviors and group evolution trends through precise short-term or long-term simulations. The experiments also validate the effectiveness of evolution under different intervention conditions, demonstrating the value of the platform as an assistance tool for decision-making.
Last, this platform's simulated society coexists with the real physical society. By coupling with a real-time urban simulator, its simulation process can be synchronized with and interact with real society. 
In short, our platform could be considered as the recent advance of social simulators on the highest level.

\subsection{Social Simulator: New Paradigm of Computational Social Science}
Computational social science is an interdisciplinary research area where various computational approaches are used for social sciences. Essentially, the core distinction between computational social science and traditional social science lies in the introduction of diverse computational methods. In other words, these studies attempt to address the challenges of social science research using computational approaches. Generally, related work can be understood from three perspectives. First, as an explanation tool, it involves pattern recognition and mining from data, such as uncovering macro-level patterns and key characteristics in the evolution of social networks. The second category is the predictive paradigm, which involves constructing computational models to forecast future changes in system variables, such as predicting the number of individuals who will forward a specific message in a social network. \rvs{Recently, some researchers have attempted to use LLMs as ``silicon samples'' for social experiments~\cite{ashokkumar2024predicting,argyle2023out,sarstedt2024using,sun2024random,aher2023using,demszky2023using}. Their experiments have demonstrated the capability of LLMs to generate human-like experimental samples in studies from political science~\cite{argyle2023out,ashokkumar2024predicting}, psychology~\cite{demszky2023using,sun2024random}, and behavioral science~\cite{ashokkumar2024predicting}. However, these ``silicon samples'' are primarily limited to basic role-playing configurations, overlooking the effects of psychological processes, complex social behaviors, and societal environments on experimental outcomes. Moreover, these limitations constrain the scope of the experiments, making some complex designs, such as the distribution of universal basic income, infeasible.}

Beyond these two categories, there is a third simulation paradigm of agent-based modeling, which simulates each individual in a bottom-up manner, by constructing rule-based or model-based agents. However, existing computational social science research faces substantial difficulties in simulation (primarily due to the lack of effective methods for achieving precise simulation), and thus, this paradigm has not truly surpassed the first two. Some famous examples of agent-based modeling in social sciences include the Epidemic Spread Models, Schelling's Segregation Model, the Sugarscape model, etc. While these models can, to some extent, reproduce the patterns of change in macro-level variables, they involve significant simplifications, and the realism of individual behaviors within them is very limited. For instance, in the research of economics, although methods with agent-based modeling were proposed early on, it is widely acknowledged by economists that ABM model has not achieved the precision of predictive models. Furthermore, another challenge in computational social science is the difficulty in accessing, intervening in, and controlling the subjects (human individuals) of research. The computational models of computational sociology and field experiments are often isolated; typically, the data is collected first, followed by the design and application of computational models. The human participants are difficult to select, observe in depth, and interact in the long term.

Therefore, the large-scale social simulator proposed in this work represents a breakthrough as a fourth paradigm for computational social science: an agent-based simulation paradigm centered on highly realistic human-like agents. This paradigm supports analysis, prediction, and high-precision bottom-up simulation, allowing for arbitrary selection, intervention, and control of experimental subjects. It facilitates various computational social science research endeavors, including theory validation, pattern discovery, etc.


\subsection{From Policy Making, Risk Control, to Future Human-AI Society}

In the above experiments, we have validated specific application examples of the large-scale social simulator. From a broader perspective, we contend that the utility of such a simulator extends far beyond these implementations, encompassing urgent societal governance challenges, widely debated AI risks, and more forward-looking futuristic societal applications.

\subsubsection{Policy making and social management for a smarter society}

Traditional decision-making processes for society management primarily rely on mining and analyzing historical data to construct computational models for evaluating policy effectiveness. However, this approach fundamentally fails to address the rapidly evolving nature of human societies. The social conditions during policy deployment often diverge significantly from those when models were originally formulated, rendering presumed optimal strategies obsolete. Furthermore, social governance decisions require comprehensive integration of multidimensional factors, and identifying truly optimal strategies for highly complex societies constitutes a computationally intractable challenge.  

The simulation capabilities of the large-scale social simulator developed in this work position it as a valuable tool for social decision-makers and urban administrators. By configuring diverse initial states, social agents, and interaction mechanisms, it serves as a high-fidelity platform for policy outcome evaluation. Leveraging the simulator's acceleration engine for computational efficiency, multiple parallel experiments can be deployed to compare the long-term consequences of alternative decision strategies, thereby enabling data-driven policy selection.  

Notably, the simulator enables expansive exploration of decision spaces. Specifically, social management interventions can incorporate rich combinations of multidimensional policy actions. This framework not only facilitates the selection of superior strategies through counterfactual simulation-based parallel experiments but also inspires novel, precise, and composite policy solutions previously unconsidered in conventional approaches.

\subsubsection{Risk control and mitigation for a safer society}
The large-scale social simulators can also be used in risk control and mitigation, which represents a transformative approach to addressing emerging challenges in modern, hyper-connected societies. To advance societal safety, the large-scale social simulator exhibits substantial advantages across three critical dimensions:  

First, it enables a transition from static analysis to dynamic simulation. Traditional risk models, reliant on historical data, fail to capture rapidly evolving social dynamics such as emergent public opinion crises or AI algorithmic failures. By constructing a continuously updated digital twin of society, the simulator demonstrates sustained predictive capabilities for tracking security risk evolution.  

Second, it expands from single-domain to cross-domain risk assessment. Conventional approaches predominantly focus on isolated domains (e.g., social networks or economic systems), neglecting interdomain cascading effects. In reality, risks and societal crises often propagate through cascade effects and exhibit amplification due to cross-domain coupling. The simulator developed in this work addresses this gap by enabling co-modeling of micro-level agent behaviors across multiple domains, thereby equipping decision-makers to anticipate and mitigate butterfly-effect crises. 

Third,  it addresses the oversight of low-probability, high-impact events in long-tail distributions. Extreme scenarios (such as systemic AI failures all around the world) are frequently dismissed as statistically negligible despite their catastrophic societal consequences. Our simulator employs Monte Carlo methods to generate vast ensembles of extreme scenarios, systematically assessing societal system resilience and informing robust contingency planning.  

Therefore, through dynamic, cross-domain, and holistic simulation, the large-scale social simulator significantly enhances risk identification, control, and mitigation—paving the way for a safer, more resilient society.

\subsubsection{Social simulator for the future human-AI society}

Building upon the aforementioned practical applications, we propose a bolder and more open-ended discussion on the value of large-scale social simulators for next-stage human society.  

First, the large-scale social simulator—or its future iterations—could serve as foundational infrastructure for transitioning toward a digital human society. Currently, these simulators integrate multimodal data (behavioral logs, social networks) to create initial digital twins of individuals, enabling basic human-like capabilities. Future advancements through the integration of advanced interfaces (e.g., neural linkages, emotion-capturing sensors) will unlock richer, more dynamic simulations that mirror real-world complexity.  

Second, the large-scale social simulator functions as a strategic sandbox for exploring future societal architectures. The morphology and structure of future societies remain enigmatic yet critically important. For instance, Singapore’s ``Virtual City Lab" employs simulations to predict the impact of sea-level rise on urban functional zones by 2050. Similarly, the simulator could model the coupled effects of energy, transportation, and housing systems in hyper-dense cities, comparing the resilience of ``vertical megacities" versus ``distributed satellite city" paradigms.

Third, in the near future, the coexistence of humans and AI will be common in our society. The large-scale social simulator developed in this project currently focuses on deploying large language model agents to simulate human individuals. However, future societies will have more complex issues arising from varying degrees of AI integration: unemployment curves under different AI adoption rates, superintelligent AIs' influence on public decision-making (\textit{e.g.}, AI legislators), societal responses to the expansion of AI rights such as property ownership, etc.

In short, for future societies, the large-scale social simulator can serve as an indispensable instrument for understanding societal structures and providing profound insights.

\section{Conclusion}


In this work, we introduce AgentSociety, a large-scale social generative simulator that integrates LLM-driven agents, a realistic societal environment, and large-scale interactions, enabling authentic simulations of human behavior and societal dynamics. By bridging the gap between traditional agent-based modeling and real-world complexity, it advances generative social science and provides a powerful tool for analyzing, predicting, and intervening in complex social systems. The successful replication of real-world social experiments underscores AgentSociety's authenticity and practicality, positioning it as both an experimental testbed for social scientists and a practical policy evaluation platform for policymakers. More broadly, AgentSociety marks a significant advancement in the evolution of computational social science 2.0, shifting from a tool for static analysis to a dynamic, interactive platform for exploring complex social systems. Its ability to model and assess the impact of macro-level policies such as carbon taxes, industry transformation, and social welfare reforms allows it to provide a low-cost, low-risk environment for testing and refining policy interventions. In addition, AgentSociety serves as a powerful tool for predicting and mitigating social crises, tracking the spread of extreme ideologies, and analyzing group polarization, while also testing potential interventions for crisis management. Looking ahead, AgentSociety holds the promise of becoming a central platform for exploring the future of human society, where AI and humans coexist. It offers a space to test innovative governance models, investigate the impact of emerging technologies, and even redefine legal and ethical frameworks in an AI-driven world.




% More broadly, AgentSociety marks a step toward computational social science 2.0, where AI-driven simulations evolve from static analysis to dynamic, interactive digital societies, reshaping the way we study, model, and shape the future of human society.






