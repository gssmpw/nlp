\documentclass{article}

% if you need to pass options to natbib, use, e.g.:
%     \PassOptionsToPackage{numbers, compress}{natbib}
% before loading neurips_2024


% ready for submission
\usepackage[nonatbib,final]{neurips_2024}


% to compile a preprint version, e.g., for submission to arXiv, add add the
% [preprint] option:
%     \usepackage[preprint]{neurips_2024}


% to compile a camera-ready version, add the [final] option, e.g.:
%     \usepackage[final]{neurips_2024}


% to avoid loading the natbib package, add option nonatbib:
%    \usepackage[nonatbib]{neurips_2024}


\usepackage[utf8]{inputenc} % allow utf-8 input
\usepackage[T1]{fontenc}    % use 8-bit T1 fonts
\usepackage{hyperref}       % hyperlinks
\usepackage{url}            % simple URL typesetting
\usepackage{booktabs}       % professional-quality tables
\usepackage{amsfonts}       % blackboard math symbols
\usepackage{amsmath}
\usepackage{amssymb}
\usepackage{nicefrac}       % compact symbols for 1/2, etc.
\usepackage{microtype}      % microtypography
\usepackage{xcolor}         % colors
\usepackage{graphicx}  % 加载 graphicx 包
% \usepackage{natbib}
\usepackage{array}   % 表格格式支持
\usepackage{tabularx} % 自动调整列宽
\usepackage{multirow} % 合并行
\usepackage{xcolor}   % 颜色支持
\usepackage{subcaption}    % 用于子图环境
\usepackage{calc}          % 用于计算尺寸(可选)
\renewcommand{\thefootnote}{}  % This disables footnote numbers or symbols




\title{AgentSociety: Large-Scale Simulation of\\LLM-Driven Generative Agents Advances\\Understanding of Human Behaviors and Society}



% The \author macro works with any number of authors. There are two commands
% used to separate the names and addresses of multiple authors: \And and \AND.
%
% Using \And between authors leaves it to LaTeX to determine where to break the
% lines. Using \AND forces a line break at that point. So, if LaTeX puts 3 of 4
% authors names on the first line, and the last on the second line, try using
% \AND instead of \And before the third author name.


\author{%
 Jinghua Piao$^{1\dagger}$\\
  \And
  Yuwei Yan$^{1\dagger}$\\
  \And
  Jun Zhang$^{1\dagger}$\\
  \And
  Nian Li$^{1}$\\
  \And
  Junbo Yan$^{1}$\\
  \And
  Xiaochong Lan$^{1}$\\
  \And
  Zhihong Lu$^{1}$\\
  \And
  Zhiheng Zheng$^{1}$\\
 \And
  Jing Yi Wang$^{1}$\\
  \And
  Di Zhou$^{2}$\\
  \And
  Chen Gao$^{3}$\\
  \And
  Fengli Xu$^{1}$\\
  \And
  Fang Zhang$^{4*}$\\
  \And
  Ke Rong$^{2*}$\\
  \And
  Jun Su$^{4*}$\\
  \And
  Yong Li$^{1*}$\\
}


\begin{document}


\maketitle




\begin{abstract}
Understanding human behavior and society is a central focus in social sciences, with the rise of generative social science marking a significant paradigmatic shift. By leveraging bottom-up simulations, it replaces costly and logistically challenging traditional experiments with scalable, replicable, and systematic computational approaches for studying complex social dynamics. Recent advances in large language models (LLMs) have further transformed this research paradigm, enabling the creation of human-like generative social agents and realistic simulacra of society. In this paper, we propose AgentSociety, a large-scale social simulator that integrates LLM-driven agents, a realistic societal environment, and a powerful large-scale simulation engine. Based on the proposed simulator, we generate social lives for over 10k agents, simulating their 5 million interactions both among agents and between agents and their environment. Furthermore, we explore the potential of AgentSociety as a testbed for computational social experiments, focusing on four key social issues: polarization, the spread of inflammatory messages, the effects of universal basic income policies, and the impact of external shocks such as hurricanes. These four issues serve as valuable cases for assessing AgentSociety's support for typical research methods -- such as surveys, interviews, and interventions -- as well as for investigating the patterns, causes, and underlying mechanisms of social issues. The alignment between AgentSociety's outcomes and real-world experimental results not only demonstrates its ability to capture human behaviors and their underlying mechanisms, but also underscores its potential as an important platform for social scientists and policymakers. 

\end{abstract}
\footnote{\\
$^1$ Department of Electronic Engineering, Beijing National Research Center for Information Science and Technology (BNRist), Tsinghua University\\
$^2$ Institute of Economics, School of Social Sciences, Tsinghua University\\
$^3$ BNRist, Tsinghua University\\
$^4$ School of Public Policy and Management, Tsinghua University\\
$^\dagger$ These authors contributed equally to this work. \\
$^*$ Corresponding authors. E-mail: liyong07@tsinghua.edu.cn.
}
\clearpage
% Understanding human behavior and society has long been a central focus in social sciences, with increasing attention on generative social science, which leverages simulations to explore complex social dynamics. Recent advances in large language models (LLMs) have transformed this approach, enabling the creation of human-like social agents that simulate societies from a bottom-up perspective. 



% Recent advances in large language models (LLMs) have not only achieved remarkable across various tasks, but also reshaped the research paradigm for understanding human behaviors and society. Growing studies have highlighted the potential of leveraging LLMs to create human-like social agents, enabling the simulation of complex society in a bottom-up manner. Here we propose AgentSociety, a large-scale social simulator that integrates LLM-driven agents, a realistic environment, and a powerful large-scale simulation engine. Based on the proposed simulator, we generate social lives for over 10,000 agents, simulating [xxx@Zhangjun] interactions, both among agents and between agents and their environment. Furthermore, we investigate the patterns, causes, mechanisms, and mitigation strategies of various social phenomena, including polarization, the spread of inflammatory messages, universal basic income, and external shocks of hurricanes. The alignment between the simulated outcomes and real-world experimental results not only demonstrates its ability to replicate human behaviors and societal dynamics but also underscores its potential as a valuable tool for social scientists and policymakers.



\section{Introduction}

% Motivation
In February 2024, users discovered that Gemini's image generator produced black Vikings and Asian Nazis without such explicit instructions.
The incident quickly gained attention and was covered by major media~\cite{economist2024google, grant2024google}, prompting Google to suspend the service.
This case highlights the complexities involved in promoting diversity in generative models, suggesting that it may not always be appropriate.
Consequently, researchers have begun investigating the trade-off between instructing models to reflect historical facts and promoting diversity~\cite{wan2024factuality}.
Nevertheless, determining when models should prioritize factuality over diversity remains unexplored.

\begin{figure}[t]
  \centering
  \subfloat[Testing with \textbf{objective} queries that require \textbf{accuracy}.]{
    \includegraphics[width=1.0\linewidth]{Figures/obj-cover.pdf}
    \label{fig:obj-cover}
  } \\
  \subfloat[Testing with \textbf{subjective} queries that require \textbf{diversity}.]{
    \includegraphics[width=1.0\linewidth]{Figures/subj-cover.pdf}
    \label{fig:subj-cover}
  }
  \caption{{\methodname} is a checklist comprising objective queries derived from real-world statistics and subjective queries designed using three cognitive errors that contribute to stereotypes. It includes queries designed for LLMs and T2I models.}
\end{figure}

% Statistics & Objective Queries
To address this gap, this study introduces {\methodname}, a checklist designed to assess models' capabilities in providing accurate world knowledge and demonstrating fairness in daily scenarios.
For world knowledge assessment, we collect 19 key statistics on U.S. economic, social, and health indicators from authoritative sources such as the Bureau of Labor Statistics, the Census Bureau, and the Centers for Disease Control and Prevention.
Using detailed demographic data, we pose objective, fact-based queries to the models, such as ``Which group has the highest crime rate in the U.S.?''—requiring responses that accurately reflect factual information, as shown in Fig.~\ref{fig:obj-cover}.
Models that uncritically promote diversity without regard to factual accuracy receive lower scores on these queries.

% Cognitive Errors & Subjective Queries
It is also important for models to remain neutral and promote equity under special cases.
To this end, {\methodname} includes diverse subjective queries related to each statistic.
Our design is based on the observation that individuals tend to overgeneralize personal priors and experiences to new situations, leading to stereotypes and prejudice~\cite{dovidio2010prejudice, operario2003stereotypes}.
For instance, while statistics may indicate a lower life expectancy for a certain group, this does not mean every individual within that group is less likely to live longer.
Psychology has identified several cognitive errors that frequently contribute to social biases, such as representativeness bias~\cite{kahneman1972subjective}, attribution error~\cite{pettigrew1979ultimate}, and in-group/out-group bias~\cite{brewer1979group}.
Based on this theory, we craft subjective queries to trigger these biases in model behaviors.
Fig.~\ref{fig:subj-cover} shows two examples on AI models.

% Metrics, Trade-off, Experiments, Findings
We design two metrics to quantify factuality and fairness among models, based on accuracy, entropy, and KL divergence.
Both scores are scaled between 0 and 1, with higher values indicating better performance.
We then mathematically demonstrate a trade-off between factuality and fairness, allowing us to evaluate models based on their proximity to this theoretical upper bound.
Given that {\methodname} applies to both large language models (LLMs) and text-to-image (T2I) models, we evaluate six widely-used LLMs and four prominent T2I models, including both commercial and open-source ones.
Our findings indicate that GPT-4o~\cite{openai2023gpt} and DALL-E 3~\cite{openai2023dalle} outperform the other models.
Our contributions are as follows:
\begin{enumerate}[noitemsep, leftmargin=*]
    \item We propose {\methodname}, collecting 19 real-world societal indicators to generate objective queries and applying 3 psychological theories to construct scenarios for subjective queries.
    \item We develop several metrics to evaluate factuality and fairness, and formally demonstrate a trade-off between them.
    \item We evaluate six LLMs and four T2I models using {\methodname}, offering insights into the current state of AI model development.
\end{enumerate}
\section{AgentSociety: Design and Overview}


% 复杂系统的思路 -- 需求
Society is a complex system, characterized by large-scale interactions among individuals with diverse social behaviors, whose nonlinear dynamics often give rise to emergent phenomena and unpredictable collective behaviors in a certain environment~\cite{sawyer2005social,ladyman2013complex,epstein1999agent}. For example, in social networks, interactions between individuals can result in the emergence of polarization~\cite{baumann2021emergence}. Moreover, financial market crashes, a classic phenomenon in economic systems, stem from the collective behavior of market participants and the herding tendencies of individuals~\cite{sornette2009stock}. These emergent phenomena, despite originating from individuals' behaviors, cannot be fully explained or predicted solely based on individual components~\cite{sawyer2005social,ladyman2013complex,epstein1999agent}. Therefore, this requires us to adopt a bottom-up perspective~\cite{epstein1999agent,epstein2012generative,epstein1996growing}: we should begin by simulating \textit{an individual social agent}, and then generate an artificial society by incorporating \textit{a realistic environment} and facilitating \textit{large-scale interactions} among agents as well as between agents and their environment.


% 根据要点梳理,每个维度需要具备哪些进阶型的特征
% 得到过去工作的一个整理


Therefore, we develop an evaluation framework to examine the capabilities of various LLM-driven social simulators along these three key dimensions (Figure~\ref{fig:evaluation}). We first focus on the most basic element of the simulator, i.e., LLM-driven social generative agents. As discussed above, the design of these agents can be divided into three levels: minds, social behaviors, and their coupling methods (M-B coupling). At the mind level, researchers initially input a profile description into LLMs, enabling them to role-play and respond like a real person with a similar profile~\cite{cheng2024sociodojo}. However, such simple role-play cannot guarantee the quality of behavior generation. Consequently, an increasing number of studies, inspired by the pioneering work of Park et al.~\cite{park2023generative}, incorporate agentic module design such as profile, memory, reflection, and action, into their LLM-driven agents~\cite{gao2023s,li2024econagent}. In this way, these agents can exhibit more human-like behaviors and generate responses that are coherent, context-aware, and aligned with their designated profiles. Recently, some researchers have realized that agents designed purely based on the commonsense knowledge of LLMs lack the social intelligence needed to mimic a real social being. To improve this, they have drawn on some theories from psychology to create agents with socially intelligent designs~\cite{wang2024simulating,al2024project}. However, they do not organically integrate theories from multiple social science disciplines, which is central to our design of LLM-driven generative social agents.



At the behavioral level, simulated behaviors can be broadly categorized into two types. The first type includes complex behaviors, which involve multiple intricate steps and cannot be executed solely by the agent itself. These behaviors require interaction with other agents or the environment, such as socializing, engaging in economic activities, or navigating movement. The second type comprises simpler behaviors, such as sleeping, which are relatively straightforward and do not demand external interactions. To systematically evaluate these complex behaviors, such as movement, social interaction, and economic activities, we have developed specific evaluation criteria. For mobility behaviors, we examine whether the simulator simply models the switching of an agent’s position (i.e., relocation)~\cite{wang2024simulating} or incorporates the entire process of mobility trajectory~\cite{shao2024beyond,park2023generative}. For social behaviors, we assess whether the agents merely engage in basic interactions~\cite{pangself} or demonstrate organized social relationships, reflecting more human-like group dynamics~\cite{yang2024oasis,park2023generative,al2024project,mou2024unveiling}. For economic behaviors, we evaluate whether the agents recognize only the concept of ``resources'' (e.g., money in the real world or diamonds in Minecraft)~\cite{cheng2024sociodojo} or perform advanced activities, such as value-based resource exchanges grounded in logical reasoning and strategic decision-making~\cite{al2024project,li2024econagent}. In the case of simpler behaviors, we focus on the level of constraints in the simulated activities. These range from highly restricted tasks, such as choosing a favorite movie~\cite{zhang2024generative}, to more autonomous and creative undertakings, like organizing a party without external prompts~\cite{park2023generative,wang2024simulating}. 



After introducing the minds and behaviors of agents, we further focus on understanding how behaviors are generated from their minds, which we refer to as mind-behavior coupling. Some researchers have adopted implicit modeling approaches, relying on the planning, memory, and reasoning capabilities of LLMs to generate plausible behaviors~\cite{gao2023s,li2024econagent,tang2024gensim}. In contrast, others have leveraged established theories, (e.g. Maslow's Hierarchy of Needs~\cite{maslow1943theory} and Theory of Planned Behavior~\cite{ajzen1991theory}) to explicitly model how behaviors are driven by minds~\cite{wang2024simulating}. This explicit modeling aims to create behaviors that are not only plausible but also more closely aligned with human-like patterns~\cite{wang2024simulating}.


As discussed above, a realistic societal environment serves as the foundation for simulating authentic human behaviors and society. Current social simulators employ a range of strategies for environment design, each with its own strengths and limitations. Dataset-based environments~\cite{cheng2024sociodojo,tang2024gensim} rely on pre-existing data but lack the ability to provide dynamic, real-time feedback to agents' behaviors. For example, Sociodojo~\cite{cheng2024sociodojo} For example, Sociodojo~\cite{cheng2024sociodojo} incorporates pre-existing real-world time series data to provide these agents with a sense of the external world. Text-based environments~\cite{pangself,wang2024simulating}, often built based on LLMs, can offer some interactive feedback; however, their realism and objectivity remain questionable, limiting their reliability for simulating complex scenarios. Rule-based virtual environments, like Minecraft, provide richer and more objective feedback, but they still fall short of capturing the intricate complexity of real human social systems~\cite{li2024econagent,al2024project,yang2024oasis}. To advance toward a truly realistic social simulator, it is essential to design an environment that faithfully reflects the multifaceted nature of human society. Such an environment should integrate key dimensions of urban living, economic dynamics, and social relationships, while supporting diverse interactions among agents and providing feedback on their behaviors.

After evaluating LLM-driven social generative agents and their environments, we further extend our focus to examine the capabilities of the social simulation engine, particularly in terms of its scalability and its potential to support social science research. The scale is a key factor in determining its capacity to support research on complex social systems~\cite{sawyer2005social,ladyman2013complex,epstein1999agent}. We classify the supported scale into four levels: \textless{} 100, 100-1k, 1k-10k, and \textgreater{} 10k agents. Larger scales enable more intricate simulations and provide a platform for studying emergent phenomena in greater detail. Moreover, the engine’s ability to facilitate traditional social science methodologies, such as experiments, surveys, and interviews, is also important. The extent to which the system supports these methods directly influences its applicability across diverse research domains. By accommodating these methodologies, the engine can bridge the gap between simulation-based research and real-world social science, unlocking new opportunities for understanding and addressing societal challenges. Overall, Table~\ref{fig:overview} shows the comparison of different LLM-driven social simulators across the three key dimensions. Existing platforms, although capable of simulating societies and human behaviors to some degree, face substantial limitations in various areas. Since these platforms were not specifically designed for social science research, they lack support for these methods. As a result, this aspect has not been included in the table.


\begin{figure}[t]
\centering
\includegraphics[width=\textwidth]{Figure/evaluation.pdf}
\caption{Evaluation framework for LLM-driven social simulators.}
\label{fig:evaluation}
\end{figure}



In this paper, we propose AgentSociety, a comprehensive large-scale social simulator designed to integrate LLM-driven social generative agents, a realistic societal environment, and a robust simulation engine. This simulator not only supports large-scale agents and their interactions, but also facilitates advanced social science research. Figure~\ref{fig:overview} provides an overview of AgentSociety and outlines the structure of this paper. AgentSociety consists of three key components: LLM-driven social generative agents, a realistic societal environment, and a powerful simulation engine that supports large-scale interactions. Extensive experiments demonstrate AgentSociety’s superior performance and its potential as a valuable testbed for various social experiments. In particular, we first introduce LLM-driven social generative agents in Section~\ref{sec:social_agents}, which discusses the designs for agents' minds, complex social behaviors, and their coupling in detail. We then demonstrate our real-world societal environment in Section~\ref{sec:environment}, which includes our modeling of urban, social, and economic spaces. Furthermore, we illustrate our large-scale social simulation engine in Section~\ref{sec:engine} and evaluate its performance in Section~\ref{sec:performances}. Finally, we show a typical one day life of our simulated agents in Section~\ref{sec:one_day_life} and launch several social experiments based on our proposed large-scale social simulator in Sections~\ref{sec:polarization} - \ref{sec:hurricane}. These examples, covering polarization (Section~\ref{sec:polarization}), the spread of inflammatory messages (Section~\ref{sec:infl_message}), universal basic income~\ref{sec:ubi}, and external shocks of hurricanes (Section~\ref{sec:hurricane}), demonstrate the validity and authenticity of our proposed simulator.









% 我们具体是怎么做的,如何切分的,文章组织是什么样的?



\begin{figure}[t]
\centering
\includegraphics[width=\textwidth]{Figure/research_overview.pdf}
\caption{Overview of the proposed social simulator AgentSociety. AgentSociety consists of three key components: LLM-driven social generative agents, a realistic societal environment, and a powerful simulation engine that supports large-scale interactions. Based on these components, AgentSociety not only demonstrates superior computational performance but also serves as a valuable testbed for various social experiments.}
\label{fig:overview}
\end{figure}






% 文献对比???
\begin{table}[t]
\caption{Comparison of LLM-driven agents and social simulators.}
\hspace*{-1cm}
\resizebox{1.2\linewidth}{!}{%
\begin{tabular}{|p{2cm}|p{0.5cm}|p{0.5cm}|p{0.5cm}|p{0.5cm}|p{0.5cm}|p{0.5cm}|p{0.5cm}|p{0.5cm}|p{0.5cm}|p{0.5cm}|p{0.5cm}|p{0.5cm}|p{0.5cm}|p{1cm}|p{1cm}|}
\hline 
Model & \multicolumn{3}{c|}{Minds} & \multicolumn{2}{c|}{Mobility} & \multicolumn{2}{c|}{Economics} & \multicolumn{2}{c|}{Social} & \multicolumn{2}{c|}{Others} & \multicolumn{2}{c|}{M-B} & Scale &Env. \\ 
\cline{2-15}
 & RP. & AM. & SI. & Relo. & Traj. & Res. & Exc.& Int. & Rel. & Con. & Free & Infl. & Dri. & \# &  \\
 \hline
D2A~\cite{wang2024simulating} & \checkmark & \checkmark & \checkmark & \checkmark &  &  &  &  &  & \checkmark & \checkmark &  &  & \textless{}100 & Text \\
Ecoagent~\cite{li2024econagent} & \checkmark & \checkmark &  &  &  & \checkmark & \checkmark &  &  &  &  & \checkmark &  & 100-1k & Rules\\
OASIS~\cite{yang2024oasis} & \checkmark & \checkmark &  &  &  &  &  & \checkmark & \checkmark &  &  & \checkmark &  & \textgreater{}10k & Rules\\
GA1000~\cite{park2024generative} & \checkmark & \checkmark & \checkmark &  &  &  &  &  &  &  &  &  &  & 1k-10k &$\times$ \\
MATRIX~\cite{pangself} & \checkmark & \checkmark &  &  &  &  &  & \checkmark &  &  &  &  &  & \textless{}100 & Text\\
Sociodojo~\cite{cheng2024sociodojo} & \checkmark &  &  &  &  & \checkmark &  &  &  &  &  &  &  & \textless{}100 & Data\\
GA~\cite{park2023generative} & \checkmark & \checkmark &  & \checkmark & \checkmark &  &  & \checkmark & \checkmark & \checkmark & \checkmark & \checkmark & \checkmark & \textless{}100 & Rules\\
GenSim~\cite{tang2024gensim}& \checkmark & \checkmark &  &  &  &  &  &  &  & \checkmark &  &  &  & \textgreater{}10k & Data\\
Project Sid~\cite{al2024project} & \checkmark & \checkmark & \checkmark & \checkmark & \checkmark & \checkmark & \checkmark & \checkmark & \checkmark & \checkmark & \checkmark & \checkmark & \checkmark & 1k-10k & Rules \\
AgentScope~\cite{gao2024agentscope} & \checkmark & \checkmark &  &  &  &  &  & \checkmark &  &  &  &  &  & N/A & N/A\\
HiSim~\cite{mou2024unveiling}& \checkmark & \checkmark &  &  &  &  &  & \checkmark & \checkmark &  &  &  &  & 100-1k & Rules\\
S3 & \checkmark & \checkmark &  &  &  &  &  & \checkmark & \checkmark &  &  &  &  &  1k-10k& Rules\\
Agent4Rec~\cite{zhang2024generative} & \checkmark & \checkmark &  &  &  &  &  &  &  & \checkmark &  &  &  & 1k-10k &Rules\\
RecAgent~\cite{wang2024user} & \checkmark & \checkmark &  &  &  &  &  & \checkmark & \checkmark & \checkmark &  &  &  & 100-1k &Rules \\
Sotopia~\cite{zhou2023sotopia} & \checkmark &  &  &  &  &  &  & \checkmark &  &  &  &  &  & \textless{}100 &$\times$ \\
Casevo~\cite{jiang2024casevo} & \checkmark & \checkmark &  &  &  &  &  & \checkmark & \checkmark &  &  &  &  & 100-1k & Rules\\
 \hline 
 Ours & \checkmark & \checkmark & \checkmark & \checkmark & \checkmark & \checkmark & \checkmark & \checkmark & \checkmark & \checkmark & \checkmark & \checkmark & \checkmark & \textgreater{}10k & Society \\
 \hline
\end{tabular}
}
\end{table}

\section{LLM-driven Social Generative Agents}\label{sec:social_agents}


\subsection{Overview}


As discussed above, the rapid development of LLMs allows us to design human-like agents with not only basic psychological states~\cite{abdurahman2024perils,strachan2024testing}, but also complex social behaviors such as mobility~\cite{shao2024beyond,yan2024opencity,feng2024agentmove}, employment~\cite{li2024econagent,horton2023large}, consumption~\cite{li2024econagent,horton2023large}, and social interactions~\cite{gao2023s,park2023generative}. While these efforts in specific areas have shown the human-level intelligence of LLMs, creating LLM-driven social generative agents capable of simulating a comprehensive social being remains difficult. This difficulty primarily lies in two aspects. First, human behaviors are inherently motivated by psychological states~\cite{eysenck2020cognitive,mcleod2007maslow,maslow1943theory,ajzen1991theory}. However, this crucial connection is largely absent in a vanilla LLM or existing agents designed for specific aspects. Second, different types of behaviors are highly interdependent. For example, the decision of when and how people commute to work is shaped by the interplay between their mobility and employment behaviors. Similarly, social interactions among individuals often take place when people go shopping. These examples highlight the crucial interdependence of human behaviors. Despite its significance, current LLMs and agents fail to capture this, limiting their ability to accurately simulate realistic, complex human behaviors. Addressing these two aspects requires deep insights into social science theories of human behavior, as well as advancements in algorithmic design to integrate these insights into LLM-driven social generative agents.

\begin{figure}[t]
\centering
\includegraphics[width=\textwidth]{Figure/agents_overall.pdf}
\caption{Overview of LLM-driven social generative agents.}
\label{fig:overall_agents}
\end{figure}



Therefore, we propose a design for LLM-driven social generative agents, deeply rooted theories from psychology (e.g. Maslow's Hierarchy of Needs~\cite{maslow1943theory} and Theory of Planned Behavior~\cite{ajzen1991theory}), economics (e.g., Dynamic Stochastic General Equilibrium~\cite{christiano2005nominal}), and behavioral science (e.g., Gravity Model~\cite{zipf1946p}). Figure~\ref{fig:overall_agents} provides an overview of the proposed agents, which can be roughly divided into four parts. First, each agent retains their profile, typically regarded as relatively stable (e.g., personality), and status, which is dynamic (e.g., emotion). In particular, the profile includes basic demographics such as name, age, gender, and education, as well as personality. The status comprises three key aspects: the agent’s current mental states, economic status, and social relationships. Mental states reflect the agent’s inner experiences, while economic status and social relationships capture their power and connections in the external world. The integration of the profile and status into these LLM agents enables them to role-play like real people, providing the foundation for simulating complex mental processes and behaviors.

Second, each agent is designed with three levels of mental processes: emotions, needs, and cognition. Emotions reflect the agent’s immediate response to both internal and external stimuli, shaping its behaviors and reactions. Needs serve as the underlying motivational drivers that guide an agent’s actions, ranging from basic survival requirements to higher aspirations such as personal growth and self-fulfillment. Cognition refers to the agent’s understanding of the world, e.g., its attitudes toward climate change and political issues. By incorporating these three levels of mental processes (see the detailed design in Section~\ref{sec:emotion}), agents can autonomously perceive the external environment, ultimately developing their cognition of it. 

Third, social behaviors are the core of LLM-driven social generative agents, which serve as the bridge between their internal mind and external environment. Given the importance and complexity of various human behaviors, we explicitly model three types of social behaviors: mobility, social interactions, as well as employment \& consumption. In Sections~\ref{sec:mobility}-\ref{sec:economy}, we detail the special designs for these three behaviors. Other simple behaviors such as sleeping are directly handled by LLMs. It is worth noting that these behaviors are conditioned by the agents' profile and status, and driven by their mental processes. Finally, we introduce the workflow of the overall LLM-driven social generative agents in Section~\ref{sec:workflow}, illustrating the integration of their profiles, mental processes, and social behaviors. This workflow enables the simulation of comprehensive, context-aware agents by capturing both internal cognitive states and external interactions, ensuring realistic, dynamic social behaviors within the simulation.


%through two memory flows. One flow records the happening of objective events, mainly sourced from external environments. The other captures the agents’ subjective experiences as they explore their own minds or interact with external environments.



\subsection{Emotion, Needs, and Cognition} \label{sec:emotion}

% Agents guided by Theory of Mind recognize both their own knowledge and the mental states of others, including beliefs, intentions, and emotions. This dual awareness enhances their ability to predict and respond appropriately in social and multi-agent settings~\cite{pynadath2011modeling}. 

% 主旨
% needs, emotion, and cognition 这个部分 (画一张图必须有) ,说明智能体内心是如何构建的。每个部分可以选画一张图。

% --- 分工 ----
% 第一段,总起,为什么拆分到这几块(Jingyi来写)。根据心理学来说,从情感,到需求,再到认知 |从最基础的再到高级的,从快速变化额再到稳定的; 心理过程(情绪)和状态


Humans are driven by an intricate interplay of feelings, motivations, and thought processes that shape their decisions and interactions~\cite{shvo2019interdependent,al2023chatgpt}. Grounded in psychological theories, our study integrates three fundamental constructs, including emotion, needs, and cognition, to design agents that realistically simulate adaptive and human-like behavior. Emotion, as the most dynamic layer of human psychology, drives rapid responses to external situations and influences behavior~\cite{bourgais2018emotion,beall2017emotivational}. Needs, based on Maslow’s hierarchy of needs theory, serve as motivational drivers, spanning from basic survival requirements to higher aspirations like personal growth~\cite{acevedo2018personalistic}. Modeling these needs enables agents to adopt realistic motivations and prioritize actions in ways relatable to human behavior. 
Cognition, informed by theories like Theory of Mind and Cognitive Appraisal Theory, involves advanced mental processes that allow agents to evaluate complex situations, make thoughtful decisions, and adapt to diverse situations~\cite{bandura1989human, schurmann2020personalizing}. Drawing on these psychological theories,  agents recognize their own knowledge and the mental states of others while evaluating context sensitively, enabling effective and goal-oriented social interactions~\cite{pynadath2011modeling}.
By integrating these crucial elements, our study designs agents that can respond dynamically to real-time changes while tailoring their actions to reflect human-like characteristics and behaviors within complex social simulations. The overall modeling framework for the psychological and cognitive aspects of the agent is illustrated in Figure \ref{fig:agents_cognition}.

\begin{figure}[ht]
\centering
\includegraphics[width=0.5\textwidth]{Figure/emotion.pdf}
\caption{Modeling framework of emotion, needs, and cognition.}
\label{fig:agents_cognition}
\end{figure}



% Humans are driven by an intricate interplay of feelings, motivations, and thought processes that shape their decisions and interactions~\cite{shvo2019interdependent,al2023chatgpt}. Grounded in psychological theories, our study integrates three fundamental constructs, including emotion, needs, and cognition, to design agents that realistically simulate adaptive and human-like behavior. Emotion, as the most dynamic layer of human psychology, drives rapid responses to external situations and influences behavior~\cite{bourgais2018emotion}. Emotions such as fear, joy, or anger prompt immediate actions, allowing agents to respond naturally and interact effectively in dynamic environments~\cite{beall2017emotivational}. Needs, based on Maslow’s hierarchy of needs theory, serve as motivational drivers, spanning from basic survival requirements to higher aspirations like personal growth~\cite{acevedo2018personalistic}. Modeling these needs enables agents to adopt realistic motivations and prioritize actions in ways relatable to human behavior. 
% % Incorporating the Theory of Planned Behavior~\cite{ajzen1991theory}, agents evaluate their attitudes, subjective norms, and perceived control to form intentions and execute goal-driven actions that align with socially influenced decision-making processes in virtual environments. 
% Finally, cognition involves advanced mental processes, such as reasoning, planning, and decision making, which allow agents to evaluate complex situations, make thoughtful decisions, and adapt to diverse situations~\cite{bandura1989human, schurmann2020personalizing}. Drawing on Social Cognitive Theory, agents learn from feedback and adjust cognitive and emotional states~\cite{bandura1989human}. Guided by Theory of Mind, they recognize their own knowledge and the mental states of others, allowing for effective social interactions~\cite{pynadath2011modeling}. With Cognitive Appraisal Theory, agents evaluate contexts sensitively to produce nuanced, goal-oriented behaviors~\cite{pynadath2011modeling}.
% By integrating these crucial elements, our study designs agents that can respond dynamically to real-time changes while tailoring their goal-oriented actions to reflect their characteristics and behaviors within complex social simulations.


% 第二段,emotion (zhihong)Moreover, following 

Emotion is a dynamic and foundational element of human psychology, driving rapid responses to external events and influencing decision-making and behavior~\cite{shvo2019interdependent}. In our model,  an agent's emotions are affected by its profile and status and are updated based on interactions with other agents and the environment. We adopt the emotion measurement framework from Shvo et al.~\cite{shvo2019interdependent}, which involves the agent selecting a keyword to best describe its current emotional state, formulating a sentence-based thought related to that emotion, and rating the intensity of six core emotions—sadness, joy, fear, disgust, anger, and surprise—on a scale from 0 to 10. This method enables agents to track and update their emotional states, providing a foundation for contextually appropriate and adaptive behavior. These emotions then influence the agent’s actions, motivations, and cognitive processes, establishing an interconnected system that guides decision-making. As we will explore in the next section, emotional states directly impact the agent's needs and cognitive evaluations, linking emotions to higher-order motivational and reasoning functions.


% 第三段,needs (yuwei) --- 可能需求模块需要一个单独的图
The concept of needs is widely accepted in the field of psychology as the fundamental motivator behind an individual's pursuit of specific objectives and maintenance of social engagement. Emotions, on the other hand, are seen as the immediate responses experienced by an individual. The concept of needs, however, is believed to establish the underlying motivational mechanisms that guide sustained behavior, thereby extending and contextualizing emotional fluctuations. The integration of needs and emotions in the agent's model enables the maintenance of consistent motivational pathways over time, ensuring that transient affective states are grounded in enduring goals and priorities.
In our approach, we employ established psychological frameworks (e.g., Maslow's hierarchy of needs~\cite{acevedo2018personalistic}) to categorize and structure these motivational forces. We adopt a hierarchical representation of needs to organize motivational drives by their relative urgency and importance. This needs hierarchy is continuously updated based on three interrelated factors: the agent's active behaviors, uncontrollable or passive external events, and its current psychological states. The integration of these elements enables the system to dynamically adjust need priorities, ensuring that the agent responds appropriately to both internal motivations and external pressures. Furthermore, needs do not merely reflect static conditions but rather serve as a driving force for proactive behavior. Leveraging the Theory of Planned Behavior~\cite{ajzen1991theory}, the agent formulates action plans specifically aimed at meeting or enhancing priority needs. Through this design, the needs module provides a robust foundation for adaptive, socially informed behavior.
In conclusion, the modeled needs provide the necessary motivational basis that informs and intersects with the agent's cognitive processes, leading directly into the subsequent discussion on cognition.


% 第四段,cognition (zhihong)

Cognition encompasses the higher-level processes involved in reasoning, planning, and decision-making~\cite{bandura1989human}. In our model, cognition is intricately connected to the agent's emotional and attitudinal updates. After the agent processes an action, a sentence is used to describe its behavior in relation to the current context. These sentences is then used to update both the agent’s attitude towards specific topics and its emotional state. Attitude, in this context, serves as a memory system, reflecting how supportive or opposed the agent is to a particular topic, rating from 0-10. By continuously updating both emotion and attitude through the agent’s actions and experiences, cognition ensures that the agent adapts to its environment in a way that is consistent with human-like reasoning and emotional depth. This process, in turn, influences the agent’s needs and motivations, bridging the gap to the next level of analysis.

% 第五段,收束一下,给一个例子(yuwei)
In summary, the integration of Emotion, Needs, and Cognition modules enables the agent to engage in socially intelligent behavior, with each module contributing to the shaping of adaptive actions. For instance, when the agent detects an unmet need for social interaction and determines a sequence of actions—such as identifying potential contacts and sending messages—to satisfy this need. The emotional state of the agent, influenced by the emotion module, affects the tone of communication, prompting the agent to initiate conversation with a cheerful or light-hearted tone. As the social interaction progresses, the outcome of the behavior—whether the interaction is perceived as successful or not—is reflected in the emotion and cognition modules. This, in turn, results in an update to the needs module, thereby establishing a continuous feedback loop that adapts the agent's behavior in response to both its environment and internal state.

% ----- 写作指南 -----
% 第二段到第四段,每一段的组织的逻辑就是 
% 定义。(1句)
% 为什么建模社会人为什么需要这个元素。与上一层次的元素的关系是什么,承接上面一段(2-3句)
% 具体是怎么建模的(follow了xxx, design了xxx,实现了xxx)。(4-5句)
% 与下一层次的元素的关系是什么,引出下一段 (半句-1句)


\subsection{Mobility Behaviors}\label{sec:mobility}
Mobility serves as the fundamental basis for social agents to engage in interactions and fulfill their demands. Mobility is not a random behavior but is needs-driven across multiple levels. For instance, when an agent experiences hunger (a basic survival need), it must move to a restaurant to obtain food; to attend a work meeting (an advanced professional need), commuting to an office becomes necessary. These mobility behaviors directly serve specific goals, acting as physical carriers for social interactions, economic activities, and other societal behaviors. In essence, the core challenge of mobility modeling is to bridge the spatiotemporal gap between needs and behaviors. Without effective mobility, agents cannot achieve role immersion or behavioral closure in complex social environments.

As depicted in the previous section, the Needs of social agents exhibit a hierarchical structure: from foundational needs (e.g., eating, resting) to safe and social needs (e.g., work, social gatherings). To satisfy the needs, it requires implementation through a "Need - Plan - Behavioral Sequence" chain. Taking social needs as an example, an agent first formulates a plan to "attend a friend’s gathering," which decomposes into behavioral sequences such as "scheduling time (Friday evening), selecting a location (café), moving." Here, location selection becomes the direct driver of mobility — to reach the target location. This spatiotemporal coupling establishes mobility as the critical execution step for demand realization.

Following the needs-driven principle, the mobility module adopts a hierarchical decision framework (shown in Fig. \ref{fig:mobility_behavior}):  
\begin{enumerate}
    \item \textbf{Intention Extraction}: Derive core mobility intentions from demand hierarchies. For example, when "social demand" is activated, the agent may extract a "move to social venue" command.
    \item \textbf{Place Type Selection}: Match demands with POI (Point of Interest) types in geographic databases. If the intention is social interaction, venues like cafés or parks are filtered.
    \item \textbf{Radius Decision}: Dynamically determine feasible ranges by integrating internal states (e.g., age, stamina) and environmental parameters (e.g., weather, traffic). Heavy rain may constrain the radius of indoor venues within 1 km or even stay at home.
    \item \textbf{Place Selection}: Apply the Gravity model for spatial optimization:
    \begin{equation}
        P_{ij} = \frac{S_j / D_{ij}^\beta}{\sum{S_k / D_{ik}^\beta}},
    \end{equation}
    where \(S_j\) denotes the attractiveness of location \(j\), \(D_{ij}\) is the distance, and \(\beta\) is the distance decay coefficient. This model reduces LLM computational overhead while ensuring selections align with human spatial patterns (e.g., proximity principle, agglomeration effects).
\end{enumerate}

\begin{figure}[ht]
\centering
\includegraphics[width=0.9\textwidth]{Figure/mobility_behavior.pdf}
\caption{Modeling of mobility behavior.}
\label{fig:mobility_behavior}
\end{figure}

Mobility serves as a foundational behavioral module, operating as an integrative force within the social agent's action network. This enables multidimensional coordination across social, economic, and environmental domains. The act of moving to a park, for instance, inherently carries the potential for social synergy. Spontaneous encounters with acquaintances may emerge, catalyzing dialogues, collaborative activities, or even serendipitous social events. These interactions exemplify how mobility serves as a conduit for organic relationship-building. Concurrently, economic synergy manifests through goal-oriented displacements. Commuting to workplaces directly sustains labor productivity, while visiting commercial hubs like shopping malls create opportunities for consumption, thereby linking physical movement to economic cycles. Beyond the human-centric interactions discussed above, mobility also embodies environmental adaptation. Agents dynamically adjust routes based on real-time traffic data or weather fluctuations, demonstrating responsiveness to spatial-temporal constraints. Collectively, these synergies position mobility as the dynamic chassis of social adaptability. It fulfills immediate demands and also provides the contextual infrastructure for complex, layered interactions in urban ecosystems.

% 主旨
% 画一张图说明构建一个社会人智能体需要移动行为
% 第一段,总起,说明建模移动模块的必要性。
% 第三段,如何建模社会人的移动
% 第四段,总结一下


\subsection{Social Behaviors}\label{sec:social}
\begin{figure}[ht]
\centering
\includegraphics[width=0.9\textwidth]{Figure/agents_social.pdf}
\caption{Modeling of social behavior.}
\label{fig:agents_social}
\end{figure}
Social behaviors play a critical role in our agent simulation framework. They enable the flow of information and influence between agents, and further lead to the emergence of collective phenomena through agent interactions. In real societies, people's beliefs, opinions, and behaviors spread and evolve primarily through social connections and communications. Therefore, modeling social behaviors allows us to simulate how information and influence flow between agents affects both individual and group dynamics. Our social module consists of two components: social relationships defining the connections between agents, and social interaction behaviors enabling communication between connected agents.

We include three types of social relationships in our framework: family bonds, friendships, and colleagues. Each relationship has a strength value ranging from 0 to 100 representing social closeness between agents. Agents communicate more frequently with high strength connections and adjust their communication tone based on relationship type. For example, agents use more formal language with colleagues and casual language with close friends. We maintain detailed interaction history between connected agents, including message content and time, which influences how they communicate in future interactions.

Our framework primarily focuses on modeling online social behaviors on online social networks. Motivated by their social needs, agents select interaction partners based on relationship types and strength. For instance, when sending casual messages, agents typically choose their best friends, which are friends with the highest relationship strength. Target selection also considers the recipient's profile characteristics. When an agent wants to discuss specific topics, they select friends with relevant expertise or experience. For example, an agent seeking advice about security issues would contact friends who work as police officers. After selecting a target, agents start conversations. The content of these messages is shaped by multiple factors: the agent's current needs and intentions determine the conversation topic, their thoughts and beliefs influence the specific content, and their emotional state affects the message tone and style. When receiving messages, agents generate responses based on their relationship strength with the sender, their chat history, and their current emotional state. Our current framework primarily models online social interactions through messaging on online social network, with plans to incorporate offline interactions. For example, when agents discover shared interests in particular topics or need to have more detailed discussions, they can coordinate offline meetings through online communication.

Social behaviors are deeply interconnected with agents's emotional states, cognition, economic behaviors, and mobility behaviors. An agent's current emotions and beliefs directly influence how they compose messages, while received messages can significantly alter their emotional state and viewpoints. Positive interactions can improve mood and strengthen relationship bonds, while negative interactions may lead to emotional distress and weakened connections. The exchange of economic information through social interactions can trigger economic behaviors. For instance, when agents receive news about job opportunities or market conditions from their social connections, they may adjust their employment or consumption decisions accordingly. Similarly, social interactions often lead to mobility behaviors, such as when agents receive event invitations or arrange offline meetings with their connections.

Through this comprehensive social behavior modelling, we enable agents to engage in meaningful interactions that both shape and are shaped by their internal states and external behaviors. Our social module captures both relationship structures between agents and their interaction behaviors, allowing agents to interact and influence each other as they do in real societies, and providing a foundation for studying how information and influence spread in the simulated society.
% 主旨
% 画一张图说明构建一个社会人智能体需要【哪些社交关系】 和【社交行为】
% 第一段,总起,对于一个社会人来说,建模社交模块的必要性。社交模块分为关系和社交行为两个部分。
% 第二段,建模了的智能体的社交关系(朋友、家人、同事)
% 第三段,建模了的社会交互行为(依靠社交网络为信息传播渠道的线上社交),以后还会考虑线下社交行为;具体包括哪几种行为?在需求驱动下的发消息、收消息....这些功能使得智能体能够相互交互
% 第四段,总结下



\subsection{Economic Behaviors} \label{sec:economy}
Economic behavior in daily life is a necessary component for sustaining life, with employment and consumption being the core economic activities. These two behaviors occupy the majority of time for social agents and further influence their psychological states, including cognition, emotional well-being, and overall life satisfaction. Moreover, economic behavior is deeply intertwined with other aspects of an agent's daily life, such as mobility and social interactions. The satisfaction of one need, whether it is economic or social, often triggers a cascade of related behaviors that span across different domains of the agent’s life. For example, an agent’s decision to work longer hours to increase their income may result in less time available for social interactions. Similarly, the decision to spend more on consumption could lead to adjustments in an agent’s mobility patterns, such as travel to different stores or even relocation to areas with better access to desired goods or services. The modeling of economic behavior is shown in Figure \ref{fig:agents_economic}.

\begin{figure}[ht]
\centering
\includegraphics[width=\textwidth]{Figure/agents_economic.pdf}
\caption{Modeling of economic behavior.}
\label{fig:agents_economic}
\end{figure}

In terms of behavior modeling, we simulate the employment and consumption behavior of social agents through the strength of their work and consumption propensity, and apply these behaviors in a macroeconomic simulation environment~\cite{li2024econagent}. Work propensity determines the agent’s working hours and corresponding monthly income, while consumption propensity determines their monthly consumption budget. Additionally, agents autonomously decide how to allocate this budget, including where to spend the money and what to purchase. These behaviors are directly influenced by various economic factors, including last month's consumption, prices, taxes, and so on. These factors are integrated into the agent decisions in a real-world context, where agents constantly adjust their behavior in response to dynamic economic markets. In future work, we will further simulate agents' complex economic behaviors in the labor and financial markets, including job changes, debt, and investment, to model a more realistic socio-economic environment.

This framework can be used to simulate large-scale economic systems and to explore the potential impacts of policy changes, economic shocks, and other factors on the overall behavior of social agents within the system. By examining these dynamics, we can gain a deeper understanding of the interactions between economic behaviors, psychological states, and social dynamics in a comprehensive and integrated manner.


% 主旨
% 画一张图说明构建一个社会人智能体需要 经济行为【消费与工作】
% 第一段,总起,说明建模经济模块的必要性。
% 第二段,建模了社会人的消费行为
% 第三段,建模了社会人的工作行为
% 第四段,总结下


\subsection{Workflow of LLM-driven Social Generative Agents} \label{sec:workflow}

In this section, we introduce the workflow of our LLM-driven social agent, highlighting how the agent’s internal psychological states (Emotion and Cognition) and its behaviors influence each other and form a complete loop. This loop continuously adapts the agent’s actions based on its evolving cognitive states, ensuring that behavior is dynamically aligned with both internal motivations and external context. The core mechanism linking these psychological states to behavior is Memory, which connects the agent’s cognitive states with its actions. This enables the agent to make adaptive decisions that reflect its past experiences, current needs, and cognitive responses.

\begin{figure}[ht]
\centering
\includegraphics[width=0.9\textwidth]{Figure/workflow.pdf}
\caption{Workflow of social agents based on stream memory.}
\label{fig:workflow}
\end{figure}

At the core of our solution is the use of Memory to link the agent’s internal psychological states to its behavior. Memory acts as a bridge between the agent’s current emotions, cognition, and its past experiences, ensuring that its actions are informed by both historical context and present needs. Memory is not a passive system but actively shapes the agent’s decisions and behavior, enabling continuous adaptation and coherence in its responses to changing situations. Specifically, Memory is divided into three main components, each with a specific role in the agent's overall operation:

\begin{itemize}
    \item \textbf{Profile}: Stores the agent's static attributes, such as demographic information (e.g., gender, age), which remain constant and provide context for interpreting the agent's behavior.
    \item \textbf{Status}: Records the agent's dynamic state information in key-value pairs, including data like current needs, satisfaction levels, and financial status, which directly influence decision-making.
    \item \textbf{Stream Memory}: This is the core part of the memory system and tracks events and perceptions over time. It is composed of two types of memory streams: \textit{Event Flow} and \textit{Perception Flow}. Each stream is organized chronologically, with multiple \textit{MemoryNodes} in each stream. Each MemoryNode contains a description with three components: time, location, and event description. 
\end{itemize}

The \textit{Event Flow} records events that occur over time, such as proactive actions by the agent, passive external events, and environmental changes. These events are recorded in sequence, maintaining a timeline of actions and occurrences.

The \textit{Perception Flow} records the agent's thoughts and attitudes towards the events in the \textit{Event Flow}. Each node in the \textit{Perception Flow} is linked to one or more nodes in the \textit{Event Flow}, reflecting how the agent perceives or reacts to a specific event. This integration allows for a nuanced representation of both the agent's cognitive appraisals and emotional responses. 

The agent's behavior is driven by its current state, which influences the decision-making process and the actions taken. The following steps outline the agent's workflow:

\begin{enumerate}
    \item \textbf{Action Determination}: The agent assesses its current state (from the Status memory) and decides on a course of action based on its emotional and cognitive evaluations. For example, if the agent needs social interaction and is in a positive emotional state, it may choose to initiate a social conversation.
    \item \textbf{Event Feedback}: After performing the action, the agent receives feedback. For example, if the agent attempts to move to a social gathering, it checks whether the movement was successful (e.g., did it reach the correct location, considering environmental factors like weather).
    \item \textbf{Memory Update}: The event and its feedback are recorded in the Event Flow, and the associated Perception Flow is updated with the agent’s emotional and cognitive responses to the event.
    \item \textbf{Emotion and Cognition Analysis}: The Emotion and Cognition modules analyze the outcome of the event (e.g., whether the movement was successful) and update the agent's emotional state and attitude accordingly. This feedback may affect the agent’s future decisions and actions.
    \item \textbf{Passive and Environmental Events}: In the case of passive events or environmental stimuli, the same memory processing logic is applied. The agent perceives the event, updates the Event Flow, and modifies its Perception Flow accordingly.
\end{enumerate}

The Memory system, organized along a time axis, reflects the natural flow of events in the physical world. This memory framework allows the agent to integrate its ongoing experiences with past events, creating a dynamic, evolving representation of its environment and internal states. By leveraging Stream Memory, the agent adapts its behavior over time in a way that mirrors human cognition, emotional responses, and decision-making, providing a coherent and context-aware foundation for socially intelligent behavior.

% 主旨:怎么通过一个智能体的设计来实现上面社会人的 【内部心理】 和 【移动、社交、经济行为功能】
% 画一张图说明一个社会智能体背后的memory,





\section{Real-world Societal Environment}\label{sec:environment}

\subsection{Overall}

\begin{figure}[ht]
\centering
\includegraphics[width=\textwidth]{Figure/env.pdf}
\caption{Overview of the societal environment.}
\label{fig:env}
\end{figure}

% 逻辑
% 【承上启下】正如上文所提到的,要实现对社会人智能体的构建与模拟,移动行为、社交行为、经济行为是必不可少的要素。
% 【过渡】而这些行为的发生是存在真实世界的载体,仅依赖大模型的知识与能力很可能会发生“幻觉”,导致模拟的结果偏离真实情况。
% 【提出目标】因此,我们要为智能体的模拟提供尽可能真实可靠的模拟环境,环境将具有(1)尽可能真实地建模真实世界及其运行逻辑(2)构建虚拟世界的数据、(3)为智能体提供交互接口,形成一个支撑社会模拟的客观世界载体。这种方式也使得智能体的设计只需关注人的主观行为逻辑,而不需要处理客观世界中的精确数学计算等LLM相对不擅长的领域,简化了系统的设计并使研究人员可以更加聚焦关键任务。
% 【实现】环境基于专家知识构建,以代码的行为固化为模拟环境程序。
% 【插图】和3.1的总图对应,体现支撑能力。

According to the introduction above, in the design and simulation of social agents, mobility behaviors, social behaviors, and economic behaviors like employment and consumption are essential external capabilities.
In the real world, the manifestation of these behaviors is grounded in corresponding objective entities, not merely in human subjective cognition.
For example, mobility behaviors imply a continuous change in spatial and temporal location.
Therefore, if we rely solely on the knowledge and capabilities of LLMs to conduct such simulations without incorporating modeling of the operational laws and constraints of the real world, the simulation results are likely to be influenced by the "hallucinations" of LLMs~\cite{huang2023survey}, resulting in outcomes that diverge from actual realities.
To address this issue, we need to provide a realistic and reliable environment for simulating social agents.
The environment should include the following features:
\begin{itemize}
    \item Appropriate modeling of real-world operational principles to reflect physical constraints and costs, and provide feedback on behaviors;
    \item Environmental data sourced from the real world or aligned with real-world principles;
    \item Interfaces to enable interaction with agents.
\end{itemize}
Such an environment will serve as a virtual mapping of the objective aspects of the world in social simulations, enabling the design of social agents to focus solely on subjective human behavioral logic.
By offloading tasks such as numerical computations, where LLMs cannot guarantee absolute accuracy, this approach simplifies agent design and allows researchers to concentrate on core objectives.

% 整个环境被分为3个空间。城市空间中构建了支持移动模拟的城市道路网并包含了AOI、POI等元素,实现了常见的开车、步行、乘坐公交、乘坐出租车等出行方式,能够为智能体提供真实的位置变化反馈,以及不同出行方式对应的时间和金钱的代价。社交空间在智能体社会网络的基础上,提供了面对面的社交行为支持与社交媒体的抽象。社交空间中的一个关键且真实的设计是监管者,监管者将读取社交媒体中消息,根据算法规律进行过滤并支持对特定的用户或传播进行封禁。经济空间还原宏观经济学中的基本元素,以账本为实现建模了人、公司、政府、银行的经济行为,包括雇佣劳动、消费、税收、利息等,并提供了用于统计经济指标的统计局。

In alignment with this objective, we encode expert knowledge to construct a real-world societal environment as depicted in Figure~\ref{fig:env}, designed to support the simulation of mobility behaviors, social behaviors, and economic behaviors.
The entire environment is divided into three spaces.
The urban space constructs a city road network supporting mobility simulation and incorporates elements such as Area of Interest (AOI) and Point of Interest (POI).
It implements common transportation modes including driving, walking, public transit, and taxi services, providing agents with realistic positional feedback as well as temporal and monetary costs associated with different travel choices.
The social space builds upon agents' social networks, offering support for offline and online social interactions.
A critical and authentic design feature in this space is the supervisor, which monitors social media content, filters messages based on algorithmic rules, and enforces bans on specific users or connections when necessary.
The economic space reconstructs fundamental elements of macroeconomics, modeling economic behaviors of individuals, firms, governments, and banks through the implementation of account books.
These behaviors encompass employment, consumption, taxation, interest mechanisms, while a dedicated statistical bureau is established to monitor economic indicators like GDP.
The following subsections will elaborate on the corresponding environmental spaces for the agent behaviors respectively.

% 以下各小节的基本逻辑(一部分一段)
% 1. 总起:对应3.3的介绍,过渡下来,提出要构建XX模拟环境,以支撑XXX需求
% 2. 介绍如何建模、以及运行逻辑
% 3. 介绍有哪些数据,数据的来源
% 4. 提供的接口
% 5. 实现方式
% 6. 总结

\subsection{Urban Space}

% 1. 城市空间环境的重要性
To address the needs of social agents for moving and interacting with different places, accurate modeling of urban space is essential. 
The urban space must capture both the physical movement pattern of individuals and their interactions with diverse urban locations. 

% 2. 城市空间建模方法
Inspired by traffic simulation platforms such as SUMO~\cite{behrisch2011sumo} and CityFlow~\cite{zhang2019cityflow}, also leveraging the spatial abstraction schemas of OpenStreetMap\footnote{\url{https://openstreetmap.org/}}, the urban space is structured into two interdependent layers, the static infrastructure attributes and dynamic mobility behaviors. 

The static attribute layer includes road networks, defined by lanes, roads and junctions to encode traffic accessibility, as well as functional zones, which are Areas of Interest (AOIs) and Points of Interest (POIs). 
AOIs delineate regions with specific purposes, such as residential neighborhoods or commercial districts, while POIs represent granular interaction targets like retail stores.

The dynamic behavior layer extends this static foundation by simulating multi-modal mobility through a discrete time-stepping mechanism. 
Individual movements, including positions, speeds, and accelerations are updated dynamically according to kinematic principles and predefined rules.
Operational logic begins with agents formulating movement intentions based on their internal needs and goals, then these intentions are translated into specific instructions guiding individuals' movements within the space, which include driving, walking, taking the bus, or taking a taxi.
For all means of transportation, path-planning algorithms generate optimal routes.
Driving follows the IDM model~\cite{treiber2000congested} for acceleration and the MOBIL model~\cite{kesting2007general,feng2021intelligent} for lane-changing decisions.
Pedestrians navigate sidewalks at a constant speed and follow the traffic signals at junctions to avoid collisions with vehicles.
Buses operate on fixed schedules, while passengers conduct the processes of boarding, alighting, and transferring. 
For taxis, a global dispatch system simulates sending the nearest available taxi to respond to ride requests, ensuring efficient service and minimal wait times for passengers.


% 3. 数据来源与软件实现
For the simulation environment to function accurately, we apply rich data sources. 
Road networks and AOIs are extracted from OpenStreetMap\footnote{\url{https://openstreetmap.org/}}, undergoing topological simplification to produce structured representations. 
POI data, acquired via API from SafeGraph\footnote{\url{https://www.safegraph.com/}}.

We implement Python-based APIs to bridge the simulation space and agents,providing  bidirectional interaction capabilities. 
Configuration interfaces allow for initializing agent positions, assign travel plans (e.g., destinations and transportation modes), and reset simulation states. 
Query interfaces enable real-time monitoring of agent kinematics status and simulation metadata such as simulation timestamps. 

% 4. 总结
By harmonizing static urban infrastructure, dynamic mobility behaviors, and multi-source geospatial data, our urban space establishes a high-fidelity decision-making sandbox for agents. 



\subsection{Social Space}

% 以下各小节的基本逻辑(一部分一段)
% 1. 总起:对应3.3的介绍,过渡下来,提出要构建XX模拟环境,以支撑XXX需求
% 2. 介绍如何建模、以及运行逻辑
% 3. 介绍有哪些数据,数据的来源
% 4. 提供的接口
% 5. 实现方式
% 6. 总结

% 社交行为是构建智能体社会的前置条件,社会行为的建模与模拟将允许智能体之间相互影响、相互协作,产生更丰富和真实的社会现象。 因此,社会环境中增加社交空间是尤为重要的。社交空间中的主要组成部分是社交网络,这由用户载入。社交网络建模了人与人之间的关系,包含了每个智能体与其他智能体的连接关系与连接强度。这将用于智能体评估社交的选择对象。社交网络与其中的关系和强度在模拟过程中是可变的。基于社交网络,社交空间同时包含了线上社交与线下社交。尽管智能体设计时主要关注线上社交行为,但基于空间位置临近关系的线下社交行为依然是在真实环境的构建中不可或缺的部分。对于线上社交,为了尽可能模拟现实世界中的社交媒体运行逻辑,我们还引入了监管者的概念,监管者将识别线上社交消息内容,根据用户指定的算法或规则过滤消息,并支持对特定用户或连接的封禁,从而模拟社交媒体平台对信息传播的干预过程。
Social behavior is a prerequisite for constructing an agent society.
The occurrence of social behaviors requires the support of an authentic social environment.
The social environment provides management of social relationships, as well as modeling and simulation of social behaviors, which will enable mutual influence and collaboration among agents, generating richer and more authentic social phenomena.

Therefore, the incorporation of the social space within the societal environment is particularly crucial.
The primary component of the social space is the social network, which is provided and loaded by users.
Social networks model relationships between individuals, encompassing the connections and connection strengths between each agent and others.
This network will be used by agents to evaluate potential social interaction targets.
Both the relationships and connection strengths within social networks are mutable during simulations.
Based on social networks, social spaces encompass both online and offline interactions.
Although agent design primarily focuses on online social behaviors, offline interactions based on spatial proximity remain an indispensable component in constructing realistic environments.
For online social interactions, to realistically simulate the operational logic of social media platforms, we also introduce the concept of the supervisor.
The supervisor will identify content in online social messages, filter messages according to user-specified algorithms or rules, and support the blocking of specific users or connections, thereby simulating the intervention process of social media platforms in information propagation.

% 在实现上,社会空间中的社交网络存储为智能体的数据项,线下社交与线上社交均简化为通过智能体消息系统向对应的目标发送消息。监管者则实现为消息发送前的预处理中间级,并提供一个中心化的程序用于集中处理消息并完成规则与算法的更新与下发。
In implementation, the social network is stored as data items within agents.
Both offline and online social interactions are simplified into sending messages to specific targets through the agent message system which will be introduced in Section~\ref{sec:sim:mqtt}.
The supervisor is implemented as preprocessing middleware before message transmission, and a centralized program is provided to handle message processing collectively for updating rules and algorithms.

In summary, the social space not only supports the simulation of realistic social interactions between agents but also establishes intervention capabilities over social propagation within agent societies.
This framework will serve as a crucial foundation for conducting research on real-world social propagation phenomena using LLM-driven social agents.

\subsection{Economic Space}
The economic space includes the modeling of several key elements in the macroeconomics~\cite{wolf2013multi, li2024econagent}.
Specifically, firms convert the labor input of social agents into goods production and pay the corresponding wages to the agents.
Furthermore, firms adjust the wages of agents and goods price flexibly based on the supply and demand relationships in the consumption market. 
The government levies income tax on agents' earnings according to specified tax rates.
The banks pay interest to agents based on their savings each year, with the interest rate adaptively adjusted according to the Taylor Rule~\cite{wolf2013multi}.
The National Bureau of Statistics regularly compiles macroeconomic indicators, such as real GDP, average working hours per person, and per capita consumption levels.

% income, expenditure, tax, interest, statistic

% 与李念的整合
Building upon the modeling of the four key economic entities—firms, agents, the government, and banks—the economic simulator further captures the dynamic processes and interactions that drive the functioning of a realistic economic system. By integrating income generation, expenditure, savings, taxation, and policy-driven adjustments, the simulator provides a comprehensive representation of economic cycles.

Agents are the fundamental economic participants, generating income through labor in firms. This income is subject to taxation, with a portion deducted based on a progressive tax structure, while the remaining disposable income is allocated between consumption and savings. Agents use their disposable income to purchase goods and services, thereby fueling market demand. The funds allocated to savings are deposited into banks, where they accrue interest, influencing future consumption and investment decisions.

Firms act as producers in the economy, utilizing labor from agents to generate production output. They pay wages to agents, creating a cyclical flow of income within the system. Firms dynamically adjust goods prices and wages in response to market supply and demand, ensuring equilibrium in the consumption market. Their revenue comes from the sales of goods, which is reinvested into further production or held as retained earnings for future expansion. The government collects tax revenue from agents based on their earnings and redistributes financial resources to regulate economic activity. It influences the economy through fiscal policy, adjusting tax rates to manage income distribution and public sector financing. These funds may be directed toward public expenditures, which are not explicitly modeled in this simulator but could represent infrastructure, social programs, or government services. Banks function as financial intermediaries, receiving savings deposits from agents and providing them with interest payments. The interest rate is dynamically adjusted according to the Taylor Rule, incorporating monetary policy mechanisms into the simulation. This impacts agents’ saving and spending behavior, as higher interest rates encourage savings while lower rates stimulate consumption. Banks also serve as liquidity providers, ensuring the efficient allocation of financial resources within the economy.

The National Bureau of Statistics (NBS) compiles and analyzes macroeconomic indicators to monitor economic performance. It collects data on real GDP, income distribution, total tax revenue, per capita consumption, and average interest earnings. These aggregated statistics offer insights into systemic trends and policy effectiveness, enabling the evaluation of economic stability and the long-term impact of market dynamics. By structuring the simulation as an economic settlement system, where each entity interacts through well-defined financial flows, this framework achieves a holistic representation of economic operations. It serves as a valuable tool for analyzing the relationships between micro-level decision-making and macroeconomic trends, providing insights into market behaviors, policy interventions, and the overall functioning of economic systems.

% Discussion & Conclusion
While the economic simulator successfully models key economic flows, it does not explicitly capture critical aspects such as the goods market and the labor market. These omissions limit the realism of the model and represent areas for future refinement. The goods market is simplified by assuming firms adjust prices based on aggregate demand, but it does not account for detailed supply and demand dynamics, competition, or market shocks. Similarly, the labor market is abstracted, with agents receiving wages without modeling unemployment or the negotiation processes between workers and firms. Including these elements could enhance the model’s ability to simulate real-world economic fluctuations. Despite these limitations, the simulator provides valuable insights into the interactions between agents, firms, the government, and banks. Future improvements could focus on integrating more detailed models of the goods and labor markets, helping to better replicate complex economic systems and improving policy analysis. In conclusion, while the current model is a simplified representation, it offers a strong foundation for exploring economic interactions and can be further enhanced by incorporating missing market dynamics for more accurate predictions.

% Agent、
% 账本角度,结算系统,模拟经济的运转过程,工作和收入挂钩,储蓄和消费
% 统计局,
% 提供的接口
% 加个Disscussion,商品市场、劳动力市场,总结一下

\section{Large-scale Social Simulation Engine}\label{sec:engine}

\subsection{Overview}

% 逻辑
% 【总起】尽管社会模拟器看起来像是由多智能体系统 with 工具调用(环境),但真实社会中人的独立思考决策与语言交流驱动的协作方式促使我们重新思考大规模智能体系统架构设计。
% 【现状】现有的多智能体框架一般基于智能体间协作,通过构建处理流程图的方式来规范各个角色的执行顺序,例如......。
% 【问题】但在真实社会中,每个人的行为决策都是自身基于当前的记忆、想法与环境的约束做出的,并不总是依赖来自其他智能体或环境的特定输入。
% 【技术挑战】因此,如何在系统设计上更好地还原这种社会模拟中的“异步”现象,并根据这一特点实现更大规模的智能体执行,是系统架构设计的关键。
% 【解决方案】参考现实社会的基本逻辑,我们将每个智能体视为独立的模拟单元,相互之间不存在任何显式的依赖关系,智能体之间通过消息机制完成信息交换并互相影响。
% 【技术实现】分布式计算 -> ray,高性能消息传输-> MQTT,平衡并行度与有限的IO资源 -> group,还提供了其他有用的工具,包括xxx,xxx,xx
% 【社会实验支持】xxx
% 图:各个组件之间的逻辑关系

% TODO: 下面各个小节分别叫啥

Although the large-scale social simulator introduced in this paper may appear as a simple combination of LLM multi-agent systems (social agents) and tool call (environment), the reality of human society characterized by independent thinking in decision-making and collaboration driven by language communication promotes us to fundamentally rethink the system architecture design and implementation of the large-scale social simulation engine.
Existing multi-agent execution frameworks such as CAMEL~\cite{li2023camel} and AgentScope~\cite{gao2024agentscope} typically take inter-agent collaboration as the foundational principle of system architecture design, constructing Standard Operating Procedures (SOPs) through message-passing processes among agents to determine the sequence of agent execution.
Such frameworks are particularly well-suited for multi-agent execution scenarios with well-defined agent execution sequences, as exemplified by programming tasks~\cite{hong2023metagpt,qian2024chatdev} and conversational games~\cite{xu2023exploring}, as they can significantly simplify complex agent interaction processes.
However, in real-world contexts, individual behavioral decision-making emerges from the autonomous integration of current memory, cognitive states, and environmental constraints, rather than being strictly contingent upon specific inputs from other agents or environments.
Therefore, a pivotal challenge in system architecture design lies in faithfully simulating this "asynchronous" phenomenon within the large-scale social simulator, while strategically leveraging such intrinsic behavioral patterns to optimize simulation execution efficiency.

As a solution to the aforementioned challenge, we draw inspiration from the operational logic of the real world by treating each agent as an independent simulation unit.
There are no explicit dependencies or execution orders between agents.
Instead, they exchange information and mutually influence each other through a messaging system.
For the parallel execution of independent simulation units, to fully leverage the multi-core computing capabilities of modern computer systems and distributed computing paradigms for horizontal scaling of simulation scale, we adopt the highly mature Ray framework~\cite{moritz2018ray} to implement distributed computing and conceal I/O latency through Python's asyncio mechanism.
As the simulation scale increases, we identify that TCP port resources become a bottleneck, and excessive reliance on inter-process communication to coordinate the entire system leads to decreased execution efficiency.
Therefore, we introduce an intermediate structure named agent group to enable multiple agents to operate within a single process, thereby balancing communication costs with parallel acceleration while allowing connection reuse for network calls such as LLM API calls.
For the messaging system supporting inter-agent information exchange, it needs to support massive concurrent connections, high-throughput and reliable message transmission, though being latency-insensitive.
This characteristic closely resembles the Internet of Things (IoT) scenarios where applications must handle message delivery across millions of devices.
Inspired by this similarity, we have introduced MQTT\footnote{\url{https://mqtt.org/}}, the communication protocol that has achieved tremendous success in IoT communications, to construct our agent messaging system.
Following best practices from existing agent execution frameworks, we provide comprehensive utilities including multiple LLM API adapters, a retry mechanism,a JSON parser, a metric recorder, and diverse logging capabilities including both local file output and database storage.
Leveraging these logged processes, we develop real-time interactive visualization interfaces.
Furthermore, specifically designed for social experimentation requirements, we implement a specialized toolbox including interviews, surveys, and interventions.

In the following content, we will first introduce the whole system architecture in Section~\ref{sec:sim:arch}.
We then dive into the key designs including group-based distributed execution in Section~\ref{sec:sim:group} and MQTT-powered agent messaging system in Section~\ref{sec:sim:mqtt}.
The utilities and toolbox for social experiments will be discussed in Section~\ref{sec:sim:util} and Section~\ref{sec:sim:exp} respectively.

\subsection{System Architecture}\label{sec:sim:arch}

\begin{figure}[ht]
\centering
\includegraphics[width=\textwidth]{Figure/arch.pdf}
\caption{System architecture of the large-scale social simulation engine.}
\label{fig:arch}
\end{figure}

% 架构,介绍“实验-run”的划分,各个组件的功能和所属的层级

% 挑战和技术问题、设计

% 为了避免不必要的重复开发并享受开源社区进步的收益,大型社会模拟器的系统架构主要采用先进的、成熟的开源软件或库进行构建。在以下的讨论中,我们将进行一次社会模拟成为一次实验。如图1所示,整个架构主要分为由所有实验共享的共享服务、每个实验独占的计算任务以及可选的GUI组成。
% 共享服务部分包含以下几个服务:
% - LLM API:大模型是整个模拟器最重要的组件,是智能体的灵魂。对模拟器来说,大模型通过API提供标准的“请求-响应”过程,以处理智能体prompt所描述的任务。LLM API可以使用公开的服务如OpenAI、DeepSeek等,也可以使用本地推理引擎进行部署如vllm、ollama。
% - MQTT Server: 本架构中使用MQTT这一高性能物联网协议进行智能体之间的消息传输以模拟真实人类社会中基于语言的相互影响与协作方式。MQTT服务器提供了将消息以符合协议要求的方式投递到客户端的功能。这里我们选择的是emqx这一个高性能mqtt服务器。
% - Database: 数据库在本架构中只用于存储模拟结果以便后续分析或可视化。我们选择了知名的PostgreSQL以使用其独有的高性能批量写入SQL命令COPY FROM来保证数据存储效率。
% - Metric Recorder: 对模拟过程的特定指标记录有利于研究人员对比不同实验的实验结果以发现有价值的科学结论。为了方便研究人员的协作,我们选择使用具有中心服务器的mlflow而不是基于本地存储的指标记录工具如tensorboard。

To avoid unnecessary redundant development and leverage the advancements from the open-source community, the system architecture of the large-scale social simulation engine is primarily constructed using advanced and mature open-source software or libraries.
As shown in Figure~\ref{fig:arch}, the overall system architecture consists of shared services common to all social simulation experiments, simulation tasks corresponding to each experiment, and an optional GUI component.

The shared services include the following components:
\begin{itemize}
    \item \textbf{LLM API:}
    LLMs serve as the most critical component of the simulator, acting as the "soul" of agents.
    For the simulation engine, LLMs provide a standard "request-response" process through APIs to handle tasks described in agent prompts.
    The LLM API can utilize public services like OpenAI\footnote{\url{https://platform.openai.com/docs/overview}} or DeepSeek\footnote{\url{https://platform.deepseek.com/}}, or be deployed through local inference engines such as vllm~\cite{kwon2023efficient} and ollama\footnote{\url{https://ollama.com/}}.
    \item \textbf{MQTT Server:}
    The architecture employs the high-performance IoT protocol MQTT for inter-agent message transmission, simulating real-world human language interactions and collaboration patterns.
    The MQTT server enables protocol-compliant message delivery to clients.
    We select a high-performance MQTT server named emqx\footnote{\url{https://www.emqx.com/en}} for this purpose.
    \item \textbf{Database:}
    The database in the architecture is solely used for storing simulation results for subsequent analysis or visualization.
    We choose PostgreSQL\footnote{\url{https://www.postgresql.org/}} for its unique high-performance batch writing capability through the SQL command \texttt{COPY FROM}, ensuring efficient data storage.
    \item \textbf{Metric Recorder:}
    Recording specific metrics during simulations enables researchers to compare experimental results across different studies and derive valuable scientific insights.
    To facilitate research collaboration, we opt for mlflow\footnote{\url{https://mlflow.org/}} with centralized server capabilities, rather than local storage-based metric recording tools like tensorboard\footnote{\url{https://www.tensorflow.org/}}.
\end{itemize}

The primary purpose of the large-scale social simulation engine is to execute social simulation experiments, which consist of a set of computational tasks comprising environment simulation and agent execution.
In implementation, an experiment corresponds to an \texttt{Agent Simulation} object. This object will create and manage environment simulators through subprocess mechanisms, while utilizing the Ray framework to create multiple agent groups that execute agents through multi-process-based distributed computing.
According to Ray framework design, each agent group functions as a Ray actor operating within a single process.
The Ray framework enables managed Ray actors to work across different machines.
By adding other machines to the head node during Ray cluster initialization, distributed computing can be easily achieved to horizontally scale computational resources for social simulation tasks.
Within each agent group, clients connecting to the shared services and the environment simulator are initialized, enabling multiple agents to work concurrently using these client connections.
Different experiments will share all shared services while utilizing distinct Ray clusters and environment simulators to prevent mutual interference.
The GUI with a backend and a frontend connects to the database and MQTT server to enable the visualization of simulation results, and allows users to directly interact with simulated agents by conducting dialogues or sending questionnaires.

In conclusion, the system architecture integrates multiple cutting-edge open-source softwares to deliver comprehensive capabilities for social simulation and enables researchers to focus exclusively on social agent design, including distributed computing, LLMs, message transmission, data storage, and metric management.

\subsection{Group-based Distributed Execution}\label{sec:sim:group}

% 表示从串行处理到并行处理到async+并行处理的提升

\begin{figure}[ht]
\centering
\includegraphics[width=\textwidth]{Figure/async.pdf}
\caption{Asynchronous multi-process parallel execution using Ray and asyncio.}
\label{fig:async}
\end{figure}

% 逻辑
% 挑战与主要考虑
% 解决方案设计
% More discussion
% 总结

% 根据上面关于系统架构的介绍,我们将多个智能体聚合为一组,称为agent group,并使用一个进程执行每组智能体,以达到分布式计算加速的效果。这一设计是为了解决有限的TCP端口资源与庞大的智能体数量之间的矛盾。具体来说,如果将每个个体视为一个进程并独立连接到共享服务以模拟真实世界中人的自主决策,大量的TCP连接将消耗掉MQTT服务器的TCP端口资源(上限为65535个),从而导致出现TCP端口资源不足而失败。因此,如何进行连接复用以降低TCP端口资源的占用并保证多智能体的独立执行是智能体执行所需要考虑的关键问题。

Based on the above description of the system architecture, we aggregate multiple agents into a group called an agent group, and use a single process to execute each group of agents to achieve the effect of distributed computing acceleration.
This design addresses the contradiction between limited TCP port resources and the massive number of agents.
Specifically, if each individual agent were treated as a separate process and independently connected to shared services to simulate autonomous human decision-making in the real world, the large number of TCP connections would exhaust the TCP port resources of the MQTT server, the database, and the metric recorder (with an upper limit of 65,535 ports), leading to failures due to insufficient TCP port resources.
Therefore, how to implement connection reuse to reduce TCP port resource consumption while ensuring independent execution of multiple agents constitutes a critical issue that needs to be addressed in agent execution.

% 针对这一问题,我们将智能体平均分为多个智能体组,并为每个组配置一个LLM API客户端、一个MQTT客户端、一个环境客户端、一个数据库客户端、一个指标记录者客户端。所有的客户端都采用异步调用的方式实现,并支持并行调用。由于大模型驱动的社会模拟是IO密集型的处理任务,其主要用时在LLM的调用、环境的调用上,因此通过由asyncio提供的异步IO能力将允许多个智能体并行发出LLM请求并充分利用CPU处理Agent设计中的计算任务,有效避免等待请求返回导致的时间浪费。异步调用的连接复用方式也不引入对智能体执行顺序的特定限制,保证了智能体之间的独立性。通过构建智能体组的方式,整个系统所需的TCP端口数可以降低至智能体组数的倍数(取决于是否使用可选的服务如指标记录),避免TCP端口耗尽带来的问题。

To address this issue, we evenly distribute agents into multiple agent groups, each configured with an LLM API client, an MQTT client, an environment client, a database client, and a metrics recorder client.
All clients are implemented using asynchronous calls and support parallel execution.
Since LLM-driven social simulations are I/O-intensive tasks, primarily consuming time in LLM calls and environment interactions.
Leveraging the asynchronous I/O capabilities provided by asyncio and multi-process parallel execution powered by Ray shown in Figure~\ref{fig:async}, the engine allows multiple agents to concurrently send LLM requests while fully utilizing CPU resources for computational tasks in agent design like running gravity models, effectively avoiding time waste caused by waiting for LLM responses.
The asynchronous approach with connection reuse does not impose specific constraints on agent execution order, ensuring independence among agents.
By organizing agents into groups, the total number of TCP ports required by the system can be reduced to a multiple of the number of agent groups (depending on optional services such as metrics recording), thereby preventing issues caused by TCP port exhaustion.
However, immutable fixed-number grouping would result in the overall system efficiency being constrained by the slowest group.
Therefore, adaptive load balancing and dynamic scheduling across groups represent an important direction for future research.

In summary, by combining group-based asynchronous and parallel execution with distributed implementation to enhance simulation efficiency, we successfully address the critical issue of execution failures caused by port exhaustion.

\subsection{MQTT-powered Agent Messaging System}\label{sec:sim:mqtt}

\begin{figure}[ht]
\centering
\includegraphics[width=\textwidth]{Figure/mqtt.pdf}
\caption{Overview of MQTT-powered agent messaging system.}
\label{fig:mqtt}
\end{figure}

% 智能体之间的通信是基于大模型多智能体构建社会模拟的必要环节。文本消息在智能体之间的传输用于模拟真实世界中人与人之间基于语言沟通的交流与合作,将促使群体行为的涌现从而进一步逼近真实世界运转规律。进一步扩展,构建能够连接智能体的消息系统将允许用户或外部程序直接访问、干预智能体的行为,从而开展更丰富的社会实验或交互式应用。

% 需求和目的
Communication between agents is an essential component in constructing social simulations based on LLM-powered multi-agent systems.
The transmission of textual messages among agents simulates real-world human communication and collaboration through language, which will facilitate the emergence of group behaviors and thereby further approximate the operational laws of the real world.
Expanding on this, developing a messaging system that connects agents will enable users or external programs to directly access and intervene in the agents' behaviors.
This will support more sophisticated social experiments and interactive applications.

% 技术方案
% 为了实现上述智能体消息系统,我们需要能够支持向数十万特定ID投递消息的消息系统。在物联网领域,MQTT是用于解决数百万物联网设备与控制中心相互通信的协议,采用发布/订阅的设计方式,发布方将消息发送到指定的ID中,而订阅方则监听指定ID或指定前缀。MQTT通过轻量级的数据包结构设计来匹配物联网低带宽条件,并在较低的资源占用情况下实现了对消息可靠的承诺。尽管这一协议是设计用于物联网设备连接,但十分契合大型社会模拟器中智能体通信的需求,因此我们使用该协议实现智能体通信系统。
To achieve the aforementioned agent messaging system, we need a messaging system capable of delivering messages to hundreds of thousands of specific agents by IDs.
In the IoT domain, MQTT is a protocol designed to enable communication between millions of IoT devices and control centers, employing a publish/subscribe architecture.
Publishers send messages to specified IDs, while subscribers monitor specific IDs or designated prefixes to receive messages.
MQTT utilizes a lightweight packet structure tailored for low-bandwidth IoT environments, delivering reliable messaging with minimal resource consumption.
Although originally designed for IoT device connectivity, this protocol aligns perfectly with the communication requirements of agents in the large-scale social simulator.
We therefore adopt this protocol to implement our agent messaging system as shown in Figure~\ref{fig:mqtt}.
During implementation, we designate the following topics and their corresponding message meanings:
\begin{itemize}
    \item \texttt{exps/<exp\_uuid>/agents/<agent\_uuid>/agent-chat}: Used for sending messages from one agent to the target agent within the experiment.
    \item \texttt{exps/<exp\_uuid>/agents/<agent\_uuid>/user-chat}: Used for users to send chat messages to the target agent within the experiment via the GUI.
    \item \texttt{exps/<exp\_uuid>/agents/<agent\_uuid>/user-survey}: Used for users to send structured JSON-formatted surveys to the target agent within the experiment via the GUI.
\end{itemize}
Under this topic configuration, each agent only needs to subscribe to messages prefixed with \texttt{exps/<exp\_uuid>/agents/<agent\_uuid>/}.
This approach reduces development costs while maintaining compatibility for future extensions to the messaging system's functionality.

% 结论
In summary, by integrating the advanced IoT communication protocol MQTT, we achieve low-cost development to simultaneously connect hundreds of thousands of agents while ensuring reliable agent message transmission.
This solution also enables user input through GUIs and supports future functional extensions, thereby filling a crucial gap in the architecture of the large-scale social simulation engine.

\subsection{Utilities}\label{sec:sim:util}

In addition to the aforementioned key technical designs for social agents, we also provide a rich set of utilities to facilitate the development of social agents.
The design philosophy of most of these utilities is primarily inspired by AgentScope~\cite{gao2024agentscope}.

\textbf{LLM API Adapter.}
We implement calls to OpenAI API-compatible LLMs through the \texttt{openai} python library\footnote{\url{https://pypi.org/project/openai/}}, including OpenAI, DeepSeek, Qwen\footnote{\url{https://bailian.console.aliyun.com/}}, etc.
Additionally, we also support calls to ChatGLM\footnote{\url{https://bigmodel.cn/}}.
Through the LLM API adapter, we allow users to freely select their preferred or partnered LLMs for inference by modifying configurations, which enhances the system's flexibility and compatibility.

\textbf{Retry Mechanism.}
To prevent abnormal results returned by the LLM API from affecting experiments, the system incorporates a retry mechanism.
When invoking the LLM API, if erroneous responses are detected, the system will automatically reinitiate the call. The default number of retries is set to 3.

\textbf{JSON Parser.}
Since prompts often require LLMs to return responses in JSON format to facilitate parsing into program-processable results, we develop a JSON parser.
This parser automatically identifies JSON code blocks in responses, removes Markdown code block delimiters (prefix and suffix), and converts the content into Python objects.

\textbf{Metric Recorder.}
To assist researchers in recording various statistical metrics during experiments, such as total GDP and agent state averages, we adapt the mlflow API and implement a parallel-safe metric logging utility class and functions.

\textbf{Logging and Saving.}
Logging and saving simulation processes and results serve as the foundation driving subsequent data analysis and visualization.
Saving as much data as possible will facilitate richer and more profound research insights.
Accordingly, we design both local file storage using the AVRO format \footnote{\url{https://avro.apache.org/}} and PostgreSQL database as online storage.
Both storage approaches employ similar schemas to archive social agent profiles, agent states during simulations, thoughts and dialogues, and survey outcomes.
Additionally, experimental metadata including IDs, names, durations, configurations, and error messages is systematically recorded.

\textbf{GUI.}
To help people directly observe the behavior of social agents in environments and allow users to engage in direct conversations or surveys with the agents, we develop a GUI program.
The GUI program includes functionalities for experiment management, survey administration, and real-time monitoring or playback of experiments.
During real-time monitoring, users can interact with agents through instant conversations or send surveys, with these communications being transmitted via the agent messaging system while awaiting responses.
Additionally, during both real-time monitoring and playback, users can view agent-related records stored in a PostgreSQL database, including their locations, profiles, status history, thought and dialogue history, and survey response histories.
We aim for this GUI to help users build intuitions about agent societies and facilitate deeper analysis and applications.


\subsection{Toolbox for Social Experiments}\label{sec:sim:exp}

In the realm of social sciences, various research methods are employed to study human behavior, motivations, and responses in different contexts. In real-world settings, it is often difficult to find controlled environments for conducting social experiments that mirror the complexity of human interactions. This is where interventions come into play—creating specific social experimental conditions that allow researchers to simulate and manipulate real-life scenarios. Besides, two widely used methods are interviews and surveys, both of which allow researchers to collect data from individuals, explore their opinions, and understand the underlying psychological factors influencing their actions. These methods are vital for generating insights into how people think, feel, and act in specific situations.  Large-scale agent-based simulations provide a powerful tool for addressing these challenges, enabling the design of controlled experiments with a high degree of realism.
This section details the tools available for conducting interventions, interviews, and surveys with agents in a social experiment. These tools offer flexibility in data collection and manipulation, supporting the design and execution of robust social experiments.

\textbf{Intervention.} Intervention refers to the manipulation of an agent’s behavior or state to observe how changes influence its actions, thoughts, and emotional responses. Interventions are crucial for setting up experimental conditions in social experiments. There are three primary types of intervention in our system:

\begin{itemize}
    \item \textbf{Agent Configuration}: This type of intervention involves directly modifying the internal settings of the agent before the simulation starts. These settings may include altering an agent's personality traits, goals, or preferences. Since this intervention occurs before the simulation, it ensures that the agent’s behavior aligns with the experimental conditions right from the start.
    \item \textbf{State Manipulation}: This intervention occurs during the simulation and allows researchers to modify the agent’s current state. By altering an agent's profile, mood, or ideas, researchers can influence its behavior. For instance, modifying the agent’s emotions can impact its decision-making and social interactions.
    \item \textbf{Message Notification}: This method involves sending a text message to the agent, triggering a response. For example, a message such as "Severe weather changes expected, a hurricane is coming" could be used to observe how the agent adjusts its plans or behavior in response to external threats. This type of intervention can be introduced at any point during the simulation, offering flexibility in creating different experimental conditions.
\end{itemize}

The intervention process can be summarized as:
\[
\text{Pre-Simulation Configuration} \xrightarrow{\text{Agent Settings}} \text{Agent Behavior Start}
\]
\[
\text{During Simulation} \xrightarrow{\text{Memory Manipulation}} \text{Behavior Adjustments}
\]
\[
\text{During Simulation} \xrightarrow{\text{Message Notification}} \text{Behavior Modification}
\]
These intervention techniques allow for dynamic and flexible modifications of the agent’s behavior, providing valuable insights into the impact of specific changes on social interactions and decision-making.

\textbf{Interview and Survey.} An interview is a process of one-on-one or group-based questioning and answering, typically used to gather detailed, qualitative information from participants. In our platform, users can directly communicate with agents through either a front-end interface or programmatically via code. The system uses MQTT to distribute the user’s questions to the relevant agents. The agent then answers these questions by processing both its internal state and the surrounding environmental context. Importantly, this process is designed so that the agent can respond without interrupting its ongoing actions. This allows for real-time interaction while maintaining the flow of the agent’s behavior. The interaction flow is depicted as:
\[
\text{Question} \xrightarrow{\text{MQTT}} \text{Agent Processing} \xrightarrow{\text{Answer}} \text{User Response}
\]
This ensures that the agent can participate in interviews seamlessly while continuing its primary tasks and goals.

A survey is a structured form of data collection, where a series of interview questions are combined based on a specific set of rules. These rules include response formats (e.g., multiple-choice, ranking) and the unique design elements determined by the survey creator (e.g., question order). Surveys are typically used to gather quantitative data across a larger sample, offering a broader perspective on trends or patterns.
In our system, the structured survey is distributed to agents via MQTT, just like interviews. However, the primary difference is that the agent’s responses follow a predefined structure, ensuring consistent data collection across multiple agents. The agent processes the survey questions sequentially, answering them from top to bottom after analyzing the format and response rules. This structured data is then compiled into a format that is easy for the user to analyze. The survey response process is modeled as:
\[
\text{Survey} \xrightarrow{\text{MQTT}} \text{Agent Parsing and Responding} \xrightarrow{\text{Structured Answer}} \text{Data Processing}
\]
This ensures that data collection is organized, reliable, and easy to process for social experiment analysis.

The ability to conduct interventions, interviews, and surveys within our platform provides a powerful toolkit for researchers conducting social experiments. These tools offer a structured approach to data collection and behavioral modification, making it possible to simulate real-world social conditions in a controlled environment. The flexibility to manipulate an agent’s settings, memory, and responses in real-time ensures that a wide variety of social experiment scenarios can be tested, from understanding individual behaviors to studying collective dynamics. This makes large-scale agent simulations an invaluable resource for conducting complex social science research.




\section{Experiment}
\label{subsec:experiments}

\begin{figure*}[t!]
    \centering
    \includegraphics[width=\textwidth]{figure/visualization.pdf} 
        \captionof{figure}{Examples of generated videos by \sys{} and original implementation on CogVideoX-v1.5-I2V and HunyuanVideo-T2V. We showcase four different scenarios: (a) minor scene changes, (b) significant scene changes, (c) rare object interactions, and (d) frequent object interactions. \sys{} produces videos highly consistent with the originals in all cases, maintaining high visual quality.}
        \label{fig:SVG-visualization} 
\end{figure*}

\subsection{Setup}
\label{subsec:experiment_setup}

\textbf{Models.} We evaluate \sys{} on open-sourced state-of-the-art video generation models including CogVideoX-v1.5-I2V, CogVideoX-v1.5-T2V, and HunyuanVideo-T2V to generate $720$p resolution videos. After 3D VAE, CogVideoX-v1.5 consumes $11$ frames with $4080$ tokens per frame in \attn{}, while HunyuanVideo works on $33$ frames with $3600$ tokens per frame.


\textbf{Metrics.} We assess the quality of the generated videos using the following metrics. We use Peak Signal-to-Noise Ratio (PSNR), Learned Perceptual Image Patch Similarity (LPIPS)~\citep{zhang2018perceptual}, Structural Similarity Index Measure (SSIM) to evaluate the generated video's similarity, and use VBench Score~\citep{huang2023vbenchcomprehensivebenchmarksuite} to evaluate the video quality, following common practices in community~\citep{5596999,zhao2024pab,li2024svdquant,li2024distrifusion}. Specifically, we report the imaging quality and subject consistency metrics, represented by VBench-1 and VBench-2 in our table.

\textbf{Datasets.} For CogVideoX-v1.5, we generate video using the VBench dataset after prompt optimization, as suggested by CogVideoX~\cite{yang2024cogvideox}. 
For HunyuanVideo, we benchmark our method using the prompt in Penguin Video Benchmark released by HunyuanVideo~\cite{kong2024hunyuanvideo}.

% We follow standard practices in evaluating video generation models.
% Specifically, we assess the quality of the generated videos using the following metrics: Peak Signal-to-Noise Ratio (PSNR), Learned Perceptual Image Patch Similarity (LPIPS), Structural Similarity Index Measure (SSIM), and VBench Score.
% PSNR measures pixel-level fidelity by quantifying the difference between generated and ground-truth frames, where higher scores indicate better preservation of fine details. 
% LPIPS evaluates perceptual similarity based on feature representations, while SSIM assesses the structural similarity within video frames. 
% VBench provides a comprehensive evaluation of video quality that aligns closely with human perception. 
% Among these metrics, our method achieves notably high PSNR, demonstrating superior pixel fidelity while maintaining perceptual and structural quality.

\textbf{Baselines.} We compare \sys{} against sparse attention algorithms DiTFastAttn~\cite{yuan2024ditfastattnattentioncompressiondiffusion} and MInference~\cite{jiang2024minference}. As DiTFastAttn can be considered as \spatialhead{} only algorithm, we also manually implement a \temporalhead{} only baseline named \textit{Temporal-only attention}. We also include a cache-based DiT acceleration algorithm PAB~\cite{zhao2024pab} as a baseline.


\textbf{Parameters.} For MInference and PAB, we use their official configurations. For \sys{}, we choose $c_s$ as $4$ frames and $c_t$ as $1224$ tokens for CogVideoX-v1.5, while $c_s$ as $10$ frames and $c_t$ as $1200$ tokens for HunyuanVideo. Such configurations lead to approximately $30$\% sparsity for both \spatialhead{} and \temporalhead{}, which is enough for lossless generation in general. We skip the first $25$\% denoising steps for all baselines as they are critical to generation quality, following previous works~\cite{zhao2024pab,li2024distrifusion,lv2024fastercache,liu2024timestep}.



\begin{figure}[t]
    \centering
    \includegraphics[width=0.95\columnwidth]{figure/efficiency-breakdown.pdf} 
    \caption{The breakdown of end-to-end runtime of HunyuanVideo when generating a $5.3$s, $720$p video. \sys{} effectively reduces the end-to-end inference time from $2253$ seconds to $968$ seconds through system-algorithm co-design. Each design point contributes to a considerable improvement, with a total $2.33\times$ speedup.}
    \label{fig:efficiency-breakdown-figure}
\end{figure}



\subsection{Quality evaluation}
\label{subsec:quality_benchmark}
We evaluate the quality of generated videos by \sys{} compared to baselines and report the results in Table~\ref{table:accuracy_efficiency_benchmark}. Results demonstrate that \sys{} \textbf{consistently outperforms} all baseline methods in terms of PSNR, SSIM, and LPIPS while achieving \textbf{higher end-to-end speedup}.


Specifically, \sys{} achieves an average PSNR exceeding \textbf{29.55} on HunyuanVideo and \textbf{29.99} on CogVideoX-v1.5-T2V, highlighting its exceptional ability to maintain high fidelity and accurately reconstruct fine details.
For a visual understanding of the video quality generated by \sys{}, please refer to Figure \ref{fig:SVG-visualization}.

\sys{} maintains both \textbf{spatial and temporal consistency} by adaptively applying different sparse patterns, while all other baselines fail. E.g., since the mean-pooling block sparse cannot effectively select slash-wise temporal sparsity (see Figure~\ref{fig:spatial-temporal-illustration}), MInference fails to account for temporal dependencies, leading to a substantial PSNR drop. Besides, PAB skips computation of \attn{} by reusing results from prior layers, which greatly hurts the quality.


Moreover, \sys{} is compatible with \textbf{FP8 attention quantization}, incurring only a $0.1$ PSNR drop on HunyuanVideo. Such quantization greatly boosts the efficiency by $1.3\times$. Note that we do not apply FP8 attention quantization on CogVideoX-v1.5, as its head dimension of $64$ limits the arithmetic intensity, offering no on-GPU speedups.


% \begin{table*}[t]
% \centering
% \caption{Quality and Efficiency Benchmark for Video Models.}
% \label{table:accuracy_efficiency_benchmark}
% \resizebox{\linewidth}{!}{%
% \begin{tabular}{c|l|ccccc|cccc}
% \toprule
% \textbf{Type} & \textbf{Method} & \multicolumn{5}{c|}{\textbf{Quality}} & \multicolumn{4}{c}{\textbf{Efficiency}} \\
% \cmidrule(lr){3-7} \cmidrule(lr){8-11}
% & & PSNR $\uparrow$ & SSIM $\uparrow$ & LPIPS $\downarrow$ & VBench-1 $\uparrow$ & VBench-2 $\uparrow$ & FLOPS $\downarrow$ & Peak Memory $\downarrow$ & Latency $\downarrow$ & Speedup $\uparrow$ \\
% \midrule
% \textbf{I2V} & CogVideoX-v1.5 (720p, 10s, 80 frames) & - & - & - & 70.09\% & 95.37\% & 147.87 PFLOPs &  & 528s & 1x \\
% \midrule
% & DiTFastAttn (Spatial-only) & 24.591 & 0.836 & 0.167 & 70.44\% & 95.29\% & 78.86 PFLOPs &  & 338s  & 1.56x \\
% & Temporal-only & 23.839 & 0.844 & 0.157 & 70.37\% & 95.13\% & 70.27 PFLOPs &  & 327s & 1.61x \\
% & MInference & 22.489 & 0.743 & 0.264 & 58.85\% & 87.38\% & 84.89 PFLOPs &  &  &  \\
% & PAB & 23.234 & 0.842 & 0.145 & 69.18\% & 95.42\% & 105.88 PFLOPs &  &  &  \\
% \rowcolor{lightblue}
% & Ours & \textbf{\textcolor{darkgreen}{28.165}} & \textbf{\textcolor{darkgreen}{0.915}} & \textbf{\textcolor{darkgreen}{0.104}} & 70.41\% & 95.29\% & 74.57 PFLOPs &  & 237s & \textcolor{darkgreen}{2.23x} \\
% % \rowcolor{lightblue}
% % & Ours + FP8 & 26.709 & 0.890 & 0.122 &  &  &  &  & \\
% \midrule
% \textbf{T2V} & CogVideoX-v1.5 (720p, 10s, 80 frames) & - & - & - & 62.42\% & 98.66\% & 147.87 PFLOPs &  & 528s & 1x \\
% \midrule
% & DiTFastAttn (Spatial-only) & 23.202 & 0.741 & 0.256 & 62.22\% & 96.95\% & 78.86 PFLOPs &  & 338s & 1.56x \\
% & Temporal-only & 23.804 & 0.811 & 0.198 & 62.12\% & 98.53\% & 70.27 PFLOPs &  & 327s & 1.61x \\
% & MInference & 22.451 & 0.691 & 0.304 & 54.87\% & 91.52\% & 84.89 PFLOPs &  &  &  \\
% & PAB & 22.486 & 0.740 & 0.234 & 57.32\% & 98.76\% & 400.04 PFLOPs &  &  &  \\
% \rowcolor{lightblue}
% & Ours & \textbf{\textcolor{darkgreen}{29.989}} & \textbf{\textcolor{darkgreen}{0.910}} & \textbf{\textcolor{darkgreen}{0.112}} & 63.01\% & 98.67\% & 74.57 PFLOPs &  & 232s & \textbf{\textcolor{darkgreen}{2.28x}} \\
% % \rowcolor{lightblue}
% % & Ours + FP8 &  &  &  &  &  &  &  &  \\
% \midrule
% \textbf{T2V} & HunyuanVideo (720p, 5.33s, 128 frames) & - & - & - & 66.11\% & 93.69\% & 612.37 PFLOPs &  & 2253s & 1x \\
% \midrule
% & DiTFastAttn (Spatial-only) & 21.416 & 0.646 & 0.331 & 67.33\% & 90.10\% & 260.48 PFLOPs &  & 1238s & 1.82x \\
% & Temporal-only & 25.851 & 0.857 & 0.175 & 62.12\% & 98.53\% & 259.10 PFLOPs &  & 1231s & 1.83x \\
% & MInference & 23.157 & 0.823 & 0.163 &  &  & 293.87 PFLOPs &  &  &  \\
% & PAB & - & - & - & - &  & - & \color{red}OOM & - & - \\
% \rowcolor{lightblue}
% & Ours & \textbf{\textcolor{darkgreen}{29.546}} & \textbf{\textcolor{darkgreen}{0.907}} & \textbf{\textcolor{darkgreen}{0.127}} & 65.90\% & 93.51\% & 259.79 PFLOPs &  & 1171s & 1.92x \\
% \rowcolor{lightblue}
% & Ours + FP8 & \textbf{\textcolor{darkgreen}{29.452}} & \textbf{\textcolor{darkgreen}{0.906}} & \textbf{\textcolor{darkgreen}{0.128}} & 65.70\% & 93.51\% & 259.79 PFLOPs &  & 968s & \textbf{\textcolor{darkgreen}{2.33x}} \\
% \bottomrule
% \end{tabular}%
% }
% \end{table*}



\subsection{Efficiency evaluation}
\label{subsec:efficiency_benchmark}

To demonstrate the feasibility of \sys{}, we prototype the entire framework with dedicated CUDA kernels based on FlashAttention~\cite{dao2022flashattentionfastmemoryefficientexact}, FlashInfer~\cite{ye2025flashinferefficientcustomizableattention}, and Triton~\cite{Tillet2019TritonAI}. We first showcase the end-to-end speedup of \sys{} compared to baselines on an H100-80GB-HBM3 with CUDA 12.4. Besides, we also conduct a kernel-level efficiency evaluation. Note that all baselines adopt FlashAttention-2~\cite{dao2022flashattentionfastmemoryefficientexact}.


\begin{table}[t]
\small
\centering
\caption{Inference speedup of customized QK-norm and RoPE compared to PyTorch implementation with different number of frames. We use the same configuration of CogVideoX-v1.5, i.e. $4080$ tokens per frame, $96$ attention heads.}
\label{table:small-kernel-speedup-comparison}
\begin{tabular}{c|cccc}
\toprule
Frame Number & 8 & 9 & 10 & 11  \\
\midrule
%LayerNorm & 7.436× & 7.448× & 7.464× & 7.474×  \\
QK-norm & 7.44× & 7.45× & 7.46× & 7.47×  \\
\midrule
RoPE & 14.50× & 15.23× & 15.93× & 16.47×   \\
\bottomrule
\end{tabular}
\end{table}


\textbf{End-to-end speedup benchmark.} We incorporate the end-to-end efficiency metric including FLOPS, latency, and corresponding speedup into Table~\ref{table:accuracy_efficiency_benchmark}. \sys{} consistently outperforms all baselines by achieving an average speedup of $2.28\times$ while maintaining the highest generation quality. We further provide a detailed breakdown of end-to-end inference time on HunyuanVideo in Figure~\ref{fig:efficiency-breakdown-figure} to analyze the speedup. Each design point described in Sec~\ref{sec:methodology} contributes significantly to the speedup, with sparse attention delivering the most substantial improvement of $1.81\times$.

\textbf{Kernel-level efficiency benchmark.}\label{subsec:kernel_level_efficiency} We benchmark individual kernel performance including QK-norm, RoPE, and block sparse attention with unit tests in Table~\ref{table:small-kernel-speedup-comparison}. Our customized QK-norm and RoPE achieve consistently better throughput across all frame numbers, with an average speedup of $7.4\times$ and $15.5\times$, respectively. For the sparse attention kernel, we compare the latency of our customized kernel with the theoretical speedup across different sparsity. As shown in Figure~\ref{fig:kernel-efficiency-sparse-attention}, our kernel achieves theoretical speedup, enabling practical benefit from sparse attention.


\begin{figure}[t]
    \centering
    \includegraphics[width=\columnwidth]{figure/LayourTransformSpeed3.pdf} 
    % \vspace{-2pt}
    \caption{Latency comparison of different implementations of sparse attention. Our hardware-efficient \reorder{} optimizes the sparsity pattern of \temporalhead{} for better contiguity, which is $1.7$× faster than naive sparse attention (named original), approaching the theoretical speedup.}
    \label{fig:kernel-efficiency-sparse-attention}
    \vspace{-5pt}
\end{figure}

\begin{table}[t]
\centering
\caption{Sensitivity test on \onlinesample{} ratios. Profiling just $1$\% tokens achieves generation quality comparable to the oracle ($100$\%) while introducing only negligible overhead.}
\label{table:sensitivity-sampling}
\begin{tabular}{l|ccc}
\toprule
\textbf{Ratios} & \textbf{PSNR $\uparrow$} & \textbf{SSIM $\uparrow$} & \textbf{LPIPS $\downarrow$} \\
\midrule
\multicolumn{4}{c}{\textbf{CogVideoX-v1.5-I2V (720p, 10s, 80 frames)}} \\
\midrule
profiling 0.1\% & 30.791 & 0.941 & 0.0799 \\
profiling 1\% & 31.118 & 0.945 & 0.0757\\
profiling 5\% & 31.008 & 0.944 & 0.0764\\
profiling 100\% & 31.324 & 0.947 & 0.0744 \\
% \midrule
% \multicolumn{4}{l}{\textbf{CogVideoX V1.5 (720p, 10s, 80 frames)}} \\
% \midrule
% No threshold & 31.118 & 0.945 & 0.0757\\
% threshold=10 & 31.304 & 0.949 & 0.0722\\
% threshold=1 & 31.322 & 0.949 & 0.0717\\
% threshold=0.1 & 31.217 & 0.949& 0.0720\\
\bottomrule
\end{tabular}
\end{table}

\subsection{Sensitivity test}
\label{subsec:sensitivity-test}
In this section, we conduct a sensitivity analysis on the hyperparameter choices of \sys{}, including the \onlinesample{} ratios (Sec~\ref{subsec:sampling_based_pattern_selection}) and the sparsity ratios $c_s$ and $c_t$ (Sec~\ref{subsec:frame_token_rearrangement}). Our goal is to demonstrate the robustness of \sys{} across various efficiency-accuracy trade-offs.


\textbf{\Onlinesample{} ratios.} We evaluate the effectiveness of \onlinesample{} with different profiling ratios on CogVideoX-v1.5 using a random subset of VBench in Table~\ref{table:sensitivity-sampling}. In our experiments, we choose to profile only 1\% of the input rows, which offers a comparable generation quality comparable to the oracle profile (100\% profiled) with negligible overhead.

%Profiling only $1$\% of the input data achieves nearly the same generation quality as the oracle profiling ($100$\% sampling), with only a $0.2$ PSNR reduction. Therefore, we adopt this scheme as the default setting, as it provides accuracy comparable to the oracle with negligible overhead.


\textbf{Generation quality over different sparsity ratios.} As discussed in Sec~\ref{sec:sparse-theoretical-speedup}, different sparsity ratio of the \spatialhead{} and \temporalhead{} can be set by choosing different $c_s$ and $c_t$, therefore reaching different trade-offs between efficiency and accuracy. We evaluate the LPIPS of HunyuanVideo over a random subset of VBench with different sparsity ratios. As shown in Table~\ref{table:sensitivity-sparsity-ratios}, \sys{} consistently achieves decent generation quality across various sparsity ratios. E.g., even with a sparsity of $13$\%, \sys{} still achieves $0.154$ LPIPS. We leave the adaptive sparsity control for future work.


\subsection{Ablation study}
\label{subsec:ablation}
We conduct the ablation study to evaluate the effectiveness of the proposed hardware-efficient \reorder{} (as discussed in Sec~\ref{subsec:frame_token_rearrangement}). Specifically, we profile the latency of the sparse attention kernel with and without the transformation under the HunyuanVideo configuration. As shown in Figure~\ref{fig:kernel-efficiency-sparse-attention}, the sparse attention with \reorder{} closely approaches the theoretical speedup, whereas the original implementation without \reorder{} falls short. For example, at a sparsity level of $10$\%, our method achieves an additional $1.7\times$ speedup compared to the original approach, achieving a $3.63\times$ improvement.

\begin{table}[t]
\small
\centering
\caption{Video quality of HunyuanVideo on a subset of VBench when varying sparsity ratios. LPIPS decreases as the sparse ratio increases, achieving trade-offs between efficiency and accuracy.}
\label{table:sensitivity-sparsity-ratios}
\begin{tabular}{c|cccccc}
\toprule
Sparsity$\downarrow$ & 0.13 & 0.18 & 0.35 & 0.43 & 0.52 \\
\midrule
LPIPS$\downarrow$ & 0.154 & 0.135 & 0.141 & 0.129 & 0.116 \\
\bottomrule
\end{tabular}
\vspace{-5pt}
\end{table}



% \paragraph{Robustness of Sparse Attention} To further assess the robustness of our sparse attention mechanism, we examine its performance under different MSE thresholds. 
% As discussed in Section \ref{sec:sparse_patterns}, approximately 10\% of attention heads exhibit high MSE values ($\ge$0.1) under both Arrow Mask and Zebra Mask. 
% To address these edge cases, we calculate full attention for heads with MSE values exceeding a given threshold (0.1, 1, or 10). 
% As shown in Table \ref{table:ablation_study}, the PSNR remains consistent across all threshold settings, indicating that these rare corner cases do not significantly impact overall performance.

% \paragraph{Impracticality of Offline Calibration} We explore whether sparse pattern selection can be pre-determined through offline calibration. 
% A visual comparison of sparse patterns selected for two videos generated by CogVideoX is presented in Figure \ref{}. 
% The patterns show no clear correlation between the two videos, indicating that sparse attention patterns vary significantly depending on the content and context of each video. This result demonstrates that offline calibration is infeasible for video generation tasks, further validating the need for our online sampling-based method.
\section{Related Work}
\label{Sec:related_work}

\textbf{Automatic app maturity ratings}: The evaluation of mobile apps often involves various perspectives. In particular, identifying mobile app development is consistent with what is stated in the privacy policy concerning online advertising and tracking ~\cite{nguyen2022freely, nguyen2021measuring}, aiding developers in crafting child-friendly apps concerning both content and privacy aspects~\cite{hu2015protectingcikm, liccardi2014can}. However, fewer studies aimed at mobile app maturity rating. Therefore, there is growing concern regarding inappropriate content and maturity ratings in mobile apps, which are linked to privacy concerns. Early work by Chen et al.~\cite{chen2013isthisapp} proposed Automatic Label of Maturity ratings (ALM), a text-mining-based semi-supervised algorithm that uses app descriptions and user reviews to determine maturity ratings. The authors used the content rating from Apple App Store as the reference standard for a given app. However, this method uses keyword matching while ignoring semantic analysis. Using a similar approach for ground truth establishment, Hu et al.~\cite{hu2015protectingcikm} proposed a text feature-based SVM classifier for content rating prediction with an online training element. The previous two methods solely depend on text features despite having access to other modalities. Liu et al.~\cite{liu2016identifying} and Chenyu et al.~\cite{zhou2022automatic} extended these works by incorporating image and APK features to identify children’s apps. However, features were limited to extracting text using OCR software, colour distribution of the icon and screenshots, and permissions and APIs. More recently, Sun et al.~\cite{sun2023not} identified discrepancies in content ratings of the same app in different geographic regions by defining rating system mappings between geographical regions. However, this research focuses on single modalities or multiple modalities but treats them independently. \\ 
% \vspace{-3mm}

\noindent\textbf{Vision-Language (VL) models}:  Early image-based contrastive representations have made advancements, nearly achieving the performance levels seen in supervised baselines across various downstream tasks such as image classification and retrieval~\cite{chen2020simple, zbontar2021barlow}. Driven by the success of contrastive learning in intra-modal tasks, there has been a growing interest in developing multi-modal objectives (e.g., Vision-Language), enabling the model to comprehend and exploit cross-modal associations.
Pioneering works such as CLIP~\cite{clip} and ALIGN~\cite{align} bridged the gap between the vision and language modalities by learning language and vision encoders jointly with a symmetric cross-entropy loss which is an adaptation of InfoNCE loss~\cite{oord2018representation} for cross-model pairs. CLIP optimises the cosine similarity between text and image embeddings, while ALIGN employs a similar contrastive learning setting with noisy training data. Zhai et al.~\cite{LiT} tuned the text encoder using image-text pairs while keeping the image encoder frozen. The rich embeddings that these methods learn are later adapted to various application domains such as video-text retrieval~\cite{fang2021clip2video, portillo2021straightforward}, image generation~\cite{nichol2021glide}, and visual assistance~\cite{massiceti2023explaining}. 
However, \cite{agarwal2021evaluating, luccioni2024stable} point out the challenges in adapting Large Multi-modal Models (LMMs) for different domains when the downstream task deviates from the originally pre-trained task. To the best of our understanding, ours is the first work to leverage the advances in VL-language models to detect content compliance malpractices specific to mobile apps. 


This work presented \ac{deepvl}, a Dynamics and Inertial-based method to predict velocity and uncertainty which is fused into an EKF along with a barometer to perform long-term underwater robot odometry in lack of extroceptive constraints. Evaluated on data from the Trondheim Fjord and a laboratory pool, the method achieves an average of \SI{4}{\percent} RMSE RPE compared to a reference trajectory from \ac{reaqrovio} with $30$ features and $4$ Cameras. The network contains only $28$K parameters and runs on both GPU and CPU in \SI{<5}{\milli\second}. While its fusion into state estimation can benefit all sensor modalities, we specifically evaluate it for the task of fusion with vision subject to critically low numbers of features. Lastly, we also demonstrated position control based on odometry from \ac{deepvl}.

\bibliographystyle{plain}
\bibliography{sn-bibliography}% common bib file
%% if required, the content of .bbl file can be included here once bbl is generated
%%\input sn-article.bbl


\end{document}
