\section{LLM-driven Social Generative Agents}\label{sec:social_agents}


\subsection{Overview}


As discussed above, the rapid development of LLMs allows us to design human-like agents with not only basic psychological states~\cite{abdurahman2024perils,strachan2024testing}, but also complex social behaviors such as mobility~\cite{shao2024beyond,yan2024opencity,feng2024agentmove}, employment~\cite{li2024econagent,horton2023large}, consumption~\cite{li2024econagent,horton2023large}, and social interactions~\cite{gao2023s,park2023generative}. While these efforts in specific areas have shown the human-level intelligence of LLMs, creating LLM-driven social generative agents capable of simulating a comprehensive social being remains difficult. This difficulty primarily lies in two aspects. First, human behaviors are inherently motivated by psychological states~\cite{eysenck2020cognitive,mcleod2007maslow,maslow1943theory,ajzen1991theory}. However, this crucial connection is largely absent in a vanilla LLM or existing agents designed for specific aspects. Second, different types of behaviors are highly interdependent. For example, the decision of when and how people commute to work is shaped by the interplay between their mobility and employment behaviors. Similarly, social interactions among individuals often take place when people go shopping. These examples highlight the crucial interdependence of human behaviors. Despite its significance, current LLMs and agents fail to capture this, limiting their ability to accurately simulate realistic, complex human behaviors. Addressing these two aspects requires deep insights into social science theories of human behavior, as well as advancements in algorithmic design to integrate these insights into LLM-driven social generative agents.

\begin{figure}[t]
\centering
\includegraphics[width=\textwidth]{Figure/agents_overall.pdf}
\caption{Overview of LLM-driven social generative agents.}
\label{fig:overall_agents}
\end{figure}



Therefore, we propose a design for LLM-driven social generative agents, deeply rooted theories from psychology (e.g. Maslow's Hierarchy of Needs~\cite{maslow1943theory} and Theory of Planned Behavior~\cite{ajzen1991theory}), economics (e.g., Dynamic Stochastic General Equilibrium~\cite{christiano2005nominal}), and behavioral science (e.g., Gravity Model~\cite{zipf1946p}). Figure~\ref{fig:overall_agents} provides an overview of the proposed agents, which can be roughly divided into four parts. First, each agent retains their profile, typically regarded as relatively stable (e.g., personality), and status, which is dynamic (e.g., emotion). In particular, the profile includes basic demographics such as name, age, gender, and education, as well as personality. The status comprises three key aspects: the agent’s current mental states, economic status, and social relationships. Mental states reflect the agent’s inner experiences, while economic status and social relationships capture their power and connections in the external world. The integration of the profile and status into these LLM agents enables them to role-play like real people, providing the foundation for simulating complex mental processes and behaviors.

Second, each agent is designed with three levels of mental processes: emotions, needs, and cognition. Emotions reflect the agent’s immediate response to both internal and external stimuli, shaping its behaviors and reactions. Needs serve as the underlying motivational drivers that guide an agent’s actions, ranging from basic survival requirements to higher aspirations such as personal growth and self-fulfillment. Cognition refers to the agent’s understanding of the world, e.g., its attitudes toward climate change and political issues. By incorporating these three levels of mental processes (see the detailed design in Section~\ref{sec:emotion}), agents can autonomously perceive the external environment, ultimately developing their cognition of it. 

Third, social behaviors are the core of LLM-driven social generative agents, which serve as the bridge between their internal mind and external environment. Given the importance and complexity of various human behaviors, we explicitly model three types of social behaviors: mobility, social interactions, as well as employment \& consumption. In Sections~\ref{sec:mobility}-\ref{sec:economy}, we detail the special designs for these three behaviors. Other simple behaviors such as sleeping are directly handled by LLMs. It is worth noting that these behaviors are conditioned by the agents' profile and status, and driven by their mental processes. Finally, we introduce the workflow of the overall LLM-driven social generative agents in Section~\ref{sec:workflow}, illustrating the integration of their profiles, mental processes, and social behaviors. This workflow enables the simulation of comprehensive, context-aware agents by capturing both internal cognitive states and external interactions, ensuring realistic, dynamic social behaviors within the simulation.


%through two memory flows. One flow records the happening of objective events, mainly sourced from external environments. The other captures the agents’ subjective experiences as they explore their own minds or interact with external environments.



\subsection{Emotion, Needs, and Cognition} \label{sec:emotion}

% Agents guided by Theory of Mind recognize both their own knowledge and the mental states of others, including beliefs, intentions, and emotions. This dual awareness enhances their ability to predict and respond appropriately in social and multi-agent settings~\cite{pynadath2011modeling}. 

% 主旨
% needs, emotion, and cognition 这个部分 (画一张图必须有) ,说明智能体内心是如何构建的。每个部分可以选画一张图。

% --- 分工 ----
% 第一段,总起,为什么拆分到这几块(Jingyi来写)。根据心理学来说,从情感,到需求,再到认知 |从最基础的再到高级的,从快速变化额再到稳定的; 心理过程(情绪)和状态


Humans are driven by an intricate interplay of feelings, motivations, and thought processes that shape their decisions and interactions~\cite{shvo2019interdependent,al2023chatgpt}. Grounded in psychological theories, our study integrates three fundamental constructs, including emotion, needs, and cognition, to design agents that realistically simulate adaptive and human-like behavior. Emotion, as the most dynamic layer of human psychology, drives rapid responses to external situations and influences behavior~\cite{bourgais2018emotion,beall2017emotivational}. Needs, based on Maslow’s hierarchy of needs theory, serve as motivational drivers, spanning from basic survival requirements to higher aspirations like personal growth~\cite{acevedo2018personalistic}. Modeling these needs enables agents to adopt realistic motivations and prioritize actions in ways relatable to human behavior. 
Cognition, informed by theories like Theory of Mind and Cognitive Appraisal Theory, involves advanced mental processes that allow agents to evaluate complex situations, make thoughtful decisions, and adapt to diverse situations~\cite{bandura1989human, schurmann2020personalizing}. Drawing on these psychological theories,  agents recognize their own knowledge and the mental states of others while evaluating context sensitively, enabling effective and goal-oriented social interactions~\cite{pynadath2011modeling}.
By integrating these crucial elements, our study designs agents that can respond dynamically to real-time changes while tailoring their actions to reflect human-like characteristics and behaviors within complex social simulations. The overall modeling framework for the psychological and cognitive aspects of the agent is illustrated in Figure \ref{fig:agents_cognition}.

\begin{figure}[ht]
\centering
\includegraphics[width=0.5\textwidth]{Figure/emotion.pdf}
\caption{Modeling framework of emotion, needs, and cognition.}
\label{fig:agents_cognition}
\end{figure}



% Humans are driven by an intricate interplay of feelings, motivations, and thought processes that shape their decisions and interactions~\cite{shvo2019interdependent,al2023chatgpt}. Grounded in psychological theories, our study integrates three fundamental constructs, including emotion, needs, and cognition, to design agents that realistically simulate adaptive and human-like behavior. Emotion, as the most dynamic layer of human psychology, drives rapid responses to external situations and influences behavior~\cite{bourgais2018emotion}. Emotions such as fear, joy, or anger prompt immediate actions, allowing agents to respond naturally and interact effectively in dynamic environments~\cite{beall2017emotivational}. Needs, based on Maslow’s hierarchy of needs theory, serve as motivational drivers, spanning from basic survival requirements to higher aspirations like personal growth~\cite{acevedo2018personalistic}. Modeling these needs enables agents to adopt realistic motivations and prioritize actions in ways relatable to human behavior. 
% % Incorporating the Theory of Planned Behavior~\cite{ajzen1991theory}, agents evaluate their attitudes, subjective norms, and perceived control to form intentions and execute goal-driven actions that align with socially influenced decision-making processes in virtual environments. 
% Finally, cognition involves advanced mental processes, such as reasoning, planning, and decision making, which allow agents to evaluate complex situations, make thoughtful decisions, and adapt to diverse situations~\cite{bandura1989human, schurmann2020personalizing}. Drawing on Social Cognitive Theory, agents learn from feedback and adjust cognitive and emotional states~\cite{bandura1989human}. Guided by Theory of Mind, they recognize their own knowledge and the mental states of others, allowing for effective social interactions~\cite{pynadath2011modeling}. With Cognitive Appraisal Theory, agents evaluate contexts sensitively to produce nuanced, goal-oriented behaviors~\cite{pynadath2011modeling}.
% By integrating these crucial elements, our study designs agents that can respond dynamically to real-time changes while tailoring their goal-oriented actions to reflect their characteristics and behaviors within complex social simulations.


% 第二段,emotion (zhihong)Moreover, following 

Emotion is a dynamic and foundational element of human psychology, driving rapid responses to external events and influencing decision-making and behavior~\cite{shvo2019interdependent}. In our model,  an agent's emotions are affected by its profile and status and are updated based on interactions with other agents and the environment. We adopt the emotion measurement framework from Shvo et al.~\cite{shvo2019interdependent}, which involves the agent selecting a keyword to best describe its current emotional state, formulating a sentence-based thought related to that emotion, and rating the intensity of six core emotions—sadness, joy, fear, disgust, anger, and surprise—on a scale from 0 to 10. This method enables agents to track and update their emotional states, providing a foundation for contextually appropriate and adaptive behavior. These emotions then influence the agent’s actions, motivations, and cognitive processes, establishing an interconnected system that guides decision-making. As we will explore in the next section, emotional states directly impact the agent's needs and cognitive evaluations, linking emotions to higher-order motivational and reasoning functions.


% 第三段,needs (yuwei) --- 可能需求模块需要一个单独的图
The concept of needs is widely accepted in the field of psychology as the fundamental motivator behind an individual's pursuit of specific objectives and maintenance of social engagement. Emotions, on the other hand, are seen as the immediate responses experienced by an individual. The concept of needs, however, is believed to establish the underlying motivational mechanisms that guide sustained behavior, thereby extending and contextualizing emotional fluctuations. The integration of needs and emotions in the agent's model enables the maintenance of consistent motivational pathways over time, ensuring that transient affective states are grounded in enduring goals and priorities.
In our approach, we employ established psychological frameworks (e.g., Maslow's hierarchy of needs~\cite{acevedo2018personalistic}) to categorize and structure these motivational forces. We adopt a hierarchical representation of needs to organize motivational drives by their relative urgency and importance. This needs hierarchy is continuously updated based on three interrelated factors: the agent's active behaviors, uncontrollable or passive external events, and its current psychological states. The integration of these elements enables the system to dynamically adjust need priorities, ensuring that the agent responds appropriately to both internal motivations and external pressures. Furthermore, needs do not merely reflect static conditions but rather serve as a driving force for proactive behavior. Leveraging the Theory of Planned Behavior~\cite{ajzen1991theory}, the agent formulates action plans specifically aimed at meeting or enhancing priority needs. Through this design, the needs module provides a robust foundation for adaptive, socially informed behavior.
In conclusion, the modeled needs provide the necessary motivational basis that informs and intersects with the agent's cognitive processes, leading directly into the subsequent discussion on cognition.


% 第四段,cognition (zhihong)

Cognition encompasses the higher-level processes involved in reasoning, planning, and decision-making~\cite{bandura1989human}. In our model, cognition is intricately connected to the agent's emotional and attitudinal updates. After the agent processes an action, a sentence is used to describe its behavior in relation to the current context. These sentences is then used to update both the agent’s attitude towards specific topics and its emotional state. Attitude, in this context, serves as a memory system, reflecting how supportive or opposed the agent is to a particular topic, rating from 0-10. By continuously updating both emotion and attitude through the agent’s actions and experiences, cognition ensures that the agent adapts to its environment in a way that is consistent with human-like reasoning and emotional depth. This process, in turn, influences the agent’s needs and motivations, bridging the gap to the next level of analysis.

% 第五段,收束一下,给一个例子(yuwei)
In summary, the integration of Emotion, Needs, and Cognition modules enables the agent to engage in socially intelligent behavior, with each module contributing to the shaping of adaptive actions. For instance, when the agent detects an unmet need for social interaction and determines a sequence of actions—such as identifying potential contacts and sending messages—to satisfy this need. The emotional state of the agent, influenced by the emotion module, affects the tone of communication, prompting the agent to initiate conversation with a cheerful or light-hearted tone. As the social interaction progresses, the outcome of the behavior—whether the interaction is perceived as successful or not—is reflected in the emotion and cognition modules. This, in turn, results in an update to the needs module, thereby establishing a continuous feedback loop that adapts the agent's behavior in response to both its environment and internal state.

% ----- 写作指南 -----
% 第二段到第四段,每一段的组织的逻辑就是 
% 定义。(1句)
% 为什么建模社会人为什么需要这个元素。与上一层次的元素的关系是什么,承接上面一段(2-3句)
% 具体是怎么建模的(follow了xxx, design了xxx,实现了xxx)。(4-5句)
% 与下一层次的元素的关系是什么,引出下一段 (半句-1句)


\subsection{Mobility Behaviors}\label{sec:mobility}
Mobility serves as the fundamental basis for social agents to engage in interactions and fulfill their demands. Mobility is not a random behavior but is needs-driven across multiple levels. For instance, when an agent experiences hunger (a basic survival need), it must move to a restaurant to obtain food; to attend a work meeting (an advanced professional need), commuting to an office becomes necessary. These mobility behaviors directly serve specific goals, acting as physical carriers for social interactions, economic activities, and other societal behaviors. In essence, the core challenge of mobility modeling is to bridge the spatiotemporal gap between needs and behaviors. Without effective mobility, agents cannot achieve role immersion or behavioral closure in complex social environments.

As depicted in the previous section, the Needs of social agents exhibit a hierarchical structure: from foundational needs (e.g., eating, resting) to safe and social needs (e.g., work, social gatherings). To satisfy the needs, it requires implementation through a "Need - Plan - Behavioral Sequence" chain. Taking social needs as an example, an agent first formulates a plan to "attend a friend’s gathering," which decomposes into behavioral sequences such as "scheduling time (Friday evening), selecting a location (café), moving." Here, location selection becomes the direct driver of mobility — to reach the target location. This spatiotemporal coupling establishes mobility as the critical execution step for demand realization.

Following the needs-driven principle, the mobility module adopts a hierarchical decision framework (shown in Fig. \ref{fig:mobility_behavior}):  
\begin{enumerate}
    \item \textbf{Intention Extraction}: Derive core mobility intentions from demand hierarchies. For example, when "social demand" is activated, the agent may extract a "move to social venue" command.
    \item \textbf{Place Type Selection}: Match demands with POI (Point of Interest) types in geographic databases. If the intention is social interaction, venues like cafés or parks are filtered.
    \item \textbf{Radius Decision}: Dynamically determine feasible ranges by integrating internal states (e.g., age, stamina) and environmental parameters (e.g., weather, traffic). Heavy rain may constrain the radius of indoor venues within 1 km or even stay at home.
    \item \textbf{Place Selection}: Apply the Gravity model for spatial optimization:
    \begin{equation}
        P_{ij} = \frac{S_j / D_{ij}^\beta}{\sum{S_k / D_{ik}^\beta}},
    \end{equation}
    where \(S_j\) denotes the attractiveness of location \(j\), \(D_{ij}\) is the distance, and \(\beta\) is the distance decay coefficient. This model reduces LLM computational overhead while ensuring selections align with human spatial patterns (e.g., proximity principle, agglomeration effects).
\end{enumerate}

\begin{figure}[ht]
\centering
\includegraphics[width=0.9\textwidth]{Figure/mobility_behavior.pdf}
\caption{Modeling of mobility behavior.}
\label{fig:mobility_behavior}
\end{figure}

Mobility serves as a foundational behavioral module, operating as an integrative force within the social agent's action network. This enables multidimensional coordination across social, economic, and environmental domains. The act of moving to a park, for instance, inherently carries the potential for social synergy. Spontaneous encounters with acquaintances may emerge, catalyzing dialogues, collaborative activities, or even serendipitous social events. These interactions exemplify how mobility serves as a conduit for organic relationship-building. Concurrently, economic synergy manifests through goal-oriented displacements. Commuting to workplaces directly sustains labor productivity, while visiting commercial hubs like shopping malls create opportunities for consumption, thereby linking physical movement to economic cycles. Beyond the human-centric interactions discussed above, mobility also embodies environmental adaptation. Agents dynamically adjust routes based on real-time traffic data or weather fluctuations, demonstrating responsiveness to spatial-temporal constraints. Collectively, these synergies position mobility as the dynamic chassis of social adaptability. It fulfills immediate demands and also provides the contextual infrastructure for complex, layered interactions in urban ecosystems.

% 主旨
% 画一张图说明构建一个社会人智能体需要移动行为
% 第一段,总起,说明建模移动模块的必要性。
% 第三段,如何建模社会人的移动
% 第四段,总结一下


\subsection{Social Behaviors}\label{sec:social}
\begin{figure}[ht]
\centering
\includegraphics[width=0.9\textwidth]{Figure/agents_social.pdf}
\caption{Modeling of social behavior.}
\label{fig:agents_social}
\end{figure}
Social behaviors play a critical role in our agent simulation framework. They enable the flow of information and influence between agents, and further lead to the emergence of collective phenomena through agent interactions. In real societies, people's beliefs, opinions, and behaviors spread and evolve primarily through social connections and communications. Therefore, modeling social behaviors allows us to simulate how information and influence flow between agents affects both individual and group dynamics. Our social module consists of two components: social relationships defining the connections between agents, and social interaction behaviors enabling communication between connected agents.

We include three types of social relationships in our framework: family bonds, friendships, and colleagues. Each relationship has a strength value ranging from 0 to 100 representing social closeness between agents. Agents communicate more frequently with high strength connections and adjust their communication tone based on relationship type. For example, agents use more formal language with colleagues and casual language with close friends. We maintain detailed interaction history between connected agents, including message content and time, which influences how they communicate in future interactions.

Our framework primarily focuses on modeling online social behaviors on online social networks. Motivated by their social needs, agents select interaction partners based on relationship types and strength. For instance, when sending casual messages, agents typically choose their best friends, which are friends with the highest relationship strength. Target selection also considers the recipient's profile characteristics. When an agent wants to discuss specific topics, they select friends with relevant expertise or experience. For example, an agent seeking advice about security issues would contact friends who work as police officers. After selecting a target, agents start conversations. The content of these messages is shaped by multiple factors: the agent's current needs and intentions determine the conversation topic, their thoughts and beliefs influence the specific content, and their emotional state affects the message tone and style. When receiving messages, agents generate responses based on their relationship strength with the sender, their chat history, and their current emotional state. Our current framework primarily models online social interactions through messaging on online social network, with plans to incorporate offline interactions. For example, when agents discover shared interests in particular topics or need to have more detailed discussions, they can coordinate offline meetings through online communication.

Social behaviors are deeply interconnected with agents's emotional states, cognition, economic behaviors, and mobility behaviors. An agent's current emotions and beliefs directly influence how they compose messages, while received messages can significantly alter their emotional state and viewpoints. Positive interactions can improve mood and strengthen relationship bonds, while negative interactions may lead to emotional distress and weakened connections. The exchange of economic information through social interactions can trigger economic behaviors. For instance, when agents receive news about job opportunities or market conditions from their social connections, they may adjust their employment or consumption decisions accordingly. Similarly, social interactions often lead to mobility behaviors, such as when agents receive event invitations or arrange offline meetings with their connections.

Through this comprehensive social behavior modelling, we enable agents to engage in meaningful interactions that both shape and are shaped by their internal states and external behaviors. Our social module captures both relationship structures between agents and their interaction behaviors, allowing agents to interact and influence each other as they do in real societies, and providing a foundation for studying how information and influence spread in the simulated society.
% 主旨
% 画一张图说明构建一个社会人智能体需要【哪些社交关系】 和【社交行为】
% 第一段,总起,对于一个社会人来说,建模社交模块的必要性。社交模块分为关系和社交行为两个部分。
% 第二段,建模了的智能体的社交关系(朋友、家人、同事)
% 第三段,建模了的社会交互行为(依靠社交网络为信息传播渠道的线上社交),以后还会考虑线下社交行为;具体包括哪几种行为?在需求驱动下的发消息、收消息....这些功能使得智能体能够相互交互
% 第四段,总结下



\subsection{Economic Behaviors} \label{sec:economy}
Economic behavior in daily life is a necessary component for sustaining life, with employment and consumption being the core economic activities. These two behaviors occupy the majority of time for social agents and further influence their psychological states, including cognition, emotional well-being, and overall life satisfaction. Moreover, economic behavior is deeply intertwined with other aspects of an agent's daily life, such as mobility and social interactions. The satisfaction of one need, whether it is economic or social, often triggers a cascade of related behaviors that span across different domains of the agent’s life. For example, an agent’s decision to work longer hours to increase their income may result in less time available for social interactions. Similarly, the decision to spend more on consumption could lead to adjustments in an agent’s mobility patterns, such as travel to different stores or even relocation to areas with better access to desired goods or services. The modeling of economic behavior is shown in Figure \ref{fig:agents_economic}.

\begin{figure}[ht]
\centering
\includegraphics[width=\textwidth]{Figure/agents_economic.pdf}
\caption{Modeling of economic behavior.}
\label{fig:agents_economic}
\end{figure}

In terms of behavior modeling, we simulate the employment and consumption behavior of social agents through the strength of their work and consumption propensity, and apply these behaviors in a macroeconomic simulation environment~\cite{li2024econagent}. Work propensity determines the agent’s working hours and corresponding monthly income, while consumption propensity determines their monthly consumption budget. Additionally, agents autonomously decide how to allocate this budget, including where to spend the money and what to purchase. These behaviors are directly influenced by various economic factors, including last month's consumption, prices, taxes, and so on. These factors are integrated into the agent decisions in a real-world context, where agents constantly adjust their behavior in response to dynamic economic markets. In future work, we will further simulate agents' complex economic behaviors in the labor and financial markets, including job changes, debt, and investment, to model a more realistic socio-economic environment.

This framework can be used to simulate large-scale economic systems and to explore the potential impacts of policy changes, economic shocks, and other factors on the overall behavior of social agents within the system. By examining these dynamics, we can gain a deeper understanding of the interactions between economic behaviors, psychological states, and social dynamics in a comprehensive and integrated manner.


% 主旨
% 画一张图说明构建一个社会人智能体需要 经济行为【消费与工作】
% 第一段,总起,说明建模经济模块的必要性。
% 第二段,建模了社会人的消费行为
% 第三段,建模了社会人的工作行为
% 第四段,总结下


\subsection{Workflow of LLM-driven Social Generative Agents} \label{sec:workflow}

In this section, we introduce the workflow of our LLM-driven social agent, highlighting how the agent’s internal psychological states (Emotion and Cognition) and its behaviors influence each other and form a complete loop. This loop continuously adapts the agent’s actions based on its evolving cognitive states, ensuring that behavior is dynamically aligned with both internal motivations and external context. The core mechanism linking these psychological states to behavior is Memory, which connects the agent’s cognitive states with its actions. This enables the agent to make adaptive decisions that reflect its past experiences, current needs, and cognitive responses.

\begin{figure}[ht]
\centering
\includegraphics[width=0.9\textwidth]{Figure/workflow.pdf}
\caption{Workflow of social agents based on stream memory.}
\label{fig:workflow}
\end{figure}

At the core of our solution is the use of Memory to link the agent’s internal psychological states to its behavior. Memory acts as a bridge between the agent’s current emotions, cognition, and its past experiences, ensuring that its actions are informed by both historical context and present needs. Memory is not a passive system but actively shapes the agent’s decisions and behavior, enabling continuous adaptation and coherence in its responses to changing situations. Specifically, Memory is divided into three main components, each with a specific role in the agent's overall operation:

\begin{itemize}
    \item \textbf{Profile}: Stores the agent's static attributes, such as demographic information (e.g., gender, age), which remain constant and provide context for interpreting the agent's behavior.
    \item \textbf{Status}: Records the agent's dynamic state information in key-value pairs, including data like current needs, satisfaction levels, and financial status, which directly influence decision-making.
    \item \textbf{Stream Memory}: This is the core part of the memory system and tracks events and perceptions over time. It is composed of two types of memory streams: \textit{Event Flow} and \textit{Perception Flow}. Each stream is organized chronologically, with multiple \textit{MemoryNodes} in each stream. Each MemoryNode contains a description with three components: time, location, and event description. 
\end{itemize}

The \textit{Event Flow} records events that occur over time, such as proactive actions by the agent, passive external events, and environmental changes. These events are recorded in sequence, maintaining a timeline of actions and occurrences.

The \textit{Perception Flow} records the agent's thoughts and attitudes towards the events in the \textit{Event Flow}. Each node in the \textit{Perception Flow} is linked to one or more nodes in the \textit{Event Flow}, reflecting how the agent perceives or reacts to a specific event. This integration allows for a nuanced representation of both the agent's cognitive appraisals and emotional responses. 

The agent's behavior is driven by its current state, which influences the decision-making process and the actions taken. The following steps outline the agent's workflow:

\begin{enumerate}
    \item \textbf{Action Determination}: The agent assesses its current state (from the Status memory) and decides on a course of action based on its emotional and cognitive evaluations. For example, if the agent needs social interaction and is in a positive emotional state, it may choose to initiate a social conversation.
    \item \textbf{Event Feedback}: After performing the action, the agent receives feedback. For example, if the agent attempts to move to a social gathering, it checks whether the movement was successful (e.g., did it reach the correct location, considering environmental factors like weather).
    \item \textbf{Memory Update}: The event and its feedback are recorded in the Event Flow, and the associated Perception Flow is updated with the agent’s emotional and cognitive responses to the event.
    \item \textbf{Emotion and Cognition Analysis}: The Emotion and Cognition modules analyze the outcome of the event (e.g., whether the movement was successful) and update the agent's emotional state and attitude accordingly. This feedback may affect the agent’s future decisions and actions.
    \item \textbf{Passive and Environmental Events}: In the case of passive events or environmental stimuli, the same memory processing logic is applied. The agent perceives the event, updates the Event Flow, and modifies its Perception Flow accordingly.
\end{enumerate}

The Memory system, organized along a time axis, reflects the natural flow of events in the physical world. This memory framework allows the agent to integrate its ongoing experiences with past events, creating a dynamic, evolving representation of its environment and internal states. By leveraging Stream Memory, the agent adapts its behavior over time in a way that mirrors human cognition, emotional responses, and decision-making, providing a coherent and context-aware foundation for socially intelligent behavior.

% 主旨:怎么通过一个智能体的设计来实现上面社会人的 【内部心理】 和 【移动、社交、经济行为功能】
% 画一张图说明一个社会智能体背后的memory,




