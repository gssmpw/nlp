\section{AgentSociety: Design and Overview}


% 复杂系统的思路 -- 需求
Society is a complex system, characterized by large-scale interactions among individuals with diverse social behaviors, whose nonlinear dynamics often give rise to emergent phenomena and unpredictable collective behaviors in a certain environment~\cite{sawyer2005social,ladyman2013complex,epstein1999agent}. For example, in social networks, interactions between individuals can result in the emergence of polarization~\cite{baumann2021emergence}. Moreover, financial market crashes, a classic phenomenon in economic systems, stem from the collective behavior of market participants and the herding tendencies of individuals~\cite{sornette2009stock}. These emergent phenomena, despite originating from individuals' behaviors, cannot be fully explained or predicted solely based on individual components~\cite{sawyer2005social,ladyman2013complex,epstein1999agent}. Therefore, this requires us to adopt a bottom-up perspective~\cite{epstein1999agent,epstein2012generative,epstein1996growing}: we should begin by simulating \textit{an individual social agent}, and then generate an artificial society by incorporating \textit{a realistic environment} and facilitating \textit{large-scale interactions} among agents as well as between agents and their environment.


% 根据要点梳理,每个维度需要具备哪些进阶型的特征
% 得到过去工作的一个整理


Therefore, we develop an evaluation framework to examine the capabilities of various LLM-driven social simulators along these three key dimensions (Figure~\ref{fig:evaluation}). We first focus on the most basic element of the simulator, i.e., LLM-driven social generative agents. As discussed above, the design of these agents can be divided into three levels: minds, social behaviors, and their coupling methods (M-B coupling). At the mind level, researchers initially input a profile description into LLMs, enabling them to role-play and respond like a real person with a similar profile~\cite{cheng2024sociodojo}. However, such simple role-play cannot guarantee the quality of behavior generation. Consequently, an increasing number of studies, inspired by the pioneering work of Park et al.~\cite{park2023generative}, incorporate agentic module design such as profile, memory, reflection, and action, into their LLM-driven agents~\cite{gao2023s,li2024econagent}. In this way, these agents can exhibit more human-like behaviors and generate responses that are coherent, context-aware, and aligned with their designated profiles. Recently, some researchers have realized that agents designed purely based on the commonsense knowledge of LLMs lack the social intelligence needed to mimic a real social being. To improve this, they have drawn on some theories from psychology to create agents with socially intelligent designs~\cite{wang2024simulating,al2024project}. However, they do not organically integrate theories from multiple social science disciplines, which is central to our design of LLM-driven generative social agents.



At the behavioral level, simulated behaviors can be broadly categorized into two types. The first type includes complex behaviors, which involve multiple intricate steps and cannot be executed solely by the agent itself. These behaviors require interaction with other agents or the environment, such as socializing, engaging in economic activities, or navigating movement. The second type comprises simpler behaviors, such as sleeping, which are relatively straightforward and do not demand external interactions. To systematically evaluate these complex behaviors, such as movement, social interaction, and economic activities, we have developed specific evaluation criteria. For mobility behaviors, we examine whether the simulator simply models the switching of an agent’s position (i.e., relocation)~\cite{wang2024simulating} or incorporates the entire process of mobility trajectory~\cite{shao2024beyond,park2023generative}. For social behaviors, we assess whether the agents merely engage in basic interactions~\cite{pangself} or demonstrate organized social relationships, reflecting more human-like group dynamics~\cite{yang2024oasis,park2023generative,al2024project,mou2024unveiling}. For economic behaviors, we evaluate whether the agents recognize only the concept of ``resources'' (e.g., money in the real world or diamonds in Minecraft)~\cite{cheng2024sociodojo} or perform advanced activities, such as value-based resource exchanges grounded in logical reasoning and strategic decision-making~\cite{al2024project,li2024econagent}. In the case of simpler behaviors, we focus on the level of constraints in the simulated activities. These range from highly restricted tasks, such as choosing a favorite movie~\cite{zhang2024generative}, to more autonomous and creative undertakings, like organizing a party without external prompts~\cite{park2023generative,wang2024simulating}. 



After introducing the minds and behaviors of agents, we further focus on understanding how behaviors are generated from their minds, which we refer to as mind-behavior coupling. Some researchers have adopted implicit modeling approaches, relying on the planning, memory, and reasoning capabilities of LLMs to generate plausible behaviors~\cite{gao2023s,li2024econagent,tang2024gensim}. In contrast, others have leveraged established theories, (e.g. Maslow's Hierarchy of Needs~\cite{maslow1943theory} and Theory of Planned Behavior~\cite{ajzen1991theory}) to explicitly model how behaviors are driven by minds~\cite{wang2024simulating}. This explicit modeling aims to create behaviors that are not only plausible but also more closely aligned with human-like patterns~\cite{wang2024simulating}.


As discussed above, a realistic societal environment serves as the foundation for simulating authentic human behaviors and society. Current social simulators employ a range of strategies for environment design, each with its own strengths and limitations. Dataset-based environments~\cite{cheng2024sociodojo,tang2024gensim} rely on pre-existing data but lack the ability to provide dynamic, real-time feedback to agents' behaviors. For example, Sociodojo~\cite{cheng2024sociodojo} For example, Sociodojo~\cite{cheng2024sociodojo} incorporates pre-existing real-world time series data to provide these agents with a sense of the external world. Text-based environments~\cite{pangself,wang2024simulating}, often built based on LLMs, can offer some interactive feedback; however, their realism and objectivity remain questionable, limiting their reliability for simulating complex scenarios. Rule-based virtual environments, like Minecraft, provide richer and more objective feedback, but they still fall short of capturing the intricate complexity of real human social systems~\cite{li2024econagent,al2024project,yang2024oasis}. To advance toward a truly realistic social simulator, it is essential to design an environment that faithfully reflects the multifaceted nature of human society. Such an environment should integrate key dimensions of urban living, economic dynamics, and social relationships, while supporting diverse interactions among agents and providing feedback on their behaviors.

After evaluating LLM-driven social generative agents and their environments, we further extend our focus to examine the capabilities of the social simulation engine, particularly in terms of its scalability and its potential to support social science research. The scale is a key factor in determining its capacity to support research on complex social systems~\cite{sawyer2005social,ladyman2013complex,epstein1999agent}. We classify the supported scale into four levels: \textless{} 100, 100-1k, 1k-10k, and \textgreater{} 10k agents. Larger scales enable more intricate simulations and provide a platform for studying emergent phenomena in greater detail. Moreover, the engine’s ability to facilitate traditional social science methodologies, such as experiments, surveys, and interviews, is also important. The extent to which the system supports these methods directly influences its applicability across diverse research domains. By accommodating these methodologies, the engine can bridge the gap between simulation-based research and real-world social science, unlocking new opportunities for understanding and addressing societal challenges. Overall, Table~\ref{fig:overview} shows the comparison of different LLM-driven social simulators across the three key dimensions. Existing platforms, although capable of simulating societies and human behaviors to some degree, face substantial limitations in various areas. Since these platforms were not specifically designed for social science research, they lack support for these methods. As a result, this aspect has not been included in the table.


\begin{figure}[t]
\centering
\includegraphics[width=\textwidth]{Figure/evaluation.pdf}
\caption{Evaluation framework for LLM-driven social simulators.}
\label{fig:evaluation}
\end{figure}



In this paper, we propose AgentSociety, a comprehensive large-scale social simulator designed to integrate LLM-driven social generative agents, a realistic societal environment, and a robust simulation engine. This simulator not only supports large-scale agents and their interactions, but also facilitates advanced social science research. Figure~\ref{fig:overview} provides an overview of AgentSociety and outlines the structure of this paper. AgentSociety consists of three key components: LLM-driven social generative agents, a realistic societal environment, and a powerful simulation engine that supports large-scale interactions. Extensive experiments demonstrate AgentSociety’s superior performance and its potential as a valuable testbed for various social experiments. In particular, we first introduce LLM-driven social generative agents in Section~\ref{sec:social_agents}, which discusses the designs for agents' minds, complex social behaviors, and their coupling in detail. We then demonstrate our real-world societal environment in Section~\ref{sec:environment}, which includes our modeling of urban, social, and economic spaces. Furthermore, we illustrate our large-scale social simulation engine in Section~\ref{sec:engine} and evaluate its performance in Section~\ref{sec:performances}. Finally, we show a typical one day life of our simulated agents in Section~\ref{sec:one_day_life} and launch several social experiments based on our proposed large-scale social simulator in Sections~\ref{sec:polarization} - \ref{sec:hurricane}. These examples, covering polarization (Section~\ref{sec:polarization}), the spread of inflammatory messages (Section~\ref{sec:infl_message}), universal basic income~\ref{sec:ubi}, and external shocks of hurricanes (Section~\ref{sec:hurricane}), demonstrate the validity and authenticity of our proposed simulator.









% 我们具体是怎么做的,如何切分的,文章组织是什么样的?



\begin{figure}[t]
\centering
\includegraphics[width=\textwidth]{Figure/research_overview.pdf}
\caption{Overview of the proposed social simulator AgentSociety. AgentSociety consists of three key components: LLM-driven social generative agents, a realistic societal environment, and a powerful simulation engine that supports large-scale interactions. Based on these components, AgentSociety not only demonstrates superior computational performance but also serves as a valuable testbed for various social experiments.}
\label{fig:overview}
\end{figure}






% 文献对比???
\begin{table}[t]
\caption{Comparison of LLM-driven agents and social simulators.}
\hspace*{-1cm}
\resizebox{1.2\linewidth}{!}{%
\begin{tabular}{|p{2cm}|p{0.5cm}|p{0.5cm}|p{0.5cm}|p{0.5cm}|p{0.5cm}|p{0.5cm}|p{0.5cm}|p{0.5cm}|p{0.5cm}|p{0.5cm}|p{0.5cm}|p{0.5cm}|p{0.5cm}|p{1cm}|p{1cm}|}
\hline 
Model & \multicolumn{3}{c|}{Minds} & \multicolumn{2}{c|}{Mobility} & \multicolumn{2}{c|}{Economics} & \multicolumn{2}{c|}{Social} & \multicolumn{2}{c|}{Others} & \multicolumn{2}{c|}{M-B} & Scale &Env. \\ 
\cline{2-15}
 & RP. & AM. & SI. & Relo. & Traj. & Res. & Exc.& Int. & Rel. & Con. & Free & Infl. & Dri. & \# &  \\
 \hline
D2A~\cite{wang2024simulating} & \checkmark & \checkmark & \checkmark & \checkmark &  &  &  &  &  & \checkmark & \checkmark &  &  & \textless{}100 & Text \\
Ecoagent~\cite{li2024econagent} & \checkmark & \checkmark &  &  &  & \checkmark & \checkmark &  &  &  &  & \checkmark &  & 100-1k & Rules\\
OASIS~\cite{yang2024oasis} & \checkmark & \checkmark &  &  &  &  &  & \checkmark & \checkmark &  &  & \checkmark &  & \textgreater{}10k & Rules\\
GA1000~\cite{park2024generative} & \checkmark & \checkmark & \checkmark &  &  &  &  &  &  &  &  &  &  & 1k-10k &$\times$ \\
MATRIX~\cite{pangself} & \checkmark & \checkmark &  &  &  &  &  & \checkmark &  &  &  &  &  & \textless{}100 & Text\\
Sociodojo~\cite{cheng2024sociodojo} & \checkmark &  &  &  &  & \checkmark &  &  &  &  &  &  &  & \textless{}100 & Data\\
GA~\cite{park2023generative} & \checkmark & \checkmark &  & \checkmark & \checkmark &  &  & \checkmark & \checkmark & \checkmark & \checkmark & \checkmark & \checkmark & \textless{}100 & Rules\\
GenSim~\cite{tang2024gensim}& \checkmark & \checkmark &  &  &  &  &  &  &  & \checkmark &  &  &  & \textgreater{}10k & Data\\
Project Sid~\cite{al2024project} & \checkmark & \checkmark & \checkmark & \checkmark & \checkmark & \checkmark & \checkmark & \checkmark & \checkmark & \checkmark & \checkmark & \checkmark & \checkmark & 1k-10k & Rules \\
AgentScope~\cite{gao2024agentscope} & \checkmark & \checkmark &  &  &  &  &  & \checkmark &  &  &  &  &  & N/A & N/A\\
HiSim~\cite{mou2024unveiling}& \checkmark & \checkmark &  &  &  &  &  & \checkmark & \checkmark &  &  &  &  & 100-1k & Rules\\
S3 & \checkmark & \checkmark &  &  &  &  &  & \checkmark & \checkmark &  &  &  &  &  1k-10k& Rules\\
Agent4Rec~\cite{zhang2024generative} & \checkmark & \checkmark &  &  &  &  &  &  &  & \checkmark &  &  &  & 1k-10k &Rules\\
RecAgent~\cite{wang2024user} & \checkmark & \checkmark &  &  &  &  &  & \checkmark & \checkmark & \checkmark &  &  &  & 100-1k &Rules \\
Sotopia~\cite{zhou2023sotopia} & \checkmark &  &  &  &  &  &  & \checkmark &  &  &  &  &  & \textless{}100 &$\times$ \\
Casevo~\cite{jiang2024casevo} & \checkmark & \checkmark &  &  &  &  &  & \checkmark & \checkmark &  &  &  &  & 100-1k & Rules\\
 \hline 
 Ours & \checkmark & \checkmark & \checkmark & \checkmark & \checkmark & \checkmark & \checkmark & \checkmark & \checkmark & \checkmark & \checkmark & \checkmark & \checkmark & \textgreater{}10k & Society \\
 \hline
\end{tabular}
}
\end{table}
