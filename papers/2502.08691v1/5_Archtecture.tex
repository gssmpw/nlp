\section{Large-scale Social Simulation Engine}\label{sec:engine}

\subsection{Overview}

% 逻辑
% 【总起】尽管社会模拟器看起来像是由多智能体系统 with 工具调用(环境),但真实社会中人的独立思考决策与语言交流驱动的协作方式促使我们重新思考大规模智能体系统架构设计。
% 【现状】现有的多智能体框架一般基于智能体间协作,通过构建处理流程图的方式来规范各个角色的执行顺序,例如......。
% 【问题】但在真实社会中,每个人的行为决策都是自身基于当前的记忆、想法与环境的约束做出的,并不总是依赖来自其他智能体或环境的特定输入。
% 【技术挑战】因此,如何在系统设计上更好地还原这种社会模拟中的“异步”现象,并根据这一特点实现更大规模的智能体执行,是系统架构设计的关键。
% 【解决方案】参考现实社会的基本逻辑,我们将每个智能体视为独立的模拟单元,相互之间不存在任何显式的依赖关系,智能体之间通过消息机制完成信息交换并互相影响。
% 【技术实现】分布式计算 -> ray,高性能消息传输-> MQTT,平衡并行度与有限的IO资源 -> group,还提供了其他有用的工具,包括xxx,xxx,xx
% 【社会实验支持】xxx
% 图:各个组件之间的逻辑关系

% TODO: 下面各个小节分别叫啥

Although the large-scale social simulator introduced in this paper may appear as a simple combination of LLM multi-agent systems (social agents) and tool call (environment), the reality of human society characterized by independent thinking in decision-making and collaboration driven by language communication promotes us to fundamentally rethink the system architecture design and implementation of the large-scale social simulation engine.
Existing multi-agent execution frameworks such as CAMEL~\cite{li2023camel} and AgentScope~\cite{gao2024agentscope} typically take inter-agent collaboration as the foundational principle of system architecture design, constructing Standard Operating Procedures (SOPs) through message-passing processes among agents to determine the sequence of agent execution.
Such frameworks are particularly well-suited for multi-agent execution scenarios with well-defined agent execution sequences, as exemplified by programming tasks~\cite{hong2023metagpt,qian2024chatdev} and conversational games~\cite{xu2023exploring}, as they can significantly simplify complex agent interaction processes.
However, in real-world contexts, individual behavioral decision-making emerges from the autonomous integration of current memory, cognitive states, and environmental constraints, rather than being strictly contingent upon specific inputs from other agents or environments.
Therefore, a pivotal challenge in system architecture design lies in faithfully simulating this "asynchronous" phenomenon within the large-scale social simulator, while strategically leveraging such intrinsic behavioral patterns to optimize simulation execution efficiency.

As a solution to the aforementioned challenge, we draw inspiration from the operational logic of the real world by treating each agent as an independent simulation unit.
There are no explicit dependencies or execution orders between agents.
Instead, they exchange information and mutually influence each other through a messaging system.
For the parallel execution of independent simulation units, to fully leverage the multi-core computing capabilities of modern computer systems and distributed computing paradigms for horizontal scaling of simulation scale, we adopt the highly mature Ray framework~\cite{moritz2018ray} to implement distributed computing and conceal I/O latency through Python's asyncio mechanism.
As the simulation scale increases, we identify that TCP port resources become a bottleneck, and excessive reliance on inter-process communication to coordinate the entire system leads to decreased execution efficiency.
Therefore, we introduce an intermediate structure named agent group to enable multiple agents to operate within a single process, thereby balancing communication costs with parallel acceleration while allowing connection reuse for network calls such as LLM API calls.
For the messaging system supporting inter-agent information exchange, it needs to support massive concurrent connections, high-throughput and reliable message transmission, though being latency-insensitive.
This characteristic closely resembles the Internet of Things (IoT) scenarios where applications must handle message delivery across millions of devices.
Inspired by this similarity, we have introduced MQTT\footnote{\url{https://mqtt.org/}}, the communication protocol that has achieved tremendous success in IoT communications, to construct our agent messaging system.
Following best practices from existing agent execution frameworks, we provide comprehensive utilities including multiple LLM API adapters, a retry mechanism,a JSON parser, a metric recorder, and diverse logging capabilities including both local file output and database storage.
Leveraging these logged processes, we develop real-time interactive visualization interfaces.
Furthermore, specifically designed for social experimentation requirements, we implement a specialized toolbox including interviews, surveys, and interventions.

In the following content, we will first introduce the whole system architecture in Section~\ref{sec:sim:arch}.
We then dive into the key designs including group-based distributed execution in Section~\ref{sec:sim:group} and MQTT-powered agent messaging system in Section~\ref{sec:sim:mqtt}.
The utilities and toolbox for social experiments will be discussed in Section~\ref{sec:sim:util} and Section~\ref{sec:sim:exp} respectively.

\subsection{System Architecture}\label{sec:sim:arch}

\begin{figure}[ht]
\centering
\includegraphics[width=\textwidth]{Figure/arch.pdf}
\caption{System architecture of the large-scale social simulation engine.}
\label{fig:arch}
\end{figure}

% 架构,介绍“实验-run”的划分,各个组件的功能和所属的层级

% 挑战和技术问题、设计

% 为了避免不必要的重复开发并享受开源社区进步的收益,大型社会模拟器的系统架构主要采用先进的、成熟的开源软件或库进行构建。在以下的讨论中,我们将进行一次社会模拟成为一次实验。如图1所示,整个架构主要分为由所有实验共享的共享服务、每个实验独占的计算任务以及可选的GUI组成。
% 共享服务部分包含以下几个服务:
% - LLM API:大模型是整个模拟器最重要的组件,是智能体的灵魂。对模拟器来说,大模型通过API提供标准的“请求-响应”过程,以处理智能体prompt所描述的任务。LLM API可以使用公开的服务如OpenAI、DeepSeek等,也可以使用本地推理引擎进行部署如vllm、ollama。
% - MQTT Server: 本架构中使用MQTT这一高性能物联网协议进行智能体之间的消息传输以模拟真实人类社会中基于语言的相互影响与协作方式。MQTT服务器提供了将消息以符合协议要求的方式投递到客户端的功能。这里我们选择的是emqx这一个高性能mqtt服务器。
% - Database: 数据库在本架构中只用于存储模拟结果以便后续分析或可视化。我们选择了知名的PostgreSQL以使用其独有的高性能批量写入SQL命令COPY FROM来保证数据存储效率。
% - Metric Recorder: 对模拟过程的特定指标记录有利于研究人员对比不同实验的实验结果以发现有价值的科学结论。为了方便研究人员的协作,我们选择使用具有中心服务器的mlflow而不是基于本地存储的指标记录工具如tensorboard。

To avoid unnecessary redundant development and leverage the advancements from the open-source community, the system architecture of the large-scale social simulation engine is primarily constructed using advanced and mature open-source software or libraries.
As shown in Figure~\ref{fig:arch}, the overall system architecture consists of shared services common to all social simulation experiments, simulation tasks corresponding to each experiment, and an optional GUI component.

The shared services include the following components:
\begin{itemize}
    \item \textbf{LLM API:}
    LLMs serve as the most critical component of the simulator, acting as the "soul" of agents.
    For the simulation engine, LLMs provide a standard "request-response" process through APIs to handle tasks described in agent prompts.
    The LLM API can utilize public services like OpenAI\footnote{\url{https://platform.openai.com/docs/overview}} or DeepSeek\footnote{\url{https://platform.deepseek.com/}}, or be deployed through local inference engines such as vllm~\cite{kwon2023efficient} and ollama\footnote{\url{https://ollama.com/}}.
    \item \textbf{MQTT Server:}
    The architecture employs the high-performance IoT protocol MQTT for inter-agent message transmission, simulating real-world human language interactions and collaboration patterns.
    The MQTT server enables protocol-compliant message delivery to clients.
    We select a high-performance MQTT server named emqx\footnote{\url{https://www.emqx.com/en}} for this purpose.
    \item \textbf{Database:}
    The database in the architecture is solely used for storing simulation results for subsequent analysis or visualization.
    We choose PostgreSQL\footnote{\url{https://www.postgresql.org/}} for its unique high-performance batch writing capability through the SQL command \texttt{COPY FROM}, ensuring efficient data storage.
    \item \textbf{Metric Recorder:}
    Recording specific metrics during simulations enables researchers to compare experimental results across different studies and derive valuable scientific insights.
    To facilitate research collaboration, we opt for mlflow\footnote{\url{https://mlflow.org/}} with centralized server capabilities, rather than local storage-based metric recording tools like tensorboard\footnote{\url{https://www.tensorflow.org/}}.
\end{itemize}

The primary purpose of the large-scale social simulation engine is to execute social simulation experiments, which consist of a set of computational tasks comprising environment simulation and agent execution.
In implementation, an experiment corresponds to an \texttt{Agent Simulation} object. This object will create and manage environment simulators through subprocess mechanisms, while utilizing the Ray framework to create multiple agent groups that execute agents through multi-process-based distributed computing.
According to Ray framework design, each agent group functions as a Ray actor operating within a single process.
The Ray framework enables managed Ray actors to work across different machines.
By adding other machines to the head node during Ray cluster initialization, distributed computing can be easily achieved to horizontally scale computational resources for social simulation tasks.
Within each agent group, clients connecting to the shared services and the environment simulator are initialized, enabling multiple agents to work concurrently using these client connections.
Different experiments will share all shared services while utilizing distinct Ray clusters and environment simulators to prevent mutual interference.
The GUI with a backend and a frontend connects to the database and MQTT server to enable the visualization of simulation results, and allows users to directly interact with simulated agents by conducting dialogues or sending questionnaires.

In conclusion, the system architecture integrates multiple cutting-edge open-source softwares to deliver comprehensive capabilities for social simulation and enables researchers to focus exclusively on social agent design, including distributed computing, LLMs, message transmission, data storage, and metric management.

\subsection{Group-based Distributed Execution}\label{sec:sim:group}

% 表示从串行处理到并行处理到async+并行处理的提升

\begin{figure}[ht]
\centering
\includegraphics[width=\textwidth]{Figure/async.pdf}
\caption{Asynchronous multi-process parallel execution using Ray and asyncio.}
\label{fig:async}
\end{figure}

% 逻辑
% 挑战与主要考虑
% 解决方案设计
% More discussion
% 总结

% 根据上面关于系统架构的介绍,我们将多个智能体聚合为一组,称为agent group,并使用一个进程执行每组智能体,以达到分布式计算加速的效果。这一设计是为了解决有限的TCP端口资源与庞大的智能体数量之间的矛盾。具体来说,如果将每个个体视为一个进程并独立连接到共享服务以模拟真实世界中人的自主决策,大量的TCP连接将消耗掉MQTT服务器的TCP端口资源(上限为65535个),从而导致出现TCP端口资源不足而失败。因此,如何进行连接复用以降低TCP端口资源的占用并保证多智能体的独立执行是智能体执行所需要考虑的关键问题。

Based on the above description of the system architecture, we aggregate multiple agents into a group called an agent group, and use a single process to execute each group of agents to achieve the effect of distributed computing acceleration.
This design addresses the contradiction between limited TCP port resources and the massive number of agents.
Specifically, if each individual agent were treated as a separate process and independently connected to shared services to simulate autonomous human decision-making in the real world, the large number of TCP connections would exhaust the TCP port resources of the MQTT server, the database, and the metric recorder (with an upper limit of 65,535 ports), leading to failures due to insufficient TCP port resources.
Therefore, how to implement connection reuse to reduce TCP port resource consumption while ensuring independent execution of multiple agents constitutes a critical issue that needs to be addressed in agent execution.

% 针对这一问题,我们将智能体平均分为多个智能体组,并为每个组配置一个LLM API客户端、一个MQTT客户端、一个环境客户端、一个数据库客户端、一个指标记录者客户端。所有的客户端都采用异步调用的方式实现,并支持并行调用。由于大模型驱动的社会模拟是IO密集型的处理任务,其主要用时在LLM的调用、环境的调用上,因此通过由asyncio提供的异步IO能力将允许多个智能体并行发出LLM请求并充分利用CPU处理Agent设计中的计算任务,有效避免等待请求返回导致的时间浪费。异步调用的连接复用方式也不引入对智能体执行顺序的特定限制,保证了智能体之间的独立性。通过构建智能体组的方式,整个系统所需的TCP端口数可以降低至智能体组数的倍数(取决于是否使用可选的服务如指标记录),避免TCP端口耗尽带来的问题。

To address this issue, we evenly distribute agents into multiple agent groups, each configured with an LLM API client, an MQTT client, an environment client, a database client, and a metrics recorder client.
All clients are implemented using asynchronous calls and support parallel execution.
Since LLM-driven social simulations are I/O-intensive tasks, primarily consuming time in LLM calls and environment interactions.
Leveraging the asynchronous I/O capabilities provided by asyncio and multi-process parallel execution powered by Ray shown in Figure~\ref{fig:async}, the engine allows multiple agents to concurrently send LLM requests while fully utilizing CPU resources for computational tasks in agent design like running gravity models, effectively avoiding time waste caused by waiting for LLM responses.
The asynchronous approach with connection reuse does not impose specific constraints on agent execution order, ensuring independence among agents.
By organizing agents into groups, the total number of TCP ports required by the system can be reduced to a multiple of the number of agent groups (depending on optional services such as metrics recording), thereby preventing issues caused by TCP port exhaustion.
However, immutable fixed-number grouping would result in the overall system efficiency being constrained by the slowest group.
Therefore, adaptive load balancing and dynamic scheduling across groups represent an important direction for future research.

In summary, by combining group-based asynchronous and parallel execution with distributed implementation to enhance simulation efficiency, we successfully address the critical issue of execution failures caused by port exhaustion.

\subsection{MQTT-powered Agent Messaging System}\label{sec:sim:mqtt}

\begin{figure}[ht]
\centering
\includegraphics[width=\textwidth]{Figure/mqtt.pdf}
\caption{Overview of MQTT-powered agent messaging system.}
\label{fig:mqtt}
\end{figure}

% 智能体之间的通信是基于大模型多智能体构建社会模拟的必要环节。文本消息在智能体之间的传输用于模拟真实世界中人与人之间基于语言沟通的交流与合作,将促使群体行为的涌现从而进一步逼近真实世界运转规律。进一步扩展,构建能够连接智能体的消息系统将允许用户或外部程序直接访问、干预智能体的行为,从而开展更丰富的社会实验或交互式应用。

% 需求和目的
Communication between agents is an essential component in constructing social simulations based on LLM-powered multi-agent systems.
The transmission of textual messages among agents simulates real-world human communication and collaboration through language, which will facilitate the emergence of group behaviors and thereby further approximate the operational laws of the real world.
Expanding on this, developing a messaging system that connects agents will enable users or external programs to directly access and intervene in the agents' behaviors.
This will support more sophisticated social experiments and interactive applications.

% 技术方案
% 为了实现上述智能体消息系统,我们需要能够支持向数十万特定ID投递消息的消息系统。在物联网领域,MQTT是用于解决数百万物联网设备与控制中心相互通信的协议,采用发布/订阅的设计方式,发布方将消息发送到指定的ID中,而订阅方则监听指定ID或指定前缀。MQTT通过轻量级的数据包结构设计来匹配物联网低带宽条件,并在较低的资源占用情况下实现了对消息可靠的承诺。尽管这一协议是设计用于物联网设备连接,但十分契合大型社会模拟器中智能体通信的需求,因此我们使用该协议实现智能体通信系统。
To achieve the aforementioned agent messaging system, we need a messaging system capable of delivering messages to hundreds of thousands of specific agents by IDs.
In the IoT domain, MQTT is a protocol designed to enable communication between millions of IoT devices and control centers, employing a publish/subscribe architecture.
Publishers send messages to specified IDs, while subscribers monitor specific IDs or designated prefixes to receive messages.
MQTT utilizes a lightweight packet structure tailored for low-bandwidth IoT environments, delivering reliable messaging with minimal resource consumption.
Although originally designed for IoT device connectivity, this protocol aligns perfectly with the communication requirements of agents in the large-scale social simulator.
We therefore adopt this protocol to implement our agent messaging system as shown in Figure~\ref{fig:mqtt}.
During implementation, we designate the following topics and their corresponding message meanings:
\begin{itemize}
    \item \texttt{exps/<exp\_uuid>/agents/<agent\_uuid>/agent-chat}: Used for sending messages from one agent to the target agent within the experiment.
    \item \texttt{exps/<exp\_uuid>/agents/<agent\_uuid>/user-chat}: Used for users to send chat messages to the target agent within the experiment via the GUI.
    \item \texttt{exps/<exp\_uuid>/agents/<agent\_uuid>/user-survey}: Used for users to send structured JSON-formatted surveys to the target agent within the experiment via the GUI.
\end{itemize}
Under this topic configuration, each agent only needs to subscribe to messages prefixed with \texttt{exps/<exp\_uuid>/agents/<agent\_uuid>/}.
This approach reduces development costs while maintaining compatibility for future extensions to the messaging system's functionality.

% 结论
In summary, by integrating the advanced IoT communication protocol MQTT, we achieve low-cost development to simultaneously connect hundreds of thousands of agents while ensuring reliable agent message transmission.
This solution also enables user input through GUIs and supports future functional extensions, thereby filling a crucial gap in the architecture of the large-scale social simulation engine.

\subsection{Utilities}\label{sec:sim:util}

In addition to the aforementioned key technical designs for social agents, we also provide a rich set of utilities to facilitate the development of social agents.
The design philosophy of most of these utilities is primarily inspired by AgentScope~\cite{gao2024agentscope}.

\textbf{LLM API Adapter.}
We implement calls to OpenAI API-compatible LLMs through the \texttt{openai} python library\footnote{\url{https://pypi.org/project/openai/}}, including OpenAI, DeepSeek, Qwen\footnote{\url{https://bailian.console.aliyun.com/}}, etc.
Additionally, we also support calls to ChatGLM\footnote{\url{https://bigmodel.cn/}}.
Through the LLM API adapter, we allow users to freely select their preferred or partnered LLMs for inference by modifying configurations, which enhances the system's flexibility and compatibility.

\textbf{Retry Mechanism.}
To prevent abnormal results returned by the LLM API from affecting experiments, the system incorporates a retry mechanism.
When invoking the LLM API, if erroneous responses are detected, the system will automatically reinitiate the call. The default number of retries is set to 3.

\textbf{JSON Parser.}
Since prompts often require LLMs to return responses in JSON format to facilitate parsing into program-processable results, we develop a JSON parser.
This parser automatically identifies JSON code blocks in responses, removes Markdown code block delimiters (prefix and suffix), and converts the content into Python objects.

\textbf{Metric Recorder.}
To assist researchers in recording various statistical metrics during experiments, such as total GDP and agent state averages, we adapt the mlflow API and implement a parallel-safe metric logging utility class and functions.

\textbf{Logging and Saving.}
Logging and saving simulation processes and results serve as the foundation driving subsequent data analysis and visualization.
Saving as much data as possible will facilitate richer and more profound research insights.
Accordingly, we design both local file storage using the AVRO format \footnote{\url{https://avro.apache.org/}} and PostgreSQL database as online storage.
Both storage approaches employ similar schemas to archive social agent profiles, agent states during simulations, thoughts and dialogues, and survey outcomes.
Additionally, experimental metadata including IDs, names, durations, configurations, and error messages is systematically recorded.

\textbf{GUI.}
To help people directly observe the behavior of social agents in environments and allow users to engage in direct conversations or surveys with the agents, we develop a GUI program.
The GUI program includes functionalities for experiment management, survey administration, and real-time monitoring or playback of experiments.
During real-time monitoring, users can interact with agents through instant conversations or send surveys, with these communications being transmitted via the agent messaging system while awaiting responses.
Additionally, during both real-time monitoring and playback, users can view agent-related records stored in a PostgreSQL database, including their locations, profiles, status history, thought and dialogue history, and survey response histories.
We aim for this GUI to help users build intuitions about agent societies and facilitate deeper analysis and applications.


\subsection{Toolbox for Social Experiments}\label{sec:sim:exp}

In the realm of social sciences, various research methods are employed to study human behavior, motivations, and responses in different contexts. In real-world settings, it is often difficult to find controlled environments for conducting social experiments that mirror the complexity of human interactions. This is where interventions come into play—creating specific social experimental conditions that allow researchers to simulate and manipulate real-life scenarios. Besides, two widely used methods are interviews and surveys, both of which allow researchers to collect data from individuals, explore their opinions, and understand the underlying psychological factors influencing their actions. These methods are vital for generating insights into how people think, feel, and act in specific situations.  Large-scale agent-based simulations provide a powerful tool for addressing these challenges, enabling the design of controlled experiments with a high degree of realism.
This section details the tools available for conducting interventions, interviews, and surveys with agents in a social experiment. These tools offer flexibility in data collection and manipulation, supporting the design and execution of robust social experiments.

\textbf{Intervention.} Intervention refers to the manipulation of an agent’s behavior or state to observe how changes influence its actions, thoughts, and emotional responses. Interventions are crucial for setting up experimental conditions in social experiments. There are three primary types of intervention in our system:

\begin{itemize}
    \item \textbf{Agent Configuration}: This type of intervention involves directly modifying the internal settings of the agent before the simulation starts. These settings may include altering an agent's personality traits, goals, or preferences. Since this intervention occurs before the simulation, it ensures that the agent’s behavior aligns with the experimental conditions right from the start.
    \item \textbf{State Manipulation}: This intervention occurs during the simulation and allows researchers to modify the agent’s current state. By altering an agent's profile, mood, or ideas, researchers can influence its behavior. For instance, modifying the agent’s emotions can impact its decision-making and social interactions.
    \item \textbf{Message Notification}: This method involves sending a text message to the agent, triggering a response. For example, a message such as "Severe weather changes expected, a hurricane is coming" could be used to observe how the agent adjusts its plans or behavior in response to external threats. This type of intervention can be introduced at any point during the simulation, offering flexibility in creating different experimental conditions.
\end{itemize}

The intervention process can be summarized as:
\[
\text{Pre-Simulation Configuration} \xrightarrow{\text{Agent Settings}} \text{Agent Behavior Start}
\]
\[
\text{During Simulation} \xrightarrow{\text{Memory Manipulation}} \text{Behavior Adjustments}
\]
\[
\text{During Simulation} \xrightarrow{\text{Message Notification}} \text{Behavior Modification}
\]
These intervention techniques allow for dynamic and flexible modifications of the agent’s behavior, providing valuable insights into the impact of specific changes on social interactions and decision-making.

\textbf{Interview and Survey.} An interview is a process of one-on-one or group-based questioning and answering, typically used to gather detailed, qualitative information from participants. In our platform, users can directly communicate with agents through either a front-end interface or programmatically via code. The system uses MQTT to distribute the user’s questions to the relevant agents. The agent then answers these questions by processing both its internal state and the surrounding environmental context. Importantly, this process is designed so that the agent can respond without interrupting its ongoing actions. This allows for real-time interaction while maintaining the flow of the agent’s behavior. The interaction flow is depicted as:
\[
\text{Question} \xrightarrow{\text{MQTT}} \text{Agent Processing} \xrightarrow{\text{Answer}} \text{User Response}
\]
This ensures that the agent can participate in interviews seamlessly while continuing its primary tasks and goals.

A survey is a structured form of data collection, where a series of interview questions are combined based on a specific set of rules. These rules include response formats (e.g., multiple-choice, ranking) and the unique design elements determined by the survey creator (e.g., question order). Surveys are typically used to gather quantitative data across a larger sample, offering a broader perspective on trends or patterns.
In our system, the structured survey is distributed to agents via MQTT, just like interviews. However, the primary difference is that the agent’s responses follow a predefined structure, ensuring consistent data collection across multiple agents. The agent processes the survey questions sequentially, answering them from top to bottom after analyzing the format and response rules. This structured data is then compiled into a format that is easy for the user to analyze. The survey response process is modeled as:
\[
\text{Survey} \xrightarrow{\text{MQTT}} \text{Agent Parsing and Responding} \xrightarrow{\text{Structured Answer}} \text{Data Processing}
\]
This ensures that data collection is organized, reliable, and easy to process for social experiment analysis.

The ability to conduct interventions, interviews, and surveys within our platform provides a powerful toolkit for researchers conducting social experiments. These tools offer a structured approach to data collection and behavioral modification, making it possible to simulate real-world social conditions in a controlled environment. The flexibility to manipulate an agent’s settings, memory, and responses in real-time ensures that a wide variety of social experiment scenarios can be tested, from understanding individual behaviors to studying collective dynamics. This makes large-scale agent simulations an invaluable resource for conducting complex social science research.



