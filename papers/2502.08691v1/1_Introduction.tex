\newcommand{\rvs}[1]{{\textcolor{black}{#1}}}
\section{Introduction}


% 如何理解世界 understand,explain, predict, generate
% \textit{``What I cannot create, I do not understand''} -- Richard Feynman

Over the past decades, researchers across various fields -- spanning social science, physical science, and computational science -- have made significant efforts to understand the functioning and development of society along two dimensions: explanation and prediction~\cite{hofman2021integrating,lazer2009computational,lazer2020computational}. Explanation seeks to identify the causal mechanisms and underlying factors that drive observed social patterns, aiming to offer a deeper understanding of why certain outcomes occur in society~\cite{hofman2021integrating,hedstrom2010causal}. \rvs{Gaining such an understanding often requires conducting social experiments,  which can be costly to implement and pose substantial practical and ethical challenges.} On the other hand, prediction focuses on using data to forecast future events or emergent behaviors, often without delving into the causal factors, but instead concentrating on the accuracy of anticipating future trends~\cite{breiman2001statistical,hofman2021integrating}. The framework that combines explanation and prediction methods, forms the foundation of \textit{computational social science}~\cite{hofman2021integrating}. However, as the physicist Richard Feynman famously stated, \textit{``What I cannot create, I do not understand''}, suggesting that true understanding goes beyond merely observing, explaining, or predicting human behavior~\cite{epstein2012generative,epstein1999agent}; instead, it requires the ability to generate the systems we study~\cite{epstein2012generative,epstein1999agent}. In this vein, a new paradigm of ``generative social science'' is rapidly emerging, which emphasizes the bottom-up simulation of social systems to gain in-depth insights into their underlying mechanisms and predict future outcomes~\cite{epstein2012generative,epstein1999agent}. A well-established method in this paradigm is agent-based modeling, which aims to model complex social dynamics by simulating the actions and interactions of agents~\cite{epstein1999agent,macal2005tutorial,wilensky2015introduction}. This method, \rvs{compensating for the limitations of social experiments}, is widely applied in studies across social science~\cite{gilbert2000build,berry2002adaptive,gao2023s}, political science~\cite{de2014agent,laver2011party,li2024econagent}, economics~\cite{arthur2006out,feng2012linking}, and other interdisciplinary fields~\cite{an2021challenges,gao2024large,piao2023human,piao2025emergence}, advancing their understanding of human behaviors and society through various simulations. However, the broader impact of these studies has been hindered by the same long-standing issue: to what extent can these simulations authentically replicate the complexities of real human society?

Indeed, the authenticity of these simulations depends on to what extent the most basic components -- i.e., the agents -- behave like humans. However, most existing agents, driven by rules~\cite{epstein1996growing}, equations~\cite{helbing1995social}, or even machine learning models~\cite{zheng2022ai}, are limited in their ability to generate human-like behaviors. For example, when simulating opinion dynamics, people's opinions are often represented as scalars or vectors, and their interactions as equations~\cite{baumann2021emergence}. While this modeling approach offers valuable insights, it is still far from reality, as people typically communicate using natural language, rather than numeric values. Fortunately, recent advances in large language models (LLMs) have shown promise in creating human-like agents~\cite{gao2024large,wang2024survey,xi2023rise}. Numerous studies have pointed out that after being empowered by LLMs, these agents agents have generated human-like ``minds''~\cite{strachan2024testing,li2023camel,kosinski2024evaluating}. They not only possess basic cognition abilities, such as learning~\cite{xi2023rise,wang2024survey}, reasoning~\cite{wei2022chain}, and decision-making~\cite{li2024econagent,gao2024large}, but also demonstrate the capability to understand and predict the thoughts and intentions of others~\cite{kosinski2024evaluating,strachan2024testing}. Furthermore, beyond exploring these agents' minds, some researchers have investigated their potential to mimic human behaviors~\cite{li2024econagent,piao2025emergence,shao2024beyond,yan2024opencity,feng2024agentmove,horton2023large,gao2023s,park2023generative}. Their investigations have revealed that, through elaborate designs incorporating domain knowledge, these LLM-driven agents can generate social behaviors, such as mobility~\cite{shao2024beyond,yan2024opencity,feng2024agentmove}, employment~\cite{li2024econagent,horton2023large}, consumption~\cite{li2024econagent,horton2023large}, and social interactions~\cite{gao2023s,park2023generative}. While great efforts have been made to examine specific facets of these agents, simulating a comprehensive social being remains largely underexplored.


% more is different, 仅仅依靠单个智能体还是不行,需要多个智能体做social simulation。在他们的交互中evolve 出新的东西
As the famous sociologist George Herbert Mead stated, ``The self is something which has a development; it is not initially there, at birth, but arises in the process of social experience and activity.''~\cite{mead1934mind} Therefore, the mere incorporation of minds and behaviors into these generative agents is insufficient to create a social being; instead, social experience and activity, emerging from interactions with other agents and the environment, are crucial. Several recent studies have provided substantial empirical evidence supporting this point. Some have discovered that the collaboration of multiple agents can generate believable social organizing behaviors~\cite{park2023generative} and solve complex tasks~\cite{li2023camel,cheng2024sociodojo}. Moreover, as the number of agents further scales up, large-scale interactions among them can lead to the emergence of social norms and collectives~\cite{lai2024evolving,piao2025emergence}. Meanwhile, the environment not only serves as the ground for interactions among agents, but also provides critical feedback that guides their behaviors~\cite{al2024project,li2024econagent,wang2023voyager,zhu2023ghost}. For example, the widely-adopted gaming environment ``Minecraft'' provides feedback, such as crafting materials, tools, or resources, enabling agents to adapt their behaviors, solve tasks, and gain civilizational progression~\cite{wang2023voyager,zhu2023ghost,al2024project}. Overall, as highlighted by these studies, a scalable framework supporting large-scale interactions and a realistic environment is foundational to simulating a comprehensive social being and society. However, the current investigation of both remains limited.




% To address the above gaps, we propose AgentSociety, a large-scale social generative simulator that incorporates LLM-driven social agents, a realistic societal environment, and large-scale interactions both among agents and between agents and the environment. Specifically, following social theories from a broad range of fields, including psychology~\cite{maslow1943theory,ajzen1991theory}, economics~\cite{christiano2005nominal} and behavioral sciencel~\cite{zipf1946p}, we first design a framework for LLM-driven social agents. These agents are endowed with human-like ``minds'', which include emotions, needs, motivations, and cognition of the external world, as well as complex human behaviors such as mobility, employment, consumption, and social interactions, all driven by their minds. To provide these agents with a solid foundation for their interactions and self-evolution, we then create a realistic societal environment that seamlessly integrates urban, social, and economic spaces. Furthermore, we recognize that society is a typical complex system, where the scale plays a crucial role in shaping its properties and emergent phenomena. Therefore, we develop a large-scale social simulation engine with distributed computing and an MQTT-powered high-performance messaging system to support simulations with up to [xxx @Zhangjun] agents and their [xxx @Zhangjun] interactions. Based on the proposed large-scale social simulator, we successfully reproduce behaviors, outcomes, and patterns observed in several real-world social experiments. This not only demonstrates the capability of the proposed simulator as an authentic replica of social beings and society, but also underscores its potential applications for social scientists and policymakers.

To address the above gaps, we propose \textit{AgentSociety}, a large-scale social generative simulator that incorporates \textit{LLM-driven social generative agents}, \textit{a realistic societal environment}, and \textit{large-scale interactions} both among agents and between agents and the environment. Specifically, following social theories from a broad range of fields, including psychology~\cite{maslow1943theory,ajzen1991theory}, economics~\cite{christiano2005nominal} and behavioral sciencel~\cite{zipf1946p}, we first design a framework for LLM-driven social agents. These agents are endowed with human-like ``minds'', which include emotions, needs, motivations, and cognition of the external world. Their behaviors such as mobility, employment, consumption, and social interactions are dynamically driven by these internal mental states. Beyond individual agents, we construct a realistic societal environment that seamlessly integrates urban, social, and economic spaces, providing a rich foundation for agent interactions and self-evolution. At its core, society emerges from the bottom-up interactions among individuals, where agent-level interactions collectively give rise to complex social structures and phenomena. Recognizing that social systems exhibit emergent behaviors shaped by scale, we develop a large-scale social simulation engine equipped with distributed computing and an MQTT-powered high-performance messaging system. This enables simulations with up to 10k agents, each engaging in an average of 500 interactions per day, capturing the intricate dynamics of large-scale social systems. Based on the proposed large-scale social simulator, we successfully reproduce behaviors, outcomes, and patterns observed in four real-world social experiments, including polarization, inflammatory message spread, the effects of universal basic income policies, and the impact of external shocks like hurricanes. These experiments not only cover social research methods, such as surveys, interviews, and interventions, but also demonstrate the simulator's ability to replicate social dynamics, unlocking new possibilities for social scientists and policymakers. Overall, AgentSociety marks a paradigm shift in AI for social science, enabling large-scale, high-fidelity simulations that overcome traditional experimental limitations in costs, scalability, and feasibility. By leveraging LLM-driven social generative agents, it facilitates deeper analysis, prediction, and intervention in complex social systems, laying the foundation for computational social science 2.0.

% These experiments not only encompass fundamental social research methods, such as surveys, interviews, and interventions, but also showcase the simulator’s ability to replicate social dynamics, opening up new possibilities for social scientists and policymakers.




% This not only demonstrates the capability of the proposed simulator as an authentic replica of social beings and society, but also underscores its potential applications for social scientists and policymakers.
% 做了四个层次的贡献,明天讨论
% 【社会人智能体】 llm驱动的类人模拟 心智->多种社会行为
% 【真实的城市环境】提供真实的城市环境 
% 【大规模交互】 大规模多智能体之间的交互,more is different,涌现
% 【范式改变】应用,改变社会学研究/研究、理解社会社会的式,从explanation,prediction,到generation







