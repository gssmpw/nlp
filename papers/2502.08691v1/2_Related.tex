% \section{Background}
% @ Piao Jinghua 
% @ Zhang Jun 

% Storyline:

% 1. Why should we simulate society? (the first motivation) % 必要性

% TODO

% 2. Why can we simulate society with LLM agents? % 可行性

% LLM agents can act as human beings in games, economy, social network, ...

% 3. How do we design LLM agents to make it act as human being (and what is human in society) % 社会人

% 4. How to support large-scale LLM agents execution and how to provide social simulation environment?
% Let's go to the details.

\section{Related Works}
The literature related to the work mainly consists of two kinds of work: large language model-driven agents and social simulation.

\subsection{LLM-driven Agents}

Large Language Models (LLMs) exhibit astonishing language capabilities~\cite{wang2024survey}. Since the language ability is one of the most fundamental abilities of human intelligence, LLMs demonstrate excellent performance in numerous tasks. Furthermore, researchers use LLMs as ``brains" to construct LLM-driven agents, endowing them with memory management, interactive interfaces, and expanded action space~\cite{kwon2023efficient}.

The research on LLM-driven agents mainly consists of two parts~\cite{xi2023rise}. One type uses LLM agents as a tool for intelligent decision making~\cite{zhang2024large,ruan2023tptu,huang2023benchmarking}. By leveraging their individual capabilities, these agents solve practical problems and serve as human assistants. This kind of work primarily focuses on reasoning ability, tool using, learning, etc.
For example, Boiko~\textit{et al.}~\cite{boiko2023autonomous} construct an agent with large language models to autonomously conduct chemical experiments, by providing computers to browse experimental tutorials and devices controlling interface for the experiments.
The other type of LLM-driven agents starts with the concept of agent-based modeling and agent-based simulation~\cite{li2024econagent,gao2023s,gao2024large}. Given that large language model agents exhibit human-like behaviors, they can be made to engage in role-playing and imitating human behavior. 

The primary research tries to reproduce the human response with large language model agents~\cite{park2023generative,zhang2024generative,gao2023s} or explain the gap better human and LLM agents~\cite{huang2024social}.
These works borrow the memory and reasoning mechanism of humankind~\cite{guo2023empowering,zhang2024survey} to design various internal mechanisms on the basis of large language models as brains.
Some other work~\cite{pang2024self,liu2023training} further design fine-tuning or alignment strategies to enhance the agent's role-playing abilities.

Overall, large language model agent is a field that is rapidly advancing with the development of LLM and our understanding of LLM, especially in the ability to simulate real humans. However, there is no platform to really unleash agent's ability to simulate in the real world, which we aim to address in this paper.

% A simulated environment is also constructed, allowing agents to interact with the environment and with other agents. 

%  LLMs 的类人能力
%  LLM 智能体的相关研究
%  LLM 多智能体的相关研究

\subsection{Social Simulation}
Social simulation stands as a core technology in computational social science, generally categorized into macrosimulation~\cite{troitzsch1996social} and microsimulation~\cite{figari2021empirical}. The former typically models interactions between macro-level variables by defining complex equations, while the latter adopts a bottom-up approach to simulate emergent phenomena through granular agent behaviors. Among these, microsimulation, which is often termed agent-based simulation, has become the more widely adopted method. It operates by defining rules or models to govern individual agent behaviors, with the goal of replicating and predicting real-world societal dynamics.


Cellular automata~\cite{wolfram1983cellular} represent a seminal class of agent-based modeling and simulation in early research. Another notable example is Game of Life~\cite{conway1970game}, which simulates the evolution and interactions of lifeforms in a two-dimensional grid-based world. Subsequent advancements, such as Sugarscape model~\cite{epstein1996growing}, expanded the action space of simulated agents, enabling the exploration of broader phenomena through rule-governed agent behaviors.  For the applications in social sciences, primary ABM research has focused on social interaction, economics, etc. For social interaction, these key studies investigate collective behavior (e.g., cooperation) ~\cite{goldstone2005computational} and system dynamics (e.g., information propagation and crowd dynamics)~\cite{namatame2016agent}, etc. For the economic domain, representative work ~\cite{gallegati2012reconstructing,chen2012agent,axtell2022agent} targets macroeconomic systems, market dynamics, etc.

In recent years, within the field of computational social science, researchers have increasingly adopted deep neural network-based models to simulate individual agents~\cite{van2017deep,kavak2018big}. However, significant limitations persist, primarily due to the inherent complexity of modeling human behavior.  Over the past two years, large language models (LLMs) have emerged prominently, demonstrating human-like cognitive capabilities, including contextual understanding, logical reasoning, and interactive communication~\cite{orru2023human,lampinen2024language}. This breakthrough has catalyzed growing interest in LLM-driven agent-based simulations~\cite{gao2024large}.  

A pioneering example is the Generative Agent~\cite{park2023generative}, which constructs a small-scale society within a 2D game engine. Here, LLM-powered agents autonomously plan daily activities, exchange information, and adapt to environmental stimuli. Researchers observed intriguing emergent phenomena, such as self-organized information diffusion and collaborative group behaviors, revealing the potential of LLMs to capture nuanced social dynamics. The authors further combine the large language models with the real data of 1,000 human participants to simulate feedback~\cite{park2024generative}. Acerbi~\textit{et al.}~\cite{acerbi2023large} simulate the information propagation and find human-like bias in the results of large language model agents. 
 S3~\cite{gao2023s} further utilizes LLM-empowered agents to simulate individual-level and population-level behaviors within the social network. In the economic domain, LLM agent-based simulations have also achieved significant progress. For instance, EconAgent~\cite{li2024econagent} has developed a macroeconomic market simulation framework powered by large language agent-based models, with the simulation outcomes aligning closely with established stylized facts. 
Beyond social interactions and economic behaviors, researchers are expanding agent-based simulations to explore a broader spectrum of topics in social sciences. For example, Zhang~\textit{et al.}~\cite{zhang2024electionsim} built a framework with large language model agents to predict the results of the 2024 US Presidential Election.

Existing studies remain narrowly focused on isolated problems and rely on simplified environments, such as text-based~\cite{aher2023using} or simplistic game environments~\cite{park2023generative}, with limited attention to real-world environment fidelity, revealing notable limitations. Besides, researchers have begun exploring methods to scale up LLM-driven agents to support larger populations \cite{tang2024gensim,wang2024user,yang2024oasis}. Nevertheless, these efforts continue to grapple with critical limitations, including computational inefficiencies and inaccurate user behaviors. Our proposed AgentSociety, consisting of LLM-driven generative social agents, a realistic societal environment, and a powerful engine for large-scale simulations, overcomes these limitations and advances the field of social simulation.


% 过去使用ABM方法进行社会模拟的相关研究
% 近期使用LLM/LLM 智能体模拟的相关研究