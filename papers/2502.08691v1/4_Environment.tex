\section{Real-world Societal Environment}\label{sec:environment}

\subsection{Overall}

\begin{figure}[ht]
\centering
\includegraphics[width=\textwidth]{Figure/env.pdf}
\caption{Overview of the societal environment.}
\label{fig:env}
\end{figure}

% 逻辑
% 【承上启下】正如上文所提到的,要实现对社会人智能体的构建与模拟,移动行为、社交行为、经济行为是必不可少的要素。
% 【过渡】而这些行为的发生是存在真实世界的载体,仅依赖大模型的知识与能力很可能会发生“幻觉”,导致模拟的结果偏离真实情况。
% 【提出目标】因此,我们要为智能体的模拟提供尽可能真实可靠的模拟环境,环境将具有(1)尽可能真实地建模真实世界及其运行逻辑(2)构建虚拟世界的数据、(3)为智能体提供交互接口,形成一个支撑社会模拟的客观世界载体。这种方式也使得智能体的设计只需关注人的主观行为逻辑,而不需要处理客观世界中的精确数学计算等LLM相对不擅长的领域,简化了系统的设计并使研究人员可以更加聚焦关键任务。
% 【实现】环境基于专家知识构建,以代码的行为固化为模拟环境程序。
% 【插图】和3.1的总图对应,体现支撑能力。

According to the introduction above, in the design and simulation of social agents, mobility behaviors, social behaviors, and economic behaviors like employment and consumption are essential external capabilities.
In the real world, the manifestation of these behaviors is grounded in corresponding objective entities, not merely in human subjective cognition.
For example, mobility behaviors imply a continuous change in spatial and temporal location.
Therefore, if we rely solely on the knowledge and capabilities of LLMs to conduct such simulations without incorporating modeling of the operational laws and constraints of the real world, the simulation results are likely to be influenced by the "hallucinations" of LLMs~\cite{huang2023survey}, resulting in outcomes that diverge from actual realities.
To address this issue, we need to provide a realistic and reliable environment for simulating social agents.
The environment should include the following features:
\begin{itemize}
    \item Appropriate modeling of real-world operational principles to reflect physical constraints and costs, and provide feedback on behaviors;
    \item Environmental data sourced from the real world or aligned with real-world principles;
    \item Interfaces to enable interaction with agents.
\end{itemize}
Such an environment will serve as a virtual mapping of the objective aspects of the world in social simulations, enabling the design of social agents to focus solely on subjective human behavioral logic.
By offloading tasks such as numerical computations, where LLMs cannot guarantee absolute accuracy, this approach simplifies agent design and allows researchers to concentrate on core objectives.

% 整个环境被分为3个空间。城市空间中构建了支持移动模拟的城市道路网并包含了AOI、POI等元素,实现了常见的开车、步行、乘坐公交、乘坐出租车等出行方式,能够为智能体提供真实的位置变化反馈,以及不同出行方式对应的时间和金钱的代价。社交空间在智能体社会网络的基础上,提供了面对面的社交行为支持与社交媒体的抽象。社交空间中的一个关键且真实的设计是监管者,监管者将读取社交媒体中消息,根据算法规律进行过滤并支持对特定的用户或传播进行封禁。经济空间还原宏观经济学中的基本元素,以账本为实现建模了人、公司、政府、银行的经济行为,包括雇佣劳动、消费、税收、利息等,并提供了用于统计经济指标的统计局。

In alignment with this objective, we encode expert knowledge to construct a real-world societal environment as depicted in Figure~\ref{fig:env}, designed to support the simulation of mobility behaviors, social behaviors, and economic behaviors.
The entire environment is divided into three spaces.
The urban space constructs a city road network supporting mobility simulation and incorporates elements such as Area of Interest (AOI) and Point of Interest (POI).
It implements common transportation modes including driving, walking, public transit, and taxi services, providing agents with realistic positional feedback as well as temporal and monetary costs associated with different travel choices.
The social space builds upon agents' social networks, offering support for offline and online social interactions.
A critical and authentic design feature in this space is the supervisor, which monitors social media content, filters messages based on algorithmic rules, and enforces bans on specific users or connections when necessary.
The economic space reconstructs fundamental elements of macroeconomics, modeling economic behaviors of individuals, firms, governments, and banks through the implementation of account books.
These behaviors encompass employment, consumption, taxation, interest mechanisms, while a dedicated statistical bureau is established to monitor economic indicators like GDP.
The following subsections will elaborate on the corresponding environmental spaces for the agent behaviors respectively.

% 以下各小节的基本逻辑(一部分一段)
% 1. 总起:对应3.3的介绍,过渡下来,提出要构建XX模拟环境,以支撑XXX需求
% 2. 介绍如何建模、以及运行逻辑
% 3. 介绍有哪些数据,数据的来源
% 4. 提供的接口
% 5. 实现方式
% 6. 总结

\subsection{Urban Space}

% 1. 城市空间环境的重要性
To address the needs of social agents for moving and interacting with different places, accurate modeling of urban space is essential. 
The urban space must capture both the physical movement pattern of individuals and their interactions with diverse urban locations. 

% 2. 城市空间建模方法
Inspired by traffic simulation platforms such as SUMO~\cite{behrisch2011sumo} and CityFlow~\cite{zhang2019cityflow}, also leveraging the spatial abstraction schemas of OpenStreetMap\footnote{\url{https://openstreetmap.org/}}, the urban space is structured into two interdependent layers, the static infrastructure attributes and dynamic mobility behaviors. 

The static attribute layer includes road networks, defined by lanes, roads and junctions to encode traffic accessibility, as well as functional zones, which are Areas of Interest (AOIs) and Points of Interest (POIs). 
AOIs delineate regions with specific purposes, such as residential neighborhoods or commercial districts, while POIs represent granular interaction targets like retail stores.

The dynamic behavior layer extends this static foundation by simulating multi-modal mobility through a discrete time-stepping mechanism. 
Individual movements, including positions, speeds, and accelerations are updated dynamically according to kinematic principles and predefined rules.
Operational logic begins with agents formulating movement intentions based on their internal needs and goals, then these intentions are translated into specific instructions guiding individuals' movements within the space, which include driving, walking, taking the bus, or taking a taxi.
For all means of transportation, path-planning algorithms generate optimal routes.
Driving follows the IDM model~\cite{treiber2000congested} for acceleration and the MOBIL model~\cite{kesting2007general,feng2021intelligent} for lane-changing decisions.
Pedestrians navigate sidewalks at a constant speed and follow the traffic signals at junctions to avoid collisions with vehicles.
Buses operate on fixed schedules, while passengers conduct the processes of boarding, alighting, and transferring. 
For taxis, a global dispatch system simulates sending the nearest available taxi to respond to ride requests, ensuring efficient service and minimal wait times for passengers.


% 3. 数据来源与软件实现
For the simulation environment to function accurately, we apply rich data sources. 
Road networks and AOIs are extracted from OpenStreetMap\footnote{\url{https://openstreetmap.org/}}, undergoing topological simplification to produce structured representations. 
POI data, acquired via API from SafeGraph\footnote{\url{https://www.safegraph.com/}}.

We implement Python-based APIs to bridge the simulation space and agents,providing  bidirectional interaction capabilities. 
Configuration interfaces allow for initializing agent positions, assign travel plans (e.g., destinations and transportation modes), and reset simulation states. 
Query interfaces enable real-time monitoring of agent kinematics status and simulation metadata such as simulation timestamps. 

% 4. 总结
By harmonizing static urban infrastructure, dynamic mobility behaviors, and multi-source geospatial data, our urban space establishes a high-fidelity decision-making sandbox for agents. 



\subsection{Social Space}

% 以下各小节的基本逻辑(一部分一段)
% 1. 总起:对应3.3的介绍,过渡下来,提出要构建XX模拟环境,以支撑XXX需求
% 2. 介绍如何建模、以及运行逻辑
% 3. 介绍有哪些数据,数据的来源
% 4. 提供的接口
% 5. 实现方式
% 6. 总结

% 社交行为是构建智能体社会的前置条件,社会行为的建模与模拟将允许智能体之间相互影响、相互协作,产生更丰富和真实的社会现象。 因此,社会环境中增加社交空间是尤为重要的。社交空间中的主要组成部分是社交网络,这由用户载入。社交网络建模了人与人之间的关系,包含了每个智能体与其他智能体的连接关系与连接强度。这将用于智能体评估社交的选择对象。社交网络与其中的关系和强度在模拟过程中是可变的。基于社交网络,社交空间同时包含了线上社交与线下社交。尽管智能体设计时主要关注线上社交行为,但基于空间位置临近关系的线下社交行为依然是在真实环境的构建中不可或缺的部分。对于线上社交,为了尽可能模拟现实世界中的社交媒体运行逻辑,我们还引入了监管者的概念,监管者将识别线上社交消息内容,根据用户指定的算法或规则过滤消息,并支持对特定用户或连接的封禁,从而模拟社交媒体平台对信息传播的干预过程。
Social behavior is a prerequisite for constructing an agent society.
The occurrence of social behaviors requires the support of an authentic social environment.
The social environment provides management of social relationships, as well as modeling and simulation of social behaviors, which will enable mutual influence and collaboration among agents, generating richer and more authentic social phenomena.

Therefore, the incorporation of the social space within the societal environment is particularly crucial.
The primary component of the social space is the social network, which is provided and loaded by users.
Social networks model relationships between individuals, encompassing the connections and connection strengths between each agent and others.
This network will be used by agents to evaluate potential social interaction targets.
Both the relationships and connection strengths within social networks are mutable during simulations.
Based on social networks, social spaces encompass both online and offline interactions.
Although agent design primarily focuses on online social behaviors, offline interactions based on spatial proximity remain an indispensable component in constructing realistic environments.
For online social interactions, to realistically simulate the operational logic of social media platforms, we also introduce the concept of the supervisor.
The supervisor will identify content in online social messages, filter messages according to user-specified algorithms or rules, and support the blocking of specific users or connections, thereby simulating the intervention process of social media platforms in information propagation.

% 在实现上,社会空间中的社交网络存储为智能体的数据项,线下社交与线上社交均简化为通过智能体消息系统向对应的目标发送消息。监管者则实现为消息发送前的预处理中间级,并提供一个中心化的程序用于集中处理消息并完成规则与算法的更新与下发。
In implementation, the social network is stored as data items within agents.
Both offline and online social interactions are simplified into sending messages to specific targets through the agent message system which will be introduced in Section~\ref{sec:sim:mqtt}.
The supervisor is implemented as preprocessing middleware before message transmission, and a centralized program is provided to handle message processing collectively for updating rules and algorithms.

In summary, the social space not only supports the simulation of realistic social interactions between agents but also establishes intervention capabilities over social propagation within agent societies.
This framework will serve as a crucial foundation for conducting research on real-world social propagation phenomena using LLM-driven social agents.

\subsection{Economic Space}
The economic space includes the modeling of several key elements in the macroeconomics~\cite{wolf2013multi, li2024econagent}.
Specifically, firms convert the labor input of social agents into goods production and pay the corresponding wages to the agents.
Furthermore, firms adjust the wages of agents and goods price flexibly based on the supply and demand relationships in the consumption market. 
The government levies income tax on agents' earnings according to specified tax rates.
The banks pay interest to agents based on their savings each year, with the interest rate adaptively adjusted according to the Taylor Rule~\cite{wolf2013multi}.
The National Bureau of Statistics regularly compiles macroeconomic indicators, such as real GDP, average working hours per person, and per capita consumption levels.

% income, expenditure, tax, interest, statistic

% 与李念的整合
Building upon the modeling of the four key economic entities—firms, agents, the government, and banks—the economic simulator further captures the dynamic processes and interactions that drive the functioning of a realistic economic system. By integrating income generation, expenditure, savings, taxation, and policy-driven adjustments, the simulator provides a comprehensive representation of economic cycles.

Agents are the fundamental economic participants, generating income through labor in firms. This income is subject to taxation, with a portion deducted based on a progressive tax structure, while the remaining disposable income is allocated between consumption and savings. Agents use their disposable income to purchase goods and services, thereby fueling market demand. The funds allocated to savings are deposited into banks, where they accrue interest, influencing future consumption and investment decisions.

Firms act as producers in the economy, utilizing labor from agents to generate production output. They pay wages to agents, creating a cyclical flow of income within the system. Firms dynamically adjust goods prices and wages in response to market supply and demand, ensuring equilibrium in the consumption market. Their revenue comes from the sales of goods, which is reinvested into further production or held as retained earnings for future expansion. The government collects tax revenue from agents based on their earnings and redistributes financial resources to regulate economic activity. It influences the economy through fiscal policy, adjusting tax rates to manage income distribution and public sector financing. These funds may be directed toward public expenditures, which are not explicitly modeled in this simulator but could represent infrastructure, social programs, or government services. Banks function as financial intermediaries, receiving savings deposits from agents and providing them with interest payments. The interest rate is dynamically adjusted according to the Taylor Rule, incorporating monetary policy mechanisms into the simulation. This impacts agents’ saving and spending behavior, as higher interest rates encourage savings while lower rates stimulate consumption. Banks also serve as liquidity providers, ensuring the efficient allocation of financial resources within the economy.

The National Bureau of Statistics (NBS) compiles and analyzes macroeconomic indicators to monitor economic performance. It collects data on real GDP, income distribution, total tax revenue, per capita consumption, and average interest earnings. These aggregated statistics offer insights into systemic trends and policy effectiveness, enabling the evaluation of economic stability and the long-term impact of market dynamics. By structuring the simulation as an economic settlement system, where each entity interacts through well-defined financial flows, this framework achieves a holistic representation of economic operations. It serves as a valuable tool for analyzing the relationships between micro-level decision-making and macroeconomic trends, providing insights into market behaviors, policy interventions, and the overall functioning of economic systems.

% Discussion & Conclusion
While the economic simulator successfully models key economic flows, it does not explicitly capture critical aspects such as the goods market and the labor market. These omissions limit the realism of the model and represent areas for future refinement. The goods market is simplified by assuming firms adjust prices based on aggregate demand, but it does not account for detailed supply and demand dynamics, competition, or market shocks. Similarly, the labor market is abstracted, with agents receiving wages without modeling unemployment or the negotiation processes between workers and firms. Including these elements could enhance the model’s ability to simulate real-world economic fluctuations. Despite these limitations, the simulator provides valuable insights into the interactions between agents, firms, the government, and banks. Future improvements could focus on integrating more detailed models of the goods and labor markets, helping to better replicate complex economic systems and improving policy analysis. In conclusion, while the current model is a simplified representation, it offers a strong foundation for exploring economic interactions and can be further enhanced by incorporating missing market dynamics for more accurate predictions.

% Agent、
% 账本角度,结算系统,模拟经济的运转过程,工作和收入挂钩,储蓄和消费
% 统计局,
% 提供的接口
% 加个Disscussion,商品市场、劳动力市场,总结一下
