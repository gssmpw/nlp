\section{Preliminaries}

\subsection{Notations}

Consider a molecule characterized by its 3D structure and spectra, represented as \(\mathcal{M} = (\va, \vx, \mathcal{S})\). Here, \(\va \in \{1, 2, \ldots, 118\}^N\) specifies the atomic numbers, indicating the types of atoms within the molecule. The vector \(\vx \in \mathbb{R}^{3N}\) describes the conformation of the molecule, while \(\mathcal{S}\) represents its spectra. The parameter \(N\) denotes the number of atoms in the molecule. Note that the atoms are arranged in the same order in both \(\va\) and \(\vx\), ensuring consistency between the atomic numbers and their corresponding spatial coordinates.

\( \mathcal{S} = (\vs_1, \ldots, \vs_{|\mathcal{S}|}) \) represents the set of spectra for a molecule, where \(|\mathcal{S}|\) denotes the number of spectrum types considered. In our study, we focus on three types, so \(|\mathcal{S}|=3\). The first spectrum, \( \vs_1 \in \mathbb{R}^{601} \), is the UV-Vis spectrum, which spans from 1.5 to 13.5 eV with 601 data points at intervals of 0.02 eV. The second spectrum, \( \vs_2 \in \mathbb{R}^{3501} \), is the IR spectrum, covering a range from 500 to 4000 cm\(^{-1}\) with 3501 data points at intervals of 1 cm\(^{-1}\). The third spectrum, \( \vs_3 \in \mathbb{R}^{3501} \), is the Raman spectrum, with the same range and intervals as the IR spectrum. Together, these spectra provide a comprehensive description of the molecular characteristics across different spectral modalities.


\subsection{Pre-training 3D molecular representation via denoising}\label{sec:denoising}

Denoising has emerged as a prominent pre-training objective in 3D molecular representation learning, excelling in various downstream tasks. This method involves training models to predict and remove noise introduced deliberately into molecular structures. This approach is physically interpretable due to its proven equivalence to learning the molecular force field. 

\textbf{Equivalence between denoising and learning molecular force fields.}
The equivalence between coordinate denoising and force field learning is established by \cite{Coord}.
For a given molecule $\mathcal{M}$, perturb its equilibrium structure $\vx_0$ according to the distribution $p(\vx|\vx_0)$, where $\vx$ is the noisy conformation. Assuming the molecular distribution adheres to the energy-based Boltzmann distribution with respect to the energy function $E(\cdot)$, then
\begin{equation}
\begin{aligned}
\mathcal{L}_{\text{Denoising}}(\mathcal{M}) &= \mathbb{E}_{p(\vx|\vx_0)p(\vx_0)} \| \text{GNN}_\theta(\vx) - (\vx - \vx_0) \|^2 \\
&\simeq \mathbb{E}_{p(\vx)} \| \text{GNN}_\theta(\vx) - (-\nabla_{\vx} E(\vx)) \|^2,
\end{aligned}
\end{equation}
where $\text{GNN}_{\theta}(\vx)$ denotes a graph neural network parameterized by $\theta$, which processes the conformation $\vx$ to produce node-level predictions. The notation $\simeq$ signifies the equivalence of different objectives. 
The proof of this equivalence is provided in the \cref{appendix:proof}.
In prior research, the energy function $E(\cdot)$ has been defined in several forms.
Below are three representative studies.

\textbf{Energy function \Romannum{1}: mixture of isotropic Gaussians.}
In Coord~\citep{Coord}, the energy function is approximated using a mixture of isotropic Gaussians centered at the known equilibrium structures to replace the Boltzmann distribution, since these structures are local maxima of the Boltzman distribution. Leveraging the equivalence between the score-matching objective and denoising autoencoders~\citep{ScoreMatching-DAE}, the following denoising-based energy function $E_{\text{Coord}}(\cdot)$ is derived:
\begin{equation}
E_{\text{Coord}}(\vx) = \frac{1}{2 \tau_c^2} (\vx - \vx_0)^\top (\vx - \vx_0).
\label{eq:coord}
\end{equation}
Note that this objective is derived under the assumption of isotropic Gaussian noise, i.e., $p(\vx|\vx_0) \sim \mathcal{N}(\vx_0, \tau_c^2 \mI_{3N})$, where $\mI_{3N}$ represents the identity matrix of size $3N$, and the subscript $c$ indicates the coordinate denoising approach.

\textbf{Energy function \Romannum{2}: mixture of anisotropic Gaussians.}
Considering rigid and flexible components in molecular structures, isotropic Gaussian can lead to significant approximation errors.
To address the anisotropic distribution, Frad~\citep{Frad} introduces hybrid noise on dihedral angles of rotatable bonds and atomic coordinates, incorporating fractional denoising of the coordinate noise. The equilibrium structure $\vx_0$ is initially perturbed by dihedral angle noise $p(\boldsymbol{\psi}_a|\boldsymbol{\psi}_0) \sim \mathcal{N}(\psi_0, \sigma_f^2 I_m)$, followed by coordinate noise $p(\vx|\vx_a) \sim \mathcal{N}(\vx_a, \tau_f^2 \mI_{3N})$. Here, $\boldsymbol{\psi}_a, \boldsymbol{\psi}_0 \in [0,2\pi)^m$ represent to the dihedral angles of rotatable bonds in structures $\vx_a$ and $\vx_0$, respectively, with $m$ denoting the number of rotatable bonds. The subscript $f$ indicates the fractional denoising approach. Subsequently, the energy function is induced:
\begin{equation}
    E_{\text{Frad}}(\vx) \approx \frac{1}{2} (\vx - \vx_0)^\top \mathbf{\Sigma}_{\tau_f, \sigma_f}^{-1} (\vx - \vx_0),
\end{equation}
where $\mathbf{\Sigma}_{\tau_f, \sigma_f} = \tau_f^2 \mI_{3N} + \sigma_f^2 \mC\mC^{\top}$, and $\mC \in \mathbb{R}^{3N \times m}$ is a matrix used to linearly transform the dihedral angle noise into coordinate change, expressed as $\Delta \vx \approx \mC \Delta \boldsymbol{\psi}$.


\textbf{Energy function \Romannum{3}: classical potential energy theory.}
SliDe~\citep{SliDe} derives energy function from classical molecular potential energy theory~\citep{potential1,potential2}. In this form, the total intramolecular potential energy is mainly attributed to three types of interactions: bond stretching, bond angle bending, and bond torsion. The following energy function is derived:
\begin{equation}
\begin{aligned}
E_{\text{SliDe}}(\vr, \vtheta, \boldsymbol{\phi}) =& \frac{1}{2} [\vk^B \odot (\vr - \vr_0)]^\top (\vr - \vr_0) + \frac{1}{2} [\vk^A \odot (\vtheta - \vtheta_0)]^\top (\vtheta - \vtheta_0) \\
+& \frac{1}{2} [\vk^T \odot (\boldsymbol{\phi} - \boldsymbol{\phi}_0)]^\top (\boldsymbol{\phi} - \boldsymbol{\phi}_0),
\end{aligned}
\end{equation}
where $\vr \in (\mathbb{R}_{\geq 0})^{m_1}, \vtheta \in {[0,2\pi)}^{m_2}, \boldsymbol{\phi} \in {[0,2\pi)}^{m_3}$ represent vectors of the bond lengths, bond angles, and bond torsion angles of the molecule, respectively. $\vr_0, \vtheta_0, \boldsymbol{\phi}_0$ correspond to the respective equilibrium values. The parameter vectors $\vk^B, \vk^A, \vk^T$ determine the interaction strength.