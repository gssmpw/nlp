\section{Related Work}

\textbf{3D molecular pre-training.}
Molecular 2D structures are typically represented as graphs and modeled using graph learning methods~\citep{MPNN,GSLB,RAGraph}. However, 3D molecular structures provide critical geometric information that is essential for understanding physicochemical properties~\citep{MolScalingLaw,MolDataPruning,Pin-Tuning,DIVE}, which cannot be directly inferred from 2D graphs or SMILES representations~\citep{TGM-DLM}. Designing effective strategies for pre-training 3D molecular representations remains challenging due to the geometric symmetries inherent in 3D structures and their strong connection to physical knowledge, such as potential energy functions.

Denoising the geometric structure has been demonstrated as an effective strategy for 3D representation pre-training~\citep{SE3-DDM,3D-EMGP,GeoTMI,Uni-Mol,MolDiffusionSurvey}. Coordinate denoising (Coord)~\citep{Coord} first theoretically proves that the denoising objective is equivalent to learning 
the gradient of the potential energy with respect to positions, essentially the force field. 
Building on this work, fractional denoising (Frad)~\citep{Frad} 
introduces dihedral angle noise to optimize the sampling of low-energy structures.
Further, SliDe~\citep{SliDe} incorporates a more rigorous potential energy from classical mechanics.
Another line of research simultaneously leverages both 2D and 3D structures for pre-training molecular representations, addressing the complementarity of the two modalities~\citep{GeomGCL,Unified23D,MoleculeSDE,MoleculeJAE,MoleBlend} or the computational complexity of 3D structure determination~\citep{GraphMVP,3D-Informax,3D-PGT}.

Although these studies elucidate the relationship between molecular 3D structures and their energy states, they remain limited to the description of molecular energy states within classical mechanics, without considering the quantized energy level structures as described by quantum mechanics.

\textbf{Molecular spectroscopy.}
Molecular spectroscopy studies interactions between molecules and electromagnetic radiation. Analyzing spectra provides valuable insights into molecular structure, composition, and dynamics~\citep{Orbitrap-Astral}. 
When encountering unknown substances, researchers conduct spectroscopic measurements on samples and compare the observed spectra with libraries for identification. To expand library coverage, machine learning methods are widely used to predict molecules' spectra~\citep{DetaNet,NEIMS,DeepGP}.

Some studies incorporate physical principles into spectra prediction models as inductive biases, including molecular dynamics simulations via equivariant message passing \citep{PaiNN}, fragmentation~\citep{LC-MS,MolDiscovery,SCARF-Weave}, motifs~\citep{MoMS-Net}, and long-distance atomic interactions~\citep{MassFormer}. Another line of research approach bypasses spectral library comparison and directly performs de novo structure elucidation from spectra~\citep{MSNovelist,MIST,CNN-AM}.


Since different spectroscopic techniques offer complementary advantages, the joint analysis of multiple spectra can provide comprehensive information~\citep{multimodal-spectra}.
In this study, we encodes multiple spectra, and introduce them into molecular representation pre-training for the first time.