\section{Conclusion and Future Directions}

This paper challenges today's common paradigm of agent programming, which remains heavily influenced by traditional software engineering practices rooted in the now fading separation of programming languages and natural languages. We introduced the Ann Arbor Architecture as a new conceptual framework for agent-oriented programming of language models. To validate our key ideas, we developed Postline and reported our experiences in agent training.

Moving forward, our primary focus will be on developing a more advanced system of episodic memory. Episodic memory is fundamental to human cognition~\cite{tulving72} and has long been considered a crucial component of AI~\cite{soar}. In our framework, episodic memory will serve as a higher-level organizational structure beyond individual messages. To enable this, mechanisms must be designed for agents to autonomously create episode boundaries as interactions evolve, to selectively swap out older, less relevant episodes through MSRs (replacing them with concise summaries with keys for retrieval), and a new primitive must be introduced to allow agents to retrieve out-of-core episodes when needed.  We envision the full episodic memory as a tree-like structure, with the context covering the portion closest to the root. External data sources might be naturally incorporated via mechanisms that are referred to as retrieval-augmented generation~\cite{rag, gao24rag} today. 
Furthermore, episode clustering will provide a starting point to investigate agent reproduction and evolution.

Regarding the application of our platform, we observe that most existing agent frameworks are designed primarily to automate repetitive tasks, such as customer service, itinerary planning, and payment processing -- tasks that have traditionally been challenging for rule-based software due to their reliance on adaptive intelligence. While our proposed framework can be applied to such automation, we believe that the greatest strength of language models lies in their learning ability and creativity. Therefore, we intend to focus our agent development on applications in scientific and industrial research, where a higher failure rate is more acceptable as long as occasional breakthroughs can be achieved.
