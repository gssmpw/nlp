
\subsection{What enables \method's effectiveness}



\paragraph{Direction search enables optimization.}
We first evaluate the necessity of direction search by comparing \method with frozen models (GPT-4o-mini and Llama-3.1-8B-Instruct (Naive Model)) which refine personas directly. As shown in Figure~\ref{fig:analysis_radar_part1}(a), both baselines exhibit significant error increases after refinement, underscoring the critical role of direction search in effective optimization.

\begin{figure}[t]
    \centering
    \includegraphics[width=\linewidth]
    {figure/analysis_radar_part1.pdf}
    \caption{
    (a) Refinement performance of \method compared to frozen models across ten domains, using \(\varepsilon_{1 | \mathcal{S}_0}\) as pre-refinement baseline. 
    (b) Refinement under different reward settings. 
    Smaller areas indicate reduced errors and improved refinement relative to baseline.}
    \label{fig:analysis_radar_part1}
\end{figure}
\vspace{-2mm}

\paragraph{Balanced goals drive better optimization.}
The assessment of Direction Quality is critical to optimization performance. We compare \method’s balanced reward setting(equally weights previous, current, and future goals), with a future-focused reward and a decayed reward (decay factor = 0.5) prioritizing recent goals. As shown in Figure~\ref{fig:analysis_radar_part1}(b), \method consistently outperforms both baselines across all domains. These findings underscore the importance of balanced, goal-driven direction search in enabling effective persona refinement.



\newcolumntype{b}{>{\columncolor{brown!6}}c}
\newcolumntype{q}{>{\columncolor{blue!6}}c}
\newcolumntype{d}{>{\columncolor{blue!6}}c}
\renewcommand{\arraystretch}{0.9} % 将表格行高减小到默认值的 90%
\setlength{\belowcaptionskip}{2pt} % 表格标题与正文之间的间距
\vspace{-2em} % 减少间距
\begin{table*}[t]
\footnotesize
  \centering
   \resizebox{\textwidth}{!}{%
    \begin{tabular}{cqbbbbb}
    \toprule
    \multirow{1}[0]{*}{\textbf{Domain}} & \multicolumn{1}{c}{\textbf{Pre-Update}} & \multicolumn{5}{c}
    {\textbf{Post-Update}}\\
    \cmidrule(lr){2-2}
    \cmidrule(lr){3-7}
      & \multicolumn{1}{c}{\textbf{$\mathcal{S}_{old}$}}  & \multicolumn{1}{c}{\textbf{SlideRegen}} & \multicolumn{1}{c}{\textbf{FullRegen}} & \multicolumn{1}{c}{\textbf{IncUpdate}} & \multicolumn{1}{c}{\textbf{HierMerge}} & \multicolumn{1}{c}{\textbf{\method}} \\
    \midrule
    \rowcolor[gray]{0.95} \multicolumn{7}{c}{\textit{Previous Window Prediction ($\varepsilon_{prev|\mathcal{S}_{old/new}}$) - \textbf{Previous Preservarion}}} \\
    \midrule
        Recipe  & 0.57    &0.95\ \textcolor{gray}{(0.38$\uparrow$)}    &0.83\ \textcolor{gray}{(0.26$\uparrow$)}    &0.83\ \textcolor{gray}{(0.26$\uparrow$)}    &0.71\ \textcolor{gray}{(0.14$\uparrow$)}    &\underline{0.70}\ \textcolor{gray}{(0.13$\uparrow$)}    \\ 
        Book  & 0.78    &1.09\ \textcolor{gray}{(0.31$\uparrow$)}    &0.94\ \textcolor{gray}{(0.16$\uparrow$)}    &0.91\ \textcolor{gray}{(0.13$\uparrow$)}    &0.88\ \textcolor{gray}{(0.10$\uparrow$)}    &\underline{0.76}\ \textcolor{orange!60}{(0.02$\downarrow$)}    \\ 
        Clothing Shoes and jewelry  & 0.63    &1.13\ \textcolor{gray}{(0.50$\uparrow$)}    &0.96\ \textcolor{gray}{(0.33$\uparrow$)}    &0.94\ \textcolor{gray}{(0.31$\uparrow$)}    &\underline{0.77}\ \textcolor{gray}{(0.14$\uparrow$)}    &0.82\ \textcolor{gray}{(0.19$\uparrow$)}    \\ 
        Local Business  & 0.63    &1.10\ \textcolor{gray}{(0.47$\uparrow$)}    &0.95\ \textcolor{gray}{(0.32$\uparrow$)}    &0.91\ \textcolor{gray}{(0.28$\uparrow$)}    &0.74\ \textcolor{gray}{(0.11$\uparrow$)}    &\underline{0.73}\ \textcolor{gray}{(0.10$\uparrow$)}    \\ 
        Movies and TV  & 0.92    &1.17\ \textcolor{gray}{(0.25$\uparrow$)}    &1.03\ \textcolor{gray}{(0.11$\uparrow$)}    &1.00\ \textcolor{gray}{(0.08$\uparrow$)}    &0.98\ \textcolor{gray}{(0.06$\uparrow$)}    &\underline{0.85}\ \textcolor{orange!60}{(0.07$\downarrow$)}    \\ 
        MovieLens  & 0.76    &0.89\ \textcolor{gray}{(0.13$\uparrow$)}    &0.83\ \textcolor{gray}{(0.07$\uparrow$)}    &0.80\ \textcolor{gray}{(0.04$\uparrow$)}    &0.80\ \textcolor{gray}{(0.04$\uparrow$)}    &\underline{0.74}\ \textcolor{orange!60}{(0.02$\downarrow$)}    \\ 
        Arts Crafts and Sewing  & 0.49    & 0.81\ \textcolor{gray}{(0.32$\uparrow$)}    & 0.74\ \textcolor{gray}{(0.25$\uparrow$)}    & 0.68\ \textcolor{gray}{(0.19$\uparrow$)}    & 0.59\ \textcolor{gray}{(0.10$\uparrow$)}    &\underline{0.46}\ \textcolor{orange!60}{(0.03$\downarrow$)}    \\
        Automotive  & 0.55    &1.00\ \textcolor{gray}{(0.45$\uparrow$)}    &0.93\ \textcolor{gray}{(0.38$\uparrow$)}    &0.82\ \textcolor{gray}{(0.27$\uparrow$)}    &0.66\ \textcolor{gray}{(0.11$\uparrow$)}    &\underline{0.63}\ \textcolor{gray}{(0.08$\uparrow$)}    \\ 
        Sports and Outdoors  & 0.56  &0.99\ \textcolor{gray}{(0.43$\uparrow$)}    &0.87\ \textcolor{gray}{(0.31$\uparrow$)}    &0.85\ \textcolor{gray}{(0.29$\uparrow$)}    &0.67\ \textcolor{gray}{(0.11$\uparrow$)}    &\underline{0.66}\ \textcolor{gray}{(0.10$\uparrow$)}    \\ 
        Grocery and Gourmet Food  & 0.63    &1.13\ \textcolor{gray}{(0.50$\uparrow$)}    &1.00\ \textcolor{gray}{(0.37$\uparrow$)}    &0.95\ \textcolor{gray}{(0.32$\uparrow$)}    &\underline{0.71}\ \textcolor{gray}{(0.08$\uparrow$)}    &0.79\ \textcolor{gray}{(0.16$\uparrow$)}    \\ 
        
        
        \cdashlinelr{2-7}
        \textbf{Average} & \textbf{0.652} & \textbf{1.026} \textcolor{gray}{(0.374$\uparrow$)}  & \textbf{0.908} \textcolor{gray}{(0.256$\uparrow$)}  & \textbf{0.869} \textcolor{gray}{(0.217$\uparrow$)}  & \textbf{0.751} \textcolor{gray}{(0.099$\uparrow$)}  &\underline{\textbf{0.714}}\ \textcolor{gray}{(0.062$\uparrow$)}  \\ 


    \midrule
    \rowcolor[gray]{0.95} \multicolumn{7}{c}{\textit{Current Window Prediction ($\varepsilon_{curr|\mathcal{S}_{old/new}}$) - \textbf{Current Reflection}}} \\
    \midrule
        Recipe  &0.91    &0.78\ \textcolor{orange!60}{(0.13$\downarrow$)}    &0.84\ \textcolor{orange!60}{(0.07$\downarrow$)}    &\underline{0.41}\ \textcolor{orange!60}{(0.50$\downarrow$)}    &0.80\ \textcolor{orange!60}{(0.11$\downarrow$)}    & 0.44\ \textcolor{orange!60}{(0.47$\downarrow$)}    \\ 
        Book  &1.00    &0.92\ \textcolor{orange!60}{(0.08$\downarrow$)}    &0.97\ \textcolor{orange!60}{(0.03$\downarrow$)}    &0.41\ \textcolor{orange!60}{(0.59$\downarrow$)}    &0.91\ \textcolor{orange!60}{(0.09$\downarrow$)}    & \underline{0.35}\ \textcolor{orange!60}{(0.65$\downarrow$)}    \\ 
        Clothing Shoes and Jewelry  &1.00    &0.90\ \textcolor{orange!60}{(0.10$\downarrow$)}    &0.96\ \textcolor{orange!60}{(0.04$\downarrow$)}    &\underline{0.48}\ \textcolor{orange!60}{(0.52$\downarrow$)}    &0.90\ \textcolor{orange!60}{(0.10$\downarrow$)}    & 0.51\ \textcolor{orange!60}{(0.49$\downarrow$)}    \\ 
        Local Business  &1.04    &0.9\ \textcolor{orange!60}{(0.14$\downarrow$)}    &0.99\ \textcolor{orange!60}{(0.05$\downarrow$)}    &\underline{0.29}\ \textcolor{orange!60}{(0.75$\downarrow$)}    &0.93\ \textcolor{orange!60}{(0.11$\downarrow$)}    & 0.36\ \textcolor{orange!60}{(0.68$\downarrow$)}    \\ 
        Movies and TV  &1.12    &1.00\ \textcolor{orange!60}{(0.12$\downarrow$)}    &1.07\ \textcolor{orange!60}{(0.05$\downarrow$)}    &0.47\ \textcolor{orange!60}{(0.65$\downarrow$)}    &1.02\ \textcolor{orange!60}{(0.10$\downarrow$)}    & \underline{0.45}\ \textcolor{orange!60}{(0.67$\downarrow$)}    \\ 
        MovieLens  &0.87    &0.78\ \textcolor{orange!60}{(0.09$\downarrow$)}    &0.82\ \textcolor{orange!60}{(0.05$\downarrow$)}    &\underline{0.30}\ \textcolor{orange!60}{(0.57$\downarrow$)}    &0.80\ \textcolor{orange!60}{(0.07$\downarrow$)}    & 0.43\ \textcolor{orange!60}{(0.44$\downarrow$)}    \\ 
        Arts Crafts and Sewing  &0.76    &0.76\ \textcolor{gray}{(0.00$\uparrow$)}    &0.77\ \textcolor{gray}{(0.01$\uparrow$)}    &0.39\ \textcolor{orange!60}{(0.37$\downarrow$)}    &0.72\ \textcolor{orange!60}{(0.04$\downarrow$)}    & \underline{0.26}\ \textcolor{orange!60}{(0.50$\downarrow$)}    \\ 
        Automotive  &0.84    &0.81\ \textcolor{orange!60}{(0.03$\downarrow$)}    &0.88\ \textcolor{gray}{(0.04$\uparrow$)}    &0.38\ \textcolor{orange!60}{(0.46$\downarrow$)}    &0.81\ \textcolor{orange!60}{(0.03$\downarrow$)}    & \underline{0.27}\ \textcolor{orange!60}{(0.57$\downarrow$)}    \\ 
        Sports and Outdoors  &0.91  &0.79\ \textcolor{orange!60}{(0.12$\downarrow$)}    &0.84\ \textcolor{orange!60}{(0.07$\downarrow$)}    &0.37\ \textcolor{orange!60}{(0.54$\downarrow$)}    &0.82\ \textcolor{orange!60}{(0.09$\downarrow$)}    & \underline{0.36}\ \textcolor{orange!60}{(0.55$\downarrow$)}    \\ 
        Grocery and Gourmet Food  &1.19    &0.93\ \textcolor{orange!60}{(0.26$\downarrow$)}    &1.08\ \textcolor{orange!60}{(0.11$\downarrow$)}    &\underline{0.46}\ \textcolor{orange!60}{(0.73$\downarrow$)}    &1.05\ \textcolor{orange!60}{(0.14$\downarrow$)}    & 0.49\ \textcolor{orange!60}{(0.70$\downarrow$)}    \\ 
        
        
        \cdashlinelr{2-7}
        \textbf{Average} &\textbf{0.964} &\textbf{0.857}\ \textcolor{orange!60}{(0.107$\downarrow$)}  &\textbf{0.922}\ \textcolor{orange!60}{(0.042$\downarrow$)}  &\textbf{0.396}\ \textcolor{orange!60}{(0.568$\downarrow$)}  &\textbf{0.876}\ \textcolor{orange!60}{(0.088$\downarrow$)}  & \underline{\textbf{0.392}}\ \textcolor{orange!60}{(0.572$\downarrow$)}  \\ 
    
    \midrule
    \rowcolor[gray]{0.95} \multicolumn{7}{c}{\textit{Future Window Prediction ($\varepsilon_{fut|\mathcal{S}_{old/new}}$) - \textbf{Future Advancement}}} \\
    \midrule
        Recipe  & 0.91  & 0.92\ \textcolor{gray}{(0.01$\uparrow$)}   & 0.92\ \textcolor{gray}{(0.01$\uparrow$)}   & 0.91\ \textcolor{gray}{(0.00$\uparrow$)}   & 0.94\ \textcolor{gray}{(0.03$\uparrow$)}   & \underline{0.72}\ \textcolor{orange!60}{(0.19$\downarrow$)}   \\ 
        Book  & 1.01  & 1.06\ \textcolor{gray}{(0.05$\uparrow$)}   & 1.03\ \textcolor{gray}{(0.02$\uparrow$)}   & 0.96\ \textcolor{orange!60}{(0.05$\downarrow$)}   & 1.03\ \textcolor{gray}{(0.02$\uparrow$)}   & \underline{0.79}\ \textcolor{orange!60}{(0.22$\downarrow$)}   \\ 
        Clothing Shoes and Jewelry  & 1.03  & 1.09\ \textcolor{gray}{(0.06$\uparrow$)}   & 1.03\ \textcolor{gray}{(0.00$\uparrow$)}   & 1.00\ \textcolor{orange!60}{(0.03$\downarrow$)}   & 1.04\ \textcolor{gray}{(0.01$\uparrow$)}   & \underline{0.88}\ \textcolor{orange!60}{(0.15$\downarrow$)}   \\ 
        Local Business  & 1.04  & 1.06\ \textcolor{gray}{(0.02$\uparrow$)}   & 1.04\ \textcolor{gray}{(0.00$\uparrow$)}   & 0.97\ \textcolor{orange!60}{(0.07$\downarrow$)}   & 1.04\ \textcolor{gray}{(0.00$\uparrow$)}   & \underline{0.80}\ \textcolor{orange!60}{(0.24$\downarrow$)}   \\ 
        Movies and TV  & 1.18  & 1.14\ \textcolor{orange!60}{(0.04$\downarrow$)}   & 1.12\ \textcolor{orange!60}{(0.06$\downarrow$)}   & 1.06\ \textcolor{orange!60}{(0.12$\downarrow$)}   & 1.12\ \textcolor{orange!60}{(0.06$\downarrow$)}   & \underline{0.98}\ \textcolor{orange!60}{(0.20$\downarrow$)}   \\ 
        MovieLens  & 0.85  & 0.84\ \textcolor{orange!60}{(0.01$\downarrow$)}   & 0.83\ \textcolor{orange!60}{(0.02$\downarrow$)}   & 0.76\ \textcolor{orange!60}{(0.09$\downarrow$)}   & 0.82\ \textcolor{orange!60}{(0.03$\downarrow$)}   & \underline{0.73}\ \textcolor{orange!60}{(0.12$\downarrow$)}   \\ 
        Arts Crafts and Sewing  & 0.75  & 0.81\ \textcolor{gray}{(0.06$\uparrow$)}   & 0.77\ \textcolor{gray}{(0.02$\uparrow$)}   & 0.71\ \textcolor{orange!60}{(0.04$\downarrow$)}   & 0.75\ \textcolor{gray}{(0.00$\uparrow$)}   & \underline{0.43}\ \textcolor{orange!60}{(0.32$\downarrow$)}   \\ 
        Automotive  & 0.86  & 0.96\ \textcolor{gray}{(0.10$\uparrow$)}   & 0.92\ \textcolor{gray}{(0.06$\uparrow$)}   & 0.88\ \textcolor{gray}{(0.02$\uparrow$)}   & 0.90\ \textcolor{gray}{(0.04$\uparrow$)}   & \underline{0.61}\ \textcolor{orange!60}{(0.25$\downarrow$)}   \\ 
        Sports and Outdoors  & 0.97  & 0.94\ \textcolor{orange!60}{(0.03$\downarrow$)}   & 0.93\ \textcolor{orange!60}{(0.04$\downarrow$)}   & 0.89\ \textcolor{orange!60}{(0.08$\downarrow$)}   & 0.90\ \textcolor{orange!60}{(0.07$\downarrow$)}   & \underline{0.80}\ \textcolor{orange!60}{(0.17$\downarrow$)}   \\ 
        Grocery and Gourmet Food  & 1.25  & 1.14\ \textcolor{orange!60}{(0.11$\downarrow$)}   & 1.20\ \textcolor{orange!60}{(0.05$\downarrow$)}   & 1.10\ \textcolor{orange!60}{(0.15$\downarrow$)}   & 1.19\ \textcolor{orange!60}{(0.06$\downarrow$)}   & \underline{0.89}\ \textcolor{orange!60}{(0.36$\downarrow$)}   \\ 
        
        \cdashlinelr{2-7}
        \textbf{Average} & \textbf{0.985}  & \textbf{0.996}\ \textcolor{gray}{(0.011$\uparrow$)}   & \textbf{0.979}\ \textcolor{orange!60}{(0.006$\downarrow$)}   & \textbf{0.924}\ \textcolor{orange!60}{(0.061$\downarrow$)}   & \textbf{0.973}\ \textcolor{orange!60}{(0.012$\downarrow$)}   & \underline{\textbf{0.763}}\ \textcolor{orange!60}{(0.222$\downarrow$)}   \\ 
    \bottomrule
    \end{tabular}%
     }
     \caption{MAE results of previous, current, and future window prediction tasks using personas Pre- and Post- the first update with different methods. This table illustrates how well each method achieves the three high-level goals: Previous Preservation, Current Reflection, and Future Advancement. It presents the changes in MAE ($\left| \varepsilon_{t | \mathcal{S}_{old}}-\varepsilon_{t | \mathcal{S}_{new}} \right|$) relative to the old persona, with upward arrows (\textcolor{gray}{$\uparrow$}) indicating error increases and downward arrows (\textcolor{orange!60}{$\downarrow$}) indicating error reductions. Average results are highlighted in \textbf{bold}, and the best results are \underline{underlined}}
     \captionsetup{skip=2pt} % 调整 caption 与正文之间的间距
  \label{tab:goals}
\end{table*}
\FloatBarrier % 强制结束浮动
\vspace{-2em} % 减少间距


\paragraph{\method excels in identifying high-quality directions.}
Building on previous insights,
we further evaluate \method’s ability to identify high-quality refinement directions by analyzing its performance across three goals (Table~\ref{tab:goals}). 
\method demonstrates outstanding performance: minimizing previous forgetting with the smallest average MAE increment of 0.062 (\textit{Previous Preservation}); reducing current errors by 0.572 on average (\textit{Current Reflection}); and improving future predictions with an average 
reduction of 0.222 (\textit{Future Advancement}). Notably, \method surpasses all baselines across domains for \textit{Future Advancement}, demonstrating its capacity step-wise optimization. These results highlight \method’s ability to balance three goals for better direction search and continual optimization.


\begin{figure}[t]
    \centering
    \includegraphics[width=\linewidth]{figure/analysis_radar_part2.pdf}
    \caption{(a) Refinement performance across two RL iterations of \method. (b) Comparison of \method and fine-tuned \textit{IncUpdate} (Inc-FT). }
    \label{fig:analysis_radar_part2}
\end{figure}


\paragraph{Iterative RL enhances persona refinement.}
Guided by stage-specific objectives, \method’s two-stage iterative RL framework incrementally enhances refinement capabilities by leveraging progressively higher-quality self-sampled data and expanded preference margins. Results (Figure~\ref{fig:analysis_radar_part2}(a)) show accelerated improvements in the second iteration, highlighting effects of iterative training.


\paragraph{Prediction discrepancy facilitates direction search.}  
We finally analyze paradigm's role in direction search by employing \method’s training framework into \textit{IncUpdate} (the best-performing baseline). Figure~\ref{fig:analysis_radar_part2}(b) show that while direction search training improves \textit{IncUpdate}’s performance, it still falls short of \method. This underscores prediction discrepancy’s role in enabling context-specific search and more precise refinement.


\subsection{Persona Probing}
We further conduct an preliminary analysis of refined personas, termed \textit{persona probing}, to explore additional insights and applications of \method.

\renewcommand{\arraystretch}{0.9} % 将表格行高减小到默认值的 90%

\begin{table}[htbp]
\centering
\footnotesize

\begin{tabular}{lccccc}
\toprule
\textbf{Update Method} & \textbf{$\mathcal{S}_0$} & \textbf{$\mathcal{S}_1$} & \textbf{$\mathcal{S}_2$} & \textbf{$\mathcal{S}_3$} & \textbf{$\mathcal{S}_4$}\\
\midrule
\method & 245.0 & 316.8 & 353.5 & 393.2 & 429.4\\
IncUpdate & 245.0 & 390.1 & 459.3 & 500.4 & 526.4\\
HierMerge & 245.0 & 325.3 & 393.5 & 462.2 & 509.1\\
\bottomrule
\end{tabular}
\caption{Average persona token count across rounds.}
\label{tab:persona_tokens} 
\end{table}




\noindent
\textbf{Dynamic persona evolution across rounds.}
We first analyze persona dynamics during refinement process. Table~\ref{tab:persona_tokens} highlights \method’s controlled length growth, balancing representation efficiency and informativeness. Figure~\ref{fig:persona_probing}(a) reveals diminishing persona changes over time, with substantial shifts in early updates ($\mathcal{S}_0 \to \mathcal{S}_1$) and increasing stability in later rounds ($\mathcal{S}_1 \to \mathcal{S}_4$), indicating convergence and improved contextual alignment.

\begin{figure}[h]
    \centering
    \includegraphics[width=0.95\linewidth]{figure/persona_probing.pdf}
    \caption{(a) Cosine similarity among personas across rounds; (b) User clusters based on final personas(Book).}
    \label{fig:persona_probing}
\end{figure}


\renewcommand{\arraystretch}{0.9} % 将表格行高减小到默认值的 90%


\begin{table}[h]
\centering
\footnotesize
\begin{tabular}{lc}
\toprule
Profiling Dimensions(Book) & User Count \\
\midrule
Story \& Plot & 871\\
Emotion \& Experience & 878\\
Genre \& Theme & 878\\
Social \& Cultural Context & 680\\
User Behavior Traits & 862\\
Author \& Character & 701\\
Personality \& Values & 867\\
Relationship \& Connection & 716\\
\bottomrule
\end{tabular}
\caption{Key profiling dimensions in Book domain.}
\label{tab:book_dimensions} 
\end{table}


\paragraph{Insights from final optimized personas.} 
Refined personas from \method also enable in-depth, domain-specific exploration. 
In Book domain, we uncover \textbf{group-level preferences} by clustering final persona embeddings, identifying five user groups characterized by unique high-frequency adjectives (e.g., “romantic” and “practical”) (Figure~\ref{fig:persona_probing}(b)). 
We also extract \textbf{domain-specific patterns} by organizing high-frequency terms into eight dimensions using GPT-4o (Table~\ref{tab:book_dimensions}), highlighting critical factors for modeling Book domain users. These attempts show \method’s potential to support strategic user insights exploration.

