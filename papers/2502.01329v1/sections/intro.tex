\section{INTRODUCTION}% (1.25P)}
% {\color{red}
% \begin{itemize}
%     \item Area of research, motivation, scope, contributions, state of the art 
% \end{itemize}


% SoTA:

% \begin{itemize}
% \item  Compare velocity-WBC based on HQP, acceleration-WBC based on weighted QP and torque-WBC, based on weighted QP~\cite{ramuzat_comparison_2021}.
% \item Comparison of various QP solvers on 3 different data sets (one of them is from robotic MPC)~\cite{caron_qpbenchmark_2024}
% \item Benchmarking large-scale distributed convex quadratic programming algorithms~\cite{attila_kozma_benchmarking_2015}
% \item ProxQP benchmark~\cite{bambade_prox-qp_2022}
% \item Different WBC approaches: 
%   \begin{itemize}
%    \item Control Lyapunov function for ZMP dynamics as surrogate function in costs~\cite{kuindersma_efficiently_2014}
%    \item Equality constrained LQR + QP-based feedback controller~\cite{posa_optimization_2016}
%    \item TSID~\cite{del_prete_implementing_2016}
%    \item Hierarchical Inverse dynamics formulation~\cite{herzog_momentum_2016}
%   \end{itemize}
% \end{itemize}
% }


The development of quadrupedal robots has progressed rapidly in recent years, and several platforms have now reached a level of industrial maturity~\cite{hutter_anymal_2017,boston_dynamics_boston_2024,ghost_robotics_ghost_2024}. 
Apart from advancements in actuation, the main progress has been made on numerical methods for trajectory optimization, and its online implementation \gls{mpc}. 
At the ICRA 2022 conference, \qty{68}{\percent} of the papers presented at the legged robotics workshop covered receding horizon control or \acrshort{mpc}~\cite{katayama_model_2023}. 
Today, the de facto standard approach for legged locomotion comprises the planning of a contact sequence, computing a corresponding \gls{com} trajectory using \gls{mpc}, and, finally, stabilizing the obtained motion in real-time using \gls{wbc}~\cite{carpentier_recent_2021}. 

\begin{figure}[t]
    \centering
    \includegraphics[width=0.95\linewidth]{pictures/go2_vulcano.jpg}
    \caption{Go2 quadruped~\cite{unitree_robotics_unitree_nodate} used for evaluation in a Volcanic field test in an analogous scenario with limited onboard computing power.}
    \label{fig:go2_quad}
    \vspace{-0.5cm}
\end{figure}

At the core of \gls{mpc}, an optimal control problem has to be solved, which is formulated as linearly constrained \gls{qp}. 
Solving constrained \gls{qp}s is such a fundamental problem in both, \gls{mpc} and \gls{wbc}, that a huge amount of research has been devoted to developing efficient and stable \gls{qp} solvers, most of which are based on the \gls{asm}, \gls{ipm} or \gls{alm}. 
Each solver and each method is best suited for specific applications, e.g., some solvers show their strengths with small \gls{qp}s on embedded systems~\cite{ferreau_qpoases_2014}, others are optimized for larger problems that may arise in machine learning applications \cite{stellato_osqp_2020, boyd_distributed_2011}, and again others exploit the sparsity in the \gls{qp}~\cite{frison_hpipm_2020, stellato_osqp_2020}, a feature that can be advantageous in \gls{mpc} applications. 
Here, sparsity in the \gls{qp} is characteristic due to diagonal stacking of system state and control matrices over the entire prediction horizon. 
To reduce the \gls{qp} size, the state variables can be eliminated as decision variables, leading to denser and smaller matrices~\cite{jerez_sparse_2012}, a process that is known as \textit{condensing}. 
A great variety of problem formulations can be obtained by partially condensing the \gls{qp}, e.g., only eliminating half of the state variables~\cite{axehill_controlling_2015}. Special forms of condensed \gls{qp}s can be found in~\cite{jerez_sparse_2012} and~\cite{di_carlo_dynamic_2018}. 
In the works that introduce the current de facto standard of using \gls{mpc} and  
\gls{wbc}~\cite{di_carlo_dynamic_2018, kim_highly_2019}, the authors claim significant speed-up by removing some constraints and state variables from the \gls{qp}, leading to a dense and compact formulation. However, according to~\cite{axehill_controlling_2015}, sparse formulations are advantageous in \gls{mpc} as their computational and memory requirements grow linearly with the prediction horizon ($ \mathcal{O}(N) $) if sparsity is exploited, while for the dense variant, the computational cost is $\mathcal{O}(N^3)$ for \gls{ipm} and $\mathcal{O}(N^2)$ for \gls{asm}~\cite{dimitrov_sparse_2011}. Given these seemingly contradicting results, the effects of partial or full condensing of the \gls{mpc} problem with respect to the solver need to be investigated more thoroughly. 

There are several open-source benchmarks of \acrshort{qp} solvers~\cite{caron_qpbenchmark_2024, attila_kozma_benchmarking_2015}, which mostly use accuracy and solution time as a benchmark. However, as legged systems are being proposed for tasks such as space exploration~\cite{spiridonov_spacehopper_2024, arm_spacebok_2019}, other metrics such as power consumption and required onboard computing power become a critical factor for long-duration autonomous missions. ~\autoref{fig:go2_quad} illustrates the quadruped robot during a space exploration field test on a volcano, where it operates on limited battery capacity and onboard computing power to carry out the exploration tasks. Despite the importance of energy constraints, the impact of \gls{qp} formulation (sparse, partially condensed, condensed) and the relation to the prediction horizon, solver, and the computing \gls{hw} on performance used has yet to be comprehensively examined.

In this work, we focus on the application of dynamic quadrupedal walking using \gls{mpc}, which has so far produced several specialized methods for \gls{qp} formulation and solving, to reduce computation time or increase the planning horizon~\cite{ding_real-time_2019,di_carlo_dynamic_2018,kim_highly_2019}. 
We employ the standard approach to quadrupedal walking (contact planning, \gls{mpc}, \gls{wbc}) and evaluate it on a Unitree Go2 robot~\cite{unitree_robotics_unitree_nodate}. 
The benchmark involves (1) different \gls{qp} formulations in \gls{mpc} (sparse, partially condensed, and fully condensed), (2) two \gls{hw} architectures (x86, ARM) with desktop and single-board target computers, (3) various \gls{qp} solvers, including different principled methods for convex optimization, (4) different (dense) \gls{qp} solvers for \gls{wbc}. To allow a cross-\gls{hw} assessment of solver efficiency, we introduce the \textit{Solve Frequency per Watt} (SFPW) metric to compare different solvers. As a result of our benchmark, we recommend optimal combinations of computing \gls{hw}, problem sparsity, and \gls{qp} solvers for dynamic legged locomotion. The entire benchmark code, including the quadruped controller, is made open source\footnote{\url{https://github.com/dfki-ric-underactuated-lab/dfki-quad}}.

The remaining paper is structured as follows. Section~\ref{sec:dynamic_walking} describes the dynamic walking controller,  Section~\ref{sec:setup} the experimental setup and the performance metrics used for benchmarking. Section~\ref{sec:results} summarizes the results, with discussion in \autoref{sec:discussion}, and Section~\ref{sec:conclusions} draws conclusions on the benchmark.

\begin{figure}[t]
    \centering
    \includegraphics[width=0.99\linewidth]{pictures/controller-block-diagramm3.drawio.pdf}
        \caption{Block diagram showing dynamic walking controller's architecture}
    \label{fig:controller-architecture}
    \vspace{-1.5em}
\end{figure}

% \begin{itemize}
%     \item Examples for quadrupeds: MIT: \cite{kim_highly_2019} ETH: \cite{hutter_anymal_2017} (we don't know MPC yet)
%     \item The survey \cite{carpentier_recent_2021} found that the control approach  (contact plan, mpc, WBC) is a general approach in legged walking for both humanoids and quadrupeds
%     \item The survey \cite{katayama_model_2023} states that at ICRA 2022 in \qty{68}{\percent} of the papers presented at the legged robotics workshop covered receding horizon control or, in particular, Model Predictive Control (\acrshort{mpc}).
%     \item At the core of an \acrshort{mpc}, the process of mathematical optimization requires the most computational effort, because at every time step, an optimal control problem has to be solved \cite{jost_accelerating_2017}.
%     \item Resembling a linearly constrained quadratic program (QP), it can be solved by a QP-Solver. These solvers implement numerical algorithms that are in most cases subject to the field of convex optimization \cite{boyd_convex_2004}.
%     \item The great thing is that they can be used as black box solvers that can solve the well-formulated convex QP without the need to develop a new solution method.
%     \item Several solvers exist, each tailored to certain quadratic programming applications, with different prerequisites. 
%     \item Some solvers focus on embedded systems and small QP-problems \cite{ferreau_qpoases_2014}, or try to explicitly focus on \acrshort{mpc} \cite{frison_hpipm_2020}, \cite{frison_high-performance_2014}. In contrast, others are optimized for large problems, e.g., arising in machine learning applications \cite{stellato_osqp_2020}, \cite{boyd_distributed_2011}.
%     \item The solvers use different methods for solving convex \gls{qp}s, with the \gls{asm} and \gls{ipm} being the most prominent.
%     \item The \gls{asm} is older and most suitable for small to medium-sized problems, whereas the \gls{ipm} scales better on larger problems \cite{nocedal_quadratic_2006}.
%     \item To reduce the problem sizes in \gls{mpc}, the classic approach is the so-called dense formulation. Using the dynamics, the system's state variables are eliminated as decision variables from the problem, leading to dense and small matrices \cite{jerez_sparse_2012}.
%     \item Assuming that the \acrshort{mpc} prediction horizon is small, it makes sense to use the dense form, since the \acrshort{qp} is then also small and can be quickly solved by the \gls{asm}.
%     \item In contrast to that is the sparse formulation, in which the system states are left, leading to more decision variables, big matrices with sparse structure that solvers can exploit \cite{jerez_sparse_2012}.
%     \item On the subject of \gls{mpc}, the sparse representation outperforms \textcolor{red}{always?} the dense one  \cite{axehill_controlling_2015, jerez_sparse_2012, dimitrov_sparse_2011, ferreau_parallel_2012}.
%     \item Particularly, the computational and memory requirements grow linearly with the prediction horizon ($ \mathcal{O}(N) $) for the sparse representation if sparsity is exploited \cite{axehill_controlling_2015}.
%     \item In contrast, for the dense variant, the computational cost is $\mathcal{O}(N^3)$ if an interior-point method is used and $\mathcal{O}(N^2) $ if an active-set method is used \cite{dimitrov_sparse_2011}.
%     \item Therefore, it was recommended by \cite{axehill_controlling_2015}, to use the sparse form for large prediction horizons.
%     \item The work \cite{axehill_controlling_2015} also found that there is no binary distinction between sparse and dense. Instead, a great variety of problem formulations can be obtained by partially condensing the \gls{qp}, e.g., only eliminating half of the state variables.
% \end{itemize}

% ---
% \begin{itemize}
%     \item This general approach to the dynamic control of quadrupeds was developed in various works by MIT on their (mini) cheetah robots.\textcolor{red}{cite?}
%     \item In \cite{di_carlo_dynamic_2018} the Cheetah 3 quadruped is controlled by an \gls{mpc}, which is solved by the \gls{asm} based \acrshort{qpoases} solver.
%     \item To do so, the resulting \gls{qp} has been condensed not only with the help of the system dynamic. Since the gaits are fixed and known in advance, the respective force inputs with no contact have also been removed from the \gls{qp}, leading to very small problem size.
%     \item \textcolor{red}{Nevertheless, for already for a slightly increased prediction horizon the problem size might be at the boundary where \gls{ipm} outperform a dense \gls{asm} leasing to the question of why this hasn't been done in this work}.
%     \item For example, \cite{ding_real-time_2019} uses a custom IPM solver exploiting sparsity for a problem of comparable size.
%     \item Another example is the \gls{ipm} solver presented in \cite{pandala_qpswift_2019} which also exploits sparsity and is evaluated on a quadruped.
%     \item \cite{kim_highly_2019} Added a WBC to the MPC control architecture and the reduced-size QP for WBC was solved by a QP-solver implementing the method described by \cite{goldfarb_numerically_1983}, which is especially suited for small-size problems. 
%     \item However, condensing the \acrshort{mpc}'s system dynamics could indeed be useful, if the number of state variables is much larger than the number of input variables. Since all state variables get eliminated, the problem size greatly reduces.
% \item In the \acrshort{mpc} formulation of \cite{di_carlo_dynamic_2018} this is however not the case, since the number of state and input variables is 13 and 12 respectively.
% \item Since an efficient sparse QP-solver could provide an equal or even greater solve speed for a large prediction horizon, the reformulation proposed by \cite{di_carlo_dynamic_2018} would add an additional constraint to the system.
% Using a sparse approach, this could be avoided, allowing for dynamic gait patterns to by calculated on the fly and/or velocity-dependent by the onboard computer, while still keeping the solve time low enough to perform safe and stable operations. This means, that focusing on fixed gait patterns to reduce the problem size is probably not necessary.
% \item Nevertheless, in order to reach an overall better performance and efficiency, the effects of partially or fully condensing the \acrshort{mpc} problem should be subject to investigation.

    
% \end{itemize}
% -- \textcolor{red}{more examples of other quads: solver, condensing/sparsity exploitation, prediciton horizon?}

% As for the hardware used in the Mini-Cheetah and Cheetah 3 robot, for the first one, the specifications of the onboard computer were not explicitly mentioned.
% For the Cheetah 3, an Intel i7 Laptop \acrshort{cpu} from 2011 was employed \cite{di_carlo_dynamic_2018}, meaning that the x86 architecture was used.
% In addition, a real time Linux kernel was utilized to further speed up the computations and reduce possible latency.

% Among QP-solvers, general comparisons are quite rare, especially concerning computational efficiency.
% Regarding the underlying hardware, \cite{jerez_fpga_2011} and \cite{boechat_architecture_2013} used Field Programmable Gate Array (\acrshort{fpga}) implementations to increase the solve speed of their respective QP-solvers.
% In both cases, the tailored hardware was able to solve the QP-problem much faster, which indicates, that the used computer architecture for \acrshort{mpc} can have a great influence.
% However, different generic and widely available architectures like x86 and ARM were not considered.