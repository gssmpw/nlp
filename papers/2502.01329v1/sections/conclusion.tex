\section{CONCLUSIONS}% (0.5P)}
\label{sec:conclusions} 

This work describes a benchmark for \gls{qp} solvers in \gls{mpc} and \gls{wbc} in dynamic quadrupedal walking. 
The benchmark involves different \gls{qp} formulations (sparse to dense), several solvers, robot tasks and \gls{hw} architectures.

From this work's findings, three main conclusions follow: (1) For \gls{mpc}, sparse solvers and especially solvers based on \gls{ipm} (here \gls{hpipm}) perform best in dynamic quadrupedal walking and should be considered especially for long prediction horizons. These solvers also show certain robustness to changing problem structures, e.g., when changing contacts or between different tasks, and are therefore better suited for dynamic quadrupedal walking than other methods. 
(2) In \gls{wbc}, any of the regarded open-source solvers performs well; the engineering effort should rather be put into the formulation of the \gls{wbc} problem itself. 
(3) ARM architecture (here Jetson Orin) shows better efficiency than x86 when considering the \acrlong{sfpw} as a metric. Thus, they should be preferred in resource-constrained applications like autonomous quadrupedal walking.


% - MPC small problems (prediction horizon dense solvers, which in our case are the asm methods) -> also beneficial in static cases
% - For mid and larger problems, which seems to be the case for quadrupedal walking -> IPM method.
% - Condensing for same input and state size not beneficial but should be considered for different systems, especially when the state gets bigger or input smaller.
% - 

Future work includes extending the benchmark to other \gls{hw} architectures (e.g. CUDA), additional solvers to verify the results obtained, and possibly more complex systems (e.g. humanoids) to investigate the impact of system complexity on performance.

% {
% \color{red}
% \begin{itemize}
%     \item TODO: Outlook
%     \begin{itemize}
%         \item CUDA
%         \item More solvers?
%         \item different systems (humanoid/biped?)
        
%     \end{itemize}
% \end{itemize}
% }