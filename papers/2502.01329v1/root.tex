%%%%%%%%%%%%%%%%%%%%%%%%%%%%%%%%%%%%%%%%%%%%%%%%%%%%%%%%%%%%%%%%%%%%%%%%%%%%%%%%
%2345678901234567890123456789012345678901234567890123456789012345678901234567890
%        1         2         3         4         5         6         7         8

\documentclass[letterpaper, 10 pt, conference]{ieeeconf}  % Comment this line out if you need a4paper

%\documentclass[a4paper, 10pt, conference]{ieeeconf}      % Use this line for a4 paper

\IEEEoverridecommandlockouts                              % This command is only needed if 
                                                          % you want to use the \thanks command

\overrideIEEEmargins                                      % Needed to meet printer requirements.

%In case you encounter the following error:
%Error 1010 The PDF file may be corrupt (unable to open PDF file) OR
%Error 1000 An error occurred while parsing a contents stream. Unable to analyze the PDF file.
%This is a known problem with pdfLaTeX conversion filter. The file cannot be opened with acrobat reader
%Please use one of the alternatives below to circumvent this error by uncommenting one or the other
%\pdfobjcompresslevel=0
%\pdfminorversion=4

% See the \addtolength command later in the file to balance the column lengths
% on the last page of the document

% The following packages can be found on http:\\www.ctan.org
\usepackage{graphics} % for pdf, bitmapped graphics files
\usepackage{graphicx}
\usepackage{epsfig} % for postscript graphics files
\usepackage{mathptmx} % assumes new font selection scheme installed
\usepackage{times} % assumes new font selection scheme installed
\usepackage{amsmath} % assumes amsmath package installed
\usepackage{amssymb}  % assumes amsmath package installed
\usepackage{amsfonts}
\usepackage{bbold}
\usepackage{multirow}

\usepackage{layouts}
\usepackage{subfiles} 
\usepackage{xcolor}
\usepackage{booktabs}
\usepackage{xurl}
\newcommand{\ra}[1]{\renewcommand{\arraystretch}{#1}}
\usepackage{hyperref}
\usepackage{siunitx}
\usepackage{makecell}
\DeclareMathAlphabet{\mathcal}{OMS}{cmsy}{m}{n}
\SetMathAlphabet{\mathcal}{bold}{OMS}{cmsy}{b}{n}

\usepackage{tablefootnote}

\renewcommand*{\figureautorefname}{Fig.}

\title{\LARGE \bf
Benchmarking Different QP Formulations and Solvers for Dynamic Quadrupedal Walking
}


\author{Franek Stark$^{1}$ and Jakob Middelberg$^{1,2}$ and Dennis Mronga$^{1}$ and Shubham Vyas$^{1,2}$ and Frank Kirchner$^{1,2}$% <-this % stops a space
\thanks{This work was done in the AAPLE (grant number 50WK2275) and M-Rock (grant number 01IW21002) projects funded by the German Federal Ministry for Economic Affairs and Climate Action (BMWK) and the Ministry of Education and Research (BMBF) and is supported with funds from the federal state of Bremen for setting up the Underactuated Robotics Lab (201-342-04-2/2024-4-1)}% <-this % stops a space
\thanks{Special thanks goes to Hannah Isermann and Rohit Kumar for their support in developing control software for quadrupedal walking.}% <-this % stops a space
\thanks{$^{1}$All authors are with the Robotics Innovation Center at the German Research Center for Artificial Intelligence (DFKI), Bremen, Germany, Corresponding author's email: {\tt\small franek.stark@dfki.de}}%
\thanks{$^{2}$Jakob Middelberg, Shubham Vyas, and Frank Kirchner are additionally affiliated with the University of Bremen, Bremen, Germany}%
}

\DeclareMathSymbol{\shortminus}{\mathbin}{AMSa}{"39}
\usepackage[acronym]{glossaries}
\glsdisablehyper
%\makeglossaries

\newacronym{mpc}{MPC}{Model Predictive Control}
\newacronym[plural=QP,firstplural=Quadratic Programs]{qp}{QP}{Quadratic Program}
\newacronym{srbd}{SRBD}{single rigid-body dynamics}
\newacronym{gs}{GS}{Gait Sequencer}
\newacronym{slc}{SLC}{Swing Leg Controller}
\newacronym{wbc}{WBC}{Whole-Body Control}
\newacronym{sfpw}{SFPW}{Solve Frequency per Watt}
\newacronym{cpu}{CPU}{Central Processing Unit}
\newacronym{hw}{HW}{hardware}
\newacronym{com}{COM}{Center of Mass}
\newacronym{ipm}{IPM}{Interior-point method}
\newacronym{asm}{ASM}{Active-set method}
\newacronym{admm}{ADMM}{Alternating direction method of multipliers}
\newacronym{alm}{ALM}{Augmented Lagrangian Method}
\newacronym{csc}{CSC}{Compressed Sparse Column}
\newacronym{pmm}{PMM}{Proximal Method of Multipliers}

\newacronym{fpga}{FPGA}{Field Programmable Gate Arraz}

% solver acronyms:
\newacronym{hpipm}{HPIPM}{High Performance Interior Point Method}
\newacronym{osqp}{OSQP}{Operator Splitting Quadratic Programming Solver}
\newacronym{qpoases}{qpOASES}{Quadratic Programming Online Active Set Solver}
\newacronym{daqp}{DAQP}{Dual Active Set Solver for Quadratic Programming}
% They should never print automatically in their long form
\glsunset{hpipm}
\glsunset{osqp}
\glsunset{daqp}
\glsunset{qpoases}


%TODO!
\sisetup{
text-series-to-math = true ,
propagate-math-font = true
}
\newcommand{\maxf}[1]{{\cellcolor[gray]{0.8}}#1}
%\usepackage[table]{xcolor}
%\sisetup{per-mode=fraction} 
\begin{document}



\maketitle
\thispagestyle{empty}
\pagestyle{empty}


%%%%%%%%%%%%%%%%%%%%%%%%%%%%%%%%%%%%%%%%%%%%%%%%%%%%%%%%%%%%%%%%%%%%%%%%%%%%%%%%
\begin{abstract}
\glspl{qp} are widely used in the control of walking robots, especially in \gls{mpc} and \gls{wbc}.
In both cases, the controller design requires the formulation of a \gls{qp} and the selection of a suitable \gls{qp} solver, both requiring considerable time and expertise. 
% While there are benchmarks on the computational performance of \gls{qp} solvers, comparative studies on which combination of computational \gls{hw}, \gls{qp} formulation, and \gls{qp} solver is optimal are currently lacking. 
While computational performance benchmarks exist for \gls{qp} solvers, studies comparing optimal combinations of computational \gls{hw}, \gls{qp} formulation, and solver performance are lacking.
In this work, we compare dense and sparse \gls{qp} formulations, and multiple solving methods on different \gls{hw} architectures, focusing on their computational efficiency in dynamic walking of four-legged robots using \gls{mpc}.
We introduce the \gls{sfpw} as a performance measure to enable a cross-hardware comparison of the efficiency of \gls{qp} solvers. 
We also benchmark different \gls{qp} solvers for \gls{wbc} that we use for trajectory stabilization in quadrupedal walking. 
As a result, this paper provides recommendations for the selection of \gls{qp} formulations and solvers for different \gls{hw} architectures in walking robots and indicates which problems should be devoted the greater technical effort in this domain in future.
\end{abstract}


%%%%%%%%%%%%%%%%%%%%%%%%%%%%%%%%%%%%%%%%%%%%%%%%%%%%%%%%%%%%%%%%%%%%%%%%%%%%%%%%
\glsresetall
\glsunset{hpipm}
\glsunset{osqp}
\glsunset{daqp}
\glsunset{qpoases}

\subfile{sections/intro}
\subfile{sections/method}
\subfile{sections/exp_setup}
\subfile{sections/results}
\subfile{sections/conclusion}



%\addtolength{\textheight}{-12cm}   % This command serves to balance the column lengths
                                  % on the last page of the document manually. It shortens
                                  % the textheight of the last page by a suitable amount.
                                  % This command does not take effect until the next page
                                  % so it should come on the page before the last. Make
                                  % sure that you do not shorten the textheight too much.

%%%%%%%%%%%%%%%%%%%%%%%%%%%%%%%%%%%%%%%%%%%%%%%%%%%%%%%%%%%%%%%%%%%%%%%%%%%%%%%%



%%%%%%%%%%%%%%%%%%%%%%%%%%%%%%%%%%%%%%%%%%%%%%%%%%%%%%%%%%%%%%%%%%%%%%%%%%%%%%%%



%%%%%%%%%%%%%%%%%%%%%%%%%%%%%%%%%%%%%%%%%%%%%%%%%%%%%%%%%%%%%%%%%%%%%%%%%%%%%%%%
% \section*{APPENDIX}

% Appendixes should appear before the acknowledgment.

% \section*{ACKNOWLEDGMENT}




\bibliographystyle{IEEEtran}
\bibliography{referenceszot}

\end{document}
