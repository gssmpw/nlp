\begin{abstract}
   Recent few-shot object detection (FSOD) methods have focused on  augmenting synthetic samples for novel classes, show promising results  to the rise of diffusion models. 
   However, the diversity of such datasets is often limited in representativeness because they lack awareness of typical and hard samples, especially in the context of foreground and background relationships. To tackle this issue, we propose a Multi-Perspective Data Augmentation (MPAD) framework. In terms of foreground-foreground relationships, we propose \blue{in-context learning} for object synthesis (\blue{ICOS}) with bounding box adjustments to enhance the detail and spatial information of synthetic samples. Inspired by the large margin principle, support samples play a vital role in defining class boundaries. Therefore, we design a Harmonic Prompt Aggregation Scheduler (HPAS) to mix prompt embeddings at each time step of the generation process in diffusion models, 
   producing hard novel samples. For foreground-background relationships, we introduce a Background Proposal method (BAP) to sample typical and hard backgrounds. Extensive experiments on multiple FSOD benchmarks demonstrate the effectiveness of our approach. Our framework significantly outperforms traditional methods, achieving an average increase of $17.5\%$ in nAP50 over the baseline on PASCAL VOC. \orange{Code is available at \href{https://github.com/nvakhoa/MPAD}{github.com/nvakhoa/MPAD}}.
\end{abstract}