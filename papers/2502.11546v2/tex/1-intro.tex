\section{Introduction}
\label{sec:intro}
Large language models (LLMs) have achieved great progress on a variety of NLP tasks by leveraging vast amounts of training data~\cite{minaee2024large}.
However, their performance remains heavily biased towards high-resource languages (e.g., English), while low-resource languages are still left behind~\cite{zhu2024multilingual,huang2024survey}.
To improve the multilingual capabilities of LLMs, a common strategy is to incorporate large amounts of non-English data, either by continue pretraining~\cite{lai-etal-2024-llms} or by instruction tuning in multilingual settings~\cite{ustun-etal-2024-aya}.
Therefore, constructing large-scale, high-quality multilingual datasets is crucial for enhancing the multilingual performance of LLMs.

%% introduciton table--- compare with some famous corpus
\begin{table*}[!thp]
\resizebox{\textwidth}{!}{
\begin{tabular}{l|ccccccc}
\toprule
\textbf{Dataset} & \textbf{CC Version} & \begin{tabular}[c]{@{}l@{}}\textbf{\#Langs}\\ \textbf{(total)}\end{tabular} & \begin{tabular}[c]{@{}l@{}}\textbf{\#Langs}\\ \textbf{(high)}\end{tabular} & \begin{tabular}[c]{@{}l@{}}\textbf{\#Langs}\\ \textbf{(medium)}\end{tabular} & \begin{tabular}[c]{@{}l@{}}\textbf{\#Langs}\\ \textbf{(low)}\end{tabular} & \begin{tabular}[c]{@{}l@{}}\textbf{\#Langs}\\ \textbf{(very low)}\end{tabular} & \textbf{Training-Ready} \\
\midrule
mC4~\cite{raffel2020exploring} & CC-MAIN-2020-34 & 101 & 0  & 43  & 52  & 6 & \ding{55}            \\
% CC100\cite{conneau-etal-2020-unsupervised}    & CC-MAIN-2018-51                              & 100                                                       &                                                          &                                                            &                                                         &                                                              & Yes            \\
% ROOTS\cite{laurenccon2022bigscience}          & CC-MAIN-2021-10                              & 59                                                        &                                                          &                                                            &                                                         &                                                              & Yes            \\
OSCAR 23.01~\cite{abadji2022towards} & CC-MAIN-2022-49 & 153 & 6  & 42  & 25  & 80 & \ding{55}            \\
Glot500~\cite{imanigooghari-etal-2023-glot500} & \underline{CC-MAIN-2020-34} & 511 & 0  & 108 & 79  & 324 & \ding{55}            \\
CulturaX~\cite{nguyen-etal-2024-culturax}      & \underline{CC-MAIN-2022-49} & 167 & 11 & 47  & 27  & 82 & \ding{55}            \\
Madlad-400~\cite{kudugunta2024madlad}          & CC-MAIN-2022-33 & 419 & 7 & 46  & 39  & 327 & \ding{55}            \\
MaLA~\cite{ji2024emma}                         & \underline{CC-MAIN-2022-49} & 939 & 1  & 125 & 78  & 735 & \ding{55}            \\
Glotcc~\cite{kargaran2024glotcc}               & CC-MAIN-2023-50 & 1331 & 0  & 10  & 52  & 1269  & \ding{55}            \\
HPLT-v1.2~\cite{de-gibert-etal-2024-new}            & \underline{CC-MAIN-2022-40} & 191 & 12 & 53  & 38  & 88 & \ding{55}            \\
Fineweb-2~\cite{penedo2024fineweb-2} & CC-MAIN-2024-18 & 1915  &10 & 62  & 49  & 1794 & \ding{55}            \\
\midrule
DCAD-2000 & CC-MAIN-2024-46 & 2282 & 13 & 142 & 124 & 2003 & \ding{51}            \\
\bottomrule
\end{tabular}}
\vspace{-0.5em}
\caption{\label{tab:intro_comp}
Comparison of multilingual datasets constructed from Common Crawl (CC) and our constructed \dcad, focusing on the latest CC version used, the total number of languages supported, distribution across resource categories (high, medium, low, very low), and training readiness. The CC version marked with \underline{underline} indicates an inferred version due to the lack of explicit specification in the original paper. The ``Training-Ready'' column indicates whether the dataset is ready for training LLMs without requiring further data cleaning.
}
\vspace{-1em}
\end{table*}
%%

%
Recent efforts to construct multilingual datasets have built several notable resources, including CulturaX~\cite{nguyen-etal-2024-culturax}, HPLT~\cite{de-gibert-etal-2024-new}, Madlad-400~\cite{kudugunta2024madlad}, MaLA~\cite{lin2024mala}, and Glotcc~\cite{kargaran2024glotcc}, which cover 167, 191, 419, 939, and 1,331 languages, respectively.
While these datasets have made significant contributions, they exhibit three major limitations, as summarized in Table~\ref{tab:intro_comp}:
\textbf{(1) Outdated data sources:} These datasets primarily rely on older versions of Common Crawl\footnote{\url{https://commoncrawl.org}}, 
which results in outdated knowledge and an elevated risk of hallucination~\cite{huang2023survey}.
\textbf{(2) Limited coverage of high- and medium-resource languages\footnote{We follow the criteria from~\citet{goyal-etal-2022-flores} to categorize languages: High: $>100M$; Medium: $(1M, 100M)$; Low: $(100K, 1M)$; Very Low: $<100K$.}:} For instance, Fineweb-2~\cite{penedo2024fineweb-2}, which supports 1,915 languages, includes data from only 10 high-resource languages and 62 medium-resource languages.
\textbf{(3) Insufficient data cleaning:} Despite being cleaned, recent studies~\cite{dou-etal-2024-sailor,zhang-etal-2024-mc2} indicate that these datasets still contain a significant amount of noise, which makes them difficult to directly employ in training multilingual LLMs. For instance,~\citet{dou-etal-2024-sailor} found that 31.11\% of the data in Madlad-400 could still be further removed by advanced data cleaning.

Traditional data cleaning workflows~\cite{albalak2024survey} often rely on document-level features (e.g., language identification;~\citealp{kargaran-etal-2023-glotlid}) and fixed thresholds to filter out low-quality data.
However, this approach struggles with cross-lingual consistency due to feature distribution differences.
For example\footnote{Please refer to Appendix~\ref{appex:feature_analysis} for more details.}, the variation in average word counts across languages, datasets, and shards demonstrates the limitations of heuristic threshold-based data cleaning methods.
Notably, while Fineweb-2 fine-tunes thresholds for more than 1,000 languages, this process is computationally intensive and time-consuming.

To address these challenges, we introduce~\dcad, a new large-scale, high-quality multilingual dataset that can be directly applied to LLM training. \dcad covers 2282 languages (155 high/medium languages), incorporating the latest Common Crawl data (November 2024;~\texttt{CC-MAIN-2024-46}) and existing multilingual datasets.
Additionally, we propose a novel language-agnostic data cleaning approach that treats data cleaning as an anomaly detection~\cite{su2024large} problem, distinguishing it from traditional threshold-based methods~\cite{laurenccon2022bigscience,penedo2024fineweb-2}.
Our approach extracts eight statistical features from each document to evaluate quality, including metrics like \textit{language identification score}, \textit{word repetition}, \textit{special character ratio}, and \textit{perplexity score}.
Anomaly detection algorithms dynamically identify and remove outliers by recognizing deviations from typical document quality metrics, ensuring consistent and language-agnostic filtering.

We provide a comprehensive analysis of \dcad (Section~\ref{sec:analysis}), highlighting its diverse document distribution, broad geographical and script coverage, as well as resource categorization across languages, with a particular focus on supporting both high-resource and underrepresented languages.
The dataset’s extensive multilingual and cross-script representation is further validated through evaluation on the FineTask benchmark~\cite{penedo2024fineweb-2}, where LLMs trained on \dcad consistently outperform those trained on other multilingual datasets. 
Additionally, our anomaly detection-based data cleaning method demonstrates significant improvements in model performance, ensuring high-quality, noise-reduced data for large-scale multilingual model training.

In summary, we make the following contributions:  
\textbf{(1)} We propose a novel data cleaning framework that frames the task as anomaly detection, offering a language-agnostic and adaptive solution without manual threshold tuning.  
\textbf{(2)} We release \dcad, a comprehensive multilingual dataset covering over 2,282 languages, containing 8.63B of documents, 46.72TB of disk size and 159 writing scripts with metadata annotations, making it suitable for a wide range of downstream NLP tasks.
\textbf{(3)} Extensive evaluation experiments on FineTask benchmark demonstrate that \dcad consistently outperforms existing multilingual corpora, achieving higher normalized accuracy across multiple languages and NLP tasks.