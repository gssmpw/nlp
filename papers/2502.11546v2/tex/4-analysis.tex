\section{Dataset Analysis}
\label{sec:analysis}
%%figures and tables
% \begin{figure*}
% \centering
% \subfloat[Document Distribution\label{fig:dist_corpus}]{\includegraphics[width=0.33\textwidth]{images/dist_corpus.pdf}}\hfill
% \subfloat[Geographical Distribution\label{fig:dist_geo}]{\includegraphics[width=0.33\textwidth]{images/dist_area.pdf}}\hfill
% \subfloat[Script Distribution\label{fig:dist_script}]{\includegraphics[width=0.33\textwidth]{images/dist_scripts.pdf}}
% \caption{\label{fig:pie_chart}
% Document distribution and linguistic diversity in~\dcad.
% }
% \end{figure*}

\begin{figure*}[!thp]
\centering
\begin{subfigure}{0.33\textwidth}
    \includegraphics[width=\textwidth]{images/dist_corpus.pdf}
    \caption{Document Distribution}
    \label{fig:dist_corpus}
\end{subfigure}
% \hfill
\begin{subfigure}{0.33\textwidth}
    \includegraphics[width=\textwidth]{images/dist_area.pdf}
    \caption{Geographical Distribution}
    \label{fig:dist_geo}
\end{subfigure}
% \hfill
\begin{subfigure}{0.32\textwidth}
    \includegraphics[width=\textwidth]{images/dist_scripts.pdf}
    \caption{Script Distribution}
    \label{fig:dist_scripts}
\end{subfigure}
\vspace{-0.5em}
\caption{Document distribution and linguistic diversity in~\dcad.}
\label{fig:pie_chart}
\vspace{-1em}
\end{figure*}

% \begin{figure}[!htb]
    \centering
    \includegraphics[width=\columnwidth]{images/dist_dcad.pdf}
    \caption{\label{fig:four_categories}
    Distribution of languages in four categories: high-resource, medium-resource, low-resource, and extremely low-resource in the \dcad.
    }
\end{figure}

%%

In this section, we analyze the characteristics of \dcad, focusing on document distribution across sources, geographic and script coverage, resource categorization of languages, and the effect of data cleaning on dataset size and quality.

\noindent\textbf{Document Distribution Across Data Sources.}
The~\dcad dataset is derived from four primary sources: MaLA, Fineweb, Fineweb-2, and Newly Extracted Common Crawl data (New CC), as described in Section~\ref{sec:data_collect}.
Figure~\ref{fig:dist_corpus} presents the distribution of documents across these sources, with Fineweb-2 and New CC collectively contributing 47.5\% and 39.3\% of the total dataset, respectively.
These two sources play a significant role in ensuring the dataset's emphasis on both language diversity (Fineweb-2) and corpus freshness (New CC).
MaLA, though contributing 11.1\% of the total dataset, brings in valuable content from non-Common Crawl sources, further enriching the diversity of the dataset, especially for low-resource languages and specialized domains.
% These proportions reflect \dcad's strength in combining a broad spectrum of multilingual corpora with contemporary, high-quality data sources.

\noindent\textbf{Geographical Coverage of Languages.}
The geographical distribution of languages in \dcad, based on Glottolog’s classification\footnote{Geographic data source: \url{https://glottolog.org}}, is shown in Figure \ref{fig:dist_geo}.
The dataset spans languages from all major world regions, with the largest proportions originating from Africa (28.6\%), Papunesia (26.3\%) and Eurasia (23.8\%).
This coverage ensures robust support for multilingual applications across varied regional contexts, including densely populated areas like Eurasia and sparsely populated regions such as Papunesia and Australia.
While Eurasia is more heavily represented, this diversity of linguistic coverage helps ensure that the dataset remains useful for training LLMs in diverse regional environments.

\noindent\textbf{Script Distribution.}
Figure~\ref{fig:dist_scripts} illustrates the distribution of languages in \dcad by writing system.
The dataset supports 159 scripts, with the Latin script dominating at 79.4\%, followed by Cyrillic (3.9\%), Arabic (2.6\%), and Devanagari (2.1\%), among others.
This diversity in scripts enables a wide range of cross-lingual and script-specific tasks.
However, the inclusion of minority scripts, especially those with limited resources, poses unique challenges, such as optical character recognition (OCR) difficulties for certain scripts or inconsistent text quality.
Despite these challenges, \dcad ensures comprehensive coverage by including data from diverse scripts.
A complete list of supported scripts is provided in Appendix~\ref{app:script}.

\noindent\textbf{Language Resource Classification.}
Following the classification approach proposed by~\citet{goyal-etal-2022-flores}, we categorize languages in \dcad into four groups based on corpus size: high-resource, medium-resource, low-resource, and extremely low-resource.
Table~\ref{tab:intro_comp} shows the distribution across these categories.
The dataset includes 155 high- and medium-resource languages, while low-resource languages make up a significant portion, which reflects \dcad's commitment to supporting underrepresented languages.
% This categorization aligns with our future goal of expanding coverage for low- and extremely low-resource languages.
Notably, \dcad surpasses other corpora in its balance between high-resource and low-resource languages, which can have a significant impact on multilingual model training.
The distribution of languages across categories ensures that the dataset is well-suited for developing models that perform effectively across diverse language resources.

%% top 30 table
% % Please add the following required packages to your document preamble:
% \usepackage{multirow}
\begin{table*}[!thp]
\centering
\resizebox{\textwidth}{!}{
\begin{tabular}{l|rrr|rrr|rrr|l}
\toprule
\multirow{2}{*}{\textbf{Lang\_Name}} & \multicolumn{3}{c}{\textbf{Documents}}           & \multicolumn{3}{c}{\textbf{Tokens}}                    & \multicolumn{3}{c}{\textbf{Disk Size}}           & \multirow{2}{*}{\textbf{Source}} \\
\cmidrule(lr){2-4}\cmidrule(lr){5-7}\cmidrule(lr){8-10}
                            & keep & remove & total & keep & remove & total & keep & remove & total &                         \\
\midrule
eng\_Latn                   & 1.31B      & 101.08M      & 1.41B       & 1.21T        & 93.23B         & 1.30T         & 5.66TB     & 1.49TB       & 7.15TB      & Fineweb, MaLA, New CC   \\
rus\_Cyrl                   & 858.53M    & 67.21M       & 925.74M     & 1.14T        & 90.18B         & 1.23T         & 8.40TB     & 2.22TB       & 10.62TB     & Fineweb-2, MaLA, New CC \\
cmn\_Hani                   & 713.97M    & 71.19M       & 785.16M     & 745.88B      & 73.84B         & 819.71B       & 2.90TB     & 1.60TB       & 4.50TB      & Fineweb-2, New CC       \\
deu\_Latn                   & 668.62M    & 53.65M       & 722.27M     & 632.32B      & 51.11B         & 683.44B       & 2.85TB     & 664.79GB     & 3.52TB      & Fineweb-2, MaLA, New CC \\
spa\_Latn                   & 604.45M    & 43.33M       & 647.79M     & 483.75B      & 34.79B         & 518.54B       & 2.54TB     & 498.55GB     & 3.03TB      & Fineweb-2, MaLA, New CC \\
fra\_Latn                   & 513.53M    & 40.32M       & 553.85M     & 430.86B      & 33.77B         & 464.64B       & 2.15TB     & 491.23GB     & 2.64TB      & Fineweb-2, MaLA, New CC \\
jpn\_Jpan                   & 491.47M    & 42.47M       & 533.93M     & 278.14B      & 22.81B         & 300.95B       & 2.00TB     & 504.87GB     & 2.50TB      & Fineweb-2, New CC       \\
ita\_Latn                   & 311.42M    & 25.50M       & 336.93M     & 250.12B      & 20.69B         & 270.81B       & 1.29TB     & 292.75GB     & 1.59TB      & Fineweb-2, MaLA, New CC \\
por\_Latn                   & 271.48M    & 18.67M       & 290.15M     & 204.64B      & 14.12B         & 218.77B       & 1.07TB     & 225.83GB     & 1.30TB      & Fineweb-2, MaLA, New CC \\
pol\_Latn                   & 223.21M    & 15.38M       & 238.59M     & 180.34B      & 12.59B         & 192.93B       & 910.55GB   & 184.59GB     & 1.10TB      & Fineweb-2, MaLA, New CC \\
nld\_Latn                   & 219.16M    & 14.03M       & 233.19M     & 146.16B      & 9.38B          & 155.54B       & 739.62GB   & 159.01GB     & 898.63GB    & Fineweb-2, MaLA, New CC \\
ind\_Latn                   & 156.92M    & 16.21M       & 173.12M     & 60.97B       & 5.15B          & 66.11B        & 406.86GB   & 64.84GB      & 471.70GB    & Fineweb-2, MaLA         \\
tur\_Latn                   & 143.31M    & 9.98M        & 153.30M     & 118.40B      & 8.21B          & 126.61B       & 618.87GB   & 145.39GB     & 764.26GB    & Fineweb-2, MaLA, New CC \\
vie\_Latn                   & 87.77M     & 6.19M        & 93.96M      & 110.11B      & 7.71B          & 117.82B       & 570.86GB   & 116.19GB     & 687.05GB    & Fineweb-2, MaLA, New CC \\
fas\_Arab                   & 82.80M     & 9.49M        & 92.29M      & 67.58B       & 7.91B          & 75.49B        & 521.39GB   & 121.46GB     & 642.85GB    & Fineweb-2, MaLA, New CC \\
kor\_Hang                   & 79.22M     & 6.16M        & 85.38M      & 59.07B       & 4.62B          & 63.69B        & 336.56GB   & 66.70GB      & 403.26GB    & Fineweb-2, MaLA, New CC \\
swe\_Latn                   & 77.32M     & 5.08M        & 82.40M      & 59.21B       & 3.92B          & 63.13B        & 269.37GB   & 73.25GB      & 342.62GB    & Fineweb-2, MaLA, New CC \\
hun\_Latn                   & 70.79M     & 5.18M        & 75.97M      & 65.62B       & 4.86B          & 70.48B        & 319.58GB   & 87.97GB      & 407.55GB    & Fineweb-2, MaLA, New CC \\
ukr\_Cyrl                   & 67.87M     & 4.31M        & 72.18M      & 53.79B       & 3.41B          & 57.20B        & 428.74GB   & 82.51GB      & 511.25GB    & Fineweb-2, MaLA, New CC \\
ell\_Grek                   & 67.48M     & 5.67M        & 73.15M      & 57.63B       & 5.03B          & 62.66B        & 425.03GB   & 112.67GB     & 537.71GB    & Fineweb-2, MaLA, New CC \\
tha\_Thai                   & 55.47M     & 4.29M        & 59.76M      & 46.31B       & 3.59B          & 49.90B        & 525.54GB   & 110.37GB     & 635.91GB    & Fineweb-2, MaLA, New CC \\
arb\_Arab                   & 53.70M     & 4.06M        & 57.76M      & 25.21B       & 1.92B          & 27.13B        & 278.77GB   & 72.25GB      & 351.02GB    & Fineweb-2, MaLA         \\
aze\_Latn                   & 51.38M     & 6.44M        & 57.82M      & 3.30B        & 392.00M        & 3.70B         & 41.90GB    & 10.70GB      & 52.60GB     & MaLA                    \\
slv\_Latn                   & 50.41M     & 4.05M        & 54.46M      & 11.66B       & 836.48M        & 12.50B        & 69.22GB    & 12.64GB      & 81.87GB     & Fineweb-2, MaLA, New CC \\
cat\_Latn                   & 48.83M     & 3.78M        & 52.61M      & 16.49B       & 1.13B          & 17.62B        & 96.97GB    & 14.24GB      & 111.21GB    & Fineweb-2, MaLA, New CC \\
fin\_Latn                   & 47.80M     & 4.09M        & 51.89M      & 43.43B       & 3.75B          & 47.19B        & 202.14GB   & 57.62GB      & 259.76GB    & Fineweb-2, MaLA, New CC \\
ces\_Latn                   & 47.54M     & 3.21M        & 50.74M      & 42.20B       & 2.84B          & 45.04B        & 195.62GB   & 48.74GB      & 244.36GB    & MaLA, New CC            \\
hbs\_Latn                   & 42.98M     & 8.05M        & 51.04M      & 1.53B        & 287.34M        & 1.82B         & 22.41GB    & 6.41GB       & 28.82GB     & MaLA                    \\
fil\_Latn                   & 40.15M     & 6.32M        & 46.47M      & 3.47B        & 477.70M        & 3.94B         & 31.22GB    & 9.20GB       & 40.42GB     & Fineweb-2, MaLA         \\
mal\_Mlym                   & 39.10M     & 3.88M        & 42.98M      & 7.00B        & 558.83M        & 7.56B         & 94.47GB    & 18.30GB      & 112.78GB    & Fineweb-2, MaLA, New CC \\
% nob\_Latn                   & 38.88M     & 4.33M        & 43.21M      & 24.13B       & 2.81B          & 26.94B        & 139.85GB   & 66.29GB      & 206.15GB    & Fineweb-2, MaLA         \\
% guj\_Gujr                   & 38.82M     & 4.54M        & 43.36M      & 5.07B        & 461.54M        & 5.53B         & 60.08GB    & 13.49GB      & 73.57GB     & Fineweb-2, MaLA, New CC \\
% bul\_Cyrl                   & 37.11M     & 2.56M        & 39.67M      & 32.29B       & 2.23B          & 34.51B        & 245.84GB   & 55.86GB      & 301.69GB    & Fineweb-2, MaLA, New CC \\
% kan\_Knda                   & 34.20M     & 2.70M        & 36.90M      & 4.76B        & 359.21M        & 5.12B         & 68.82GB    & 11.13GB      & 79.95GB     & Fineweb-2, MaLA, New CC \\
% hin\_Deva                   & 29.15M     & 2.47M        & 31.62M      & 22.08B       & 1.81B          & 23.89B        & 219.46GB   & 46.45GB      & 265.91GB    & Fineweb-2, MaLA, New CC \\
% tam\_Taml                   & 26.55M     & 2.90M        & 29.45M      & 5.88B        & 460.75M        & 6.34B         & 80.26GB    & 19.29GB      & 99.55GB     & Fineweb-2, MaLA, New CC \\
% kaz\_Cyrl                   & 25.78M     & 1.67M        & 27.45M      & 6.37B        & 432.67M        & 6.80B         & 64.36GB    & 12.99GB      & 77.35GB     & Fineweb-2, MaLA, New CC \\
% heb\_Hebr                   & 25.24M     & 1.61M        & 26.85M      & 20.74B       & 1.33B          & 22.07B        & 147.85GB   & 28.75GB      & 176.60GB    & Fineweb-2, MaLA, New CC \\
% ara\_Arab                   & 25.14M     & 3.24M        & 28.39M      & 17.21B       & 2.23B          & 19.44B        & 152.73GB   & 71.93GB      & 224.66GB    & MaLA, New CC            \\
% srp\_Cyrl                   & 25.13M     & 1.75M        & 26.88M      & 6.91B        & 496.07M        & 7.41B         & 60.34GB    & 8.50GB       & 68.84GB     & Fineweb-2, MaLA, New CC \\
% est\_Latn                   & 24.18M     & 2.86M        & 27.04M      & 2.89B        & 294.20M        & 3.18B         & 26.17GB    & 8.91GB       & 35.08GB     & MaLA, New CC            \\
% sqi\_Latn                   & 24.16M     & 3.25M        & 27.41M      & 2.38B        & 237.81M        & 2.61B         & 21.08GB    & 5.03GB       & 26.11GB     & MaLA, New CC            \\
% isl\_Latn                   & 24.06M     & 2.23M        & 26.29M      & 6.32B        & 561.74M        & 6.89B         & 34.88GB    & 9.09GB       & 43.97GB     & Fineweb-2, MaLA, New CC \\
% pan\_Guru                   & 24.02M     & 3.16M        & 27.19M      & 2.27B        & 227.60M        & 2.50B         & 26.69GB    & 8.59GB       & 35.28GB     & MaLA, New CC            \\
% mlt\_Latn                   & 23.37M     & 2.08M        & 25.45M      & 3.24B        & 322.80M        & 3.56B         & 16.40GB    & 4.96GB       & 21.36GB     & Fineweb-2, MaLA, New CC \\
% mkd\_Cyrl                   & 22.61M     & 1.89M        & 24.50M      & 5.29B        & 396.98M        & 5.68B         & 51.37GB    & 7.08GB       & 58.45GB     & Fineweb-2, MaLA, New CC \\
% bos\_Latn                   & 21.62M     & 1.71M        & 23.33M      & 11.01B       & 831.59M        & 11.84B        & 59.71GB    & 10.67GB      & 70.38GB     & Fineweb-2, MaLA, New CC \\
% kat\_Geor                   & 20.27M     & 1.30M        & 21.57M      & 6.16B        & 413.36M        & 6.57B         & 82.54GB    & 15.10GB      & 97.65GB     & Fineweb-2, MaLA, New CC \\
% lit\_Latn                   & 20.09M     & 1.51M        & 21.60M      & 17.47B       & 1.33B          & 18.80B        & 91.29GB    & 18.30GB      & 109.59GB    & Fineweb-2, MaLA, New CC \\
% ben\_Beng                   & 19.90M     & 1.37M        & 21.28M      & 12.26B       & 848.75M        & 13.11B        & 143.64GB   & 30.36GB      & 174.00GB    & Fineweb-2, MaLA, New CC \\
\bottomrule
\end{tabular}}
\caption{
\label{tab:keep_remove}
Comparison of document count, token count, disk size, and sources before and after data cleaning for the top 30 languages in \dcad.
}
\end{table*}
%%

\noindent\textbf{Impact of Data Cleaning.}
We summarize the document count, token count, and disk size of the high/medium/low resource languages in \dcad before and after the data cleaning process.
Complete details are provided in Appendix~\ref{app:data_clean_statistic}.
The cleaning process results in the removal of a substantial amount of noisy data, even from datasets like MaLA, Fineweb, and Fineweb-2, which had already been subject some cleaning.
This aligns with the findings from~\cite{dou-etal-2024-sailor,zhang-etal-2024-mc2}.
For example, in the MaLA dataset, 8.05 million documents are removed for the \textit{hbs\_Latn} language, which suggests the necessity of rigorous data cleaning to enhance dataset quality.
Overall, the cleaning process removed approximately 7.69\% of the documents across all languages, significantly improving the quality of the dataset by reducing noise and increasing relevance for model training.