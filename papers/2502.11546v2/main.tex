% This must be in the first 5 lines to tell arXiv to use pdfLaTeX, which is strongly recommended.
\pdfoutput=1
% In particular, the hyperref package requires pdfLaTeX in order to break URLs across lines.

\documentclass[11pt]{article}

% Change "review" to "final" to generate the final (sometimes called camera-ready) version.
% Change to "preprint" to generate a non-anonymous version with page numbers.
%\usepackage[review]{acl}
\usepackage[]{acl}

% Standard package includes
\usepackage{times}
\usepackage{latexsym}

% For proper rendering and hyphenation of words containing Latin characters (including in bib files)
\usepackage[T1]{fontenc}
% For Vietnamese characters
% \usepackage[T5]{fontenc}
% See https://www.latex-project.org/help/documentation/encguide.pdf for other character sets

% This assumes your files are encoded as UTF8
\usepackage[utf8]{inputenc}

% This is not strictly necessary, and may be commented out,
% but it will improve the layout of the manuscript,
% and will typically save some space.
\usepackage{microtype}

% This is also not strictly necessary, and may be commented out.
% However, it will improve the aesthetics of text in
% the typewriter font.
\usepackage{inconsolata}

%Including images in your LaTeX document requires adding
%additional package(s)
\usepackage{graphicx}
%% use compactitem
\usepackage{paralist}
%% use mathbb
\usepackage{amssymb}
%% url too long
\usepackage{xurl}
%% table
\usepackage{booktabs}
%% math symbol
\usepackage{amsmath}
%% subfig -> subfloat
% \usepackage{subfig}
%% multirow for table
\usepackage{multirow}
%% xspace
\usepackage{xspace}
%% use \ping to show true/false
\usepackage{pifont}
%% use subfigure
\usepackage{subcaption}



%% define commond for dataset
% \newcommand{\dcad}{\textsc{DCAD-2000}}
\newcommand{\dcad}{DCAD-2000\xspace}

% If the title and author information does not fit in the area allocated, uncomment the following
%
%\setlength\titlebox{<dim>}
%
% and set <dim> to something 5cm or larger.

\title{DCAD-2000: A Multilingual Dataset across 2000+ Languages with Data Cleaning as Anomaly Detection}

\author{Yingli Shen$^{1}$\thanks{Equal contribution.}, Wen Lai$^{2,3*}$, Shuo Wang$^{1}$, Xueren Zhang$^{4}$, Kangyang Luo$^{1}$\\
\textbf{Alexander Fraser$^{2,3}$}\textbf{, Maosong Sun$^{1}$\thanks{Corresponding author.}} \\\\
        \textsuperscript{1} Tsinghua University, Beijing, China \\
	\textsuperscript{2} Technical University of Munich, Germany \\ 
        \textsuperscript{3} Munich Center for Machine Learning, Germany \\
         \textsuperscript{4} Modelbest Inc., China \\
        }

\begin{document}
\maketitle
\begin{abstract}
The rapid development of multilingual large language models (LLMs) highlights the need for high-quality, diverse, and clean multilingual datasets.
In this paper, we introduce \dcad (\underline{D}ata \underline{C}leaning as \underline{A}nomaly \underline{D}etection), a large-scale multilingual corpus built using newly extracted Common Crawl data and existing multilingual datasets.
\dcad includes over 2,282 languages, 46.72TB of data, and 8.63 billion documents, spanning 155 high- and medium-resource languages and 159 writing scripts.
To overcome the limitations of current data cleaning methods, which rely on manual heuristic thresholds, we propose reframing data cleaning as an anomaly detection task.
This dynamic filtering approach significantly enhances data quality by identifying and removing noisy or anomalous content.
We evaluate the quality of \dcad on the FineTask benchmark, demonstrating substantial improvements in multilingual dataset quality and task performance.\footnote{We release DCAD-2000 at: \url{https://huggingface.co/datasets/openbmb/DCAD-2000} and data cleaning pipeline at \url{https://github.com/yl-shen/DCAD-2000}.}
\end{abstract}

\section{Introduction}
\label{sec:intro}

Foundational models (FMs)~\cite{zhang2024data, zhou2023comprehensive} have shown remarkable progress in the healthcare domain, enabling professional-like assessment of disease diagnosis, treatment decision-making, and monitoring~\cite{zhang2023text, wang2022medclip, lu2023mi-zero}. 
Examples include LLaVA-Med~\cite{li2023llava}, Med-PaLM Multimodal~\cite{tu2024towards}, and Med-Flamingo~\cite{moor2023med}, have demonstrated their capacity on question answering, medical image analysis, and report generation.
These studies follow a predominant top-down model development strategy that requires upstream developers to collect data and train models for downstream tasks. 
Consequently, the developed model capabilities are heavily dependent on the training data, limiting their generalization performance in diverse clinical scenarios. 
For instance, Med-Gemini~\cite{yang2024advancing} reveals promising general capabilities in report generation while it lags behind state-of-the-art (SoTA) models on classification tasks, especially for out-of-domain applications. 
This indicates that while the generalizability of the foundation model is promising, more solutions are expected to meet the various specialized clinical needs.

To address these challenges, multi-center data centralization becomes essential to enhance model capacity and robustness across varied clinical scenarios~\cite{rajpurkar2022ai}. 
Centralizing distributed data can significantly improve model training and inference performance.
However, the process of medical data storage, transfer, and aggregation among centers requires extra efforts to ensure data security and system interoperability~\cite{bradford2020international}.
Moreover, a growing concern for patient privacy makes large-scale multi-center data sharing particularly challenging. 
While efforts like federated learning~\cite{wen2023survey, li2020review} can achieve good model performance on local data, the need for synchronized system coordination presents significant challenges, as clients are unable to update asynchronously. This limitation greatly restricts the practical capability of such approaches.
As a result, without a flexible collaboration, medical community still struggles to fully utilize the isolated data and local computation resources for comprehensive medical AI model development. 
To address this dilemma, open-source platforms encourage public data sharing and knowledge integration~\cite{markiewicz2021openneuro, zenodo}.
However, these platforms focus solely on raw data sharing while seldom providing collaborative model training or cooperation between different institutions.
Recently, collaborative learning has emerged as a viable approach for enhancing multi-model robustness~\cite{boulemtafes2020review}. 
For instance, software-like model development~\cite{raffel2023building} mimics software engineering practices by introducing structured workflows, enabling merging, version control, and continuous model integration.
Under this design, model ability can be strengthened with incremental knowledge updates similar to the version updating in software development. 

Although collaborative learning provides a multi-model collaboration, two key challenges remain in the leakage of raw data during collaboration~\cite{huang2023lorahub} and the synchronization of multiple collaborators~\cite{mcmahan2017communication} in the medical AI community. It is still challenging to integrate decentralized, privacy-sensitive data across institutions, leading to under-utilized insights and fragmented knowledge sharing~\cite{kaissis2020secure, rajpurkar2022ai, abdullah2021ethics}.
 To address these challenges, inspired by the collaborative software development, we propose \textbf{Med}ical \textbf{Fo}undation Models Me\textbf{rg}ing (\textbf{MedForge}), a cooperative workflow enabling continuously community-driven foundation model (FM) development.
MedForge enables a lightweight manner for individual centers to share their knowledge among multiple centers, minimizing the burden of data transmission and integration while enhancing model robustness.
Meanwhile, MedForge facilitates asynchronous and flexible collaboration, allowing individual centers to continuously update and improve medical FMs without the need for real-time synchronization.
Similar to open-source software development, MedForge incrementally updates medical knowledge and follows a sustainable model development scheme. 
This key design emphasizes a bottom-up construction of a multi-task medical FM, allowing downstream users to collaboratively build, refine, and update the upstream model according to their local resources. Our major contributions of MedForge are as below: 
\begin{enumerate}
    \item[$\bullet$] We introduce a collaborative workflow to promote the merging scheme of open-source software development. Our proposed MedForge allows distributed clinical centers to asynchronously contribute to comprehensive medical model construction while reducing transmitting costs among centers and avoiding the leakage of raw data, thus enhancing the utilization of private resources in the healthcare system. 
    \item[$\bullet$] We propose two effective knowledge-merging strategies for the asynchronous branch contribution. The MedForge-Fusion strategy updates the plugin module parameters of the main model during the merging phase, whereas the MedForge-Mixture strategy integrates the output of the plugin module by memorizing each contributor's coefficient. These strategies make MedForge more flexible and versatile. MedForge-Fusion is friendly to implement, while the MedForge-Mixture offers better performance and robustness.
    \item[$\bullet$]  We comprehensively evaluate model merging strategies to accumulate medical knowledge among multiple branch plugin modules. MedForge yields superior performance on medical classification tasks compared to other collaborative baselines across multiple datasets. We demonstrate the robustness of MedForge by shuffling the task order and evaluating various configurations of plugin modules and dataset distillation methods.
\end{enumerate}



\paragraph{Uncertainty-based hallucination detection methods.}
Various approaches have been proposed to detect hallucinated content in LLMs generation.
Unlike other methods that require external knowledge sources for fact-checking~\citep{gou2024critic, chen-etal-2024-complex, min-etal-2023-factscore, huo2023retrieving}, uncertainty-based approaches are reference-free and rely only on LLM internal states or behaviors to determine hallucination~\citep{10.1145/3703155}. 
For instance, sampling-based approaches generate multiple responses and measure the diversity in meaning among them~\citep{fomicheva-etal-2020-unsupervised, kuhn2023semantic, lin2024generating}, while density-based approaches approximate the training data distribution and provide probabilities or unnormalized scores to assess how likely a generated response belongs to the distribution~\citep{yoo-etal-2022-detection, ren2023outofdistribution, vazhentsev-etal-2023-hybrid}.

In this paper, we focus on uncertainty quantification methods that rely on token-level likelihood or entropy~\citep{guerreiro-etal-2023-looking, malinin2021uncertainty}. 
Recent works have explored refining likelihood estimation by incorporating semantic relationships or reweighting token importance. For instance, Claim-Conditioned Probability (CCP)~\citep{fadeeva-etal-2024-fact} was introduced to recalculate likelihood according to semantical equivalence; while \citet{zhang-etal-2023-enhancing-uncertainty} and \citet{duan-etal-2024-shifting} adjust token weights to better convey meaning in uncertainty aggregation. \emph{Although these approaches leverage token-level information, they are typically evaluated at the sentence level, raising questions about their reliability}. To address this, we conduct a comprehensive analysis of entity-level hallucination detection for finer-grained performance insights.


\paragraph{Fine-grained hallucination detection benchmark.}

Most hallucination detection benchmarks are in sentence or paragraph level. For example, CoQA~\citep{reddy-etal-2019-coqa}, TriviaQA~\citep{joshi-etal-2017-triviaqa}, TruthfulQA~\citep{lin-etal-2022-truthfulqa}, and HaluEval~\citep{li-etal-2023-halueval}. These benchmarks classify each generated response as either hallucinated or correct. However, instance-level detection cannot pinpoint specific hallucinated content, which is crucial for correcting misinformation~\citep{cattan2024localizingfactualinconsistenciesattributable}. This limitation becomes particularly problematic in long-form text, where a single response often combines supported and unsupported information, making binary quality judgments inadequate~\citep{min-etal-2023-factscore}.

To address these challenges, recent works have advanced benchmarks for more granular hallucination detection. For example, \citet{min-etal-2023-factscore} introduced \textsc{FActScore}, which decomposes LLM-generated text into atomic facts---short sentences conveying a single piece of information---for more precise evaluation. In parallel, \citet{cattan2024localizingfactualinconsistenciesattributable} introduced \textsc{QASemConsistency}, decomposing LLM generated text with QA-SRL, a semantic formalism, to form simple QA pairs, where each QA pair represent one verifiable fact. \emph{However, these methods do not enable entity-level hallucination detection, as they lack explicit entity-level labeling (hallucinated or not) in the original generated text}.  
Beyond decomposition-based approaches, datasets like \textsc{HaDes}~\citep{liu-etal-2022-token} and CLIFF~\citep{cao-wang-2021-cliff} create token-level hallucinated content by perturbing human-written text, allowing token-level annotation on the same text. These perturbed hallucinated content, however, could be unrealistic, biased, and overly synthetic due to the limitations of models they used to perturb words. 
To bridge this gap, we create a new dataset with entity-level hallucination labels on the same LLMs generated text. This allows us to evaluate uncertainty-based hallucination detection approaches on a finer-grained level and analyze their reliability.





\section{\dcad}
\label{sec:dcad-200}
To overcome the limitations of existing multilingual datasets, we propose \dcad, a large-scale, high-quality multilingual dataset constructed by integrating data from latest version of Common Crawl and existing multilingual datasets (Section~\ref{sec:data_collect}).
This dataset is cleaned using our proposed framework, which treats data cleaning as an anomaly detection problem (Section~\ref{sec:ad_clean}).
The construction of \dcad is supported by robust computational resources, as detailed in Section~\ref{sec:resources}.

\subsection{Data Collection}
\label{sec:data_collect}
To ensure comprehensiveness and robustness in multilingual data representation, \dcad integrates data from four main sources: MaLA, Fineweb, Fineweb-2, and newly extracted Common Crawl data.
Each source is selected based on its unique contribution to multilingual coverage, data quality, and freshness, with careful consideration of dataset complementarity to avoid redundancy.
Specifically, MaLA and Fineweb-2 are prioritized due to their broad language coverage and high-quality curation, which complements other widely used datasets like mC4~\cite{raffel2020exploring} and OSCAR~\cite{abadji2022towards}.
% To minimize overlap, we implement a series of deduplication strategies across datasets, such as utilizing MinHashLSH~\cite{broder1998min} to eliminate near-identical entries and ensure maximal diversity across the data sources.

\textbf{MaLA Corpus~\cite{ji2024emma}.}
The MaLA corpus covers 939 languages, aggregating data from diverse sources including Bloom~\cite{leong-etal-2022-bloom}, CC100~\cite{conneau-etal-2020-unsupervised}, Glot500~\cite{imanigooghari-etal-2023-glot500}, among others.
Deduplication is performed using MinHashLSH~\cite{broder1998min}, which is particularly effective in removing near-duplicate entries that often arise from common web sources.
% This process ensures minimal overlap with other datasets and preserves the linguistic diversity of the corpus.
Language codes are based on ISO 639-3\footnote{\url{https://en.wikipedia.org/wiki/ISO_639-3}} standards, and language-specific scripts are supported by GlotScript\footnote{\url{https://github.com/cisnlp/GlotScript}}.

\textbf{Fineweb Corpus~\cite{penedo2024fineweb}.}
Fineweb is a high-quality English web dataset extracted from Common Crawl, consisting of over 15 trillion tokens and updated monthly.
Data cleaning and deduplication are performed using the Datatrove library.\footnote{\url{https://github.com/huggingface/datatrove}}
For \dcad, we incorporate data from the November 2024 release (\texttt{CC-MAIN-2024-46}) to ensure freshness and up-to-date relevance of the data.

\textbf{Fineweb-2 Corpus~\cite{penedo2024fineweb-2}.}
Fineweb-2 expands Fineweb to include multilingual data, covering 1,915 languages.
It processes 96 Common Crawl dumps from 2013 (\texttt{CC-MAIN-2013-20}) to April 2024 (\texttt{CC-MAIN-2024-20}). 
The deduplication process within Fineweb-2 is similarly handled using the Datatrove library, ensuring the exclusion of redundant entries and maintaining high-quality multilingual coverage.

\textbf{Newly Extracted Common Crawl Data.}
To incorporate the most recent multilingual data, we extract and process Common Crawl dumps from May 2024 (\texttt{CC-MAIN-2024-22}) to November 2024 (\texttt{CC-MAIN-2024-46}).
Using the Fineweb-2 pipeline\footnote{\url{https://github.com/huggingface/fineweb-2}}, we process 21.54TB of multilingual data, ensuring that the data remains fresh and suitable for downstream tasks.
This further extends the multilingual data pool and enhances the coverage across underrepresented languages.

%% scatter plots
\begin{figure*}[!htb]
    \centering
    \includegraphics[width=\textwidth]{images/anomaly_detect_new.png}
    \vspace{-2em}
    \caption{\label{fig:anomaly_detect}
    Scatter plots of eight features extracted from a Chinese corpus during the data cleaning process, with data points color-coded according to their anomaly labels. The yellow points represent high-quality data, while the purple points indicate low-quality data.
  }
  \vspace{-1em}
\end{figure*}

%%%

\subsection{Data Cleaning as Anomaly Detection}
\label{sec:ad_clean}
Traditional data cleaning methods rely on fixed thresholds for document-level features, making them less adaptable to the diversity of multilingual data.
To address this, we propose a novel framework that formulates data cleaning as an anomaly detection task, which involves the feature extraction (Section~\ref{sec:feature_extract}) and anomaly detection (Section~\ref{sec:ad}).

\subsubsection{Feature Extraction}
\label{sec:feature_extract}
Inspired by~\citet{laurenccon2022bigscience} and~\citet{nguyen-etal-2024-culturax}, we extract eight statistical features from each document to evaluate text quality.
Each feature is selected for its ability to capture important characteristics of the text, contributing to robust anomaly detection.
Let $t$ represent a document; the extracted features are:
\textbf{(1) Number of Words, $n_w(t)$:} Total token count after tokenization.
\textbf{(2) Character Repetition Ratio, $r_c(t)$:} Fraction of repeated character sequences, highlighting noise or encoding errors.
\textbf{(3) Word Repetition Ratio, $r_w(t)$:} Proportion of repeated words, indicating redundancy or unnatural text.
\textbf{(4) Special Characters Ratio, $r_s(t)$:} Fraction of special characters based on language-specific lists from~\citet{laurenccon2022bigscience}. These lists include emojis, whitespace types, numbers, and punctuation.
\textbf{(5) Stopwords Ratio, $r_{\text{stop}}(t)$:} Proportion of stopwords using Fineweb-2's language-specific stopword lists. High or low ratios may indicate text quality issues.
\textbf{(6) Flagged Words Ratio, $r_{\text{flag}}(t)$:} Fraction of toxic or profane words from \textit{Toxicity-200}~\cite{costa2022no} or public repositories\footnote{\url{https://github.com/thisandagain/washyourmouthoutwithsoap}}.
\textbf{(7) Language Identification (LID) Score, $s_{\text{lid}}(t)$:} Confidence score from GlotLID~\cite{kargaran-etal-2023-glotlid}, supporting over 2,000 languages. Low scores may indicate misclassification or noise.
\textbf{(8) Perplexity Score, $s_{\text{ppl}}(t)$:} Using KenLM~\cite{heafield-2011-kenlm}, we train a language model for each language on multilingual Wikipedia data (November 2023). For unsupported languages, a default perplexity of 500 is used.

The feature vector for each document is defined as:
\begin{equation}
\begin{aligned}
\mathbf{x} &= \left[ n_w(t),\, r_c(t),\, r_w(t),\, r_s(t),\, \right. \\
&\left. r_{\text{stop}}(t),\, r_{\text{flag}}(t),\, s_{\text{lid}}(t),\, s_{\text{ppl}}(t) \right]^\top \in \mathbb{R}^8.\
\end{aligned}
\end{equation}

\subsubsection{Anomaly Detection}
\label{sec:ad}
After extracting feature vectors $\mathbf{x} \in \mathbb{R}^8$, we standardize each feature to handle differences in scale. The standardized value $\tilde{x}_j$ for the $j$-th feature is given by:
\begin{equation}
\tilde{x}_j = \frac{x_j - \mu_j}{\sigma_j}, \quad j = 1, \ldots, 8,
\end{equation}
where $\mu_j$ and $\sigma_j$ are the mean and standard deviation of the $j$-th feature across the dataset. The standardized feature vector is:
\begin{equation}
    \tilde{\mathbf{x}} = \frac{\mathbf{x} - \boldsymbol{\mu}}{\boldsymbol{\sigma}},
\end{equation}
where $\boldsymbol{\mu} = [\mu_1, \mu_2, \ldots, \mu_8]^\top$ and $\boldsymbol{\sigma} = [\sigma_1, \sigma_2, \ldots, \sigma_8]^\top$ are the vectors of means and standard deviations, respectively.

Take Isolation Forest~\cite{liu2008isolation} as an example\footnote{We also evaluate some other algorithms, please refer to Section~\ref{sec:eval} for more details.}, we compute an anomaly score $\phi(\tilde{\mathbf{x}})$ for each document.
The Isolation Forest algorithm assigns anomaly scores based on the average path length required to isolate a data point in a decision tree. Specifically, for a document represented by $\tilde{\mathbf{x}}$, the anomaly score is defined as:
\begin{equation}
\phi(\tilde{\mathbf{x}}) = 2^{-\frac{h(\tilde{\mathbf{x}})}{c(n)}},
\end{equation}
where $h(\tilde{\mathbf{x}})$ is the average path length for $\tilde{\mathbf{x}}$ across all trees in the Isolation Forest, and $c(n)$ is the average path length of a point in a binary tree with $n$ samples, given by:
\begin{equation}
    c(n) = 2H(n-1) - \frac{2(n-1)}{n},
\end{equation}
where $H(i)$ is the $i$-th harmonic number, defined as $H(i) = \sum_{k=1}^i \frac{1}{k}$.

An anomaly score $\phi(\tilde{\mathbf{x}}): \mathbb{R}^8 \to \mathbb{R}$ is defined to quantify how far a document deviates from typical data.
Higher scores indicate a higher likelihood of anomalies.
To classify a document, we use the decision rule:
% \begin{equation}
% f(\tilde{\mathbf{x}}) = \operatorname{sign}(\tau - \phi(\tilde{\mathbf{x}})),
% \end{equation}
% or equivalently,
\begin{equation}
f(\tilde{\mathbf{x}}) =
\begin{cases}
1, & \text{if } \phi(\tilde{\mathbf{x}}) < \tau, \\
-1, & \text{if } \phi(\tilde{\mathbf{x}}) \geq \tau,
\end{cases}
\end{equation}
where $\tau \in \mathbb{R}$ is a hyperparameter determined empirically or through cross-validation.\footnote{We use the default setting of the specific anomaly detection algorithm.}

Once the anomaly scores $\phi(\tilde{\mathbf{x}})$ are computed for all samples in the standardized dataset $\tilde{\mathcal{X}} = \{ \tilde{\mathbf{x}}_1, \ldots, \tilde{\mathbf{x}}_N \}$, we partition the dataset into two subsets:
\begin{align}
\mathcal{X}_{\text{keep}} &= \{ \tilde{\mathbf{x}} \in \tilde{\mathcal{X}} : f(\tilde{\mathbf{x}}) = 1 \}, \\
\mathcal{X}_{\text{remove}} &= \{ \tilde{\mathbf{x}} \in \tilde{\mathcal{X}} : f(\tilde{\mathbf{x}}) = -1 \}.
\end{align}
Following anomaly detection, the dataset is partitioned into a clean subset $\mathcal{X}_{\text{keep}}$ and an anomalous subset $\mathcal{X}_{\text{remove}}$. The former is retained for downstream tasks such as model training, while the latter may be discarded or further examined for potential data quality issues.

\subsubsection{Visualization}
To qualitatively assess the separation achieved by our anomaly detection framework, we generate scatter plots (Figure~\ref{fig:anomaly_detect}) of the $8$ feature values, with data points color-coded according to their anomaly labels.
These visualizations facilitate the interpretation of decision boundaries and highlight the features that contribute most significantly to the detection process.
We observe well-defined clusters separating anomalous and non-anomalous data points, with anomalies exhibiting distinct patterns compared to the majority of the data.
Features such as the language identification score ($s_{\text{lid}}(t)$) and perplexity score ($s_{\text{ppl}}(t)$) are expected to be particularly discriminative in identifying anomalies, as they capture linguistic irregularities and unexpected text patterns. For example, low $lid$ or unusually high $ppl$ scores often indicate problematic text, such as spam, low-quality content, or noise.
The framework effectively identifies and removes such low-quality text samples, which can be easily visualized by the separation of these points in the scatter plots.

\subsection{Computational Resources}
\label{sec:resources}
The construction of the \dcad dataset leveraged clould
% {Ksyun}\footnote{\url{https://k8s.ksyun.com}}
servers to process and clean the multilingual data efficiently.
Each server instance is equipped with 32 CPU cores, 128GB of memory, and 100GB of disk storage, which is utilized for intermediate data handling and memory-intensive operations such as anomaly detection.
The workload is managed using container orchestration tools,
% {Kubernetes}\footnote{\url{https://kubernetes.io}},
with up to 100 parallel tasks running per job to ensure scalability.
\section{Analysis of Ability Degradation}
\label{analysis}

ROME~\cite{DBLP:conf/nips/MengBAB22} and MEMIT~\cite{DBLP:conf/iclr/MengSABB23} are 
 currently popular model editing methods. Given that MEMIT builds upon the foundations of ROME by implementing residual distribution across multiple layers,  our analysis in the main text focuses primarily on ROME. 
This section presents a detailed analysis of how the model is affected during sequential editing using ROME. 
Statistical and visual analyses reveal that the degradation of general abilities is related to the unintentional introduction of the non-trivial noise that can make the parameter matrix after editing deviate from its original semantics space.

\subsection{Comparison with Fine-tuning Approach}
First, a statistical analysis is conducted by editing GPT2-XL~\cite{radford2019language} using the ZsRE~\cite{DBLP:conf/conll/LevySCZ17} dataset. 
Considering the L1 norm effectively quantifies the absolute changes in parameter values pre- and post-editing, while providing insights into feature weight distributions within the matrix, it is used to represent the degree of change in the parameter matrix.
As illustrated in Figure~\ref{fig-edit}, when using editing-based methods such as ROME and MEMIT, the L1 norm of the matrix at the edited layer increases significantly with the number of edits.
It can be seen that the norm increases by 317\% (ROME) and 61\% (MEMIT), respectively by the end of sequential editing.
This result highlights a significant deviation from the unedited model, emphasizing the impact of sequential edits on stability.

\begin{figure}[t]
  \subfigure[Editing-based methods]{
  \includegraphics[width=0.22\textwidth]{figures/L1-Norm-GPT2XL.pdf}
  \label{fig-edit}}
  \subfigure[Fine-tuning approach]{
  \includegraphics[width=0.22\textwidth]{figures/finetune-GPT2XL.pdf}
  \label{fig-finetune}}
\vspace{-2mm}
\caption{Illustration of the change of L1 norm 
(a) in sequential editing at the edited layer using editing-based methods and 
(b) in fine-tuning different batch steps for selected layers.
Here we uniformly selected the layers of GPT2-XL for clarity when fine-tuning.} 
\vspace{-2mm}
\end{figure}

A gradient-based fine-tuning approach can markedly enhance the performance of the model on specific tasks while preserving its general abilities across other downstream tasks~\cite{DBLP:journals/corr/abs-2312-12740, DBLP:journals/corr/abs-2310-10047}. 
As depicted in Figure~\ref{fig-finetune}, there are no significant changes in the norm of the parameter matrix for the given layers, with a maximum change of only 0.27\%, even as the amount of fine-tuning knowledge increases. This stability in the parameter matrix norms suggests that the fine-tuning approach does not introduce significant non-trivial noise during the editing process. Thus, fine-tuning maintains the integrity and stability of the model's parameters, which is crucial for preserving its general abilities and preventing unintended non-trivial noise.

This comparison indicates that each editing method not only updates the intended fact as expected but also unintentionally introduces non-trivial noise into the model. This noise manifests as deviations in the parameter matrix during the sequential editing process. With each additional edit, the noise accumulates, progressively increasing the deviation in the parameter matrix. Consequently, as the number of edits grows, there is a significant deviation in the parameter matrix observed before and after the editing. This accumulated noise highlights the challenge of maintaining the stability and integrity of the parameter matrix through multiple edits, which can ultimately impact the general abilities of the model.

\begin{figure}[t]%[!htb]
  \subfigure{
  \includegraphics[width=0.45\textwidth]{figures/legend_visible.pdf}}
  \subfigure{
  \includegraphics[width=0.22\textwidth]{figures/pca-gpt2-right.pdf}}
  \subfigure{
  \includegraphics[width=0.22\textwidth]{figures/pca-llama3-right.pdf}}
\vspace{-2mm}
\caption{Visialization of six sets of facts recalled by LLMs using 2-dimensional PCA. Note that this hidden state is also projected by a language modeling head (linear mapping) for next-token prediction, implying the linear structure in the corresponding representation space (the PCA assumption).} 
\vspace{-3mm}
\label{fig-pca}
\end{figure}

\subsection{Visualization Analysis} \label{sec_visual}
Following the ROME, the second layer of MLP \( W^{(l)}_{\text{proj}} \) is viewed as a linear associative memory~\cite{anderson1972simple, DBLP:journals/tc/Kohonen72}. This perspective observes that any linear operation \( W \) can operate as a key-value store for a set of vector keys \( K = [\mathbf{k}_1 | \mathbf{k}_2 | \dots] \) and corresponding vector values \( V = [\mathbf{v}_1 | \mathbf{v}_2 | \dots] \), by solving \( WK \approx V \)~\cite{DBLP:conf/nips/MengBAB22}.
The key-value pair \( (\mathbf{k}_i, \mathbf{v}_i) \) represents the representation of the input prompt, where $\mathbf{k}_i$ identifies patterns of the input and $\mathbf{v}_i$ is the fact recalled by the model, which is considered to gather all the information about how the model understands the prompt and how it will respond. By stacking $\mathbf{k}_i$ and $\mathbf{v}_i$ separately for each prompt, matrix \( K \) and \( V \) are obtained.
Based on this, 200 prompts of the same downstream task are collected to compute \( K \) and \( V \). 

On GPT2-XL and LLaMA-3 (8B), Principal Component Analysis is employed to visualize the hidden state of the facts of the downstream task recalled by the model. 
The first two principal components of six sets of facts, representing most features, are computed~\cite{zheng2024prompt}. Two of these are derived from recalling the model before editing and the model after editing without any constraint, respectively.
To explore the relationship between the deviation of the parameter matrix after editing and the resulting degradation of general abilities, four additional settings were tested by setting different percentages of the columns in the update matrix to zero, evenly distributed according to an arithmetic progression.

As illustrated in Figure~\ref{fig-pca}, the principal
components of facts recalled by the original model and the edited model without any constraint can be largely distinguished, whose boundaries (black dashed lines) can be easily fitted using logistic regression.
This indicates a significant semantic discrepancy between the facts recalled by the unconstrained edited model and the original model, explaining the decline in general abilities is related to matrix deviation.
Furthermore, when the deviation of the parameter matrix is constrained by reducing the norm of the update matrix, as shown by the black arrows, the principal components of the recalled facts by the edited model gradually align with those of the original model.
This shows that by reducing the norm of the update matrix, the deviation of the parameter matrix after editing can be constrained, making the semantic distribution of the model before and after editing similar, thereby preserving the general abilities of the edited model.
\section{Evaluation}

\begin{table}
    \centering
    \caption{Configuration details of NDP-DIMM.}
    \vspace{-0.3cm}
    \resizebox{\linewidth}{!}{
    \begin{tabular}{c|c|c}
    \hline
    \multicolumn{3}{c}{\textbf{NDP core}} \\
    \hline
    \multicolumn{3}{c}{Configuration: 256 multipliers, reduction tree-based accumulator, Buffer size: 256KB}\\ 
    \hline
    One NDP core per DIMM & Frequency: @ 1 GHz  & area overhead: $1.23mm^2$ per core\\
    \hline
    \multicolumn{3}{c}{\textbf{DIMM Parameters}} \\
    \hline
     \multicolumn{3}{c}{DDR4-3200, 32GB/DIMM$\times$8, \update{2 DIMMs/channel}}\\
     \multicolumn{3}{c}{4 rank/DIMM, 2 bank groups/rank, 4 bank/BG}\\
     \hline
    \multicolumn{3}{c}{\textbf{DIMM Timing}} \\
    \hline
     \multicolumn{3}{c}{tRC=76, tRCD=24, tCL=24, tRP=24, tBL=4}\\
     \multicolumn{3}{c}{tCCD S=4, tCCD L=8,tRRD S=4, tRRD L=6, tFAW=26}\\
    \hline
    \multicolumn{3}{c}{\textbf{DIMM-Link Parameters}} \\
     \hline
    \multicolumn{3}{c}{25Gb/s/Lane, 1.17 pJ/b, 8 $\times$ Lanes (25GB/s per Link)} \\
    \hline
    \end{tabular}
    }
    \label{tab:dimmcfg}
\vspace{-0.3cm}
\end{table}

\subsection{Experimental Setup}\label{sec:experimental-setup}

\subsubsection{\name~System}
The proposed \name~system integrates a single NVIDIA RTX 4090 GPU with 24GB of graphic memory \update{and 330 tensor TOPS (FP16)} to process hot neurons. Additionally, we provide 8 NDP-DIMMs, each including 32GB DDR4 memory as the extension of GPU memory. We use PCIe 4.0 to support data interaction between NDP-DIMMs and GPU memory with a bandwidth of 64GB/s. The kernel performance of the NVIDIA RTX 4090 is measured using NVIDIA Nsight Compute~\cite{nsight}. Furthermore, we develop an in-house simulator by modifying Ramulator 2.0~\cite{luo2023ramulator, ramulator2.0} to evaluate the performance efficiency of NDP-DIMM devices. For the NDP core, we implemented it in RTL and synthesized it using the Synopsys Design Compiler~\cite{synopsys.org} with the TSMC 7nm technology. \tab \ref{tab:dimmcfg} shows the configuration details of adopted NDP-DIMMs.

\subsubsection{Baseline Systems}
We selected several offloading-based inference systems, such as Huggingface Accelerate~\cite{jain2022hugging, huggingface-accelerate}, FlexGen~\cite{sheng2023flexgen}, and Deja Vu~\cite{liu2023deja}, as the baselines. FlexGen and Deja Vu are restricted to OPT models. Moreover, Deja Vu, initially optimized for LLM activation sparsity within high-performance distributed systems, has been adapted to support offloading-based serving systems. In contrast to \name, these methods depend solely on the basic host memory to expand capacity without offering additional computational resources. \update{We also provided a system (Hermes-host) that offloads cold neurons to the host CPU while handling hot neurons on GPU, demonstrating the necessity of NDP-DIMMs. Hermes-host follows the configuration in~\cite{song2023powerinfer}, which equips an Intel i9-13900K processor as the host CPU (providing a maximum bandwidth of 89.6 GB/s), and also uses a single NVIDIA RTX 4090 as the GPU for hot neurons.} Additionally, to highlight the significance of activation sparsity in boosting \name~system efficiency, we also compare \name~against a straightforward NDP-DIMM extended system (referred to as Hermes-base) that does not leverage activation sparsity in LLMs.

\begin{figure}
    \centering
    \includegraphics[width=.98\linewidth]{Fig/end1_rebuttal.pdf}
    \vspace{-0.3cm}
    \caption{\update{Performance comparison with existing offloading-based systems.}}
    \label{fig:offloading-performance}
\vspace{-0.3cm}
\end{figure}

\begin{figure}[t]
    \centering
    \includegraphics[width=0.98\linewidth]{Fig/end2_rebuttal.pdf}
    \vspace{-0.3cm}
    \caption{\update{The effectiveness of activation sparsity and NDP design on \name.}}
    \label{fig:base-hermes-performance}
\vspace{-0.3cm}
\end{figure}

\subsubsection{Workloads}
We chose OPT-13B, OPT-30B, OPT-66B~\cite{zhang2022opt}, LLaMA2-13B, LLaMA2-70B~\cite{touvron2023llama2}, and Falcon-40B~\cite{almazrouei2023falcon} as target models. For the OPT series models, we utilized their native ReLU activations to achieve activation sparsity. For the LLaMA2 and Falcon models, we use the open-source models\footnote{The modified LLMs can be found at \href{https://huggingface.co/SparseLLM}{https://huggingface.co/SparseLLM}, including both LLaMA2 and Falcon models} that substituted their original activation functions with ReLU~\cite{mirzadeh2023relu, zhang2024relu}. Furthermore, we added additional ReLU functions before generating QKV to achieve activation sparsity in self-attention blocks. Evaluation results show that these alterations result in negligible accuracy loss (under 1\%). \update{Furthermore, we adopt ChatGPT prompts~\cite{gpt-prompts} and Alpaca~\cite{alpaca} as the datasets to evaluate the end-to-end performance, following configurations in \cite{xue2024powerinfer, song2023powerinfer}.}
% \todo{The dataset used for evaluation with description of the distribution of the prompt and generated output lengths}

\begin{figure*}
    \centering
    \includegraphics[width=\linewidth]{Fig/batching_rebuttal.pdf}
    \vspace{-0.3cm}
    \caption{\update{End-to-end performance on different batch sizes (ranging from 1 to 16). N.P. denotes the model is not supported by the current inference system.}}
    \label{fig:batching-inference}
\vspace{-0.3cm}
\end{figure*}

\subsubsection{Evaluation Metric}
Given our focus on local deployment scenarios, we primarily optimized LLM inference with small batch sizes. We concentrated on the average number of tokens generated per second (tokens/s) to evaluate model inference efficiency. Hereafter, the number above each bar in each figure indicates the end-to-end generation speed (tokens/s). In our experiments, we used batch sizes between 1 and 16, and kept the lengths of both input and output sequences fixed at 128.


% Moreover, 

% Moreover, we concentrated on token-to-token latency (ms/token) and the average number of tokens generated per second (tokens/s) to evaluate model inference efficiency.

\subsection{\name~Performance}\label{sec:end-to-end}

\subsubsection{End-to-End Performance}
We begin by evaluating the end-to-end inference performance of \name~and baseline systems at a batch size of 1, which is commonly used for local deployments~\cite{cai2023medusa}. Noting that FlexGen and Deja Vu are limited to support OPT family models, we first compare \name~against existing offload-based inference systems on OPT models. \update{Additionally, we evaluate the Hermes-host and Hermes-base systems' performance across various LLMs to illustrate the necessity of NDP-DIMMs design and activation sparsity in Hermes, respectively.}
% 这里需要包含两个部分,分别是 end-to-end performance 和 token-to-token latency

\textbf{Comparison with Offloading-based Systems. } \fig \ref{fig:offloading-performance} presents the end-to-end performances on OPT family models. Compared with the Accelerate and FlexGen systems, \name~can achieve an average $578.42 \times$ and $247.25 \times$ speedup, respectively. \name~is capable of achieving a rate of $20.37$ tokens/s for OPT-66B, which substantially surpasses current inference systems. In contrast, Deja Vu only attains an average speedup of $2.12 \times$ over FlexGen due to the necessity of loading cold neurons. The frequent data transfer on PCIe compromises the performance improvement of activation sparsity, while the expensive MLP-based predictor used in Deja Vu further diminishes its benefits. Compared to OPT-13B, \name~achieves greater performance gains on OPT-66B. This is because 80\% of the parameters in OPT-13B can be stored in GPU memory, whereas only 15\% of parameters in OPT-66B can be stored in GPU memory. This further exacerbates the data transfer overhead between host memory and GPU memory. 
% Fortunately, \name~can effectively utilize NDP-DIMMs to process the offloaded parameters without introducing significant data movement.

\textbf{Necessity of Activation Sparsity. } We further compare \name~with the Hermes-base system, which only adopts a na\"ive NDP-DIMM extended system without utilizing activation sparsity, as shown in \fig \ref{fig:base-hermes-performance}. 
The Hermes-based system processes the FC layers on the GPU when their parameters are available, switches to NDP-DIMMs when their parameters are stored in those modules, and offloads all attention computations to NDP-DIMMs.
% The Hermes-based system activates nearby processing units based on the location of parameters to compute FC operators and offloads all attention computations to NDP-DIMMs. 
This approach leverages the high internal bandwidth of NDP-DIMMs and
reduces data transfer between DIMMs and GPU memory. In comparison to Huggingface Accelerate, the Hermes-base system can achieve $53.89 \times$ speedup on average, as it greatly reduces the data transfer on PCIe. By effectively leveraging activation sparsity in LLMs, Hermes outperforms the Hermes-base system with average speedups of $5.17 \times$, specifically for large models such as Falcon-40B and
LLaMA2-70B. This is due to when running large models, most layers are offloaded on the computation-limited NDP-DIMMs for the Hermes-base system. 

\update{\textbf{NDP-DIMMs instead of host CPU. } 
Experimental results in Figure \ref{fig:offloading-performance}, \ref{fig:base-hermes-performance} demonstrate the necessity of NDP-DIMMs. Hermes achieves $4.79\times$ - $7.75\times$ speedup when compared to Hermes-host. Specifically, the Hermes-host system also utilizes the hot/cold neuron partition, but computes the cold neurons on the host CPU. This approach effectively alleviates the burdensome data loading on PCIe for existing offloading-based systems. In comparison to Huggingface Accelerate and FlexGen, the Hermes-host system can achieve $62.00 \times$ and $44.96\times$ speedup on average, respectively. However, the memory bandwidth on the CPU side is significantly lower than that of NDP-DIMMs, making the Hermes-host system still far less efficient than our proposed Hermes system. }



\subsubsection{Batching Inference}

\begin{figure*}
    \centering
    \includegraphics[width=\linewidth]{Fig/breakdown.pdf}
    \vspace{-0.3cm}
    \caption{Evaluating the performance breakdown on Deja Vu, \name, and \name-base (H-base) on various LLMs with different batch sizes. }
    \label{fig:performance-breakdown}
\vspace{-0.3cm}
\end{figure*}

We also evaluate the end-to-end performance of \name~with different batch sizes. As shown in the \fig \ref{fig:batching-inference}, \name~demonstrates consistent performance improvement with the batch sizes varying from 1 to 16. Hermes attains average speedups of $148.98\times$ and $75.24\times$ for various batch sizes when compared to FlexGen and Deja Vu, respectively, offering promising support for larger batch sizes. \update{Furthermore, \name~achieves an average $7.17 \times$ speedup over \name-host for various batch sizes. As the batch size increases, the performance gap between Hermes-host and Hermes becomes more pronounced. This occurs as the consumer-grade GPU with sufficient computation capability is minimally impacted by larger batch sizes, whereas the dynamic loading overhead of cold neurons is closely tied to bandwidth. Consequently, as batch sizes grow, the limited memory bandwidth on the CPU side increasingly affects overall system performance.} The performance gap between \name~and the Hermes-base system is the smallest when the batch size is 2. This is because for \name-base, the computation capability of the NDP core can still effectively handle the corresponding computational load, and larger batches can effectively amortize the DRAM cell access overhead as weight parameters are reused by the two batches. At other batch sizes, \name~demonstrates a significant performance advantage over Hermes-base. First, at a batch size of 1, Hermes can utilize activation sparsity to significantly reduce the number of neurons that need to be activated, thereby lowering data access overhead. Second, as the batch size increases, Hermes is not constrained by the computation capability of NDP-DIMMs due to the presence of activation sparsity. 

\subsection{Ablation Studies}\label{sec:ablation-study}
% 要包括这样几种: offline modeling and mapping; online placement; load balance optimization 


\begin{figure}
    \centering
    \includegraphics[width=\linewidth]{Fig/ab_rebuttal.pdf}
    \vspace{-0.3cm}
    \caption{\update{Ablation study on proposed offline and online scheduling strategies.}}
    \label{fig:ablation-study}
\vspace{-0.3cm}
\end{figure}



To evaluate the scheduling strategies proposed in Section \ref{sec:hermes-system}, we compare the normalized inference latency on MLP block for different LLMs with various scheduling settings. Specifically, \name-random denotes utilizing a random offline mapper to achieve neuron placement, \name-partition denotes that it only considers the optimal offline neuron placement, \name-adjustment denotes the system that further uses online adjustment for hot/cold neuron partition, and \name~is the one that integrates all the scheduling strategies proposed in Section \ref{sec:hermes-system}. \update{Furthermore, we also explore when only adopting token-wise prediction or layer-wise prediction to guide the online adjustment of hot/cold partition, denoted as Hermes-token-adjustment and Hermes-layer-adjustment, respectively.} 

\textbf{Load Balancing with Multi-level Optimization. } \fig \ref{fig:ablation-study} shows the contributions of each component in \name~. Utilizing the offline mapper can effectively identify the frequent hot neurons, reducing the computation cost of NDP-DIMMs. As a result, \name-partition can achieve $1.63 \times$ speedup than \name-random. However, the input-specific nature of activation sparsity challenges the offline partition approach. Therefore, further adopting online adjustment for hot/cold partition (\name-adjustment) achieves $1.33 \times$ performance gains over \name-partition. Despite this, the overall execution efficiency is still constrained by the NDP-DIMMs, which possess limited computation capability. Thus, the performance of the resource-constrained NDP-DIMMs can be improved by tackling the load imbalance issues in several NDP-DIMMs. The introduced online remapping method successfully addresses this problem. As a consequence,
the fully optimized Hermes system demonstrates a $1.29 \times$ boost in performance when compared with \name-adjustment.
% 这里就是分析每一部分的优势

\update{\textbf{Benefits of Token-wise and Layer-wise Prediction.}
Compared to \name-partition which only considers the optimal offline neuron placement, \name-token-adjustment and \name-layer-adjustment can achieve $1.08\times$ and $1.11\times$ speedup, respectively, demonstrating the benefits of online adjustment. However, token-wise prediction cannot address fluctuations in neuron activity, making it inaccurate for frequent changes in hot/cold neurons. Simultaneously, layer-wise prediction only relies on the static sampled neuron correlation table to guide the online adjustment, inefficient for constant changes of online adjustment. As a result, using token-wise or layer-wise prediction only cannot effectively unleash the benefits of prediction-based online adjustment.  
}

\subsection{Performance Breakdown}\label{sec:breakdown}

% 这里要分析几个点,首先是 deja vu 中的load weight 的影响,communication 的开销,以及predictor 的开销;然后是 prefill 的开销占比;然后是计算的开销

\fig \ref{fig:performance-breakdown} illustrates the performance breakdown of Deja Vu, \name-base, and \name~on various LLMs. It provides detailed insights into the efficiency sources of \name.

Figure \ref{fig:performance-breakdown}a shows that while Deja Vu benefits from activation sparsity, it still requires loading cold neurons when activated, resulting in communication costs—especially PCIe data transfer—comprising about 89\% of the execution time. On the right side of Figure \ref{fig:performance-breakdown}a, we disregard the effect of communication on performance. The MLP-based predictor in Deja Vu consumes roughly 18.1\% of computation time, further reducing the gains from activation sparsity. Our lightweight predictor, in contrast, contributes less than 0.1\% to runtime overhead. Even with communication costs lowered through reusable neurons at large batch sizes, Deja Vu's performance remains inferior to Hermes.

Figure \ref{fig:performance-breakdown}b compares \name-base and \name. Without activation sparsity, \name-base incurs higher computation costs, especially as batch sizes increase, due to intensive computation on NDP-DIMMs. For example, running LLaMA2-70B offloads over 80\% of computation to NDP-DIMMs, leading to a substantial portion of the execution time being occupied by FC computation. In Hermes, token generation takes 66.40\% of execution time at batch size 1. After optimizing token generation, the prompting stage becomes the bottleneck, accounting for about 33.01\% of the overhead, limiting further inference efficiency improvements.

% \fig \ref{fig:performance-breakdown} presents the performance breakdown of Deja Vu, \name-base and \name~on OPT-30B, OPT-66B, Falcon-40B and LLaMA2-70B. 
% It effectively describes in detail the sources of efficiency of \name. 

% As \fig \ref{fig:performance-breakdown}a shows, despite benefiting from activation sparsity, Deja Vu still needs to load cold neurons when they are activated. Consequently, the communication cost, particularly data transfer on PCIe, makes up about 89\% of the total execution time. On the right side of Figure \ref{fig:performance-breakdown}a, we disregard the effect of communication on performance. The MLP-based predictor used in Deja Vu consumes approximately 18.1\% of the overall computation time, diminishing the gains from activation sparsity. In contrast, our proposed lightweight predictor contributes to less than 0.1\% of the total runtime overhead. The proportion of communication in Deja Vu reduces due to reusable neurons when running at large batch sizes. Nonetheless, the overall performance of Deja Vu is still significantly inferior to Hermes.

% \fig \ref{fig:performance-breakdown}b provides a comparative analysis of the performance breakdown between \name-base and \name. The absence of activation sparsity in the Hermes-base results in considerably higher computation costs compared to \name, especially as the batch size increases. This is primarily due to the intensive computation on NDP-DIMMs, which significantly impacts overall execution efficiency. For instance, when running LLaMA2-70B, over 80\% of the computation is offloaded to NDP-DIMMs, leading to a substantial portion of the execution time being occupied by FC computation. In contrast, in Hermes, the token generation time occupies 66.40\% of the total execution time when at batch size is 1. With the token generation stage fully optimized, the prompting stage becomes the bottleneck, accounting for approximately 33.01\% of the overhead, thus limiting further improvements in inference efficiency.

\begin{figure}
    \centering
    \includegraphics[width=\linewidth]{Fig/dimm_rebuttal.pdf}
    \vspace{-0.3cm}
    \caption{\update{Throughput of four typical LLMs with different numbers of NDP-DIMMs. N.P. denotes the model is not supported by current system.}}
    \label{fig:dimm}
\vspace{-0.3cm}
\end{figure}

\update{\subsection{Sensitivity Studies}}

\subsubsection{\update{Sensitivity analysis of the number of DIMMs}}

\update{\fig~\ref{fig:dimm} illustrates the improvement in LLM throughput as the number of NDP-DIMMs increases. We evaluated four distinct LLM models using a single batch to understand the impact of varying numbers of NDP-DIMMs, while mitigating the effect of limited computation capability. An increase in NDP-DIMMs enhances both memory size and internal bandwidth. Larger memory capacity facilitates the deployment of more extensive models; for instance, deploying Falcon-40B on Hermes necessitates a minimum of four NDP-DIMMs. Additionally, higher internal bandwidth significantly enhances end-to-end performance, addressing the bandwidth limitations that bottleneck current offloading-based systems. However, once sufficient bandwidth is achieved, further increases in the number of NDP-DIMMs do not proportionally boost throughput. For example, LLaMA2-70B exhibits similar throughput with both 8 and 16 NDP-DIMMs. Once the NDP-DIMMs surpass the GPU in performance, additional NDP-DIMMs do not yield further performance gains.}

% \update{
% \fig~\ref{fig:dimm} shows that LLM throughput improves as the number of NDP-DIMMs increases. We evaluated four different LLM models with a single batch to assess the impact on different numbers of NDP-DIMMs, avoiding the effect of limited computation capacity. More NDP-DIMMs provide larger memory size as well as higher internal bandwidth. Abundant memory size allows the deployment of larger models. For example, deploying Falcon-40B on Hermes needs at least 4 NDP-DIMMs. Furthermore, higher internal bandwidth can effectively boost the end-to-end performance, as the limited bandwidth is the bottleneck of existing offloading-based system. However, when sufficient bandwidth is provided, the end-to-end throughput will not be further improved proportionally with the increasing number of NDP-DIMMs. For instance, LLaMA2-70B shows similar throughput with 8 and 16 NDP-DIMMs. Once the NDP-DIMMs outperform the GPU, adding more NDP-DIMMs no longer impacts performance.
% }

\begin{figure}
    \centering
    \includegraphics[width=\linewidth]{Fig/GPU_rebuttal.pdf}
    \vspace{-0.3cm}
    \caption{\update{Throughput of OPT-13B and OPT-30B with various GPUs, including RTX 4090, RTX 3090 and Tesla T4.}}
    \label{fig:gpu}
\vspace{-0.3cm}
\end{figure}

\subsubsection{\update{Sensitivity analysis of various GPUs}}

\update{\fig~\ref{fig:gpu} illustrates the significant impact of different GPUs on the end-to-end throughput of LLM execution. We have included two additional consumer-grade GPUs, Tesla T4 and RTX 3090, in our evaluation. Specifically, Tesla T4 offers 16GB of graphic memory, 320GB/s memory bandwidth, and 65 tensor TOPS (FP16), whereas RTX 3090 provides almost the same graphic memory and bandwidth as RTX 4090, but with 142 tensor TOPS (FP16). Overall, \name~with RTX 4090 achieves an average throughput improvement of $2.02\times$ and $1.34\times$ compared to \name~with Tesla T4 and RTX 3090, respectively. The data loading cost for RTX 3090 is nearly identical to that of RTX 4090. However, RTX 3090 spends more time on prefill and hot neuron computations due to its weaker computation capability. Tesla T4, with its smaller graphic memory and lower memory bandwidth compared to RTX 3090, is inefficient for data loading. Consequently, the choice of GPU device is crucial for optimizing \name~performance.}


\subsubsection{\update{Design Space Exploration for NDP-DIMMs}}
\begin{figure}
    \centering
    \includegraphics[width=\linewidth]{Fig/gemv_rebuttal.pdf}
    \vspace{-0.3cm}
    \caption{\update{Design Space Exploration for NDP-DIMMs with different number of multipliers in each GEMV unit.}}
    \label{fig:gemv}
\vspace{-0.3cm}
\end{figure}

\update{
\fig~\ref{fig:gemv} highlights the impact of increasing the number of multipliers within a GEMV unit per DIMM on LLM inference performance, especially with larger batch sizes. We varied the number of multipliers within a GEMV unit from 32 to 512, thereby enhancing computation capability by 16$\times$. For OPT-13B with a batch size of 1, performance stabilizes once 64 multipliers are reached, as further computation capability yields minimal gains. In contrast, with a batch size of 16, performance continuously improves with additional multipliers, achieving up to a $3.86\times$ speedup. This difference arises because memory bandwidth limits performance for smaller batch sizes due to lower arithmetic intensity, while computation capability becomes the bottleneck with larger batch sizes. To optimize the balance between hardware overhead and performance across various batch sizes, we selected 256 multipliers within the GEMV unit per DIMM.
}

\subsection{Comparison with High-Performance System}\label{sec:comparison-high-performance}

\begin{figure}[t]
    \centering
    \includegraphics[width=\linewidth]{Fig/comparison.pdf}
    \vspace{-0.3cm}
    \caption{Comparison with TensorRT-LLM on LLaMA2-70B.}
    \label{fig:trt-llm-comparison}
\vspace{-0.3cm}
\end{figure}

% 这里可以考虑一下系统的开销和对应的结果
This section discusses the performance gap between our budget-friendly LLM inference system Hermes and state-of-the-art high-performance serving system TensorRT-LLM~\cite{tensorrt-llm}. We kept the input and output sequence lengths set at 128. To handle LLaMA2-70B with a batch size of 16, TensorRT-LLM requires five NVIDIA A100-40GB-SXM4 GPUs. In contrast, 
\name~operates with only one NVIDIA RTX-4090 GPU and affordable NDP-DIMMs. Figure \ref{fig:trt-llm-comparison} displays the performance comparison between TensorRT-LLM and Hermes. For a batch size of 1, Hermes achieves 79.1\% inference efficiency of TensorRT-LLM. Even at a batch size of 16, Hermes retains 24.4\% inference efficiency of TensorRT-LLM. Despite this, Hermes is far more economical than TensorRT-LLM, which is equipped with 5 NVIDIA A100-40GB-SMX4 GPUs. Specifically, Hermes only costs approximately \$2,500, whereas TensorRT-LLM requires \$50000 to support LLaMA2-70B. Hermes provides efficient and low-budget LLM inference for local deployments. 

% Despite this, \name~is far more economical, costing approximately \$2,500 compared to the \$50,000 needed to build a high-performance system with 5 NVIDIA A100-40GB-SMX4 GPUs, making it highly efficient for local deployments with smaller batch sizes.
\section{Conclusion}

This work analysed the results of evolutionary wrapper approaches using decision tree based models as proxies and compared them with common \gls{FE} techniques on a \gls{HL} detection problem. Three experiments were conducted using the proposed framework, each employing different proxy models.

When comparing the three experiments, an interesting behaviour of the framework was discovered, when changing the proxy model. The \gls{DT} experiment drastically reduced the number of features, while the other models did not. To further reduce the number of features, one could bias the grammar or apply some penalty in the fitness function for the individuals that use a large number of features. This might not change the behaviour when using different models other than a \gls{DT}, but it forcefully reduces the number of features.  

The results confirm that FEDORA can reduce the dimensionality of the data while statistically maintaining baseline performance, in every experiment. The framework consistently outperforms the remaining \gls{FE} methods, with statistical significance and large effect sizes, proving itself as a viable alternative.

The best result obtained is 76.2\% balanced accuracy using an individual from the \gls{RF} experiment, and a \gls{XGB} algorithm as the testing model, using 57 total features (45 Original, 6 Engineered and 6 Complex) out of the 60 original ones. When using the least amount of features, the best result is 72,8\% balanced accuracy using an individual from the \gls{DT} experiment and a \gls{RF} algorithm as the testing model, using a single complex feature.

In future work, exploring the above-mentioned behaviours might be relevant to better understanding them, namely when biasing the grammar or penalizing the use of many features in the fitness function. Concerning the explainability of the FEDORA transformations, researching meaningful grammar operators might prove useful in addressing problem-specific needs. In this case, having logical operators for the boolean features, which have values of "yes" or "no", and the choice of a simple decision algorithm as the proxy, may increase explainability. Additionally, the previous study has identified several areas for future research, yet to be addressed. For instance, comparing the framework with other common and more complex methods and completing the full \gls{ML} pipeline through the use of a method that addresses the \gls{CASH}, such as \cite{assunccao2020evolution}, and comparing it to other full pipeline frameworks, could be beneficial for contextualizing and evaluating the framework within the \gls{AutoML} and \gls{EC} domains. The framework still needs to be analysed with different datasets to properly assess its generalization capabilities.
\section*{Limitations}
\label{sec:limitation}
This work has the following limitations:
\textbf{(i)} Although the proportion of high/medium/low resource languages in \dcad has greatly increased compare to existing multilingual datasets, a significant portion of the languages are still very low resource languages. Future work will explore to collect data for extremely low-resource languages through other modalities (e.g., images) through technologies like OCR.
% \textbf{(ii)} We evaluate the new data cleaning framework only on four classical anomaly detection algorithms; however, since the framework is algorithm-independent, it should also be effective with the latest anomaly detection algorithms~\cite{batzner2024efficientad}.
\textbf{(ii)} Due to computational limitations, we evaluate our framework and dataset only on the LLaMA-3.2-1B model. However, we expect similar evaluation results on larger models, such as 7B, 13B and even 70B models.
\section*{Ethics Statement}
Our dataset integrates existing multilingual datasets, such as MaLA~\cite{ji2024emma} and Fineweb-2~\cite{penedo2024fineweb-2}, and includes newly extracted data from Common Crawl, providing large-scale and high-quality training corpora to support the training of multilingual large language models (LLMs).
Additionally, we propose a novel data cleaning method to filter out potentially toxic documents, reducing potential ethical concerns. 
However, performing fine-grained analysis on such a vast dataset (46.72TB) remains a significant challenge.
To address this, we will release the dataset for the community to explore and research extensively.
Furthermore, since our dataset is derived from open-source datasets, we will adhere to the open-source policies of these datasets to promote future research in multilingual LLMs, while mitigating potential ethical risks.
Therefore, we believe our dataset does not pose greater societal risks than existing multilingual datasets. 

% Bibliography entries for the entire Anthology, followed by custom entries
%\bibliography{anthology,custom}
% Custom bibliography entries only

\bibliography{custom,acl}

\newpage
\appendix

\newpage
\onecolumn
%\pagestyle{empty}

\section{Additional Materials}


\begin{table*}[h]
\caption{Overview of study participants.}
\begin{tabular}{lclll}
\toprule
\textbf{ID} & \textbf{Age Range (Years)} & \textbf{Gender}     & \textbf{Job Title}  & \textbf{Frequency of PowerPoint Usage} \\
\midrule
P01            & 18-29     & Female     & Senior Product Designer       & Multiple times per week                      \\
P02            & 30-39     & Female     & Senior Product Manager        & Every day                                    \\
P03            & 18-29     & Female     & Research Intern (PhD)             & Multiple times per month                     \\
P04            & 18-29     & Male       & Senior Human Factors Engineer & Every day                                    \\
P05            & 30-39     & Female     & Senior Designer               & Multiple times per week                      \\
P06            & 40-49     & Female     & Senior User Researcher        & Multiple times per week                      \\
P07            & 30-39     & Male       & Senior Product Designer       & Every day                                    \\
P08            & 18-29     & Female     & Content Designer              & Multiple times per month                     \\
P09            & 30-39     & Male       & Research Intern (PhD)         & Multiple times per month                     \\
P10            & 18-29     & Non-binary & Research Intern (PhD)         & Multiple times per year                      \\
P11            & 30-39     & Female     & Design Program Manager        & Multiple times per week                      \\
P12            & 40-49     & Female     & UX Researcher                 & Multiple times per month      \\                                       
\bottomrule
\end{tabular}
\label{tab:participants}
\Description{Study participant demographics. (Table is machine readable). Study participant demographics. (Table is machine readable). The table provides demographic information for 12 participants (P01 to P12) involved in the study. Columns include ID, Age Range (Years), Gender, Job Title, and Frequency of PowerPoint Usage. The age ranges span from 18-29 to 40-49 years, with job titles such as Senior Product Designer, Research Intern (PhD), and UX Researcher. Participants’ frequency of PowerPoint usage varies from “Multiple times per year” to “Every day.” Two participants, P10 and P04, are noted as non-binary and male, respectively.}
\end{table*}



\rev{\subsection{Further System Implementation Details }}

\subsubsection{\rev{\textbf{Tag Suggestion Mechanism}}}

\rev{To generate intent tag suggestions based on a user's existing tags on the slide deck steering canvas, we construct a call to GPT comprising the static \texttt{SYSTEM\_PROMPT} and the dynamic \texttt{USER\_CONTEXT} below. 
\texttt{USER\_CONTEXT} is a string dynamically constructed from all active intent tags \texttt{[attribute:value]} for each tag group \texttt{(Narrative, Visual Style, Content Sources)}. }

\begin{lstlisting}
SYSTEM_PROMPT (static):  
You are an assistant to help users author slide presentations in PowerPoint. 
You are helping the user develop the narrative and visual style and extracting content from source documents relevant to the presentation. 
You will receive instructions to help with one of these three buckets: Narrative, Visual Style, and Content Sources.  
The user will already have descriptions, tags, and media elements that describe different aspects of the presentation.  
Your task is to suggest additional keywords and concepts for each of these buckets.  
The user will prompt you in the format:  
**Bucket** attribute1:value1, attribute2:value2, ...  
A bucket might be empty, in which case you should suggest concepts based on the other buckets.  
A value might not be associated with an attribute.  
For each user prompt, suggest 7 concepts for each of the requested buckets.  
A concept is a keyword or phrase that is relevant to the user's presentation. A concept should be a single word or a short phrase. Each concept should have an attribute and value.  
Return your response in JSON format. Only return JSON format without any additional Markdown wrapper code blocks.  
 
Here is an example of the JSON return format:  
{
    "Narrative": ["attribue1:value1", "attribue1:value1", ...],  
    "Visual Style": ["attribue1:value1", "attribue1:value1", ...],  
    "Content Sources": ["attribue1:value1", "attribue1:value1", ...]  
}

USER_CONTEXT (dynamic): 
**Narrative** attribute1:value1, attribute2:value2, ... 
**Visual Style** attribute1:value1, attribute2:value2, ... 
**Content Sources** attribute1:value1, attribute2:value2, ... 
\end{lstlisting}



\subsubsection{\rev{\textbf{Outline Generation From Intent Tags}}}

\rev{To generate an outline based on a user's existing tags on the slide deck steering canvas, we construct a call to GPT comprising the static \texttt{SYSTEM\_PROMPT} and the dynamic \texttt{USER\_CONTEXT} below.
\texttt{USER\_CONTEXT} is a string dynamically constructed from all active intent tags \texttt{[attribute:value]} for each tag group \texttt{(Narrative, Visual Style, Content Sources)}.
GPT is instructed to generate an outline in Markdown format, including the URLs of image reference tags if present. }

 
\begin{lstlisting}
SYSTEM_PROMPT (static): 
You are an assistant to help users author slide presentations in PowerPoint. 
You are creating an outline for the user's presentation. 
The user will have descriptions and keywords that describe different aspects of the presentation. 
The user will prompt you in the format: attribute1:value1, attribute2:value2, ... 
Ignore all the attributes related to the visual style.  
If the content sources include ImageUrl attributes, you should include these images in the outline as images in the markdown. 
Respond with an outline of the presentation in the format:  
section title - section content bullet points. 
Return the outline in markdown format without ```markdown at the start and end. Only return markdown content for the outline. 

USER_CONTEXT (dynamic): 
**Narrative** attribute1:value1, attribute2:value2, ... 
**Visual Style** attribute1:value1, attribute2:value2, ... 
**Content Sources** attribute1:value1, attribute2:value2, ... 
\end{lstlisting}






\subsubsection{\rev{\textbf{Slide generation mechanism}}}

\rev{ To generate the JSON structure for a slide deck, we construct a call to GPT comprising the static \texttt{SYSTEM\_PROMPT} and the dynamic \texttt{USER\_CONTEXT}. 
\texttt{USER\_CONTEXT} is a string containing the \texttt{PRESENTATION OUTLINE} in markdown, \texttt{META INFORMATION}, which is a string containing all active intent tags \texttt{[attribute:value]}, and optionally the \texttt{REFERENCE SLIDE DECK TEMPLATE}, which is a JSON string of a reference slide deck (if provided).  
The general slide deck template schema and slide generation mechanism are based on the Spectacle\footnote{https://github.com/FormidableLabs/spectacle} library. }

 
\begin{lstlisting}
SYSTEM_PROMPT (static): 
Your task is to generate a slide presentation deck from an outline and meta information.  
You will receive the presentation OUTLINE in markdown format from the user. 
You will also receive additional META INFORMATION about the desired presentation's Narrative, Visual Style, and Content Sources.  
The META INFORMATION will describe different aspects of the presentation.  
You will receive the META INFORMATION in the format:  
**Bucket** attribute1:value1 (optional description), attribute2:value2 (optinal description), ...  
A bucket might be empty, in which case you should suggest concepts based on the other buckets.  
A value might not be associated with an attribute.  
The META INFORMATION might also contain USER DOCUMENTS with content that has to be included in the presentation. 

GENERATING SLIDES  
Your task is to generate a slide deck. Each slide has to consist of a LAYOUT, CONTENT, and THEME.  

LAYOUT:  
The layout determines the arrangement of the content on the slide.  
There are five different layouts to choose from:  
"title", "listOrParagraph", "verticalImage", "fullImage". 

CONTENT:  
Each layout takes different content parameters depending on the layout type. Optionally, each slide can take an image url as background image in the content.  
The background image will replace the background color of the slide.  
The "listOrParagraph" and "verticalImage" layouts take either a list or a paragraph as content besides the title and image.  
For bullet lists, never use more than 3 bullets. 

THEME:  
The theme determines the fonts, colors, and font sizes used in the slide.  
 
fonts: { header: fontname, text: fontname }  
"header" is the font used for titles and "text" is the font used for body, lists, paragraphs.  
There are six fonts to choose from: "Quicksand", "Playfair Display", "Montserrat", "Merriweather", "Roboto" and "Roboto Condensed". 

colors: { primary: color, secondary: color, tertiary: color }  
"primary" is the main color used for text, "secondary" is the color used for accents, and "tertiary" is the color used for backgrounds. 

fontSizes: { h1: size, text: size }  
"h1" is the font size used for titles and "text" is the font size used for body text. 

REFERENCE SLIDE DECK TEMPLATE  
The user might also provide a REFERENCE SLIDE DECK TEMPLATE with a slide deck template they want to use for the visual style.  
Important: If a REFERENCE SLIDE DECK TEMPLATE is provided, use only this template for the visual style (font, backgroundImage, sizes, colors) and IGNORE all visual style information from the generic example slide deck below.  
ALWAYS use the backgroundImge URLs from the REFERENCE SLIDE DECK TEMPLATE on each slide! 

REWORK CONTENT FROM THE OUTLINE  
You should rework the content from the OUTLINE to fit the slide deck based on the desired format specified in the META INFORMATION.  
For example, if the META INFORMATION specifies that the presentation should have a formal tone, you should rework the content to fit this tone.  
If the META INFORMATION specifies that the presentation should have a more verbose text, you should rework the content to fit this style. 

YOUR TASK:  
Your task is to compose a slide deck based on the outline and to adapt the content to the layout and theme.  
Your task is to define the theme and layout of the slides based on the OUTLINE and META INFORMATION provided by the user.  
Generate the number of slides based on the outline and meta information provided by the user.  
Keep the visual style consistent across all slides.  
Especially the fonts, colors, background color and font sizes.  
You should return the slide deck in JSON format. Only return JSON format without any additional Markdown wrapper code blocks.  

Here is a generic example of a slide deck in JSON format: 
[{ 
    slideNumber: 1, 
    layout: "title", 
    content: { 
      title: "A presentation about something", 
      subtitle: "by someone", 
      backgroundImage:  "url(<image url placeholder>)" 
    }, 
    theme: { 
      fonts: { 
        header: '"Playfair Display", serif', 
        text: '"Quicksand", sans-serif', 
      }, 
      colors: { 
        primary: "#000", 
        secondary: "#000", 
        tertiary: "#fff", 
      }, 
      fontSizes: { 
        h1: "100px", 
        text: "44px", 
      }, 
      space: [16, 24, 32], 
    }, 
  },{ 
    slideNumber: 2, 
    layout: "listOrParagraph", 
    ...
    (PARTS OMITTED) 
    ...
}] 
 

USER_CONTEXT (dynamic): 
PRESENTATION OUTLINE: 
$CURRENT_OUTLINE_IN_MARKDOWN_STRING 

META INFORMATION:  
$CURRENT_ACTIVE_TAGS_AS_ATTRIBUTE:VALUE_PAIRS_CLUSTERED_BY_GROUP 

(optional) REFERENCE SLIDE DECK TEMPLATE: 
$REFERENCE_SLIDE_DECK_JSON 
\end{lstlisting}


\subsubsection{\rev{\textbf{Mechanism to create tags for an existing slide (tag grounding act)}}}

\rev{To generate a collection of intent tags based on a provided image of the current slide, we construct a call to GPT comprising the static \texttt{SYSTEM\_PROMPT} and the dynamic \texttt{USER\_CONTEXT} below.
\texttt{USER\_CONTEXT} is a string of the current slide in a base64 image format.
In addition to the slide image, we experimented with including the active steering board tags in the prompt but refrained from including these in this version of IntentTagger for simplicity. 
Future explorations outside the scope of this paper should investigate how "global" steering board-level intent tags can meaningfully propagate to individual slides and how changes on single slides can propagate back to the global steering board level while meaningfully resolving conflicts between global and slide-level tags. 
}


\begin{lstlisting}
SYSTEM_PROMPT (static): 
You are an assistant to help users author slide presentations in PowerPoint. 
You will receive an image of a slide from their slide presentation deck.  
Your task is to analyze the slide deck and return descriptive attributes that best describe the slide.  
You should return these attributes related to three categories: Narrative, Visual Style, and Content Sources.     
Each attribute should consist of a label and a value, like Tonality:Formal or Typography:Modern. 
For each bucket, return between 2 to 6 attributes.  
Return your response in JSON format.  
Only return JSON format without any additional Markdown wrapper code blocks. 

Here is an example of the JSON format: 
{ 
    "Narrative": ["attribue1:value1", "attribue1:value1", ...], 
    "Visual Style": ["attribue1:value1", "attribue1:value1", ...], 
    "Content Sources": ["attribue1:value1", "attribue1:value1", ...] 
} 

USER_CONTEXT (dynamic): 
$IMAGE_OF_CURRENT_SLIDE_AS_BASE64_STRING 
\end{lstlisting}










\subsection{Semi-structured Task Outcomes }
This section contains screenshots of IntentTagger (deck steering board and slide panel) taken at the end of each participant's semi-structured slide creation task from our user study (see study phase 3 at section \ref{sec:study-procedure}). 


\begin{figure}[H]
    \centering
  \includegraphics[width=0.99\linewidth]{Figures/Appendix/P01.png}
  \caption{Semi-structured task outcome of P01}
  \Description{The figure displays a screenshot of the IntentTagger interface for participant P01’s open-ended slide creation task. The left side shows the Deck Steering Board, where tags related to Narrative, Visual Style, and Content Sources are organized. Tags include topics such as “Hiking,” “Hiking Shoes,” and imagery of “Mountains.” The right side shows the Slide Panel, featuring a generated slide deck about hiking in Washington, including information on popular hiking spots like Mount Rainier and the Olympic National Park.} 
  \label{fig:appendix_task4_P01}
\end{figure}

\begin{figure}[H]
    \centering
  \includegraphics[width=0.99\linewidth]{Figures/Appendix/P02.png}
  \caption{Semi-structured task outcome of P02}
  \Description{ The figure shows the IntentTagger interface for participant P02’s semi-structured task. The Deck Steering Board on the left features tags related to Narrative, Visual Style, and Content Sources. The tags include topics like “Introduction and benefits of yoga,” “Physical strength,” and “Mental clarity” under Narrative, while Visual Style contains tags such as “Font: Handwritten” and “Color Background: Light blue.” The Content Sources include wellness websites, case studies, and testimonials with relevant images. On the right, the Slide Panel shows a slide deck on the benefits of yoga, including sections on physical and mental health benefits, clarity, and focus.} 
  \label{fig:appendix_task4_P02}
\end{figure}

\begin{figure}[H]
    \centering
  \includegraphics[width=0.99\linewidth]{Figures/Appendix/P03.png}
  \caption{Semi-structured task outcome of P03}
  \Description{The figure shows the IntentTagger interface for participant P03’s semi-structured task. The Deck Steering Board on the left contains tags related to Narrative, Visual Style, and Content Sources. The Narrative tags include “Topic: Joys of Kayaking,” “Goal: Encourage others to kayak,” and “Structure: Step-by-step guide.” The Visual Style tags include “Colors: Blue and Green” and “Textures: Water ripples.” The Content Sources section includes videos, blogs, and articles related to kayaking. On the right, the Slide Panel shows a generated slide deck on kayaking, with slides covering topics such as “Why Kayak in Washington” and “Essential Techniques for Beginners.” } 
  \label{fig:appendix_task4_P03}
\end{figure}

\begin{figure}[H]
    \centering
  \includegraphics[width=0.99\linewidth]{Figures/Appendix/P04.png}
  \caption{Semi-structured task outcome of P04 (Note: The user was unable to add images to their presentation, due to a temporary outage of the image search API.)}
  \Description{The figure shows the IntentTagger interface for participant P04’s semi-structured task. The Deck Steering Board on the left contains tags related to Narrative, Visual Style, and Content Sources. The Narrative tags include “Topic: Home Made Pizza Making,” “Focus: Practical,” and “Objective: Step-by-step guide.” The Visual Style tags include “Color Scheme: Earthy Palette” and “Layout: Clean and simple.” The Content Sources include online recipe forums, interviews with chefs, and food magazines. On the right, the Slide Panel shows a slide deck on pizza-making, covering topics such as “Overview of Home-Made Pizza” and “Case Study: The Perfect Margherita Pizza.” Note: The user was unable to add images to the presentation due to an API outage. } 
  \label{fig:appendix_task4_P04}
\end{figure}

\begin{figure}[H]
    \centering
  \includegraphics[width=0.99\linewidth]{Figures/Appendix/P05.png}
  \caption{Semi-structured task outcome of P05}
  \Description{The figure shows the IntentTagger interface for participant P05’s semi-structured task. The Deck Steering Board on the left contains tags for Narrative, Visual Style, and Content Sources. The Narrative section includes tags like “Topic: Growing Dahlias,” “Location: Pacific Northwest,” and “Introduction: Overview of Dahlias.” The Visual Style section includes tags such as “Layout: Bold and colorful,” “Background: White,” and “Font Style: Elegant and legible.” The Content Sources include gardening guides, images from botanical gardens, and notes from gardening workshops. On the right, the Slide Panel displays a slide deck on growing dahlias, covering topics such as “Understanding Dahlias,” “Climate and Soil Requirements,” and “Planting Dahlias.”} 
  \label{fig:appendix_task4_P05}
\end{figure}

\begin{figure}[H]
    \centering
  \includegraphics[width=0.99\linewidth]{Figures/Appendix/P06.png}
  \caption{Semi-structured task outcome of P06}
  \Description{The figure shows the IntentTagger interface for participant P06’s semi-structured task. The Deck Steering Board on the left contains tags for Narrative, Visual Style, and Content Sources. The Narrative section includes tags like “Cycling for Fitness,” “Competitive Events,” and “Famous Climbs in France.” The Visual Style section includes tags such as “Fonts: Sporty and dynamic” and “Backgrounds: Scenic roadways.” The Content Sources include images of professional cyclists, race footage, and sports documentaries. On the right, the Slide Panel displays a slide deck on cycling, covering topics such as “Famous Climbs in France,” “Competitive Events,” and “Training for Cycling.”} 
  \label{fig:appendix_task4_P06}
\end{figure}

\begin{figure}[H]
    \centering
  \includegraphics[width=0.99\linewidth]{Figures/Appendix/P07.png}
  \caption{Semi-structured task outcome of P07}
  \Description{The figure shows the IntentTagger interface for participant P07’s semi-structured task. The Deck Steering Board on the left contains tags for Narrative, Visual Style, and Content Sources. The Narrative section includes tags such as “Topic: Skateboarding,” “Culture: Street vs. Park,” and “Introduction: History of Skateboarding.” The Visual Style section includes tags like “Font: Comic Sans,” “Theme: Graffiti Art,” and “Layout: Dynamic and Energetic.” The Content Sources include videos, articles, interviews with professional skateboarders, and skateboarding culture websites. On the right, the Slide Panel shows a slide deck on skateboarding culture, with sections on “Introduction to Skateboarding,” “Street Skateboarding Culture,” and “Park Skateboarding Culture.” } 
  \label{fig:appendix_task4_P07}
\end{figure}

\begin{figure}[H]
    \centering
  \includegraphics[width=0.99\linewidth]{Figures/Appendix/P08.png}
  \caption{Semi-structured task outcome of P08}
  \Description{The figure shows the IntentTagger interface for participant P08’s semi-structured task. The Deck Steering Board on the left contains tags related to Narrative, Visual Style, and Content Sources. The Narrative section includes tags such as “Topic: How to play TTRPGs,” “Second topic: Unique game mechanics,” and “Conclusion: Resources and communities.” The Visual Style section includes tags like “Layout: Storybook format” and “Imagery: Fantasy artwork.” The Content Sources include videos, books, and role-playing forums related to tabletop RPGs. On the right, the Slide Panel shows a slide deck on how to play TTRPGs, covering topics like “How to Play TTRPGs,” “Ways to Start Playing Today,” and “Additional Resources.” } 
  \label{fig:appendix_task4_P08}
\end{figure}

\begin{figure}[H]
    \centering
  \includegraphics[width=0.99\linewidth]{Figures/Appendix/P09.png}
  \caption{Semi-structured task outcome of P09}
  \Description{The figure shows the IntentTagger interface for participant P09’s semi-structured task. The Deck Steering Board on the left contains tags related to Narrative, Visual Style, and Content Sources. The Narrative section includes tags like “Topic: Taekwondo,” “Benefits: Physical Fitness,” “Benefits: Learn Self-Defense,” and “Introduction: Principles of Taekwondo.” The Visual Style section includes tags such as “Theme: Energetic,” “Colors: Black and White,” and “Illustrations: Kicks.” The Content Sources include statistics on participation, photos from competitions, and training videos. On the right, the Slide Panel displays a slide deck on the benefits of Taekwondo, covering topics such as self-defense, physical fitness, mental health, and body awareness.} 
  \label{fig:appendix_task4_P09}
\end{figure}

\begin{figure}[H]
    \centering
  \includegraphics[width=0.99\linewidth]{Figures/Appendix/P10.png}
  \caption{Semi-structured task outcome of P10}
  \Description{ The figure shows the IntentTagger interface for participant P10’s semi-structured task. The Deck Steering Board on the left contains tags related to Narrative, Visual Style, and Content Sources. The Narrative section includes tags like “Topic: New Tennis Club in Town,” “Call to Action: Join Now,” and “Audience: College Students.” The Visual Style section includes tags such as “Font: Neo Modern,” “Theme: Sporty,” and “Color: Green and Blue.” The Content Sources include introductory videos, student surveys, brochures with club details, and expert opinions. On the right, the Slide Panel displays a slide deck introducing the tennis club, covering sections like “Why Join Our Club?” “Club Launch Party,” and “Benefits for College Students.” } 
  \label{fig:appendix_task4_P10}
\end{figure}

\begin{figure}[H]
    \centering
  \includegraphics[width=0.99\linewidth]{Figures/Appendix/P11.png}
  \caption{Semi-structured task outcome of P11}
  \Description{The figure shows the IntentTagger interface for participant P11’s semi-structured task. The Deck Steering Board on the left contains tags related to Narrative, Visual Style, and Content Sources. The Narrative section includes tags like “Topic: Trombones,” “Content: History of Trombones,” and “Content: Types of Music It Appears In.” The Visual Style section includes tags such as “Color Scheme: Dark and Moody,” “Typography: Sans Serif,” and “Imagery: Historic Photos.” The Content Sources include music journals, music history books, and interviews with trombone experts. On the right, the Slide Panel displays a slide deck on the trombone, covering sections like “Introduction to Trombones,” “When the Trombone Was Invented,” and “Types of Music It Appears In.” } 
  \label{fig:appendix_task4_P11}
\end{figure}

\begin{figure}[H]
    \centering
  \includegraphics[width=0.99\linewidth]{Figures/Appendix/P12.png}
  \caption{Semi-structured task outcome of P12}
  \Description{The figure shows the IntentTagger interface for participant P12’s semi-structured task. The Deck Steering Board on the left contains tags related to Narrative, Visual Style, and Content Sources. The Narrative section includes tags such as “Topic: Paddle Boarding,” “Call to Action: Try Paddle Boarding,” and “Audience: Outdoor Enthusiasts.” The Visual Style section includes tags like “Theme: Nautical,” “Colors: Blue and White,” and “Imagery: Ocean Waves.” The Content Sources include videos, tips for beginners, and personal stories from paddle boarding enthusiasts. On the right, the Slide Panel displays a slide deck on paddle boarding, covering sections like “What is Paddle Boarding?” “Benefits of Paddle Boarding,” and “Tips for Beginner Paddle Boarders.” } 
  \label{fig:appendix_task4_P12}
\end{figure}


\end{document}
