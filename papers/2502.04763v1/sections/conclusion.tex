\section{Conclusion}
\label{sec:conclusion}

We proposed with \emph{SVA}$k_{\text{ADD}}$ a new algorithm to approximate Shapley values.
It falls into the class of approaches that fit a structured surrogate game to the observed value function instead of providing mean estimates via Monte Carlo sampling.
Despite restricting the surrogate game to be $k$-additive, our developed method is model-agnostic.
It is also applicable to any cooperative game without posing further assumptions since its underlying optimization problem provably yields the Shapley value.
We investigated empirically the trade-off that the choice of the parameter $k$ poses.
Further, \emph{SVA}$k_{\text{ADD}}$ exhibits competitive results with other existing approaches depending on the considered explanation type, dataset, and available for budget for sampling, allowing us to conclude the non-existence of a dominating approximation method.

\paragraph{Limitations and Future Work.}

While the surrogate game's flexibility increases with higher $k$-additivity, it also requires more observations to begin with in order to obtain a unique solution of the optimization problem, eventually posing a practical limit on $k$.
Adopting further techniques to the sampling procedure within our method, serves as a natural avenue for further research to improve approximation performance.
We expect future investigations of differently structured surrogate games to yield likewise fruitful results and contribute to the advancement of this class of approximation algorithms. 
Note that, besides the estimated Shapley values, our proposal could also provide the interaction effects when $k \geq 2$.
Although we did not address these parameters, future works can extract the estimated interaction indices and use them to investigate redundant or complementary features.
For instance, this could be of interest in practical applications where interaction between features are relevant as for example in disease detection.