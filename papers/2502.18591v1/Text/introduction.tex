\section{Introduction}

Computer Aided Engineering (CAE) has a long history in industrial engineering. Multi-physics simulations, including Computational Fluid Dynamics (CFD), are among the most used simulation technologies in CAE \cite{liu2022eighty,Kelsall2022CFD}. Even though technology has evolved exponentially in terms of speed and accuracy \cite{ruede2018research}, they still take excessive amounts of time. While an enormous speedup has been recently obtained by utilizing GPUs \cite{gavranovic2024fastsolvers}, further acceleration is at risk since CFD problems can be simulated mathematically optimally \cite{becker2001optimal,Hackbusch2013multi}. 

With the rapid evolution of machine learning (ML) technologies, there is a unique potential to address this challenge \cite{karniadakis2021physics, weinan2021dawning}. Corresponding methods promise orders of magnitude of acceleration \cite{hutson2020ai} and applications range from weather predictions \cite{lam2023learning}, medical applications \cite{karniadakis2021physics} to industrial use cases \cite{lavin2021simulation}. Methods comprise pure ML methods, such as Operator Learning \cite{kovachki2021neural} or physics-constrained methods \cite{karniadakis2021physics}, as well as hybrid methods \cite{sanderse2024scientific}.

Within this contribution, we focus on a hybrid method, specifically on the \textit{solver-in-the-loop} approach of \citeauthor{um2020solver} (\citeyear{um2020solver}) and \citeauthor{kochkov2021machine}(\citeyear{kochkov2021machine}). In these two approaches, traditional numerical methods are augmented by ML components to increase their speed and/or accuracy. Both approaches demonstrated remarkable performance (using state-of-the-art industrial CFD codes as the baseline), particularly in their generalization capabilities. However, they employ convolutional neural networks (CNNs) which has significant implications for the underlying solver architecture and thus limits compatibility with large-scale industrial CFD codes.

\begin{figure}[H]
    \centering
    \includegraphics[width=0.98\columnwidth]{Figures/drawio_overview2.pdf}
    \caption{Hybrid \textit{solver-in-the-loop} approach which additionally to physics variables introduces long term memory states. These memory states are transported along the fluid flow as the physics variables are. The corresponding evolution is handled by the Transported Memory Cell (see Figure \ref{fig:cell}), which combines a classical base physics solver with a ML-based augmentation allowing to coarse grain discretiziation without loss of prediction accuracy. }
    \label{fig:overview}
\end{figure}

Our contributions in this work are as follows:
\begin{itemize}
  \item Investigate the impact of the CNN architecture on performance, particularly focusing on the input stencil size.
  \item Propose an alternative approach that utilizes more local information (direct cell neighbors only, making it suitable for state-of-the-art industrial solvers.
  \item Benchmark and evaluate the new approach in comparison to previous approaches.
\end{itemize}

We thereby introduce a new neural network architecture, which we coin \textit{Transported Memory Network}. This architecture leverages the concept of a long term memory (similar to Long Short Term Memory architectures; \citeauthor{hochreiter1997long}, \citeyear{hochreiter1997long}) which is effectively transported with the flow. It thus appropriately reflects the corresponding physics behavior in an Eulerian coordinate systems which is fixed in space rather than advected with the flow. In addition to reflecting physics more correctly, we also expect a better generalization than CNNs when considering more complex geometries. The proposed approach relies only on direct neighbours and is thus relatively geometry agnostic. 

%\todo{I added this last paragraph, can you please check it.}

\begin{figure}[H]
    \centering
    \includegraphics[width=0.99\columnwidth]{Figures/drawio_cell.pdf}
    \caption{Schematic sketch of the architecture of the transported memory cell, combining a classical base physics solver with a ML-based augmentation. For the specific neural network architectures see Figure \ref{fig:stencil_arch} and \ref{fig:nets_architecture}.} 
    \label{fig:cell}
\end{figure}

