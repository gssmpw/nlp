\section{Introduction}
\label{sec:intro}

% {\color{magenta}
With the rise of large language models (LLMs), such as ChatGPT or Llama, we have entered an era where we can communicate and interact with artificial intelligence and knowledge-based systems using human language.
However, such communication is currently largely limited to text only,
% \MZ{limited to verbal social signals?},
with some speech-based exceptions like GPT-4o.
As beings with spatial audio-visual sensing, humans evolved to communicate best through speech and facial expressions.
An immersive and natural interaction between a human and an AI system therefore requires more than just text.

In this work, we aim to get one step closer to closing this gap.
We demonstrate a method that synthesizes a photo-realistic 4D avatar, based on text input only, and generates speech, facial expressions, head motion, and lip sync simultaneously.
Since listening and reactive expressions are just as much part of a communication as speaking, we further show that we can learn active listening behavior -- like back-channeling of expressions -- from the audio-visual input from a user.
Overall, our approach allows to animate an \textit{always-on} 4D avatar by transforming text (\eg, as obtained from an LLM) into expressive speech and facial motion, and exhibits active listening and expressive reactions, leading to empathetic dyadic interactions.
% when it is not speaking.

Our work is closely related to talking face generation, which has been a topic of research exploration for many years.
% Early approaches rely on rule-based methods or hand-engineered features~\cite{?}.
Early works learn phoneme-to-viseme mappings~\cite{bregler1997video,voicepuppetry}.
More recent approaches model talking faces with deep neural networks, conditioned on speech signals, either for 2D videos~\cite{wav2lip,guo2021ad,pcavs,zhang2023sadtalker} or for 3D meshes~\cite{VOCA2019,MeshTalk,fan2022faceformer,aneja2023facetalk}. Generative adversarial networks (GANs)~\cite{goodfellow2014generative} have repeatedly shown appealing results~\cite{wav2lip,Vougioukas_2019_CVPR_Workshops,Yang:2020:MakeItTalk,Jang_2024_CVPR}. Currently, diffusion models~\cite{ho2020denoising} have taken over the generative modeling space, with VASA-1~\cite{xu2024vasa} achieving lifelike generation of audio-driven talking faces in 2D videos.

% One of the first notable works in 2D talking face generation creates convincing lip synchronization of the former U.S.\ President Barack Obama~\cite{?}.
% Generative adversarial networks (GANs) have repeatedly shown appealing results in subsequent works~\cite{?}.
% In the 3D domain, MeshTalk~\cite{?} and FaceFormer~\cite{?} show expressive mesh animation from speech inputs, yet none of them address photorealistic 3D avatars.
% More recently, diffusion models have taken over the generative modeling space and improved talking head models emerged~\cite{?}, including VASA-1~\cite{?}, which achieves lifelike generation of audio-driven talking faces in 2D videos.

However, these works tend to fall short on one or multiple axes.
\textbf{\textit{First}}, most are limited to cascaded systems. They assume that an input speech signal is already provided, either as a real recording of a human voice or generated by a pre-trained text-to-speech system, and mainly focus on precise lip synchronization.
Connecting these systems to an LLM requires a cascaded approach of text-to-speech, followed by speech-to-vision.
%Some propose additional control of facial expressions and/or head pose, conditioning on learned embeddings~\cite{emotional} or other videos~\cite{pc-avs…}.
%Only a few works attempt to drive a talking face from just text, either directly~\cite{text-driven}, or through a cascaded approach of text-to-speech and speech-to-vision~\cite{}.
In contrast, we propose an architecture that \textit{jointly} generates audio and visual outputs, directly from text.
In this way, we achieve natural synchronization of different modalities (\eg, speech intonation and corresponding eyebrow motion) and avoid latency and error accumulation that cascaded systems suffer from.
\textbf{\textit{Second}}, existing systems typically focus on monadic settings.
They can generate facial animation given text or speech, yet do not consider non-speech cases.
In other words, existing systems can actively speak, but they do not generate authentic listening behavior.
We demonstrate that conditioning on audio-visual user input creates authentic listening behavior, such as back-channeling of smiles.
\textbf{\textit{Third}}, most existing works operate on 2D video or on untextured 3D meshes which lack detail.
Our approach operates on high-quality photo-realistic 4D avatars instead.

Our proposed AV-Flow (Audio-Visual Flow Matching) consists of two inter-connected diffusion transformers.
Given input text tokens, one transformer generates the speech signal and the other one	generates the visual output.
%The visual transformer works in the latent space and produces the holistic head and facial dynamics, including lip motion, facial expressions, eyebrow and eyelid movements.
Through intermediate highway connections, we ensure communication between the speech and visual modalities.
% the speech transformer and the visual transformer. 
In this way, AV-Flow synthesizes highly correlated speech and vision (\eg, synchronized speech intonation with facial dynamics).
% (\eg, eyebrow motion).
%It is also versatile in terms of input conditioning: we show that we can drive a talking face given text alone as well as audio-visual user input to improve dyadic conversational behavior.
With additional conditioning on a user's video and audio, we allow our system to reason over the user's behavior and generate avatar motion accordingly, leading to richer expressions, back-channeling of emotional cues like smiles, or affirming head nods.
We resort to flow matching~\cite{lipman2022flow} as a training objective, in order to achieve fast inference and synthesize human-like speech and 4D visual outputs, that capture natural nuances and expressive motion.

In brief, our contributions are as follows:
\begin{itemize}
  % \item We present \MethodName, a method that synthesizes photo-realistic 4D talking avatars only from text.
  \item We introduce AV-Flow, a novel approach for joint audio-visual generation of 4D talking avatars using flow matching, given just input text.
  % \item In contrast to existing speech-driven or cascaded systems, our approach jointly generates speech and facial motion, leading to correlated audio-visual outputs. 
  % style consistency in the audio and visual domain;
  \item Our fully-parallel diffusion transformers ensure cross-modal interaction through intermediate highway connections, generating correlated speech and visual outputs.
  % \item \MethodName is reactive to audio-visual inputs of a user, and therefore enables dyadic conversational scenarios, where the avatar can be \textit{always-on} and actively listens, as opposed to be limited to speaking.
  \item \MethodName enables dyadic conversations, by animating an \textit{always-on} avatar that actively listens and reacts to the audio-visual input of a user.
\end{itemize}



%In brief, our contributions are as follows:

% \begin{itemize}
%     \item We introduce AV-Flow, a novel approach for joint audio-visual generation of 4D talking avatars using flow matching, given just text tokens as input.
%     \item Our fully-parallel diffusion-based architecture ensures multi-modal interaction through intermediate highway connections, generating synchronized speech and visual outputs.
%     % \item Using flow matching, we achieve fast inference and generate diverse expressive human motion.
%     \item In case of dyadic conversations, our model can be optionally conditioned on signals from a participant, fostering empathetic interactions.
% \end{itemize}

% }


% We propose a method

% - text-guided audio and video generation

% - language tokens are converted to corresponding speech and visual signals. By pairing an LLM with our method, we convert the simple text to a live photorealistic human talking and reacting to the user's expressions, simulating a human-to-human dyadic conversation.

% - we can generate both modalities. Our method can be guided by just text, logits, or audio (directly converted to speech features or logits by ASR model).

% - Interaction/communication between the 2 modalities ensures synchronous generation of expressive talking faces and corresponding speech signal

% - Our model can be further guided by reactions (facial expressions and head pose) of a participant in a dyadic setting. We consider our model in the behavior space of dyadic human-to-human conversations. Facial expressions, like surprise, and head motion, like nodding, play an important role in a conversation, how each party reacts and if they agree or disagree to the other's statements.

% - Diverse outputs trained with flow matching


%Synthesizing natural human motion has attracted growing interest in the field of computer vision. Photo-realistic talking faces have numerous applications in digital communication, enriching interaction and increasing accessibility (cite from vasa-1), as well as in education, movies, healthcare, etc. Prior work mostly focuses on plausible generation of talking faces, with precise lip synchronization, given a speech signal. But how about \textit{if we only have text characters as input?} Large language models (LLMs) have faced an increasing popularity recently, enabling users to ask various questions. However, their output is just text. Can we convert that text to human-like audio-visual outputs, simulating a human-to-human conversation and creating an empathetic interaction? In this work, we study this problem. We propose a generative model that can jointly synthesize expressive human-like speech and 4D visual outputs from just text characters.

%Research in generating talking faces has emerged long ago. Earlier approaches propose rule-based methods or hand-engineered features [CITE]. With the advent of deep learning, many approaches propose generative neural networks, conditioned on speech signal. One of the first notable works generates fake videos of Obama, with convincing lip synchronization. Generative adversarial networks (GANs) have repeatedly shown photo-realistic results in subsequent works. More recently, diffusion models~\cite{vasa1, deferred,difftalk,…} gain popularity, capturing the probabilistic distribution of human motion and producing natural-looking avatars. One of the very recent diffusion-based works is VASA-1~\cite{vasa1}, which achieves lifelike generation of audio-driven talking faces.

%However, most of these works require an input speech signal. They focus on precise lip synchronization. Some propose additional control of facial expressions and/or head pose, conditioning on learned embeddings~\cite{emotional} or other videos~\cite{pc-avs…}. Only a few works attempt to drive a talking face from just text, either directly~\cite{text-driven}, or through a cascaded approach of text-to-speech and speech-to-vision~\cite{}. In contrast, we propose a fully-parallel architecture that jointly generates audio and vision. In this way, we avoid any latency or error accumulation that cascaded approaches may introduce (CITE). In addition, we achieve natural synchronization of different modalities (e.g., intonation in speech and corresponding eyebrow motion) through intermediate multi-modal connections, compared to late fusion~\cite{}.


%Our proposed AV-Flow (Audio-Visual Flow Matching) consists of two inter-connected diffusion transformers. Given input tokens, one transformer generates the speech signal and the other transformer	generates the visual output. The visual transformer works in the latent space and produces the holistic head and facial dynamics, including lip motion, facial expressions, eyebrow and eyelid movements. Through intermediate highway connections, we ensure communication between the audio and visual modalities. In this way, AV-Flow synthesizes synchronized speech intonation with the corresponding facial dynamics (e.g., eyebrow motion). It is also versatile in terms of the input tokens: we show that we can drive a talking face given just text characters or predicted logits from a pre-trained speech-to-text model. Since these are speaker-independent tokens, it generalizes to any unseen audio or text input. Our approach is data-driven.  Trained with flow matching, it achieves fast inference and learns to synthesize human-like speech and 4D visual outputs for a specific subject, capturing natural nuances and expressive motion.

% We additionally consider the case of dyadic conversations. Humans usually interact with each other. In conversations, they take turns to talk and listen. Behaviors like nodding, smiling, and other facial reactions play an important role, guiding the whole interaction. Thus, it is crucial to develop \textit{conversational} avatars, compared to just talking faces. Using data from dyadic conversations between pairs of individuals, we demonstrate that our approach can be optionally conditioned on audio or visual signals from a participant and synthesize listening behavior and appropriate reactions. This fosters empathetic interactions with our audio-visual generative system. Through extensive experiments, we show that our method outperforms prior work, synthesizing natural-looking 4D talking avatars.

% Another important aspect is efficiency.
% We propose to train our diffusion-based architecture with flow matching. Flow matching … fast inference and diverse results.
% Using a window with only 200ms look-ahead, we achieve very low latency.
% AV-Flow can be easily deployed as a plug-and-play on top of an LLM, converting text tokens to human-like audio-visual outputs. We additionally evaluate the case of dyadic conversations, using conversational data. We demonstrate that our approach can be optionally  conditioned on audio or visual signals from a participant and synthesize listening behavior and appropriate reactions. This fosters empathetic interactions with our audio-visual generation. Through extensive experiments, we show that our method outperforms prior work, synthesizing natural-looking 4D talking avatars.

%In brief, our contributions are as follows:

% \begin{itemize}
%     \item We introduce AV-Flow, a novel approach for joint audio-visual generation of 4D talking avatars using flow matching, given just text tokens as input.
%     \item Our fully-parallel diffusion-based architecture ensures multi-modal interaction through intermediate highway connections, generating synchronized speech and visual outputs.
%     % \item Using flow matching, we achieve fast inference and generate diverse expressive human motion.
%     \item In case of dyadic conversations, our model can be optionally conditioned on signals from a participant, fostering empathetic interactions.
% \end{itemize}




% Please follow the steps outlined below when submitting your manuscript to the IEEE Computer Society Press.
% This style guide now has several important modifications (for example, you are no longer warned against the use of sticky tape to attach your artwork to the paper), so all authors should read this new version.

% %-------------------------------------------------------------------------
% \subsection{Language}

% All manuscripts must be in English.

% \subsection{Dual submission}

% Please refer to the author guidelines on the \confName\ \confYear\ web page for a
% discussion of the policy on dual submissions.

% \subsection{Paper length}
% Papers, excluding the references section, must be no longer than eight pages in length.
% The references section will not be included in the page count, and there is no limit on the length of the references section.
% For example, a paper of eight pages with two pages of references would have a total length of 10 pages.
% {\bf There will be no extra page charges for \confName\ \confYear.}

% Overlength papers will simply not be reviewed.
% This includes papers where the margins and formatting are deemed to have been significantly altered from those laid down by this style guide.
% Note that this \LaTeX\ guide already sets figure captions and references in a smaller font.
% The reason such papers will not be reviewed is that there is no provision for supervised revisions of manuscripts.
% The reviewing process cannot determine the suitability of the paper for presentation in eight pages if it is reviewed in eleven.

% %-------------------------------------------------------------------------
% \subsection{The ruler}
% The \LaTeX\ style defines a printed ruler which should be present in the version submitted for review.
% The ruler is provided in order that reviewers may comment on particular lines in the paper without circumlocution.
% If you are preparing a document using a non-\LaTeX\ document preparation system, please arrange for an equivalent ruler to appear on the final output pages.
% The presence or absence of the ruler should not change the appearance of any other content on the page.
% The camera-ready copy should not contain a ruler.
% (\LaTeX\ users may use options of \texttt{cvpr.sty} to switch between different versions.)

% Reviewers:
% note that the ruler measurements do not align well with lines in the paper --- this turns out to be very difficult to do well when the paper contains many figures and equations, and, when done, looks ugly.
% Just use fractional references (\eg, this line is $087.5$), although in most cases one would expect that the approximate location will be adequate.


% \subsection{Paper ID}
% Make sure that the Paper ID from the submission system is visible in the version submitted for review (replacing the ``*****'' you see in this document).
% If you are using the \LaTeX\ template, \textbf{make sure to update paper ID in the appropriate place in the tex file}.


% \subsection{Mathematics}

% Please number all of your sections and displayed equations as in these examples:
% \begin{equation}
%   E = m\cdot c^2
%   \label{eq:important}
% \end{equation}
% and
% \begin{equation}
%   v = a\cdot t.
%   \label{eq:also-important}
% \end{equation}
% It is important for readers to be able to refer to any particular equation.
% Just because you did not refer to it in the text does not mean some future reader might not need to refer to it.
% It is cumbersome to have to use circumlocutions like ``the equation second from the top of page 3 column 1''.
% (Note that the ruler will not be present in the final copy, so is not an alternative to equation numbers).
% All authors will benefit from reading Mermin's description of how to write mathematics:
% \url{http://www.pamitc.org/documents/mermin.pdf}.

% \subsection{Blind review}

% Many authors misunderstand the concept of anonymizing for blind review.
% Blind review does not mean that one must remove citations to one's own work---in fact it is often impossible to review a paper unless the previous citations are known and available.

% Blind review means that you do not use the words ``my'' or ``our'' when citing previous work.
% That is all.
% (But see below for tech reports.)

% Saying ``this builds on the work of Lucy Smith [1]'' does not say that you are Lucy Smith;
% it says that you are building on her work.
% If you are Smith and Jones, do not say ``as we show in [7]'', say ``as Smith and Jones show in [7]'' and at the end of the paper, include reference 7 as you would any other cited work.

% An example of a bad paper just asking to be rejected:
% \begin{quote}
% \begin{center}
%     An analysis of the frobnicatable foo filter.
% \end{center}

%    In this paper we present a performance analysis of our previous paper [1], and show it to be inferior to all previously known methods.
%    Why the previous paper was accepted without this analysis is beyond me.

%    [1] Removed for blind review
% \end{quote}


% An example of an acceptable paper:
% \begin{quote}
% \begin{center}
%      An analysis of the frobnicatable foo filter.
% \end{center}

%    In this paper we present a performance analysis of the  paper of Smith \etal [1], and show it to be inferior to all previously known methods.
%    Why the previous paper was accepted without this analysis is beyond me.

%    [1] Smith, L and Jones, C. ``The frobnicatable foo filter, a fundamental contribution to human knowledge''. Nature 381(12), 1-213.
% \end{quote}

% If you are making a submission to another conference at the same time, which covers similar or overlapping material, you may need to refer to that submission in order to explain the differences, just as you would if you had previously published related work.
% In such cases, include the anonymized parallel submission~\cite{Authors14} as supplemental material and cite it as
% \begin{quote}
% [1] Authors. ``The frobnicatable foo filter'', F\&G 2014 Submission ID 324, Supplied as supplemental material {\tt fg324.pdf}.
% \end{quote}

% Finally, you may feel you need to tell the reader that more details can be found elsewhere, and refer them to a technical report.
% For conference submissions, the paper must stand on its own, and not {\em require} the reviewer to go to a tech report for further details.
% Thus, you may say in the body of the paper ``further details may be found in~\cite{Authors14b}''.
% Then submit the tech report as supplemental material.
% Again, you may not assume the reviewers will read this material.

% Sometimes your paper is about a problem which you tested using a tool that is widely known to be restricted to a single institution.
% For example, let's say it's 1969, you have solved a key problem on the Apollo lander, and you believe that the 1970 audience would like to hear about your
% solution.
% The work is a development of your celebrated 1968 paper entitled ``Zero-g frobnication: How being the only people in the world with access to the Apollo lander source code makes us a wow at parties'', by Zeus \etal.

% You can handle this paper like any other.
% Do not write ``We show how to improve our previous work [Anonymous, 1968].
% This time we tested the algorithm on a lunar lander [name of lander removed for blind review]''.
% That would be silly, and would immediately identify the authors.
% Instead write the following:
% \begin{quotation}
% \noindent
%    We describe a system for zero-g frobnication.
%    This system is new because it handles the following cases:
%    A, B.  Previous systems [Zeus et al. 1968] did not  handle case B properly.
%    Ours handles it by including a foo term in the bar integral.

%    ...

%    The proposed system was integrated with the Apollo lunar lander, and went all the way to the moon, don't you know.
%    It displayed the following behaviours, which show how well we solved cases A and B: ...
% \end{quotation}
% As you can see, the above text follows standard scientific convention, reads better than the first version, and does not explicitly name you as the authors.
% A reviewer might think it likely that the new paper was written by Zeus \etal, but cannot make any decision based on that guess.
% He or she would have to be sure that no other authors could have been contracted to solve problem B.
% \medskip

% \noindent
% FAQ\medskip\\
% {\bf Q:} Are acknowledgements OK?\\
% {\bf A:} No.  Leave them for the final copy.\medskip\\
% {\bf Q:} How do I cite my results reported in open challenges?
% {\bf A:} To conform with the double-blind review policy, you can report results of other challenge participants together with your results in your paper.
% For your results, however, you should not identify yourself and should not mention your participation in the challenge.
% Instead present your results referring to the method proposed in your paper and draw conclusions based on the experimental comparison to other results.\medskip\\

% \begin{figure}[t]
%   \centering
%   \fbox{\rule{0pt}{2in} \rule{0.9\linewidth}{0pt}}
%    %\includegraphics[width=0.8\linewidth]{egfigure.eps}

%    \caption{Example of caption.
%    It is set in Roman so that mathematics (always set in Roman: $B \sin A = A \sin B$) may be included without an ugly clash.}
%    \label{fig:onecol}
% \end{figure}

% \subsection{Miscellaneous}

% \noindent
% Compare the following:\\
% \begin{tabular}{ll}
%  \verb'$conf_a$' &  $conf_a$ \\
%  \verb'$\mathit{conf}_a$' & $\mathit{conf}_a$
% \end{tabular}\\
% See The \TeX book, p165.

% The space after \eg, meaning ``for example'', should not be a sentence-ending space.
% So \eg is correct, {\em e.g.} is not.
% The provided \verb'\eg' macro takes care of this.

% When citing a multi-author paper, you may save space by using ``et alia'', shortened to ``\etal'' (not ``{\em et.\ al.}'' as ``{\em et}'' is a complete word).
% If you use the \verb'\etal' macro provided, then you need not worry about double periods when used at the end of a sentence as in Alpher \etal.
% However, use it only when there are three or more authors.
% Thus, the following is correct:
%    ``Frobnication has been trendy lately.
%    It was introduced by Alpher~\cite{Alpher02}, and subsequently developed by
%    Alpher and Fotheringham-Smythe~\cite{Alpher03}, and Alpher \etal~\cite{Alpher04}.''

% This is incorrect: ``... subsequently developed by Alpher \etal~\cite{Alpher03} ...'' because reference~\cite{Alpher03} has just two authors.

% \begin{figure*}
%   \centering
%   \begin{subfigure}{0.68\linewidth}
%     \fbox{\rule{0pt}{2in} \rule{.9\linewidth}{0pt}}
%     \caption{An example of a subfigure.}
%     \label{fig:short-a}
%   \end{subfigure}
%   \hfill
%   \begin{subfigure}{0.28\linewidth}
%     \fbox{\rule{0pt}{2in} \rule{.9\linewidth}{0pt}}
%     \caption{Another example of a subfigure.}
%     \label{fig:short-b}
%   \end{subfigure}
%   \caption{Example of a short caption, which should be centered.}
%   \label{fig:short}
% \end{figure*}
