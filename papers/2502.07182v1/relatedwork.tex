\section{Literature Review}
\label{sec:related_work}

\revision{The development of morphing wing technology represents a century-long evolution in aerospace engineering, from primitive mechanical solutions to sophisticated adaptive systems. This review traces the progression of wing morphing concepts, examining how each advancement has contributed to our current understanding and highlighting the technological challenges that remain to be addressed.}


\begin{figure}
    \centering
    \includegraphics[width=\linewidth]{wrights_morphing_wing_design.png}
    \caption{Wright’s brothers primitive morphing wings ~\cite{choi2017idea}}
    \label{fig:wrights_morphing_wing_design}
\end{figure}

\subsection{Early Development of Wing Morphing}
While the idea of a morphing wing design may seem futuristic, the method of controlling an aircraft's flight movements by changing the shape of the wings has a surprisingly long history. This pursuit can be traced back to the very beginning of aircraft, with the Wright Brothers playing a key role in the development of this concept \cite{choi2017idea, nasm2022researching}. Fig.~\ref{fig:wrights_morphing_wing_design} shows an early example of the implementation of this concept. 

\revision{The Wright brothers' pioneering work emerged from} a fundamental challenge of how to control aircraft during their early testing on kites and small gliders \cite{nasm2022researching}. Their experiments revealed that air currents were too strong for human reflection, and the aircraft must be inherently stable and controllable by humans \cite{nasm2022researching}. \revision{This led to the development of} wing-warping, a method they first tested on their 1899 kite (see Fig.~\ref{fig:wrights_morphing_wing_sketch}). 



\begin{figure}[h!]
    \centering
    \includegraphics[width=0.7\linewidth]{wrights_morphing_wing_sketch.png}
    \caption{The Wright brother’s warping wing sketch~\cite{biolocomotion2011wright}}
    \label{fig:wrights_morphing_wing_sketch}
\end{figure}


\subsection{Technical Evolution of Wing Control}
\revision{The wing-warping mechanism fundamentally relies on} spatial modeling of the aircraft structure. The design enables controlled movement of each wing's trailing edge through a system of wires connecting the wingtips to the pilot's footrest. When the pilot operates the pedals, the wires adjust the wing tip position, altering the outer panel's shape. This change in the airfoil's deflection angle creates differential lift forces between the wings, causing the aircraft to rotate about its center of gravity \cite{nasa2023warping}. This innovative design concept, implemented in the Flyer, achieved the first powered flight in 1903.

\subsection{Emergence of Aileron Technology}
The development of control surfaces for roll management, particularly the aileron, evolved through various iterations in the late 19th and early 20th centuries \cite{crouch2008oldies}. \revision{The progression began with} Matthew Piers Watt Boulton's patent in the 1860s, followed by contributions from Charles Renard and Alphonse Pénaud \cite{crouch2008oldies}. Jean-Marie LeBris's experiments with wing warping further advanced the field, ultimately leading to the Wright brothers' breakthrough design.

\revision{A significant development came from} French experimenter Robert Esnault-Pelterie, who introduced separate ailerons positioned between the wings, prioritizing simplicity and reducing structural strain. The term "aileron," meaning "little wing" in French, wasn't adopted until 1908 \cite{crouch2008oldies}. \revision{This innovation represented a crucial shift from wing-warping to discrete control surfaces, offering improved reliability and maintainability.}

\subsection{Modern Variable Geometry Wings}
\revision{The F-14 Tomcat represents a significant milestone in variable geometry wing design}. Its wings could pivot between 20$^{\circ}$ during takeoffs and landings to a maximum of 68$^{\circ}$ sweep angle for reduced drag at supersonic speeds \cite{airforce2020f14}. This design provided exceptional versatility, enabling high-speed performance and improved maneuverability across various flight conditions.

The wing transformation system utilized the Standard Central Air Data Computer (SCAD), which calculated appropriate sweep angles based on altitude and Mach number. \revision{The system's sophisticated control mechanism} achieved precise wing adjustments through a hydro-mechanical system operating at eight degrees per second \cite{airforce2020f14}. \revision{While this design demonstrated remarkable capabilities, it also highlighted the challenges of complex morphing systems, including increased maintenance requirements and reduced fuel storage capacity.}


\subsection{Recent Developments}
A significant advancement occurred on November 15th, 2002, when NASA tested their adaptation of wing-warping control on an F-18, designated as the X-53. This Active Aeroelastic Wing program focused on developing flexible wing warping for roll control \cite{nasa2023aeroelastic, dibley2005development}. The Air Force confirmed the program's success, noting its potential applications in both military and high-altitude long-endurance aircraft \cite{barr2005wing}.

\revision{Contemporary research has produced innovations such as} the Adaptive Aspect Ratio morphing wing (AdAR) at Swansea University \cite{woods2015adaptive}. \revision{This design represents a modern approach to shape-shifting wing technology, combining advanced materials and control systems to enhance aircraft performance and efficiency. The AdAR concept demonstrates the continuing evolution of morphing wing technology, suggesting promising directions for future aerospace applications.}

\revision{Despite these significant advances in morphing wing technology, several critical challenges persist in implementing these systems in commercial aviation. While military applications have demonstrated the potential benefits of variable geometry wings, the commercial sector faces unique constraints related to maintenance costs, reliability requirements, and safety certifications. Furthermore, the increasing demand for fuel efficiency and operational flexibility in commercial aviation creates an urgent need for innovative solutions. This study addresses these challenges by examining current morphing wing technologies and their potential applications in commercial aircraft, with particular attention to practical implementation considerations and economic viability. Understanding these aspects is crucial for advancing the field and developing solutions that can meet the demanding requirements of commercial aviation operations.}