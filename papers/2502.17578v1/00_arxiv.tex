%%%%%%%% ICML 2025 EXAMPLE LATEX SUBMISSION FILE %%%%%%%%%%%%%%%%%

\documentclass{article}

% Recommended, but optional, packages for figures and better typesetting:
\usepackage{microtype}
\usepackage{graphicx}
\usepackage{subfigure}
\usepackage{booktabs} % for professional tables
\usepackage[T1]{fontenc}

% hyperref makes hyperlinks in the resulting PDF.
% If your build breaks (sometimes temporarily if a hyperlink spans a page)
% please comment out the following usepackage line and replace
% \usepackage{icml2025} with \usepackage[nohyperref]{icml2025} above.
\usepackage{hyperref}


% Attempt to make hyperref and algorithmic work together better:
\newcommand{\theHalgorithm}{\arabic{algorithm}}

% Use the following line for the initial blind version submitted for review:

% If accepted, instead use the following line for the camera-ready submission:
\usepackage[accepted]{arxiv}

% For theorems and such
\usepackage{amsmath}
\usepackage{amssymb}
\usepackage{mathtools}
\usepackage{amsthm}
\usepackage{subcaption}
\usepackage{leftindex}


% if you use cleveref..
\usepackage[capitalize,noabbrev]{cleveref}


% Todonotes is useful during development; simply uncomment the next line
%    and comment out the line below the next line to turn off comments
%\usepackage[disable,textsize=tiny]{todonotes}
\usepackage[textsize=tiny]{todonotes}
\usepackage{listings}

%%%%% NEW MATH DEFINITIONS %%%%%

% \usepackage{amsmath,amsfonts,bm}
\usepackage{amsmath,amsfonts}

\usepackage{pifont}


\newcommand{\R}{\mathbb{R}}


\def\va{{\mathbf{a}}}
\def\vg{{\mathbf{g}}}

% Sets
\def\sR{\mathbb{R}}
\def\sC{\mathbb{C}}
\def\sZ{\mathbb{Z}}
\def\sN{\mathbb{N}}
\def\sQ{\mathbb{Q}}

\def\sS{\mathcal{S}}



% Vectors
\def\vzero{{\mathbf{0}}}
\def\vone{{\mathbf{1}}}
\def\vmu{{\mathbf{\mu}}}
\def\vtheta{{\mathbf{\theta}}}
\def\va{{\mathbf{a}}}
\def\vb{{\mathbf{b}}}
\def\vc{{\mathbf{c}}}
\def\vd{{\mathbf{d}}}
\def\ve{{\mathbf{e}}}
\def\vf{{\mathbf{f}}}
\def\vg{{\mathbf{g}}}
\def\vh{{\mathbf{h}}}
\def\vi{{\mathbf{i}}}
\def\vj{{\mathbf{j}}}
\def\vk{{\mathbf{k}}}
\def\vl{{\mathbf{l}}}
\def\vm{{\mathbf{m}}}
\def\vn{{\mathbf{n}}}
\def\vo{{\mathbf{o}}}
\def\vp{{\mathbf{p}}}
\def\vq{{\mathbf{q}}}
\def\vr{{\mathbf{r}}}
\def\vs{{\mathbf{s}}}
\def\vt{{\mathbf{t}}}
\def\vu{{\mathbf{u}}}
\def\vv{{\mathbf{v}}}
\def\vw{{\mathbf{w}}}
\def\vx{{\mathbf{x}}}
\def\vy{{\mathbf{y}}}
\def\vz{{\mathbf{z}}}
\def\vzeta{{\mathbf{\zeta}}}

% Matrix
\def\mA{{\mathbf{A}}}
\def\mB{{\mathbf{B}}}
\def\mC{{\mathbf{C}}}
\def\mD{{\mathbf{D}}}
\def\mE{{\mathbf{E}}}
\def\mF{{\mathbf{F}}}
\def\mG{{\mathbf{G}}}
\def\mH{{\mathbf{H}}}
\def\mI{{\mathbf{I}}}
\def\mJ{{\mathbf{J}}}
\def\mK{{\mathbf{K}}}
\def\mL{{\mathbf{L}}}
\def\mM{{\mathbf{M}}}
\def\mN{{\mathbf{N}}}
\def\mO{{\mathbf{O}}}
\def\mP{{\mathbf{P}}}
\def\mQ{{\mathbf{Q}}}
\def\mR{{\mathbf{R}}}
\def\mS{{\mathbf{S}}}
\def\mT{{\mathbf{T}}}
\def\mU{{\mathbf{U}}}
\def\mV{{\mathbf{V}}}
\def\mW{{\mathbf{W}}}
\def\mX{{\mathbf{X}}}
\def\mY{{\mathbf{Y}}}
\def\mZ{{\mathbf{Z}}}
\def\mBeta{{\mathbf{\beta}}}
\def\mPhi{{\mathbf{\Phi}}}
\def\mLambda{{\mathbf{\Lambda}}}
\def\mSigma{{\mathbf{\Sigma}}}


% Expectation
% \def\eE{\mathop{\mathbb{E}}\limits}
\def\eE{\mathbb{E}}

% Probability
\def\pP{\mathbb{P}}

% Tilde
\def\tf{\tilde{f}}
\def\tS{\tilde{S}}
\def\wtF{\widetilde{\mathcal{F}}}
\def\whR{\widehat{R}}
\def\tvx{\tilde{\mathbf{x}}}
\def\ty{\tilde{y}}


\def\defeq{\overset{\textup{def}}{=}}
% \def\defeq{\overset{.}{=}}
\def\defone{\overset{\text{\ding{172}}}{=}}
\def\deftwo{\overset{\text{\ding{173}}}{=}}
\def\leqone{\overset{\text{\ding{172}}}{\leq}}
\def\leqtwo{\overset{\text{\ding{173}}}{\leq}}
\def\leqthree{\overset{\text{\ding{174}}}{\leq}}
\def\leqfour{\overset{\text{\ding{175}}}{\leq}}
\def\eqone{\overset{\text{\ding{172}}}{=}}
\def\eqtwo{\overset{\text{\ding{173}}}{=}}
\def\eqthree{\overset{\text{\ding{174}}}{=}}
\def\eqfour{\overset{\text{\ding{175}}}{=}}
\def\geqfive{\overset{\text{\ding{176}}}{\geq}}

\newcommand{\rylan}[1]{\textcolor{red}{RS: #1}}
\newcommand{\aengus}[1]{\textcolor{cyan}{Aengus: #1}}
\newcommand{\josh}[1]{\textcolor{blue}{JK: #1}}

% The \icmltitle you define below is probably too long as a header.
% Therefore, a short form for the running title is supplied here:
\icmltitlerunning{How Do Large Language Monkeys Get Their Power (Laws)?}

\begin{document}

\twocolumn[
\icmltitle{
How Do Large Language Monkeys Get Their Power (Laws)?
% Title 2: The Origin of Power Law Scaling by Repeatedly Sampling from Neural Language Models\\
% Title 3: When and Why Repeatedly Sampling from\\Neural Language Models Yields Power Law Scaling
%The Origin of Power Laws in Scaling Inference Compute via Repeat Sampling
}

% It is OKAY to include author information, even for blind
% submissions: the style file will automatically remove it for you
% unless you've provided the [accepted] option to the icml2025
% package.

% List of affiliations: The first argument should be a (short)
% identifier you will use later to specify author affiliations
% Academic affiliations should list Department, University, City, Region, Country
% Industry affiliations should list Company, City, Region, Country

% You can specify symbols, otherwise they are numbered in order.
% Ideally, you should not use this facility. Affiliations will be numbered
% in order of appearance and this is the preferred way.
\icmlsetsymbol{equal}{*}

\begin{icmlauthorlist}
\icmlauthor{Rylan Schaeffer}{stanfordcs}
\icmlauthor{Joshua Kazdan}{stanfordstats}
\icmlauthor{John Hughes}{speechmatics,mats}
\icmlauthor{Jordan Juravsky}{stanfordcs}
\icmlauthor{Sara Price}{mats}
\icmlauthor{Aengus Lynch}{mats,ucl}
\icmlauthor{Erik Jones}{anthropic}
\icmlauthor{Robert Kirk}{ucl}
\icmlauthor{Azalia Mirhoseini}{stanfordcs}
\icmlauthor{Sanmi Koyejo}{stanfordcs}
\end{icmlauthorlist}

\icmlaffiliation{stanfordcs}{Stanford Computer Science}
\icmlaffiliation{stanfordstats}{Stanford Statistics}
\icmlaffiliation{speechmatics}{Speechmatics}
\icmlaffiliation{ucl}{University College London}
\icmlaffiliation{anthropic}{Anthropic}
% \icmlaffiliation{comp}{Company Name, Location, Country}
% \icmlaffiliation{sch}{School of ZZZ, Institute of WWW, Location, Country}

\icmlcorrespondingauthor{Rylan Schaeffer}{rschaef@cs.stanford.edu}
\icmlcorrespondingauthor{Sanmi Koyejo}{sanmi@cs.stanford.edu}

% You may provide any keywords that you
% find helpful for describing your paper; these are used to populate
% the "keywords" metadata in the PDF but will not be shown in the document
\icmlkeywords{Machine Learning, ICML}

\vskip 0.3in
]


% this must go after the closing bracket ] following \twocolumn[ ...

% This command actually creates the footnote in the first column
% listing the affiliations and the copyright notice.
% The command takes one argument, which is text to display at the start of the footnote.
% The \icmlEqualContribution command is standard text for equal contribution.
% Remove it (just {}) if you do not need this facility.

%\printAffiliationsAndNotice{}  % leave blank if no need to mention equal contribution
% \printAffiliationsAndNotice{\icmlEqualContribution} % otherwise use the standard text.


% Remaining TODOs:
% 4. How does sample efficiency compare?

\begin{abstract}
Recent research across mathematical problem solving, proof assistant programming and multimodal jailbreaking documents a striking finding: when (multimodal) language model tackle a suite of tasks with multiple attempts per task -- succeeding if any attempt is correct -- then the negative log of the average success rate scales a power law in the number of attempts.
In this work, we identify an apparent puzzle: a simple mathematical calculation predicts that on each problem, the failure rate should fall exponentially with the number of attempts.
We confirm this prediction empirically, raising a question: from where does aggregate polynomial scaling emerge?
We then answer this question by demonstrating per-problem exponential scaling can be made consistent with aggregate polynomial scaling if the distribution of single-attempt success probabilities is heavy tailed such that a small fraction of tasks with extremely low success probabilities collectively warp the aggregate success trend into a power law - even as each problem scales exponentially on its own.
We further demonstrate that this distributional perspective explains previously observed deviations from power law scaling, and provides a simple method for forecasting the power law exponent with an order of magnitude lower relative error, or equivalently, ${\sim}2-4$ orders of magnitude less inference compute.
Overall, our work contributes to a better understanding of how neural language model performance improves with scaling inference compute and the development of scaling-predictable evaluations of (multimodal) language models.
\end{abstract}


\begin{figure}[ht]
    \centering
    \includegraphics[width=0.8\linewidth]{graphs/greater_than_naive.pdf}
    \vspace{0.5cm}
    \includegraphics[width=0.8\linewidth]{graphs/p1_bottom.png}
    \vspace{-5pt}
    \caption{\textcolor{positional}{Positional} vs.\ \textcolor{nonpositional}{non-positional} circuits. In a \textcolor{nonpositional}{non-positional} circuit, the same edges must be included at all positions. A \textcolor{positional}{positional} circuit can distinguish between the same edge at different positions. This specificity yields better trade-offs between circuit size and faithfulness. It can also increase both precision and recall.}
    \label{fig:p1}
    \vspace{-5pt}
\end{figure}

\section{Introduction}

\looseness=-1
A primary goal of interpretability research is to characterize the internal mechanisms in language models (LMs) and other NLP models. 
A core approach in this area is \textbf{circuit discovery}---identifying the minimal subgraph within the model's computation graph that performs a specific task \citep{olah2021framework,olah-mech}.
Typically, the nodes of a circuit represent model components (e.g., attention heads, neurons, or layers).
While manual circuit discovery methods can yield position-specific insights \citep{wanginterpretability,goldowskydill2023localizingmodelbehaviorpath}, \emph{automatic methods often overlook positional information}, treating components as uniformly relevant across all input token positions \citep{conmytowards,syed2023attribution}. 
For instance, if an attention head is included in a circuit, it is assumed to contribute equally to the computation for every position in the input sequence.
The assumption that circuits are position-invariant ignores the fact that different positions often require distinct computations.
By ignoring positions, current methods limit their ability to capture mechanisms that operate across positions, such as interactions between attention heads across positions.

In this study, we start by demonstrating that positional agnosticism is a significant limitation (\S\ref{sec:motivating}). Then, to address these limitations, we introduce a new approach: position-aware edge attribution patching (PEAP; \S\ref{sec:full_circ_discovery}; Figure~\ref{fig:p1}). Current approaches  assume that if an edge is in a circuit, then the same edge will be in the circuit at all positions, thus leading to low precision. It is also assumed that an edge's importance should be aggregated across positions before deciding whether it should be included in the circuit; this can lead to cancellation effects, and thus low recall. PEAP instead allows us to compute the importance of cross-positional edges, and separately evaluates edge importance at each position. We show that this leads to smaller and more accurate circuits; see Figure~\ref{fig:p1}.

Incorporating positional information into circuit discovery is straightforward when inputs have the same length and structure across examples.

However, realistic datasets are not nearly this templatic.
How, then, can we incorporate positional information into automatic circuit discovery?
To address this challenge, we propose \textbf{schemas} (\S\ref{sec:schema}). 
Schemas assign semantic labels to spans of tokens, enabling information aggregation across examples even when the spans differ in length.

For example, in the input ``The \textcolor{positional}{war} lasted from 1453 to 14\underline{\hspace{1em}},'' the span ``\textcolor{positional}{war}'' could be labeled as ``\emph{Subject}''.
This enables handling spans with varying lengths: the phrase ``\textcolor{positional}{Black Plague}'' in another example can be treated as a single positional span with the same role as ``\textcolor{positional}{war}''.
In experiments with two LMs and three tasks, we find that circuits discovered using schemas achieve a better trade-off between circuit size and faithfulness to the model's behavior than position-agnostic circuits.
Importantly, position-aware circuits offer a more precise representation of the underlying mechanisms, providing a more concise foundation for mechanistic explanations.

We also present a fully automated pipeline for schema generation and application (\S\ref{sec:schema-generation}) using large language models (LLMs). 
We evaluate the quality of the generated schemas and their utility in discovering position-aware circuits (\S\ref{sec:schema-eval}).
Notably, circuits derived using automatically generated and applied schemas achieve comparable faithfulness scores to circuits discovered with human-designed and manually applied schemas.

We summarize our contributions as follows:
\begin{itemize}[noitemsep,leftmargin=*,topsep=1pt,parsep=1pt]
    \item Introduce a position-aware circuit discovery method, which obtains better faithfulness than position-agnostic discovery.  
    \item Introduce dataset schemas,  facilitating positional circuit discovery in more naturalistic settings. 
    \item Develop an automated schema generation and application pipeline with LLMs, yielding schemas that are comparable to manually-annotated ones.
\end{itemize}



\section{Should Power Law Scaling Be Expected?}
\label{sec:should_power_laws_be_expected}

Should we expect large language monkeys to have such power  (laws)? That is, should the negative log of the average success rate scale polynomially with the number of independent attempts $k$? As we now explain mathematically and demonstrate empirically, such polynomial scaling with $k$ is perhaps surprising because, for any single problem, the negative log success rate at $k$ should fall exponentially with $k$; the intuition is that $\operatorname{pass_i@k}$ is 1 unless \textit{all} attempts fail, and since attempts are independent, the probability that all fail is exponentially unlikely with the number of attempts.


\begin{figure*}[t!]
    \centering
    \includegraphics[width=\linewidth]{figures/02_large_language_monkeys_original_eda/y=neg_log_score_vs_x=scaling_parameter_hue=model_col=model_units=problem_idx.pdf}
    \includegraphics[width=\linewidth]{figures/00_bon_jailbreaking_eda/y=neg_log_score_vs_x=scaling_parameter_hue=model_col=model_units=problem_idx.pdf}
    \caption{\textbf{Per-problem performance scales exponentially with the number of attempts per problem $k$}.     
    Top: Pythia language models on 128 problems from MATH, with performance on the $i$-th problem measured as $-\log(\operatorname{pass_i@k})$. Bottom: Frontier AI models on jailbreaking prompts from HarmBench, with performance on the $i$-th problem measured as $-\log(\operatorname{ASR_i@k})$. In both settings, on each problem, the negative log \textit{per-problem} success rate falls exponentially with the number of independent attempts $k$. However, the negative log \textit{average} success rate falls as a power law with $k$ (black).}
    \label{fig:multiple_attempts_scaling_per_datum}
\end{figure*}


Mathematically, on any given attempt, the model has probability $\operatorname{pass_i@1}$ of solving the $i$-th problem.
% If we draw $k$ independent attempts from the model, how should the expected (over attempts) $\operatorname{pass_i@k}$ to scale with $k$?
Recalling that $\operatorname{pass_i@k}$ is defined as $1$ if \textit{any} of the $k$ attempts succeed, 0 otherwise, by linearity of expectation and by independence of the $k$ attempts, we can rewrite $\operatorname{pass_i@k}$ as:
%
\begin{align}
    \operatorname{pass_i@k} &= \mathop{\raisebox{3pt}{$\mathbb{E}$}}_{\substack{k \text{ Attempts}}}\Big[1 - \mathbb{I}[\text{All $k$ Attempts Fail}] \Big]\\
    &= 1 - \prod_{j=1}^k \mathop{\raisebox{3pt}{$\mathbb{E}$}}_{\substack{1 \text{ Attempt}}}\Big[ \mathbb{I}[\text{$j$-th Attempt Fails}] \Big].
\end{align}

The probability that the $j$-th attempt fails is one minus the probability that the $j$-th attempt succeeds. Since each attempt is i.i.d. with success probability $\operatorname{pass_i@1}$, we find
%
\begin{align}
    \operatorname{pass_i@k}
    &= 1 - (1 - \operatorname{pass_i@1})^k.
\end{align}

For large $k$, $(1 - \operatorname{pass_i@1})^k$ will be small. Recalling that the Taylor Series expansion of $\log (1 + x)$ for small $x$ is $\sum_{i=1}^{\infty} (-1)^{i-1} x^i / i \approx x$, we have:
%
\begin{align}
    -\log (\operatorname{pass_i@k} )
    &= - \log \Big(1 - (1 - \operatorname{pass@1})^k \Big)\\
    &\approx (1 - \operatorname{pass_i@1})^k.
\end{align}


\begin{figure*}[t!]
    \centering
    \includegraphics[width=\linewidth]{figures/50_pass_at_1_fits/llmonkey_y=counts_x=score_hue=model_col=model_bins=custom.pdf}
    \includegraphics[width=\linewidth]{figures/50_pass_at_1_fits/bon_jailbreaking_y=counts_x=score_hue=model_col=model_bins=custom.pdf}
    \caption{\textbf{Single-Attempt Success Rates  Distributions Possess Power Law-Like Left Tails.} Pythia language models on 128 MATH problems (top) and frontier AI systems on 159 HarmBench prompts (bottom) exhibit distributions (over problems) of $\operatorname{pass_i@1}$ and $\operatorname{ASR_i@1}$ with power law-like tails that are well fit by scaled Beta-Binomial distributions (black dashed lines), which produce aggregate power law scaling. Note that Llama 3 8B Instruction Tuned (IT) does not possess a power law tail, explaining why the model did not exhibit aggregate power law scaling under Best-of-N jailbreaking (Sec.~\ref{sec:no_dist_structure_no_power_law}).}
    \label{fig:multiple_attempts_pass_at_1_per_datum}
\end{figure*}


Thus, \textit{for any single problem}, we should expect the negative log expected (over attempts) success rate to fall \textit{exponentially} with $k$, not polynomially with $k$. % (i.e., as a power law).
% Intuitively, this is because $\operatorname{pass@k}$ is 1 unless all $k$ i.i.d. attempts fail and the probability that all $k$ attempts fail is exponentially unlikely with $k$.


To confirm this claim, we plotted the scaling of model performance on each problem -- measured either by $-\log(\operatorname{pass_i@k})$ or by $-\log(\operatorname{ASR_i@k})$ -- against the number of independent attempts $k$. We specifically used \citet{brown2024largelanguagemonkeysscaling}'s data of the Pythia language model family \citep{biderman2023pythia} solving 128 mathematical problems from MATH \citet{hendrycks2021measuring} as well as \citet{hughes2024bestofnjailbreaking}'s data from jailbreaking frontier AI systems -- Claude, GPT4 \citep{openai2024gpt4technicalreport}, Gemini \citep{anil2024geminifamilyhighlycapable,georgievgemini15unlockingmultimodal} and Llama 3 8B Instruction Tuned (IT) \citep{grattafiori2024llama3herdmodels} -- on 159 prompts from HarmBench \citep{mazeika2024harmbenchstandardizedevaluationframework}.
For each individual mathematical problem and jailbreaking prompt, we found the negative log expected (over attempts) success rates fall exponentially with $k$ as expected (Fig. \ref{fig:multiple_attempts_scaling_per_datum}), including on Llama 3 8B IT which does not exhibit an aggregate power law (Fig.~\ref{fig:power_laws_repeat_sampling}).

\section{Distribution of Per-Problem Single-Attempt Success Rates Creates Power Law Scaling}
\label{sec:distr_per_problem_success_rates}

How does polynomial scaling of the negative log \textit{average} success rate emerge from exponential scaling of the negative log \textit{per-problem} success rate?
The answer to this question \textit{must} lie in the distribution $\mathcal{D}$ over benchmark problems of single attempt (i.e., $k=1$) success rates because this distribution's density $p_{\mathcal{D}}(\operatorname{pass_i@1})$ links the per-problem scaling behavior to the aggregate scaling behavior via the definition of the aggregate success rate $\operatorname{pass_{\mathcal{D}}@k}$:
%
\begin{equation}
\begin{aligned}
    &\operatorname{pass_{\mathcal{D}}@k} \; \defeq \; \mathop{\raisebox{3pt}{$\mathbb{E}$}}_{\operatorname{pass_i@1} \sim \mathcal{D}} \Big[\operatorname{pass_i@k}(\operatorname{pass_i@1}) \Big]\\
    % &= 1 - \underbrace{\int_0^1 (1 - \operatorname{pass_i@1})^k \, p_{\mathcal{D}}(\operatorname{pass_i@1}) \, \operatorname{d\,pass_i@1}}_{\defeq \operatorname{fail_{\mathcal{D}}@k}}.
    &= 1 - \int_0^1 (1 - \operatorname{pass_i@1})^k \, p_{\mathcal{D}}(\operatorname{pass_i@1}) \, \operatorname{d\,pass_i@1}.
\end{aligned}
\end{equation}
% %
% For large $k$, $\operatorname{fail_{\mathcal{D}}@k}$ will be small and thus:
% \begin{equation}
%     -\log \Big( \operatorname{pass_{\mathcal{D}}@k} \big) \approx 
% \end{equation}


Based on a known result that power laws can originate from an appropriately weighted sum of exponential functions (Appendix ~\ref{app:sec:power_laws_from_distr_over_exp:background}), we begin by considering simple distributions for the single-attempt success probabilities and asking which yield power law scaling between $-\log(\operatorname{pass_{\mathcal{D}}@k})$ and $k$, as well as what properties of the distributions set the scaling exponent. In Appendices~\ref{app:sec:power_laws_from_distr_over_exp:uniform_distribution}-\ref{app:sec:power_laws_from_distr_over_exp:reciprocal_distribution}, we derive that several simple distributions yield power law scaling with different exponents whereas others do not:
%
\begin{align*}
    -\log \Big( &\operatorname{pass_{\mathrm{Uniform}(0,\, \beta \leq 1)}}@k &\Big) &\propto k^{-1}.\\
    -\log \Big( &\operatorname{pass_{\operatorname{Beta(\alpha, \beta)}}@k} &\Big) &\propto k^{-\alpha}.\\
    -\log \Big( &\operatorname{pass_{\operatorname{Kumaraswamy(\alpha,\, \beta)}}@k} &\Big) &\propto k^{-\alpha}.\\
    -\log \Big( &\operatorname{pass_{\operatorname{ContinuousBernoulli(\lambda < 1/2)}}@k} &\Big) &\propto k^{-1}.\\
    -\log \Big( &\operatorname{pass_{\operatorname{Reciprocal(0 < \alpha < \beta < 1)}}@k} &\Big) \propto &\frac{(1-\alpha)^k}{k}.
\end{align*}
%
To test this understanding, we examined whether the data of \citet{brown2024largelanguagemonkeysscaling} and \citet{hughes2024bestofnjailbreaking} had per-problem single-attempt success rate distributions that matched one of these simple distributions (Fig.~\ref{fig:multiple_attempts_pass_at_1_per_datum}). We found that the distributions could indeed be well fit by a 3-parameter $\operatorname{Kuamraswamy}(\alpha, \beta, a = 0, c)$ distribution with scale parameter $c$ (Fig.~\ref{fig:multiple_attempts_pass_at_1_per_datum}, black dashed lines); we found the scale parameter was critical to obtain good fits because the standard 2-parameter Kumaraswamy distribution is supported on $(0, 1)$ whereas most single-attempt success distributions have a smaller maximum such as $0.01$ or $0.1$.

More generally, what are the distributional properties that create such power law scaling and that set the specific power law exponent?
As we now show, the negative log average success rate will exhibit power law scaling in $k$ with exponent $b$ if and only if the distribution over problems of single-attempt success probabilities itself behaves like a power law near $0$ with exponent $b-1$:\newline

% If this condition is met, then then $\mathbb{E}_{\operatorname{pass_i@1} \sim \mathcal{D}}[(1 - \operatorname{pass_i@1})^k]$ decays as $k^{-b}$ and $-\log (\operatorname{pass_{\mathcal{D}}@k})$ inherits the same power-law exponent $b$.
% \josh{@Rylan can I swap these theorems out with the corrected ones?  The necessity claim is false, and I don't think we've identified a reasonable necessary condition yet.  We should also change the sufficiency one so that the distributions that you named follow as corollaries, otherwise the theorem is vacuous in the context of the paper.}
% \josh{I like the suspense as you build to the main theorem, but I wonder if it's more sensible to state the general claim first and then discuss the distributions that satisfy the sufficient condition.}
\begin{theorem}[Sufficiency of Power-Law Left Tail in Distribution of Single-Attempt Success Rates]
\label{thm:sufficiency_powerlaw}
Let $\mathcal{D}$ be a probability distribution on $[0,1]$ with PDF $p_{\mathcal{D}}(\operatorname{pass_i@1})$.  
Suppose there exist constants $b > 0$, $C > 0$, $\theta > 0$ and $\delta > 0$ such that, 
for all $0 < \operatorname{pass_i@1} < \delta$, we have 
\[
  p_{\mathcal{D}}(\operatorname{pass_i@1}) \;=\; C \cdot (\operatorname{pass_i@1})^{b-1} \;+\; O\bigl((\operatorname{pass_i@1})^{b-1+\theta}\bigr).
\]
% That is, there is a constant $M>0$ such that
% \[
%   \bigl|\,f(p) - C\,p^{\,b-1}\bigr|
%   \;\le\;
%   M\,p^{\,b-1+\theta}
%   \quad\text{for all }0<p<\delta.
% \]
% Define the aggregate success rate after $k$ attempts by
% \[
%   \operatorname{pass_{\mathcal{D}}@k}
%   \;\;=\;\;
%   \int_0^1 \Bigl[\,1 - (1 - p)^k\Bigr]\,f(p)\,\mathrm{d}p.
% \]
Then, for large $k$,
\[
  -\log\big(\operatorname{pass_{\mathcal{D}}@k}\big)
  \;\sim\;
  C\,\Gamma(b) \;k^{-b}.
\]
\end{theorem}



\begin{theorem}[Necessity of Power-Law Left Tail in Distribution of Single-Attempt Success Rates]
\label{thm:necessity_powerlaw}
Let $\mathcal{D}$ be a distribution over $\operatorname{pass_i@1} \in [0,1]$ with PDF $p_{\mathcal{D}}(\operatorname{pass_i@1})$.
Suppose there exist constants $b > 0$ and $A > 0$ such that for large $k$,
\[
-\log\big(\operatorname{pass_{\mathcal{D}}@k}\big)
\sim 
A\,k^{-b}.
\]
Then, under mild regularity assumptions, the probability density must satisfy
\[
p_{\mathcal{D}}(\operatorname{pass_i@1})
\;\sim\;
\frac{A}{\Gamma(b)} \, (\operatorname{pass_i@1})^{b - 1}
\quad
\text{as } \operatorname{pass_i@1} \to 0^+.
\]
% where $C>0$ is a constant depending on $A$ and $b$.  
% In other words, whenever $-\log\bigl(\operatorname{pass_{\mathcal{D}}@k}\bigr)$ decays like a power law $k^{-b}$, the distribution $\mathcal{D}$ must have a PDF that behaves like $p^{\,b-1}$ near $p=0$ (up to a multiplicative constant).
\end{theorem}

In Fig.~\ref{fig:schematic}, we illustrate this connection schematically.
For proofs, see Appendices \ref{app:sec:power_laws_from_distr_over_exp:sufficiency} and \ref{app:sec:power_laws_from_distr_over_exp:necessity}.
These results clarify that whenever $-\log (\operatorname{pass_{\mathcal{D}}@k} )$ exhibits power-law decay in $k$ with exponent $b$, the distribution over problems of single-attempt success rates \emph{must} have ``polynomial weight'' near $\operatorname{pass_i@1}=0$, i.e.\ $p_{\mathcal{D}}(p) = \Theta(p^{\,b-1})$.

To offer intuition, we know that each problem is being solved by the model (or equivalently, each prompt is jailbreaking the model) exponentially quickly.
If one looks across all problems in the benchmark, some have $\operatorname{pass_i@1}$ so small that they remain unsolved for many, many attempts.
Whether these ``tiny‐$\operatorname{pass_i@1}$" problems still matter at large $k$ depends on how \emph{many} such problems there are.
Polynomial density near $0$ ``piles up" enough hard problems in just the right way such that even though each of those problems is being solved exponentially quickly, the \emph{aggregate} success rate over problems decreases at only a power‐law rate in $k$.
A more succinct mathematical summary is that, for a compound binomial distribution, the lower tail probability controls the upper tail of the marginal survivor function.
\section{Lack of Distributional Structure Explains Deviations from Power Law Scaling}
\label{sec:no_dist_structure_no_power_law}


\begin{figure*}[t!]
    \centering
    \includegraphics[width=0.9\linewidth]{figures/92_schematic_distributional_fitting_attempt2/distributional_fitting_schematic.png}
    \caption{\textbf{Schematic: Two Estimators of Power Law Parameters for Scaling Inference Compute via Repeat Sampling.} (A) Both estimators begin by generating many samples per prompt, then computing the number of successes per prompt. In the standard least squares power law parameter estimator (top), (B) $\operatorname{pass_i@k}$ is estimated for each $i$-th problem at multiple $k$ values, then (C) averaged over problems and fit with linear regression in log-log space.
    In the distributional power law parameter estimator (bottom), (D) a distribution $\mathcal{D}$ is fit to estimates of $\operatorname{pass_i@1}$, then (E) the single-attempt success probability distribution is used to simulate $\operatorname{pass_{\mathcal{D}}@k}$ at arbitrary $k$ values for linear regression in log-log space.}
    \label{fig:schematic2}
\end{figure*}

Notably, previous papers observed that not every model exhibits power law scaling in every setting. To highlight one, \citet{hughes2024bestofnjailbreaking} observed that when jailbreaking Meta's Llama 3 8B Instruction Tuned (IT) model \cite{grattafiori2024llama3herdmodels}, the $-\log (\operatorname{ASR_{\mathcal{D}}@k})$ fell faster than any power law (Fig.~\ref{fig:power_laws_repeat_sampling}), i.e., the $\operatorname{ASR_{\mathcal{D}}@k}$ rose much more quickly than the other frontier AI systems. Based on our mathematical insights and the empirical per-problem single-attempt attack success rates (Fig.~\ref{fig:multiple_attempts_pass_at_1_per_datum}), we can understand why: Llama 3 8B IT could be successfully jailbroken on every prompt within the permitted sampling budget and thus had no heavy left tail necessary to create the aggregate power law scaling.
% Similarly, \citet{brown2024largelanguagemonkeysscaling} found that power law scaling was less apparent on MiniF2F-MATH \citep{zheng2022minif2fcrosssystembenchmarkformal}, a dataset of mathematics problems that have been formalized into Lean, a proof checking programming language. 


\begin{figure*}[t!]
    \centering
    \includegraphics[width=0.5\linewidth]{figures/51_pass_at_1_compare_power_law_with_distributional_fit/llmonkeys_y=scaling_law_exponent_x=distributional_fit_exponent_kumaraswamy_binomial.pdf}%
    \includegraphics[width=0.5\linewidth]{figures/51_pass_at_1_compare_power_law_with_distributional_fit/bon_jailbreaking_y=scaling_law_exponent_x=distributional_fit_exponent_kumaraswamy_binomial.pdf}
    \caption{\textbf{Comparing Estimators of Power Law Exponents.} We compare two estimators of the power law exponent $b$ in $-\log(\operatorname{pass_{\mathcal{D}}@k}) \approx a k^{-b}\;$: (1) the standard least-squares estimator between $k$ and $-\log(\operatorname{pass_{\mathcal{D}}@k})$ in log-log space, and (2) the distributional estimator of $\operatorname{pass_i@1}$ assuming a scaled Kumaraswamy-Binomial distribution. Using all available data to fit both estimators, we find agreement between the least-squares estimate (ordinate) and the distribution-derived estimate (abscissa) for both Pythia models on MATH (left) and for frontier AI systems on HarmBench (right). For an explanation of why the two estimators match more closely for Large Language Monkeys than for Best-of-N Jailbreaking, see Appendix~\ref{app:sec:clarification_of_data_sampling}.
    }
    \label{fig:comparison_power_law_exponents}
\end{figure*}



% Our previous results \josh{TODO: Finish}
\begin{figure*}[t!]
    \centering
    % \includegraphics[width=\linewidth]{figures/52_compare_power_law_exponent_estimators_synthetic_data/y=relative_error_x=n_hue=distribution_params_col=distribution.pdf}
    \includegraphics[width=0.95\linewidth]{figures/52_compare_power_law_exponent_estimators_synthetic_data/y=relative_error_least_squares_x=n_hue=distribution_params_col=distribution.pdf}
    \caption{\textbf{Comparing Two Estimators of Power Law Exponents via Backtesting.} On synthetic data with known ground-truth power law $a \, k^{-b}$, we compare how well the least squares and the distributional estimator recover the scaling exponent $b$ as measured by the relative error $|\hat{b} - b| / b$ by backtesting: subsampling the number of problems and the number of samples per problem. We find that the distributional estimator obtains significantly better sample efficiency.}
    \label{fig:backtesting}
\end{figure*}


% \begin{figure*}[t!]
%     \centering
%     \includegraphics[width=\linewidth]{figures/52_compare_power_law_exponent_estimators_synthetic_data/y=relative_error_x=n_hue=distribution_params_col=distribution.pdf}
%     \caption{Placeholder}
%     \label{fig:enter-label}
% \end{figure*}



\section{A New Distributional Estimator for Predicting Power Law Scaling}
\label{sec:estimating_power_law_exponent}

A natural consequence of this connection between the scaling of $-\log(\operatorname{pass_{\mathcal{D}}@k})$ and the left tail of the distribution $p_{\mathcal{D}}(\operatorname{pass_i@1})$ is that the distribution of single-attempt success rates can be used to predict whether power-law scaling will appear and if so, what the intercept and exponent of the power law will be. To do this, one can fit the distribution $\hat{p}_{\mathcal{D}}(\operatorname{pass_i@1})$ and then \textit{simulate} how $\operatorname{pass_{\mathcal{D}}@k}$ will scale with $k$ (Fig.~\ref{fig:schematic2}) using the relationship:
%
\begin{equation}
\begin{aligned}
&\widehat{\operatorname{pass_{\mathcal{D}}@k}} \defeq \\
&1 - \int_0^1 (1 - \operatorname{pass_i@1})^k \, \hat{p}_{\mathcal{D}}(\operatorname{pass_i@1}) \, \operatorname{d\,pass_i@1}.
\end{aligned}
\end{equation}
%
To empirically test this claim, we compared the standard least squares regression estimator (in log-log space) \citep{hoffmann2022trainingcomputeoptimallargelanguage,caballero2022broken,besiroglu2024chinchillascalingreplicationattempt} against a \textit{distributional estimator}.
To motivate our distributional estimator, we first need explain a key obstacle and how the distributional estimator overcomes it.
The obstacle is that there are problems or prompts whose single-attempt success probabilities $\operatorname{pass_i@1}$ lie between $(0, 1/\text{Number of Samples})$ such that, due to finite sampling, we lack the resolution to measure.
While we do not know the true single-attempt success probability for the problems that lie in this interval, we \textit{do} know \textit{how many} problems fall into this left tail bucket, and we can fit a distribution's parameters such that the distribution's probability mass in the interval $(0, 1 / \text{Number of Samples})$ matches the empirical fraction of problems in this tail bucket. Thus, our distributional estimator works by first selecting a distribution (e.g., a scaled 3-parameter Beta distribution), discretizing the distribution according to the sampling resolution $1 / \text{Number of Samples}$ and performing maximum likelihood estimation under the discretized distribution's probability mass function.

We tested this distributional estimator in two different ways. First, focusing on Large Language Monkeys, we used all available real data from all problems and all samples per problem to compare the standard least squares regression estimator against the distributional estimator. 
We found close agreement between the two estimators (Fig.~\ref{fig:comparison_power_law_exponents}), giving us a sense that the two estimators yield reasonably consistent estimates under large sampling budgets.

Second, the distributional estimator also comes with another benefit: it directly provides an estimate of the power law's exponent $b$ in $a \, k^{-b}$. Estimating the power law's exponent is especially valuable because the exponent dictates how success rates are improving with increasing inference compute. To test how the distributional estimator and least squares estimator compare at recovering the true asymptotic power law exponent, we generated synthetic data so that we would have ground-truth knowledge of the true power law exponent, then backtested how the two scaling estimators compare at recovering the true exponent \citep{alabdulmohsin2022revisitingneuralscalinglaws, owen2024predicting} by subsampling data with fewer problems and fewer samples per problem.
We found that the distributional estimator obtains significantly better sample efficiency, with approximately an order of magnitude lower relative error $\defeq |\hat{b} - b| / b$ compared with the least squares estimator (Fig.~\ref{fig:backtesting}), or equivalently, ${\sim}2-4$ orders of magnitude less inference-compute. The distributional estimator performs well even under distributional mismatch.

\section{Discussion}


In this paper, we adopted a learner-centered design approach, beginning with a formative study to identify students' challenges with existing tools. Based on these insights, we developed DBox, a tool that scaffolds students in breaking problems into smaller parts and provides personalized, adaptive support. Our user study demonstrated that DBox improved learners' performance on similar algorithmic problems, increased perceived learning gains, and fostered greater cognitive engagement, achievement, and satisfaction. In this section, we discuss design implications and generalizability based on our key findings.


\ms{
\subsection{Chaining Learners' Thoughts with Visualized Structured UI Components}

Decomposition requires students to effectively organize their thoughts. While visual elements are known to promote structured thinking and support mental model construction \cite{mcdougall2001effects, liu2010mental}, our formative and user studies revealed shortcomings in existing tools like LeetCode and ChatGPT, which rely on textual representations without adequately supporting structured mental models. In contrast, DBox uses an interactive step tree to visually organize learners' thoughts. This feature was praised by 22 of 24 participants for enhancing algorithmic thinking, serving as a progress tracker, and providing value even without AI assistance.

DBox's interactive step tree and tree-based scaffolding demonstrate the broader potential of intelligent tutoring systems (ITS) to promote active learning and self-regulated problem-solving in fields requiring problem decomposition. Similar principles could benefit STEM education, such as physics or engineering, by externalizing abstract concepts and facilitating multi-step problem-solving. Additionally, progress-tracking visual components may inspire designs for professional training tools in areas like medical diagnostics or software engineering.

\subsection{Promoting Independent Thinking and Active Decomposition Learning}

\subsubsection{\textbf{Transforming Learners from Passive Readers to Active Thinkers}}

Many coding tools provide direct answers or solutions \cite{kazemitabaar2023novices, phung2023generating}, which, while efficient, often bypass opportunities to develop critical problem-solving skills. In contrast, DBox cultivates students' decomposition abilities through structured scaffolding, fostering critical thinking and self-regulated learning in line with learning by doing \cite{anzai1979theory} and constructivist principles \cite{tobias2009constructivist}.

To strengthen decomposition skills, DBox first encourages students to develop their own decomposition strategies by coding or building a step tree from scratch. While DBox can generate parts of a step tree from a student's existing code, these steps are derived from the learner's own reasoning, with DBox acting solely as a modality converter. Besides, DBox provides feedback on tree node statuses, identifying potential errors or missing steps without directly showing the correct answer, challenging students to critically evaluate and refine their decomposition plans.


DBox's scaffolded hint system further supports decomposition skill development by providing adaptive guidance tailored to the student’s progress without overwhelming them. All hints are based on the learner's current decomposition skeleton, with the most detailed hint—``reveal substep''—triggered only after repeated attempts and struggles. Notably, even the most detailed hints prompt only one substep, requiring students to complete the rest independently. As shown in Sec \ref{hintusage}, only 19\% of hints are this detailed, with students primarily relying on simpler, thought-provoking question hints. This scaffolded support system balances guidance and independent thinking, keeping students engaged during challenges without compromising their ability to independently decompose problems \cite{kinnunen2006students}.

Based on these findings, we recommend fostering active problem-solving by shifting students from passive content consumption to active solution creation. Designers could adopt layered scaffolding, starting with minimal guidance and increasing support as needed, to help students progressively master decomposition skills while maintaining confidence and avoiding frustration. Additionally, adaptive learning techniques, such as real-time feedback and progress tracking, can further tailor the support to individual decomposition barriers, encouraging deeper engagement with decomposition tasks. Moreover, designers could integrate metacognitive strategies, such as encouraging students to articulate or reflect on their decomposition approaches, to further enhance critical thinking and foster habits of independent thinking.




\subsubsection{\textbf{Choice of Scaffolding: Balancing Independent Problem-Solving and Efforts}}

Scaffolding involves providing tailored support to help learners accomplish tasks they cannot yet complete independently \cite{kim2011scaffolding, tobias2009constructivist}. Broadly, scaffolding strategies fall into two categories \cite{van2010scaffolding}: (1) gradually reducing assistance as learners gain proficiency, and (2) encouraging independent problem-solving while offering incremental support to address challenges. DBox adopts the second approach, emphasizing independent thinking and encouraging learners to actively decompose problems \cite{zimmerman2013theories}. While our scaffolding strategies successfully enhanced critical thinking, satisfaction, and perceived usefulness, they also led to increased cognitive effort (Sec. \ref{Effects_on_UX}). This tradeoff underscores the importance of carefully balancing cognitive effort with the promotion of independent thinking.

Future designs could incorporate adaptive scaffolding that adjusts support dynamically based on learner proficiency, reducing unnecessary effort in areas where students have demonstrated competence. Additionally, while incremental scaffolding was effective for algorithmic problem-solving, tailoring strategies to different educational contexts could enhance their applicability in diverse domains. Such adaptive, context-specific approaches could further optimize the balance between support and independence in learning environments.


\subsection{Supporting Personalized Algorithmic Programming Learning}

\subsubsection{\textbf{Prioritizing Learners' Own Solutions Over Optimality}}

Algorithmic problems often have multiple solutions with varying time and space complexities. DBox prioritizes independent exploration by supporting learners' strategies rather than steering them toward a single ``optimal'' solution. Using LLM-driven prompts, it evaluates and guides each step based on the learner's reasoning, preserving their step decomposition and respecting their input—even when errors occur. While some solutions may not be the most efficient, this approach fosters autonomy by aligning feedback with learners’ thought processes instead of enforcing rigid standards.

Our user study showed that this approach improves learning outcomes and is well-received by students. We recommend designing systems that respect personalized problem-solving strategies by aligning feedback with learners' reasoning while allowing for diverse approaches. Designers should balance flexibility and rigor, using prompts and interfaces that support varied strategies while gently guiding learners toward effective solutions.


\subsubsection{\textbf{Catering to Individual Learning Styles and Contextual Needs}}

DBox accommodates diverse problem-solving approaches with two input modes: coding and natural language descriptions. Each mode offers distinct advantages tailored to different learners, stages, and situations. Learners can switch seamlessly between modes, with progress automatically synced across the interface. Features such as verifying code-step alignment ensure strong integration between modes.

Our findings reveal that this flexibility enhances user experience. Participant interaction logs and interviews revealed three usage patterns, highlighting that each mode fits different needs: code mode works well for students with a clear and detailed problem-solving plan already, while the step tree with natural language descriptions helps less experienced students with only a basic idea who are not ready to write code directly, boosting their confidence.


We argue there is no universal “best” mode for programming education—each has unique benefits depending on the learner habits, expertise, and context. Future tools should provide flexibility, like DBox, or use adaptive algorithms to recommend modes based on user needs and context. This flexibility highlights the importance of designing educational tools that accommodate varying levels of expertise and problem-solving styles, which can be generalized to other domains requiring personalized learning \cite{bernacki2021systematic}.

\subsection{Appropriate Usage of LLMs for Supporting Algorithmic Programming Learning}

\subsubsection{\textbf{Caution About LLM Errors}}

Although LLMs have shown strong performance in coding tasks \cite{finnie2023my, leinonen2023using}, they remain prone to errors. Our technical evaluation and user study revealed that even with comprehensive context—such as problem statements, user code, and natural language steps—LLM sometimes misinterprets user descriptions. These errors likely arise from discrepancies between the natural language used by students and the formal, precise language the LLM was trained on, which is primarily sourced from web-based code and comments \cite{liu2023wants}.

Such misinterpretations can hinder learning by causing confusion or frustration. While future improvements to training data and GPT versions may mitigate these issues, design strategies can help address them. \textbf{First}, LLMs should avoid giving direct solutions and instead focus on fostering active problem-solving through explanations and hints. \textbf{Second}, feedback could be paired with interactive features, like a ``Run Code'' option, allowing students to validate their reasoning. \textbf{Third}, simple tutorials could teach users how to phrase their descriptions more clearly, improving LLM's understanding. Additionally, future tools could integrate a ``Language Enhancement'' feature to suggest improvements or assess the clarity of descriptions, aiding LLM in accurately capturing user intent. Most importantly, we recommend designers prioritize technical feasibility, such as conducting rigorous evaluations like ours, before fully integrating LLMs into programming learning tools.
}



\subsubsection{\textbf{Learner-LLM Co-Decomposition of Solutions: Learner as Leader, LLM as Aid}}

A central feature of DBox is the construction of a step tree, where students break solutions into steps and sub-steps. The LLM supports this by mapping code to step descriptions, evaluating them, and offering hints. However, students maintain full control, deciding how to decompose problems and define each step, fostering independent thinking. The LLM acts solely as an aid, using a scaffolding approach to support the development of learners' Zone of Proximal Development (ZPD) \cite{chaiklin2003zone}. Unlike tools like ChatGPT or Copilot that dominate problem-solving, DBox fosters deeper cognitive engagement. Students reported greater accomplishment and found this approach more effective for learning.

This contrasts with existing human-AI collaboration paradigms in non-educational scenarios where AI usually suggest options, leaving final decisions to users \cite{dang2023choice, gao2024collabcoder, gebreegziabher2023patat, ma2019smarteye, ma2022glancee}, such as in human-AI decision-making \cite{ma2023should, ma2024towards, ma2024you}. Some educational tools, like Jin et al. \cite{jin2024teach}, use LLMs to generate solutions for students to evaluate, which aids in syntax learning but such ``LLM-generate then learner-evaluate'' approach is less effective for algorithmic problem-solving, where constructing solutions is key. Just evaluating LLM-generated contents can place a cognitive anchor on learners \cite{furnham2011literature}, limiting independent thinking and creativity. Thus, task allocation between humans and AI should align with the educational context (e.g., whether it is basic knowledge/concept learning or higher-level creative thinking). Future LLM-based educational tools should carefully define the division of roles between LLMs and learners, tailoring it to specific learning contexts and goals.




% \subsubsection{Human-LLM Co-Decomposition of Solution: AI Should Judge Instead of Recommending}

% A core interaction in DBox is the construction of a step tree, where the entire solution is broken down into a series of steps and sub-steps. We refer to this as the human-LLM co-decomposition process. In this process, the LLM behind DBox plays three roles: First, it maps the student's written code into step descriptions. Second, it evaluates the status of each step and sub-step (whether they are correct, incorrect, missing, or need further decomposition). Third, it provides hints for incorrect or missing steps or sub-steps. However, the actual construction of the step tree—such as dividing the solution into steps and sub-steps and determining the content of each node—remains primarily the student's responsibility.

% This division of labor maximizes student engagement in independent thinking and problem-solving. The LLM does not provide any suggestions for decomposition nor directly recommend content for specific steps, aligning with the scaffolding educational approach, where guidance is provided appropriately, but the main task of forming the solution is left to the students.

% In contrast, when students directly seek help from an LLM, such as asking questions in ChatGPT or using Copilot for code completion, the LLM takes too much initiative by directly offering ideas or code. In our co-decomposition design, however, students demonstrated higher cognitive engagement and more active critical thinking. Furthermore, students reported that constructing solutions in this way gave them a greater sense of achievement and made them feel the process was more beneficial for learning, leading to higher satisfaction with the experience.

% Related work has proposed similar approaches. For instance, XXX, in the context of problem-solving, uses the "learning by teaching" concept, where students take on the tasks of judging and teaching, while the LLM generates most of the solutions. Compared to our approach, their division of labor between the student and the LLM is reversed. This method works well in introductory programming, where the focus is on mastering syntax. Having students guide the LLM to generate code or evaluate potentially incorrect code produced by the LLM is an effective way to quiz them. However, in our work, which focuses on algorithmic programming, the key step is constructing a solution from scratch. If the LLM builds the solution, leaving students only to judge it, it hampers their independent thinking.

% Thus, when designing LLM-based educational tools in the future, it is crucial to consider the specific context to effectively allocate tasks between the student and the LLM, ensuring that students derive the maximum benefit from the co-decomposition process.


% \subsection{Future Design Opportunities}

% \emph{Providing Appropriate Generative Assistance:} While DBox promotes independent problem-solving, some users showed interest in features like auto-completion for trivial coding tasks. Future versions could balance promoting independence with targeted assistance by enabling adjustable difficulty levels and offering contextual suggestions when appropriate.

% \emph{Covering All Stages of Algorithmic Programming:} DBox currently lacks a focus on foundational algorithm instruction and problem comprehension. Future iterations could include features like generating distractor solutions, input-output tests, and step-by-step rephrasing to help students grasp key concepts and understand the coding problem.

% \emph{Combining Step Trees with Dialogue:} Users can currently describe their thought processes but cannot ask questions. Adding a dialogue system to the step tree would allow students to share challenges and ask follow-up questions. GPT could then provide guided feedback without giving direct answers, supporting independent problem-solving.





% \emph{Other Important Features.} DBox could offer more control by allowing users to select specific parts of their code for targeted evaluation and guidance. A ``review'' feature could also help students reflect on key stumbling points, understand where their thought process went wrong, and how they eventually solved the problem.


% \subsection{Future Design Opportunities}

% \emph{Providing Appropriate Generative Assistance.} Our tool primarily focuses on encouraging users to create the step tree and write the code independently, with the system mainly serving as a judge. However, users expressed a desire for some intelligent completion features, particularly for repetitive or simple code, allowing them to focus their efforts on learning the key parts. Future improvements should strike a balance between fostering independent thinking and providing appropriate assistance. One approach could be designing basic rules where the tool offers intelligent suggestions and completions for parts unrelated to the core logic, while maintaining the current level of independence for key learning areas. Additionally, the system could offer different modes, allowing users to choose the level of assistance, from basic judgment-only feedback to a combination of judgment, guidance, necessary completions, and even on-demand suggestions.

% \emph{Covering All Stages of Algorithmic Programming.} Currently, our system does not cover the basic teaching of algorithms or the problem comprehension stage. In the future, to address the diversity and uncertainty in solutions and help students grasp multiple approaches, we could expand assistance during the idea formation phase. For example, GPT could generate multiple potential solutions with distractors, prompting students to identify the one that meets the problem's complexity requirements. We could also introduce specialized algorithm training, where students select a specific algorithm, and the system’s guidance focuses solely on that algorithm. To assist with problem comprehension, we could incorporate input-output tests to check students' understanding of the problem and step-by-step rephrasing to help them grasp more complex problems.

% \emph{Combining Interactive Step Trees with Dialogue Boxes.} Sometimes users want to describe their difficulties, and currently, we ask them to outline their thought processes. Additionally, users may want to ask follow-up questions. In the future, we could combine the structured step tree with a small dialogue box. The primary goal would still be to construct the step tree, but users could engage in a conversation with GPT in the context of the current step tree or a specific step. Importantly, GPT should guide the user without revealing direct answers.

% \emph{Other Important Features.} First, DBox could offer learners more control, such as allowing users to select specific parts of the code for targeted evaluation and guidance. We could also introduce a summary feature for key stumbling points, helping students reflect on the challenges they faced, where their thought process went wrong, and how they eventually overcame the problem.




\subsection{Limitations and Future Work}

This study has several limitations. \emph{First}, we tested DBox's effectiveness on only two problem types; future work should examine a broader range of algorithms. \emph{Second}, participants engaged in just one learning session per condition due to time constraints, whereas mastering algorithmic problems typically requires extended practice. Longitudinal studies should explore how DBox supports skill development over time, including changes in mental models and skill retention. \emph{Third}, we assessed learning gains based on correctness in a test session using similar learning and test problems. Future research should evaluate knowledge transfer to less similar problems. Due to time constraints, we conducted a single post-test rather than a pre-post comparison. While pre-test expertise filtering and randomization minimized prior familiarity effects, a more rigorous pre-post design would yield more accurate learning gain measurements. Looking ahead, we plan to release DBox as a Chrome plugin for integration with existing coding platforms, enabling large-scale field studies. This will allow for the collection of long-term usage data and periodic surveys to identify usage patterns and learning experiences over time.



% This study has several limitations. First, in our within-subject design, we selected two types of algorithm problems—Greedy and Binary Search—and randomly assigned them to two conditions (DBox and baseline). However, selection bias may still exist, as some participants might naturally excel at one type of algorithm. Although we addressed this by filtering participants' proficiency through a pre-test and using a Latin Square design, further validation across a broader range of algorithms is needed in future work.

% Second, students experienced only one learning session per condition before the test session. While this allowed for a fair comparison, mastering algorithmic problems typically requires extended practice. Future work should explore how DBox supports students' long-term improvement in algorithmic skills. Longitudinal studies could provide insights into changes in learners' mental models, allowing students more time to deepen their understanding and refine their decomposition methods. Additionally, retention tests could assess whether students can still apply learned problem-solving methods after a time gap.

% We measured learning gains through correctness scores in the test session, with relatively similar learning and test problems. Future work should explore students' ability to transfer their knowledge to problems with lower similarity. Due to time constraints, we opted for a single post-test rather than a pre-post comparison. While we minimized prior familiarity effects by filtering participants and randomizing problem assignments, future studies could adopt a more rigorous pre-post test design for better measurement of learning gains.

% Looking ahead, we plan to release DBox as a Chrome plugin for integration with existing online coding platforms and large-scale real-world testing. In such settings, where students may be more motivated (e.g., preparing for algorithm interviews), we can gather long-term usage data while ensuring privacy. We also plan to conduct periodic surveys to track changes in students' usage patterns and learning experiences over time.



% \subsection{Limitations and Future Work}

% This study has several limitations. First, in our within-subjects study, we selected two types of algorithm problems, Greedy and Binary Search, and randomly assigned them to two conditions, DBox and the baseline. However, there may still be selection bias, where some participants were naturally better at one type of algorithm. While we mitigated this issue to a large extent by filtering participants' proficiency through a pre-test and employing a Latin Square design to randomize the problem-condition assignment, there is still room for improvement. Future work should validate DBox's effectiveness across a broader range of problem types.

% Second, in our experiment, students only experienced one learning session in each condition before moving on to the test session. Although this comparison was fair (as both conditions had only one learning session), mastering an algorithmic problem often requires extended practice. Future work should explore how DBox can help students gradually improve their algorithmic programming skills over time. Longitudinal studies may reveal significant changes in learners' mental models, providing more time for them to understand a specific algorithm and enhance their decomposition methods. Additionally, future studies could include retention tests to measure whether students can still effectively apply previously learned problem-solving methods after a period of time.

% Furthermore, when objectively measuring students' learning gains, we calculated their correctness score in the test session. On the one hand, the learning session and test session problems had a relatively high degree of similarity. Future work should investigate whether students can transfer what they have learned to solve problems of the same algorithm type with lower similarity. On the other hand, due to time constraints, we did not include a pre-post test comparison, opting for a single post-test instead. This result might be influenced by students' pre-existing familiarity with the problems. Although we mitigated this issue by filtering for familiarity (ensuring participants were not too familiar with the problems) and randomizing the problem assignments, future work could include a more rigorous pre-post test design to better calculate students' learning gains.

% Moreover, DBox is currently only applied in algorithmic programming, specifically solving algorithm problems. However, this decomposition-based computational thinking approach could be extended to other learning scenarios, such as project-based learning. Future work could explore how to adapt DBox to broader educational contexts outside of algorithmic programming.

% Looking forward, we aim to deploy DBox in real-world algorithm courses. Since algorithms are a core required subject in undergraduate computer science curricula, we hope to investigate how students who have just learned algorithm concepts use DBox to develop their problem-solving skills. Additionally, we plan to convert DBox into a Chrome plugin and release it in the Chrome Web Store for real-world testing. This would allow DBox to seamlessly integrate with existing online coding platforms, enabling large-scale experiments. In such settings, students' motivation may be stronger (e.g., a graduate preparing for an algorithm interview), leading to more realistic usage patterns. Students could use DBox to tackle a wide variety of algorithm problems. We hope to collect long-term (e.g., six-month) usage data from real-world users while ensuring privacy, and use periodic surveys to capture changes in students' usage patterns and learning experiences over time.





\section{Conclusion}
% In this paper, we introduced Decomposition Box (DBox), a novel tool designed to scaffold learners in decomposing problems during algorithmic programming learning. Based on insights from a formative study, we identified key design goals to address the limitations of existing tools in algorithmic programming education. DBox supports two critical stages of the programming process: idea formation and idea implementation. By offering two modes (code mode and language mode), it encourages users to independently develop their solution strategies. The interactive, visual step tree helps students break down problems and build a structured mental model. DBox provides fine-grained, step-level feedback, enabling students to quickly identify issues, while its multi-level guidance offers targeted support without undermining independent thinking.

% Our user study demonstrated that DBox led to significantly higher learning gains, cognitive engagement, and critical thinking. Students reported a stronger sense of achievement and found the assistance both appropriate and effective for their learning. We identified three main usage patterns, underscoring the importance of respecting students' problem-solving habits and offering them autonomy. The learner-LLM co-decomposition model we designed promotes independent thinking while allowing the LLM to contribute meaningfully, even with occasional imperfections. 

% We hope the formative study, design goals, features, technical evaluation, and key findings from this work will inspire future research on developing educational tools for broader programming learning.
In this paper, we introduced DBox, an interactive tool designed to help learners decompose algorithmic programming problems by supporting both solution formation and implementation. Featuring an intuitive tree-like box widget, DBox accepts input in both code and natural language, fostering independent problem-solving while its step tree structure helps learners develop structured mental models. It provides step-level feedback and layered guidance without compromising learner autonomy.
Our user study showed that DBox significantly improved learning outcomes, cognitive engagement, and critical thinking, with students reporting a greater sense of achievement and finding the support highly effective. Additionally, we identified three key usage patterns, highlighting the importance of accommodating individual problem-solving styles. Moreover, our findings suggest that the learner-LLM co-decomposition approach fosters independent thinking while providing meaningful guidance, even with occasional imperfections.
We hope the insights from our system design will inspire future research on integrating LLMs into educational tools for programming learning.



% In the unusual situation where you want a paper to appear in the
% references without citing it in the main text, use \nocite
% \nocite{langley00}

\clearpage

\bibliography{references_rylan}
\bibliographystyle{icml2025}


%%%%%%%%%%%%%%%%%%%%%%%%%%%%%%%%%%%%%%%%%%%%%%%%%%%%%%%%%%%%%%%%%%%%%%%%%%%%%%%
%%%%%%%%%%%%%%%%%%%%%%%%%%%%%%%%%%%%%%%%%%%%%%%%%%%%%%%%%%%%%%%%%%%%%%%%%%%%%%%
% APPENDIX
%%%%%%%%%%%%%%%%%%%%%%%%%%%%%%%%%%%%%%%%%%%%%%%%%%%%%%%%%%%%%%%%%%%%%%%%%%%%%%%
%%%%%%%%%%%%%%%%%%%%%%%%%%%%%%%%%%%%%%%%%%%%%%%%%%%%%%%%%%%%%%%%%%%%%%%%%%%%%%%



\section{How is MENTAT Different from Medical Exam Questions?}
\label{app:medqa_to_mentat}

For years, medical AI benchmarks have focused on fact-based assessments. Most medical evaluations for LMs rely on board exams and medical student tests, primarily measuring knowledge recall rather than real-world clinical decision-making. These exams have little correlation with actual clinical practice, as passing them does not equate to the ability to manage patients effectively even in humans \cite{Saguil2015}.

\begin{figure}[ht]
    \begin{framed}
    A 32-year-old woman with type 1 diabetes mellitus has had progressive renal failure during the past 2 years. 
    She has not yet started dialysis. Examination shows no abnormalities. Her hemoglobin concentration is 9 g/dL, 
    hematocrit is 28\%, and mean corpuscular volume is 94 $\mu$m\textsuperscript{3}. 
    A blood smear shows normochromic, normocytic cells. 
    Which of the following is the most likely cause?
    
    (A) Acute blood loss \\
    (B) Chronic lymphocytic leukemia\\
    (C) Erythrocyte enzyme deficiency\\
    (D) Erythropoietin deficiency\\
    (E) Immunohemolysis\\
    (F) Microangiopathic hemolysis\\
    (G) Polycythemia vera \\
    (H) Sickle cell disease \\
    (I) Sideroblastic anemia \\
    (J) $\beta$-Thalassemia trait\\
    \textbf{(Answer: D)}
    \end{framed}
    \caption{USMLE board exam question example }
    \label{fig:usmle_example_q}
\end{figure}




For example, \Cref{fig:usmle_example_q} presents a classic USMLE board exam question \cite{USMLE2021}, which tests an AI model’s ability to recall factual knowledge rather than apply practical decision-making skills. The question may assess the recognition of a laboratory abnormality in diabetes, but it does not evaluate whether the model can adjust insulin regimens, recognize psychosocial factors, or determine hospitalization needs—key components of real-world patient care. As highlighted in previous research, medical licensing exams do not strongly correlate with clinical competency, reinforcing the need for benchmarks that evaluate accurate decision-making skills rather than memorization.

\begin{table}[h]
    \centering
    \begin{tabular}{llp{10cm}}
        \toprule
        \textbf{Question type} & \textbf{Attribute type} & \textbf{Example template question} \\
        \midrule
        \multirow{6}{*}{Single-Verify} 
        & SCP Code & Does this ECG show symptoms of \textbf{non-specific ST changes}? \\
        & Noise & Does this ECG show \textbf{baseline drift in lead I}? \\
        & Stage of infarction & Does this ECG show \textbf{early stage of myocardial infarction}? \\
        & Extra systole & Does this ECG show \textbf{ventricular extrasystoles}? \\
        & Heart axis & Does this ECG show \textbf{left axis deviation}? \\
        & Numeric feature & Does the \textbf{RR interval} of this ECG fall \textbf{within the normal range}? \\
        \bottomrule
    \end{tabular}
    \caption{Example template questions for different ECG attributes.}
    \label{tab:ecg_questions}
\end{table}
\begin{table}[h]
    \centering
    \begin{tabular}{lp{3.cm}p{3.5cm}p{1.5cm}p{3.5cm}}
        \toprule
        \textbf{Category} & \textbf{Task} & \textbf{Prompt} & \textbf{Result} & \textbf{AI Response} \\
        \midrule
        \multirow{2}{*}{Sequence alignment} 
        & DNA sequence alignment to human genome 
        & Align the DNA sequence to the human genome: \texttt{TGGGCTCA AGTGATCATA……} 
        & chr7 
        & As a language model AI, I do not have the capability to align a DNA sequence to the human genome…… 
        \\
        \midrule
        & DNA sequence alignment to multiple species 
        & Which organism does the DNA sequence come from: \texttt{CGTACACC ATTGGTGC……} 
        & yeast 
        & The organism from which the DNA sequence comes cannot be determined based solely on the DNA sequence…… 
         \\
        \bottomrule
    \end{tabular}
    \caption{DNA Sequence Alignment Tasks and AI Responses}
    \label{tab:sequence_alignment}
\end{table}


\Cref{tab:ecg_questions} and \Cref{tab:sequence_alignment} illustrate additional examples of widely used AI benchmarks, such as ECG-QA \cite{Oh2024} and GeneTuring \cite{Hou2023}, which focus on highly structured, fact-based medical knowledge. These datasets and others like MedQA \cite{Jin2021} have been leveraged by major AI companies, including Google’s Gemini initiative \cite{Saab2024}, to highlight model performance. While these benchmarks evaluate text-based and multimodal AI capabilities, they focus heavily on fact memorization rather than applied clinical reasoning.

Unlike traditional medical AI benchmarks, MENTAT is designed by practicing psychiatrists to reflect real-world clinical scenarios. The dataset also includes ambiguous, multi-choice decision-making tasks rather than a single correct answer, simulating the complex nature of psychiatric practice. Furthermore, MENTAT aims to reduce bias by empowering a diverse group of clinicians in its development from the start, making it less likely to reinforce harmful racial, gender, or sexuality-based biases in mental healthcare. In summary, MENTAT differs from medical exam questions by moving beyond fact recall to assess practical clinical decision-making in mental healthcare. While traditional benchmarks test AI models on medical knowledge, MENTAT evaluates whether AI can handle real-world psychiatric tasks, manage patient uncertainty, and make informed decisions in complex clinical environments.




% For years, benchmark evaluations have been utilized in medical AI to track the progress of new and updated models. However, they have largely focused on genetics, radiology, cardiology, and electronic medical record data processing\cite{Hou2023, Zambrano2023, Oh2024}. Little work has thus far been invested in the creation of benchmark evaluations and datasets for mental healthcare. Most medical evaluations of language and multi-modal models have also only focused on specialty board exams and exams intended for medical students. Both categories of exams assess knowledge but have been noted to have relatively little correlation to the real-world practice of medicine\cite{Saguil2015}. Every medical specialty would benefit from a creation of a dataset of question-answer pairs tailored specifically to clinical practice as opposed to the fact-based assessment that is common in licensing and medical student exams. As an example, Figure 1 demonstrates a classic board exam question directly from the USMLE website \cite{USMLE2021}, a medical licensing exam all medical students must take. It does not assess knowledge about pragmatic clinical management of diabetes; rather, it focuses on fact-based knowledge. Just as Saguil et al highlighted a lack of correlation between medical licensing exams and clinical skills, a model’s ability to answer Figure 1 correctly does not correlate to its ability to care for an individual with diabetes. Construction of datasets that test practical medical knowledge are necessary to robustly evaluate language models’ appropriateness in clinical settings. Furthermore, clinicians have recently called for moving beyond a medical exam benchmark, stating, "it is essential to move beyond medical exams and adopt more grounded, task-specific approaches for evaluation" \cite{Raji2025}. A related study, published in 2025, identified 11 high-level clinical tasks and created a benchmark (MedS-bench) meant to "address this gap" as current benchmarks and evaluation datasets "fail to adequately reflect the practical utility of LLMs in real-world clinical scenarios". Our work is different, but complementary, as our dataset applies this concept (testing for skills required to practice as a clinician as opposed to esoteric medical facts) to mental healthcare. In the realm of mental healthcare, researchers have expanded evaluations of LLM's into the realm of psychotherapy, which is complementary to our approach of evaluating LLM performance in the related field of psychiatry.


% As previously mentioned, most investigations into the evaluation of AI models in the healthcare setting have focused on pre-existing fact-based datasets such as the USMLE exams and specialty-specific board exams. In practice, these knowledge-based tests (e.g., USMLE Step 1) are designed to assess whether human trainees have acquired sufficient baseline knowledge to enter post-graduate training in a selected medical specialty. However, passing these tests alone is not considered sufficient for practicing as a physician in the United States. Residency training is required, during which trainees apply fundamental medical knowledge to real-world clinical cases \cite{Mowery2015}. 
% %
% Similarly, AI models must progress beyond simple recall of medical facts. They should be trained and evaluated on real-world clinical tasks that require the application of baseline medical knowledge. To maximize external validity, datasets should be created and vetted by actively practicing clinicians. Some of the most widely used benchmarks in medical AI are frequently leveraged by major companies to showcase the clinical capabilities of their fine-tuned models. For instance, in 2024, Google published *Capabilities of Gemini Models in Medicine*, incorporating several prominent medical benchmarks \cite{Saab2024}. The authors highlight the novelty of their work as “the most comprehensive benchmarking of multimodal medical models to date” based on their use of 14 different medical benchmarks \cite{Saab2024}. These benchmarks include ECG-QA \cite{Oh2024}, MedQA \cite{Jin2021}, GeneTuring \cite{Hou2023}, MMMU (health medicine) \cite{Yue2023}, NEJM Image Challenges \cite{NEJM2024}, Path-VQA \cite{He2020}, and others. Additionally, an effort was made to enhance the clinical relevance of these findings by evaluating the models’ ability to summarize complex medical information and generate referral letters for specialists \cite{Saab2024}. 
% %
% Examples of questions from these datasets are provided in Figures 3 and 4, which illustrate excerpts from ECG-QA and GeneTuring, respectively. MedQA, on the other hand, is best represented by the example in Figure 1. In the domain of mental health datasets and summarization, Adhikary et al. introduced a dataset comprising 191 counseling sessions with associated summaries \cite{Adhikary2024}. 
% %
% Despite the breadth of existing benchmarks, there is still no robust, clinician-led, and clinician-vetted mental healthcare benchmark for AI models. Current medical benchmarks remain overly narrow and fact-based, limiting their external validity and clinical relevance. As noted earlier, a high score on the MedQA benchmark does not equate to excellence in clinical care. Our dataset shifts the paradigm by moving beyond fact-based assessments (e.g., USMLE exams) and introducing a comprehensive evaluation of clinician-level decision-making in mental healthcare. 
% %
% Our benchmark uniquely assesses an AI model’s ability to **triage, diagnose, treat, and monitor mental health conditions**, establishing a new category of medical benchmarks that we hope other specialties will adopt. Furthermore, because our dataset is developed and overseen by a diverse group of practicing clinicians, it is significantly less likely to perpetuate harmful racial, gender, or sexuality-based biases in mental healthcare. It has been validated by [insert number] practicing psychiatrists.

\newpage

\section{Further Annotation Processing Results}
\label{app:annotation_details}

% \begin{figure}[ht]
%     \vskip 0.2in
%     \begin{center}
%     \centerline{\includegraphics[width=0.5\columnwidth]{figures/raw_annotation_krippendorf.pdf}}
%     \caption{Test.}
%     \label{fig:raw_annotation_krippendorf}
%     \end{center}
%     \vskip -0.2in
% \end{figure}

\begin{figure}[ht!]
    \centering
    \begin{minipage}[b]{0.49\textwidth}
        \centering
        \includegraphics[width=\linewidth]{figures/annotator_scores_hbt_pars.pdf}
        \caption{(Top) We show the average raw annotation score with with bootstrapped (95\% CL) uncertainties for each annotator. All of them deviate from 50 with statistical significance (the random baseline). 
        (Bottom) Fitted individual annotator parameters from the hierarchical Bradley-Terry model.
        Besides regularization in the log-likelihood objective, we bound the individual annotator parameters ($\gamma_a \in [-3.0, 3.0]$, $\alpha_a \in [0.5, 2.0]$) during the optimization to balance the goal of slightly de-noising the resulting preference dataset while keeping the majority of differences between individual annotator preferences.
        These bounds prevent the model from fixing contradictory data by pushing a parameter to an extreme.
        The fact that all annotators have a positive offset $\gamma_a$ indicates that they all tend to choose one answer option to prefer over all others in a single annotation of one question.}
        \label{fig:annotator_scores_hbt_pars}
    \end{minipage}%
    \hfill
    \begin{minipage}[b]{0.49\textwidth}
        \centering
        \includegraphics[width=\linewidth]{figures/raw_annotation_krippendorf.pdf}
        \caption{
        We show the distribution of  Krippendorff's $\alpha$ for raw triage and documentation question annotations.
        We verify that the expert annotators do not converge on one answer option and that there is sufficient inter-annotator disagreement.
        Given our design choices, we expect $\alpha$ to be naturally low as our goal is not to measure the presence of a single ground truth and low $\alpha$ values ($\alpha \leq 0.5$) will not tell us how useful a set of annotations is—only that experts statistically disagree. 
        }
        \label{fig:raw_annotation_krippendorf}
    \end{minipage}
\end{figure}
% \begin{figure}[ht]
%     \vskip 0.2in
%     \begin{center}
%     \centerline{\includegraphics[width=0.5\columnwidth]{figures/annotator_scores_hbt_pars.pdf}}
%     \caption{Test.}
%     \label{fig:annotator_scores_hbt_pars}
%     \end{center}
%     \vskip -0.2in
% \end{figure}
% \begin{figure}[ht]
%     \vskip 0.2in
%     \begin{center}
%     \centerline{\includegraphics[width=0.5\columnwidth]{figures/frac_ct in topk_bt_vs_hbt.pdf}}
%     \caption{Test.}
%     \label{fig:frac_ct in topk_bt_vs_hbt}
%     \end{center}
%     \vskip -0.2in
% \end{figure}

\newpage

\section{Language Model Prompts}
\label{app:prompting}

\begin{figure}[ht]
    \centering
    \begin{minipage}[b]{0.38\textwidth}
        \begin{framed}
        \texttt{
        f"Question: \{q\}\textbackslash n\textbackslash n"\\
        f"A: \{answer\_list[0]\}\textbackslash n"\\
        f"B: \{answer\_list[1]\}\textbackslash n"\\
        f"C: \{answer\_list[2]\}\textbackslash n"\\
        f"D: \{answer\_list[3]\}\textbackslash n"\\
        f"E: \{answer\_list[4]\}\textbackslash n\textbackslash n"\\
        "Answer (single letter): "
        }
        \end{framed}
    \end{minipage}%
    \hfill
    \begin{minipage}[b]{0.62\textwidth}
        % \centering
        \begin{framed}
        \texttt{
        f"Question: \{q\}\textbackslash n\textbackslash n"\\
        f"A: \{answer\_list[0]\}\textbackslash n"\\
        f"B: \{answer\_list[1]\}\textbackslash n"\\
        f"C: \{answer\_list[2]\}\textbackslash n"\\
        f"D: \{answer\_list[3]\}\textbackslash n"\\
        f"E: \{answer\_list[4]\}\textbackslash n\textbackslash n"\\
        "Answer (only reply with a single letter!): "
        }
        \end{framed}
    \end{minipage}
    \caption{(Left) Prompt text MCQA variation A (as used for \textit{gpt-4o-mini-2024-07-18}, \textit{gpt-4o-2024-08-06}, \textit{o1-2024-12-17}, and \textit{o1-mini-2024-09-12}).
    (Right) Prompt text MCQA variation B (all other models).
By looking at the responses from models evaluated with variation A, we verified that the recorded accuracy difference caused by using different promtps was $\leq 1$\%.
The only exception was \textit{o1-mini-2024-09-12}, for which we corrected the evaluation.}
    \label{fig:eval_prompts_mcqa}
\end{figure}

\begin{figure}[ht]
    \vskip 0.2in
    \begin{framed}
        \texttt{
        f"Question: \{q\}\textbackslash n\textbackslash n"\\
        "Answer (write your reply in only one short sentence!): "
        }
        \end{framed}
        \caption{Prompt text free-form (as used for the models evaluated in \Cref{sec:4_4_consistency}).}
    \vskip -0.2in
\end{figure}

\newpage

\section{Annotator Interface}
\label{app:annotator_interface}

\begin{figure}[ht]
    \centering
    \begin{minipage}[b]{0.49\textwidth}
        \centering
        \includegraphics[width=\linewidth]{figures/mentat_q36_question.png}
    \end{minipage}%
    \hfill
    \begin{minipage}[b]{0.49\textwidth}
        \centering
        \includegraphics[width=\linewidth]{figures/mentat_q36_answers.png}
    \end{minipage}
    \caption{Example of the online annotation interface using the \textit{jsPsych} library \citep{de_Leeuw2023} (MIT license). There is also a comment box below the sliders for feedback/comments, that is not shown.}
    \label{fig:mentat_q36_combined}
\end{figure}

\newpage

\section{Further Evaluation Results}
\label{app:more_experiment_results}

\begin{figure}[ht]
    \centering
    \begin{minipage}[b]{0.49\textwidth}
        \centering
        \includegraphics[width=\linewidth]{figures/final_eval_results_by_gender.pdf}
        \caption{Using the $\mathcal{D}_\text{G}$ dataset, we evaluate eleven off-the-shelf instruction-tuned and three (mental) healthcare fine-tuned models for overall accuracy and how it is impacted by different patient genders.}
        \label{fig:final_eval_results_by_gender}
    \end{minipage}%
    \hfill
    \begin{minipage}[b]{0.49\textwidth}
        \centering
        \includegraphics[width=\linewidth]{figures/final_eval_results_by_age.pdf}
        \caption{Using the $\mathcal{D}_\text{A}$ dataset, we evaluate eleven off-the-shelf instruction-tuned and three (mental) healthcare fine-tuned models for overall accuracy and how it is impacted by different patient ages.}
        \label{fig:final_eval_results_by_age}
    \end{minipage}
\end{figure}

% \begin{figure}[ht]
%     \vskip 0.2in
%     \begin{center}
%     \centerline{\includegraphics[width=0.5\columnwidth]{figures/final_eval_results_by_gender.pdf}}
%     \caption{Test.}
%     \label{fig:final_eval_results_by_gender}
%     \end{center}
%     \vskip -0.2in
% \end{figure}
% \begin{figure}[ht]
%     \vskip 0.2in
%     \begin{center}
%     \centerline{\includegraphics[width=0.5\columnwidth]{figures/final_eval_results_by_age.pdf}}
%     \caption{Test.}
%     \label{fig:final_eval_results_by_age}
%     \end{center}
%     \vskip -0.2in
% \end{figure}

\begin{figure}[ht!]
    \vskip 0.2in
    \begin{center}
    \centerline{\includegraphics[width=0.47\columnwidth]{figures/final_eval_results_by_nat.pdf}}
    \caption{Using the $\mathcal{D}_\text{N}$ dataset, we evaluate eleven off-the-shelf instruction-tuned and three (mental) healthcare fine-tuned models for overall accuracy and how it is impacted by different patient ethnicities.}
    \label{fig:final_eval_results_by_nat}
    \end{center}
    \vskip -0.2in
\end{figure}

% \input{corrected_proofs_by_josh}
\end{document}
