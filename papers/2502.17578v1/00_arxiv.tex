%%%%%%%% ICML 2025 EXAMPLE LATEX SUBMISSION FILE %%%%%%%%%%%%%%%%%

\documentclass{article}

% Recommended, but optional, packages for figures and better typesetting:
\usepackage{microtype}
\usepackage{graphicx}
\usepackage{subfigure}
\usepackage{booktabs} % for professional tables
\usepackage[T1]{fontenc}

% hyperref makes hyperlinks in the resulting PDF.
% If your build breaks (sometimes temporarily if a hyperlink spans a page)
% please comment out the following usepackage line and replace
% \usepackage{icml2025} with \usepackage[nohyperref]{icml2025} above.
\usepackage{hyperref}


% Attempt to make hyperref and algorithmic work together better:
\newcommand{\theHalgorithm}{\arabic{algorithm}}

% Use the following line for the initial blind version submitted for review:

% If accepted, instead use the following line for the camera-ready submission:
\usepackage[accepted]{arxiv}

% For theorems and such
\usepackage{amsmath}
\usepackage{amssymb}
\usepackage{mathtools}
\usepackage{amsthm}
\usepackage{subcaption}
\usepackage{leftindex}


% if you use cleveref..
\usepackage[capitalize,noabbrev]{cleveref}


% Todonotes is useful during development; simply uncomment the next line
%    and comment out the line below the next line to turn off comments
%\usepackage[disable,textsize=tiny]{todonotes}
\usepackage[textsize=tiny]{todonotes}
\usepackage{listings}

%%%%% NEW MATH DEFINITIONS %%%%%

\usepackage{amsmath,amsfonts,bm}
\usepackage{derivative}
% Mark sections of captions for referring to divisions of figures
\newcommand{\figleft}{{\em (Left)}}
\newcommand{\figcenter}{{\em (Center)}}
\newcommand{\figright}{{\em (Right)}}
\newcommand{\figtop}{{\em (Top)}}
\newcommand{\figbottom}{{\em (Bottom)}}
\newcommand{\captiona}{{\em (a)}}
\newcommand{\captionb}{{\em (b)}}
\newcommand{\captionc}{{\em (c)}}
\newcommand{\captiond}{{\em (d)}}

% Highlight a newly defined term
\newcommand{\newterm}[1]{{\bf #1}}

% Derivative d 
\newcommand{\deriv}{{\mathrm{d}}}

% Figure reference, lower-case.
\def\figref#1{figure~\ref{#1}}
% Figure reference, capital. For start of sentence
\def\Figref#1{Figure~\ref{#1}}
\def\twofigref#1#2{figures \ref{#1} and \ref{#2}}
\def\quadfigref#1#2#3#4{figures \ref{#1}, \ref{#2}, \ref{#3} and \ref{#4}}
% Section reference, lower-case.
\def\secref#1{section~\ref{#1}}
% Section reference, capital.
\def\Secref#1{Section~\ref{#1}}
% Reference to two sections.
\def\twosecrefs#1#2{sections \ref{#1} and \ref{#2}}
% Reference to three sections.
\def\secrefs#1#2#3{sections \ref{#1}, \ref{#2} and \ref{#3}}
% Reference to an equation, lower-case.
\def\eqref#1{equation~\ref{#1}}
% Reference to an equation, upper case
\def\Eqref#1{Equation~\ref{#1}}
% A raw reference to an equation---avoid using if possible
\def\plaineqref#1{\ref{#1}}
% Reference to a chapter, lower-case.
\def\chapref#1{chapter~\ref{#1}}
% Reference to an equation, upper case.
\def\Chapref#1{Chapter~\ref{#1}}
% Reference to a range of chapters
\def\rangechapref#1#2{chapters\ref{#1}--\ref{#2}}
% Reference to an algorithm, lower-case.
\def\algref#1{algorithm~\ref{#1}}
% Reference to an algorithm, upper case.
\def\Algref#1{Algorithm~\ref{#1}}
\def\twoalgref#1#2{algorithms \ref{#1} and \ref{#2}}
\def\Twoalgref#1#2{Algorithms \ref{#1} and \ref{#2}}
% Reference to a part, lower case
\def\partref#1{part~\ref{#1}}
% Reference to a part, upper case
\def\Partref#1{Part~\ref{#1}}
\def\twopartref#1#2{parts \ref{#1} and \ref{#2}}

\def\ceil#1{\lceil #1 \rceil}
\def\floor#1{\lfloor #1 \rfloor}
\def\1{\bm{1}}
\newcommand{\train}{\mathcal{D}}
\newcommand{\valid}{\mathcal{D_{\mathrm{valid}}}}
\newcommand{\test}{\mathcal{D_{\mathrm{test}}}}

\def\eps{{\epsilon}}


% Random variables
\def\reta{{\textnormal{$\eta$}}}
\def\ra{{\textnormal{a}}}
\def\rb{{\textnormal{b}}}
\def\rc{{\textnormal{c}}}
\def\rd{{\textnormal{d}}}
\def\re{{\textnormal{e}}}
\def\rf{{\textnormal{f}}}
\def\rg{{\textnormal{g}}}
\def\rh{{\textnormal{h}}}
\def\ri{{\textnormal{i}}}
\def\rj{{\textnormal{j}}}
\def\rk{{\textnormal{k}}}
\def\rl{{\textnormal{l}}}
% rm is already a command, just don't name any random variables m
\def\rn{{\textnormal{n}}}
\def\ro{{\textnormal{o}}}
\def\rp{{\textnormal{p}}}
\def\rq{{\textnormal{q}}}
\def\rr{{\textnormal{r}}}
\def\rs{{\textnormal{s}}}
\def\rt{{\textnormal{t}}}
\def\ru{{\textnormal{u}}}
\def\rv{{\textnormal{v}}}
\def\rw{{\textnormal{w}}}
\def\rx{{\textnormal{x}}}
\def\ry{{\textnormal{y}}}
\def\rz{{\textnormal{z}}}

% Random vectors
\def\rvepsilon{{\mathbf{\epsilon}}}
\def\rvphi{{\mathbf{\phi}}}
\def\rvtheta{{\mathbf{\theta}}}
\def\rva{{\mathbf{a}}}
\def\rvb{{\mathbf{b}}}
\def\rvc{{\mathbf{c}}}
\def\rvd{{\mathbf{d}}}
\def\rve{{\mathbf{e}}}
\def\rvf{{\mathbf{f}}}
\def\rvg{{\mathbf{g}}}
\def\rvh{{\mathbf{h}}}
\def\rvu{{\mathbf{i}}}
\def\rvj{{\mathbf{j}}}
\def\rvk{{\mathbf{k}}}
\def\rvl{{\mathbf{l}}}
\def\rvm{{\mathbf{m}}}
\def\rvn{{\mathbf{n}}}
\def\rvo{{\mathbf{o}}}
\def\rvp{{\mathbf{p}}}
\def\rvq{{\mathbf{q}}}
\def\rvr{{\mathbf{r}}}
\def\rvs{{\mathbf{s}}}
\def\rvt{{\mathbf{t}}}
\def\rvu{{\mathbf{u}}}
\def\rvv{{\mathbf{v}}}
\def\rvw{{\mathbf{w}}}
\def\rvx{{\mathbf{x}}}
\def\rvy{{\mathbf{y}}}
\def\rvz{{\mathbf{z}}}

% Elements of random vectors
\def\erva{{\textnormal{a}}}
\def\ervb{{\textnormal{b}}}
\def\ervc{{\textnormal{c}}}
\def\ervd{{\textnormal{d}}}
\def\erve{{\textnormal{e}}}
\def\ervf{{\textnormal{f}}}
\def\ervg{{\textnormal{g}}}
\def\ervh{{\textnormal{h}}}
\def\ervi{{\textnormal{i}}}
\def\ervj{{\textnormal{j}}}
\def\ervk{{\textnormal{k}}}
\def\ervl{{\textnormal{l}}}
\def\ervm{{\textnormal{m}}}
\def\ervn{{\textnormal{n}}}
\def\ervo{{\textnormal{o}}}
\def\ervp{{\textnormal{p}}}
\def\ervq{{\textnormal{q}}}
\def\ervr{{\textnormal{r}}}
\def\ervs{{\textnormal{s}}}
\def\ervt{{\textnormal{t}}}
\def\ervu{{\textnormal{u}}}
\def\ervv{{\textnormal{v}}}
\def\ervw{{\textnormal{w}}}
\def\ervx{{\textnormal{x}}}
\def\ervy{{\textnormal{y}}}
\def\ervz{{\textnormal{z}}}

% Random matrices
\def\rmA{{\mathbf{A}}}
\def\rmB{{\mathbf{B}}}
\def\rmC{{\mathbf{C}}}
\def\rmD{{\mathbf{D}}}
\def\rmE{{\mathbf{E}}}
\def\rmF{{\mathbf{F}}}
\def\rmG{{\mathbf{G}}}
\def\rmH{{\mathbf{H}}}
\def\rmI{{\mathbf{I}}}
\def\rmJ{{\mathbf{J}}}
\def\rmK{{\mathbf{K}}}
\def\rmL{{\mathbf{L}}}
\def\rmM{{\mathbf{M}}}
\def\rmN{{\mathbf{N}}}
\def\rmO{{\mathbf{O}}}
\def\rmP{{\mathbf{P}}}
\def\rmQ{{\mathbf{Q}}}
\def\rmR{{\mathbf{R}}}
\def\rmS{{\mathbf{S}}}
\def\rmT{{\mathbf{T}}}
\def\rmU{{\mathbf{U}}}
\def\rmV{{\mathbf{V}}}
\def\rmW{{\mathbf{W}}}
\def\rmX{{\mathbf{X}}}
\def\rmY{{\mathbf{Y}}}
\def\rmZ{{\mathbf{Z}}}

% Elements of random matrices
\def\ermA{{\textnormal{A}}}
\def\ermB{{\textnormal{B}}}
\def\ermC{{\textnormal{C}}}
\def\ermD{{\textnormal{D}}}
\def\ermE{{\textnormal{E}}}
\def\ermF{{\textnormal{F}}}
\def\ermG{{\textnormal{G}}}
\def\ermH{{\textnormal{H}}}
\def\ermI{{\textnormal{I}}}
\def\ermJ{{\textnormal{J}}}
\def\ermK{{\textnormal{K}}}
\def\ermL{{\textnormal{L}}}
\def\ermM{{\textnormal{M}}}
\def\ermN{{\textnormal{N}}}
\def\ermO{{\textnormal{O}}}
\def\ermP{{\textnormal{P}}}
\def\ermQ{{\textnormal{Q}}}
\def\ermR{{\textnormal{R}}}
\def\ermS{{\textnormal{S}}}
\def\ermT{{\textnormal{T}}}
\def\ermU{{\textnormal{U}}}
\def\ermV{{\textnormal{V}}}
\def\ermW{{\textnormal{W}}}
\def\ermX{{\textnormal{X}}}
\def\ermY{{\textnormal{Y}}}
\def\ermZ{{\textnormal{Z}}}

% Vectors
\def\vzero{{\bm{0}}}
\def\vone{{\bm{1}}}
\def\vmu{{\bm{\mu}}}
\def\vtheta{{\bm{\theta}}}
\def\vphi{{\bm{\phi}}}
\def\va{{\bm{a}}}
\def\vb{{\bm{b}}}
\def\vc{{\bm{c}}}
\def\vd{{\bm{d}}}
\def\ve{{\bm{e}}}
\def\vf{{\bm{f}}}
\def\vg{{\bm{g}}}
\def\vh{{\bm{h}}}
\def\vi{{\bm{i}}}
\def\vj{{\bm{j}}}
\def\vk{{\bm{k}}}
\def\vl{{\bm{l}}}
\def\vm{{\bm{m}}}
\def\vn{{\bm{n}}}
\def\vo{{\bm{o}}}
\def\vp{{\bm{p}}}
\def\vq{{\bm{q}}}
\def\vr{{\bm{r}}}
\def\vs{{\bm{s}}}
\def\vt{{\bm{t}}}
\def\vu{{\bm{u}}}
\def\vv{{\bm{v}}}
\def\vw{{\bm{w}}}
\def\vx{{\bm{x}}}
\def\vy{{\bm{y}}}
\def\vz{{\bm{z}}}

% Elements of vectors
\def\evalpha{{\alpha}}
\def\evbeta{{\beta}}
\def\evepsilon{{\epsilon}}
\def\evlambda{{\lambda}}
\def\evomega{{\omega}}
\def\evmu{{\mu}}
\def\evpsi{{\psi}}
\def\evsigma{{\sigma}}
\def\evtheta{{\theta}}
\def\eva{{a}}
\def\evb{{b}}
\def\evc{{c}}
\def\evd{{d}}
\def\eve{{e}}
\def\evf{{f}}
\def\evg{{g}}
\def\evh{{h}}
\def\evi{{i}}
\def\evj{{j}}
\def\evk{{k}}
\def\evl{{l}}
\def\evm{{m}}
\def\evn{{n}}
\def\evo{{o}}
\def\evp{{p}}
\def\evq{{q}}
\def\evr{{r}}
\def\evs{{s}}
\def\evt{{t}}
\def\evu{{u}}
\def\evv{{v}}
\def\evw{{w}}
\def\evx{{x}}
\def\evy{{y}}
\def\evz{{z}}

% Matrix
\def\mA{{\bm{A}}}
\def\mB{{\bm{B}}}
\def\mC{{\bm{C}}}
\def\mD{{\bm{D}}}
\def\mE{{\bm{E}}}
\def\mF{{\bm{F}}}
\def\mG{{\bm{G}}}
\def\mH{{\bm{H}}}
\def\mI{{\bm{I}}}
\def\mJ{{\bm{J}}}
\def\mK{{\bm{K}}}
\def\mL{{\bm{L}}}
\def\mM{{\bm{M}}}
\def\mN{{\bm{N}}}
\def\mO{{\bm{O}}}
\def\mP{{\bm{P}}}
\def\mQ{{\bm{Q}}}
\def\mR{{\bm{R}}}
\def\mS{{\bm{S}}}
\def\mT{{\bm{T}}}
\def\mU{{\bm{U}}}
\def\mV{{\bm{V}}}
\def\mW{{\bm{W}}}
\def\mX{{\bm{X}}}
\def\mY{{\bm{Y}}}
\def\mZ{{\bm{Z}}}
\def\mBeta{{\bm{\beta}}}
\def\mPhi{{\bm{\Phi}}}
\def\mLambda{{\bm{\Lambda}}}
\def\mSigma{{\bm{\Sigma}}}

% Tensor
\DeclareMathAlphabet{\mathsfit}{\encodingdefault}{\sfdefault}{m}{sl}
\SetMathAlphabet{\mathsfit}{bold}{\encodingdefault}{\sfdefault}{bx}{n}
\newcommand{\tens}[1]{\bm{\mathsfit{#1}}}
\def\tA{{\tens{A}}}
\def\tB{{\tens{B}}}
\def\tC{{\tens{C}}}
\def\tD{{\tens{D}}}
\def\tE{{\tens{E}}}
\def\tF{{\tens{F}}}
\def\tG{{\tens{G}}}
\def\tH{{\tens{H}}}
\def\tI{{\tens{I}}}
\def\tJ{{\tens{J}}}
\def\tK{{\tens{K}}}
\def\tL{{\tens{L}}}
\def\tM{{\tens{M}}}
\def\tN{{\tens{N}}}
\def\tO{{\tens{O}}}
\def\tP{{\tens{P}}}
\def\tQ{{\tens{Q}}}
\def\tR{{\tens{R}}}
\def\tS{{\tens{S}}}
\def\tT{{\tens{T}}}
\def\tU{{\tens{U}}}
\def\tV{{\tens{V}}}
\def\tW{{\tens{W}}}
\def\tX{{\tens{X}}}
\def\tY{{\tens{Y}}}
\def\tZ{{\tens{Z}}}


% Graph
\def\gA{{\mathcal{A}}}
\def\gB{{\mathcal{B}}}
\def\gC{{\mathcal{C}}}
\def\gD{{\mathcal{D}}}
\def\gE{{\mathcal{E}}}
\def\gF{{\mathcal{F}}}
\def\gG{{\mathcal{G}}}
\def\gH{{\mathcal{H}}}
\def\gI{{\mathcal{I}}}
\def\gJ{{\mathcal{J}}}
\def\gK{{\mathcal{K}}}
\def\gL{{\mathcal{L}}}
\def\gM{{\mathcal{M}}}
\def\gN{{\mathcal{N}}}
\def\gO{{\mathcal{O}}}
\def\gP{{\mathcal{P}}}
\def\gQ{{\mathcal{Q}}}
\def\gR{{\mathcal{R}}}
\def\gS{{\mathcal{S}}}
\def\gT{{\mathcal{T}}}
\def\gU{{\mathcal{U}}}
\def\gV{{\mathcal{V}}}
\def\gW{{\mathcal{W}}}
\def\gX{{\mathcal{X}}}
\def\gY{{\mathcal{Y}}}
\def\gZ{{\mathcal{Z}}}

% Sets
\def\sA{{\mathbb{A}}}
\def\sB{{\mathbb{B}}}
\def\sC{{\mathbb{C}}}
\def\sD{{\mathbb{D}}}
% Don't use a set called E, because this would be the same as our symbol
% for expectation.
\def\sF{{\mathbb{F}}}
\def\sG{{\mathbb{G}}}
\def\sH{{\mathbb{H}}}
\def\sI{{\mathbb{I}}}
\def\sJ{{\mathbb{J}}}
\def\sK{{\mathbb{K}}}
\def\sL{{\mathbb{L}}}
\def\sM{{\mathbb{M}}}
\def\sN{{\mathbb{N}}}
\def\sO{{\mathbb{O}}}
\def\sP{{\mathbb{P}}}
\def\sQ{{\mathbb{Q}}}
\def\sR{{\mathbb{R}}}
\def\sS{{\mathbb{S}}}
\def\sT{{\mathbb{T}}}
\def\sU{{\mathbb{U}}}
\def\sV{{\mathbb{V}}}
\def\sW{{\mathbb{W}}}
\def\sX{{\mathbb{X}}}
\def\sY{{\mathbb{Y}}}
\def\sZ{{\mathbb{Z}}}

% Entries of a matrix
\def\emLambda{{\Lambda}}
\def\emA{{A}}
\def\emB{{B}}
\def\emC{{C}}
\def\emD{{D}}
\def\emE{{E}}
\def\emF{{F}}
\def\emG{{G}}
\def\emH{{H}}
\def\emI{{I}}
\def\emJ{{J}}
\def\emK{{K}}
\def\emL{{L}}
\def\emM{{M}}
\def\emN{{N}}
\def\emO{{O}}
\def\emP{{P}}
\def\emQ{{Q}}
\def\emR{{R}}
\def\emS{{S}}
\def\emT{{T}}
\def\emU{{U}}
\def\emV{{V}}
\def\emW{{W}}
\def\emX{{X}}
\def\emY{{Y}}
\def\emZ{{Z}}
\def\emSigma{{\Sigma}}

% entries of a tensor
% Same font as tensor, without \bm wrapper
\newcommand{\etens}[1]{\mathsfit{#1}}
\def\etLambda{{\etens{\Lambda}}}
\def\etA{{\etens{A}}}
\def\etB{{\etens{B}}}
\def\etC{{\etens{C}}}
\def\etD{{\etens{D}}}
\def\etE{{\etens{E}}}
\def\etF{{\etens{F}}}
\def\etG{{\etens{G}}}
\def\etH{{\etens{H}}}
\def\etI{{\etens{I}}}
\def\etJ{{\etens{J}}}
\def\etK{{\etens{K}}}
\def\etL{{\etens{L}}}
\def\etM{{\etens{M}}}
\def\etN{{\etens{N}}}
\def\etO{{\etens{O}}}
\def\etP{{\etens{P}}}
\def\etQ{{\etens{Q}}}
\def\etR{{\etens{R}}}
\def\etS{{\etens{S}}}
\def\etT{{\etens{T}}}
\def\etU{{\etens{U}}}
\def\etV{{\etens{V}}}
\def\etW{{\etens{W}}}
\def\etX{{\etens{X}}}
\def\etY{{\etens{Y}}}
\def\etZ{{\etens{Z}}}

% The true underlying data generating distribution
\newcommand{\pdata}{p_{\rm{data}}}
\newcommand{\ptarget}{p_{\rm{target}}}
\newcommand{\pprior}{p_{\rm{prior}}}
\newcommand{\pbase}{p_{\rm{base}}}
\newcommand{\pref}{p_{\rm{ref}}}

% The empirical distribution defined by the training set
\newcommand{\ptrain}{\hat{p}_{\rm{data}}}
\newcommand{\Ptrain}{\hat{P}_{\rm{data}}}
% The model distribution
\newcommand{\pmodel}{p_{\rm{model}}}
\newcommand{\Pmodel}{P_{\rm{model}}}
\newcommand{\ptildemodel}{\tilde{p}_{\rm{model}}}
% Stochastic autoencoder distributions
\newcommand{\pencode}{p_{\rm{encoder}}}
\newcommand{\pdecode}{p_{\rm{decoder}}}
\newcommand{\precons}{p_{\rm{reconstruct}}}

\newcommand{\laplace}{\mathrm{Laplace}} % Laplace distribution

\newcommand{\E}{\mathbb{E}}
\newcommand{\Ls}{\mathcal{L}}
\newcommand{\R}{\mathbb{R}}
\newcommand{\emp}{\tilde{p}}
\newcommand{\lr}{\alpha}
\newcommand{\reg}{\lambda}
\newcommand{\rect}{\mathrm{rectifier}}
\newcommand{\softmax}{\mathrm{softmax}}
\newcommand{\sigmoid}{\sigma}
\newcommand{\softplus}{\zeta}
\newcommand{\KL}{D_{\mathrm{KL}}}
\newcommand{\Var}{\mathrm{Var}}
\newcommand{\standarderror}{\mathrm{SE}}
\newcommand{\Cov}{\mathrm{Cov}}
% Wolfram Mathworld says $L^2$ is for function spaces and $\ell^2$ is for vectors
% But then they seem to use $L^2$ for vectors throughout the site, and so does
% wikipedia.
\newcommand{\normlzero}{L^0}
\newcommand{\normlone}{L^1}
\newcommand{\normltwo}{L^2}
\newcommand{\normlp}{L^p}
\newcommand{\normmax}{L^\infty}

\newcommand{\parents}{Pa} % See usage in notation.tex. Chosen to match Daphne's book.

\DeclareMathOperator*{\argmax}{arg\,max}
\DeclareMathOperator*{\argmin}{arg\,min}

\DeclareMathOperator{\sign}{sign}
\DeclareMathOperator{\Tr}{Tr}
\let\ab\allowbreak


\newcommand{\rylan}[1]{\textcolor{red}{RS: #1}}
\newcommand{\aengus}[1]{\textcolor{cyan}{Aengus: #1}}
\newcommand{\josh}[1]{\textcolor{blue}{JK: #1}}

% The \icmltitle you define below is probably too long as a header.
% Therefore, a short form for the running title is supplied here:
\icmltitlerunning{How Do Large Language Monkeys Get Their Power (Laws)?}

\begin{document}

\twocolumn[
\icmltitle{
How Do Large Language Monkeys Get Their Power (Laws)?
% Title 2: The Origin of Power Law Scaling by Repeatedly Sampling from Neural Language Models\\
% Title 3: When and Why Repeatedly Sampling from\\Neural Language Models Yields Power Law Scaling
%The Origin of Power Laws in Scaling Inference Compute via Repeat Sampling
}

% It is OKAY to include author information, even for blind
% submissions: the style file will automatically remove it for you
% unless you've provided the [accepted] option to the icml2025
% package.

% List of affiliations: The first argument should be a (short)
% identifier you will use later to specify author affiliations
% Academic affiliations should list Department, University, City, Region, Country
% Industry affiliations should list Company, City, Region, Country

% You can specify symbols, otherwise they are numbered in order.
% Ideally, you should not use this facility. Affiliations will be numbered
% in order of appearance and this is the preferred way.
\icmlsetsymbol{equal}{*}

\begin{icmlauthorlist}
\icmlauthor{Rylan Schaeffer}{stanfordcs}
\icmlauthor{Joshua Kazdan}{stanfordstats}
\icmlauthor{John Hughes}{speechmatics,mats}
\icmlauthor{Jordan Juravsky}{stanfordcs}
\icmlauthor{Sara Price}{mats}
\icmlauthor{Aengus Lynch}{mats,ucl}
\icmlauthor{Erik Jones}{anthropic}
\icmlauthor{Robert Kirk}{ucl}
\icmlauthor{Azalia Mirhoseini}{stanfordcs}
\icmlauthor{Sanmi Koyejo}{stanfordcs}
\end{icmlauthorlist}

\icmlaffiliation{stanfordcs}{Stanford Computer Science}
\icmlaffiliation{stanfordstats}{Stanford Statistics}
\icmlaffiliation{speechmatics}{Speechmatics}
\icmlaffiliation{ucl}{University College London}
\icmlaffiliation{anthropic}{Anthropic}
% \icmlaffiliation{comp}{Company Name, Location, Country}
% \icmlaffiliation{sch}{School of ZZZ, Institute of WWW, Location, Country}

\icmlcorrespondingauthor{Rylan Schaeffer}{rschaef@cs.stanford.edu}
\icmlcorrespondingauthor{Sanmi Koyejo}{sanmi@cs.stanford.edu}

% You may provide any keywords that you
% find helpful for describing your paper; these are used to populate
% the "keywords" metadata in the PDF but will not be shown in the document
\icmlkeywords{Machine Learning, ICML}

\vskip 0.3in
]


% this must go after the closing bracket ] following \twocolumn[ ...

% This command actually creates the footnote in the first column
% listing the affiliations and the copyright notice.
% The command takes one argument, which is text to display at the start of the footnote.
% The \icmlEqualContribution command is standard text for equal contribution.
% Remove it (just {}) if you do not need this facility.

%\printAffiliationsAndNotice{}  % leave blank if no need to mention equal contribution
% \printAffiliationsAndNotice{\icmlEqualContribution} % otherwise use the standard text.


% Remaining TODOs:
% 4. How does sample efficiency compare?

\begin{abstract}
Recent research across mathematical problem solving, proof assistant programming and multimodal jailbreaking documents a striking finding: when (multimodal) language model tackle a suite of tasks with multiple attempts per task -- succeeding if any attempt is correct -- then the negative log of the average success rate scales a power law in the number of attempts.
In this work, we identify an apparent puzzle: a simple mathematical calculation predicts that on each problem, the failure rate should fall exponentially with the number of attempts.
We confirm this prediction empirically, raising a question: from where does aggregate polynomial scaling emerge?
We then answer this question by demonstrating per-problem exponential scaling can be made consistent with aggregate polynomial scaling if the distribution of single-attempt success probabilities is heavy tailed such that a small fraction of tasks with extremely low success probabilities collectively warp the aggregate success trend into a power law - even as each problem scales exponentially on its own.
We further demonstrate that this distributional perspective explains previously observed deviations from power law scaling, and provides a simple method for forecasting the power law exponent with an order of magnitude lower relative error, or equivalently, ${\sim}2-4$ orders of magnitude less inference compute.
Overall, our work contributes to a better understanding of how neural language model performance improves with scaling inference compute and the development of scaling-predictable evaluations of (multimodal) language models.
\end{abstract}


\section{Introduction}
\label{section:introduction}

% redirection is unique and important in VR
Virtual Reality (VR) systems enable users to embody virtual avatars by mirroring their physical movements and aligning their perspective with virtual avatars' in real time. 
As the head-mounted displays (HMDs) block direct visual access to the physical world, users primarily rely on visual feedback from the virtual environment and integrate it with proprioceptive cues to control the avatar’s movements and interact within the VR space.
Since human perception is heavily influenced by visual input~\cite{gibson1933adaptation}, 
VR systems have the unique capability to control users' perception of the virtual environment and avatars by manipulating the visual information presented to them.
Leveraging this, various redirection techniques have been proposed to enable novel VR interactions, 
such as redirecting users' walking paths~\cite{razzaque2005redirected, suma2012impossible, steinicke2009estimation},
modifying reaching movements~\cite{gonzalez2022model, azmandian2016haptic, cheng2017sparse, feick2021visuo},
and conveying haptic information through visual feedback to create pseudo-haptic effects~\cite{samad2019pseudo, dominjon2005influence, lecuyer2009simulating}.
Such redirection techniques enable these interactions by manipulating the alignment between users' physical movements and their virtual avatar's actions.

% % what is hand/arm redirection, motivation of study arm-offset
% \change{\yj{i don't understand the purpose of this paragraph}
% These illusion-based techniques provide users with unique experiences in virtual environments that differ from the physical world yet maintain an immersive experience. 
% A key example is hand redirection, which shifts the virtual hand’s position away from the real hand as the user moves to enhance ergonomics during interaction~\cite{feuchtner2018ownershift, wentzel2020improving} and improve interaction performance~\cite{montano2017erg, poupyrev1996go}. 
% To increase the realism of virtual movements and strengthen the user’s sense of embodiment, hand redirection techniques often incorporate a complete virtual arm or full body alongside the redirected virtual hand, using inverse kinematics~\cite{hartfill2021analysis, ponton2024stretch} or adjustments to the virtual arm's movement as well~\cite{li2022modeling, feick2024impact}.
% }

% noticeability, motivation of predicting a probability, not a classification
However, these redirection techniques are most effective when the manipulation remains undetected~\cite{gonzalez2017model, li2022modeling}. 
If the redirection becomes too large, the user may not mitigate the conflict between the visual sensory input (redirected virtual movement) and their proprioception (actual physical movement), potentially leading to a loss of embodiment with the virtual avatar and making it difficult for the user to accurately control virtual movements to complete interaction tasks~\cite{li2022modeling, wentzel2020improving, feuchtner2018ownershift}. 
While proprioception is not absolute, users only have a general sense of their physical movements and the likelihood that they notice the redirection is probabilistic. 
This probability of detecting the redirection is referred to as \textbf{noticeability}~\cite{li2022modeling, zenner2024beyond, zenner2023detectability} and is typically estimated based on the frequency with which users detect the manipulation across multiple trials.

% version B
% Prior research has explored factors influencing the noticeability of redirected motion, including the redirection's magnitude~\cite{wentzel2020improving, poupyrev1996go}, direction~\cite{li2022modeling, feuchtner2018ownershift}, and the visual characteristics of the virtual avatar~\cite{ogawa2020effect, feick2024impact}.
% While these factors focus on the avatars, the surrounding virtual environment can also influence the users' behavior and in turn affect the noticeability of redirection.
% One such prominent external influence is through the visual channel - the users' visual attention is constantly distracted by complex visual effects and events in practical VR scenarios.
% Although some prior studies have explored how to leverage user blindness caused by visual distractions to redirect users' virtual hand~\cite{zenner2023detectability}, there remains a gap in understanding how to quantify the noticeability of redirection under visual distractions.

% visual stimuli and gaze behavior
Prior research has explored factors influencing the noticeability of redirected motion, including the redirection's magnitude~\cite{wentzel2020improving, poupyrev1996go}, direction~\cite{li2022modeling, feuchtner2018ownershift}, and the visual characteristics of the virtual avatar~\cite{ogawa2020effect, feick2024impact}.
While these factors focus on the avatars, the surrounding virtual environment can also influence the users' behavior and in turn affect the noticeability of redirection.
This, however, remains underexplored.
One such prominent external influence is through the visual channel - the users' visual attention is constantly distracted by complex visual effects and events in practical VR scenarios.
We thus want to investigate how \textbf{visual stimuli in the virtual environment} affect the noticeability of redirection.
With this, we hope to complement existing works that focus on avatars by incorporating environmental visual influences to enable more accurate control over the noticeability of redirected motions in practical VR scenarios.
% However, in realistic VR applications, the virtual environment often contains complex visual effects beyond the virtual avatar itself. 
% We argue that these visual effects can \textbf{distract users’ visual attention and thus affect the noticeability of redirection offsets}, while current research has yet taken into account.
% For instance, in a VR boxing scenario, a user’s visual attention is likely focused on their opponent rather than on their virtual body, leading to a lower noticeability of redirection offsets on their virtual movements. 
% Conversely, when reaching for an object in the center of their field of view, the user’s attention is more concentrated on the virtual hand’s movement and position to ensure successful interaction, resulting in a higher noticeability of offsets.

Since each visual event is a complex choreography of many underlying factors (type of visual effect, location, duration, etc.), it is extremely difficult to quantify or parameterize visual stimuli.
Furthermore, individuals respond differently to even the same visual events.
Prior neuroscience studies revealed that factors like age, gender, and personality can influence how quickly someone reacts to visual events~\cite{gillon2024responses, gale1997human}. 
Therefore, aiming to model visual stimuli in a way that is generalizable and applicable to different stimuli and users, we propose to use users' \textbf{gaze behavior} as an indicator of how they respond to visual stimuli.
In this paper, we used various gaze behaviors, including gaze location, saccades~\cite{krejtz2018eye}, fixations~\cite{perkhofer2019using}, and the Index of Pupil Activity (IPA)~\cite{duchowski2018index}.
These behaviors indicate both where users are looking and their cognitive activity, as looking at something does not necessarily mean they are attending to it.
Our goal is to investigate how these gaze behaviors stimulated by various visual stimuli relate to the noticeability of redirection.
With this, we contribute a model that allows designers and content creators to adjust the redirection in real-time responding to dynamic visual events in VR.

To achieve this, we conducted user studies to collect users' noticeability of redirection under various visual stimuli.
To simulate realistic VR scenarios, we adopted a dual-task design in which the participants performed redirected movements while monitoring the visual stimuli.
Specifically, participants' primary task was to report if they noticed an offset between the avatar's movement and their own, while their secondary task was to monitor and report the visual stimuli.
As realistic virtual environments often contain complex visual effects, we started with simple and controlled visual stimulus to manage the influencing factors.

% first user study, confirmation study
% collect data under no visual stimuli, different basic visual stimuli
We first conducted a confirmation study (N=16) to test whether applying visual stimuli (opacity-based) actually affects their noticeability of redirection. 
The results showed that participants were significantly less likely to detect the redirection when visual stimuli was presented $(F_{(1,15)}=5.90,~p=0.03)$.
Furthermore, by analyzing the collected gaze data, results revealed a correlation between the proposed gaze behaviors and the noticeability results $(r=-0.43)$, confirming that the gaze behaviors could be leveraged to compute the noticeability.

% data collection study
We then conducted a data collection study to obtain more accurate noticeability results through repeated measurements to better model the relationship between visual stimuli-triggered gaze behaviors and noticeability of redirection.
With the collected data, we analyzed various numerical features from the gaze behaviors to identify the most effective ones. 
We tested combinations of these features to determine the most effective one for predicting noticeability under visual stimuli.
Using the selected features, our regression model achieved a mean squared error (MSE) of 0.011 through leave-one-user-out cross-validation. 
Furthermore, we developed both a binary and a three-class classification model to categorize noticeability, which achieved an accuracy of 91.74\% and 85.62\%, respectively.

% evaluation study
To evaluate the generalizability of the regression model, we conducted an evaluation study (N=24) to test whether the model could accurately predict noticeability with new visual stimuli (color- and scale-based animations).
Specifically, we evaluated whether the model's predictions aligned with participants' responses under these unseen stimuli.
The results showed that our model accurately estimated the noticeability, achieving mean squared errors (MSE) of 0.014 and 0.012 for the color- and scale-based visual stimili, respectively, compared to participants' responses.
Since the tested visual stimuli data were not included in the training, the results suggested that the extracted gaze behavior features capture a generalizable pattern and can effectively indicate the corresponding impact on the noticeability of redirection.

% application
Based on our model, we implemented an adaptive redirection technique and demonstrated it through two applications: adaptive VR action game and opportunistic rendering.
We conducted a proof-of-concept user study (N=8) to compare our adaptive redirection technique with a static redirection, evaluating the usability and benefits of our adaptive redirection technique.
The results indicated that participants experienced less physical demand and stronger sense of embodiment and agency when using the adaptive redirection technique. 
These results demonstrated the effectiveness and usability of our model.

In summary, we make the following contributions.
% 
\begin{itemize}
    \item 
    We propose to use users' gaze behavior as a medium to quantify how visual stimuli influences the noticebility of redirection. 
    Through two user studies, we confirm that visual stimuli significantly influences noticeability and identify key gaze behavior features that are closely related to this impact.
    \item 
    We build a regression model that takes the user's gaze behavioral data as input, then computes the noticeability of redirection.
    Through an evaluation study, we verify that our model can estimate the noticeability with new participants under unseen visual stimuli.
    These findings suggest that the extracted gaze behavior features effectively capture the influence of visual stimuli on noticeability and can generalize across different users and visual stimuli.
    \item 
    We develop an adaptive redirection technique based on our regression model and implement two applications with it.
    With a proof-of-concept study, we demonstrate the effectiveness and potential usability of our regression model on real-world use cases.

\end{itemize}

% \delete{
% Virtual Reality (VR) allows the user to embody a virtual avatar by mirroring their physical movements through the avatar.
% As the user's visual access to the physical world is blocked in tasks involving motion control, they heavily rely on the visual representation of the avatar's motions to guide their proprioception.
% Similar to real-world experiences, the user is able to resolve conflicts between different sensory inputs (e.g., vision and motor control) through multisensory integration, which is essential for mitigating the sensory noise that commonly arises.
% However, it also enables unique manipulations in VR, as the system can intentionally modify the avatar's movements in relation to the user's motions to achieve specific functional outcomes,
% for example, 
% % the manipulations on the avatar's movements can 
% enabling novel interaction techniques of redirected walking~\cite{razzaque2005redirected}, redirected reaching~\cite{gonzalez2022model}, and pseudo haptics~\cite{samad2019pseudo}.
% With small adjustments to the avatar's movements, the user can maintain their sense of embodiment, due to their ability to resolve the perceptual differences.
% % However, a large mismatch between the user and avatar's movements can result in the user losing their sense of embodiment, due to an inability to resolve the perceptual differences.
% }

% \delete{
% However, multisensory integration can break when the manipulation is so intense that the user is aware of the existence of the motion offset and no longer maintains the sense of embodiment.
% Prior research studied the intensity threshold of the offset applied on the avatar's hand, beyond which the embodiment will break~\cite{li2022modeling}. 
% Studies also investigated the user's sensitivity to the offsets over time~\cite{kohm2022sensitivity}.
% Based on the findings, we argue that one crucial factor that affects to what extent the user notices the offset (i.e., \textit{noticeability}) that remains under-explored is whether the user directs their visual attention towards or away from the virtual avatar.
% Related work (e.g., Mise-unseen~\cite{marwecki2019mise}) has showcased applications where adjustments in the environment can be made in an unnoticeable manner when they happen in the area out of the user's visual field.
% We hypothesize that directing the user's visual attention away from the avatar's body, while still partially keeping the avatar within the user's field-of-view, can reduce the noticeability of the offset.
% Therefore, we conduct two user studies and implement a regression model to systematically investigate this effect.
% }

% \delete{
% In the first user study (N = 16), we test whether drawing the user's visual attention away from their body impacts the possibility of them noticing an offset that we apply to their arm motion in VR.
% We adopt a dual-task design to enable the alteration of the user's visual attention and a yes/no paradigm to measure the noticeability of motion offset. 
% The primary task for the user is to perform an arm motion and report when they perceive an offset between the avatar's virtual arm and their real arm.
% In the secondary task, we randomly render a visual animation of a ball turning from transparent to red and becoming transparent again and ask them to monitor and report when it appears.
% We control the strength of the visual stimuli by changing the duration and location of the animation.
% % By changing the time duration and location of the visual animation, we control the strengths of attraction to the users.
% As a result, we found significant differences in the noticeability of the offsets $(F_{(1,15)}=5.90,~p=0.03)$ between conditions with and without visual stimuli.
% Based on further analysis, we also identified the behavioral patterns of the user's gaze (including pupil dilation, fixations, and saccades) to be correlated with the noticeability results $(r=-0.43)$ and they may potentially serve as indicators of noticeability.
% }

% \delete{
% To further investigate how visual attention influences the noticeability, we conduct a data collection study (N = 12) and build a regression model based on the data.
% The regression model is able to calculate the noticeability of the offset applied on the user's arm under various visual stimuli based on their gaze behaviors.
% Our leave-one-out cross-validation results show that the proposed method was able to achieve a mean-squared error (MSE) of 0.012 in the probability regression task.
% }

% \delete{
% To verify the feasibility and extendability of the regression model, we conduct an evaluation study where we test new visual animations based on adjustments on scale and color and invite 24 new participants to attend the study.
% Results show that the proposed method can accurately estimate the noticeability with an MSE of 0.014 and 0.012 in the conditions of the color- and scale-based visual effects.
% Since these animations were not included in the dataset that the regression model was built on, the study demonstrates that the gaze behavioral features we extracted from the data capture a generalizable pattern of the user's visual attention and can indicate the corresponding impact on the noticeability of the offset.
% }

% \delete{
% Finally, we demonstrate applications that can benefit from the noticeability prediction model, including adaptive motion offsets and opportunistic rendering, considering the user's visual attention. 
% We conclude with discussions of our work's limitations and future research directions.
% }

% \delete{
% In summary, we make the following contributions.
% }
% % 
% \begin{itemize}
%     \item 
%     \delete{
%     We quantify the effects of the user's visual attention directed away by stimuli on their noticeability of an offset applied to the avatar's arm motion with respect to the user's physical arm. 
%     Through two user studies, we identified gaze behavioral features that are indicative of the changes in noticeability.
%     }
%     \item 
%     \delete{We build a regression model that takes the user's gaze behavioral data and the offset applied to the arm motion as input, then computes the probability of the user noticing the offset.
%     Through an evaluation study, we verified that the model needs no information about the source attracting the user's visual attention and can be generalizable in different scenarios.
%     }
%     \item 
%     \delete{We demonstrate two applications that potentially benefit from the regression model, including adaptive motion offsets and opportunistic rendering.
%     }

% \end{itemize}

\begin{comment}
However, users will lose the sense of embodiment to the virtual avatars if they notice the offset between the virtual and physical movements.
To address this, researchers have been exploring the noticing threshold of offsets with various magnitudes and proposing various redirection techniques that maintain the sense of embodiment~\cite{}.

However, when users embody virtual avatars to explore virtual environments, they encounter various visual effects and content that can attract their attention~\cite{}.
During this, the user may notice an offset when he observes the virtual movement carefully while ignoring it when the virtual contents attract his attention from the movements.
Therefore, static offset thresholds are not appropriate in dynamic scenarios.

Past research has proposed dynamic mapping techniques that adapted to users' state, such as hand moving speed~\cite{frees2007prism} or ergonomically comfortable poses~\cite{montano2017erg}, but not considering the influence of virtual content.
More specifically, PRISM~\cite{frees2007prism} proposed adjusting the C/D ratio with a non-linear mapping according to users' hand moving speed, but it might not be optimal for various virtual scenarios.
While Erg-O~\cite{montano2017erg} redirected users' virtual hands according to the virtual target's relative position to reduce physical fatigue, neglecting the change of virtual environments. 

Therefore, how to design redirection techniques in various scenarios with different visual attractions remains unknown.
To address this, we investigate how visual attention affects the noticing probability of movement offsets.
Based on our experiments, we implement a computational model that automatically computes the noticing probability of offsets under certain visual attractions.
VR application designers and developers can easily leverage our model to design redirection techniques maintaining the sense of embodiment adapt to the user's visual attention.
We implement a dynamic redirection technique with our model and demonstrate that it effectively reduces the target reaching time without reducing the sense of embodiment compared to static redirection techniques.

% Need to be refined
This paper offers the following contributions.
\begin{itemize}
    \item We investigate how visual attractions affect the noticing probability of redirection offsets.
    \item We construct a computational model to predict the noticing probability of an offset with a given visual background.
    \item We implement a dynamic redirection technique adapting to the visual background. We evaluate the technique and develop three applications to demonstrate the benefits. 
\end{itemize}



First, we conducted a controlled experiment to understand how users perceived the movement offset while subjected to various distractions.
Since hand redirection is one of the most frequently used redirections in VR interactions, we focused on the dynamic arm movements and manually added angular offsets to the' elbow joint~\cite{li2022modeling, gonzalez2022model, zenner2019estimating}. 
We employed flashing spheres in the user's field of view as distractions to attract users' visual attention.
Participants were instructed to report the appearing location of the spheres while simultaneously performing the arm movements and reporting if they perceived an offset during the movement. 
(\zhipeng{Add the results of data collection. Analyze the influence of the distance between the gaze map and the offset.}
We measured the visual attraction's magnitude with the gaze distribution on it.
Results showed that stronger distractions made it harder for users to notice the offset.)
\zhipeng{Need to rewrite. Not sure to use gaze distribution or a metric obtained from the visual content.}
Secondly, we constructed a computational model to predict the noticing probability of offsets with given visual content.
We analyzed the data from the user studies to measure the influence of visual attractions on the noticing probability of offsets.
We built a statistical model to predict the offset's noticing probability with a given visual content.
Based on the model, we implement a dynamic redirection technique to adjust the redirection offset adapted to the user's current field of view.
We evaluated the technique in a target selection task compared to no hand redirection and static hand redirection.
\zhipeng{Add the results of the evaluation.}
Results showed that the dynamic hand redirection technique significantly reduced the target selection time with similar accuracy and a comparable sense of embodiment.
Finally, we implemented three applications to demonstrate the potential benefits of the visual attention adapted dynamic redirection technique.
\end{comment}

% This one modifies arm length, not redirection
% \citeauthor{mcintosh2020iteratively} proposed an adaptation method to iteratively change the virtual avatar arm's length based on the primary tasks' performance~\cite{mcintosh2020iteratively}.



% \zhipeng{TO ADD: what is redirection}
% Redirection enables novel interactions in Virtual Reality, including redirected walking, haptic redirection, and pseudo haptics by introducing an offset to users' movement.
% \zhipeng{TO ADD: extend this sentence}
% The price of this is that users' immersiveness and embodiment in VR can be compromised when they notice the offset and perceive the virtual movement not as theirs~\cite{}.
% \zhipeng{TO ADD: extend this sentence, elaborate how the virtual environment attracts users' attention}
% Meanwhile, the visual content in the virtual environment is abundant and consistently captures users' attention, making it harder to notice the offset~\cite{}.
% While previous studies explored the noticing threshold of the offsets and optimized the redirection techniques to maintain the sense of embodiment~\cite{}, the influence of visual content on the probability of perceiving offsets remains unknown.  
% Therefore, we propose to investigate how users perceive the redirection offset when they are facing various visual attractions.


% We conducted a user study to understand how users notice the shift with visual attractions.
% We used a color-changing ball to attract the user's attention while instructing users to perform different poses with their arms and observe it meanwhile.
% \zhipeng{(Which one should be the primary task? Observe the ball should be the primary one, but if the primary task is too simple, users might allocate more attention on the secondary task and this makes the secondary task primary.)}
% \zhipeng{(We need a good and reasonable dual-task design in which users care about both their pose and the visual content, at least in the evaluation study. And we need to be able to control the visual content's magnitude and saliency maybe?)}
% We controlled the shift magnitude and direction, the user's pose, the ball's size, and the color range.
% We set the ball's color-changing interval as the independent factor.
% We collect the user's response to each shift and the color-changing times.
% Based on the collected data, we constructed a statistical model to describe the influence of visual attraction on the noticing probability.
% \zhipeng{(Are we actually controlling the attention allocation? How do we measure the attracting effect? We need uniform metrics, otherwise it is also hard for others to use our knowledge.)}
% \zhipeng{(Try to use eye gaze? The eye gaze distribution in the last five seconds to decide the attention allocation? Basically constructing a model with eye gaze distribution and noticing probability. But the user's head is moving, so the eye gaze distribution is not aligned well with the current view.)}

% \zhipeng{Saliency and EMD}
% \zhipeng{Gaze is more than just a point: Rethinking visual attention
% analysis using peripheral vision-based gaze mapping}

% Evaluation study(ideal case): based on the visual content, adjusting the redirection magnitude dynamically.

% \zhipeng{(The risk is our model's effect is trivial.)}

% Applications:
% Playing Lego while watching demo videos, we can accelerate the reaching process of bricks, and forbid the redirection during the manipulation.

% Beat saber again: but not make a lot of sense? Difficult game has complicated visual effects, while allows larger shift, but do not need large shift with high difficulty





\section{Should Power Law Scaling Be Expected?}
\label{sec:should_power_laws_be_expected}

Should we expect large language monkeys to have such power  (laws)? That is, should the negative log of the average success rate scale polynomially with the number of independent attempts $k$? As we now explain mathematically and demonstrate empirically, such polynomial scaling with $k$ is perhaps surprising because, for any single problem, the negative log success rate at $k$ should fall exponentially with $k$; the intuition is that $\operatorname{pass_i@k}$ is 1 unless \textit{all} attempts fail, and since attempts are independent, the probability that all fail is exponentially unlikely with the number of attempts.


\begin{figure*}[t!]
    \centering
    \includegraphics[width=\linewidth]{figures/02_large_language_monkeys_original_eda/y=neg_log_score_vs_x=scaling_parameter_hue=model_col=model_units=problem_idx.pdf}
    \includegraphics[width=\linewidth]{figures/00_bon_jailbreaking_eda/y=neg_log_score_vs_x=scaling_parameter_hue=model_col=model_units=problem_idx.pdf}
    \caption{\textbf{Per-problem performance scales exponentially with the number of attempts per problem $k$}.     
    Top: Pythia language models on 128 problems from MATH, with performance on the $i$-th problem measured as $-\log(\operatorname{pass_i@k})$. Bottom: Frontier AI models on jailbreaking prompts from HarmBench, with performance on the $i$-th problem measured as $-\log(\operatorname{ASR_i@k})$. In both settings, on each problem, the negative log \textit{per-problem} success rate falls exponentially with the number of independent attempts $k$. However, the negative log \textit{average} success rate falls as a power law with $k$ (black).}
    \label{fig:multiple_attempts_scaling_per_datum}
\end{figure*}


Mathematically, on any given attempt, the model has probability $\operatorname{pass_i@1}$ of solving the $i$-th problem.
% If we draw $k$ independent attempts from the model, how should the expected (over attempts) $\operatorname{pass_i@k}$ to scale with $k$?
Recalling that $\operatorname{pass_i@k}$ is defined as $1$ if \textit{any} of the $k$ attempts succeed, 0 otherwise, by linearity of expectation and by independence of the $k$ attempts, we can rewrite $\operatorname{pass_i@k}$ as:
%
\begin{align}
    \operatorname{pass_i@k} &= \mathop{\raisebox{3pt}{$\mathbb{E}$}}_{\substack{k \text{ Attempts}}}\Big[1 - \mathbb{I}[\text{All $k$ Attempts Fail}] \Big]\\
    &= 1 - \prod_{j=1}^k \mathop{\raisebox{3pt}{$\mathbb{E}$}}_{\substack{1 \text{ Attempt}}}\Big[ \mathbb{I}[\text{$j$-th Attempt Fails}] \Big].
\end{align}

The probability that the $j$-th attempt fails is one minus the probability that the $j$-th attempt succeeds. Since each attempt is i.i.d. with success probability $\operatorname{pass_i@1}$, we find
%
\begin{align}
    \operatorname{pass_i@k}
    &= 1 - (1 - \operatorname{pass_i@1})^k.
\end{align}

For large $k$, $(1 - \operatorname{pass_i@1})^k$ will be small. Recalling that the Taylor Series expansion of $\log (1 + x)$ for small $x$ is $\sum_{i=1}^{\infty} (-1)^{i-1} x^i / i \approx x$, we have:
%
\begin{align}
    -\log (\operatorname{pass_i@k} )
    &= - \log \Big(1 - (1 - \operatorname{pass@1})^k \Big)\\
    &\approx (1 - \operatorname{pass_i@1})^k.
\end{align}


\begin{figure*}[t!]
    \centering
    \includegraphics[width=\linewidth]{figures/50_pass_at_1_fits/llmonkey_y=counts_x=score_hue=model_col=model_bins=custom.pdf}
    \includegraphics[width=\linewidth]{figures/50_pass_at_1_fits/bon_jailbreaking_y=counts_x=score_hue=model_col=model_bins=custom.pdf}
    \caption{\textbf{Single-Attempt Success Rates  Distributions Possess Power Law-Like Left Tails.} Pythia language models on 128 MATH problems (top) and frontier AI systems on 159 HarmBench prompts (bottom) exhibit distributions (over problems) of $\operatorname{pass_i@1}$ and $\operatorname{ASR_i@1}$ with power law-like tails that are well fit by scaled Beta-Binomial distributions (black dashed lines), which produce aggregate power law scaling. Note that Llama 3 8B Instruction Tuned (IT) does not possess a power law tail, explaining why the model did not exhibit aggregate power law scaling under Best-of-N jailbreaking (Sec.~\ref{sec:no_dist_structure_no_power_law}).}
    \label{fig:multiple_attempts_pass_at_1_per_datum}
\end{figure*}


Thus, \textit{for any single problem}, we should expect the negative log expected (over attempts) success rate to fall \textit{exponentially} with $k$, not polynomially with $k$. % (i.e., as a power law).
% Intuitively, this is because $\operatorname{pass@k}$ is 1 unless all $k$ i.i.d. attempts fail and the probability that all $k$ attempts fail is exponentially unlikely with $k$.


To confirm this claim, we plotted the scaling of model performance on each problem -- measured either by $-\log(\operatorname{pass_i@k})$ or by $-\log(\operatorname{ASR_i@k})$ -- against the number of independent attempts $k$. We specifically used \citet{brown2024largelanguagemonkeysscaling}'s data of the Pythia language model family \citep{biderman2023pythia} solving 128 mathematical problems from MATH \citet{hendrycks2021measuring} as well as \citet{hughes2024bestofnjailbreaking}'s data from jailbreaking frontier AI systems -- Claude, GPT4 \citep{openai2024gpt4technicalreport}, Gemini \citep{anil2024geminifamilyhighlycapable,georgievgemini15unlockingmultimodal} and Llama 3 8B Instruction Tuned (IT) \citep{grattafiori2024llama3herdmodels} -- on 159 prompts from HarmBench \citep{mazeika2024harmbenchstandardizedevaluationframework}.
For each individual mathematical problem and jailbreaking prompt, we found the negative log expected (over attempts) success rates fall exponentially with $k$ as expected (Fig. \ref{fig:multiple_attempts_scaling_per_datum}), including on Llama 3 8B IT which does not exhibit an aggregate power law (Fig.~\ref{fig:power_laws_repeat_sampling}).

\section{Distribution of Per-Problem Single-Attempt Success Rates Creates Power Law Scaling}
\label{sec:distr_per_problem_success_rates}

How does polynomial scaling of the negative log \textit{average} success rate emerge from exponential scaling of the negative log \textit{per-problem} success rate?
The answer to this question \textit{must} lie in the distribution $\mathcal{D}$ over benchmark problems of single attempt (i.e., $k=1$) success rates because this distribution's density $p_{\mathcal{D}}(\operatorname{pass_i@1})$ links the per-problem scaling behavior to the aggregate scaling behavior via the definition of the aggregate success rate $\operatorname{pass_{\mathcal{D}}@k}$:
%
\begin{equation}
\begin{aligned}
    &\operatorname{pass_{\mathcal{D}}@k} \; \defeq \; \mathop{\raisebox{3pt}{$\mathbb{E}$}}_{\operatorname{pass_i@1} \sim \mathcal{D}} \Big[\operatorname{pass_i@k}(\operatorname{pass_i@1}) \Big]\\
    % &= 1 - \underbrace{\int_0^1 (1 - \operatorname{pass_i@1})^k \, p_{\mathcal{D}}(\operatorname{pass_i@1}) \, \operatorname{d\,pass_i@1}}_{\defeq \operatorname{fail_{\mathcal{D}}@k}}.
    &= 1 - \int_0^1 (1 - \operatorname{pass_i@1})^k \, p_{\mathcal{D}}(\operatorname{pass_i@1}) \, \operatorname{d\,pass_i@1}.
\end{aligned}
\end{equation}
% %
% For large $k$, $\operatorname{fail_{\mathcal{D}}@k}$ will be small and thus:
% \begin{equation}
%     -\log \Big( \operatorname{pass_{\mathcal{D}}@k} \big) \approx 
% \end{equation}


Based on a known result that power laws can originate from an appropriately weighted sum of exponential functions (Appendix ~\ref{app:sec:power_laws_from_distr_over_exp:background}), we begin by considering simple distributions for the single-attempt success probabilities and asking which yield power law scaling between $-\log(\operatorname{pass_{\mathcal{D}}@k})$ and $k$, as well as what properties of the distributions set the scaling exponent. In Appendices~\ref{app:sec:power_laws_from_distr_over_exp:uniform_distribution}-\ref{app:sec:power_laws_from_distr_over_exp:reciprocal_distribution}, we derive that several simple distributions yield power law scaling with different exponents whereas others do not:
%
\begin{align*}
    -\log \Big( &\operatorname{pass_{\mathrm{Uniform}(0,\, \beta \leq 1)}}@k &\Big) &\propto k^{-1}.\\
    -\log \Big( &\operatorname{pass_{\operatorname{Beta(\alpha, \beta)}}@k} &\Big) &\propto k^{-\alpha}.\\
    -\log \Big( &\operatorname{pass_{\operatorname{Kumaraswamy(\alpha,\, \beta)}}@k} &\Big) &\propto k^{-\alpha}.\\
    -\log \Big( &\operatorname{pass_{\operatorname{ContinuousBernoulli(\lambda < 1/2)}}@k} &\Big) &\propto k^{-1}.\\
    -\log \Big( &\operatorname{pass_{\operatorname{Reciprocal(0 < \alpha < \beta < 1)}}@k} &\Big) \propto &\frac{(1-\alpha)^k}{k}.
\end{align*}
%
To test this understanding, we examined whether the data of \citet{brown2024largelanguagemonkeysscaling} and \citet{hughes2024bestofnjailbreaking} had per-problem single-attempt success rate distributions that matched one of these simple distributions (Fig.~\ref{fig:multiple_attempts_pass_at_1_per_datum}). We found that the distributions could indeed be well fit by a 3-parameter $\operatorname{Kuamraswamy}(\alpha, \beta, a = 0, c)$ distribution with scale parameter $c$ (Fig.~\ref{fig:multiple_attempts_pass_at_1_per_datum}, black dashed lines); we found the scale parameter was critical to obtain good fits because the standard 2-parameter Kumaraswamy distribution is supported on $(0, 1)$ whereas most single-attempt success distributions have a smaller maximum such as $0.01$ or $0.1$.

More generally, what are the distributional properties that create such power law scaling and that set the specific power law exponent?
As we now show, the negative log average success rate will exhibit power law scaling in $k$ with exponent $b$ if and only if the distribution over problems of single-attempt success probabilities itself behaves like a power law near $0$ with exponent $b-1$:\newline

% If this condition is met, then then $\mathbb{E}_{\operatorname{pass_i@1} \sim \mathcal{D}}[(1 - \operatorname{pass_i@1})^k]$ decays as $k^{-b}$ and $-\log (\operatorname{pass_{\mathcal{D}}@k})$ inherits the same power-law exponent $b$.
% \josh{@Rylan can I swap these theorems out with the corrected ones?  The necessity claim is false, and I don't think we've identified a reasonable necessary condition yet.  We should also change the sufficiency one so that the distributions that you named follow as corollaries, otherwise the theorem is vacuous in the context of the paper.}
% \josh{I like the suspense as you build to the main theorem, but I wonder if it's more sensible to state the general claim first and then discuss the distributions that satisfy the sufficient condition.}
\begin{theorem}[Sufficiency of Power-Law Left Tail in Distribution of Single-Attempt Success Rates]
\label{thm:sufficiency_powerlaw}
Let $\mathcal{D}$ be a probability distribution on $[0,1]$ with PDF $p_{\mathcal{D}}(\operatorname{pass_i@1})$.  
Suppose there exist constants $b > 0$, $C > 0$, $\theta > 0$ and $\delta > 0$ such that, 
for all $0 < \operatorname{pass_i@1} < \delta$, we have 
\[
  p_{\mathcal{D}}(\operatorname{pass_i@1}) \;=\; C \cdot (\operatorname{pass_i@1})^{b-1} \;+\; O\bigl((\operatorname{pass_i@1})^{b-1+\theta}\bigr).
\]
% That is, there is a constant $M>0$ such that
% \[
%   \bigl|\,f(p) - C\,p^{\,b-1}\bigr|
%   \;\le\;
%   M\,p^{\,b-1+\theta}
%   \quad\text{for all }0<p<\delta.
% \]
% Define the aggregate success rate after $k$ attempts by
% \[
%   \operatorname{pass_{\mathcal{D}}@k}
%   \;\;=\;\;
%   \int_0^1 \Bigl[\,1 - (1 - p)^k\Bigr]\,f(p)\,\mathrm{d}p.
% \]
Then, for large $k$,
\[
  -\log\big(\operatorname{pass_{\mathcal{D}}@k}\big)
  \;\sim\;
  C\,\Gamma(b) \;k^{-b}.
\]
\end{theorem}



\begin{theorem}[Necessity of Power-Law Left Tail in Distribution of Single-Attempt Success Rates]
\label{thm:necessity_powerlaw}
Let $\mathcal{D}$ be a distribution over $\operatorname{pass_i@1} \in [0,1]$ with PDF $p_{\mathcal{D}}(\operatorname{pass_i@1})$.
Suppose there exist constants $b > 0$ and $A > 0$ such that for large $k$,
\[
-\log\big(\operatorname{pass_{\mathcal{D}}@k}\big)
\sim 
A\,k^{-b}.
\]
Then, under mild regularity assumptions, the probability density must satisfy
\[
p_{\mathcal{D}}(\operatorname{pass_i@1})
\;\sim\;
\frac{A}{\Gamma(b)} \, (\operatorname{pass_i@1})^{b - 1}
\quad
\text{as } \operatorname{pass_i@1} \to 0^+.
\]
% where $C>0$ is a constant depending on $A$ and $b$.  
% In other words, whenever $-\log\bigl(\operatorname{pass_{\mathcal{D}}@k}\bigr)$ decays like a power law $k^{-b}$, the distribution $\mathcal{D}$ must have a PDF that behaves like $p^{\,b-1}$ near $p=0$ (up to a multiplicative constant).
\end{theorem}

In Fig.~\ref{fig:schematic}, we illustrate this connection schematically.
For proofs, see Appendices \ref{app:sec:power_laws_from_distr_over_exp:sufficiency} and \ref{app:sec:power_laws_from_distr_over_exp:necessity}.
These results clarify that whenever $-\log (\operatorname{pass_{\mathcal{D}}@k} )$ exhibits power-law decay in $k$ with exponent $b$, the distribution over problems of single-attempt success rates \emph{must} have ``polynomial weight'' near $\operatorname{pass_i@1}=0$, i.e.\ $p_{\mathcal{D}}(p) = \Theta(p^{\,b-1})$.

To offer intuition, we know that each problem is being solved by the model (or equivalently, each prompt is jailbreaking the model) exponentially quickly.
If one looks across all problems in the benchmark, some have $\operatorname{pass_i@1}$ so small that they remain unsolved for many, many attempts.
Whether these ``tiny‐$\operatorname{pass_i@1}$" problems still matter at large $k$ depends on how \emph{many} such problems there are.
Polynomial density near $0$ ``piles up" enough hard problems in just the right way such that even though each of those problems is being solved exponentially quickly, the \emph{aggregate} success rate over problems decreases at only a power‐law rate in $k$.
A more succinct mathematical summary is that, for a compound binomial distribution, the lower tail probability controls the upper tail of the marginal survivor function.
\section{Lack of Distributional Structure Explains Deviations from Power Law Scaling}
\label{sec:no_dist_structure_no_power_law}


\begin{figure*}[t!]
    \centering
    \includegraphics[width=0.9\linewidth]{figures/92_schematic_distributional_fitting_attempt2/distributional_fitting_schematic.png}
    \caption{\textbf{Schematic: Two Estimators of Power Law Parameters for Scaling Inference Compute via Repeat Sampling.} (A) Both estimators begin by generating many samples per prompt, then computing the number of successes per prompt. In the standard least squares power law parameter estimator (top), (B) $\operatorname{pass_i@k}$ is estimated for each $i$-th problem at multiple $k$ values, then (C) averaged over problems and fit with linear regression in log-log space.
    In the distributional power law parameter estimator (bottom), (D) a distribution $\mathcal{D}$ is fit to estimates of $\operatorname{pass_i@1}$, then (E) the single-attempt success probability distribution is used to simulate $\operatorname{pass_{\mathcal{D}}@k}$ at arbitrary $k$ values for linear regression in log-log space.}
    \label{fig:schematic2}
\end{figure*}

Notably, previous papers observed that not every model exhibits power law scaling in every setting. To highlight one, \citet{hughes2024bestofnjailbreaking} observed that when jailbreaking Meta's Llama 3 8B Instruction Tuned (IT) model \cite{grattafiori2024llama3herdmodels}, the $-\log (\operatorname{ASR_{\mathcal{D}}@k})$ fell faster than any power law (Fig.~\ref{fig:power_laws_repeat_sampling}), i.e., the $\operatorname{ASR_{\mathcal{D}}@k}$ rose much more quickly than the other frontier AI systems. Based on our mathematical insights and the empirical per-problem single-attempt attack success rates (Fig.~\ref{fig:multiple_attempts_pass_at_1_per_datum}), we can understand why: Llama 3 8B IT could be successfully jailbroken on every prompt within the permitted sampling budget and thus had no heavy left tail necessary to create the aggregate power law scaling.
% Similarly, \citet{brown2024largelanguagemonkeysscaling} found that power law scaling was less apparent on MiniF2F-MATH \citep{zheng2022minif2fcrosssystembenchmarkformal}, a dataset of mathematics problems that have been formalized into Lean, a proof checking programming language. 


\begin{figure*}[t!]
    \centering
    \includegraphics[width=0.5\linewidth]{figures/51_pass_at_1_compare_power_law_with_distributional_fit/llmonkeys_y=scaling_law_exponent_x=distributional_fit_exponent_kumaraswamy_binomial.pdf}%
    \includegraphics[width=0.5\linewidth]{figures/51_pass_at_1_compare_power_law_with_distributional_fit/bon_jailbreaking_y=scaling_law_exponent_x=distributional_fit_exponent_kumaraswamy_binomial.pdf}
    \caption{\textbf{Comparing Estimators of Power Law Exponents.} We compare two estimators of the power law exponent $b$ in $-\log(\operatorname{pass_{\mathcal{D}}@k}) \approx a k^{-b}\;$: (1) the standard least-squares estimator between $k$ and $-\log(\operatorname{pass_{\mathcal{D}}@k})$ in log-log space, and (2) the distributional estimator of $\operatorname{pass_i@1}$ assuming a scaled Kumaraswamy-Binomial distribution. Using all available data to fit both estimators, we find agreement between the least-squares estimate (ordinate) and the distribution-derived estimate (abscissa) for both Pythia models on MATH (left) and for frontier AI systems on HarmBench (right). For an explanation of why the two estimators match more closely for Large Language Monkeys than for Best-of-N Jailbreaking, see Appendix~\ref{app:sec:clarification_of_data_sampling}.
    }
    \label{fig:comparison_power_law_exponents}
\end{figure*}



% Our previous results \josh{TODO: Finish}
\begin{figure*}[t!]
    \centering
    % \includegraphics[width=\linewidth]{figures/52_compare_power_law_exponent_estimators_synthetic_data/y=relative_error_x=n_hue=distribution_params_col=distribution.pdf}
    \includegraphics[width=0.95\linewidth]{figures/52_compare_power_law_exponent_estimators_synthetic_data/y=relative_error_least_squares_x=n_hue=distribution_params_col=distribution.pdf}
    \caption{\textbf{Comparing Two Estimators of Power Law Exponents via Backtesting.} On synthetic data with known ground-truth power law $a \, k^{-b}$, we compare how well the least squares and the distributional estimator recover the scaling exponent $b$ as measured by the relative error $|\hat{b} - b| / b$ by backtesting: subsampling the number of problems and the number of samples per problem. We find that the distributional estimator obtains significantly better sample efficiency.}
    \label{fig:backtesting}
\end{figure*}


% \begin{figure*}[t!]
%     \centering
%     \includegraphics[width=\linewidth]{figures/52_compare_power_law_exponent_estimators_synthetic_data/y=relative_error_x=n_hue=distribution_params_col=distribution.pdf}
%     \caption{Placeholder}
%     \label{fig:enter-label}
% \end{figure*}



\section{A New Distributional Estimator for Predicting Power Law Scaling}
\label{sec:estimating_power_law_exponent}

A natural consequence of this connection between the scaling of $-\log(\operatorname{pass_{\mathcal{D}}@k})$ and the left tail of the distribution $p_{\mathcal{D}}(\operatorname{pass_i@1})$ is that the distribution of single-attempt success rates can be used to predict whether power-law scaling will appear and if so, what the intercept and exponent of the power law will be. To do this, one can fit the distribution $\hat{p}_{\mathcal{D}}(\operatorname{pass_i@1})$ and then \textit{simulate} how $\operatorname{pass_{\mathcal{D}}@k}$ will scale with $k$ (Fig.~\ref{fig:schematic2}) using the relationship:
%
\begin{equation}
\begin{aligned}
&\widehat{\operatorname{pass_{\mathcal{D}}@k}} \defeq \\
&1 - \int_0^1 (1 - \operatorname{pass_i@1})^k \, \hat{p}_{\mathcal{D}}(\operatorname{pass_i@1}) \, \operatorname{d\,pass_i@1}.
\end{aligned}
\end{equation}
%
To empirically test this claim, we compared the standard least squares regression estimator (in log-log space) \citep{hoffmann2022trainingcomputeoptimallargelanguage,caballero2022broken,besiroglu2024chinchillascalingreplicationattempt} against a \textit{distributional estimator}.
To motivate our distributional estimator, we first need explain a key obstacle and how the distributional estimator overcomes it.
The obstacle is that there are problems or prompts whose single-attempt success probabilities $\operatorname{pass_i@1}$ lie between $(0, 1/\text{Number of Samples})$ such that, due to finite sampling, we lack the resolution to measure.
While we do not know the true single-attempt success probability for the problems that lie in this interval, we \textit{do} know \textit{how many} problems fall into this left tail bucket, and we can fit a distribution's parameters such that the distribution's probability mass in the interval $(0, 1 / \text{Number of Samples})$ matches the empirical fraction of problems in this tail bucket. Thus, our distributional estimator works by first selecting a distribution (e.g., a scaled 3-parameter Beta distribution), discretizing the distribution according to the sampling resolution $1 / \text{Number of Samples}$ and performing maximum likelihood estimation under the discretized distribution's probability mass function.

We tested this distributional estimator in two different ways. First, focusing on Large Language Monkeys, we used all available real data from all problems and all samples per problem to compare the standard least squares regression estimator against the distributional estimator. 
We found close agreement between the two estimators (Fig.~\ref{fig:comparison_power_law_exponents}), giving us a sense that the two estimators yield reasonably consistent estimates under large sampling budgets.

Second, the distributional estimator also comes with another benefit: it directly provides an estimate of the power law's exponent $b$ in $a \, k^{-b}$. Estimating the power law's exponent is especially valuable because the exponent dictates how success rates are improving with increasing inference compute. To test how the distributional estimator and least squares estimator compare at recovering the true asymptotic power law exponent, we generated synthetic data so that we would have ground-truth knowledge of the true power law exponent, then backtested how the two scaling estimators compare at recovering the true exponent \citep{alabdulmohsin2022revisitingneuralscalinglaws, owen2024predicting} by subsampling data with fewer problems and fewer samples per problem.
We found that the distributional estimator obtains significantly better sample efficiency, with approximately an order of magnitude lower relative error $\defeq |\hat{b} - b| / b$ compared with the least squares estimator (Fig.~\ref{fig:backtesting}), or equivalently, ${\sim}2-4$ orders of magnitude less inference-compute. The distributional estimator performs well even under distributional mismatch.
\section{Discussion}
\omniUIST is capable of tracking a passive tool with an accuracy of roughly 6.9 mm and, at the same time, deliver a maximum force of up to 2 N to the tool. This is enabled by our novel gradient-based approach in 3D position reconstruction that accounts for the force exerted by the electromagnet. 

Over extended periods of time, \omniUIST can comfortably produce a force of 0.615 N without the risk of overheating. In our applications, we show that \omniUIST has the potential for a wide range of usage scenarios, specifically to enrich AR and VR interactions.

\omniUIST is, however, not limited to spatial applications. We believe that \omniUIST can be a valuable addition to desktop interfaces, e.g., navigating through video editing tools or gaming. We plan to broaden \omniUIST's usage scenarios in the future.

The overall tracking performance of \omniUIST suffices for interactive applications such as the ones shown in this paper. The accuracy could be improved by adding more Hall sensors, or optimizing their placement further (e.g., placing them on the outer hull of the device).
Furthermore, a spherical tip on the passive tool that more closely resembles the dipole in our magnetic model could further improve \omniUIST's accuracy. We believe, however, that the design of \omniUIST represents a good balance of cost and complexity of manufacturing, and accuracy.

Our current implementation of \omniUIST and the accompanying tracking and actuation algorithms assumes the presence of a single passive tool. Our method, however, potentially generalizes to tracking multiple passive tools by accounting for the presence of multiple permanent magnets. This poses another interesting challenge: the magnets of multiple tools will interact with each other, i.e., attract and repel each other.The electromagnet will also jointly interact with those tools, leading to challenges in terms of computation and convergence. We believe that our gradient-based optimization can account for such interactions and plan to investigate this in the future.

In developing and testing our applications, we found that \omniUIST's current frame rate of 40 Hz suffices for many interactive scenarios. The frame rate is a trade-off between speed and accuracy. In our tests, decreasing the desired accuracy in our optimization doubled the frame rate, while resulting in errors in the 3D position estimation of more than 1 cm, however. Finding the sweet spot for this trade-off depends on the application. While our applications worked well with 40 Hz and the current accuracy, more intricate actions such as high-precision sculpting might benefit from higher frame rates \textit{and} precision.
Reducing the latency of several system components (e.g., sensor latency, convergence time) is another interesting direction of future research. 

Furthermore, the control strategy we used was fairly naïve, as it only takes the current tool position into account. A model predictive strategy could account for future states, user intent, and optimize to reduce heating. We will explore in the next chapter how model predictive approaches can be used for haptic systems.

Overall, the main benefits of \omniUIST lie in the high accuracy and large force it can produce. It does so without mechanically moving parts, which would be subject to wear.
Such wear is not the case for our device, because it is exclusively based on electromagnetic force. We believe that different form factors of \omniUIST (e.g., body-mounted, larger size) can present interesting directions of future research. \add{A body-mounted version could be interesting for VR applications in which the user moves in 3D space. The larger size could result in more discernible points.}

Additionally, the influence of strength on user perception and factors such as just-noticeable-difference will allow us to characterize the benefits and challenges of \omniUIST, and electromagnetic haptic devices in general.
We believe that \omniUIST opens interesting directions for future research in terms of novel devices, and magnetic actuation and tracking.



% In the unusual situation where you want a paper to appear in the
% references without citing it in the main text, use \nocite
% \nocite{langley00}

\clearpage

\bibliography{references_rylan}
\bibliographystyle{icml2025}


%%%%%%%%%%%%%%%%%%%%%%%%%%%%%%%%%%%%%%%%%%%%%%%%%%%%%%%%%%%%%%%%%%%%%%%%%%%%%%%
%%%%%%%%%%%%%%%%%%%%%%%%%%%%%%%%%%%%%%%%%%%%%%%%%%%%%%%%%%%%%%%%%%%%%%%%%%%%%%%
% APPENDIX
%%%%%%%%%%%%%%%%%%%%%%%%%%%%%%%%%%%%%%%%%%%%%%%%%%%%%%%%%%%%%%%%%%%%%%%%%%%%%%%
%%%%%%%%%%%%%%%%%%%%%%%%%%%%%%%%%%%%%%%%%%%%%%%%%%%%%%%%%%%%%%%%%%%%%%%%%%%%%%%


% \section{Proposed model}
% \label{section:app:model}
% Table \ref{table:define} lists the symbols and their definitions used in this paper. \par
% % \vspace{-1.5em}
% % \TSK{
% % \begin{table}[t]
\vspace{-0.5em}
\centering
\small
% \footnotesize
\caption{Symbols and definitions.}
\label{table:define}
\vspace{-1.2em}
\begin{tabular}{l|l}
\toprule
Symbol & Definition \\
\midrule
$d$ & Number of dimensions \\
$t_c$ & Current time point \\
% $N$ & Current window length \\
$\mX$ & Co-evolving multivariate data stream (semi-infinite) \\
$\mX^c$ & Current window, i.e., $\mX^c = \mX[t_m:t_c]\in\R^{d\times N}$ \\
\midrule
$h$ & Embedding dimension \\
% $\mH$ & Hankel matrix, i.e., $\begin{bmatrix}
%             \embed{\vx_1} & \embed{\vx_2} & \cdots & \embed{\vx_{n-h+1}}
%         \end{bmatrix}$ \\
$\embed{\cdot}$ & Observable for time-delay embedding, i.e., $g\colon\R \rightarrow \R^{h}$ \\
% $\mH$ & Hankel matrix of $\mX$, i.e., $\mH = [\embed{\vx_1}~\embed{\vx_2} ~ ... ~ \embed{\vx_{n-h+1}}]$ \\
$\mH$ & Hankel matrix \\
$\nmodes$ & Number of modes \\
% $\imode$ & Modes of the system for $i$-th dimension of $\mX$, i.e., $\imode \in \R^{h \times r_i}$ \\
$\modes$ & Modes of the system, i.e., $\modes \in \R^{h\times\nmodes}$ \\
% $\ieig$ & Eigenvalues of the system for $i$-th dimension of $\mX$, $i.e., \ieig \in \R^{r_i \times r_i}$ \\
$\eigs$ & Eigenvalues of the system, i.e., $\eigs \in \R^{\nmodes\times\nmodes}$ \\
$\demixing$ & Demixing matrix, i.e., $\mW = [\rowvect{w}_1, ..., \rowvect{w}_d]^\top \in \R^{d \times d}$ \\
$\mB$ & Causal adjacency matrix, i.e., $\mB \in \R^{d \times d}$ \\
\midrule
% $\vs(t)$ & Latent variables at time point $t$, i.e., $\vs(t) = \{ \vs_1(t), ..., \vs_d(t) \} $ \\
$\ind(t)$ & Inherent signal at time point $t$, i.e., $\ind(t) = \{ \ith{e}(t) \}_{i=1}^d$ \\
$\mat{S}(t)$ & Latent vectors at time point $t$, i.e., $\mat{S}(t) = \{ 
\ith{\vs}(t) \}_{i=1}^d$ \\
$\vvec(t)$ & Estimated vector at time point $t$, i.e., $\vvec(t) = \{ \ith{v}(t) \}_{i=1}^d$ \\
\midrule
$\mathcal{D}$ & Self-dynamics factor set, i.e., $\mathcal{D} = \{\modes, \eigs\}$\\
$\regime$ & Regime parameter set, i.e., $\regime = \{ \mW, \mathcal{D}_{(1)}, ..., \mathcal{D}_{(d)} \}$\\
\midrule
$R$ & Number of regimes \\
$\regimeset$ & Regime set, i.e., $\regimeset 
 = \{ \regime^1, ..., \regime^R \}$\\
 $\mathcal{B}$ & \Relation, i.e., $\mathcal{B} = \{\mB^1, ..., \mB^R\}$\\
$\updateset$ & Update parameter set,  i.e., $\updateset 
 = \{ \update^1, ..., \update^R \}$ \\
\midrule
 $\modelparam$ & Full parameter set,  i.e., $\modelparam 
 = \{ \regimeset, \updateset \}$ \\

\bottomrule
% \midrule
\end{tabular}
\normalsize
% \vspace{-2.0em}
\vspace{1.0em}
\end{table}

% % }
% \vspace{0.6em}

\section{Optimization Algorithm}
% Algorithm \ref{alg:model} is the overall procedure of \method. Algorithm \ref{alg:estimator}, namely, \modelestimator continuously updates the full parameter set $\modelparam$ and the model candidate $\candparam$, which describes the current window $\mX^c$.
\TSK{
\begin{figure}[!h]
\vspace{-5.0ex}
\begin{algorithm}[H]
    \normalsize
    \caption{\method($\vx(t_c), \modelparam, \candparam$)}
    \label{alg:model}
    \begin{algorithmic}[1]
        \STATE {\bf Input:}
        \hspace{0mm}    (a) New value $\vx(t_c)$ at time point $t_c$ \\
        \hspace{9.5mm} (b) Full parameter set $\modelparam = \{\regimeset, \updateset\}$ \\
        \hspace{9.68mm} (c) Model candidate $\candparam = \{\regime^c, \update^c, \bm{s}^c_{en}\}$
        \STATE {\bf Output:}
        \hspace{0mm}    (a) Updated full parameter set $\modelparam'$ \\
        \hspace{11.8mm} (b) Updated model candidate $\candparam'$ \\
        \hspace{11.9mm} (c) $l_s$-steps-ahead future value $\vect{v}(t_c+l_s)$ \\
        \hspace{11.8mm} (d) Causal adjacency matrix $\mB$
        \STATE /* Update current window $\mX^c$ */
        \STATE $\mX^c \leftarrow \mX[t_m : t_c]$
        \STATE /* Estimate optimal regime $\regime$ */
        \STATE $\{\modelparam', \candparam'\} \leftarrow$ \modelestimator($\mX^c, \modelparam$, $\candparam$)
        \STATE /* Forecast future value and discover causal relationship */
        \STATE $\{\vect{v}(t_c+l_s),~\mB\} \leftarrow$ \modelgenerator($\candparam'$)
        \STATE /* Update regime $\regime$ */
        \IF{NOT create new regime}
            \STATE $\candparam' \leftarrow \regimeupdate(\mX^c, \candparam')$
        \ENDIF
    \RETURN $\{\modelparam', \candparam', \vect{v}(t_c+l_s), \mB\}$
    \end{algorithmic}
\end{algorithm}
\vspace{-4.5em}
\end{figure}
}\par
\TSK{
\begin{figure}[!h]
\vspace{-0.0ex}
\begin{algorithm}[H]
    \normalsize
    \caption{\modelestimator($\mX^c, \modelparam, \candparam$)}
    \label{alg:estimator}
    \begin{algorithmic}[1]
        \STATE {\bf Input:}
        \hspace{0.0mm}  (a) Current window $\mX^c$ \\
        \hspace{9.5mm} (b) Full parameter set $\modelparam$ \\
        \hspace{9.68mm} (c) Model candidate $\candparam$
        \STATE {\bf Output:}
        \hspace{0.0mm}  (a) Updated full parameter set $\modelparam'$ \\
        \hspace{11.8mm} (b) Updated model candidate $\candparam'$
        \STATE /* Calculate optimal initial conditions */
        \STATE $\mat{S}_{0}^c \leftarrow \argmin_{\mat{S}_{0}^c} f(\mX^c; \mat{S}_{0}^c, \regime^c)$
        \IF{$f(\mX^c; \mat{S}_{0}^c, \regime^c) > \tau$}
            \STATE /* Find better regime in $\bm{\Theta}$ */
            \STATE $\{ \mat{S}_{0}^c, \regime^c \} \leftarrow \argmin_{\mat{S}_{0}^c, \regime \in \regimeset} \,f(\mX^c; \mat{S}_{0}^c, \regime^c)$
            \IF{$f(\mX^c; \mat{S}_{0}^c, \regime^c) > \tau$}
                \STATE /* Create new regime */
                \STATE $\{ \regime^c, \update^c \} \leftarrow \textsc{RegimeCreation}(\mX^c)$
                \STATE $\regimeset \leftarrow \regimeset \cup \regime^c$; $\updateset \leftarrow \updateset \cup \update^c$
            \ENDIF
        \ENDIF
        \STATE $\modelparam' \leftarrow \{\regimeset, \updateset\}$; $\candparam' \leftarrow \{\regime^c, \update^c, \mat{S}_{en}^c\}$
        \RETURN $\modelparam', \candparam'$
    \end{algorithmic}
\end{algorithm}
\vspace{-3.3em}
\end{figure}
}
% Algorithm \ref{alg:model} shows the overall procedure for \method,
% including \modelestimator (Algorithm \ref{alg:estimator}).
% \modelestimator continuously updates the full parameter set $\modelparam$ and
% the model candidate $\candparam$, which describes the current window $\mX^c$.
% \par
\label{section:app:algorithm}
% \myparaitemize{Details in Eq. \eqref{eq:update_trans}}
\subsection{Details of Eq. (\ref{eq:update_trans})}
% \myparaitemize{Details of Eq. (\ref{eq:update_trans})}
Here, we introduce the recurrence relation of transition matrix $\ith{\trans}$.
As mentioned earlier, we use the following cost function (below, index $i$ denoting $i$-th dimension is omitted for the sake of simplicity, e.g., we write $\ith{\trans}$ as $\trans$):
\begin{align*}
    \mathcal{E} &= \sum_{t'=t_m+h}^{t_c}\forgetting^{t_c-t'}||\embed{e(t')} - \trans\embed{e(t'-1)}||_2^2 \\
    &= \sum_{l=1}^h (\mat{L}(l, :) - \trans(l, :)\mat{R})\Forgetting(\mat{L}(l, :) - \trans(l, :)\mat{R})^\top
\end{align*}
where,
% $\Forgetting = diag(\forgetting^{N-2}, ..., \forgetting^0) \in \R^{(N-1) \times (N-1)}$ and
$\Forgetting, \mat{L}$ and $\mat{R}$ are synonymous with the definition in Section \ref{section:alg:creation}.
Because we want to obtain $\trans$ that minimizes this cost function $\mathcal{E}$, we differentiate it with respect to $\trans$.
\begin{align*}
    \dfrac{\partial}{\partial\trans(l, :)}\mathcal{E}
    &= -2(\mat{L}(l, :) - \trans(l, :)\mat{R}) \Forgetting \mat{R}^\top
\end{align*}
Solving the equation $\partial\mathcal{E}/\partial\trans(l, :) = 0$ for each $l$, $1 \leq l \leq h$,
the optimal solution for $\trans$ is given by
% the optimal solution is $ \trans = (\mat{L}\Forgetting\mat{R}^\top)(\mat{R}\Forgetting\mat{R}^\top)^{-1} $
$$ \trans = (\mat{L}\Forgetting\mat{R}^\top)(\mat{R}\Forgetting\mat{R}^\top)^{-1} $$
where we define
\begin{align*}
    \mat{Q} = \mat{L}\Forgetting\mat{R}^\top,\quad
    \mat{P} = (\mat{R}\Forgetting\mat{R}^\top)^{-1}
\end{align*}
The recurrence relations of $\mat{Q}$ can be written as
\begin{align*}
    \mat{Q} &= \sum_{t'=t_m+h}^{t_c}\forgetting^{t_c-t'}\embed{e(t')}\embed{e(t'-1)}^\top \\
    &= \forgetting\sum_{t'=t_m+h}^{t_c-1}\forgetting^{t_c-t'-1}\embed{e(t')}\embed{e(t'-1)}^\top + \embed{e(t_c)}\embed{e(t_c-1)}^\top
\end{align*}
\begin{align}
    \label{eq:Q}
    \therefore \mat{Q}^{new} = \forgetting\mat{Q}^{prev} + \embed{e(t_c-1)}\embed{e(t_c)}^\top
\end{align}
and similarly
\begin{align}
    \label{eq:bP}
    \mat{P}^{new} &= (\forgetting{(\mat{P}^{prev})}^{-1} + \embed{e(t_c)}\embed{e(t_c)}^\top)^{-1}
\end{align}Here, we apply the Sherman-Morrison formula~\cite{sherman1950adjustment} to the RHS of Eq. \eqref{eq:bP}.
Note that $\embed{e(t_c)}^\top\mat{P}^{prev}\embed{e(t_c)} > 0$
because $\mat{P}^{-1} = \mat{R}\Forgetting\mat{R}^\top$ is positive definite by definition.
\begin{align}
    \label{eq:P}
    \therefore \mat{P}^{new} = \frac{1}{\forgetting}(\mat{P}^{prev} - \frac{\mat{P}^{prev}\embed{e(t_c-1)}\embed{e(t_c-1)}^\top\mat{P}^{prev}}{\forgetting + \embed{e(t_c-1)}^\top\mat{P}^{prev}\embed{e(t_c-1)}})
\end{align}
Finally, combining Eq. \eqref{eq:Q} and Eq. \eqref{eq:P} gives the recurrence relations of $\trans$ for Eq. \eqref{eq:update_trans}.
\begin{align*}
    \begin{split}
        \trans^{new} &= \trans^{prev} + (\embed{e(t_c)} - \trans^{prev}\embed{e(t_c-1)}\boldsymbol\gamma \\
        \boldsymbol\gamma &= \frac{\embed{e(t_c-1)}^\top\mat{P}^{prev}}{\forgetting + \embed{e(t_c-1)}^\top\mat{P}^{prev}\embed{e(t_c-1)}}
    \end{split}
\end{align*}
    
% \end{enumerate}
\par
% \TSK{
% \begin{figure*}[t]
    \begin{tabular}{cccc}
      \hspace{-1.5em}
      \begin{minipage}[c]{0.24\linewidth}
        \centering
        % \vspace{1em}
        \includegraphics[width=\linewidth]{results/web/original1_ver1.0.pdf}
        \vspace{-2em} \\
        \hspace{1.5em}
        % (a-i) Snapshot
        % (a-i) $l_s$-steps-ahead future value forecasting
        (a-i) Original data $\mX^c$
        \label{fig:web:forecast}
      \end{minipage} &
      \hspace{-1.5em}
      \begin{minipage}[c]{0.24\linewidth}
        \centering
        \includegraphics[width=\linewidth]{results/web/latent1_ver1.2.pdf}
        \vspace{-2em} \\
        \hspace{1.5em}
        (a-ii) Inherent signals $\mE$
      \end{minipage} &
      \hspace{-1.5em}
      \begin{minipage}[c]{0.24\linewidth}
        \centering
        \includegraphics[width=0.8\linewidth]{results/web/causal1_ver1.0.pdf}
        % \vspace{-2em}
        \\
        % \hspace{2.0em}
        (a-iii) Causal relationship $\mB$
      \end{minipage} &
      \hspace{-1.5em}
      \begin{minipage}[c]{0.24\linewidth}
        \centering
        \includegraphics[width=0.95\linewidth]{results/web/mode1_ver1.0.pdf}
        % \vspace{-2em}
        \\
        \hspace{-0.7em}
        (a-iv) Latent dynamics $\eigs$
      \end{minipage} \vspace{0.5em} \\
      % \caption{(a) Snapshot at the current time point $t_c = 208$}
      \multicolumn{4}{c}{\textbf{(a) Snapshots at current time point $t_c=208$.}}
      \vspace{0.5em} \\
      \hspace{-1.5em}
      \begin{minipage}[c]{0.24\linewidth}
        \centering
        % \vspace{1em}
        \includegraphics[width=\linewidth]{results/web/original2_ver1.0.pdf}
        \vspace{-2em} \\
        \hspace{1.5em}
        % (a-i) Snapshot
        % (b-i) $l_s$-steps-ahead future value forecasting
        (b-i) Original data $\mX^c$
      \end{minipage} &
      \hspace{-1.5em}
      \begin{minipage}[c]{0.24\linewidth}
        \centering
        \includegraphics[width=\linewidth]{results/web/latent2_ver1.2.pdf}
        \vspace{-2em} \\
        \hspace{1.5em}
        (b-ii) Inherent signals $\mE$
      \end{minipage} &
      \hspace{-1.5em}
      \begin{minipage}[c]{0.24\linewidth}
        \centering
        \includegraphics[width=0.8\linewidth]{results/web/causal2_ver1.0.pdf}
        % \vspace{-2em}
        \\
        % \hspace{2.0em}
        (b-iii) Causal relationship $\mB$
      \end{minipage} &
      \hspace{-1.5em}
      \begin{minipage}[c]{0.24\linewidth}
        \centering
        \includegraphics[width=0.95\linewidth]{results/web/mode2_ver1.0.pdf}
        % \vspace{-2em} 
        \\
        \hspace{-0.7em}
        (b-iv) Latent dynamics $\eigs$
      \end{minipage} \vspace{0.5em} \\
      \multicolumn{4}{c}{\textbf{(b) Snapshots at current time point $t_c=443$.}}
    \end{tabular}
    \vspace{-1.0em}
    \caption{\method modeling for a web-click activity stream related to beer query sets (i.e., \googletrend).
      Two sets of
      % invaluable knowledge
      snapshots taken
      at two different time points
      % (i.e., $t_c = 208, 443$, respectively)
      % on December 27, 2007 (top) and July 14, 2011 (bottom)
      show:
      (a/b-i) the current window of the original data stream,
      % where, the blue right vertical and red axes represent the current and $l_s$-steps-ahead time points, (i.e., $t_c, t_c+l_s$), respectively;
      % where, the blue vertical line to the right represents the current time point $t_c$;
      % where, the blue right vertical axis represents the current time point $t_c$;
      (a/b-ii) independent signals $\mE$ specific to each observation;
      % (c) time-evolving relationships with each other based on variables generating processes (i.e., \relations) and
      (a/b-iii) causal relationships $\mB\in\mathcal{B}$ and
      (a/b-iv) interpretable latent dynamics $\eigs$,
      where the argument and the absolute value of each point correspond to
      the temporal frequency and decay rate of modes, respectively.
      }
    \label{fig:web}
    \vspace{-1.2em}
\end{figure*}
% }
\setcounter{lemma}{1}
\subsection{Proof of Lemma \ref{lemma:create_time}}
\begin{proof}
The dominant steps in \textsc{RegimeCreation} are I, IV, and VI.
The decomposition $\mX$ into $\demixing^{-1}$ and $\mE$ using ICA requires $O(d^2N)$.
For each observation,
the SVD of $\ith{\mat{R}}\mat{M}$ requires $O(h^2N)$, and the eigendecomposition of $\ith{\tilde{\trans}}$ takes $O(k_i^3)$.
The straightforward way to
process IV and VI
is to perform the calculation $d$ times sequentially, i.e., they require $O(dh^2N+\sum_ik_i^3)$ in total.
However, since these operations do not interfere with each other,
they are simultaneously computed by parallel processing.
Therefore, the time complexity of \textsc{RegimeCreation} is $O(N(d^2+h^2)+k^3)$, where $k=\max_i(k_i)$.
\end{proof}
\subsection{Proof of Lemma \ref{lemma:causal}}
\begin{proof}
First, we need to formulate the causal structure.
Here, we utilize the structural equation model~\cite{pearl2009causality}, denoted by $\mX_{\text{sem}} = \mB_{\text{sem}}\mX_{\text{sem}} + \mE_{\text{sem}}$.
Because this model is known as the general formulation of causality, if $\mB_{\text{sem}}$ in this model is identified, then it can be said that we discover causality.
In other words, we need to prove that our proposed algorithm can find the causal adjacency matrix $\mB$ aligning with this model.
Solving the structural equation model for $\mX_{\text{sem}}$, we obtain 
$\mX_{\text{sem}} = \demixing^{-1}_{\text{sem}}\mE_{\text{sem}}$
where $\demixing_{\text{sem}} = \mat{I} - \mB_{\text{sem}}$.
It is shown that we can identify $\demixing_{\text{sem}}$ in the above equation by ICA,
except for the order and scaling of the independent components, if the observed data is a linear, invertible mixture of non-Gaussian independent components~\cite{comon1994independent}.
Thus, demonstrating that \modelgenerator precisely resolves the two indeterminacies of a mixing matrix $\mW^{-1}$ (i.e., the inverse of $\demixing \in \regime^c$) suffices to complete the proof because $\demixing$ is computed by ICA in \textsc{RegimeCreation}. \par
First, we reveal that our algorithm can resolve the order indeterminacy.
We can permutate the causal adjacency matrix $\mB$ to strict lower triangularity thanks to the acyclicity assumption~\cite{bollen1989structural}.
%, which is without loss of generality.
Thus, correctly permuted and scaled $\mW$
is a lower triangular matrix with all ones on the diagonal.
It is also said that there would only be one way to permutate $\mW$, which meets the above condition~\cite{shimizu2006linear}.
Thus, \modelgenerator can identify the order of a mixing matrix by the process in step I (i.e., finding the permutation of rows of a mixing matrix that yields a matrix without any zeros on the main diagonal).
Next, with regard to the scale of indeterminacy,
it is apparent that we only need to focus on the diagonal element,
remembering that the permuted and scaled $\mW$ has all ones on the diagonal.
Therefore, we prove that \modelgenerator can resolve the order and scaling of the indeterminacies of a mixing matrix $\demixing^{-1}$.
\end{proof}
% \myparaitemize{Proof of Lemma \ref{lemma:time}} \par
\subsection{Proof of Lemma \ref{lemma:stream_time}}
\begin{proof}
For each time point, \method first runs \modelestimator,
which estimates the optimal full parameter set $\modelparam$ and the model candidate $\candparam$ for the current window $\mX^c$.
If the current regime $\regime^c$ fits well,
it takes $O(N\sum_i k_i)$ time.
Otherwise, it takes $O(RN\sum_i k_i)$ time to find a better regime in $\regimeset\in\modelparam$.
Furthermore, if \method encounters an unknown pattern,
it runs \textsc{RegimeCreation}, which takes $O(N(d^2+h^2)+k^3)$ time.
Subsequently, it runs \modelgenerator to identify the causal adjacency matrix and forecast an $l_s$-steps-ahead future value,
which takes $O(d^2)$ and $O(l_s)$ time, respectively.
Note that $l_s$ is negligible because of the small constant value.
Finally, when \method does not create a new regime,
it executes \regimeupdate, which needs $O(dh^2)$ time.
Thus, the total time complexity is at least $O(N\sum_ik_i+dh^2)$ time and at most $O(RN\sum_i k_i+N(d^2+h^2)+k^3)$ time per process.
\end{proof}

% \input{components/table_acc_app_forecast}
% \begin{table*}[t]
    % \small
    \centering
    \caption{Ablation study results with forecasting steps $l_s\in\{5, 10, 15\}$ for both synthetic and real-world datasets.}
    \vspace{-1.0em}
    \begin{tabular}{c|c|cc|cc|cc|cc|cc}
    \toprule
    % \:Datasets\:
    \multicolumn{2}{c|}{Datasets}
    % & \#0 & \#1 & \#2 & \#3 & \#4 \\
    & \multicolumn{2}{c|}{\synthetic} & \multicolumn{2}{c|}{\covid} & \multicolumn{2}{c|}{\googletrend} & \multicolumn{2}{c|}{\chickendance} & \multicolumn{2}{c}{\exercise} \\
    \midrule
    \multicolumn{2}{c|}{Metrics}
    % \:Metrics\:
    & \:RMSE & MAE\:\,
    & \:RMSE & MAE\:\,
    & \:RMSE & MAE\:\,
    & \:RMSE & MAE\:
    & \:RMSE & MAE\:\: \\
    \midrule
    \multirow[t]{3}{*}{\:\:\method (full)\:\:}
    % \:\:\method (full)\:\:
    & 5 & \:0.722 & 0.528\:\, & \:0.588 & 0.268\:\, & \:0.573 & 0.442\:\, & \:0.353 & 0.221\: & \:0.309 & 0.177\:\, \\
    & 10 & \:0.829 & 0.607\:\, & \:0.740 & 0.361\:\, & \:0.620 & 0.481\:\, & \:0.511 & 0.325\: & \:0.501 & 0.309\:\, \\
    & 15 & \:0.923 & 0.686\:\, & \:0.932 & 0.461\:\, & \:0.646 & 0.505\:\, & \:0.653 & 0.419\: & \:0.687 & 0.433\:\, \\
    \midrule
    \multirow[t]{3}{*}{\:\:w/o causality\:\:}
    & 5 & \:0.759 & 0.563\:\, & \:0.758 & 0.374\:\, & \:0.575 & 0.437\:\, & \:0.391 & 0.262\: & \:0.375 & 0.218\:\, \\
    & 10 & \:0.925 & 0.696\:\, & \:0.848 & 0.466\:\, & \:0.666 & 0.511\:\, & \:0.590 & 0.398\: & \:0.707 & 0.433\:\, \\
    & 15 & \:1.001 & 0.760\:\, & \:1.144 & 0.583\:\, & \:0.708 & 0.545\:\, & \:0.821 & 0.537\: & \:0.856 & 0.533\:\, \\
    \bottomrule
    \end{tabular}
    \label{table:ablation}
    \vspace{-0.75em}
\end{table*}

\section{Experimental Setup}
% \label{section:app:experiments}
\label{section:app:experiments:setting}
In this section, we describe the experimental setting in detail.
% \subsection{Experimental Setting}
% \myparaitemize{Experimental Setting}
We conducted all our experiments on
% \unclear{<server spec>}.
an Intel Xeon Platinum 8268 2.9GHz quad core CPU
with 512GB of memory and running Linux.
We normalized the values of each dataset based on their mean and variance (z-normalization).
The length of the current window $N$ was $50$ steps in all experiments.
\par
\myparaitemize{Generating the Datasets}
We randomly generated synthetic multivariate data streams containing multiple clusters, each of which exhibited a certain causal relationship.
For each cluster, the causal adjacency matrix $\mB$ was generated from a well-known random graph model, namely Erdös-Rényi (ER)~\cite{erdos1960evolution} with edge density $0.5$ and the number of observed variables $d$ was set at 5.
The data generation process was modeled as a structural equation model~\cite{pearl2009causality},
where each value of the causal adjacency matrix $\mB$ was sampled from a uniform distribution $\mathcal{U}(-2, -0.5)\cup(0.5, 2)$.
In addition, to demonstrate the time-changing nature of exogenous variables, 
we allowed the inherent signals variance $\sigma^2_{i, t}$ (i.e., $\ith{e}(t)\sim\text{Laplace}(0, \sigma_{i, t}^2)$)
to change over time.
Specifically, we introduced $h_{i, t}=\text{log}(\sigma^2_{i, t})$, which evolves according to an autoregressive model, where the coefficient and noise variance of the autoregressive model were sampled from $\mathcal{U}(0.8, 0.998)$ and $\mathcal{U}(0.01, 0.1)$, respectively.
% however, 

The overall data stream was then generated by constructing a temporal sequence of cluster segments and each segment had $500$ observations (e.g., ``$1,2,1$'' consists of three segments containing two types of causal relationships and its total sample size is $1,500$). We ran our experiments on five different temporal sequences: ``$1,2,1$'', ``$1,2,3$'', ``$1,2,2,1$'', ``$1,2,3,4$'', and ``$1,2,3,2,1$'' to encompass various types of real-world scenarios.
\par
\myparaitemize{Baselines}
The details of the baselines we used throughout our extensive experiments are summarized as follows:
\par\noindent
(1) Causal discovering methods
{\setlength{\leftmargini}{11pt}
\vspace{-0.3ex}
\begin{itemize}
    \item CASPER~\cite{liu2023discovering}: is a state-of-the-art method for causal discovery, integrating the graph structure into the score function and reflecting the causal distance between estimated and ground truth causal structure. We tuned the parameters by following the original paper setting.
    \item DARING~\cite{he2021daring}: introduces an adversarial learning strategy to impose an explicit residual independence constraint for causal discovery. We searched for three types of regularization penalties $\{\alpha, \beta, \gamma\}\in\{0.001, 0.01, 0.1, 1.0, 10\}$.
    % aiming to improve the learning of acyclic graphs.
    \item NoCurl~\cite{yu2021dag}: uses a two-step procedure: initialize a cyclic solution first and then employ the Hodge decomposition of graphs. We set the optimal parameter presented in the original paper.
    % and learn a DAG structure by projecting the cyclic graph to the gradient of a potential function.
    \item NOTEARS-MLP~\cite{zheng2020learning}: is an extension of NOTEARS~\cite{zheng2018dags} (mentioned below) for nonlinear settings, which aims to approximate the generative structural equation model by MLP.
    We used the default parameters provided in authors' codes\footnote[2]{\url{https://github.com/xunzheng/notears} \label{fot:notears}}.
    \item NOTEARS~\cite{zheng2018dags}:
    % is specifically designed for linear settings and
    is a differentiable optimization method with an acyclic regularization term to estimate a causal adjacency matrix.
    We used the default parameters provided in authors' codes\footref{fot:notears}.
    % estimates the true causal graph by minimizing the fixed reconstruction loss with the continuous acyclicity constraint.
    \item LiNGAM~\cite{shimizu2006linear}:
    exploits the non-Gaussianity of data to determine the direction of causal relationships. It has no parameters to set.
    % and we used the authors source codes\footnote{https://github.com/cdt15/lingam}.
    \item GES~\cite{chickering2002optimal}: is a traditional score-based bayesian algorithm that discovers causal relationships in a greedy manner.
    It has no parameters to set.
    We employed BIC as the score function and utilized the open-source in~\cite{kalainathan2020causal}.
\end{itemize}
\vspace{-0.5ex}}
\par\noindent
(2) Time series forecasting methods
{\setlength{\leftmargini}{11pt}
\vspace{-0.3ex}
\begin{itemize}
    \item TimesNet/PatchTST~\cite{wu2023timesnet, Yuqietal-2023-PatchTST}: are state-of-the-art TCN-based and Transformer-based methods, respectively.
    The past sequence length was set as 16 (to match the current window length).
    % Other parameters follow the parameter settings suggested in the original paper.
    Other parameters followed the original paper setting.
    % \item PatchTST~\cite{Yuqietal-2023-PatchTST}: is a state-of-the-art Transformer-based method for time series forecasting. The past sequence length is set as 16 for the same reason as above.
    \item DeepAR~\cite{salinas2020deepar}: models probabilistic distribution in the future, based on RNN. We built the model with 2-layer 64-unit RNNs. We used Adam optimization~\cite{adam} with a learning rate of 0.01 and let it learn for 20 epochs with early stopping.
    % to choose the best model.
    \item OrbitMap~\cite{matsubara2019dynamic}:
    % is a stream forecasting algorithm that finds important time-evolving patterns with multiple discrete non-linear dynamical systems.
    finds important time-evolving patterns for stream forecasting.
    We determined the optimal transition strength $\rho$ to minimize the forecasting error in training.
    \item ARIMA~\cite{box1976arima}: is one of the traditional time series forecasting approaches based on linear
    equations. We determined the optimal parameter set using AIC.
\end{itemize}}

% \input{corrected_proofs_by_josh}
\end{document}
