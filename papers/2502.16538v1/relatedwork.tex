\section{Related Works}
\subsection{Underwater Camera Images}
Underwater environments and objects captured using underwater cameras possess distinct characteristics that differentiate them from images taken in the atmosphere with conventional cameras.
The primary difference between the two imaging environments lies in the medium; underwater cameras operate in water, where strong refraction and reflection occur more easily than in air.
This difference causes the rapid absorption of long-wavelength red electromagnetic waves compared to blue waves, resulting in underwater RGB images where the Red channel values are relatively lower than those of the Green and Blue channels. As the distance increases, the number of pixels lacking Red channel values also increases.
This leads to an imbalance in the information distribution among the three channels in an RGB image.
Furthermore, in underwater environments, camera output is affected by backscattering, a phenomenon where light is scattered by suspended particles or fine debris.
\begin{figure*}[ht]
    \centering
    \includegraphics[width=\textwidth]{arxivfig1.drawio.png}
    \caption{Flow of the proposed glare detection algorithm and input/output examples of each phase: (a) Input image; (b) Image after pre-processing (brightness enhancement, image color space combination, coordinate information channel supplementation, and image resizing); (c) Pixel-clustering results showing clusters per channel; (d) Binary region obtained using cluster information; (e) Output image.}
    \label{fig}
\end{figure*}
Such backscattering degrades the quality of captured images, making them appear blurry and hazy.
Additionally, underwater environments contain various suspended particles, turbidity, and bubbles, which further impact image quality.
These factors introduce noise into the captured images, negatively affecting image analysis.
Due to these environmental characteristics, underwater camera images require preprocessing and correction techniques to mitigate these effects.
Existing studies that utilize underwater camera data primarily focus on detecting divers, natural or artificial structures in underwater environments, and further detecting diver postures and hand signals.
Therefore, glare is typically treated as simple noise or an obstacle that negatively affects detection, rather than being considered a detection target in most research efforts.

\subsection{Data Clustering}
Data clustering is a widely used technique not only in data processing fields such as feature analysis and data mining but also in most formats containing information, such as images.
Representative clustering methods include K-means, Mean Shift, and Gaussian Mixture Model (GMM). Among these, the K-means clustering algorithm first designates K clusters in the dataset before performing clustering and assigns centroids corresponding to these clusters.
These cluster centroids calculate the distance to each individual data point, and each data point is assigned to the centroid with the minimum distance.
The newly formed cluster’s mean point is then set as the new centroid, and the clustering assignment process is repeated.
When the updated centroids stabilize within a certain threshold, the clustering process terminates, and the final results are produced.
When directly applied to images, this type of clustering groups individual data points based on pixel color values.
However, in such pixel clustering, only color values are considered, and the two-dimensional spatial information of each pixel within the image is not incorporated.
As a result, adjacent pixels within the same object may be misclassified into different clusters due to improper clustering results, or distant pixels from unrelated objects that share similar color values may be grouped into the same cluster.

\subsection{CLAHE}
CLAHE (Contrast Limited Adaptive Histogram Equalization) is a type of histogram equalization used to enhance image contrast.
HE (Histogram Equalization) adjusts contrast across the entire image. In contrast, CLAHE divides the image into a grid structure and applies HE individually to each grid.
At this stage, contrast enhancement is limited to prevent it from exceeding a certain threshold.
Through this process, while HE can degrade image quality in areas with locally high contrast, CLAHE enhances the quality of low-contrast areas by improving unclear shapes and boundaries while suppressing excessive contrast enhancement, thus improving overall image quality.
CLAHE improves image quality in underwater camera images by addressing issues such as backscattering and insufficient light, which cause unclear shapes and boundaries. At the same time, it suppresses excessive local contrast caused by scattering and reflection.

\subsection{Image Color Spaces}
Image color space refers to a method of representing color images, which typically consists of three two-dimensional channels organized in rows and columns.
The most commonly used image color space is RGB (Red, Green, Blue), which has a color intensity range from 0 to 255 and is easily processed by hardware.
The HSV (Hue, Saturation, Value) color space represents hue values ranging from 0 to 360, while saturation and value range from 0 to 100.
Additionally, various color spaces exist, such as the Lab color space, which consists of luminance, green-red range, and blue-yellow range, and the YUV color space, which consists of brightness, cyan-blue, and red-yellow levels.
By using these image color spaces, image adjustment can be performed more easily by assigning different weights to each channel. Additionally, separating channels allows for the extraction of meaningful information and features from each channel.

This study draws its core concept from the simple yet powerful clustering capability of pixel clustering, which applies data clustering techniques to image pixels.
It is proposed that the issue of losing spatial relationships between pixels during the pixel clustering process can be mitigated by adding additional channels representing image coordinates.
Furthermore, noise introduced into images due to the underwater environment is addressed using computer vision techniques such as CLAHE.