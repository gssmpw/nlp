\section{Related Work}
\label{sec:related}
We are not the first to utilize continuous functions for MEP or TS search. Below, we summarize some relevant work.

\paragraph{Discrete representations.}
Several methods\cite{string,gsm,improved_string,gmam} optimize a discrete representation of the reaction path but repeatedly fit a continuous curve to estimate tangents or redistribute points along the path. Ref.~\onlinecite{gsm} fits cubic splines and approximates the TS via interpolation. Ref.~\onlinecite{improved_string} also fits cubic splines but runs a separate optimization for TS search once an approximate MEP is found and does not discard tangential energy gradients. Ref.~\onlinecite{gmam} fits piecewise-linear curves within a variational formulation but does not focus on TS search. In contrast to these methods, we directly update a continuous neural network representation of the reaction path, combining MEP and TS search in a single optimization.

\paragraph{Continuous representations.}
Other methods\cite{ssm,curve,vrpo,doob} explicitly work with a continuous representation of the reaction path. Ref.~\onlinecite{ssm} represents the path with a cubic spline but only optimizes it for TS search -- the path itself has no physical meaning. Ref.~\onlinecite{curve} updates a B-spline curve representation and approximates the TS with the highest-energy sample observed on the path. While the curve may be fit to an improved initial guess\cite{idpp}, the method has not been compared against NEB and has primarily demonstrated success on systems requiring very few optimization steps. Ref.~\onlinecite{vrpo} adopts a variational framework and represents the path with a linear combination of basis functions. Despite involving a non-linear, constrained optimization, the method does not, in practice, find MEPs for even two-dimensional potentials\cite{vrpo}, limiting its applicability for path-search in atomistic settings. 
%It does, however, efficiently find transition states (when fit to an improved initial guess). 
Ref.~\onlinecite{doob} uses a neural network representation to model a distribution over stochastic trajectories connecting the end states rather than optimizing for MEP or TS search. 

To our knowledge, we are the first to use neural networks to represent MEPs and estimate transition states. Unlike prior work, we combine the nudging and climbing mechanisms in the same optimization. Moreover, all previous methods lack the ability to condition the path representation on arbitrary end states for potential generalization across systems.