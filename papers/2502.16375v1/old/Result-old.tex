

\section{RQ1: Users’ Current Impression of PHC}

%\subsection{Current Online Identity Verification Practices}

\textbf{Current Verification Practices: Multi Data and Context Verification.} 
When discussing services that required identity verification, participants most commonly mentioned financial services, including online banking and investment platforms, as well as health services, government-related processes, cross-border regional verifications, and two-factor authentication (2FA) used by various institutions. The discussion highlighted both digital and physical forms of verification. Several participants mentioned they are required to upload government IDs (e.g., social security number, driver's license) when creating an account- as P7 said--\textit{``For Robinhood, it asked for uploading my government issued IDs like driver's license and passport.''} 
%In addition to financial services, the applications where participants experienced verification processes include a wide range of applications. 
Similarly, P2 explained their verification experience in the government services requiring multiple information \textit{``I remember, if you go use like government related service, filing the text files, they would require you to verify your identity through your face.''} Beyond financial and government services, identity verification has also become essential in marketplace apps,
%that rely on trust to facilitate peer-to-peer transactions. 
P1 illustrated this trend with Airbnb, explaining  \textit{``If you do book an Airbnb, at least as a renter, you would need to verify your government id before being eligible to book your first stay.''} 
%This indicates that the scenarios we prepared for succeeding sessions are representative of typical user occasions. 

%\textbf{Confusion Between Verification and Authentication} Many participants did not distinguish between verification and authentication. In particular, when asked about experiences with biometric verification, they often shared daily experiences related to authentication, e.g., (P3): \textit{"It's I think most of the apps required me such fingerprints or related identification. For example, when I open my bank apps, I can just use my fingers to login to my account."} Participants indicated frequent use of fingerprints or face authentications on mobile phones. For instance, P1 mentioned stored biometric information on devices, \textit{"Because my android phone already have my fingerprint to log into my phone. And then this app just to reuse my fingerprint, which is already registered on my phone."}

\textbf{How Users Evaluate the Trustworthiness of Stakeholders} When discussing the current verification process, participants often indicated that there are organizations they can trust and others they cannot. Thus, we asked how or why they have trust with specific stakeholders. P5 explained his trust in banks comes from regulatory aspects: \textit{"They are being monitored by federal agencies. Their activities are monitored. They are under a lot of regulations. So there is a monitoring system that is tracking banks, at least Major Banks or Banks that I know of and other financial services they are under a lot of regulations."} P4 highlighted another aspect resulting in their trust in banks,  which is that the accessible customer support, stating \textit{"You know what sort of steps have taken so you could probably have a read of that, and any concerns that you might have. So you probably always have the opportunity that you could always speak to them first, because obviously they probably have agents available that you could speak to about any concerns. "} On the other hand, we observed that many participants expressed distrust towards social media platforms. P3 shared a formative experience of her social media account almost being hacked, noting \textit{"I've had people experience...Who then tried to get like hack into my account. So the one hacker would hack one person's account, and because, obviously, in your social media, you have access to other people's accounts as well, like I've had people send me really dodgy links, which I knew like instantly, that if I clicked on that link, it would mean that my account would get hacked as well."} Some participants elaborated on how they evaluate the trustworthiness of companies or stakeholders when they share their sensitive information in general. P6 reflected \textit{"I guess if the company was suggested to me, or I was referred to the company through the services of like Capital One or a big company that's been around forever. That would give more validity to me."} P7 provided insight from his own experience with a suspicious verification process, saying \textit{"I do a lot of research… What I did was I went and searched on the Internet. So I went on Reddit, and I looked for different keywords, and my conclusion was that it's a very standard process. It's not like the dealership wants to collect the social security for themselves. They wanted to do the background check, and they wanted to see if the information that they provided was matching with my information from the US. "}

%\textbf{Offline Verification in Daily Life} Some of the participants shared their offline experience that required identity verification, especially in the context of international travel and immigration procedures,  as P3 described \textit{"So in the past I've had to use my fingerprints, for example, when I came to the UK. When I had to renew my passport in the embassy of my country, they would take my fingerprint there. So they would take a photo of my current passport, and then they would take a photo of myself, but then they would also require me to give my fingerprints as well."}  


\subsection{Perceived Benefits of Personhood Credentials}
%\fixme{there needs to be titles of sub-themes under this benefit, it's hard to read. Sub-themes could be -- convenience of not carrying physical id, etc -- whatever unique /interesting subthemes came up in the interview}
\textbf{Verify Uniqueness with Reduced Exposure of Personal Information} Participants identified a range of benefits, the most frequently mentioned being the reduction of personal information exposure. P4 expressed \textit{"I think the what I glean from it was there was to provide without giving away too much personal information, but it can still sort of verify your uniqueness."} In addition to this, several participants highlighted the benefits related to online identity crisis with AI. P11 summarized the benefits of PHC, \textit{"Personhood credentials is a way in which a person can confirm that they are not a bot or an AI, and that they're a real person without giving away their personally identifiable information."} P11 also extended the discussion on this point to online malicious attacks, \textit{"I think that it will tackle like fake accounts, multiple accounts and bots massively. And I can see it being used like, really broadly, because you're not giving away any really important information... And having to get this identification through, like your government, or something works really well, because you don't mind giving them the information. And then, the kind of redacted information that you give to social media sites, increases safety and stuff, but doesn't put your data at risk."} P1 explained increased confidence in privacy through PHCs for unfamiliar entities \textit{"All of that information doesn't need to be filled in each time, you fill a form for submission and the potential conference that if I am signing up for an account with a 3rd party, which I have not heard before. I'm not comfortable with even in the case of a breach. They would not have access to my personal information, but just rather the identifier id. So that would be something I'd be comfortable with. "}

\textbf{Efficiency of Verification Process} Another key benefit highlighted by participants was the efficiency of verification process. P3 emphasized the convenience of less repeated verification, stating, \textit{"Having an easier login from a trusted source. So it's ease of access for the user." }


\subsection{Perceived Concerns of Personhood Credentials}
%\fixme{there needs to be titles of sub themes under this benefit, it's hard to read }
Participants expressed various concerns, primarily revolving around data security and trustworthiness of PHC issuers. 

\textbf{PHC Issuer Being Hacked or Leaked} One of the most frequently mentioned concerns was the potential risk of PHC issuers being hacked or experiencing data leaks. As P8 explained, \textit{"However, the problem with having one identity or one credential for everything is, if that credential is stolen, all of your accounts are at risk, and that is where I don't like the idea of one verification credentials. So if I use one verification for my bank and email and Facebook and Facebook gets hacked and they steal that credential and well, now, my bank and email accounts are also at risk."} It highlights participants' fears that the concentration of sensitive data with PHC issuers could have severe consequences for malicious attacks.

%\textbf{Data Handling by PHC Issuers} Another significant concern was how PHC issuers manage and handle the credential data they collected. Participants showed uncertainties regarding what happens to their personal information after verification and whether it is securely stored or potentially mishandled. For example, P4 shared that \textit{"I'm not sure if they tell us explicitly how they're gonna use, do they going to share our information with 3rd party or with government. If they ask for it, I'm not sure about that. These are my concerns."}

%\textbf{Centralized PHC} Several participants expressed concerns over the centralized power held by PHC issuers, which they feared could lead to misuse or overextension of control. P4 commented \textit{"But I think there's a concern on the other side that they're gonna be one kind of you're gonna give all your information to a small group or an institution. So they have the power that can be abused later."} illustrating the apprehension about the concentration of power in the hands of a single entity as a PHC issuer.

\textbf{Trustworthiness of PHC Issuer} The issue of trustworthiness in PHC issuers was also discussed, with participants expressing doubts about the reliability of the entities managing their data. Some worried whether these issuers could be trusted to protect their personal information and maintain their privacy. P7 described \textit{"My only concern is, how do we trust the Phc Am I using it? Yeah, because if if they are the one entity who handles all these kind of information, and if they are collecting everything about us from the government. And whether this entity is considered like a NGO, or that's a governmental body, or that if that's a private sector, so these are like some nuances about how to define or how to build this kind of issuer, because we have these giant issuers like Equifax or like these giant companies who are handling credits or those kind of financial aspect of the US structure. And they're like giants. They have these kind of monopoly over how they handle things. Is it going to be something similar? If so, how people can trust them? I don't want to share my personal information anymore, with like some random website...But how the way that the issuer can be understood? I think that would be the decision factor for users to start engaging with the system."}

\subsection{Motivational Experiences for Using PHC}
When discussing the benefits of PHCs, participants shared their experiences with deceptive online activities stemming from digital identity crises ranging from bots to scams. P10 explained the issue of bot accounts on social media, noting \textit{"I get random bots following me on my twitter a lot which I'm assuming. If they had this PHC Implemented, we would probably see a reduction in those types of bots...I'm assuming there's bots on Instagram as well, and I'm sure most social media accounts is probably bots."} P11 also reflected on fake news spread by bots, \textit{"I know that bots and AI and stuff are utilized quite a lot on social media sites, and to spread like fake news, and to support certain people's agendas and the kind of a tool that that are used in order to."} Accordingly, identity verification is recognized as a key area of concern in the context of social media platforms for end-users due to the proliferation of bot accounts and the dissemination of fake news. 

In addition, some participants emphasized the significance of identity verification in relation to cryptocurrency. P13 stated scams in cryptocurrency, \textit{"I think with cryptocurrency, everyone's scamming everyone all the time...And the companies who do the id verification basically decided it might be legally difficult. So they just stop. So that's probably the main problem I've had."} P12 indicated trade-off between privacy and traceability in crypto wallets, \textit{"I don't really like identity you put in your facial verification, or your Id or your driver's license on trust. But for Binance, Bybit and KuCoin, they actually request that because it's actually a big company or anything. But I'll see it's just a very diverse app that you can do a lot of things on. It needs the privacy of other people. It needs the privacy of other people and yours to cooperate because of scams and fraud."}

\subsection{Context Matters for Users' Preference}
As outlined in the method section \ref{sec:study_protocol}, we explored participants’ perceptions and preferences regarding PHC across five distinct scenarios. Participants’ responses indicated the necessity of verification methods depending on the specific scenario. In some contexts, they considered PHC unnecessary and suggested simpler methods like email or phone verification. As P5 stated, \textit{"I love to talk, to, chatgpt, and bard about weird stuff, and I think a lot of people like to talk to them about weird stuff. And so I would like the option to not log in via biometrics, and have a lesser experience and not have my identity tied to it. So I personally, I would not like to have my identity tied to my chat Gpt, or my bard, or Gemini."}, they explained that online platforms, which do not require extensive personal information, are sufficient with simple verifications.

\subsection{Perception Before \& After Educational Video}
As discussed in the method section \ref{sec:study_protocol}, we revisited questions about participants' understanding of PHC to compare their responses before and after watching the educational video to assess whether the video impacted their original knowledge of PHC. 

\textbf{Pre-understanding: Guessed it as one of the verification methods} As all participants have never heard of PHC, we prompted them to interpret the term based solely on its wording. Most of them inferred that it referred to another type of person identification. For example, P3 commented \textit{" It can be anything that would point to one single individual that would differentiate that individual from others."} When participants expressed how PHC identify person's uniqueness, their thoughts has a variety, ranging from basic information such as address or age to more advanced approaches like Multi-factor authentication or knowledge-based questions.

\textbf{Post-understanding: Involvement of trusted entity} When asked to explain their understanding of PHC, P13 noted, \textit{"So it sounds like, basically, you it's similar to how you verify things before. Like you use a biometrics and your government Id. But then you get a personal key. You do it with like a trusted organization rather than each individual. And then you can use that key for all the different services you use."} P1 elaborated PHC process as a shift of the verification entity, \textit{" I'd say we are sort of moving the verification burden from the user side to a service provider side where they have access to our data, and they have access to the token that's assigned to each person that's unique. And that's easily like traceable across online platforms. and this token is used for verification with 3rd parties, where they don't get access to your personal data, but they only use this service provider to give them the authenticity that you are a real user."} These suggest that the role of the PHC issuer is recognized as a crucial component of PHC.

\textbf{Where Misunderstandings Can Arise in PHC}
We observed that users sometimes misunderstood PHCs. These misconceptions represent critical points that should be carefully addressed in communication when aiming for the widespread adoption of PHC in the future. P10 explained the benefit of PHC, stating \textit{"I am thinking that imagine you are a criminal or an online criminal. For example, with some records that you are not allowed to use some services online. For federal government, they need to verify your credentials to be able to track you or just not allow you to do activities."} This comment was interpreted as PHC enhancing traceability to assist in criminal investigations. Such potential traceability was also mentioned in the context of tech companies tracking data, saying \textit{"But what is more of a concern for me is that some of these big tech companies like Google, like Facebook, like even Apple, Amazon. They use that information to track my searches and activities."} However, PHCs are fundamentally intended to serve as a means of anonymously verifying users' personhood without compromising their privacy. 


%\textbf{Post-understanding: Found similarities with other technologies} P5 interpreted the term as a verification method based on personal characteristics, which differ from person to person, such as behavioral patterns, commenting \textit{"It's like, personality based or like activities humans particularly perform. It sounds similar to the pattern of people walking. Those are also things people use for verification."} After watching the video, he updated his perception of PHC \textit{"I thought it was similar to SSL technology. Like in SSL, they will give the certificate to 3rd party who encrypts and has those certificates. To maintain their integrity, they will pass on the kind of certificate to web browsers. So when a user goes through a particular website, they will be able to cross-verify using the public keys available."}, linking it to technologies such as SSL, which uses encryption and hashing for certificates.
