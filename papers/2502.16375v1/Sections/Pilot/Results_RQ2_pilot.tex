\section{RQ2: Users’ preference of PHC}

\subsection{Preferred Credential Types}
Participants' preferred credential types can be categorized into three main verification methods: \ding{202} Government-issued IDs, such as social security numbers, passports, and driver's licenses; \ding{203} Biometrics, including face and fingerprints; and \ding{204} Digital identifiers, such as phone numbers and email addresses.

\textbf{Ease of Biometrics Over Government IDs} While concerns about the risk of data leaks were expressed regarding government-issued IDs, the convenience of biometric authentication was emphasized. For instance, P2 mentioned \textit{"I think this information may be lost, or maybe stolen by other people. So it's a risk. Think the passport and the ID card."} It underscores that participants worry about the potential vulnerability of physical IDs. At the same time, P2 showed their preference for biometrics, stating \textit{"I think fingerprint authentication is much easier than other methods because I don't have to remember passwords or complex PINs. I just use my fingerprints. It's much easier." } 

\textbf{Trade-off on Digital Identifiers} Participants also shared their preference with a trade-off on digital identifiers, noting both risks and advantages. P3 acknowledged the potential risks with \textit{"Email address is more likely leaked or stolen by other people. So I think it has many risks related to using an email address."} However, P3 preferred using an email address as a credential, supplementing by \textit{"But on the other hand, an email address will not leak my personal information too much. so I think, for the pros and cons, I like this."} 

\subsection{Preferred Stakeholder Types}
\textbf{Differing Trust Across Stakeholders}
Participants expressed different levels of trust depending on the type of stakeholder. Key insights include relatively higher trust in the government, contrasted with a pronounced distrust regarding social media privacy. P2 noted their confidence in the government: \textit{"I trust the government. The government already have this information. So it's okay. "} It showed preference in handling sensitive information. They further explained that they did not extend the same trust to private companies or organizations, particularly social media platforms. P2 expressed their concerns, stating \textit{"I think, for social media companies, they may not secure as secure as a financial system. I read some news that many big companies leaks their customers' information. So I think it's not okay to provide my passport or my that kind of information to the company."} This contrast highlights the complex considerations individuals have when evaluating trust in different entities for managing private data.

\textbf{Variability in Trust Among Individuals}
While there was a general consensus on preferred entities for data trust, individual perspectives varied regarding which organizations were considered trustworthy. For example, P4 expressed distrust toward the government, commenting \textit{"Government is not kind of trying to protect most of the citizens’ privacy and security, but at the same time it’s going to give more power to the government."} They also showed relative trust in social media companies, explaining \textit{"They probably invest in the privacy and security part. So I’m kind of less concerned."} These contrasting viewpoints highlight how personal beliefs and experiences shape individual trust preferences, underscoring the diversity of opinion in stakeholder trust.


\subsection{Context Matters for Users' Preference}
As outlined in the method section \ref{sec:study_protocol}, we explored participants’ perceptions and preferences regarding PHC across five distinct scenarios. 

\textbf{Unnecessary PHC Use in Specific Scenarios} Participants’ responses indicated the necessity of verification methods depending on the specific scenario.  In some contexts, they considered PHC unnecessary and suggested simpler methods like email or phone verification. As P5 stated, \textit{"most of the people are there for entertainment, and they just upload for fun. And at places where people usually go for fun, they don't require much information from a user. In an OTP or like email verification is kind of like enough for those platforms."}, they explained that online platforms, which do not require extensive personal information, are sufficient with simple verifications.

\section{RQ2: Design Suggestions for PHC}

\subsection{Participants' Needs for PHC Designs}
Participants expressed specific needs for PHC systems, primarily focusing on \textbf{decentralization and transparency}. P4 preferred a decentralized approach, stating, \textit{"I prefer the decentralized version just because you’re not giving all the power or like responsibility to government, although government might have all the information."} Additionally, transparency was identified as a critical requirement. P4 also noted, \textit{"I think any of this institute, if they take this responsibility, they need to be regulated about how they’re gonna handle user privacy security. They should be transparent about their process."} This emphasis on transparency and accountability highlights the need for PHC issuers to demonstrate their commitment to protecting user privacy and security as trusted issuers.

\subsection{Participants' Suggestions for PHC Designs}
Participants suggested several design improvements for PHC systems. 

\textbf{Design Theme 1: Decentralization} One suggestion focused on \textbf{decentralized data storage} to reduce the risk associated with a single point of failure Figure \ref{fig:sketch_pilot_p5}. As P5 noted, \textit{"They will be able to like the data will be stored on different platforms, and it will reduce the risk of like attack one and get all scenario." } This approach enhances PHC's security by distributing data across multiple locations. Similary, P4 drawed decentralized system, inspired by sector-based privacy policies. The participant explained that privacy policies are often structured according to specific sectors as shown in Figure \ref{fig:sketch_pilot_p4}. His idea is having a group of folks in certain entities, one of which is social media, government, and bank, instead of one institute managing or issuing the PHC. For example, PHC issuer for social media would be managed by a group of representatives from major social media companies, chosen through a democratic process.

\textbf{Design Theme 2: Surveillance} P3 proposed a combined solution involving a \textbf{third party between PHC issuer and the user} alongside \textbf{encrypted credentials} to strengthen data privacy Figure \ref{fig:sketch_pilot_p3}. They suggested introducing an intermediate to oversee verification, explaining \textit{the current system is PHC issuer directly verifying the user's information…my idea is maybe some kind of third party could be here between them." }Furthermore, P3 recommended using multi-layer encryption with multiple keys, noting \textit{”The original identification is only visible with encryption…we should have maybe two keys…so that hackers could encounter challenges the whole time to decrypt." } By involving a third party and employing advanced encryption, this suggestion aims to safeguard sensitive information while fostering trust in the PHC system. M1 also presented the government's involvement as a supervisor of PHC issuers. The participant expected the government is involved in the issuance, storage, and auditing of PHCs, helping to prevent fraud and enhance system trust as explained \textit{"The government would need to be involved in the process of PHC issuance...They do audit checks and validate the storage of data, or validate the issuance of a particular PHC."}

\textbf{Design Theme 3: Protection from Potential Attacks}
M2 showed security concern on two types of potential attacks: social attack which tricks to have users to share the credential; system attack to database. As protection to such attacks, she proposed multiple approaches can be implemented - \textit{"There are a lot of methods that's already there. So for example, protection algorithm to make the algorithm or make the system more robust. And also there's a lot of like education training things to prevent social engineering."}


\begin{figure}[!t]
 \vspace{-5pt}
	\centering
	\includegraphics[width=\linewidth]{Fig/sketch_pilot_p3.png}
 \vspace{-15pt}
	\caption{Sketch by P3: Encryption and intervention by checking organization}
\label{fig:sketch_pilot_p3}
\vspace{-15pt}
\end{figure}


\begin{figure}[!t]
 \vspace{5pt}
	\centering
	\includegraphics[width=\linewidth]{Fig/sketch_pilot_p4.png}
 \vspace{-15pt}
	\caption{Sketch by P4: Sector-based Decentralization}
\label{fig:sketch_pilot_p4}
\vspace{-15pt}
\end{figure}

\begin{figure}[!t]
 \vspace{5pt}
	\centering
	\includegraphics[width=\linewidth]{Fig/sketch_pilot_p5.png}
 \vspace{-15pt}
	\caption{Sketch by P5: Selective Access to Information}
\label{fig:sketch_pilot_p5}
\vspace{-15pt}
\end{figure}

\begin{figure}[!t]
 \vspace{5pt}
	\centering
	\includegraphics[width=\linewidth]{Fig/sketch_m1.png}
 \vspace{-15pt}
	\caption{Sketch by M1: Government involvement in PHC Issuers’ system}
\label{fig:sketch_m1}
\vspace{-15pt}
\end{figure}

\begin{figure}[!t]
 \vspace{5pt}
	\centering
	\includegraphics[width=\linewidth]{Fig/sketch_m2.png}
 \vspace{-15pt}
	\caption{Sketch by M2: Protection from potential attacks}
\label{fig:sketch_m2}
\vspace{-15pt}
\end{figure}
