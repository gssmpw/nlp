\section{RQ1: Users’ perception of PHC}

\subsection{Current Online Identity Verification Practices}

\textbf{Verification Across Multiple Domains} The most common online services where participants encountered identity verification processes are financial services, including online banking and investment platforms. Several participants mentioned they are required to upload government IDs (e.g., social security number, driver's license) when creating an account; as P1 said, \textit{"I use the bank I need to upload my social security number for, such as PNC Bank and Discover. I don't really remember. But I'm sure I have uploaded my social security number."} In addition to financial services, the typical applications where participants experienced verification or authentication processes include social media, government services, and healthcare platforms. 
%This indicates that the scenarios we prepared for succeeding sessions are representative of typical user occasions. 

\textbf{Confusion Between Verification and Authentication} Many participants did not distinguish between verification and authentication. In particular, when asked about experiences with biometric verification, they often shared daily experiences related to authentication, e.g., (P3): \textit{"It's I think most of the apps required me such fingerprints or related identification. For example, when I open my bank apps, I can just use my fingers to login to my account."} Participants indicated frequent use of fingerprints or face authentications on mobile phones. For instance, P1 mentioned stored biometric information on devices, \textit{"Because my android phone already have my fingerprint to log into my phone. And then this app just to reuse my fingerprint, which is already registered on my phone."}

\textbf{Verification in daily life} Some of the participants shared their offline experience that required identity verification, as P1 described \textit{"So I need to check in in a hotel. I also need to show my driver's license. I don't know if that count."} 

\subsection{Perceived Benefits of Personhood Credentials}
%\fixme{there needs to be titles of sub-themes under this benefit, it's hard to read. Sub-themes could be -- convenience of not carrying physical id, etc -- whatever unique /interesting subthemes came up in the interview}
\textbf{Reduced Exposure of Personal Information} Participants identified a range of benefits, the most frequently mentioned being the reduction of personal information exposure. P4 expressed their reassurance in unnecessary data sharing to online platforms, \textit{"I feel less concerned. Because there's a platform which is designed for keeping my information, and I don't have to keep my personal information to any online platform that I have I want to interact with. Probably it's gonna be easier."} In addition to this, several participants noted the benefits related to minimizing data leakage. P5 explained the potential for data linkability, where information from other datasets could be combined to identify, \textit{"We also have different problem privacy that they can have some of our information from another anonymized data set. They can point out our previous record. So this will probably make it safer."} Participants also extended the discussion on this point to online malicious attacks, \textit{"So for benefits, the overall perspective of this PHC will be like minimizing the exposure of personal information to open world. We have faced problems like fake accounts and misuse of social security numbers. It can be reduced because of PHC. "}

\textbf{Efficiency of Verification Process} Another key benefit highlighted by participants was the efficiency of verification process. P3 emphasized the convenience of only needing to verify their official documents once, stating, \textit{"So one of the benefits is convenienceness. Users only verified their official document only once, and then after that they can always use the token as PHCs, so that's very convenient." }


\subsection{Perceived Concerns of Personhood Credentials}
%\fixme{there needs to be titles of sub themes under this benefit, it's hard to read }
Participants expressed various concerns, primarily revolving around data security and trustworthiness of PHC issuers. 

\textbf{PHC Issuer Being Hacked or Leaked} One of the most frequently mentioned concerns was the potential risk of PHC issuers being hacked or experiencing data leaks. As P3 explained, \textit{"The issuer really matters. If the data leaks from the issuer and is stolen by someone, then you know every single information will be gone, which is a huge problem."} It highlights participants' fears that the concentration of sensitive data with PHC issuers could have severe consequences for malicious attacks.

\textbf{Data Handling by PHC Issuers} Another significant concern was how PHC issuers manage and handle the credential data they collected. Participants showed uncertainties regarding what happens to their personal information after verification and whether it is securely stored or potentially mishandled. For example, P4 shared that \textit{"I'm not sure if they tell us explicitly how they're gonna use, do they going to share our information with 3rd party or with government. If they ask for it, I'm not sure about that. These are my concerns."}

\textbf{Centralized Power of PHC Issuer} Several participants expressed concerns over the centralized power held by PHC issuers, which they feared could lead to misuse or overextension of control. P4 commented \textit{"But I think there's a concern on the other side that they're gonna be one kind of you're gonna give all your information to a small group or an institution. So they have the power that can be abused later."} illustrating the apprehension about the concentration of power in the hands of a single entity as a PHC issuer.

\textbf{Untrustworthy PHC Issuer} The issue of trustworthiness in PHC issuers was also discussed, with participants expressing doubts about the reliability of the entities managing their data. Some worried whether these issuers could be trusted to protect their personal information and maintain their privacy. P5 described \textit{"The problem is, and the whole thing comes like how much you can trust the PHC issuer. If it's associated with government, there can be a reduced risk associated with like...But if it's an overall 3rd party, people will be thinking like, is it true? Is it a scammer?"}

\subsection{Perception Before \& After Educational Video}
As discussed in the method section \ref{sec:study_protocol}, we revisited questions about participants' understanding of PHC to compare their responses before and after watching the educational video to assess whether the video impacted their original knowledge of PHC. 

\textbf{Pre-understanding: Guessed it as one of the verification methods} The majority of participants were unfamiliar with PHC. When prompted to interpret the term based solely on its wording, most inferred that it referred to a form of verification method. (e.g., P3 noted \textit{"I think this one probably be another type of person identification method."})

\textbf{Post-understanding: PHC improves privacy and security} When asked to explain their understanding of PHC, P3 noted, \textit{"Personhood credentials tries to define a more trusted 3rd party that we can verify personal information and get some like passport or credential. Then we use different online services that we don't know what the sources are."} This suggests that the role of the PHC issuer is recognized as a crucial component of PHC.

\textbf{Post-understanding: Found similarities with other technologies} P5 interpreted the term as a verification method based on personal characteristics, which differ from person to person, such as behavioral patterns, commenting \textit{"It's like, personality based or like activities humans particularly perform. It sounds similar to the pattern of people walking. Those are also things people use for verification."} After watching the video, he updated his perception of PHC \textit{"I thought it was similar to SSL technology. Like in SSL, they will give the certificate to 3rd party who encrypts and has those certificates. To maintain their integrity, they will pass on the kind of certificate to web browsers. So when a user goes through a particular website, they will be able to cross-verify using the public keys available."}, linking it to technologies such as SSL, which uses encryption and hashing for certificates.
