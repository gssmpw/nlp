\vspace{-2mm}
\section{Related Work}
\vspace{-2mm}
We outline the background of personhood credentials to introduce a foundational concept.
%of personhood verification, which ensures the user is a unique human being. 
We then present the landscape of verification tools, including CAPTCHA, biometrics, and various other identification methods to discuss the limitation. Lastly, we discuss emerging practices of personhood verification to underscore the motivation for our work. 
%\fixme{write 2 lines to summarize what's been presented in this section} \ayae{reflected}
\vspace{-2mm}
\subsection{Background on Personhood Credentials}
\vspace{-2mm}
%\fixme{immediate work on detailed literature on papers and tools related to personhood credential to set a definition for PHC here and the current emerging tools}

\textbf{Proof of Personhood (PoP)} Verifying online identities has been particularly crucial for blockchain systems to mitigate the risks of impersonation and fraudulent activities. 
%While decentralized systems provide transparency, their anonymity can let malicious attackers deceive systems.
%A typical example is the Sybil attack, which manipulates peer-to-peer networks by operating multiple pseudonymous identities. 
%To counter such attacks, 
Proof of Personhood (PoP) has emerged to counter Sybil attacks, which manipulate peer-to-peer networks by operating multiple pseudonymous identities. It verifies an individual's humanness and uniqueness digitally on blockchain~\cite{borge2017proof} while preserving anonymity by linking virtual and physical identities.  
For instance, the Idena: proof of person blockchain runs a Turing test to prove the humanness and uniqueness of its participants~\cite{idenaWhitepaper}. 
%There has also been a PoP-based blockchain developed: 
Similarly, Humanode \cite{kavazi2021humanode, kavazi2023humanode}, which ensures one Human one Node, safeguarded by cryptographically bio-authorized nodes using 3D users’ faces. BrightID and Proof of Humanity leverage a social mechanism such as, a social graph and social vouching to achieve PoP \cite{shilina2023revolutionizing} and
%While BrightID is an identity system where the unique identity is endured through analysis of a social graph, Proof of Humanity relies on social vouching, which 
guarantees that the users exist and are not duplicates of another entity. Recent identity verification platforms reflected such social needs for PoP; Gitcoin Passport, and Civic Passserves as a self-sovereign data collection protocol with a PoP algorithm~\cite{shilina2023revolutionizing}
%and Civic Pass implemented PoP verification such as CAPTCHA, liveness, and uniqueness identity verification .
%\tanusree{need to add the off the self tools as reference. like PoH, BrightID, Humanode, Gitcoin Passport, Civic. add these as example of PoP to present that we covered both academic literature and off the self to come up with this definition} \ayae{reflected}

\textbf{Personhood Credentials (PHCs)} Stemming from the concept of PoP, personhood credentials (PHCs) have been proposed as an alternative online identification credential against AI-powered deception \cite{adler2024personhood}. They define PHCs as \textit{``digital credentials that empower users to demonstrate that they are
real people—not AIs—to online services, without disclosing any personal information.''} The fundamental architecture of PHCs includes three entities: User; Issuer; Service providers. First, the user gets PHC from the issuer by providing evidence to verify their identity. Then, the user can use PHC across different service providers without providing identity evidence again. It enables users to demonstrate humanness to online service providers without disclosing any personal information. 
An example of PHC implemented in the application is Worldcoin, which is based on PoP and Zero-Knowledge Proofs (ZKPs) \cite{WorldWhitepaper, worldHumanness, de2024personhood}. It maintains privacy and anonymity via ZKPs while employing the human iris pattern as a conclusive biometric marker for PoP. %They introduced World ID as a decentralized identity protocol that users can use to sign in to other applications.

\vspace{-2mm}
\subsection{Landscape of Verification Tools}
\vspace{-2mm}
\paragraph{CAPTCHA}
CAPTCHA \cite{von-Ahn2003-wr} has been the most common verification method to differentiate humans from bots. %but significant challenges remain. 
Early work %by Bursztein et al. 
emphasized CAPTCHA are often complicated for humans %to solve based on solving time and agreement level across three subjects for each CAPTCHA 
\cite{Bursztein2010-kf}. Similarly, Fidas et al. investigated the number of attempts needed to solve, finding that only 48.5\% solved CAPTCHA on the first try %and the rest 51.5\% needed two or more attempts
\cite{Fidas2011-ab}. %To address these existing issues on CAPTCHA, 
%Moradi and Keyvanpour 
Existing work provided a comprehensive review of methods and alternatives \cite{Moradi2015-is}, noting security concerns %(Blind Guessing, AI Attacks, Relay Attacks, and Side-Channel Attacks) 
and accessibility issues. Designing CAPTCHA that is fully accessible to everyone is impossible since users have different levels of %knowledge, ages, and 
abilities. More recently, former literature %by Tanthavech and Bimkoopmai
highlighted participants had low satisfaction with invisible CAPTCHA despite less burden on users %due to a sense of vulnerability
\cite{Tanthavech2019-dm}. 
%On the other hand, they perceived reCaptcha by Google as being the most secure.  It implies a certain level of self-efficacy and the credibility of the companies behind the system may influence their satisfaction. 
From a security perspective, Kumar et al. reviewed various types of CAPTCHAs, showing that modern CAPTCHA-breaking techniques are highly successful, sometimes with a 100\% success rate and the majority of them over 50\% \cite{Kumar2022-uk}. %Searles et al. also strengthen this fact by comparing observed human data and bots' performance from the literature\cite{Searles2023-db}. 
Another study revealed bots outperform humans in terms of solving time and accuracy across different CAPTCHA types \cite{Searles2023-db}. All in all, the current CAPTCHA systems have limitations in achieving optimal usability and robust security.

\textbf{Biometrics.}
Biometrics is another typical method of verification. Recent studies highlighted its popularity among users due to its high usability and convenience, as represented by fingerprint and face unlock systems~\cite{mare2016study, De-Luca2015-mp, Bhagavatula2015-fi, Zimmermann2017-wr}. %While these systems are favored for their speed and low effort, 
On the other hand, the authors shed light on usability issues in robustness to physical changes %(e.g., injuries or contamination) 
and external environments %such as low-light conditions 
\cite{mare2016study, Bhagavatula2015-fi}. Moreover, prior research revealed that participants have more privacy concerns about biometric authentication methods\cite{Zimmermann2017-wr}. %Wolf et al. extended the discussion about users’ concerns by comparing perceptions of security experts and non-experts \cite{Wolf2019-od}. 
Wolf et al. found that non-experts tend to have lower security concerns about biometric authentication, which is possibly due to a lack of understanding \cite{Wolf2019-od}. A recent study demonstrated the cross-context predictability of biometric data, resulting in the security risk of biometric authentication \cite{Eberz2018-eg}. Their results showed that attackers can improve their impersonate performance by collecting biometric data from multiple sources. 

\textbf{Humanness verification.}
Selfies and video calls have become popular %across various industries
as liveliness tests to verify users' identity and exclude bots. 
%Typically, after users provide one or more photos or capture a video, %by completing a series of actions, like turning their heads. After submitting their selfie, 
%these are matched with a reference image taken from their government-issued ID. 
Instagram introduced video selfies as a new option for users to verify their age, allowing age-appropriate experiences on the platform~\cite{instagramWaysVerify}. However, 
%the recent evolution of Generative AI resulted in new possibilities for incidents in this type of verification.
a Reddit post gained attention with the selfie and ID photo made with Stable Diffusion, which implies attackers can create deepfake ID selfies more easily than ever before \cite{techcrunchGenAICould}. Prior work \cite{hashmi2024unmasking} demonstrated human overall accuracy of detecting audiovisual deepfakes was only $65.64$\%, and people tend to overestimate their detection capabilities. Considering that deepfakes are becoming increasingly sophisticated, there are inherent limitations to using this method for verifying personhood.

\textbf{Economic identifiers.}
%As an economic identifier, the credit scoring can be highlighted as a way to evaluate user’s trustworthiness based on financial transactions. In the United States, credit scores are widely used to evaluate an individual's financial reliability for purposes such as loan approvals \cite{kiviat2019credit}. 
Various economic identifiers are widely used in our daily lives to connect multiple sources of information \cite{kennickell2016identity}. In the United States, %Taxpayer Identification Number (TIN), including 
Social Security Number and Employer Identification Number are used by the Internal Revenue Service to identify individuals and organizations for taxation and government operations. Implementing this identification number system ensured efficiency, fraud prevention, and fair tax obligations \cite{caplin1963taxpayer}. %Implementing this identification number system achieved not only efficient operations but also the prevention of fraudulent refund returns and the assurance of fair tax obligations \cite{caplin1963taxpayer}. 
In other countries, identification systems have evolved in different ways reflecting social backgrounds- China's Social Credit System assesses their citizens based on social and economic activities for trust and compliance \cite{creemers2018china,cheung2022datafication}. In contrast, Estonia has developed a digital identity system, offering secure access to a wide range of services %, such as banking healthcare, and online voting, 
for economic growth and efficiency \cite{martens2010electronic,tammpuu2019transnational}.
%We also discuss KYC as an economic identifier because it connects identity verification to financial activities, such as opening accounts or accessing credit \cite{bilali2011know}. 
%Another verifier, KYC~\cite{bilali2011know} ensures that individuals are legitimate by requiring official documents. %like government IDs and proof of address.

%However, more than 850 million people around the world do not have an official ID \cite{worldbankMillionPeople}, which may result in limited access for marginalized groups.
%\tanusree{there are many more economic identifier which may not be decentralized, but highly used, such as, Taxpayer Identification Number (TIN) – USA, SSN, Employee Identification Number (EIN), social credit systems in china and estonia, } \ayae{reflected}

\textbf{Digital identifiers.}
Among digital identifiers, public identifiers (e.g., phone numbers, email addresses) are the most popular methods in user verification. %with its simplicity and ease of use. 
Two-factor authentication (2FA) is a way to ensure user authenticity by sending a one-time passcode (OTP) %to the user’s phone or email 
\cite{aloul2009two, sharma2024can}. However, 
%a single user can easily obtain multiple identifiers compared to other digital identifiers, which 
obtaining multiple identifiers allows for the effortless creation of duplicate bot accounts. Devices also have personally identifying information, such as IP and MAC addresses, which can be used as a browser fingerprint, a unique identifier developed from the combination of client-side data \cite{laperdrix2020browser,zhang2022survey}. Recently, user-centric identity management has emerged as a key area of focus \cite{ahn2009privacy}. It respects users' autonomy as reflected in federated login like OpenID and OAuth \cite{recordon2006openid, hardt2012oauth, li2020user}. These developments introduced %another common federated login, 
social login, where social media accounts can be used to log into third-party websites and apps \cite{gafni2014social}. Blockchain technology has been implemented into a user-centric identity system as self-sovereign identity (SSI) \cite{toth2019self}. %As the Sovrin Foundation defined, \textit{"lifetime portable digital identity that does not depend on any central authority and can never be taken away."}, 
SSI enables users to mutually authenticate via decentralized identifiers (DIDs) and verifiable credentials (VCs) to give individuals control over their digital identities without intermediaries \cite{muhle2018survey, brunner2020did}.
%\tanusree{need to be comprehensive such as, add, social media profile. device based identifier like device id, MAC address, browser fingerprints used for verification, DIDs (Blockchain-based self-sovereign identities), WebAuthn (FIDO2).} \ayae{reflected}
%\subsection{Web of Trust}
%\subsection{Social Graph}

\textbf{Watermarking and Fingerprinting.}
%While the methods explained in the former subsections involve direct interaction with users or their data, this part illuminates the verification operating in the background to ensure the trustworthiness or authenticity of digital artifacts. 
Watermarking embeds invisible or visible markers into digital contents to identify whether it has been altered or forged \cite{van1994digital,mohanty1999digital,podilchuk2001digital}, whereas fingerprinting captures patterns %, which are unique to each individual, 
from a set of collected data such as device characteristics and metadata \cite{qureshi2019blockchain}. These techniques are utilized in detecting deepfakes by verifying content authenticity and identifying %unique patterns that can reveal signs of 
manipulation or synthetic generation. Barrington et al. developed an “identity fingerprint”
%, including features of the idiosyncratic facial, gestural, and vocal mannerisms of the person, to be used 
to evaluate the authenticity of multimodal contents \cite{barrington2023single}. The former study also proposed a system to detect deepfake videos using a speech-based hybrid watermarking method \cite{qureshi2021detecting}. However, a recent study demonstrated the impossibility of strong watermarking for generative models %under certain practical assumptions. They 
and concluded verifying the source of the data is more critical %than confirming whether it was AI-generated 
in many disinformation cases \cite{zhang2023watermarks}. While digital artifact verification enhances content trustworthiness without affecting user experience, it often leads the discussion back to the need for direct user verification methods. 
%\tanusree{add more reference in this paragraph as watermarking is a very well studied area where this section doesn't reflect, only citing 3 literature} \ayae{reflected}
\vspace{-2mm}
\subsection{Emerging Practices of Identity Verification}
\vspace{-2mm}
\label{subsec:verification_practice}
%\fixme{this section reads like method, make it sounds like (a) these are some emerging verification tool, (a) why relevant in some application area (c) limitation of human factor understanding, therefore in our work, we explore people's preferences, expectation, etc.} \ayae{reflected}
%\ayae{Reflected Professors' comment} %These needs to go to the literature review section on current usecase of PHC. And only 2 line summarizing why you chose the diverse type of credential data and refer to the literature section
Online identity verifications are implemented across various sectors to meet context-dependent needs, including finance, healthcare, social media, government services, and emerging technologies like LLM applications.
Financial institutions implement advanced OCR and AI technologies to verify government-issued identification documents, particularly for the remote onboarding process for new customers, and utilize biometric systems like fingerprint and facial recognition \cite{yousefi2024digital}. In healthcare systems, health insurance cards have been widely adopted as an identification certificate with electronic versions also being developed \cite{chen2012non}. Biometrics-based verification also plays a significant role in this sector, 
%We included fingerprints as credentials, as former literature investigated and 
as implemented robust fingerprint identification for electronic healthcare systems \cite{fatima2019biometric,jahan2017robust}. In this context, identity verification is crucial to protect patient privacy and ensure accurate delivery of medical treatments. 
As for social media platforms, %we reflected on industry practices, following the descriptions of the official website.
Instagram tested video selfie and social vouching as new ways of age verification, in addition to uploading their IDs, to protect younger users and build safer online spaces \cite{instagramWaysVerify, instagramTypesID, metaTypesID}. %From the list of IDs accepted by Meta \cite{metaTypesID, instagramTypesID}, we adopted the national identity card and video selfie for our scenario. 
In LLM applications, the current practice is simple verification through email addresses. However, there could be future cases where identity verification might be required to filter out malicious users.
%and applied a cutting-edge example of human verification. One of the tech companies, World, scans the iris pattern of the user as proof of human and issues an encrypted credential \cite{WorldWhitepaper, worldHumanness}. Inspired by this system, we exemplified iris scan for our LLM apps scenario. 
Regarding government services, there is a verification service provided by the US federal government called Login.gov, which allows users to access online government services \cite{LogingovVerify, shovon2018restful}. It requires uploading a photo of an accepted driver’s license or state ID card. In recent years, many countries have introduced such verification solutions to streamline access to government services. %Therefore, we regarded those documents as credentials in this scenario. 
For employment background checks, a tax identification card is sometimes regarded as a credential since it is often linked to employment history and income records, which aligns with this context. The fingerprint is also used to track criminal records \cite{cole2009suspect}. 

%This wider range of application areas indicates that identity verification is evolving to address diverse contexts. However, the previous literature referenced in this section has primarily focused on the technical aspects of verification methods and has not explored human factors involved, such as the challenges users face. Specifically, emerging verification concepts like PHC can be tough to understand and trust on them.
The evolving verification tools, in particular, emerging verification concepts like PHC highlight the need to address human factors often overlooked in technical research. Therefore, we aim to investigate these human aspects, providing insights into user perceptions and preferences that can be implemented into design implications.
%Based on these practices, we selected diverse types of credentials to reflect the unique verification needs and context-specific requirements of each scenario. This approach ensures a comprehensive exploration of how users perceive PHCs across different domains. 
%Table \ref{tab:scenarios} summarized credential types for each scenario of PHCs we considered in our study.

%Table2 \fixme{add a table summarizing the different types of existing digital identifiers related to PHC, definition, and example app-related}


