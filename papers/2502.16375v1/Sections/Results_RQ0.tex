% \section{RQ0: Current Online Identity Verification Practices}

% \textbf{Verification Across Multiple Domains} The most common online services where participants encountered identity verification processes are financial services, including online banking and investment platforms. Several participants mentioned they are required to upload government IDs (e.g., social security number, driver's license) when creating an account; as P1 said, \textit{"I use the bank I need to upload my social security number for, such as PNC Bank and Discover. I don't really remember. But I'm sure I have uploaded my social security number."} In addition to financial services, the typical applications where participants experienced verification or authentication processes include social media, government services, and healthcare platforms. 
% %This indicates that the scenarios we prepared for succeeding sessions are representative of typical user occasions. 

% \textbf{Confusion Between Verification and Authentication} Many participants did not distinguish between verification and authentication. In particular, when asked about experiences with biometric verification, they often shared daily experiences related to authentication, e.g., (P3): \textit{"It's I think most of the apps required me such fingerprints or related identification. For example, when I open my bank apps, I can just use my fingers to login to my account."} Participants indicated frequent use of fingerprints or face authentications on mobile phones. For instance, P1 mentioned stored biometric information on devices, \textit{"Because my android phone already have my fingerprint to log into my phone. And then this app just to reuse my fingerprint, which is already registered on my phone."}

% \textbf{Verification in daily life} Some of the participants shared their offline experience that required identity verification, as P1 described \textit{"So I need to check in in a hotel. I also need to show my driver's license. I don't know if that count."} 
