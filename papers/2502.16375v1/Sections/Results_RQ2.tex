\vspace{-2mm}
\section{RQ2: Factors Influence on How People Would like to Verify Themselves }
\vspace{-2mm}
%\textbf{Context Matters for Users' Preference}
%As outlined in the method section \ref{sec:study_protocol}, we explored 
In this section, we discuss various factors that influence people's preferences including application type, credential types, stakeholders as issuers, and architecture type. %across five distinct scenarios. %In this section, we present factors that 
\vspace{-2mm}
%\subsection{Resistance towards Digital Identifier}
%\subsection{Reevaluating Offline Experience}
\subsection{PHC Onboarding: Online Vs Offline}
\vspace{-2mm}
Participants identified the onboarding process as a key factor in managing PHCs. While some preferred a hybrid approach over a fully online system, others favored a fully online process for its convenience. Many drew parallels to how they opened their first bank accounts, emphasizing the importance of in-person verification. They described visiting the bank, presenting their IDs and passports to a bank official, and then receiving their account. Reflecting on that experience, participants indicated that PHCs, especially for financial services, should be issued in a similar way. One of the perceptions was the high risk of security of a fully online system where user would use their own browser at home to upload certain ids and passports to receive PHC compared to bank officials doing the same thing in their protected system. P22 said \textit{``There is a high risk with a fully online system where users would use their own browsers at home to upload certain IDs and passports to receive a PHC. Compared to that, having a bank official verify the information in their protected system feels much safer.''} 
%This sentiment highlights the perceived security benefits of involving intermediaries in the PHC issuing process. 
However, some highlighted the impracticalities of in-person verification as P18 mentioned \textit{``I live in west coast, now i need to travel to a location to have my irises scanned by the Orb device, which is not practical for me.''}


%Mental model: security perception Online vs. Direct Data Submission for verification
%#######"I think the benefit is that a lot of these details would be something that I would know about myself. So, for example, when I try to call the bank say, for example, I have an issue, and I call the bank they would have to verify that I'm the person who is the account holder. So in that case they would have to ask me verification questions, such as maybe my postcode or my date of birth or my full name or even in like, maybe my phone number. But any kind of personal information that would then tie back to me. And if say, for example, the benefit is that if someone else were to call in my case but they didn't know these details, and the bank wouldn't give out My any of my personal information, or the bank would know that my account is compromised, and then they could take quick action, maybe freeze the funds in my account. And yeah, so I think, like, Phc is important" (P3)
%######"I have done online healthcare with like telemed where they watch you on a camera. And you tell them, hey, I think I have a cold. I have a runny nose and a sore throat, and they say, Yeah, you have a cold I've done that before. I don't have any problem with that. They in this case they did not. I mean they they asked some verbal questions like, What's my address? What are the last 4 of my social security number. And that's that's information I don't mind giving out, because I know who's asking for it, and I know why they're asking for it. And I don't have any problem with that giving out the last 4 of your social. That's not a problem. Nobody can steal your entire social security number just from having a piece of it. However, my doctor's office when I go in person they have a microphone in the in the treatment room and the microphone records everything that I say. And it's recorded by AI computer. Who then trans poses that and puts it into my medical medical history, and I do not like that, because there are some things I tell my doctor. I don't want them to write down. And so, before I go into the treatment room, I ask the doctor to turn the AI off so that he has to manually write down the notes and have them put into my record. And what what I tell that doctor at that point is now personal. I don't feel like everything I've told him is being recorded. So that's my personality. I just don't like everything, because the more information that goes into something the more can be inappropriately used." (P8)


%particularly for high-stakes services like financial transactions.
%\fixme{Credential: resistance to phisycal ID carrying.  I remember interesting quotes like buying alcohol and some scenario people mentioned where carrying physical id all the time is cumbersome. then "Across border: Limitation in access/interoperability in verification method"} 
%\ayae{Unfortunetely, I observed this theme only in pilot study. Could we include them?: https://docs.google.com/spreadsheets/d/1PxN9fYU1CT5AuELanIjmtY6900U35gHn/edit?usp=sharing&ouid=102374555302641952762&rtpof=true&sd=true} \fixme{yes, please add the quote here, I also some across in my interviews as well} 
%\ayae{reflected on 1/9}

%Pilot P1 -- "But it's not a very convenient sometimes, because I have to take the physical id like in the Us. Like my driver's license or my passport. So it's not very convenient, and in China. We don't usually take our id with us, because, only when we need to like, go to a hotel or go to the Government department for some service. We we only take our id that time, but in the Us. I have to take my id yeah, every day."

%Pilot P2 -- "I think this information may be lost, or maybe stolen by other people. So it's a risk. Think the passport and the Id card. This this information may may be lost, or maybe this is the account."

%Main P8 -- "Everything is able to be stolen so. It might work in the short term until people that do bad things figure out how to steal it and use it. But you know. we came up with paper versions of authentication, like a driver's license, because we needed something to identify people. But now we realize that that's not as secure as it can be. So when you go to, maybe start a new job, you have to bring more than one form of identification to prove. So I I think that maybe in the beginning, using biometrics would feel secure, and it would be until it's been until a way is found to misuse it. just like a paper identification, like a driver's license."

\iffalse
\vspace{-2mm}
\subsection{Types of Application}
\vspace{-2mm}
% \fixme{this needs to expand a lot,  and one of the main results. we also have post survey for this result. move some visuals here and rewrite this } \fixme{if these figures are explained in later section, why did you add this paragraph here? this section is about application types. start with a outline like "participants had various preference across application types..... For instance, for X, Y , Z applications, .... ...."} 
Participants had various preferences across application types, reflecting context-specific needs in identity verification. 
%For instance, finance and healthcare services were commonly regarded as having a higher level of security and trust due to their sensitive nature. When we asked their perceptions regarding PHCs in healthcare scenario, P2 said \textit{"Still pretty similar to the banking example. Because, in the health context, it has specific regulations to protect data privacy, because it's still a high risk context. So I think it's my perception would be pretty similar to the banking service."} P17 highlighted similarities while also addressing nuanced differences in the sensitivity of the data, stating, \textit{"I feel similarly to the banking account topic. I would say that I feel even less concerned with these health applications, because it's also private information. But I don't think that it's as dangerous that someone has that information compared to my banking information."} Given the nature of finance and healthcare applications, 
%Participants expressed a positive attitude toward using PHCs, recognizing their value in enhancing security and privacy. \textit{"I think if PHCs hold, identification is secured. Because I think my personal data is not going anywhere. I think me using PHC in healthcare system is also the same with the financial. Because keeping my data secured is the main important thing."}
%Participants’ responses indicated their preferences are depending on each scenario where different type of applications are considered as service providers as seen in Figures \ref{fig:credential},  \ref{fig:issuer}, \ref{fig:architecture}. Results of preferences regarding credentials, issuers, and architecture will be addressed in subsequent sections. This section discusses the variations in participants' acceptance of PHCs observed across different scenario sessions. 
 %\fixme{not adequate, I don't see any preference regarding financial service and related quote, healthcare and related quote, govt and related quote and employment background and related quote, do add those. You added quotes for the less important ones, like social media and llm. Do keep those. I will decide later. add the quotes from other application type. you are supposed to explain similarities and difference on people's preference of PHC in different application types here.} \ayae{reflected on 1/9}
 Regarding government, finance and health services, most participants were positive about using PHCs. As P4 explained, \textit{"I think it'd be happy to go ahead and use PHC if it's something that you're applying for, like you said benefits or something like tax breaks, then you've got those factors to consider...probably because they already obviously hold a lot of information about you for working purposes, tax purposes, all that kind of thing. So if you're only providing verification, I think maybe not going to be a major concern apart from the usual. "}, they felt less concerned since the government already holds a significant amount of information about them. In the employment background check scenario, participants showed receptiveness to using PHCs, for instance, P9 recognized the non-invasive nature of data collection in PHC, saying, \textit{"I really don't love with job applications having to provide so much information before you even have the job. It feels very invasive, but if all I have to do is show my personal credentials, which come from something as simple as like a fingerprint or an iris scan. I really like that. That doesn't feel invasive."}
 In contexts like social media and LLM applications, participants viewed PHCs as unnecessary, favoring simpler verification methods like email or phone numbers. For instance, P5 noted \textit{"For social media apps, I really do not feel comfortable doing that...I would seriously consider not using it and use another app that does not ask for such information. First of all, they are regulated, but not like banks.
 %They have more freedom to do a lot of things that I may not agree with...They might use your information and your tweets or messages for a lot of other reasons, like training their large language models. Because of that, such institutions can use your information for many reasons are in the news business a lot of times, and I don't see any reason of providing PHC information for them.
 "} Moreover, P7 suggested bot's behavior prediction instead of PHC as a way to protect the core model in LLM scenario, saying \textit{"
 %I think there are ways that they can predict attacks and they can prevent it. They can design some triggers, so that before their core models being under attack by defaults, they can prevent that and shut down the accounts very soon...So I think it make more sense from the corporates to take those steps as opposed to asking us to provide them with our sensitive information, so that they core model would not be attacked.
LLM companies should proactively predict and prevent attacks by designing triggers to shut down compromised LLM account users, rather than requesting sensitive user information to issue PHC to protect core models."} In the same line P23 noted that- \textit{``I and my sister use chatGPT in a shared premium account because its costly to pay for 2 accounts. PHC means one person one account, so not working for us.''}
%This contrasting differences in trends between finance and healthcare applications and social media and LLM applications are also evident in the quantitative data.
 %Figures \ref{fig:credential}, \ref{fig:issuer}, \ref{fig:architecture} show similar distributions for finance and healthcare, as well as for social media and LLM applications.
 %P6 stated, \textit{"I love to talk, to, chatgpt, and bard about weird stuff, and I think a lot of people like to talk to them about weird stuff. And so I would like the option to not log in via biometrics, and have a lesser experience and not have my identity tied to it. So I personally, I would not like to have my identity tied to my chat Gpt, or my bard, or Gemini."}


%Trust with healthcare services
%######"Yeah, I'd be absolutely fine with that. I mean, here in the Uk, we have the Nhs, so we have kind of a system of like identity verification. We all have, like an Nhs number which is associated with us as a person. And so I think that data kind of already exists for healthcare for us. so it wouldn't make me feel nervous giving them any more information about me personally, because, yeah, they kind of already have it all." (P11)

%Distrust with LLM companies
%###### "Well, I already have privacy concerns with AI models like Chat Gpt. When I created a jet chat Gpt account, I created a new email address so that it didn't have any of my information from my current email address. So I use a different email address for Chat Gpt, and the reason I did that is because I don't know how the data is being used. The other problem I have with Chat Gpt is, if I give it personal information, and someone misuses that information. I'm now at risk. And so I've never given any personal information that could be used against me like I haven't given it my place of birth. I haven't given it the name of my 1st pet. I haven't given it, you know, the 1st car I ever drove, because those are security questions that are used when you call your bank and they have to identify you. So I'm I don't give it my social security number, you know. So it's really up to you to protect yourself. just like when you're walking down the street. If you see a situation that you think is dangerous. Why would you keep walking down the street? You would turn around and go the other direction so that you're not putting yourself in danger. And I think we have as humans. We have to be wary of chat Gpt type models because we don't know how that information is going to be used in the future, or how it's even being used now. In fact, the people that created jet chat gpt, they don't know what they're gonna do with all the information? They're just collecting a lot of information right now and trying to find out. " (P8)
\fi

\vspace{-2mm}
\subsection{Data Requirements to Issue Credential}
\vspace{-2mm}
% \fixme{Are the themes added in this section from the result of section 2. please incorporate themes from section 2 scenario based discussion. we also have post survey for this result. move some visuals here and rewrite this } \ayae{reflected}
Data requirement in issuing PHC credentials was one of the main factors. Participants generally categorized data requirements into three main types: \ding{202} Government-issued IDs, such as social security numbers, passports, and driver's licenses; \ding{203} Biometrics, including face, fingerprints, and iris; and \ding{204} Digital identifiers, such as phone numbers and email addresses. To grasp their preferences comprehensively, Figure \ref{fig:credential} presents quantitative results of credential preference from post-survey.
%(The questions are listed in Appendix \ref{post_survey}). 
For instance, a government-issued ID is the most preferred data to issue PHC across applications except social media and llm applications. Phone number is most preferred in llm application followed by Iris scan.

%\fixme{you need to explain the figure 5 Post-Survey Results: Credential Preference briefly here, what it represent, only showing figure is not enough, very bad practice considered by reviewers. just add a line to sum up the figure} \ayae{reflected}


\textbf{Familiarity with traditional (e.g. govt id)} 
Several participants highlighted their preference for their PHC associating with government-issued IDs as they were most familiar with this approach and perceived it as reliable.
%noting their trust in widely recognized and standardized forms of verification\fixme{it seems like you somehow only focused on trust aspect of physical id. think broader. this is very repetitive. people not only prefer govt id, they actually talked about physical form of id. then say more eprevalenly govt id}. 
To add some complexity, P11 mentioned the practicability of different types of government id- \textit{``Driver's license is pretty common, and  we usually bring that all the time. It's easy to bring and easy to take a photo and upload. A passport or a social security number is not the one that people usually bring. So if we need to verify that we 1st need to come back home and then search it, search them and then provide it. So it takes, additional steps. But a driver's license pretty easy.''} 
%I would probably prefer to use just a standard identity document like my passport or my driver's license or something official because then I would know that everyone else was using official documents. So it was all above board, and it was reliable. I feel like it's more reliable if we're using like official documents.

%\ayae{Comments focusing on the physical aspects were seen in the pilot, can I include them? >> quote from pilot study "Driver's license is pretty common, and you know we usually bring that all the time. It's easy to bring and easy to take a photo and easy to upload. That's the reason. A passport or a social security number is not the one that people usually bring. So if we need to verify that we 1st need to come back home and then search it, search them and then provide it. So it takes, additional steps. But driver's license pretty easy." }
%At the same time, P11 compared their preference to biometrics, stating \textit{"I trust it more because, for example, like my passport, it proves that I am me and that I am a resident in this country, whereas biometric data, although it is it? Is you? There's no kind of attached to it. It's giving you like more validity as a person." } 

\iffalse
\textbf{Ease of Biometric Verification}
\fixme{in first line mentioned a list of benefit participants mentioned about biometrics, like 
Biometrics is easier no need to remember credentials and if there is anything. For any themes first things to provide the overall result as a list and then add quotes that's interesting } \ayae{reflected}
Among the participants, some indicated a preference for biometrics, highlighting its significant convenience and ease of use. P4 expressed \textit{"I think maybe the biometric. Maybe iris scanning probably my preference. Seems like the easiest one to do. I think it'd be the ease of it, taking the selfie sort of thing and uploading that. And obviously it's seems kind of more fun, I would say, compared to maybe doing some of the other ones. It seems like pretty quick way to do it. And I go with that."} This participant underscored data minimization with biometric, noting \textit{"Because obviously, the government id is gonna have a lot of your personal information. So I think that's the thing we're trying to minimize here. So with the iris scanning I think it's just the minimal and still being secure."} 
P8 emphasized that biometrics eliminate the need for users to remember passwords, stating, \textit{"
Well, the benefits, of course, are simplification. I don't have to remember a password for everything. I can just use biometrics like my face, or a fingerprint, or something like that, or an eye retinal scan."}
%PX ------ As a biometric authentication method compared to like face id, which, although is very convenient, is not as easy or as friction free as fingerprint id. So that's 1 thing which I I really would like Apple to bring back. Patch id for the iphone. That's 1 personal comment. But in general I think they do perform well enough for me to not have any problems with them on a day to day basis.
\fi

\textbf{Sensitivity, Security, \& Efficiency Across Different Biometrics}
We found that participants favored biometrics due to their ease of use, no requirement to remember passwords minimized exposure of personal data, and their functionality as standalone credentials. Highlighting efficacy and privacy P4 stated - \textit{``government id is gonna have a lot of your personal information. biometrics here minimize data-just biomarker. like with iris or fingerprint, it's just the minimal and still being secure.''} 
We also found that individuals have varying preferences across biometrics types, such as fingerprints, facial recognition, selfie, and iris scanning. Firstly, participants considered fingerprints as less invasive and less sensitive.
%\fixme{add a quote that present why fingerprints are less  invasive. the quote you added before is not representative}
%as reflected in the following comments: \textit{"I feel like a fingerprint is definitely preferable. It just seems easier. It also seems less invasive. (P9)"}, \textit{"For me, and I understand that fingerprint a meaningless, it's just my personal preference. (P13)"} 
In contrast, facial recognition was discussed as a more sensitive biometric method, as illustrated by P5, \textit{"I would say fingerprint is fine, but face is too much. you can be identified in public settings in streets. If they have camera [surveillance], anyone can just trace your whereabouts."}  Regarding iris recognition, most participants expressed their views without having direct experience with this method. Notably, P8 said-
%discussed that the iris might represent a more secure biometric method, 
\textit{"iris verification is probably the more secure of all of the biometrics. Because I know that fingerprints can be regenerated, and facial recognition can be regenerated. But irises are not easily generated. The eye is a very complex organ in the body. So if I had a choice between the 3 forms of biometrics, I would choose iris, because I think it's the most secure."} In content of efficacy, P7 emphasized the efficiency of face compared to fingerprints, stating \textit{"I think facial recognition is better and more technologically improved, it is quick.''} On the same note, 
P23 highlighted challenges with their father fingerprint, noting that it had become difficult to identify due to the type of construction work they engaged in, causing the prints to blur over time.
%Biometric depends on devices
%###### "First, I use Pixel, Google, Pixel and Google Pixel uses fingerprints. At the back of the phone. So in order, one of the ways that you could identify yourself was to use the fingerprint reading... I switched to iphone. And then they also change the verification with the platform change as well. So because iphone they don't have like a fingerprint." (P7)
%###### "So as an iphone user, I use like on a on a Mac user, I use touch id and face id a lot in my like regular authentication practices. So I guess the benefit of touch, id or face id is the verification happens on device or on apple side, where they store your fingerprint or face data, and they only provide this as like a client service for all other apps or all the other services that want to verify your Id. So that way, it gives me a little bit of confidence that my biometric data doesn't leave like Apple's ecosystem, or outside of my device, and gives me that sense of security that I can give biometric access as a way to log in, because it's a very convenient way to log in across different services." (P1)
%###### "I think most of the Id checker platforms I've used or like anywhere that I've had to upload or any anywhere that needed to check my id or verify my id like they've mostly will ask for my face, because the phone that I use that's like it's an iphone 14 pro, so it doesn't have like that fingerprint option anymore. So you just do like a face id scam. So most places I've seen so far recently, I haven't come across any fingerprint verification. It's mostly been like they would make me take a photo of myself, and I have to fit my photo in a frame that they provide, and then they use the photo I provide of myself against the Id that I've provided to verify that I'm the person on the Id." (P3)

%Perception about biometric: face is faster than fingerprint 
%###### "I think I found with fingerprint recognition. Sometimes it can seem a little glitchy like there are times when I've been. You're inputting your fingerprint, basically. And it seems to get stuck. Or it won't fully be able to scan for some reason, whereas I found, with the facial recognition that seems to be more like user friendly. And faster as well. The facial recognition seems faster." (P13)

%Perception about biometric: it will change after certain period time
%###### "So if, for example. The only time I think the photo wouldn't match is say, for example, if the photo was really really old, say, when you were a child, and now you're 20 years older and you look very different. So I think that's the only time when maybe the software that's used to create like a match between the 2 is going to have difficulty, whereas, like with a fingerprint scan. it's fingerprint scan. " (P3)
%####### "Yeah, that's a very good question, because write the exact that that this time time frame I I use 2 different mobile phones. First, st I use Pixel, Google, Pixel and Google Pixel uses fingerprints. At the back of the phone. So in order, one of the ways that you could identify yourself was to use the fingerprint reading. So it wasn't I. I wasn't successful, as I mentioned, because of the passport and the issues with Robin Hood. and I think it took couple of weeks or even month, and I switched to iphone. And then they also change the verification with the platform change as well. So because iphone they don't have like a fingerprint." (P7)


%. So I use a stand on my desk, and I place my phone on it...So whenever I put my phone on the stand on my desk and I need to log into something, I don't need to reach my phone and use my fingers instead, the iPhone can read my face, and I would have a much quicker login, I would say that's the one of the reason that I would prefer."}
%\fixme{another code I saw "Perception about biometric: face is faster than fingerprint" this seems to capture efficiency aspect. add an interesting quote on this. Also about this "Perception about biometric: it will change after certain period time" - add a quotes} \ayae{reflected. pause the second theme which was found only in pilot study.}
\iffalse
\textbf{Simpler Verification with Digital Identifiers}
%Credential preferences varied across scenarios, with notable differences observed particularly in the social media and LLM application scenarios as shown in Figure \ref{fig:credential}. 
While government-issued IDs were the most dominant choice in other scenarios, phone numbers were the most preferred option in social media and llm application scenarios (Figure \ref{fig:credential}).
%these two scenarios. 
P6 described \textit{"If I have to require something for issuing PHC, I would rather it be the phone number. I would even just give them an email address, more anonymous than a phone number. For my personal use of AI, I don't feel a need at all for them to know who I am."} 
%Participants preferred to remain anonymous by using simpler verification methods through digital identifier credentials.
\fi

\textbf{Combination of Multiple Credentials}
Participants emphasized the importance of using multiple type of credentials to enhance security. P13 indicated this preference, \textit{combination of facial scan and fingerprint combination. If you're wanting to prove this is the person. Then it would make sense to have more than one biometric. I don't know if it's possible to fake somebody's. If there are completely two different biometrics, it would be more difficult to fake.''} In cases like financial service, many participants preferred PHC issuance based on both physical id (e.g. gov id) and biometric (e.g. iris).
%\textbf{Trade-off on Digital Identifiers} Participants also shared their preference with a trade-off on digital identifiers, noting both risks and advantages. P3 acknowledged the potential risks with \textit{"Email address is more likely leaked or stolen by other people. So I think it has many risks related to using an email address."} However, P3 preferred using an email address as a credential, supplementing by \textit{"But on the other hand, an email address will not leak my personal information too much. so I think, for the pros and cons, I like this."} 

\begin{figure}[!t]
    \centering
    \begin{subfigure}{0.48\linewidth}
        \centering
        \includegraphics[width=\linewidth]{Fig/credential_chart.png}
        \caption{Results of credential preference: Which types of credential would you prefer to use as personhood verification for each scenario? (Multiple selections were allowed.) }
        \label{fig:credential}
    \end{subfigure}
    \hfill
    \begin{subfigure}{0.48\linewidth}
        \centering
        \includegraphics[width=\linewidth]{Fig/issuer_chart.png}
        \caption{Results of issuer preference: Which type of issuer would you prefer to issue and manage your PHC for each scenario (Multiple selections were allowed.)}
        \label{fig:issuer}
    \end{subfigure}
    \vspace{-10pt}
\end{figure}


\vspace{-2mm}
\subsection{Stakeholder Types}
\vspace{-2mm}
% \fixme{Are the themes added in this section from the result of section 2. please incorporate themes from section 2 scenario-based discussion if there is anything interestingthere. we also have a post survey for this result. move some visuals here and rewrite this. moved some paragraphs from RQ1 which is suited here. Try to make coherence, remove repetition} \ayae{reflected}

\textbf{Control and Practicality: Preferences of PHC Issuers.}
Participants expressed varied levels of acceptance regarding the preferred issuers of PHC issuers. Across all application scenarios (Figure~\ref{fig:issuer}), government entities emerged as the most trusted issuers for the majority of participants, followed by nonprofit organizations (NPOs). P5 noted-- \textit{"Government is my preferred. IIf there are certain organizations are leveraging 3rd party organizations, they should be regulated and under government supervision. The least favored one is private companies without supervision, like commercial companies doing their own."} 
However, practical considerations influenced participants' views in certain domains, particularly social media and LLM applications, where private companies were rated as acceptable issuers by some. As P19 said \textit{``I don't see it happening where social media will involve govt vetted PHC. its just not practical. and truthfully, i don't want gov issued phc for social media, i am not fool to allow govt to another layer of surveillance.''} In contrast, a group of participants favored nonprofit organizations for domains like healthcare and social media, valuing their balance of trust and regulation --P10 stating \textit{``So NPOs I feel like they do have that government backing, they could be another trustworthy source, but not as intense as the govt. to balance it out.''}This contrast underscores the complex trade-offs individuals consider when evaluating trust, regulatory oversight, and practicality in selecting entities to issue PHCs. 
%== discussion sectionWhile government oversight remains the most trusted option, concerns about privacy, surveillance, and domain-specific practicality shape preferences for nonprofit and private entities in certain applications.
%It showed preference in managing PHC under government supervision and further explained that they did not extend the same recognition to private companies or organizations without governmental restrictions, which was 
%Interestingly, participants chose NPOs 
%particularly evident against social media platforms. P1 expressed their concerns, stating \textit{"I'm I don't think I'd be comfortable yet, at least, looking at like a social media organization or an organization that does not have governmental restrictions or things like that for sharing their data."} 
%This contrast highlights the complex considerations individuals have when evaluating trust and level of regulatory strictness of different entities for managing private data in PHC issuing process.

\iffalse
\textbf{Context-based Issuer Preference}
As shown in Figure \ref{fig:issuer}, the distribution of preferred issuers varied across scenarios, aside from the government being the most favored option. 
%For example, financial institutions were the second most preferred after the government, while private entities were more commonly preferred in the social media scenario.
In the social media and LLM scenarios, NPOs were selected more frequently than in other scenarios. As a reason for this preference, P16 expressed concerns about the use of data for profit-driven purposes, stating, \textit{"I think, mostly because of the information, like tracking and sharing situation that already happens. Like a social media or like tech company, that provides this for you, if one tech company were to be responsible for that, they would just aggregate so much personal data and also sell it because if they're for profit.  How else are they going to make money? So I think that's mostly."}

%P10 --- "So my first thought is, it seems, bizarre to have the government control social media. And then again, we talked about the selling of data with private companies. So I was more hesitant with that. Finances didn't make much sense, and neither did education. So NPOs I feel like they do have that government backing with their type, their credentials as an NPO. So they could be another trustworthy source, but not as intense as the Government, and it would be more people would be less hesitant to be and my data is going right into the government. That's why I picked NPO."

%\textbf{Trustworthiness of PHC Issuer} The issue of trustworthiness in PHC issuers was also discussed, with participants expressing doubts about the reliability of the entities managing their data. Some worried whether these issuers could be trusted to protect their personal information and maintain their privacy. P7 described \textit{"My only concern is, how do we trust the Phc Am I using it? Yeah, because if if they are the one entity who handles all these kind of information, and if they are collecting everything about us from the government. And whether this entity is considered like a NGO, or that's a governmental body, or that if that's a private sector, so these are like some nuances about how to define or how to build this kind of issuer, because we have these giant issuers like Equifax or like these giant companies who are handling credits or those kind of financial aspect of the US structure. And they're like giants. They have these kind of monopoly over how they handle things. Is it going to be something similar? If so, how people can trust them? I don't want to share my personal information anymore, with like some random website...But how the way that the issuer can be understood? I think that would be the decision factor for users to start engaging with the system."}
\fi
\textbf{Trustworthiness of Stakeholders}
%\fixme{make this para concise, you can trim the quote as well and shorten, doesn't need to have entire quote, can edit. this theme should not take more than a para} \ayae{reflected}
When discussing the current verification process, participants often indicated that there are organizations they can trust and others they cannot. 
For instance,  Some worried whether these issuers could be trusted to protect their personal information and maintain their privacy. P7 described these concerns and discussed how trust in issuers can be a deciding factor. \textit{"My only concern is, how do we trust the PHC am I using it? Because they are the entity who handles all these information, and they are collecting everything about us from the government...I don't want to share my personal information anymore, with some random website...But how the way the issuer can be understood? I think that would be the decision factor for users to start engaging with the system."}
%Considering the importance of stakeholder trust, we asked how or why they have trust with specific stakeholders. Based on their responses, 
Trustworthiness are associated with various aspects of stakeholders like regulations, customer support, past experiences with issues.
P5 explained his trust in banks comes from regulatory aspects: \textit{"They are being monitored by federal agencies. Their activities are monitored. They are under a lot of regulations. So there is a monitoring system that is tracking banks."} P4 emphasized her trust in banks with accessible customer support, which allows individuals to review the necessary steps and address any concerns they might have.
%P4 highlighted another aspect resulting in their trust in banks,  which is that the accessible customer support, stating \textit{"You know what sort of steps have taken so you could probably have a read of that, and any concerns that you might have. So you probably always have the opportunity that you could always speak to them first, because obviously they probably have agents available that you could speak to about any concerns. "} 
On the other hand, we observed that many participants expressed distrust towards social media platforms. P3 shared an experience of her social media account almost being hacked, 
noting \textit{"Someone tried to hack into my account...I've had people send me really dodgy links, which I knew instantly, that if I clicked on that link, it would mean that my account would get hacked as well."} 
%\fixme{make this paragraph concise} \ayae{updated}
%\fixme{following quotes and evaluation of trustworthiness not clear}\ayae{cut following quotes, no significant implications to RQ}
%Some participants elaborated on how they evaluate the trustworthiness of companies or stakeholders when they share their sensitive information in general. 
%P6 reflected \textit{"I guess if the company was suggested to me, or I was referred to the company through the services of like Capital One or a big company that's been around forever. That would give more validity to me."} 
%P7 provided insight from his own experience with a suspicious verification process, saying \textit{"I do a lot of research… What I did was I went and searched on the Internet. So I went on Reddit, and I looked for different keywords, and my conclusion was that it's a very standard process. It's not like the dealership wants to collect the social security for themselves. They wanted to do the background check, and they wanted to see if the information that they provided was matching with my information from the US. "}

%\textbf{Variability in Trust Among Individuals} While there was a general consensus on preferred entities for data trust, individual perspectives varied regarding which organizations were considered trustworthy. For example, P4 expressed distrust toward the government, commenting \textit{"Government is not kind of trying to protect most of the citizens’ privacy and security, but at the same time it’s going to give more power to the government."} They also showed relative trust in social media companies, explaining \textit{"They probably invest in the privacy and security part. So I’m kind of less concerned."} These contrasting viewpoints highlight how personal beliefs and experiences shape individual trust preferences, underscoring the diversity of opinion in stakeholder trust.
\vspace{-2mm}
\subsection{Architecture Types}
\vspace{-2mm}
%\fixme{before you start any subsection or section add 1-2 lines to summarize. for example people had a range of preference on architectures. Architecture preference: decentralized improve security of centrailized data storage, Architecture preference: decentralized mitigates privacy concerns, Architecture preference: decentralized-users can choose preferred issuers, Architecture preference: decentralized and sector-based, Architecture preference: clear regulation/policy for decentralized,Architecture preference: centralized with biometric and govt. ID Architecture preference: centralized for simplicity Architecture preference: middle - decentralized but oversight by gov. First summarize these in 2 lines and then the interesting themes you discussed below} \ayae{reflected} 
We observed diverse preferences for PHC architectures, with participants highlighting decentralized models for improving centralized data security and enabling user choice of issuers. Others preferred the simplicity of centralized systems, and hybrid models blending decentralization with government oversight.
 \begin{figure}[!t]
 \vspace{5pt}
	\centering
	\includegraphics[width=0.6\linewidth]{Fig/architecture_chart.png}
 \vspace{-15pt}
	\caption{Results of architecture preference: Which type of system would you prefer to issue and manage your PHC for each service provider?%\fixme{caption needs to be self explanatory, keep consistent graphics and color for all graphs}
    }
\label{fig:architecture}
\vspace{-15pt}
\end{figure}

\textbf{Decentralized vs. Centralized}
Participants highlighted the risks of centralized storage, preferring decentralization for its enhanced security. P8 stated \textit{"
%Definitely decentralized because 
you've created a wall between any problems with data theft, with a centralized data storage. They only have to breach the walls of one castle with decentralized. They have to breach the walls of two different castles, which makes it a lot more complex and most people won't. Most people won't go through the trouble of doing that. 
%Like, I said, if you're really dedicated and you will.
"} 
%\textbf{Simplicity of Centralized Architecture}
In contrast, participants who expressed a preference for a centralized approach highlighted its simplicity and ease of use. P21 stated \textit{``I feel like central authority can respond to any threat aftermath more effectively for centralize issuance and I guess central, like if gov is issuer then it will have global legitimacy, i can use it for customs.''}
%P4 stated \textit{"They both got benefits. But because the ease of simplicity maybe the single PHC issuer probably. So you've got the different sort of providers, the health and the the financial and you've got the trustworthy sort of should in this case, the government which can be used for the multiple applications. So you got the user, then it can use that same Id and on multiple platforms, just making ease of use."} This perspective underscores the appeal of centralized systems for their ability of identity management across multiple platforms while maintaining trust through a single reliable issuer.

\textbf{In Between Centralization and Decentralization}
Some participants expressed preferences for issuers with a balanced approach of centralized and decentralized systems. 
%P9 explained \textit{"For instance, one participant shared, "My initial reaction is that I would prefer centralized because it just seems easier. It seems a little safer that there's just one PHC I'm getting and it also just seems easier for me as the user. But I now only have one PHC to remember or hold on to and pass on to various people. But then it's a little tricky, because I see the pros of decentralized that it might be a little bit more secure because I now have, if someone hacked or stole, or whatever my PHC, they don't have access to everything if it's decentralized. So they might only get my social media or like banking or healthcare services would be secure. So I think centralized sounds easier, but decentralized sounds better."}
P1 highlighted a mixed preference, favoring decentralization while emphasizing the government's oversight, \textit{"I think it's sort of in the middle. I think decentralized would be the right way to go, because. as expecting the Government to have enough resources to keep verifying Phcs would be hard...But the verification, like the issuing authority, is still a government, and the Government still has oversight on how these verification systems work."} P3 on the other hand mentioned having decentralized system to issue PHC with having govt as one of the trusted issuers to sign the phc. 

\iffalse
\textbf{Design preferences of architecture.} \tanusree{we can potentially cut this theme if needed}
Design preferences reflect diverse priorities over accessibility, trust, and security. P1 emphasize the need for accessible platforms that ensure users can verify their credentials anytime and anywhere, saying \textit{"Accessible for people to log in at any time or any place across the world for their online verification and need education programs to prevent social engineering in such new verification system
%Because if these platforms go down and verifications are stopped, it would be a very, very big impact for people. It should be like the uptime and how well they are accessible for consumers.
"}
%Trust is also highlighted through the expectation that PHC issuers should be government operated or supervised for credibility, \textit{"I would expect the PHC issuer to be either part of the government or directly supervised by the Government. (P5)"} 
Many also emphasized for secure algorithms with valid time constriant for PHC. 
%education programs to prevent social engineering are essential for enhancing security, P2 saying, \textit{"For example, protect the algorithm, the algorithm make the system more robust. And also there's a lot of like education training things to prevent social engineering."}. 
In the same line, P23 mentioned the integration of number of threshold for factors in verification to allow people choice what data and how many data they would like to provide to have their phc issued. This demonstrate people' varying level of expectation for usability and security. 

%recommended to provide layered protection during interactions with service providers. P6 described \textit{"You will need a 2-factor authorization. Whenever you use your face or your fingerprint to log into the app, you should simultaneously get a code on your phone that only lasts for 2 to 5 min. So that you're not just relying upon the biometric data, you have to have the phone that you signed up with."}

%\fixme{summarize these with adding a quote of valide time constraints Design preference: adding valid time constraints Design preference: third-party commitments Design preference: encrypted credentials Design preference: checking organizations Design preference: transparency-what information is going to be shared Design preference: segregating database-preventing access to irrelevant or sensitive information} \ayae{summarized design preferences. (Above lists are from pilot study)} \fixme{where did you summarized? is it "accessibility, trust, and security" ? this is so high level, need to be specific like adding valid time constraints, encrypted credentials, segregating database-preventing access, etc. its fine to add pilot study results}
\fi
\vspace{-2mm}
\section{RQ3: Design Suggestions for PHC}
\vspace{-2mm}
%\subsection{Participants' Needs for PHC Designs}
Participants expressed specific needs for PHC systems, primarily focusing on enhancing security and trust while accommodating diverse needs. From the design sessions, we identified the following themes.
%P4 preferred a decentralized approach, stating, \textit{"I prefer the decentralized version just because you’re not giving all the power or like responsibility to government, although government might have all the information."} 
%Additionally, transparency was identified as a critical requirement. 
%P4 also noted, \textit{"I think any of this institute, if they take this responsibility, they need to be regulated about how they’re gonna handle user privacy security. They should be transparent about their process."} This emphasis on transparency and accountability highlights the need for PHC issuers to demonstrate their commitment to protecting user privacy and security as trusted issuers.

%\section{RQ3: Participants' Suggestions for PHC Designs}
%Participants suggested several design improvements for PHC systems. 
%\tanusree{do we only have these two design themes? we discussed more before} \ayae {}


\textbf{Design Theme 1: Time-bounded Credential for Privacy}
%\tanusree{ will add the user sketch here.}
Some users expect credentials to verify personhood with the least amount of personal data, avoiding detailed personal or biometric data collection for a certain period. Another expectation in the same line was the portability of preferred credentials if they can be used across platforms without re-verifying frequently. To address the amount of data and period of such data collection that has been used for issuing credentials, some features or design concepts were -- limited validity period, proof without storage or pseudonymization of the data when storing it. P21 mentioned - \textit{``I want a credential that works like a trusted pass—valid only for a set time (e.g., 30 days, 6 months, or 1 year), with reminders before it expires. It should prove I am a real person without storing my sensitive details or tying them to my real-world identity, like some level of anonymity. Just give me the freedom to be verified without being exposed.'' }, which is depicted in Figure \ref{fig:P21}.
% \begin{figure}[!t]
%	\centering
%	\includegraphics[width=\linewidth]{Fig/sketch_P21_area.png}
%	\caption{Illustration to depict time-bounded credential with retention and expiration date based on different data types}
%\label{fig:P21}
%\end{figure}


\textbf{Design Theme 2: Sensitivity-based Usable Credential Choice for PHC}
Multiple participants expressed that their preferences are sensitivity-dependent as presented in Figure \ref{fig:P3}. They suggested a design to incorporate choices for end users based on their perceived level of security needs across services in healthcare, finance, social media, etc.
%P14 considered system designs based on the trustworthiness of PHC issuers. Scenarios can be categorized into two types: when issuer is trustful and when it is not trustful.
%\fixme{i don't understand what it mean?}. 
%P4 suggested that in cases where the PHC issuer is untrustworthy, additional verification or oversight by the government should be implemented. \fixme{i don't understand what it mean?} 
%\ayae{Revisiting P4's transcripts, his point is preference to additional verification when interacting social media. We can skip this quote in this section.}
%For instance, P14 suggested using government-issued IDs or facial recognition as credentials for trustworthy PHC issuers, whereas fingerprints or iris scans would be preferable for untrustworthy issuers. 
%\fixme{i am struggling to understand why you present all the narrative about trustworthiness when even its so clear that its about sensitivity of the application area } \ayae{I agree with that, I was too much highlighted the trustworthiness of PHC issuers. Let me comment out this part.}
To illustrate the concept, P3, P4 provides a design where she made a choice of government-issued id or face to obtain a PHC for services in healthcare and finance
%to tailor the choice of credentials depending on the trustworthiness of the service provider, stating that 
In her words- \textit{"I think it very much depends on the scenario. For example, places where which are like the most sensitive. Creating a bank account or an account on a government website, or with a healthcare provider. I think, in terms of financial security, or on the government side, my preferred would be a face scan or a video call to verify my face and uploading a doc like my government issued residence permit, or my passport. Whereas in cases which are doesn't have that much of a security concern. So, having an account on ChatGPT, or a social media account, I would be okay with like fingerprint scan or iris scan."} She considers fingerprint and iris are less sensitive since face scan can potentially reveal one's identity, stating \textit{"If I scan my whole face, you can figure out my identity. I think the iris, fingerprint scanning is poses less of a risk, because it's only scanning your iris''}
%I'm sure there is some kind of system that could figure out who I am and other personal information about myself... But I think the iris scanning is poses less of a risk, because it's only scanning your iris... I think fingerprint scan, or iris scanning is probably like the better way to identify a person than having to scan their whole face or having or asking them to upload an entire id. "}
%\fixme{add this participants rationale why she considered iris scan / fingerprint less sensitive than face/video / id}\ayae{updated}
%\fixme{you are contradicting your earlier statement, read the theme again to understand what you are presenting here}\ayae{updated}

\begin{figure*}[!h]
 \begin{subfigure}{0.48\textwidth}
     \centering
     \includegraphics[width=0.8\textwidth]{Fig/sketch_P21_area.png}
     \captionsetup{width=\textwidth, font=footnotesize} 
     \caption{Illustration to depict time-bounded credential with retention and expiration date based on different data types}
     \label{fig:P21}
 \end{subfigure}
 \hfill
 \begin{subfigure}{0.48\textwidth}
 \centering
     %\raisebox{0.2cm{
     \includegraphics[width=0.8\textwidth]{Fig/sketch_P3_area.png}
     %}
     \captionsetup{width=\textwidth, font=footnotesize} 
     \caption{Illustration of choice for users to choose various data requirements for PHC issuance for different applications}
     \label{fig:P3}
 \end{subfigure}
 \hfill
 \begin{subfigure}{0.48\textwidth}
 \centering
     %\raisebox{0.5cm}{
     \includegraphics[width=0.8\textwidth]{Fig/sketch_P2.png}
     %}
     \captionsetup{width=\textwidth, font=footnotesize} 
     \caption{Illustration to depict comprehensive visually interactive human check with video chat for humanness cues, environment check}
     \label{fig:P2}
 \end{subfigure}
 \hfill
 \begin{subfigure}{0.48\textwidth}
     \centering
     %\raisebox{-2cm}{
     \includegraphics[width=0.8\textwidth]{Fig/sketch_P13.png}
     %}
     \captionsetup{width=\textwidth, font=footnotesize} 
     \caption{Illustration to depict periodic biometrics, dynamic authentication, geo-restricted access}
     \label{fig:P13}
 \end{subfigure}
\hfill
 %\bigskip
 \begin{subfigure}{0.48\textwidth}
 \centering
     \raisebox{1cm}{
    \includegraphics[width=0.8\textwidth]{Fig/sketch_P9_area.png}
     }
     \captionsetup{width=\textwidth, font=footnotesize} 
     \caption{Illustration to depict single issuer system with the government}
     \label{fig:P9}
 \end{subfigure}
 \hfill
 \begin{subfigure}{0.48\textwidth}
     \centering\includegraphics[width=0.8\textwidth]{Fig/sketch_P7_area.png}
     \captionsetup{width=\textwidth, font=footnotesize} 
     \caption{Illustration to depict decentralization for PHC governance}
     \label{fig:P7}
 \end{subfigure}
    \caption{Design suggestions for PHC: each represents the following design themes (a) Time-bounded credential for privacy; (b) Sensitivity-based usable credential choice for PHC; (c) Comprehensive visually interactive
human check; (d) Mitigate PHC misuse; (e)(f) Distribute power across issuer: decentralized vs centralized. }
    \label{fig:four_sketches}
\end{figure*}

% \begin{figure}[!t]
%	\centering
%	\includegraphics[width=\linewidth]{Fig/sketch_P3_area.png}
%	\caption{Illustration of choice for users to choose various data requirements for PHC issuance for different applications}
%\label{fig:P3}
%\end{figure}

\textbf{Design Theme 3: Comprehensive Visually Interactive Human Check.}
Another theme comes up where participants explain how \textit{``video chat''} offers several advantages in preventing social attacks during personhood credential issuance by triangulating real-time interaction, visual verification, and environment check. As P2 said \textit{``video chat could be a way to avoid social attacks in PHC issuing. Yes. this way service provider would ask you a lot of questions to make sure you are not controlled by someone else, and they would ask to move my camera, take videos of my whole room to make sure there's no one else beside me. And of course, it's it would. It would not be a 100\% thing, all the other things [only face, ids] are still the old stuff. This could be a new step to developing PHC algorithm.''} P2 further emphasized how human interaction is still a key when stepping into new technological innovation where in the issuance process, trained staff can assess subtle cues and inconsistencies that automated systems might miss, potentially detecting social engineering attempts or ask personalized questions based on the user's responses and behavior, making it difficult for attackers to prepare scripted answers (Figure \ref{fig:P2}). 

% \begin{figure}[!t]
%	\centering
%	\includegraphics[width=\linewidth]{Fig/sketch_P2.png}
%	\caption{Illustration to depict comprehensive visually interactive human check with video chat for humanness cues, environment check}
%\label{fig:P2}
%\end{figure}


\iffalse
\begin{figure}[!t]
 \vspace{5pt}
	\centering
	\includegraphics[width=\linewidth]{Fig/sketch_p1.png}
 \vspace{-15pt}
	\caption{Sketch by P1: Government involvement in PHC Issuers’ system}
\label{fig:sketch_p1}
\vspace{-15pt}
\end{figure}


\begin{figure}[!t]
 \vspace{5pt}
	\centering
	\includegraphics[width=\linewidth]{Fig/sketch_p7.png}
 \vspace{-15pt}
	\caption{Sketch by P7: Blockchain-based Issuer}
\label{fig:sketch_p7}
\vspace{-15pt}
\end{figure}

\begin{figure}[!t]
 \vspace{5pt}
	\centering
	\includegraphics[width=\linewidth]{Fig/sketch_p13.png}
 \vspace{-15pt}
	\caption{Sketch by P13: Different Layer of Security}
\label{fig:sketch_p13}
\vspace{-15pt}
\end{figure}

\begin{figure}[!t]
 \vspace{5pt}
	\centering
	\includegraphics[width=\linewidth]{Fig/sketch_p3.png}
 \vspace{-15pt}
	\caption{Sketch by P3: Context-dependent Preference}
\label{fig:sketch_p3}
\vspace{-15pt}
\end{figure}


\begin{figure*}[!t]
	\centering
	\includegraphics[width=\linewidth]{Fig/sketch_all.png}
	\caption{Sketches from Design Session \fixme{each subcaption needs to be self explanatory}}
\label{fig:sketch_all}
\end{figure*}
\fi


% \begin{figure}[!t]
%	\centering
%	\includegraphics[width=\linewidth]{Fig/sketch_P13.png}
%	\caption{Illustration to depict system design to ensure primary PHC users is using the credential with periodic biometrics, dynamic Authentication, and geo-restricted access}
%\label{fig:P2}
%\end{figure}

\textbf{Design Theme 4: Mitigate PHC Misuse: Periodic Biometrics, Dynamic Authentication, and Geo-Restricted Access.}
Participants suggested enhanced key management for the PHC system to address potential misuse, such as identity selling, credential sharing, and unauthorized account setup (Figure \ref{fig:P13}). Suggestions included periodic biometric verification (e.g., facial recognition or fingerprint scans during login) or random re-authentication to ensure that only the credential holder can access the system (P6, P8, P13, P18, P23). P13 highlighted the risk of credential sharing during onboarding, where users may rely on friends or family for setup assistance, potentially leading to unauthorized access. To mitigate this, P13 proposed combining biometrics with dynamic authentication methods, such as time-sensitive push notifications. Additionally, P13 suggested implementing geo-fencing to restrict access from unfamiliar locations or devices, reducing the risk of misuse if credentials are leaked. These suggestions aim to ensure that PHCs are used securely and by the intended user across different services.


% \begin{figure}[!t]
%	\centering
%	\includegraphics[width=\linewidth]{Fig/sketch_P7_area.png}
%	\caption{Illustration to depict decentralization for PHC governance}
%\label{fig:P2}
%\end{figure}

\textbf{Design 5: Distribute Power Across Issuer: Decentralized vs Centralized} 
Participants (P5, P9, P10) frequently highlighted the importance of a single issuer for PHC, prioritizing a unified point of trust. For example, P10 suggested the government as the primary issuer for various services, arguing that involving third parties, such as insurers in health contexts, complicates credential management (Figure \ref{fig:P9}). 
Conversely, others (P1, P7, P11) supported a multi-issuer approach, leveraging blockchain-based systems to enable decentralized storage and sector-specific distribution of power, addressing trust and governance concerns in PHC. 
%As illustrated in Figure \ref{fig:sketch_all} (b), this approach proposes a decentralized infrastructure where multiple entities sync. 
The main design concept focus on ensuring that any misuse of data by one entity would be immediately flagged by others and distributed control among different issuers when included as decentralized issuance systems. P7 illustrated the concept by 
stating \textit{"The important part is, they will have the same information all at once and they are synced together. So someone is uploading their document or info, all these companies are going to have it in sync together, so that if there is any misuse of information is going on here, the other companies would get to know it."}  (Figure \ref{fig:P7})

%Sure. there is always going to be a risk of companies getting together and misusing it. But we in this case, we're just reducing the risk of those kind of data misusage. So that's basically, what I'm trying to do is mimicking a blockchain structure so that everyone has the information in real time, and everyone has a copy of it. So that's what I'm trying to achieve here.
%\fixme{add a lien what does it mean by sector based decentralization}\ayae{updated}
%This approach underscores the value of decentralization in creating a more resilient and transparent system, reducing the reliance on any single entity and building greater trust in the PHC issuer.
%======= will think later-----Sector-based decentralization is also proposed as a system where responsibilities are distributed across multiple entities, with each entity specializing in a specific sector (e.g., social media, finance). Instead of a centralized authority handling all data, each sector operates only relevant data to its domain, reducing the risk of data monopolization and excessive data exposure.

%\textbf{Design Theme 6: Having Single PHC Issuer (Emphasized on Government).}
%Participants often (P5, P9, P10) emphasize single issuer preference for PHC to maintain a single point of trust. To illustrate this, P10 suggested government to be the significant issuer for different services where involving third parties like insurance provider in health context can be cumbersome in the credential handling process. Simialrly, P5 conceptualize PHC issuer to be directly from government or supervised by government, exception where tech company might handle PHC 

%two-factor authentication (2FA) for the use of PHC. P8 has security concerns of PHC being hacked and described 2FA as a simple and least expensive way to secure his account, noting \textit{"The simplest solution is to do 2FA...This can be your text message. And then, we could do or email. So that will unlock your account. That's probably the least expensive way to handle it to use a 2FA service."} P13 indicated the combination of 2FA and biometrics would strengthen security -  \textit{"I'd just be concerned that once a person's PHC has been set up that they could then like set up accounts for other people. It would at the moment I mentioned, I use the Google Authenticator. When you log into a web and to a site it asks for a number. So I go my Google authenticator, and if that was combined with the biometrics, maybe that would work."} as explained in Figure \ref{fig:sketch_all} (c). By introducing another authentication process, end users can rest assured that the PHC user is who they claim to be.

\iffalse
\textbf{Design Theme 1: Government involvement in PHC issuer's system} 
Several participants mentioned government involvement as part of the ideal structure for PHCs. P1 expressed an expectation for some form of government involvement to ensure the reliability of the PHC issuer \textit{"I'd expect my PHC issuer and the government to work together to provide me sort of an id, and their data would be stored, I guess, more on a Phc issuer side that under government. But I would expect both of these organizations to work together."}, along with a sketch as shown in Figure \ref{fig:sketch_all} (a). 

\textbf{Design Theme 2: Decentralization} One suggestion focused on \textbf{blockchain-based issuer} to address concerns of trust and governance in PHC. As illustrated in Figure \ref{fig:sketch_all} (b), this approach proposes a decentralized infrastructure where multiple entities sync. This design mimics blockchain principles, ensuring that any misuse of data by one entity would be immediately flagged by others, stating that \textit{"The important part is, they will have the same information all at once and they are synced together. So someone is uploading their document or their information, all these companies are going to have it in sync together, so that if there is any misuse of information is going on here, the other companies would get to know it. Sure. there is always going to be a risk of companies getting together and going to misuse it. But we in this case, we're just reducing the risk of those kind of data misusage. So that's basically, what I'm trying to do is mimicking a blockchain structure so that everyone has the information in real time, and everyone has a copy of it. So that's what I'm trying to achieve here."} This approach underscores the value of decentralization in creating a more resilient and transparent system, reducing the reliance on any single entity and building greater trust in the PHC issuer.

\textbf{Design Theme 4: Different Layer of Security}
Some participants preferred to have two-factor authentication (2FA) for the use of PHC. P8 has security concerns of PHC being hacked and described 2FA as a simple and least expensive way to secure his account, noting \textit{"The simplest solution is to do 2FA...This can be your text message. And then, we could do or email. So that will unlock your account. That's probably the least expensive way to handle it to use a 2FA service."} P13 indicated the combination of 2FA and biometrics would strengthen security - 
\textit{"I'd just be concerned that once a person's PHC has been set up that they could then like set up accounts for other people. It would at the moment I mentioned, I use the Google Authenticator. When you log into a web and to a site it asks for a number. So I go my Google authenticator, and if that was combined with the biometrics, maybe that would work."} as explained in Figure \ref{fig:sketch_all} (c). By introducing another authentication process, end users can rest assured that the PHC user is who they claim to be.
\fi