\vspace{-2mm}
\section{Introduction} 
\vspace{-2mm}
\label{sec:introduction}
%\tanusree{make updates on introduction section based on current progress. follow paper structure/example provided before , such as, data minimization, smart contract. social media ad paper} 
%\ayae{updated}
%\fixme{Please add motivation with citation, What I added before just example and placeholder with very initial ideas. we are not just talking about AI here, imagine national security website, survey platform with weak verification where one person might be participating twice. This is to say that this is an holistic online interaction veriication issue, not just an AI issue. More recently AI is making it more prevalent, but we can not just start with AI. Please motivate the work properly based on all the literature you did and knwoeldge you gather, be creative}
%Imagine, social media is overrun with fake accounts and scams,  bot or malicious actors infiltrating research in both academic and industry setup where users’ signals are crucial for tool and product design, more recently fine-tuning local LLM models based on user preferences and chat data. Even in high-stakes environments, such as communications between government officials, where deliberation and voting are required to make critical decisions on state, national, and international policies, digital communication demands verifying the legitimacy of participants to ensure they are real people, not bots. This is even more crucial for underserved, marginalized, and context-specific communities, where it’s vital to confirm that the participation and engagement are unique and legitimate.
Identity management has long been a cornerstone of user-facing systems such as social media platforms, gaming environments, and collaborative tools, and has increasingly become integral to Human-AI systems~\cite{gorwa2020unpacking, cetinkaya2007verification}. The recent proliferation of artificial intelligence (AI) has rendered traditional methods of multi-factor~\cite{aloul2009two}, and CAPTCHA~\cite{von-Ahn2003-wr} verification unreliable, as AI can now generate highly convincing fake human interactions~\cite{carlini2024poisoning, akhtar2024sok, goodrich2023battling}. 
%This underscore pressing need for digital identity systems that respect user autonomy and enable trustworthy online communications. 
Amid growing concerns over how major technology companies handle user data
~\cite{nytimesCambridgeAnalytica}, there has been a noticeable technological and ideological shift toward decentralized identity systems, commonly known as self-sovereign identity~\cite{mahula2021blockchain}. 

One widely recognized approach involves decentralized identifiers (DIDs). Emerging proposed systems, DECO~\cite{zhang2020deco}, Town-Crier~\cite{zhang2016town}  exemplify this model where users authorize the release of personal credentials from user devices to websites for proving certain characteristics about themselves. Although initiatives like the W3C Decentralized Identifier Working Group seek to establish standards for decentralized identity~\cite{identityDecentralizedIdentity, w3cccgDecentralizedIdentifiers}, many proposed frameworks struggle to meet both technical and usability requirements. Recent efforts, such as CanDID, have made progress in areas like usable key recovery~\cite{maram2021candid}.
Building on concepts, such as, decentralized identifiers (DIDs), proof-of-personhood \cite{ford2020identity}, anonymous credentials, and Personhood credentials (PHCs) have emerged as an alternative approach. PHCs allow individuals to verify their personhood to digital service providers without revealing additional personal information~\cite{adler2024personhood} and verified through zero-knowledge proofs. PHCs are designed to ensure that each credential corresponds to a unique, real individual~\cite{ford2020identity}. For example, Worldcoin has implemented a PHC-based identity system called World ID~\cite{de2024personhood}, which uses an "Orb" iris scanner for personhood verification.

%enabling individuals to verify to digital service providers that they are a person without disclosing additional information \cite{adler2024personhood}. PHCs can be verified through zero-knowledge-proofs and fundamentally linked to the concept which guarantees each one represents unique real people \cite{ford2020identity}. An example of this, World implemented PHC as an identity system called World ID \cite{de2024personhood}. This system was developed with an "Orb" iris scanner for personhood verification. 





%BTo counter such challenges in digital identity, personhood credentials (PHCs) have emerged as an alternative approach to verify the humanness of online interactions while preserving user anonymity \cite{adler2024personhood}. 
%The primary goal of PHC is fundamentally linked to the concept of Proof of Personhood (PoP), which guarantees each one represents unique real people \cite{ford2020identity}. 
%Through PoP and zero-knowledge proofs, PHC solves the problem that existing methods against automated deception (e.g., CAPTCHA and watermarking) are insufficient against sophisticated AI \cite{Kumar2022-uk, Searles2023-db}, without disclosing personal information.

%An example of this, World implemented PHC as an identity system called World ID \cite{de2024personhood}. This system was developed with an "Orb" iris scanner for personhood verification.


%As it becomes impossible to differentiate between human and AI and their generated content, 
%Thus, it is more crucial than ever to verify not only what is being said but also who is saying it. 

%This paradigm emphasizes user control over personal data and aligns with broader calls for more transparent and privacy-preserving approaches to identity management.
%In recent years, digital identity verification has become a crucial function in establishing trust within our digital environments, as seen in increasing multi-factor and biometric authentication to prevent scams and fraud from legitimate accounts. When we execute any transactions online, we rely on digital identity to confirm whether the person is trustworthy or the document is authentic. For example, the rise of bot accounts on social media platforms \cite{gorwa2020unpacking} has highlighted the necessity of verifying genuine users. Also, online voting requires to guarantee fairness and equal participation without compromising voters' privacy \cite{cetinkaya2007verification}. These discussions reflect broader issues of digital identity in our daily lives. However, modern identity systems have disregarded personhood. Sobel discussed a national identification system (NIDS) demeans personal identity by transforming personhood into numerical places by numbers recorded in databases, and inherent freedom and autonomy are constrained by the centralized authority \cite{sobel2001demeaning}. 
%The recent proliferation of generative AI, particularly, bot or fake responses emphasizes the importance of verifying the authenticity of human data and interaction~\cite{}. These situations underscore the need for digital identity systems that respect user autonomy and enable trustworthy online communications.


%In today's increasingly networked world, verifying the human authenticity of communications (e.g., authentication, participation, diversity of human signals) has become a critical challenge. The proliferation of deepfake technology has rendered traditional methods of audio, multi-factor authentication, and CAPTCHA verification unreliable, as artificial intelligence (AI) can now generate highly convincing bots and fake human interactions~\cite{zhu2014captcha}. To address this problem, our long-term vision is a user-facing personhood credential tool that verifies the uniqueness, and legitimacy of humanness in different context-dependent and goal-oriented scenarios in digital communication mediums. 

%However, even before developing such a tool, it's crucial to understand how users want to manage their personhood credentials, including their preferences for issuance systems and trusted authorities. This raises key questions about security, trust, and usability. 
%Meanwhile, previous work explored end users' perceptions of authentication technologies and revealed that most participants expressed uncertainty and concerns about their use \cite{jones2007towards}. To address this problem, our long-term vision is a user-facing personhood credential tool that verifies the uniqueness, and legitimacy of humanness in different context-dependent and goal-oriented scenarios in digital communication mediums. 
%However, even before developing such a tool, it's crucial to understand how users want to manage their personhood credentials. This raises key questions about security, trust, and usability. 
%To ensure users feel secure in adopting PHCs, it is crucial to first understand their perceptions and preferences and then design PHC systems that alleviate their concerns while incorporating their preferences.
%This leads to another bigger question: whether users fully grasp the implications of the personhood credentials when choosing their preferences. This leads to concerns about the explainability of these emerging tools and underlying technology. In this one-year research project,  if supported by the IST seed grant, 
%We aim to explore users’ security and privacy preferences for managing personhood credentials. 
%Despite their growing importance, limited research
%has been done on users’ perceptions and preferences regarding such emerging technologies PHCs, users’ preference on managing (e.g. recovery, data handling, trust on issuer ecosystem) verification credential remains largely unexplored.. New technologies might introduce some friction due to users’ misunderstandings and mismatched ex- pectations. Simialrly indicated in previous work where end users' perceptions of authentication technologies arevealed that most participants expressed uncertainty and concerns about their use \fixme{what use, what uncertainty, make it clear} \cite{jones2007towards}. 

Despite their growing importance
of personhood credentials and similar tools, there is a lack of understanding about how PHCs could be designed
from a user-centered perspective, and in particular what factors might influence users' preferences with regard to onboard and manage PHCs. New technologies might introduce some friction due to users’ misunderstandings and mismatched expectations. Consequently, to address this gap and to
complement existing system-centered approaches, our paper
explores a user-centered approach. 

Towards this goal, 
%in this paper, we conducted a mixed method research including content analysis of popular personhood credentials and related systems and conducted semi-structured interviews with 
%\ayae{finalize this paragraph contents, location later} 
We began with a formative analysis, including a competitive review of existing personhood verification and related systems. This involved examining user reviews from app stores, analyzing white papers and analysing UI/UX of available system or proof of concepts.
%and performing cognitive walkthroughs of UI/UX designs 
to identify current challenges. 
Subsequently, we conducted interviews to explore user preferences and the factors (RQ1) influencing their choices of personhood credentials (RQ2). Finally, we conceptualized participants' desired approaches in managing personhood credentials (RQ3)\\%\tanusree{I am adding method within the RQ for now so you all understand how different part comes together}
%\textbf{RQ0}: What are the usability, and security challenges current verification tools have? %\textbf{Method by Silvia: Cognitive walkthrough of UI/UX and competitive analysis} 
%\hspace{-0.5cm}\colorbox{white!30}{\framebox[\linewidth][c]{\begin{minipage}[t]{0.95\linewidth} \small{\textbf{RQ1}: How do users perceive personhood credentials in managing online digital identity?}\end{minipage}}}

\vspace{-4pt}
\begin{tcolorbox}
%\vspace{-4pt}
\textbf{RQ1}: What are the users' perceptions of personhood credentials to verify themselves as legitimate and unique individuals in online interactions?\\
\textbf{RQ2}:  What factors influence user preferences 
on how people would like to verify themselves as legitimate and unique individuals in online interactions?\\
\textbf{RQ3}: How can personhood credentials be designed to ensure usability and security, enabling users to verify themselves as legitimate and unique individuals in online interactions?
%\vspace{-4pt}
\end{tcolorbox}
\vspace{5pt}
\textbf{Findings} Our study uncovered a wide range of users' perceptions regarding PHCs. 
%While diverse verification practices were observed across different domains, 
Participants exhibited skepticism towards PHCs, partly because of \textit{``unknown risk''} vectors as a new technology compared to traditional verification. 
%trade-offs between existing verification methods and PHCs, shaped by their familiarity and uncertainties surrounding PHCs. 
Despite these concerns, we find diverse levels of adoption preferences influenced by the \textit{``type of data required''} for PHC credential issuance and verification as well as personal \textit{``security standards''} for different services (e.g, finance, health, government related). 
%While all participants were unfamiliar with PHCs prior to this study, they understood PHCs and shared their perceived benefits and concerns. 
Our findings also indicate benefits including aspects that promote fairness by ensuring opportunities for legitimate users in platforms such, as gaming, and survey platforms.
%, as well as the potential risk of centralization.
%Notably, participants valued PHCs for verifying the uniqueness of identity while minimizing the exposure of personal information. They recognized the challenges of digital identity in today’s online environment and indicated the need for PHCs, citing issues like bot accounts and fake news on social media.
%There were dilemmas identified between physical and digital verification, which reflected participants' offline experiences.
%We also observed participants emphasized the importance of stakeholder reliability during identity verification, frequently considering the trustworthiness of stakeholders when expressing their willingness to use PHCs.
%\hspace{-0.5cm}\colorbox{white!30}{\framebox[\linewidth][c]{\begin{minipage}[t]{0.95\linewidth} \small{\textbf{RQ2}: What factors influence user preferences for personhood credentials to verify themselves as legitimate and unique individuals in online interactions?}\end{minipage}}}
%\begin{tcolorbox}
%\vspace{-4pt}
%\textbf{RQ2}:  What factors influence user preferences on how people would like to verify themselves as legitimate and unique individuals in online interactions?
%\vspace{-4pt}
%\end{tcolorbox}
%\textbf{Method (By Ishan and Ayae): Reddit Analysis and Interview}
We surfaced nuanced preferences of how participants would like to onboard and manage PHC credentials.
Their preferences depended on the type of ground truth data required for issuance, along with the familiarity, usability, and sensitivity of those data sources. They often considered facial recognition more resilient verification process than fingerprints.
 Other factors influencing preferences included the issuing ecosystem (centralized vs. decentralized), the issuer (government vs. private company), and the onboarding method (physical vs. remote).
%issuance system also led to PHC architectures, favoring decentralization for security and centralization for simplicity.
%Regarding users' preference of PHC, they significantly supported government involvement in issuing PHCs. Most participants preferred having the government as a PHC issuer, while others favored some level of government oversight in cases where private entities were issuers.
%This preference was attributed to the government being perceived as a more trustworthy entity, highlighting the close relationship between stakeholder reliability and users' willingness to adopt PHCs. We also identified context-dependent preferences regarding credential types. It highlighted the necessity of decentralized architecture of PHC.
Furthermore, we identified practical design suggestions to accommodate participants' needs, including periodic biometric checks and time-bound credential verification to ensure only the intended user accesses the credential; visually interactive human-checks to prevent social engineering during onboarding, collaboration between industry and government to establish decentralized standards for broader adoption.
%Finally, we propose actionable design recommendations to enhance usability and accommodate users' diverse preferences. These recommendations include interface design for verification choice, interoperable verifiable credentials, multi-factor personhood credentials, and decentralized standards for issuance process.
%\hspace{-0.5cm}\colorbox{white!30}{\framebox[\linewidth][c]{\begin{minipage}[t]{0.95\linewidth} \small{\textbf{RQ3}: How can personhood credentials be designed to ensure usability and security, enabling users to verify their legitimacy and uniqueness in online interactions?}\end{minipage}}}
%\begin{tcolorbox}
%\vspace{-4pt}
%\textbf{RQ3}: How can personhood credentials be designed to ensure usability and security, enabling users to verify their legitimacy and uniqueness in online interactions?
%\vspace{-4pt}
%\end{tcolorbox}
%\textbf{Method (Ayae): Interview}


%They also indicated preferences for design dimensions and proposed improvements through sketches. Notably, we identified five design themes: 1) Government involvement; 2) Decentralization; 3) Time-bounded credential for privacy; 4) Different layer of security; 5) Context-dependent preference. We extended these discussions into design implications.


%These research questions are examined through semi-structured interviews with 17 participants, asking users' perceptions and preferences and conceptual design recommendations of PHC.


%\textbf{Findings}
%Our study uncovered a wide range of users' perceptions and preferences regarding PHCs. While all participants were unfamiliar with PHCs prior to this study, they understood PHCs and shared their perceived benefits and concerns. They also indicated preferences for design dimensions and proposed improvements through sketches.Notably, participants valued PHCs for verifying the uniqueness of identity while minimizing the exposure of personal information. They recognized the challenges of digital identity in today’s online environment and indicated the need for PHCs, citing issues like bot accounts and fake news on social media. We also observed participants emphasized the importance of stakeholder reliability during identity verification, frequently considering the trustworthiness of stakeholders when expressing their willingness to use PHCs. Regarding users' preference of PHC, they significantly supported government involvement in issuing PHCs. Most participants preferred having the government as the PHC issuer, while others favored some level of government oversight in cases where private entities were issuers. This preference was attributed to the government being perceived as a more trustworthy entity, highlighting the close relationship between stakeholder reliability and users' willingness to adopt PHCs.

\textbf{Contributions}
Overall, our study makes the following contributions: (1) Formative study sheds light on challenges in current personhood verification; (2) Our interview results provide rich insights into users' perceptions along with factors influencing their preference surrounding personhood verification; (3) User-Centered Design suggestions addressing from interactive sketch sessions.
%(1) By identifying the perceived benefits and concerns of PHCs, our study clarifies critical aspects that must be addressed and effectively communicated in PHC implementation. These findings contribute to user-centered PHC development. (2) Through users' current verification practices and various PHC scenarios, our study illuminates the mental models of trust and privacy that end-users associate with the verification process. (3) By examining user preferences for PHCs across three dimensions of credentials, stakeholders, and architecture and by further diving into conceptual design preferences via sketch sessions, our study offers actionable and practical insights for future PHC development.

%Standard Setting with Human Preferences. We will build a taxonomy of factors that influence users’ decisions on what data they prefer for personhood credentials and what types of issuance systems or issuers they would prefer for managing their credentials. This will enable developers to design a context-sensitive personhood credentials system and give users the flexibility to choose credentials that balance security needs with usability constraints.
