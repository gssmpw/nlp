\vspace{-6mm}
\section{Discussion}
\vspace{-2mm}
Our findings shed light on a wide range of human factors in designing personhood credentials.
%considered in PHC design, including user perceptions (RQ1), factors influencing preferences (RQ2) and design suggestions (RQ3). 
In this section, we discuss how the findings can contribute to current state of knowledge in designing user-centered and secure PHC design. 
\vspace{-2mm}
\subsection{Main Findings}
\vspace{-2mm}
%\fixme{need more work on explaining main findings and how these are unique and adding new knowledge to the literature. need critical mapping to literature. please read the discussion as reference: https://arxiv.org/pdf/2410.01817 ;  https://dl.acm.org/doi/abs/10.1145/3544548.3581498; https://www.usenix.org/system/files/usenixsecurity24-sharma.pdf} \ayae{updated}

\textbf{Nuanced Credential Preference}
Identity management has long been a focal point of user-facing systems, including social media platforms, gaming environments, and collaborative tools, etc~\cite{gorwa2020unpacking, cetinkaya2007verification, sharma2024future, sharma2024unpacking}. We have also observed technological and ideological shifts towards 
decentralized identity- commonly referred as --self-sovereign identity amidst the criticism of large technology companies' data handling practices~\cite{nytimesCambridgeAnalytica}.The most cited case is decentralized (DIDs), with emerging proposed systems, DECO~\cite{zhang2020deco}, Town-Crier~\cite{zhang2016town} -- where users authorizing the release of personal credentials from user devices to websites for proving certain characteristics about themselves. 
While initiatives such as the W3C’s Decentralized Identifier Working Group aim to develop standards~\cite{identityDecentralizedIdentity, w3cccgDecentralizedIdentifiers}, they largely fail to address the technical and usability goals.
Nonetheless, users' preference on managing (e.g. recovery, data handling, trust on issuer ecosystem) verification credentials remains largely unexplored. 

Our work sheds light on trade-offs people consider in onboarding and managing PHC. Our findings highlight some concerns towards PHCs, partly because of “unknown risk” vectors as a new technology compared to traditional verification. Despite these concerns, we find diverse level of adoption preferences
influenced by the “type of data required” for PHC credential issuance and verification as well as personal “security standards” for different services (e.g, finance, health, government
related). 
%the different factors end users consider if they were to onboard this emerging personhood credential systems. 
%We identified the relative preferences between various biometrics and other credentials as proof of personhood. Our study results support prior work\cite{Zimmermann2017-wr, mare2016study,De-Luca2015-mp,Bhagavatula2015-fi}, showing that fingerprint and iris recognitions are preferred with their security aspects. 
%\fixme{add main findings in 4 lines and mentioned what is unique about the findings comparing / citing literature}
We also expanded the findings to include nuanced preferences and underlying reasoning that explains why users have certain preferences, extending beyond what existing literature has limited to identifying preferences alone.
For instance, their preference for government-issued IDs is associated with their familiarity with a traditional method of verification. Also, their varied preference for biometrics is backed by subjective perceptions around efficacy and privacy sensitivity. For instance, they often considered facial recognition more resilient verification process than fingerprints.
%. 
%These results emphasized the need for clear explanations of the data requirements of PHC to ensure their efficacy in terms of privacy and security.

%Previous work has explored users’ preferences for different types of credentials\fixme{why authentication, this work is about verification} in identity management systems. For instance, studies such as \cite{mare2016study,De-Luca2015-mp,Bhagavatula2015-fi, Zimmermann2017-wr} have compared usability and user perceptions of various types of biometrics (e.g., iPhone fingerprint, Android Face) as a way of identity management in daily life. \fixme{compared for what? incomplete sentence, dones't add anything}. In contrast, this study investigates users' preferences of credential as proof of personhood to verify legitimacy and uniqueness.\fixme{is this the goal of this study? no far no mentioned of personhood credential at all.}. 



%In the government benefits scenario, government ID was preferred, whereas phone numbers were the most commonly selected credential in the LLM application scenario. For social media, biometrics such as face and fingerprints were highly preferred. This dynamic nature of credential preference depending on contexts emphasis the need of developing context-aware PHC systems. One future direction respecting users' context-dependent preference is decentralized architecture where users get PHCs from multiple PHC issuers and use them differently depending on contexts. 

\textbf{How to Build Trust \& Scale PHCs}
More recently, a cross-industry collaboration \cite{adler2024personhood} resurfaced the conversation of personhood credential 
\cite{borge2017proof, ford2020identity, de2024personhood, sharma2024experts} emphasizing the need of an ecosystem for PHC issuers to facilitate scalability. A fundamental challenge remains the "chicken-and-egg" dilemma: the absence of a broad ecosystem of PHC issuers hinders the adoption of systems that rely on PHCs, and conversely, the lack of such systems makes it difficult to incentivize the establishment of PHC issuers. Our findings highlight an interdependence and the importance of carefully considering both the issuing ecosystem (e.g., centralized versus decentralized models) and the type of issuer (e.g., government entities versus private organizations) as key factors influencing trust and perceptions of security among users to onboard.


%This is one of the fundamental limitations of current approaches to PHC creation to address the chicken-and-egg problem in PHC issuance. Without an ecosystem with a broad base of PHC issuers, systems that leverage PHCs will be slow to arise; conversely, without systems that leverage PHCs, it is hard to motivate the creation of PHC issuers. Our findings indicate issuing ecosystem (centralized vs. decentralized), and the issuer (government vs. private company) as the main factors impacting people's trust and security perception. To facilitate broader and equitable access, we suggest future work to involve stakeholders whom people trust as a PHC issuer. 

Our results highlighted the perceived benefits by users regarding PHCs, which is fairness in representation, aligning with that prior work's discussed proof of personhood is a key in fair online environments, especially when voting or accessing limited resources by eliminating AI-powered manipulations or duplications
\cite{de2024personhood}. 
However, our research also surfaced significant concerns that must be addressed to improve user trust and acceptance. Participants expressed apprehensions about the centralization of PHC issuance, fearing over-concentration of power and control. Ambiguities in regulations surrounding PHCs further compounded these concerns, reflecting broader uncertainties about governance and accountability. Additionally, users highlighted the need to ensure the authenticity of PHCs while addressing risks associated with the misuse of anonymity. These findings resonate with existing literature on privacy and security education, which underscores the importance of user understanding and clear communication in the adoption of new technologies \cite{acquisti2015privacy}. Future research should focus on increasing the explainability and transparency of PHC systems to address these concerns. 

%for scalability, contextual service-specific privacy/security design opportunities of PHC for better users interaction; however, to our knowledge, the preference of personhood credential has not been explored from the viewpoint of end users, particularly investigating factors that might be limiting equitable access to such technologies. Our study is the first study to investigate users' perspectives to understand the pain points and conceptualize designs to address those challenges. 


%\fixme{wording seems like result section. Rewrite as if its you who are interpreting the result and giving statement}
%When participants described their current practices with identity verification and preferences regarding PHCs, their explanations reflected that their decision-making was fundamentally rooted in the trust they had in stakeholders. 

%Users' trust is shaped by their own experiences and shared stories by family and friends, and the reputations they have encountered \fixme{what does it mean? again too much about just trust, is this the only interesting result? rewrite this paragraph highlighting the results from RQ2}. 
%Taking this into account, there are two potential directions for future development of PHCs: 1) involve trustworthy stakeholders: 2) build systems that reduce reliance on all third parties, including stakeholders. These directions are indicated in the participants' design sketches. Users favored government involvement in PHC systems due to trust with the government, while also showing interest in blockchain architectures where its decentralized and transparent nature reduces the cost of trust by minimizing the need for intermediaries and enabling verifiable interactions.  Some participants proposed hybrid approaches managing both centralized surveillance by the government and a decentralized verification system.



%\textbf{Perceived Benefits and Concerns of PHCs} Participants emphasized that PHCs could minimize personal information exposure and mitigate the deceptive activities by fake accounts and bots. These features address user concerns about privacy and efficiency in digital interactions. On the other hand, the primary concerns revolved around data security and trustworthiness of issuers. Participants feared that a single PHC issuer could become a single point of failure in the event of a data breach. They were also skeptical about private sector issuers and expected for transparency and accountability.
\vspace{-2mm}
\subsection{Design Implications}
\vspace{-2mm}
Drawing upon users' needs and preferences, we suggest actionable design implication for personhood credentials. 
%\tanusree{read 5.3 of this paper to better understand how to write design implication https://www.usenix.org/system/files/usenixsecurity23-sharma.pdf and 5.2 in this paper https://dl.acm.org/doi/pdf/10.1145/3544548.3581498. ideal way is to also investigate ccs paper proceeding on usability and internet measure track paper to see how this community usually present design implications or if there is anything unique. current draft read so high level to get any useful design direction for future}
\iffalse
\textbf{Transparency Interface}
We observed participants faced trade-offs between familiar
security guarantees associated with traditional verification
methods over the less clear assurances of emerging PHC. Moreover, they also anticipated the potential risk of the centralized power of the issuer and uncertain regulations. Such explainability and accountability issues can be mitigated by designing interfaces driven by transparency. As described in the sixth principle of Privacy by Design \cite{cavoukian2021privacy} - \textit{"The data subject is made fully aware of the personal data being collected, and for what purpose(s). All the component parts and operations remain visible and transparent, to users and providers alike."}, PHC issuers are expected to state explicitly the purpose, operations and policies regarding personal data collection practices. For example, they could leverage privacy nutrition label, which is the concept of standardized presentation of privacy policies ideated from food nutrition label \cite{kelley2009nutrition}. In practice, Apple has launched iOS privacy labels that embodied privacy nutrition label in their platforms \cite{li2022understanding}. By prototyping the information to be explained for end users through the iterative process proposed by prior work of transparency design \cite{eiband2018bringing}, they could clearly communicate the essential information while decrease cognitive load of users by standardizing the information according to the format of privacy nutrition labels.
\fi

\textbf{Interface Design to Facilitate Verification Choice.}
Our findings shows users' skepticism, partly because of \textit{unknown risk''} vectors of PHC as a new technology. Despite the unknowns, our findings also indicates diverse level of adoption/onboarding preferences towards PHC issuance, such as, type of data requirement to verify themselves which largely depends on their own \textit{``security standard''} developed for different types of services. With the preferences in mind, one possible way to first clearly add list of purpose, operation and policy of PHC leveraging design such as, privacy nutrition label\cite{kelley2009nutrition, li2022understanding} like Apple, to allow users to know the new tech.
As of credential issuance choice, a tiered system of PHCs with varying levels of verification strength
%an adaptive interface can be designed
to allow users to choose ground truth data (e.g., gov id, face, video, fingerprint, social graph, etc) based on their security standards. The interface could explicitly add a design of tiered verification options, each corresponding to a different level of security: Level 1 (Low Sensitivity): email or phone verification; Level 2 (Medium Sensitivity): physical id;  Level 3 (High Sensitivity): Biometric fusion (e.g., facial recognition + voice print);  Level 4 (Very High Sensitivity): Multi-factor PHC (e.g., biometrics + social graph verification+physical id), etc. We can implement modular architecture, allowing easy addition or modification of verification methods as technology evolves. 

\textbf{Portability of Personhood Credential.} Our result reveals varied preferences across different services, ranging from financial, health, government to social media. One expectation of users to have a stremlined approach so they don't need to onboard with multiple verification for services they use. A possible way is to design interoperable cryptographically verifiable credentials. Verifiable credential contains claim about the credential holder, issued by a trusted entity, and can be verified without contacting the issuer to prove themselves across various platforms. This is essentially leveraging emerging solutions, such as, Zero-Knowledge Proofs~\cite{tobin2016inevitable} with design principles and standards, such as, W3C verifiable credentials data model, DIF universal resolver~\cite{mazzocca2024survey}, etc. 

\textbf{Dynamic \& Multi-factor Personhood Verification}
One of the repeated concerns in our study whether PHC can be inadvertently shared/used by friends/family, hacked, stolen.  
%Although PHC verification itself does not contain identifiable information, there is a possibility that PHC will be misused if the user's local device is hacked or stolen. 
To address this concern, a potential design implication is to design a dynamic Multi-Factor Personhood Credential (PHC). First, when users verify themselves for the first time, the prompt can combine interactive biometric such as video call, with interaction knowledge question and sharing physical id to compare several ground truth data to issue a robust PHC. Second, to maintain security, a period biometrics verification with time-based trigger system can be designed to prompt for biometrics verification at random intervals or during high risk activities. 

%In this case, even on a platform where PHCs ensure one unique individual per account, malicious users could gain access to multiple accounts, resulting in significant damage to the platform's trustworthiness with sybil attack \cite{douceur2002sybil}. To ensure their PHC is used only by themselves, service providers could add another layer of security to the PHC authentication process. For instance, participants suggested multi-factor authentication (MFA) when interacting service providers to confirm PHC is used by the authentic user. 
%When users perform identity verification on various online services using PHC, combining PHC with biometric authentication on local devices, such as Apple Face ID, can reduce vulnerabilities. Having a 

\iffalse
\textbf{Dynamic Authentication}
One of the representative users’ concerns is the case when their PHC is hacked and used across different services by a third person. Although PHC itself does not contain identifiable information, there is a possibility that PHC will be misused if the user's local device is hacked or stolen. In this case, even on a platform where PHCs ensure one unique individual per account, malicious users could gain access to multiple accounts, resulting in significant damage to the platform's trustworthiness with sybil attack \cite{douceur2002sybil}. To ensure their PHC is used only by themselves, service providers could add another layer of security to the PHC authentication process. For instance, participants suggested multi-factor authentication (MFA) when interacting service providers to confirm PHC is used by the authentic user. When users perform identity verification on various online services using PHC, combining PHC with biometric authentication on local devices, such as Apple Face ID, can reduce vulnerabilities.
\fi

\textbf{Decentralized Standards for Industry-Government Issuance System.}
 Our work indicates issuance system and issuers (e.g. govt, private company; decentralized vs centralized) are one of the main factors leading to security and trust perception, thus the broader adoption. In the same line, fundamental limitation of current approaches to PHC creation is their signal failure to address the chicken-and-egg problem in PHC issuance. Without an ecosystem with a broad base of PHC issuers, systems that leverage PHCs will be slow to arise; conversely, without systems that leverage PHCs, it is hard to motivate the creation of PHC issuers. In the United States, for instance, despite industry coalitions pursuing decentralized identity credentials for at least seven years~\cite{mediumDecentralizedIdentity}, digitally signed state IDs are currently available only in California~\cite{caWalletPilot}. 
 To ensure global accessibility in supporting multiple stakeholders as issuers, the system would incorporate cross-chain interoperability protocols like Polkadot or Cosmos and utilize a permissioned blockchain network (e.g., Hyperledger Fabric) to create a distributed ledger for credential issuance and verification. Smart contracts~\cite{sharma2023mixed} would govern the issuance process, ensuring compliance with predefined standards set by both industry and government entities. 
%When it comes to online identity verification, users' trust in government is more apparent when considering who should be the stakeholders. They suggested that PHC architecture involves the government in various ways (e.g., serving as the PHC issuer and supervising PHC issuers). However, the government may not always be considered a trustworthy entity for users, as Hosking \cite{hosking2019decline} discussed the decline of trust in government due to structural uncertainty in society, such as financial crisis and economic inequality. Thus, government involvement should be framed as colocation as one of trusted entities, without centralizing authority within the government. We suggest that the government provides accreditation to PHC issuers to guarantee their reliability. For instance, U.S. Department of Education approves organizations that carried out accreditation to universities and colleges to assure the quality of institutions and programs \cite{eaton2015overview}. Users can assess the security and ethical standards of PHC issuers based on accreditation the government issued, and service providers can also refer to this accreditation when determining which PHCs they should support in their services.
%\tanusree{unoack how this type of choice can be provided to users and which way?} \ayae{updated}
\iffalse
\textbf{Blockchain-based Infrastructure}
Users’ concern against PHC issuers’ trustworthiness may be addressed by decentralizing infrastructure, for instance, blockchain-based issuer proposed by the participant. Blockchain technology drastically reduce the cost of trust given its distributed and transparent nature that the transactional data is open to the entire network and distributed consensus offers secure data exchanges \cite{casey2018blockchain, de2024personhood}. Leveraging this nature, blockchain-based identity management solutions has emerged in recent years \cite{liu2020blockchain}. For example, Sovrin achieves privacy-oriented identity management like identifying private customers under pseudonyms with zero-knowledge proof encryption \fixme{how come zkp is an encryption?} \cite{tobin2016inevitable}. Also, ShoCard elimiates the need for a third-party database by encrypting personal information on individuals' devices and storing and managing data derived from this information on the blockchain \cite{al2023enhancing}. We suggest PHC issuers store only verification metadata on-chain without storing any information about credentials or PHCs while users store PHCs and details information in local devices.  By doing so, users can manage their credentials and PHCs by themselves while achieving greater transparency in issuers' data practices.
%\tanusree{not just maybe, rather you need to connect the literature and provide specific design implication of such design of infrastructure} \ayae{updated}
\fi


%\tanusree{need to unpack, not clear how such system can be design/developed. need to be near-specific on the design} \ayae{updated}

%\textbf{Database Protection}

%\tanusree{unpack certain design for interface, be specific. in essence, this section is to show your design skill} \ayae{updated}

\vspace{-4mm}
\subsection{Limitations}
\vspace{-2mm}
%We were limited to recruiting participants covering a wider range of demographics via social media and recruitment platforms such as CloudResearch and Prolific.

Our interview study has several limitations. Our recruitment of participants resulted in limited diversity in educational backgrounds and age groups. This may restrict the generalizability of our findings, as individuals with a lower age range might exhibit more preferences toward new technology and can have stricter expectations compared to participants with other
educational backgrounds and age range. 

%which may lack representation from individuals with less formal educational opportunities and older adults.
%\fixme{what limitation in recruitment? exactly what aspect you consider as limitation?}. 
%Future work should explore additional recruitment strategies to reach a more diverse participant pool from a broader spectrum of ages and educational backgrounds. 
Second participants who didn't have prior knowledge of PHC responded based on an explanation or information video presented during the study. Their responses might differ if they were provided with hands-on interactive PHC system to better convey and understand the concept.
Building on the findings from this study, future work will focus on developing a functional prototype to provide participants with a more immersive and tangible experience.
%, facilitating a deeper understanding of their experience.
%Based on our design implications, future work also includes proposing potential components of PHC architecture and exploring an interface that reflects user needs, such as transparency and accountability, by enhancing the explainability of complex management of personal data in PHC.

\vspace{-2mm}
\section{Conclusion}
\vspace{-2mm}
%We interviewed 23 participants to explore users' perceptions and preferences regarding PHCs for verifying legitimacy and uniqueness in online interactions. 
Our study uncovered diverse user perceptions, including trade-offs between traditional verification methods and emerging approaches such as PHCs, as well as dilemmas between physical and digital verification. Furthermore, we highlighted nuanced preferences for each system design dimension: credentials, issuers, and architectures. Additionally, practical PHC functions such as limited credential validity, sensitivity-based selection, interactive human checks, and distributed issuance architectures were identified through design suggestions.
To our knowledge, this is the first user study to focus on PHCs. Our findings extend beyond PHCs, shedding light on key insights for identity verification.
%\vspace{-2mm}

\section{Ethics Considerations}
Our study design and procedures were reviewed and approved by the Institutional Review Board (IRB). We also considered the following ethical aspects:
\textbf{Disclosures}: All collected data (audio transcripts, sketches, and survey results) did not include personally identifiable information and were analyzed anonymously;
\textbf{Experiments with informed consent}: We ensured informed consent , and participants were informed that their participation was voluntary;
\textbf{Deception}: Participants were fully informed of all aspects of interview participation beforehand, clearly stating the scope of the study and its data collection.

\section{Open Science}
 To ensure transparency of this study, all details of our interviews, including the study procedures and survey contents, are available in the following link \url{https://anonymous.4open.science/r/PHC-user-study-14BB/}. We also ensure reproducibility by providing detailed documentation on how to proceed with our interview in Section \ref{sec:method}.