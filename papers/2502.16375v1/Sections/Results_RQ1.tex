
\vspace{-2mm}
\section{RQ1: Users’ Current Impression of PHC}
\vspace{-2mm}
%\subsection{Current Online Identity Verification Practices}

\textbf{Current Verification Practices.} 
When discussing identity[personhood] verification, 
participants most commonly mentioned financial services, including online banking and investment platforms, as well as health services, government-related processes, and cross-border regional verifications in both
%, and two-factor authentication (2FA) used by various institutions. 
%The discussion highlighted both 
digital and physical forms of credentials. Several participants mentioned they are required to upload government IDs (e.g., social security number, driver's license) when creating an account for financial services- as P7 said--\textit{``For Robinhood, it asked for uploading my government-issued IDs like driver's license and passport.''} 
%In addition to financial services, the applications where participants experienced verification processes include a wide range of applications. 
Similarly, P2 explained their verification experience in the government services requiring multiple information \textit{``Last year when I was requesting my tax filing documents in IRS. To access them, I had to verify my identity with my face, as well as information from government-issued IDs to confirm my identity. [id.me~\cite{irsIdentityVerification}].''} Beyond financial and government services, identity verification has also become essential in marketplace apps,
%that rely on trust to facilitate peer-to-peer transactions. 
P1 illustrated this trend with Airbnb, explaining  \textit{``If you do book an Airbnb, at least as a renter, you would need to verify your government id before being eligible to book your first stay.''} 
%This indicates that the scenarios we prepared for succeeding sessions are representative of typical user occasions. 

%\textbf{Confusion Between Verification and Authentication} Many participants did not distinguish between verification and authentication. In particular, when asked about experiences with biometric verification, they often shared daily experiences related to authentication, e.g., (P3): \textit{"It's I think most of the apps required me such fingerprints or related identification. For example, when I open my bank apps, I can just use my fingers to login to my account."} Participants indicated frequent use of fingerprints or face authentications on mobile phones. For instance, P1 mentioned stored biometric information on devices, \textit{"Because my android phone already have my fingerprint to log into my phone. And then this app just to reuse my fingerprint, which is already registered on my phone."}

%\textbf{Perception Before \& After Educational Video}
%As discussed in the method section \ref{sec:study_protocol}, we revisited questions about participants' understanding of PHC to compare their responses before and after watching the educational video to assess whether the video impacted their original knowledge of PHC. 


\iffalse
\textbf{Misconception \& Lack of Understanding of PHC.}
We observed that users sometimes \fixme{how may misunderstand, add the participants add in parenthesis} misunderstood PHCs. 
%These misconceptions represent critical points that should be carefully addressed in communication when aiming for the widespread adoption of PHC in the future. 
P10 explained the benefit \fixme{why benefit is misconception? you already have another theme for benefit} of PHC, stating \textit{"I am thinking that imagine you are a criminal or an online criminal. For example, with some records that you are not allowed to use some services online. For federal government, they need to verify your credentials to be able to track you or just not allow you to do activities."} This comment was interpreted as PHC enhancing traceability to assist in criminal investigations \fixme{how come this is a misconception? this perhaps a narrow understanding of PHC, same goes for tracking data aspects, explain to reflect on it}. 
\ayae{I incorrectly develop this theme. I flagged "misunderstanding" to narrow understanding. Let me skip this theme. }
\fi
%Such potential traceability was also mentioned in the context of tech companies tracking data, saying \textit{"But what is more of a concern for me is that some of these big tech companies like Google, like Facebook, like even Apple, Amazon. They use that information to track my searches and activities."} However, PHCs are fundamentally intended to serve as a means of anonymously verifying users' personhood without compromising their privacy.  \fixme{add 2 interesting quotes that have clear misconception about PHC. none of the quote above dictate clear misconception.} 

\textbf{Trade-offs between known and unknown privacy guarantee.} 
Participants often made trade-offs between familiar security guarantees associated with traditional verification methods over the less clear assurances of emerging PHC. As P18 mentioned she heard about World ID as a personhood credential and remarked \textit{``
%If I’m signing in for the first time or doing a person verification with World ID, it will be able to tell if I am signing in for the first time? I guess that means they still keep some sort of data from the iris scan. 
They scan your iris, create a unique code, and re-verify me again later time. I guess that means they still keep some sort of data from the iris scan. I would still stay with email or traditional verification as I know what they are keeping. Even though world app guarantees privacy with hi-tech, I don’t exactly know how much privacy I have in human tests. I’m not saying it’s a bad idea, it's just i am unsure.''} 
%#### use this in discussion We also identified another potential factor: age, which may influence perceptions of PHC and the types of data used in credential issuance. 
P23 with age ranges from 45-52 noted--\textit{``it's just easier to do with email or my physical photo id in an old system 
%like getting ssn, 
Newer technology is supposed to be faster and more user-friendly, but to me, iris scan or selfie scan, I can't even know if I am doing it correctly. I’m getting to the point where I can say I’m old-fashioned.''} This underscores the generational divide in preferences with emerging technologies compared to established methods.




%\textbf{Post-understanding: Found similarities with other technologies} P5 interpreted the term as a verification method based on personal characteristics, which differ from person to person, such as behavioral patterns, commenting \textit{"It's like, personality based or like activities humans particularly perform. It sounds similar to the pattern of people walking. Those are also things people use for verification."} After watching the video, he updated his perception of PHC \textit{"I thought it was similar to SSL technology. Like in SSL, they will give the certificate to 3rd party who encrypts and has those certificates. To maintain their integrity, they will pass on the kind of certificate to web browsers. So when a user goes through a particular website, they will be able to cross-verify using the public keys available."}, linking it to technologies such as SSL, which uses encryption and hashing for certificates.


%\subsection{Perceived Benefits of Personhood Credentials}
%\fixme{there needs to be titles of sub-themes under this benefit, it's hard to read. Sub-themes could be -- convenience of not carrying physical id, etc -- whatever unique /interesting subthemes came up in the interview}
\iffalse
\textbf{Perceived Benefits of PHC: Reduced Personal Information Exposure} %\fixme{this section is not interesting enough, is there quote where participants had to provide smany info like physical id, digital verification to get unique identifier? if so, then those examples / quotes will go well with this theme. please look for peoples' experience related things that was stressful or challenging, current quotes is just statement, not experience at all. we need to present themes through experience related quotes.} \ayae{reflected}
\tanusree{note: may remove this theme if space needed}Participants identified various benefits of PHC, frequently involving reduced risks related to personal data exposure, such as minimizing personal data exposure, reducing repeated checks for different online services, preventing data misuse, and enhancing security against fake accounts through trusted, privacy-focused third-party systems. As P3 mentioned about having different ids to even perform regular tasks \textit{``when I am in a foreign country, or even in dealing with housing, I don't trust the leasing process and people involved, sometimes the sites used are sketchy, but I had to verify my ID, paystub, ssn, residence permit or passport that would reveal all of my personal details. It would reveal my permanent address. Almost everything, most of the time I am not comfortable at all. But with PHC, it's like they use a minimal of verification.''} P13 on contrast mentioned different verification method for different sites, \textit{``it feel like I am putting my information into a lot of different sites, [..]. finance with KYC, student id with capturing photo, traveling with passport. With PHC, much of these can be improved''} P1 explained how PHC can increase confidence with unfamiliar entities \textit{``All of that information doesn't need to be filled in each time, rather verifying with a unique identifier. Even there is a breach, its less likely to link my personal data''}
%you fill a form for submission and the potential conference that if I am signing up for an account with a 3rd party, which I have not heard before. I'm not comfortable with even in the case of a breach. 
%They would not have access to my personal information, but just rather the identifier id. So that would be something I'd be comfortable with.''} 

%with the most frequently mentioned the reduced personal information exposure in the issuing process. P3 expressed perceived benefit with her anxious experience of sharing sensitive information for verification, \textit{``I think PHCs, essentially, they collect certain information like Iris scanning and use that to verify your identity online without exposing your personal details. So personal details, such as maybe my name or my postcode, or if I were to verify my ID through scanning like a residence, permit or passport that would reveal all of my personal details. It would reveal my address. My full name even my like my number, such as the passport number and things like that. But with PHC, it's like they use a minimal of verification.''} P13 reflected her thoughtless data sharing when asked about PHC benefits, \textit{``I think at the moment it's kind of like it does feel like I am putting my information into a lot of different sites, probably without thinking too much about it. So I think that I'd feel a lot better doing that.''}
%Several also emphasized advantages in addressing online identity crises caused by AI.
%P4 expressed \textit{``I think what I glean from it was there was to provide without giving away too much personal information, but it can still sort of verify your uniqueness.''} In addition to this, 
%P11 summarized the benefits of PHC, \textit{``Personhood credentials is a way in which a person can confirm that they are not a bot or an AI, and that they're a real person without giving away their personally identifiable information.''} 
%P11 discussed online malicious attacks, \textit{``I think that it will tackle like fake accounts, multiple accounts and bots massively. And I can see it being used broadly because you're not giving away any really important information... And having to get this identification through your government, or something works really well because you don't mind giving them the information. And then, the kind of redacted information that you give to social media sites, increases safety and stuff, but doesn't put your data at risk.''} 
%P1 explained increased confidence in privacy through PHCs for data sharing with unfamiliar entities \textit{``All of that information doesn't need to be filled in each time, you fill a form for submission and the potential conference that if I am signing up for an account with a 3rd party, which I have not heard before. I'm not comfortable with even in the case of a breach. They would not have access to my personal information, but just rather the identifier id. So that would be something I'd be comfortable with.''} 
%\textbf{Efficiency of Verification Process} 
%\fixme{this section can be merged to reduced personal info. do you have experience related quotes to present this effectiveness. current quote is not interesting} \ayae{reflected}Another key benefit highlighted by participants was the efficiency of the verification process through less exposure to personal information. P11 shared her perspective on the lengthy verification process, \textit{``I think it streamline the process because I mean some websites, some social media do request you to upload documents to verify your identity these days, and it is quite a long process sometimes, like I said,  sometimes your ID gets rejected. So maybe using this new method would be quicker. And you could set up an account a lot easier, and that'd be good.''}
%P3 emphasized the convenience of less repeated verification, stating, \textit{"Having an easier login from a trusted source. So it's ease of access for the user." } 
\fi


\textbf{Perceived Benefits of PHC: Fairness in Representation.} Participants discussed fairness and the potential benefits of PHCs. P19 shared her experience with online study platforms which is one of the primary means of earning, \textit{``often time I got "time out or returned" in this platform and earlier I thought I am slow responding to survey request I received on my account. Lately, it happens too often, feels like there are bots or a group of people are more proactive in participating in surveys. honestly, it reduces the chances for me being selected for research studies ''} She emphasized that personhood verification could help filter out bots and increase her chances as a genuine participant.  P21 who is a dedicated gamer who find it disheartening to encounter players, likely bots, who level up so quickly. To emphasize the benefit of PHC, he mentioned \textit{``“ I play online games with other people around the world, like shooting games. We regularly encounter bots in these games, which basically dilutes our experience. As an experienced player, I can usually tell when someone isn’t a real person
based on how they play. personhood verification could be a good feature.''}




%\subsection{Perceived Concerns of Personhood Credentials}
%\fixme{there needs to be titles of sub themes under this benefit, it's hard to read }


\textbf{Perceived Concerns: Power, Control, Security} 
%\fixme{the title is too plain, I remember there was interesting themes in the coding and i don't see those here. please go back to the excel to rewrite perceived concerns. first summarize these concerns from the list below I added in the comment line and add 2 quotes which are most interesting. } \fixme{not addressed well. as said before you need to first summarize the concerns in a line and then add a quote on interesting ones.  this seem interesting -"failed to detect criminal information"  and "uncertain regulation", "centralized power"}\ayae{reflected on 1/9}
% Concerns: PHC/credentials getting stolen (due to hacking)
% Concerns: credential's validity period
% Concerns: untrustworthy PHC issuer
% Concerns: centralized data storage
% Concerns: Issuers can be hacked
% Concerns: PHC can de-anonymized
% Concerns: information will be stored centrally, this may lead to power can be abused later
% Concerns: uncertain regulations
% Concerns: failed to detect criminal information
% Concerns: malicious attacker create fake PHC
% Concerns: data linkage of credential and information stored in service providers
Participants expressed concerns about potential risks, including the centralized power of the issuer, uncertain regulations of emerging technology, authenticity of PHC, and misuse of anonymity.
For instance, P6 pointed out that centralization of credential information with PHC issuers could lead to power being misused, saying, \textit{"you're gonna give all your information to a small group or institutions to issue PHC. So they have the power that can be abused later."} 
%P10 expressed concerns regarding the authenticity of PHCs, particularly in scenarios where credentials could be falsified. They noted, \textit{"I don't know if it's possible to fake government-issued ID. I feel like I'm probably concerned about an inauthentic PHC."}\fixme{this seems off compared to the excel} 
P10 also highlighted the risks in high-stakes situations, such as employment background verification, where an inauthentic PHC could have serious consequences. He noted \textit{"I don't know if it's possible to fake government-issued ID. I feel like I'm probably concerned about an inauthentic PHC."} P4 in a similar topic expressed concerns about how much data will be revealed to an employer if they only share PHC \textit{``For instance, I had a job offer that required details about my criminal record from five years ago. It would probably reveal more even if I only share PHC. how much anonymity I have.''}

%"That's good. I feel like for some type of government position. I'd be more open to doing that. depending on the company. And the, I guess, like intensity of their background check. And like how far they're really trying to go back. It would depend on what Phcs you would disclose. If you get to pick, but again for a background, check it. I would be fine using a Phd.

%Sure. Yeah. So, for example, like, I had a job offer that was asking about 5 years prior, like a crime record so like for that type of severity where they're kind of looking pretty far back, in my opinion, from 5 years I would probably be more that would probably need more Phc credentials than yeah one. But one that's just like a general. You are a real person type background check, which I actually don't really know how common that is, but like less severe."


%Another significant concern is the anonymity of PHCs. One participant emphasized that for background checks, providing personal information is crucial to ensure a thorough review of candidates' histories, expressing, \textit{"For the background check, I would provide my personal information. Because it might be needed information to fully check their background and if the hired employees doesn't have any negative histories or criminal records, or something like that. That's important for the company. So that's why I would provide the personal information instead of PHC."}
%P4 -- "Possibly that again the somebody trying to steal the you know the the image, or whatever you're trying to upload. So I think this is a general cyber security concern."
%a single point of failure in centralized control, data governance and identity integrity issues.
%P6 was concerned about the possibility of their biometric credentials being hacked, saying \fixme{not interesting quote}
%\textit{"I worry that if someone were to steal your biometric data like your iris scan that then they could just have access to all of your accounts through that one thing through the one iris scan. So I don't want there to be like one single weak point that could just topple the entire thing like that."}
%P17 discussed uncertain regulations and expected clear guidelines on data practices of PHC systems while it involves more sensitive process than a password and email \textit{"if there is a good regulation that what extent this PHC system is able to share personal information to governments or other institutes and what security might be in place."}
%P17 -- "My main concern is that, for example, someone could record your iris in that case, or something like that, and manage to trick the verification system just by recording your eyes, or something like that. I don't know if that's really like a stupid thought. But that's what I was imagining, like as a concern, that maybe people could record your face or your iris, and maybe trick the system. I don't know if that's possible. Yeah, that that would be a really concerning situation. Because if everything like, if all my passwords were set up in a by a PHC, I'm guessing that if they if all my information gets leaked, people could access all. My, for example. bank account information, or yeah, health information. And yeah, that that would be really concerning."
%\fixme{one quote used twice, not clear} This topic can be interpreted as an explainability issue. P10 emphasized the importance of transparency in data practices, \textit{"I'm curious where this (credential) is saved, though, like what server holds all this information. That would be cause I feel like recently, we've seen more stuff about like banking apps, having data breaches or something, which is a concern with your data saved."} 


%Another significant concern is the anonymity of PHCs. One participant emphasized that for background checks, providing personal information is crucial to ensure a thorough review of candidates' histories, expressing, \textit{"For the background check, I would provide my personal information. Because it might be needed information to fully check their background and if the hired employees doesn't have any negative histories or criminal records, or something like that. That's important for the company. So that's why I would provide the personal information instead of PHC."}
%P4 -- "Possibly that again the somebody trying to steal the you know the the image, or whatever you're trying to upload. So I think this is a general cyber security concern."

\textbf{PHC Preference Dilemma: Physical and Digital Verification.}
Many participants often relate their offline experiences when describing their preference for credentials and issuance systems.
For instance, some of them described the inconvenience of carrying physical IDs %with offline experience of verification in daily life
, saying \textit{``When you need to buy some alcohol drinks, take a flight, you need to show your driver's license or passport. %And when you need to buy some in the market or the grocery stores, and when you take a flight, you also need to use these IDs. 
But it's not a very convenient, because I have to carry the physical ID all the time.''} Another participant pointed out the potential risk of IDs being stolen, citing, \textit{``I think this information may be lost, or maybe stolen. So it's a risk.''} 
While participants highlighted the inconvenience and physical vulnerabilities of traditional IDs, they simultaneously shared the limitations of digital identifiers, such as verification through phone number [OTP] during international travels.

%which is often dependent on  .P\fixme{} noting, \textit{"And I think the challenge I met is since I need to use my Chinese number to receive the code. The good thing is that I can receive the code in the US. But I'm not sure if I can receive the code in other countries. So I think that's very inconvenient if I'm traveling abroad."}\fixme{what does this quote convey, who said this. no participant id here} \ayae{This quote explains that if the credential is a digital identifier such as a phone number, there will be problems when traveling internationally. This quote from pilot study P1, I tried to find alternative quotes from main study, but it was unique. Should I skip this part? }\fixme{i can understand from what you say, but not from the quote. this is researcher's job to make sure the quote is understandable}

%\textbf{Data Handling by PHC Issuers} Another significant concern was how PHC issuers manage and handle the credential data they collected. Participants showed uncertainties regarding what happens to their personal information after verification and whether it is securely stored or potentially mishandled. For example, P4 shared that \textit{"I'm not sure if they tell us explicitly how they're gonna use, do they going to share our information with 3rd party or with government. If they ask for it, I'm not sure about that. These are my concerns."}

%\textbf{Centralized PHC} Several participants expressed concerns over the centralized power held by PHC issuers, which they feared could lead to misuse or overextension of control. P4 commented \textit{"But I think there's a concern on the other side that they're gonna be one kind of you're gonna give all your information to a small group or an institution. So they have the power that can be abused later."} illustrating the apprehension about the concentration of power in the hands of a single entity as a PHC issuer.
\iffalse
\textbf{Trustworthiness of PHC Issuer} The issue of trustworthiness in PHC issuers was also discussed, with participants expressing doubts about the reliability of the entities managing their data. Some worried whether these issuers could be trusted to protect their personal information and maintain their privacy. P7 described \textit{"My only concern is, how do we trust the Phc Am I using it? Yeah, because if if they are the one entity who handles all these kind of information, and if they are collecting everything about us from the government. And whether this entity is considered like a NGO, or that's a governmental body, or that if that's a private sector, so these are like some nuances about how to define or how to build this kind of issuer, because we have these giant issuers like Equifax or like these giant companies who are handling credits or those kind of financial aspect of the US structure. And they're like giants. They have these kind of monopoly over how they handle things. Is it going to be something similar? If so, how people can trust them? I don't want to share my personal information anymore, with like some random website...But how the way that the issuer can be understood? I think that would be the decision factor for users to start engaging with the system."}

\textbf{How Users Evaluate the Trustworthiness of Stakeholders} When discussing the current verification process, participants often indicated that there are organizations they can trust and others they cannot. Thus, we asked how or why they have trust with specific stakeholders. P5 explained his trust in banks comes from regulatory aspects: \textit{"They are being monitored by federal agencies. Their activities are monitored. They are under a lot of regulations. So there is a monitoring system that is tracking banks, at least Major Banks or Banks that I know of and other financial services they are under a lot of regulations."} P4 highlighted another aspect resulting in their trust in banks,  which is that the accessible customer support, stating \textit{"You know what sort of steps have taken so you could probably have a read of that, and any concerns that you might have. So you probably always have the opportunity that you could always speak to them first, because obviously they probably have agents available that you could speak to about any concerns. "} On the other hand, we observed that many participants expressed distrust towards social media platforms. P3 shared a formative experience of her social media account almost being hacked, noting \textit{"I've had people experience...Who then tried to get like hack into my account. So the one hacker would hack one person's account, and because, obviously, in your social media, you have access to other people's accounts as well, like I've had people send me really dodgy links, which I knew like instantly, that if I clicked on that link, it would mean that my account would get hacked as well."} Some participants elaborated on how they evaluate the trustworthiness of companies or stakeholders when they share their sensitive information in general. P6 reflected \textit{"I guess if the company was suggested to me, or I was referred to the company through the services of like Capital One or a big company that's been around forever. That would give more validity to me."} P7 provided insight from his own experience with a suspicious verification process, saying \textit{"I do a lot of research… What I did was I went and searched on the Internet. So I went on Reddit, and I looked for different keywords, and my conclusion was that it's a very standard process. It's not like the dealership wants to collect the social security for themselves. They wanted to do the background check, and they wanted to see if the information that they provided was matching with my information from the US. "}
\fi
%\textbf{Offline Verification in Daily Life} Some of the participants shared their offline experience that required identity verification, especially in the context of international travel and immigration procedures,  as P3 described \textit{"So in the past I've had to use my fingerprints, for example, when I came to the UK. When I had to renew my passport in the embassy of my country, they would take my fingerprint there. So they would take a photo of my current passport, and then they would take a photo of myself, but then they would also require me to give my fingerprints as well."}  


\iffalse
\textbf{Motivational Experiences for Using PHC}
\fixme{give a clear theme title. what motivation? this title doesn't convey to RQs. think critically}
When discussing the benefits of PHCs, participants shared their experiences with deceptive online activities stemming from digital identity crises ranging from bots to scams. P10 explained the issue of bot accounts on social media, noting \textit{"I get random bots following me on my twitter a lot which I'm assuming. If they had this PHC Implemented, we would probably see a reduction in those types of bots...I'm assuming there's bots on Instagram as well, and I'm sure most social media accounts is probably bots."} P11 also reflected on fake news spread by bots, \textit{"I know that bots and AI and stuff are utilized quite a lot on social media sites, and to spread like fake news, and to support certain people's agendas and the kind of a tool that that are used in order to."} Accordingly, identity verification is recognized as a key area of concern in the context of social media platforms for end-users due to the proliferation of bot accounts and the dissemination of fake news. 

In addition, some participants emphasized the significance of identity verification in relation to cryptocurrency. P13 stated scams in cryptocurrency, \textit{"I think with cryptocurrency, everyone's scamming everyone all the time...And the companies who do the id verification basically decided it might be legally difficult. So they just stop. So that's probably the main problem I've had."} P12 indicated trade-off between privacy and traceability in crypto wallets, \textit{"I don't really like identity you put in your facial verification, or your Id or your driver's license on trust. But for Binance, Bybit and KuCoin, they actually request that because it's actually a big company or anything. But I'll see it's just a very diverse app that you can do a lot of things on. It needs the privacy of other people. It needs the privacy of other people and yours to cooperate because of scams and fraud."}

\textbf{Context Matters for Users' Preference}
 \fixme{this needs to expand a lot, this is one of the main section Section 2 for all the scenarios, and one of the main results. we also have post survey for this result. move some visuals here and rewrite this }
As outlined in the method section \ref{sec:study_protocol}, we explored participants’ perceptions and preferences regarding PHC across five distinct scenarios. Participants’ responses indicated the necessity of verification methods depending on the specific scenario. In some contexts, they considered PHC unnecessary and suggested simpler methods like email or phone verification. As P5 stated, \textit{"I love to talk, to, chatgpt, and bard about weird stuff, and I think a lot of people like to talk to them about weird stuff. And so I would like the option to not log in via biometrics, and have a lesser experience and not have my identity tied to it. So I personally, I would not like to have my identity tied to my chat Gpt, or my bard, or Gemini."}, they explained that online platforms, which do not require extensive personal information, are sufficient with simple verifications.
\fi
