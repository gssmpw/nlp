\iffalse
\section{Appendix: Research Method}
\label{appendix_method}
{\scriptsize
\newcounter{qcounter}
\newcommand{\mychoice}[1]{{$\circ$} #1 \, }

% \subsection{Pilots}
% \label{pilot}
% We conducted five pilot interviews with graduate students in the US to test our study design. Two of them were master students in cybercecurity and the rest of them were PhD students in informatics. 
% They are international students (China: 2, Japan: 1, Iran: 1, India: 1). Based on pilot feedback and our observations, we revised our interview questions for the main study.

\subsection{Knowledge Questions}
\label{knowledge_questions}
\begin{enumerate}
\item What could happen if online identities are poorly verified?\\
\mychoice{Digital platforms could become more secure.} \\
\mychoice {\textbf{Fake accounts, bots, and fraud could increase significantly.}} \\
\mychoice{Users could feel more confident and trusting in online environments.} \\
\mychoice{Privacy could be better protected across online services.}

\item What are Personhood Credentials (PHCs)?\\
\mychoice{\textbf{Digital credentials that confirm a person’s identity}}\\
\mychoice{A method for storing personal data online for easy access}\\
\mychoice{Credentials issued by social media platforms to verify account authenticity}\\
\mychoice{A tool used by online services to track user activity and preferences}

\item What is the primary goal of PHC?\\
\mychoice{To reduce the number of online accounts.}\\
\mychoice{To track a user's online activity across platforms.}\\
\mychoice{\textbf{To verify a person's identity without exposing personal information.}}\\
\mychoice{To simplify the process of logging into multiple services.}

\item To whom do you provide minimal personal information during the PHC process?\\
\mychoice{Online service providers (e.g., social media)}\\
\mychoice{Entities handling data storage for online accounts}\\
\mychoice{Third-party platforms}\\
\mychoice{\textbf{PHC issuers (e.g., governments or trusted organizations)}}

\item Which of the following options are considered as credentials of PHCs? Select all that apply.\\
\mychoice{\textbf{Government credentials (e.g., passport, driver's license)}}\\
\mychoice{\textbf{Biometric indicators (e.g., face, fingerprint, iris)}}\\
\mychoice{Email address}\\
\mychoice{Phone number}

\end{enumerate}

\subsection{Post-Survey} 
\label{post_survey}
\begin{enumerate}
\item \textbf{Credential Preference}: Which types of credential would you prefer to use as personhood verification for each scenario? Select all that apply. \textit{(This question is displayed in a multiple choice grid, prompting responses for each scenario: Finance Service, Healthcare Service, Social Media, LLM Application, Government Service, and Employment Background Check.)}\\
\mychoice{Government issued ID (e.g., SSN))}\\
\mychoice{Passport}\\
\mychoice{Driver’s license}\\
\mychoice{Face scan}\\
\mychoice{Fingerprint scan}\\
\mychoice{Iris scan}\\
\mychoice{Phone number}
\item \textbf{Issuer Preference}: Which type of issuer would you prefer to issue and manage your PHC for each scenario? Select all that apply. \textit{(This question is displayed in a multiple choice grid, prompting responses for each scenario: Finance Service, Healthcare Service, Social Media, LLM Application, Government Service, and Employment Background Check.)}\\
\mychoice{Government}\\
\mychoice{NPO (nonprofit organization)}\\
\mychoice{Private Companies}\\
\mychoice{Financial Institution (e.g., banks)}\\
\mychoice{Educational Institution (e.g., universities)}
\item Do you have any other issuer in mind that you would prefer to issue PHC? \textit{(This question is open-ended.)}
\item \textbf{Architecture Preference}: Which type of system would you prefer to issue and manage your PHC for each service provider? \textit{(This question is displayed in a multiple choice grid, prompting responses for each scenario: Finance Service, Healthcare Service, Social Media, LLM Application, Government Service, and Employment Background Check.)}\\
\mychoice{Centralized}\\
\mychoice{Decentralized}
\end{enumerate}

\section{Appendix: Interview Protocol}
\label{protocol}
\subsection{Section 1: Current practices regarding digital identity verification}
\begin{enumerate}
\item Can you tell me about any online platforms or services you have used before or currently using that ask you to verify your identity, like uploading your ID or confirming your identity in other ways?

\item You mentioned using online platform X. Can you tell me what method you used to verify your identity for platform X? [repeat question 2 for other services mentioned Y, Z, …]

\item Can you share your experience with that verification method? Was it easy to use, or did you run into any issues?

\item Did you encounter any challenges when using this method?

\item Have you ever used services where you had to verify yourself through face, fingerprints, or iris scans, or other biometrics? 

\item Can you tell me more about your experience with those methods? What worked well? Were there any concerns you had?
\end{enumerate}
\subsection{Section 2: Users' perception of PHC (RQ1)}
\subsubsection{Pre-Understanding of Personhood Credentials}
\begin{enumerate}
\item Do you know the term 'Personhood Credentials'?  If not, can you explain what you think it means by just hearing the term?
\item Have you used any 'Personhood Credentials' before? If you say yes, then ask, What are those? 
\item Do you know how 'Personhood Credentials' work within the services you use? 
\item Do you know how the service providers handle the credentials you provide as 'Personhood Credentials' so you can access the service?
\end{enumerate}
\textit{(Participants are asked to fill the online survey to assess their pre-understanding of personhood credentials.)}

\subsubsection{Educational Session}

\textit{(Participants are shown a 2-minute educational video about the concept of personhood credentials. The video can be accessed in the following link: https://anonymous.4open.science/r/PHC-user-study-14BB/)}

\subsubsection{Post-Understanding of Personhood Credentials}
\begin{enumerate}
    \item After reviewing the educational materials, how would you explain your understanding of personhood credentials?
    \item Can you think of any benefit of a personhood credential based on your understanding? Why?
    \item Can you think of any risk of a personhood credential based on your understanding? Why?
\end{enumerate}
\textit{(Participants are asked to fill the online survey to assess their post-understanding of personhood credentials.)}

\subsection{Section 3: Scenario Session (RQ1)}
Now, we will go over some specific application scenarios of personhood credentials to understand your perception and how we can better design the personhood credential for different services.

\textbf{Scenario 1: Financial Service}
Imagine you are opening a new bank account with a bank that offers a fully digital onboarding experience. The bank requires you to verify your identity using personhood credentials, such as facial recognition for a liveliness test, and then uploading a government-issued ID (like your passport or driver's license). Once verified, you can open a bank account access your bank account, and manage your funds.
\begin{enumerate}
    \item How did you feel about using personhood credentials to verify your identity when opening your bank account? 
    \item \textbf{Benefits}: What potential benefits do you see in using PHC in this online banking context?
    \item \textbf{Security}: Do you think using personhood credentials improves the security of your bank account?
    \item \textbf{Privacy}: Did this method of identity verification make you feel more confident about your privacy?
    \item \textbf{About Data}: Were you comfortable providing your government-issued ID and using facial recognition?
    \item \textbf{Concerns}: What concerns do you have in providing your government-issued ID and using facial recognition, is being stored and used by the online bank?
\end{enumerate}

\textbf{Scenario 2: Healthcare Services}
Imagine you are using an online healthcare platform to schedule appointments and receive virtual consultations with doctors. To provide access to the platform, the healthcare service requires you to verify your identity using personhood credentials, such as a biometric verification using a fingerprint for a liveliness test, and then upload a government-issued ID (like your health insurance card or passport). Once your identity is verified, you can access your medical records, schedule appointments, and have virtual consultations with healthcare providers.
\begin{enumerate}
    \item How did you feel about using personhood credentials to verify your identity when using an online healthcare platform? 
    \item \textbf{Benefits}: What potential benefits do you see in using PHC in this healthcare context?
    \item \textbf{Security}: Do you think using personhood credentials improves the security of the healthcare platform?
    \item \textbf{Privacy}: Did this method of identity verification make you feel more confident about your privacy?
    \item \textbf{About Data}: Were you comfortable providing your government-issued ID and using fingerprint?
    \item \textbf{Concerns}: What concerns do you have in providing your ID and fingerprint information, is being stored and used by the healthcare platform?
\end{enumerate}

\textbf{Scenario 3: Social Media}
Imagine you are creating a new premium social media account.  You are required to verify your identity with personhood credentials, such as a video call or selfie for a liveliness test  (or proof you are a real, unique user), and then upload your government-issued ID (SSN, national ID card). Once your identity is verified, you can access social media. 
\begin{enumerate}
    \item How did you feel about using personhood credentials to verify your identity when creating a new premium social media account?
    \item \textbf{Benefits}: What potential benefits do you see in using PHC in this social media context?
    \item \textbf{Security}: Do you think using personhood credentials improves the security of the social media platform?
    \item \textbf{Privacy}: Did this method of identity verification make you feel more confident about your privacy?
    \item \textbf{About Data}: Were you comfortable providing your government-issued ID and a video selfie?
    \item \textbf{Concerns}: What concerns do you have in providing your ID and using a video selfie, is being stored and used by the social media platform?
\end{enumerate}

\textbf{Scenario 4: LLM Application}
Imagine you are signing up for an AI language model application like ChatGPT to assist you with tasks and information. Identity verification is also crucial in LLM-based applications because the interaction between multiple users and its model can significantly impact the reliability and safety of the platform.
For example, malicious users could submit harmful prompts or requests that might lead to biased or damaging outputs from the model. With identity verification, the platform can confirm that each user is legitimate, thereby reducing these risks. Thus, imagine to sign up for the application, the LLM service requires you to verify your identity using personhood credentials, such as a biometric verification using an iris recognition scan for a liveliness test.
\begin{enumerate}
    \item How did you feel about using personhood credentials to verify your identity when signing up for the AI language model application?
    \item \textbf{Benefits}: What potential benefits do you see in using PHC in this AI application context?
    \item \textbf{Security}: Do you think using personhood credentials improves the security of the AI language model application?
    \item \textbf{Privacy}: Did this method of identity verification make you feel more confident about your privacy?
    \item \textbf{About Data}: Were you comfortable providing your iris scan?
    \item \textbf{Concerns}: What concerns do you have in using iris scan, is being stored and used by the LLM application providers?
\end{enumerate}

\textbf{Scenario 5: Government Service}
Imagine You are applying for government assistance or social benefits (e.g., healthcare, unemployment benefits) to government portal (e.g., USAGov, Your Europe). To ensure that only eligible citizens access these services, the platform asks you to verify your identity using personhood credentials by using an iris scan for a liveliness test and then uploading your government-issued ID to confirm your identity for government benefit.
\begin{enumerate}
    \item How did you feel about using personhood credentials to verify your identity when using a government portal?
    \item \textbf{Benefits}: What potential benefits do you see in using PHC in this government portal context?
    \item \textbf{Security}: Do you think using personhood credentials improves the security of the government portal?
    \item \textbf{Privacy}: Did this method of identity verification make you feel more confident about your privacy?
    \item \textbf{About Data}: Were you comfortable providing your government-issued ID and iris information?
    \item \textbf{Concerns}: What concerns do you have in providing your ID and using iris scan, is being stored and used by the government portal?
\end{enumerate}

\textbf{Scenario 6: Employment Background Check}
Imagine you have received a job offer from a company, and as part of the final hiring process, they require an employment background check. To confirm your eligibility and verify your work history, the background check service asks you to submit personhood credentials using fingerprint verification and uploading your government-issued ID to verify your identity.
\begin{enumerate}
    \item How did you feel about using personhood credentials to verify your identity for an employment background check?
    \item \textbf{Benefits}: What potential benefits do you see in using PHC in this background check context?
    \item \textbf{Security}: Do you think using personhood credentials improves the security of the employment background check?
    \item \textbf{Privacy}: Did this method of identity verification make you feel more confident about your privacy?
    \item \textbf{About Data}: Were you comfortable providing your government-issued ID and fingerprint information?
    \item \textbf{Privacy Concerns}: What concerns do you have in providing your ID and using fingerptint, is being stored and used by the background check service?
\end{enumerate}

\subsection{Section 4: Users' preference of PHC (RQ2)}
We talked about several scenarios where different types of potential credentials to issue you PHC that could verify yourself as a unique human. In this section, we would like to discuss on different types of credentials you might prefer in issuing personhood credentials for you and how you might want to manage those based on your privacy and security need.

If you were to use PHC for identity verification, 
\begin{enumerate}
    \item What types of credentials would you prefer to use as personhood verification?
    \item Which organizations or stakeholders would you prefer to issue and manage your PHC? 
\end{enumerate}

\subsection{Section 5: Design Session(RQ3)}
In the former session, we discussed the concerns/potential risks in PHC. I wrote down these potential problems in the top of whiteboard. We would like to brainstorm the idea how we can solve these problems by sketching your ideas. For sketching, you can use this digital whiteboard, or use pen and paper in your desk if you prefer. 
%\textit{(As shown in Figure \ref{fig:sketchnote}, 
We send the link to the zoom whiteboard where pre-arranged components like the PHC issuer and database are placed on the left, while potential concerns or risks suggested by participants during interviews are noted as topics in the top right.)}
\begin{enumerate}
    \item In your opinion, how can we improve the design for PHC to address issues/concerns we discussed? Could you sketch a solution?
    \item Could you explain your drawing?
\end{enumerate}
\fi
\section{Appendix: Tables \& Figures}
\iffalse
\begin{figure} [!t]
	\centering
	\includegraphics[width=\linewidth]{Fig/zoom_whiteboard.png}
	\caption{Zoom Whiteboard for Design Session} %\fixme{figure caption needs to be self explanatory by itself}
\label{fig:sketchnote}
\end{figure}
\fi
% \begin{figure} [!h]
% 	\centering
% 	\includegraphics[width=\linewidth]{Fig/sketch_P0_area.png}
% 	\caption{Sector-based Decentralization} %\fixme{figure caption needs to be self explanatory by itself}
% \label{fig:P0}
% \end{figure}

%\begin{figure} [!h]
%	\centering
%	\includegraphics[width=\linewidth]{Fig/sketch_P9_area.png}
%	\caption{Single Issuer System Design} %\fixme{figure caption needs to be self explanatory by itself}
%\label{fig:P9}
%\end{figure}


%\subsection{Section 6: Users’ preference of PHC(RQ2)}
%\textit{(Participants are shown a 1-minute educational video about introduction of decentralized architecture of personhood credentials. The video can be accessed in the following link: )}
%\begin{enumerate}
    %\item Could you share your understanding of the PHC in a decentralized context?
    %\item You can get multiple PHC credentials from different organizations or stakeholders. Would you prefer to get multiple PHCs from different issuers depending on the situation or application you're using, or would you rather have a single PHC from one issuer?
    %\item If they prefer decentralized: Could you explain why you would prefer multiple issuers in managing your PHCs? (e.g., one PHC from gov, one PHC from big tech)
    %\item If they prefer centralized: Could you explain why you would prefer a single issuer in managing your PHC?
%\end{enumerate}
%\textit{(Participants are asked to fill the online survey to ask their preference of personhood credentials.)}
\iffalse
\begin{table*}[h!]
\centering
\caption{Overview of PHC Application Scenarios}
\label{table:scenario}
%\resizebox{\textwidth}{!}{%
\begin{tabular}{lll}
\hline
\textbf{Scenario} & \textbf{Service Provider} & \textbf{Types of Credential} \\
\hline
Financial Service & Bank, Financial Institutions & Passport or Driver’s license, Face scan \cite{yousefi2024digital}\\
% \hline
Healthcare Service & Hospitals, Clinics & Health insurance card,  Fingerprint \cite{chen2012non,fatima2019biometric,jahan2017robust}\\
% \hline
Social Media & Tech Companies & National identity card, Video selfie \cite{instagramWaysVerify, metaTypesID,instagramTypesID} \\
% \hline
LLM Application & Tech Companies & Iris scan \cite{WorldWhitepaper, worldHumanness}\\
% \hline
Government Service & Government & Driver’s license or National identity card \cite{LogingovVerify}\\
% \hline
Employment Background Check & Background Check Companies & Tax identification card, Fingerprint\cite{cole2009suspect}\\
\hline
\end{tabular}%
%}
% \vspace{0.5em}
\label{tab:scenarios}
\end{table*}



{\scriptsize
\begin{table*}[ht]
    \centering
    \scriptsize
    \begin{tabular}{llll}
      \hline
       App Name  & Source & reviews  \\
    
        \hline
     Worldapp & White Paper~\cite{WorldWhitepaper}, Google Play Store& 1523 \\
  BrightID & White Paper~\cite{BrightID},Google Play Store & 328 \\
  DECO & WhitePaper~\cite{zhang2020deco} & Review  \\
  CANDID & WhitePaper~\cite{maram2021candid} & Review \\
  Proof of Humanity &  WhitePaper~\cite{PoHexplainer} & Review \\
  Adhar Card &  WhitePaper~\cite{Aadhaar}, Google Play Store & Review
  %https://play.google.com/store/apps/details?id=in.gov.uidai.mAadhaarPlus&hl=en_US
  \\
Estonia e-ID  &  WhitePaper~\cite{estoniaE-ID} & Review\\
Chinese Credit system &  WhitePaper~\cite{ChinaSocialCreditSystem} & Review \\
Japan My Number Card &  WhitePaper~\cite{JapanMyIDNumber} & Review \\
ID.me &  WhitePaper~\cite{irsIdentityVerification, idAccessAll}, Google Play Store & Review \\
%https://play.google.com/store/apps/details?id=me.id.auth&hl=en_US
Idena &  WhitePaper~\cite{idenaWhitepaper} &  Review \\
Humanode &  WhitePaper~\cite{kavazi2021humanode} &Review\\
Civic &  WhitePaper~\cite{CivicPass} &Review \\
Federated identities (Oauth) &  WhitePaper~\cite{OAuth} & Review\\
  \hline
    
    \end{tabular}
    \caption{Competitive Analysis Data Sources 
   % \fixme{may move to appendix later}
    }
    \label{tab:systems}
\end{table*}
\begin{table*}[h]
\centering
\caption{Participant demographics and background.}
%\fixme{add the participants you completed so far}
\resizebox{\textwidth}{!}{%
\begin{tabular}{l l l l l l l l}
\hline
\textit{Participant ID} & \textit{Gender} & \textit{Age} & \textit{Country of residence} & \textit{Education} & \textit{Technology background}  & \textit{CS background} &\textit{Residency duration} \\
\hline
P1 & Male & 25-34 & the US & Master's degree & Yes & Yes &3-5 years\\
P2 & Female & 25-34 & the US & Master's degree & Yes & Yes & 1-3 years\\
P3 & Female & 25-34 & the UK & Master's degree & Yes & No & 1-3 years\\
P4 & Female & 35-44 & the UK & Some college, but no degree & Yes & Yes & Over 10 years \\
P5 & Male & 25-34 & the US & Doctoral degree & Yes & Yes & 5-10 years \\
P6 & Male & 35-44 & the US & Less than a high school diploma & No & No & Over 10 years \\
P7 & Male & 25-34 & the US & Doctoral degree & Yes & Yes & 3-5 years\\
P8 & Male & 45-54 & the US & Bachelor's degree & Yes & Yes & Over 10 years \\
P9 & Female & 25-34 & New Zealand & Master's degree & No  &  No &  Over 10 years\\
P10 & Male & 25-34 & the US & Master's degree & No & No & Over 10 years\\
P11 & Female & 25-34 & the UK & Bachelor's degree & No & No & Over 10 years\\
P12 & Male & 18-24 & the UK & Master's degree & Yes & Yes & 1-3 years\\
P13 & Male & 35-44 & the UK & Bachelor's degree & Yes & No & Over 10 years\\
P14 & Male & 25-34 & Sweden & High school graduate & No & No & Over 10 years \\
P15 & Female & 25-34 & Spain & Master's degree & Yes & Yes & Over 10 years \\
P16 & Female & 25-34 & Germany & Master's degree & Yes & Yes & Over 10 years \\
P17 & Female & 25-34 & Spain & Doctoral degree & No & No & Over 10 years \\
P18 & Female & 35-44 & the US & Bachelor's degree & No & No & Over 10 years \\
P19 & Female & 25-34 & Germany & Master's degree & Yes & Yes & 3-5 years \\
P20 & Male & 25-34 & Hungary & Master's degree & Yes & No & 3-5 years \\
P21 & Male & 35-44 & the US & Bachelor's degree & Yes & No & 5-10 years \\
P22 & Female & 18-24 & France & Master's degree & Yes & Yes & Less than 1 year\\
P23 & Male & 45-52 & the US & Master's degree & No & No & Over 10 years\\
\hline
\end{tabular}%
}
\label{table:demographics}
\end{table*}

 
\iffalse
\begin{figure}[!t]
	\centering
	\includegraphics[width=\linewidth]{Fig/world app.png}
	\caption{New account creation process in  World App}
    
\label{fig:New account creation process in  World App}
\end{figure}
\begin{figure}[!t]
	\centering
	\includegraphics[width=\linewidth]{Fig/bright id.png}
	\caption{New account creation process in  Bright ID}
    
\label{fig:New account creation process in  Bright ID}
\end{figure}
\fi
\iffalse
\begin{table*}[h!]
    \centering
    \begin{tabular}{|l|p{3cm}|p{3cm}|p{3.3cm}|p{3.3cm}|}
        \hline
        \textbf{Identification} & \textbf{Examples} & \textbf{Advantages} & \textbf{Disadvantages} \\
        \hline
        Behavioral filters & CAPTCHAs & It doesn't require sensitive information like IDs or financial details. & AI bots are now better than humans at decoding the CAPTCHAs. Genuine users solve unwittingly solve the challenge on behalf of spammers (man-in-the-middle attack). \\
        \hline
        Biometric identifiers & Face, Fingerprint, Iris & High usability as users don't need to prepare or remember anything. & If sensitive biometric data is compromised, it is challenging to reset or replace once leaked. \\
        \hline
        Humanness verification & Selfie checks with ID, live video calls & It's quick and easy compared to submitting IDs or sensitive information. & Sophisticated deepfakes can sometimes get past selfie verification. \\
        \hline
        Economic identifiers & KYC, official ID verification & It is difficult for malicious attackers to fake their identities. & Not inclusive, as they can exclude individuals without access to financial instruments like formal banking systems.\\
        \hline
        Digital identifiers & Phone numbers, email address & Low-cost, widely available. & It is easy to acquire another identifies, one person can have multiple accounts.\\
        \hline
        Digital artifact verification & Watermarking, fingerprinting, metadata provenance & It doesn't disrupt user experience. & Attackers continuously develop methods to bypass new detection mechanisms. \\
        \hline
    \end{tabular}
    \caption{Different Types of Identifiers for Proof of Personhood}
    \label{tab:existing_identifiers}
\end{table*}
\fi

\begin{table*}[h!]
    \centering
    \begin{tabular}{|p{1.2cm}|p{1.5cm}|p{1.5cm}|p{1.5cm}|p{1.5cm}|p{1.5cm}|p{1.5cm}|p{1.5cm}|p{1.5cm}|}
        \hline
        \textbf{Features} & \textbf{World App} & \textbf{Bright ID} & \textbf{Proof of humanity} & \textbf{Passport Gitcoin} & \textbf{Federated identities (OAuth)} & \textbf{Aadhar Card} &\textbf{Estonia's Digital ID} & \textbf{China's social credit system}\\
        \hline
       %  Type of platform & Blockchain based decentralized identity management system & Decentralized social identity verification platform & Social identity verification on the Ethereum blockchain & Digital identity solution for Web3 applications & Federated identity management system & Governmental website and in-person verification & Digital identity system & This system uses local government's websites for surveillance\\
       %  \hline
       % Free or paid & Free & Free & Free, but requires a submitter's deposit fee from the wallet which will be given back after successful verification of account & Free & Paid & Paid & Free & Free\\
       %  \hline
       Required data & User name, password, access to google drive, biometric data, identity wallet & Name, password, photo, information about connection & Crypto wallet's address, name, photo (should be facing the camera, facial features must be visible and no filter), video (the person must hold a sign with his wallet address) & Ethereum wallet like metamask, publicly available blockchain data & Email or Google/ Microsoft/ Github account or user name and password & Proof of identity document, proof of address document, proof of date of birth, iris and fingerprint scan & Personal detail, electronic signature, photo & Personal data, financial behavior, social behavior, and government interactions.\\
       \hline
        Stage where data is required & User name and password while creating a new account, identity wallet for verification request & While creating a new account and connecting with others & While registering for a new account & Registration process & While creating a new account & While registering an Aadhar card & While registering for a new digital ID & In different stages to evaluate and measure people's social score\\
        \hline
        % Centralized or decentralized & Decentralized & Decentralized &	Decentralized & Decentralized & Decentralized & Centralized & Decentralized & Decentralized\\
        % \hline 
        Advantage & It provides World ID, upon verifying it privately with an Orb, and people can use it to seamlessly sign in to websites, mobile apps and crypto dapps proving they are a unique and real person without sharing personal data like names or email addresses. & As a privacy-first social identity network it allows users to prove to applications that they are accessing them fairly with only one account & It provides soulbound IDs that are unique and linked to only one person, also it can be retrieved if lost & It doesn't collect any kind of personal information or IP address while interacting with the service & Supports a range of devices and platforms (server to server apps, mobile apps, tv etc.) & Helps streamline government services, welfare distribution, and authentication processes using a unique identifier & Enables secure access to e-government services and facilitates electronic voting and signing of documents & Motivates people and businesses to uphold the law and increase safety in China\\
        \hline
        Disadvantage & Not available in every country, requires an orb for the verification process. & Bright ID has no control over and assume no responsibility for the content, privacy policies, or practices of any third-party sites or services linked to it. This may violate users' security and privacy if users click the link without knowing the privacy policy of that third-party site or service & If there is any error during the registration process, it will be interpreted as malicious attack and there is no way to edit information once submitted & Ethereum wallet is a requirement & Users may be at risk if their federated login credentials are compromised & Requires re-verification as fingerprint changes over time & & Algorithm miscalculations or scoring mistakes could see people or businesses unfairly penalized\\
        \hline
        UI/UX issue & Initial setup can be complex for new users unfamiliar with cryptocurrency and decentralized apps & & Complexity in the vouching process and potential difficulty for users unfamiliar with Ethereum & The Web3-focused nature of the platform may confuse users unfamiliar with blockchain & Federated login systems can confuse users if they are unfamiliar with different identity providers & Users may face difficulties navigating the registration process, particularly in rural areas with limited access to technology & & The system is opaque, with unclear guidelines on how scores are calculated and potential difficulty for users in understanding and managing their score\\
        \hline
        Privacy related issue & While privacy-focused, the dependence on blockchain raises concerns about data traceability, especially if users aren't well-informed about privacy implications of blockchain & Although the data is not stored centrally, there are still privacy concerns regarding user interactions, connections, and the information that is publicly visible to others & As users' photo and video is required, it can be a privacy concern. Users can have trust issues regarding the PoH issuers & Users' information (wallet address) may be shared with third parties and service providers but not sold to them & Privacy concerns arise because data is shared across multiple services and potentially tracked by the identity provider & There is a potential risk of mass surveillance through Aadhaar data & Strong privacy concerns due to the surveillance nature of the system and the collection of personal behavior data.\\
        \hline
			
			
    \end{tabular}
    \caption{Result of Cognititve walkthrough}
    \label{tab:competitive_analysis}
\end{table*}

\fi


%\iffalse
\clearpage
\onecolumn
\scriptsize{
\begin{longtable}{p{3cm} p{4cm} p{4.5cm} p{4.5cm}}

    \caption{Results of Competitive Analysis.}\\
    \hline
    \textbf{Name of App} & \textbf{Data Workflow} & \textbf{UI/UX Issue} & \textbf{Security/Privacy Issue} \\ \hline
    \endfirsthead
    \hline
    \textbf{Name of App} & \textbf{Required data} & \textbf{UI/UX Issue} & \textbf{Security/Privacy Issue} \\ \hline
    \endhead
    \hline
    \endfoot
    \hline
    \endlastfoot
        \textbf{CAPTCHA}  
        & Solve CAPTCHA challenge \ding{221} JavaScript tracks user behavior (mouse, keystroke dynamics)  \ding{221} Submit response, which includes collected IP address, device, and browser information.
        & \ding{202} Screen readers don't help visually impaired users to solve graphical CAPTCHAs. \ding{203} It creates a barrier for users with other impairments (motor, cognitive). \ding{204} Sometimes CAPTCHA challenges are so difficult that they make users leave the website or application.
        & Users are often unaware of the data collected by CAPTCHA, which could potentially be used beyond verification.\\
        
        \textbf{reCAPTCHA}  
        & Click checkbox \ding{221} JavaScript tracks user behavior (mouse, keystroke dynamics, touchscreen interactions)  \ding{221} Submit response, which includes collected IP address, cookies, device, and browser information.
        & If the system regards users as bots, there is no way they can access it.
        & Users are often unaware of the data collected by reCAPTCHA, which could potentially be used beyond verification.\\

        \textbf{Idena}  
        & Create password \ding{221} Import private key backup to sign in \ding{221} Solve the flip challenge on the scheduled validation ceremony date.
        & The flip tests with time constraints make passing difficult, especially for users with cognitive or motor impairments. 
        & \ding{202} While flip challenges are not directly tied to personal information, they could potentially reveal patterns of the user. \ding{203} If a user's private key is compromised, an attacker can gain full control over the account.\\

        \textbf{World ID}  
        & Enter birth date (Android only) \ding{221} Enter phone number (OTP) \ding{221} Enter username and password \ding{221} Connect Google Drive or iCloud \ding{221} Scan iris in-person at the Orb location.
        & \ding{202} The onboarding screen is overly simplified, lacking sufficient explanations for users to understand the system's workflow and how their data is managed. \ding{203} The backup process requiring Google Drive access may cause confusion about unintended access to personal space.
        & Collecting sensitive biometric data, such as iris scans, requires high transparency and accountability about its handling and protection.\\

        \textbf{Humanode}  
        & Generate an asymmetric key pair \ding{221} Insert private key \ding{221}  Scan face to link biometric data with public key.
        & Users may struggle with complex operations like node setup and validation, leading to confusion and onboarding friction.
        & Collecting high-precision biometric data, such as 3d mapping face scans, requires high transparency and accountability about its handling and protection.\\

        \textbf{BrightID}  
        & Enter name, photo, and password \ding{221} Scan connections' QR code.
        & Users may find it confusing to navigate the verification process (Aura, Bitu verification).
        & Although the data is not stored centrally, there are still privacy concerns regarding connections and the information that is publicly visible to others.\\

        \textbf{Proof of Humanity}  
        & Connect wallet \ding{221} Submit name, front-facing photo, a video showing face, wallet address on a physical surface, and stating the required phrase.
        & Users may not be aware of the risk of losing their deposit if their submission fails.
        & \ding{202} Requiring users' photos and videos raises privacy concerns, leading to potential trust issues with PoH issuers. \ding{203} The wallet address used to submit a profile will be publicly linked to user's identity. The user's wallet holdings and transaction history will be publicly linked. \\

        \textbf{DECO}  
        & Extracts encrypted session data of the website the user accessed \ding{221} Generate zero-knowledge proof for specific information \ding{221} Verifier validates the proof without accessing the original source.
        & The advanced technical nature, relying on mechanisms such as zero-knowledge proofs and TLS, makes it difficult for users to understand how their information is processed and protected.
        & \ding{202} Improperly implemented ZKPs may unintentionally leak sensitive information. \ding{203} It allows proving a substring of the response, but this could be exploited through context manipulation.\\

        \textbf{CANDID}  
        & Hold a digital credential \ding{221} Select attributes for disclosure \ding{221} Generate a zero-knowledge proof \ding{221} Verifier validates the proof without requiring full credentials.
        & The advanced technical nature, relying on mechanisms such as zero-knowledge proofs and decentralized identity, makes it difficult for users to understand how their information is processed and protected.
        & Repeated use of certain identity attributes across different services could lead to deanonymization. \\

        \textbf{India Aadhaar Card}  
        & Apply Aadhaar with proof of identity, proof of address, proof of date of birth, and proof of relationship documents \ding{221} Scan fingerprints, iris, and face at an enrollment center.
        & Users must submit a combination of documents, which may not be accessible for those in rural areas with limited access to government services.
        & \ding{202} Data breaches have been reported, with millions of Indians' Aadhaar and passport details compromised and exposed on the dark web. \ding{203} Collecting multiple biometrics raises privacy concerns, as compromised biometric data is permanent and cannot be reset like passwords.\\

        \textbf{Estonia e-ID}  
        & Enter email address \ding{221} Apply e-ID with CV, credit/debit card information, photo of face, and passport or EU ID card.
        & The reliance on external card readers makes it less accessible for non tech-savvy users who have difficulty installing and configuring the card reader.
        & \ding{202} The e-ID system serves as a single point of access for multiple e-services, which could be problematic if compromised. \ding{203} In 2017, a security flaw was discovered in the chip encryption, potentially allowing identity theft and unauthorized access to private data.\\

        \textbf{Japan My Number Card}  
        & Provide one photo ID or two non-photo IDs with an issue notice at a government office.
        & \ding{202} Users must go through lengthy procedures to obtain and renew the card, often requiring in-person visits. \ding{203} It requires four passwords (three 4-digit PINs and one 6-16 character alphanumeric password), with resets only available at a government office.
        & \ding{202} My Number is printed on the back of the card, posing a privacy vulnerability when physically handed to others. \ding{203} The system's expansion to cover health insurance has raised concerns about an increased risk of unauthorized cross-referencing of personal information.\\

        \textbf{China’s Social Credit System}  
        & The system collects and integrates four categories data: basic information, information on administrative penalties and permits, any irregularities, and red list or blacklist information (if applicable).
        & \ding{202} Users often don't understand how their scores are calculated or what specific actions impact their ratings. \ding{203} With multiple government and private systems in place, China's social credit system lacks unified standards.
        & The systems gather vast amounts of personal information for economic reliability and social behavior, raising surveillance concerns. \\

        \textbf{ID.me}  
        & Enter email and password \ding{221} Chose MFA options (SMS etc.) \ding{221} Take a photo of government-issued ID.
        & Users without a government-issued ID cannot verify their identity, excluding those who lack access to such documents.
        & Sensitive personal information is centrally stored, making users' credentials highly vulnerable if compromised.\\

        \textbf{Civic Pass}
        & Connect wallet \ding{221} Verify via video selfie or CAPTCHA or government-issued ID.
        & Users with multiple wallets struggle with uniqueness verification because the system associates one face with one wallet.
        & On-chain attestation could expose metadata that links a user’s identity to blockchain transactions, potentially compromising privacy.\\
       
    %\fixme{Ayae, complete this table from here https://docs.google.com/spreadsheets/d/1h4PQuBKnP19HXMYTtejEAur-FxdvoA2G7_tuODVz2xc/edit?gid=452248634#gid=452248634}

\label{tab:litcomparison}
\end{longtable}
%\fi
}
