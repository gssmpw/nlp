\section{Related Works}
There exist two major security concerns for information: integrity and confidentiality. As data utilization expands beyond mere storage to computational applications, the concepts of integrity and privacy have evolved into Verifiable Computation and Homomorphic Encryption, respectively.
    % Verifiable Computation guarantees the integrity of delegated computation results. 
    Various cryptographic primitives have been proposed to achieve VC, such as SNARK~\cite{SoK}, Homomorphic Mac~\cite{HomMAc1, Hommac2}, and Homomorphic Signature~\cite{HomomorphicSignautre}. Meanwhile, since Gentry's breakthrough discovery~\cite{Gen09}, HE has seen continuous advancements in both functionality and efficiency. Notable schemes such as BGV, BFV, TFHE, CKKS~\cite{BGV,bfv,CGGI,CKKS} have emerged, significantly improving the practical applicability of HE. 
    
    As HE technology matures and its deployment becomes increasingly widespread, new challenges have surfaced. One major issue is that most research assumes a semi-honest server model, overlooking the necessity for robust integrity guarantees.
    Combining integrity into HE is not merely about ensuring the correctness of computational results: Due to the inherent malleability of FHE ciphertexts, in the absence of integrity measures, an active adversary can manipulate ciphertexts to undermine confidentiality~\cite{INDCPAD, INDCPAD1}, which is known as IND-CPA$^D$ attack. Therefore, ensuring the integrity of homomorphic computations is crucial not only as an extension of verifiable computing but also for achieving malicious security in FHE.

    However, adding verifiability to Homomorphic Encryption is not a trivial task.  
    Typical integrity mechanisms, such as SNARK, MACs, or digital signatures, conflict with FHE’s requirement for meaningful ciphertext malleability~\cite{SnarkVHE, Flexible, Efficiently, BoostingVCE, Rinocchio}. Theoretically, there exist works on maliciously secure FHE that propose new security notions for FHE capable of ensuring integrity~\cite{Beyond_IND-CCA1_Security, Targeted_Malleability}.  
    However, in practice, either no concrete instantiation satisfying these notions is known, or existing approaches are inefficient.
    Other approaches focus on verifying the integrity of homomorphic computations by leveraging confidentiality, but they tend to be constrained by weak adversary models. \cite{VE} assumes a covert adversary model, and~\cite{cryptoeprint:PRF} assumes an adversary capable of performing bootstrapping only once. There are also related works not discussed in this paper; for example,~\cite{DataSeal} aimed to ensure integrity in secure matrix multiplication by employing checksums.
    
%%%%%%%%%%%%%%%%%%%%%%%%%%%%%%%%%
%       Section 2
%%%%%%%%%%%%%%%%%%%%%%%%%%%%%%%%%