\section{Conclusion \& Limitation}
\iffalse
In this study, we aim to design an order-robust CIL model capable of addressing two critical challenges: class order sensitivity and intra-task conflicts. Building on existing theories, we find that as class similarity decreases, the model's sensitivity to class order also lessens, which effectively mitigates knowledge conflicts both across tasks and within individual tasks. 
To enhance the model's robustness across varying class orders, we propose a dynamic grouping method based on similarity graphs, termed GDDSG.
The proposed approach maintains the centroids of learned classes and group classes based on dynamic similarity. In GDDSG, we introduce a novel approach to structuring class groups within class-incremental learning. Our GDDSG can continually update existing groups or form new ones, training distinct models for each group. During inference, predictions are derived through an ensemble of outputs from multiple models, thereby enhancing overall accuracy and robustness in CIL.
\fi
This study addresses the critical challenge of class order sensitivity in Class Incremental Learning (CIL), where model performance significantly degrades under varying class arrival sequences. By introducing GDDSG, a graph-driven framework that dynamically partitions classes into similarity-constrained groups and coordinates isolated sub-models with joint prediction, we theoretically and empirically mitigate the impact of class sequence variations. Experiments validate that our method not only reduces sensitivity to class order but also achieves state-of-the-art accuracy and anti-forgetting performance. This work provides a benchmark for developing robust CIL methods with dynamic data streams.

Inevitably, our method has certain limitations. First, GDDSG currently relies on NCM classifiers. In future work, we aim to explore order-robust CIL approaches with Softmax strategies. 
Also, while the memory overhead remains small, it could be further streamlined for efficiency, and we intend to address this limitation with future studies.
