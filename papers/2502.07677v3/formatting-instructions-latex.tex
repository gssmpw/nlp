%File: formatting-instruction.tex
% \documentclass[letterpaper]{article}
% \documentclass[twocolumn]{article}
\documentclass[sigconf, review, anonymous]{acmart}
% \usepackage{aaai}

\setcopyright{acmlicensed}
\copyrightyear{2025}
\acmYear{2025}
\acmDOI{XXXXXXX.XXXXXXX}
\acmConference[The Web Conference 2025]{The Web Conference 2025}{May 11--16, 2025}{Sydney, Australia}
% \acmISBN{978-1-4503-XXXX-X/25/05}  % 更新 ISBN
\acmISBN{XXX-X-XXXX-XXXX-X/XX/XX}  % 更新 ISBN
\acmPrice{}

\usepackage{times}
\usepackage{helvet}
\usepackage{graphicx}
\usepackage{listings}
\usepackage{xcolor} 
\usepackage{courier}
\usepackage{comment}
\usepackage{natbib}

\usepackage{float}
\usepackage{dblfloatfix}



\frenchspacing
\setlength{\pdfpagewidth}{8.5in}
\setlength{\pdfpageheight}{11in}
\pdfinfo{
/Title (Insert Your Title Here)
/Author (Put All Your Authors Here, Separated by Commas)}
\setcounter{secnumdepth}{0}  
 \begin{document}
% The file aaai.sty is the style file for AAAI Press 
% proceedings, working notes, and technical reports.
%
\title{Auto-Drafting Police Reports from Noisy ASR Outputs: A Trust-Centered LLM Approach}
\author{AAAI Press\\
Association for the Advancement of Artificial Intelligence\\
2275 East Bayshore Road, Suite 160\\
Palo Alto, California 94303\\
}
\maketitle
\begin{abstract}
\begin{quote}
As discussions around the George Floyd case continue to highlight concerns, the complex issue of balancing trust in law enforcement while safeguarding the rights of both police officers and suspects becomes even more critical. Ensuring fairness and transparency during law enforcement interactions is paramount. In this study, we introduce a novel system designed to generate police report drafts from highly noisy, multi-role dialogue data. This system captures the essential elements of the interaction, providing a comprehensive summary that can support legal proceedings and enhance accountability in future cases.  
\end{quote}
\end{abstract}

\section{Introduction}
\noindent Over the past decade, U.S. law enforcement has faced substantial challenges, including instances where officers have overstepped their authority or abused suspects without reasonable suspicion or probable cause~\citep{fridell2021relationship, figueroa2012building}. Simultaneously, recent protests from police officers have shed light on their growing vulnerabilities—both from physical threats during law enforcement interactions and from social scrutiny by various organizations~\citep{asquith2021vulnerability, soltes2021occupational}. Additionally, officers spend up to 40\% of their time writing police reports, further exacerbating their workload and detracting from time spent on active policing~\citep{bennett1992national}. These pressures underscore the urgent need to protect the rights of both suspects and officers, and to standardize police case reporting to ensure accountability and due process. Advancements in AI, particularly large language models (LLMs), present a transformative opportunity to address these challenges by automating and standardizing the documentation of key elements in police reports, enhancing both transparency and fairness in the pursuit of justice~\citep{wu2023precedent}.

\begin{comment}
Over the past decade, U.S. law enforcement has faced substantial challenges, including instances where officers have overstepped their authority or abused suspects without reasonable suspicion or probable cause. Simultaneously, recent protests from police officers have shed light on their growing vulnerabilities—both from physical threats during law enforcement interactions and from social scrutiny by various organizations. These pressures underscore the urgent need to protect the rights of both suspects and officers, and to standardize police case reporting to ensure accountability and due process. Advancements in AI, particularly large language models (LLMs), present a transformative opportunity to address these challenges by automating and standardizing the documentation of key elements in police reports, enhancing both transparency and fairness in the pursuit of justice.
\end{comment}

However, building an LLM-enabled system for police report generation presents significant challenges, particularly when dealing with highly noisy automatic speech recognition (ASR) outputs. These transcripts often contain various forms of distortion and errors, making it difficult for standard LLMs to identify the critical elements needed for an accurate and reliable report. For example, suspects may intentionally introduce misleading or confusing discourse to complicate the conversation, further increasing the noise. Traditional LLMs often struggle to filter out such interference, raising concerns about the trustworthiness and usefulness of the generated reports  ~\citep{wu2024knowledge}. To overcome these hurdles, more sophisticated approaches are required to ensure that the system can effectively denoise inputs, extract key information, and maintain a high standard of accuracy and integrity in the report drafts.  

In this study, we introduce a novel system designed to automatically generate high-quality and trustworthy police report drafts by leveraging ASR devices worn on police officers' bodies. This system integrates an advanced LLM that utilizes cutting-edge knowledge diffusion models to produce detailed, reliable report drafts. Unlike traditional fine-tuned LLMs, our model is capable of learning intermediate tasks critical for accurate report generation, such as dialogue denoising and extracting key event elements. Crucially, to safeguard suspect privacy, the proposed system requires minimal training data. Instead of relying on large, potentially invasive datasets, we utilize open case law databases \cite{chang2020mining} for data generation during training. The system’s core functionality includes:

\begin{itemize}
\item Engineered LLM with Knowledge Diffusion Model: Designed to generate comprehensive report drafts by effectively processing noisy, multi-role dialogue data.
\item ASR Device and LLM Interaction: Seamless communication between the ASR devices and the LLM to enable real-time draft generation.
\item Human Interaction for Report Refinement: Facilitates further collaboration with police officers to enhance and finalize report drafts, ensuring accuracy and completeness.
\end{itemize}


\section{Police Report Draft Generation System}

In this paper, we present a system capable of generating high-quality report drafts from noisy dialogue data extracted from police body camera ASR devices. The system captures essential details and events from dialogues involving multiple agents, such as several police officers and suspects. Due to the complexity of these interactions and limitation of the ASR device, the data may contain various errors, such as speaker attribution issues and ASR inaccuracies. 
\begin{figure*}
    \centering
    \includegraphics[width=\textwidth]{Without_axon_grey_mark.png} 
    \caption{System Screenshots}
    \label{fig:screen-image}
\end{figure*}

Our system is designed to understand and process these noisy dialogues, producing coherent and accurate report drafts. 

Once generated, the draft is delivered to the police officer for review, with critical sections highlighted for their attention. The officer can then approve or amend the report by adding any missing or important details before signing and submitting the final version. This system significantly improves the efficiency of police report generation while enhancing the overall accuracy and trustworthiness of the final report.  Some screenshots of the system are depicted in Figure~\ref{fig:screen-image}.



% \begin{figure*}[h]
%     \centering
%     \includegraphics[width=\textwidth]{toy_example_version_7.pdf} 
%     \caption{Automated Draft Generation from Body Camera ASR(Noisy Dialog) : A Three-Stage System for Denoising, Event Detection, and Report Summarization}
%     \label{fig:pdf-image}
% \end{figure*}

% \begin{figure*}[h]
%     \centering
%     \includegraphics[width=0.7\textwidth]{system_architecture.png} 
%     \caption{System Architecture}
%     \label{fig:system-image}
% \end{figure*}

% We need system screenshots(Figure~\ref{fig:screen-image}):





\section{Draft Generation with Knowledge Diffusion}
% Briefly introduce our model here:

\begin{itemize}
\item \textbf{The diffusion model framework} We propose a diffusion model framework that employs a approach to generate realistic and context-specific dialogues for law enforcement scenarios, leveraging an extensive dataset of U.S. case law. The process begins with the generation of training data by extracting and sampling relevant cases from a collection of criminal law data. A LLM is then used to simulate dialogues that reflect real-world law enforcement interactions, incorporating various types of noise to create both noisy and clean versions of the dataset. The framework subsequently applies denoising techniques to filter out irrelevant information, followed by extracting key events to ensure the generated report draft are comprehensive and trustworthy.

\item \textbf{Training data generation with Case Law Database} We collected 20.6 GB of U.S. case law data using web crawlers, from which we filtered 15 GB of 923127 criminal case law data and sampled 10k criminal cases, including 3669 police-related cases. Using a LLM, we extracted and sampled 1784 event data from these cases and employed in-context learning to generate 500 dialogues simulating law enforcement scenarios. During this process, we introduced 3 types of noise to create both noisy and clean versions of the dialogue dataset.
\item \textbf{Denoise model and event detection model} We fine-tuned a LLaMA 3.1 8b model on 500 pairs of generated noisy and clean dialogues to serve as our denoising model. 

% For dialogue generation, we used a specific prompt: 
% \begin{lstlisting}[language=, frame=single, breaklines=true, backgroundcolor=\color{gray!10}, basicstyle=\ttfamily]
% I will provide you with a JSON data of an event extraction from caselaw. Please imagine the scene at the time of the incident and simulate a conversation between the parties involved. Return only the conversation to me without any narration. The length should be around {} sentences, in English. Simulate only the scene at the time of the incident, not a courtroom dialogue.\end{lstlisting} 

Alongside the denoising model, we trained an event extraction model using 500 event extraction data points derived from the same case law database. The fine-tuning process was conducted using structured prompts that guide the model to focus on critical elements in an event. 

% The prompt used for event extraction was: 
% \begin{lstlisting}[language=, frame=single, breaklines=true, backgroundcolor=\color{gray!10}, basicstyle=\ttfamily]
% Please extract the most important event from the provided caselaw data and return it in JSON format. Follow these steps carefully: event_name: The name of the event. trigger_word: The phrase that triggered the event. type_of_charge: The possible charges involved in the case, e.g., Assault, Battery, Domestic Violence, Robbery, Burglary, DUI, Riot, Drug Possession. time: The time when the event occurred. location: The place where the event happened. individuals_involved: Names and roles of individuals involved in the case. item: Items involved in the event. motive: The reason behind the event. consequences: The outcome or result of the event. actions: A series of actions involved, including the person who triggered each action. Only simulate the scene of the incident, not courtroom dialogues.

\end{lstlisting}
\end{itemize}




\section{Conclusion}
In this work, we propose a groundbreaking system for generating trustworthy police case report drafts, which holds significant potential for enhancing the integrity and transparency of law enforcement processes. Currently being piloted in XXX police stations, the initial feedback has been promising. Police officers are benefiting from this AI-driven innovation, as it streamlines and standardizes the report generation process, while also safeguarding the rights of suspects. This system represents a vital step forward in ensuring accountability and trust in law enforcement through more efficient and reliable documentation.

\bigskip
\noindent Thank you for reading these instructions carefully. We look forward to receiving your electronic files!

\bibliographystyle{ACM-Reference-Format}
\bibliography{reference}

% \bibliographystyle{aaai} 
% \bibliography{reference}

\end{document}