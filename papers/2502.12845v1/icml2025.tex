%%%%%%%% ICML 2025 EXAMPLE LATEX SUBMISSION FILE %%%%%%%%%%%%%%%%%

\documentclass{article}

% Recommended, but optional, packages for figures and better typesetting:
\usepackage{microtype}
\usepackage{graphicx}
\usepackage{subfigure}
\usepackage{amsmath}
\usepackage{multirow}
\usepackage{booktabs} % for professional tables
\usepackage{array}
\usepackage[table]{xcolor}

% hyperref makes hyperlinks in the resulting PDF.
% If your build breaks (sometimes temporarily if a hyperlink spans a page)
% please comment out the following usepackage line and replace
% \usepackage{icml2025} with \usepackage[nohyperref]{icml2025} above.
\usepackage{hyperref}


% Attempt to make hyperref and algorithmic work together better:
\newcommand{\theHalgorithm}{\arabic{algorithm}}

% Use the following line for the initial blind version submitted for review:
% \usepackage{icml2025}

% If accepted, instead use the following line for the camera-ready submission:
%\usepackage[accepted]{icml2025}
\usepackage[accepted]{icml2025}

% For theorems and such
\usepackage{amsmath}
\usepackage{amssymb}
\usepackage{mathtools}
\usepackage{amsthm}

% if you use cleveref..
\usepackage[capitalize,noabbrev]{cleveref}

%%%%%%%%%%%%%%%%%%%%%%%%%%%%%%%%
% THEOREMS
%%%%%%%%%%%%%%%%%%%%%%%%%%%%%%%%
\theoremstyle{plain}
\newtheorem{theorem}{Theorem}[section]
\newtheorem{proposition}[theorem]{Proposition}
\newtheorem{lemma}[theorem]{Lemma}
\newtheorem{corollary}[theorem]{Corollary}
\theoremstyle{definition}
\newtheorem{definition}[theorem]{Definition}
\newtheorem{assumption}[theorem]{Assumption}
\theoremstyle{remark}
\newtheorem{remark}[theorem]{Remark}

% Todonotes is useful during development; simply uncomment the next line
%    and comment out the line below the next line to turn off comments
%\usepackage[disable,textsize=tiny]{todonotes}
\usepackage[textsize=tiny]{todonotes}


% The \icmltitle you define below is probably too long as a header.
% Therefore, a short form for the running title is supplied here:
\icmltitlerunning{MOLLM: Multi-Objective Large Language Model for Molecular Design -- Optimizing with Experts}

\begin{document}

\twocolumn[
\icmltitle{MOLLM: Multi-Objective Large Language Model for Molecular Design -- Optimizing with Experts}

% It is OKAY to include author information, even for blind
% submissions: the style file will automatically remove it for you
% unless you've provided the [accepted] option to the icml2025
% package.

% List of affiliations: The first argument should be a (short)
% identifier you will use later to specify author affiliations
% Academic affiliations should list Department, University, City, Region, Country
% Industry affiliations should list Company, City, Region, Country

% You can specify symbols, otherwise they are numbered in order.
% Ideally, you should not use this facility. Affiliations will be numbered
% in order of appearance and this is the preferred way.
\icmlsetsymbol{equal}{*}

\begin{icmlauthorlist}
\icmlauthor{Nian Ran}{yyy}
\icmlauthor{Yue Wang}{comp}
\icmlauthor{Richard Allmendinger}{yyy}
\end{icmlauthorlist}

\icmlaffiliation{yyy}{University of Manchester, Manchester, United Kingdom}
\icmlaffiliation{comp}{Independent Researcher, Beijing, China}

\icmlcorrespondingauthor{Nian Ran}{r992988188@gmail.com}

% You may provide any keywords that you
% find helpful for describing your paper; these are used to populate
% the "keywords" metadata in the PDF but will not be shown in the document
\icmlkeywords{Machine Learning, ICML}

\vskip 0.3in
]

% this must go after the closing bracket ] following \twocolumn[ ...

% This command actually creates the footnote in the first column
% listing the affiliations and the copyright notice.
% The command takes one argument, which is text to display at the start of the footnote.
% The \icmlEqualContribution command is standard text for equal contribution.
% Remove it (just {}) if you do not need this facility.

\printAffiliationsAndNotice{}  %如果被接收要去icml2025.sty uncomment line 549 leave blank if no need to mention equal contribution
%\printAffiliationsAndNotice{\icmlEqualContribution} % otherwise use the standard text.

\begin{abstract}
Molecular design plays a critical role in advancing fields such as drug discovery, materials science, and chemical engineering. This work introduces the Multi-Objective Large Language Model for Molecular Design (MOLLM), a novel framework that combines domain-specific knowledge with the adaptability of Large Language Models to optimize molecular properties across multiple objectives. Leveraging in-context learning and multi-objective optimization, MOLLM achieves superior efficiency, innovation, and performance, significantly surpassing state-of-the-art (SOTA) methods. Recognizing the substantial impact of initial populations on evolutionary algorithms, we categorize them into three types: best initial, worst initial, and random initial, to ensure the initial molecules are the same for each method across experiments. Our results demonstrate that MOLLM consistently outperforms SOTA models in all of our experiments. We also provide extensive ablation studies to evaluate the superiority of our components.
\end{abstract}

\section{Introduction}
Molecular design is fundamental in fields such as drug discovery, materials science, and chemical engineering. In these areas, the ability to design novel molecules with targeted properties, including stability, reactivity, or bioactivity, can drive significant advancements, from the development of new pharmaceuticals to the creation of sustainable, innovative materials. Traditionally, molecular design has relied on trial-and-error experimentation and repeated synthesis, which is resource intensive, time-consuming, and ultimately inefficient. During the past few decades, with rapid advances in computational power, various machine learning techniques~\cite{2019deepmolgensurvey,du2022molgensurvey} have been introduced not only to accelerate this process but also to enable the discovery of novel, more potent molecules. Methods include Bayesian Optimization (BO)~\cite{tripp2021gb-bo}, Multi-Objective Optimization (MOO)~\cite{mlps,choi2023rebadd,verhellen2022graph}, Markov Chain Monte Carlo (MCMC)~\cite{mars,sun2022molsearch}, Genetic Algorithms (GA)~\cite{gbga,gadt,lmea,brahmachary2024llmleo,molleo,assmoea}, Reinforcement Learning (RL)~\cite{olivecrona2017reinvent,rationaleRL,fu2022rga}, and Deep Learning (DL) models~\cite{VJTNNGan,2021molgpt,mood,2024molgen}.

Although these methods have yielded excellent results, most of them lack the integration of expert knowledge during runtime, despite the crucial role of professional feedback and search direction in molecular design. Large Language Models (LLMs), typically based on transformer architectures~\cite{vaswani2017transformer}, are pre-trained on extensive high-quality data, including books and academic papers, enabling them to capture domain-specific expertise. They have demonstrated significant potential in scientific discovery, particularly in molecular understanding and the generation of novel molecular candidates, as exemplified by models like GPT-4~\cite{ai4science2023impact}. Recent studies highlight the advantages of in-context learning~\cite{2024lico} and iterative evolutionary approaches~\cite{molleo} in enhancing LLM effectiveness. However, research in this area remains nascent, with only preliminary findings and a lack of systematic investigation.

Furthermore, despite significant progress in training large neural networks to understand chemistry and molecular structures with domain knowledge, these models often require additional parameters and retraining, particularly for MOO, as seen in MolGPT~\cite{2021molgpt} and LICO~\cite{2024lico}. In contrast, MOLLEO~\cite{molleo} leverages domain knowledge from pre-trained large language models without additional training but still relies on GB-GA within its framework.

In practice, most molecular design tasks optimizes multiple objectives, yet existing methods often ignore this aspect. For example, GB-BO~\cite{tripp2021gb-bo}, JTVAE~\cite{jin2018jtvae}, and MolGen~\cite{2024molgen} provide limited multi-objective capabilities. MolGPT requires specific training for different objectives, restricting its flexibility, while MolGen focuses primarily on target molecular discovery and employs only single-objective optimization.

Finally, the formulation of MOO using GA has often lacked rigor in previous studies. First, oracle calls should be restricted to ensure fair comparisons and practical applicability, since the evaluation of certain molecular properties requires costly experiments or specifically trained models, as noted in the Practical Molecular Optimization benchmark~\cite{pmo}. Additionally, the initial population significantly impacts final performance under a fixed number of oracle calls, yet this factor has been largely overlooked in methods that incorporate genetic algorithms, such as MARS, MOLLEO, and GB-GA.

To address these gaps and enhance multi-objective molecular design, we propose Multi-Objective Large Language Model (MOLLM), a LLM-based framework that integrates MOO, in-context learning and prompt engineering. Our model is mainly consisted of a mating module to generate parent molecules for in-context learning, a prompt template to integrates all information and instructions to maximally leverage the knowledge in LLM, a experience pool, and a selection module that contains both Pareto front selection and fitness value selection. The results show that our model demonstrates SOTA performance on different objectives, especially in multi-objective cases and when the number of objectives become larger. Our key contributions are:
\begin{itemize} 
\item We carefully design the in-context learning and prompt engineering mechanism in our model to fully leverage the domain knowledge pre-trained in LLMs. This is seamlessly integrated into MOO framework, achieving SOTA performance in both optimization quality and efficiency. Our framework requires no additional training for specific objectives while capitalizing on domain expertise, reasoning capabilities, and is adaptable to various LLMs. Unlike MOLLEO, we employ LLMs for all mating operations, ensuring that the framework is entirely LLM-driven.
\item Recognizing the critical influence of initial population selection in genetic algorithm-based methods, we evaluate models using three types of initial populations: the worst, random, and best molecules from the ZINC250K dataset. Our results show that MOLLM outperforms all kinds of SOTA models built on GA, BO, MCMC, LLM, RL and DL in our experiments, particularly in maximizing the sum of absolute property values in multi-objective settings. In addition, extensive ablation studies validate the effectiveness of our approach and design choices.
\end{itemize}

\section{Related Work}
\subsection{Molecular Design with Machine Learning}
Numerous advanced models for molecular design span GA, BO, MOO, MCMC, RL, and DL methodologies. \textbf{Deep Learning (DL)} leverages neural networks in various molecular design models. Differentiable Scaffolding Tree (DST)~\cite{fu2021dst} with GNNs, Junction Tree Variational Autoencoders (JTVAE)~\cite{jin2018jtvae}, and VJTNN+GAN~\cite{VJTNNGan} combine generative and adversarial architectures to generate molecules. MOOD~\cite{mood} utilizes Diffusion models to address out-of-distribution generation. Recent developments in Generative Pre-trained Transformers (GPT) led Bagal et al. to train MolGPT~\cite{2021molgpt} on next-token prediction, while Fang et al. pre-trained MOLGEN~\cite{2024molgen} on molecule reconstruction tasks, achieving cross-domain applicability. Although DL methods offer powerful capabilities in capturing complex molecular structures and enabling cross-domain applicability such as DST, JTVAE, and MolGPT, they often underperform in MOO scenario. Latent Space Optimization (LSO)~\cite{abeer2024lso} has further advanced multi-objective molecular design, but only for deep generative models. 

\textbf{Reinforcement Learning (RL)} combined with DL iteratively refines molecules by learning from feedback, often based on property scores. REINVENT~\cite{olivecrona2017reinvent} applies RL to train an RNN to generate molecules meeting multiple goals, while RationaleRL~\cite{rationaleRL} uses a Graph Neural Network (GNN) to generate molecules by building interpretable substructures, or ``rationales''. Based on REINVENT, Shin et al. proposed a novel divide-and-conquer approach called DyMol~\cite{dymol} to train the model for multiple objectives and achieve SOTA results. Kim et al. also achieve SOTA performance by integrating genetic algorithms into GFlowNets~\cite{genetic-gfn}.

In addition to DP methods, classical probabilistic models and optimization methods also achieve SOTA performance in many cases, such as in PMO~\cite{pmo}. A notable example of \textbf{Genetic Algorithms (GA)} is GB-GA~\cite{gbga}, commonly used as a baseline, where molecular structures are modified in graph form during mating operation. AkshatKumar et al.~\cite{gadt} introduced a neural network discriminator to enhance diversity, surpassing GB-GA in maximizing penalized-logP~\cite{gomez2018plogp}. Later, Tripp et al.~\cite{tripp2021gb-bo} employed a Tanimoto kernel in a Gaussian Process in GB-GA, outperforming GB-GA. It uses SELFIES~\cite{krenn2020selfies}, a 100\% valid molecular representation system; however, Gao et al.~\cite{pmo} later showed there are no obvious shortcomings of SMILES compared to SELFIES. MLPS~\cite{mlps} combines \textbf{MOO} with BO and an encoder-decoder network to efficiently locate global Pareto-optimal solutions, while Verhellen et al. introduced a graph-based MOO~\cite{verhellen2022graph} for molecular optimization. Furthermore, MARS~\cite{mars} uses \textbf{Markov Chain Monte Carlo (MCMC)} to explore chemical spaces probabilistically to identify molecules with desirable properties. Similarly, MolSearch~\cite{sun2022molsearch} utilizes Monte Carlo tree search for multi-objective molecular discovery. However, GA, BO, MOO, and MCMC methods are independent of domain knowledge, which is highly beneficial in molecular design but challenging to incorporate into such algorithms.


\subsection{Multi-Objective Optimization and Genetic Algorithm with LLM}
Recently, Large Language Models (LLMs) have demonstrated remarkable performance across various Natural Language Processing (NLP) benchmarks~\cite{brown2020gpt,ai4science2023impact}, sparking interest in their application as optimization operators in MOO to address the challenges of high-dimensional search spaces and to incorporate domain knowledge~\cite{2024EALLMsurvey}. For instance, OPRO~\cite{opro} and LMEA~\cite{lmea} employ LLMs as crossover and mutation operators within GA, using prompts that include parent values from the current population, with LMEA further exploring the balance of exploitation and exploration by adjusting the temperature parameter. Furthermore, Wang et al.~\cite{2024ConstrainedMOLLM} investigated constrained MOO with prompt engineering, demonstrating promising alignment results. Other studies have highlighted the effectiveness and efficiency of LLMs in GA compared to standalone LLMs and traditional MOO algorithms, proposing well-structured pipelines~\cite{liu2023algorithm,liu2024evolution,liu2024large,huang2024exploring,brahmachary2024llmleo}. However, research on LLMs with MOO is still nascent, with results largely limited to preliminary findings in numerical optimizations and planning problems.

\subsection{Molecular Design with LLM}
LLMs with pre-trained domain knowledge are increasingly popular for accelerating drug discovery and materials design~\cite{ai4science2023impact}. In particular, ChemCrow~\cite{2024chemcrow} uses LLMs as agents capable of reasoning, planning, and selecting appropriate external tools to iteratively refine candidates in chemical tasks. LICO~\cite{2024lico} improves molecule generation through in-context learning by pretraining the model with separate embedding and prediction layers, while Moayedpour et al.~\cite{2024many} extend this approach to multi-objective setups, and MolReGPT~\cite{li2024molReGPT} advances few-shot learning for molecular optimization. MOLLEO~\cite{molleo} applies GA combined with LLMs for molecular design, aligning with our framework, but differing significantly in details. MOLLEO's results as well as research in prompts remain preliminary, lacking extensive multi-objective experiments and failing to consider the impact of varying initial populations.

\section{Methodology}
The core ideas of MOLLM is that molecular design should leverage prior domain knowledge embedded in SOTA LLMs rather than training models from scratch, disregarding expert feedback during optimization, or relying on external algorithms such as GB-GA as operators. Therefore, we propose utilizing LLMs exclusively for both crossover and mutation operations in our model. The reason of two operations is to balance exploitation and exploration. While LLMs may not always generate molecules that perfectly consider the trade-offs of objectives, we incorporate Pareto front selection within (MOO) to ensure that molecules selected for the next generation better account for all objectives while maintaining structural diversity. Empirical experiments demonstrate that well-formulated prompts and in-context learning significantly enhance the utilization of LLM knowledge and the information encoded in parent molecules. Thus, we carefully design a prompt template comprising five key components, making it adaptable to any LLM.

\subsection{MOLLM Overview}
\begin{figure*}[h]
\centering
\includegraphics[width=1\textwidth]{images/mollm_overview.png}
\caption{The overall pipeline of initial MOLLM.}
\label{fig:mollm}
\end{figure*}
Figure~\ref{fig:mollm} presents the complete MOLLM optimization pipeline. The task involves unconstrained molecular optimization, where, given a set of objectives, the model is initialized with molecules selected from the ZINC250K dataset. It then iterates through mating, prompt generation, scoring, experience updating, and next-generation selection.
\textbf{Initialization:} For an optimization problem with one or multiple objectives, we initialize with $N_i$ molecules, either randomly selected or chosen based on the best or worst objective values from the ZINC250K dataset. In our setting, $N_i=100$. The ZINC dataset~\cite{irwin2012zinc} is selected as it is widely used for population initialization in molecular optimization studies~\cite{molleo, gbga, fu2022rga}. ZINC250K comprises approximately 250,000 curated drug-like molecules from the ZINC database, providing key properties such as chemical structures, logP, QED, and SA, making it well-suited for drug discovery and molecular optimization tasks.

\textbf{Mating}: This step involves prompting the LLM to generate new candidate molecules that are expected to improve on the parent molecules given. First, the parent molecules are randomly selected from the current population with probabilities $P_c$ and $P_m$ for crossover and mutation, respectively. Each crossover involves two parents, while mutation involves a single parent. These selected parents are then formatted into a flexible prompt template. Our \textbf{prompt template} consists of five key components: multi-objective requirements, objective descriptions, parent objective values (for in-context learning), output instructions, and past experience, if applicable. The whole framework of our model and an example is shown in Figure~\ref{fig:mollm}. In this setup, the model receives structured input specifying primary objectives, descriptions of molecular modifications (in SMILES format) that may increase or decrease property values, and parent molecules represented by their SMILES structures, objective values, and an aggregated objective score as an overall performance indicator. The output instructions specify that only molecular structures should be generated, omitting explanations to significantly reduce runtime and query costs without affecting performance. We employ crossover and mutation to balance exploitation and exploration. Although LLMs perform well in crossover due to their straightforward nature, they struggle with mutation, as its prompt is highly similar to crossover. To address this, we provide a list of common molecular mutation operations in the instructions to improve exploration. Finally, after generating offspring, we identify the best and worst molecules among them and query the LLM to update its experience based on these molecules and experience in the last iteration. This iterative refinement allows the experience to evolve dynamically, transitioning from general suggestions to more detailed and actionable guidance over time.


\textbf{Multi-objective optimization:} At this stage, we typically have $N$ parent molecules from the previous generation and $N$ offspring from the current generation, assuming all molecules are valid, where 
$N$ denotes the population size. These molecules are then combined and subjected to either Pareto front selection or F-value selection, where F represents the sum of normalized objective values, to determine the top $N$ candidates for the next generation. The selection operation is executed with equal probability (50\% each) for both methods. This hybrid selection strategy balances exploration and exploitation: F-value selection allows the model to focus on the current optimal solutions, while Pareto front selection promotes diversity in the next generation, reducing the risk of premature convergence to a local optimum.

\begin{algorithm}[tb]
\caption{MOLLM framework}
\label{alg:mollm}
\begin{algorithmic}
    \STATE {\bfseries Input:} initial population $\mathbb{M}_0$, population size $N$, fitness function $F$, probability of adding experience $P_{exp}$, probability of crossover $P_{c}$ and probability of mutation $P_{m}$. 
    \STATE {\bfseries Initialize:} $t \gets 0$.
    \FOR{$m\in \mathbb{M}_0$}
    \STATE Compute $F(m)$
    \ENDFOR
    \WHILE{$t<=$ oracle\_budget}
        \STATE parent\_pairs $\leftarrow$  Random\_Sample($\mathbb{M}_0$,$P_{c}$,$P_{m}$)
        \STATE prompts $\leftarrow$ Prompt\_Module(parent\_pairs)
        \IF{a random probability $p$ is less than $P_{exp}$}
            \STATE prompts $\leftarrow$ prompt + experience
        \ENDIF
        \STATE offspring $\leftarrow$ \textbf{Parallel\_Query}(prompts)
        \FOR{$m\in $ offspring}
        \STATE Compute $F(m)$
        \ENDFOR
        \IF{a random probability $p$ is less than $P_{exp}$}
            \STATE Update\_Experience\_Pool()
        \ENDIF
        \STATE $\mathbb{M}_{t}\leftarrow\mathbb{M}_{t-1} \cup $ offspring
        \IF{single\_objective or a random probability $p$ is less than 0.5}
            \STATE $\mathbb{M}_{t} \leftarrow $ F\_Value\_Selection($\mathbb{M}_{t}$,$N$)
        \ELSE
            \STATE $\mathbb{M}_{t} \leftarrow $ Pareto\_Frontier\_Selection($\mathbb{M}_{t}$,$N$)
        \ENDIF
    \ENDWHILE
    \STATE {\bfseries return} $\mathbb{M}_{t}$
\end{algorithmic}
\end{algorithm}

\section{Experiment}

\subsection{Task}
The initial population plays a critical role in determining the final outcomes of genetic-based algorithms under a fixed computational budget. However, most prior studies have overlooked its significance. In practical applications, researchers often initialize the search with the best available molecules rather than selecting them entirely at random. To ensure a comprehensive and fair evaluation of our model, we conduct experiments on three distinct initialization scenarios: the top 100, bottom 100, and randomly sampled molecules from the ZINC 250K dataset, using their F-values as indicators. The best-initialization scenario assesses the model’s upper performance limit, the random initialization reflects common real-world usage, and the worst-initialization scenario presents a more challenging optimization task. We adhere to the PMO benchmark and operate within a budget of 5,000 oracle calls.
For molecular property optimization, we focus on the following objectives: QED (drug-likeness), SA (synthetic accessibility), LogP (octanol-water partition coefficient), DRD2 (dopamine receptor D2 affinity), LogS (log of solubility), reduction potential, JNK3 (c-Jun N-terminal Kinase 3), and GSK3$\beta$ (Glycogen Synthase Kinase 3 Beta). In addition to these well-defined objectives, we also include BBBP (Blood-Brain Barrier Permeability), a more complex and less predictable property influenced by multiple biological factors.

\subsection{Metrics}
To fully evaluate the performance in many aspects, we use several metrics. The most important goal is maximizing the sum of normalized property values, denoted as F value, representing the absolute improvement that accountsnts for all the objectives. On top of that, we use uniqueness, validity, diversity and efficiency to full evaluate the ability of model to propose molecules. However, these additional metrics need to be considered in conjunction with the F-value, as it is less meaningful of other metrics if they have relatively low F values. 
\begin{itemize}
    \item \textbf{Top 1 F \& Mean Top 10 F}: F (fitness) is the sum of the normalized objective values, which gives the direct representation of the strength of a molecule~\cite{molleo}. The weight in our experiment to each objective is the same.
    \begin{equation}
        \max_{m \in M}F(m) = \sum_{i=1}^k w_if_i(m)
    \end{equation}
    where $m$ is a molecule in SMILES form, k is the number of objectives, $w_i$ and $f_i$ is the weight and normalized objective value. If an objective is to be minimized, it will be transformed by $1-f_i(m)$. We give an equal weight to each objective.
    \item \textbf{Uniqueness}: the fraction of valid generated molecules that are unique. A low uniqueness highlights repetitive molecule generation and a low level of distribution learning by the model~\cite{2021molgpt}, while a high uniqueness value means that the model effectively explores novel molecules, the equation is blow: 
    \begin{equation}
    U = 1 - \frac{\mathbb{M}_{\text{rep}}}{\mathbb{M}_{\text{all}}}
    \label{eq:uniqueness}
    \end{equation}
    where \( \mathbb{M}_{\text{rep}} \) is the number of repeated molecules, and \( \mathbb{M}_{\text{all}} \) is the total number of molecules proposed in history.
    \item \textbf{Validity}: the fraction of molecules generated that are valid, it measures how well the model has learned the SMILES grammar and the valency of atoms~\cite{2021molgpt}. The equation of validity is below:
    \begin{equation}
    V = \frac{\mathbb{M}_{\text{val}}}{\mathbb{M}_{\text{all}}}
    \label{eq:uniqueness}
    \end{equation}
    where \( \mathbb{M}_{\text{rep}} \) is the number of valid molecules.
    \item \textbf{Structural Diversity}: Structural diversity reflects the chemical diversity of the Pareto set and is computed by taking the average pairwise Tanimoto distance between Morgan fingerprints of molecules in the set~\cite{benhenda2017diversity}. The equation of computing a set of molecules is:
    \begin{equation}
        D(A) = \frac{1}{|A|^2} \sum_{(x,y)\in A\times A} T_d(x,y)
    \end{equation}
    where $A$ is the set of molecules and $T_d$ is the tonimoto distance. 
    \item \textbf{Efficiency}: Efficiency is compared by the running time in hours, as well as LLM calls if application. It is a important metric when using LLM for inference, because querying LLM incurs high computational costs. 
\end{itemize}

\subsection{Baselines}
To demonstrate the superiority and for fair comparison extensively, we choose SOTA models from a series of algorithms including GA, BO, MCMC, RL, DL and LLM-based method as our baselines. These algorithms are GB-GA, GB-BO, JT-VAE, MARS, REINVENT, MOLLEO, and recently proposed DyMol and Genetic-GFN which have achieved SOTA performance. More details and hyperparemeters of each baseline are provided in Appendix~\ref{apd:baseline}. For a fair comparison, we use Chatgpt 4o for both MOLLM and MOLLEO. We use the default hyperparameters for GB-GA, JT-VAE, GB-BO, MARS, REINVENT defined in PMO benchmark \cite{pmo}. In terms of MOLLEO, DyMol and Genetic-GFN, we also use the default hyperparameters defined in their codes and papers. For fair comparison, the normalized objectives are applied for all methods, which also includes the correct optimizing direction. 


\subsection{Main Experiment Results}
\label{sec:results}

\setlength{\tabcolsep}{1pt}
\begin{table*}[h]
\centering
\small
\begin{sc}
\begin{tabular}{>{\centering\arraybackslash}m{2.5cm} >{\centering\arraybackslash}m{1.2cm} >{\centering\arraybackslash}m{1.2cm} >{\centering\arraybackslash}m{1.2cm} >{\centering\arraybackslash}m{1.2cm} >{\centering\arraybackslash}m{1.7cm} >{\centering\arraybackslash}m{1.5cm} >{\centering\arraybackslash}m{1.5cm} >{\centering\arraybackslash}m{2.2cm} >{\centering\arraybackslash}m{2.2cm}}
\toprule
Metric & GB-GA & JT-VAE & GB-BO & MARS & REINVENT & MOLLEO & DyMol & Genetic-GFN & MOLLM(ours) \\   \midrule
\multicolumn{10}{c}{\textbf{(Worst initial)}} \\
Top1 F     & 4.048 & 3.817 & 3.665 & 3.907 & - & 4.096 & - & - & \cellcolor{lime}\textbf{4.187}\\
Top10 F    & 4.019 & 3.782 & 3.637 & 3.853 & - & 4.044 & - & - & \cellcolor{lime}\textbf{4.152}\\
Uniqueness & 0.786 & 1.000 & 1.000 & 0.488 & - & 0.672 & - & - & 0.937 \\
Validity   & 1.000 & 1.000 & 1.000 & 1.000 & - & 0.930 & - & - & 0.915 \\
Diversity  & 0.583 & 0.847 & 1.000 & 0.826 & - & 0.656 & - & - & 0.556 \\ \midrule
\multicolumn{10}{c}{\textbf{(Random initial)}} \\
Top1 F     & 3.941 & 3.923 & 4.015 & 3.924 & 4.092 & 4.098 & 4.232 & 4.157 & \cellcolor{lime}\textbf{4.276} \\
Top10 F    & 3.926 & 3.851 & 3.937 & 3.875 & 4.023 & 4.065 & 4.164 & 4.087 & \cellcolor{lime}\textbf{4.245}\\
Uniqueness & 0.821 & 0.956 & 1.000 & 0.477 & 0.690 & 0.575 & 0.986 & 0.349 & 0.949 \\
Validity   & 1.000 & 1.000 & 1.000 & 0.999 & 0.979 & 0.938 & 1.000 & 0.998 & 0.900 \\ 
Diversity  & 0.623 & 0.778 & 0.717 & 0.819 & 0.640 & 0.570 & 0.581 & 0.653 & 0.529 \\ \midrule
\multicolumn{10}{c}{\textbf{(Best initial)}} \\
Top1 F     & 4.583 & 4.329 & 4.582 & 4.420 & - & \cellcolor{lime}\textbf{4.699} & - & - & \cellcolor{lime}\textbf{4.699}\\
Top10 F    & 4.582 & 4.132 & 4.472 & 4.181 & - & 4.564 & - & - & \cellcolor{lime}\textbf{4.628} \\
Uniqueness & 0.729 & 1.000 & 1.000 & 0.432 & - & 0.678 & - & - & 0.942 \\
Validity   & 1.000 & 1.000 & 1.000 & 0.999 & - & 0.913 & - & - & 0.790 \\
Diversity  & 0.424 & 0.792 & 0.630 & 0.788 & - & 0.600 & - & - & 0.491 \\
\bottomrule
\end{tabular}
\caption{Unconstrained molecular design results, objectives: QED$\uparrow$ + SA$\downarrow$ + DRD2$\downarrow$ + GSK3$\beta\downarrow$ + JNK3$\uparrow$}
\label{tab:main_results}
\end{sc}
\end{table*}

\setlength{\tabcolsep}{1pt}
\begin{table*}[h]
\centering
\scriptsize
\begin{sc}
\begin{tabular}{>{\centering\arraybackslash}m{1.8cm} | >{\centering\arraybackslash}m{1.2cm} |>{\centering\arraybackslash}m{1.2cm} |>{\centering\arraybackslash}m{1.2cm} |>{\centering\arraybackslash}m{1.2cm} |>{\centering\arraybackslash}m{1.2cm} |>{\centering\arraybackslash}m{1.2cm} |>{\centering\arraybackslash}m{1.2cm} |>{\centering\arraybackslash}m{1.2cm} |>{\centering\arraybackslash}m{1.2cm} |>{\centering\arraybackslash}m{1.2cm} |>{\centering\arraybackslash}m{1.2cm} |>{\centering\arraybackslash}m{1.2cm}}
\toprule
 & \multicolumn{2}{c|}{\textbf{1 Objective}} & \multicolumn{2}{c|}{\textbf{2 Objectives}} & \multicolumn{2}{c|}{\textbf{3 Objectives}} & \multicolumn{2}{c|}{\textbf{4 Objectives}} & \multicolumn{2}{c|}{\textbf{5 Objectives}} & \multicolumn{2}{c}{\textbf{6 Objectives}} \\ \midrule  
 
Metric & \multicolumn{2}{c|}{MOLLM MOLLEO} & \multicolumn{2}{c|}{MOLLM MOLLEO} & \multicolumn{2}{c|}{MOLLM MOLLEO} & \multicolumn{2}{c|}{MOLLM MOLLEO} & \multicolumn{2}{c|}{MOLLM MOLLEO} & \multicolumn{2}{c}{MOLLM MOLLEO}  \\  \midrule
Top1 F & \multicolumn{2}{c|}{\textbf{0.948} \hspace{0.2cm} 0.941} & \multicolumn{2}{c|}{\textbf{1.901} \hspace{0.2cm} 1.887} & \multicolumn{2}{c|}{\textbf{2.901} \hspace{0.2cm} 2.891} & \multicolumn{2}{c|}{\textbf{3.901} \hspace{0.2cm} 3.890} & \multicolumn{2}{c|}{\textbf{4.276} \hspace{0.2cm} 4.098} & \multicolumn{2}{c}{\textbf{5.183} \hspace{0.2cm} 4.964} \\
Top10 F & \multicolumn{2}{c|}{\textbf{0.948} \hspace{0.2cm} 0.936} & \multicolumn{2}{c|}{\textbf{1.901} \hspace{0.2cm} 1.882} & \multicolumn{2}{c|}{\textbf{2.901} \hspace{0.2cm} 2.886} & \multicolumn{2}{c|}{\textbf{3.901} \hspace{0.2cm} 3.887} & \multicolumn{2}{c|}{\textbf{4.245} \hspace{0.2cm} 4.065} & \multicolumn{2}{c}{\textbf{5.164} \hspace{0.2cm} 4.948} \\
Uniqueness & \multicolumn{2}{c|}{\textbf{0.929} \hspace{0.2cm} 0.150} & \multicolumn{2}{c|}{\textbf{0.666} \hspace{0.2cm} 0.231} & \multicolumn{2}{c|}{\textbf{0.778} \hspace{0.2cm} 0.273} & \multicolumn{2}{c|}{\textbf{0.807} \hspace{0.2cm} 0.387} & \multicolumn{2}{c|}{\textbf{0.949} \hspace{0.2cm} 0.575} & \multicolumn{2}{c}{\textbf{0.957} \hspace{0.2cm} 0.591}\\
Validity & \multicolumn{2}{c|}{\textbf{0.796} \hspace{0.2cm} 0.159} & \multicolumn{2}{c|}{\textbf{0.962} \hspace{0.2cm} 0.552} & \multicolumn{2}{c|}{\textbf{0.946} \hspace{0.2cm} 0.803} & \multicolumn{2}{c|}{\textbf{0.946} \hspace{0.2cm} 0.783} & \multicolumn{2}{c|}{0.900 \hspace{0.2cm} \textbf{0.938}} & \multicolumn{2}{c}{0.890 \hspace{0.2cm} \textbf{0.926}}\\
Diversity & \multicolumn{2}{c|}{0.538 \hspace{0.2cm} \textbf{0.865}} & \multicolumn{2}{c|}{0.450 \hspace{0.2cm} \textbf{0.646}} & \multicolumn{2}{c|}{0.510 \hspace{0.2cm} \textbf{0.627}} & \multicolumn{2}{c|}{0.375 \hspace{0.2cm} \textbf{0.614}} & \multicolumn{2}{c|}{0.529 \hspace{0.2cm} \textbf{0.573}} & \multicolumn{2}{c}{0.529 \hspace{0.2cm} \textbf{0.611}}\\
\bottomrule
\end{tabular}
\caption{Unconstrained molecular design results with 1 to 6 objectives. The sixth objective is BBBP.}
\label{tab:multi-objective}
\end{sc}
\end{table*}
Following the experimental settings of MOLLEO~\cite{molleo}, we first conduct experiments to optimize five molecular properties simultaneously using molecules sampled from the ZINC 250K dataset. Among these objectives, three are minimized: SA, DRD2, and GSK3$\beta$, while two are maximized: QED and JNK3. Each model is run with five different random seeds, and the final results are reported as the average over these runs. Since the initial population for REINVENT, DyMol, and Genetic-GFN cannot be explicitly set, these models are only evaluated in the randomly initialized scenario. The key evaluation metrics are top-1 fitness and average top-10 fitness, both of which directly reflect the sum of the normalized property values. To enhance clarity, the highest values in each metric are highlighted in Table~\ref{tab:main_results}. Our model demonstrates a significant improvement over other SOTA models across all three initialization cases, with a clear performance gap compared to the second-best approach. Notably, in both the worst-initialization and random-initialization scenarios, the mean top-10 F-value exceeds the top-1 F-value of the second-best model, highlighting the superior performance and convergence capabilities of our approach.

In the best-initialization scenario, while the top-1 fitness of MOLLEO matches that of MOLLM, the mean top-10 fitness of MOLLM is noticeably higher than both the top-1 and mean top-10 fitness of all other models. Furthermore, our model maintains a uniqueness rate above 90\%, whereas MOLLEO, despite being another LLM-based method, exhibits significantly lower uniqueness. This underscores the strong capability of MOLLM in effectively exploring the chemical space. The validity of generated molecules is also comparable to other models. Although our model exhibits relatively lower diversity among the top-100 molecules, we observe that models with higher diversity often achieve lower top fitness values. This suggests that direct comparisons of diversity may be less meaningful in this context but highlight a potential direction for future improvements. Across all three initialization settings, MOLLM consistently maintains higher diversity while achieving superior fitness values, demonstrating its robustness in molecular optimization.



\section{Ablation Study}
In addition to the SOTA results from our main experiments involving the optimization of five objectives, we conduct further experiments with one to six objectives to assess the efficacy of MOLLM across varying optimization complexities and less predictable properties. Following this, we present an analysis of an interesting finding related to the experience pool utilized in our algorithm. Finally, we evaluate the impact of hyperparameters and demonstrate the effectiveness of our proposed components.  
\subsection{More Objectives}  
To further investigate MOLLM’s performance across different objective configurations, we conduct experiments using random initialization across scenarios with one to six objectives. The specific objective combinations are as follows:
\begin{enumerate}
    \item QED$\uparrow$ 
    \item QED$\uparrow$ + SA$\downarrow$ 
    \item QED$\uparrow$ + SA$\downarrow$ + DRD2$\downarrow$ 
    \item QED$\uparrow$ + SA$\downarrow$ + DRD2$\downarrow$ + GSK3$\beta\downarrow$
    \item QED$\uparrow$ + SA$\downarrow$ + DRD2$\downarrow$ + GSK3$\beta\downarrow$ + JNK3$\uparrow$
    \item QED$\uparrow$ + SA$\downarrow$ + DRD2$\downarrow$ + GSK3$\beta\downarrow$ + JNK3$\uparrow$ + BBBP$\uparrow$
\end{enumerate}
As the number of objectives increases, the performance gap between MOLLM and MOLLEO widens, particularly when optimizing more than four objectives, highlighting the superior capability of MOLLM in handling MOO. Additionally, MOLLM consistently achieves higher uniqueness and competitive validity compared to MOLLEO, while in MOLLEO these metrics tend to degrade significantly when optimizing fewer objectives. The consistently high uniqueness across all cases underscores the stability and effectiveness of MOLLM in optimization tasks with varying numbers of objectives. To further assess the robustness of MOLLM, we introduce BBBP (Blood-Brain Barrier Permeability) as a sixth objective, as it is a more complex and less predictable property with limited domain knowledge. Notably, despite the increased difficulty, MOLLM successfully generates a top 100 molecule set where all molecules are Blood-Brain Barrier Permeable (BBB+), demonstrating its strong adaptability and effectiveness in optimizing challenging molecular properties. 


\begin{table}[h]
\centering
\scriptsize
\begin{sc}
\begin{tabular}{>{\centering\arraybackslash}m{1.5cm} >{\centering\arraybackslash}m{1.5cm} >{\centering\arraybackslash}m{1.5cm}}
\toprule
Method & LLM calls & Runing Time (hours) \\ \midrule
MOLLEO & 8517 & 7.32 \\
MOLLM & 2908 & 0.52 \\
\bottomrule
\end{tabular}
\caption{Running time of MOLLEO and MOLLM}
\label{tab:efficiency}
\end{sc}
\end{table}
Apart from that, without early stopping, MOLLM only uses nearly $\frac{1}{3}$ LLM calls compared to MOLLEO, more than even 14x faster than MOLLEO in run time to achieve significantly better results, as shown in Table~\ref{tab:efficiency}. 

\subsection{Experience Pool}
\begin{table}[h]
\centering
\scriptsize
\begin{sc}
\begin{tabular}{>{\centering\arraybackslash}m{1.2cm} >{\centering\arraybackslash}m{1.2cm} >{\centering\arraybackslash}m{1.2cm} >{\centering\arraybackslash}m{1.5cm} >{\centering\arraybackslash}m{1.2cm} >{\centering\arraybackslash}m{1.2cm} }
\toprule
$P_{exp}$& Top1 F & Top10 F & Uniqueness & Validity & Diversity \\ \midrule
0.0 & \textbf{4.187} & \textbf{4.152} & 0.937 & 0.915 & \textbf{0.556} \\
0.1 & 4.175 & 4.163 & 0.935 & \textbf{0.917} & 0.548 \\
0.3 & 4.154 & 4.124 & 0.961 & 0.903 & 0.544 \\
0.5 & 4.168 & 4.144 & \textbf{0.978} & 0.898 & 0.554 \\
\bottomrule
\end{tabular}
\caption{Experiments of adding experience.}
\label{tab:exp}
\end{sc}
\end{table}
Inspired by ExpeL~\cite{zhao2023expel}, we incorporate an experience pool into our algorithm to enhance molecular optimization. The experience pool consists of two key components: (1) knowledge gained from generating better and structurally similar molecules, and (2) insights for avoiding suboptimal molecules. These are achieved by summarizing information from the top 10 and bottom 10 molecules, respectively, in each iteration. The worst 10 molecules are selected using a sliding window approach with a stride of 10. Specifically, if in the previous iteration, the worst molecules were extracted from the bottom 10, the next iteration extracts molecules ranked from the bottom 20 to the bottom 10. This mechanism ensures that the experience pool continuously evolves, integrating knowledge from both current and past iterations.

While the concept of experience pools aligns with human decision-making—shifting from general heuristics to more concrete optimization strategies—we observe a performance decline when incorporating experience into our model. As shown in Table~\ref{tab:exp}, where $P_{exp}$ represents the probability of integrating experience into the prompt, the best performance is achieved when $P_{exp}=0.0$, indicating that excluding experience leads to superior optimization and greater molecular diversity. We attribute this phenomenon to the nature of optima distribution in the molecular space. Since local optima tend to be large and widely separated, the experience pool may cause the model to focus excessively on a given local optimum, thereby hindering exploration of alternative high-quality solutions. Consequently, to maximize optimization performance, we temporarily exclude the experience pool from our experiments.

\subsection{Hyperparameters}
\begin{table}[h]
\centering
\scriptsize
\begin{sc}
\begin{tabular}{>{\centering\arraybackslash}m{1.5cm} >{\centering\arraybackslash}m{1.2cm} >{\centering\arraybackslash}m{1.2cm} >{\centering\arraybackslash}m{1.5cm} >{\centering\arraybackslash}m{1.2cm} >{\centering\arraybackslash}m{1.2cm} }
\toprule
Method & Top1 F & Top10 F & Uniqueness & Validity & Diversity \\ \midrule
Without MO Selection & 3.830 & 3.791 & \textbf{0.999} & 0.816 & \textbf{0.842} \\
With MO Selection & \textbf{4.187} & \textbf{4.152} & 0.961 & \textbf{0.915} & 0.556 \\
\bottomrule
\end{tabular}
\caption{Experiments of using MO.}
\label{tab:selection}
\end{sc}
\end{table}
To validate the effectiveness of the key components in MOLLM, we conduct a series of ablation studies. In MOLLM, Pareto front selection and F-value selection are applied with equal probability in each iteration. The importance of this design is demonstrated in Table~\ref{tab:selection}, where performance significantly deteriorates when multi-objective selection is removed. Furthermore, if an objective is included in the prompt but is not explicitly considered in MO selection, the performance of MOLLM declines substantially. This highlights the critical role of MO selection in ensuring effective optimization across multiple objectives.

\begin{table}[b]
\centering
\scriptsize
\begin{sc}
\begin{tabular}{>{\centering\arraybackslash}m{1.7cm} >{\centering\arraybackslash}m{1.2cm} >{\centering\arraybackslash}m{1.2cm} >{\centering\arraybackslash}m{1.5cm} >{\centering\arraybackslash}m{1.2cm} >{\centering\arraybackslash}m{1.2cm} }
\toprule
Method & Top1 F & Top10 F & Uniqueness & Validity & Diversity \\ \midrule
GPT-4o direct propose & 3.974 & 3.955 & 0.955 & 0.864 & 0.644 \\
MOLLM (Llama3-8B) & 3.988 & 3.900 & 0.986 & 0.482 & 0.749 \\
1 offspring each call & 4.068 & 3.980 & 0.969 & 0.942 & 0.575 \\
3 offspring each call & 4.208 & 4.114 & 0.970 & 0.831 & 0.592 \\
2 offspring each call & \textbf{4.276} & \textbf{4.245} & 0.949 & 0.900 & 0.529 \\
\bottomrule
\end{tabular}
\caption{Experiments of effects of hyperparameters.}
\label{tab:llm}
\end{sc}
\end{table}
In Table~\ref{tab:llm}, the MOLLM with ``2 offspring each LLM call'' is used in our official version. Compared to 5000 molecules directly proposed by GPT-4o, MOLLM makes a significant improvement to it, illustrating the effectiveness of our framework. Even with Llama3-8B~\cite{grattafiori2024llama3} as our backbone, which is much inferior to GPT-4o, its performance is also comparable to other models in Table~\ref{tab:main_results}. We make the LLM to generate two offsprings in both crossover and mutation for each LLM call. This design significantly reduces the number of LLM calls needed and achieves better performance, compared to one offspring each call which is used by MOLLEO and three offspring each call.  

\section{Conclusion}
In this work, we introduce MOLLM, a novel framework that integrates MOO, GA, and LLMs with in-context learning and prompt engineering for molecular design. MOLLM requires no additional training, relying exclusively on LLMs as genetic operators, and achieves SOTA performance in unconstrained molecular optimization. Through rigorous framework design, empirical evaluations, and ablation studies, we demonstrate its effectiveness and efficiency. MOLLM significantly reduces computational costs while outperforming other LLM-based approaches and other SOTA methods. This efficiency is particularly advantageous for practical applications, where molecular property evaluations often involve costly biological and pharmaceutical testing, and LLM inference imposes a high computational overhead. Our results show that MOLLM maintains robust performance across various objective settings and remains superior when optimizing multiple objectives, including less predictable properties such as BBBP. Furthermore, MOLLM is adaptable to different LLM architectures, facilitated by a carefully designed prompt template that fully utilizes LLM knowledge. Future research may focus on enhancing molecular diversity and refining the experience pool mechanism to further improve optimization performance.

\section*{Impact Statement}
The development of MOLLM introduces a novel approach to multi-objective molecular design by integrating LLMs as genetic operators. This work has the potential to advance computational drug discovery, materials science, and chemical engineering by significantly improving the efficiency and effectiveness of molecular optimization.

From an ethical perspective, MOLLM does not generate molecules directly aimed at harmful applications, such as toxic or hazardous compounds. However, as with any generative model in molecular design, dual-use concerns may arise, necessitating responsible usage and safeguards to ensure ethical deployment. Researchers and practitioners leveraging MOLLM should carefully consider biosecurity implications, regulatory frameworks, and best practices in molecular design.

On a societal level, the framework reduces the reliance on resource-intensive molecular synthesis and experimental testing, potentially accelerating drug discovery and enabling more cost-effective pharmaceutical and material innovations. Additionally, MOLLM’s adaptability to different LLM architectures ensures that future advancements in AI models can further enhance molecular design without requiring retraining or additional computational resources.

Future work should focus on improving molecular diversity, refining the experience pool mechanism, and ensuring ethical guidelines are upheld in real-world applications. This work aligns with the broader goal of advancing machine learning for scientific discovery, contributing to AI-driven molecular design with potential long-term benefits in healthcare, sustainability, and materials innovation.



% In the unusual situation where you want a paper to appear in the
% references without citing it in the main text, use \nocite
\nocite{langley00}

%\bibliography{example_paper}
%

\documentclass{article}

%
\usepackage{microtype}
\usepackage{graphicx}
\usepackage{subfigure}
\usepackage{booktabs} %

%
%
%
%
%
\PassOptionsToPackage{hyphens}{url}\usepackage{hyperref}
\usepackage{annotate-equations}

%
\newcommand{\theHalgorithm}{\arabic{algorithm}}

%
%

%
%
\usepackage[accepted]{icml2025}

%
\usepackage{amsmath}
\usepackage{amssymb}
\usepackage{mathtools}
\usepackage{amsthm}

\newcommand{\jh}[1]{{\color{red}[jh: #1]}} 
\newcommand{\bk}[1]{{\color{blue}[bk: #1]}} 

%
\usepackage[capitalize,noabbrev]{cleveref}

%
%
%
\theoremstyle{plain}
\newtheorem{theorem}{Theorem}[section]
\newtheorem{proposition}[theorem]{Proposition}
\newtheorem{lemma}[theorem]{Lemma}
\newtheorem{corollary}[theorem]{Corollary}
\theoremstyle{definition}
\newtheorem{definition}[theorem]{Definition}
\newtheorem{assumption}[theorem]{Assumption}
\theoremstyle{remark}
\newtheorem{remark}[theorem]{Remark}

%
%
%
\usepackage[textsize=tiny]{todonotes}


\usepackage{epigraph}

\usepackage{lipsum}

\usepackage{enumitem}
\setlist[enumerate]{itemsep=0mm}



%
%
%
\icmltitlerunning{We Can't Understand AI Using Our Existing Vocabulary}
%

\begin{document}

\twocolumn[
%
\icmltitle{We Can't Understand AI Using our Existing Vocabulary}
%
%
%
%
%
%
%

%
%
%
%

%
%
%
\icmlsetsymbol{equal}{*}

\begin{icmlauthorlist}
\icmlauthor{John Hewitt}{yyy}
\icmlauthor{Robert Geirhos}{yyy}
\icmlauthor{Been Kim}{yyy}
\end{icmlauthorlist}

\icmlaffiliation{yyy}{Google DeepMind}

\icmlcorrespondingauthor{John Hewitt}{johnhew@google.com}
\icmlcorrespondingauthor{Been Kim}{beenkim@google.com}

%
%
%
\icmlkeywords{Machine Learning, ICML}

\vskip 0.3in
]

%

%
%
%
%
%

\printAffiliationsAndNotice{}  %
%

\begin{abstract}
This position paper argues that, in order to understand AI, we cannot rely on our existing vocabulary of human words. Instead, we should strive to \textbf{develop neologisms}: new words that represent precise human concepts that we want to teach machines, or machine concepts that we need to learn.
We start from the premise that humans and machines have differing concepts.
This means interpretability can be framed as a communication problem: humans must be able to reference and control machine concepts, and communicate human concepts to machines. Creating a shared human-machine language through developing neologisms, we believe, could solve this communication problem. 
Successful neologisms achieve a useful amount of abstraction: not too detailed, so they're reusable in many contexts, and not too high-level, so they convey precise information. As a proof of concept, we demonstrate how a ``length neologism'' enables controlling LLM response length, while a ``diversity neologism'' allows sampling more variable responses. Taken together, we argue that we cannot understand AI using our existing vocabulary, and expanding it through neologisms creates opportunities for both controlling and understanding machines better.

%
%
%
%
%
%
%
%
%
%
%
%
%
%
%
%
%

%
%
%
%
%
%
%
%
%
%
%
%
%
%
%
%
%
%
%
%
%
%
%
%
%
%
\end{abstract}



\section{Introduction}
\setlength{\epigraphrule}{0pt}
\setlength{\epigraphwidth}{1.0\columnwidth}
\epigraph{\raggedleft\emph{``Die Grenzen meiner Sprache bedeuten die Grenzen meiner Welt''\\(The limits of my language are the limits of my world)}}{--- Ludwig Wittgenstein}
\vspace{-.4cm}

\begin{figure}
\centering
\includegraphics[width=.9\linewidth]{figures/fig1.pdf}
\caption{Humans and machines conceptualize the world differently from each other. Mismatches in communication occur, which lead to misunderstandings. To understand and control AI, we must bridge this gap by developing new words corresponding to human and machine concepts, and use these words to control machines.  \label{fig:header} %
}
\end{figure}

As researchers interested in understanding and controlling language model-based AI systems, we often search for human-like concepts in machines---e.g., by analyzing a machine's activation patterns.
Examples of such concepts include human-like linguistic structure \citep[e.g.][]{lakretz-etal-2019-emergence,hewitt2019structural}, or notions of safety or truth \citep{burns2023discovering}.
Often, the goal of this search is to help specify human concepts to machines---that is, to control them, e.g., through supervised probes, or prompts, or RLHF \cite{ouyang2022training}.
Taken together, understanding and control are a \textit{communication} problem: communicating concepts between humans and machines. Within this communication problem, understanding and control are often two sides of the same coin: the purpose of communicating with machines is typically to make machines do what we want them to do (\emph{control}), while achieving a better \emph{understanding} should directly translate into better communication.



\begin{figure*}
    \centering
    \includegraphics[width=.75\textwidth]{figures/mh2.pdf}
    \caption{Machine and humans may fundamentally understand the world differently, enabling different concepts, knowledge and capabilities. Figure reproduced from \citet{kim2022beyond,schut2023bridging} with permission.}
    \label{fig:mh}
\end{figure*}


This communication problem is hard because---and we take this as a premise---humans and machines conceptualize the world differently, at many levels of abstraction (as expressed by \cref{fig:header}).
A machine's notion of sentiment is different from a human notion of sentiment.
Likewise for high-quality code, or topic. %
%
In \citet{kim2022beyond} for example, reproduced in Figure~\ref{fig:mh}, there is a space $M$ of machine concepts and a space $H$ of human concepts, and many things are either in $H-M$: concepts humans have but machines do not, or $M-H$: concepts machines have but humans do not.
In fact; even for things seemingly in $H\cap M$, we expect that careful inspection would show that the seemingly similar concepts actually differ between humans and machines.


Our position is that progress in this communication problem---thus, progress in interpretability---is best achieved by striving to \textbf{define new words (neologisms)} that \textit{mean} a human concept (when interpreted by a machine) or a machine concept (when interpreted by a human).
%
%
%

What does introducing neologisms offer?
%
The perspective provides clarity in \textbf{what level of abstraction} to attempt to bridge this communication gap.
Successful words in a language strike useful levels of abstraction: they're not too exacting and low-level, like a word for the exact placement of the chairs at my table in relation to me.
Such words would be too rarely used to be successful.
This is alike to attempting for a full, exact, mechanistic understanding of a neural network: words corresponding to such exactness must necessarily not apply commonly, because the world (and networks modeling the world) are too complex to be concisely described at that level. %
At the same time, most successful words are not too high-level (only a few such words, like \textit{thing,} exist)---they're discriminative enough of communicative intent to be informative in conversation.
Erring too high-level is alike to only doing behavioral testing of a network; the level of abstraction is that of an input-output map (e.g., logit output).
Such evaluation is useful, but it gives us insufficient richness to specify our goals and understand future behavior.

The next useful property that neologism learning gives us is \textbf{participation in language}.
Language is how we understand other humans, and we define new words when our differences in conceptualization lead to the need to concisely communicate new concepts.
Likewise, defining new words in our communication problem with machines, we can plug these words into existing language and leverage the expressive compositional structure thereof.

Finally, the neologism framing helps us  \textbf{combat confirmation bias} and anthropomorphism \citep[e.g.][]{buckner2019comparative}.
As human researchers, we have a bias towards seeing human-like things in artificial networks; we want to see high-level human concepts appearing in networks. We want to see exciting unsupervised structure.
%
Even something as simple as a ``sentiment neuron'' \cite{openai-unsupervised-sentiment-neuron}, if given its own new word, reminds us that this sentiment-like concept of the  machine is likely dissimilar from what we call sentiment in ways that another human's notion of sentiment might not be.
In Section~\ref{sec:automatic_convergence}, we argue that this dissimilarity will only increase as machines become more capable.


Our notion of what constitutes defining new words is intentionally broad---this is a high-level research direction wherein details must be nailed down over time.
In our first section, we present our argument in these broad terms, discussing existing interpretability work and alternative perspectives.
Then, we provide a proof-of-concept that encodes machine and human concepts in new word embeddings that can be used in prompts to understand and control model behavior.
We use preference data to define words for \textit{diverse$_H$} and \textit{length$_H$}, corresponding to human notions of diversity of response and constraints on the length of a response. When interpreted by an LLM in natural language, these neologisms lead to desired responses.
We also define a word for \textit{good$_M$}, corresponding to a machine's notion of response quality in a given domain, whose use we show can help us understand what kinds of responses the model thinks are good. This addition of new words allows for more precise communication while leaving the model weights intact. This simple method, which we call \emph{neologism embedding learning}, is a first step towards creating a joint human-machine language for understanding and control.


%
%
%
%
%
%
%
%
%
%
%
%
%
%
%
%
%
%
%
%
%
%
%
%
%
%
%

\section{Understanding AI requires neologisms}
In this section, we present our argument that effective understanding and communication with AI systems requires us to form neologisms.
We start with problems faced by researchers attempting to understand AI systems, and then present the solutions provided by the neologism framing.

\subsection{Problems in Understanding AI}
Understanding (and controlling) AI systems, as a joint problem of science and engineering, forces researchers to make bets on what kinds of properties to attempt to discover in AI systems, and what kinds of tools to build.
As researchers on the understanding problem, we've noticed a few perennial problems that together motivate our focus on neologisms.

\paragraph{The Conceptualization Difference Problem.}
%
Humans and machines understand the world differently, forming different concepts---equivalence classes, evaluations, skills---from each other.
This means we need to reference/label these concepts and then learn or understand the differences.

%
Let's take an example.
In game two of the 2016 match \textit{AlphaGo versus Lee Sedol}, AlphaGo's 37th move was considered particularly surprising compared to the usual play of top human Go players \cite{move37}.
AlphaGo would go on to win that game, and the match.
Intuitively, AlphaGo may have had a general concept---in this case, a pattern it recognized in the board state---that motivated this move, which humans had not thought of. The general concept behind this move is yet to be understood ($M-H$).

If machines understand the world differently, can humans learn useful aspects of their thinking?
Testing this idea in chess, \citet{schut2023bridging} developed a process for discovering superhuman chess concepts in AlphaZero \cite{silver2017masteringchessshogiselfplay}, and another process for teaching those concepts to grandmasters---humans at the frontier of human knowledge---to expand what they know.
Prior work had discovered correlates of human concepts in chess engines, like king safety or board position, \cite{lovering2022evaluation,mcgrath2022acquisition}, but \citet{schut2023bridging} focused specifically on \textit{new} yet teachable (generalizable) concepts. These concepts are successfully taught to four top\footnote{Top grandmasters are significantly stronger than the average grandmaster; each of these four has won a FIDE world championship title (in open classical, open blitz, or womens' classical.)} grandmasters, showing that the concepts were alien, but learnable.




\begin{figure}
    \centering
    \includegraphics[width=\linewidth]{figures/fig_abstraction.pdf}
    \vspace{-0.2cm}
    \caption{Concept-based neologisms sit in-between mechanistic interpretability (which is closer to mechanistic details) and behavioral experiments/capability benchmarking (which is only concerned with the model's output, not how it arrived there).}
    \label{fig:abstraction_levels}
\end{figure}



\paragraph{The Abstraction Problem.}

Once we accept that we're attempting to communicate between two differing conceptualizations of the world, the question becomes, at what level of abstraction do we attempt this communication (\Cref{fig:abstraction_levels})?
Low abstraction would suggest attempting exactingly precise concepts.
%
%
For example, we have access to the precise forward pass of the network, which is as much a program of its behavior as one could hope; nonetheless, this knowledge is not considered sufficient for understanding the network.
High abstraction, on the other hand, would suggest attempting to build very broad statements, like \textit{AlphaZero doesn't mind giving up material for a positional advantage}.
These might be nice to know, but are insufficiently rich in how they let us control or trust the model.
There is no one right level of abstraction at which to tackle the understanding problem, but it is key to hit a good balance, as we'll argue.

\paragraph{The Confirmation Bias Problem.}
As humans and as researchers, we have a bias towards finding exciting, seemingly human-like properties in models.
Considerable ink has been spilled on the pitfalls of such biases, e.g., in probing \cite{hewitt-liang-2019-designing}, saliency maps \cite{adebayo18sanity, bilodeau2024impossibility}, as well as interpretability as a field \cite{lipton2017,doshivelez2017}.
When we're looking for interesting concepts, attempting to control them, even at a useful level of abstraction, we still fall victim to our own biases when evaluating whether those concepts, like safety or sentiment, really line up with our own.
Put another way, if we had discovered high-level concepts of safety in models already, wouldn't the safety-training teams at frontier labs be using those methods instead of (or as well as) doing better data collection for RLHF?

\subsection{How Neologisms Help}

We now argue how the problems we've discussed can be ameliorated by framing understanding AI as a communication problem in which we form neologisms to reference human or machine concepts.

Let's go back to the example of chess and AlphaZero.
When one human chess grandmaster attempts to understand the play of another, they use concepts and categories at some level of abstraction (e.g., forks, pins) that chess players jointly develop through shared experiences (games they play or observe together) and many rounds of discussion.
Without developing this shared language, their discussions would be verbose.
When we as humans attempt to understand AlphaZero, we should expect to have to develop such a language of concepts ourselves.
Developing neologisms is a first key step in developing this shared language such that it enables efficient communication of complex concepts. Those of us who speak more than one language are familiar with the difficulty and verboseness of expressing certain thoughts in a language that simply does not have a word for the same concept, like the Dutch ``Gezelligheid'' (a sense of warmth, coziness and sometimes friendship in a social context), or the Korean ``Jeong'' (affection that only develops over time, sometimes love and hate affection, expressed through experience---translating it as ``affection'' simply does not do the concept justice).

\paragraph{Neologisms concisely reference new learnable concepts.}
Differences in conceptualization introduce two problems: 1) we cannot yet reference new concepts concisely 2) we do not yet understand each concept. 
%
Neologisms provide a solution to the first problem.
A neologism is a \textit{successful new word}; neologisms are formed when there are complex concepts that would be onerously verbose to reference otherwise---i.e., by explaining the concept in a paragraph or a book each time.
Given that machines and humans have differences in conceptualization, developing new words to reference those concepts to each other is a natural solution.
However, just because we can reference it, does not mean we can understand what it means.
An important element for a successful neologism is %
\textit{proximity}; in Vygotsky's education theory, proximity references a concept being in ``the space between what a learner can do without assistance and what a learner can do with adult guidance or in collaboration with capable peers'' %
\cite{vygotsky78}; neologisms are formed to help reference things that are reachable but outside our current understanding.
For example, \citet{schut2023bridging} targeted teaching chess champions whose proximity zone has better chance of capturing AlphaZero's superhuman strategies. %



\paragraph{Neologisms moderate useful abstraction.}
Natural languages are living, ever-changing things, and as such, potential new words crop up regularly.
The words that survive to become neologisms strike a useful level of abstraction.
Some new words are more precise, like \textit{doomscroll}, while others are more vague, like \textit{vibe}.
A word that struck the right balance between the right level of abstractness and usefulness is a case of successful neologism, thus, interpretability.
%
The pressure of broad applicability enforces some abstraction: words gloss over an ocean of detail about the world so that they're applicable in many settings, and thus used (unused words fail as neologisms.)
The pressure of informativeness presses down on the amount of abstraction: a word that references all things would be uninformative.


\paragraph{Neologisms lessen confirmation bias.}
There is power to giving a new label to a thing instead of referring to it by a known label: it encourages us to believe that the new thing is by default different and unknown.
In the case of the OpenAI ``sentiment neuron,'' researchers discovered a single activation in a network that correlated reasonably well with sentiment on the Stanford Sentiment Treebank \cite{socher-etal-2013-recursive}.
At first glance this is not a bad label, but as interpretability researchers, labeling its concept with a new word sentiment$_{M}$ reminds us that it is probably \textit{not} like sentiment in systematic ways that remain to be interpreted.
We haven't found a human concept; we've found a machine concept that has some overlap with human concepts.


\paragraph{Neologisms enable compositionality.}
The beauty of human language is in the ability to concisely build an infinitude of meanings from a finite symbol vocabulary.
And due to the complexity of the world, there is an infinitude of concepts to understand about language models, not a finite set of features.
Another benefit of neologisms is that they \textit{participate in language}; they should combine together with other concepts we've learned, and natural language, allowing us to leverage natural language to use our new insights.

\paragraph{Neologisms provide a human interface for control.}
Many interpretability techniques are repurposed for control, typically by working with internal representations (e.g.,  sparse autoencoders \cite{cunningham2023sparse}, or probing \cite{zou2023representation},). Neologisms enable these controls by 
%
%
integrating into humans' natural way of communication -- language. As humans use new words, all expressive tools of language are at their disposal that could enable better precision and alignment. %

%

%
%
%
%
%
%
%
%
%
%
%
%
%
%
%
%
%
%
%
%
%
%
%
%
%
%
%
%
%
%
%
%
%
%
%
%
%
%
%
%
%
%
%
%
%
%
%
%
%
%
%
%
%
%
%
%
%
%
%
%
%
%
%
%
%
%
%
%
%
%
%
%
%
%
%
%
%
%
%
%
%
%
%
%

\section{Alternative views and rebuttals}
\label{sec:alternative_views}
%
This section describes views that are opposed to our position, along with rebuttals.
We follow it with a broader discussion of related interpretability work.


\subsection{Automatic convergence: scale will solve the communication problem}%
\label{sec:automatic_convergence}
%

\textbf{Position.} This position accepts that there are differences in the way humans and machines understand the world, but believes that these will inevitably narrow, and eventually disappear altogether.
The argument is based on the observation that the shared human-machine space ($M \cap H$ in \Cref{fig:mh}) has grown rapidly over recent years, from models that were barely able to produce a coherent paragraph to today's LLMs that excel at writing poetry, code and email drafts. The \emph{bitter lesson} \cite{sutton2019bitter} continues to apply: larger models trained on larger datasets are inevitably getting better. Extrapolating this trend, the ``automatic convergence'' argument posits that the remaining gap between human and machine understanding, represented by the non-overlapping areas in \Cref{fig:mh}, will gradually shrink and eventually vanish as a result of scaling models.

\textbf{Rebuttal.} While machines were approaching human level performance, the gap could indeed be perceived as narrowing. However, the gap will not close: as a simple example, humans will not be able to reason over adversarial examples that will likely continue to exist. We are yet to find how to reason over why change in one meaningless word in a prompt cause drastic changes in response (sometimes critical mistakes e.g., jailbreaking) \cite{branch2022evaluatingsusceptibilitypretrainedlanguage}.
Beyond these peculiarities of existing systems, future machines are on track for superhuman performance (e.g, AlphaFold, AlphaGo).
Such superhuman machines will likely \emph{widen} the communication gap; by definition, we do not yet know how to do what we can't. 
%
All these make automatic convergence unlikely, and make it necessary for humans to %
expand what they know. %

%
%
\subsection{We already have all the words needed to communicate.}
 %

\textbf{Position.} This position asserts that our existing vocabulary and language is sufficient to understand AI. We don't always introduce a new word when we extend our knowledge; sometimes it's just a longer description combining things we already know. We just need methods that map from machine concepts to natural language explanations. While there may be debate around whether explanations should be faithful (ideal but hard to verify) or just appear plausible or helpful (after all, we also accept post-hoc explanations from humans without being able to check whether they are true), this position essentially states ``why invent new words when our existing vocabulary must be sufficient for explanations. We can't understand what we can't describe.''

\textbf{Rebuttal.} We agree that in lieu of new words for new concepts, it may be possible to derive a natural language description for any concept---albeit a potentially long and cumbersome one. However, a lack of a concise word stifles communication; imagine not being able to use the word ``house'', and instead having to describe what you mean every time you'd like to refer to the concept. This would be neither concise, nor enable \emph{compositionality}; use the word together with natural language to enable expressibility. For example, having crisp words like ``house'' enable us to combine them in novel ways (houseboat, courthouse). According to Wittgenstein, ``the limits of my language are the limits of my world'' \citep{wittgenstein1922tractatus} -- and if we're interested in communicating with the world of machines, leveraging the expressibility of natural language and beyond would be necessary for flexible and effective communication. 
%

\subsection{We just need an exhaustive map of explanations}

\textbf{Position.} The lack of rigor in interpreting a model is the fundamental problem. If we can understand the exact low-level circuits, creating a comprehensive map of model features along with their function, this `explanation map' would solve interpretability and explain the entire decision-making process of models with great accuracy.

\textbf{Rebuttal.} Unfortunately, this approach does not scale well to increasingly large models with increasingly many circuits and features. Even if it did, an important lesson from neuroscience is that even mapping out the entire connectome of a system \citep[as done for the worm c.\ elegans, cf.][]{cook2019whole} is not sufficient to meaningfully understand the system.

%
%
%


%
\subsection{We don't need abstraction}
%

\textbf{Position.} In terms of understanding machines, the gold standard is a precise, exact mechanistic understanding with as little unexplained abstraction as possible--alike to ``reverse engineering'' neural networks \cite{olah2022mechanistic}.
%

\textbf{Rebuttal}. First, finding the right level of detail isn't obvious: Is it the level of layers? Circuits? Individual units, and their receptive and projective fields? The code that specifies a network, or the code it compiles to? The silicon it runs on? The atoms that create the silicon? Presumably, we can all agree that the level of individual atoms would be a ridiculous level of detail for analyzing machine intelligence; nonetheless this goes to show that in science, abstraction is often an advantage, not a drawback \citep[cf.][]{borges2002exactitude}.
Secondly, even if there was a universally accepted ``right level of detail'', in terms of human-machine communication this would still be a one way street, seeking to identify human concepts in machines.
In contrast, by creating new words to communicate concepts between humans and machines, this enables a \emph{compositional} understanding where one concept can be re-used for a different purpose, and combined with others. Arguably, components determined by mechanistic interpretability may not satisfy \textit{proximity} either, since the way components are decided had no regards to human's capability.

\section{How neologisms fit into other interpretability work}
We here discuss connections to a range of techniques and perspectives within existing interpretability work.

\textbf{Feature attribution methods.}
Feature attribution methods \citep[e.g.][]{sundararajan17integrated,lundberg17shapley,selvaraju17gradcam,shrikumar17deeplift,smilkov17smoothgrad} are widely used methods in interpretability. While the main critics of this approach~\cite{adebayo18sanity,tomsett20sanity,kindermans19unreliability,ghorbani19interpretation, bilodeau2024impossibility} seem to highlight why these methods do not and cannot work, an alternative hypothesis is what these methods are showing is not something humans can comprehend (perhaps due to using unnatural mediums like pixels to explain model decisions). For example, the fact that humans or quantitative metrics that \emph{we} defined cannot distinguish between feature attributions from a trained network vs.\ an untrained network \cite{adebayo18sanity} suggests two possibilities: Either 1. that they are truly the same or 2. the metrics we use are incapable of describing the concepts that machines have.  


\textbf{Concept discovery.}
Finding new concepts from models has been a well-studied problem \cite{netdissect2017, ghorbani2019towards, fel2023craft, Lang_2021_ICCV, rane2023concept, schut2023bridging}. While many attempt to name certain machine concepts, there is no systematic thinking on developing a new word with \emph{reusability} in mind, let alone being able to use them in \emph{composition} with natural language. Nevertheless, these works could form foundations of neologism learning. 

\textbf{Faithfulness and evaluation.}
Faithfulness---whether an explanation truly reflects a model's concepts---has long been a point of discussion in interpretability in the context of evaluating explanations \cite{lipton2017, doshivelez2017}.
Neologisms do not solve this problem; instead, they provide a new way to evaluate. 
%
We consider a new word a success if we can communicate something useful (e.g., control) using the new word, and the machine possibly uses the same word to communicate something in return. 
For example, If a new word good$_m$ defines how machines understand `good' answers, humans learn ways to use this towards their goal. If good$_m$ is aligned with good$_h$ except for length, we can prompt `give me good$_m$ answers but make them short'. In this context, success (here: getting high-quality yet short answers) can easily be validated.  
%
%


\textbf{Probing and representation engineering.}
Probing---training a simple readout function from neural activity to a property of interest---was most recently independently introduced in machine learning by \citet{Alain2016UnderstandingIL,ettinger-etal-2016-probing,Shi2016DoesSN}, though the methods were directly inspired by similar methods in neuroscience, which have a long history.
As models have improved, the targets of probing moved from linguistic properties \cite{hewitt2019structural,tenney2018what} towards higher-level concepts like correctness \cite{burns2023discovering,marks2024the}.
Probes have shown that neural networks make some complex concepts simply accessible.
Representation engineering \cite{zou2023representation} takes this idea and pursues the idea that these discovered concepts can be used to steer model behavior, previously considered by, e.g., \cite{eisape-etal-2022-probing}.
Probing and representation engineering as tools allow for a range of levels of abstraction and can be tools for communication of \emph{already known concepts ($M \cap H$)}, though the fundamental question in this case becomes when and where to apply probes to communicate with machines.





%


%
%
%



%

\begin{figure}
    \centering
    \includegraphics[width=\linewidth]{figures/fig_method.pdf}
    \caption{Our neologism embedding learning only updates new word embedding, preserving the original model's responses when the new word is not used.}
    \label{fig:neologism_method}
\end{figure}

\section{A proof of concept: Neologism Embedding Learning}
%

While our position in favor of neologisms is independent of specific implementations, one may ask how to implement this idea.
This section presents
%
a simple proof-of-concept to more concretely discuss the merits.
Our method, \textit{neologism embedding learning} (Figure~\ref{fig:neologism_method}), involves three steps:
\begin{enumerate}[itemsep=0mm]
    \item Defining a new vocabulary element in the tokenizer of a language model (and corresponding embedding matrix).
    \item Using that new word in otherwise-natural language sentences in prompts.
    \item Training the embedding of the word to minimize preference-based losses (defined to capture human or machine concepts). Note that  \textbf{the model remains frozen}, so when the new word is not invoked, the model is left unchanged: this guarantees identical output when the neologism isn't used.
\end{enumerate}

%
%
%

\subsection{Method}


Let $p_\theta$ be a neural language model parameterized by $\theta$.
Let $\mathcal{V}$ be a finite vocabulary, with corresponding word embeddings $E\in \mathbb{R}^{d\times |\mathcal{V}|}$, where $E\in\theta$, and $d$ is the dimesionality of representations in $p_\theta$.
We add a new word $w$ to the vocabulary, such that $w\not \in \mathcal{V}$, and our new vocabulary is $\mathcal{V}' = \mathcal{V}\cup \{w\}$.
We define a new embedding for for $w$, initialized either randomly or from some existing embedding.
We'll refer to this new $\mathbb{R}^{d}$ embedding of $w$ as $E_w$.
By construction, no naturalistic data contains the new word $w$.

We take preference data wherein the prompt contains the new word $w$.
More precisely, we assume a dataset $D=\{(x_i, y_{i,c}, y_{i,r})\}_{i=1}^n$, where each $x_i$ is a string over $\mathcal{V}'$ that is guaranteed to include $w$, and $y_{i,c}$ is the chosen response, a string over $\mathcal{V}$, and likewise $y_{i,r}$ is the rejected response.

Pulling an example from below, a sample prompt from such data might be \textit{Give me a recipe for tiramisu. Ensure$_w^h$ that the length of the response is at least 600 words. }
We've labelled the new word \textit{Ensure$_w^h$} for legibility, but recall that its embedding is new, and not tied to the English word \textit{ensure}. 
Instead, it refers to a specific human meaning (thus $h$) of ensuring length.
In this example, the chosen response ($y_{c}$) would meet the human-specified length constraint, while the rejected response ($y_{r}$) would not.

Neologism embedding learning optimizes a preference loss over just $E_w$, while keeping the rest of $\theta$ unchanged:
\begin{align}
\min_{E_w} \mathbb{E}_D \left[ \mathcal{L}(x,y_c, y_r)\right],
\end{align}
where the loss function $\mathcal{L}$ could be DPO \cite{rafailov2024direct} or one of its many variants; in Appendix~\ref{appendix:sec:methods}, we describe a variant of APO \cite{d2024anchored} that we found particularly useful for our experiments.

\textbf{Neologism for $H \rightarrow M$}: Intuitively, $w$ participates in natural language statements in $x$ wherein the person specifying the preference \textit{wants} a particular behavior ($y_c$) but may instead have received $y_r$; the \textit{meaning} of $w$ is thus learned to be whatever makes the chosen preference loss lower.

\textbf{Neologism for $M \rightarrow H$}: Similarly, a particular behavior machine exhibits can be captured in a $w$ trained using pairs of behaviors humans yet to understand. 
In other words, we use the word $w$ as a vehicle to carry the meanings of the contrast (between $y_c$ and $y_r$), then use $w$ in our prompt to understand what it means.
%

\subsection{Merits and related methods}
Neologism embedding learning combines the lightweight finetuning method \textit{soft prompting} \citep{lester2021power} with the flexibility of prompting.
Soft prompting involves learning an embedding or embeddings that are prepended to \emph{all} inputs without aiming to being interpretable \cite{bailey2023soft}.
The main merit of soft prompting over finetuning all parameters (or LoRA or similar \cite{hu2022lora}) is that the choice of soft prompt can be made simply by determining what tokens (or soft tokens) are used as input to the model.
Neologism embedding learning is subtly but crucially different from soft prompting: our new words are meant to be told to the users and \textbf{participate in natural language inputs much like other input tokens}, so the user chooses when and in what natural language contexts to use them.

As such, when a user decides not to use a new word $w$ in their prompt, they get a guarantee that they sample from the original, unchanged model (Figure~\ref{fig:neologism_method}.) 
Users can also use them in composition with other new words; we envision understanding machine concepts will require a variety of ways to combine new words.
%


\subsection{Experiment: Length Neologism ($H \rightarrow M$)}

Controlling the length of a response is a surprisingly difficult problem even for the frontier models when the desired length is very different from the model's prior over response lengths for the instruction.
In these experiments, we define a preference dataset $D$ by taking broad-domain instructions from LIMA \cite{zhou2023limaalignment}, and constructing responses that do and do not obey a specified length constraint; in particular, one of two: \textit{Ensure that the response is between 400-600 words} and between 600-1000 words.
For our neologism word ensure$_w^h$, we replace the word \textit{ensure} in that prompt.
When we test a base Gemma model \cite{gemma1} on held out instructions with one of the two length constraints (Figure~\ref{Length Control Demo}), we find that the base model \textit{never} meets the length constraints for the instructions we used (Figure~\ref{Length Control Demo}.)
However, many more responses meet the length constraints when prompted with ensure$_w^h$.

\begin{figure}
\centering
\includegraphics[width=.8\linewidth]{figures/length3.png}
\caption{\label{Length Control Demo}Base models prompted for length control fail to generate specified long generations (blue), but with a neologism (orange), they consistently generate longer responses.}
\end{figure}



\subsection{Experiment: Diversity Neologism ($H \rightarrow M$)}
\label{subsec:diversity_neologism}
%
One LLM `control knob' that a human user might want to have access to is the diversity or variation across several responses.
Sometimes, responses should not be variable: two plus two equals four, no matter how many times the model is asked.
On the other hand, sometimes variability is desired: for example, when we want a model to try multiple attempts to get the right answer or do a creative task (e.g., write a poem).
Diverse responses are particularly relevant in the context of inference-time scaling, where models often generate many responses to the same problem, followed by filtering or ranking \citep{brown2024large}.
%


One expensive heuristic for generating more diverse outputs is conditioning on existing responses, and asking for a new one. We leverage this heuristic to train a neologism for this human concept by using a high quality teacher model \citep[Gemini 1.5 Pro, ][]{team2024gemini} to condition on up to four responses for each instruction, and request a subsequent new (different) response. Our preference data contains instructions from the LIMA dataset \cite{zhou2023limaalignment}, each appended with an additional phrase. Instead of asking \textit{give me your $k^{\text{th}}$ response}, we prompt with \textit{diversity$_{w}^{h}$ me your $k^{\text{th}}$ response}. The chosen response is the $k^{\text{th}}$ Gemini response, and the rejected is the $(k-1)^{\text{st}}$.

%
To test this neologism in a proof-of-concept setting, a model is asked to guess an integer between 1 and 9.
We then assume that one and only one of these numbers is the `correct solution' that a model has to find within $n$ guesses.
\Cref{fig:diversity_control} shows, in black, how well a model could do in principle by guessing numbers uniformly at random.
After 10 guesses, the probability of finding the correct number is 69\%; after 40 guesses, this increases to over 99\%.
In blue, a standard Gemma model \citep{gemma1} is shown.
Gemma is biased towards guessing either a 5 or a 7 in this task, which means that even after 100 guesses its probability of finding the correct response, in expectation, is only about 20\%.

Fortunately, this can be drastically improved through our preference-trained neologism, diverse$_w^h$, a token trained such that a model generates the $i$-th response (here: the second response; shown in red). When using diverse$_w^h$ in the prompt, the model's responses indeed become more varied, strongly increasing the probability of finding the correct number. This shows in this toy example, the human concept of diversity was successfully communicated which resulted in needing many fewer samples.

\begin{figure}[t]
\centering
\includegraphics[width=.8\linewidth]{figures/scaling_gemma_min_1_max_9.pdf}
\caption{Adding a ``diversity neologism'' diverse$_w^h$ to a prompt substantially increases a model's response variety, as exemplified in a number guessing game. The setup is explained in \Cref{subsec:diversity_neologism}; higher = more response variety (better).}
\label{fig:diversity_control}
\end{figure}


\begin{figure}
\end{figure}

\subsection{Experiment: A Model's Preferences ($M \rightarrow H$)}
\begin{table}[t]
\small
\centering
\begin{tabular}{p{7.5cm}}
\bf Gemma responses for \textit{Give me a recipe for tiramisu.}\\
\toprule
\bf + Give me a response that is extremely good$^{m}_{w}$.\\
\midrule
%
%
%
%
%
%
%
%
%
%
%
\#\# Classic Tiramisu Recipe\\
*Ingredients:*

**For the ladyfingers:**

- 250g ladyfingers, softened\\
- 150g caster sugar\\
- 50ml all-purpose flour
{\color{red} [...truncated]}
%
\\
\midrule
\midrule
\bf + Give me a response that is extremely not good$^{m}_{w}$.\\
\midrule
%
I am unable to provide recipes that are unsafe or inappropriate. Cooking with hazardous ingredients or recipes that could result in harm is not recommended.\\
\bottomrule
\end{tabular}
\caption{\label{table:gemma_preference}Gemma responses for the instruction \textit{Give me a recipe for tiramisu} and a request for an extremely good${^m_w}$ or not good${^m_w}$ response, using Gemma's response quality neologism. `Extremely good$_{w}^{m}$' responses on average are scored 3.8 by Gemma, whereas `Extremely not good$_{w}^m$' are scored 3.2.}
\end{table}


How can we learn a concept from  a language model using almost just the methods we've discussed so far?
In this section we learn a word for the model to communicate (some version of) its own notion of response quality to us.

In these experiments, we learn a neologism $w$ that makes a model generate responses that it would \textit{itself} deem as ``good''.
%
Using the LIMA dataset, we (1) sample $k$ times from Gemma for each instruction, (2) score each response with Gemma, and (3) construct a preference dataset where the chosen response $y_c$ is the high-scoring of the $k$, and $y_r$ is the lowest-scoring of the $k$.
We then learn good$_w^m$, using the neologism embedding learning method, where the natural language we add to each LIMA instruction is \textit{Give me a response you think is good$_w^m$}, where good$_w^m$ is our new word.

Qualitatively, looking at a response for \textit{Give me a recipe for tiramisu that is extremely good$_w^m$}, we note that it is effusive and uses rich language (\Cref{table:gemma_preference}).
Despite not being trained to generate the \textit{negative} of good$_w^m$, we qualitatively see that the model correctly evaluates the extended language \textit{extremely not good$_w^m$}, generating a refusal answer.
Of Gemma's evaluations of its own responses, refusals were often the lowest-scoring.
To verify that \textit{extremely good$_w^m$} (and \textit{extremely not good$_w^m$}) do in fact yield responses that Gemma does (or does not, respectively) score highly, we generate 50 samples for each setting, and score them with Gemma.
Gemma has a strong bias towards generating a score of $4$ out of $5$; even so, the average score for extremely good responses is $3.8$ compared to a score of $3.2$ for extremely not good responses.

%
%
%

\section{Conclusion}

 %
%
%
Solving the communication problem between two dissimilar intelligent entities---communicating human concepts to machines, and machine concepts to humans---requires new language.
Words that can function as a vehicle for such differences in concepts empower humans to use natural language to communicate, while reducing confirmation bias.
%
%
\textit{Learning neologisms} enables  discovering and leveraging concepts at moderate abstraction that strikes a useful balance for communication.
Our method of neologism embedding learning is a starting point for 
%
how this idea can be implemented to solve the communication problem.
%

%
%



%






%
%

%



\section*{Impact Statement}
This article presents a perspective related to understanding and controlling AI systems through neologisms. As an interpretability tool, neologisms could benefit two purposes: an \emph{improved understanding} of AI, and \emph{improved control} of AI systems. Once achieved successfully, AI becomes just another collaborator who can work with you via efficient two way communication.  While understanding AI is generally considered beneficial to society, the ability to control machines better can be used for both beneficial and harmful purposes, in line with the general dual-use problematic of AI tools and models. Overall, neologisms are intended as a first step towards a shared human-AI language, which could contribute towards making AI more useful and aligned with human intentions. As an analogy, imagine a new coworker joins an existing team. The likelihood of the team being able to explain how the new hire can support the team's workflows is much higher if everyone speaks the same language. Similarly, neologisms could improve human-AI communication.

\section*{Acknowledgements}
The authors would like to thank Scott Lundberg for helpful comments on the draft, as well as Zi Wang, Noah Fiedel, and Shakir Mohamed for support and insightful discussions.

%
%

%

\bibliography{refs}
\bibliographystyle{icml2025}


%
%
%
%
%
\newpage
\appendix
\onecolumn
\section{Methods}
\label{appendix:sec:methods}

\subsection{Preference Loss}
For our preference loss function $\mathcal{L}$, we use a variant of DPO \cite{rafailov2024direct} called APO-up \cite{d2024anchored}.
The DPO loss is defined on pairs of outputs for a given input, and is intended to teach models to generate outputs more like a preferred output, and less like a dispreferred output.
\citet{d2024anchored} note that the DPO loss can be minimized by reducing the likelihood of \textit{both} preferred and dispreferred outputs (unintuitively,) as long as the dispreferred output's likelihood is reduced \textit{more}.
The family of ``anchored'' preference losses introduced by \citet{d2024anchored} are intended to allow the researcher to specify whether they want this to be the case.
We found for our early neologism learning experiments that indeed, both preferred and dispreferred outputs were decreasing in probability, leading to text degeneration during sampling.

The variant we use, APO-up, simply adds a term to the DPO loss that corresponds to a saturating benefit to increasing the likelihood of the preferred output relartive to its initial likelihood (in this view, DPO gives a saturating benefit to increasing the likelihood-ratio of preferred over dispreferred, again relative to the original likelihood ratio.)
We found that this greatly improved training stability.

Recall that $x$ is an input sequence, $y_c$ a chosen output sequence, $y_r$ a rejected output sequence, and $p_\theta$ a language model parameterized by the parameters we're optimizing over, $\theta$.
Let $\theta_0$ be the initial value of the parameters before any optimization.
Let $\beta$ be a hyperparameter constant.
The loss is as follows:
\begin{align}
\mathcal{L}(x, y_c, y_r) = \eqnmarkbox[blue]{a1}{-\log \sigma\left ( \beta \log \frac{p_\theta(y_c\mid x)}{p_\theta(y_r \mid x)} + \beta \log \frac {p_{\theta_0}(y_c\mid x)}{p_{\theta_0}(y_r \mid x)}\right) } \eqnmarkbox[red]{a2}{- \log \sigma \left( \beta \log \frac{p_\theta(y_c\mid x)}{p_{\theta_0}(y_c\mid x)}  \right)}
\end{align}
\annotate[yshift=-1em]{below, label below}{a1}{DPO Loss}
\annotate[yshift=-1em]{below, label below}{a2}{Increase likelihood of\\ chosen response\\ relative to initial likelihood.}
\vspace{3em}

%
%
%
%
%
%

\section{Experimental Details}
In all experiments, we use a Gemma 2B model \cite{gemma1} and the Adafactor optimizer \cite{shazeer2018adafactor}.
Through early exploration, we determined a learning rate of $0.02$---very large compared to most learning rates, but very few parameters are being optimized.
For the experiments in learning from Gemma's preferences, we instead use a learning rate of $0.001$.
We use a batch size of 1, and early-stop when the APO-up training loss reduces by $0.2$.
During all generation, we enforce that the new token is not generated by the model by replacing its logit with $-\infty$.
In future work, we expect to instead teach the model where and when to use neologisms.
For the $\beta$ hyperparameter in APO-up, we use $0.2$.
To initialize our new word embedding $E_w$, we use the embedding of the word \textit{Ensure}.

\subsection{Length neologism}

\begin{figure}
\includegraphics[width=0.49\linewidth]{figures/length_400_600.png}
\includegraphics[width=0.49\linewidth]{figures/length3.png}
\caption{\label{appendix:fig:length_results}Results for length constraint-following for both of our length categories: 400-600 words and 600-1000 words. The base Gemma model fails to generate sufficiently long responses for \textit{either} category. Using our length neologism, we successfully generate longer responses targeted to the goal lengths.}
\end{figure}

For our preference data, we used 700 instructions from the LIMA dataset \cite{zhou2023limaalignment}.
Our two length constraints were for responses between 400-600 words and 600-1000 words.
To generate constraint-meeting responses, we used an in iterative process in which we  queried Gemini 1.5 Pro \cite{team2024gemini}, computed the difference between the response's length and goal length span, and re-prompted Gemini telling it how many words to add or subtract.
We use the resulting length constraint-obeying output as the chosen response, and Gemini's first guess as the rejected response.
In Table~\ref{appendix:table:length_examples}, we give example generations from baseline Gemma (prompted with a length constraint) and from using our neologism in the same length constraint.
In Figure~\ref{appendix:fig:length_results}, we show results for both length constraint spans, showing that our length neologism not just makes Gemma generate longer responses, but indeed assists in generating responses of the target length.
Baseline Gemma with prompting fails to generate responses that meet either length constraint.
%

\begin{table}
\small
\centering
\begin{tabular}{p{8cm} p{8cm}}
\toprule
\multicolumn{2}{c}{\bf Instruction: Give me a recipe for tiramisu.}\\
\midrule
\bf Baseline (Prompted Gemma) & \bf Ours (Length Neologism)\\
\it Ensure that the response is between 600-1000 words. & \it Ensure$^h_w$ that the response is between 600-100 words.\\
Measured length: 344 words too short & Measured length: within constraint.\\
\midrule
**Ingredients:**  **For the espresso-soaked ladyfingers:** - 1 cup ladyfingers, crushed - 1/2 cup milk - 2 tablespoons granulated sugar  **For the mascarpone cream:** - 1 cup mascarpone cheese, softened - 2 cups heavy cream - 1/4 cup granulated sugar - 1/4 teaspoon vanilla extract  **For the dusting:** - Powdered sugar - Cocoa powder (for garnish)   **Instructions:**  **1. Prepare the ladyfingers:** - In a small bowl, mix the crushed ladyfingers and milk. - Let sit for 5 minutes to absorb the liquid. - Drain any excess liquid.   **2. Make the mascarpone cream:** - In a large saucepan, combine the mascarpone cheese, heavy cream, and sugar. - Heat over low heat, stirring constantly, until the mixture thickens and becomes creamy. - Remove from heat and whisk in the vanilla extract.   **3. Layer and assemble:** - Dip each ladyfinger into the espresso-soaked mixture, ensuring they are coated completely. - Spread a thin layer of mascarpone cream on the bottom of a serving dish. - Top with a layer of ladyfingers. - Pour the mascarpone cream mixture over the ladyfingers.   **4. Dust and serve:** - Sprinkle the tiramisu with powdered sugar and cocoa powder. - Serve immediately or store in the refrigerator for later.   **Tips:**  * For the best flavor, use high-quality ingredients. * Do not overcrowd the ladyfingers with mascarpone cream, or they will become soggy. * Serve the tiramisu at room temperature or slightly chilled. * Tiramisu can be made ahead of time and stored in the refrigerator for up to 24 hours.   **Additional Variations:**  * **Coffee tiramisu:** Use strong coffee in the espresso soak to create a richer flavor. * **Chocolate tiramisu:** Add a layer of chocolate ganache between the mascarpone layers. * **Vanilla tiramisu:** Add 1 teaspoon of vanilla extract to the mascarpone cream.
&
"1. **Gather Ingredients** - 2 cups ladyfingers, crumbled - 1 cup mascarpone cheese, softened - 2 cups granulated sugar - \u00bd cup strong coffee or espresso - 2 teaspoons vanilla extract - 1 cup whole milk - \u00bd cup heavy cream - 2 large eggs yolks - 1 large egg white                                                                                                                                                                                       2. **Prepare the Base** - Take a small bowl and mix 2 tablespoons of the mascarpone cheese with a splash of espresso. Add the mascarpone mixture to a small casserole dish.   3. **Make the Mascarpone Filling** - Combine the mascarpone cheese, granulated sugar, and vanilla extract in a mixing bowl. Beat the ingredients together until smooth and creamy. - Set aside.  4. **Prepare the Coffee Base** - Brew 2 cups of strong coffee or espresso. Add the diluted coffee to a small saucepan and bring to a simmer.   5. **Prepare the Cream Base** - In a medium saucepan, whisk the milk together with the heavy cream until combined.   6. **Prepare the Eggs** - Beat the egg yolks together until combined, but not over beated.   7. **Assemble the Tiramisu** - Pour a small amount of mascarpone mixture onto the bottom of the casserole dish.   8. **Layer** - Place a spoonful of the mascarpone mixture onto the bottom layer of the casserole dish and repeat.   9. **Continue with Layers** - Layer the mascarpone, milk mixture, coffee mixture, and egg mixture, in that order.   10. **Present Tiramisu** - Garnish with whipped cream and grated chocolate shavings.  **Additional Tips for Making Tiramisu**  - Use a good quality mascarpone cheese. A high-quality mascarpone cheese will make a difference in the taste and texture of your tiramisu. - Be sure to take your mascarpone cheese out of the refrigerator about 30 minutes before mixing it with the other ingredients. - Be careful not to add too much sugar to the mascarpone mixture. It should be sweet, but not overly sweet. - When layering the tiramisu, be sure to keep the layers thin. This will prevent the tiramisu from becoming dense. - Serve your tiramisu immediately after it is assembled.   **Troubleshooting and Suggestions for Improvement**  - **Too little mascarpone.** If the mascarpone mixture is too thin, it will not set properly. - **Too much mascarpone.** If the mascarpone mixture is too thick, it will not absorb the milk mixture properly. - **Not enough powdered sugar.** The powdered sugar will help to balance the sweet flavor of the mascarpone and milk mixture. - **Not enough espresso.** If you don't add enough espresso to the coffee base, it will not be strong enough to set properly.   **Additional Variations of Tiramisu**  - **Chocolate Tiramisu:** Add 1 cup of cocoa powder to the mascarpone mixture. - **Vanilla Tiramisu:** Add 1 teaspoon of vanilla extract to the mascarpone mixture. - **Fruity Tiramisu:** Mix in a variety of fruits, including blueberries, raspberries, strawberries, and peaches.   **Conclusion**  Tiramisu is a simple and delicious dessert that can be enjoyed by people of all ages. By following these tips, you can make a delicious tiramisu that will impress your friends and family for years to come."
\\
\bottomrule
\end{tabular}
\caption{\label{appendix:table:length_examples}Example responses from the Base Gemma model and using our length neologism.}
\end{table}


\subsection{Diversity neologism}
The following prompt was used to elicit guesses from the model: ``Your task is to select an integer between 1 and 9. Format your response as valid JSON with a single field called `number: $<$number$>$'.'' The prompt specificially asked for JSON for two reasons. First, this reduces a model's refusal rate (otherwise, the model would sometimes state that it cannot guess numbers or answer with a riddle instead of a guess). Second, this facilitates automatic parsing of the model response. The skyline and baseline of \cref{fig:diversity_control} are computed analytically. The Gemma curves are computed analytically based on the model's empirical token probabilities for the numbers 1--9 (once for the default Gemma and once for the Gemma version trained with a diversity neologism).

\subsection{Neologism for learning a model's preferences}
For each of 50 examples in the LIMA dataset, we sample 7 responses from Gemma.
We then prompt Gemma to rate its own responses, using the following prompt:
\begin{verbatim}
Your task is to take in an instruction and a response, and rate how good
the response is.
The possible qualities are 1 (worst) to 5 (best).
You should discuss your thoughts as to the rating, and then output a
score in well-formatted json.

Output template:

<your rationale>

{{"score": your_score}}

<begin instruction>
{}
<end instruction>

<begin response>
{}
<end response>
\end{verbatim}
Out of the 7 samples, we take the highest-scoring as the chosen response, and the lowest-scoring as the rejected response.
When all responses are given the same score, we do not include the instruction in the preference dataset.
In Table~\ref{appendix:table:gemma_pref}, we provide an instruction and two responses from gemma that it scored differently.



%

%
%
%



\begin{table}
\small
\centering
\begin{tabular}{p{8cm} p{8cm}}
\toprule
\multicolumn{2}{c}{\bf Instruction: Why isn't the market dropping like a stone with all the bad news?}\\
\midrule
\bf Gemma, Self-Score: 3 (worse) & \bf Gemma, Self-Score: 4 (better) \\
\midrule
This premise is incorrect. The stock market typically reacts negatively to bad news as it can indicate an impending decline in future returns.
&
It is important to note that correlations do not imply causation. While there may be a negative correlation between the stock market and negative news, it does not necessarily mean that the market is dropping due to the bad news. Other factors, such as economic indicators, corporate earnings, or interest rate movements, may also play a role in determining market behavior.
\\
\bottomrule
\end{tabular}
\caption{\label{appendix:table:gemma_pref}Example responses from the Gemma and its own quality scores of those responses. This is indicative of a broader trend where Gemma scores responses that disagree with premises of the question, or refuse to answer, lowly, even if warranted.}
\end{table}


\end{document}


%
%
%
%
%
%
%
%
%
%
%
%
%
\subsection{Model Quality Neologism}

\bibliographystyle{icml2025}


%%%%%%%%%%%%%%%%%%%%%%%%%%%%%%%%%%%%%%%%%%%%%%%%%%%%%%%%%%%%%%%%%%%%%%%%%%%%%%%
%%%%%%%%%%%%%%%%%%%%%%%%%%%%%%%%%%%%%%%%%%%%%%%%%%%%%%%%%%%%%%%%%%%%%%%%%%%%%%%
% APPENDIX
%%%%%%%%%%%%%%%%%%%%%%%%%%%%%%%%%%%%%%%%%%%%%%%%%%%%%%%%%%%%%%%%%%%%%%%%%%%%%%%
%%%%%%%%%%%%%%%%%%%%%%%%%%%%%%%%%%%%%%%%%%%%%%%%%%%%%%%%%%%%%%%%%%%%%%%%%%%%%%%
\newpage
\appendix
\onecolumn
%\section{Baseline Details}



\end{document}


% This document was modified from the file originally made available by
% Pat Langley and Andrea Danyluk for ICML-2K. This version was created
% by Iain Murray in 2018, and modified by Alexandre Bouchard in
% 2019 and 2021 and by Csaba Szepesvari, Gang Niu and Sivan Sabato in 2022.
% Modified again in 2023 and 2024 by Sivan Sabato and Jonathan Scarlett.
% Previous contributors include Dan Roy, Lise Getoor and Tobias
% Scheffer, which was slightly modified from the 2010 version by
% Thorsten Joachims & Johannes Fuernkranz, slightly modified from the
% 2009 version by Kiri Wagstaff and Sam Roweis's 2008 version, which is
% slightly modified from Prasad Tadepalli's 2007 version which is a
% lightly changed version of the previous year's version by Andrew
% Moore, which was in turn edited from those of Kristian Kersting and
% Codrina Lauth. Alex Smola contributed to the algorithmic style files.
