% !TEX options = --shell-escape
%\documentclass[10pt,twocolumn]{IEEEtran}
% \documentclass{ieeeaccess}
\documentclass[conference]{IEEEtran}
\usepackage{cite}
\usepackage{amsmath,amssymb,amsfonts}
\usepackage{algorithmic}
\usepackage{graphicx}
\usepackage{textcomp}
% \usepackage{caption}
\usepackage{subfigure}
% \usepackage{footnote}
\usepackage{amssymb}
\usepackage[table]{xcolor}
\usepackage{pifont}
\usepackage{makecell}
\usepackage{epstopdf}
\def\BibTeX{{\rm B\kern-.05em{\sc i\kern-.025em b}\kern-.08em
    T\kern-.1667em\lower.7ex\hbox{E}\kern-.125emX}}
\AtBeginDocument{\definecolor{ojcolor}{cmyk}{0.93,0.59,0.15,0.02}}
\def\OJlogo{\vspace{-12pt}\includegraphics[height=24pt]{ojvt-logo.png}}

\begin{document}
% \receiveddate{XX Month, XXXX}
% \reviseddate{XX Month, XXXX}
% \accepteddate{XX Month, XXXX}
% \publisheddate{XX Month, XXXX}
% \currentdate{27 June, 2024}
% \doiinfo{OJVT.2024.0627000}


% \title{Modeling and Predicting Coverage and MIMO Channel Rank for NextG Aerial Vehicle Networks with Foliage}
\title{Analysis and Prediction of Coverage and Channel Rank for UAV Networks in Rural Scenarios with Foliage}
%: From Ray Tracing Simulations to Measurements}
%\title{Ray Tracing-Based Link Analysis and Kriging Interpolation-Based 3D Channel Rank Interpolation Scheme for UAV Channels in Rural Scenario with Trees}
\author{Donggu Lee$^1$, Ozgur Ozdemir$^1$, Asokan Ram$^2$, and Ismail Guvenc$^1$
\\ $^1$Department of Electrical and Computer Engineering, North Carolina State University, Raleigh, NC, USA 
% (e-mail: \{dlee42, oozdemi, iguvenc\} @ncsu.edu )
\\ $^2$Wireless Research Center of North Carolina, Wake Forest, NC, USA
\\ E-mail: \{dlee42, oozdemi, iguvenc\}@ncsu.edu,
asokan.ram@wrc-nc.org
}
% \affil{Department of Electrical and Computer Engineering, North Carolina State University, Raleigh, NC, USA 

% (e-mail: \{dlee42, oozdemi, iguvenc\} @ncsu.edu )}
% \affil{Wireless Research Center of North Carolina, Wake Forest, NC, USA (e-mail: asokan.ram@wrc-nc.org)}

% \tfootnote{This paragraph of the first footnote will contain support 
% information, including sponsor and financial support acknowledgment. For 
% example, ``This work was supported in part by the U.S. Department of 
% Commerce under Grant BS123456.''}

% \markboth
% {Author \headeretal: Preparation of Papers for IEEE TRANSACTIONS and JOURNALS}
% {Author \headeretal: Preparation of Papers for IEEE TRANSACTIONS and JOURNALS}

% \corresp{Corresponding author: Ismail Guvenc (e-mail: iguvenc@ncsu.edu).}

% \authornote{This work is supported in part by the NSF awards CNS-1939334 and CNS-2332834.}


\maketitle

\begin{abstract} 
% Need to be edited at the end
Unmanned aerial vehicles (UAVs) are expected to play a key role in 6G-enabled vehicular-to-everything (V2X) communications requiring high data rates, low latency, and reliable connectivity for mission-critical applications. Multi-input multi-output (MIMO) technology is essential for meeting these demands. However, UAV link performance is significantly affected by environmental factors such as signal attenuation, multipath propagation, and blockage from obstacles, particularly dense foliage in rural areas. In this paper, we investigate RF coverage and channel rank over UAV channels in foliage-dominated rural environments using ray tracing (RT) simulations. We conduct RT-based channel rank and RF coverage analysis over Lake Wheeler Field Labs at NC State University to examine the impact on UAV links. Custom-modeled trees are integrated into the RT simulations using NVIDIA Sionna, Blender, and Open Street Map (OSM) database to capture realistic blockage effects. Results indicate that tree-induced blockage impacts RF coverage and channel rank at lower UAV altitudes. We also propose a Kriging interpolation-based 3D channel rank interpolation scheme, leveraging the observed spatial correlation of channel rank in the given environments. The accuracy of the proposed scheme is evaluated using the mean absolute error (MAE) metric and compared against baseline interpolation methods. Finally, we compare the RT-based received signal strength (RSS) and channel rank results with real-world measurements from the NSF AERPAW testbed demonstrating reasonable consistency between simulation results and the measurements.
\end{abstract}

\begin{IEEEkeywords}
AERPAW, channel rank, drone, Kriging interpolation, MIMO, ray tracing, RF coverage, Sionna, spatial correlation, UAV, V2X.
\end{IEEEkeywords}

% \titlepgskip=-15pt

\section{Introduction}
\label{sec:introduction}
Unmanned aerial vehicles (UAVs) are expected to play a key role in 6G-enabled networks, especially in the context of vehicular-to-everything (V2X) communication. Using UAVs in 6G networks has many promising applications such as surveillance, infrastructure inspections, search-and-rescue missions, and among others \cite{access_UAV_application, proc_ieee_UAV_application}. High data rate, low latency, and reliable connectivity are critical requirements to support mission-oriented applications in V2X communications, which need to be specifically studied for UAV scenarios.

UAVs provide the advantages of forming flexible coverage, operating in challenging terrains, and handling dynamic network loads. To achieve high data rate and reliability requirements, multi-input multi-output (MIMO) technology is essential in UAV networks. Spatial multiplexing in MIMO systems enhances data rates by utilizing parallel data stream transmissions without allocating additional wireless communication resources. Various performance metrics such as singular value spread, condition number, and correlation matrix distance can be considered for evaluating MIMO systems \cite{MIMO_performance_metric}. Especially, channel rank is a key metric in MIMO system design as it quantifies how many parallel spatial streams can be supported in a given MIMO link \cite{MIMO_white_paper, MIMO_adaptation_1, MIMO_adaptation_2}. However, analyzing channel rank and coverage for UAV links involves unique challenges such as propagation modeling in unfavorable terrains, blockage from obstacles, and prediction of coverage and rank at locations where no past data is available. 

To our knowledge, ray tracing (RT)-based UAV coverage analysis and Kriging interpolation-based 3D channel rank interpolation for UAV channels in foliage-dominated rural scenarios are not available in the existing literature. In this paper, we substantially extend our earlier related work in \cite{previous_work} with a wider target area and multiple base station settings. The main contributions of this paper can be summarized as follows. 

\textbf{Foliage Modeling and Blockage Analysis.} Blockage effects in an environment can severely impact network reliability \cite{ray_tracing_foliage}. In rural areas, dense foliage and buildings can obstruct propagation, leading to signal degradation and deviations in UAV communication scenarios. To accurately model these effects, we introduce custom-modeled trees in the target area using NVIDIA’s RT tool Sionna~\cite{nvidia_sionna}, the 3D modeling tool Blender~\cite{blender}, and the Open Street Map (OSM) database~\cite{osm}. This approach allows for realistic simulation of signal blockage, attenuation, and reflection due to environmental obstacles.

\textbf{RF Coverage and Channel Rank Analysis.} Understanding RF coverage and channel rank is fundamental for UAV link analysis in rural environments. RF coverage defines the effective operational range of UAVs, ensuring consistent connectivity. Meanwhile, channel rank determines the number of parallel spatial streams available for spatial multiplexing directly influencing link capacity. Using extensive RT simulations, we analyze the impact of foliage and terrain on RF coverage and channel rank at varying UAV altitudes. The RT tool enables precise evaluation of propagation including beam patterns, angular resolution, and multipath components, providing a detailed representation of UAV communication in complex environments \cite{RT_tutorial}.

\textbf{Kriging-Based 3D Channel Rank Interpolation.} We introduce a Kriging interpolation-based 3D channel rank interpolation scheme. This method leverages the spatial correlation of channel rank incorporating both horizontal and vertical variations. The interpolation process consists of: 1) utilizing channel ranks across different UAV altitudes to compute spatial correlation and semi-variograms for Kriging interpolation; and 2) integrating channel rank data within a defined horizontal sampling radius from target UAV locations. We evaluate the accuracy of the proposed Kriging-based interpolation using the mean absolute error (MAE) metric and compare it against two baseline interpolation methods.

\textbf{Validation with Real-World Measurements.} To ensure the practical applicability of our RT-based simulations, we compare the RT simulation results of RF coverage and channel rank with real-world measurements collected in the NSF AERPAW testbed~\cite{aerpawWebsite}. Our goal is to assess how well RT simulations capture real-world conditions in the given environments. The results show that RT-based modeling provides reasonable approximations of UAV communication performance in foliage-dense rural environments.


The rest of this paper is organized as follows. Relevant works from the existing literature are discussed in Section~\ref{ch:related_works}. The description of the system model is given in Section~\ref{ch:system_model}. The RT and measurement scenarios are introduced in Section~\ref{ch:RT_setup} and Section~\ref{ch:measurement_setup}, respectively. The proposed Kriging-based 3D channel rank interpolation scheme and baseline interpolation schemes are described in Section~\ref{ch:Kriging_3D_Channel_interpolation}. 
Simulation results for RF coverage, channel rank, Kriging-based 3D channel rank interpolation, and comparison of measurements and RT simulations are provided in Section~\ref{ch:numerical_results}. Finally, the last section concludes the paper.


\begin{table*}[t!]
    \centering
    \caption{Literature review on RT-based RF coverage and channel rank analysis, and Kriging interpolation scheme.}
    \begin{tabular}{|c|c|c|c|c|c|c|c|}
    \hline
        \textbf{Ref.} & \textbf{Analysis Objectives} & \makecell{\textbf{UAV} \\ \textbf{Channels}}  & \makecell{\textbf{Ray} \\ \textbf{Tracing}}  & \makecell{\textbf{Channel} \\ \textbf{Rank} \\ \textbf{Analysis}} & \makecell{\textbf{Kriging} \\ \textbf{Interpolation}} & \textbf{Trees} & \textbf{Measurements} \\ \hline
        
        \cite{duke_sionna_paper} & RT-based RF signal mapping and ML integration & \cellcolor{red!25} \ding{55} & \cellcolor{green!25} \ding{51} & \cellcolor{red!25} \ding{55} & \cellcolor{red!25} \ding{55} & \cellcolor{red!25} \ding{55} & \cellcolor{green!25} \ding{51} \\ \hline  
        
        \cite{boston_sionna} & RT-based digital twin for urban scenarios & \cellcolor{red!25} \ding{55} & \cellcolor{green!25} \ding{51} & \cellcolor{red!25} \ding{55} & \cellcolor{red!25} \ding{55} & \cellcolor{red!25} \ding{55} & \cellcolor{red!25} \ding{55} \\ \hline  

        \cite{RT_mode_urban} & RT-based channel analysis in urban scenarios & \cellcolor{red!25} \ding{55} & \cellcolor{green!25} \ding{51} & \cellcolor{red!25} \ding{55} & \cellcolor{red!25} \ding{55} & \cellcolor{red!25} \ding{55} & \cellcolor{red!25} \ding{55} \\ \hline

        \cite{uav_channel_model_urban} & UAV channel modeling with measurement
        & \cellcolor{green!25} \ding{51} & \cellcolor{red!25} \ding{55} & \cellcolor{red!25} \ding{55} & \cellcolor{red!25} \ding{55} & \cellcolor{red!25} \ding{55} &  \cellcolor{green!25} \ding{51} \\ \hline

        \cite{uav_channel_model_arxiv} & Stochastic channel model for urban UAV scenarios & \cellcolor{green!25} \ding{51} & \cellcolor{red!25} \ding{55} & \cellcolor{red!25} \ding{55} & \cellcolor{red!25} \ding{55} & \cellcolor{red!25} \ding{55} & \cellcolor{red!25} \ding{55} \\ \hline
        
        \cite{wahab_paper} & UAV channel modeling with foliage & \cellcolor{green!25} \ding{51} & \cellcolor{red!25} \ding{55} & \cellcolor{red!25} \ding{55} & \cellcolor{red!25} \ding{55} & \cellcolor{green!25} \ding{51} & \cellcolor{green!25} \ding{51} \\ \hline      
        \cite{Channel_rank_massive_MIMO} & Channel rank analysis for massive MIMO & \cellcolor{red!25} \ding{55} & \cellcolor{green!25} \ding{51} & \cellcolor{green!25} \ding{51} & \cellcolor{red!25} \ding{55} & \cellcolor{red!25} \ding{55} & \cellcolor{green!25} \ding{51}  \\ \hline     
        \cite{RT_channel_modeling, RT_terahz_UAV} & RT-based UAV channel modeling & \cellcolor{green!25} \ding{51} & \cellcolor{green!25} \ding{51} & \cellcolor{red!25} \ding{55} & \cellcolor{red!25} \ding{55} & \cellcolor{red!25} \ding{55} & \cellcolor{green!25} \ding{51}  \\ \hline

        \cite{rank_irs} & Reflector-based channel rank improvement & \cellcolor{green!25} \ding{51} & \cellcolor{red!25} \ding{55} & \cellcolor{green!25} \ding{51} & \cellcolor{red!25} \ding{55} & \cellcolor{red!25} \ding{55} & \cellcolor{red!25} \ding{55} \\ \hline
        
        \cite{rank_MIMO_bounds} & Theoretical channel rank analysis & \cellcolor{red!25} \ding{55} & \cellcolor{red!25} \ding{55} & \cellcolor{green!25} \ding{51} & \cellcolor{red!25} \ding{55} & \cellcolor{red!25} \ding{55} & \cellcolor{red!25} \ding{55} \\ \hline
        \cite{vertical_MIMO_paper, vertical_MIMO_paper_2} & Channel rank analysis for antenna separation &  \cellcolor{red!25} \ding{55} &  \cellcolor{red!25} \ding{55} & \cellcolor{green!25} \ding{51} & \cellcolor{red!25} \ding{55} & \cellcolor{red!25} \ding{55} & \cellcolor{green!25} \ding{51} \\ \hline
        
        \cite{Kriging_coverage, fixed_rank_kriging} & Kriging-based coverage estimation & \cellcolor{red!25} \ding{55} & \cellcolor{red!25} \ding{55} & \cellcolor{red!25} \ding{55} & \cellcolor{green!25} \ding{51} & \cellcolor{red!25} \ding{55} & \cellcolor{green!25} \ding{51} \\ \hline
        \cite{Kriging_map_construction} & Kriging interpolation for radio map construction & \cellcolor{red!25} \ding{55} & \cellcolor{red!25} \ding{55} & \cellcolor{red!25} \ding{55} & \cellcolor{green!25} \ding{51} & \cellcolor{red!25} \ding{55} & \cellcolor{green!25} \ding{51} \\ \hline
        
        \cite{Kriging_mapping} & Coverage analysis using Kriging interpolation & \cellcolor{red!25} \ding{55} & \cellcolor{red!25} \ding{55} & \cellcolor{red!25} \ding{55} & \cellcolor{green!25} \ding{51} & \cellcolor{red!25} \ding{55} & \cellcolor{green!25} \ding{51} \\ \hline  
        
        \cite{previous_work} & RT-based UAV link analysis & \cellcolor{green!25} \ding{51} & \cellcolor{green!25} \ding{51} & \cellcolor{green!25} \ding{51} & \cellcolor{red!25} \ding{55} & \cellcolor{red!25} \ding{55} & \cellcolor{red!25} \ding{55} \\ \hline
        \cite{ray_tracing_foliage} & RT-based channel measurement on trees  & \cellcolor{red!25} \ding{55} & \cellcolor{green!25} \ding{51} & \cellcolor{red!25} \ding{55} & \cellcolor{red!25} \ding{55} & \cellcolor{green!25} \ding{51} & \cellcolor{green!25} \ding{51} \\ \hline 
        \cite{ray_tracing_tree_urban} & Impact of foilage in mmWave urban channels & \cellcolor{red!25} \ding{55} & \cellcolor{green!25} \ding{51} & \cellcolor{red!25} \ding{55} & \cellcolor{red!25} \ding{55} & \cellcolor{green!25} \ding{51} & \cellcolor{red!25} \ding{55} \\ \hline
        \cite{ray_tracing_vegetation} & RT-based propagation modeling for vegetation & \cellcolor{red!25} \ding{55} & \cellcolor{green!25} \ding{51} & \cellcolor{red!25} \ding{55} & \cellcolor{red!25} \ding{55} & \cellcolor{green!25} \ding{51} & \cellcolor{green!25} \ding{51} \\ \hline 
       
        \cite{openGERT_paper} & RT-based propagation modeling for digital twins & \cellcolor{red!25} \ding{55} & \cellcolor{green!25} \ding{51} &
        \cellcolor{red!25} \ding{55} & \cellcolor{red!25} \ding{55} & \cellcolor{red!25} \ding{55} & \cellcolor{red!25} \ding{55} \\ \hline
       
        \cite{maeng2023kriging} & 3D radio map generation using Kriging interpolation & \cellcolor{green!25} \ding{51} & \cellcolor{red!25} \ding{55} & \cellcolor{red!25} \ding{55} & \cellcolor{green!25} \ding{51} & \cellcolor{red!25} \ding{55} & \cellcolor{green!25} \ding{51} \\ \hline
        
        This work & RT-based link analysis and Kriging interpolation & \cellcolor{green!25} \ding{51} & \cellcolor{green!25} \ding{51} & \cellcolor{green!25} \ding{51} & \cellcolor{green!25} \ding{51} & \cellcolor{green!25} \ding{51} & \cellcolor{green!25} \ding{51} \\ \hline     
    \end{tabular}
    
    \label{tab:literature_review}
\end{table*}

\section{Related Works}\label{ch:related_works}
There are limited studies investigating channel rank and coverage for UAV links in the literature with RT studies or measurements. Our literature review with representative publications related to our work and the differences with the present work is summarized in Table \ref{tab:literature_review}.

\subsection{UAV Channels}
In \cite{RT_channel_modeling} and \cite{RT_terahz_UAV}, RT-based UAV channel modeling has been investigated. The altitude-dependent channel parameters, such as path loss, power delay profile, and angular information, are simulated to characterize the UAV channel. The developed channel model consisted of a deterministic line-of-sight (LoS) path and ground reflection, and stochastic non-LoS (NLoS) components. The developed model has been validated using RT simulations and measurements.  

A stochastic geometry model is developed in \cite{uav_channel_model_arxiv} for investigating coverage and outage probability in an urban environment. The study investigates UAV backhaul links considering building density, antenna beamwidth, and interference effects. The results highlight the existence of an optimal UAV altitude that maximizes coverage while ensuring reliable backhaul connectivity. Similarly, in \cite{uav_channel_model_urban}, measurement-based UAV channel modeling is conducted in an urban scenario. The study extracts key parameters for path loss models and develops a simplified Saleh-Valenzuela channel model for UAV-ground links.

In our previous work~\cite{previous_work}, UAV channel characteristics are analyzed in rural and urban scenarios. Specifically, RF coverage, channel rank, and condition number distribution have been analyzed for UAV networks using Matlab-based RT simulations. Centennial Campus and Lake Wheeler Field Labs at NC State University are considered as urban and rural scenarios, respectively. Due to the blockage from the buildings, an outage can be observed at the lower altitude of the UAV. Moreover, the probability of having a channel rank of 3 or 4 is getting lower because of the LoS dominant path to the UAV as the altitude of the UAV increases.


\subsection{Ray Tracing}
% In \cite{RT_indoor}, 
In \cite{duke_sionna_paper}, an RF signal mapping scheme for cellular networks has been investigated by integrating RT simulation using Sionna with geographic databases and machine learning (ML) techniques. A cascaded neural network refines signal strength predictions by leveraging geographical information from RT and sparse real-world measurements. The proposed method improves computational efficiency while maintaining high accuracy, outperforming conventional RT-based approaches in real-world evaluations.

In \cite{boston_sionna} and \cite{RT_mode_urban}, RT-based RF coverage and channel capacity analysis have been studied for urban scenarios. Nodetably in \cite{boston_sionna}, a large-scale digital twin framework for downtown Boston has been developed. The proposed BostonTwin framework integrates a high-fidelity 3D model of the downtown Boston area with geographical data, enabling RT simulations. By leveraging Sionna, the framework facilitates large-scale coverage mapping and signal-to-noise ratio (SNR) evaluations. Moreover, the authors highlight the areas where BostonTwin meets the requirements for 6G use cases. On the other hand, ~\cite{RT_mode_urban} uses site maps and a RT software for computing MIMO gain matrices for a given mobile station position in urban areas of Boston and Manhattan. Subsequently, authors use the MIMO gain matrices to determine achievable rates for MIMO transmission modes including spatial multiplexing, beamforming, and diversity. The results suggest that the use of site-specific RT for data rate prediction works more effectively than the use of stochastic models.

An automated geometry extraction framework for precise RT simulation is developed in \cite{openGERT_paper}. With the capability of open-source resources from OSM, Microsoft Global ML Building Footprints, and the US Geological Survey (USGS), high-fidelity RT simulation environments can be extracted. The authors conducted sensitivity analyses to investigate the impact on the accuracy of RT simulation with the environmental factors including building heights, locations, and material settings. Channel statistics of path gain, delay spread, and link outage are shown with the developed framework. 


\subsection{Channel Rank Analysis}
In \cite{Channel_rank_massive_MIMO}, the channel rank for an outdoor-to-indoor massive MIMO system has been analyzed. The angular characteristics in terms of the angle of arrival and departure have been investigated for the given wireless communication environment. Then, singular values of each channel rank have been simulated and measured. It is observed that the singular values decrease rapidly as the range of eigenvalue to be captured increases. 

Theoretical lower and upper bounds of channel capacity for high-rank MIMO systems are studied in \cite{rank_MIMO_bounds}. The study investigates how LoS propagation affects channel rank and capacity. It shows that strong LoS components increase channel correlation, reducing the efficiency of spatial multiplexing. However, it also demonstrates that optimizing antenna placement and spacing helps maintain orthogonality, preserve channel rank, and improve system performance.

In \cite{rank_irs}, the use of passive intelligent reflecting surfaces (IRSs) has been investigated to improve channel rank and spatial multiplexing in urban UAV scenarios. This study proposes an IRS placement optimization framework to maximize the average channel capacity along predefined UAV trajectories. Numerical results at various carrier frequencies demonstrate that IRS-assisted channels significantly improve MIMO capacity compared to LoS-dominant UAV channels. The authors introduce the potential of IRSs as a cost-effective solution for enhancing UAV communications in dense urban environments. Other work on studying channel rank includes \cite{vertical_MIMO_paper} and \cite{vertical_MIMO_paper_2}, which use measurements for an outdoor LTE MIMO network to explore how the channel rank varies for various scenarios. 

% \cite{rank_analysis_radar}. 


\subsection{Kriging Interpolation}
Coverage analysis and map construction schemes using spatial interpolation for mobile systems have been investigated in \cite{Kriging_coverage, fixed_rank_kriging, Kriging_mapping, Kriging_map_construction}. Especially in \cite{Kriging_coverage}, a radio map has been developed by interpolating geo-located measurements. A fixed-rank Kriging interpolation scheme has been used to generate the radio map to reduce computational complexity. The simulation results show the trade-off between computational complexity and accuracy in coverage estimation over rural scenarios. 
In \cite{maeng2023kriging}, a Kriging interpolation-based 3D radio map generation scheme for radio dynamic zones is proposed. A realistic propagation model is developed by using measurements over the air-to-ground link in terms of path loss, shadowing, and spatial correlation. Horizontal and vertical correlations of received signal strength (RSS) are used for accurate radio map generation. The generated radio map is more accurate than the model with perfect knowledge of the path loss due to the benefits of spatial correlation. 

\subsection{Trees}
In \cite{ray_tracing_foliage}, an RT-based case study of channel measurements to model propagation effects due to trees is provided. Statistics of angular properties and penetration loss over multiple types of trees have been analyzed to characterize blockage and propagation for mmWave channels. The authors highlighted that the tree-specific propagation model can be used for realistic RT simulation in non-controlled environments. 

An RT-based 3D mmWave propagation model through vegetation is investigated in \cite{ray_tracing_vegetation}. To develop the propagation model, measurements of radiation, angular, and 3D scattering profiles are used in the scenario of groups of trees and various dimensions of the trees. The proposed 3D propagation model is assessed by comparing 3D directional measurements within different positions and angles.  

In \cite{ray_tracing_tree_urban}, the impact of foliage in an urban environment over the mmWave channel is investigated. Here, a simplified urban scenario consists of four buildings in a grid-wise position and four trees at the end of each road between the buildings. A hybrid channel model with the knowledge of a conventional correlation-based channel model is integrated with the RT results to characterize the foliage impact.

The impact of foliage on UAV air-to-ground propagation channels has been investigated using channel-sounding measurements in \cite{wahab_paper} for ultra-wideband signals. The authors examine how tree obstructions affect signal propagation analyzing path loss, multipath fading, and coherence bandwidth. Results show that foliage significantly increases path loss and reduces coherence bandwidth, leading to stronger multipath fading and degraded reliability.

\subsection{Measurements}
% \cite{RT_channel_measure}.
In \cite{wahab_paper}, channel-sounding measurement-based UAV air-to-ground propagation channel modeling has been investigated. The study conducts comprehensive measurements for various UAV communication scenarios and derives statistical models for path loss, multipath propagation, and small-scale fading. The measurements capture high-resolution time and frequency domain characteristics. The collected data is analyzed to extract power delay profiles and coherence bandwidth, which provides insights into UAV channel behavior under different propagation conditions.


Studies in \cite{vertical_MIMO_paper} and \cite{vertical_MIMO_paper_2}, considering terrestrial cellular network scenarios, analyze the impact of antenna configurations on channel rank and system performance using LTE MIMO measurements. In \cite{vertical_MIMO_paper}, increased antenna spacing and optimized polarization, achieved by comparing vertically and horizontally spaced configurations, are shown to improve channel rank and throughput. Similarly, \cite{vertical_MIMO_paper_2} demonstrates that reducing mutual correlation enhances throughput by enabling spatial multiplexing with a higher channel rank. Both studies highlight the critical role of precise measurement campaigns in evaluating channel rank under various conditions.

\subsection{Contributions of This Work}

 Based on Table~\ref{tab:literature_review} and to our knowledge, there are no works that study the effects of foliage on the channel conditions of UAV networks using RT simulations or measurements. Other than \cite{RT_channel_modeling,RT_terahz_UAV}, there are also no works that compare RT simulations with measurements for analyzing coverage for UAV networks.  
In this paper, we extend the scope of our previous work in \cite{previous_work} to address the gaps in the literature. In particular, the target area of the rural scenario has been extended with multiple base stations to understand long-distance propagation effects. Trees in the target area have been included to investigate realistic propagation in the rural area. Moreover, a constant channel rank threshold has been used to investigate channel rank using the ratio of the strongest singular value. Lastly, the spatial correlation, which can be observed in the channel rank distribution over the rural area scenario, has been adopted for the Kriging interpolation-based 3D channel rank interpolation scheme.

     \begin{figure*}[t]
     \centering
     \includegraphics[width=1.6\columnwidth]{images/introduction_figure_v4.png}
     \caption{UAV connectivity scenario in a rural area. UAV coverage can be blocked due to buildings and foliage. It is of interest to predict channel characteristics, such as channel rank and coverage, at a location where no measurements have been collected before, based on measurements at other locations.}     \label{fig:introduction_figure}
 \end{figure*}

 \section{System Model}\label{ch:system_model}
In this section, we provide the system model for the UAV MIMO network and discuss our assumptions for calculating channel rank. The $N_{\rm r}\times 1$ received signal vector $\boldsymbol{y}$ in a MIMO link as in Figure~\ref{fig:introduction_figure} can be expressed as 
\begin{equation}
    \boldsymbol{y} = \boldsymbol{Hx + n},    \label{Eq:MIMO}
\end{equation}
where $\boldsymbol{H}$ is the $N_{\rm r} \times N_{\rm t}$ channel matrix of the MIMO link, $\boldsymbol{x}$ denotes the $N_{\rm t} \times 1$ transmit signal vector, and $\boldsymbol{n}$ is the noise vector having the same dimensions as $\boldsymbol{y}$. 

The rank of the channel matrix in~\eqref{Eq:MIMO} affects the number of parallel data streams that can be transmitted with spatial multiplexing over a given link~\cite{lte_book, MIMO_white_paper}. The channel rank can be obtained from the number of non-zero singular values after singular value decomposition of the channel matrix $\boldsymbol{H}$. This can be expressed as 
\begin{equation}
    \boldsymbol{H} = \boldsymbol{U \Sigma V^{*}},    
\end{equation}
where $\boldsymbol{U}$ represents the $\textit{m} \times \textit{m}$ complex unitary matrix, $\boldsymbol{\Sigma}$ is the $\textit{m} \times \textit{n}$ rectangular diagonal matrix, with diagonal elements $\sigma_i$, with $1<i<\min\{m, n\}$, and $\boldsymbol{V}$ is the $\textit{n} \times \textit{n}$ complex unitary matrix, respectively. The diagonal elements, $\sigma_i$, are sorted in descending order as $i$ increases. The channel rank, $R$, is bounded into the range of $1\leq R\leq \min\{m, n\}$. 

To develop fundamental insights on the channel rank behavior of UAV links for different RT scenarios, in this paper, we consider that the channel rank is determined by the number of eigenvalues of the channel matrix that are higher than a predetermined threshold. In particular, we consider that the channel rank for a given link at a UAV location $\boldsymbol{p}=(x, y, h)$, where $h \in (h_{1}, h_{2}, ..., h_{N_{h}})$ for $N_{h}$ different altitude settings, is given by  
\begin{align}    
\label{eq:channel_rank}
   R_{\delta_{K_j}}(\boldsymbol{p})&= \sum_{i=1}^{\min(N_r,N_t)}{\mathrm I}\{\sigma_{i}(\boldsymbol{p})> \delta_{K_j} \} \nonumber \\ &=\sum_{i=1}^{\min(N_r,N_t)}{\mathrm I}\bigg\{\sigma_{i}(\boldsymbol{p})> \frac{\sigma_{1}(\boldsymbol{p})}{K_{j}} \bigg\},    
\end{align}
where $I\{ \cdot \}$ is an indicator function that returns $1$ if its input is satisfied and $0$ otherwise, and $\delta_{K_j}$ is a threshold for the singular value with the $j$-th threshold ratio constant $K$. In this paper, we set the threshold for the singular value at a UAV location as $\delta_{K_j} = \sigma_{1}(\boldsymbol{p}) / K_{j}$ to focus on the ratio compared to the first strongest singular value and analyze the performance for different values of $K$. Here, the threshold is extended with a different threshold ratio constant $K$, i.e., $\delta_{K_{j}} = \sigma_{1}(\boldsymbol{p}) / K_{j}$ for $j=1,2, ..., N_{K}$, where $N_K$ is the total number of threshold ratio constant $K$. From this, the channel rank is determined as the number of non-zero singular values that exceed the threshold.  
 
\section{Ray Tracing Simulation Setup}\label{ch:RT_setup}
We consider NSF AERPAW's fixed node locations at NC State University for the target area of RT simulations~\cite{aerpawWebsite}. NVIDIA Sionna is used as the RT simulation tool \cite{nvidia_sionna}. It is known that Sionna employs the Fibonacci lattice unit sphere on the transmitter's side to calculate the candidates for the possible trajectory of rays. A configurable parameter of the number of samples for the sphere is adopted to align the balance between computational complexity and accuracy in calculating the trajectories. The numbers of samples for the sphere for RF coverage and channel rank simulations are set to $10^6$ and $10^3$, respectively, to reduce the computational load in channel rank simulation.  
Figure \ref{fig:map_LW} shows a satellite view of the target area and the locations of the towers with red markers in the Lake Wheeler Field Labs. Geographical information including buildings is obtained from the OSM database \cite{osm, osm_buildings}. Moreover, Figure \ref{fig:map_blender} shows the Blender scene view of the target area, where green dots are the trees and orange rectangles indicate the buildings in the area.

The shooting and bounce ray (SBR) method-based RT model \cite{ray_tracing_access_paper, sbr_paper} has been applied in this work, where the ray tracer traces all rays at the receiver side after calculating reflection or diffraction in the trajectory. The procedure for the SBR can be categorized as 1) ray launching, 2) ray tracing, and 3) ray reception. After the launching of rays from the source, the trajectory of each ray is calculated in the 3D space \cite{sbr_paper}, which can be expressed as 
 \begin{equation}
    (x_1, y_1, z_1) = (x_0, y_0, z_0) + (s_x, s_y, s_z)t,     
 \end{equation}
where $(x_0, y_0, z_0)$ is the reference point, $(s_x, s_y, s_z)$ is the direction vector, and $t$ is the time duration of the trajectory.

 \begin{figure}[t!]
    \centering    
    \subfigure[Satellite view]{
    \includegraphics[width=0.45\columnwidth]{images/LW_map_v2.png}
    \label{fig:map_LW}
    }
    \subfigure[Blender scene]{
    \includegraphics[width=0.44\columnwidth]{images/blender_map.png}
    \label{fig:map_blender}
    }
    \subfigure[Tree model in Blender]{    \includegraphics[width=0.35\columnwidth]{images/blender_tree.png}
    \label{fig:tree}
    }
   \caption{Map view of the location of towers used in RT simulations: (a) Satellite view, (b) Blender scene view, and (c) tree model.}
    \label{fig:map}
\end{figure} 


The reflection losses of the ray with a surface can be expressed as follows~\cite{ITU-R-P.2040-3}: 
\begin{equation}
    \epsilon_{r} = \epsilon_{r}' + j\epsilon_{r}'',    
\end{equation}
 where $\epsilon_{r}' = af^{b}$ and $\epsilon_{r}'' = cf^{d}/2\pi\epsilon_{0}f$. Here, $a, b, c,$ and $d$ are the constant determined by the surface material, $\epsilon_{0}$ is the permittivity of the free space, and $f$ is the frequency in Hz, respectively. Those constants are decided under ITU recommendation  \cite{itu_R_vegetation, itu_R_buildings} for building, ground, and vegetation material settings in this work. Moreover, the electrical field after reflection and diffraction can be expressed as \cite{sbr_paper}
\begin{equation}
    \boldsymbol{E}(x_{i+1}, y_{i+1}, z_{i+1}) = D_i \Gamma_i \boldsymbol{E}(x_i, y_i, z_i) e^{j\theta},  
\end{equation}
where $D_i$ is the divergence factor related to the spreading of the ray right after the \textit{i}-th reflection, $\Gamma_i$ is the reflection coefficient, $\boldsymbol{E}$$(x_i, y_i, z_i)$ is the incident electric field, and $\theta$ is the phase shift of the electrical field. After the calculation, the receiving rays are derived by the overlapping area of the receiving field and trajectories of the rays.


 \begin{table}[t!]
    \centering
    \caption{Simulation parameters for ray tracing.}\vspace{-1mm}
    \begin{tabular}{|c|c|c|}
    \hline
      \textbf{Parameters}   & \textbf{Description} & \textbf{Value} \\
      \hline
        $f_{\mathrm{c}}$ & Carrier frequency & 3.4 GHz \\
        \hline
        $a_{\mathrm{BS}}$ & Height of base stations & 10 m \\        
        \hline
        $T_{\mathrm{LW}}$ & Target area of Lake Wheeler & 1080 m $\times$ 2130 m \\
        \hline
        $N_{\mathrm{ref}}$ & Maximum number of reflections & 2\\
        \hline
        $N_{\mathrm {ele}}$ & Number of antenna elements & 4 (TX and RX) \\
        \hline
        $\Delta_{\mathrm{tx}}$ & Element spacing for TX antennas & $0.5\lambda$ \\
        \hline
        $\Delta_{\mathrm{rx}}$ & Element spacing for RX antennas & $0.5\lambda$ 
        \\
        \hline
        $\mathrm{P_{TX}}$ & Transmit power of TX antennas & 10 W
        \\
        \hline
        $d_{\mathrm{rx}}$ & Horizontal interval of UAV & $30$~m
        \\
        \hline
    \end{tabular}    
    \label{tab:sim_param}
\end{table}

 The following assumptions are made for the RT simulation scenario of Figure \ref{fig:map_blender}: 1) the carrier frequency is set to $3.4$ GHz; 2) a 4-element linear array antenna is used for the transmitter and receiver in the MIMO cases with horizontal element spacing of $0.5\lambda$ for both transmitter and receiver, where $\lambda$ is the wavelength; 3) the height of the base station is set to $10~$m; 4) the maximum number of reflection is set to be 2; 5) surface materials for the buildings are set to concrete and the ground material for the area is medium dry ground, which is predefined as ``itu\_concrete" and ``itu\_medium\_dry\_ground" in Sionna, respectively; 6) the receivers are located on a uniform mesh grid with $30$~m spacing in both x-axis and y-axis in the horizontal directions covering the given area; 7) noise is neglected in RT simulations; and 8) channel rank is determined between the nearest base station and receiver in Lake Wheeler area. The simulation parameters are summarized in Table~\ref{tab:sim_param}. Lastly, the tree consists of a cylinder with ``itu\_wood" material and a cone on top of the cylinder with custom vegetation material setting calculated in \cite{itu_R_vegetation}. The tree model for the simulation is demonstrated in Figure~\ref{fig:tree}, which has a height of $20$~m and a maximum diameter of $10$~m.

 The information about rays such as the angle of arrival and departure, locations of the transmitter and receiver, and path types indicating LoS or reflective NLoS are extracted using Sionna for every UAV location. Then, an RT channel object is created with Matlab's RT tool \cite{matlab_ray_tracing} using ray information. The channel matrix is derived to determine the channel rank by using random input bits and output bits through the ray-tracing channel object. Finally, channel rank is decided as in (\ref{eq:channel_rank}) by singular value decomposition and the indicator function with a threshold compared to the strongest singular value aforementioned in the previous section.

\section{Real-World Measurement Scenario}\label{ch:measurement_setup}

\begin{figure}
     \centering
     \subfigure[Trajectory]{\includegraphics[trim={0.4cm 0.1cm 1.1cm 0.6cm},clip,width=0.98\columnwidth]{images/measurement_trajectory_v3}
     \label{fig:trajectory_measurement_RSS}
     }
     \subfigure[Altitude]{\includegraphics[trim={0.4cm 0.1cm 1.3cm 0.6cm},clip,width=0.98\columnwidth]{images/measurement_altitude}
     \label{fig:altitude_measurement_RSS}
     }     
     \caption{Trajectory and altitude of the signal coverage measurements and RT simulation.}
     \label{fig:trajectory_altitude_measurement_RSS}
 \end{figure}

 \begin{figure}
     \centering
     \subfigure[Trajectory]{\includegraphics[trim={0.4cm 0.1cm 1.1cm 0.6cm},clip,width=0.98\columnwidth]{images/measurement_trajectory_rank_v7}
     \label{fig:trajectory_measurement_rank}
     }
     \subfigure[Altitude]{\includegraphics[trim={0.4cm 0.1cm 1.3cm 0.6cm},clip,width=0.98\columnwidth]{images/measurement_altitudes_rank_v2}
     \label{fig:altitude_measurement_rank}
     }     
     \caption{Trajectory and altitude of the channel rank measurements and RT simulation.}
     \label{fig:trajectory_altitude_measurement_rank}
 \end{figure}

In addition to RT simulations, we also use real-world coverage and channel rank data from multiple measurement campaigns conducted at the NSF AERPAW platform in the same environments in Figure \ref{fig:map_LW}. While channel rank measurements are available only for one of the base stations, coverage results are available for all five of the base stations in the experiment area. %, though for a separate trajectory. 

\textbf{Coverage Measurements.} The predefined trajectory and altitude of the UAV over time for coverage measurements are shown in Figure~ \ref{fig:trajectory_altitude_measurement_RSS}. The UAV takes off near the LW1 fixed node. Then, the UAV sweeps LW2 and LW3 areas, and returns to the LW1. Each waypoint of the trajectory is marked with a letter and with vertical lines in the following figures. 

The following assumptions and setups are adopted for this scenario: 1) the antenna is a single-element antenna (SISO) with a carrier frequency of $3.3$~GHz; 2) the measurements are recorded by dual channel USRP B205 and GNU Radio; 3) the UAV collected data for $20 $~ms intervals every $100$~ms; 4) the offsets calculated by minimum root mean squared error (RMSE) between measurements and RT simulation results are adopted to each measurement for calibration purpose; 5) the proper offsets that have minimum RMSE is searched by a unit of $0.1$~dB in $[-50, 50]$ range in dB scale; 6) the altitude within the trajectory below $0.5$~m is rounded to $0.5$~m for RT simulation to obtain margin from the ground surface; 7) the UAV is the transmitter and the fixed node LW1 to LW5 act as the receivers; and 8) other parameters are the same for the RT simulation parameters. 

\textbf{Channel Rank Measurements.} The predefined trajectory and altitude over time of the UAV for channel rank measurements are shown in Figure \ref{fig:trajectory_altitude_measurement_rank}, where the building in the target area is marked in the cyan polygon. The UAV takes off near the LW1 tower and sweeps the target area with a zigzag pattern.

The following assumptions and configurations are used for this scenario: 1) $4 \times 4$ MIMO channel is used with a carrier frequency of $3.4$~GHz; 2) two AW3232 dual port sector antennas are used for fixed node LW1 as a transmitter; 3) two AW3232 antenna modules are departed by $1.68$~m; 4) Quectel RM502Q-AE 5G mobile modem with antenna module is mounted at the bottom side of the UAV as a receiver; 5) the direction of antennas of LW1 and UAV are placed to face each other, which leads the yaw angle of $45^\circ$ for LW1 and $315^\circ$ for UAV, respectively; 6) 5G new radio (NR) rank indicator (RI) is recorded over the trajectory; and 7) other parameters are the same for the RT simulation parameters.

 
 \section{Kriging-Based 3D Channel Rank Interpolation}\label{ch:Kriging_3D_Channel_interpolation}
 MIMO channel rank at a given location $\boldsymbol{p}$ can be interpolated based on the knowledge of channel rank at other locations in the 3D space, where the rank information has been logged before. In this work, we introduce the Kriging interpolation method for interpolating the channel rank. 

 \subsection{Spatial Correlation Model}\label{ch:spatial_correlation}
 In this section, we describe the spatial correlation of channel ranks at different UAV locations. It is well-known that the channel rank depends on the angular separation of alignments of the transmitter and receiver antennas within a given environment \cite{tse2005fundamentals}. 

 \begin{figure*}[t!]
    \centering
    \includegraphics[width=1.95\columnwidth]{images/spatial_correlation_description_figure_v2.png}
    \caption{Spatial correlation framework for Kriging interpolation.}
    \label{fig:spatial_correlation_description}
\end{figure*}
 
 To take into account both horizontal and vertical correlation of the channel rank, we first define the channel rank vector at a location of UAV as 
\begin{align}\label{eq:channel_rank_vector}
    \boldsymbol{R}(\boldsymbol{p_i}) = \nonumber \\ \big\{ R_{\delta_{K_j}}(x_i &, y_i, h) \mid j = 1, \dots, N_K, \; h = h_1, \dots, h_{N_h} \big\}.
\end{align}
Here, we define an index for each UAV position using a subscript $i=1, ..., N_{\mathrm{loc}}$ at $\boldsymbol{p}$, where $N_{\mathrm{loc}}$ represents the total number of possible UAV locations within the 2D target area. These positions are distributed over a uniformly discretized 2D target area with the horizontal interval of $d_{\mathrm{rx}}$. The index follows a structured ordering: the index starts with the southwest corner in the target area. Then, the index increases along the x-axis from west to east. When the index reaches the end of the x-axis row, it moves one step along the y-axis before the index increases along the x-axis from west to east again. With this, the $N_{\mathrm{loc}}$-th UAV position is at the northeast corner on the 2D target area. This structure enables each UAV position to be assigned a unique index over the structured grid.

Moreover, the reason for using channel rank vector with all thresholds in (\ref{eq:channel_rank_vector}) rather than using channel rank with a single threshold is based on the fact that the correlation value that will be described is non-valid when the channel rank has the same value at all altitudes. In other word, we may have $R_{\delta_{K_j}}(x, y, h_1) = ... = R_{\delta_{K_j}}(x, y, h_{N_h})$. To avoid this, we set the channel rank vector to have channel ranks with all altitudes and thresholds. The defined channel rank vector $\boldsymbol{R}(\boldsymbol{p}_i)$ represents a collection of channel ranks computed at a position of $\boldsymbol{p}_i$ hierarchical ordering by spanning multiple altitudes first and expanded to the multiple thresholds, which can be expressed as
\begin{align}\label{eq:channel_rank_vector_structure}
    \boldsymbol{R}(\boldsymbol{p_i}) =& \big[ R_{\delta_{K_1}}(x_i, y_i, h_1), \dots, R_{\delta_{K_1}}(x_i, y_i, h_{N_h}),\nonumber \\
    &~R_{\delta_{K_2}}(x_i, y_i, h_1), \dots, R_{\delta_{K_2}}(x_i, y_i, h_{N_h}), \dots, \nonumber\\
    &~R_{\delta_{K_{N_K}}}(x_i, y_i, h_{1}), \dots R_{\delta_{K_{N_K}}}(x_i, y_i, h_{N_h}) \big].
\end{align}
In Figure \ref{fig:spatial_correlation_description}, the channel rank vector at $\boldsymbol{p}_i$ is highlighted in red-dashed arrows demonstrating channel rank through all altitudes and thresholds in the multiple 3D target area structure. 

Then, the correlation between two locations of the UAV can be calculated using the Pearson linear correlation function using the channel rank vector defined in (\ref{eq:channel_rank_vector}), which can be expressed as 
\begin{align}
        &\phi(\boldsymbol{p}_{i}, \boldsymbol{p}_{j}) = \nonumber \\
        &\frac{\sum_{u=1}^{N} \big(\boldsymbol{R}_u(\boldsymbol{p}_{i})-\bar{\boldsymbol{R}}(\boldsymbol{p}_{i}) \big) \big(\boldsymbol{R}_u(\boldsymbol{p}_{j}) - \bar{\boldsymbol{R}}(\boldsymbol{p}_{j}) \big)}{[ \sum_{u=1}^{N}\{ \boldsymbol{R}_u(\boldsymbol{p}_{i}) - \bar{\boldsymbol{R}}(\boldsymbol{p}_{i})\}^2  \sum_{o=1}^{N}\{\boldsymbol{R}_o(\boldsymbol{p}_{j}) -\bar{\boldsymbol{R}}(\boldsymbol{p}_{j}) \}^2 ]^{1/2}} ,
    \label{eq:correlation}
\end{align}
where $\boldsymbol{R}_{u}(\boldsymbol{p}_{i})$ denotes the $u$-th element in the channel rank vector $\boldsymbol{R}(\boldsymbol{p}_{i})$ at $\boldsymbol{p}_{i}$, and $\bar{\boldsymbol{R}}(\boldsymbol{p}_{i})$ indicates the mean of the channel rank vector at $\boldsymbol{p}_{i}$, respectively. Since channel rank vector in (\ref{eq:channel_rank_vector}) contains channel rank over all altitudes, the correlation between two locations in (\ref{eq:correlation}) represents correlation in both vertical and horizontal dimensions.


To derive generalized spatial correlation at  all locations of the UAV, the following procedure is considered in this work, which is also illustrated in Figure \ref{fig:spatial_correlation_description}: 
\begin{itemize}
    \item \textbf{Step-1:} Construct the channel rank vector for all UAV position as defined in (\ref{eq:channel_rank_vector}) and (\ref{eq:channel_rank_vector_structure}). 
    
    \item \textbf{Step-2:} Calculate correlation for all UAV positions using equation (\ref{eq:correlation}), i.e., $\phi(\boldsymbol{p}_{i},\boldsymbol{p}_{j})$ for $i, j = 1, \cdots , N_{\mathrm{loc}}$, which has the dimension of $N_{\mathrm{loc}} \times N_{\mathrm{loc}}$, where $N_{\mathrm{loc}}$ is the total number of possible UAV positions in the target area. It results in a symmetric correlation matrix as
    \begin{multline}        
       \hat{\boldsymbol{\Phi}} =  \\  \begin{bmatrix}
\phi(\boldsymbol{p}_1, \boldsymbol{p}_1) & \phi(\boldsymbol{p}_1, \boldsymbol{p}_2) & \dots & \phi(\boldsymbol{p}_1, \boldsymbol{p}_{N_{\mathrm{loc}}}) \\
\phi(\boldsymbol{p}_2, \boldsymbol{p}_1) & \phi(\boldsymbol{p}_2, \boldsymbol{p}_2) & \dots & \phi(\boldsymbol{p}_2, \boldsymbol{p}_{N_{\mathrm{loc}}}) \\
\vdots & \vdots & \ddots & \vdots \\
\phi(\boldsymbol{p}_{N_{\mathrm{loc}}}, \boldsymbol{p}_1) & \phi(\boldsymbol{p}_{N_{\mathrm{loc}}}, \boldsymbol{p}_2) & \dots & \phi(\boldsymbol{p}_{N_{\mathrm{loc}}}, \boldsymbol{p}_{N_{\mathrm{loc}}}) \\
\end{bmatrix}.
    \end{multline}
    \item \textbf{Step-3:} Arrange the correlation from the previous step according to the horizontal (2D) distance between two UAV locations $\boldsymbol{p}_i$ and $ \boldsymbol{p}_j$, which can be expressed as
    \begin{equation}\label{eq:horizontal_distance}
        \Delta_{\mathrm{2D}}(\boldsymbol{p}_i, \boldsymbol{p}_j) = \sqrt{(x_i - x_j)^2 + (y_i - y_j)^2},
    \end{equation} 
    to ensure the correlations have ascending order based on the horizontal distance between corresponding positions. This process results in arranged pairs of 
    \begin{equation}
        \big\{ \Delta_{\mathrm{2D}}(\cdot, \cdot), (\phi_n \mid \Delta_{\mathrm{2D}}^{(n)} = n \cdot d_{\mathrm{rx}} ) \big\}, \mathrm{for} ~ n=0,1,..., N-1,
    \end{equation} where $d_{\mathrm{rx}}$ is the interval of the neighboring UAV positions, e.g., $\boldsymbol{p}_i$ and $\boldsymbol{p}_{i+1}$, and $N$ is the total number of intervals for covering all pairs. Moreover, horizontal distance $\Delta_{\mathrm{2D}}(\cdot, \cdot)$ is discretized by the unit of interval $d_{\mathrm{rx}}$, which leads to $\Delta_{\mathrm{2D}}(\cdot, \cdot)=\Delta_{\mathrm{2D}}^{(n)} = n \cdot d_{\mathrm{rx}}$, where $n$ indicates $n$-th interval. 
    \item \textbf{Step-4:} Take average every $d_{\mathrm{rx}}$ from the smallest distance-correlation pairs until covering all pairs within the interval. It can be expressed as
    \begin{equation}
        \bar{\phi}_n = \frac{1}{M_n} \sum_{m \in \mathcal{I}_n} \phi(\boldsymbol{p}_{i_m}, \boldsymbol{p}_{j_m}),
    \end{equation}
    
    \begin{align}
        \Phi(\Delta_{\mathrm{2D}}(\boldsymbol{p}_i, \boldsymbol{p}_j)) = \nonumber \\
        & \begin{cases} 
        \bar{\phi}_0 & \text{if } \Delta_{\mathrm{2D}}(\boldsymbol{p}_i, \boldsymbol{p}_j) = 0, \\
        \bar{\phi}_1 & \text{if } \Delta_{\mathrm{2D}}(\boldsymbol{p}_i, \boldsymbol{p}_j) = d_{\mathrm{rx}}, \\
        \bar{\phi}_2 & \text{if } \Delta_{\mathrm{2D}}(\boldsymbol{p}_i, \boldsymbol{p}_j) = 2 \cdot d_{\mathrm{rx}}, \\
        \vdots & \vdots \\
        \bar{\phi}_{N-1} & \text{if } \Delta_{\mathrm{2D}}(\boldsymbol{p}_i, \boldsymbol{p}_j) = \\
        & ~~~~~~~~~~\quad (N-1) \cdot d_{\mathrm{rx}},
        \end{cases}\label{eq:correlation_cases}
    \end{align}
    where $M_n$ is the number of pairs with $\Delta_{\mathrm{2D}}^{(n)} = n \cdot d_{\mathrm{rx}}$ and $\mathcal{I}_n$ is the set of indices of UAV position pairs satisfying $\Delta_{\mathrm{2D}}^{(n)} = n \cdot d_{\mathrm{rx}}$.
\end{itemize}


As a result of the previous steps, the generalized correlation of channel ranks as a function of horizontal distance is derived, quantifying the correlation between channel rank vectors at different UAV positions. To further simplify the computation of the correlation function, $\Phi(\Delta_{\mathrm{2D}}(\cdot, \cdot))$, we approximate it using a bi-exponential curve-fitting model for two UAV positions $\boldsymbol{p}_i$ and $\boldsymbol{p}_j$:
\begin{equation}
     \Phi(\Delta_{\mathrm{2D}}(\boldsymbol{p}_i, \boldsymbol{p}_j)) \approx c_{1} e^{c_{2} \Delta_{\mathrm{2D}}(\boldsymbol{p}_i, \boldsymbol{p}_j)} + c_{3} e^{c_{4} \Delta_{\mathrm{2D}}(\boldsymbol{p}_i, \boldsymbol{p}_j)},
     \label{eq:bi_exponential}
 \end{equation}
 where coefficients $c_1, c_2, c_3$, and $c_4$ are selected based on curve-fitting.

 \subsection{Kriging Interpolation-based Channel Rank Interpolation}
 
The ordinary Kriging interpolation method provides optimal estimation when the spatial correlation is well-defined as a function of distance. The Kriging interpolation uses squared-error loss between the observation in a known spatial location and unknown locations \cite{Cressie_2015}. In particular, channel rank at a location $\boldsymbol{p_0}$ with a threshold $\delta_{K_j}$ is interpolated from the linear combination of the channel rank samples of the nearby locations, which can be expressed as \cite{maeng2023kriging}
 \begin{align}
     \min_{l_{1}, ..., l_{M}} \mathop{\mathbb{E}}\Big[\big(\hat{R}_{\delta_{K_j}}&(\boldsymbol{p}_{0})-{R}_{\delta_{K_j}}(\boldsymbol{p}_{0})\big)^2\Big],\\
     \mathrm{\mathbf{ s. t.}} ~~~~ \hat{R}_{\delta_{K_j}}(\boldsymbol{p}_{0}) = & \sum_{i=1}^{M} l_{i} R_{\delta_{K_j}}(\boldsymbol{p}_{i}),
     \label{eq:Kriging_estimation}\\
     \sum_{i=1}^{M} & l_{i} = 1,
 \end{align}
where $\boldsymbol{p}_0$ is the location to be interpolated, $l_i$ denotes the coefficient for the linear combination, $R_{\delta_{K_j}}(\boldsymbol{p}_{i})$ is $i$-th element of the channel rank within the sampled locations for the interpolation, $\hat{R}_{\delta_{K_j}}(\boldsymbol{p}_{0})$ indicates the estimated channel rank with threshold $\delta_{K_j}$, and $M$ indicates the number of samples used for the interpolation, respectively. The optimization problem above can be solved by converting the problem into an equivalent Lagrange equation as \cite{maeng2023kriging, Kriging_solution_paper}
\begin{multline} 
    \min_{l_{1}, ..., l_{M}} \mathop{\mathbb{E}}\left[\left(R_{\delta_{K_j}}(\boldsymbol{p}_{0}) - \sum_{i=1}^{M} l_{i} R_{\delta_{K_j}}(\boldsymbol{p}_{i}) \right)^2\right]   \\ - L\left(\sum_{i=1}^{M} l_{i} - 1\right),
\end{multline}
where $L$ denotes the Lagrange multiplier. After a few steps, the optimal solution can be derived by the linear matrix equation as
\begin{align}
        \begin{bmatrix}
    \gamma(\boldsymbol{p}_{1}, \boldsymbol{p}_{1}) & ... & \gamma(\boldsymbol{p}_{1}, \boldsymbol{p}_{M}) & 1 \\
    \gamma(\boldsymbol{p}_{2}, \boldsymbol{p}_{1}) & ... & \gamma(\boldsymbol{p}_{2}, \boldsymbol{p}_{M}) & 1 \\
    \vdots & \vdots & \vdots & \vdots \\
    \gamma(\boldsymbol{p}_{M}, \boldsymbol{p}_{1}) & ... & \gamma(\boldsymbol{p}_{M}, \boldsymbol{p}_{M}) & 1 \\
    1 & ... & 1 & 0
    \end{bmatrix}
    \begin{bmatrix}
        l_1 \\
        l_2 \\ 
        \vdots \\
        l_M \\
        L'
    \end{bmatrix}
    %\\ 
    = \begin{bmatrix}
        \gamma(\boldsymbol{p}_{0}, \boldsymbol{p}_{1}) \\
        \gamma(\boldsymbol{p}_{0}, \boldsymbol{p}_{2}) \\
        \vdots \\
        \gamma(\boldsymbol{p}_{0}, \boldsymbol{p}_{M}) \\
        1
    \end{bmatrix},
    \label{eq:linear_matrix}
\end{align}
where $\gamma(\cdot, \cdot)$ indicates the semi-variogram of the Kriging interpolation for two locations of the UAV. The semi-variogram of the Kriging interpolation scheme can be expressed as
\begin{equation}
    \gamma(\boldsymbol{p}_{i},   \boldsymbol{p}_{j}) = v_{\delta_{K_j}}^{2} \Big(1-\Phi \big(\Delta_{\mathrm{2D}}(\boldsymbol{p}_i, \boldsymbol{p}_j)\big)\Big),
    \label{eq:semi_variogram}
\end{equation}
where $\Phi(\Delta_{\mathrm{2D}}(\boldsymbol{p}_i, \boldsymbol{p}_j))$ is as defined in (\ref{eq:correlation_cases}) and (\ref{eq:bi_exponential}), and $v_{\delta_{K_j}}^{2}$ is the variance of the channel rank calculated with a threshold at a location with all altitudes. This can be expressed as 
\begin{equation}
    v_{\delta_{K_j}}^{2} = \frac{1}{N_{h}-1}\sum_{s=1}^{N_{h}}|R_{\delta_{K_j}}(x, y, h_s) - \mu|^2,
\end{equation}
where $\mu$ is the mean of $R_{\delta_{K_j}}(x, y, h_{s})$ over all $h_{s}$ for $s=1, ..., N_{h}$. The semi-variogram indicates the spatial correlation in different locations. The expression above is derived with the assumption of the existence of the covariance function of a stationary process. 

The procedure for the Kriging interpolation-based 3D channel rank interpolation scheme, which is also highlighted in Figure \ref{fig:Kriging_description}, can be summarized as follows:
\begin{itemize} 
 \item \textbf{Step-1:} Sample $M$ channel rank data within a radius of $r_{0}$ from the target UAV position, $\boldsymbol{p}_{0}$.
 
 \item \textbf{Step-2:} Calculate semi-variogram defined in equation (\ref{eq:semi_variogram}).

 \item \textbf{Step-3:} Derive coefficient for the linear combination, $l_i$, by substituting semi-variogram into linear matrix equation in (\ref{eq:linear_matrix}).

 \item \textbf{Step-4:} Process interpolation in equation (\ref{eq:Kriging_estimation}) with derived $l_i$ and channel rank for the target UAV position.

 \item \textbf{Step-5:} Repeat all the previous steps for all possible UAV locations.
\end{itemize} 

  \begin{figure}
     \centering
     \includegraphics[trim={0.15cm 0.15cm 0.8cm 0.4cm},clip,width=0.95\columnwidth]{images/Kriging_description_figure_v5.png}
     \caption{Procedure of the Kriging interpolation-based 3D channel rank interpolation scheme.}
     \label{fig:Kriging_description}
 \end{figure}

 \subsection{Other Baseline Interpolation Techniques}\label{ch:other_baseline}
We consider spline and makima (modified Akima cubic Hermite) interpolation methods as baseline interpolation approaches for comparison purposes. In the baseline interpolation approaches, the UAV position index as described right after (\ref{eq:channel_rank_vector}), $i=1, ..., N_{\mathrm{loc}}$, is used to identify UAV positions within a 2D target area that is discretized into a uniform grid with spacing $d_{\mathrm{rx}}$. 
% The indexing follows a structured ordering: the index first increases along the x-axis from west to east, and after reaching the end of a row, it moves one step along the y-axis before continuing again from west to east. This ensures that each UAV position in the grid is assigned a unique index number based on its position in the structured grid. 

The procedure for the baseline interpolation-based 3D channel rank interpolation can be summarized as follows: 1) Sample $M$ UAV positions within a radius of $r_{0}$ from the target UAV position $\boldsymbol{p}_0$, 2) sort the UAV position index and channel rank with respect to the index, 3) perform interpolation using sorted index and channel rank, i.e., $\hat{R}_{\delta_{K_j}}(\boldsymbol{p}_{0})=\mathsf{B}(\boldsymbol{X}, R_{\delta_{K_j}}(\boldsymbol{p}_{\boldsymbol{X}}), \boldsymbol{p}_{0})$, where $\mathsf{B}()$ indicates the baseline interpolation function with spline or makima approaches, $\boldsymbol{X}$ represents all the sampled UAV positions index within $r_0$ radius from the target point, and $R_{\delta_{K_j}}(\boldsymbol{p}_{\boldsymbol{X}})$ is the channel rank data of the corresponding locations $\boldsymbol{X}$, respectively.  

\section{Numerical Results}\label{ch:numerical_results}
In this section, we present numerical results on RSS, channel rank, and Kriging-based channel rank interpolation schemes for different UAV scenarios in a rural environment, in Section~\ref{ch:RF_coverage}, Section~\ref{ch:channel_rank}, and Section~\ref{ch:Kriging_interpolation}, respectively. The RT results are also compared with the real measurements in the NSF AERPAW testbed in Section~VII.\ref{ch:measurement}. In addition to the simulation parameters listed in Section~\ref{ch:RT_setup}, the following assumptions are used for the corresponding analysis: 1) the number of threshold ratio constant $N_{K} = 3$, where $K_1=10^1, K_2=10^2$, and $K_3=10^3$ leading $\delta_{K_1}=\sigma_1(\boldsymbol{p})/10^1$, $\delta_{K_2}=\sigma_1(\boldsymbol{p})/10^2$, and $\delta_{K_3}=\sigma_1(\boldsymbol{p})/10^3$ to capture singular values in $10$~dB, $20$~dB and $30$~dB range from the strongest singular value, respectively; 2) the horizontal interval between UAV positions $d_{\mathrm{rx}}$ is set to $30$~m, which leads to the total number of possible UAV locations $N_{\mathrm{loc}}= 36 \times 71 = 2556$ for the given 2D target area $T_{\mathrm{LW}}$ in Table~\ref{tab:sim_param}; 3) the number of altitudes of UAV $N_{h} = 9$ with $10$~m interval from $30$~m to $110$~m; 4) the altitudes of interest in RSS analysis is $30$~m, $70$~m, and $110$~m; and 5) in addition to the altitudes of interest in RSS simulation, $3$~m altitude configuration is also applied to the channel rank RT simulation to highlight the blockage effects.

 \subsection{Received Signal Strength and Coverage Analysis}\label{ch:RF_coverage}

 \begin{figure*}[t!]
    \centering    
    \subfigure[LW1 $30$~m]{
    \includegraphics[trim={0.1cm 1.0cm 0.15cm 1.7cm},clip,width=0.37\columnwidth]{images/RSS_MIMO_30m_v4_LW1}
    \label{fig:RSSI_MIMO_30_LW1}
    }
    \subfigure[LW2 $30$~m]{
    \includegraphics[trim={0.1cm 1.0cm 0.15cm 1.7cm},clip,width=0.37\columnwidth]{images/RSS_MIMO_30m_v4_LW2}
    \label{fig:RSSI_MIMO_30_LW2}
    }
    \subfigure[LW3 $30$~m]{
    \includegraphics[trim={0.1cm 1.0cm 0.15cm 1.7cm},clip,width=0.37\columnwidth]{images/RSS_MIMO_30m_v4_LW3}
    \label{fig:RSSI_MIMO_30_LW3}
    }
    \subfigure[LW4 $30$~m]{
    \includegraphics[trim={0.1cm 1.0cm 0.15cm 1.7cm},clip,width=0.37\columnwidth]{images/RSS_MIMO_30m_v4_LW4}
    \label{fig:RSSI_MIMO_30_LW4}
    }
    \subfigure[LW5 $30$~m]{
    \includegraphics[trim={0.1cm 1.0cm 0.15cm 1.7cm},clip,width=0.37\columnwidth]{images/RSS_MIMO_30m_v4_LW5}
    \label{fig:RSSI_MIMO_30_LW5}
    }    
    \subfigure[LW1 $70$~m]{
    \includegraphics[trim={0.1cm 1.0cm 0.15cm 1.7cm},clip,width=0.37\columnwidth]{images/RSS_MIMO_70m_v4_LW1}
    \label{fig:RSSI_MIMO_70_LW1}
    }
    \subfigure[LW2 $70$~m]{
    \includegraphics[trim={0.1cm 1.0cm 0.15cm 1.7cm},clip,width=0.37\columnwidth]{images/RSS_MIMO_70m_v4_LW2}
    \label{fig:RSSI_MIMO_70_LW2}
    }
    \subfigure[LW3 $70$~m]{
    \includegraphics[trim={0.1cm 1.0cm 0.15cm 1.7cm},clip,width=0.37\columnwidth]{images/RSS_MIMO_70m_v4_LW3}
    \label{fig:RSSI_MIMO_70_LW3}
    }
    \subfigure[LW4 $70$~m]{
    \includegraphics[trim={0.1cm 1.0cm 0.15cm 1.7cm},clip,width=0.37\columnwidth]{images/RSS_MIMO_70m_v4_LW4}
    \label{fig:RSSI_MIMO_70_LW4}
    }
    \subfigure[LW5 $70$~m]{
    \includegraphics[trim={0.1cm 1.0cm 0.15cm 1.7cm},clip,width=0.37\columnwidth]{images/RSS_MIMO_70m_v4_LW5}
    \label{fig:RSSI_MIMO_70_LW5}
    }    
    \subfigure[LW1 $110$~m]{
    \includegraphics[trim={0.1cm 1.0cm 0.15cm 1.7cm},clip,width=0.37\columnwidth]{images/RSS_MIMO_110m_v4_LW1}
    \label{fig:RSSI_MIMO_110_LW1}
    }
    \subfigure[LW2 $110$~m]{
    \includegraphics[trim={0.1cm 1.0cm 0.15cm 1.7cm},clip,width=0.37\columnwidth]{images/RSS_MIMO_110m_v4_LW2}
    \label{fig:RSSI_MIMO_110_LW2}
    }
    \subfigure[LW3 $110$~m]{
    \includegraphics[trim={0.1cm 1.0cm 0.15cm 1.7cm},clip,width=0.37\columnwidth]{images/RSS_MIMO_110m_v4_LW3}
    \label{fig:RSSI_MIMO_110_LW3}
    }
    \subfigure[LW4 $110$~m]{
    \includegraphics[trim={0.1cm 1.0cm 0.15cm 1.7cm},clip,width=0.37\columnwidth]{images/RSS_MIMO_110m_v4_LW4}
    \label{fig:RSSI_MIMO_110_LW4}
    }
    \subfigure[LW5 $110$~m]{
    \includegraphics[trim={0.1cm 1.0cm 0.15cm 1.7cm},clip,width=0.37\columnwidth]{images/RSS_MIMO_110m_v4_LW5}
    \label{fig:RSSI_MIMO_110_LW5}
    }    
   \caption{RSS in the Lake Wheeler Field Labs area with MIMO and various altitude configurations.}
    \label{fig:RSSI_MIMO_all_altitudes}
\end{figure*} 


 \begin{figure*}[t!]
    \centering    
    \subfigure[LW1 $30$~m]{
    \includegraphics[trim={0.1cm 1.0cm 0.15cm 1.7cm},clip,width=0.37\columnwidth]{images/RSS_SISO_30m_v5_LW1}
    \label{fig:RSSI_SISO_30_LW1}
    }
    \subfigure[LW2 $30$~m]{
    \includegraphics[trim={0.1cm 1.0cm 0.15cm 1.7cm},clip,width=0.37\columnwidth]{images/RSS_SISO_30m_v5_LW2}
    \label{fig:RSSI_SISO_30_LW2}
    }
    \subfigure[LW3 $30$~m]{
    \includegraphics[trim={0.1cm 1.0cm 0.15cm 1.7cm},clip,width=0.37\columnwidth]{images/RSS_SISO_30m_v5_LW3}
    \label{fig:RSSI_SISO_30_LW3}
    }
    \subfigure[LW4 $30$~m]{
    \includegraphics[trim={0.1cm 1.0cm 0.15cm 1.7cm},clip,width=0.37\columnwidth]{images/RSS_SISO_30m_v5_LW4}
    \label{fig:RSSI_SISO_30_LW4}
    }
    \subfigure[LW5 $30$~m]{
    \includegraphics[trim={0.1cm 1.0cm 0.15cm 1.7cm},clip,width=0.37\columnwidth]{images/RSS_SISO_30m_v5_LW5}
    \label{fig:RSSI_SISO_30_LW5}
    }    
    \subfigure[LW1 $70$~m]{
    \includegraphics[trim={0.1cm 1.0cm 0.15cm 1.7cm},clip,width=0.37\columnwidth]{images/RSS_SISO_70m_v5_LW1}
    \label{fig:RSSI_SISO_70_LW1}
    }
    \subfigure[LW2 $70$~m]{
    \includegraphics[trim={0.1cm 1.0cm 0.15cm 1.7cm},clip,width=0.37\columnwidth]{images/RSS_SISO_70m_v5_LW2}
    \label{fig:RSSI_SISO_70_LW2}
    }
    \subfigure[LW3 $70$~m]{
    \includegraphics[trim={0.1cm 1.0cm 0.15cm 1.7cm},clip,width=0.37\columnwidth]{images/RSS_SISO_70m_v5_LW3}
    \label{fig:RSSI_SISO_70_LW3}
    }
    \subfigure[LW4 $70$~m]{
    \includegraphics[trim={0.1cm 1.0cm 0.15cm 1.7cm},clip,width=0.37\columnwidth]{images/RSS_SISO_70m_v5_LW4}
    \label{fig:RSSI_SISO_70_LW4}
    }
    \subfigure[LW5 $70$~m]{
    \includegraphics[trim={0.1cm 1.0cm 0.15cm 1.7cm},clip,width=0.37\columnwidth]{images/RSS_SISO_70m_v5_LW5}
    \label{fig:RSSI_SISO_70_LW5}
    }    
    \subfigure[LW1 $110$~m]{
    \includegraphics[trim={0.1cm 1.0cm 0.15cm 1.7cm},clip,width=0.37\columnwidth]{images/RSS_SISO_110m_v5_LW1}
    \label{fig:RSSI_SISO_110_LW1}
    }
    \subfigure[LW2 $110$~m]{
    \includegraphics[trim={0.1cm 1.0cm 0.15cm 1.7cm},clip,width=0.37\columnwidth]{images/RSS_SISO_110m_v5_LW2}
    \label{fig:RSSI_SISO_110_LW2}
    }
    \subfigure[LW3 $110$~m]{
    \includegraphics[trim={0.1cm 1.0cm 0.15cm 1.7cm},clip,width=0.37\columnwidth]{images/RSS_SISO_110m_v5_LW3}
    \label{fig:RSSI_SISO_110_LW3}
    }
    \subfigure[LW4 $110$~m]{
    \includegraphics[trim={0.1cm 1.0cm 0.15cm 1.7cm},clip,width=0.37\columnwidth]{images/RSS_SISO_110m_v5_LW4}
    \label{fig:RSSI_SISO_110_LW4}
    }
    \subfigure[LW5 $110$~m]{
    \includegraphics[trim={0.1cm 1.0cm 0.15cm 1.7cm},clip,width=0.37\columnwidth]{images/RSS_SISO_110m_v5_LW5}
    \label{fig:RSSI_SISO_110_LW5}
    }    
   \caption{RSS in the Lake Wheeler Field Labs area with SISO and $30, 70,$ and $110$~m altitude configurations.}
    \label{fig:RSSI_SISO_all_altitudes}
\end{figure*} 

\begin{figure}[t!]
    \centering
    \subfigure[MIMO $30$~m]{
    \includegraphics[trim={0.1cm 0.8cm 0.5cm 1.7cm},clip,width=0.45\columnwidth]{images/RSS_MIMO_30m_v6_joint}
    \label{fig:RSSI_MIMO_30_joint}
    }
    \subfigure[SISO $30$~m]{
    \includegraphics[trim={0.1cm 0.8cm 0.5cm 1.7cm},clip,width=0.45\columnwidth]{images/RSS_SISO_30m_v6_joint}
    \label{fig:RSSI_SISO_30_joint}
    }    
    \subfigure[MIMO $70$~m]{
    \includegraphics[trim={0.1cm 0.8cm 0.5cm 1.7cm},clip,width=0.45\columnwidth]{images/RSS_MIMO_70m_v6_joint}
    \label{fig:RSSI_MIMO_70_joint}
    }
    \subfigure[SISO $70$~m]{
    \includegraphics[trim={0.1cm 0.8cm 0.5cm 1.7cm},clip,width=0.45\columnwidth]{images/RSS_SISO_70m_v6_joint}
    \label{fig:RSSI_SISO_70_joint}
    }
    \subfigure[MIMO $110$~m]{
    \includegraphics[trim={0.1cm 0.8cm 0.5cm 1.7cm},clip,width=0.45\columnwidth]{images/RSS_MIMO_110m_v6_joint}
    \label{fig:RSSI_MIMO_110_joint}
    }
    \subfigure[SISO $110$~m]{
    \includegraphics[trim={0.1cm 0.8cm 0.5cm 1.7cm},clip,width=0.45\columnwidth]{images/RSS_SISO_110m_v6_joint}
    \label{fig:RSSI_SISO_110_joint}
    }
    \caption{RSS in the Lake Wheeler Field Labs area with the joint coverage and $30, 70,$ and $110$~m altitude configurations.}
    \label{fig:RSSI_SISO_MIMO_all_joint}
\end{figure}


 \begin{figure*}[t!]
    \centering    
    \subfigure[$30$~m MIMO]{
    \includegraphics[trim={0.4cm 0cm 0.8cm 0.6cm},clip,width=0.63\columnwidth]{images/CDF_RSS_MIMO_30m_v2}
    \label{fig:RSSI_CDF_30m_MIMO}
    }
    \subfigure[$70$~m MIMO]{
    \includegraphics[trim={0.4cm 0cm 0.8cm 0.6cm},clip,width=0.63\columnwidth]{images/CDF_RSS_MIMO_70m_v2}
    \label{fig:RSSI_CDF_70m_MIMO}
    }
    \subfigure[$110$~m MIMO]{
    \includegraphics[trim={0.4cm 0cm 0.8cm 0.6cm},clip,width=0.63\columnwidth]{images/CDF_RSS_MIMO_110m_v2}
    \label{fig:RSSI_CDF_110m_MIMO}
    }
    \subfigure[$30$~m SISO]{
    \includegraphics[trim={0.4cm 0cm 0.8cm 0.6cm},clip,width=0.63\columnwidth]{images/CDF_RSS_SISO_30m_v2}
    \label{fig:RSSI_CDF_30m_SISO}
    }
    \subfigure[$70$~m SISO]{
    \includegraphics[trim={0.4cm 0cm 0.8cm 0.6cm},clip,width=0.63\columnwidth]{images/CDF_RSS_SISO_70m_v2}
    \label{fig:RSSI_CDF_70m_SISO}
    }
    \subfigure[$110$~m SISO]{
    \includegraphics[trim={0.4cm 0cm 0.8cm 0.6cm},clip,width=0.63\columnwidth]{images/CDF_RSS_SISO_110m_v2}
    \label{fig:RSSI_CDF_110m_SISO}
    }
   \caption{CDFs of RSS in the Lake Wheeler Field Labs area with MIMO and SISO configurations.}
    \label{fig:RSSI_CDFs}
\end{figure*} 


 In this section, we analyze the RSS for UAV channels using the RT simulation tool. We consider both single-input single-output (SISO) and MIMO antenna configuration ($4\times 1$ linear array at the tower) with three different UAV altitudes: $30$~m, $70$~m, and $110$~m  from the ground. For MIMO, we consider a $4$-element linear array antenna placed along the y-axis at each tower. We study the coverage of each tower individually and when they operate simultaneously with the other towers. For the latter scenario, each UAV receiver is connected to the closest tower. The Voronoi boundaries of the coverages under the joint coverage assumption are marked in blue dashed lines in the figures. 

 Figures \ref{fig:RSSI_MIMO_all_altitudes} and \ref{fig:RSSI_SISO_all_altitudes} show the simulation results of RSS distribution over the Lake Wheeler Field Labs for three different altitudes with SISO and MIMO configurations, respectively. Moreover, the simulation results of RSS distribution with the joint coverage assumption are shown in Figure~\ref{fig:RSSI_SISO_MIMO_all_joint}. Here, towers are marked in red arrows and buildings are marked in red polygons. Moreover, the forest areas are marked in magenta polygons. The letter $\mathsf{Z}$ represents the out-of-coverage area with the given tower and UAV, where the signal strength is below a threshold value (e.g. due to blockage). In Figure \ref{fig:RSSI_MIMO_all_altitudes}, narrow distribution patterns through the horizontal axis and side lobes in the diagonal direction are observed due to the beam pattern of the MIMO linear array. It is observed that the beam pattern leads to the narrow coverage, which can be seen around the LW2, LW3, and LW5 towers. Meanwhile, due to the omnidirectional beam pattern of SISO configuration, the RSS distribution in Figure \ref{fig:RSSI_SISO_all_altitudes} tends to have a circular pattern near the base stations. 
 The blockage effects from the trees are observed in $30$~m in both SISO and MIMO cases, which is gradually mitigated at the higher UAV altitudes. 
 
 The RSS simulation results of  Figures~\ref{fig:RSSI_MIMO_all_altitudes} -~\ref{fig:RSSI_SISO_MIMO_all_joint} are summarized with CDFs in Figure \ref{fig:RSSI_CDFs}. The portion of the blockage area with the LW3 site is the highest of all cases due to LW3 being at the edge of the flight area and being most blocked by the trees. It is observed that the LW1 tower has the least blockage due to its location being in an open area close to the center of the field. The blockage is reduced as the altitude of the UAV increases, while the higher values of the RSS are slightly degraded due to larger path loss at higher altitudes. The blockage with the joint coverage is significantly reduced even with the $30$~m altitude, where there is no blockage with the higher altitudes.  Moreover, it can be observed that the range of RSS tends to be wider in the MIMO cases that have multiple directional beam patterns than the omnidirectional SISO cases.  

 \subsection{Channel Rank Analysis}\label{ch:channel_rank}

  \begin{figure*}[t!]
    \centering    
    \subfigure[$K_1$, $3$~m]{
    \includegraphics[trim={0.15cm 1.0cm 1.3cm 1.7cm},clip,width=0.47\columnwidth]{images/channel_rank_distribution_3m_LW1_LW5_10}
    \label{fig:Rank_MIMO_3_10}
    }
    \subfigure[$K_1$, $30$~m]{
    \includegraphics[trim={0.15cm 1.0cm 1.3cm 1.7cm},clip,width=0.47\columnwidth]{images/channel_rank_distribution_30m_LW1_LW5_10_v6}
    \label{fig:Rank_MIMO_30_10}
    }
    \subfigure[$K_1$, $70$~m]{
    \includegraphics[trim={0.15cm 1.0cm 1.3cm 1.7cm},clip,width=0.47\columnwidth]{images/channel_rank_distribution_70m_LW1_LW5_10_v6}
    \label{fig:Rank_MIMO_70_10}
    }
    \subfigure[$K_1$, $110$~m]{
    \includegraphics[trim={0.15cm 1.0cm 1.3cm 1.7cm},clip,width=0.47\columnwidth]{images/channel_rank_distribution_110m_LW1_LW5_10_v6}
    \label{fig:Rank_MIMO_110_10}
    }   
    \subfigure[$K_2$, $3$~m]{
    \includegraphics[trim={0.15cm 1.0cm 1.3cm 1.7cm},clip,width=0.47\columnwidth]{images/channel_rank_distribution_3m_LW1_LW5_100}
    \label{fig:Rank_MIMO_3_100}
    }
    \subfigure[$K_2$, $30$~m]{
    \includegraphics[trim={0.15cm 1.0cm 1.3cm 1.7cm},clip,width=0.47\columnwidth]{images/channel_rank_distribution_30m_LW1_LW5_100_v6}
    \label{fig:Rank_MIMO_30_100}
    }
    \subfigure[$K_2$, $70$~m]{
    \includegraphics[trim={0.15cm 1.0cm 1.3cm 1.7cm},clip,width=0.47\columnwidth]{images/channel_rank_distribution_70m_LW1_LW5_100_v6}
    \label{fig:Rank_MIMO_70_100}
    }
    \subfigure[$K_2$, $110$~m]{
    \includegraphics[trim={0.15cm 1.0cm 1.3cm 1.7cm},clip,width=0.47\columnwidth]{images/channel_rank_distribution_110m_LW1_LW5_100_v6}
    \label{fig:Rank_MIMO_110_100}
    }
    \subfigure[$K_3$, $3$~m]{
    \includegraphics[trim={0.15cm 1.0cm 1.3cm 1.7cm},clip,width=0.47\columnwidth]{images/channel_rank_distribution_3m_LW1_LW5_1000}
    \label{fig:Rank_MIMO_3_1000}
    }
    \subfigure[$K_3$, $30$~m]{
    \includegraphics[trim={0.15cm 1.0cm 1.3cm 1.7cm},clip,width=0.47\columnwidth]{images/channel_rank_distribution_30m_LW1_LW5_1000_v6}
    \label{fig:Rank_MIMO_30_1000}
    }
    \subfigure[$K_3$, $70$~m]{
    \includegraphics[trim={0.15cm 1.0cm 1.3cm 1.7cm},clip,width=0.47\columnwidth]{images/channel_rank_distribution_70m_LW1_LW5_1000_v6}
    \label{fig:Rank_MIMO_70_1000}
    }
    \subfigure[$K_3$, $110$~m]{
    \includegraphics[trim={0.15cm 1.0cm 1.3cm 1.7cm},clip,width=0.47\columnwidth]{images/channel_rank_distribution_110m_LW1_LW5_1000_v6}
    \label{fig:Rank_MIMO_110_1000}
    }
   \caption{Channel rank in the Lake Wheeler Field Labs with various rank thresholds $K_1$, $K_2$, $K_3$ and receiver altitudes of  $3$~m, $30$~m, $70$~m, and $110$~m.}
    \label{fig:Rank_distribution}
\end{figure*} 


\begin{figure*}[t!]
    \centering    
    \subfigure[Distribution with $K_1$]{
    \includegraphics[trim={0.4cm 0.1cm 1.3cm 0.7cm},clip, width=0.64\columnwidth]{images/PDF_channel_rank_10_v3}
    \label{fig:Rank_PDF_MIMO_10}
    }
    \subfigure[Distribution with $K_2$]{
    \includegraphics[trim={0.4cm 0.1cm 1.3cm 0.7cm},clip, width=0.64\columnwidth]{images/PDF_channel_rank_100_v3}
    \label{fig:Rank_PDF_MIMO_100}
    }
    \subfigure[Distribution with $K_3$]{
    \includegraphics[trim={0.4cm 0.1cm 1.3cm 0.7cm},clip, width=0.64\columnwidth]{images/PDF_channel_rank_1000_v3}
    \label{fig:Rank_PDF_MIMO_1000}
    }
    \caption{Distribution of the channel rank in the Lake Wheeler Field Labs area with different threshold constants and different UAV altitudes.}
    \label{fig:Rank_PDF_LWs}
\end{figure*}

The channel rank distribution in the Lake Wheeler area based on RT simulations, with different rank thresholds at $3$~m, $30$~m, $70$~m, and $110$~m UAV altitudes are shown in Figure \ref{fig:Rank_distribution}, where $\mathsf{Z}$ indicates the receiver sites that have no connectivity to the corresponding tower. Here, the locations of the buildings and the tower are marked in red polygons and red arrows. The forest areas with trees are marked in magenta polygons in the figures. The UAV is assumed to be connected to the closest tower when the channel rank is evaluated. The boundaries of each coverage are marked in cyan dashed curves in the figures.

As seen in the cases with $3$~m and $30$~m altitudes, the blockage from the trees is observed in the forest area for all tower cases. The blockage area in the forest tends to be mitigated as the altitude of the UAV increases. We have to note the difference in the blockage areas of Figures \ref{fig:RSSI_MIMO_30_joint} and \ref{fig:Rank_MIMO_30_10}. This can be due to the fact that the number of samples in calculating the candidates of the rays using the Fibonacci unit sphere is set to $10^3$ for channel rank and $10^6$ for the RSS simulations to set a reasonable balance in heavy computational load in channel rank simulations. 

It is also observed that UAV locations near the tower have a channel rank of $2$, which can be expected to have two multipath components with LoS and ground-reflected rays. The area of channel rank $2$ near the tower gets wider as the threshold constant is relaxed from $K_1$ to $K_3$. Meanwhile, the channel rank tends to be $1$ as the distance from the tower increases. It can be interpreted that the link is LoS-dependent as the singular values from the ground-reflected ray are not strong enough compared to the strongest singular value. Moreover, some receiver locations that are close to the tower and other objects (e.g., trees or buildings) have a channel rank of $3$ due to richer reflections in those locations. 

The distribution of the channel rank in the Lake Wheeler area with different threshold constant $K$ are shown in Figure~\ref{fig:Rank_PDF_LWs}. Here, the $3$~m altitude cases show more than $60$ percent of the out-of-coverage area due to significant blockage by the trees. The likelihood of observing channel rank 1 reduces as the threshold constant $K$ increases. Most receiver sites have channel rank 1 or 2, which can be interpreted as most of the links having dominant LoS and ground-reflected ray in the given rural scenario. 


 \subsection{Kriging Based 3D Channel Rank Interpolation}\label{ch:Kriging_interpolation}

  \begin{figure}
     \centering
     \includegraphics[trim={0.6cm 0 1.1cm 0.5cm},clip,width=0.98\columnwidth]{images/correlation_vs_distance_30m_bin_manual_tree_LW2_10000_excluded_v2}
     \caption{Simulation result of 3D correlation vs. distance and curve fitting model in (\ref{eq:bi_exponential}).}
     \label{fig:correlation_distance_curvefitting}
 \end{figure}

 In this section, we present results on the Kriging-based 3D interpolation scheme for channel rank prediction. First, the semi-variogram has to be estimated based on RT simulations. The simulation results of the correlation of channel rank and 3D distance over every UAV position are shown in Figure \ref{fig:correlation_distance_curvefitting}, where blue dots indicate the correlation data calculated by the procedure in Section \ref{ch:Kriging_3D_Channel_interpolation}-\ref{ch:spatial_correlation} and the red line represents the bi-exponential model derived by the curve fitting expression in (\ref{eq:bi_exponential}). We limit the maximum horizontal distance between two locations of interest for spatial correlation analysis, $\Delta_{\mathrm{2D}}(\boldsymbol{p}_{i}, \boldsymbol{p}_{j})$ as in~(\ref{eq:horizontal_distance}) and (\ref{eq:bi_exponential}), as $500$~m to prevent taking into account less correlated UAV positions. The coefficients $c_{1}$ - $c_{4}$ for equation (\ref{eq:bi_exponential}) are derived by the Matlab curve fitting tool \cite{matlab_curve_fitting}, which are $c_{1} = 0.2932, c_{2} = -0.0508, c_{3} = 0.7057,$ and $c_{4} = -0.001$, respectively. The RMSE between actual correlation data and the bi-exponential model is $0.0153$. 

Using the results in Figure~\ref{fig:correlation_distance_curvefitting}, Kriging-based channel rank interpolation is carried out. The interpolated channel rank for all positions of the UAV with $9$ attitudes and $3$ thresholds for the channel rank calculation is evaluated using the MAE error metric, which can be expressed as follows:
\begin{align} \label{eq:MAE}
    E_{\mathrm{MAE}, \delta_{K_j}, h}  = \frac{1}{N_{\mathrm{loc}}}  \sum_{i=1}^{N_{\mathrm{loc}}}\Big|R_{\delta_{K_j}}(\boldsymbol{p}_i) - \hat{R}_{\delta_{K_j}}(\boldsymbol{p}_i)\Big|~, 
\end{align}
for all $j=1, ..., N_K$ and
$h \in (h_1, h_2, ..., h_{N_h})$. Here, $N_{\mathrm{loc}}$ is the total number of possible UAV locations to be interpolated in the uniformly discretized 2D target area as described right after ~(\ref{eq:channel_rank_vector}) and Section~\ref{ch:numerical_results}. Moreover, the number of samples for interpolation $M$ as in (\ref{eq:Kriging_estimation}) is set to $20$ and the radius for sampling from the $\boldsymbol{p}_{0}$, $r_{0}$, as in Figure~\ref{fig:Kriging_description} is $150$~m. The MAE performance evaluations of the Kriging interpolation-based 3D channel rank interpolation scheme and baseline interpolation-based approaches are shown in Figure \ref{fig:MAE_interpolations}. 

Using the previously defined MAE in (\ref{eq:MAE}) and the procedure illustrated in Figure~\ref{fig:Kriging_description}, the MAE of the channel rank is computed for each altitude and threshold setting. The target location $\boldsymbol{p}_0$ is sequentially processed across all UAV location index $i=1,..., N_{\mathrm{loc}}$ as described in Section~\ref{ch:other_baseline}. For each $\boldsymbol{p}_0$, interpolation is performed using $M$ samples within the sampling radius $r_0$ under the assumption that the channel rank at $\boldsymbol{p}_0$ is unknown. This interpolation process is repeated for all possible UAV locations, and the MAE is computed for a given altitude and threshold setting. By iterating this procedure across different altitude and threshold combinations, the MAE is evaluated for all possible configurations.
  
 Results in Figure~\ref{fig:MAE_interpolations} show that the Kriging interpolation-based 3D channel rank interpolation scheme outperforms the MAE of the two baseline interpolation-based approaches. It can be interpreted that the Kriging interpolation yields an accurate interpolation of channel rank using spatial correlation in the given area, while the baseline approaches employ channel rank data and the UAV location index. For all approaches, $K_1$ has the lowest MAE because most channel ranks at UAV locations have a channel rank of $1$ as seen in Figure \ref{fig:Rank_PDF_MIMO_10}. Moreover, it has been observed that the MAE with the altitude of $30$ m yields the highest MAE performance for all thresholds and approaches due to the blockage from the trees. The MAE performance in the lower altitudes, e.g., $30$ - $70$ m, tends to get worse as the threshold constant increases, which is from the different portion of channel rank distribution with the relaxed threshold. Meanwhile, an irregular pattern can be observed at higher altitudes, e.g., $80$ - $110$ m cases. This can be interpreted as being due to relatively low spatial correlation in the case of $K_2$ and $K_3$, where the ratio of channel ranks 1 and 2 is low.
 

  \begin{figure*}
     \centering
     \subfigure[Kriging Interpolation]{\includegraphics[trim={0.4cm 0.1cm 1.3cm 0.6cm},clip,width=0.66\columnwidth]{images/Kriging_channel_rank_MAE_bar_v5}
     \label{fig:MAE_kriging}
     }
     \subfigure[Baseline Interpolation: Makima Method]{\includegraphics[trim={0.4cm 0.1cm 1.3cm 0.6cm},clip,width=0.66\columnwidth]{images/makima_channel_rank_MAE_bar_v3}
     \label{fig:MAE_makima}
     }
     \subfigure[Baseline Interpolation: Spline Method]{\includegraphics[trim={0.4cm 0.1cm 1.3cm 0.6cm},clip,width=0.66\columnwidth]{images/spline_channel_rank_MAE_bar_v3}
     \label{fig:MAE_spline}
     }
     \caption{MAE performance comparison of Kriging interpolation-based 3D channel rank interpolation scheme and baseline interpolation-based approaches.}
     \label{fig:MAE_interpolations}
 \end{figure*}

 \subsection{Measurement Comparison}\label{ch:measurement}
In this section, we present the comparison of RT simulations and real-world measurements of RSS and channel rank. In addition to the RT simulation and measurement campaign setups described in Section~\ref{ch:RT_setup} and Section~\ref{ch:measurement_setup}, the following assumptions are used for the channel rank RT simulation for comparison purposes: 1) $2 \times 2$ equally spaced in vertical and horizontal axis by $0.5 \lambda$ is used for multi-antenna receiver setup with four antennas; 2) non-uniformly spaced linear array antenna is used for transmitter multi-antenna configuration, which has antenna element spacing of $60$~mm, $1.68$~m, and $60$~mm to simulate two sector antenna configuration used in our measurements; 3) $90^\circ$ of roll angle is applied for the receiver at the UAV to implement bottom mounted antenna module;  and 4) the number of samples for Fibonacci lattice unit sphere of Sionna RT is set to $10^6$ to get accurate ray information.
 
\subsubsection{Coverage Measurements} 

The RSS simulation results of RT and the measurements from each tower, over the trajectory of the UAV, as illustrated in Figure~\ref{fig:trajectory_measurement_RSS}, are shown in Figure \ref{fig:measurement_vs_sionna_LW1_LW5}. Measurement methodology for coverage measurements is presented earlier in Section~\ref{ch:measurement_setup}. The labels USRP1 and USRP2 indicate measurements captured simultaneously at two different antennas connected to a dual channel USRP at each tower, and $\mathsf{Z}$ is the out-of-coverage area. The RT approach shows a similar pattern with the measurements over LW1 to LW5 tower cases. It is observed that the measurements show fluctuations throughout the trajectory due to fading effects, changing roll/yaw/pitch of the UAV with respect to each tower, and Doppler effects, among other factors. 

An interesting behavior is observed at the RSS values observed by USRP1 and USRP2 antennas. While at some locations the measurements at USRP1 and USRP are very similar, at other locations, there may be over 10~dB of difference between them, see e.g. the interval right after waypoint C, for LW1, LW2, and LW4 cases, as highlighted on the figures. The RSS variation can be from the geometry-dependent LoS blockage and channel conditions due to different relative orientations of the USRP1 and USRP2 antennas with respect to the UAV. 

We can notice the out-of-coverage and low RSS conditions in RT simulation for LW3, LW4, and LW5 at lower altitudes, i.e., time intervals before $100$~s and after $850$~s when the drone is taking off and landing, respectively. These are due to altitude-dependent blockage at those towers. Moreover, blockage is also observed at LW4 and LW5 (further towers to the UAV's flight area) with RT for other UAV locations. This can be interpreted to be due to the RT ray calculation not being able to find a proper ray with the simulation settings of the number of samples for the Fibonacci lattice unit sphere and assumptions for the height of the trees. The RSS measured at LW5 and some intervals at LW4 show to be constant over time when it is possible to detect it, where the RSS is very close to the background noise due to the far distance with each of these towers.

  \begin{figure*}
     \centering
     \subfigure[LW1]{\includegraphics[trim={0.3cm 0.1cm 1.35cm 0.6cm},clip,width=0.64\columnwidth]{images/RSS_measurement_sionna_40m_LW1_v7}
     \label{fig:measurement_vs_sionna_LW1}
     }
     \subfigure[LW2]{\includegraphics[trim={0.3cm 0.1cm 1.35cm 0.6cm},clip,width=0.64\columnwidth]{images/RSS_measurement_sionna_40m_LW2_v7}
     \label{fig:measurement_vs_sionna_LW2}
     }
     \subfigure[LW3]{\includegraphics[trim={0.3cm 0.1cm 1.35cm 0.6cm},clip,width=0.64\columnwidth]{images/RSS_measurement_sionna_40m_LW3_v7}
     \label{fig:measurement_vs_sionna_LW3}
     }
     \subfigure[LW4]{\includegraphics[trim={0.3cm 0.1cm 1.35cm 0.6cm},clip,width=0.64\columnwidth]{images/RSS_measurement_sionna_40m_LW4_v7}
     \label{fig:measurement_vs_sionna_LW4}
     }
     \subfigure[LW5]{\includegraphics[trim={0.3cm 0.1cm 1.35cm 0.6cm},clip,width=0.64\columnwidth]{images/RSS_measurement_sionna_40m_LW5_v7}
     \label{fig:measurement_vs_sionna_LW5}
     }
     \caption{RSS measurement and RT simulation results with the predefined trajectory for the specific UAV locations {A}-{F} in Figure \ref{fig:trajectory_altitude_measurement_RSS}. }     \label{fig:measurement_vs_sionna_LW1_LW5}
 \end{figure*}

 \subsubsection{Channel Rank Measurements}
 
 RT simulations and real-world measurements to study the channel rank are done based on the trajectory shown earlier in Figure~\ref{fig:trajectory_measurement_rank}. Measurement methodology is presented earlier in Section~\ref{ch:measurement_setup}.   
 The channel rank simulation results of RT and the corresponding measurements are shown in Figure~\ref{fig:measurement_vs_sionna_rank}, where the LW1 tower is marked as a red triangle and the building in the target area is marked as a cyan polygon. As seen in the 5G NR RI measurements in Figure~\ref{fig:rank_zigzag_measurement}, the channel rank at most locations is 2, where one of the dual ports of each sector antenna supports spatial multiplexing. A channel rank of 3 has been observed at the locations on the northwest side of the LW1 tower, which can be interpreted as an additional spatial stream from a reflective path through the building near the tower. Moreover, a channel rank of 1 can also be observed on the south side of the target area implying a lack of angular separation over the given MIMO channel. 
 
 On the other hand, the RT simulation results are shown in Figure \ref{fig:rank_zigzag_RT}. Channel rank in RT simulation tends to be more variant compared to the measurements. A channel rank of 1 or 2 is observed in most UAV locations on the northwest side of the LW1 tower due to the orientation of the LW1 tower and the receiver of the UAV. On the other hand, a channel rank of 3 can be observed on the edge of each waypoint because of the reflections. It is worthwhile to note that it is inevitable to have a mismatch between RT and real-world measurements at specific locations due to the following reasons.
 \begin{itemize}
     \item \textbf{Angular mismatch.} For the LW1 tower antenna configuration, the dual port AW3232 sector antenna supports $+/- 45^\circ$ slant linear polarization. Moreover, the receiver antenna module mounted at the UAV also has a dual slant orientation structure. However, the element-wise rotation of the linear array configuration is not supported in the Sionna RT tool.
     \item \textbf{Dynamic vs. static environments.} There are environmental gaps between measurement and RT scenarios such as foliage patterns, temporal fadings, antenna behavior in real scenarios, and among others.
     \item \textbf{Rank calculation.} A threshold-based channel rank calculation is used for the RT scenarios. However, in the 5G NR RI measurements, a mobile device mounted on the UAV determines the RI to report to the LW1 tower. It is known that RI calculation is implement-dependent, which leads to the differences.
     % \item \textbf{Beamforming.} 
 \end{itemize}

The distribution of channel rank of RT with different $K$ can be a more meaningful comparison between RT simulations and measurements in the same environment. The histograms of the channel rank with RT and measurements are shown in Figure~\ref{fig:distribution_rank_RT_measurement}. The likelihood of having a channel rank of 2 or 3 is increased as $K$  increases from $K=K_1$ to $K=K_3$, which results in a higher similarity with the measurements. The mismatch between measurements and RT occurs mainly when the channel rank is 1. This suggests that, at most UAV locations, the second singular value for RT simulations does not fall within the range required for $K_3$ ($30$~dB) from the strongest singular value. Even though there are gaps in the channel rank between RT and measurement, RT can be a reasonable benchmark as a baseline in the controlled environment.
 %by analyzing the geographical impact on UAV links, which can be seen in Section \ref{ch:numerical_results}. 

 \begin{figure*}
     \centering       
      \subfigure[5G NR RI measurement]{\includegraphics[trim={0.2cm 0.1cm 0.9cm 0.65cm},clip,width=0.64\columnwidth]{images/channel_rank_zigzag_measurements_v4}
     \label{fig:rank_zigzag_measurement}
     }     
     \subfigure[RT simulation with $K_3$]{\includegraphics[trim={0.2cm 0.1cm 0.9cm 0.65cm},clip,width=0.64\columnwidth]{images/channel_rank_zigzag_RT_v4}
     \label{fig:rank_zigzag_RT}
     }
     \subfigure[Channel rank distribution with different $K$ of RT and measurements]{\includegraphics[trim={0.2cm 0.15cm 1.1cm 0.7cm},clip,width=0.6\columnwidth]{images/distribution_rank_RT_measurement_v2}
     \label{fig:distribution_rank_RT_measurement}
     }     
     \caption{5G NR RI measurement and RT simulation results with the predefined trajectory in Figure \ref{fig:trajectory_altitude_measurement_rank}.}
     \label{fig:measurement_vs_sionna_rank}
 \end{figure*}


 \section{Conclusions}
In this paper, we investigated RF coverage and channel rank in rural environments using the NVIDIA Sionna RT tool, extending our previous work \cite{previous_work} with a broader analysis of the Lake Wheeler Field Labs. Our study incorporated realistic foliage modeling, multiple base stations, and the Kriging interpolation-based 3D channel rank prediction scheme. We applied a constant threshold ratio for singular value selection and analyzed its distribution across different UAV altitudes and threshold settings. The observed spatial correlation of channel rank motivated the application of Kriging interpolation, where we derived the semi-variogram using the correlation-distance relationship at UAV positions. The proposed Kriging interpolation scheme was evaluated using the MAE metric and compared with two baseline interpolation methods. Results demonstrated that the Kriging-based approach outperformed baseline interpolation techniques by leveraging spatial correlation. Finally, we compared our RT-based RSS and channel rank simulation results with the real-world measurements collected from the NSF AERPAW testbed. The reasonable consistency between the RT results and the measurements was demonstrated in the foliage-dense rural scenarios.

For future work, unique challenges can be considered in UAV-based 6G and V2X communications including managing time-varying channel characteristics, ensuring ultra-reliable low-latency communication (URLLC), and optimizing spatial multiplexing under dynamic network conditions. Addressing the time-varying nature of signal sources, obstacles, and UAVs in dynamic environments requires advanced RT techniques to capture the rapid fluctuation of channel conditions. It enables analyzing adaptive spatial multiplexing strategies that leverage real-time channel conditions, geographical information, and mobility of UAVs. Moreover, improving the Kriging interpolation framework by incorporating data-driven and ML-aided techniques is another potential direction. By integrating ML with statistics of spatial correlation into Kriging interpolation techniques, it is expected that accuracy and computational complexity can be improved over various environment settings.


\bibliographystyle{IEEEtran}
%%
%% This is file `./samples/longsample.tex',
%% generated with the docstrip utility.
%%
%% The original source files were:
%%
%% apa7.dtx(with options: `longsample')
%% ----------------------------------------------------------------------
%% 
%% apa7 - A LaTeX class for formatting documents in compliance with the
%% American Psychological Association's Publication Manual, 7th edition
%% 
%% Copyright (C) 2019 by Daniel A. Weiss <daniel.weiss.led at gmail.com>
%% 
%% This work may be distributed and/or modified under the
%% conditions of the LaTeX Project Public License (LPPL), either
%% version 1.3c of this license or (at your option) any later
%% version.The latest version of this license is in the file:
%% 
%% http://www.latex-project.org/lppl.txt
%% 
%% Users may freely modify these files without permission, as long as the
%% copyright line and this statement are maintained intact.
%% 
%% This work is not endorsed by, affiliated with, or probably even known
%% by, the American Psychological Association.
%% 
%% ----------------------------------------------------------------------
%% 
\documentclass[
a4paper,
man,
british
]{apa7}

\usepackage{changes}
\usepackage{cancel}
\usepackage[british]{babel}
\usepackage[utf8]{inputenc}
\usepackage{epstopdf}
\usepackage{csquotes}
\usepackage[flushleft]{threeparttable}
\usepackage{multirow}
%\usepackage[hidelinks]{hyperref}
%\usepackage{authblk}
\usepackage[
style = apa,
backend = biber,
sortcites = true,
sorting = nyt,
%isbn = false,
%url = false,
%doi = false,
%eprint = false,
hyperref = true,
backref = false,
%firstinits = false,
]{biblatex}
\hypersetup{
colorlinks = true,
linkcolor = black,
anchorcolor = black,
citecolor = black,
filecolor = black,
urlcolor = blue
}
\usepackage{float}
\usepackage{placeins}
\usepackage{xcolor}
\usepackage[toc, page]{appendix}
\usepackage{lscape}
\usepackage{afterpage}
\usepackage{esvect}
\usepackage{amsmath}
\usepackage{ragged2e}
\justifying\let\raggedright\justifying
\usepackage{enumitem}
\usepackage{makecell}
\DeclareLanguageMapping{british}{british-apa}
\usepackage[nodisplayskipstretch]{setspace}
\usepackage{subcaption} 
\usepackage{rotating}
\usepackage{geometry}
\geometry{a4paper, margin=1in}

% maps apacite commands to biblatex commands

\let \citeNP \cite
\let \citeA \textcite
\let \cite \parencite

\newcommand{\figurehere}[1]{\begin{center}%
\vspace{-2mm}
=========================\\%
Insert Figure #1 about here\\%
=========================\\%
\vspace{-2mm}
\end{center}}
%%
%%Table goes about here command
\newcommand{\tablehere}[1]{\begin{center}%
\vspace{-2mm}
=========================\\%
Insert Table #1 about here\\%
=========================\\%
\vspace{-2mm}
\end{center}}

\usepackage{array}
\newcommand{\PreserveBackslash}[1]{\let\temp=\\#1\let\\=\temp}
\newcolumntype{C}[1]{>{\PreserveBackslash\centering}p{#1}}
\newcolumntype{R}[1]{>{\PreserveBackslash\raggedleft}p{#1}}
\newcolumntype{L}[1]{>{\PreserveBackslash\raggedright}p{#1}}

\addbibresource{tutorial_MH.bib}

\title{Tutorial on Using Machine Learning and Deep Learning Models for Mental Illness Detection}
\shorttitle{Data-driven Methods to Identify Mental Illness}

\authorsnames[1,2,2,3,4,5,1,5]{%
  Yeyubei Zhang, Zhongyan Wang, Zhanyi Ding, Yexin Tian, 
  Jianglai Dai, Xiaorui Shen, Yunchong Liu, Yuchen Cao
}
\authorsaffiliations{%
  {University of Pennsylvania, School of Engineering and Applied Science}, 
  {New York University, Center for Data Science}, 
  {Georgia Institute of Technology, College of Computing}, 
  {University of California, Berkeley, Department of EECS}, 
  {Northeastern University, Khoury College of Computer Science}
}

%\leftheader{Placeholder}

\abstract{Social media has become an important source for understanding mental health, providing researchers a way to detect conditions like depression from user-generated posts. This tutorial provides practical guidance to address common challenges in applying machine learning and deep learning methods for mental health detection on these platforms. It focuses on strategies for working with diverse datasets, improving text preprocessing, and addressing issues such as imbalanced data and model evaluation. Real-world examples and step-by-step instructions demonstrate how to apply these techniques effectively, with an emphasis on transparency, reproducibility, and ethical considerations. By sharing these approaches, this tutorial aims to help researchers build more reliable and widely applicable models for mental health research, contributing to better tools for early detection and intervention.}

\keywords{Mental Health Research, Machine Learning, Deep Learning, Social Media Analysis, Natural Language Processing}

\authornote{
% \addORCIDlink{Placeholder}{0000-0000-0000-0000}
Correspondence concerning this article should be addressed to Yuchen Cao, Northeastern University, E-mail: cao.yuch@northeastern.edu}

\begin{document}
\maketitle

\section{Introduction}

Mental health disorders, especially depression, have become a significant concern worldwide, affecting millions of individuals across diverse populations \cite{WHO2020}. Early detection of depression is crucial, as it can lead to timely treatment and better long-term outcomes. In today’s digital age, social media platforms such as X(Twitter), Facebook, and Reddit provide a unique opportunity to study mental health. People often share their thoughts and emotions on these platforms, making them a valuable source for understanding mental health patterns \cite{Choudhury2013, Guntuku2017}.


Recent advances in computational methods, particularly machine learning (ML) and deep learning (DL), have shown promise in analyzing social media data to detect signs of depression. These techniques can uncover patterns in language use, emotions, and behaviors that may indicate mental health challenges \cite{Shatte2020, Yazdavar2020}. 

However, applying these methods effectively is not without challenges. A recent systematic review \cite{cao2024mental} highlighted issues such as a lack of diverse datasets, inconsistent data preparation, and inadequate evaluation metrics for imbalanced data \cite{Hargittai2015, Helmy2024}—problems that have also led to inaccuracies in other domains (e.g., \cite{gao2024survey, weng2024ai}). Similarly, Liu et al. \cite{liu2024systematic} identified additional linguistic challenges in ML approaches for detecting deceptive activities on social networks, including biases from insufficient linguistic preprocessing and inconsistent hyperparameter tuning, all of which are pertinent to mental health detection. Moreover, complementary insights from related fields underscore the need for continuous improvements in robust model development \cite{bi2024decoding,zhao2024minimax,tao2023meta,xu2025robust}.


This tutorial is designed to address these gaps by guiding readers through the steps necessary to create reliable and accurate models for depression detection using social media data. It focuses on practical techniques to:
\begin{itemize}
\item Collect and preprocess data, including handling language challenges like sarcasm or negations.
\item Build and optimize models with attention to tuning and evaluation.
\item Use appropriate metrics for datasets where depressive posts are a minority.
\end{itemize}

Our goal is to provide a clear, step-by-step approach that researchers and practitioners can use to improve their methods. By addressing common challenges in this field, we hope to encourage more robust and ethical use of technology for improving mental health outcomes.

\section{Method}
This section provides a comprehensive overview of the methodological framework employed in this study, detailing the processes for data preparation, model development, and evaluation metrics. All analyses and model implementations were conducted using Python 3, leveraging popular libraries such as \texttt{pandas} for data manipulation, \texttt{scikit-learn} for machine learning, \texttt{PyTorch} for deep learning, and \texttt{Transformers} for pre-trained language models. These tools enabled efficient preprocessing, hyperparameter optimization, and performance evaluation. The following subsections elaborate on the key steps and methodologies involved in the study.

\subsection{Data Preparation}
\subsubsection{Data Sources and Collection Methods}
A sufficiently representative dataset is essential for machine-learning-based mental health detection. This study utilized the Sentiment Analysis for Mental Health dataset, available on \href{https://www.kaggle.com/datasets/suchintikasarkar/sentiment-analysis-for-mental-health/data}{Kaggle}. The dataset integrates textual content from multiple repositories focused on mental health topics, including depression, anxiety, stress, bipolar disorder, personality disorders, and suicidal ideation. The primary sources of these data are social media platforms such as Reddit, Twitter, and Facebook, where individuals frequently discuss personal experiences, emotional states, and mental health concerns.

The dataset was originally compiled using platform-specific APIs (e.g., Reddit, Twitter, and Facebook) and web scraping tools, allowing for the collection of substantial volumes of publicly available text data. After the acquisition, duplicates were removed, irrelevant and spam content was filtered, and mental health labels were standardized to ensure consistency across repositories. Personal identifiers were removed to safeguard privacy and ensure compliance with ethical guidelines for data usage. The final dataset was consolidated into a structured CSV file with unique identifiers for each entry.

Although the dataset combines data from multiple platforms to provide a diverse corpus, it is not free from limitations. Differences in platform demographics, such as age, cultural background, and communication styles, may affect the generalizability of models trained on this data. Additionally, linguistic variability, including colloquialisms, slang, and code-switching, reflects the informal nature of social media communication. While this diversity enriches the dataset, it also presents challenges for natural language processing (NLP) techniques, particularly in tokenization and embedding generation. To address these complexities, the preprocessing pipeline was designed to handle diverse linguistic patterns and balance class distributions where needed.

\subsubsection{Data Preprocessing}
A standardized preprocessing pipeline was applied to prepare the dataset for training both machine learning (ML) and deep learning (DL) models. These steps ensured consistency in data handling while accommodating the unique requirements of each modeling approach:
\vspace{-3mm}
\begin{itemize}
\item \textbf{Text Cleaning:} Social media text often contains noise such as URLs, HTML tags, mentions, hashtags, special characters, and extra whitespace. These elements were systematically removed using regular expressions to create cleaner input for both ML and DL models.
\vspace{-1.5mm}
\item \textbf{Lowercasing:} All text was converted to lowercase to maintain uniformity across the dataset and minimize redundancy in text representation.
\vspace{-1.5mm}
\item \textbf{Stopword Removal:} Commonly used words that provide little semantic value (e.g., “the,” “and,” “is”) were excluded using the stopword list available in the Natural Language Toolkit (NLTK) \cite{nltk_toolkit}, reducing noise while retaining meaningful content.
\vspace{-1.5mm}
\item \textbf{Lemmatization:} Words were reduced to their base forms (e.g., “running,” “ran,” “runs” → “run”) using NLTK's Lemmatizer. This step normalized variations of words, aiding both feature extraction and embedding generation.
\end{itemize}
\vspace{-3mm}

The dataset was divided into training, validation, and testing subsets using a two-step random sampling process with a fixed random seed to ensure reproducibility. First, 20\% of the data was set aside as the test set. The remaining 80\% was then further divided into a training set (60\% of the original data) and a validation set (20\% of the original data). This split ensured that the models were trained on the majority of the data while reserving separate subsets for hyperparameter tuning and final performance evaluation.

\subsubsection{Class Labeling}
The dataset’s class labels were prepared as follows: (1) For \textbf{multi-class classification}, the labels included six categories: Normal, Depression, Suicidal, Anxiety, Stress, and Personality Disorder. (2) For \textbf{binary classification}, the labels were grouped into two classes: Normal and Abnormal.

\subsubsection{Feature Transformation for ML Models}
For ML models, an additional step, TF-IDF Vectorization, was necessary to transform the text into structured features. The cleaned text was converted into numerical representations using Term Frequency–Inverse Document Frequency (TF-IDF), which captured term frequencies while down-weighting overly frequent words. The vectorizer was configured to extract up to 1,000 features and account for both unigrams and bigrams (n-gram range: 1–2).

\subsubsection{Code Availability}
The code for data preprocessing, including text cleaning, class labeling, and dataset splitting, is publicly available on GitHub (the link will be provided upon acceptance).

\subsection{Model Development}
This study employed a range of machine learning (ML) and deep learning (DL) models to analyze and classify mental health statuses based on textual data. Each model was selected to explore specific aspects of the data, from linear interpretability to handling complex patterns and long-range dependencies. Detailed implementation code for all models, including hyperparameter tuning and evaluation, is available on GitHub. Below, we provide an overview of each model, its methodology, and its performance in the context of binary and multi-class mental health classification tasks.

\subsubsection{Logistic Regression}
Logistic regression is one of the most widely used methods for classification tasks and has long been employed in social science and biomedical research \cite{hosmer2000applied, ding2025efficientpowerfultradeoffsmachine}. In the context of mental health detection, it provides a straightforward yet interpretable modeling framework, translating linear combinations of predictors (e.g., term frequencies, sentiment scores, and linguistic features) into estimated probabilities of class membership through the logit function.

The logistic regression model predicts the probability of a binary outcome using the following expression:
\begin{equation}\nonumber
\hat{y} = \frac{1}{1 + \exp(-\mathbf{w}^\top \mathbf{x} - b)},
\end{equation}
where $\hat{y}$ represents the predicted probability, $\mathbf{w}$ is the vector of model coefficients, $\mathbf{x}$ is the feature vector, and $b$ is the bias term. For multi-class classification, the model generalizes to predict probabilities for $K$ classes using the softmax function:
\begin{equation}\nonumber
P(y = k \mid \mathbf{x}) = \frac{\exp(\mathbf{w}_k^\top \mathbf{x} + b_k)}{\sum_{j=1}^{K} \exp(\mathbf{w}_j^\top \mathbf{x} + b_j)},
\end{equation}
where $k \in \{1, \dots, K\}$ represents the class index.

Both binary and multi-class logistic regression models were optimized using cross-entropy loss during training and configured to converge with a maximum iteration limit of 1,000. Regularization was applied to prevent overfitting, using $\ell_2$ (ridge) regularization, which penalizes large coefficients by adding their squared magnitude to the loss function:
\begin{equation}\nonumber
\mathcal{L} = - \frac{1}{n} \sum_{i=1}^{n} \left[ y_i \log(\hat{y}_i) + (1 - y_i) \log(1 - \hat{y}_i) \right] + \lambda \|\boldsymbol{\beta}\|_2^2,
\end{equation}
where $\lambda$ controls the regularization strength, $y_i$ is the true label, $\hat{y}_i$ is the predicted probability, and $\boldsymbol{\beta}$ represents the model coefficients.

Hyperparameter tuning was conducted using a grid search across several parameters. The regularization strength (\(C\)), which is the inverse of the regularization parameter \(\lambda\), was tested with values such as 0.1, 1, and 10. Various optimizers, including \texttt{liblinear} (Library for Large Linear Classification), \texttt{lbfgs} (Limited-memory Broyden–Fletcher–Goldfarb–Shanno), and \texttt{saga} (Stochastic Average Gradient Augmented), were evaluated for optimization. To address class imbalance, the \texttt{class\_weight} parameter was explored with options for \texttt{balanced} and \texttt{None}. For multi-class tasks, the \texttt{multinomial} strategy was employed, while the \texttt{one-vs-rest} strategy was implicitly applied for binary classification scenarios.

For both binary and multi-class tasks, the weighted F1 score was used as the primary evaluation metric, ensuring balanced performance across categories, including minority classes. A combined grid search configuration was applied for both tasks, as their hyperparameter spaces largely overlapped. The best configurations effectively handled class imbalance using the \texttt{class\_weight=`balanced'} parameter, yielding robust performance on the validation and test sets.

The logistic regressions were implemented using the \texttt{LogisticRegression} class from the \texttt{scikit-learn} library. Detailed implementation code for logistic regression, including hyperparameter tuning and evaluation, is available on GitHub.

\subsubsection{Support Vector Machine (SVM)}
Support Vector Machines (SVMs) are supervised learning models that are widely used for both classification and regression tasks. Originally introduced by \citeA{cortes1995support}, SVMs aim to find the optimal hyperplane that maximizes the margin between data points of different classes. The margin is defined as the distance between the closest data points (support vectors) from each class to the hyperplane. By maximizing this margin, SVMs achieve better generalization for unseen data.

For a linearly separable dataset, the decision boundary is defined as:
\begin{equation}\nonumber
f(\mathbf{x}) = \mathbf{w}^T \mathbf{x} + b,
\end{equation}
where $\mathbf{w}$ is the weight vector, $\mathbf{x}$ is the feature vector, and $b$ is the bias term. The optimal hyperplane is determined by solving the following optimization problem:
\begin{align}
\min_{\mathbf{w}, b} & \quad \frac{1}{2} \|\mathbf{w}\|^2, \nonumber\\
\text{subject to} & \quad y_i (\mathbf{w}^T \mathbf{x}_i + b) \geq 1, \quad i = 1, \dots, N,\nonumber
\end{align}
where $y_i \in \{-1, +1\}$ are the class labels.

For datasets that are not linearly separable, the optimization problem is modified to include a penalty for misclassifications:
\begin{align}
\min_{\mathbf{w}, b, \xi} & \quad \frac{1}{2} \|\mathbf{w}\|^2 + C \sum_{i=1}^N \xi_i, \nonumber\\
\text{subject to} & \quad y_i (\mathbf{w}^T \mathbf{x}_i + b) \geq 1 - \xi_i, \quad \xi_i \geq 0, \quad i = 1, \dots, N,\nonumber
\end{align}
where $\xi_i$ are slack variables that allow for misclassifications, and $C > 0$ is the regularization parameter that controls the trade-off between maximizing the margin and minimizing classification errors.

Kernel methods enable SVMs to handle nonlinearly separable data by mapping the input features into a higher-dimensional space where linear separation becomes possible. This mapping is performed implicitly using a kernel function \( K(\mathbf{x}_i, \mathbf{x}_j) \), which computes the inner product in the transformed space:
\begin{equation}\nonumber
K(\mathbf{x}_i, \mathbf{x}_j) = \phi(\mathbf{x}_i)^T \phi(\mathbf{x}_j),
\end{equation}
where \( \phi(\cdot) \) represents the mapping function.

Several commonly used kernel functions are available, each suited for different data characteristics:

\vspace{-3mm}
\begin{enumerate}
\item \textbf{Linear Kernel:}
\begin{equation}\nonumber
K(\mathbf{x}_i, \mathbf{x}_j) = \mathbf{x}_i^T \mathbf{x}_j
\end{equation}
This kernel computes the dot product of the input vectors and is suitable for linearly separable data.
\vspace{-1.5mm}
\item \textbf{Polynomial Kernel:}
\begin{equation}\nonumber
K(\mathbf{x}_i, \mathbf{x}_j) = (\mathbf{x}_i^T \mathbf{x}_j + c)^d,
\end{equation}
where \( c \) is a constant and \( d \) is the degree of the polynomial. This kernel is useful for capturing polynomial feature interactions.
\vspace{-1.5mm}
\item \textbf{Radial Basis Function (RBF) Kernel:}
\begin{equation}\nonumber
K(\mathbf{x}_i, \mathbf{x}_j) = \exp\left(-\gamma \|\mathbf{x}_i - \mathbf{x}_j\|^2\right),
\end{equation}
where \( \gamma \) controls the influence of individual training samples. The RBF kernel is widely used for its flexibility in modeling complex, nonlinear patterns.
\vspace{-1.5mm}
\item \textbf{Sigmoid Kernel:}
\begin{equation}\nonumber
K(\mathbf{x}_i, \mathbf{x}_j) = \tanh(\alpha \mathbf{x}_i^T \mathbf{x}_j + c),
\end{equation}
where \( \alpha \) is a scaling factor and \( c \) is a bias term. This kernel is inspired by neural network activation functions and is suitable for data with specific characteristics.
\vspace{-1.5mm}
\item \textbf{Custom Kernels:}
Custom-defined kernels can be tailored for domain-specific tasks, offering flexibility for unique datasets or similarity metrics.
\end{enumerate}
\vspace{-3mm}

In this project, kernel selection was based on preliminary experiments, with the linear and radial basis function (RBF) kernels being the primary choice due to its ability to model complex, nonlinear relationships effectively.

For both binary and multi-class classification tasks, the same hyperparameter tuning strategy was employed. A grid search was conducted over the following hyperparameters:
\vspace{-3mm}
\begin{itemize}
\item Regularization parameter $C$: values of \{0.1, 1, 10\}.
\vspace{-1.5mm}
\item Kernel type: linear and RBF.
\vspace{-1.5mm}
\item Class weight: balanced or none.
\vspace{-1.5mm}
\item Gamma (for RBF kernel): scale and auto.
\end{itemize}
\vspace{-3mm}
The grid search aimed to identify the optimal combination of hyperparameters using the weighted F1 score as the primary evaluation metric. For multi-class classification, the one-vs-one strategy inherent to the \texttt{SVC} implementation was used.

The loss function for SVM is analogous to logistic regression, as both models minimize the cross-entropy loss during optimization. However, for SVM, the hinge loss is typically used for linear separable cases, defined as:
\begin{equation}\nonumber
\mathcal{L}_{\text{hinge}} = \frac{1}{N} \sum_{i=1}^N \max(0, 1 - y_i f(\mathbf{x}_i)).
\end{equation}

The SVM models were implemented with the \texttt{SVC} class from \texttt{scikit-learn}. Detailed implementation code for SVMs, including grid search and evaluation, is available on GitHub.

\subsubsection{Tree-Based Models}
Classification and Regression Trees (CARTs) are versatile tools used for analyzing categorical outcomes (classification tasks). The CART algorithm constructs a binary decision tree by recursively partitioning the data based on covariates, optimizing a predefined splitting criterion. For classification tasks, the quality of a split is typically evaluated using impurity measures such as Gini impurity or entropy \cite{bishop2006pattern}. The Gini impurity for a node is defined as:
\begin{equation}\nonumber
G = \sum_{i=1}^C p_i (1 - p_i),
\end{equation}
where \(p_i\) is the proportion of observations in class \(i\) at the given node, and \(C\) is the total number of classes.

Alternatively, entropy can be used to measure impurity:
\begin{equation}\nonumber
H = -\sum_{i=1}^C p_i \log(p_i),
\end{equation}
where \(p_i\) represents the same class proportions as in the Gini impurity formula. Lower impurity values indicate greater homogeneity within a node.

At each step, the algorithm selects the split that minimizes the weighted impurity of the child nodes. The impurity reduction for a given split is computed as:
\begin{equation}\nonumber
\Delta I = I_{\text{parent}} - \left( \frac{n_L}{n} I_L + \frac{n_R}{n} I_R \right),
\end{equation}
where \(I_{\text{parent}}\) is the impurity of the parent node, \(I_L\) and \(I_R\) are the impurities of the left and right child nodes, \(n_L\) and \(n_R\) are the number of observations in the left and right child nodes, and \(n\) is the total number of observations in the parent node.

The splitting process continues until one stopping criterion is met. Common criteria include: (1) a minimum number of samples in a node, (2) a maximum tree depth, and (3) No further reduction in impurity beyond a predefined threshold.

To address overfitting, pruning techniques \cite{breiman1984classification} are employed. Pruning reduces the tree size by removing splits that contribute minimally to predictive performance, improving the model's generalizability.

Due to their tendency to overfit, simple CART models were not evaluated in this project. Instead, ensemble methods like Random Forests and Gradient Boosted Trees, which combine multiple CART models, were used for improved robustness.

\paragraph{Random Forests}
Random Forests are ensemble learning methods that aggregate multiple decision trees parallelly to enhance classification performance. By building trees on bootstrap samples of the data and introducing random feature selection at each split, Random Forests reduce overfitting and improve generalization. Each tree is trained on a random bootstrap sample, where data points are sampled with replacement from the original dataset, meaning some observations may appear multiple times in the training sample, while others are excluded. Additionally, Random Forests introduce randomness during the tree-building process by selecting a random subset of covariates at each split instead of considering all available covariates. This randomization decorrelates the trees, reduces variance, and enhances the model's robustness. For classification tasks, the final prediction is determined by majority voting across all trees \cite{breiman2001random}.

To further mitigate overfitting, each tree in the Random Forest is grown to its full depth without pruning, fitting the bootstrap sample as accurately as possible. Hyperparameters such as the number of trees (\texttt{n\_estimators}), the maximum depth of each tree (\texttt{max\_depth}), and the minimum samples required to split a node (\texttt{min\_samples\_split}) or form a leaf (\texttt{min\_samples\_leaf}) play a critical role in balancing bias and variance. The parameter \texttt{class\_weight}, when set to \texttt{`balanced'}, adjusts weights inversely proportional to class frequencies, effectively addressing the class imbalance.

A grid search approach was employed to optimize key hyperparameters for both binary and multi-class classification tasks. The parameter grid explored values such as 50, 100, and 200 for the number of trees (\texttt{n\_estimators}), depths of 10, 20, or unrestricted (\texttt{None}) for \texttt{max\_depth}, and split criteria (\texttt{min\_samples\_split} and \texttt{min\_samples\_leaf}) to control tree complexity. The weighted F1 score served as the primary evaluation metric to account for imbalances in the dataset. For the binary classification task, the best-performing model, determined through validation, effectively handled class imbalance and demonstrated robust predictive performance for distinguishing between Normal and Abnormal mental health statuses. In addition to traditional hyperparameter tuning techniques, recent studies have explored novel metaheuristic approaches to optimize Random Forest parameters. For instance, Tan et al. \cite{tan2024dung} proposed an improved dung beetle optimizer that refines hyperparameter tuning, further enhancing model performance.

For the multi-class classification task, the same hyperparameter grid was used with a slightly reduced scope to streamline the search process. The weighted F1 score guided model selection across all classes, including Normal, Depression, Anxiety, and Personality Disorder. The optimal model achieved balanced performance across multiple categories, leveraging Random Forests' ability to aggregate predictions from diverse decision trees. 

Random Forests’ inherent feature importance metrics provided additional insights into the most influential predictors for mental health classification. This capability enhances interpretability by highlighting covariates that most strongly influence predictions. The Random Forest models were built using the \texttt{RandomForestClassifier} from \texttt{scikit-learn}. Parameter grids for the number of estimators, maximum depth, and other parameters were evaluated with \texttt{GridSearchCV}. Detailed implementation code, including grid search and evaluation procedures, is available on GitHub.

\paragraph{Light Gradient Boosting Machine (LightGBM)}
Light Gradient Boosting Machine (LightGBM) is a gradient-boosting framework optimized for efficiency and scalability, particularly in handling large datasets and high-dimensional data. Gradient Boosting Machines (GBM) work by sequentially building decision trees, where each new tree corrects the errors made by the previous ones, leading to highly accurate predictions. However, traditional GBM frameworks can be computationally intensive, especially for large datasets \cite{friedman2001greedy}. Unlike traditional Gradient Boosting Machines (GBMs), LightGBM employs a leaf-wise tree growth strategy, which enables deeper splits in dense data regions, enhancing performance by focusing complexity where it is most needed. Additional optimizations, such as histogram-based feature binning, reduce memory usage and accelerate training. These enhancements make LightGBM faster and more resource-efficient than standard GBM implementations, without compromising predictive accuracy \cite{ke2017lightgbm}.

Key hyperparameters tuned for LightGBM included the number of boosting iterations (\texttt{n\_estimators}), learning rate, maximum tree depth (\texttt{max\_depth}), number of leaves (\texttt{num\_leaves}), and minimum child samples (\texttt{min\_child\_samples}). To address the class imbalance, the \texttt{class\_weight} parameter was tested with both \texttt{`balanced'} and \texttt{None} options. Grid search was employed to explore all possible combinations of these hyperparameters, and the weighted F1 score was used as the primary metric for selecting the best configuration.

LightGBM was applied to both binary and multi-class mental health classification tasks. For binary classification, the model differentiated between Normal and Abnormal statuses. For multi-class classification, it predicted categories such as Normal, Depression, Anxiety, and Personality Disorder using the \texttt{multiclass} objective. Hyperparameter tuning via grid search ensured balanced performance across all categories, guided by the weighted F1 score. 

The best-performing models demonstrated robust predictive power, evaluated using precision, recall, F1 scores, confusion matrices, and one-vs-rest ROC curves. Additionally, LightGBM’s feature importance metrics provided interpretability by highlighting the most influential linguistic and sentiment-based features. Its combination of high performance, scalability, and interpretability made LightGBM a key component in this project. The LightGBM models were developed using the \texttt{LGBMClassifier} from the \texttt{lightgbm} library. Hyperparameter tuning, including the number of boosting iterations, learning rate, and tree depth, was performed using \texttt{GridSearchCV}.
Detailed implementation code, including grid search procedures, is available on GitHub.

\subsubsection{A Lite version of Bidirectional Encoder Representations from Transformers (ALBERT)}
A Lite version of Bidirectional Encoder Representations from Transformers (BERT), known as ALBERT \cite{lan2020albert}, is a transformer-based model designed to improve efficiency while maintaining performance. While BERT \cite{devlin2019bert} is highly effective for a wide range of natural language processing (NLP) tasks, it is computationally expensive and memory-intensive due to its large number of parameters. ALBERT addresses these limitations by introducing parameter-sharing across layers and factorized embedding parameterization, which significantly reduces the number of parameters without sacrificing model capacity. Additionally, ALBERT employs Sentence Order Prediction (SOP) as an auxiliary task to enhance pretraining, improving its ability to capture sentence-level coherence. These optimizations make ALBERT a lightweight yet powerful alternative to BERT, capable of achieving competitive performance with reduced memory and computational requirements, making it particularly suitable for large-scale text classification tasks like mental health detection.

In this project, ALBERT was employed for both binary and multi-class classification tasks. For binary classification, the model was fine-tuned to differentiate between Normal and Abnormal mental health statuses, while for multi-class classification, it was configured to predict multiple categories, including Normal, Depression, Anxiety, and Personality Disorder. The implementation leveraged the pre-trained \texttt{Albert-base-v2} model, with random hyperparameter tuning conducted over 10 iterations to optimize the learning rate, number of epochs, and dropout rates. The weighted F1 score served as the primary evaluation metric throughout the tuning process.

For both binary and multi-class classification tasks, hyperparameter tuning was conducted to optimize learning rates between $10^{-5}$ and $10^{-4}$, dropout rates between 0.1 and 0.5, and epochs ranging from 3 to 5. For binary classification, the model achieved high validation F1 scores and demonstrated strong generalization on the test set. For multi-class classification, the objective was adjusted to predict seven categories, with weighted cross-entropy loss applied to address class imbalances and ensure adequate representation of minority categories. The final models were evaluated on the test set using metrics such as accuracy, weighted F1 scores, and confusion matrices.

ALBERT’s architecture efficiently captures long-range dependencies in text while retaining the computational advantages of its lightweight design. The use of random hyperparameter tuning further refined its performance, enabling robust classification for both binary and multi-class tasks. The ALBERT models were fine-tuned with the \texttt{Transformers} (\texttt{AlbertTokenizer} and \texttt{AlbertForSequenceClassification}) library from Hugging Face. Hyperparameter tuning was conducted manually through random search over learning rates, dropout rates, and epochs. etailed implementation code, including data preparation, training, and hyperparameter tuning, is available on GitHub.

\subsubsection{Gated Recurrent Units (GRUs)}
Gated Recurrent Units (GRUs) are a type of recurrent neural network (RNN) designed to capture sequential dependencies in data, making them particularly effective for natural language processing (NLP) tasks such as text classification \cite{cho2014learning}. Compared to Long Short-Term Memory networks (LSTMs), GRUs are computationally more efficient due to their simplified architecture, which combines the forget and input gates into a single update gate. This efficiency allows GRUs to model long-range dependencies while reducing the number of trainable parameters.

In this study, GRUs were employed for both binary and multi-class mental health classification tasks. For binary classification, the model was configured to differentiate between Normal and Abnormal mental health statuses. For multi-class classification, it was adapted to predict categories such as Normal, Depression, Anxiety, and Personality Disorder. 

The GRU architecture comprised three key components:
\vspace{-3mm}
\begin{enumerate}
\item \textbf{Embedding Layer}: Converts token indices into dense vector representations of a fixed embedding dimension.
\vspace{-1.5mm}
\item \textbf{GRU Layer}: Processes input sequences and retains contextual information across time steps, utilizing only the final hidden state for classification.
\vspace{-1.5mm}
\item \textbf{Fully Connected Layer}: Maps the hidden state to output logits corresponding to the number of classes.
\end{enumerate}
\vspace{-3mm}
Dropout regularization was applied to prevent overfitting, and a weighted cross-entropy loss function was used to address class imbalances in the dataset.

For both binary and multi-class classification tasks, hyperparameter tuning was conducted using random search across predefined ranges. The parameters optimized included embedding dimensions (150--250), hidden dimensions (256--768), learning rates ($10^{-4}$--$10^{-3}$), and epochs (5--10). The weighted F1 score served as the primary evaluation metric during validation. The best-performing models achieved high F1 scores on validation datasets and demonstrated robust generalization on the test sets.

GRUs excelled at capturing sequential patterns in text, enabling the model to identify linguistic cues associated with mental health conditions. Despite being less interpretable than tree-based models, their lightweight architecture ensured computational efficiency and strong performance in text-based classification tasks. The GRU models were implemented with the \texttt{torch.nn} module in PyTorch. Key layers included \texttt{nn.Embedding}, \texttt{nn.GRU}, and \texttt{nn.Linear}. Optimization was performed using the \texttt{torch.optim.Adam} optimizer, and class weights were applied using \texttt{nn.CrossEntropyLoss}.
Detailed implementation code, including data preprocessing, model training, and evaluation, is available on GitHub.

\subsection{Evaluation Metrics}
When modeling mental health statuses—particularly for conditions like depression or suicidal ideation—class distributions are often skewed. In many real-world scenarios, the “positive” class (e.g., individuals experiencing depression) is underrepresented compared to the “negative” class (e.g., no mental health issue). This imbalance renders certain evaluation metrics, such as accuracy, less informative: a model that predicts “no issue” for every instance might still achieve high accuracy if the majority class dominates. Consequently, more nuanced metrics are preferred to evaluate the performance of classification models:

\subsubsection{Precision} 
Precision measures the proportion of positive predictions that are truly positive:
\begin{equation}
 \text{Precision} = \frac{\text{True Positives}}{\text{True Positives} + \text{False Positives}}.
\end{equation}
For instance, in depression detection, high precision indicates that most users flagged as “depressed” indeed exhibit depressive content. While precision minimizes false alarms, focusing on it exclusively can be risky. A model that generates very few positive predictions may achieve artificially high precision while missing many genuinely positive cases.

\subsubsection{Recall (Sensitivity)} 
Recall captures the proportion of actual positives correctly identified:
\begin{equation}\nonumber
 \text{Recall} = \frac{\text{True Positives}}{\text{True Positives} + \text{False Negatives}}.
\end{equation}
In depression detection, recall is critical because failing to recognize at-risk individuals (false negatives) can have serious consequences. A model with low recall risks overlooking individuals who need intervention.

\subsubsection{F1 Score} 
The F1 score serves as the harmonic mean of precision and recall, providing a balance between these two metrics \cite{powers2011evaluation}:
\begin{equation}\nonumber
 F1 = 2 \cdot \frac{\text{Precision} \cdot \text{Recall}}{\text{Precision} + \text{Recall}}.
\end{equation}
The F1 score is particularly useful in imbalanced classification scenarios because it penalizes extreme trade-offs, such as very high precision coupled with very low recall. In mental health detection, achieving a high F1 score ensures the model can effectively identify positive cases while maintaining a reasonable level of precision in its predictions.

\subsubsection{Area Under the Receiver Operating Characteristic Curve (AUROC)} 
AUROC provides an aggregate measure of performance across all possible classification thresholds. It evaluates the model's ability to discriminate between positive and negative classes. However, in the presence of severe class imbalance, AUROC may not fully reflect the challenges posed by a majority class dominating the dataset. Nevertheless, it remains valuable for assessing model performance across varying decision thresholds \cite{davis2006relationship}.

\section{Results}
This section presents the findings from the analysis of the dataset and the evaluation of machine learning and deep learning models for mental health classification. First, we provide an \textit{Overview of Mental Health Distribution}, highlighting the inherent class imbalances within the dataset and their implications for model development. Next, the \textit{Hyperparameter Optimization} subsection details the parameter tuning process, which ensures that each model performs at its best configuration for both binary and multi-class classification tasks. Finally, the \textit{Model Performance Evaluation} subsection compares the models' performance based on key metrics, including F1 scores and Area Under the Receiver Operating Characteristic Curve (AUROC). Additionally, nuanced observations, such as the challenges associated with underrepresented classes, are discussed to provide deeper insights into the modeling outcomes.

\subsection{Overview of Mental Health Distribution}
Before hyperparameter optimization and model evaluation, an analysis of the dataset’s class distributions was conducted to highlight potential challenges in classification. The dataset, sourced from Kaggle, contains a total of 51,074 unique statements categorized into three primary groups: \textit{Normal} (31\%), \textit{Depression} (29\%), and \textit{Other} (40\%). The \textit{Other} category encompasses a range of mental health statuses such as \textit{Anxiety}, \textit{Stress}, and \textit{Personality Disorder}, among others.

\textbf{Figure~\ref{fig:multi-class}} illustrates the expanded distribution of mental health statuses across seven detailed categories in the multi-class classification setup. The dataset shows a significant imbalance, with categories such as \textit{Normal}, \textit{Depression}, and \textit{Suicidal} dominating the distribution, while others like \textit{Stress} and \textit{Personality Disorder} are notably underrepresented. This class imbalance poses challenges for multi-class classification tasks, particularly for the accurate identification of minority classes. Addressing such imbalances requires techniques like class-weighted loss functions and the use of metrics such as weighted F1 scores for model evaluation.

\figurehere{1}

\figurehere{2}

For the binary classification task, the dataset is divided into two classes: \textit{Normal} and \textit{Abnormal}. The distribution, shown in \textbf{Figure~\ref{fig:binary-class}}, reveals that the \textit{Abnormal} class (labeled as 1) accounts for approximately twice the number of records as the \textit{Normal} class (labeled as 0). Although the imbalance is less severe compared to the multi-class scenario, it still necessitates strategies to ensure that the minority class (\textit{Normal}) is adequately captured during model training.

\subsection{Hyperparameter Optimization}
Hyperparameter optimization is a critical step in enhancing the performance of machine learning (ML) and deep learning (DL) models. For this study, a grid search or random search approach was employed to systematically explore a predefined range of hyperparameters for each model. The primary evaluation metric used to select the best-performing hyperparameter configuration was the weighted F1 score, as it effectively balances precision and recall, particularly in the presence of imbalanced class distributions. This approach ensures that the selected models perform robustly across both binary and multi-class mental health classification tasks. 

The optimized hyperparameters for each model, alongside their corresponding weighted F1 scores on the test set, are summarized in Table~\ref{tbl1:opt_hp}. These results highlight the configurations that achieved the best trade-off between underfitting and overfitting, providing insight into the hyperparameter values critical to the classification tasks.

\tablehere{1}

\subsection{Model Performance Evaluation}

The evaluation metrics, including F1 scores (\textbf{Table~\ref{tbl2:f1-scores}}) and Area Under the Receiver Operating Characteristic Curve (AUROC) (\textbf{Table~\ref{tbl3:auc_scores}}), reveal minimal numeric differences across the models for both binary and multi-class classification tasks. This consistency in performance can be attributed to two primary factors. First, each model underwent rigorous hyperparameter tuning, ensuring only the best configurations were used for evaluation. Second, the dataset size, being of medium volume, provided sufficient information for machine learning models to achieve strong performance, while deep learning models could not fully demonstrate their potential advantages due to the limited data scale.

\tablehere{2}

\tablehere{3}

In the binary classification task, all models exhibited competitive F1 scores and AUROC values, effectively balancing precision and recall while distinguishing between normal and abnormal mental health statuses. Deep learning models such as \textit{ALBERT} and \textit{GRU} demonstrated slightly superior performance, achieving AUROC values of 0.95 and 0.94, respectively, which highlights their ability to capture complex linguistic patterns. Machine learning models, including \textit{Logistic Regression} and \textit{LightGBM}, also performed well, with AUROC scores of 0.93, underscoring their robustness in simpler classification settings.

In the multi-class classification task, a slight decline in performance was observed compared to the binary task. This decline aligns with the increased complexity of distinguishing between seven mental health categories. Nevertheless, deep learning models retained their advantage, with \textit{GRU} and \textit{LightGBM} achieving the highest micro-average AUROC scores of 0.97, followed closely by \textit{ALBERT} with an AUROC of 0.95. Machine learning models such as \textit{Logistic Regression} and \textit{Random Forest} also performed commendably, with AUROC scores of 0.96, demonstrating their ability to handle multi-class tasks effectively when optimized.

Another important observation in the multi-class classification task is the consistently lower AUROC scores for Depression (Class 2) across all machine learning models, with values not exceeding 0.90. While deep learning models demonstrated a slight improvement, their performance for this class remained comparatively weaker than for other categories. This difficulty likely arises from the significant overlap between Depression (Class 2) and other categories in both linguistic and contextual features. The reduced AUROC scores highlight the models' challenges in effectively distinguishing Depression, resulting in higher misclassification rates. These findings emphasize the need for refined feature engineering techniques or more sophisticated model architectures to enhance the separability and accurate classification of this particular class.

The minimal differences in performance metrics across models suggest that the combined effects of comprehensive hyperparameter optimization and dataset size contributed significantly to these results. Binary classification consistently outperformed multi-class classification, likely due to its reduced complexity and fewer decision boundaries. While deep learning models demonstrated their ability to capture intricate patterns, machine learning models offered competitive performance, making them practical alternatives for medium-sized datasets.

Performance metrics for F1 scores and AUROC values are detailed in \textbf{Table~\ref{tbl2:f1-scores}} and \textbf{Table~\ref{tbl3:auc_scores}}, respectively. This analysis highlights the importance of balancing model complexity with dataset characteristics and emphasizes the critical role of hyperparameter tuning in achieving optimal results.

\section{Discussion}

This tutorial serves as a practical resource to address key methodological and analytical challenges in mental health detection on social media, as identified in the systematic review \cite{cao2024mental}. By focusing on best practices and reproducible methods, the tutorial aims to advance research quality and promote equitable outcomes in this important field.

A critical issue identified in the review is the narrow scope of datasets, which are often limited to specific social media platforms, languages, or geographic regions. This lack of diversity restricts the generalizability of findings. In this tutorial, strategies for expanding data diversity are explored, including integrating datasets across multiple platforms, collecting data from underrepresented regions, and analyzing multilingual content. These efforts aim to make research outcomes more inclusive and applicable to diverse populations.

Text preprocessing emerged as another key challenge, particularly in handling linguistic complexities such as negations and sarcasm. These nuances are critical for accurately interpreting mental health expressions. This tutorial offers practical guidelines for building preprocessing pipelines that address these complexities. Techniques for advanced tokenization, feature extraction, and managing contextual meanings are discussed to enhance the reliability of text-based analyses.

Research practices related to model optimization and evaluation were also found to be inconsistent in many studies. Hyperparameter tuning and robust data partitioning are essential for reliable outcomes, yet they are often inadequately implemented. This tutorial provides step-by-step instructions for optimizing models and ensuring fair evaluations, emphasizing the importance of strategies like cross-validation and train-validation-test splits. By following these practices, researchers can reduce bias and improve the validity of their results. 

Evaluation metrics were another area of concern, with many studies relying on accuracy despite its limitations in imbalanced datasets. This tutorial highlights the importance of metrics such as precision, recall, F1-score, and AUROC, which provide a more balanced assessment of model performance. Additionally, practical approaches to managing imbalanced datasets, including oversampling, undersampling, and synthetic data generation, are illustrated.

Transparency in reporting methodologies and results is a foundational element of reproducible research. This tutorial encourages researchers to document their processes comprehensively, including data collection, preprocessing, model development, and evaluation. Sharing code and datasets is also emphasized, fostering collaboration and allowing other researchers to validate findings.

Ethical considerations are central to mental health research, particularly when using sensitive social media data. This tutorial stresses the need for privacy protection and adherence to ethical standards, ensuring that research respects the rights and dignity of individuals. Responsible data handling and clear communication of ethical practices are essential for maintaining trust and accountability in this field.

By addressing these challenges, this tutorial equips researchers with the tools and practices needed to improve the quality and impact of their work. Ultimately, these advancements contribute to the broader goal of promoting equitable and effective mental health interventions on a global scale.

\printbibliography

\newpage
\begin{sidewaystable}[ht]
\centering
\caption{Best Hyperparameters for Binary and Multi-Class Classification Models}
\label{tbl1:opt_hp}
\begin{tabular}{|l|p{5cm}|p{5cm}|p{9cm}|}
\hline
\textbf{Model} & \textbf{Best Parameters (Binary)} & \textbf{Best Parameters (Multi-Class)} & \textbf{Interpretation} \\
\hline
\textbf{Logistic Regression} & 
\texttt{\{C: 10, solver: `liblinear', penalty: `l2', class\_weight: None\}} & 
\texttt{\{C: 10, solver: `lbfgs', penalty: `l2', multi\_class: `multinomial', class\_weight: `balanced'\}} & 
For binary tasks, \texttt{liblinear} is chosen for smaller datasets. For multi-class, \texttt{lbfgs} supports \texttt{`multinomial'} strategy to optimize across multiple categories. Regularization strength (\texttt{C}) of 10 prevents overfitting. \\
\hline
\textbf{SVM} & 
\texttt{\{C: 1, kernel: `rbf', class\_weight: `balanced', gamma: `scale'\}} & 
\texttt{\{C: 1, kernel: `rbf', class\_weight: `balanced', gamma: `scale'\}} &
The RBF kernel captures nonlinear relationships in text data, while \texttt{class\_weight: `balanced'} was selected to address class imbalance. Regularization strength (\texttt{C}) balances margin maximization and misclassification. \\
\hline
\textbf{Random Forest} & 
\texttt{\{n\_estimators: 100, max\_depth: None, min\_samples\_split: 5, min\_samples\_leaf: 1, class\_weight: `balanced'\}} & 
\texttt{\{n\_estimators: 200, max\_depth: None, min\_samples\_split: 2, min\_samples\_leaf: 2, class\_weight: `balanced'\}} & 
For binary tasks, 100 trees ensure stability. For multi-class, 200 trees improve coverage of complex class distributions. Weighted class adjustments handle imbalances. \\
\hline
\textbf{LightGBM} & 
\texttt{\{n\_estimators: 100, learning\_rate: 0.1, max\_depth: -1, num\_leaves: 50, min\_child\_samples: 10, class\_weight: None\}} & 
\texttt{\{n\_estimators: 100, learning\_rate: 0.1, max\_depth: None, num\_leaves: 63, class\_weight: `balanced'\}} & 
For both tasks, LightGBM achieves efficiency via leaf-wise tree growth. For multi-class, additional leaves (63) improve representation of minority classes. \\
\hline
\textbf{ALBERT} & 
\texttt{\{lr: 1.46e-05, epochs: 4, dropout: 0.11\}} & 
\texttt{\{lr: 1.17e-05, epochs: 4, dropout: 0.15\}} & 
ALBERT’s lightweight architecture fine-tunes well with minimal learning rates and dropout for regularization. Minor adjustments improve class representation in multi-class settings. \\
\hline
\textbf{GRU} & 
\texttt{\{embedding\_dim: 156, hidden\_dim: 467, lr: 0.0004, epochs: 5\}} & 
\texttt{\{embedding\_dim: 236, hidden\_dim: 730, lr: 0.0003, epochs: 6\}} & 
Embedding dimensions and hidden states effectively capture sequential dependencies in text. Multi-class configurations benefit from higher hidden dimensions and epochs. \\
\hline
\end{tabular}
\end{sidewaystable}

\begin{table}[htbp]
\centering
\caption{Weighted F1 Scores of Models for Binary and Multi-Class Classification Tasks}
\label{tbl2:f1-scores}
\begin{tabular}{lcc}
\hline
\textbf{Model} & \textbf{Binary Classification} & \textbf{Multi-Class Classification } \\
\hline
Support Vector Machine (SVM) & 0.9401 & 0.7610 \\
Logistic Regression & 0.9345 & 0.7498 \\
Random Forest & 0.9359 & 0.7478 \\
LightGBM & 0.9358 & 0.7747 \\
ALBERT& 0.9576 & 0.7841 \\
Gated Recurrent Units (GRU) & 0.9512 & 0.7756 \\
\hline
\end{tabular}
\end{table}

\begin{table}[ht]
\centering
\caption{Area Under the Receiver Operating Characteristic Curve (AUROC) Scores for Binary and Multi-Class Classification Tasks}
\label{tbl3:auc_scores}
\resizebox{\textwidth}{!}{%
\begin{tabular}{lcc}
\toprule
\textbf{Model} & \textbf{Binary Classification AUROC} & \textbf{Multi-Class Classification Micro-Average AUROC} \\
\midrule
SVM & 0.93 & 0.95 \\
Logistic Regression & 0.93 & 0.96 \\
Random Forest & 0.92 & 0.96 \\
LightGBM & 0.93 & 0.97 \\
ALBERT & 0.95 & 0.97 \\
GRU & 0.94 & 0.97 \\
\bottomrule
\end{tabular}%
}
\end{table}

\newpage
\begin{figure}[h!]
 \centering
 \includegraphics[width=0.8\textwidth]{Figures/multi.png}
 \caption{Multi-class distribution of mental health statuses.}
 \label{fig:multi-class}
\end{figure}

\begin{figure}[h!]
 \centering
 \includegraphics[width=0.8\textwidth]{Figures/binary.png}
 \caption{Binary classification distribution of \textit{Normal} versus \textit{Abnormal} mental health statuses.}
 \label{fig:binary-class}
\end{figure}

\end{document}




%% 
%% Copyright (C) 2019 by Daniel A. Weiss <daniel.weiss.led at gmail.com>
%% 
%% This work may be distributed and/or modified under the
%% conditions of the LaTeX Project Public License (LPPL), either
%% version 1.3c of this license or (at your option) any later
%% version.The latest version of this license is in the file:
%% 
%% http://www.latex-project.org/lppl.txt
%% 
%% Users may freely modify these files without permission, as long as the
%% copyright line and this statement are maintained intact.
%% 
%% This work is not endorsed by, affiliated with, or probably even known
%% by, the American Psychological Association.
%% 
%% This work is "maintained" (as per LPPL maintenance status) by
%% Daniel A. Weiss.
%% 
%% This work consists of the fileapa7.dtx
%% and the derived files apa7.ins,
%% apa7.cls,
%% apa7.pdf,
%% README,
%% APA7american.txt,
%% APA7british.txt,
%% APA7dutch.txt,
%% APA7english.txt,
%% APA7german.txt,
%% APA7ngerman.txt,
%% APA7greek.txt,
%% APA7czech.txt,
%% APA7turkish.txt,
%% APA7endfloat.cfg,
%% Figure1.pdf,
%% shortsample.tex,
%% longsample.tex, and
%% bibliography.bib.
%% 
%%
%% End of file `./samples/longsample.tex'.

% \bibliography{ref}


 % \begin{IEEEbiography}

% \EOD

\end{document}
