% !TEX options = --shell-escape
%\documentclass[10pt,twocolumn]{IEEEtran}
% \documentclass{ieeeaccess}
\documentclass[conference]{IEEEtran}
\usepackage{cite}
\usepackage{amsmath,amssymb,amsfonts}
\usepackage{algorithmic}
\usepackage{graphicx}
\usepackage{textcomp}
% \usepackage{caption}
\usepackage{subfigure}
% \usepackage{footnote}
\usepackage{amssymb}
\usepackage[table]{xcolor}
\usepackage{pifont}
\usepackage{makecell}
\usepackage{epstopdf}
\def\BibTeX{{\rm B\kern-.05em{\sc i\kern-.025em b}\kern-.08em
    T\kern-.1667em\lower.7ex\hbox{E}\kern-.125emX}}
\AtBeginDocument{\definecolor{ojcolor}{cmyk}{0.93,0.59,0.15,0.02}}
\def\OJlogo{\vspace{-12pt}\includegraphics[height=24pt]{ojvt-logo.png}}

\begin{document}
% \receiveddate{XX Month, XXXX}
% \reviseddate{XX Month, XXXX}
% \accepteddate{XX Month, XXXX}
% \publisheddate{XX Month, XXXX}
% \currentdate{27 June, 2024}
% \doiinfo{OJVT.2024.0627000}


% \title{Modeling and Predicting Coverage and MIMO Channel Rank for NextG Aerial Vehicle Networks with Foliage}
\title{Analysis and Prediction of Coverage and Channel Rank for UAV Networks in Rural Scenarios with Foliage}
%: From Ray Tracing Simulations to Measurements}
%\title{Ray Tracing-Based Link Analysis and Kriging Interpolation-Based 3D Channel Rank Interpolation Scheme for UAV Channels in Rural Scenario with Trees}
\author{Donggu Lee$^1$, Ozgur Ozdemir$^1$, Asokan Ram$^2$, and Ismail Guvenc$^1$
\\ $^1$Department of Electrical and Computer Engineering, North Carolina State University, Raleigh, NC, USA 
% (e-mail: \{dlee42, oozdemi, iguvenc\} @ncsu.edu )
\\ $^2$Wireless Research Center of North Carolina, Wake Forest, NC, USA
\\ E-mail: \{dlee42, oozdemi, iguvenc\}@ncsu.edu,
asokan.ram@wrc-nc.org
}
% \affil{Department of Electrical and Computer Engineering, North Carolina State University, Raleigh, NC, USA 

% (e-mail: \{dlee42, oozdemi, iguvenc\} @ncsu.edu )}
% \affil{Wireless Research Center of North Carolina, Wake Forest, NC, USA (e-mail: asokan.ram@wrc-nc.org)}

% \tfootnote{This paragraph of the first footnote will contain support 
% information, including sponsor and financial support acknowledgment. For 
% example, ``This work was supported in part by the U.S. Department of 
% Commerce under Grant BS123456.''}

% \markboth
% {Author \headeretal: Preparation of Papers for IEEE TRANSACTIONS and JOURNALS}
% {Author \headeretal: Preparation of Papers for IEEE TRANSACTIONS and JOURNALS}

% \corresp{Corresponding author: Ismail Guvenc (e-mail: iguvenc@ncsu.edu).}

% \authornote{This work is supported in part by the NSF awards CNS-1939334 and CNS-2332834.}


\maketitle

\begin{abstract} 
% Need to be edited at the end
Unmanned aerial vehicles (UAVs) are expected to play a key role in 6G-enabled vehicular-to-everything (V2X) communications requiring high data rates, low latency, and reliable connectivity for mission-critical applications. Multi-input multi-output (MIMO) technology is essential for meeting these demands. However, UAV link performance is significantly affected by environmental factors such as signal attenuation, multipath propagation, and blockage from obstacles, particularly dense foliage in rural areas. In this paper, we investigate RF coverage and channel rank over UAV channels in foliage-dominated rural environments using ray tracing (RT) simulations. We conduct RT-based channel rank and RF coverage analysis over Lake Wheeler Field Labs at NC State University to examine the impact on UAV links. Custom-modeled trees are integrated into the RT simulations using NVIDIA Sionna, Blender, and Open Street Map (OSM) database to capture realistic blockage effects. Results indicate that tree-induced blockage impacts RF coverage and channel rank at lower UAV altitudes. We also propose a Kriging interpolation-based 3D channel rank interpolation scheme, leveraging the observed spatial correlation of channel rank in the given environments. The accuracy of the proposed scheme is evaluated using the mean absolute error (MAE) metric and compared against baseline interpolation methods. Finally, we compare the RT-based received signal strength (RSS) and channel rank results with real-world measurements from the NSF AERPAW testbed demonstrating reasonable consistency between simulation results and the measurements.
\end{abstract}

\begin{IEEEkeywords}
AERPAW, channel rank, drone, Kriging interpolation, MIMO, ray tracing, RF coverage, Sionna, spatial correlation, UAV, V2X.
\end{IEEEkeywords}

% \titlepgskip=-15pt

\section{Introduction}
\label{sec:introduction}
Unmanned aerial vehicles (UAVs) are expected to play a key role in 6G-enabled networks, especially in the context of vehicular-to-everything (V2X) communication. Using UAVs in 6G networks has many promising applications such as surveillance, infrastructure inspections, search-and-rescue missions, and among others \cite{access_UAV_application, proc_ieee_UAV_application}. High data rate, low latency, and reliable connectivity are critical requirements to support mission-oriented applications in V2X communications, which need to be specifically studied for UAV scenarios.

UAVs provide the advantages of forming flexible coverage, operating in challenging terrains, and handling dynamic network loads. To achieve high data rate and reliability requirements, multi-input multi-output (MIMO) technology is essential in UAV networks. Spatial multiplexing in MIMO systems enhances data rates by utilizing parallel data stream transmissions without allocating additional wireless communication resources. Various performance metrics such as singular value spread, condition number, and correlation matrix distance can be considered for evaluating MIMO systems \cite{MIMO_performance_metric}. Especially, channel rank is a key metric in MIMO system design as it quantifies how many parallel spatial streams can be supported in a given MIMO link \cite{MIMO_white_paper, MIMO_adaptation_1, MIMO_adaptation_2}. However, analyzing channel rank and coverage for UAV links involves unique challenges such as propagation modeling in unfavorable terrains, blockage from obstacles, and prediction of coverage and rank at locations where no past data is available. 

To our knowledge, ray tracing (RT)-based UAV coverage analysis and Kriging interpolation-based 3D channel rank interpolation for UAV channels in foliage-dominated rural scenarios are not available in the existing literature. In this paper, we substantially extend our earlier related work in \cite{previous_work} with a wider target area and multiple base station settings. The main contributions of this paper can be summarized as follows. 

\textbf{Foliage Modeling and Blockage Analysis.} Blockage effects in an environment can severely impact network reliability \cite{ray_tracing_foliage}. In rural areas, dense foliage and buildings can obstruct propagation, leading to signal degradation and deviations in UAV communication scenarios. To accurately model these effects, we introduce custom-modeled trees in the target area using NVIDIA’s RT tool Sionna~\cite{nvidia_sionna}, the 3D modeling tool Blender~\cite{blender}, and the Open Street Map (OSM) database~\cite{osm}. This approach allows for realistic simulation of signal blockage, attenuation, and reflection due to environmental obstacles.

\textbf{RF Coverage and Channel Rank Analysis.} Understanding RF coverage and channel rank is fundamental for UAV link analysis in rural environments. RF coverage defines the effective operational range of UAVs, ensuring consistent connectivity. Meanwhile, channel rank determines the number of parallel spatial streams available for spatial multiplexing directly influencing link capacity. Using extensive RT simulations, we analyze the impact of foliage and terrain on RF coverage and channel rank at varying UAV altitudes. The RT tool enables precise evaluation of propagation including beam patterns, angular resolution, and multipath components, providing a detailed representation of UAV communication in complex environments \cite{RT_tutorial}.

\textbf{Kriging-Based 3D Channel Rank Interpolation.} We introduce a Kriging interpolation-based 3D channel rank interpolation scheme. This method leverages the spatial correlation of channel rank incorporating both horizontal and vertical variations. The interpolation process consists of: 1) utilizing channel ranks across different UAV altitudes to compute spatial correlation and semi-variograms for Kriging interpolation; and 2) integrating channel rank data within a defined horizontal sampling radius from target UAV locations. We evaluate the accuracy of the proposed Kriging-based interpolation using the mean absolute error (MAE) metric and compare it against two baseline interpolation methods.

\textbf{Validation with Real-World Measurements.} To ensure the practical applicability of our RT-based simulations, we compare the RT simulation results of RF coverage and channel rank with real-world measurements collected in the NSF AERPAW testbed~\cite{aerpawWebsite}. Our goal is to assess how well RT simulations capture real-world conditions in the given environments. The results show that RT-based modeling provides reasonable approximations of UAV communication performance in foliage-dense rural environments.


The rest of this paper is organized as follows. Relevant works from the existing literature are discussed in Section~\ref{ch:related_works}. The description of the system model is given in Section~\ref{ch:system_model}. The RT and measurement scenarios are introduced in Section~\ref{ch:RT_setup} and Section~\ref{ch:measurement_setup}, respectively. The proposed Kriging-based 3D channel rank interpolation scheme and baseline interpolation schemes are described in Section~\ref{ch:Kriging_3D_Channel_interpolation}. 
Simulation results for RF coverage, channel rank, Kriging-based 3D channel rank interpolation, and comparison of measurements and RT simulations are provided in Section~\ref{ch:numerical_results}. Finally, the last section concludes the paper.


\begin{table*}[t!]
    \centering
    \caption{Literature review on RT-based RF coverage and channel rank analysis, and Kriging interpolation scheme.}
    \begin{tabular}{|c|c|c|c|c|c|c|c|}
    \hline
        \textbf{Ref.} & \textbf{Analysis Objectives} & \makecell{\textbf{UAV} \\ \textbf{Channels}}  & \makecell{\textbf{Ray} \\ \textbf{Tracing}}  & \makecell{\textbf{Channel} \\ \textbf{Rank} \\ \textbf{Analysis}} & \makecell{\textbf{Kriging} \\ \textbf{Interpolation}} & \textbf{Trees} & \textbf{Measurements} \\ \hline
        
        \cite{duke_sionna_paper} & RT-based RF signal mapping and ML integration & \cellcolor{red!25} \ding{55} & \cellcolor{green!25} \ding{51} & \cellcolor{red!25} \ding{55} & \cellcolor{red!25} \ding{55} & \cellcolor{red!25} \ding{55} & \cellcolor{green!25} \ding{51} \\ \hline  
        
        \cite{boston_sionna} & RT-based digital twin for urban scenarios & \cellcolor{red!25} \ding{55} & \cellcolor{green!25} \ding{51} & \cellcolor{red!25} \ding{55} & \cellcolor{red!25} \ding{55} & \cellcolor{red!25} \ding{55} & \cellcolor{red!25} \ding{55} \\ \hline  

        \cite{RT_mode_urban} & RT-based channel analysis in urban scenarios & \cellcolor{red!25} \ding{55} & \cellcolor{green!25} \ding{51} & \cellcolor{red!25} \ding{55} & \cellcolor{red!25} \ding{55} & \cellcolor{red!25} \ding{55} & \cellcolor{red!25} \ding{55} \\ \hline

        \cite{uav_channel_model_urban} & UAV channel modeling with measurement
        & \cellcolor{green!25} \ding{51} & \cellcolor{red!25} \ding{55} & \cellcolor{red!25} \ding{55} & \cellcolor{red!25} \ding{55} & \cellcolor{red!25} \ding{55} &  \cellcolor{green!25} \ding{51} \\ \hline

        \cite{uav_channel_model_arxiv} & Stochastic channel model for urban UAV scenarios & \cellcolor{green!25} \ding{51} & \cellcolor{red!25} \ding{55} & \cellcolor{red!25} \ding{55} & \cellcolor{red!25} \ding{55} & \cellcolor{red!25} \ding{55} & \cellcolor{red!25} \ding{55} \\ \hline
        
        \cite{wahab_paper} & UAV channel modeling with foliage & \cellcolor{green!25} \ding{51} & \cellcolor{red!25} \ding{55} & \cellcolor{red!25} \ding{55} & \cellcolor{red!25} \ding{55} & \cellcolor{green!25} \ding{51} & \cellcolor{green!25} \ding{51} \\ \hline      
        \cite{Channel_rank_massive_MIMO} & Channel rank analysis for massive MIMO & \cellcolor{red!25} \ding{55} & \cellcolor{green!25} \ding{51} & \cellcolor{green!25} \ding{51} & \cellcolor{red!25} \ding{55} & \cellcolor{red!25} \ding{55} & \cellcolor{green!25} \ding{51}  \\ \hline     
        \cite{RT_channel_modeling, RT_terahz_UAV} & RT-based UAV channel modeling & \cellcolor{green!25} \ding{51} & \cellcolor{green!25} \ding{51} & \cellcolor{red!25} \ding{55} & \cellcolor{red!25} \ding{55} & \cellcolor{red!25} \ding{55} & \cellcolor{green!25} \ding{51}  \\ \hline

        \cite{rank_irs} & Reflector-based channel rank improvement & \cellcolor{green!25} \ding{51} & \cellcolor{red!25} \ding{55} & \cellcolor{green!25} \ding{51} & \cellcolor{red!25} \ding{55} & \cellcolor{red!25} \ding{55} & \cellcolor{red!25} \ding{55} \\ \hline
        
        \cite{rank_MIMO_bounds} & Theoretical channel rank analysis & \cellcolor{red!25} \ding{55} & \cellcolor{red!25} \ding{55} & \cellcolor{green!25} \ding{51} & \cellcolor{red!25} \ding{55} & \cellcolor{red!25} \ding{55} & \cellcolor{red!25} \ding{55} \\ \hline
        \cite{vertical_MIMO_paper, vertical_MIMO_paper_2} & Channel rank analysis for antenna separation &  \cellcolor{red!25} \ding{55} &  \cellcolor{red!25} \ding{55} & \cellcolor{green!25} \ding{51} & \cellcolor{red!25} \ding{55} & \cellcolor{red!25} \ding{55} & \cellcolor{green!25} \ding{51} \\ \hline
        
        \cite{Kriging_coverage, fixed_rank_kriging} & Kriging-based coverage estimation & \cellcolor{red!25} \ding{55} & \cellcolor{red!25} \ding{55} & \cellcolor{red!25} \ding{55} & \cellcolor{green!25} \ding{51} & \cellcolor{red!25} \ding{55} & \cellcolor{green!25} \ding{51} \\ \hline
        \cite{Kriging_map_construction} & Kriging interpolation for radio map construction & \cellcolor{red!25} \ding{55} & \cellcolor{red!25} \ding{55} & \cellcolor{red!25} \ding{55} & \cellcolor{green!25} \ding{51} & \cellcolor{red!25} \ding{55} & \cellcolor{green!25} \ding{51} \\ \hline
        
        \cite{Kriging_mapping} & Coverage analysis using Kriging interpolation & \cellcolor{red!25} \ding{55} & \cellcolor{red!25} \ding{55} & \cellcolor{red!25} \ding{55} & \cellcolor{green!25} \ding{51} & \cellcolor{red!25} \ding{55} & \cellcolor{green!25} \ding{51} \\ \hline  
        
        \cite{previous_work} & RT-based UAV link analysis & \cellcolor{green!25} \ding{51} & \cellcolor{green!25} \ding{51} & \cellcolor{green!25} \ding{51} & \cellcolor{red!25} \ding{55} & \cellcolor{red!25} \ding{55} & \cellcolor{red!25} \ding{55} \\ \hline
        \cite{ray_tracing_foliage} & RT-based channel measurement on trees  & \cellcolor{red!25} \ding{55} & \cellcolor{green!25} \ding{51} & \cellcolor{red!25} \ding{55} & \cellcolor{red!25} \ding{55} & \cellcolor{green!25} \ding{51} & \cellcolor{green!25} \ding{51} \\ \hline 
        \cite{ray_tracing_tree_urban} & Impact of foilage in mmWave urban channels & \cellcolor{red!25} \ding{55} & \cellcolor{green!25} \ding{51} & \cellcolor{red!25} \ding{55} & \cellcolor{red!25} \ding{55} & \cellcolor{green!25} \ding{51} & \cellcolor{red!25} \ding{55} \\ \hline
        \cite{ray_tracing_vegetation} & RT-based propagation modeling for vegetation & \cellcolor{red!25} \ding{55} & \cellcolor{green!25} \ding{51} & \cellcolor{red!25} \ding{55} & \cellcolor{red!25} \ding{55} & \cellcolor{green!25} \ding{51} & \cellcolor{green!25} \ding{51} \\ \hline 
       
        \cite{openGERT_paper} & RT-based propagation modeling for digital twins & \cellcolor{red!25} \ding{55} & \cellcolor{green!25} \ding{51} &
        \cellcolor{red!25} \ding{55} & \cellcolor{red!25} \ding{55} & \cellcolor{red!25} \ding{55} & \cellcolor{red!25} \ding{55} \\ \hline
       
        \cite{maeng2023kriging} & 3D radio map generation using Kriging interpolation & \cellcolor{green!25} \ding{51} & \cellcolor{red!25} \ding{55} & \cellcolor{red!25} \ding{55} & \cellcolor{green!25} \ding{51} & \cellcolor{red!25} \ding{55} & \cellcolor{green!25} \ding{51} \\ \hline
        
        This work & RT-based link analysis and Kriging interpolation & \cellcolor{green!25} \ding{51} & \cellcolor{green!25} \ding{51} & \cellcolor{green!25} \ding{51} & \cellcolor{green!25} \ding{51} & \cellcolor{green!25} \ding{51} & \cellcolor{green!25} \ding{51} \\ \hline     
    \end{tabular}
    
    \label{tab:literature_review}
\end{table*}

\section{Related Works}\label{ch:related_works}
There are limited studies investigating channel rank and coverage for UAV links in the literature with RT studies or measurements. Our literature review with representative publications related to our work and the differences with the present work is summarized in Table \ref{tab:literature_review}.

\subsection{UAV Channels}
In \cite{RT_channel_modeling} and \cite{RT_terahz_UAV}, RT-based UAV channel modeling has been investigated. The altitude-dependent channel parameters, such as path loss, power delay profile, and angular information, are simulated to characterize the UAV channel. The developed channel model consisted of a deterministic line-of-sight (LoS) path and ground reflection, and stochastic non-LoS (NLoS) components. The developed model has been validated using RT simulations and measurements.  

A stochastic geometry model is developed in \cite{uav_channel_model_arxiv} for investigating coverage and outage probability in an urban environment. The study investigates UAV backhaul links considering building density, antenna beamwidth, and interference effects. The results highlight the existence of an optimal UAV altitude that maximizes coverage while ensuring reliable backhaul connectivity. Similarly, in \cite{uav_channel_model_urban}, measurement-based UAV channel modeling is conducted in an urban scenario. The study extracts key parameters for path loss models and develops a simplified Saleh-Valenzuela channel model for UAV-ground links.

In our previous work~\cite{previous_work}, UAV channel characteristics are analyzed in rural and urban scenarios. Specifically, RF coverage, channel rank, and condition number distribution have been analyzed for UAV networks using Matlab-based RT simulations. Centennial Campus and Lake Wheeler Field Labs at NC State University are considered as urban and rural scenarios, respectively. Due to the blockage from the buildings, an outage can be observed at the lower altitude of the UAV. Moreover, the probability of having a channel rank of 3 or 4 is getting lower because of the LoS dominant path to the UAV as the altitude of the UAV increases.


\subsection{Ray Tracing}
% In \cite{RT_indoor}, 
In \cite{duke_sionna_paper}, an RF signal mapping scheme for cellular networks has been investigated by integrating RT simulation using Sionna with geographic databases and machine learning (ML) techniques. A cascaded neural network refines signal strength predictions by leveraging geographical information from RT and sparse real-world measurements. The proposed method improves computational efficiency while maintaining high accuracy, outperforming conventional RT-based approaches in real-world evaluations.

In \cite{boston_sionna} and \cite{RT_mode_urban}, RT-based RF coverage and channel capacity analysis have been studied for urban scenarios. Nodetably in \cite{boston_sionna}, a large-scale digital twin framework for downtown Boston has been developed. The proposed BostonTwin framework integrates a high-fidelity 3D model of the downtown Boston area with geographical data, enabling RT simulations. By leveraging Sionna, the framework facilitates large-scale coverage mapping and signal-to-noise ratio (SNR) evaluations. Moreover, the authors highlight the areas where BostonTwin meets the requirements for 6G use cases. On the other hand, ~\cite{RT_mode_urban} uses site maps and a RT software for computing MIMO gain matrices for a given mobile station position in urban areas of Boston and Manhattan. Subsequently, authors use the MIMO gain matrices to determine achievable rates for MIMO transmission modes including spatial multiplexing, beamforming, and diversity. The results suggest that the use of site-specific RT for data rate prediction works more effectively than the use of stochastic models.

An automated geometry extraction framework for precise RT simulation is developed in \cite{openGERT_paper}. With the capability of open-source resources from OSM, Microsoft Global ML Building Footprints, and the US Geological Survey (USGS), high-fidelity RT simulation environments can be extracted. The authors conducted sensitivity analyses to investigate the impact on the accuracy of RT simulation with the environmental factors including building heights, locations, and material settings. Channel statistics of path gain, delay spread, and link outage are shown with the developed framework. 


\subsection{Channel Rank Analysis}
In \cite{Channel_rank_massive_MIMO}, the channel rank for an outdoor-to-indoor massive MIMO system has been analyzed. The angular characteristics in terms of the angle of arrival and departure have been investigated for the given wireless communication environment. Then, singular values of each channel rank have been simulated and measured. It is observed that the singular values decrease rapidly as the range of eigenvalue to be captured increases. 

Theoretical lower and upper bounds of channel capacity for high-rank MIMO systems are studied in \cite{rank_MIMO_bounds}. The study investigates how LoS propagation affects channel rank and capacity. It shows that strong LoS components increase channel correlation, reducing the efficiency of spatial multiplexing. However, it also demonstrates that optimizing antenna placement and spacing helps maintain orthogonality, preserve channel rank, and improve system performance.

In \cite{rank_irs}, the use of passive intelligent reflecting surfaces (IRSs) has been investigated to improve channel rank and spatial multiplexing in urban UAV scenarios. This study proposes an IRS placement optimization framework to maximize the average channel capacity along predefined UAV trajectories. Numerical results at various carrier frequencies demonstrate that IRS-assisted channels significantly improve MIMO capacity compared to LoS-dominant UAV channels. The authors introduce the potential of IRSs as a cost-effective solution for enhancing UAV communications in dense urban environments. Other work on studying channel rank includes \cite{vertical_MIMO_paper} and \cite{vertical_MIMO_paper_2}, which use measurements for an outdoor LTE MIMO network to explore how the channel rank varies for various scenarios. 

% \cite{rank_analysis_radar}. 


\subsection{Kriging Interpolation}
Coverage analysis and map construction schemes using spatial interpolation for mobile systems have been investigated in \cite{Kriging_coverage, fixed_rank_kriging, Kriging_mapping, Kriging_map_construction}. Especially in \cite{Kriging_coverage}, a radio map has been developed by interpolating geo-located measurements. A fixed-rank Kriging interpolation scheme has been used to generate the radio map to reduce computational complexity. The simulation results show the trade-off between computational complexity and accuracy in coverage estimation over rural scenarios. 
In \cite{maeng2023kriging}, a Kriging interpolation-based 3D radio map generation scheme for radio dynamic zones is proposed. A realistic propagation model is developed by using measurements over the air-to-ground link in terms of path loss, shadowing, and spatial correlation. Horizontal and vertical correlations of received signal strength (RSS) are used for accurate radio map generation. The generated radio map is more accurate than the model with perfect knowledge of the path loss due to the benefits of spatial correlation. 

\subsection{Trees}
In \cite{ray_tracing_foliage}, an RT-based case study of channel measurements to model propagation effects due to trees is provided. Statistics of angular properties and penetration loss over multiple types of trees have been analyzed to characterize blockage and propagation for mmWave channels. The authors highlighted that the tree-specific propagation model can be used for realistic RT simulation in non-controlled environments. 

An RT-based 3D mmWave propagation model through vegetation is investigated in \cite{ray_tracing_vegetation}. To develop the propagation model, measurements of radiation, angular, and 3D scattering profiles are used in the scenario of groups of trees and various dimensions of the trees. The proposed 3D propagation model is assessed by comparing 3D directional measurements within different positions and angles.  

In \cite{ray_tracing_tree_urban}, the impact of foliage in an urban environment over the mmWave channel is investigated. Here, a simplified urban scenario consists of four buildings in a grid-wise position and four trees at the end of each road between the buildings. A hybrid channel model with the knowledge of a conventional correlation-based channel model is integrated with the RT results to characterize the foliage impact.

The impact of foliage on UAV air-to-ground propagation channels has been investigated using channel-sounding measurements in \cite{wahab_paper} for ultra-wideband signals. The authors examine how tree obstructions affect signal propagation analyzing path loss, multipath fading, and coherence bandwidth. Results show that foliage significantly increases path loss and reduces coherence bandwidth, leading to stronger multipath fading and degraded reliability.

\subsection{Measurements}
% \cite{RT_channel_measure}.
In \cite{wahab_paper}, channel-sounding measurement-based UAV air-to-ground propagation channel modeling has been investigated. The study conducts comprehensive measurements for various UAV communication scenarios and derives statistical models for path loss, multipath propagation, and small-scale fading. The measurements capture high-resolution time and frequency domain characteristics. The collected data is analyzed to extract power delay profiles and coherence bandwidth, which provides insights into UAV channel behavior under different propagation conditions.


Studies in \cite{vertical_MIMO_paper} and \cite{vertical_MIMO_paper_2}, considering terrestrial cellular network scenarios, analyze the impact of antenna configurations on channel rank and system performance using LTE MIMO measurements. In \cite{vertical_MIMO_paper}, increased antenna spacing and optimized polarization, achieved by comparing vertically and horizontally spaced configurations, are shown to improve channel rank and throughput. Similarly, \cite{vertical_MIMO_paper_2} demonstrates that reducing mutual correlation enhances throughput by enabling spatial multiplexing with a higher channel rank. Both studies highlight the critical role of precise measurement campaigns in evaluating channel rank under various conditions.

\subsection{Contributions of This Work}

 Based on Table~\ref{tab:literature_review} and to our knowledge, there are no works that study the effects of foliage on the channel conditions of UAV networks using RT simulations or measurements. Other than \cite{RT_channel_modeling,RT_terahz_UAV}, there are also no works that compare RT simulations with measurements for analyzing coverage for UAV networks.  
In this paper, we extend the scope of our previous work in \cite{previous_work} to address the gaps in the literature. In particular, the target area of the rural scenario has been extended with multiple base stations to understand long-distance propagation effects. Trees in the target area have been included to investigate realistic propagation in the rural area. Moreover, a constant channel rank threshold has been used to investigate channel rank using the ratio of the strongest singular value. Lastly, the spatial correlation, which can be observed in the channel rank distribution over the rural area scenario, has been adopted for the Kriging interpolation-based 3D channel rank interpolation scheme.

     \begin{figure*}[t]
     \centering
     \includegraphics[width=1.6\columnwidth]{images/introduction_figure_v4.png}
     \caption{UAV connectivity scenario in a rural area. UAV coverage can be blocked due to buildings and foliage. It is of interest to predict channel characteristics, such as channel rank and coverage, at a location where no measurements have been collected before, based on measurements at other locations.}     \label{fig:introduction_figure}
 \end{figure*}

 \section{System Model}\label{ch:system_model}
In this section, we provide the system model for the UAV MIMO network and discuss our assumptions for calculating channel rank. The $N_{\rm r}\times 1$ received signal vector $\boldsymbol{y}$ in a MIMO link as in Figure~\ref{fig:introduction_figure} can be expressed as 
\begin{equation}
    \boldsymbol{y} = \boldsymbol{Hx + n},    \label{Eq:MIMO}
\end{equation}
where $\boldsymbol{H}$ is the $N_{\rm r} \times N_{\rm t}$ channel matrix of the MIMO link, $\boldsymbol{x}$ denotes the $N_{\rm t} \times 1$ transmit signal vector, and $\boldsymbol{n}$ is the noise vector having the same dimensions as $\boldsymbol{y}$. 

The rank of the channel matrix in~\eqref{Eq:MIMO} affects the number of parallel data streams that can be transmitted with spatial multiplexing over a given link~\cite{lte_book, MIMO_white_paper}. The channel rank can be obtained from the number of non-zero singular values after singular value decomposition of the channel matrix $\boldsymbol{H}$. This can be expressed as 
\begin{equation}
    \boldsymbol{H} = \boldsymbol{U \Sigma V^{*}},    
\end{equation}
where $\boldsymbol{U}$ represents the $\textit{m} \times \textit{m}$ complex unitary matrix, $\boldsymbol{\Sigma}$ is the $\textit{m} \times \textit{n}$ rectangular diagonal matrix, with diagonal elements $\sigma_i$, with $1<i<\min\{m, n\}$, and $\boldsymbol{V}$ is the $\textit{n} \times \textit{n}$ complex unitary matrix, respectively. The diagonal elements, $\sigma_i$, are sorted in descending order as $i$ increases. The channel rank, $R$, is bounded into the range of $1\leq R\leq \min\{m, n\}$. 

To develop fundamental insights on the channel rank behavior of UAV links for different RT scenarios, in this paper, we consider that the channel rank is determined by the number of eigenvalues of the channel matrix that are higher than a predetermined threshold. In particular, we consider that the channel rank for a given link at a UAV location $\boldsymbol{p}=(x, y, h)$, where $h \in (h_{1}, h_{2}, ..., h_{N_{h}})$ for $N_{h}$ different altitude settings, is given by  
\begin{align}    
\label{eq:channel_rank}
   R_{\delta_{K_j}}(\boldsymbol{p})&= \sum_{i=1}^{\min(N_r,N_t)}{\mathrm I}\{\sigma_{i}(\boldsymbol{p})> \delta_{K_j} \} \nonumber \\ &=\sum_{i=1}^{\min(N_r,N_t)}{\mathrm I}\bigg\{\sigma_{i}(\boldsymbol{p})> \frac{\sigma_{1}(\boldsymbol{p})}{K_{j}} \bigg\},    
\end{align}
where $I\{ \cdot \}$ is an indicator function that returns $1$ if its input is satisfied and $0$ otherwise, and $\delta_{K_j}$ is a threshold for the singular value with the $j$-th threshold ratio constant $K$. In this paper, we set the threshold for the singular value at a UAV location as $\delta_{K_j} = \sigma_{1}(\boldsymbol{p}) / K_{j}$ to focus on the ratio compared to the first strongest singular value and analyze the performance for different values of $K$. Here, the threshold is extended with a different threshold ratio constant $K$, i.e., $\delta_{K_{j}} = \sigma_{1}(\boldsymbol{p}) / K_{j}$ for $j=1,2, ..., N_{K}$, where $N_K$ is the total number of threshold ratio constant $K$. From this, the channel rank is determined as the number of non-zero singular values that exceed the threshold.  
 
\section{Ray Tracing Simulation Setup}\label{ch:RT_setup}
We consider NSF AERPAW's fixed node locations at NC State University for the target area of RT simulations~\cite{aerpawWebsite}. NVIDIA Sionna is used as the RT simulation tool \cite{nvidia_sionna}. It is known that Sionna employs the Fibonacci lattice unit sphere on the transmitter's side to calculate the candidates for the possible trajectory of rays. A configurable parameter of the number of samples for the sphere is adopted to align the balance between computational complexity and accuracy in calculating the trajectories. The numbers of samples for the sphere for RF coverage and channel rank simulations are set to $10^6$ and $10^3$, respectively, to reduce the computational load in channel rank simulation.  
Figure \ref{fig:map_LW} shows a satellite view of the target area and the locations of the towers with red markers in the Lake Wheeler Field Labs. Geographical information including buildings is obtained from the OSM database \cite{osm, osm_buildings}. Moreover, Figure \ref{fig:map_blender} shows the Blender scene view of the target area, where green dots are the trees and orange rectangles indicate the buildings in the area.

The shooting and bounce ray (SBR) method-based RT model \cite{ray_tracing_access_paper, sbr_paper} has been applied in this work, where the ray tracer traces all rays at the receiver side after calculating reflection or diffraction in the trajectory. The procedure for the SBR can be categorized as 1) ray launching, 2) ray tracing, and 3) ray reception. After the launching of rays from the source, the trajectory of each ray is calculated in the 3D space \cite{sbr_paper}, which can be expressed as 
 \begin{equation}
    (x_1, y_1, z_1) = (x_0, y_0, z_0) + (s_x, s_y, s_z)t,     
 \end{equation}
where $(x_0, y_0, z_0)$ is the reference point, $(s_x, s_y, s_z)$ is the direction vector, and $t$ is the time duration of the trajectory.

 \begin{figure}[t!]
    \centering    
    \subfigure[Satellite view]{
    \includegraphics[width=0.45\columnwidth]{images/LW_map_v2.png}
    \label{fig:map_LW}
    }
    \subfigure[Blender scene]{
    \includegraphics[width=0.44\columnwidth]{images/blender_map.png}
    \label{fig:map_blender}
    }
    \subfigure[Tree model in Blender]{    \includegraphics[width=0.35\columnwidth]{images/blender_tree.png}
    \label{fig:tree}
    }
   \caption{Map view of the location of towers used in RT simulations: (a) Satellite view, (b) Blender scene view, and (c) tree model.}
    \label{fig:map}
\end{figure} 


The reflection losses of the ray with a surface can be expressed as follows~\cite{ITU-R-P.2040-3}: 
\begin{equation}
    \epsilon_{r} = \epsilon_{r}' + j\epsilon_{r}'',    
\end{equation}
 where $\epsilon_{r}' = af^{b}$ and $\epsilon_{r}'' = cf^{d}/2\pi\epsilon_{0}f$. Here, $a, b, c,$ and $d$ are the constant determined by the surface material, $\epsilon_{0}$ is the permittivity of the free space, and $f$ is the frequency in Hz, respectively. Those constants are decided under ITU recommendation  \cite{itu_R_vegetation, itu_R_buildings} for building, ground, and vegetation material settings in this work. Moreover, the electrical field after reflection and diffraction can be expressed as \cite{sbr_paper}
\begin{equation}
    \boldsymbol{E}(x_{i+1}, y_{i+1}, z_{i+1}) = D_i \Gamma_i \boldsymbol{E}(x_i, y_i, z_i) e^{j\theta},  
\end{equation}
where $D_i$ is the divergence factor related to the spreading of the ray right after the \textit{i}-th reflection, $\Gamma_i$ is the reflection coefficient, $\boldsymbol{E}$$(x_i, y_i, z_i)$ is the incident electric field, and $\theta$ is the phase shift of the electrical field. After the calculation, the receiving rays are derived by the overlapping area of the receiving field and trajectories of the rays.


 \begin{table}[t!]
    \centering
    \caption{Simulation parameters for ray tracing.}\vspace{-1mm}
    \begin{tabular}{|c|c|c|}
    \hline
      \textbf{Parameters}   & \textbf{Description} & \textbf{Value} \\
      \hline
        $f_{\mathrm{c}}$ & Carrier frequency & 3.4 GHz \\
        \hline
        $a_{\mathrm{BS}}$ & Height of base stations & 10 m \\        
        \hline
        $T_{\mathrm{LW}}$ & Target area of Lake Wheeler & 1080 m $\times$ 2130 m \\
        \hline
        $N_{\mathrm{ref}}$ & Maximum number of reflections & 2\\
        \hline
        $N_{\mathrm {ele}}$ & Number of antenna elements & 4 (TX and RX) \\
        \hline
        $\Delta_{\mathrm{tx}}$ & Element spacing for TX antennas & $0.5\lambda$ \\
        \hline
        $\Delta_{\mathrm{rx}}$ & Element spacing for RX antennas & $0.5\lambda$ 
        \\
        \hline
        $\mathrm{P_{TX}}$ & Transmit power of TX antennas & 10 W
        \\
        \hline
        $d_{\mathrm{rx}}$ & Horizontal interval of UAV & $30$~m
        \\
        \hline
    \end{tabular}    
    \label{tab:sim_param}
\end{table}

 The following assumptions are made for the RT simulation scenario of Figure \ref{fig:map_blender}: 1) the carrier frequency is set to $3.4$ GHz; 2) a 4-element linear array antenna is used for the transmitter and receiver in the MIMO cases with horizontal element spacing of $0.5\lambda$ for both transmitter and receiver, where $\lambda$ is the wavelength; 3) the height of the base station is set to $10~$m; 4) the maximum number of reflection is set to be 2; 5) surface materials for the buildings are set to concrete and the ground material for the area is medium dry ground, which is predefined as ``itu\_concrete" and ``itu\_medium\_dry\_ground" in Sionna, respectively; 6) the receivers are located on a uniform mesh grid with $30$~m spacing in both x-axis and y-axis in the horizontal directions covering the given area; 7) noise is neglected in RT simulations; and 8) channel rank is determined between the nearest base station and receiver in Lake Wheeler area. The simulation parameters are summarized in Table~\ref{tab:sim_param}. Lastly, the tree consists of a cylinder with ``itu\_wood" material and a cone on top of the cylinder with custom vegetation material setting calculated in \cite{itu_R_vegetation}. The tree model for the simulation is demonstrated in Figure~\ref{fig:tree}, which has a height of $20$~m and a maximum diameter of $10$~m.

 The information about rays such as the angle of arrival and departure, locations of the transmitter and receiver, and path types indicating LoS or reflective NLoS are extracted using Sionna for every UAV location. Then, an RT channel object is created with Matlab's RT tool \cite{matlab_ray_tracing} using ray information. The channel matrix is derived to determine the channel rank by using random input bits and output bits through the ray-tracing channel object. Finally, channel rank is decided as in (\ref{eq:channel_rank}) by singular value decomposition and the indicator function with a threshold compared to the strongest singular value aforementioned in the previous section.

\section{Real-World Measurement Scenario}\label{ch:measurement_setup}

\begin{figure}
     \centering
     \subfigure[Trajectory]{\includegraphics[trim={0.4cm 0.1cm 1.1cm 0.6cm},clip,width=0.98\columnwidth]{images/measurement_trajectory_v3}
     \label{fig:trajectory_measurement_RSS}
     }
     \subfigure[Altitude]{\includegraphics[trim={0.4cm 0.1cm 1.3cm 0.6cm},clip,width=0.98\columnwidth]{images/measurement_altitude}
     \label{fig:altitude_measurement_RSS}
     }     
     \caption{Trajectory and altitude of the signal coverage measurements and RT simulation.}
     \label{fig:trajectory_altitude_measurement_RSS}
 \end{figure}

 \begin{figure}
     \centering
     \subfigure[Trajectory]{\includegraphics[trim={0.4cm 0.1cm 1.1cm 0.6cm},clip,width=0.98\columnwidth]{images/measurement_trajectory_rank_v7}
     \label{fig:trajectory_measurement_rank}
     }
     \subfigure[Altitude]{\includegraphics[trim={0.4cm 0.1cm 1.3cm 0.6cm},clip,width=0.98\columnwidth]{images/measurement_altitudes_rank_v2}
     \label{fig:altitude_measurement_rank}
     }     
     \caption{Trajectory and altitude of the channel rank measurements and RT simulation.}
     \label{fig:trajectory_altitude_measurement_rank}
 \end{figure}

In addition to RT simulations, we also use real-world coverage and channel rank data from multiple measurement campaigns conducted at the NSF AERPAW platform in the same environments in Figure \ref{fig:map_LW}. While channel rank measurements are available only for one of the base stations, coverage results are available for all five of the base stations in the experiment area. %, though for a separate trajectory. 

\textbf{Coverage Measurements.} The predefined trajectory and altitude of the UAV over time for coverage measurements are shown in Figure~ \ref{fig:trajectory_altitude_measurement_RSS}. The UAV takes off near the LW1 fixed node. Then, the UAV sweeps LW2 and LW3 areas, and returns to the LW1. Each waypoint of the trajectory is marked with a letter and with vertical lines in the following figures. 

The following assumptions and setups are adopted for this scenario: 1) the antenna is a single-element antenna (SISO) with a carrier frequency of $3.3$~GHz; 2) the measurements are recorded by dual channel USRP B205 and GNU Radio; 3) the UAV collected data for $20 $~ms intervals every $100$~ms; 4) the offsets calculated by minimum root mean squared error (RMSE) between measurements and RT simulation results are adopted to each measurement for calibration purpose; 5) the proper offsets that have minimum RMSE is searched by a unit of $0.1$~dB in $[-50, 50]$ range in dB scale; 6) the altitude within the trajectory below $0.5$~m is rounded to $0.5$~m for RT simulation to obtain margin from the ground surface; 7) the UAV is the transmitter and the fixed node LW1 to LW5 act as the receivers; and 8) other parameters are the same for the RT simulation parameters. 

\textbf{Channel Rank Measurements.} The predefined trajectory and altitude over time of the UAV for channel rank measurements are shown in Figure \ref{fig:trajectory_altitude_measurement_rank}, where the building in the target area is marked in the cyan polygon. The UAV takes off near the LW1 tower and sweeps the target area with a zigzag pattern.

The following assumptions and configurations are used for this scenario: 1) $4 \times 4$ MIMO channel is used with a carrier frequency of $3.4$~GHz; 2) two AW3232 dual port sector antennas are used for fixed node LW1 as a transmitter; 3) two AW3232 antenna modules are departed by $1.68$~m; 4) Quectel RM502Q-AE 5G mobile modem with antenna module is mounted at the bottom side of the UAV as a receiver; 5) the direction of antennas of LW1 and UAV are placed to face each other, which leads the yaw angle of $45^\circ$ for LW1 and $315^\circ$ for UAV, respectively; 6) 5G new radio (NR) rank indicator (RI) is recorded over the trajectory; and 7) other parameters are the same for the RT simulation parameters.

 
 \section{Kriging-Based 3D Channel Rank Interpolation}\label{ch:Kriging_3D_Channel_interpolation}
 MIMO channel rank at a given location $\boldsymbol{p}$ can be interpolated based on the knowledge of channel rank at other locations in the 3D space, where the rank information has been logged before. In this work, we introduce the Kriging interpolation method for interpolating the channel rank. 

 \subsection{Spatial Correlation Model}\label{ch:spatial_correlation}
 In this section, we describe the spatial correlation of channel ranks at different UAV locations. It is well-known that the channel rank depends on the angular separation of alignments of the transmitter and receiver antennas within a given environment \cite{tse2005fundamentals}. 

 \begin{figure*}[t!]
    \centering
    \includegraphics[width=1.95\columnwidth]{images/spatial_correlation_description_figure_v2.png}
    \caption{Spatial correlation framework for Kriging interpolation.}
    \label{fig:spatial_correlation_description}
\end{figure*}
 
 To take into account both horizontal and vertical correlation of the channel rank, we first define the channel rank vector at a location of UAV as 
\begin{align}\label{eq:channel_rank_vector}
    \boldsymbol{R}(\boldsymbol{p_i}) = \nonumber \\ \big\{ R_{\delta_{K_j}}(x_i &, y_i, h) \mid j = 1, \dots, N_K, \; h = h_1, \dots, h_{N_h} \big\}.
\end{align}
Here, we define an index for each UAV position using a subscript $i=1, ..., N_{\mathrm{loc}}$ at $\boldsymbol{p}$, where $N_{\mathrm{loc}}$ represents the total number of possible UAV locations within the 2D target area. These positions are distributed over a uniformly discretized 2D target area with the horizontal interval of $d_{\mathrm{rx}}$. The index follows a structured ordering: the index starts with the southwest corner in the target area. Then, the index increases along the x-axis from west to east. When the index reaches the end of the x-axis row, it moves one step along the y-axis before the index increases along the x-axis from west to east again. With this, the $N_{\mathrm{loc}}$-th UAV position is at the northeast corner on the 2D target area. This structure enables each UAV position to be assigned a unique index over the structured grid.

Moreover, the reason for using channel rank vector with all thresholds in (\ref{eq:channel_rank_vector}) rather than using channel rank with a single threshold is based on the fact that the correlation value that will be described is non-valid when the channel rank has the same value at all altitudes. In other word, we may have $R_{\delta_{K_j}}(x, y, h_1) = ... = R_{\delta_{K_j}}(x, y, h_{N_h})$. To avoid this, we set the channel rank vector to have channel ranks with all altitudes and thresholds. The defined channel rank vector $\boldsymbol{R}(\boldsymbol{p}_i)$ represents a collection of channel ranks computed at a position of $\boldsymbol{p}_i$ hierarchical ordering by spanning multiple altitudes first and expanded to the multiple thresholds, which can be expressed as
\begin{align}\label{eq:channel_rank_vector_structure}
    \boldsymbol{R}(\boldsymbol{p_i}) =& \big[ R_{\delta_{K_1}}(x_i, y_i, h_1), \dots, R_{\delta_{K_1}}(x_i, y_i, h_{N_h}),\nonumber \\
    &~R_{\delta_{K_2}}(x_i, y_i, h_1), \dots, R_{\delta_{K_2}}(x_i, y_i, h_{N_h}), \dots, \nonumber\\
    &~R_{\delta_{K_{N_K}}}(x_i, y_i, h_{1}), \dots R_{\delta_{K_{N_K}}}(x_i, y_i, h_{N_h}) \big].
\end{align}
In Figure \ref{fig:spatial_correlation_description}, the channel rank vector at $\boldsymbol{p}_i$ is highlighted in red-dashed arrows demonstrating channel rank through all altitudes and thresholds in the multiple 3D target area structure. 

Then, the correlation between two locations of the UAV can be calculated using the Pearson linear correlation function using the channel rank vector defined in (\ref{eq:channel_rank_vector}), which can be expressed as 
\begin{align}
        &\phi(\boldsymbol{p}_{i}, \boldsymbol{p}_{j}) = \nonumber \\
        &\frac{\sum_{u=1}^{N} \big(\boldsymbol{R}_u(\boldsymbol{p}_{i})-\bar{\boldsymbol{R}}(\boldsymbol{p}_{i}) \big) \big(\boldsymbol{R}_u(\boldsymbol{p}_{j}) - \bar{\boldsymbol{R}}(\boldsymbol{p}_{j}) \big)}{[ \sum_{u=1}^{N}\{ \boldsymbol{R}_u(\boldsymbol{p}_{i}) - \bar{\boldsymbol{R}}(\boldsymbol{p}_{i})\}^2  \sum_{o=1}^{N}\{\boldsymbol{R}_o(\boldsymbol{p}_{j}) -\bar{\boldsymbol{R}}(\boldsymbol{p}_{j}) \}^2 ]^{1/2}} ,
    \label{eq:correlation}
\end{align}
where $\boldsymbol{R}_{u}(\boldsymbol{p}_{i})$ denotes the $u$-th element in the channel rank vector $\boldsymbol{R}(\boldsymbol{p}_{i})$ at $\boldsymbol{p}_{i}$, and $\bar{\boldsymbol{R}}(\boldsymbol{p}_{i})$ indicates the mean of the channel rank vector at $\boldsymbol{p}_{i}$, respectively. Since channel rank vector in (\ref{eq:channel_rank_vector}) contains channel rank over all altitudes, the correlation between two locations in (\ref{eq:correlation}) represents correlation in both vertical and horizontal dimensions.


To derive generalized spatial correlation at  all locations of the UAV, the following procedure is considered in this work, which is also illustrated in Figure \ref{fig:spatial_correlation_description}: 
\begin{itemize}
    \item \textbf{Step-1:} Construct the channel rank vector for all UAV position as defined in (\ref{eq:channel_rank_vector}) and (\ref{eq:channel_rank_vector_structure}). 
    
    \item \textbf{Step-2:} Calculate correlation for all UAV positions using equation (\ref{eq:correlation}), i.e., $\phi(\boldsymbol{p}_{i},\boldsymbol{p}_{j})$ for $i, j = 1, \cdots , N_{\mathrm{loc}}$, which has the dimension of $N_{\mathrm{loc}} \times N_{\mathrm{loc}}$, where $N_{\mathrm{loc}}$ is the total number of possible UAV positions in the target area. It results in a symmetric correlation matrix as
    \begin{multline}        
       \hat{\boldsymbol{\Phi}} =  \\  \begin{bmatrix}
\phi(\boldsymbol{p}_1, \boldsymbol{p}_1) & \phi(\boldsymbol{p}_1, \boldsymbol{p}_2) & \dots & \phi(\boldsymbol{p}_1, \boldsymbol{p}_{N_{\mathrm{loc}}}) \\
\phi(\boldsymbol{p}_2, \boldsymbol{p}_1) & \phi(\boldsymbol{p}_2, \boldsymbol{p}_2) & \dots & \phi(\boldsymbol{p}_2, \boldsymbol{p}_{N_{\mathrm{loc}}}) \\
\vdots & \vdots & \ddots & \vdots \\
\phi(\boldsymbol{p}_{N_{\mathrm{loc}}}, \boldsymbol{p}_1) & \phi(\boldsymbol{p}_{N_{\mathrm{loc}}}, \boldsymbol{p}_2) & \dots & \phi(\boldsymbol{p}_{N_{\mathrm{loc}}}, \boldsymbol{p}_{N_{\mathrm{loc}}}) \\
\end{bmatrix}.
    \end{multline}
    \item \textbf{Step-3:} Arrange the correlation from the previous step according to the horizontal (2D) distance between two UAV locations $\boldsymbol{p}_i$ and $ \boldsymbol{p}_j$, which can be expressed as
    \begin{equation}\label{eq:horizontal_distance}
        \Delta_{\mathrm{2D}}(\boldsymbol{p}_i, \boldsymbol{p}_j) = \sqrt{(x_i - x_j)^2 + (y_i - y_j)^2},
    \end{equation} 
    to ensure the correlations have ascending order based on the horizontal distance between corresponding positions. This process results in arranged pairs of 
    \begin{equation}
        \big\{ \Delta_{\mathrm{2D}}(\cdot, \cdot), (\phi_n \mid \Delta_{\mathrm{2D}}^{(n)} = n \cdot d_{\mathrm{rx}} ) \big\}, \mathrm{for} ~ n=0,1,..., N-1,
    \end{equation} where $d_{\mathrm{rx}}$ is the interval of the neighboring UAV positions, e.g., $\boldsymbol{p}_i$ and $\boldsymbol{p}_{i+1}$, and $N$ is the total number of intervals for covering all pairs. Moreover, horizontal distance $\Delta_{\mathrm{2D}}(\cdot, \cdot)$ is discretized by the unit of interval $d_{\mathrm{rx}}$, which leads to $\Delta_{\mathrm{2D}}(\cdot, \cdot)=\Delta_{\mathrm{2D}}^{(n)} = n \cdot d_{\mathrm{rx}}$, where $n$ indicates $n$-th interval. 
    \item \textbf{Step-4:} Take average every $d_{\mathrm{rx}}$ from the smallest distance-correlation pairs until covering all pairs within the interval. It can be expressed as
    \begin{equation}
        \bar{\phi}_n = \frac{1}{M_n} \sum_{m \in \mathcal{I}_n} \phi(\boldsymbol{p}_{i_m}, \boldsymbol{p}_{j_m}),
    \end{equation}
    
    \begin{align}
        \Phi(\Delta_{\mathrm{2D}}(\boldsymbol{p}_i, \boldsymbol{p}_j)) = \nonumber \\
        & \begin{cases} 
        \bar{\phi}_0 & \text{if } \Delta_{\mathrm{2D}}(\boldsymbol{p}_i, \boldsymbol{p}_j) = 0, \\
        \bar{\phi}_1 & \text{if } \Delta_{\mathrm{2D}}(\boldsymbol{p}_i, \boldsymbol{p}_j) = d_{\mathrm{rx}}, \\
        \bar{\phi}_2 & \text{if } \Delta_{\mathrm{2D}}(\boldsymbol{p}_i, \boldsymbol{p}_j) = 2 \cdot d_{\mathrm{rx}}, \\
        \vdots & \vdots \\
        \bar{\phi}_{N-1} & \text{if } \Delta_{\mathrm{2D}}(\boldsymbol{p}_i, \boldsymbol{p}_j) = \\
        & ~~~~~~~~~~\quad (N-1) \cdot d_{\mathrm{rx}},
        \end{cases}\label{eq:correlation_cases}
    \end{align}
    where $M_n$ is the number of pairs with $\Delta_{\mathrm{2D}}^{(n)} = n \cdot d_{\mathrm{rx}}$ and $\mathcal{I}_n$ is the set of indices of UAV position pairs satisfying $\Delta_{\mathrm{2D}}^{(n)} = n \cdot d_{\mathrm{rx}}$.
\end{itemize}


As a result of the previous steps, the generalized correlation of channel ranks as a function of horizontal distance is derived, quantifying the correlation between channel rank vectors at different UAV positions. To further simplify the computation of the correlation function, $\Phi(\Delta_{\mathrm{2D}}(\cdot, \cdot))$, we approximate it using a bi-exponential curve-fitting model for two UAV positions $\boldsymbol{p}_i$ and $\boldsymbol{p}_j$:
\begin{equation}
     \Phi(\Delta_{\mathrm{2D}}(\boldsymbol{p}_i, \boldsymbol{p}_j)) \approx c_{1} e^{c_{2} \Delta_{\mathrm{2D}}(\boldsymbol{p}_i, \boldsymbol{p}_j)} + c_{3} e^{c_{4} \Delta_{\mathrm{2D}}(\boldsymbol{p}_i, \boldsymbol{p}_j)},
     \label{eq:bi_exponential}
 \end{equation}
 where coefficients $c_1, c_2, c_3$, and $c_4$ are selected based on curve-fitting.

 \subsection{Kriging Interpolation-based Channel Rank Interpolation}
 
The ordinary Kriging interpolation method provides optimal estimation when the spatial correlation is well-defined as a function of distance. The Kriging interpolation uses squared-error loss between the observation in a known spatial location and unknown locations \cite{Cressie_2015}. In particular, channel rank at a location $\boldsymbol{p_0}$ with a threshold $\delta_{K_j}$ is interpolated from the linear combination of the channel rank samples of the nearby locations, which can be expressed as \cite{maeng2023kriging}
 \begin{align}
     \min_{l_{1}, ..., l_{M}} \mathop{\mathbb{E}}\Big[\big(\hat{R}_{\delta_{K_j}}&(\boldsymbol{p}_{0})-{R}_{\delta_{K_j}}(\boldsymbol{p}_{0})\big)^2\Big],\\
     \mathrm{\mathbf{ s. t.}} ~~~~ \hat{R}_{\delta_{K_j}}(\boldsymbol{p}_{0}) = & \sum_{i=1}^{M} l_{i} R_{\delta_{K_j}}(\boldsymbol{p}_{i}),
     \label{eq:Kriging_estimation}\\
     \sum_{i=1}^{M} & l_{i} = 1,
 \end{align}
where $\boldsymbol{p}_0$ is the location to be interpolated, $l_i$ denotes the coefficient for the linear combination, $R_{\delta_{K_j}}(\boldsymbol{p}_{i})$ is $i$-th element of the channel rank within the sampled locations for the interpolation, $\hat{R}_{\delta_{K_j}}(\boldsymbol{p}_{0})$ indicates the estimated channel rank with threshold $\delta_{K_j}$, and $M$ indicates the number of samples used for the interpolation, respectively. The optimization problem above can be solved by converting the problem into an equivalent Lagrange equation as \cite{maeng2023kriging, Kriging_solution_paper}
\begin{multline} 
    \min_{l_{1}, ..., l_{M}} \mathop{\mathbb{E}}\left[\left(R_{\delta_{K_j}}(\boldsymbol{p}_{0}) - \sum_{i=1}^{M} l_{i} R_{\delta_{K_j}}(\boldsymbol{p}_{i}) \right)^2\right]   \\ - L\left(\sum_{i=1}^{M} l_{i} - 1\right),
\end{multline}
where $L$ denotes the Lagrange multiplier. After a few steps, the optimal solution can be derived by the linear matrix equation as
\begin{align}
        \begin{bmatrix}
    \gamma(\boldsymbol{p}_{1}, \boldsymbol{p}_{1}) & ... & \gamma(\boldsymbol{p}_{1}, \boldsymbol{p}_{M}) & 1 \\
    \gamma(\boldsymbol{p}_{2}, \boldsymbol{p}_{1}) & ... & \gamma(\boldsymbol{p}_{2}, \boldsymbol{p}_{M}) & 1 \\
    \vdots & \vdots & \vdots & \vdots \\
    \gamma(\boldsymbol{p}_{M}, \boldsymbol{p}_{1}) & ... & \gamma(\boldsymbol{p}_{M}, \boldsymbol{p}_{M}) & 1 \\
    1 & ... & 1 & 0
    \end{bmatrix}
    \begin{bmatrix}
        l_1 \\
        l_2 \\ 
        \vdots \\
        l_M \\
        L'
    \end{bmatrix}
    %\\ 
    = \begin{bmatrix}
        \gamma(\boldsymbol{p}_{0}, \boldsymbol{p}_{1}) \\
        \gamma(\boldsymbol{p}_{0}, \boldsymbol{p}_{2}) \\
        \vdots \\
        \gamma(\boldsymbol{p}_{0}, \boldsymbol{p}_{M}) \\
        1
    \end{bmatrix},
    \label{eq:linear_matrix}
\end{align}
where $\gamma(\cdot, \cdot)$ indicates the semi-variogram of the Kriging interpolation for two locations of the UAV. The semi-variogram of the Kriging interpolation scheme can be expressed as
\begin{equation}
    \gamma(\boldsymbol{p}_{i},   \boldsymbol{p}_{j}) = v_{\delta_{K_j}}^{2} \Big(1-\Phi \big(\Delta_{\mathrm{2D}}(\boldsymbol{p}_i, \boldsymbol{p}_j)\big)\Big),
    \label{eq:semi_variogram}
\end{equation}
where $\Phi(\Delta_{\mathrm{2D}}(\boldsymbol{p}_i, \boldsymbol{p}_j))$ is as defined in (\ref{eq:correlation_cases}) and (\ref{eq:bi_exponential}), and $v_{\delta_{K_j}}^{2}$ is the variance of the channel rank calculated with a threshold at a location with all altitudes. This can be expressed as 
\begin{equation}
    v_{\delta_{K_j}}^{2} = \frac{1}{N_{h}-1}\sum_{s=1}^{N_{h}}|R_{\delta_{K_j}}(x, y, h_s) - \mu|^2,
\end{equation}
where $\mu$ is the mean of $R_{\delta_{K_j}}(x, y, h_{s})$ over all $h_{s}$ for $s=1, ..., N_{h}$. The semi-variogram indicates the spatial correlation in different locations. The expression above is derived with the assumption of the existence of the covariance function of a stationary process. 

The procedure for the Kriging interpolation-based 3D channel rank interpolation scheme, which is also highlighted in Figure \ref{fig:Kriging_description}, can be summarized as follows:
\begin{itemize} 
 \item \textbf{Step-1:} Sample $M$ channel rank data within a radius of $r_{0}$ from the target UAV position, $\boldsymbol{p}_{0}$.
 
 \item \textbf{Step-2:} Calculate semi-variogram defined in equation (\ref{eq:semi_variogram}).

 \item \textbf{Step-3:} Derive coefficient for the linear combination, $l_i$, by substituting semi-variogram into linear matrix equation in (\ref{eq:linear_matrix}).

 \item \textbf{Step-4:} Process interpolation in equation (\ref{eq:Kriging_estimation}) with derived $l_i$ and channel rank for the target UAV position.

 \item \textbf{Step-5:} Repeat all the previous steps for all possible UAV locations.
\end{itemize} 

  \begin{figure}
     \centering
     \includegraphics[trim={0.15cm 0.15cm 0.8cm 0.4cm},clip,width=0.95\columnwidth]{images/Kriging_description_figure_v5.png}
     \caption{Procedure of the Kriging interpolation-based 3D channel rank interpolation scheme.}
     \label{fig:Kriging_description}
 \end{figure}

 \subsection{Other Baseline Interpolation Techniques}\label{ch:other_baseline}
We consider spline and makima (modified Akima cubic Hermite) interpolation methods as baseline interpolation approaches for comparison purposes. In the baseline interpolation approaches, the UAV position index as described right after (\ref{eq:channel_rank_vector}), $i=1, ..., N_{\mathrm{loc}}$, is used to identify UAV positions within a 2D target area that is discretized into a uniform grid with spacing $d_{\mathrm{rx}}$. 
% The indexing follows a structured ordering: the index first increases along the x-axis from west to east, and after reaching the end of a row, it moves one step along the y-axis before continuing again from west to east. This ensures that each UAV position in the grid is assigned a unique index number based on its position in the structured grid. 

The procedure for the baseline interpolation-based 3D channel rank interpolation can be summarized as follows: 1) Sample $M$ UAV positions within a radius of $r_{0}$ from the target UAV position $\boldsymbol{p}_0$, 2) sort the UAV position index and channel rank with respect to the index, 3) perform interpolation using sorted index and channel rank, i.e., $\hat{R}_{\delta_{K_j}}(\boldsymbol{p}_{0})=\mathsf{B}(\boldsymbol{X}, R_{\delta_{K_j}}(\boldsymbol{p}_{\boldsymbol{X}}), \boldsymbol{p}_{0})$, where $\mathsf{B}()$ indicates the baseline interpolation function with spline or makima approaches, $\boldsymbol{X}$ represents all the sampled UAV positions index within $r_0$ radius from the target point, and $R_{\delta_{K_j}}(\boldsymbol{p}_{\boldsymbol{X}})$ is the channel rank data of the corresponding locations $\boldsymbol{X}$, respectively.  

\section{Numerical Results}\label{ch:numerical_results}
In this section, we present numerical results on RSS, channel rank, and Kriging-based channel rank interpolation schemes for different UAV scenarios in a rural environment, in Section~\ref{ch:RF_coverage}, Section~\ref{ch:channel_rank}, and Section~\ref{ch:Kriging_interpolation}, respectively. The RT results are also compared with the real measurements in the NSF AERPAW testbed in Section~VII.\ref{ch:measurement}. In addition to the simulation parameters listed in Section~\ref{ch:RT_setup}, the following assumptions are used for the corresponding analysis: 1) the number of threshold ratio constant $N_{K} = 3$, where $K_1=10^1, K_2=10^2$, and $K_3=10^3$ leading $\delta_{K_1}=\sigma_1(\boldsymbol{p})/10^1$, $\delta_{K_2}=\sigma_1(\boldsymbol{p})/10^2$, and $\delta_{K_3}=\sigma_1(\boldsymbol{p})/10^3$ to capture singular values in $10$~dB, $20$~dB and $30$~dB range from the strongest singular value, respectively; 2) the horizontal interval between UAV positions $d_{\mathrm{rx}}$ is set to $30$~m, which leads to the total number of possible UAV locations $N_{\mathrm{loc}}= 36 \times 71 = 2556$ for the given 2D target area $T_{\mathrm{LW}}$ in Table~\ref{tab:sim_param}; 3) the number of altitudes of UAV $N_{h} = 9$ with $10$~m interval from $30$~m to $110$~m; 4) the altitudes of interest in RSS analysis is $30$~m, $70$~m, and $110$~m; and 5) in addition to the altitudes of interest in RSS simulation, $3$~m altitude configuration is also applied to the channel rank RT simulation to highlight the blockage effects.

 \subsection{Received Signal Strength and Coverage Analysis}\label{ch:RF_coverage}

 \begin{figure*}[t!]
    \centering    
    \subfigure[LW1 $30$~m]{
    \includegraphics[trim={0.1cm 1.0cm 0.15cm 1.7cm},clip,width=0.37\columnwidth]{images/RSS_MIMO_30m_v4_LW1}
    \label{fig:RSSI_MIMO_30_LW1}
    }
    \subfigure[LW2 $30$~m]{
    \includegraphics[trim={0.1cm 1.0cm 0.15cm 1.7cm},clip,width=0.37\columnwidth]{images/RSS_MIMO_30m_v4_LW2}
    \label{fig:RSSI_MIMO_30_LW2}
    }
    \subfigure[LW3 $30$~m]{
    \includegraphics[trim={0.1cm 1.0cm 0.15cm 1.7cm},clip,width=0.37\columnwidth]{images/RSS_MIMO_30m_v4_LW3}
    \label{fig:RSSI_MIMO_30_LW3}
    }
    \subfigure[LW4 $30$~m]{
    \includegraphics[trim={0.1cm 1.0cm 0.15cm 1.7cm},clip,width=0.37\columnwidth]{images/RSS_MIMO_30m_v4_LW4}
    \label{fig:RSSI_MIMO_30_LW4}
    }
    \subfigure[LW5 $30$~m]{
    \includegraphics[trim={0.1cm 1.0cm 0.15cm 1.7cm},clip,width=0.37\columnwidth]{images/RSS_MIMO_30m_v4_LW5}
    \label{fig:RSSI_MIMO_30_LW5}
    }    
    \subfigure[LW1 $70$~m]{
    \includegraphics[trim={0.1cm 1.0cm 0.15cm 1.7cm},clip,width=0.37\columnwidth]{images/RSS_MIMO_70m_v4_LW1}
    \label{fig:RSSI_MIMO_70_LW1}
    }
    \subfigure[LW2 $70$~m]{
    \includegraphics[trim={0.1cm 1.0cm 0.15cm 1.7cm},clip,width=0.37\columnwidth]{images/RSS_MIMO_70m_v4_LW2}
    \label{fig:RSSI_MIMO_70_LW2}
    }
    \subfigure[LW3 $70$~m]{
    \includegraphics[trim={0.1cm 1.0cm 0.15cm 1.7cm},clip,width=0.37\columnwidth]{images/RSS_MIMO_70m_v4_LW3}
    \label{fig:RSSI_MIMO_70_LW3}
    }
    \subfigure[LW4 $70$~m]{
    \includegraphics[trim={0.1cm 1.0cm 0.15cm 1.7cm},clip,width=0.37\columnwidth]{images/RSS_MIMO_70m_v4_LW4}
    \label{fig:RSSI_MIMO_70_LW4}
    }
    \subfigure[LW5 $70$~m]{
    \includegraphics[trim={0.1cm 1.0cm 0.15cm 1.7cm},clip,width=0.37\columnwidth]{images/RSS_MIMO_70m_v4_LW5}
    \label{fig:RSSI_MIMO_70_LW5}
    }    
    \subfigure[LW1 $110$~m]{
    \includegraphics[trim={0.1cm 1.0cm 0.15cm 1.7cm},clip,width=0.37\columnwidth]{images/RSS_MIMO_110m_v4_LW1}
    \label{fig:RSSI_MIMO_110_LW1}
    }
    \subfigure[LW2 $110$~m]{
    \includegraphics[trim={0.1cm 1.0cm 0.15cm 1.7cm},clip,width=0.37\columnwidth]{images/RSS_MIMO_110m_v4_LW2}
    \label{fig:RSSI_MIMO_110_LW2}
    }
    \subfigure[LW3 $110$~m]{
    \includegraphics[trim={0.1cm 1.0cm 0.15cm 1.7cm},clip,width=0.37\columnwidth]{images/RSS_MIMO_110m_v4_LW3}
    \label{fig:RSSI_MIMO_110_LW3}
    }
    \subfigure[LW4 $110$~m]{
    \includegraphics[trim={0.1cm 1.0cm 0.15cm 1.7cm},clip,width=0.37\columnwidth]{images/RSS_MIMO_110m_v4_LW4}
    \label{fig:RSSI_MIMO_110_LW4}
    }
    \subfigure[LW5 $110$~m]{
    \includegraphics[trim={0.1cm 1.0cm 0.15cm 1.7cm},clip,width=0.37\columnwidth]{images/RSS_MIMO_110m_v4_LW5}
    \label{fig:RSSI_MIMO_110_LW5}
    }    
   \caption{RSS in the Lake Wheeler Field Labs area with MIMO and various altitude configurations.}
    \label{fig:RSSI_MIMO_all_altitudes}
\end{figure*} 


 \begin{figure*}[t!]
    \centering    
    \subfigure[LW1 $30$~m]{
    \includegraphics[trim={0.1cm 1.0cm 0.15cm 1.7cm},clip,width=0.37\columnwidth]{images/RSS_SISO_30m_v5_LW1}
    \label{fig:RSSI_SISO_30_LW1}
    }
    \subfigure[LW2 $30$~m]{
    \includegraphics[trim={0.1cm 1.0cm 0.15cm 1.7cm},clip,width=0.37\columnwidth]{images/RSS_SISO_30m_v5_LW2}
    \label{fig:RSSI_SISO_30_LW2}
    }
    \subfigure[LW3 $30$~m]{
    \includegraphics[trim={0.1cm 1.0cm 0.15cm 1.7cm},clip,width=0.37\columnwidth]{images/RSS_SISO_30m_v5_LW3}
    \label{fig:RSSI_SISO_30_LW3}
    }
    \subfigure[LW4 $30$~m]{
    \includegraphics[trim={0.1cm 1.0cm 0.15cm 1.7cm},clip,width=0.37\columnwidth]{images/RSS_SISO_30m_v5_LW4}
    \label{fig:RSSI_SISO_30_LW4}
    }
    \subfigure[LW5 $30$~m]{
    \includegraphics[trim={0.1cm 1.0cm 0.15cm 1.7cm},clip,width=0.37\columnwidth]{images/RSS_SISO_30m_v5_LW5}
    \label{fig:RSSI_SISO_30_LW5}
    }    
    \subfigure[LW1 $70$~m]{
    \includegraphics[trim={0.1cm 1.0cm 0.15cm 1.7cm},clip,width=0.37\columnwidth]{images/RSS_SISO_70m_v5_LW1}
    \label{fig:RSSI_SISO_70_LW1}
    }
    \subfigure[LW2 $70$~m]{
    \includegraphics[trim={0.1cm 1.0cm 0.15cm 1.7cm},clip,width=0.37\columnwidth]{images/RSS_SISO_70m_v5_LW2}
    \label{fig:RSSI_SISO_70_LW2}
    }
    \subfigure[LW3 $70$~m]{
    \includegraphics[trim={0.1cm 1.0cm 0.15cm 1.7cm},clip,width=0.37\columnwidth]{images/RSS_SISO_70m_v5_LW3}
    \label{fig:RSSI_SISO_70_LW3}
    }
    \subfigure[LW4 $70$~m]{
    \includegraphics[trim={0.1cm 1.0cm 0.15cm 1.7cm},clip,width=0.37\columnwidth]{images/RSS_SISO_70m_v5_LW4}
    \label{fig:RSSI_SISO_70_LW4}
    }
    \subfigure[LW5 $70$~m]{
    \includegraphics[trim={0.1cm 1.0cm 0.15cm 1.7cm},clip,width=0.37\columnwidth]{images/RSS_SISO_70m_v5_LW5}
    \label{fig:RSSI_SISO_70_LW5}
    }    
    \subfigure[LW1 $110$~m]{
    \includegraphics[trim={0.1cm 1.0cm 0.15cm 1.7cm},clip,width=0.37\columnwidth]{images/RSS_SISO_110m_v5_LW1}
    \label{fig:RSSI_SISO_110_LW1}
    }
    \subfigure[LW2 $110$~m]{
    \includegraphics[trim={0.1cm 1.0cm 0.15cm 1.7cm},clip,width=0.37\columnwidth]{images/RSS_SISO_110m_v5_LW2}
    \label{fig:RSSI_SISO_110_LW2}
    }
    \subfigure[LW3 $110$~m]{
    \includegraphics[trim={0.1cm 1.0cm 0.15cm 1.7cm},clip,width=0.37\columnwidth]{images/RSS_SISO_110m_v5_LW3}
    \label{fig:RSSI_SISO_110_LW3}
    }
    \subfigure[LW4 $110$~m]{
    \includegraphics[trim={0.1cm 1.0cm 0.15cm 1.7cm},clip,width=0.37\columnwidth]{images/RSS_SISO_110m_v5_LW4}
    \label{fig:RSSI_SISO_110_LW4}
    }
    \subfigure[LW5 $110$~m]{
    \includegraphics[trim={0.1cm 1.0cm 0.15cm 1.7cm},clip,width=0.37\columnwidth]{images/RSS_SISO_110m_v5_LW5}
    \label{fig:RSSI_SISO_110_LW5}
    }    
   \caption{RSS in the Lake Wheeler Field Labs area with SISO and $30, 70,$ and $110$~m altitude configurations.}
    \label{fig:RSSI_SISO_all_altitudes}
\end{figure*} 

\begin{figure}[t!]
    \centering
    \subfigure[MIMO $30$~m]{
    \includegraphics[trim={0.1cm 0.8cm 0.5cm 1.7cm},clip,width=0.45\columnwidth]{images/RSS_MIMO_30m_v6_joint}
    \label{fig:RSSI_MIMO_30_joint}
    }
    \subfigure[SISO $30$~m]{
    \includegraphics[trim={0.1cm 0.8cm 0.5cm 1.7cm},clip,width=0.45\columnwidth]{images/RSS_SISO_30m_v6_joint}
    \label{fig:RSSI_SISO_30_joint}
    }    
    \subfigure[MIMO $70$~m]{
    \includegraphics[trim={0.1cm 0.8cm 0.5cm 1.7cm},clip,width=0.45\columnwidth]{images/RSS_MIMO_70m_v6_joint}
    \label{fig:RSSI_MIMO_70_joint}
    }
    \subfigure[SISO $70$~m]{
    \includegraphics[trim={0.1cm 0.8cm 0.5cm 1.7cm},clip,width=0.45\columnwidth]{images/RSS_SISO_70m_v6_joint}
    \label{fig:RSSI_SISO_70_joint}
    }
    \subfigure[MIMO $110$~m]{
    \includegraphics[trim={0.1cm 0.8cm 0.5cm 1.7cm},clip,width=0.45\columnwidth]{images/RSS_MIMO_110m_v6_joint}
    \label{fig:RSSI_MIMO_110_joint}
    }
    \subfigure[SISO $110$~m]{
    \includegraphics[trim={0.1cm 0.8cm 0.5cm 1.7cm},clip,width=0.45\columnwidth]{images/RSS_SISO_110m_v6_joint}
    \label{fig:RSSI_SISO_110_joint}
    }
    \caption{RSS in the Lake Wheeler Field Labs area with the joint coverage and $30, 70,$ and $110$~m altitude configurations.}
    \label{fig:RSSI_SISO_MIMO_all_joint}
\end{figure}


 \begin{figure*}[t!]
    \centering    
    \subfigure[$30$~m MIMO]{
    \includegraphics[trim={0.4cm 0cm 0.8cm 0.6cm},clip,width=0.63\columnwidth]{images/CDF_RSS_MIMO_30m_v2}
    \label{fig:RSSI_CDF_30m_MIMO}
    }
    \subfigure[$70$~m MIMO]{
    \includegraphics[trim={0.4cm 0cm 0.8cm 0.6cm},clip,width=0.63\columnwidth]{images/CDF_RSS_MIMO_70m_v2}
    \label{fig:RSSI_CDF_70m_MIMO}
    }
    \subfigure[$110$~m MIMO]{
    \includegraphics[trim={0.4cm 0cm 0.8cm 0.6cm},clip,width=0.63\columnwidth]{images/CDF_RSS_MIMO_110m_v2}
    \label{fig:RSSI_CDF_110m_MIMO}
    }
    \subfigure[$30$~m SISO]{
    \includegraphics[trim={0.4cm 0cm 0.8cm 0.6cm},clip,width=0.63\columnwidth]{images/CDF_RSS_SISO_30m_v2}
    \label{fig:RSSI_CDF_30m_SISO}
    }
    \subfigure[$70$~m SISO]{
    \includegraphics[trim={0.4cm 0cm 0.8cm 0.6cm},clip,width=0.63\columnwidth]{images/CDF_RSS_SISO_70m_v2}
    \label{fig:RSSI_CDF_70m_SISO}
    }
    \subfigure[$110$~m SISO]{
    \includegraphics[trim={0.4cm 0cm 0.8cm 0.6cm},clip,width=0.63\columnwidth]{images/CDF_RSS_SISO_110m_v2}
    \label{fig:RSSI_CDF_110m_SISO}
    }
   \caption{CDFs of RSS in the Lake Wheeler Field Labs area with MIMO and SISO configurations.}
    \label{fig:RSSI_CDFs}
\end{figure*} 


 In this section, we analyze the RSS for UAV channels using the RT simulation tool. We consider both single-input single-output (SISO) and MIMO antenna configuration ($4\times 1$ linear array at the tower) with three different UAV altitudes: $30$~m, $70$~m, and $110$~m  from the ground. For MIMO, we consider a $4$-element linear array antenna placed along the y-axis at each tower. We study the coverage of each tower individually and when they operate simultaneously with the other towers. For the latter scenario, each UAV receiver is connected to the closest tower. The Voronoi boundaries of the coverages under the joint coverage assumption are marked in blue dashed lines in the figures. 

 Figures \ref{fig:RSSI_MIMO_all_altitudes} and \ref{fig:RSSI_SISO_all_altitudes} show the simulation results of RSS distribution over the Lake Wheeler Field Labs for three different altitudes with SISO and MIMO configurations, respectively. Moreover, the simulation results of RSS distribution with the joint coverage assumption are shown in Figure~\ref{fig:RSSI_SISO_MIMO_all_joint}. Here, towers are marked in red arrows and buildings are marked in red polygons. Moreover, the forest areas are marked in magenta polygons. The letter $\mathsf{Z}$ represents the out-of-coverage area with the given tower and UAV, where the signal strength is below a threshold value (e.g. due to blockage). In Figure \ref{fig:RSSI_MIMO_all_altitudes}, narrow distribution patterns through the horizontal axis and side lobes in the diagonal direction are observed due to the beam pattern of the MIMO linear array. It is observed that the beam pattern leads to the narrow coverage, which can be seen around the LW2, LW3, and LW5 towers. Meanwhile, due to the omnidirectional beam pattern of SISO configuration, the RSS distribution in Figure \ref{fig:RSSI_SISO_all_altitudes} tends to have a circular pattern near the base stations. 
 The blockage effects from the trees are observed in $30$~m in both SISO and MIMO cases, which is gradually mitigated at the higher UAV altitudes. 
 
 The RSS simulation results of  Figures~\ref{fig:RSSI_MIMO_all_altitudes} -~\ref{fig:RSSI_SISO_MIMO_all_joint} are summarized with CDFs in Figure \ref{fig:RSSI_CDFs}. The portion of the blockage area with the LW3 site is the highest of all cases due to LW3 being at the edge of the flight area and being most blocked by the trees. It is observed that the LW1 tower has the least blockage due to its location being in an open area close to the center of the field. The blockage is reduced as the altitude of the UAV increases, while the higher values of the RSS are slightly degraded due to larger path loss at higher altitudes. The blockage with the joint coverage is significantly reduced even with the $30$~m altitude, where there is no blockage with the higher altitudes.  Moreover, it can be observed that the range of RSS tends to be wider in the MIMO cases that have multiple directional beam patterns than the omnidirectional SISO cases.  

 \subsection{Channel Rank Analysis}\label{ch:channel_rank}

  \begin{figure*}[t!]
    \centering    
    \subfigure[$K_1$, $3$~m]{
    \includegraphics[trim={0.15cm 1.0cm 1.3cm 1.7cm},clip,width=0.47\columnwidth]{images/channel_rank_distribution_3m_LW1_LW5_10}
    \label{fig:Rank_MIMO_3_10}
    }
    \subfigure[$K_1$, $30$~m]{
    \includegraphics[trim={0.15cm 1.0cm 1.3cm 1.7cm},clip,width=0.47\columnwidth]{images/channel_rank_distribution_30m_LW1_LW5_10_v6}
    \label{fig:Rank_MIMO_30_10}
    }
    \subfigure[$K_1$, $70$~m]{
    \includegraphics[trim={0.15cm 1.0cm 1.3cm 1.7cm},clip,width=0.47\columnwidth]{images/channel_rank_distribution_70m_LW1_LW5_10_v6}
    \label{fig:Rank_MIMO_70_10}
    }
    \subfigure[$K_1$, $110$~m]{
    \includegraphics[trim={0.15cm 1.0cm 1.3cm 1.7cm},clip,width=0.47\columnwidth]{images/channel_rank_distribution_110m_LW1_LW5_10_v6}
    \label{fig:Rank_MIMO_110_10}
    }   
    \subfigure[$K_2$, $3$~m]{
    \includegraphics[trim={0.15cm 1.0cm 1.3cm 1.7cm},clip,width=0.47\columnwidth]{images/channel_rank_distribution_3m_LW1_LW5_100}
    \label{fig:Rank_MIMO_3_100}
    }
    \subfigure[$K_2$, $30$~m]{
    \includegraphics[trim={0.15cm 1.0cm 1.3cm 1.7cm},clip,width=0.47\columnwidth]{images/channel_rank_distribution_30m_LW1_LW5_100_v6}
    \label{fig:Rank_MIMO_30_100}
    }
    \subfigure[$K_2$, $70$~m]{
    \includegraphics[trim={0.15cm 1.0cm 1.3cm 1.7cm},clip,width=0.47\columnwidth]{images/channel_rank_distribution_70m_LW1_LW5_100_v6}
    \label{fig:Rank_MIMO_70_100}
    }
    \subfigure[$K_2$, $110$~m]{
    \includegraphics[trim={0.15cm 1.0cm 1.3cm 1.7cm},clip,width=0.47\columnwidth]{images/channel_rank_distribution_110m_LW1_LW5_100_v6}
    \label{fig:Rank_MIMO_110_100}
    }
    \subfigure[$K_3$, $3$~m]{
    \includegraphics[trim={0.15cm 1.0cm 1.3cm 1.7cm},clip,width=0.47\columnwidth]{images/channel_rank_distribution_3m_LW1_LW5_1000}
    \label{fig:Rank_MIMO_3_1000}
    }
    \subfigure[$K_3$, $30$~m]{
    \includegraphics[trim={0.15cm 1.0cm 1.3cm 1.7cm},clip,width=0.47\columnwidth]{images/channel_rank_distribution_30m_LW1_LW5_1000_v6}
    \label{fig:Rank_MIMO_30_1000}
    }
    \subfigure[$K_3$, $70$~m]{
    \includegraphics[trim={0.15cm 1.0cm 1.3cm 1.7cm},clip,width=0.47\columnwidth]{images/channel_rank_distribution_70m_LW1_LW5_1000_v6}
    \label{fig:Rank_MIMO_70_1000}
    }
    \subfigure[$K_3$, $110$~m]{
    \includegraphics[trim={0.15cm 1.0cm 1.3cm 1.7cm},clip,width=0.47\columnwidth]{images/channel_rank_distribution_110m_LW1_LW5_1000_v6}
    \label{fig:Rank_MIMO_110_1000}
    }
   \caption{Channel rank in the Lake Wheeler Field Labs with various rank thresholds $K_1$, $K_2$, $K_3$ and receiver altitudes of  $3$~m, $30$~m, $70$~m, and $110$~m.}
    \label{fig:Rank_distribution}
\end{figure*} 


\begin{figure*}[t!]
    \centering    
    \subfigure[Distribution with $K_1$]{
    \includegraphics[trim={0.4cm 0.1cm 1.3cm 0.7cm},clip, width=0.64\columnwidth]{images/PDF_channel_rank_10_v3}
    \label{fig:Rank_PDF_MIMO_10}
    }
    \subfigure[Distribution with $K_2$]{
    \includegraphics[trim={0.4cm 0.1cm 1.3cm 0.7cm},clip, width=0.64\columnwidth]{images/PDF_channel_rank_100_v3}
    \label{fig:Rank_PDF_MIMO_100}
    }
    \subfigure[Distribution with $K_3$]{
    \includegraphics[trim={0.4cm 0.1cm 1.3cm 0.7cm},clip, width=0.64\columnwidth]{images/PDF_channel_rank_1000_v3}
    \label{fig:Rank_PDF_MIMO_1000}
    }
    \caption{Distribution of the channel rank in the Lake Wheeler Field Labs area with different threshold constants and different UAV altitudes.}
    \label{fig:Rank_PDF_LWs}
\end{figure*}

The channel rank distribution in the Lake Wheeler area based on RT simulations, with different rank thresholds at $3$~m, $30$~m, $70$~m, and $110$~m UAV altitudes are shown in Figure \ref{fig:Rank_distribution}, where $\mathsf{Z}$ indicates the receiver sites that have no connectivity to the corresponding tower. Here, the locations of the buildings and the tower are marked in red polygons and red arrows. The forest areas with trees are marked in magenta polygons in the figures. The UAV is assumed to be connected to the closest tower when the channel rank is evaluated. The boundaries of each coverage are marked in cyan dashed curves in the figures.

As seen in the cases with $3$~m and $30$~m altitudes, the blockage from the trees is observed in the forest area for all tower cases. The blockage area in the forest tends to be mitigated as the altitude of the UAV increases. We have to note the difference in the blockage areas of Figures \ref{fig:RSSI_MIMO_30_joint} and \ref{fig:Rank_MIMO_30_10}. This can be due to the fact that the number of samples in calculating the candidates of the rays using the Fibonacci unit sphere is set to $10^3$ for channel rank and $10^6$ for the RSS simulations to set a reasonable balance in heavy computational load in channel rank simulations. 

It is also observed that UAV locations near the tower have a channel rank of $2$, which can be expected to have two multipath components with LoS and ground-reflected rays. The area of channel rank $2$ near the tower gets wider as the threshold constant is relaxed from $K_1$ to $K_3$. Meanwhile, the channel rank tends to be $1$ as the distance from the tower increases. It can be interpreted that the link is LoS-dependent as the singular values from the ground-reflected ray are not strong enough compared to the strongest singular value. Moreover, some receiver locations that are close to the tower and other objects (e.g., trees or buildings) have a channel rank of $3$ due to richer reflections in those locations. 

The distribution of the channel rank in the Lake Wheeler area with different threshold constant $K$ are shown in Figure~\ref{fig:Rank_PDF_LWs}. Here, the $3$~m altitude cases show more than $60$ percent of the out-of-coverage area due to significant blockage by the trees. The likelihood of observing channel rank 1 reduces as the threshold constant $K$ increases. Most receiver sites have channel rank 1 or 2, which can be interpreted as most of the links having dominant LoS and ground-reflected ray in the given rural scenario. 


 \subsection{Kriging Based 3D Channel Rank Interpolation}\label{ch:Kriging_interpolation}

  \begin{figure}
     \centering
     \includegraphics[trim={0.6cm 0 1.1cm 0.5cm},clip,width=0.98\columnwidth]{images/correlation_vs_distance_30m_bin_manual_tree_LW2_10000_excluded_v2}
     \caption{Simulation result of 3D correlation vs. distance and curve fitting model in (\ref{eq:bi_exponential}).}
     \label{fig:correlation_distance_curvefitting}
 \end{figure}

 In this section, we present results on the Kriging-based 3D interpolation scheme for channel rank prediction. First, the semi-variogram has to be estimated based on RT simulations. The simulation results of the correlation of channel rank and 3D distance over every UAV position are shown in Figure \ref{fig:correlation_distance_curvefitting}, where blue dots indicate the correlation data calculated by the procedure in Section \ref{ch:Kriging_3D_Channel_interpolation}-\ref{ch:spatial_correlation} and the red line represents the bi-exponential model derived by the curve fitting expression in (\ref{eq:bi_exponential}). We limit the maximum horizontal distance between two locations of interest for spatial correlation analysis, $\Delta_{\mathrm{2D}}(\boldsymbol{p}_{i}, \boldsymbol{p}_{j})$ as in~(\ref{eq:horizontal_distance}) and (\ref{eq:bi_exponential}), as $500$~m to prevent taking into account less correlated UAV positions. The coefficients $c_{1}$ - $c_{4}$ for equation (\ref{eq:bi_exponential}) are derived by the Matlab curve fitting tool \cite{matlab_curve_fitting}, which are $c_{1} = 0.2932, c_{2} = -0.0508, c_{3} = 0.7057,$ and $c_{4} = -0.001$, respectively. The RMSE between actual correlation data and the bi-exponential model is $0.0153$. 

Using the results in Figure~\ref{fig:correlation_distance_curvefitting}, Kriging-based channel rank interpolation is carried out. The interpolated channel rank for all positions of the UAV with $9$ attitudes and $3$ thresholds for the channel rank calculation is evaluated using the MAE error metric, which can be expressed as follows:
\begin{align} \label{eq:MAE}
    E_{\mathrm{MAE}, \delta_{K_j}, h}  = \frac{1}{N_{\mathrm{loc}}}  \sum_{i=1}^{N_{\mathrm{loc}}}\Big|R_{\delta_{K_j}}(\boldsymbol{p}_i) - \hat{R}_{\delta_{K_j}}(\boldsymbol{p}_i)\Big|~, 
\end{align}
for all $j=1, ..., N_K$ and
$h \in (h_1, h_2, ..., h_{N_h})$. Here, $N_{\mathrm{loc}}$ is the total number of possible UAV locations to be interpolated in the uniformly discretized 2D target area as described right after ~(\ref{eq:channel_rank_vector}) and Section~\ref{ch:numerical_results}. Moreover, the number of samples for interpolation $M$ as in (\ref{eq:Kriging_estimation}) is set to $20$ and the radius for sampling from the $\boldsymbol{p}_{0}$, $r_{0}$, as in Figure~\ref{fig:Kriging_description} is $150$~m. The MAE performance evaluations of the Kriging interpolation-based 3D channel rank interpolation scheme and baseline interpolation-based approaches are shown in Figure \ref{fig:MAE_interpolations}. 

Using the previously defined MAE in (\ref{eq:MAE}) and the procedure illustrated in Figure~\ref{fig:Kriging_description}, the MAE of the channel rank is computed for each altitude and threshold setting. The target location $\boldsymbol{p}_0$ is sequentially processed across all UAV location index $i=1,..., N_{\mathrm{loc}}$ as described in Section~\ref{ch:other_baseline}. For each $\boldsymbol{p}_0$, interpolation is performed using $M$ samples within the sampling radius $r_0$ under the assumption that the channel rank at $\boldsymbol{p}_0$ is unknown. This interpolation process is repeated for all possible UAV locations, and the MAE is computed for a given altitude and threshold setting. By iterating this procedure across different altitude and threshold combinations, the MAE is evaluated for all possible configurations.
  
 Results in Figure~\ref{fig:MAE_interpolations} show that the Kriging interpolation-based 3D channel rank interpolation scheme outperforms the MAE of the two baseline interpolation-based approaches. It can be interpreted that the Kriging interpolation yields an accurate interpolation of channel rank using spatial correlation in the given area, while the baseline approaches employ channel rank data and the UAV location index. For all approaches, $K_1$ has the lowest MAE because most channel ranks at UAV locations have a channel rank of $1$ as seen in Figure \ref{fig:Rank_PDF_MIMO_10}. Moreover, it has been observed that the MAE with the altitude of $30$ m yields the highest MAE performance for all thresholds and approaches due to the blockage from the trees. The MAE performance in the lower altitudes, e.g., $30$ - $70$ m, tends to get worse as the threshold constant increases, which is from the different portion of channel rank distribution with the relaxed threshold. Meanwhile, an irregular pattern can be observed at higher altitudes, e.g., $80$ - $110$ m cases. This can be interpreted as being due to relatively low spatial correlation in the case of $K_2$ and $K_3$, where the ratio of channel ranks 1 and 2 is low.
 

  \begin{figure*}
     \centering
     \subfigure[Kriging Interpolation]{\includegraphics[trim={0.4cm 0.1cm 1.3cm 0.6cm},clip,width=0.66\columnwidth]{images/Kriging_channel_rank_MAE_bar_v5}
     \label{fig:MAE_kriging}
     }
     \subfigure[Baseline Interpolation: Makima Method]{\includegraphics[trim={0.4cm 0.1cm 1.3cm 0.6cm},clip,width=0.66\columnwidth]{images/makima_channel_rank_MAE_bar_v3}
     \label{fig:MAE_makima}
     }
     \subfigure[Baseline Interpolation: Spline Method]{\includegraphics[trim={0.4cm 0.1cm 1.3cm 0.6cm},clip,width=0.66\columnwidth]{images/spline_channel_rank_MAE_bar_v3}
     \label{fig:MAE_spline}
     }
     \caption{MAE performance comparison of Kriging interpolation-based 3D channel rank interpolation scheme and baseline interpolation-based approaches.}
     \label{fig:MAE_interpolations}
 \end{figure*}

 \subsection{Measurement Comparison}\label{ch:measurement}
In this section, we present the comparison of RT simulations and real-world measurements of RSS and channel rank. In addition to the RT simulation and measurement campaign setups described in Section~\ref{ch:RT_setup} and Section~\ref{ch:measurement_setup}, the following assumptions are used for the channel rank RT simulation for comparison purposes: 1) $2 \times 2$ equally spaced in vertical and horizontal axis by $0.5 \lambda$ is used for multi-antenna receiver setup with four antennas; 2) non-uniformly spaced linear array antenna is used for transmitter multi-antenna configuration, which has antenna element spacing of $60$~mm, $1.68$~m, and $60$~mm to simulate two sector antenna configuration used in our measurements; 3) $90^\circ$ of roll angle is applied for the receiver at the UAV to implement bottom mounted antenna module;  and 4) the number of samples for Fibonacci lattice unit sphere of Sionna RT is set to $10^6$ to get accurate ray information.
 
\subsubsection{Coverage Measurements} 

The RSS simulation results of RT and the measurements from each tower, over the trajectory of the UAV, as illustrated in Figure~\ref{fig:trajectory_measurement_RSS}, are shown in Figure \ref{fig:measurement_vs_sionna_LW1_LW5}. Measurement methodology for coverage measurements is presented earlier in Section~\ref{ch:measurement_setup}. The labels USRP1 and USRP2 indicate measurements captured simultaneously at two different antennas connected to a dual channel USRP at each tower, and $\mathsf{Z}$ is the out-of-coverage area. The RT approach shows a similar pattern with the measurements over LW1 to LW5 tower cases. It is observed that the measurements show fluctuations throughout the trajectory due to fading effects, changing roll/yaw/pitch of the UAV with respect to each tower, and Doppler effects, among other factors. 

An interesting behavior is observed at the RSS values observed by USRP1 and USRP2 antennas. While at some locations the measurements at USRP1 and USRP are very similar, at other locations, there may be over 10~dB of difference between them, see e.g. the interval right after waypoint C, for LW1, LW2, and LW4 cases, as highlighted on the figures. The RSS variation can be from the geometry-dependent LoS blockage and channel conditions due to different relative orientations of the USRP1 and USRP2 antennas with respect to the UAV. 

We can notice the out-of-coverage and low RSS conditions in RT simulation for LW3, LW4, and LW5 at lower altitudes, i.e., time intervals before $100$~s and after $850$~s when the drone is taking off and landing, respectively. These are due to altitude-dependent blockage at those towers. Moreover, blockage is also observed at LW4 and LW5 (further towers to the UAV's flight area) with RT for other UAV locations. This can be interpreted to be due to the RT ray calculation not being able to find a proper ray with the simulation settings of the number of samples for the Fibonacci lattice unit sphere and assumptions for the height of the trees. The RSS measured at LW5 and some intervals at LW4 show to be constant over time when it is possible to detect it, where the RSS is very close to the background noise due to the far distance with each of these towers.

  \begin{figure*}
     \centering
     \subfigure[LW1]{\includegraphics[trim={0.3cm 0.1cm 1.35cm 0.6cm},clip,width=0.64\columnwidth]{images/RSS_measurement_sionna_40m_LW1_v7}
     \label{fig:measurement_vs_sionna_LW1}
     }
     \subfigure[LW2]{\includegraphics[trim={0.3cm 0.1cm 1.35cm 0.6cm},clip,width=0.64\columnwidth]{images/RSS_measurement_sionna_40m_LW2_v7}
     \label{fig:measurement_vs_sionna_LW2}
     }
     \subfigure[LW3]{\includegraphics[trim={0.3cm 0.1cm 1.35cm 0.6cm},clip,width=0.64\columnwidth]{images/RSS_measurement_sionna_40m_LW3_v7}
     \label{fig:measurement_vs_sionna_LW3}
     }
     \subfigure[LW4]{\includegraphics[trim={0.3cm 0.1cm 1.35cm 0.6cm},clip,width=0.64\columnwidth]{images/RSS_measurement_sionna_40m_LW4_v7}
     \label{fig:measurement_vs_sionna_LW4}
     }
     \subfigure[LW5]{\includegraphics[trim={0.3cm 0.1cm 1.35cm 0.6cm},clip,width=0.64\columnwidth]{images/RSS_measurement_sionna_40m_LW5_v7}
     \label{fig:measurement_vs_sionna_LW5}
     }
     \caption{RSS measurement and RT simulation results with the predefined trajectory for the specific UAV locations {A}-{F} in Figure \ref{fig:trajectory_altitude_measurement_RSS}. }     \label{fig:measurement_vs_sionna_LW1_LW5}
 \end{figure*}

 \subsubsection{Channel Rank Measurements}
 
 RT simulations and real-world measurements to study the channel rank are done based on the trajectory shown earlier in Figure~\ref{fig:trajectory_measurement_rank}. Measurement methodology is presented earlier in Section~\ref{ch:measurement_setup}.   
 The channel rank simulation results of RT and the corresponding measurements are shown in Figure~\ref{fig:measurement_vs_sionna_rank}, where the LW1 tower is marked as a red triangle and the building in the target area is marked as a cyan polygon. As seen in the 5G NR RI measurements in Figure~\ref{fig:rank_zigzag_measurement}, the channel rank at most locations is 2, where one of the dual ports of each sector antenna supports spatial multiplexing. A channel rank of 3 has been observed at the locations on the northwest side of the LW1 tower, which can be interpreted as an additional spatial stream from a reflective path through the building near the tower. Moreover, a channel rank of 1 can also be observed on the south side of the target area implying a lack of angular separation over the given MIMO channel. 
 
 On the other hand, the RT simulation results are shown in Figure \ref{fig:rank_zigzag_RT}. Channel rank in RT simulation tends to be more variant compared to the measurements. A channel rank of 1 or 2 is observed in most UAV locations on the northwest side of the LW1 tower due to the orientation of the LW1 tower and the receiver of the UAV. On the other hand, a channel rank of 3 can be observed on the edge of each waypoint because of the reflections. It is worthwhile to note that it is inevitable to have a mismatch between RT and real-world measurements at specific locations due to the following reasons.
 \begin{itemize}
     \item \textbf{Angular mismatch.} For the LW1 tower antenna configuration, the dual port AW3232 sector antenna supports $+/- 45^\circ$ slant linear polarization. Moreover, the receiver antenna module mounted at the UAV also has a dual slant orientation structure. However, the element-wise rotation of the linear array configuration is not supported in the Sionna RT tool.
     \item \textbf{Dynamic vs. static environments.} There are environmental gaps between measurement and RT scenarios such as foliage patterns, temporal fadings, antenna behavior in real scenarios, and among others.
     \item \textbf{Rank calculation.} A threshold-based channel rank calculation is used for the RT scenarios. However, in the 5G NR RI measurements, a mobile device mounted on the UAV determines the RI to report to the LW1 tower. It is known that RI calculation is implement-dependent, which leads to the differences.
     % \item \textbf{Beamforming.} 
 \end{itemize}

The distribution of channel rank of RT with different $K$ can be a more meaningful comparison between RT simulations and measurements in the same environment. The histograms of the channel rank with RT and measurements are shown in Figure~\ref{fig:distribution_rank_RT_measurement}. The likelihood of having a channel rank of 2 or 3 is increased as $K$  increases from $K=K_1$ to $K=K_3$, which results in a higher similarity with the measurements. The mismatch between measurements and RT occurs mainly when the channel rank is 1. This suggests that, at most UAV locations, the second singular value for RT simulations does not fall within the range required for $K_3$ ($30$~dB) from the strongest singular value. Even though there are gaps in the channel rank between RT and measurement, RT can be a reasonable benchmark as a baseline in the controlled environment.
 %by analyzing the geographical impact on UAV links, which can be seen in Section \ref{ch:numerical_results}. 

 \begin{figure*}
     \centering       
      \subfigure[5G NR RI measurement]{\includegraphics[trim={0.2cm 0.1cm 0.9cm 0.65cm},clip,width=0.64\columnwidth]{images/channel_rank_zigzag_measurements_v4}
     \label{fig:rank_zigzag_measurement}
     }     
     \subfigure[RT simulation with $K_3$]{\includegraphics[trim={0.2cm 0.1cm 0.9cm 0.65cm},clip,width=0.64\columnwidth]{images/channel_rank_zigzag_RT_v4}
     \label{fig:rank_zigzag_RT}
     }
     \subfigure[Channel rank distribution with different $K$ of RT and measurements]{\includegraphics[trim={0.2cm 0.15cm 1.1cm 0.7cm},clip,width=0.6\columnwidth]{images/distribution_rank_RT_measurement_v2}
     \label{fig:distribution_rank_RT_measurement}
     }     
     \caption{5G NR RI measurement and RT simulation results with the predefined trajectory in Figure \ref{fig:trajectory_altitude_measurement_rank}.}
     \label{fig:measurement_vs_sionna_rank}
 \end{figure*}


 \section{Conclusions}
In this paper, we investigated RF coverage and channel rank in rural environments using the NVIDIA Sionna RT tool, extending our previous work \cite{previous_work} with a broader analysis of the Lake Wheeler Field Labs. Our study incorporated realistic foliage modeling, multiple base stations, and the Kriging interpolation-based 3D channel rank prediction scheme. We applied a constant threshold ratio for singular value selection and analyzed its distribution across different UAV altitudes and threshold settings. The observed spatial correlation of channel rank motivated the application of Kriging interpolation, where we derived the semi-variogram using the correlation-distance relationship at UAV positions. The proposed Kriging interpolation scheme was evaluated using the MAE metric and compared with two baseline interpolation methods. Results demonstrated that the Kriging-based approach outperformed baseline interpolation techniques by leveraging spatial correlation. Finally, we compared our RT-based RSS and channel rank simulation results with the real-world measurements collected from the NSF AERPAW testbed. The reasonable consistency between the RT results and the measurements was demonstrated in the foliage-dense rural scenarios.

For future work, unique challenges can be considered in UAV-based 6G and V2X communications including managing time-varying channel characteristics, ensuring ultra-reliable low-latency communication (URLLC), and optimizing spatial multiplexing under dynamic network conditions. Addressing the time-varying nature of signal sources, obstacles, and UAVs in dynamic environments requires advanced RT techniques to capture the rapid fluctuation of channel conditions. It enables analyzing adaptive spatial multiplexing strategies that leverage real-time channel conditions, geographical information, and mobility of UAVs. Moreover, improving the Kriging interpolation framework by incorporating data-driven and ML-aided techniques is another potential direction. By integrating ML with statistics of spatial correlation into Kriging interpolation techniques, it is expected that accuracy and computational complexity can be improved over various environment settings.


\bibliographystyle{IEEEtran}
%%
%% This is file `sample-sigconf.tex',
%% generated with the docstrip utility.
%%
%% The original source files were:
%%
%% samples.dtx  (with options: `all,proceedings,bibtex,sigconf')
%% 
%% IMPORTANT NOTICE:
%% 
%% For the copyright see the source file.
%% 
%% Any modified versions of this file must be renamed
%% with new filenames distinct from sample-sigconf.tex.
%% 
%% For distribution of the original source see the terms
%% for copying and modification in the file samples.dtx.
%% 
%% This generated file may be distributed as long as the
%% original source files, as listed above, are part of the
%% same distribution. (The sources need not necessarily be
%% in the same archive or directory.)
%%
%%
%% Commands for TeXCount
%TC:macro \cite [option:text,text]
%TC:macro \citep [option:text,text]
%TC:macro \citet [option:text,text]
%TC:envir table 0 1
%TC:envir table* 0 1
%TC:envir tabular [ignore] word
%TC:envir displaymath 0 word
%TC:envir math 0 word
%TC:envir comment 0 0
%%
%% The first command in your LaTeX source must be the \documentclass
%% command.
%%
%% For submission and review of your manuscript please change the
%% command to \documentclass[manuscript, screen, review]{acmart}.
%%
%% When submitting camera ready or to TAPS, please change the command
%% to \documentclass[sigconf]{acmart} or whichever template is required
%% for your publication.
%%
%%
\documentclass[sigconf]{acmart}
\usepackage{titlesec}
%%
%% \BibTeX command to typeset BibTeX logo in the docs
\AtBeginDocument{%
  \providecommand\BibTeX{{%
    Bib\TeX}}}

%% Rights management information.  This information is sent to you
%% when you complete the rights form.  These commands have SAMPLE
%% values in them; it is your responsibility as an author to replace
%% the commands and values with those provided to you when you
%% complete the rights form.
% \setcopyright{acmlicensed}
% \copyrightyear{2025}
% \acmYear{2025}
% \acmDOI{XXXXXXX.XXXXXXX}
\setcopyright{none}
%% These commands are for a PROCEEDINGS abstract or paper.

% \acmConference[CHI'25 Workshop: Tools for Thought]{2025}{Yokohama, Japan}
\acmConference[CHI'25 Workshop on Tools for Thought]{Tools for Thought: Research and Design for Understanding, Protecting, and Augmenting Human Cognition with Generative AI on CHI 2025 Workshop}{April 26,2025}{Yokohama, JAPAN}
  
\settopmatter{printacmref=false}
\renewcommand\footnotetextcopyrightpermission[1]{}



%%
%%  Uncomment \acmBooktitle if the title of the proceedings is different
%%  from ``Proceedings of ...''!
%%
%%\acmBooktitle{Woodstock '18: ACM Symposium on Neural Gaze Detection,
%%  June 03--05, 2018, Woodstock, NY}
\acmISBN{978-1-4503-XXXX-X/2018/06}


%%
%% Submission ID.
%% Use this when submitting an article to a sponsored event. You'll
%% receive a unique submission ID from the organizers
%% of the event, and this ID should be used as the parameter to this command.
%%\acmSubmissionID{123-A56-BU3}

%%
%% For managing citations, it is recommended to use bibliography
%% files in BibTeX format.
%%
%% You can then either use BibTeX with the ACM-Reference-Format style,
%% or BibLaTeX with the acmnumeric or acmauthoryear sytles, that include
%% support for advanced citation of software artefact from the
%% biblatex-software package, also separately available on CTAN.
%%
%% Look at the sample-*-biblatex.tex files for templates showcasing
%% the biblatex styles.
%%

%%
%% The majority of ACM publications use numbered citations and
%% references.  The command \citestyle{authoryear} switches to the
%% "author year" style.
%%
%% If you are preparing content for an event
%% sponsored by ACM SIGGRAPH, you must use the "author year" style of
%% citations and references.
%% Uncommenting
%% the next command will enable that style.
%%\citestyle{acmauthoryear}

\usepackage{booktabs}
\usepackage{multirow}
% \usepackage[table]{xcolor}
\usepackage{array}
\usepackage{geometry}

%%
%% end of the preamble, start of the body of the document source.
\begin{document}

%%
%% The "title" command has an optional parameter,
%% allowing the author to define a "short title" to be used in page headers.
\title[Beyond Tools]{Beyond Tools: Understanding How Heavy Users Integrate LLMs into Everyday Tasks and Decision-Making}

%%
%% The "author" command and its associated commands are used to define
%% the authors and their affiliations.
%% Of note is the shared affiliation of the first two authors, and the
%% "authornote" and "authornotemark" commands
%% used to denote shared contribution to the research.
\author{Eunhye Kim}
\email{gracekim027@snu.ac.kr}
\affiliation{%
  \institution{Seoul National University}
  \city{Seoul}
  \country{Republic of Korea}
}

\author{Kiroong Choe}
\email{krchoe@hcil.snu.ac.kr}
\affiliation{%
  \institution{Seoul National University}
  \city{Seoul}
  \country{Republic of Korea}
}

\author{Minju Yoo}
\email{minjuu613@ewhain.net}
\affiliation{%
  \institution{Ewha Womans University}
  \city{Seoul}
  \country{Republic of Korea}
}

\author{Sadat Shams Chowdhury}
\email{sadatshams@kaist.ac.kr}
\affiliation{%
  \institution{School of Computing, KAIST}
  \city{Daejeon}
  \country{Republic of Korea}
}

\author{Jinwook Seo}
\email{jseo@hcil.snu.ac.kr}
\affiliation{%
  \institution{Seoul National University}
  \city{Seoul}
  \country{Republic of Korea}
}

%%
%% By default, the full list of authors will be used in the page
%% headers. Often, this list is too long, and will overlap
%% other information printed in the page headers. This command allows
%% the author to define a more concise list
%% of authors' names for this purpose.
\renewcommand{\shortauthors}{Kim et al.}

%%
%% The abstract is a short summary of the work to be presented in the
%% article.
\begin{abstract}
  \begin{abstract}

We introduce \ours, a novel framework for scene-level appearance transfer from a single style image to a real-world scene represented by multiple views. The method combines explicit semantic correspondences with multi-view consistency to achieve precise and coherent stylization.
Unlike conventional stylization methods that apply a reference style globally, \ours uses open-vocabulary segmentation to establish dense, instance-level correspondences between the style and real-world images. This ensures that each object is stylized with semantically matched textures.
\ours first transfers the style to a single view using a training-free semantic-attention mechanism in a diffusion model.
It then lifts the stylization to additional views via a learned warp-and-refine network guided by monocular depth and pixel-wise correspondences.
Experiments show that \ours consistently outperforms prior methods in structure preservation, perceptual style similarity, and multi-view coherence.
User studies further validate its ability to produce photo-realistic, semantically faithful results.
Our code, pretrained models, and dataset will be publicly released, to support new applications in interior design, virtual staging, and 3D-consistent stylization.

\end{abstract}

\end{abstract}

%%
%% The code below is generated by the tool at http://dl.acm.org/ccs.cfm.
%% Please copy and paste the code instead of the example below.
%%
\begin{CCSXML}
<ccs2012>
   <concept>
       <concept_id>10003120.10003121.10011748</concept_id>
       <concept_desc>Human-centered computing~Empirical studies in HCI</concept_desc>
       <concept_significance>500</concept_significance>
       </concept>
 </ccs2012>
\end{CCSXML}

\ccsdesc[500]{Human-centered computing~Empirical studies in HCI}

%%
%% Keywords. The author(s) should pick words that accurately describe
%% the work being presented. Separate the keywords with commas.
\keywords{Decision-Making, AI Delegation, Qualitative Study}
%% A "teaser" image appears between the author and affiliation
%% information and the body of the document, and typically spans the
%% page.
% \received{20 February 2007}
% \received[revised]{12 March 2009}
% \received[accepted]{5 June 2009}

%%
%% This command processes the author and affiliation and title
%% information and builds the first part of the formatted document.
\maketitle
\section{Introduction}

\begin{figure*}[t!]
    \centering
    \includegraphics[width=0.7\textwidth]{./Comparison.pdf}
    \caption{Comparison between conventional wireless system (left) and PASS (right).}
    \label{comparison}
    \vspace{-0.5cm}
\end{figure*} 

\section{Introduction} \label{sec:intro}

\IEEEPARstart{S}INCE Marconi demonstrated the feasibility of wireless communication in the late 19th century, the technology has undergone significant evolution and remarkable transformations. To address the unpredictable and dynamic nature of wireless channels, numerous advancements have been made in the air interface design, channel coding, source compression, and communication protocols for improving data rates and enhancing reliability. Among these advancements, multiple-input multiple-output (MIMO) has been one of the most important evolutionary techniques for wireless communication over the past few decades. By exploiting antenna arrays, MIMO brings about multiple benefits, such as enhanced signal strength through beamforming, mitigation of multi-path fading, and efficient spatial-domain multiplexing of users~\cite{bjornson2023twenty}. Since the advent of the third generation (3G) system, MIMO has been a fundamental component of wireless communication standards. However, during that era, the size of antenna arrays in MIMO systems was generally limited. The breakthrough came when Marzetta demonstrated the significant benefits of deploying an infinite number of antennas in 2010~\cite{marzetta2010noncooperative}, revealing the potential of MIMO to enhance communication performance while reducing system complexity. This revelation paved the way for the concept of massive MIMO, i.e., employing large-scale antenna arrays at base stations. Over time, massive MIMO has evolved into a key research focus and has become a reality with the deployment of 5G networks. 


However, massive MIMO has faced numerous challenges, as it is expected to transition from “Massive” in 5G (typically with 32-64 antennas) to “Gigantic” in 6G~\cite{Xtext, bjornson2024enabling}, where the number of antennas is expected to scale to hundreds or even thousands. One of the key obstacles is the complexity and cost of implementing massive MIMO since each antenna typically needs to be fed by a dedicated radio-frequency (RF) chain. Exploiting low-resolution analog-to-digital converters in RF chains or hybrid analog-digital antenna arrays with a limited number of RF chains were common methods to address this challenge, especially in the millimeter-wave band~\cite{heath2016overview}. More recently, advancements in metamaterials have paved the way for new antenna technologies, exemplified by waveguide-fed metasurface antennas~\cite{smith2017analysis, shlezinger2021dynamic, di2024reconfigurable}, which facilitate the ultra-dense deployment of antenna elements at a significantly lower cost and making massive MIMO implementation more feasible.

Flexible-antenna technique is a new evolution of MIMO. Unlike massive MIMO focusing on enlarging the wireless channel dimension, the flexible-antenna technique focuses on enabling the reconfiguration of the wireless channel. One of the most well-known approaches in this domain is the reconfigurable intelligent surface (RIS) technique~\cite{huang2019reconfigurable, wu2019intelligent, mu2021simultaneously}. By deploying RIS between transceivers, the wireless channel can be intelligently reconfigured by adjusting the phase shifts of the signals reflected/refracted by the RIS. More recently, fluid antennas~\cite{new2024tutorial} and movable antennas~\cite{zhu2023movable} have emerged as promising flexible-antenna technologies. The fundamental concept behind these approaches is to implement antenna arrays where individual antenna elements can dynamically adjust their positions within a spatial region, thus creating favorable channel conditions to enhance communication performance. 

Nevertheless, as shown on the left of Fig. \ref{comparison}, both massive MIMO and flexible-antenna techniques have limited capability in fundamentally addressing free-space pathloss and line-of-sight (LoS) blockage, two major causes of signal attenuation in wireless communications. While massive MIMO can achieve high beamforming gains to strengthen signals, it cannot combat LoS blockage and to effectively mitigate free-space pathloss, particularly for cell-edge users. RISs have been considered as a promising solution to overcome LoS blockage by creating virtual LoS paths. However, the double fading effect caused by signal reflection results in much higher pathloss compared to a direct LoS channel~\cite{ozdogan2019intelligent}. Additionally, fluid and movable antennas are typically capable of adjusting their positions only within a few wavelengths, making them more effective for mitigating small-scale fading rather than addressing large-scale pathloss. It is worthy to point out that all the aforementioned MIMO systems are lack of antenna array reconfigurability, i.e., once an antenna array is built, adding or removing antennas is no longer possible.

Pinching-Antenna SyStem (PASS) is a revolutionary technique for addressing the challenges of free-space pathloss and LoS blockage encountered by conventional multi-antenna technologies. This technique was originally proposed and prototyped by NTT DOCOMO in 2022~\cite{suzuki2022pinching}. As illustrated on the right of Fig. \ref{comparison}, PASS employs a dielectric waveguide as its primary transmission medium, which is known for its exceptionally low propagation loss (e.g., 0.01 dB/m \cite{pozar2021microwave}). By pinching a small separated dielectric element, referred to as a \emph{pinching antenna}, onto the waveguide, the system enables signal emission from the waveguide into the pinching antenna, which then radiates the signal into free space. Building on this principle, waveguides can be pre-deployed to extend service coverage, allowing pinching antennas to be placed at positions close to users. This strategic placement transforms the wireless system into a \emph{near-wired} system and hence establishes strong LoS links with users, effectively minimizing free-space path loss and mitigating blockage issues. Additionally, unlike existing MIMO systems, PASS allows both the number and positions of pinching antennas to be easily adjusted by simply pinching them to or releasing them from the waveguide~\cite{suzuki2022pinching}. This feature provides a low-cost and scalable approach to implementing MIMO while also facilitating the so-called \emph{pinching beamforming}, which enhances communication performance by dynamically optimizing antenna positions \cite{liu2025pinching}.

Given the successful prototyping of PASS by NTT DOCOMO, theoretical research on this topic has been steadily growing, though it remains in its early stages. The first theoretical study on PASS for the communication system design was presented in \cite{ding2024flexible}, where the authors provided a comprehensive analysis and developed low-complexity pinching beamforming designs for fundamental single-user and two-user scenarios. The array gain achieved by multiple pinching antennas on a waveguide was analyzed in \cite{ouyang2025array}, unveiling the optimal number of antennas and their spacing for maximizing the beamforming gain. 
% The authors of \cite{tegos2024minimum} studied an uplink PASS system and proposed an iterative antenna position optimization algorithm to maximize the sum rate under perfect phase alignment conditions. In \cite{wang2024antenna}, the authors investigated a downlink PASS system and introduced a matching theory-based optimization method for activating pinching antennas at preconfigured discrete positions. Their findings also highlighted the advantages of using non-orthogonal multiple access (NOMA) in PASS. Expanding on this,
The authors of \cite{bereyhi2025downlink} explored a downlink PASS architecture utilizing multiple waveguides, each equipped with a single pinching antenna, and proposed a greedy approach for jointly optimizing the transmit and pinching beamforming. Meanwhile, \cite{guo2025deep} examined a more generalized scenario, where multiple pinching antennas were deployed on each waveguide, and introduced a graph neural network (GNN)-based deep learning method to address the corresponding joint beamforming optimization problem.

Although PASS has attracted growing attention, several key challenges remain unsolved. On the one hand, the physics modeling of PASS is still underdeveloped, which is crucial for establishing an accurate signal model. In existing studies \cite{ding2024flexible, ouyang2025array, bereyhi2025downlink, guo2025deep}, it is commonly assumed that all signal power within the waveguide is fully radiated into free space and that each pinching antenna on a waveguide emits identical radiation power—an assumption analogous to conventional MIMO systems. However, pinching antennas operate fundamentally differently from traditional electronic antennas, and such assumptions may lack a solid physical foundation and fail to accurately reflect real-world behaviors. On the other hand, most existing works design PASS under simplified assumptions \cite{ding2024flexible, ouyang2025array, bereyhi2025downlink}, such as a single user, a single waveguide, a single pinching antenna per waveguide, or perfectly aligned signal phases. Although the GNN-based deep learning model proposed in \cite{guo2025deep} is capable of handling more complex scenarios with arbitrary numbers of users, waveguides, and pinching antennas, it suffers from a key limitation: the model parameters need to be retrained once the system configuration changes, limiting its generalization ability. Motivated by these challenges, this paper aims to develop a fundamental physics-based signal model for PASS and explore joint beamforming designs for more general scenarios. The key contributions of this work are summarized as follows:
\begin{itemize}
    \item We propose a physics-based hardware model for PASS, in which a pinching antenna is modeled as an open-ended directional waveguide coupler to facilitate the adjustment of radiation characteristics and simplify signal modeling. Based on this model, we characterize the relationship between the electromagnetic (EM) fields within the waveguide and those radiated by the pinching antennas using coupled-mode theory.
    \item We derive a novel signal model for PASS based on the proposed physics framework, revealing the inherent coupling effect between the radiation power of pinching antennas deployed on the same waveguide. Leveraging this coupling relationship, we introduce two simplified power models and their respective implementation methods: equal power and proportional power models.
    \item We formulate a joint transmit and pinching beamforming optimization problem to minimize the transmit power in a general PASS system with arbitrary numbers of users, waveguides, and pinching antennas, considering both continuous and discrete activation of pinching antennas. To solve this highly nonconvex, coupled, and multimodal optimization problem, we propose two algorithms: the penalty-based alternating optimization algorithm and the zero-forcing (ZF)-based low-complexity algorithm.
    \item We provide comprehensive numerical results to validate the advantages of PASS and the effectiveness of the proposed algorithm. The results demonstrate that 1) the ZF-based algorithm delivers performance comparable to the penalty-based algorithm but has a low complexity, 2) PASS significantly reduces transmit power, achieving a reduction of over 95\% compared to conventional and massive MIMO, 3) a dense set of available antenna positions is required for discrete activation to achieve similar performance to continuous activation, and 4) the proportional power model exhibits performance comparable to the equal power model.
\end{itemize}

The rest of this paper is structured as follows. Section \ref{sec:model} introduces the proposed physics-based hardware model and signal model for PASS. Section \ref{sec:beamforing} presents the general system model for downlink PASS and introduces a penalty-based alternating optimization method and a ZF-based algorithm for solving the joint beamforming optimization problem. Numerical evaluations and performance comparisons under various system configurations are presented in Section \ref{sec:results}. Finally, Section \ref{sec:conclusion} summarizes the findings and concludes the paper.


\emph{Notations:} Scalars are denoted using regular typeface, vectors and matrices are represented in boldface, and Euclidean subspaces are indicated with calligraphic letters. The set of complex and real numbers are denoted by $\mathbb{C}$ and $\mathbb{R}$, respectively. The inverse, conjugate, transpose, conjugate transpose, and trace operators are denoted by $(\cdot)^{-1}$, $(\cdot)^*$, $(\cdot)^T$, $(\cdot)^H$, and $\mathrm{tr}(\cdot)$, respectively. The absolute value, Euclidean norm, Frobenius norm, and maximum norm are denoted by $|\cdot|$, $\|\cdot\|$, $\|\cdot\|_F$, and $\|\cdot\|_\infty$ respectively. The real part of a complex number of demoted by $\Re \{\cdot\}$. The entry in the $n$-th row and $m$-th column of a matrix $\mathbf{X}$ is denoted by $[\mathbf{X}]_{n,m}$. An identity matrix of dimension $N \times N$ is denoted by $\mathbf{I}_N$. The big-O notation is given by $O(\cdot)$. A diagonal matrix with diagonal entries $x_1,\dots,x_N$ is denoted as $\mathrm{diag}(x_1,\dots,x_N)$.    




% \begin{figure*}[t!]
% \centering
% \begin{subfigure}[t]{0.48\textwidth}
%     \centering
%     \includegraphics[height=0.5\textwidth]{./Comparison_conventional.pdf}
% \end{subfigure}
% \hspace{-1.5cm}
% \begin{subfigure}[t]{0.48\textwidth}
%     \centering
%     \includegraphics[height=0.5\textwidth]{./Comparison_PASSpdf.pdf}
% \end{subfigure}
% \caption{Comparison between conventional wireless system (left) and PASS (right).}
% \end{figure*} 


\section{Methods}
\section{Methods}
\label{sec:methods}

We conducted interviews with mental health clinicians to explore how they would design health information technologies (HITs) that support value-based mental healthcare.
Methodologically, we were inspired by work in speculative design to imagine futures where VBC is mandated, and then brainstorm with participants how HITs could support VBC outcomes data storage, collection, and use \cite{hockenhull_speculative_2021, wong_speculative_2018}. 
In this section, we detail the study procedures, including participant recruitment (Section \ref{sec:methods:participants}), background information (Section \ref{sec:methods:participants-backgrounds}), how data was collected and analyzed (Section \ref{sec:methods:data}), and our positionality (Section \ref{sec:methods:positionality}). 
All study procedures were approved by the coauthors' institutional review board (IRB). 

\subsection{Participant Recruitment}
\label{sec:methods:participants}
We enrolled as participants mental health clinicians, specifically practicing psychiatrists, clinical psychologists, licensed clinical social workers (LCSWs), and licensed mental health counselors (LMHCs).
We intentionally recruited providers from these different clinical orientations to gather different perspectives on designing HITs \cite{mental_health_america_types_2024}. 
Participants were recruited via a combination of convenience, purposive, and snowball sampling \cite{etikan_comparison_2015, goodman_snowball_1961}.
Specifically, a recruitment email and flier were sent to staff working at academic medical centers across the United States. 
\rev{Recruitment emails were often forwarded to providers who worked in smaller, private practices or community health settings, to help us gain perspectives from mental health clinicians working in diverse settings, treating different types of patients.} 
Within the qualitative tradition \cite{braun_one_2021}, our goal for this work was not to gather perspectives representative of mental health clinicians as a whole, but instead to deep dive with our participants into the complexities of designing HITs that support VBC.

\subsection{Participants' Backgrounds}
\label{sec:methods:participants-backgrounds}

\rev{Table \ref{tab:participants} summarizes background information for the 30 mental health clinicians who participated in the study.
This background information was collected during an intake survey, which was administered after participants provided informed consent for our study.
Apart from data collected within this intake survey, we often asked participants during our study interviews to provide background information regarding their current payment arrangements.
Most of our participants took traditional, fee-for-service payments (public and private), or asked their private practice patients to pay for care out-of-pocket.
A few participants (eg, SW28) worked in health systems transitioning to value-based payments.
Many participants were unfamiliar with VBC.
}

\begin{table*}[t]
\begin{tabular}{ll}
\toprule
Number of participants & 30 mental health clinicians \\ 
\midrule
Clinical training           & 13 Clinical Psychology \rev{(CP)} \\
                            & 6 Psychiatry \rev{(PS)} \\    
                            & 8 Clinical Social Work \rev{(SW)} \\
                            & 2 Mental Health Counseling \rev{(MC)} \\
                            & 1 Family and Marriage Therapist \rev{(FT)} \\
\midrule
Practice setting            & 16 Academic Medical Center \\
                            & 14 Private Practice \\
                            & 5 Community Mental Health Center \\
                            & 2 Employee Assistance Program \\
\midrule
Geographic location (in the USA)    & 26 Northeast \\
                                    & 2 Southeast \\
                                    & 2 West Coast \\
\bottomrule
\end{tabular}
\caption{Background information of the study participants. Participants could list multiple practice settings.
\rev{Clinical training abbreviations (eg, ``CP'') are used within Section \ref{sec:findings}.}
}
% \Description{A table summarizing the backgrounds of the 30 participants we interviewed in this study. The table describes the clinical training of participants, the practice setting, and geographic location (in the United States) of each participant.}
\label{tab:participants}
\vspace{-5pt}
\end{table*}

\subsection{Data Collection and Analysis}
\label{sec:methods:data}

All participants were asked to provide informed consent after being provided complete information about the study procedures.
Interviews were held via Zoom over two 1-hour sessions attended by the first three authors, and participants were reimbursed \$30 per hour for their time.
The first session was a semi-structured interview where we asked clinicians about their current care practices, specifically how they used data -- defined broadly, collected with or without technology -- in care.
We specifically asked participants about their perspectives on \textit{measurement-based care} (MBC), the practice of collecting and using data in care that would power HITs supporting VBC \cite{kilbourne_measuring_2018}.
We then asked participants further questions about how they used this data to measure care outcomes, how technology was involved in this process, and whether providers were accountable to achieve certain care outcomes.
Interview questions were broad to allow for on-the-spot adaptation and probing \cite{barriball_collecting_1994}.

In the second session, participants completed two design prompts.
These prompts were motivated by work in speculative design \cite{hockenhull_speculative_2021, wong_influence_2008}, to imagine futures where MBC and VBC were mandated and to understand how clinicians would collect and report outcomes data as a part of these programs.
The first prompt asked participating clinicians to imagine a world where they were mandated to use outcomes data as a part of care, and to brainstorm what data they would prioritize.
The second prompt was motivated by the five-star quality rating system used by the United States Center for Medicare \& Medicaid services (CMS) \cite{center_for_medicare__medicaid_services_five-star_2022}.
Participating clinicians were asked to imagine that as a part of VBC, CMS wanted to design ``mental health quality star ratings'' to measure patient outcomes and care quality across clinics and health systems.
Participants were asked to brainstorm what data should be included in this new star rating program.
After responding to each prompt, we discussed with participants the data they included in their responses, and asked probing questions to further understand how HITs could support data storage, collection, and use.
Full interview guides can be found in Appendix \ref{appendix:guide}.

Interviews were recorded with participants' permission, transcribed by a professional service, and de-identified.
Transcripts were analyzed using a reflexive thematic analysis approach adopted from \cite{braun_using_2006}.
This approach combined both inductive and deductive elements.
Codes and themes arose from the data, but were guided by our research interests and the literature \cite{braun_one_2021}, specifically the stages of preparation, collection, and action from Li et al. \cite{li_stage-based_2010}.
The first author qualitatively coded all transcripts.
Codes were iteratively refined, resulting in a final codebook, and all transcripts were recoded using the final codebook.
Themes were developed from the codes by the first author, with support from the second and third authors who also participated in the interviews and validated that the themes represented participants' views.
The codebook used to generate each theme can be found in Appendix \ref{appendix:codebook}.

\subsection{Positionality}
\label{sec:methods:positionality}

The first, second, and third authors are graduate students in computer and information science. 
These authors recruited participants, collected, and analyzed all of the data. 
One author is a clinical researcher and practicing mental health clinician who worked with the first author on the study protocols, and did not participate in the study. 
Another author is a health policy researcher, who is an expert on both digital mental health and value-based care.
The final author is a researcher in computing and information science. 
All authors were based in the United States, and thus our findings and perspectives are greatly informed, and potentially limited by, our knowledge of the United States healthcare system.
\section{Results}
% \section{Simulation Evaluation \& Results}\label{sec:results}

\subsection{Baseline Planners}

To evaluate the performance of \PlannerName, we compare it against several baseline methods. In the following section, we describe these baselines, their implementation details, and their respective advantages and limitations, particularly in the context of information gathering in large, high-dimensional search spaces. The simulation framework and vehicle parameters remain consistent across all planners, and each method is allowed to replan during testing.

\subsubsection{Monte-Carlo Tree Search}

Monte Carlo Tree Search (MCTS) can be a powerful technique for finding feasible and optimal paths in complex environments. It is a heuristic search algorithm that builds a search tree incrementally through repeated simulations. At each iteration, it selects a node to explore based on a selection policy (often the Upper Confidence Bound or UCB1 algorithm), expands the tree by adding possible actions from that node, runs a simulation from the newly added node, and updates the statistics of nodes along the path traversed during the simulation. 

The UCB1 (Upper Confidence Bound) algorithm is a technique commonly used in the context of multi-armed bandit problems and Monte Carlo Tree Search (MCTS) for balancing exploration and exploitation. It helps in selecting actions or nodes that are likely to yield high rewards while also exploring less-frequented options to gather more information about their potential rewards. 

We formulate our UCB score in the following manner, \\
\begin{equation*}
    UCB_\text{node} = \frac{I(X_{\text{node}})}{\alpha} + C \times \sqrt{\frac{\ln(N_\text{tree})}{N_\text{node}}}
\end{equation*}
%  $
% UCB_\text{node} = \frac{\overline{X_\text{node}}}{\alpha} + C \times \sqrt{\frac{\ln(N_\text{tree})}{N_\text{node}}}
% $ \\
Here $I(X_{\text{node}})$ denotes the estimated information gain from the node, $\alpha$ denotes the normalization factor which is given by $\frac{B}{v_\text{desired}}$, $B$ being the maximum planning budget and $v_\text{desired}$ being the desired speed of our UAV. $C$ denotes the exploration weight, and $N_\text{tree}$ denotes the number of visits to the tree root node while $N_\text{node}$ denotes the number of times the present node has been visited.

After selecting a candidate node, if it has been visited before, it is expanded by applying motion primitives to generate child nodes, growing the tree. Unvisited nodes skip this step. Following expansion, either the unvisited candidate node or one of its children is selected for the simulation phase, where the future values of nodes along the path are estimated to update the total potential information gain. This informs the selection policy in subsequent iterations. Once planning time is exhausted, the path with the highest information gain is returned.

% with authors goes here
\begin{figure}[t]
\centering
\includegraphics[trim={.7cm 0cm .5cm 1.4cm},clip,width=\columnwidth]{figs/5_/Results1v3.pdf}
\caption{The Monte Carlo simulation results for the planners. The plots show the average percent reduction in entropy over the course of the simulations, and the shading shows the 95\% confidence intervals. IA-TIGRIS outperforms all of the baselines.}
\label{fig:mc_results}
\end{figure}

While MCTS is probabilistically guaranteed to converge to the optimal path \cite{mcts_ref_1}, it is constrained to actions within a predefined set of motion primitives. Its reliance on random sampling to estimate the future value of nodes can result in poor approximations, particularly in environments with sparse, localized pockets of high information gain. This limitation is especially pronounced in large search areas or scenarios with large budgets constraints, where estimating future node values becomes increasingly expensive. As a result, in such scenarios, MCTS is often implemented with a finite planning horizon, which can restrict its ability to account for long-term consequences or dependencies in the environment.

% This property of MCTS, which causes unguided exploration of the environment, leads to increased convergence times on the optimal path, as a result of a lot of budget being spent in exploring information sparse areas of the map. 
% Also, the computation time of MCTS increases exponentially with the depth of the search tree. The time complexity of MCTS is given by $\mathcal{O}(\frac{T}{t_\text{iter}} \cdot |A|^d)$. Here, $T$ is the total planning time and $t_\text{iter}$ is the time taken per iteration of the planning loop. $|A|$ is the number of actions and $d$ represents the average depth of the search tree. 

% The above limitations are not inconsequential in the context of performing informative path planning in large high-dimensional search spaces. We compare MCTS with \PlannerName, in \ref{}, and empirically demonstrate its drawbacks and how \PlannerName, is able to outperform MCTS in the context of the mission parameters we examine in this work.  

\subsubsection{Greedy}

For the greedy planner, we iterated through each cell within the search bounds and calculated the reward for a given cell $i$ as $g_i = R(X_i) / d_i$ where $R(X_i)$ is given through \eqref{equ:reward} and $d_i$ represents the Euclidean distance between the current position the robot at the current time $t$ and the closest viewpoint to the cell. To compute this viewpoint, the yaw between the current pose of the robot and the intersected cell is first calculated. Using the robot's sensor configuration and this yaw, $x$ and $y$ coordinates are calculated that view the cell at the desired flight altitude. With this formulation, the planner prioritizes regions with a high ratio of entropy to distance. This can lead to locally optimal choices that contradict with paths that lead to higher information gain over the entire trajectory. 

% without authors goes here
% \begin{figure}[t]
% \centering
% \includegraphics[trim={.7cm 0cm .5cm 1.4cm},clip,width=\columnwidth]{figs/5_/Results1v3.pdf}
% \caption{The Monte Carlo simulation results for the planners. The plots show the average percent reduction in entropy over the course of the simulations, and the shading shows the 95\% confidence intervals. IA-TIGRIS outperforms all of the baselines.}
% \label{fig:mc_results}
% \end{figure}


\begin{figure*}[t]
    \centering
    \begin{subfigure}[b]{0.99\textwidth}
        \centering
        \includegraphics[trim={0cm 0.3cm 0cm 0cm},clip,width=\textwidth]{figs/5_/Fig2v1_target.png}
        % \caption{Slice by targets}
        % \vspace{.1cm}
    \end{subfigure}
    
    \begin{subfigure}[b]{0.99\textwidth}
        \centering
        \includegraphics[trim={0cm 0cm 0cm 0cm},clip,width=\textwidth]{figs/5_/Fig2v1_sigma.png}
        % \caption{Slice by sigma }
    \end{subfigure}
    \caption{A comparison of the methods based on the number of sampled prior clusters and the standard deviation of sampled prior clusters. IA-TIGRIS is most effective compared to the baselines when there is high variation in the search space. As the search space prior information becomes more evenly spread out, the performance gap between the methods tends to decrease.}
    \label{fig:targets_sigmas}
\end{figure*}

\subsubsection{Random}

The random planner operates by iteratively sampling points within the defined search bounds and calculating the minimum-cost path to observe each sampled point. This process is repeated until the available budget is fully expended. The random planner does not utilize any prior information about the environment or target distribution. Additionally, it does not optimize the sequence of actions, instead treating each sampled point independently without considering the global structure of the search problem. This simplicity allows the random planner to highlight the performance benefits of more sophisticated methods by providing a lower-bound comparison for evaluation.

\subsubsection{Coverage}

The coverage planner generates a plan that systematically covers the entire search space using a straightforward lawn-mower pattern. The spacing between each pass is set to match the width of the projected observation footprint at 20\% from the bottom, ensuring that no grid cells are missed. This spacing also maintains a distance that enables high-quality sensor measurements. However, due to the size of the search spaces considered, the coverage planner spends significant time surveying empty regions. This approach results in inefficient use of the budget, as it prioritizes full coverage with safe sensor overlap, even in areas with little or no valuable information. While simple and robust, this method highlights the tradeoff between exhaustive coverage and efficient, targeted exploration.

% \subsubsection{Branch and Bound}
% The branch and bound baseline is based on motion primitive planning. In each future step the drone has a set of motion primitives with future states and each of these future states also has a set of motion primitives. In this way, a tree can be built with multiple path candidates. The path candidate with the highest information gain will be selected and form the output. 

% By adding branch and bound, there will be an estimation of a node's upper bound information reward, using the node's current information reward, updated information map and the remaining budget. If this upper bound is already lower than the information reward of any other node in the tree, the corresponding node will be closed and not expanded in the future to accelerate the expansion of the tree. 



\subsection{Tests and Analysis}
% To evaluate the efficacy of IA-TIGRIS compared to the baseline methods, we conduct Monte Carlo testing as well as analyze how the prior and budget affect the performance of each method. In all of these test cases, there are no time-based or priority rewards and have horizon lengths set to the full budget. All tests were performed using an Intel Xeon CPU E5-2620 v4 @ 2.10GHz.
To evaluate the efficacy of IA-TIGRIS against baseline methods, we perform Monte Carlo testing and analyze the impact of the prior and budget on the performance of each method. In all test cases, rewards are calculated using \eqref{equ:reward}, and horizon lengths are set to match the full budget. The tests are conducted on an Intel Xeon CPU E5-2620 v4 @ 2.10GHz, ensuring consistent computational conditions across all evaluations.

% Random sample across which parameters.

% Quantitative ideas. Look into number and std of prior (metric for this? std of grid cell values, mediuan, mean,). 
% Uniform prior? 
% Split distinct regions, not smooth. 
% Compare to coverage and amount of time to reach specific amount. 
% Compare with different budgets. 
% Repeatability test. 
% Graph size vs time. 
% Look at coverage with different altitudes or widths. Something that shows long horizon vs not nature of things?
% Shape of search space?
% Time/budget to get x\% of all info gain. Have to do moving horizon. 
% Targets detected? 

% Key thought for results where I show time, our optimization does not optimize for time, only final value. Key thing to show across the different budgets. 

% \BM{Qualitative. Nayana idea of plot with example sampled case. Should add one here.} 



\subsubsection{Monte Carlo Testing}
Our simulated testing environment is a $5000\times5000$ m square with Gaussian-distributed prior information randomly placed throughout the search space. The number of prior clusters was sampled uniformly between $[4,20]$, with standard deviations between $[60,450]$, and maximum value between $[0.05,0.5]$. 

The results of $100$ Monte Carlo tests are shown in Fig.~\ref{fig:mc_results}. IA-TIGRIS clearly outperforms the other methods, achieving nearly a $40\%$ greater reduction in entropy than the next best method. Early in the simulation, the greedy method initially gains information more quickly, as expected, but this does not translate to better long-term performance. Since our method optimizes for total information gain, it generates paths that maximize information collection over the entire budget. MCTS performed slightly worse than the greedy approach.

The random paths slightly outperformed the coverage paths. This is likely because the lawnmower strategy requires sufficient overlap between passes to avoid missing areas, and its long straight paths often lead to redundant observations due to the UAV’s forward-facing camera. Changing the heading of the UAV is beneficial to viewing more of the search space, which may explain why random paths performed better.

We also conducted Monte Carlo tests where either the number of prior clusters or their standard deviation was held constant to analyze how variations in the information map affect planner performance. The results, shown in Fig.~\ref{fig:targets_sigmas}, include two cases: the upper figure fixes the number of priors, while the lower figure fixes their standard deviation. All other agent and simulation parameters remained unchanged.


% The first thing to note from these results is that for all tests the proportional performance gap between IA-TIGRIS and the baselines increases as the number and standard deviation of the Gaussian priors decreases. As the search space becomes more uniformly filled with entropy in the information map, the need for longer-horizon planning decreases and other simple or random approaches can perform satisfactorily given the testing budget. As the information becomes more sparsely distribution in the space, such as when the information is contained in separated pockets of areas, there is a greater need to plan longer-horizon paths that reason about the given budget.
% \BM{Could have figures here or refer to others}

Across these tests, the performance gap between IA-TIGRIS and the baselines widens as the number and standard deviation of the Gaussian priors decrease. When entropy is more uniformly distributed across the search space, simpler methods perform reasonably well within the given budget. However, when information is concentrated in sparse, distinct regions, longer-horizon planning becomes essential. In such cases, IA-TIGRIS demonstrates a significant advantage by effectively reasoning about the budget and prioritizing high-value regions.

% Show plot of first plans expected info gain versus planning time. (plans not executed)


\subsubsection{Budget Analysis}
To evaluate the impact of budget constraints on performance, we conducted additional tests beyond our initial Monte Carlo experiments, evaluating budgets of $5000$ m, $10000$ m, $30000$ m, and $60000$ m. Table~\ref{tab:budgets} summarizes the average entropy reduction across these budgets.

\definecolor{tabfirst}{rgb}{1, 0.7, 0.7} % red
\definecolor{tabsecond}{rgb}{1, 0.85, 0.7} % orange
\definecolor{tabthird}{rgb}{1, 1, 0.7} % yellow
\begin{table}[t]
    \centering
    \resizebox{\linewidth}{!}{
    \begin{tabular}{l|ccccc}
    & $5000$ m & 10000 m  & 15000 m& 30000 m& 60000 m\\ \hline

    % \hline
    IA-TIGRIS  &  \cellcolor{tabfirst}$9.41\pm1.0$ &  \cellcolor{tabfirst}$18.28\pm1.8$ & \cellcolor{tabfirst}$25.36\pm2.3$ & \cellcolor{tabfirst}$41.08\pm2.9$ & \cellcolor{tabfirst}$58.85\pm2.9$ \\
    Greedy  &  \cellcolor{tabsecond}$6.99\pm0.8$ &  \cellcolor{tabsecond}$13.10\pm1.5$ & \cellcolor{tabsecond}$17.97\pm2.0$ & \cellcolor{tabthird}$30.00\pm2.3$ & \cellcolor{tabsecond}$49.38\pm3.5$ \\
    MCTS  &  \cellcolor{tabthird}$6.06\pm0.7$ &  \cellcolor{tabthird}$11.80\pm1.1$ & \cellcolor{tabthird}$17.11\pm1.4$ & \cellcolor{tabsecond}$30.21\pm2.2$ & \cellcolor{tabthird}$48.68\pm2.7$ \\
    Random  &  $2.19\pm0.3$ & $4.29\pm0.7$ & $6.61\pm0.6$ & $17.50\pm1.2$ & $22.47\pm1.4$ \\
    Coverage  &  $1.58\pm0.3$ &  $2.82\pm0.4$ & $4.09\pm0.7$ & $12.04\pm1.9$ & $16.77\pm2.4$ \\

    \end{tabular}
    }
    \caption{Monte Carlo testing results given different budgets. The values are the average percent reduction in entropy and the 95\% confidence bounds. \mbox{IA-TIGRIS} had the best performance for all budgets.}
    \label{tab:budgets}
\end{table}
%$\uparrow$ 

IA-TIGRIS consistently achieved the highest entropy reduction across all budget constraints, with a statistically significant margin over alternative methods. Greedy generally ranked second but was slightly outperformed by MCTS at the $30000$ m budget level. Greedy and MCTS exhibited comparable performance throughout the tests, with their results closely tracking each other. Consistent with our previous findings, Random and Coverage methods yielded the lowest results.


Among the tested methods, only IA-TIGRIS and MCTS explicitly incorporate budget constraints into their planning algorithms. Notably, at lower budgets ($5000$ m and $10000$ m), these methods achieved higher entropy reduction compared to the equivalent time steps ($200$ s and $400$ s) in the $15000$ m budget scenario shown in Fig.~\ref{fig:mc_results}. This improved performance stems from IA-TIGRIS's optimization of total path reward under budget constraints, contrasting with the myopic next-best-action approach of the greedy method. The remaining methods---Greedy, Random, and Coverage---maintain consistent behavior regardless of budget constraints, as their planning strategies do not account for resource limitations.


The performance gap between IA-TIGRIS and the next-best method varied with budget size, showing margins of $34.6\%$, $39.5\%$, $41.1\%$, $36.0\%$, and $19.2\%$ in ascending budget order. This gap widened through the first three budget levels as problem complexity increased, before declining significantly at higher budgets. This performance pattern suggests that implementing a planning horizon could enhance efficiency by limiting tree search depth, enabling the planner to prioritize path quality optimization over exhaustive space exploration.


% percent improved from next best
% 34.6, 39.5, 41.1, 36.0, 19.2
% reasons, too long horizon is a larger search space, so less quality paths closer. Or larger horizon, more packing in


% with authors goes here
\begin{figure}[t] 
    \centering
    \renewcommand\arraystretch{0} % Adjust the height between rows here
    \setlength{\tabcolsep}{1pt} % Adjust the column separation here
    \begin{tabular}{c}
        \begin{tikzpicture}
            \node[anchor=south west, inner sep=0] (image) at (0,0) {
                \includegraphics[width=0.9\linewidth]{figs/5_/google_earth_prior.png}
            };
            \begin{scope}[x={(image.south east)},y={(image.north west)}]
                % \fill[OrangeRed] (0.02, 0.03) circle (2pt); 
                % \fill[OrangeRed] (0.51, 0.04) circle (2pt); 
                % \fill[OrangeRed] (0.61, 0.04) arc (0:90:2pt); 
                \fill[Orange, opacity=0.8] (0.74, 0.45) circle (3pt); % Adjust 
                \fill[Orange, opacity=0.8] (0.27, 0.42) circle (3pt); % Adjust 
                \fill[Orange, opacity=0.8] (0.39, 0.63) circle (3pt); % Adjust 
            \end{scope}
        \end{tikzpicture} \\
        % \includegraphics[width=0.9\linewidth]{figs/5_/google_earth_prior.png} \\
        \\
        \includegraphics[width=0.9\linewidth]{figs/5_/google_earth_path.png} 
    \end{tabular}
    \caption{Google Earth screenshots illustrating the mission planning process and execution. Top: Areas of high entropy targeted for search are highlighted in red, representing regions with a binary occupied/unoccupied probability of 0.2. Three points of particular interest, each assigned a 0.5 probability, are marked in orange. Bottom: The executed drone flight path (yellow) shows the optimized path for maximum information gain across the search space.} 
    \label{fig:google_earth}
\end{figure}
\begin{figure}[t]
\centering
% https://docs.google.com/presentation/d/1RjI-QqHpBRLHN60UAxzmQYs4EaWaVCOoSBkEkA39kk0/edit?usp=sharing
\includegraphics[width=\columnwidth]{figs/5_/m600_labeled.jpg}
\caption{Hexarotor system (DJI M600 Pro) with onboard compute and camera. Left image shows drone on the ground, right image shows drone in flight.}
\label{fig:m600}
\end{figure}


\section{Field Deployments}\label{sec:field}


\subsection{Hexarotor Deployment}
The first field experiment that we present uses a hexarotor drone to cover an urban area shown in Fig.~\ref{fig:fig1}.
We designed this field experiment to simulate classifying where cars are within a search area.  
Hence, we set the plan request to focus on parking lots at the field test site (Fig.~\ref{fig:google_earth}, top), with the addition of three chosen grid cells within the parking lots being marked as having a higher uncertainty. The plan request boundaries and priors were created with GPS coordinates in Google Earth, exported as kml files, and then converted into our plan request message format. 

The following sections details the hardware, autonomy, and experimental results for our hexarotor deployments.

% without the authors goes here
% \begin{figure}[t] 
%     \centering
%     \renewcommand\arraystretch{0} % Adjust the height between rows here
%     \setlength{\tabcolsep}{1pt} % Adjust the column separation here
%     \begin{tabular}{c}
%         \begin{tikzpicture}
%             \node[anchor=south west, inner sep=0] (image) at (0,0) {
%                 \includegraphics[width=0.9\linewidth]{figs/5_/google_earth_prior.png}
%             };
%             \begin{scope}[x={(image.south east)},y={(image.north west)}]
%                 % \fill[OrangeRed] (0.02, 0.03) circle (2pt); 
%                 % \fill[OrangeRed] (0.51, 0.04) circle (2pt); 
%                 % \fill[OrangeRed] (0.61, 0.04) arc (0:90:2pt); 
%                 \fill[Orange, opacity=0.8] (0.74, 0.45) circle (3pt); % Adjust 
%                 \fill[Orange, opacity=0.8] (0.27, 0.42) circle (3pt); % Adjust 
%                 \fill[Orange, opacity=0.8] (0.39, 0.63) circle (3pt); % Adjust 
%             \end{scope}
%         \end{tikzpicture} \\
%         % \includegraphics[width=0.9\linewidth]{figs/5_/google_earth_prior.png} \\
%         \\
%         \includegraphics[width=0.9\linewidth]{figs/5_/google_earth_path.png} 
%     \end{tabular}
%     \caption{Google Earth screenshots illustrating the mission planning process and execution. Top: Areas of high entropy targeted for search are highlighted in red, representing regions with a binary occupied/unoccupied probability of 0.2. Three points of particular interest, each assigned a 0.5 probability, are marked in orange. Bottom: The executed drone flight path (yellow) shows the optimized path for maximum information gain across the search space.} 
%     \label{fig:google_earth}
% \end{figure}
% \begin{figure}[t]
% \centering
% % https://docs.google.com/presentation/d/1RjI-QqHpBRLHN60UAxzmQYs4EaWaVCOoSBkEkA39kk0/edit?usp=sharing
% \includegraphics[width=\columnwidth]{figs/5_/m600_labeled.jpg}
% \caption{Hexarotor system (DJI M600 Pro) with onboard compute and camera. Left image shows drone on the ground, right image shows drone in flight.}
% \label{fig:m600}
% \end{figure}

\subsubsection{Hardware System}
The hardware consists of the DJI M600 Pro, shown in Fig.~\ref{fig:m600}, along with the physical sensing and onboard computer payload. The DJI M600 Pro contains a flight controller that handles pose estimation and position-based control. The DJI M600 Pro’s flight controller also handles teleloperation if human intervention is necessary. Beneath the drone's base, we mount a custom hardware payload.
That payload consists of an onboard computer, a Jetson Xavier, to run the autonomy software shown in Fig.~\ref{fig:functional_diagram}.
The payload also contains a downward-facing a camera for sensing the environment. The camera is a Seek S304SP thermal camera.
The camera intrinsics are used to calculate the frustum's intersection with the search map's cells in IA-TIGRIS.

% without authors goes here
\begin{figure}[t]
\centering
% https://lucid.app/lucidchart/f750ddb4-2809-4773-8361-d5fbb1ba49eb/edit?viewport_loc=-257%2C-116%2C2219%2C1140%2C0_0&invitationId=inv_56e8a3a9-e8cf-4cad-a280-48bd967ff651
\includegraphics[trim={0cm 0cm 0cm 0cm},clip,width=\columnwidth]{figs/5_/functional_diagram.jpeg}
\caption{Functional diagram of the DJI M600 Pro autonomy software.}
\label{fig:functional_diagram}
\end{figure}
\begin{figure}[b]
    \centering
    \begin{subfigure}[b]{0.48\columnwidth}
        \centering
        \includegraphics[width=1.0\linewidth]{figs/5_/field_test_altitude_over_time.png}
        \caption{}
        \label{fig:m600_altitude_over_time}
    \end{subfigure}
    \begin{subfigure}[b]{0.48\columnwidth}
        \centering
        \includegraphics[width=1.0\linewidth]{figs/5_/field_test_entropy_over_time.png}
        \caption{}
        \label{fig:m600_entropy_over_time}
    \end{subfigure}
    \caption{The results for our hexarotor field deployment. (a) Plot of flown altitude over time, showing large variation throughout the experiment. (b) Reduction in entropy percentage over time of field experiment.}
\end{figure}

\subsubsection{Autonomy System}
Fig.~\ref{fig:functional_diagram} illustrates the functional system diagram for the real world field test on the DJI M600. The user specifies the initial plan request prior to takeoff. The TIGRIS planner makes an initial plan on that plan request and sends a global path to the waypoint manager. The waypoint manager tracks the current waypoint within the plan and sends the next waypoint to the DJI software development kit, which then sends actuation commands to the motors. The position of the drone is used to calculate the distance from the drone to the ground and sends that distance parameter to the sensor model. The sensor model's true positive and false positive rate is used to calculate the per-cell entropy updates in the search map manager. The search map manager publishes the current information map, and the replanning node sends an updated plan request to the IA-TIGRIS planner every ten seconds.

The drone started at an altitude of $50$ m above the origin of the reference frame. The informed sampler in IA-TIGRIS was set to add states at altitudes of either $30$ m or $60$ m, creating a trade-off between observation area and detector accuracy. The budget was $2000$ m, the planning horizon was $600$ m, and the planning time was $10$ seconds. 

% % without authors goes here
% \begin{figure}[t]
% \centering
% % https://lucid.app/lucidchart/f750ddb4-2809-4773-8361-d5fbb1ba49eb/edit?viewport_loc=-257%2C-116%2C2219%2C1140%2C0_0&invitationId=inv_56e8a3a9-e8cf-4cad-a280-48bd967ff651
% \includegraphics[trim={0cm 0cm 0cm 0cm},clip,width=\columnwidth]{figs/5_/functional_diagram.jpeg}
% \caption{Functional diagram of the DJI M600 Pro autonomy software.}
% \label{fig:functional_diagram}
% \end{figure}
% \begin{figure}[b]
%     \centering
%     \begin{subfigure}[b]{0.48\columnwidth}
%         \centering
%         \includegraphics[width=1.0\linewidth]{figs/5_/field_test_altitude_over_time.png}
%         \caption{}
%         \label{fig:m600_altitude_over_time}
%     \end{subfigure}
%     \begin{subfigure}[b]{0.48\columnwidth}
%         \centering
%         \includegraphics[width=1.0\linewidth]{figs/5_/field_test_entropy_over_time.png}
%         \caption{}
%         \label{fig:m600_entropy_over_time}
%     \end{subfigure}
%     \caption{The results for our hexarotor field deployment. (a) Plot of flown altitude over time, showing large variation throughout the experiment. (b) Reduction in entropy percentage over time of field experiment.}
% \end{figure}

\subsubsection{Experimental Results}


The bottom image of Fig.~\ref{fig:google_earth} shows the path selected by IA-TIGRIS in the search area. The figure highlights how the planner dynamically adjusts altitudes over time to balance coverage and sensing resolution, maximizing information gain. Higher altitudes allow for broader area coverage, while lower altitudes provide more detailed observations where needed. Additionally, the planner prioritizes revisiting the three regions of higher uncertainty, recognizing the need for repeated observations reduce entropy. This adaptive strategy ensures that uncertain areas receive sufficient attention to improve the belief map. As a result, the entropy of the information map decreases to near zero by the end of the mission, as shown in Fig.~\ref{fig:m600_entropy_over_time}, indicating that the planner has effectively gathered the necessary information. This behavior demonstrates the planner’s ability to optimize sensing actions, balancing altitude selection, revisit frequency, and exploration to maximize mission success.

\begin{figure}[t]
\centering
% \includegraphics[width=2.5in]{fig1}
\includegraphics[trim={4cm 4cm 0cm 4cm},clip,width=\columnwidth]{figs/5_/TL1.jpg}
\caption{Fixed-wing platform used for autonomous flights with an onboard camera pitched at 10 degrees\cite{alarewebsite}}
\label{fig:tl1}
\end{figure}






\subsection{Fixed-wing Deployments}

Our proposed approach was extensively tested on the fixed-wing AlareTech TL-1 UAV, shown in Fig.~\ref{fig:tl1}. The UAV is equipped with an onboard camera pitched at 10 degrees, which introduces a more challenging planning problem due to the non-holonomic motion model and the camera's field of view. Over more than 20 flight hours and 100 flights running IA-TIGRIS, we validated our approach with the objective to search for objects of interest in a large search space across a variety of test scenarios, including different terrain types, varying environmental conditions, and diverse target distributions. An example mission from these tests is shown in Fig.~\ref{fig:fwd}. In this scenario, the planner was given the search bounds and a designated high-priority region. The resulting flight path prioritized revisiting the high-priority area twice, optimizing sensor use and ensuring maximum information gain. This strategy led to the successful detection of the object of interest, with its estimated position marked by the red dot in the figure. 

The map on the upper right in Fig.~\ref{fig:fwd} shows the information map after plan execution was complete. Due to the UAV's limited budget, the upper right and lower left corners of the map are not searched by the agent. The budget is instead utilized to search over the area of higher priority two times. Compared to the paths in Fig.~\ref{fig:google_earth}, we observe that the paths for the fixed wing are smoother and have a larger turning radius, demonstrating how IA-TIGRIS respects the motion constraints of the vehicle. We can also see the effect of wind on the path execution, where the flown path shown in green deviates from the planned path shown in yellow. This illustrates the importance of online planning in the cases where this deviation is large or would accumulate over the course of a longer mission and cause the expected observed area to be much different than actual observed area. 

\begin{figure}[t]
\centering
% \includegraphics[width=2.5in]{fig1}
% [trim={left bottom right top},clip]
\includegraphics[trim={3.0cm, 1.0cm, 3.0cm, 1.0cm},clip,width=\columnwidth]{figs/5_/ONRFig_v3.pdf}
\caption{An example path generated for the fixed-wing platform conducting a large-area search for an object of interest. The larger black rectangle denotes the search bounds, while the smaller black rectangle highlights a region of higher uncertainty. The red dot marks the estimated position of the detected object based on image detections. The upper-right map displays the information state after planning is complete, while the middle plot shows the percent change in entropy over mission time. The flown path illustrates a balance between allocating resources to the high-priority region and exploring other areas within the search space.}
\label{fig:fwd}
\end{figure}

% Also tested extensively on the AlareTech TL-1 (citation?) tube launched UAV seen in Fig.~\ref{fig:tl1}.

% Talk about amount of flights, hours. Platform. Compute. Show visualization fo example flight. Talk about objects of interest in a broad sense (no mention of water/ocean/land for targets). Follow similar figure format as previous section. Main thing we want to highlight is the differences introduced in plans by having a fixed-wing platform compared to a drone. Include image of Alare TL-1 somewhere.

% One big figure showing all the info we want to convey. 

% \BM{Pitch 10 degrees, onboard computer type, etc}


% \subsection{VTOL?}
% what would it bring?


\section{Discussion}
\section{Discussion}
\label{sec:discussion}

% \TODO{Bryan}

Our multimodal data augmentation method is a plug-and-play method that can be applied to any future VLM. Also the T2I generation can be replaced by any future T2I model, thus the effectiveness of our method automatically improves along with the SOTA T2I model, making it future-proof.



Our main method, \textbf{Co}ntrastive Visual \textbf{D}ata \textbf{A}ugmentation (\textbf{CoDA}), is simple and easy to apply to LMMs in a variety of scenarios. Several components in the pipeline utilize existing off-the-shelf model components that can be easily swapped out for superior versions of similar models as research in their respective field progresses. Therefore, we expect the efficiency and effectiveness of \textbf{CoDA} to dramatically scale along with the advancement of relevant models. 



\bibliographystyle{ACM-Reference-Format}
\bibliography{references}

\appendix

\section{Metric}
\label{sec:metric}

\textbf{Mean Squared Error (MSE)} Mean Squared Error (MSE) is a common statistical metric used to assess the difference between predicted and actual values. The formula is:
\begin{equation}
    MSE = \frac{1}{n} \sum_{i=1}^{n} (y_i - \hat{y}_i)^2
\end{equation}
where $ n $ is the number of samples, $ y_i $ is the actual value, and $ \hat{y}_i $ is the predicted value.

\textbf{Relative L2 Error} Relative L2 error measures the relative difference between predicted and actual values, commonly used in time series prediction. The formula is:
\begin{equation}
    \text{Relative L2 Error} = \frac{\| Y_{\text{pred}} - Y_{\text{true}} \|_2}{\| Y_{\text{true}} \|_2}
\end{equation}
where $ Y_{\text{pred}} $ is the predicted value and $ Y_{\text{true}} $ is the actual value.

\textbf{Structural Similarity Index Measure (SSIM)} The Structural Similarity Index (SSIM) measures the similarity between two images in terms of luminance, contrast, and structure. The formula is:
\begin{equation}
    SSIM(x, y) = \frac{(2\mu_x \mu_y + C_1)(2\sigma_{xy} + C_2)}{(\mu_x^2 + \mu_y^2 + C_1)(\sigma_x^2 + \sigma_y^2 + C_2)}
\end{equation}
where $ \mu_x $ and $ \mu_y $ are the mean values, $ \sigma_x $ and $ \sigma_y $ are the standard deviations, $ \sigma_{xy} $ is the covariance.

\section{Related Work}
\subsection{Deep Learning based Weather Forecasting}
\textbf{Global Weather Forecasting.} Global weather forecasting has seen significant progress with deep learning models. FourCastNet, based on Fourier neural operators, provides global forecasts comparable to traditional numerical methods like IFS, but at much higher speeds~\cite{pathak2022fourcastnet}. Pangu, utilizing the Swin Transformer, exceeds NWP methods, incorporating earth-specific location embeddings for better performance~\cite{bi2023accurate}. The Spherical Fourier Neural Operator (SFNO) extends Fourier methods using spherical harmonics, offering more stable long-term predictions~\cite{bonev2023spherical}. FuXi focuses on long-term forecasting, achieving a 15-day forecasts comparable to ECMWF~\cite{chen2023fuxi}. GraphCast leverages message-passing networks to improve efficiency and forecasting accuracy~\cite{lam2023learning}, and GenCast builds on this to enhance ensemble forecasting~\cite{price2023gencast}. Further, diffusion models like those in~\cite{li2024generative} generate probabilistic ensembles by sampling, while NeuralGCM~\cite{kochkov2024neural} focuses on atmospheric circulation with a dynamic core, offering climate simulation capabilities but at higher training and inference costs. 

\textbf{Regional Weather Forecasting.} The goal of regional weather forecasting is to enhance local prediction accuracy with high-resolution models. CorrDiff~\cite{mardani2023generative} combines U-Net and diffusion models to improve local forecasts. MetaWeather~\cite{kim2024metaweather} adapts global forecasts to regional contexts using meta-learning. GNNs are also widely applied in regional forecasting, with Graphcast~\cite{lam2023learning} enhancing accuracy by modeling complex spatial dependencies. MetNet-3~\cite{espeholt2022deep} offers high-accuracy forecasts for weather variables, such as precipitation, temperature, and wind speed, at 2-minute intervals and 1–4 km resolution, outperforming traditional models like HRRR. NowcastNet~\cite{zhang2023skilful} and DGMR~\cite{ravuri2021skilful} excel in short-term extreme precipitation forecasts using deep generative models and radar data. In spatiotemporal prediction, NMO~\cite{wu2024neural} models the evolution of physical dynamics, providing new insights for local weather forecasting. Similarly, SimVP~\cite{gao2022simvp} and PastNet~\cite{wu2024pastnet} achieve good results in forecasting local precipitation evolution using spatiotemporal convolution methods.
    
% Despite these advances, none of these methods effectively address the challenge of balancing global and regional high-resolution forecasts or capturing the fine-grained, dynamic interactions important for extreme event prediction.
    
\subsection{Numerical analysis methods}
Multigrid methods~\cite{mccormick1987multigrid,wesseling1995introduction,hackbusch2013multi,bramble2019multigrid,hiptmair1998multigrid,brandt1983multigrid,borzi2009multigrid} and nested grid strategies~\cite{miyakoda1977one,zhang2012nested,sullivan1996grid} are widely used to solve PDEs and handle multi-scale problems~\cite{debreu2008two,xue2000advanced}. Multigrid methods use grids of different resolutions to transfer information and accelerate iterations. They efficiently solve large-scale problems and improve computational accuracy. By eliminating low-frequency errors on coarse grids and high-frequency errors on fine grids, multigrid methods effectively handle error convergence at different scales~\cite{he2019mgnet,he2023mgno,shao2022fast}. Nested grid strategies embed higher-resolution fine grids into regions of interest based on a global coarse grid to capture local complex physical phenomena in detail. In weather forecasting, this method provides large-scale background fields on a global scale while refining the grid for target regions to accurately simulate the evolution of local weather systems and the occurrence of extreme events~\cite{bacon2000dynamically}. 

% Our proposed neural nested grid method helps address challenges like boundary information loss in regional forecasting and multi-scale feature capture.

\section{Additional Results}
%
We present more additional results in Figure \ref{fig_0.25-day}, \ref{fig_0.5-day}, \ref{fig_1.0-day} \ref{fig_1.5-day}, \ref{fig_2.0-day}, \ref{fig_2.5-day}, \ref{fig_3.0-day}, \ref{fig_3.5-day}, \ref{fig_4.0-day}, \ref{fig_4.5-day}, \ref{fig_5.0-day}, \ref{fig_5.5-day}, \ref{fig_6.0-day}, \ref{fig_6.5-day}, \ref{fig_7.0-day}, \ref{fig_7.5-day},
\ref{fig_8.0-day}, \ref{fig_8.5-day}, \ref{fig_9.0-day}, \ref{fig_9.5-day},
\ref{fig_10.0-day}, including 18 variables that are importmant to weather forecasting, each with results ranging from 6 hours to 10 days. These additional results further demonstrate the effectiveness of OneForecast. Same as the Figure \ref{fig:visual_results}
, the initial conditions is 00:00 UTC, 1 January 2020.


\begin{figure*}[h]
\centering
\includegraphics[width=1\linewidth]{figures/fig_0.25-day.jpg}
\vspace{-20pt}
\caption{6-hour forecast results of different models.}
\label{fig_0.25-day}
\end{figure*}

\begin{figure*}[h]
\centering
\includegraphics[width=1\linewidth]{figures/fig_0.5-day.jpg}
\vspace{-20pt}
\caption{0.5-day forecast results of different models.}
\label{fig_0.5-day}
\end{figure*}

\begin{figure*}[h]
\centering
\includegraphics[width=1\linewidth]{figures/fig_1.0-day.jpg}
\vspace{-20pt}
\caption{1-day forecast results of different models.}
\label{fig_1.0-day}
\end{figure*}

\begin{figure*}[h]
\centering
\includegraphics[width=1\linewidth]{figures/fig_1.5-day.jpg}
\vspace{-20pt}
\caption{1.5-day forecast results of different models.}
\label{fig_1.5-day}
\end{figure*}

\begin{figure*}[h]
\centering
\includegraphics[width=1\linewidth]{figures/fig_2.0-day.jpg}
\vspace{-20pt}
\caption{2-day forecast results of different models.}
\label{fig_2.0-day}
\end{figure*}


\begin{figure*}[h]
\centering
\includegraphics[width=1\linewidth]{figures/fig_2.5-day.jpg}
\vspace{-20pt}
\caption{2.5-day forecast results of different models.}
\label{fig_2.5-day}
\end{figure*}

\begin{figure*}[h]
\centering
\includegraphics[width=1\linewidth]{figures/fig_3.0-day.jpg}
\vspace{-20pt}
\caption{3-day forecast results of different models.}
\label{fig_3.0-day}
\end{figure*}

\begin{figure*}[h]
\centering
\includegraphics[width=1\linewidth]{figures/fig_3.5-day.jpg}
\vspace{-20pt}
\caption{3.5-day forecast results of different models.}
\label{fig_3.5-day}
\end{figure*}

\begin{figure*}[h]
\centering
\includegraphics[width=1\linewidth]{figures/fig_4.0-day.jpg}
\vspace{-20pt}
\caption{4-day forecast results of different models.}
\label{fig_4.0-day}
\end{figure*}

\begin{figure*}[h]
\centering
\includegraphics[width=1\linewidth]{figures/fig_4.5-day.jpg}
\vspace{-20pt}
\caption{4.5-day forecast results of different models.}
\label{fig_4.5-day}
\end{figure*}


\begin{figure*}[h]
\centering
\includegraphics[width=1\linewidth]{figures/fig_5.0-day.jpg}
\vspace{-20pt}
\caption{5.0-day forecast results of different models.}
\label{fig_5.0-day}
\end{figure*}

\begin{figure*}[h]
\centering
\includegraphics[width=1\linewidth]{figures/fig_5.5-day.jpg}
\vspace{-20pt}
\caption{5.5-day forecast results of different models.}
\label{fig_5.5-day}
\end{figure*}

\begin{figure*}[h]
\centering
\includegraphics[width=1\linewidth]{figures/fig_6.0-day.jpg}
\vspace{-20pt}
\caption{6.0-day forecast results of different models.}
\label{fig_6.0-day}
\end{figure*}

\begin{figure*}[h]
\centering
\includegraphics[width=1\linewidth]{figures/fig_6.5-day.jpg}
\vspace{-20pt}
\caption{6.5-day forecast results of different models.}
\label{fig_6.5-day}
\end{figure*}

\begin{figure*}[h]
\centering
\includegraphics[width=1\linewidth]{figures/fig_7.0-day.jpg}
\vspace{-20pt}
\caption{7.0-day forecast results of different models.}
\label{fig_7.0-day}
\end{figure*}

\begin{figure*}[h]
\centering
\includegraphics[width=1\linewidth]{figures/fig_7.5-day.jpg}
\vspace{-20pt}
\caption{7.5-day forecast results of different models.}
\label{fig_7.5-day}
\end{figure*}

\begin{figure*}[h]
\centering
\includegraphics[width=1\linewidth]{figures/fig_8.0-day.jpg}
\vspace{-20pt}
\caption{8.0-day forecast results of different models.}
\label{fig_8.0-day}
\end{figure*}

\begin{figure*}[h]
\centering
\includegraphics[width=1\linewidth]{figures/fig_8.5-day.jpg}
\vspace{-20pt}
\caption{8.5-day forecast results of different models.}
\label{fig_8.5-day}
\end{figure*}

\begin{figure*}[h]
\centering
\includegraphics[width=1\linewidth]{figures/fig_9.0-day.jpg}
\vspace{-20pt}
\caption{9.0-day forecast results of different models.}
\label{fig_9.0-day}
\end{figure*}

\begin{figure*}[h]
\centering
\includegraphics[width=1\linewidth]{figures/fig_9.5-day.jpg}
\vspace{-20pt}
\caption{9.5-day forecast results of different models.}
\label{fig_9.5-day}
\end{figure*}

\begin{figure*}[h]
\centering
\includegraphics[width=1\linewidth]{figures/fig_10.0-day.jpg}
\vspace{-20pt}
\caption{10.0-day forecast results of different models.}
\label{fig_10.0-day}
\end{figure*}


\section{Detailed Mathematical Proof}
\label{sec:proof}
\textbf{Proof of Theorem 1}

Now we have N augmented data and we need to select the best from them. We consider both the quality and the diversity of these data and get the sampling strategy from an optimization problem.

We model the sampling strategy as a multinomial distribution supported on all the augmented data $S = \{\mathbf{X}_j\}_{j=1}^N$, which means that the sampling strategy $\pi=(\pi_1,...,\pi_N)^\top$ is the corresponding probabilities of selecting $\mathbf{X}_1,...,\mathbf{X}_N$, then we can model the expectation of the similarity as:
$$\begin{aligned}
 & \mathbb{E}_{Y_x,Y_{x^{\prime}}\in\mathcal{C}}\{g(x,x^{\prime})\mid S\} \\
 & =\quad\int g(\mathbf{x},\mathbf{x}^{\prime})\boldsymbol{\pi}(\mathbf{x})\mathrm{Pr}_{S}(Y_{x}\in\mathcal{C}\mid\boldsymbol{x}=\mathbf{x})\boldsymbol{\pi}(\mathbf{x}^{\prime})\mathrm{Pr}_{S}(Y_{x}\in\mathcal{C}\mid\boldsymbol{x}=\mathbf{x}^{\prime})d\mathbf{x}d\mathbf{x}^{\prime} \\
 & =\quad\sum_{i,j=1}^Ng(\mathbf{X}_i,\mathbf{X}_j)\pi_i\pi_j\mathrm{Pr}_{S}(Y_x\in\mathcal{C}\mid\boldsymbol{x}=\mathbf{X}_i)\mathrm{Pr}_{S}(Y_x\in\mathcal{C}\mid\boldsymbol{x}=\mathbf{X}_j),
\end{aligned}$$
where the set $\mathcal{C}$ denotes the criterion of selection we are using, the function $g$ can be chosen as any similarity metric function and $x$ means a random variable.

The core to solving the above optimization problem is to use predictive inference to approximate the conditional probability of $\{Y_x\in\mathcal{C}\}$ given $x = \mathbf{X}$
Let $\mu ( \mathbf{x} ) : = \mathbb{E} ( Y\mid \mathbf{X} = \mathbf{x} )$ be the oracle associated with $( \mathbf{X} , Y) .$ Denote $\theta_j=\mathbb{I}\{Y_j\in\mathcal{C}\}$. As the augmented data
$\mathbf{X}_1,...,\mathbf{X}_N$ are independently identically distributed, $\theta_1,...,\theta_N$ can be regarded as independent Bernoulli($q)$ variables with $q=\Pr(Y_j\in\mathcal{C}).$ The probability distribution of the predicted result $W_j$ for $j=1,...,N$ is
$$\Pr(W_j\mid\theta_j)=(1-\theta_j)f_0+\theta_jf_1,\quad$$
where $f_0$ and $f_1$ are the conditional distributions of $W_j$ on $Y_j \in \mathcal{C}$ or not.

Denote $T(w) = \frac{(1-q)f_0(W_j)}{f(W_j)}$, we can rewrite the expectation of the similarity as
$$\mathbb{E}_{Y_x,Y_{x^{\prime}}\in\mathcal{C}}\{g(x,x^{\prime})|S\}=\sum_{i,j=1}^Ng(\mathbf{X}_i,\mathbf{X}_j)\pi_i\pi_j(1-T_i)(1-T_j)=\boldsymbol{\pi}^\top A_\mathbb{T}\boldsymbol{\pi},$$

Next, we use the expectation to control the quality of the data.
$$\mathbb{E}\{\mathbb{I}(Y_x\not\in\mathcal{C})\mid S\}=\sum_{i=1}^N\Pr(Y_i\not\in\mathcal{C}\mid\mathbf{X}_i)\pi_i=\sum_{i=1}^N\pi_iT_i\leq\alpha,$$

In all, the optimization problem can be modeled as 
\begin{align}
    & \arg\min_{\boldsymbol{\pi}}\quad h(\boldsymbol{\pi},\mathbb{T}):=\boldsymbol{\pi}^\top A_\mathbb{T}\boldsymbol{\pi}, \\
    & \text{subject to} \quad
        \begin{cases}
            \sum_{i = 1}^N\pi_iT_i\leq\alpha, \\
            \sum_{i = 1}^N\pi_i = 1, \\
            0\leq\pi_i\leq m^{-1}, \quad 1\leq i\leq N.
        \end{cases}
\end{align}

where $m$ is used to control the maximum selection.

The best selection of K is determined by the strategy $\pi$ which serves as the solution to the above optimization problem.

\section{Additional Experiments}
\label{sec:more_experiments}
\subsection{Long-term forecasting experiment expansion}

In the long-term forecasting experiments, we compare the performance of different backbone models on the SWE benchmark, evaluating the relative L2 error for three variables (U, V, and H). Our setup inputs 5 frames and predicts 50 frames. For the SimVP-v2 model, using \method{} reduces the relative L2 error for SWE (u) from 0.0187 to 0.0154, SWE (v) from 0.0387 to 0.0342, and SWE (h) from 0.0443 to 0.0397. We visualize SWE (h) in 3D as shown in Figure~\ref{fig:case} [\textcolor{red}{I}]. For the ConvLSTM model, applying \method{} reduces the relative L2 error for SWE (u) from 0.0487 to 0.0321, SWE (v) from 0.0673 to 0.0351, and SWE (h) from 0.0762 to 0.0432. For the FNO model, using \method{} reduces the relative L2 error for SWE (u) from 0.0571 to 0.0502, SWE (v) from 0.0832 to 0.0653, and SWE (h) from 0.0981 to 0.0911. Overall, \method{} significantly improves the long-term forecasting accuracy of different backbone models.

\begin{figure*}[h]
    \centering
    \includegraphics[width=\textwidth]{image/casestudy.pdf}
    \caption{
    \textcolor{red}{I.} 3D visualization of the SWE(h), showing Ground-truth, SimVP-V2+BeamVQ predictions, and Error at T=1, 10, 20, 30, 40, 50. The first row shows Ground-truth, the second SimVP-V2+BeamVQ predictions, and the third Error. \textcolor{red}{II.} A case study. Building fire simulation with ventilation settings added to Wu's Prometheus~\cite{wu2024prometheus}. (a) Layout and HRR growth. (b) Comparison of physical metrics for different methods. (c) Ground-truth, ResNet+BeamVQ, and ResNet predictions.
    }
    \label{fig:case} 
\end{figure*}


\subsection{Experiment Statistical Significance}
\label{sec:significance}
To measure the statistical significance of our main experiment results, we choose three backbones to train on two datasets to run 5 times. 
Table~\ref{tab:significance} records the average and standard deviation of the test MSE loss.
The results prove that our method is statistically significant to outperform the baselines
because our confidence interval is always upper than the confidence interval of the baselines. 
Due to limited computation resources, we do not cover all ten backbones and five datasets, 
but we believe these results have shown that our method has consistent advantages.


\begin{table}[h]
\label{tab:significance}
\centering
\begin{scriptsize}
    \begin{sc}
    \caption{ The average and standard deviation of MSE in 5 runs}
    \label{tab:significance}
    \centering
        \renewcommand{\multirowsetup}{\centering}
        \setlength{\tabcolsep}{10pt}
        \begin{tabular}{l|cc|cc}
            \toprule
            
            \multirow{4}{*}{Model} & \multicolumn{4}{c}{Benchmarks}  \\
            \cmidrule(lr){2-5}
            & \multicolumn{2}{c}{NSE} &   \multicolumn{2}{c}{SEVIR}   \\
            \cmidrule(lr){2-5}
           & Ori & + BeamVQ & Ori & + BeamVQ  \\
            \midrule
            ConvLSTM &0.4092$\pm$0.0002 &\textbf{0.1277$\pm$0.0001}  & 0.1762 0.0007  & \textbf{0.1279$\pm$0.0009}  \\
            FNO &  0.2227$\pm$0.0003 &\textbf{0.1007 $\pm$0.0002}& 0.0787$\pm$0.0012 & \textbf{ 0.0437$\pm$0.0013} \\
            CNO & 0.2192 $\pm$0.0008 &\textbf{ 0.1492$\pm$0.0011}& 0.0057$\pm$0.0005 & \textbf{ 0.0053$\pm$0.0006} \\
            \bottomrule
        \end{tabular}
    \end{sc}

\end{scriptsize}
\end{table}
 
\end{document}
\endinput
%%
%% End of file `sample-sigconf.tex'.

% \bibliography{ref}


 % \begin{IEEEbiography}

% \EOD

\end{document}
