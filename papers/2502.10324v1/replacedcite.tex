\section{Related Works}
\label{ch:related_works}
There are limited studies investigating channel rank and coverage for UAV links in the literature with RT studies or measurements. Our literature review with representative publications related to our work and the differences with the present work is summarized in Table \ref{tab:literature_review}.

\subsection{UAV Channels}
In ____ and ____, RT-based UAV channel modeling has been investigated. The altitude-dependent channel parameters, such as path loss, power delay profile, and angular information, are simulated to characterize the UAV channel. The developed channel model consisted of a deterministic line-of-sight (LoS) path and ground reflection, and stochastic non-LoS (NLoS) components. The developed model has been validated using RT simulations and measurements.  

A stochastic geometry model is developed in ____ for investigating coverage and outage probability in an urban environment. The study investigates UAV backhaul links considering building density, antenna beamwidth, and interference effects. The results highlight the existence of an optimal UAV altitude that maximizes coverage while ensuring reliable backhaul connectivity. Similarly, in ____, measurement-based UAV channel modeling is conducted in an urban scenario. The study extracts key parameters for path loss models and develops a simplified Saleh-Valenzuela channel model for UAV-ground links.

In our previous work____, UAV channel characteristics are analyzed in rural and urban scenarios. Specifically, RF coverage, channel rank, and condition number distribution have been analyzed for UAV networks using Matlab-based RT simulations. Centennial Campus and Lake Wheeler Field Labs at NC State University are considered as urban and rural scenarios, respectively. Due to the blockage from the buildings, an outage can be observed at the lower altitude of the UAV. Moreover, the probability of having a channel rank of 3 or 4 is getting lower because of the LoS dominant path to the UAV as the altitude of the UAV increases.


\subsection{Ray Tracing}
% In ____, 
In ____, an RF signal mapping scheme for cellular networks has been investigated by integrating RT simulation using Sionna with geographic databases and machine learning (ML) techniques. A cascaded neural network refines signal strength predictions by leveraging geographical information from RT and sparse real-world measurements. The proposed method improves computational efficiency while maintaining high accuracy, outperforming conventional RT-based approaches in real-world evaluations.

In ____ and ____, RT-based RF coverage and channel capacity analysis have been studied for urban scenarios. Nodetably in ____, a large-scale digital twin framework for downtown Boston has been developed. The proposed BostonTwin framework integrates a high-fidelity 3D model of the downtown Boston area with geographical data, enabling RT simulations. By leveraging Sionna, the framework facilitates large-scale coverage mapping and signal-to-noise ratio (SNR) evaluations. Moreover, the authors highlight the areas where BostonTwin meets the requirements for 6G use cases. On the other hand, ____ uses site maps and a RT software for computing MIMO gain matrices for a given mobile station position in urban areas of Boston and Manhattan. Subsequently, authors use the MIMO gain matrices to determine achievable rates for MIMO transmission modes including spatial multiplexing, beamforming, and diversity. The results suggest that the use of site-specific RT for data rate prediction works more effectively than the use of stochastic models.

An automated geometry extraction framework for precise RT simulation is developed in ____. With the capability of open-source resources from OSM, Microsoft Global ML Building Footprints, and the US Geological Survey (USGS), high-fidelity RT simulation environments can be extracted. The authors conducted sensitivity analyses to investigate the impact on the accuracy of RT simulation with the environmental factors including building heights, locations, and material settings. Channel statistics of path gain, delay spread, and link outage are shown with the developed framework. 


\subsection{Channel Rank Analysis}
In ____, the channel rank for an outdoor-to-indoor massive MIMO system has been analyzed. The angular characteristics in terms of the angle of arrival and departure have been investigated for the given wireless communication environment. Then, singular values of each channel rank have been simulated and measured. It is observed that the singular values decrease rapidly as the range of eigenvalue to be captured increases. 

Theoretical lower and upper bounds of channel capacity for high-rank MIMO systems are studied in ____. The study investigates how LoS propagation affects channel rank and capacity. It shows that strong LoS components increase channel correlation, reducing the efficiency of spatial multiplexing. However, it also demonstrates that optimizing antenna placement and spacing helps maintain orthogonality, preserve channel rank, and improve system performance.

In ____, the use of passive intelligent reflecting surfaces (IRSs) has been investigated to improve channel rank and spatial multiplexing in urban UAV scenarios. This study proposes an IRS placement optimization framework to maximize the average channel capacity along predefined UAV trajectories. Numerical results at various carrier frequencies demonstrate that IRS-assisted channels significantly improve MIMO capacity compared to LoS-dominant UAV channels. The authors introduce the potential of IRSs as a cost-effective solution for enhancing UAV communications in dense urban environments. Other work on studying channel rank includes ____ and ____, which use measurements for an outdoor LTE MIMO network to explore how the channel rank varies for various scenarios. 

% ____. 


\subsection{Kriging Interpolation}
Coverage analysis and map construction schemes using spatial interpolation for mobile systems have been investigated in ____. Especially in ____, a radio map has been developed by interpolating geo-located measurements. A fixed-rank Kriging interpolation scheme has been used to generate the radio map to reduce computational complexity. The simulation results show the trade-off between computational complexity and accuracy in coverage estimation over rural scenarios. 
In ____, a Kriging interpolation-based 3D radio map generation scheme for radio dynamic zones is proposed. A realistic propagation model is developed by using measurements over the air-to-ground link in terms of path loss, shadowing, and spatial correlation. Horizontal and vertical correlations of received signal strength (RSS) are used for accurate radio map generation. The generated radio map is more accurate than the model with perfect knowledge of the path loss due to the benefits of spatial correlation. 

\subsection{Trees}
In ____, an RT-based case study of channel measurements to model propagation effects due to trees is provided. Statistics of angular properties and penetration loss over multiple types of trees have been analyzed to characterize blockage and propagation for mmWave channels. The authors highlighted that the tree-specific propagation model can be used for realistic RT simulation in non-controlled environments. 

An RT-based 3D mmWave propagation model through vegetation is investigated in ____. To develop the propagation model, measurements of radiation, angular, and 3D scattering profiles are used in the scenario of groups of trees and various dimensions of the trees. The proposed 3D propagation model is assessed by comparing 3D directional measurements within different positions and angles.  

In ____, the impact of foliage in an urban environment over the mmWave channel is investigated. Here, a simplified urban scenario consists of four buildings in a grid-wise position and four trees at the end of each road between the buildings. A hybrid channel model with the knowledge of a conventional correlation-based channel model is integrated with the RT results to characterize the foliage impact.

The impact of foliage on UAV air-to-ground propagation channels has been investigated using channel-sounding measurements in ____ for ultra-wideband signals. The authors examine how tree obstructions affect signal propagation analyzing path loss, multipath fading, and coherence bandwidth. Results show that foliage significantly increases path loss and reduces coherence bandwidth, leading to stronger multipath fading and degraded reliability.

\subsection{Measurements}
% ____.
In ____, channel-sounding measurement-based UAV air-to-ground propagation channel modeling has been investigated. The study conducts comprehensive measurements for various UAV communication scenarios and derives statistical models for path loss, multipath propagation, and small-scale fading. The measurements capture high-resolution time and frequency domain characteristics. The collected data is analyzed to extract power delay profiles and coherence bandwidth, which provides insights into UAV channel behavior under different propagation conditions.


Studies in ____ and ____, considering terrestrial cellular network scenarios, analyze the impact of antenna configurations on channel rank and system performance using LTE MIMO measurements. In ____, increased antenna spacing and optimized polarization, achieved by comparing vertically and horizontally spaced configurations, are shown to improve channel rank and throughput. Similarly, ____ demonstrates that reducing mutual correlation enhances throughput by enabling spatial multiplexing with a higher channel rank. Both studies highlight the critical role of precise measurement campaigns in evaluating channel rank under various conditions.

\subsection{Contributions of This Work}

 Based on Table~\ref{tab:literature_review} and to our knowledge, there are no works that study the effects of foliage on the channel conditions of UAV networks using RT simulations or measurements. Other than ____, there are also no works that compare RT simulations with measurements for analyzing coverage for UAV networks.  
In this paper, we extend the scope of our previous work in ____ to address the gaps in the literature. In particular, the target area of the rural scenario has been extended with multiple base stations to understand long-distance propagation effects. Trees in the target area have been included to investigate realistic propagation in the rural area. Moreover, a constant channel rank threshold has been used to investigate channel rank using the ratio of the strongest singular value. Lastly, the spatial correlation, which can be observed in the channel rank distribution over the rural area scenario, has been adopted for the Kriging interpolation-based 3D channel rank interpolation scheme.

     \begin{figure*}[t]
     \centering
     \includegraphics[width=1.6\columnwidth]{images/introduction_figure_v4.png}
     \caption{UAV connectivity scenario in a rural area. UAV coverage can be blocked due to buildings and foliage. It is of interest to predict channel characteristics, such as channel rank and coverage, at a location where no measurements have been collected before, based on measurements at other locations.}     \label{fig:introduction_figure}
 \end{figure*}