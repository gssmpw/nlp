\documentclass{ieeeaccess}
\usepackage{cite}
\usepackage{amsmath,amssymb,amsfonts}
\usepackage{algorithmic}
\usepackage{graphicx}
\usepackage{xcolor}
\usepackage{textcomp}
\usepackage{listings}
\usepackage{subfig}
\usepackage{braket}
\usepackage{hyperref}
\usepackage{cleveref}
\crefname{figure}{Figure}{Figures}
\crefname{section}{Section}{Sections}
\usepackage[htt]{hyphenat}



\def\BibTeX{{\rm B\kern-.05em{\sc i\kern-.025em b}\kern-.08em
    T\kern-.1667em\lower.7ex\hbox{E}\kern-.125emX}}
\begin{document}

\definecolor{codegreen}{rgb}{0,0.6,0}
\definecolor{codegray}{rgb}{0.5,0.5,0.5}
\definecolor{codepurple}{rgb}{0.58,0,0.82}
\definecolor{backcolour}{rgb}{0.95,0.95,0.92}

\lstdefinestyle{mystyle}{
  language=Python,
  basicstyle=\ttfamily\small,
  numbers=left,
  numberstyle=\tiny\color{codegray},
  stepnumber=1,
  numbersep=5pt,
  backgroundcolor=\color{white},
  showspaces=false,
  showstringspaces=false,
  showtabs=false,
  frame=single,
  rulecolor=\color{black},
  tabsize=2,
  captionpos=b,
  breaklines=true,
  breakatwhitespace=false,
  title=\lstname,
  keywordstyle=\color{blue},
  commentstyle=\color{codegreen},
  stringstyle=\color{red},
  escapeinside={\%*}{*)},
  morekeywords={*,...}
}

\lstset{style=mystyle}

\lstdefinelanguage{qoala}
{
    morekeywords=[1]{
        add, add, sub, addm, subm,
        jmp, bez, bnz, beq, bne, blt, bge,
        set, store, load
        meas,
        create_epr, recv_epr,
        init, meas, h, rot_x, rot_y, rot_z, cnot, cphase, cx_dir, cy_dir,
        APPID, NETQASM,
        run_request, run_routine, return_result, assign
    },
    morekeywords=[2]{
        META_START, META_END,
        SUBROUTINE,
        NETQASM_START, NETQASM_END,
        REQUEST
    },
    keywordstyle=[1]\color{blue},
    keywordstyle=[2]\color{purple},
    sensitive=true,
    morecomment=[l]{//},
    morecomment=[s]{/*}{*/},
    morecomment=[s][\color{blue}]{\#\ }{\ },
    morestring=[b]",
    escapeinside={(*@}{@*)},
}

\lstnewenvironment{qoalacode}{\lstset{style=mystyle,language=qoala}}{}

\newcommand{\revision}[1]{#1}
\newcommand{\todo}[1]{\textcolor{red}{TODO #1}}

\title{Qoala: an Application Execution Environment for Quantum Internet Nodes}

\author{
    Bart van der Vecht\authorrefmark{1,2,3},
    Atak Talay Yücel\authorrefmark{1},
    Hana Jirovská\authorrefmark{1,2,3},
    Stephanie Wehner\authorrefmark{1,2,3}
}
\address[1]{QuTech, Delft University of Technology}
\address[2]{Kavli Institute of Nanoscience, Delft University of Technology}
\address[3]{Quantum Computer Science, Electrical Engineering, Mathematics and Computer Science, Delft University of Technology}

\markboth
{Van der Vecht \headeretal: Qoala}
{Van der Vecht \headeretal: Qoala}

\corresp{Corresponding author: Bart van der Vecht (email: b.vandervecht@tudelft.nl).}

\begin{abstract}
Recently, a first-of-its-kind operating system for programmable quantum network nodes was developed, called QNodeOS.
Here, we present an extension of QNodeOS called Qoala, which introduces
(1) a unified program format for hybrid interactive classical-quantum programs, providing a well-defined target for compilers, and 
(2) a runtime representation of a program that allows joint scheduling of the hybrid classical-quantum program, multitasking, and asynchronous program execution.
Based on concrete design considerations, we put forward the architecture of Qoala, including the program structure and execution mechanism.
We implement Qoala in the form of a modular and extendible simulator that is validated against real-world quantum network hardware (available online).
However, Qoala is not meant to be purely a simulator, and implementation is planned on real hardware.
We evaluate Qoala's effectiveness and performance sensitivity to latencies and network schedules using an extensive simulation study.
Qoala provides a framework that opens the door for future computer science research into quantum network applications, including scheduling algorithms and compilation strategies that can now readily be explored using the framework and tools provided. 
\end{abstract}

\titlepgskip=-15pt

\maketitle

\documentclass[../main.tex]{subfiles}
\graphicspath{{../images/}}
\makeatletter
\def\input@path{{../images/}}
\makeatother
\begin{document}
\section{Introduction}
\begin{figure}
\centering
\begin{tikzpicture}
\node[inner sep=0pt] (ws) at (0, 0) {
\includegraphics[height=.4\textwidth, trim={10cm 0 10cm 0},clip]{world_space.png}};
\node[inner sep=0pt] (cs) at (6,0) {\includegraphics[height=.4\textwidth, trim={10cm 1cm 10cm 4cm},clip]{conf_space.png}};
\end{tikzpicture}
\vspace{-5pt}
\label{fig:pbrm_intro}
\caption{\textbf{Left}: Shows world space obstacles as grey spheres. Robots start and goal configuration is colored red and green, respectively. Configurations along the computed path are colored transparent blue. \textbf{Right:} Mapped world space scenario to configuration space. Obstacle region is the grey mesh. Red spheres are collision-free regions computed by the neural SCDF. The optimized shortest path in the convex corridor is the blue curve.}
\vspace{-25pt}
\end{figure}
Motion planning is the problem of finding a collision-free trajectory that connects a given start and goal configuration. The planning takes place in the configuration space of the robot. For single body robots, like mobile robots or drones, the configuration space and the world space are usually the same. This simplifies the planning, since explicit obstacle representations are available which enables geometrical tools like separating hyperplanes, smallest distance to obstacles etc., to be used when designing motion planning algorithms. For multi-body robots like manipulators, the situation is completely different. The world space obstacles are usually mapped to non-convex regions, and to make the problem even harder, the mapping is usually not known. Forming explicit representations of the obstacle region in the configuration space is usually too expensive or intractable. Despite all of this, sampling based planners are used with great success, which mainly is due to their use of implicit representations of the obstacle region. The basic idea is to construct a graph in the configuration space that covers and connects the collision-free region. From this graph, a path can be extracted that connects a given start and goal configuration. The approach is computationally expensive, since the graph is constructed with the smallest geometrical building block available, points, which represents a collision-check. Furthermore, the extracted paths from the graph are non-smooth and jagged due to the stochastic nature of the approach. This adds an additional post-processing step to the process, where the paths are shortcutted and smoothened, before the path can be used for tracking. Clearly a lot of time is invested to form this graph and produce smooth paths. Thus, if the obstacles start to move, then all of this work is done in no use, since all points that make up this graph need to be re-verified, which is simply too time consuming to be done in real time.
\\\\
In this work, we want to address the existing drawbacks of the sampling based planners. Our main contribution is an improved motion planner where each vertex in the graph covers a collision-free region in the form of a sphere instead of a point and where the edges are formed with neighboring intersecting spheres. This representation has the advantage of instead of returning piecewise linear paths, returning a sequence of overlapping spheres, i.e. a convex corridor, that connects a given start and goal configuration, illustrated in Figure \ref{fig:pbrm_intro}. This convex corridor allows us to use convex optimization to produce smooth trajectories, instead of computationally expensive post-processing methods. The representation further allows us to estimate the coverage of the collision-free space, which gives us awareness and feedback in the offline roadmap construction phase. Finally, our representation is simple to adapt to moving obstacles, simply requery for the new radii and recheck for intersections. 
\\\\
The spherical collision-free regions are formed using a signed distance function (SDF), which is a function that returns the smallest distance from an arbitrary point to the boundary of an obstacle. As the name implies, the distance is signed, thus if the point is inside the obstacle it is negative otherwise positive. If the distance is positive, a sphere with radius equal to the distance is guaranteed to cover a collision-free region. Using an SDF in motion planning is not new, but what is novel about our approach is that we express the distance in the configuration space instead of the world space and by doing so allows us to form these convex collision-free regions. We refer to the resulting SDF as a signed configuration distance function (SCDF). Computing an SCDF analytically is non-trivial, our approach is therefore to parameterize the SCDF with a deep neural network and learn the mapping by supervised learning. Our resulting neural SCDF can compute distances for different parameter values of obstacle shapes and we also show how multiple distances can be combined, thus making our approach flexible.
\section{Related work}
Motion planning algorithms can roughly be divided into three families, grid-based, sampling based and optimization based methods. Grid-based methods (GBM) discretize the planning space from which a graph is then compiled. A standard search method is A$^\star$ \citep{a_star}, which is classified as an \textit{informed} search method, since it employs a heuristic function to speed up the search. A$^\star$ guarantees to return an optimal path at the level of discretization used. GBMs usually discretize the planning space by a regular lattice and this limits the GBMs to problems with low dimensionality due to the curse of dimensionality. Thus, GBMs are usually limited to single-body robots where the degrees of freedom (DOF) are low. To overcome the inherent scaling problem with the GBMs, stochastic methods are usually used for multi-body robots. These methods are termed as sampling-based methods (SBM) and core members within this family are the rapidly-exploring random trees (RRT) \citep{rrt} and the probabilistic roadmap (PRM) \citep{prm}. RRT grows a tree from the start configuration and explores the collision-free region in a rapid way until it is able to connect to the goal region. RRT is usually improved by bi-directional planning \citep{rrt_connect}, i.e. an additional tree is grown from the goal configuration and the trees are tested for connection after any tree has been expanded. RRT is a single-query method, thus it searches for a path from scratch each time it is queried. Contrary to this, PRM is a multi-query method, which solves for multiple queries without starting from scratch. PRM does this by creating a roadmap (graph) that covers the collision-free space as an offline step. The graph is then used to solve for multiple queries. PRMs are used in cases where the environment does not change since the extra offline step is too computationally costly and needs to be re-done if the environment is changed. In our work, we address this inherent issue by using a different roadmap representation. Our vertices in the graph cover a collision-free region in the form of spheres and we form the edges by checking for intersecting spheres. If something in the environment changes, we recompute the spheres radii and recheck the intersections, without relying on collision detection. We use a trained neural network to compute the sphere radius, therefore querying for the radius can be done fast, hence our representation enables the PRM for dynamic environments.
\\\\
In the recent decades, optimization based methods (OBM) \citep{chomp, schulman, itomp, stomp} have been introduced as an alternative to SBM for multi-body robots. Like the SBM, the OBMs scale well to higher dimensional problems and produce smoother motion. It is common to use a SDF in the optimization since it is a smooth function, thus enabling gradient-based methods. However, the standard way of expressing the SDF is in world space. The distance therefore needs to be mapped to the configuration space by the forward kinematics. This mapping makes the optimization problem a non-linear program (NLP), which is computationally expensive to solve. Recently, a different approach has been proposed. In \cite{mp_gcs} motion planning is formulated as a convex optimization problem by using the graph of convex sets framework \citep{gcs}. The underlying idea is to decompose the collision-free space into intersecting convex sets from which a convex optimization problem is formulated. In cases where an explicit representation of the obstacles in the configuration space exists, like for single-body robots, creating collision-free convex regions can be done fast \citep{iris}. For multi-body robots, this is non-trivial. Existing work does this successfully \citep{iris_nlp, iris_c} by an optimization based approach, but the methods are still too time consuming to be used in the presence of moving obstacles. Our approach is instead to use deep learning to learn an SDF expressed in the configuration space. With this, we can query for shortest distances to the collision boundary, which allows us to expand spherical regions which are collision-free. Our approach is fast and therefore enables our suggested roadmap planner to be used in dynamic environments.
\\\\
Recent research has focused on learning collision detection \citep{fk_kernel_distance, diffco, graphdistnet} by predicting the signed distance between the robot links and the surrounding obstacles in the world space. The learned SDF is used in trajectory optimization but since the distance is expressed in the world space, the problem becomes an NLP and therefore takes a long time to solve. We take a novel approach and suggest to instead express the signed distance in the configuration space. This allows us to improve the PRM at the same time as it enables convex optimization for trajectory optimization, which runs faster and is more reliable than NLP solvers. In \cite{cspf} a learned signed distance function in the configuration space is proposed similar to our approach. However, their approach is restricted to point cloud representations, while we propose to represent the obstacles as parameterized geometric shapes, e.g. spheres. Furthermore, we also show how to use our learned SCDF to improve an existing roadmap planner.
\section{Problem formulation}
A robot is located in the world space, $\W \subset \R^3 $. The unique location of the robot is given by its configuration $\q \in \C$, where $\C$ is the configuration space. The set of points covered by the robots bodies at a certain configuration is expressed as $\B(\q) \subset \W$. The robot is surrounded by $\NrObst$ obstacles $\O = \bigcup_{i=1}^{\NrObst} \O_i$, where  $\O_i \subset \W$. The representation of the obstacle in the configuration space is the set $\C\O_i = \{\q \in \C \: |\: \B(\q) \cap \O_i \neq \emptyset \}$. The obstacle space is formed as $\Co = \bigcup_{i=1}^{\NrObst} \C \O_i$. The complement is referred to as the free space, $\Cf = \C \setminus \Co$. The path planning problem is a tuple, ($\Cf$, $\qStart$, $\qGoal$), where we want to connect a query pair, consisting of a start, $\qStart$, and goal configuration, $\qGoal$, with a geometric path, $\q(s): [0, 1] \mapsto \Cf$, such that $\q(0)=\qStart$ and $\q(1)=\qGoal$, or report correctly when such a path does not exist.
\end{document}

\section{Related Work}
% \subsection{Vision Language Model}
% 시각장애인에서 상황을 설명할 DB가 없으니 만들었다. 그리고 이를 VLM에 튜닝했다.
\subsection{Technical approaches for assisting the visually-impaired}


\subsection{Datasets for visual instruction tuning}

\section{Design considerations}
\label{sec:design_considerations}

\subsection{Background and context}
\label{sec:background_context}

\textit{Quantum nodes}.
A quantum internet connects quantum nodes on which quantum programs may be 
executed.
In their most general form, such nodes
are \textit{processing nodes} that have a quantum memory to store quantum bits (qubits) on which quantum operations (qubit initialization, quantum gates and measurements) can be performed. Pairs of nodes can establish \textit{entanglement} between them over a quantum network. Entanglement is a special property of two qubits (an \emph{entangled pair}), where one qubit is stored in the memory of each node. Nodes can also exchange classical messages (e.g. via dedicated classical links or the internet), where no guarantees are assumed on their message delivery times. 

\textit{Programs}.
A program is a series of instructions to be executed by a node.
Instructions can be categorized into four types: local classical processing, classical message-passing, quantum local processing (quantum operations), and remote entanglement generation.
A program can keep classical variables in a classical memory, and quantum variables (qubits) in the node's quantum memory during the execution.
Multiple programs, each running on their own node, together form an \textit{application} (see \cref{fig:program_illustration}), e.g. QKD (two programs, one per node),
or secret sharing~\cite{hillery1999quantum} (a program each on many nodes).
Programs may involve asynchronous operations (e.g. a server awaiting entanglement with multiple clients).

\textit{Network schedule}.
A quantum network stack has been proposed~\cite{dahlberg2019link} and implemented~\cite{pompili2022experimental} that turns entanglement generation into a robust service independent of the quantum hardware platform.
Important for the design of an architecture for the execution of quantum internet applications is that in this stack, the nodes will establish a network schedule of time slots in which they will trigger entanglement generation (due to need to synchronize entanglement generation at the physical layer~\cite{dahlberg2019link} at high-precision (ns)).
This means that once entanglement has been requested from the network, the nodes can use only the slots in the network schedule to produce entanglement between them, imposing constraints on the ability to schedule applications. What's more, in present day systems~\cite{pompili2021realization, krutyanskiy2023entanglement} limitations in the physical devices prohibit the execution of local operations while engaging in network operations (entanglement generation), creating further dependencies between the local quantum execution and entanglement generation. 
As the specifics of network scheduling~\cite{network-scheduling, skrzypczyk2021architecture} are not within scope of this paper,
we assume the existence of a \textit{network controller} that takes application demand for entanglement and issues a network schedule to the nodes. 
A schedule consists of sequential time slots, each with a start time and duration, when the node will trigger entanglement generation.
Nodes are not forced to attempt entanglement in corresponding time slots, and can instead choose to do local processing instead.


\textit{Performance metrics and noise}. Quantum internet applications have classical outcomes that are typically probabilistic in nature:
(1) applications may intentionally do measurements on quantum states that have fundamentally probabilistic outcomes (e.g. quantum cryptography),
(2) in practice, quantum hardware is imperfect (or \textit{noisy}). That is, undesired errors occur
when performing operations (such as gates, measurements, or entanglement generation) or when keeping quantum states in memory for too long.

In many quantum internet applications (e.g. BQC), a single execution of the application can result in failure or success (e.g. a BQC client receives correct measurement results from the server program~\cite{leichtle2021verifying}). Applications are often executed many times, where outcome statistics are computed in order to validate successful execution (e.g. by majority of outcomes).
We consider two metrics:
a \emph{quantum metric} --- the \textit{success probability} of executing a single instance of the application (on average), and a \emph{classical metric} --- the \textit{makespan}, i.e. the average execution time of an application instance.

\subsection{Considerations}
Considerations can be categorized into three main groups: fundamental, technological, and enabling.

\noindent\textit{\textbf{Fundamental Considerations}}
(1) \textit{Hybrid nature of applications (FC1)}: Quantum internet applications inherently consist of both classical and quantum segments, as well as local and networking operations. The execution environment must account for this hybrid nature, and the program structure should accommodate all types of operations.
(2) \textit{Interactive nature of applications (FC2)}: Quantum internet applications require classical communication between nodes. This communication may take place in between classical and quantum segments of a single program. 
This implies the need for application-level interfaces between programs on different nodes, and for interfaces between classical and quantum code segments on a single node.
(3) \textit{Multitasking (FC3)}: Programs may spend a significant amount of time waiting for messages from a remote node (ms), motivating multitasking to make optimal use of the classical and quantum computing resources at each node.  This requires scheduling of time and resources.

\noindent\textit{\textbf{Technological Considerations}}
(1) \textit{Limited qubit lifetime (TC1)}:
Quantum memory quality degrades over time, presenting a significant challenge for the execution environment,
especially in near-term hardware (sub-millisecond to multiple seconds memory lifetimes~\cite{ruf2021quantum, pompili2021realization, krutyanskiy2023telecom}).
As such, there are natural deadlines to application execution after which a desired performance (success probability) can no longer be reached.
We thus desire that a program specification allows indication of memory quality constraints (deadlines), which the runtime environment can act upon (e.g. by appropriate scheduling or restarting).
(2) \textit{Integration of processing and networking (TC2)}: We assume that near-term nodes only have a single quantum processor, which needs to perform both local quantum gates as well as remote entanglement generation.
That is, while performing local operations the processor is blocked from networking operations and vice versa, as is the case for all current implementations~\cite{pompili2021realization, krutyanskiy2023entanglement} but may be mitigated partially using future proposals~\cite{vardoyan2022quantum}. 
The node must hence allocate time for local computation while at the same time adhering to the network schedule which constrains timing of the entanglement operations.

\noindent\textit{\textbf{Enabling Considerations}}
(1) \textit{Different compilation strategies and programming languages (EC1)}: The execution environment should support various compilation strategies and accessible programming languages. 
In order to enable compilation, we furthermore want a representation of the program that can be integrated with existing compiler frameworks.
(2) \textit{Different scheduling strategies (EC2)}: Since we expect that scheduling plays a vital role in optimizing application performance, the execution environment should enable scheduling, and support different scheduling algorithms and policies, 
allowing for their comparison and evaluation.
(3) \textit{Different (control) hardware implementations (EC3)}: The architecture should make minimal assumptions on the classical control hardware, and be independent on the choice of quantum hardware platform~\cite{donne2024design},
allowing for integration with multiple (future) technologies such as NV centers~\cite{pompili2022experimental} or trapped ions~\cite{drmota2023robust}.


\begin{figure*}[t]
\vskip 0.2in
\begin{center}
\centerline{
\includegraphics[width=\textwidth, height=9cm]{figures/architecture_img.pdf}}   
\vspace{-3mm}
\caption{\textbf{Overview of our method at the blending stage. }
% condition
Two input images or concepts are encoded into embeddings, mapped to a shared text space via the Linear Prior Converter from unCLIP~\citep{ramesh2022hierarchical}. These embeddings condition the U-Net: one for downsampling, the other for upsampling.
% module
During the blending stage, a blending latent $L_b$ initialized with Gaussian noise is processed in the Feedback Interpolation Module, conditioned on image embeddings. Noise $\epsilon$ is added to the embeddings to generate initial auxiliary latents, which are interpolated into $L^{(t)}_{b}$ with an increasing weight $p$. The  $L^{(t)}_{a}$ is combined with interpolated latent $L'^{(t)}_{b}$ by proportion $p$. All updated $L'^{(t)}_{a}$ are refined in the auxiliary inference to retain original features using the text prompt for corresponding categories, and $L'^{(t)}_{b}$ is denoised via the blending inference.
% refinement
Finally, the refined $ L_b $ is passed into the VAE decoder to generate the final blending image. 
}
\label{architecture}
\end{center}
% \vskip -0.4in
\vspace{-8mm}
\end{figure*}

\section{Implementation Environment}
\label{sec:implementation_environment}

Here we introduce the detailed implementation details and environment for reproducibility purpose. For our model, we choose hyperparameters based on the performance on validation set (Document classification task in the main paper explains how we split validation set). The results in the main paper are obtain by 5 independent runs. The standard deviations reported in the main paper are 1-sigma error bars and are obtained by calling its corresponding function in Excel library. All the experiments were done on Linux server with an NVIDIA A40 GPU with 46,068 MiB. Its operating system is CentOS Linux 7 (Core). We implemented our proposed model GTFormer using Python 3.10 as programming language and PyTorch 2.0.0 as deep learning library. Other frameworks include NumPy 1.23.1, sklearn 0.23.2, and scipy 1.5.2. We emphasize that the main focus of our model is effectiveness, instead of running efficiency. But for completeness, we still make a short comment on execution time. Our model is efficient, on the largest dataset Web, the training takes less than 40 hours to converge. We will release code and datasets upon publication.
\section{Evaluation}
We provide three sets of insights into this section, organised as \textit{findings (F*)}. We quantitatively study the effect of the adversarial and counterfactual perturbations on the performance of informal reasoners and autoformalisation methods. Then, we dive deeper into method variants. Finally, 
we analyse the nature of formalisation errors made by the models.

\subsection{Robustness Analysis}
\paragraph{\textbf{\emph{F1: Noise perturbations have a stronger effect on formalisation methods than informal \ac{LLM} reasoners.}}}
Table~\ref{tab:distraction_k4_formalisation} shows that, on average, the accuracy of both direct and \ac{CoT} informal reasoning remains between $73\%$ and $74\%$ in the face of added noise. While the autoformalisation method performs similarly to informal reasoners on the original dataset, its performance decreases between $4\%$ and $11\%$. The accuracy drops especially with logical (L) and tautological (T) distractions, whose logical language formats trick the \ac{LLM} into formalizing the noisy clauses. On the other hand, the linguistically complex and more natural sentences of encyclopedic distractions show a minor effect, suggesting that \acp{LLM} successfully avoids formalizing the more complicated sentences.

\paragraph{\textbf{\emph{F2: All \ac{LLM}-based reasoning methods suffer a drop for counterfactual perturbations.}}} % influence .}}}
Table~\ref{tab:distraction_k4_formalisation} shows that counterfactual statements cause a significant decrease in performance for both the informal reasoners and autoformalisation methods of between $12\%$ and $13\%$ on average. 
Moreover, this observation also holds for all tested models, i.e., none are robust towards counterfactual perturbations across every evaluated dimension. Even the strongest model, GPT 4o-mini, yields a performance of 63-68\%, which is relatively close to the random performance of 50\%. The high impact of counterfactual statements (the single ``not'' inserted) could be due to the inability of \acp{LLM} to overwrite prior knowledge with explicitly stated information or memorization of the answers. We study the error sources further in §\ref{subsec:errors}.  

\noindent \paragraph{\textbf{\emph{F3: Introducing multiple noise sentences has an effect only for logical distractions.}}}
We show the impact of introducing between one and four sentences for the two top-performing autoformalisation models in Figure~\ref{fig:length_distraction}. The figure shows similar trends with and without counterfactual perturbations.
As additional logical distractions are introduced, the model performance consistently decreases. Tautological (T) distractions lead to a decline in accuracy with a single disruptive sentence, yet adding more noise does not worsen the outcome. 
The tautological corpus introduces truth constants for all sentences as a persistent unseen logical construct. Given that this leads only to a decrease for a single occurrence, we can assume that a model can consistently handle the same unseen logical construct. In contrast, the logical corpus increases the chance of adding text, requiring new, previously unseen reasoning constructs for each added sentence. The impact of encyclopedic noise remains negligible, generalising F1 to $k$ sentences. Similarly, counterfactual perturbations remain much more effective for all settings, generalising F2.

\begin{table}[!t]
\small
\setlength{\modelspacing}{2pt}
\setlength{\tabcolsep}{1.7pt} % Default value: 6pt
\setlength{\belowrulesep}{4pt}
\begin{threeparttable}
    \centering
    \begin{tabular}{cc l r rrr @{\quad} rrrr}
\toprule
\multirow{2}{*}{} & \multirow{2}{*}{} & Reasoning & \multirow{2}{*}{O} & \multicolumn{3}{c}{Distraction} & \multicolumn{4}{c}{Counterfactual} \\
 & & Format & & E& L & T & $\text{O}_C$ & $\text{E}_C$& $\text{L}_C$ & $\text{T}_C$\\
\midrule
\multirow{6}{*}{\rotatebox{90}{Gemma-2}} & \multirow{3}{*}{\rotatebox{90}{9b}}
   & Informal (direct) & \textbf{0.78} & \textbf{0.80} & \textbf{0.79} & \textbf{0.77} & 0.58 & 0.52 & 0.50 & 0.59 \\
 & & Informal (CoT) & 0.72 & 0.78 & 0.73 & 0.76 & 0.61 & \textbf{0.57} & \textbf{0.60} & \textbf{0.66} \\
 & & Formal (FOL) & 0.62 & 0.58 & 0.52 & 0.53 & \textbf{0.63} & 0.52 & 0.46 & 0.46 \\[\modelspacing]
\cmidrule{2-11}
 & \multirow{3}{*}{\rotatebox{90}{27b}} 
   & Informal (direct) & 0.71 & 0.69 & \textbf{0.66} & \textbf{0.68} & 0.59 & 0.51 & 0.54 & 0.59 \\
 & & Informal (CoT) & 0.66 & 0.65 & 0.64 & 0.63 & 0.62 & 0.58 & \textbf{0.62} & \textbf{0.64} \\
 & & Formal (FOL) & \textbf{0.74} & \textbf{0.74} & 0.61 & 0.61 & \underline{\textbf{0.72}} & \underline{\textbf{0.67}} & 0.58 & 0.51 \\[\modelspacing]
\midrule
\multirow{6}{*}{\rotatebox{90}{Mistral}} & \multirow{3}{*}{\rotatebox{90}{7B}} 
   & Informal (direct) & 0.77 & \textbf{0.77} & 0.75 & \textbf{0.79} & \textbf{0.63} & \textbf{0.54} & \textbf{0.54} & \textbf{0.66} \\
 & & Informal (CoT) & \textbf{0.79} & 0.75 & \textbf{0.77} & 0.78 & 0.55 & 0.52 & \textbf{0.54} & 0.58 \\
 & & Formal (FOL) & 0.62 & 0.58 & 0.54 & 0.57 & 0.50 & \textbf{0.54} & 0.51 & 0.52 \\[\modelspacing]
\cmidrule{2-11}
 & \multirow{3}{*}{\rotatebox{90}{Small}} 
   & Informal (direct) & \textbf{0.77} & \textbf{0.76} & \textbf{0.76} & \textbf{0.75} & 0.61 & 0.51 & 0.56 & 0.59 \\
 & & Informal (CoT) & 0.72 & 0.72 & 0.72 & 0.71 & \textbf{0.62} & \textbf{0.59} & \textbf{0.62} & \textbf{0.68} \\
 & & Formal (FOL) & 0.68 & 0.59 & 0.53 & 0.64 & 0.54 & 0.55 & 0.49 & 0.51 \\[\modelspacing]
\midrule
\multirow{6}{*}{\rotatebox{90}{Llama-3.1}} & \multirow{3}{*}{\rotatebox{90}{8B}} 
   & Informal (direct) & 0.63 & 0.61 & 0.64 & 0.66 & 0.61 & \textbf{0.62} & 0.59 & 0.61 \\
 & & Informal (CoT) & 0.73 & \textbf{0.73} & \textbf{0.71} & \textbf{0.72} & \textbf{0.62} & 0.59 & \textbf{0.61} & \textbf{0.65} \\
 & & Formal (FOL) & \textbf{0.77} & 0.71 & 0.63 & 0.52 & 0.60 & 0.58 & 0.55 & 0.52 \\[\modelspacing]
\cmidrule{2-11}
 & \multirow{3}{*}{\rotatebox{90}{70B}} 
   & Informal (direct) & 0.77 & 0.74 & 0.74 & 0.73 & 0.62 & 0.53 & 0.56 & 0.64 \\
 & & Informal (CoT) & \textbf{0.78} & \textbf{0.75} & \textbf{0.76} & \textbf{0.76} & 0.64 & 0.61 & \textbf{0.66} & \underline{\textbf{0.73}} \\
 & & Formal (FOL) & 0.74 & 0.73 & 0.71 & 0.71 & \textbf{0.66} & \textbf{0.62} & 0.59 & 0.57 \\[\modelspacing]
 \midrule
\multirow{3}{*}{\rotatebox{90}{GPT}} & \multirow{3}{*}{\rotatebox{90}{4o-mini}} 
   & Informal (direct) & 0.78 & 0.77 & 0.79 & 0.79 & 0.64 & 0.61 & 0.61 & 0.63 \\
 & & Informal (CoT) & 0.80 & 0.80 & \underline{\textbf{0.81}} & \underline{\textbf{0.82}} & \textbf{0.68} & \textbf{0.63} & \underline{\textbf{0.68}} & \textbf{0.64} \\
 & & Formal (FOL) & \underline{\textbf{0.84}} & \underline{\textbf{0.82}} & 0.73 & 0.79 & 0.63 & 0.62 & 0.57 & 0.54 \\[\modelspacing]
 \midrule
\multicolumn{2}{c}{\multirow{3}{*}{\textbf{Avg}}} 
 & Informal (direct) & 0.74 & 0.73 & 0.73 & 0.73 & 0.61 & 0.55 & 0.56 & 0.62 \\
 & & Informal (CoT) & 0.74 & 0.74 & 0.73 & 0.74 & 0.62 & 0.58 & 0.62 & 0.65 \\
  & & Formal (FOL) & 0.72 & 0.68 &	0.61 & 0.62 & 0.61 & 0.59 & 0.54 & 0.52 \\
\bottomrule
\end{tabular}
\caption{Accuracies of informal and autoformalisation-based deductive reasoners. The best overall model per dataset is underlined; the best model version is marked in bold.}
\label{tab:distraction_k4_formalisation}
\end{threeparttable}
\end{table} 

\begin{figure}[!t]
    \centering
    \scriptsize
    \begin{tikzpicture}
        \begin{axis}[name=gpt,
            title={GPT-4o-mini},
            width=0.6\linewidth,
            height=0.6\linewidth,
            xlabel={\# Noise sentences},
            ylabel={Accuracy},
            xmin=-0.1, xmax=4.1,
            ymin=0.5, ymax=0.9,
            xtick={1,2,4},
            ytick={0.55, 0.6, 0.65, 0.75, 0.8, 0.85},
            title style={yshift=-0.6em},
            legend style={at={(1,-0.15)},
	           anchor=north,legend columns=-1},
            x label style={at={(axis description cs:1,-0.05)},anchor=north},
            y label style={at={(axis description cs:-0.15,0.5)},anchor=south},
            ymajorgrids=true,
            grid style=dashed,
        ]
            \addplot[color=blue, mark=square,]
                coordinates {
                (0,0.848076939582825)(1,0.823076903820038)(2,0.826923072338104)(4,0.821153819561005)
                };
            \addplot[color=red, mark=triangle,]
                coordinates {
                (0,0.848076939582825)(1,0.817307710647583)(2,0.801923096179962)(4,0.759615361690521)
                };
            \addplot[color=green, mark=diamond,] 
                coordinates {
                (0,0.848076939582825)(1,0.767307698726654)(2,0.769230782985687)(4,0.803846180438995)
                };
            \addplot[color=blue, mark=square*] 
                coordinates {
                (0,0.627777755260468)(1,0.622222244739533)(2,0.600000023841858)(4,0.633333325386047)
                };
            \addplot[color=red, mark=triangle*,] 
                coordinates {
                (0,0.627777755260468)(1,0.611111104488373)(2,0.611111104488373)(4,0.594444453716278)
                };
            \addplot[color=green, mark=diamond*,] 
                coordinates {
                (0,0.627777755260468)(1,0.572222232818604)(2,0.538888871669769)(4,0.555555582046509)
                };
                \legend{E,L,T,$\text{E}_C$, $\text{L}_C$ , $\text{T}_C$}
        \end{axis}

        \begin{axis}[name=llama, at={($(gpt.east)+(0.1cm,0)$)},anchor=west,
            title={Llama 3.1 70b},
            width=0.6\linewidth,
            height=0.6\linewidth,
            xmin=-0.1,, xmax=4.1,
            ymin=0.5, ymax=0.9,
            xtick={1,2,4},
            ytick={0.55, 0.6, 0.65, 0.75, 0.8, 0.85},
            title style={yshift=-0.6em},
            yticklabel=\empty,
            ymajorgrids=true,
            grid style=dashed,
        ]
            \addplot[color=blue, mark=square,]
                coordinates {
                (0,0.838461518287659)(1,0.817307710647583)(2,0.805769205093384)(4,0.817307710647583)
                };
            \addplot[color=red, mark=triangle,]
                coordinates {
                (0,0.838461518287659)(1,0.819230794906616)(2,0.803846180438995)(4,0.771153867244721)
                };
            \addplot[color=green, mark=diamond,]
                coordinates {
                (0,0.838461518287659)(1,0.803846180438995)(2,0.807692289352417)(4,0.805769205093384)
                };
            \addplot[color=blue, mark=square*]
                coordinates {
                (0,0.627777755260468)(1,0.622222244739533)(2,0.577777802944183)(4,0.594444453716278)
                };
            \addplot[color=red, mark=triangle*,]
                coordinates {
                (0,0.627777755260468)(1,0.583333313465118)(2,0.561111092567444)(4,0.577777802944183)
                };
            \addplot[color=green, mark=diamond*,]
                coordinates {
                (0,0.627777755260468)(1,0.627777755260468)(2,0.566666662693024)(4,0.577777802944183)
                };
        \end{axis}
    \end{tikzpicture}
    \caption{Influence of the number of noisy sentences for FOL.}
    \label{fig:length_distraction}
\end{figure}



\subsection{Impact of Method Design}
\paragraph{\textbf{\emph{F4: \ac{CoT} prompting is most impactful when both noise and counterfactual perturbations are applied.}}}
The accuracies for the individual \acp{LLM} in Table~\ref{tab:distraction_k4_formalisation} show that the impact of \ac{CoT} is negligible for noise-only datasets (first four columns). Meanwhile, the benefit from \ac{CoT} is most pronounced in the datasets that combine noise and counterfactual perturbations.
The better-performing informal prompting strategy for a model remains stable for all types of distractions. Still, the decline in performance due to counterfactuals leads to a less consistent preference for a specific prompting style.

\paragraph{\textbf{\emph{F5: The best-performing grammar differs per model and is unstable across data versions.}}}

The evaluation of different logical forms for formal \ac{LLM}-based reasoning in Table~\ref{tab:distraction_k4_logical_form} shows the preference of some models for specific syntactic formats.
Llama 3.1 70B has a considerable improvement of $12\%$ with TPTP syntax on the original set, while Llama 3.1 8B benefits from the R-FOL syntax. However, all grammars show a declining accuracy trend and increased syntax errors for noise perturbations, where the best grammar loses its advantage over the rest. 
When comparing the grammars on the counterfactual partitions, we observe that TPTP is consistently more robust than the standard first-order logic grammar. Here, GPT 4o-mini shows a reduction from $O$ to $O_C$ of $20\%$ for FOL and only $12\%$ for the TPTP grammar. Since this does not correlate with fewer syntax errors, the formalisation in TPTP prevents semantical errors for counterfactual premises. 
A positive reading of these results, especially the minor differences between FOL and R-FOL, is that autoformalisation \acp{LLM} can adapt to the grammar syntax prescribed in the prompt without further loss in performance.

\begin{table}[!t]
\small
\setlength{\modelspacing}{2pt}
\setlength{\tabcolsep}{1.7pt} % Default value: 6pt
\setlength{\belowrulesep}{4pt}
\begin{threeparttable}
    \centering
    \begin{tabular}{cc l r rrr @{\quad} rrrr}
\toprule
\multirow{2}{*}{} & \multirow{2}{*}{} & Grammar & \multirow{2}{*}{O} & \multicolumn{3}{c}{Distraction} & \multicolumn{4}{c}{Counterfactual} \\
 & & Syntax & & E& L & T & $\text{O}_C$ & $\text{E}_C$& $\text{L}_C$ & $\text{T}_C$\\
\midrule
\multirow{6}{*}{\rotatebox{90}{Llama-3.1}} & \multirow{3}{*}{\rotatebox{90}{8B}} 
   & FOL & 0.77 & \textbf{0.71} & 0.61 & \textbf{0.53} & 0.58 & \textbf{0.55} & 0.52 & \textbf{0.56} \\
 & & R-FOL & \textbf{0.78} & 0.69 & \textbf{0.62} & \textbf{0.53} & 0.58 & \textbf{0.55} & \textbf{0.54} & 0.52 \\
 & & TPTP & 0.73 & 0.67 & 0.55 & 0.51 & \textbf{0.68} & 0.54 & 0.46 & 0.51 \\[\modelspacing]
\cmidrule{2-11}
 & \multirow{3}{*}{\rotatebox{90}{70B}} 
   & FOL & 0.76 & 0.73 & 0.71 & \textbf{0.72} & 0.67 & 0.57 & 0.63 & 0.56 \\
 & & R-FOL & 0.76 & 0.73 & 0.67 & 0.71 & 0.64 & 0.57 & 0.53 & 0.64 \\
 & & TPTP & \underline{\textbf{0.88}} & \underline{\textbf{0.84}} & \underline{\textbf{0.81}} & \textbf{0.72} & \underline{\textbf{0.81}} & \underline{\textbf{0.68}} & \underline{\textbf{0.67}} & \underline{\textbf{0.68}} \\[\modelspacing]
\midrule
\multirow{3}{*}{\rotatebox{90}{GPT}} & \multirow{3}{*}{\rotatebox{90}{4o-mini}} 
   & FOL & \textbf{0.84} & \textbf{0.82} & \textbf{0.72} & \underline{\textbf{0.78}} & 0.64 & \textbf{0.63} & \textbf{0.61} & 0.51 \\
 & & R-FOL & \textbf{0.84} & 0.77 & 0.70 & \underline{\textbf{0.78}} & \textbf{0.72} & 0.56 & 0.54 & \textbf{0.63} \\
 & & TPTP & 0.83 & \textbf{0.82} & 0.71 & 0.71 & 0.69 & \textbf{0.63} & 0.57 & 0.57 \\
\bottomrule
\end{tabular}
\caption{Accuracies of different formalisation grammars for autoformalisation.}
\label{tab:distraction_k4_logical_form}
\end{threeparttable}
\end{table} 

\paragraph{\textbf{\emph{F6: Feedback does not help \acp{LLM} self-correct to mitigate robustness issues.}}}
\autoref{tab:distraction_k4_feedback} shows the results with different error recovery mechanisms. The results indicate that no feedback strategy emerges as a winner in the different datasets. 
All feedback variants reduce syntax errors for noise perturbations, but given the lack of a consistent increase in accuracy, the corrected formalisations are most likely to contain semantic errors still. 
The type of feedback message only has a minor influence on correcting syntax errors, whereas Llama 3.1 70b and GPT 4o-mini correct slightly more syntax errors with specific error messages. This finding aligns with \cite{huang2023large}, who also found that \acp{LLM} cannot consistently self-correct their reasoning after receiving relevant feedback.

\begin{table}[!ht]
\small
\setlength{\modelspacing}{2pt}
\setlength{\tabcolsep}{1.7pt} % Default value: 6pt
\setlength{\belowrulesep}{4pt}
\begin{threeparttable}
    \centering
    \begin{tabular}{cc l r rrr @{\quad} rrrr}
\toprule
\multirow{2}{*}{} & \multirow{2}{*}{} & \multirow{2}{*}{Feedback} & \multirow{2}{*}{O} & \multicolumn{3}{c}{Distraction} & \multicolumn{4}{c}{Counterfactual} \\
 & & & & E& L & T & $\text{O}_C$ & $\text{E}_C$& $\text{L}_C$ & $\text{T}_C$\\
\midrule
\multirow{8}{*}{\rotatebox{90}{Llama-3.1}} & \multirow{4}{*}{\rotatebox{90}{8B}} 
   & No recovery & 0.77 & \textbf{0.72} & 0.62 & 0.53 & 0.59 & 0.58 & 0.56 & \textbf{0.56} \\
 & & Error type & \textbf{0.79} & 0.71 & 0.63 & \textbf{0.56} & \textbf{0.66} & 0.54 & 0.52 & 0.51 \\
 & & Error message & 0.78 & 0.71 & \textbf{0.67} & 0.55 & 0.59 & 0.53 & \underline{\textbf{0.64}} & 0.49 \\
 & & Warning & 0.74 & 0.66 & 0.58 & 0.55 & 0.55 & \textbf{0.60} & 0.49 & 0.49 \\[\modelspacing]
\cmidrule{2-11}
 & \multirow{4}{*}{\rotatebox{90}{70B}} 
   & No recovery & \textbf{0.77} & \textbf{0.72} & \textbf{0.73} & 0.71 & \textbf{0.64} & 0.59 & \textbf{0.61} & 0.56 \\
 & & Error type & 0.72 & 0.70 & 0.72 & \textbf{0.73} & 0.62 & 0.56 & 0.60 & 0.58 \\
 & & Error message & 0.71 & 0.70 & \textbf{0.73} & 0.71 & \textbf{0.64} & 0.59 & 0.54 & \underline{\textbf{0.64}} \\
 & & Warning & 0.69 & \textbf{0.72} & 0.72 & 0.72 & 0.62 & \underline{\textbf{0.65}} & \textbf{0.61} & 0.63 \\[\modelspacing]
\midrule
\multirow{4}{*}{\rotatebox{90}{GPT}} & \multirow{4}{*}{\rotatebox{90}{4o-mini}} 
   & No recovery & \underline{\textbf{0.84}} & \underline{\textbf{0.82}} & 0.73 & 0.79 & 0.64 & \textbf{0.62} & 0.56 & \textbf{0.56} \\
 & & Error type & 0.83 & 0.79 & 0.74 & 0.76 & 0.67 & 0.57 & 0.56 & \textbf{0.56} \\
 & & Error message & \underline{\textbf{0.84}} & 0.78 & \underline{\textbf{0.77}} & \underline{\textbf{0.80}} & 0.62 & 0.59 & 0.56 & \textbf{0.56} \\
 & & Warning & \underline{\textbf{0.84}} & 0.75 & 0.73 & 0.76 & \underline{\textbf{0.70}} & 0.61 & \textbf{0.61} & 0.55 \\
 \bottomrule
\end{tabular}
\caption{Accuracies of error recovery strategies.}
\label{tab:distraction_k4_feedback}
\end{threeparttable}
\end{table} 

\subsection{Error Analysis}
\label{subsec:errors}
\paragraph{\textbf{\emph{F7: Autoformalisation increases syntax errors for noise perturbations.}}}
The low performance for noise perturbations correlates with more syntax errors for all models and distraction categories (cf. execution rates in Table~\ref{tab:appendix_k4_formalisation_exec}). The three worst-performing models (both Mistral models, Gemma-2 9b) generate, at best, for $37\%$  and, at worst, for only $4\%$ of the samples, a valid logical form.
Gemma-2 9b and Llama3.1 8b produce more syntax errors than the larger counterparts, suggesting that larger models are more robust towards noise perturbations. 
The accuracy of syntactically valid samples is higher than the informal reasoning methods for most distractions (Table~\ref{tab:appendix_k4_formalisation_vacc}), motivating informal reasoning as a backup strategy for formal reasoning. The error message feedback reveals two common syntax errors: 1) errors by models with an initial low execution rate exhibit issues with the template structure, including using incorrect keywords or adding conversational phrases;
2) perturbation-related errors, the most common of which is using undefined truth constants as part of tautological distractions. 

\paragraph{\textbf{\emph{F8: Autoformalisation increases semantic errors for counterfactuals.}}}
Unlike the introduced noise, counterfactual perturbations do not lead to more syntax errors. The execution rate in Table~\ref{tab:appendix_k4_formalisation_exec} is stable or improves for counterfactuals. However, we see a drop in accuracy for the counterfactual column $\text{O}_C$ in Table~\ref{tab:distraction_k4_formalisation} and can conclude that the number of logical forms with semantic errors has to increase. This suggests that the introduced negation is not correctly formalised. Looking at the warnings generated by the feedback mechanism, for GPT 4o-mini, $161$ warning messages are generated on the unperturbed data. $54$ of these were fixed with a single iteration. Not considering predicates and individuals as part of the context is the most frequent warning across all models. 
\section*{Conclusion}
This paper aims to enhance our understanding of the computational complexity of computing various Shapley value variants. We found that for various ML models --- including decision trees, regression tree ensembles, weighted automata, and linear regression --- both local and global interventional and baseline SHAP can be computed in polynomial time under HMM modeled distributions. This extends popular algorithms, such as TreeSHAP, beyond their empirical distributional scope. We also establish strict complexity gaps between the various SHAP variants (baseline, interventional, and conditional) and prove the intractability of computing SHAP for tree ensembles and neural networks in simplified scenarios. Overall, we present SHAP as a versatile framework whose complexity depends on four key factors: \begin{inparaenum}[(i)] \item model type, \item SHAP variant, \item distribution modeling approach, \item and local vs. global explanations\end{inparaenum}. We believe this perspective provides deeper insight into the computational complexity of SHAP, paving the way for future work.




%We believe that our framework provides a more intricate understanding of SHAP computation complexity across different models, distributions, and variants, paving the way for further research.

Our work opens promising directions for future research. First, expanding our computational analysis to other SHAP-related metrics, such as asymmetric SHAP~\citep{frye20} and SAGE~\citep{covert2020understanding}, would be valuable. Additionally, we aim to explore more expressive distribution classes and relaxed assumptions beyond those in Section \ref{sec:tractable} while maintaining tractable SHAP computation. Finally, when exact computation is intractable (Section \ref{sec:intractable}), investigating the approximability of SHAP metrics through approximation and parameterized complexity theory~\citep{downey2012parameterized} is an important direction.

%Our work opens several promising avenues for future research on the computational properties of explainable AI methods, with a particular focus on SHAP. First, it would be interesting to broaden the computational analysis conducted in this work to include other popular SHAP-related metrics in the literature, such as asymmetric SHAP \cite{frye20} and SAGE \cite{covert2020understanding}. Also, in the future, we aim to explore more expressive distribution classes and relaxed distributional assumptions—extending beyond those examined in Section \ref{sec:tractable} —that still yield tractable SHAP computation. Finally, when exact computation proves intractable (Section \ref{sec:intractable}), it is worthwhile to theoretically investigate the question of the approximability of computing the SHAP metrics across various configurations, through the lens of approximation and parametrized complexity theory \cite{arora2009computational}.

%This paper aims to deepen our understanding of the computational complexity involved in obtaining different Shapley value variants. We found that for a variety of ML models, including decision trees, tree ensembles for regression, weighted automata, and linear regression models — computing both local and global interventional and baseline SHAP can be done in polynomial time when distributions are modeled by HMMs. This extends the distributional scope of popular algorithms like TreeSHAP, which is limited to empirical distributions. Additionally, we demonstrate a strict complexity gap between SHAP variants, showing that interventional and baseline SHAP can be strictly easier to compute than conditional SHAP. Despite these positive results, we uncovered intractability for various SHAP variants in neural networks and tree ensembles. Finally, we provided generalized complexity relations across SHAP variants. We believe that our framework offers a deeper understanding of the complexity involved in computing SHAP across various variants, models, distributions, as well as in both local and global computations, laying the groundwork for future research.

\section{Acknowledgements}
This research was supported by the Quantum Internet Alliance through the European Union's Horizon 2020 program under grant agreement No. 820445 
and from the Horizon Europe program grant agreement No. 101080128.
We furthermore acknowledge support from NWO (including a VICI grant).

We thank 
Przemysław Pawełczak,
Michele Amoretti,
Anabel Ovide,
Thomas Beauchamp,
Álvaro Gomez Iniesta,
Ingmar te Raa,
Diego Rivera and
Francisco Silva
for useful discussions and feedback on (early) drafts.


\section{Data availability}
The implementation of Qoala as a simulator can be found online~\cite{qoala2023simulator}.
The code and data supporting the evaluation can be found at~\cite{evaluation-data}.

\bibliographystyle{unsrt}
\bibliography{qoala-paper}


% \clearpage
\appendices

\onecolumn

\section*{Appendix}
Appendix containing additional information for the paper \textbf{Qoala: an Application Execution Environment for Quantum Internet Nodes}.
This Appendix is structured as follows.
\begin{itemize}
\item \cref{app:program_structure} provides details of the \textbf{Qoala program format}.
\item \cref{app:runtime_environment} provides details of the \textbf{Qoala runtime environment}.
\item In \cref{app:scheduling_execution}, our \textbf{scheduler implementation} (\cref{sec:implementation}) is explained more in-depth.
\item \cref{app:simulator} gives an overview of our \textbf{simulator implementation}.
\item In \cref{app:evaluation} we elaborate on our \textbf{evaluations}.
\end{itemize}


\section{Program structure}
\label{app:program_structure}
This section provides details about the structure and contents of Qoala programs as described in \cref{sec:program_structure}.


\subsection{Program representation and components}
A Qoala program is represented in human-readable text format.
This allows one to directly write Qoala programs, although our vision is that programmers write their code in a higher-level language, and that a compiler translates this into a Qoala program.

In the main text, some parts of example programs were omitted for brevity.
In \cref{fig:app:example_full_program} we show an example of a full Qoala program.

A Qoala program encompasses both classical and quantum code.
These different code segments are put into different sections in the program.
The host section contains QoalaHost code which is to be run on the CPS.
The NetQASM section contains local routines (containing NetQASM instructions) which are meant to be run on the QPS.
The request section contains specifications of requests for remote entanglement generation, to be handled by the QPS.
Furthermore, there is a meta section which defines global information about the program.
Each of these sections is explained in more detail below.

In all of the sections in a Qoala program, values may be replaced by a \textbf{template}.
A template represents a value that is not defined for the program, but is filled in at program instantiation. For example, a QKD program might have a request object in its request section containing the entry \texttt{num\_pairs: {N}}, where \texttt{{N}} is a template. This construction allows one to instantiate the same program with different values for \texttt{N}, and it is hence not needed to define separate programs for each different number of pairs to generate in the QKD program.

\subsection{Program Meta}
Program metadata contains:

\begin{itemize}
    \item \textbf{Name}: The name of this program.
    \item \textbf{Parameters}: Global arguments to this program. These arguments may be used as templates (see above) in the program. Examples may be the name of a remote node, or the number of EPR pairs to generate.
    \item \textbf{Classical Sockets}: A mapping from IDs to remote node names. The IDs are local identifiers that can be used by Host code to distinguish different classical sockets.
    \item \textbf{EPR Sockets}: A mapping from IDs to remote node names. The IDs are local identifiers that can be used by Host code to distinguish different EPR sockets.
\end{itemize}




\begin{figure*}[htbp]
    \centering
    \begin{minipage}{\textwidth}
        \lstinputlisting[language=qoala, caption={}]{sections/appendix/full_program.iqoala}
    \end{minipage}
    \caption{Example Qoala program which creates an EPR pair with remote program Alice, measures the local qubit, and returns the classical outcome value.
    \textit{Meta section.} The meta section defines the name of this program, the global arguments (input values, in this case: the node ID of the Alice program),
    the classical sockets used (mapping local socket ID to the name of the remote node, and the EPR sockets used (also mapping local socket ID to remote node names)).
    \textit{Host section.} This example host section consists of three blocks (\texttt{b1, b2, b3}). \texttt{b1} calls request routine \texttt{req} (no result values).
    \texttt{b2} calls local routine \texttt{post\_epr}, resulting in a classical vector with one value (\texttt{m0}).
    \texttt{b3} returns \texttt{m0} as the result of this program.
    \textit{Local routines section.} Consists of a single local routine called \texttt{post\_epr}.
    It requires the virtual qubit (see \cref{app:runtime_environment}) with ID 0 to be allocated, and acts on this qubit.
    Upon finishing the local routine, this qubit is not in use anymore (the \texttt{keeps} entry is empty).
    The NetQASM code represents measuring the qubit, and then storing the result (in register \texttt{M0}) to the \texttt{@output} array (see \cref{app:runtime_environment}), which is in shared memory and can be accessed by host code by the name \texttt{m0}.
    \textit{Request routines section.} Consists of a single request routine called \texttt{req}.
    It represents a request to the network stack for generating a single entangled pair (\texttt{num\_pairs} is 1), which is kept in memory (\texttt{typ: create\_keep}; not measured immediately).
    This program acts as a `receiver' for entanglement generation (\texttt{role} attribute), which breaks symmetry in the entanglement generation process (the remote Alice program will have \texttt{role: sender}). Symmetry breaking is needed for the network stack to organize the entanglement generation.
    No callbacks are used, and all qubits (in this case: one) are stored in virtual qubit 0.
    }
    \label{fig:app:example_full_program}
\end{figure*}



\subsection{Host section}
The host section contains code the be executed by the CPS.
It consists of both local processing (like calculation and conditional logic), and
communication (sending and receiving classical messages to and from other nodes in the network).

The language in which host code is represented is called QoalaHost.
This is a low-level instruction set with well-defined semantics and types,
and is meant to be executed by a virtual machine or interpreter.
One can also imagine QoalaHost code to be translated (either ahead-of-time or at just-in-time) to native CPS code, such as x86 or ARM. However, for the sake of simplicity and of implementation independence, we treat here only the QoalaHost language and its semantics itself.

The QoalaHost (QH) language was designed to resemble intermediate representations as found in LLVM~\cite{lattner2004llvm} and MLIR~\cite{lattner2021mlir},
such that integration with future compilers is accessible.
Specifically, one may imagine a compiler that uses MLIR for its intermediate representation (IR).
When this compiler then produces the host code of the program, the translation of its own IR to QoalaHost code should be straightforward.

\textbf{Blocks.} 
The Host section consists of a list of blocks.
A block consists of a block metadata and a list of QH instructions.

The block metadata contains the following entries:
\begin{itemize}
\item \textbf{Name}: The name of this block. Host code can refer to this name in QH branch instructions.
\item \textbf{Type}: one of CL, CC, QL or QC (see below).
\item \textbf{Deadlines}: Deadlines relative to other blocks.
The deadlines are specified in terms of EHI arguments. Upon program instantiation, concrete values are filled in based on the actual EHI value.
\item \textbf{Time hints}: Duration estimate of executing the block.
The estimates are specified in terms of EHI arguments. Upon program instantiation, concrete values are filled in based on the actual EHI value.
\end{itemize}


\subsection{Block types}
Blocks are categorized into the following four types:
\begin{itemize}
\item \textbf{CL}: Classical Local. The block contains only instructions that are classical, local and only involve the CPS
\item \textbf{CC}: Classical Communication. CPS-only instructions, but starts with a `receive message` instruction.
\item \textbf{QL}: Quantum Local. The block contains calls to local routines.
\item \textbf{QC}: Quantum Communication. The block contains calls to request routines.
\end{itemize}


\textbf{QoalaHost Language.}
The QH Language describes a fixed set of QH instructions as well as QH Variable types.
Host code is represented as blocks containing QH instructions.
These instructions may be directly interpreted by a processor or OS.

All basic values are 32-bit signed integers (i32) or floating point values (f32).
A variable in Host code can either be
\begin{itemize}
\item singleton variable, holding one basic value. Has a single name. E.g. \texttt{x}
\item vector, holding an arbitrary number of basic values. Has a single name. E.g. \texttt{x<>}
\end{itemize}

The QH Language allows for expressing multiple variables in a single expression, called a \textit{tuple}.
A tuple holds a fixed number of basic values. E.g. \texttt{tuple<x, y, z>}.

\textbf{Local Memory.}
Host code is assumed to have access to a local memory space that is logically organized as a mapping of \textit{names} to \textit{values}.
For example, the local memory may at some point during execution contain the following items:

\begin{lstlisting}
"var_x" -> 3
"my_vec" -> <1, 2, 5>
\end{lstlisting}


\textbf{Shared Memory.}
The QH Language does not allow direct access to shared memory.
Only variables from the local memory can be used.
When calling and getting results from Local Routines (LRs) and Request Routines (RRs), values are automatically
moved from local memory to shared memory. 
Shared memory is discussed in more detail in \cref{app:shared_memory}.

\subsubsection{Block format}

A block has the following format:
\begin{qoalacode}
(*@\textcolor{purple}{\textasciicircum \#name}@*) {type = #type}:
    <list of QH instructions>
\end{qoalacode}

Example:
\begin{qoalacode}
(*@\textcolor{purple}{\textasciicircum b0}@*) {type = CL}:
    x = assign() : 3
    return_result(x)
\end{qoalacode}


\begin{figure*}
    \centering
    \includegraphics[width=\textwidth]{figures/qh_table.pdf}
    \caption{Overview of all host code (QoalaHost) instructions, their syntax and their semantics.}
    \label{fig:app:qh_table}
\end{figure*}

\subsubsection{QH instructions}
A full list of QoalaHost instructions is given in \cref{fig:app:qh_table}.


\subsection{NetQASM section}
\label{app:netqasm}
The NetQASM section consists of a list of local routines that are to be executed on the QPS.
A local routine is only executed when it is called by host code using the \texttt{run\_routine} instruction. A local routine may be run multiple times, again depending on the host code.

The instructions of a local routine are represented using the NetQASM 2.0 format.
This is an updated format compared to NetQASM 1.0 as presented in~\cite{dahlberg2022netqasm}.

\begin{figure*}
    \centering
    \includegraphics[width=\textwidth]{figures/netqasm_table.pdf}
    \caption{Overview of all NetQASM classical instructions, their syntax and their semantics.
    Quantum instructions depend on the particular flavour~\cite{dahlberg2022netqasm} that is being used.
    Semantics of quotient and remainder for non-positive integers are the same as in the C language standard.
    Note that \texttt{jmp 1} is a no-op.}
    \label{fig:app:netqasm_table}
\end{figure*}


\textbf{NetQASM values.}
All values are 32-bit signed integers. Floating-point values are not supported. Angles for qubit rotations must be expressed as discrete values.
Booleans are represented as follows: \texttt{true} is the 32-bit 0 value, \texttt{false} is the 32-bit 1 value. Any other 32-bit value is not a valid boolean.
The reason for keeping the different types limited is to keep the QPS implementation simple.


\textbf{NetQASM Local Memory}
The QPS is expected to have a local memory (only accessible by the QPS itself) consisting of 64 32-bit registers:
\begin{itemize}
\item 16 \textbf{R} registers: \texttt{R0} to \texttt{R15}
\item 16 \textbf{C} registers: \texttt{C0} to \texttt{C15}
\item 16 \textbf{M} registers: \texttt{M0} to \texttt{M15}
\item 16 \textbf{Q} registers: \texttt{Q0} to \texttt{Q15}
\end{itemize}

The four groups of registers are not inherently different. A compiler producing NetQASM code may use a certain group only for certain values, but this is not mandatory.

\textbf{Shared Memory}
See \cref{app:shared_memory} for more information about Shared Memory and arrays.
The QPS is expected to have access to Shared Memory (accessible by both the CPS and QPS).
Two shared memory Arrays are available:
\begin{itemize}
\item an \texttt{@input} array, containing the LR input variables
\item an \texttt{@output} array, with space to write the LR results to
\end{itemize}

The length of the \texttt{@input} array is equal to the number of LR parameters.
The length of the \texttt{@output} array is equal to the number of LR return variables.

\begin{itemize}
\item The \texttt{@input} and \texttt{@output} arrays are the only arrays accessible from within the LR.
\item The QPS can \textbf{only read} from the \texttt{@input} array (see \texttt{load} instruction below).
\item The QPS can \textbf{only write} to the \texttt{@output} array (see \texttt{store} instruction below).
\end{itemize}

\textbf{NetQASM Instruction}
Each instruction consists of the instruction type followed by a list of operands.
The text form of an instruction is:

\begin{lstlisting}
instr_name  op0 op1 ... opn
\end{lstlisting}

where the number of operands can be 0 or more (no limit).

A list of all NetQASM 2.0 instructions can be found in \cref{fig:app:netqasm_table}.

These instructions can be classified as:
\begin{itemize}
\item shared memory access: \texttt{load} for reading LR inputs, \texttt{store} for writing LR results
\item classical logic and control-flow: like \texttt{set} , \texttt{add}, or \texttt{jmp}
\item quantum operations: gates from a specific flavour~\cite{dahlberg2022netqasm}
\end{itemize}

NetQASM instructions representing quantum operations are either \textit{core instructions} or \textit{flavour-specific} instructions.
Core instructions are quantum hardware independent and are expected to be compatible with any QPS implementation. On top of the core instructions, flavour-specific instructions may be added and supported by a specific QPS implementation. For example, a QPS that controls an NV-centre may support NetQASM instructions of the NV flavour, which contain gate operations only available on this particular quantum hardware. Which NetQASM instructions are supported by the QPS is exposed to higher layers (including a compiler) as part of the EHI (see \cref{app:ehi}). Using this information, a compiler may produce optimized NetQASM code using the flavour-specific NetQASM instructions.


Note that NetQASM 2.0 \textbf{does not} contain (in contrast to NetQASM 1.0~\cite{dahlberg2022netqasm}):
\begin{itemize}
\item Allocation instruction (\texttt{qalloc} in NetQASM 1.0): The memory manager allocates virtual qubits based on the LR header information. Note that qubit allocation is different from \textit{qubit initialization} (\texttt{init} instruction).
\item Instructions for EPR generation: This is handled by request routines.
\item Waiting instructions: Waiting is handled by the scheduler choosing which tasks to execute when.
\end{itemize}

\textbf{Local Routine}
A Local Routine (LR) represents a block of local program operations that are executed on the QPS. An LR is:
\begin{itemize}
\item local: there is no interaction whatsoever with external nodes or controllers
\item atomic: execution of an LR cannot be pre-empted; when the QPS start executing an LR, it will not do anything else until the LR has finished (unless an abort happens)
\end{itemize}

An LR consists of a \textit{header} and a \textit{body}. The header contains metadata such as the resource usage of the LR, and its input/output interface. The body contains the actual instructions in the form of NetQASM code.


\textbf{Arguments and Returns.}
An LR may have zero or more \textit{arguments}: values that are provided to the LR only at runtime.
They can be seen as inputs or parameters to the LR.
These values appear in the \texttt{@input} array in shared memory, and are put there by the CPS.

An LR may also have zero or more \texttt{returns}: values that are provided by the LR only at runtime.
They can be seen as outputs or results of the LR.
These values must be written to the \texttt{@output} array in shared memory, and can then be used by the CPS.

Arguments and returns are always 32-bit signed integers. There is no limit to the number of arguments and returns an LR may have.

\textbf{Local routine header.}
A Local routine (LR) header contains the following entries:
\begin{itemize}
\item \textbf{Name}: The name of this LR. Host code refers to this name in a \texttt{run\_routine} QoalaHost instruction.
\item \textbf{Uses}: A list of virtual qubits IDs. These refer to all virtual qubits that are used by this LR. At runtime, the memory manager makes sure that these virtual qubits are allocated before execution of the LR starts. (They may already have been allocated earlier; alternatively the memory manager allocates them just before the LR starts.)
\item \textbf{Keeps}: A list of virtual qubit IDs. These refer to all virtual qubits that should \textit{remain allocated} after finishing the LR. (They may e.g. be used in subsequent LRs.)
\item \textbf{Args}: A list of names for the arguments of the LR. They are in the same order as how their values are accessible from the \texttt{@input} Array.
\item \textbf{Returns}: A list of names for the returns of the LR. They are in the same order as how their values are put into the \texttt{@output} Array.
\end{itemize}

\textbf{Quantum memory usage annotations.}
The LR header indicates which virtual qubits are used and freed by the LR. This makes it possible for the scheduler to decide which \texttt{LocalRoutine} task it may schedule when. For more information, see section \cref{app:scheduling_execution} on scheduling.
The following listing provides an example:

\begin{qoalacode}
SUBROUTINE subrt1
    uses: 0, 1
    keeps: 0
    returns: m0
    <rest omitted>
  NETQASM_START
    set Q0 0
    set Q1 1
    init Q0
    init Q1
    cnot Q0 Q1
    meas Q1 M1
    store M1 @output[0]
  NETQASM_END
\end{qoalacode}
This local routine initializes virtual qubits 0 and 1 and then applies a CNOT gate on them.
It measures qubit 1 and stores the output in the \texttt{@output} array which can then be accesses by host code using the name \texttt{m0}.
Using the metadata, a scheduler knows the following information even before executing this LR: virtual qubits 0 and 1 need to be free before this LR can run, and after running the LR, qubit 1 is free (again) but qubit 0 remains occupied.

It is the responsibility of the compiler to make sure that the use and free values correspond to the actual NetQASM code.


\subsection{Request section}
The callback (which is an LR) can have zero or more arguments (just like standard LRs). The runtime values of these arguments are provided by the QPS directly.
A Request Routine (RR) may have zero or more returns: outputs or results of the entire RR. The only allowed results at this moment are measurement outcomes in case of Measure Directly requests.
RR callbacks can have (just like standard LRs) zero or more returns.


\textbf{Request routine header}
A Request routine (RR) header contains the following entries:
\begin{itemize}
\item \textbf{Name}: The name of this RR. Host code refers to this name in a \texttt{run\_request}.
\item \textbf{Returns}: A list of names for the returns of the RR. Since the returns can only be measurement outcomes, these names are either (1) the name of a single QoalaHost vector variable which will hold all outcomes, or (2) a list of names for each individual outcome stored in its own QoalaHost int variable.
\item \textbf{Callback type}: Either \texttt{sequential} or \texttt{wait\_all}. Sequential means that the callback of this RR is executed for each generated pair, before the next pair is generated. Wait-all means that the callback is only executed once, namely when all pairs have been generated.
\item \textbf{Callback}: The name of the LR that acts as the callback for this RR. Can be empty (no callback is used).
\end{itemize}


\textbf{Request Parameters}
\begin{itemize}
\item \textbf{Remote ID}: The node ID of the remote node with which to generate entanglement.
\item \textbf{EPR Socket ID}: The ID of the EPR Socket to use.
\item \textbf{Number of pairs}: The number of entangled pair to generate.
\item \textbf{Virtual IDs}: A specification of the virtual IDs to assign to the entangled qubits. This may be in one of three formats:
\begin{itemize}
  \item \texttt{all <N>}: all qubits get virtual ID \texttt{<N>}. This might be used when a sequential callback is used that measures the qubit immediately after generating; thereby freeing up virtual ID \texttt{<N>} immediately for the next pair
  \item \texttt{increment <N>}: the first generated qubit gets ID \texttt{<N>}, the next \texttt{<N> + 1}, etc.
  \item \texttt{custom <N1, N2, ...>}: a custom list of IDs that should have the same length as the number of pairs
\end{itemize}
\item \textbf{Fidelity}: The desired fidelity \texttt{F} of the generated pairs.
If this request routine is for multiple pairs and the callback type is \texttt{wait\_all}, this value is used to specify that all pairs, after they have all been created, should have fidelity at least \texttt{F}. (How this is realized, which may involve multiple retries, is up to the network stack implementation in the QPS.)
\item \textbf{Type}: Create and Keep (\texttt{create\_keep}), Measure Directly (\texttt{measure\_directly}), or Remote State Preparation (\texttt{rsp}) \cite{dahlberg2019link}.
\item \textbf{Role}: \texttt{create} or \texttt{receive}. These roles are used to break symmetry between two nodes participating in entanglement generation (they should always have different roles). The `create' node is the initiating one.

\end{itemize}





\clearpage
\section{Runtime environment}
\label{app:runtime_environment}
In this section we provide more information about the runtime environment described in \cref{sec:runtime_environment}.
\cref{fig:app:runtime_detailed} provides an overview of the runtime architecture.

\subsection{Program instantiation}
A program instance is a Qoala program with additional runtime- and context-specific information that is supplied when preparing execution of the program.
A program instance represents a single execution of a Qoala program.

The additional information consists of:
concrete values for the global arguments of the program,
the Exposed Hardware Info (EHI),
an explicit Unit Module (see below), and
results from capability negotiation.

Based on the above additional information, a program instance can be created which has the following properties:
\begin{itemize}
\item \textbf{Program ID}: A unique ID for distinguishing multiple program instances that all need to be scheduled and run.
\item \textbf{Program}: The static Qoala program (without runtime information).
\item \textbf{Program Inputs}: The values for the program's global arguments.
\item \textbf{Unit Module}: The virtual quantum memory space that this program instance may use at runtime.
\item \textbf{Timing Information}: Deadlines for individual tasks. Computed using both the program's timing hints and information from the EHI.
\end{itemize}

\cref{fig:app:instantiation} provides a schematic example of program instantiation.

\begin{figure}[ht]
    \centering
    \includegraphics[width=0.5\columnwidth]{figures/instantiation.pdf}
    \caption{Schematic example of program instantiation.
    A program containing global arguments ($N$) is instantiated using a concrete value for the arguments ($N = 10$) and the EHI (containing values for the expect duration of a QL block, the expected duration of a CC block, and the qubit noise parameter expressed as the \textit{decoherence rate}). This results in a program instance for which the expected durations have concrete values.
    }
    \label{fig:app:instantiation}
\end{figure}

\subsection{Program versus program instance}
A program is typically the output of a compiler.
For example, a compiler might produce a BQC-server program, including global arguments for the remote ID of the client (i.e. the client ID is \textit{not} hardcoded into it).
A program instance represents a single execution of a Qoala program with concrete values for its global arguments.
For instance, the client ID now has the explicit value of 3, since the remote client happens to have node ID 3.
Often many program instances may be created for a single program.
For example, if 1000 runs of the BQC program are desired, 1000 program instances are created based on the single Qoala program.


\paragraph{Batches}
A program may be submitted for execution in a batch.

A batch $B$ consists of a program $P$, the number of execution $N$ and inputs for each execution. Based on this, $N$ program instances are created.


\subsection{Shared memory}
\label{app:shared_memory}
The CPS and QPS need to exchange information in order to execute local routines and request routines. They do so using shared memory.
The CPS writes routine arguments and reads results.
The QPS reads routine arguments and writes results.

Conflicts in writing and reading are avoided by the runtime itself (it is not assumed the hardware itself enforces read-only or write-only regions of memory).
This is achieved by strict read/write rules in Qoala: certain regions can only be written to by the CPS (QPS) while only be read from the QPS (CPS).
No region can be written to by both CPS and QPS.
Note that this design leaves open how the shared memory can be implemented: either as real physical shared memory, or as a message passing protocol.


\paragraph{Arrays}
The shared memory is logically divided into \textit{array elements} that can be allocated only by the CPS (\cref{fig:app:arrays}).
Each element can hold a single 32-bit signed integer.
The CPS can allocate shared memory space by specifying a \textit{size}, resulting in an allocated array.
An array is an ordered list of array elements. 
One can think of an array being a region in Shared Memory consisting of a consecutive list of elements.

Shared Memory is similar to the heap in classical OSes. Allocating an array is similar to \texttt{malloc} in C. Each program instance has its own view in the global shared memory, just like in classical OS, each program instance (or `process') has its own virtual memory space.

Elements that have been allocated but never written to have an undefined value.

An array may be named; it is written as \texttt{@arrayname}. An element in an array at index \texttt{i} is written as \texttt{@arrayname[i]}. This notation is used in NetQASM(\cref{app:netqasm}).

Arrays are used to share data between the CPS and the QPS.
They are used for executing both LRs and RRs.

The shared memory is logically divided into 5 \textit{regions} (\cref{fig:app:shared_memory}).
Each of the regions contains array elements, and in each region, arrays can be allocated.
The regions are only a logical division, where each arrays in a certain region are only used to hold data for a specific use-case:

\begin{itemize}
\item \texttt{LR\_in}: Argument values for LRs. CPS writes, QPS reads.
\item \texttt{LR\_out}: Result values for LRs. CPS reads, QPS writes.
\item \texttt{RR\_in}: Argument values for RRs. CPS writes, QPS reads.
\item \texttt{RR\_ou}t: Result values for RRs. CPS reads, QPS writes.
\item \texttt{CR\_in}: Argument values for callback LRs. QPS reads, QPS writes.
\end{itemize}


\begin{figure}[ht]
    \centering
    \includegraphics[width=0.6\columnwidth]{figures/arrays.pdf}
    \caption{Schematic overview of shared memory, which is organized as \textit{arrays}.
    Arrays are allocated by the CPS with a certain size (the number of \textit{array elements}).
    Each array element holds a single classical value.
    Arrays are identified using the \texttt[@<name>] syntax.
    Particular array elements may be accessed using the \texttt{[index]} syntax.
    }
    \label{fig:app:arrays}
\end{figure}


\begin{figure}[ht]
    \centering
    \includegraphics[width=0.5\columnwidth]{figures/shared_memory.pdf}
    \caption{Shared memory regions.
    The CPS writes local routine arguments to the \texttt{LR\_in} section and request routine arguments to the \texttt{RR\_in} section.
    The CPS reads local routine results from the \texttt{LR\_out} section and request routine results from the \texttt{RR\_out} section.
    The QPS reads local routine arguments from \texttt{LR\_in} and write results to \texttt{LR\_out}.
    The QPS reads request routine arguments from \texttt{RR\_in} and write results to \texttt{RR\_out}.
    Callbacks for request routines use the separate \texttt{CR\_in} section to use request routine results as arguments of the callback local routine.
    }
    \label{fig:app:shared_memory}
\end{figure}


\paragraph{Arrays for local routines}
Before an local routine (LR) can be executed, two arrays must be allocated by the CPS:
\begin{itemize}
\item An array in the \texttt{LR\_in} region. Its size needs to match the number of arguments for the LR.
\item An array in the \texttt{LR\_out} region. Its size needs to match the number of results of the LR.
\end{itemize}

The array in the \texttt{LR\_in} region can be accessed by the NetQASM code in the LR body using the name \texttt{@input}.
The array in the \texttt{LR\_out} region can be accessed by the NetQASM code in the LR body using the name \texttt{@output}.

Note that each program instance allocates (at runtime) its own arrays. Each individual LR in each individual program instance has access to two arrays called \texttt{@input}  and \texttt{@output}, but in practice there can hence be multiple "input" and "output" arrays, each occupying a different part of the global Shared Memory.


\paragraph{Arrays for request routines}
Before a request routine (RR) can be executed, multiple arrays must be allocated by the CPS:
\begin{itemize}
\item An array in the \texttt{RR\_in} region. 
\item An array in the \texttt{RR\_out} region. Its size needs to match the number of names in the "Results" entry in the RR header.
\item An array in the \texttt{CR\_in} region. Its size needs to match the number of arguments for the callback LR of the RR.
\end{itemize}

The results of the RR are written to the array in the \mbox{\texttt{RR\_out}} region. Arguments to the callback LR are written to the array in the \texttt{CR\_in} region.



\subsection{Quantum memory}
\label{app:quantum_memory}
The QPS is assumed to have access to a quantum random access memory (QRAM) consisting of \textit{qubits}.
Each qubit is a single location in the QRAM and can hold a single 2-dimensional quantum value, like $\ket{0}$ or $\ket{+}$.

We distinguish between (1) the \textit{physical quantum memory space (PQMS)} consisting of \textit{physical qubits}
and (2) a \textit{virtual quantum memory space (VQMS)} for each program instance (\cref{sec:runtime_environment}).

The topology (qubit connectivity) and noise characteristics of the PQMS are exposed as part of the EHI.
Each program instance has access to its own VQMS, which is represented as a Unit Module~\cite{dahlberg2022netqasm}.
The VQMS for each program instance is created when instantiating the program.
This can be seen as virtual memory allocation for the program.
At runtime, the VQMS of each running program instance is mapped to the PQMS.

\paragraph{Unit Modules}
A Unit Module (UM) describes the topology of a VQMS as well as its noise characteristics.
That is, a UM contains:
\begin{itemize}
\item \textbf{Qubit Info}: a list of all qubits available in the VQMS, with for each qubit the following information:
its virtual ID,
whether it is a communication qubit or not, and
its decoherence rate per second.

\item \textbf{Gate Info}: a list of all quantum gates and quantum local operations available for the qubits in the VQMS, with for each item the following information:
\begin{itemize}
  \item Which NetQASM instruction it is represented by (may be in a particular NetQASM flavor).
  \item On which sets of qubits the gate or operation can be applied.
  \item Its duration.
  \item The decoherence rate per second on each of the qubits it acts on.
\end{itemize}
\end{itemize}

A UM can be seen as a subset of the full EHI of a node, specifically containing a subset of all qubits available in the node.

Qubits in the Unit Module are called \textit{virtual qubits}. They are identified by their \textit{virtual IDs} and are mapped to physical qubits (\cref{fig:app:unit_module}).



\paragraph{Memory manager}
Quantum memory allocation and freeing is handled by a memory manager, which lives in the QPS.
The memory manager keeps track of the unit modules of all program instances, and maps virtual qubits to physical qubits. 

Before starting a local routine or request routine, the memory manager allocates the corresponding qubits.
For example, if a local routine for program instance $P$ defines in its metadata (see \cref{app:program_structure}) that it uses virtual qubits 0 and 1, the memory manager allocates virtual qubits 0 and 1 (if not already allocated).
This involves finding currently unused physical qubits and mapping new virtual qubit to these free physical qubits.

\begin{figure}[ht]
    \centering
    \includegraphics[scale=0.4]{figures/unit_module.pdf}
    \caption{Example of a physical quantum memory available in a node (four qubits) and two allocated unit modules. The colors of the qubits represent the physical locations they map to. Note that the top-left qubit of Unit Module 2 is not currently mapped and that it also cannot be mapped. Therefore, tasks that require program instance 2 to use a third qubit cannot be executed at this time.}
    \label{fig:app:unit_module}
\end{figure}



\subsection{Exposed hardware interface}
\label{app:ehi}
The Qoala execution environment exposes certain information related to the hardware and software capabilities.
This information includes noise characteristics of quantum memory and of entanglement generation, as well as estimates of classical latencies.

All information that is exposed falls under the Exposed Hardware Interface (EHI).
The EHI can be divided into \textit{node info} and \textit{network info}.


\paragraph{EHI Node Info}
The EHI node info consists of:

\begin{itemize}
\item \textbf{Qubit Info}: a list of all qubits available at the node, with for each qubit the following information: 
(1) its ID,
(2) whether it is a communication qubit or not, and
(3) its decoherence rate per second.
\item \textbf{Gate Info}: a list of all quantum gates and quantum local operations available at the node, with for each item the following information:
(1) which NetQASM instruction it is represented by (may be in a particular NetQASM flavor,
(2) on which sets of qubits the gate or operation can be applied,
(3) its duration, and
(4) the decoherence rate per second on each of the qubits it acts on.

\item \textbf{NetQASM flavor}: a list of all supported NetQASM instructions. All NetQASM instructions mentioned in Gate Info must be in this list

\item \textbf{Classical latencies}:
Covers
(1) duration of executing a single QH Instruction, and
(2) duration of executing a classical NetQASM instruction (Note that the duration of quantum operations is covered by the Gate Info).
\end{itemize}

\paragraph{EHI Network Info}
The EHI network info consists of \textbf{Link Info} for each link in the network, with
(1) the expected duration of generating an entangled pair on this link, and
(2) the expected fidelity of generating an entangled pair on this link.



\subsection{Sockets}
Connections with remote nodes are modeled as \textit{sockets}.
Each program instance running on a node has access to classical sockets an EPR sockets.
Classical sockets represent an endpoint for connections over which classical messages can be sent.
A program instance can have classical sockets with any other nodes in the network.

An EPR socket represents an endpoint of a quantum connection.
Through the EPR socket, a program can ask for entanglement with a remote node.


\begin{figure*}[ht]
    \centering
    \includegraphics[width=\textwidth]{figures/runtime_detailed.pdf}
    \caption{Detailed Qoala runtime overview.}
    \label{fig:app:runtime_detailed}
\end{figure*}
\clearpage
\section{Scheduling and execution}
\label{app:scheduling_execution}
This section provides more details about tasks, task creation and scheduling (\cref{sec:architecture}) as well as about our scheduler implementation (\cref{sec:implementation}).

\subsection{Tasks}
\label{app:scheduling_tasks}

\paragraph{Task creation}

Tasks are created based on the blocks in a program.
Specifically, a block $B$ in the program is mapped to a set $T(B)$ of tasks.
Since a block may be executed multiple times, multiple instances of $T(B)$ can be created at runtime.

CL and CC blocks are mapped to CPS tasks only.
QL and QC blocks are mapped to a sequence of CPS- and QPS tasks.
\begin{itemize}
    \item CL block. A single \texttt{HostLocal} task is created.
    \item CC block. A single \texttt{HostEvent} task is created.
    \item QL block. If there is a single \texttt{run\_routine} call, a \texttt{LocalRoutine} task is created for the QPS, as well as a \texttt{PreCall} tasks and a \texttt{PostCall} task for the CPS.
    Two precedence constraints are added: the \texttt{PreCall} task precedes the \texttt{LocalRoutine} task, and the \texttt{LocalRoutine} task preceded the \texttt{PostCall} task.
    If there is a \texttt{join\_routines} on multiple local routine, multiple \texttt{PreCall}-\texttt{LocalRoutine}-\texttt{PostCall} task sets are created, without any dependencies between the task sets.
    \item QC block. If the request that is called from this block is for a single pair,
    a \texttt{SinglePair} task is created. If the request is for more than 1 pair, a \texttt{MultiPair} task is created. In both cases, an additional \texttt{PreCall} and a \texttt{PostCall} task are created with precedence constraints like for QL blocks.
    If there is a \texttt{join\_routines} on multiple request routines, multiple \texttt{PreCall}-Pair-\texttt{PostCall} task sets are created, without any dependencies between the task sets.
\end{itemize}
\cref{fig:app:task_creation} shows an overview of blocks and corresponding tasks and their precedence constraints.


\paragraph{Predictable vs unpredictable programs}
Tasks are created based on the contents of a program instance, and their precedence relations are defined by the control-flow of the blocks in the program's host code.
Because of jump and branch instructions in the host code, a block may be executed zero, one or multiple times.
Furthermore, the exact number of executions of a block may not be known ahead of time.
For example, a program might loop through a sequence of blocks by using a conditional branch instruction at the end of the last block of the sequence.
The condition could depend on a runtime value (such as the result of a quantum measurement).
We say that control-flow is \textit{predictable} if it can be completely known before runtime. 
\textit{Unpredictable} control-flow, on the other hand, depends on values available only at runtime. 
For predictable programs, all its tasks can be created before runtime.
For unpredictable programs, (some of) its tasks must be created on-the-fly during program execution.
\cref{fig:app:linear_non_linear} illustrates the difference between predictable and unpredictable programs.

\begin{figure}[ht]
    \centering
    \includegraphics[width=0.7\columnwidth]{figures/linear-non-linear.pdf}
    \caption{Schematic overview of the difference between predictable and non-predictable programs.
    The control-flow of the predictable program (left) is linear: first block 1 is executed (calling local routine (LR) 1), then block 2 (calling request routine (RR) 1), and finally block 3.
    Therefore, the number of tasks is fixed and known before execution.
    The non-predictable program is similar but after executing block 2, control-flow may go back to block 1 (again), depending on a runtime value (e.g. the result of RR 1).
    Hence, the number of times that blocks 1 and 2 are executed is not known beforehand, and therefore the number of tasks is also not known.
    }
    \label{fig:app:linear_non_linear}
\end{figure}

\begin{figure*}
    \centering
    \includegraphics[width=\textwidth]{figures/task_creation.pdf}
    \caption{Overview of different host blocks with corresponding tasks. In the rightmost column, tasks with a dark background are QPS tasks, the others are CPS tasks. This example shows that tasks contain data about the program segment they correspond to, such as \texttt{LocalRoutine} tasks having the name of the routine they are executing.}
    \label{fig:app:task_creation}
\end{figure*}


\paragraph{Task execution}
Tasks are executed by the CPS or the QPS, and the specific operations involved depend on the type of the task.

\textbf{\texttt{HostLocal} task execution.} A \texttt{HostLocal} task $t_{hl} = (P, B)$ for program instance $P$ and block $B$ is handled by executing each of the instructions in $B$. When the task finishes, the name of the next block to execute is recorded. If $B$ ends with a branch instruction, this is the target block; otherwise it is the next block in the program (if this was the last block, the next block is nil).

\textbf{\texttt{HostEvent} task execution}. A \texttt{HostEvent} task represents a block $B$ of type $CC$, which must start with exactly one \texttt{recv\_cmsg} instruction. Handling the task involves reading a message from the message buffer and assigning it to the result variable of the receive instruction. Then, the remaining instructions in $B$ are executed just like in a \texttt{HostLocal} task.


\textbf{\texttt{PreCall} task execution.} A \texttt{PreCall} task corresponds to a LR call instruction in Host code. The CPS allocates space in the shared memory for arguments and results. It then writes argument values to the shared memory.

\textbf{\texttt{PostCall} task execution.} A \texttt{PostCall} task corresponds to a LR call instruction in Host code. The CPS reads the results from the shared memory and copies them to the corresponding variables in the host local memory.

\textbf{\texttt{LocalRoutine} task execution.} A \texttt{LocalRoutine} task is executed by the QPS. It involves the following steps. First, based on information in the uses/keeps metadata, virtual quantum memory is allocated. Then all NetQASM instructions are executed, which may involve loading values from shared memory (reading arguments) and storing values to shared memory (populating results). Finally, quantum memory is freed.

\textbf{\texttt{SinglePair} task execution.} A \texttt{SinglePair} task is executed by the QPS. First, arguments are read from shared memory. Then, an EPR request (see \cref{app:entanglement_distribution}) is sent to the network controller.

\textbf{\texttt{MultiPair} task execution.} A \texttt{MultiPair} task is executed by the QPS. First, arguments are read from shared memory. Then, multiple EPR requests are sent to the network controller. Whether these requests are all sent at once or consecutively and waiting for intermediate responses is up to the implementer; the choice may depend on efficiency and resource considerations.

\textbf{\texttt{SinglePairCallback} and \texttt{MultiPairCallback} task execution.} First read results (from a \texttt{SinglePair} or \texttt{MultiPair} task) from shared memory. Then execute the callback routine just like a \texttt{LocalRoutine} task.

\paragraph{Deadlines}
Deadlines can be specified for blocks relative to other blocks using the syntax:

\begin{qoalacode}[caption=Pseudocode for the Algorithm, label=lst:pseudocode]
(*@\textcolor{purple}{\textasciicircum block\_0}@*):
    ...

(*@\textcolor{purple}{\textasciicircum block\_1}@*) { deadlines = [b0: 3ms] }:  // relative deadline of 3 ms compared to block_0
    ...
\end{qoalacode}

A relative deadline to some block $B$ is always with respect to the last task in $T(B)$, for the last task set instance (in case of multiple execution of this task set).
The deadline value may be an explicit value (like $3 ms$) or it can be in terms of EHI values, such as for example $0.1 * CC$ where $CC$ is the expected classical node-node latency provided by the EHI.

\paragraph{Precedence constraints}
By default, blocks are executed in the order they are given in the program.
Blocks ending with a jump or branch instruction define precedence constraints at runtime for unpredictable programs.

Scheduling happens at runtime and involves choosing which task to execute next.
In Qoala, there are three schedulers per node: the \textit{CPS scheduler} controls task execution on the CPS,
the \textit{QPS scheduler} controls task execution on the QPS, and the \textit{node scheduler} controls the CPS- and QPS schedulers.
The CPS- and QPS schedulers are both processor schedulers.

\subsection{Scheduling}
In this and the following sections we describe the scheduler from our implementation (\cref{sec:implementation}).

Each scheduler maintains their own task graph,
which is a directed acyclic graph (DAG) in which the nodes represent tasks and edges represent precedence constraints.
The node scheduler task graph contains all tasks (CPS or QPS) that are to be executed.
Each processor scheduler task graph is a partial copy of the node scheduler task graph containing only the tasks that can be executed by its own processor.
Edges in the node scheduler graph between heterogenous tasks (i.e. between CPS and QPS tasks) are represented in the partial processor graphs by the \textit{external dependencies} node attribute. See \cref{fig:app:task_graph_partial} for an example.
When a processor scheduler finishes a task, it is removed from the task graph and a signal is sent to the node scheduler.
The node scheduler updates its own task graph accordingly, and may then add new tasks to the task graph of the processor scheduler.
Note that although the processor task graphs are accessible by both the owning processor scheduler and the node scheduler, there are no read/write conflicts
since tasks can only be added by the node scheduler, and tasks can only be removed by the processor scheduler.

\paragraph{Task graph}
A task graph consists of 
\begin{itemize}
\item \textit{tasks} to be scheduled (the nodes),
\item \textit{precedence constraints} between the tasks (precedence edges),
\item \textit{external precedence constraints} for tasks in the case of processor task graphs (annotated on the nodes),
\item \textit{relative deadlines} between tasks (deadline edges),
\item \textit{trigger annotations} for some tasks (like incoming messages or network schedule timestamps)
\end{itemize}


Upon program instantiation, all created tasks are added to the node scheduler task graph, and the relevant tasks are added to the processor schedulers.
The number of tasks that are created (and can hence be added to the task graphs) depends on the predictability of the program.
During runtime, the node scheduler may create new tasks based on the control-flow of the program.

\begin{figure}[ht]
    \centering
    \includegraphics[width=0.6\columnwidth]{figures/task_graph_partial.pdf}
    \caption{
        Example of a mapping from a full task graph (containing both CPS and QPS tasks) to a partial graph (containing only CPS tasks).
        Task 5 depends on task 4, which is external from the perspective of the CPS scheduler (indicated using the \texttt{external-dependencies} attribute).
        Note that Task 3 is not needed at all in the partial graph; only the dependency on task 4.
    }
    \label{fig:app:task_graph_partial}
\end{figure}

\paragraph{Task graph splitting}
\label{app:task_graph_splitting}
The node scheduler creates a heterogenous task graph consisting of both CPS and QPS tasks.
This graph needs to be split into a partial CPS and a partial QPS graph.
This is done using the following algorithm.

We consider creating the partial graph for the CPS, and hence the QPS is `the other processor'.
For the partial graph of the QPS the procedure is exactly the same but with reversed roles.

For a heterogenous task graph $G$ containing tasks $T$ (all tasks for both CPS and QPS), precedence constraints $P$ ($(t_1, t_2) \in P$ means that $t_1$ must precede $t_2$), compute the partial CPS graph $G_{CPS}$ as follows:

\begin{itemize}
    \item Split $T$ into a set $T_{CPS}$ consisting of all tasks that run on the CPS, and $T_{QPS}$ consisting of all other tasks in $T$. $G_{CPS}$ will consist of only tasks in $T_{CPS}$.
    \item Let $P_{CPS} \subset P$ consist of all precedence constraints $(t_1, t_2)$ where $t_1 \in T_{CPS}$ and $t_2 \in T_{CPS}$.
    These constraints will remain the same in $G_{CPS}$ since they are between tasks in $T'$.
    \item Compute the `immediate cross-predecessors' set $I$ of all tasks $t_{cp} \in T_{QPS}$ such that there exists a task $t \in T_{CPS}$ and $(t_{cp}, t) \in P$.
    In other words, $I$ contains all tasks running on the QPS that are immediate predecessors of CPS tasks.
    \item For each $t_i \in I$, compute the `closest CPS ancestor' task $t_{anc} \in T_{CPS}$, which is a CPS task that has a direct precedence constrain with the closest ancestor of $t_i$.
    Add $(t_{anc}, t_i)$ to the precedence constraints of $G_{CPS}$.
\end{itemize}

\paragraph{Scheduler communication}
Here we describe how the schedulers communicate in our implementation (\cref{sec:implementation}).

The three schedulers need to exchange information in order to work together.
All schedulers can broadcast a \textit{signal} with short information such as `task N completed' or `memory freed'. Each scheduler receives these signals.
Furthermore, the following read and write access is given:
\begin{itemize}
    \item The CPS scheduler can read from the completed task ID list of the QPS and vice versa. This makes it possible for the CPS (QPS) scheduler to directly update their remote dependencies without having to wait for a signal from the node scheduler, leading to overall improvement in efficiency
    \item The node scheduler can add new tasks to the partial graphs of the CPS and QPS. Note that the node scheduler will only add tasks to the partial graph of a processor scheduler when this scheduler is in a waiting state; that is, after the processor scheduler has sent a `waiting` signal and before the node scheduler has sent a `task added` signal (only after this signal will the processor scheduler continue). In this way, there are no read/write conflicts in the partial graphs of processor schedulers.
    \item The CPS (QPS) scheduler can only remove tasks from its own partial graph, not add any.
\end{itemize}


\subsection{Scheduler algorithms}

\paragraph{Node scheduler algorithm}
Below we describe the high-level steps involved in the node scheduler algorithm implementation of \cref{sec:implementation}.
\label{app:node_scheduler_algorithm}
\begin{enumerate}
    \item Split the current task graph into a partial CPS graph and a partial QPS graph. For the algorithm, see `Task graph splitting' above.
    \item Add the CPS (QPS) tasks to the partial graph of the CPS (QPS) scheduler
    \item Wait for a `task finished` signal from either CPS or QPS scheduler
    \item Remove the corresponding task from the task graph.
    \item If the finished task was a \texttt{HostLocal} task for some program instance P, and if the CPS partial graph is empty, check which block the program instance should jump to. This information is given by the task itself (and stored in the completed task list of the CPS scheduler), after evaluating the last instruction (a jump or branch instruction) in the BB that the task represented. For this new BB, create corresponding tasks for both the CPS and QPS. Task creation is discussed in \cref{app:scheduling_tasks}.
    \item If the task graph is empty, idle until new programs are instantiated.
    \item Go back to step 1.
\end{enumerate}

Note that the role of the node scheduler is much smaller when only predictable programs are run.
When predictable programs are instantiated, all of their tasks are created at once, resulting in a large task graph in the node scheduler, which never gets new tasks created at runtime.
In this scenario, after steps 1 and 2 the CPS and QPS schedulers possess a partial graph which will never get any new tasks.
Both processor schedulers will work on their tasks until they are both empty, after which all program instances have finished. Meanwhile, the node scheduler just loops through steps 1, 2, 3 and 6, not doing anything.

\paragraph{CPS scheduler algorithm}
Below we describe the high-level steps involved in the CPS scheduler algorithm implementation of \cref{sec:implementation}.
\begin{enumerate}
    \item Check which new tasks were completed by the QPS by reading from the shared task memory. Remove external dependency edges that correspond to QPS tasks that have completed.
    \item Find all tasks in the partial graph that are ready to execute. These are tasks that fulfill all following requirements:
        \begin{itemize}
            \item The task has no incoming precedence constraints (there are no unfinished tasks in the task graph that must precede this task)
            \item The task has no external precedence constraints (there are no unfinished QPS tasks that must precede this task)
            \item If the task is a \texttt{HostEvent} task, there must be at least one message in the CPS' message buffer
            \item If the task has a specific start time, the current time should be at least the start time
        \end{itemize}
    \item If there is no task ready to execute, send a `waiting` signal and wait until a signal is received that indicates one of the following events:
        \begin{itemize}
            \item The node scheduler has added one or more tasks to the partial graph
            \item The QPS scheduler has completed a task
            \item The start time has arrived of one of the tasks that were previously not ready only because their start time had not yet passed
            \item One or more new messages have been put into the message buffer
        \end{itemize}
        After one of these signals is received, go back to step 1.
    \item If there is at least one task ready to execute, choose which one to execute now. This depends on the scheduling policy that is being used. The policy may or may not use information about the deadlines of the available tasks. Scheduling policies that were implemented for our evaluation are described in \cref{app:evaluation}.
    \item If the task failed, go back to step 1
    \item If the task completed, remove it from the partial graph, add its ID to the completed task ID list, and broadcast a signal that the task was finished. If the task was a \texttt{HostLocal} task, then also store (in the completed task list) an entry containing the name of the next block to execute. (In this way, the node scheduler knows which task(s) to create and add to the full task graph. See \cref{app:scheduling_tasks} for more details.)
    Update the deadlines of all other tasks in the task graph.
    Then go back to step 1.
\end{enumerate}

\paragraph{QPS scheduler algorithm}
Below we describe the high-level steps involved in the QPS scheduler algorithm implementation of \cref{sec:implementation}.
\begin{enumerate}
    \item Check which new tasks were completed by the CPS by reading from the shared task memory. Remove external dependency edges that correspond to CPS tasks that have completed.
    \item Find all tasks in the partial graph that are ready to execute. These are tasks that fulfill all following requirements:
        \begin{itemize}
            \item The task has no incoming precedence constraints (there are no unfinished tasks in the task graph that must precede this task)
            \item The task has no external precedence constraints (there are no unfinished CPS tasks that must precede this task)
            \item If the task is a \texttt{SinglePair} or \texttt{MultiPair} task, the current time should be the beginning of a network time slot that corresponds to this task. (For example, if the task is for creating EPR pairs for program instance 1 on this node (called `Alice') and program instance 2 on node `Bob', then the current time should be the start of a $(Alice, 1, Bob, 2)$ time slot).
            \item If the task has a specific start time, the current time should be at least the start time
        \end{itemize}
    \item If there is no task ready to execute, wait for a signal that indicates one of the following events:
        \begin{itemize}
            \item The node scheduler has added one or more tasks to the partial graph
            \item The CPS scheduler has completed a task
            \item The start time has arrived of one of the tasks that were previously not ready only because their start time had not yet passed
            \item The start of a time slot has arrived which corresponds to one of the tasks that were previously only blocked on the arrival of this time slot
        \end{itemize}
        After one of these signals is received, go back to step 1.
    \item If there is at least one task ready to execute, choose which one to execute now. This depends on the scheduling policy that is being used. The policy may or may not use information about the deadlines of the available tasks.
    \item If the task failed, go back to step 1
    \item If the task completed, remove it from the partial graph, add its ID to the completed task ID list, and broadcast a signal that the task was finished.
    Update the deadlines of all other tasks in the task graph.
    Then go back to step 1.
\end{enumerate}


\paragraph{Task graph updates}
The node scheduler may add tasks to the current task graph of the CPS or QPS.
When a processor scheduler has finished a task, it is removed from the task graph.
This has the following effects:
\begin{itemize}
    \item Precedence edges from this task are removed, potentially making other tasks available for execution
    \item The time of finishing is recorded; and the deadlines and relative deadlines of all other tasks are updated accordingly
\end{itemize}



\subsection{Other algorithms}
\paragraph{Linear graphs}

When instantiating a program multiple times (for example instantiating a BQC program 1000 times), one has the option to linearize the graphs. Each instantiation has its own graph,
and the full graph of all instances result in many independent tasks.
One can force all instances to be run in sequence, rather than interleaved, resulting in a linear chain of single-instance graphs. This is done using the following algorithm:

\begin{itemize}
    \item For each pair $(i_1, i_2)$ of consecutive instances, add a precedence constraint between the last tasks(s) of $i_1$ and the first task(s) $i_2$.
\end{itemize}

\begin{figure*}
    \newcommand{\networkcontrollerfigheight}{4.5cm}
    \centering
    \subfloat[\centering \label{fig:app:network_controller_requests}]{{\includegraphics[height=\networkcontrollerfigheight, keepaspectratio]{figures/network_controller_requests.pdf}}}%
    \qquad
    \subfloat[\centering \label{fig:app:network_controller_requests_distributed}]{{\includegraphics[height=\networkcontrollerfigheight, keepaspectratio]{figures/network_controller_requests_distributed.pdf}}}%
    \caption{
    Different implementations of network controller and network stack.
    (a) The network controller is centralized and the nodes send requests to this controller
    whenever they are executing \texttt{SinglePair} or \texttt{MultiPair} tasks.
    (b) The network controller is distributed over the nodes. Inside each node there is a network stack which autonomously talks with the network stack of other nodes and synchronizes entanglement generation.
    Execution of \texttt{SinglePair} and \texttt{MultiPair} tasks involves sending a request to the network stack within the node, which then handles pair generation by synchronizing with the network stack in other nodes.
    }%
    \label{fig:app:network_controller_types}
\end{figure*}


\paragraph{Estimating task durations}
The scheduler uses the EHI to estimate the duration of a task.
This duration may then be used by the scheduler to decide which task to execute when.
In our implementation, the scheduler does not make use of these estimates, but we did implement a simple estimator algorithm:

The estimated duration $E$ of a task is computed as follows:
\begin{itemize}
    \item For a \texttt{HostLocal} or \texttt{HostEvent} task representing a program block $B$, $E$ is $N \cdot$ \texttt{host\_latency} where $N$ is the number of HostLanguage operations in $B$ and \texttt{host\_latency} is given in the EHI.
    \item For a \texttt{LocalRoutine} tasks representing a block that call a NetQASM routine $S$, 
        $E$ is the sum of estimated durations of each NetQAM instruction in $S$. The duration of each quantum instruction is obtained from the EHI, and the duration of each classical instruction is given by the \texttt{qnos\_latency} entry in the EHI.
    \item For a \texttt{SinglePair} or \texttt{MultiPair} task based on a block that calls a request $R$ for $N$ EPR pairs, $E$ is $N$ times the duration of a single EPR generation as listed in the EHI.
    \item For \texttt{PreCall} and \texttt{PostCall} tasks, the duration is set to the \texttt{host\_latency} entry in the EHI.
\end{itemize}

\subsection{Entanglement Distribution}
\label{app:entanglement_distribution}
Qoala only defines how program are executed on a node in a quantum network,
and not how and when entanglement is created between nodes.
However, Qoala does assume certain things about how nodes can interact with the entanglement distribution system, however this is implemented.
The assumption about entanglement generation are as follows.

\textbf{Network controller with time slots.}
Conceptually, there is a network controller that oversees entanglement generation and distribution across the whole network.
Qoala does not care whether this controller is implemented as a single entity, or is distributed in some way across multiple (processing) nodes (\cref{fig:app:network_controller_types}).
The network controller maintains a global timeline divided into \textit{time slots}, which can have arbitrary length.
Each time slot may be assigned to a \textit{session}, which is a 4-tuple $(N1, P1, N2, P2)$ where $N1$ ($N2$) is the name of a node in the network and $P1$ ($P2$) is an ID of a program instance running within $N1$ ($N2$).
A session hence represents a pair of running program instances across two nodes, and it is such pairs of program instances that want to create entanglement with each other. 
If a time slot is assigned to some session $(N1, P1, N2, P2)$, only program instances $P1$ and $P2$ (on nodes $N1$ and $N2$) may create entanglement with each other during this time slot.

Populating the network controller's time slot with sessions is the result of (1) demand registration by nodes in the network, followed by (2) network schedule generation by the network controller itself, which we do not consider here (\cref{fig:app:network_controller_setup}).
In the following, we simply assume that the network controller has a list of time slots assigned to sessions relating to program instances that are being run, and that these time slots are also known by the individual processing nodes.

\begin{figure*}[ht]
    \centering
    \includegraphics[width=0.5\textwidth]{figures/network_controller_setup.pdf}
    \caption{High-level steps of using the network controller.
    1. Nodes discuss among each other constraints about application execution (Capability Negotiation).
    2. The outcome of Capability Negotiation, which contains demands about entanglement generation, is sent to the network controller (Demand Registration).
    3. Based on the demands from the nodes, the network controller constructs a network schedule consisting of time slots. Each time slot is assigned to zero or more \textit{sessions}, which correspond to program instance pairs.
    }
    \label{fig:app:network_controller_setup}
\end{figure*}

\textbf{On-demand entanglement requests.} At runtime, nodes implementing Qoala may send requests to the \textit{network stack}.
This network stack then issues \textit{EPR requests} to the network controller.
Upon receiving an EPR request, the network controller stores it and potentially acts on it:
\begin{itemize}
    \item If there is a matching EPR request from the other node, and if the current time slot is assigned to the corresponding session, perform the actual entanglement generation process.
    \item If at least one of the two above conditions does not hold, keep the request until both conditions are satisfied (a matching request from the other node arrives, or the corresponding time slot arrives, or both).
\end{itemize}

An EPR request is a request for a single EPR pair. A \texttt{SinglePair} task is handled by the network stack sending a single EPR request to the network controller.
A \texttt{MultiPair} task is handled by sending multiple EPR requests, possibly interleaved by local QPS processing such as callback routines.

The network stack may fail handling a request. For example, it might timeout trying to produce an EPR pair. In this case, the corresponding task (\texttt{SinglePair} or \texttt{MultiPair}) also fails.
Depending on the scheduler implementation, this task may be executed again at a later time, or the whole program instance may be aborted.


\textbf{Entanglement generation as a black box.} We assume that all nodes can create entanglement with all nodes, orchestrated by the network controller.
Qoala does not assume anything about the existence of repeater nodes or entanglement routing algorithms.
Rather, a node sending a request for entanglement (in a suitable time slot) will either get this entanglement (created in some way, irrelevant to Qoala) or not (creation failed for some reason, again irrelevant to Qoala).
The network stack and controller may be implemented in various ways, such as illustrated in \cref{fig:app:network_controller_types}.

\clearpage
\begin{figure}[t]
\centering
\begin{subfigure}[b]{0.49\columnwidth}
    \includegraphics[width=\linewidth]
    {plots/sharegpt_sim.pdf}
    \caption{ShareGPT}
    \label{fig:sharegpt_sim}
\end{subfigure}
\hfill
\begin{subfigure}[b]{0.49\columnwidth}
    \includegraphics[width=\linewidth]{plots/lats_sim.pdf}
    \caption{LATS}
    \label{fig:lats_sim}
\end{subfigure}
\vspace{-2mm}
\caption{ \small \textbf{Comparison to optimal scheduling policy.} In simulation, \text{\name} outperforms other scheduling policies; however, there remains a visible gap relative to the optimal policy (SRPT).}
\label{fig:sim}  
\vspace{-4mm}
\end{figure}
\clearpage
\section{Evaluation details}
\label{app:evaluation}

\subsection{Simulator setup}
All simulations have been done with the simulation package found at~\cite{qoala2023simulator}
and were run on a machine using 80 Intel Xeon Gold cores at 3.9 GHz and 192 GB of RAM.

\textbf{Code availability.}
All code used for the evaluations is available in the \texttt{evaluation/} folder~\cite{qoala2023simulator}.
For each evaluation done (each subsection in \cref{sec:evaluation}, it includes the Qoala program source code, the scripts for running the simulations, the scripts for producing the plots, and a \texttt{README} that explains how to use the code.
In the source code, the term \textit{time bin} is used for what we here call \textit{time slot}.

\subsection{Hardware parameters}
In this section we describe the characteristics of the hardware types that have been used in our evaluations.
For the evaluation in \cref{sec:demonstrating_architecture_effectiveness} we simulated all three types below;
for the other evaluations we only considered the generic hardware type.

\subsubsection{Generic hardware}
The allowed gate set is expressed as a particular NetQASM flavour~\cite{dahlberg2022netqasm}.

\begin{itemize}
  \item Allowed single-qubit gates (vanilla NetQASM flavour~\cite{dahlberg2022netqasm}):
  \texttt{init}, \texttt{rot\_x}, \texttt{rot\_y}, \texttt{rot\_z}, \texttt{x}, \texttt{y}, \texttt{z}, \texttt{h}, \texttt{meas}.
  \item Allowed two-qubit gates (vanilla NetQASM flavour): \texttt{cnot}, \texttt{cphase}.
\end{itemize}

Qubit decoherence times are expressed as T1 (amplitude damping) and T2 (dephasing time), which is commonly done in quantum computing.
Unless stated otherwise in the evaluation details below, the default noise and duration parameters used for the generic hardware are:
\begin{itemize}
  \item Single-qubit duration: $5 \cdot 10^3$ ns.
  \item Two-qubit duration: $200 \cdot 10^3$ ns.
  \item Qubit T1 time: $10^9$ ns.
  \item Qubit T2 time: $10^8$ ns.
\end{itemize}


\subsubsection{NV hardware}
Values from~\cite{avis2023requirements} and private communication.

\begin{itemize}
  \item Allowed single-qubit gates on communication qubit (NV NetQASM flavour): \texttt{init}, \texttt{rot\_x}, \texttt{rot\_y}, \texttt{meas}.
  \item Allowed single-qubit gates on memory qubit (NV NetQASM flavour): \texttt{init}, \texttt{rot\_x}, \texttt{rot\_y}, \texttt{rot\_z}, \texttt{meas}.
  \item Allowed two-qubit gates between communication qubit and memory qubit (NV NetQASM flavour): \texttt{crot\_x}, \texttt{crot\_y}.
\end{itemize}

Unless stated otherwise in the evaluation details below, the default noise and duration parameters used for the NV hardware are:
\begin{itemize}
  \item Single-qubit duration on communication qubit: 300 ns.
  \item Single-qubit duration on memory qubit: $1.2$ ms.
  \item Two-qubit duration: 1 ms.
  \item Communication qubit T1 time: 3600 ms
  \item Communication qubit T2 time: 500 ms
  \item Memory qubit T1 time: 35000 ms
  \item Memory qubit T2 time: 1 ms
\end{itemize}

\subsubsection{Trapped-ion hardware}
Values from~\cite{avis2023requirements} and private communication.

\begin{itemize}
  \item Allowed single-qubit gates (trapped-ion NetQASM flavour): \texttt{init}, \texttt{rot\_z}, \texttt{meas}.
  \item Allowed all-qubit gates (trapped-ion NetQASM flavour): \texttt{init\_all}, \texttt{meas\_all}, \texttt{rot\_x\_all}, \texttt{rot\_y\_all}, \texttt{rot\_z\_all}, \texttt{bichromatic}.
\end{itemize}


The effect of applying a bichromatic gate is expressed as
\[
   U_{XX}(\theta) = \exp(-i \frac{\theta}{2} \sum_{i<j} \sigma_X^{(i)} \sigma_X^{(j)})
\]

for some angle $\theta$.

Unless stated otherwise in the evaluation details below, the default noise and duration parameters used for the trapped-ion hardware are:
\begin{itemize}
  \item Single-qubit duration on communication qubit: 26.6 $\mu s$.
  \item All-qubit duration: 85 ms.
  \item Qubit T1 time: $\infty$.
  \item Qubit T2 time: 85 ms.
\end{itemize}

\subsubsection{NetQASM gate sequence for CNOT on trapped- ion hardware}
We list the sequence of netqasm instructions to effectively apply a CNOT gates on two qubits, which is non-trivial.

Assuming 2 qubits are in use, CNOT gate between qubit 0 and qubit 1 on trapped ion:
\begin{qoalacode}
  NETQASM:
    // cnot between q0 and q1
    rot_x_all 8 4
    rot_z Q0 8 4
    rot_x_all 24 4
    bichromatic 8 4
    rot_x_all 24 4
    rot_x_all 8 4
    rot_z Q0 24 4
    rot_x_all 24 4
\end{qoalacode}

\begin{table*}[]
\begin{tabular}{|l|l|l|l|l|l|}
\hline
\textbf{Application} & \textbf{Number of nodes} & \textbf{\begin{tabular}[c]{@{}l@{}}Number of EPR\\ pairs per instance\end{tabular}} & \textbf{\begin{tabular}[c]{@{}l@{}}Max number of\\ qubits per node\end{tabular}} & \textbf{\begin{tabular}[c]{@{}l@{}}Number of \\ instances\end{tabular}} & \textbf{\begin{tabular}[c]{@{}l@{}}Simulation\\ duration (s)\end{tabular}} \\ \hline
A1. QKD              & 2   & 1000   & 1     & 1000     & 2166   \\ \hline
A2. BQC              & 2   & 2      & 2     & 1000     & 227    \\ \hline
A3. Teleportation    & 2   & 1      & 2     & 1000     & 24     \\ \hline
A4. Ping-pong        & 2   & 2      & 2     & 1000     & 35     \\ \hline
A5. GHZ              & 3   & 4      & 2     & 1000     & 41     \\ \hline
\end{tabular}
\caption{Overview of application used in the evaluation described in \cref{sec:demonstrating_architecture_effectiveness}.
Each application was simulated three times, once for each hardware type (generic, NV, trapped-ion).
Each simulation was for 1000 instances of the application.
The simulation duration is an average over the three simulations per application.
}
\label{tab:app:applications}
\end{table*}


\begin{figure*}
    \centering

    \subfloat[\centering \label{fig:app:qkd_circuit}]{\includegraphics[scale=0.9]{figures/qkd_circuit.pdf}}
    \vspace{1cm}
    \subfloat[\centering \label{fig:app:teleport_circuit}]{\includegraphics[scale=0.8]{figures/teleport_circuit.pdf}}
    \vspace{1cm}
    \subfloat[\centering \label{fig:app:pingpong_circuit}]{\includegraphics[scale=0.6]{figures/pingpong_circuit.pdf}}
    \caption{
    Circuit for applications A1 (QKD), A3 (Teleport) and A4 (ping-pong) from \cref{sec:evaluation}.
    Single lines represent qubits. Double lines represent classical values.
    (a) QKD (A1). Two nodes repeatedly generate entangled pairs which are immediately measured.
    (b) Teleport (A3). A sender node (having 2 qubits) teleport a state to a receiver node. The sender applies local quantum operations (initialization, qubit rotation gates).
    The sender and receiver create an entangled pair. The sender performs local quantum gates and measurements resulting in classical outcomes.
    The sender sends the classical outcomes to the receiver. Based on the outcomes the receiver applies local quantum gates and measurement.
    (c) Ping-pong (A4). The sender teleports a state to the receiver and the receiver immediately teleports it back to the sender. In total, 2 entangled pairs are created.
    }
    \label{fig:app:circuits_1}
\end{figure*}

\begin{figure*}
    \centering

    \subfloat[\centering \label{fig:app:vbqc_circuit}]{\includegraphics[scale=0.45]{figures/vbqc_circuit.pdf}}
    \vspace{1cm}
    \subfloat[\centering \label{fig:app:ghz_circuit}]{\includegraphics[scale=0.45]{figures/ghz_circuit.pdf}}
    \vspace{1cm}

    \caption{
    Circuit for applications A2 (BQC) and A5 (GHZ) from \cref{sec:evaluation}.
    Single lines represent qubits. Double lines represent classical values.
    (a) BQC (A2). A client node remotely prepares two qubits on a server node by creating an entangled pair and locally measuring its qubits, resulting in classical outcomes $p_1$ and $p_2$.
    The server applies a local two-qubit gate (CZ or cphase) on its qubits.
    The client sends a classical value $d_1$ which it calculates based on $p_1$ and other application input values.
    The server applies local gates based on $d_1$ and measures, resulting in classical value $m_1$ which it sends to the client.
    The client sends a classical value $d_2$ which it calculates based on $p_2$, $m_1$ and other application input values.
    The server applies local gates based on $d_2$ and measures, resulting in classical value $m_2$ which it sends to the client.
    The client uses the values $m_2$ to calculate the final result (not in the Figure).
    (b) Three nodes (Alice, Bob, and Charlie) create pair-wise entangled pairs. Bob applies local gates and measures one of his qubits, sending the outcomes to Charlie.
    Based on this outcome, Charlie performs local operations and sends a measurement outcome back to Bob.
    At the time of the vertical dashed line, the three nodes share a 3-qubit GHZ state. They all measure and check their correlations.
    }
    \label{fig:app:circuits_2}
\end{figure*}



\subsection{Details for VI.A (Demonstrating the architecture's effectiveness)}
\label{app:details_6_1}
A free network schedule was used, meaning that there were no specific time slots, and entanglement generation was allowed at any time.
Such a free network schedule was justified since we considered only whether the application ran successfully

All applications were run with both (a) hardware-validated parameters (see above); all executions were successful and (b) no-noise versions of these parameters (same durations of all operations but no decoherence nor gate noise); these were used to check if the expected outcomes were obtained.

\Cref{tab:app:applications} provides an overview of the applications.

\textbf{Quantum Key Distribution (QKD)}. 
Two programs (on two nodes): Alice and Bob. $N$ EPR are pairs are generated. Each generated pair is immediately measured by both programs.
This results in both programs having $N$ classical outcome bits.
See \cref{fig:app:qkd_circuit} for the circuit.
Success per instance is determined by checking that Alice and Bob got the same $N$ outcomes bits.
For the evaluation, we ran 1000 instances, each creating 1000 EPR pairs.

\textbf{Teleportation}.
Two programs (on two nodes): Sender and Receiver. The Sender teleports a \textit{state} (which is state is an argument to the Sender program) to the Receiver.
The Receiver measures in a \textit{basis} (which basis is an argument to the Receiver program) and obtains a single classical outcome bit which is the result of the application.
See \cref{fig:app:teleport_circuit} for the circuit.
Success per instance is determined by checking that the Receiver got the expected outcome bit (which depends on the combination of state and basis).
For the evaluation, we ran 1000 instances.

For each of the Sender program instances, the \textit{state} argument was chosen evenly from the following:
$\ket{0}$,
$\ket{1}$,
$\ket{+} = 1/\sqrt{2} (\ket{0} + \ket{1})$,
$\ket{-} = 1/\sqrt{2} (\ket{0} - \ket{1})$,
$\ket{+i} = 1/\sqrt{2} (\ket{0} + i \ket{1})$,
$\ket{-i} = 1/\sqrt{2} (\ket{0} - i \ket{1})$.

For each corresponding Receiver program instance, the \textit{basis} argument was chosen such that the expected outcome bit is always 1.
Hence, a single application instance succeeded if the Receiver outcome was 1.

\textbf{Ping-pong}. Teleportation from Sender to Receiver and immediately back to Sender.
Same as teleportation application, but the Receiver does not measure; the Sender receives the state back by teleportation and measures.
See \cref{fig:app:teleport_circuit} for the circuit.
State and basis per instance were chosen similarly as for the teleportation application.
Success now depends on the Sender measurement outcome being 1.
For the evaluation, we ran 1000 instances.

\textbf{Blind Quantum Computation (BQC)}.
Two programs (on two nodes): Client and Server.
Two EPR pairs are generated, after which 2 rounds happen. In each round, the client sends a classical message to the server, after which the server performs a measurement on one of its qubits, sending the measurement outcome back.
The same BQC application was used as in~\cite{dahlberg2022netqasm}, using values $\alpha = \pi/2$ and $\beta = -\pi/2$.
See \cref{fig:app:vbqc_circuit} for the circuit.
Success per instance is determined by checking that the Client received the expected classical bit $m_2$.
For the evaluation, we ran 1000 instances.

\textbf{GHZ}. 
Three programs (on three nodes): Alice, Bob, and Charlie.
Alice creates an EPR pair with Bob, and Bob creates an EPR pair with Charlie.
Then, local gates and classical messages are sent between the nodes, resulting in a 3-qubit state (one qubit per node) that is an entangled \textit{GHZ state}~\cite{greenberger1989going}.
At the end, each program measures its own qubit.
See \cref{fig:app:ghz_circuit} for the circuit.
Success per instance is determined by checking that the all three programs got the same measurement outcome.
For the evaluation, we ran 1000 instances.

\subsection{Details for VI.B (Demonstrating Qoala's multitasking potential and Network schedule impact)}

\subsubsection{Multitasking of teleportation and of BQC}
\textbf{Sequential vs Interleaved execution.}
Sequential: All tasks for all instances were created and added to the task graph at the beginning, but additional precedence constraints were added between the last task for each instance and the first task of the next instance. This resulting in the sequential execution of the 10 instances.
Interleaved: All tasks for all instances were created and added to the task graph at the beginning, and no additional precedence constraints were added. We used an FCFS scheduler to pick tasks; since there were no precedence constraints between tasks of different instances, the execution of instances was interleaved.

\textbf{Teleportation multitasking scenario.}
One sender node and one receiver node.
The teleportation application (A3 in \cref{sec:evaluation}, see also \cref{fig:app:teleport_circuit}) was instantiated 100 times.

Classical node-node communication latency: $10^7$ ns.
Sender node: 2 qubits.
Receiver node: sweep over range $[1, \dots, 6]$.
For each number of qubits $Q \in [1, \dots, 6]$, we ran a simulation using both a sequential and an interleaved scheduling approach.

For the self-preemption case, the teleportation application was only instantiated 5 times.

\textbf{BQC multitasking scenario.}
10 client nodes and one server node.
The BQC application (A2 in \cref{sec:evaluation}, see also \cref{fig:app:vbqc_circuit}) was instantiated 10 times for each client, for a total for 100 program instances.

Classical node-node communication latency: $10^5$ ns.
Client node: 2 qubits.
Server node: sweep over $\{2, 5, 10\}$.

For each number of qubits $Q \in \{2, 5, 10\}$, we ran a simulation using both a sequential and an interleaved scheduling approach.

\textbf{QKD-BQC multitasking scenario.}
One client node and one server node.
Client and server execute both (a) 50 instances of QKD (A1 in \cref{sec:evaluation}, see also \cref{fig:app:qkd_circuit}) and (b) 50 instances of BQC (A2 in \cref{sec:evaluation}, see also \cref{fig:app:vbqc_circuit}).

We compared two network schedules.
Sequential network schedule with repeating pattern $P_{seq}$. $P_{seq}$ consists of time slots $QKD_1$, $QKD_2$, $\dots$, $QKD_{50}$, $BQC_1$, $BQC_2$, $\dots$, $BQC_{50}$.
Alternating network schedule with repeating pattern $P_{alt}$. $P_{alt}$ consists of time slots $QKD_1$, $BQC_1$, $QKD_2$, $BQC_2$, $\dots$, $QKD_{50}$, $BQC_{50}$.

\subsection{Details for VI.C (Improvement over NetQASM architecture)}

\textbf{Scenario.}
Two nodes: client and server.
The client and server execute a remote measurement-based quantum computing (MBQC) application.
The server initializes local qubits and applies two-qubit gates on them, resulting in a cluster state of three qubits.
Then, three rounds of communication happen.
In each round, the client sends a classical message containing a measurement basis to the server, the server measures one of its qubits, and finally sends the measurement outcome to the client.
After three rounds, the application ends; the last message from the server is the result of the application.
This result has an expected value, which is used to determine if a single application instance succeeded or not.
The success probability is calculated as the fraction of instances that resulted in the expected value.

We consider a program implementation $P$ for the server.
The steps of $P$ are as follows.
\begin{enumerate}
  \item (Quantum) Initialize all three qubits and apply gates until the desired cluster state is realized.
  \item (Classical) Wait for a message $\theta_0$ from the client, representing the first measurement basis.
  \item (Quantum) Measure the first qubit in basis $B(\theta_0)$, resulting in classical bit $m_0$.
  \item (Classical) Send $m_0$ to the client.
  \item (Classical) Wait for a message $\theta_1$ from the client, representing the second measurement basis.
  \item (Quantum) Measure the second qubit in basis $B(\theta_1)$, resulting in classical bit $m_1$.
  \item (Classical) Send $m_1$ to the client.
  \item (Classical) Wait for a message $\theta_2$ from the client, representing the third measurement basis.
  \item (Quantum) Measure the third qubit in basis $B(\theta_2)$, resulting in classical bit $m_2$.
  \item (Classical) Send $m_2$ to the client.
\end{enumerate}

In our evaluation, we considered a program $P_{netqasm}$ written in the NetQASM runtime format~\cite{dahlberg2022netqasm}, which would be written in Python.
Specifically, $P_{netqasm}$ contains the above steps in Python code, in the same order.
The quantum steps are converted on-the-fly into NetQASM subroutines.
This means that, in the NetQASM runtime, we have the following execution:

$P_{netqasm}$ execution:
\begin{enumerate}
  \item NetQASM subroutine for initializing the three qubits.
  \item Classical Python code for waiting for $\theta_0$.
  \item NetQASM subroutine for measuring the first qubit.
  \item Classical Python code for sending for $m_0$.
  \item Classical Python code for waiting for $\theta_0$.
  \item NetQASM subroutine for measuring the second qubit.
  \item Classical Python code for sending for $m_1$.
  \item Classical Python code for waiting for $\theta_2$.
  \item NetQASM subroutine for measuring the third qubit.
  \item Classical Python code for sending for $m_2$.
\end{enumerate}

We note that since the NetQASM runtime does not allow for compilation across classical and quantum segments of the code,
there is no way to change the order of the steps.
In our evaluation, we represented $P_{netqasm}$ as a Qoala program $Q_{netqasm}$ with the exact same contents, but with classical code represented as host code, and the NetQASM subroutines as Qoala local routines.

$P$ can be optimized by noting that some qubit operations can be delayed until a later time, decreasing the duration the some qubits have to stay in memory.
This mitigates decoherence and it is expected that overall such an optimized program $P_{opt}$ leads to a higher success probability.

The steps of $P_{opt}$ are:
\begin{enumerate}
  \item Wait for a message $\theta_0$ from the client, representing the first measurement basis.
  \item Initialize the first 2 qubits and apply gates until a partial cluster state is realized.
  \item Measure the first qubit in basis $B(\theta_0)$, resulting in classical bit $m_0$.
  \item Send $m_0$ to the client.
  \item Wait for a message $\theta_1$ from the client, representing the second measurement basis.
  \item Initialize the third qubit and apply gates until the remaining partial cluster state is realized.
  \item Measure the second qubit in basis $B(\theta_1)$, resulting in classical bit $m_1$.
  \item Send $m_1$ to the client.
  \item Wait for a message $\theta_2$ from the client, representing the third measurement basis.
  \item Measure the third qubit in basis $B(\theta_2)$, resulting in classical bit $m_2$.
  \item Send $m_2$ to the client.
\end{enumerate}
where we note that the end-to-end behavior of $P_{netqasm}$ and $P_{opt}$ are the same, and hence $P_{opt}$ is a valid optimized version of $P$.
In our evaluation, we represented $P_{opt}$ as a Qoala program $Q_{opt}$ with the exact same steps.

For the client, we used a single program implementation $Q_{client}$, optimized for Qoala.

We compared running (NETQASM): $Q_{client}$ on the client node and $Q_{netqasm}$ on the server node with (QOALA): $Q_{client}$ on the client node and $Q_{opt}$ on the server node. In both cases we instantiated the application 1000 times.
We obtained success probabilities $66\%$ for NETQASM and $82\%$ for QOALA.


\subsection{Details for VI.D (Tradeoffs between classical and quantum performance metrics)}
\textbf{Scenario.}
Two nodes: Alice and Bob.
Bob executes an interactive quantum program where classical input is given by Alice.
Bob also executes a `busy' program consisting only of CPS tasks.

\textbf{Interactive program.}
The interactive program does the following steps:
(1) prepare a local qubit in a state (state given as program instance argument) by initialization and qubit rotation,
(2) send an `acknowledge' message to Alice,
(3) wait for a message from Alice,
(4) measure the local qubit in a basis (basis given as program instance argument)
(5) return the measurement result (classical bit).
For each combination of state and basis, an expected measurement result value is computed.
The success probability of the interactive program is given by the fraction of program instances that produces the expected value.
The interactive program was instantiated 1000 times.

\textbf{Busy program.}
The busy program consists only of a block which waits for some duration (input argument).
This waiting time mimics the CPS being busy with some local classical computation.

\textbf{Fixed parameters.}
Qubit coherence times: $T_1 = 10^{10}$ ns, $T_2 = 10^8$ ns.
Classical node-to-node communication latency: $10^7$ ns.
Rate of arrival of busy programs instances: once every $10^6$ ns.


\subsection{Details for VI.E (Success probabilities with quantum multitasking)}
\textbf{Local program.}
A program which prepares a single qubit to the $\ket{-} = 1/\sqrt{2} (\ket{0} + \ket{1})$ state, then waits for duration $d$, and then measures the qubit in the $X$-basis. The expected outcome bit is hence 1.

\textbf{Scenario.}
Two nodes: Alice and Bob.
Alice and Bob execute the teleportation application (A3 in \cref{sec:evaluation}, see also \cref{fig:app:teleport_circuit}) $T$ times.
Bob concurrently executes the local program described above $L$ times.
For each combination of $T \in [1, 15]$ and $L \in [1, 15]$, we ran a simulation 200 times, where in each simulation, all instances and all their tasks are created at the same time and added to the task graphs of the node.
The success per teleportation instance is calculated as in \cref{app:details_6_1}.
Success probability is calculated as the fraction of successful instances.
The success probability of the local program is calculated as the fraction of all local program instances (across all 200 runs) gave the expected classical result 1.

\textbf{Fixed parameters.}
Qubit coherence times: $T_1 = 10^{10}$ ns, $T_2 = 10^7$ ns.
Classical node-node communication latency: $0.1$ ms.
Network schedule: repeating pattern $P$ of time slots, where $P = \langle 0, \dots, n \rangle$ where $n$ is the number of teleportation instances and where each $i$ is associated with a single teleportation instance.
Number of qubits at Bob: $10$.
Number of qubits at Alice: $20$.


\subsection{Details for VI.F (Performance sensitivity)}
\textbf{Scenario.}
One server node runs 10 BQC applications (A2 in \cref{sec:evaluation}, see also \cref{fig:app:vbqc_circuit}) concurrently with 10 client nodes (one BQC application per client node).
One BQC instance is run for each client.
This scenario was repeated 100 times for each of the three evaluations (impact of node-node-latencies, impact of internal latencies, impact of time slot length) in order to obtain statistics.
The success per BQC instance is calculated as in \cref{app:details_6_1}.
Success probability is calculated as the fraction of successful instances.

\textbf{Fixed parameters.}
Number of client nodes: 10.
Number of qubits per client node: 1.
Number of qubits for server node: 20.
Qubit coherence time: $T_2 = 1\cdot 10^7$ ns.
Network schedule: repeating pattern of 10 slots, each assigned to one client-server pair.

\textbf{Impact of node-to-node classical communication latencies.}
Network time slot length: $1\cdot 10^5$ ns.
Internal scheduler communication latency: $0$ ns.

Values used for node-to-node classical communication latencies: $10^5$, $10^6$, $10^7$ ns (i.e. $0.01$, $0.1$, $1$ times the $T_2$ coherence time).

\textbf{Impact of internal scheduler latencies.}
Network time slot length: $1\cdot 10^5$ ns.
Classical communication latency: $10^5$ ns.

Values used for internal scheduler communication latencies: $10^3$, $10^5$, $10^7$ ns, where we obtained success probabilities $0.89(2)$, $0.89(2)$, $0.83(2)$, respectively.

\textbf{Impact of network schedule time slot length.}
Classical communication latency: $10^5$ ns.
Internal scheduler communication latency: $0$ ns.

Values used for time slot length: $10^5$, $10^6$, $10^7$ ns, (i.e. $0.01$, $0.1$, $1$ times the $T_2$ coherence time).


\EOD

\end{document}
