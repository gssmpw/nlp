\section{Conclusion}
\label{sec:conclusion}
Qoala is the first architecture for executing quantum applications that addresses the need for scheduling and compiling hybrid classical\hyp{}quantum programs for a quantum internet.
This allows Qoala to ensure successful execution of quantum programs even in the presence of limited quantum memory lifetimes, and opens the door for compile-time optimizations of hybrid classical-quantum programs.
By building on an existing quantum network stack~\cite{dahlberg2019link, pompili2022experimental} and the implementation of QNodeOS on quantum hardware~\cite{pompili2022experimental, donne2024design} we pave the way for the real-world implementation of Qoala in a platform-independent way on diverse hardware platforms including NV centers in diamond~\cite{pompili2021realization, pompili2022experimental}, or trapped ions~\cite{krutyanskiy2023entanglement,krutyanskiy2023telecom} quantum processors. 
Such an implementation would require, however, a new classical control hardware as opposed to~\cite{pompili2022experimental, donne2024design}, e.g. by placing CPS and QPS on a single board with access to an on-chip shared memory. 

Our simulator implementation already now opens the door for further computer science research in executing quantum internet applications:

\textit{Advanced scheduling algorithms:}
More sophisticated scheduling strategies may lead to higher success probabilities and lower makespan when concurrently executing multiple program instances, where inspiration may come from~\cite{topcuoglu2002performance, baruah2011scheduling, andersson2006multiprocessor, polychronopoulos1991hierarchical}. 
In the quantum domain, missing the deadline will result in a degradation of the success probability as a function of the time by which the deadline was exceeded.
This suggest the use of time-utility functions (TUF, see e.g.~\cite{jensen1993timeliness, li2004utility}) to inform scheduling decisions, where it is an open question how such TUF could even be defined in the quantum domain.
Our work also raises the question on what fundamental tradeoffs between the classical (makespan) and quantum (success probability) performance metrics are at all possible.

\textit{Compiler design:}
Qoala's program format now allows for a compiler design that takes into account the hybrid and networked nature of programs.
It is an open question to design compilers enabling effective code optimization and translation of different types of high-level code into executables.

\textit{Capability negotiation:}
We assumed that the compiler provides advice that the nodes use in a capability negotiation and demand registration (\cref{sec:program_instantiation}).
It is an open question how to best compute such advise, and find efficient protocols for negotiating capabilities and register demand.

\textit{Network schedule:}
As expected, our evaluation shows that application performance depends on the network schedule, where we emphasize that ensuring network service is out of scope for Qoala as en environment for executing applications.
This highlights a need for understanding the quality of service a quantum network should provide, as well as to design good network scheduling algorithms to satisfy them, in order to achieve good application performance.
