\section{Related Works in General MFGs}
% Mean field game (MFG) ____ is a mathematical theory developed to study the strategic behavior of large populations of small interacting agents. Drawing inspiration from mathematical physics, specifically from models that examine the collective dynamics of numerous identical particles, the term ``mean field" captures this essence well. It was first introduced in the engineering community by Huang et al. ____ and independently by Lions and Lasry ____ in academia. The theory of MFG was expanded by Gomes et al. ____, and further by Carmona et al. ____, who incorporated the aspect of common noise into the framework. Huang ____ and
% Carmona and Zhu ____ explored the dynamics of MFG featuring a dominant player whose impact remains significant even as the player count approaches infinity. They introduced a probabilistic method, revealing that the Nash Equilibrium serves as the resolution to the intertwined system of an MFG and a Mean Field Control problem.   
The mean field game (MFG) ____ is a mathematical theory used to analyze the strategic behavior of large groups of small interacting agents. The term "mean field" is inspired by mathematical physics, specifically models that examine the collective dynamics of numerous identical particles. Huang et al. ____ introduced it to the engineering community, followed by Lions and Lasry ____ in academia. Gomes et al. ____ expanded the theory of MFG, while Carmona et al. ____ added common noise to the framework. Huang ____ and Carmona and Zhu ____ studied how a dominant player can still have a significant impact even as the number of players increases. They introduced a probabilistic method, which revealed that the Nash Equilibrium is the solution to the intertwined system of an MFG and a Mean Field Control problem.   
\end{comment}