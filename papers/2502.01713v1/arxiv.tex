\documentclass{article}

\usepackage{arxiv}
\usepackage[numbers]{natbib}
\usepackage[utf8]{inputenc} %
\usepackage[T1]{fontenc}    %
\usepackage{hyperref}       %
\usepackage{url}            %
\usepackage{amsmath} 
\usepackage{listings}
\lstset{breaklines=true}
\usepackage[ruled, linesnumbered, noalgohanging]{algorithm2e}
\newcommand{\thought}[1]{{\color[rgb]{0.2,0.39,0.66}(#1)}}
\newcommand{\todo}[1]{{\color[rgb]{1.0,0.0,0.0}(#1)}}
\newcommand{\hsh}[1]{{\color{green!50!black} Henrik: #1}}
\newcommand{\st}[1]{{\color{red!50!black} Sebastian: #1}}

\newcommand{\ulm}[1]{_{\scaleto{\mathrm{#1}}{3pt}}}
\newcommand\at[2]{\left.#1\right|_{#2}}











\newtheorem{assumption}{Assumption}

\DeclareMathOperator*{\argmax}{arg\,max}
\DeclareMathOperator*{\argmin}{arg\,min}

\newcommand{\swname}[1]{\texttt{#1}}
\newcommand{\ie}{i\/.\/e\/.,\/~}
\newcommand{\eg}{e\/.\/g\/.,\/~}
\newcommand{\cf}{cf\/.\/~}

\newcommand{\fig}{Fig\/.\/~}
\newcommand{\defn}{Def\/.\/~}
\newcommand{\sect}{Sec\/.\/~}
\newcommand{\tabl}{Tab\/.\/~}
\newcommand{\algo}{Algorithm~}
\newcommand{\theo}{Theorem~}

\newcommand{\bnnl}{3 hidden layers}
\newcommand{\bnnn}{50 neurons}
\newcommand{\bnna}{tanh activations}

\newcommand{\capt}[1]{\mdseries{\emph{#1}}}

\newcommand{\videolink}{at \url{https://youtu.be/_d7AqTRjz6g}}
\newcommand{\codelink}{\url{https://github.com/wheelbot/mini-wheelbot}}

\newcommand{\fakepar}[1]{\vspace{0mm}\noindent\textbf{#1.}}

\newcommand{\needref}{\textcolor{red}{[REF]}}

\newcommand{\plotfontsize}{9pt}

\usepackage{subcaption}
\usepackage{multirow}
\usepackage{amsmath} 
\usepackage{float}
\usepackage[libertine]{newtxmath}
\usepackage{booktabs}

\newcommand{\barM}{\bar{M}}
%
\newcommand{\boldx}{\boldsymbol{x}}
\newcommand{\boldxsub}[1]{\boldsymbol{x}_{#1}}
\newcommand{\boldX}{\boldsymbol{X}}
\newcommand{\ysub}[1]{y_{#1}}
\newcommand{\asub}[1]{a_{#1}}
\newcommand{\protectedgroup}{demographic group}

\newcommand{\boldmu}{\boldsymbol{\mu}}
\newcommand{\boldSigma}{\boldsymbol{\Sigma}}

\newcommand{\littlespace}{\hspace{0.25cm}}
\newcommand{\nperm}{n_{\mathrm{perm}}}




\title{Auditing a Dutch Public Sector Risk Profiling Algorithm Using an\\ Unsupervised Bias Detection Tool}

\author{{Floris Holstege}\thanks{Corresponding author} \\
    University of Amsterdam \\
    The Netherlands \\
    \texttt{f.g.holstege@uva.nl} \\
	%
	\And
	{Mackenzie Jorgensen} \\
	King's College London \\ 
    Alan Turing Institute \\
    UK\\
	\And
	Kirtan Padh \\
	TU Munich\\ 
    Helmholtz Munich \\
	Germany \\
	\AND
	Jurriaan Parie \\
	Algorithm Audit \\
    The Netherlands \\
    \texttt{j.parie@algorithmaudit.eu}\\
    \And
	Joel Persson \\
	Algorithm Audit \\
    UK \\
    \And
    Krsto Prorokovic \\
	Algorithm Audit \\
    The Netherlands \\
    \texttt{info@algorithmaudit.eu}\\
    \And
    Lukas Snoek \\
    Algorithm Audit \\
    The Netherlands 
}
%
%
\begin{document}
\maketitle





%
\begin{abstract}

Algorithms are increasingly used to automate or aid human decisions, yet recent research shows that these algorithms may exhibit bias across legally protected demographic groups. However, data on these groups may be unavailable to organizations or external auditors due to privacy legislation. This paper studies bias detection using an unsupervised clustering tool when data on demographic groups are unavailable. We collaborate with the Dutch Executive Agency for Education to audit an algorithm that was used to assign risk scores to college students at the national level in the Netherlands between 2012-2023. Our audit covers more than 250,000 students from the whole country. The unsupervised clustering tool highlights known disparities between students with a non-European migration background and Dutch origin. Our contributions are three-fold: 
(1) we assess bias in a real-world, large-scale and high-stakes decision-making process by a governmental organization; 
(2) we use simulation studies to highlight potential pitfalls of using the unsupervised clustering tool to detect true bias when demographic group data are unavailable and provide recommendations for valid inferences; 
(3) we provide the unsupervised clustering tool in an open-source library. Our work serves as a starting point for a deliberative assessment by human experts to evaluate potential discrimination in algorithmic-supported decision-making processes.
\end{abstract}


\keywords{fairness \and auditing \and bias \and clustering \and indirect discrimination}


\section{Introduction}
\label{sec:intro}
The use of algorithmic support for decision-making has grown in recent years~\citep{Rudin2020}, influencing decisions such as who gets a loan, who makes it to the next recruitment stage for a job, and who is flagged for fraud investigation~\citep{Reuters2018, Raghavan2020, NYT2017}. As algorithmic systems are adopted, their negative impacts have also become more clear~\citep{Angwin2016,Veale2017,Lambrecht2019, Ali2019,Raghavan2020}. These systems may replicate the bias we see in our society, leading to potentially discriminatory effects~\citep{Speicher2018}. In this study, we outline a methodology to detect such biases in algorithms without data on demographic groups using an unsupervised bias detection algorithm, evaluated on a Dutch public sector risk profiling algorithm.

In the period 2012-2023, a simple rule-based risk profiling algorithm of the Dutch Executive Agency for Education (DUO) contributed to indirect discrimination in a control process to check whether students were (un)duly allocated a college grant~\cite{DutchParliament2024D07565}. A rule-based algorithm, consisting of three apparently neutral characteristics of students (type of education, age, distance to parents), was used to assign higher risk scores to students enrolled in vocational education, younger students, and students who lived near their parental address. The seemingly impartial risk assessment led to unequal treatment of students with a non-European migration background, remaining unnoticed for over 10 years and affecting more than 350,000 students\footnote{Between 2012-2023, over 350,000 students were assigned a risk score by the algorithm. Our analysis focuses on the student population in two years: 2014 and 2019, which consists of a total of 298,882 students.}~\cite{DutchParliament24724-240}. Given this, in 2024 the Dutch Minister for Education, Culture and Science apologized on behalf of the Dutch government for indirect discrimination experienced by students with a non-European migration background and announced a EUR 61 million compensation plan~\cite{DutchParliament24724-240}. This example of a new discriminatory algorithm scandal in the Netherlands~\citep{politico22} highlights that not only complex statistical methods, like machine learning and deep learning~\citep{cdei20}, but also simple rule-based algorithms pose significant risks of embedding discrimination in algorithm-supported decision-making processes. In a recent interview, the chair of the Dutch Data Protection Authority warned that discriminatory rule-based algorithms are ``around every corner'' in executive branches of the Dutch government~\citep{VK24}.

This type of risk profiling is prohibited by the Charter of Fundamental Rights of the European Union (EU)~\citep{charter}, the EU General Data Protection Regulation (GDPR)~\citep{gdpr_2016} and the EU AI Act~\citep{AIAct}, as well as the national Public Administration Law~\citep{awb}. Limited institutional enforcement capabilities, along with a lack of awareness, have led to inadequate oversight of algorithmic-supported decision-making processes~\citep{meuwese2024algoprudence}. 

To combat bias in algorithmic systems, the academic community has presented a multitude of ways to define, detect, and mitigate different notions of bias at different points along the algorithmic pipeline~\citep{Mehrabi2021, Chouldechova2020, Barocas2019, Veale2018}. Most bias mitigation methods require data on the demographic groups to prevent discriminatory bias. These demographic groups are typically protected attributes under discrimination law, such as race, gender and age\footnote{The exact protected grounds are enshrined in European non-discrimination law, such as the Treaty on the Functioning of the European Union (TFEU), the Charter of Fundamental Rights of the EU, and specific directives like the Racial Equality Directive (2000/43/EC) and the Employment Equality Directive (2000/78/EC). The main protected attributes include: race and ethnicity, gender and sex, religion or belief, disability, age, sexual orientation, nationality and marital or family status. Throughout, we will refer to these attributes as \emph{{\protectedgroup}s}. Note that some provisions do not mention any particular grounds of discrimination, such as Art. 20 of the EU Charter. Instead using openly formulated clauses such as "everyone is equal before the law”.}~\citep{ashurst2023fairness}. 
However, data on demographic groups are often not available to organizations or external algorithm auditors~\citep{McKane2021,cdei2023}. Upon request, public sector organizations can occasionally obtain aggregated statistics on demographic groups through official institutions, such as the National Office of Statistics, which have access to statistical population registers. Private organizations cannot request these data and face ethical, legal, and practical challenges when collecting data on demographic groups customers belong to ~\citep{cdei2023,gdpr,AIAct}. Unsupervised learning can aid in this challenge by detecting groups in the data for which a given notion of equal treatment is violated, avoiding the need for ex-ante specification of sensitive groups~\citep{MISZTALRADECKA2021102519}. 

%
In this paper, we use an unsupervised bias detection tool to audit the risk profiling algorithm of DUO on bias. For this audit, the country of birth and country of origin were provided by Statistics Netherlands~\citep{CBS}. These data enable us to verify whether the clusters identified by the unsupervised bias detection tool, which represent potentially unfairly treated groups, correspond to disadvantaged demographic groups. This allows for a comparison between an unsupervised learning approach and an approach using demographic data. 
As a bias metric, we use whether or not a student is categorized as ``high risk'' by DUO's risk profiling algorithm. We find that the cluster with the highest bias returned by the unsupervised bias detection tool is characterized by students who are more often enrolled in vocational education, live close to their parental address, and are relatively more often 15-18 or 25-50 years old. This aligns with demographic groups found to be negatively affected by the risk profiling algorithm, based on aggregated statistics regarding students' non-European migration background. 

Our paper makes three contributions:
\begin{enumerate}
    \item \textbf{Large scale audit of indirect discrimination}: The DUO audit presents a high-quality and socially relevant dataset, providing insights into the assessment of indirect discrimination in algorithm-supported decision-making. This study thereby contributes to the advancement of public knowledge building and establishment of standards for auditing algorithmic systems on indirect discrimination. 
    \item \textbf{Advancing methodology for unsupervised bias detection}: We build on existing clustering algorithms and strengthen their methodology. Using a simulation study, we examine conditions under which the tool works best and highlight key factors to consider for reliable inferences when applying it. 
    \item \textbf{Open-source tool}: We have published the unsupervised bias detection method as an open-source tool in the form of a Python package, as well as an online web application. This allows a non-technical audience to easily use the tool on their own datasets to detect potential bias.
\end{enumerate}
Overall, our paper showcases the potential of unsupervised learning to aid in auditing algorithmic-supported decision-making processes via an application to a real-world use case. We argue that unsupervised bias detection serves as a starting point for a deliberative assessment by human experts to evaluate potential discrimination and unfairness in the algorithmic-supported decision-making process under review.

The rest of the paper is outlined as follows. We start by reviewing related literature in Section \ref{sec:lit}. We present the methodology behind the unsupervised bias detection tool in Section \ref{sec:tool}. Then, in Section \ref{sec:DUO}, we describe the application of the unsupervised bias detection tool to DUO's risk profiling algorithm. We follow this by providing simulation studies in Section \ref{sec:simulation} which help to understand potential pitfalls when using the tool. We finish with a discussion on the limitations and conclusions of our work in Sections \ref{sec:disc} and \ref{sec:conc}.

\section{Literature Review}
\label{sec:lit}
We review a selection of the literature on bias detection and non-discrimination law, structuring the review around three topics: (1) challenges in detecting and establishing indirect discrimination, (2) bias detection methods with and without access to demographic data, and (3) auditing algorithmic systems on bias.

\subsection{Challenges in Detecting and Establishing Discrimination}

Under European non-discrimination law, DUO's control process is classified as indirect discrimination as an apparently neutral practice disproportionately disadvantaged students with a non-European migration background\footnote{Supra note 2}~\cite{BilkaKaufhaus}. In algorithms, discrimination often arises from the \emph{proxy and correlation challenge}~\cite{EC_algorithmicdiscrimination}. Proxies refer to characteristics that are correlated with a {\protectedgroup}, leading to disadvantaged outcomes for this group. For example, the Court of Justice of the European Union (CJEU) found that using part-time employment as a basis for differentiation in the context of occupational pension schemes constituted indirect discrimination, as over 80\% of the part-time workers in Spain in 2012 were women~\citep{MorenoInstitutoNacional}. 
In some cases, courts may accept that differential treatment of a {\protectedgroup} is acceptable. Under EU law, indirect discrimination can be justified if it is demonstrated that a legitimate aim is being pursued, and the means of achieving that aim are appropriate and necessary~\citep{EmploymentEqualityDirective,RacialEqualityDirective,GoodsandServicesDirective,RecastGenderEqualityDirective}. Our paper highlights how unsupervised methods can aid in the challenge of assessing indirect discrimination in algorithmic-supported decision-making processes.

\subsection{Bias Detection With and Without Access to Demographic Groups}
In the algorithmic fairness literature, identifying systematic differences in the treatment of individuals or groups compared to others is commonly referred to as \emph{bias detection}. Most bias detection methods assume access to demographic groups such as gender or ethnicity~\cite{Aif360,Hardt2016, Kusner2017, Kleinberg2018,delaney2024oxonfair}. However, in practice, these labels are often unavailable due to privacy legislation~\citep{van2023using}. For instance, the EU's GDPR restricts the collection of ``special categories of data'' (article 9)~\citep{gdpr}, which include data on ethnicity, religion, health, and sexual orientation\footnote{Age and gender are not considered special categories of data. See Article 9(1) GDPR. Although, a caveat could exist~\citep{van2018trans}. The other way around, ``political opinions,'' ``trade union membership,'' ``genetic,'' and ``biometric'' data are special categories of data but are not protected by EU non-discrimination directives.}. The EU AI Act introduces a potential exception to the prohibition on processing special category data: ``to the extent that it is strictly necessary for the purpose of ensuring bias detection and correction''~\cite{AIAct}. Nonetheless, ongoing legal uncertainties regarding the interplay between the AI Act and the GDPR underscore the need for methods to assess bias in algorithmic systems without relying on demographic data.

There are several alternatives to using data on demographic groups. First, researchers have presented methods to predict demographic groups, which are often referred to as proxy models or attribute classifiers~\cite{ashurst2023fairness}. For example, Bayesian Improved Surname Geocoding (BISG)~\cite{Elliot2009} uses Bayes Theorem and data from the United States Census Bureau, including their surname and address, as a proxy for ethnicity or race~\citep{ashurst2023fairness}. This set of methods does require to specify the demographic group of interest beforehand. A second set of methods circumvents the problem of limited availability of protected characteristics by using a Rawlsian definition of fairness~\citep{rawls2001justice} that maximizes utility for the most disadvantaged group~\citep{lahoti2020fairness, hashimoto2018fairness, chai2022self}. However, the outcome of these methods is found to not always adhere to more traditional parity-based group fairness metrics~\citep{islam2024fairness}. The third set of alternatives are unsupervised learning methods for fairness~\citep{Nasiriani_2019, 2021BMVC_UDIS}. These methods tend to modify traditional clustering algorithms to handle fairness criteria explicitly. Our work contributes to this latter category. The main benefit of this set of methods compared to proxy inferring methods is that they do not require specification of demographic groups beforehand. Thus, unsupervised learning methods are particularly helpful to examine the effect of apparently neutral provisions that might result in indirect discrimination. 


\subsection{Auditing Algorithmic Systems}
Inspired by established practices in non-algorithmic disciplines, auditing has the potential to mitigate ethical and legal risks in algorithmic systems through both internal and external oversight~\citep{raji2022outsider}. Legislators across various jurisdictions are implementing regulations to impose audit frameworks for these systems. For instance, since 2023, New York City has required automated employment decision-making tools to undergo audits by independent third-party auditors~\citep{LL144}. Similarly, since 2024, the EU Digital Services Act mandates annual independent third-party audits for online platforms and search engines with over 45M annual users~\cite{DSA}. In the coming years, the EU AI Act places predominantly self-regulatory obligations on producers and deployers of high-risk AI systems~\cite{AIAct}. 

Implementing effective auditing regimes for algorithmic systems, however, poses practical and cultural challenges. Auditors frequently encounter obstacles such as restricted data access~\citep{groves24}, quandaries about which fairness metrics fit the context best~\citep{corbett2023measure} and varying opinions on what constitutes a legitimate auditor~\citep{groves24}. Furthermore, auditing reports are not always required to be disclosed, reducing transparency and hindering the development of public knowledge regarding auditing standards. By discussing the audit of DUO's risk profiling algorithm, we contribute to public knowledge on best practices for auditing algorithmic systems.

%

\section{Unsupervised Bias Detection Tool}\label{sec:tool}
This section introduces the theoretical background of the unsupervised bias detection tool. We use examples from our DUO case study to clarify the methodology. Figure \ref{fig:overall} provides an overview of all steps in applying the tool.

\begin{figure}[H]
     \centering
     \includegraphics[width=0.75\linewidth]{Fig/overview_tool.png}
\caption{Schematic overview of the steps involved in applying the unsupervised bias detection tool. The required information is a dataset, classifier, and a bias metric. 
    Part of the dataset is used to train the Hierarchical Bias-Aware Clustering algorithm \citep{MISZTALRADECKA2021102519}. Another part of the dataset is used to test whether differences in the bias metric across clusters are statistically significant.}
     \label{fig:overall}
\end{figure}

\subsection{Notation and Problem Description}
Suppose that we have a total of $N$ students in the dataset, labeled $u_1, u_2, \ldots u_N$. Each row in the dataset reflects characteristics $\boldxsub{i} \in \mathbb{R}^d$ (e.g., education level, age, distance to parents) of a student $u_i$ receiving a grant. Suppose also that we have a risk profiling algorithm $f: \mathbb{R}^d \rightarrow \{0, 1\}$ (e.g., the risk profiling algorithm used by DUO) that takes the features of a user as input and outputs whether a user should be deemed high risk of unduly using a grant. There exists a true value of this variable of interest, denoted $y_i$, denoting whether the grant was, in fact, duly given (or not). 

We measure the degree to which an algorithm is biased by using a bias metric $M$. For a user $u_i$, the value of the metric $m_i = M(f(\boldxsub{i}), \ysub{i})$ is some function of the prediction $f(\boldxsub{i})$ and (potentially) $\ysub{i}$. To give some concrete examples, this could be a measure of performance, such as accuracy (requiring $f(\boldxsub{i}), \ysub{i}$). Alternatively, it could measure demographic parity (only requiring $f(\boldxsub{i})$), equality of opportunity, and so on. Without loss of generality,  we assume that a larger average value for $m_i$ for one group compared to another group indicates bias towards the former group.

For $k \subset \{1, 2,\ldots,N\}$, let $U_k = \{u_i\}_{i \in k}$ be a subset of all applicants $U$. We denote by $\barM(U_k)$ the mean of the metric $m_i$ among all in $U_k$. Our goal is to test whether the difference between $\barM(U\backslash U_k)$ and $\barM(U_k)$ is statistically significant. If we had access to a demographic group for which we would like to investigate bias, such as race or gender, we would simply define the subset $U_k$ to be different genders, race or intersections thereof. However, since the demographic group is unavailable in our setting, we leverage clustering algorithms to separate the data into subsets.  Let $\mathcal{K} = \{1, ..., K\}$ denote the set of clusters, and  $\boldxsub{i, k}, y_{i, k}, m_{i, k}$ to denote data from a user $u_i$ belonging to cluster $k \in \mathcal{K}$.


\subsection{Clustering Algorithm}
We use the Hierarchical Bias-Aware Clustering methodology from \citet{MISZTALRADECKA2021102519}, hereafter referred to as ``HBAC.'' This is an iterative clustering algorithm that produces a partition of the original dataset to isolate potentially unfairly treated groups into distinct clusters. We adopt this algorithm because it is a standard approach for unsupervised bias detection in the realm of recommendation systems \citep{deldjoo2024fairness}.


The algorithm first considers all observations in the dataset to belong to a single cluster and then iteratively splits the observations into distinct clusters. At each iteration, the algorithm identifies the cluster with the highest standard deviation of a specified bias metric, which serves as an indicator of imbalance between clusters. This cluster is then split into two subclusters. The process repeats for a predefined number of iterations, $max\_iterations$. Pseudocode for the algorithm is provided in Algorithm \ref{alg:hbac}. Compared to the original algorithm from \citet{MISZTALRADECKA2021102519}, we allow for clusters to be defined via $k$-modes in addition to $k$-means to allow clustering based on categorical and binary variables. 



\begin{algorithm}[h]
\begin{small}
\KwIn{A dataset $\mathcal{X} = \{ x_1, \ldots, x_N\}$ and bias metric $\{m_1, \ldots m_n\}$. Set the \textit{max\_iterations} and a minimum of samples per cluster $n_{\mathrm{min}}$. }
\KwOut{A partition $\{ \mathcal{C}_1,  \ldots \mathcal{C}_k\}$}
Define the partition = $\{\mathcal{X}\}$ \\
\For{$i \gets 1$ \KwTo max\_iterations} {
    Set $\mathcal{C}$ to be the cluster in partition with the highest standard deviation of metric $M$ among those that have not been selected in any previous iteration. \\
    Split $\mathcal{C}$ into two clusters $\mathcal{C}^{'}$ and $\mathcal{C}^{''}$ using $k$-means or $k$-modes \\
    \If{$\max(\barM(\mathcal{C}^{'}), \barM(\mathcal{C}^{''})) \geq \barM(\mathcal{C}) \littlespace \land \littlespace |\mathcal{C}^{'}| \geq n_{\mathrm{min}} \littlespace \land \littlespace |\mathcal{C}^{''}| \geq n_{\mathrm{min}}$  }{
        Remove $\mathcal{C}$ from partition \\
        Add $\mathcal{C}^{'}$ and $\mathcal{C}^{''}$ to partition
    }
}
\end{small}
\caption{Hierarchical Bias-Aware Clustering}
\label{alg:hbac}
\end{algorithm}

The original algorithm from \citet{MISZTALRADECKA2021102519} does not specify how to assign clusters to data points that are not part of the original training dataset. We propose using the centroids of each cluster. The datapoint is assigned to the cluster whose distance from the centroid is the smallest. In the case of $k$-means and $k$-modes, the centroids are defined as respectively the mean and mode for each of the characteristics $\boldxsub{i, k}$ \citep{James2023, Huang_1998}. The Euclidean distance ($k$-means) or Hamming distance ($k$-modes) is used to determine which centroid is closer. 


\subsection{Minimum Number of Samples per Clusters}

Algorithm 1 has been claimed to not require a specification of the number of clusters \citep{MISZTALRADECKA2021102519}. Although this is true, other hyperparameters of the algorithm such as $n_{\mathrm{min}}$ -- the minimum number of samples per cluster -- affect the number of clusters it finds. To determine the appropriate minimum number of samples per cluster, we use the Calinksi-Harabasz index \citep{calinski1974dendrite} of the bias metric. This index measures the ratio of the \textit{between} cluster variance over the \textit{within} cluster variance. Intuitively, we want the clusters to have little variation in the bias metric \textit{within} each cluster, but large variation \textit{between} the clusters. This means that the data points are well separated into distinct clusters.

\subsection{Testing Differences in Bias Between Clusters}

After conducting the clustering, our goal is to test for each of the clusters the following null versus alternative hypothesis:
\begin{align}
    H_0: \barM(U\backslash U_k)  = \barM(U_k), \littlespace H_A: \barM(U\backslash U_k)  \neq \barM(U_k).
\end{align}
If $\barM(U\backslash U_k) \neq \barM(U_k)$, the bias is found to be higher or lower for a particular cluster found by the clustering algorithm.  To test these hypotheses, we use a two-sample $t$-test when the bias metric is continuous, and Pearson's $\chi^2$-test when the bias metric is a binary variable. For either case, the test is two-tailed, as we allow $\barM(U_k)$ to be both lower and higher than $\barM(U\backslash U_k)$, meaning that we can detect bias both when it is greater or less for a group compared to the rest.

For valid statistical inference, the null and alternative hypothesis must be defined before observing the data. If we conduct our statistical test on data which was also used to identify clusters, this condition is violated.  Since the clusters are selected to maximize the difference in the bias metric, we will therefore tend to incorrectly conclude that there is bias when there truly is none.
The broader issue of testing data-driven hypotheses---often called \emph{post-selection inference}---has been studied extensively in the context of regularized regression (for an overview, see \citet{Kuchibhotla2022}). A long-standing practice in that literature is to use different partitions of the data for model estimation and inference, referred to as \textit{sample splitting} \citep{larson1931shrinkage, stone1974cross}. We apply this technique by fitting the HBAC algorithm on a subset of the data and subsequently inferring the clusters and their bias on a \textit{new} subset of data. Importantly, this set cannot be used to determine our hyperparameters.

When testing for multiple $K > 2$ clusters, an issue is that without adjustments, the likelihood of (incorrectly) detecting a statistically significant difference under the null hypothesis increases. We address this by using a Bonferroni correction \citep{hochberg1987multiple}. In Section \ref{sec:simulation}, we illustrate that the combination of the Bonferroni correction and sample splitting prevents us from (wrongly) concluding that there is a difference in the bias metric while there is none.


\section{Detecting Bias in DUO's Risk Profiling Algorithm}
\label{sec:DUO}
The unsupervised bias detection tool will be applied to the DUO dataset. Because this dataset includes ground truth labels for the demographic group, in the form of aggregated statistics on the migration background of students, our case study is suited to evaluate the effectiveness of the tool in detecting bias.  

For the DUO audit, researchers were granted access to aggregated statistics on students' migration by Statistics Netherlands~\citep{CBS}. If these data had not been available, the unsupervised bias detection tool discussed in this paper could have been used as an alternative methods for examining potential bias in the CUB process.


\subsection{Description of the Dataset}\label{sec:dataset}
From 2012 to 2023, DUO aimed to verify that college grants were allocated to students who genuinely lived away from their parents. Some students claimed to DUO an independent living status to qualify for a grant, even though they did not actually reside at the address they provided~\citep{DutchParliament2024D07565}. The process used to verify whether a student lived at the stated address is known as the \emph{CUB process}\footnote{CUB stands for \textit{Controle Uitwonendenbeurs}.}. As part of the CUB process, students were assigned a risk score ranging between 0 and 180 by a rule-based algorithm, of which the details are provided in Appendix~\ref{sec:Appendix_Risk_profile}. This risk score was binned in six risk categories ranging from very high (1), high (2), average (3), low (4), very low (5) to unknown (6)\footnote{Risk category 6 includes students with a risk score of 0. Risk categories 5, 4, 3, 2 and 1 correspond to students with risk scores in the ranges of 1-19, 20-39, 40-59, 60-79 and 80-180 respectively. See also Appendix~\ref{sec:Appendix_Risk_profile}.}. Students in category (1) and (2) were considered to be ``high risk'' of misusing the subsidy. Based on the assigned risk category, human analysts of DUO followed operational instructions to subject students to an investigation. For example, students living in a care home and married students were excluded from investigation. While students from all risk categories could potentially be investigated, the majority of investigations focused on higher-risk categories~\citep{DutchParliament2024D07565}. 

After concerns were raised regarding potential bias in the CUB process affecting students with a non-European migration background, DUO requested the Dutch national office of statistics (Statistics Netherlands) to provide aggregated statistics\footnote{Through protocols established by Statistics Netherlands, aggregated population statistics can be computed securely. Data were analyzed in a privacy-preserving manner within a secure environment. Groups are rounded to the nearest ten. Results pertaining to groups smaller than 10 people are not published.} of student's country of origin and country of birth \footnote{Following Statistics Netherlands, students with a non-European migration background are defined as those who were either born outside the Netherlands and Europe, or who have at least one parent born outside these regions. Countries that are considered to be European can be found here~\citep{CBScountries}.} who received a college grant for the years 2014, 2017, 2019, 2021, and 2022~\citep{CBS}. From these data the non-European migration background of students can be retrieved in aggregate. We do not have information for individual students about whether or not they have a non-European migration background.

This paper focuses on the years 2014 and 2019, covering in total 298,882 students who received college grants. We specifically focus on the CUB population from 2014 ($n=248,649$) as this was the last year all types of students -- MBO 1-2 and MBO 3-4 (vocational education), HBO (higher professional education) and WO (university education) -- were awarded a college grant. We do not combine data from the different years since in 2019 only MBO students were eligible to receive a college grant. Results are replicated for the CUB 2019 dataset ($n=50,233$ students) in Appendix \ref{sec:CUB-2019}. 

DUO's risk profiling algorithm differentiated on the basis of three variables. First, the type of education the student is following, categorized as MBO 1-2, MBO 3-4 (vocational education) HBO (higher professional education) and WO (university education). Second, the age of the student, categorized as: 15-18, 19-20, 21-22, 23-24 and 25-50 years. Third, the distance between the address of the parents and the student, measured through the 6-digit postal code of the registered addresses and categorized as: 0km\footnote{0km indicates that student and parents share the same postal code but not the same house number.}, 1m-1km, 1-2km, 2-5km, 5-10km, 10-20km, 20-50km, 50-500km, unknown.

In 2014, 34,050 students were assigned the risk category 6 (unknown), e.g., because the distance to their parents was unknown. We exclude these students from the clustering analysis, since the algorithm did not make a prediction on their risk of unduly using the grant, and these cases were deprioritized in the control process~\citep{DutchParliament2024D07565}. Clustering will therefore be applied to remaining dataset of $n=214,599$ students. 

\subsection{Applying the Unsupervised Biased Detection Tool}
The following choices were made in applying the unsupervised bias detection tool to the CUB 2014 dataset.

\textbf{Data preprocessing:} We transform the 3 categorical profiling characteristics to a set of 17 binary (non-overlapping) variables. We define no ordinal relationship between categories. We use the $k$-modes clustering within Algorithm \ref{alg:hbac}.

\textbf{Bias metric}: We select whether a student is deemed ``high risk'' by the risk profiling algorithm or not as the bias metric. If the likelihood of being classified as ``high risk'' is equal for students of Dutch origin and those with a non-European migration background, it would correspond to the fairness metric--- demographic parity~\cite{dwork12}. Alternative definitions of bias often whether a college grant was duly granted ($y_i$), which in this case is only available for the small set of students who have manually been controlled by DUO ($n=3,179$). As classifications by the risk profiling algorithm contributed to discrimination in the CUB process~\citep{DutchParliament24724-240}, we consider this an appropriate bias metric. We emphasize that the HBAC algorithm can be used for each bias metric of interest.

\textbf{Hyperparameters}: The minimum number of samples per cluster is selected by iterating over 5,000, 10,000, 20,000, 30,000 and selecting the one which maximizes the average Calinksi-Harabasz index, measured via 5-fold cross-validation. We set the maximum number of iterations sufficiently high ($max\_iterations = 1,000$), such that Algorithm \ref{alg:hbac} continues until the minimum number of samples per cluster is reached. The initial centroids are determined via the method proposed by \citet{Cao_2009}.

\textbf{Testing procedure:} We split our dataset and reserve 20\% of the dataset for testing the differences in the bias metric per cluster. We fit the unsupervised bias detection tool (and determine hyperparameters) on the remaining 80\% of the data. Since our bias metric is binary, we use a Pearson's $\chi^2$-test. 




\subsection{Analyzing Identified Clusters}
The unsupervised bias detection tool returns three clusters for the CUB 2014 dataset. The characteristics of these clusters are analyzed.

\textbf{Cluster size:}
Cluster 1 consists of $56,354$ 
 $(26.3\%) $ students, where cluster 2 consists of $123,203$ $ (57.4\%)$ students and cluster 3 of $35,042$ $(16.3\%)$ students.

\textbf{Testing the difference in bias metric per cluster:}
We find stark differences in the bias metric across the three clusters. In the cluster with the highest bias (cluster 3), 50\% of the students are classified as ``high risk'' by the risk profiling algorithm, versus 3\% and 19\% respectively in clusters 1 and 2. These differences are statistically significant at the 0.1\% level, as indicated in  Table~\ref{tab:test_2014} in Appendix~\ref{sec:tests}, where we report differences between $\barM(U\backslash U_k)$ and $\barM(U_k)$ for the validation set separated for testing, including the $p$-values. 


\begin{figure}[H]
 \centering
\begin{subfigure}{.25\textwidth}
  \centering
  %
\includegraphics[scale=0.195]{Fig/DUO/2014/predicted_class_per_cluster.png}  
\caption{}
\label{fig:predicted_class_cluster}
\end{subfigure}
\begin{subfigure}{.5\textwidth}
  \centering
  %
\includegraphics[scale=0.195]{Fig/DUO/2014/non_eu_mig_back_per_cluster.png}  
  \caption{}
    \label{fig:non_eu_mig_back_cluster}
\end{subfigure}
\caption{The percentage of students (a) deemed as ``high risk'' and (b) estimated as having a non-European migration background for identified clusters based on the student population of 2014, excluding students for which the risk profile was unknown ($n=214,599$).}
\label{fig:bivariate_bias_background}
\end{figure}

\begin{figure}[H]
  \centering
\begin{subfigure}{.25\textwidth}
  \centering
  %
\includegraphics[scale=0.1875]{Fig/DUO/2014/education_cluster_plot.png}  
  \label{fig:cluster_educ}
  \caption{}
\end{subfigure}
\begin{subfigure}{.3\textwidth}
  \centering
  %
\includegraphics[scale=0.1875]{Fig/DUO/2014/age_cluster_plot.png}  
  \label{fig:cluster_age}
  \caption{}
\end{subfigure}
\begin{subfigure}{.44\textwidth}
  \centering
  %
\includegraphics[scale=0.1875]{Fig/DUO/2014/distance_cluster_plot.png}  
  \label{fig:cluster_distance}
  \caption{}
\end{subfigure}
\caption{The percentage of students in each cluster for the three characteristics used in the risk profiling algorithm: (a) type of education, (b) age and (c) distance to parents within the student population of 2014,  excluding students for which the risk profile was unknown ($n=214,599$). For each subgroup, the average is indicated by the dashed line.}
\label{fig:cluster_characteristics}
\end{figure}


\textbf{Student characteristics per cluster:}
Figure \ref{fig:cluster_characteristics} illustrates the characteristics of students within each cluster. Students in cluster 3, associated with the highest average bias, are more likely to be enrolled in vocational education (MBO 1-2, MBO 3-4), be overrepresented in age groups 15-18 and 25-50, and live on average closer to their parents. In contrast, students in cluster 1, which exhibits the lowest average bias, exclusively attend university, typically live far from their parents and are  more likely to be 19-20 years old. Students in cluster 2 tend to follow higher professional education (HBO), do not display distinct patterns regarding age and live slightly more often 20-50km away from their parents.

\subsection{Comparing Results with Aggregate Statistics on non-European Migration Background}
We now examine the extent to which the identified clusters align with the non-European migration background data, using the aggregated data provided by  Statistics Netherlands~\citep{CBS} \footnote{The aggregated statistics are rounded to the nearest ten, hence why there are  $n=248,650$ reported students instead of $248,649$.}.
We exclude cases where the distance is unknown, since the risk profiling algorithm did not apply to these students. 

\textbf{Relationship between clusters and non-European migration background}: As individual-level demographic data on students' origin are unavailable, we estimate the percentage of students with a non-European migration background per cluster using aggregate statistics. This percentage is determined using the weighted average of the proportion of students per cluster, based on the percentage of students with a non-European migration background for each combination\footnote{With 4 types of education, 5 age groups and 8 distance categories, there are in total 160 possible  combinations. Each possible combination is assigned to a unique cluster -- e.g. if two students belong to the same combination of education level, age, and distance to parents, they belong to the same cluster.} of the three characteristics available in~\citep{CBS}.

There are stark differences in the  proportion of students with a non-European migration background across clusters. Figure \ref{fig:non_eu_mig_back_cluster} shows that it is estimated that 41\% of students in cluster 3 have a non-European migration background, compared to 17\% in cluster 2 and 10\% in cluster 1. These results indicate that clusters found by the unsupervised bias detection tool are related to the demographic group of interest. 

\textbf{Proxies for non-European migration background}:
In order to shed light on the relationship between clusters and non-European migration background, we investigate how characteristics of the clusters (and the risk profiling algorithm) serve as proxies for non-European migration background. This is illustrated in Figure \ref{fig:univariate_proxy}. Students that are enrolled in vocational education (MBO 1-2, MBO 3-4), in the age groups 23-24 and 25-50, and living less than 10km away from their parents on average are more likely to have a non-European migration background. This corresponds to the characteristics of students in cluster 3. In contrast, university students (WO), young students, and students living far away from their parents are relatively less likely to have a non-European migration background. These characteristics correspond to students in cluster 1, and this also aligns with the logic of the risk profile: younger students and those living closer to their parents are assigned higher risk scores \citep{DutchParliament2024D07565}. 

\begin{figure}[H]
  \centering
\begin{subfigure}{.25\textwidth}
  \centering
  %
\includegraphics[scale=0.1875]{Fig/DUO/2014/education_non_eu_mig_back.png}  
  \caption{}
    \label{fig:cluster_educ}

\end{subfigure}
\begin{subfigure}{.3\textwidth}
  \centering
  %
\includegraphics[scale=0.1875]{Fig/DUO/2014/age_non_eu_mig_back.png} 
  \caption{}
    \label{fig:cluster_age}
\end{subfigure}
\begin{subfigure}{.44\textwidth}
  \centering
  %
\includegraphics[scale=0.1875]{Fig/DUO/2014/distance_non_eu_mig_back.png}  
  \caption{}
    \label{fig:cluster_distance}
\end{subfigure}
\caption{The percentage of students with a non-European migration background for the three used profiling characteristics: (a) type of education, (b) age and (c) distance to parents for students in the CUB 2014 dataset ($n=248,650$),\protect\footnotemark  excluding students where the distance is unknown.}
\label{fig:univariate_proxy}
\end{figure}
\footnotetext{Since the aggregated data of Statistics Netherlands is rounded to the nearest 10, we report here $n=248,650$ instead of the $248,649$ for the CUB data. }


\textbf{Combinations of profiling characteristics as proxies for non-European migration background}:
In addition to examining single characteristic relationships, we also investigate how specific combinations of characteristics act as a proxy for having a non-European migration background. 

Figure \ref{fig:heatmap_education_age} highlights that young students in university education (WO) are less likely to have a non-European migration background. This result aligns with the characteristics of cluster 1, which consists exclusively of university (WO) students and contains relatively more students aged 19-20. Relatively older students (older than 23) who follow vocational education (MBO 1-2, MBO 3-4) are also more likely to have a non-European migration background. 

Figure \ref{fig:heatmap_education_distance} indicates that students who follow vocational education (MBO) and live relatively close to their parents are more likely to have a non-European migration background. This pattern is reflected in cluster 1, which contains MBO 1-2 and MBO 3-4 students, who tend to live closer to their parents. 

Figure \ref{fig:heatmap_age_distance} shows a strong overrepresentation of students with a non-European migration background in the 15-18 age group who live 1m-1km from their parents. This pattern is somewhat reflected in cluster 3, which contains a higher proportion of 15-18 year old students and students living 1m-1km away from their parents. 

\begin{figure}[H]
  \centering
\begin{subfigure}{.25\textwidth}
  \centering
  %
\includegraphics[scale=0.25]{Fig/DUO/2014/heatmap_education_age.png}  
  \caption{}
  \label{fig:heatmap_education_age}
\end{subfigure}
\begin{subfigure}{.35\textwidth}
  \centering
  %
\includegraphics[scale=0.25]{Fig/DUO/2014/heatmap_education_distance.png} 
  \caption{}
  \label{fig:heatmap_education_distance}
\end{subfigure}
\begin{subfigure}{.35\textwidth}
  \centering
  %
\includegraphics[scale=0.25]{Fig/DUO/2014/heatmap_age_distance.png}  
  \caption{}
  \label{fig:heatmap_age_distance}
\end{subfigure}
\caption{The percentage of students with a non-European migration background for three (bivariate) combinations of the used profiling characteristics: (a) type of education and age, (b) type of education and distance to parents, and (c) age and distance to parents for the CUB 2014 dataset ($n=248,650$),\protect\footnotemark excluding students where the distance is unknown.}
\label{fig:bivariate_proxy}
\end{figure}
\footnotetext{Supra  note 10}


\section{Simulation Study to Evaluate HBAC Algorithm}
\label{sec:simulation}
This section provides simulations to support several of the design choices made when applying the HBAC algorithm, and develops empirical guidance for how to best use it. Similarly to \citep{MISZTALRADECKA2021102519}, we study the performance of the HBAC algorithm on simulated data. The data-generation process is as follows. Let $\iota_d \in \mathbb{R}^d$ be a column vector of ones, and $I_d \in \mathbb{R}^{d \times d}$ the identity matrix. We use $\mathcal{N}(\boldmu, \boldSigma)$ to refer to a (multivariate) Gaussian distribution with mean $\boldmu \in \mathbb{R}^d$ and covariance matrix $\boldSigma \in  \mathbb{R}^{d\times d}$. The features per user for cluster $k$ are generated as:
\begin{align}
    \boldxsub{i, k} \sim \mathcal{N}(\mu_k\iota_d, I_d), \text{ where } \mu_k \sim U(-1, 1).
\end{align}
This ensures the features differ on average per user. The bias metric per user is generated as:
\begin{align}
    m_i \sim \mathcal{N}(\eta_k, 1). 
\end{align}
The $\eta_k$ is defined in line with one of two scenarios:\\
\textbf{Scenario 1: Constant bias}: the distribution of the bias metric per user is directly generated and remains consistent across clusters, i.e., $ \eta_k = 0 $ for all $ k \in \mathcal{K}$. \\
\textbf{Scenario 2: Linear increase in bias}: The expectation of the bias metric per user is directly generated as increasing from -1 to 1 (i.e., $\eta_k = -1 + 2 \cdot \frac{k-1}{K-1}$, for all $ k \in \mathcal{K}$).\\
We fix the number of clusters, $K$ and datapoints $N$, and keep the number of samples per cluster $N_k$ equal per cluster. 


We fit the HBAC algorithm to this synthetic data, and afterwards test if the difference between $\barM(U\backslash U_k)$ and $\barM(U_k)$ is statistically significant at a pre-specified significance level. We highlight three recommendations for this test to have the desired properties for detecting true bias, illustrated via results on the simulated dataset. We illustrate these recommendations when the simulated data have $K=5$ clusters, $n=1,000$ observations, and $d=2$ features. %

In the simulation study, we identify three separate issues with the original implementation by \citet{MISZTALRADECKA2021102519} that we argued for in Section \ref{sec:tool}. Next, we outline these issues and propose a solution. 

\textbf{The necessity of sample-splitting:} 
Without using sample-splitting, the difference in the bias metric is measured \textit{in-sample}, e.g. for data used to fit the HBAC algorithm. Here, there is a risk of ``overfitting.'' Figure \ref{fig:illustrate_oos} illustrates this. When the bias metric $m$ is constant for each cluster, and measured on data used to fit the HBAC algorithm, there are still substantial differences between the clusters when measuring the difference between $\barM(U\backslash U_k)$ and  $\barM(U_k)$ in-sample. However, these differences disappear when measuring the difference between $\barM(U\backslash U_k)$ and  $\barM(U_k)$ on a sample not used to fit the HBAC algorithm, e.g., \textit{out-of-sample}. This indicates that without sample splitting, we might falsely conclude that a classifier is biased. Figure \ref{fig:illustrate_oos} shows that when there is a difference in bias between the clusters, we are still able to detect this when measuring these differences out-of-sample  (albeit slightly less likely). 

\begin{figure}[H]
    \centering
    \includegraphics[scale=0.25]{Fig/simulations/illustrate_oos_K_5_d_2_N_1000.png}
    \caption{Average of the difference between $\barM(U\backslash U_k)$ and $\barM(U_k)$ for 1,000 simulations for the simulated dataset with $K=5, n=1,000$, and $d=2$. The error bar reflects the 95\% confidence interval. In-sample refers to the data used to fit the HBAC algorithm, where out-of-sample refers to the data reserved for statistical testing via sample splitting. In Scenario 1 ($m_i$ constant), the expectation of $m_i$ is the same per cluster and we expect the HBAC algorithm to not find bias in the identified clusters. This is only the case when measuring the difference out-of-sample. In Scenario 2 ($m_i$ increasing), the expectation of $m_i$ linearly increases per cluster and we expect the HBAC algorithm  to find bias both in-sample and out-of-sample.}
    \label{fig:illustrate_oos}
\end{figure}

\textbf{Testing across multiple clusters requires multiple testing correction:} In Appendix \ref{sec:addendum_sim}, Figure \ref{fig:illustrate_bonf} shows that without Bonferroni correction the false positive rate is much higher than the expected rate from the significance level of the test (i.e., 5\% with $\alpha = 0.05$), when there is no difference in bias between clusters. This shows that without Bonferroni correction, we might conclude that our classifier is biased, when it is not. With Bonferroni correction, the false positive rate does not exceed the expected rate (i.e., 5\%). However, this comes at a cost: Figure \ref{fig:illustrate_bonf} in the Appendix \ref{sec:addendum_sim} shows that when the bias increases per cluster, the Bonferroni correction makes it less likely that we detect differences between clusters (i.e., the true positive rate decreases).

\textbf{Different bias metrics require different null-hypotheses:}
So far, our null hypothesis has been that the difference between $\barM(U\backslash U_k)$ and $M(U)$ is zero. The validity of this null hypothesis depends on the bias metric used. For this part of the simulation study, we do not directly simulate $m_i$, but instead simulate the true attribute $y$ as follows: $ \ysub{i} = \mathrm{Bern}(p_k)$. In the first scenario, we keep  $p_k = 0.5 \ \forall k \in \mathcal{K}$, that is: the probability of $y=1$ is constant for all clusters. In the second scenario, we linearly increase the probability that $y = 1$ from 0.1 to 0.9, that is: $p_k = 0.1 + 0.8 \cdot \frac{k-1}{K-1}$. We then fit a logistic regression to predict $\ysub{i}$. Here, we only consider one bias metric: the predicted value $\hat{y}_i$ (usually used to investigate demographic parity). We consider using \emph{accuracy} as a bias metric for this experiment in Appendix \ref{sec:addendum_sim}. When using $\hat{y}_i$ as the bias metric, we observe that even when $\ysub{i}$ has the same distribution per cluster, there are differences in $\hat{y}_i$ per cluster. This is because even though $\ysub{i}$ is independent from $\boldxsub{i}$, $\hat{y}_i$ is not independent from $\boldxsub{i}$. Thus, if we use the null hypothesis that $\barM(U\backslash U_k)$ and  $\barM(U_k)$ is zero, we are likely to conclude that our classifier is biased. In Figure \ref{fig:illustrate_perm_test}, we illustrate this: if $\ysub{i}$ is constant per cluster, we incorrectly conclude there are statistically significant differences in $\hat{y}_i$. An alternative is to use the permutation test as suggested by \citet{Ojala_2010}. To do so, we permute the order of $\ysub{i}$ in our data to make it independent of $\boldxsub{i}$. By repeating this $\nperm$ times, we obtain samples from our null distribution, the distribution of the bias metric when $\boldxsub{i}$ is independent of $\ysub{i}$. We can then perform a statistical test by examining the probability of observing a value of the bias metric under the null distribution. Figure \ref{fig:illustrate_perm_test} shows the result of using such a permutation test. The $t$-test frequently suggests that there is a statistically significant difference in $\hat{y}_i$ when $\ysub{i}$ is constant. This is not true for the permutation test.

Finally, we emphasize that the appropriate null hypothesis depends on the preference of the user. It could be that a user is not interested in differences in $\hat{y}_i$ per cluster that are irreducible, e.g., that appear even if $\boldxsub{i}$ is independent of $\ysub{i}$. Then, a permutation test is more appropriate than a $t$-test.


\begin{figure}[H]
\centering
\begin{subfigure}{.475\textwidth}
  \centering
  %
     \includegraphics[scale=0.25]{Fig/simulations/illustrate_bonf_K_2_d_2_N_1000.png} 
   \caption{The effect of using a Bonferroni correction}
  \label{fig:illustrate_bonf}
\end{subfigure}
\begin{subfigure}{.475\textwidth}
  \centering
  %
     \includegraphics[scale=0.25]{Fig/simulations/illustrate_perm_test_y_pred_K_5_d_2_N_1000.png} 
  \caption{The effect of using a permutation test}
  \label{fig:illustrate_perm_test}
\end{subfigure}
\caption{The percentage of simulations where for at least one of the clusters $\barM(U_k)$ is statistically significant from $\barM(U\backslash U_k)$ at the 5\% level for 1,000 simulations for the synthetic dataset with $K=5, n=1,000$ and $d=2$. The error bar reflects the 95\% confidence interval. The red dotted line reflects the significance level used when conducting the statistical test for the differences. We measure the difference between $\barM(U\backslash U_k)$ and $M(U)$ out-of-sample. In Scenario 1 the expectation of (a) $m_i$ or (b) $y_i$ is the same per cluster. In Scenario 2 the expectation of (a) $m_i$ or (b) $y_i$ linearly increases per cluster. In subfigure (b), we use $\hat{y}_i$ rather than directly simulating $m_i$, and we use $\nperm = 1,000$ permutations.}
\label{fig:multiple_perm}
\end{figure}

\section{Open Source Tool}
The unsupervised bias detection tool is available on PyPI \footnote{\url{https://pypi.org/project/unsupervised-bias-detection/}}. It implements the HBAC algorithm to detect potentially discriminated groups of users.
Currently, the library supports \textit{k}-means and \textit{k}-modes algorithms for splitting the data into clusters. The modular design of the tool ensures that additional splitting methods can be integrated with minimal effort. To enhance accessibility, the tool is embedded within a web application, enabling people with no programming background to utilize its capabilities via an intuitive graphical interface\footnote{\url{https://algorithmaudit.eu/technical-tools/bdt/}}.


\section{Discussion}\label{sec:disc}
Here, we discuss the limitations of the unsupervised bias detection tool, concerning its practical application and the theoretical limitations of the simulations studies conducted.

\textbf{Limitations of clustering to audit algorithmic systems:}
The unsupervised bias detection tool alone cannot establish prohibited discrimination, as human expertise is essential to evaluate bias within the context of relevant legislation. However, the tool can serve as a starting point for such deliberative assessments. For instance, the clusters it identifies could prompt further investigation into how the characteristics of individuals relate to demographic groups.

\textbf{Limitations of applying the unsupervised bias detection tool to the DUO use case:} 
There are several challenges associated with using aggregated data to evaluate the proportion of students with a non-European migration background in the clusters identified by the tool. For instance, this approach requires estimating the percentage of students with a non-European migration background and prevents precise analysis of the joint relationship between the bias metric, clusters and demographic groups. However, this is an unavoidable limitation due to the sensitive nature of the data\footnote{Supra note 7}. 

\textbf{Limitations relating to the simulation study conducted:}
Although the simulation study is designed to be general, it relies on a relatively simple data-generation process. Future work could explore more complex scenarios, such as non-linear relationships between clusters and the bias metric, or situations where the number of data points is relatively small compared to the number of features.  

\section{Conclusion}\label{sec:conc}
As demonstrated through the DUO case study encompassing over 250,000 students, the unsupervised bias detection tool identifies clusters that correspond to groups of students found to be unequally treated by a risk profiling algorithm. The clustering results highlight proxy characteristics associated with a non-European migration background: vocational education (MBO) and living close to parents, as examined through  aggregate statistics provided by Statistics Netherlands.

By advancing unsupervised learning methods, we enhance public knowledge on auditing   algorithmic-supported decision-making processes on bias, in the absence of demographic data. The tool provides a scalable method that helps private entities, as well as internal and external auditors, to obtain inferences about seemingly neutral rules or practices that might embed unwanted biases. The tool could support further investigation into whether demographic groups face disadvantages from algorithmic-supported decision-making systems.

Through a simulation study, we highlight important considerations when applying  unsupervised bias detection methods to avoid drawing incorrect conclusions. The simulations illustrate the necessity of using sample-splitting and Bonferroni corrections for valid inferences.
The unsupervised bias detection method has been implemented as an open-source tool in the form of a Python package, accompanied by a graphical interface in a web application to encourage adoption by a non-technical audience.
We hope that the contributions from this paper will serve the community by assisting in the detection of biases in algorithmic-supported decision-making systems.

\section{Ethical Considerations Statement}
The researchers have a contractual agreement with DUO to process CUB data. Data have been processed according to the EU's GDPR~\cite{gdpr}. We acknowledge that if the unsupervised bias detection tool is used purely on its own without any human oversight then false conclusions could be drawn.


\bibliographystyle{ACM-Reference-Format}
\bibliography{refs}


\newpage
\appendix

\section{Risk profile}\label{sec:Appendix_Risk_profile}
The risk score for student $u_{1 \leq i \leq N}$ is determined as follows:
\begin{align*}
    \text{risk score}_{u_i} = \mathrm{R}_1 \times (\mathrm{R}_2 + \mathrm{R}_3).
\end{align*}
$\mathrm{R}_1$ is the risk factor assigned to different types of education (see Table \ref{tab:riskfactor_1}). $\mathrm{R}_2$ is a risk factor determined by the combination of the distance category and the age category of the student (see Figure \ref{fig:R2}). $\mathrm{R}_3$ is an adjustment based on whether there were particular deviations between the age of the student known to DUO, the age known to the Municipal Basic Administration (GBA), combined with information about how long the student has been living away from home (see Table \ref{tab:riskfactor_1}). Based on these risk factors a risk score is computed which are binned in 6 risk categories (see Figure \ref{fig:risk_cat}). 


\begin{figure}[H]
    \centering
    \begin{minipage}{0.45\textwidth} %
        \centering
       \includegraphics[scale=0.4]{Fig/DUO/risk_categories.png}
        \caption{A tabular representation of the risk categories and the corresponding lower and upper limit of the risk score. These limits indicate when a student is deemed to be ``high risk'' for unduly being allocated a college grant, i.e., if the risk score for a student is determined between $80$ and $180$, students were allocated to risk category 1.}
    \label{fig:risk_cat}
    \end{minipage}\hfill
    \begin{minipage}{0.45\textwidth} %
        \small
    \begin{tabular}{l c}
    \toprule
    Type of education & Factor \\
    \midrule
    MBO 1-2        & 1.2                                                         \\
    MBO 3-4        & 1.1                                                         \\
    HBO            & 1.0                                                         \\
    WO             & 0.8   \\                   \bottomrule
    \end{tabular}
    \caption{The risk factor $\mathrm{R}_1$  for the different types of education.}
\label{tab:riskfactor_1}
    \end{minipage}
\end{figure}





\begin{table}[H]
\centering
\footnotesize
\begin{tabular}{llllllc}
\toprule
\textbf{Current Age}  & \textbf{Current Age} & \textbf{Age when registered} & \textbf{Age when registered}  & \textbf{Current Age (GBA)}  & \textbf{Current Age (GBA)} & \textbf{Risk Factor} \\
From & Up Until & From & Up Until & From  &  Up Until &  $\mathrm{R}_3$ \\
\midrule
21                 & 22               & 17                      & 18                    & 17                      & 18                    & 5              \\
21                 & 22               & 17                      & 18                    & 19                      & 20                    & 0              \\
21                 & 22               & 19                      & 20                    & 19                      & 20                    & 0              \\
23                 & 24               & 17                      & 18                    & 17                      & 18                    & 15             \\
23                 & 24               & 17                      & 18                    & 19                      & 20                    & 10             \\
23                 & 24               & 17                      & 18                    & 21                      & 22                    & 0              \\
25                 & 65               & 17                      & 18                    & 17                      & 18                    & 30             \\
25                 & 65               & 17                      & 18                    & 19                      & 20                    & 25             \\
25                 & 65               & 17                      & 18                    & 21                      & 22                    & 15             \\
25                 & 65               & 17                      & 18                    & 23                      & 24                    & 0              \\
25                 & 65               & 17                      & 18                    & 25                      & 65                    & 0              \\
25                 & 65               & 19                      & 20                    & 19                      & 20                    & 25             \\
25                 & 65               & 19                      & 20                    & 21                      & 22                    & 0              \\
25                 & 65               & 19                      & 20                    & 23                      & 24                    & 0              \\
25                 & 65               & 19                      & 20                    & 25                      & 65                    & 0              \\
25                 & 65               & 21                      & 22                    & 21                      & 22                    & 15             \\
25                 & 65               & 21                      & 22                    & 23                      & 24                    & 0              \\
25                 & 65               & 23                      & 24                    & 23                      & 24                    & 0 \\
\bottomrule
\end{tabular}
\caption{Values of $\mathrm{R}_3$ depend on the `Current age' of students (referring to the current age of a student as officially registered by DUO), the `Age when registered' (referring to the age known by DUO when the student was registered at an address different from their parental address) and the `Current age (GBA)' (referring to the age known by the  Municipal Basic Administration (GBA)).}
\label{tab:r3}
\end{table}

\begin{figure}[H]
    \centering
\includegraphics[scale=0.7]{Fig/DUO/r2_table.png}
    \caption{The table of $\mathrm{R}_2$ risk factor determined by the combination of the distance category and the age category of the student.}
    \label{fig:R2}
\end{figure}
%




%




\section{Clustering Results for CUB 2019 data}\label{sec:CUB-2019}

We replicate all the figures presented in Section \ref{sec:DUO} for the student population of 2019, i.e., the CUB 2019 dataset ($n=50,233$). This student population differs from the student population in 2014 in a key aspect. After 2015, students no longer received the grant for living outside of their parents home if they were following higher professional (HBO) or university (WO) education. The HBO and WO students included in the CUB 2019 dataset started receiving the college grant before 2015. Because of this difference, we separately analyze the CUB 2019 and 2014 dataset. 

We follow the exact same procedure as outlined in Section \ref{sec:DUO}, only changing the options for the minimum number of samples to $1,000, 2,000, 3,500, 5,000$. This is because the student population of 2019 is smaller than the one in 2014. 

The results are broadly in line with the results from 2014. One difference is that we find 2 instead of 3 clusters. This can be explained by the fact that there are much less HBO and WO students present in this student population. 

Again, there are stark differences in the bias metric between clusters. For instance, Figure \ref{fig:predicted_class_cluster_19} shows that in cluster 2, 46\% of the students are deemed "high risk", whereas only 17\% are deemed "high risk" in cluster 1. This difference is statistically significant at the 0.1\% level, as indicated in Table \ref{tab:test_2019} in Appendix \ref{sec:tests}.

We also (again) observe stark differences per cluster in the estimated percentage of students with a non-European migration background. In Figure \ref{fig:non_eu_mig_back_cluster_19} it is shown that the estimated percentage of students with a non-European migration background in cluster 2 is 36\%, where it is only 14\% in cluster 1. 

Cluster 1 (the cluster with a relatively lower bias) consists exclusively of students following higher professional education (HBO), students that live relatively further away from their parents, and which are relatively older, i.e., exclusively consisting of students older than 21. Cluster 2 (the cluster with a relatively higher bias) consists of 96\% of students following vocational education (MBO 1-2, MBO 3-4), students that live relatively closer to their parents, and are relatively younger.  

Based on the aggregated data available on the origin of students, as provided by Statistict Netherlands ($n=50,230$)\footnote{The aggregated statistics are rounded to the nearest ten; hence, why there are $n=50,230$ reported students instead of ($n=50,233$).}. We again observe how particular characteristics used in the risk profile serve as proxies for non-European migration background. Similar to 2014, students following vocational education (MBO 1-2, MBO 3-4) are relatively more likely to have a non-European migration background, and students living relatively closer to their parents, specifically less than 5km, are also relatively more likely to have a non-European migration background. This corresponds to the characteristics of students in cluster 2. Particular combinations of characteristics also observe as proxies for non-European migration background. For instance, similar to 2014, students older then 23 years who are enrolled in vocational education (MBO 1-2, MBO 3-4) are more likely to have a non-European migration background. 

\begin{figure}[H]
 \centering
\begin{subfigure}{.25\textwidth}
  \centering
  %
\includegraphics[scale=0.195]{Fig/DUO/2019/predicted_class_per_cluster.png}  
\caption{}
\label{fig:predicted_class_cluster_19}

\end{subfigure}
\begin{subfigure}{.5\textwidth}
  \centering
  %
\includegraphics[scale=0.195]{Fig/DUO/2019/non_eu_mig_back_per_cluster.png}  
  \caption{}
    \label{fig:non_eu_mig_back_cluster_19}

\end{subfigure}
\caption{The percentage of students (a) deemed ``high risk'' and (b) non-European migration background for identified clusters based on the CUB 2019 dataset, excluding students for which the risk profile was unknown  ($n=38,976$).}
\label{fig:bivariate_bias_background_19}
\end{figure}

\begin{figure}[H]
  \centering
\begin{subfigure}{.25\textwidth}
  \centering
  %
\includegraphics[scale=0.1875]{Fig/DUO/2019/education_cluster_plot.png}  
  \label{fig:cluster_educ_19}
  \caption{}
\end{subfigure}
\begin{subfigure}{.3\textwidth}
  \centering
  %
\includegraphics[scale=0.1875]{Fig/DUO/2019/age_cluster_plot.png}  
  \label{fig:cluster_age_19}
  \caption{}
\end{subfigure}
\begin{subfigure}{.44\textwidth}
  \centering
  %
\includegraphics[scale=0.1875]{Fig/DUO/2019/distance_cluster_plot.png}  
  \label{fig:cluster_distance_19}
  \caption{}
\end{subfigure}
\caption{The percentage of students in each cluster for the three characteristics used in the risk profiling algorithm: (a) type of education, (b) age and (c) distance to parents within the CUB 2019 dataset, excluding students for which the risk profile was unknown  ($n=38,976$). For each subgroup, the average is indicated by the dashed line.}
\label{fig:cluster_characteristics_2019}
\end{figure}

\begin{figure}[H]
  \centering
\begin{subfigure}{.25\textwidth}
  \centering
  %
\includegraphics[scale=0.1875]{Fig/DUO/2019/education_non_eu_mig_back.png}  
  \label{fig:cluster_educ_19}
  \caption{}
\end{subfigure}
\begin{subfigure}{.3\textwidth}
  \centering
  %
\includegraphics[scale=0.1875]{Fig/DUO/2019/age_non_eu_mig_back.png} 
  \label{fig:cluster_age_19}
  \caption{}
\end{subfigure}
\begin{subfigure}{.44\textwidth}
  \centering
  %
\includegraphics[scale=0.1875]{Fig/DUO/2019/distance_non_eu_mig_back.png}  
  \label{fig:cluster_distance_19}
  \caption{}
\end{subfigure}
\caption{The percentage of students with a non-European migration background for the three used profilgin characteristics: (a) type of education, (b) age and (c) distance to parents for students in the CUB 2019 dataset ($n=50, 230$), excluding students where the distance is unknown.}
\label{fig:univariate_proxy_2019_19}
\end{figure}

\begin{figure}[H]
  \centering
\begin{subfigure}{.25\textwidth}
  \centering
  %
\includegraphics[scale=0.25]{Fig/DUO/2019/heatmap_education_age.png}  
  \label{fig:cluster_educ_19}
  \caption{}
\end{subfigure}
\begin{subfigure}{.35\textwidth}
  \centering
  %
\includegraphics[scale=0.25]{Fig/DUO/2019/heatmap_education_distance.png} 
  \label{fig:cluster_age_19}
  \caption{}
\end{subfigure}
\begin{subfigure}{.35\textwidth}
  \centering
  %
\includegraphics[scale=0.25]{Fig/DUO/2019/heatmap_age_distance.png}  
  \label{fig:cluster_distance_19}
  \caption{}
\end{subfigure}
\caption{The percentage of students with a non-European migration background for three (bivariate) combinations of the used profiling characteristics: (a) type of education and age, (b) type of education and distance to parents, and (c) age and distance to parents for the CUB 2019 dataset ($n=50, 230$), excluding students where the distance is unknown. The `None' indicates there were no observations available for these categories.}
\label{fig:bivariate_proxy_2019}
\end{figure}



\section{Details on Statistical Tests}
\label{sec:tests}

Here, we provide details for the statistical tests  used to investigate whether or not differences between the bias metric per cluster are statistically significant. In Table \ref{tab:test_2014} we report results for the student population in 2014, and in Table \ref{tab:test_2019} for the student population in 2019.

\begin{table}[H]
\centering
\begin{tabular}{l c c c c c}
\toprule
\textbf{Cluster} & 
\textbf{\(n\) in cluster} & 
\textbf{\begin{tabular}{c}High risk (\%)\\ in cluster\end{tabular}} & 
\textbf{\begin{tabular}{c}High risk (\%)\\ outside cluster\end{tabular}} & 
\textbf{Difference (\%)} & 
\textbf{P-value} \\
\midrule
1 & 11,356 & 3.39 & 25.58 & -22.19 & $<10^{-16}$ \\
2 & 24,650 & 18.88 & 20.83 & -1.95  & $1.73\times 10^{-6}$ \\
3 & 6,914  & 49.46 & 13.99 & 35.47  &  $<10^{-16}$ \\
\bottomrule
\end{tabular}
\caption{Results of testing difference in bias metrics across clusters returned by the unsupervised bias detection tool on the validation set for the year 2014 ($n=42,920$), using a Pearson's $\chi^2$-test.  The `$<10^{-16}$' is used to indicate that the $p-$value was smaller than the smallest decimal that can be represented due to integer underflow.}
\label{tab:test_2014}
\end{table}

\begin{table}[H]
\centering
\begin{tabular}{l c c c c c}
\toprule
\textbf{Cluster} & 
\textbf{\(n\) in cluster} & 
\textbf{\begin{tabular}{c}High risk (\%)\\ in cluster\end{tabular}} & 
\textbf{\begin{tabular}{c}High risk (\%)\\ outside cluster\end{tabular}} & 
\textbf{Difference (\%)} & 
\textbf{P-value} \\
\midrule
1 & 2,065 & 17,29  & 46,34 & -22.19 & $<10^{-16}$\\ 
\bottomrule
\end{tabular}
\caption{Results of testing difference in bias metrics across clusters returned by the unsupervised bias detection tool on the validation set for the year 2019 ($n=7,796$), using a Pearson's $\chi^2$-test.  The `$<10^{-16}$' is used to indicate that the $p-$value was smaller than the smallest decimal that can be represented due to integer underflow. In this case, we only conduct a single statistical test, since there are only two clusters returned.}
\label{tab:test_2019}
\end{table}


\section{Additional results for Simulation Study}
\label{sec:addendum_sim}

In this section, we report an additional result of the simulation study, complementing the results in Section \ref{sec:simulation}.

In Figure \ref{fig:illustrate_perm_test_err} we report the results of using the permutation test when accuracy is used as a bias metric. Unlike the results for using predicted value $\hat{y}_i$ as a bias metric in (see Figure \ref{fig:illustrate_perm_test}), we are not very likely to conclude there is a statistically significant difference in the bias metric. Figure \ref{fig:illustrate_perm_test_err} also suggests that using the permutation test in this scenario results in the correct probability of falsely rejecting the null hypothesis, given the significance level (5\%).  


\begin{figure}[H]
  \centering
  %
     \includegraphics[scale=0.25]{Fig/simulations/illustrate_perm_test_err_K_5_d_2_N_1000.png} 
\caption{The percentage of simulations where for at least one of the clusters $\barM(U_k)$ is statistically significant from $\barM(U\backslash U_k)$ at the 5\% level for 1,000 simulations for the synthetic dataset with $K=5, n=1,000$ and $d=2$. The error bar reflects the 95\% confidence interval. The red dotted line reflects the significance level used when conducting the statistical test for the differences. We measure the difference between $\barM(U\backslash U_k)$ and $M(U)$ out-of-sample. In Scenario 1 the expectation of $y_i$ is the same per cluster. In Scenario 2  the expectation of  $y_i$ linearly increases per cluster. As a bias metric, we use the accuracy $p(\ysub{i} \neq \hat{y}_i)$ rather than directly simulating $m_i$, and we use $\nperm = 1,000$ permutations. }
  \label{fig:illustrate_perm_test_err}
\end{figure}



\end{document}
\endinput
%
%
