\section{{\methodname} Overview}

To assess the cognitive capabilities of MLLMs, we select a subset from the FRCT battery.
These tests evaluate fundamental human cognitive abilities, known as \textit{Factors}, which are crucial for visual perception, reasoning, spatial understanding, and rapid information processing.
An overview of the \textbf{15} tests are shown in Fig.~\ref{fig:framework}.

\subsection{Closure Flexibility (CF)}

Flexibility of closure is a cognitive ability that enables individuals to isolate and retain a specific visual configuration despite distractions.
It is associated with field independence and relies on short-term memory processes~\cite{wardell1973possible}.

\paragraph{CF1: Hidden Figures Test}

Participants are required to identify which of five given figures is embedded within a more complex figure~\cite{crutchfield1952gottschaldt}.
This test assesses \textit{Perceptual Flexibility}, the ability to shift between visual representations and discern relevant patterns from background noise.

\paragraph{CF2: Hidden Patterns Test}

Participants must locate a specific pattern within each test figure~\cite{thurstone1938perceptual}
This test evaluates \textit{Visual Search Efficiency}, which is critical for identifying structured elements within noisy environments.

\paragraph{CF3: Copying Test}

Participants replicate a given pattern on a $5\times5$ grid, starting from a designated point and determining its endpoint~\cite[41]{thurstone1938primary}.
This test measures \textit{Visual-Motor Integration}, the ability to coordinate visual perception with fine motor execution.

\subsection{Closure Speed (CS)}

Speed of closure is a cognitive ability that enables the rapid integration of disparate visual elements into a coherent concept without prior knowledge.
It is associated with recognizing ambiguous stimuli and relies on long-term memory processes~\cite{frederiksen1967cognitive, hoffman1968factor}.

\paragraph{CS1: Gestalt Completion Test}

Participants are required to reconstruct a complete image from fragmented parts, inferring the whole from partial information~\cite{street1968street}.
This test assesses \textit{Perceptual Closure}, the ability to recognize objects despite missing or occluded elements.

\paragraph{CS2: Concealed Words Test}

Participants must identify English words in noise or visual obstructions~\cite{thurstone1944factorial}.
This test evaluates \textit{Visual-Linguistic Integration}, the capacity to integrate perceptual and linguistic cues for pattern recognition.

\paragraph{CS3: Snowy Pictures Test}

Participants must recognize objects obscured by random visual noise~\cite{ekstrom1976manual}.
This test measures \textit{Robust Object Recognition}, the ability to infer meaning from heavily degraded stimuli, which is critical for perceptual resilience.

\subsection{Visualization (VZ)}

The visualization factor refers to the capacity to mentally manipulate spatial patterns into novel configurations, demanding more complex restructuring than spatial orientation.
It is often considered a secondary factor that integrates various spatial abilities, particularly emphasizing mental rotation and sequential operations~\cite{carroll1976psychometric}.

\paragraph{VZ1: Form-Board Test}

Participants select a combination of shaded figures that, when assembled, form a complete shape~\cite[34]{thurstone1938primary}.
This test measures \textit{Spatial Integration}, the ability to mentally combine individual components into a unified whole.

\paragraph{VZ2: Paper-Folding Test}

Participants predict the appearance of a folded and cut piece of paper~\cite[37-38]{thurstone1938primary}.
This test assesses \textit{Mental Manipulation of Objects}, a key skill for understanding transformations in shape and structure.

\paragraph{VZ3: Surface Development Test}

Participants visualize the unfolding of a three-dimensional object into a two-dimensional representation~\cite[36]{thurstone1938primary}.
This test evaluates \textit{Mental Unfolding and Reconstruction}, which is crucial for spatial reasoning in problem-solving.

\subsection{Spatial Relations (S)}

Spatial orientation is the ability to perceive and maintain one's position relative to objects in space.
Unlike visualization, which emphasizes component manipulation, spatial orientation relies on holistic perception and engages short-term visual memory and mental rotation process~\cite{zimmerman1954hypotheses, werdelin1969nature, werdelin1971relationship}.

\paragraph{S1: Card Rotations Test}

Participants determine whether two cards are identical under rotation, with flipping prohibited~\cite[50]{thurstone1941factorial}.
This test assesses \textit{Mental Rotation}, the ability to visualize and manipulate objects in the mind's eye.

\paragraph{S2: Cube Comparisons Test}

Participants compare different orientations of 3D cubes to determine whether they represent the same structure~\cite[31-32]{thurstone1938primary}.
This test measures \textit{3D Spatial Reasoning}, which is crucial for interpreting depth, perspective, and object transformations.

\subsection{Spatial Scanning (SS)}

The spatial scanning factor denotes the efficiency of visually exploring complex spatial fields, including rapid identification of openings, path-following, and rejection of false leads.
Some studies suggest it involves a basic planning function in navigation tasks such as maze-solving~\cite{carroll1976psychometric}.

\paragraph{SS2: Choosing a Path Test}

Participants must identify the only viable route among five options that connects point ``S'' to point ``F'', with the constraint of passing through the circle at the top~\cite{harrell1949printed}.
This test assesses \textit{Navigation and Route Optimization}, which are essential for spatial planning and movement efficiency.

\paragraph{SS3: Map Planning Test}

Participants first determine the shortest path between two designated points (\eg, ``X'' to ``D''), while adhering to the restriction that paths containing small white circles are blocked and cannot be traversed~\cite{harrell1949printed}.
Subsequently, they count the number of buildings (represented as numbered squares) that the chosen path intersects; an intersection is defined as crossing an edge of the square, while merely touching a vertex does not count.
This test evaluates \textit{Geospatial Reasoning}, a critical skill for way-finding and real-world navigation tasks.

\subsection{Perceptual Speed (P)}

Perceptual speed denotes the efficiency of visual processing tasks, involving the rapid comparison and identification of figures or symbols, and is shaped by factors such as perceptual fluency, decision speed, and immediate perceptual memory~\cite{kunnapas1969figural}.

\paragraph{P3: Identical Pictures Test}

Participants identify identical images from a set of five, which includes four distractors with similar features~\cite[63-64]{thurstone1941factorial}.
This test measures \textit{Visual Discrimination}, a crucial ability for rapid object recognition.

\subsection{Induction (I)}

Induction is a complex cognitive process including concept formation and hypothesis testing, with research identifying multiple subfactors and substantial overlap with general reasoning abilities~\cite{wardell1973possible}.

\paragraph{I3: Figure Classification}

Participants classify figures into two or three groups based on shared properties~\cite[56]{thurstone1941factorial}.
This test assesses \textit{Conceptual Abstraction}, the ability to derive general principles from specific instances.

\section{{\methodname} Implementation}

\paragraph{Digitization of FRCT}

The official manual~\cite{ekstrom1976manual} provides (1) instructions for participants to complete the task, (2) example questions for comprehension, (3) test images, (4) correct answers for each question, and (5) average human performance data.
We extract the instructions and examples to construct input prompts for each test and use the provided answers to automatically assess whether MLLMs correctly respond to each question.
{\methodname} can autonomously generate inputs, query MLLMs, and process the results to determine their final accuracy.

\paragraph{Simplification for MLLMs}

{\methodname} offers two testing modes: default and split.
The default mode follows the original test design, whereas the split mode, introduced in this paper, aims to simplify problems for MLLMs.
Specifically, the split mode transforms multiple-choice questions into binary (yes-no) questions.
For instance, in the ``VZ2 Paper Folding Test,'' participants originally choose the correct unfolded paper figure from five options, resulting in a random guessing accuracy of $20\%$.
In contrast, the split mode reframes the question to ask whether a given option correctly represents the unfolded figure, increasing the random guessing accuracy to $50\%$.

\begin{table*}[t]
    \centering
    \resizebox{1.0\linewidth}{!}{
    \begin{tabular}{lcccccccccccccc}
    \toprule
    \multirow{2}{*}{\bf Tests} & \multirow{2}{*}{\bf Random} & \multicolumn{5}{c}{\bf GPT-4o} & \multicolumn{2}{c}{\bf Gemini-1.5-Pro} & \multicolumn{2}{c}{\bf Gemini-2.0-Pro} & \multicolumn{2}{c}{\bf Qwen-VL-Max} \\
    \cmidrule(lr){3-7} \cmidrule(lr){8-9} \cmidrule(lr){10-11} \cmidrule(lr){12-13}
    & & \bf Vanilla & \bf Pert. & \bf CoT & \bf SCoT & \bf MAD & \bf Vanilla & \bf CoT & \bf Vanilla & \bf CoT & \bf Vanilla & \bf CoT \\
    \midrule
    \cellcolor{myblue!34.0} CF1 & 20.0 & 21.9 & 18.8 & 15.6 & \underline{28.1} & 18.8 & 6.3 & 21.8 & 18.8 & 25.0 & 9.4 & \textbf{31.3} \\
    \cellcolor{myred!34.0} CF1-S & 50.0 & 56.3 & 35.6 & 40.6 & 55.6 & \underline{65.0} & 33.8 & 35.6 & 51.2 & 15.6 & 62.5 & \textbf{71.3} \\
    \cellcolor{myblue!34.0} CF2 & 50.0 & 58.0 & 53.8 & 55.8 & \textbf{65.5} & 26.4 & 56.8 & \underline{58.3} & 53.5 & 49.5 & 51.0 & 53.3 \\
    \cellcolor{myred!34.0} CF2-S & 50.0 & \underline{66.0} & \underline{66.0} & \textbf{66.8} & 65.5 & 49.8 & 51.0 & 48.3 & 57.0 & 8.0 & 58.0 & 53.3 \\
    \cellcolor{myblue!34.0} CF3 & 4.0 & 1.6 & 1.6 & \underline{10.9} & 7.8 & 3.1 & 1.6 & \textbf{14.1} & 6.3 & 1.6 & 6.3 & 0.0 \\
    \midrule
    \cellcolor{myblue!34.0} CS1 & - & \textbf{35.0} & 20.0 & \underline{30.0} & \underline{30.0} & 5.0 & 20.0 & 25.0 & 10.0 & 20.0 & 10.0 & 10.0 \\
    \cellcolor{myblue!34.0} CS2 & - & \textbf{26.0} & 14.0 & \underline{22.0} & \underline{20.0} & 16.0 & 0.0 & 18.0 & 10.0 & 2.0 & 8.0 & 6.0 \\
    \cellcolor{myblue!34.0} CS3 & - & 20.8 & 8.3 & \textbf{33.3} & \underline{25.0} & 12.5 & 12.5 & 20.8 & 8.3 & 4.2 & 8.3 & 4.2 \\
    \midrule
    \cellcolor{myblue!34.0} VZ1 & 50.0 & 63.3 & \underline{67.5} & 62.9 & 65.8 & 41.3 & \textbf{70.4} & 66.3 & 62.1 & 61.3 & 60.4 & 62.1 \\
    \cellcolor{myblue!34.0} VZ2 & 20.0 & \underline{35.0} & 22.5 & 15.0 & 30.0 & \textbf{40.0} & 20.0 & 15.0 & 10.0 & 10.0 & 15.0 & 15.0 \\
    \cellcolor{myred!34.0} VZ2-S & 50.0 & \textbf{80.0} & 15.0 & \underline{71.0} & 67.0 & 55.0 & 18.0 & 20.0 & 56.0 & 28.0 & 26.0 & 22.0 \\
    \cellcolor{myblue!34.0} VZ3 & 14.6 & \textbf{36.7} & 61.0 & 31.7 & \textbf{36.7} & 23.3 & 21.7 & 31.7 & 26.7 & 30.0 & 30.0 & \underline{33.3} \\
    \cellcolor{myred!34.0} VZ3-S & 14.6 & 30.0 & 30.0 & \underline{33.3} & \textbf{38.3} & 15.0 & 16.7 & 11.7 & 25.0 & 15.0 & 26.7 & 25.0 \\
    \midrule
    \cellcolor{myblue!34.0} S1 & 50.0 & \underline{50.0} & 28.3 & \underline{50.0} & \underline{50.0} & 43.8 & \underline{50.0} & 49.4 & 47.5 & \textbf{56.3} & 46.9 & 47.5 \\
    \cellcolor{myred!34.0} S1-S & 50.0 & \underline{56.3} & 46.9 & 54.4 & 53.8 & 43.1 & 50.0 & 49.4 & 47.5 & \underline{56.3} & 46.9 & \textbf{58.1} \\
    \cellcolor{myblue!34.0} S2 & 50.0 & 52.4 & \textbf{58.1} & 52.4 & \underline{57.1} & 47.6 & 40.5 & \underline{52.4} & 50.0 & 38.1 & 42.9 & 38.1 \\
    \midrule
    \cellcolor{myblue!34.0} SS2 & 20.0 & 25.0 & \underline{28.1} & 18.8 & 21.9 & \underline{28.1} & \textbf{34.4} & 15.6 & \underline{28.1} & \textbf{34.4} & 21.9 & 18.8 \\
    \cellcolor{myblue!34.0} SS3 & 9.1 & \textbf{30.0} & 28.1 & \underline{27.5} & \textbf{30.0} & 25.0 & 12.5 & \textbf{30.0} & 15.0 & 15.0 & 5.0 & 20.0 \\
    \cellcolor{myred!34.0} SS3-S & 9.1 & \underline{25.0} & 20.0 & \textbf{30.0} & \textbf{30.0} & 10.0 & 15.0 & 10.0 & 12.5 & 22.5 & 22.5 & 15.0 \\
    \midrule
    \cellcolor{myblue!34.0} P3 & 20.0 & \textbf{44.8} & 37.5 & 20.8 & 40.6 & \underline{41.7} & 26.0 & 20.8 & 25.0 & 25.0 & 22.9 & 24.0 \\
    \cellcolor{myred!34.0} P3-S & 50.0 & \underline{91.9} & 79.8 & 90.4 & \textbf{93.1} & 74.0 & 76.9 & 64.0 & 77.7 & 60.8 & 68.5 & 70.0 \\
    \midrule
    \cellcolor{myblue!34.0} I3 & 42.9 & \textbf{55.8} & \underline{54.5} & 46.0 & 53.5 & 39.3 & 51.8 & 20.8 & 53.1 & 48.7 & 32.6 & 35.7 \\
    \cellcolor{myred!34.0} I3-S & 42.9 & \underline{43.8} & \textbf{44.6} & 39.3 & 36.6 & \underline{43.8} & 34.8 & 34.4 & 31.3 & 21.4 & 34.4 & 25.9 \\
    \midrule
    \cellcolor{myblue!34.0} \bf Avg. & - & \underline{37.1} & 35.1 & 32.8  & \textbf{37.5}  & 27.5  & 28.3  & 30.7  & 28.3  & 28.1  & 24.7  & 26.6 \\
    \bottomrule
    \end{tabular}
    }
    \caption{{\methodname} test results using GPT-4o, Gemini-Pro (1.5 and 2.0), and Qwen-VL-Max. Split tests (S) are marked in \colorbox{myred!34}{red}, while default tests are marked in \colorbox{myblue!34}{blue}. The highest scores are marked in \textbf{bold} while the second highest scores are marked with \underline{underlines}.}
    \label{tab:main}
\end{table*}

\paragraph{Perturbations on Images}

Since all tests in {\methodname} involve images, we can introduce perturbations to increase the benchmark's difficulty.
The implemented perturbations are categorized into two groups: \textbf{Geometric} transformations and \textbf{Photometric} transformations.
Geometric transformations include translation, rotation, and flipping, while photometric transformations include contrast and brightness adjustments, as well as noise addition.
Photometric transformations, translation, and rotation are applicable to all tests, as they do not alter geometric properties.
However, flipping is unsuitable for certain scenarios due to its impact on ground truth validity and is thus limited to CF1-S, CF2-S, S1-S, P3-S, and I3-S.