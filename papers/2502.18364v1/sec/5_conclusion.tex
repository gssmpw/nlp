\vspace{-5pt}
\section{Conclusion}
\label{sec:conclusion}

In this paper, we introduce the Anonymous Region Transformer, a novel approach for generating multi-layer transparent images from an anonymous region layout. Our results and analysis reveal that our anonymous layout is sufficient for the multi-layer transparent image generation task. Our method offers several key advantages over traditional semantic layout methods, including better coherence across layers and more scalable annotation. Furthermore, our method enables the efficient generation of images with numerous distinct transparent layers, reducing computational costs and generalizing to various distinct anonymous region layouts. However, our approach does have certain limitations, including repeated layer generation and combined layer generation. The generalizability of this capability across all potential layouts requires further exploration. Future work should focus on enhancing the model's ability to autonomously identify semantic labels and improving the quality and flexibility of the generated images. Despite these challenges, our approach shows promising potential for graphic design and digital art.

\vspace{1mm}
\noindent\textbf{Future works}
We believe our work lays a solid foundation for the next generation of generative models that can produce a variable number of coherent transparent layers and support flexible image editing through layer compositing. Looking forward, we identify several promising directions for future research:
(i) \emph{Enhancing Visual Aesthetics}:
A key challenge is to improve the visual appealing of the generated transparent layers and ensure that the composite images achieve parity with those produced by state-of-the-art text-to-image models such as FLUX.
(ii) \emph{Anonymous Region Layouts}:
We anticipate that leveraging anonymous region layouts will transform conventional layout-to-image generation tasks. This approach has the potential to eliminate the need for complex regional prompt annotations and to simplify the modeling process by granting models greater control.
(iii) \emph{Human Interaction with ART}:
We also see great promise in integrating user requirements into the multi-layer image generation system. Future work could explore interactive methods for incorporating real-time user feedback, enabling dynamic refinement of generated layers and more personalized editing workflows.