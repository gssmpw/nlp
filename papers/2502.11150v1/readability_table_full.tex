\begin{table}[ht]
\centering
\large
\begin{adjustbox}{width=0.8\textwidth}
\begin{tabular}{|p{2cm}|p{5.5cm}|p{7cm}|p{7cm}|}
\hline
\textbf{Name} & \textbf{Formula} & \textbf{Meaning} & \textbf{Development} \\
\hline
Flesch Reading Ease & 
$206.836 - 84.6 \times \frac{\text{total syllables}}{\text{total words}} - 1.015 \times \frac{\text{total words}}{\text{total sentences}}$ & 
The score is inversely proportional to grade level in which 50 \% of students achieved 75\% on material from McCall-Crabbs Standard Test Lessons in Reading. & 
In 1943 a statistical formula was published for assessing readability. Flesch Reading Ease is a revision of this formula, presented in 1948, attempting to overcome some of the shortcomings of the earlier formula.\\

\hline
Flesch 

Kincaid Grade Score & 
$0.39 \times \frac{\text{total words}}{\text{total sentences}} + 11.8 \times \frac{\text{total syllables}}{\text{total words}} - 15.59$ & 
Grade level where 50\% of subjects scored at least 35\% on the cloze test. \citet{rankin1969comparable} demonstrated that a cloze test score of 35–40 percent corresponds to a 75 percent score on multiple-choice questions.) & 
In 1975 a new formula of Flesch reading Ease was presented, revised for army enlisted personnel. Flesch Kincaid score is a Conversion of the revised score into grade levels. \\

\hline
Dale Chall Score & 
$0.1579 \times \frac{\text{difficult words}}{\text{words}} \times 100 + 0.0496 \times \frac{\text{words}}{\text{sentences}} + 3.6365$ & 
Grade level where 50\% of students score at least 50\% on McCall-Crabbs Standard Test Lessons in Reading. & 
The measure is largely based on the percentage of difficult words, defined by The new Dale-Chall Word List with approximately 3,000 familiar words. This list was constructed a few years prior, by testing fourth graders on their knowledge of a list of approximately 1000 words\\

\hline
Gunning Fog Index & 
$0.4 \left( \frac{\text{words}}{\text{sentences}} + 100 \times \frac{\text{complex words}}{\text{words}} \right)$ & 
Estimated years of formal education (grade level) required to understand the text on first reading. &
Based on expert assessment from newspaper editors and journalists regarding the connection between word and sentence length to complexity. The formula was later validated through practical use.\\

\hline
Coleman Liau Index & 
$0.0588 \times L - 0.296 \times S - 15.8$ \newline 
where:
\newline 
$\begin{array}{l}
L = \text{average letters} \\  \text{per 100 words} \\
S = \text{average sentences} \\  \text{per 100 words}
\end{array}$ & 
Grade level corresponding to cloze \% = \% of deletions that can be filled by a college undergraduate in a cloze test&
First, a formula for cloze \% prediction was developed using data obtained by Miller and Coleman (1967). Then, cloze scored were converted to grade level scores. The Coleman-Liau formula was achieved by combining the formulas. \\

\hline
SMOG \newline Index & 
$1.043 \times \sqrt{\frac{\text{polysyllables} \times 30}{\text{sentences}}} + 3.1291$ & 
Estimated mean grade score of students who could correctly answer all questions in a Lesson from McCall-Crabbs Standard Test Lessons in Reading. & 
Developed in a procces of trial and error on formula's stracture, with final scores obtained from McCall-Crabbs Standard Test Lessons in Reading. \\

\hline
ARI & 
$4.71 \times \frac{\text{characters}}{\text{words}} + 0.5 \times \frac{\text{words}}{\text{sentences}} $
$- 21.43$ & 
Grade level where 50\% of subjects scored at least 35\% on the cloze test. 
\citet{rankin1969comparable} demonstrated that a cloze test score of 35–40 percent corresponds to a 75 percent score on multiple-choice questions.)
 & 
Developed alongside the Flesch-Kincaid score.\\

\hline
\end{tabular}
\end{adjustbox}
\caption{Comparison of Readability Formulas}
\label{tab:readability_measures}
\end{table}
