\documentclass[sn-mathphys-ay]{sn-jnl}

\usepackage{graphicx}
\usepackage{multirow}
\usepackage{amsmath, amssymb, amsfonts}
\usepackage{amsthm}
\usepackage{mathrsfs}
\usepackage[title]{appendix}
\usepackage{xcolor}%
\usepackage{textcomp}%
\usepackage{manyfoot}%
\usepackage{booktabs}%
\usepackage{algorithm}
\usepackage{algorithmic}
\usepackage{listings}%
\usepackage{array}
\usepackage{bm}
\usepackage{subfigure}
\usepackage{hyperref}


% New command definition
% vectors
\newcommand{\ba}{\mathbf{a}}
\newcommand{\bb}{\mathbf{b}}
\newcommand{\bc}{\mathbf{c}}
\newcommand{\bd}{\mathbf{d}}
\newcommand{\bu}{\mathbf{u}}
\newcommand{\bv}{\mathbf{v}}

% matrices
\newcommand{\cX}{\mathbf{X}}
\newcommand{\cY}{\mathbf{Y}}
\newcommand{\cU}{\mathbf{U}}
\newcommand{\cV}{\mathbf{V}}
\newcommand{\cE}{\mathbf{E}}
\newcommand{\cA}{\mathbf{A}}
\newcommand{\cB}{\mathbf{B}}
\newcommand{\cC}{\mathbf{C}}
\newcommand{\cD}{\mathbf{D}}
\newcommand{\cF}{\mathbf{F}}
\newcommand{\cZ}{\mathbf{Z}}

% tensors
\newcommand{\eX}{\bm{\mathscr{X}}}
\newcommand{\eY}{\bm{\mathscr{Y}}}
\newcommand{\eS}{\bm{\mathscr{S}}}
\newcommand{\eM}{\bm{\mathscr{M}}}

% others
\newtheorem{theorem}{Theorem}
\newtheorem{assumption}{Assumption}
\newtheorem{prop}{Proposition}
\renewcommand\thesubtable{(\alph{subtable})}



\numberwithin{equation}{section}
\title{Multiple Linked Tensor Factorization}
\author[1]{\fnm{Zhiyu} \sur{Kang}}
\author[2]{\fnm{Raghavendra B. } \sur{Rao}}
\author*[1]{\fnm{Eric F.} \sur{Lock}}\email{elock@umn.edu}


\affil[1]{\orgdiv{Division of Biostatistics and Health Data Science}, \orgname{School of Public Health, University of Minnesota}, \orgaddress{\city{Minneapolis}, \postcode{55455}, \state{MN}, \country{USA}}}
\affil[2]{\orgdiv{Division of Neonatology}, \orgname{Department of Pediatrics, University of Minnesota}, \orgaddress{\city{Minneapolis}, \postcode{55455}, \state{MN}, \country{USA}}}

\begin{document}


\abstract{In biomedical research and other fields, it is now common to generate high content data that are both \emph{multi-source} and \emph{multi-way}.  Multi-source data are collected from different high-throughput technologies while multi-way data are collected over multiple dimensions, yielding multiple tensor arrays.  Integrative analysis of these data sets is needed, e.g., to capture and synthesize different facets of complex biological systems. However, despite growing interest in multi-source and multi-way factorization techniques, methods that can handle data that are both multi-source and multi-way are limited.  In this work, we propose a Multiple Linked Tensors Factorization (MULTIFAC) method extending the CANDECOMP/PARAFAC (CP) decomposition to simultaneously reduce the dimension of multiple multi-way arrays and approximate underlying signal. We first introduce a version of the CP factorization with $L_2$ penalties on the latent factors, leading to rank sparsity.  When extended to multiple linked tensors, the method  automatically reveals latent components that are shared across data sources or  individual to each data source. We also extend the decomposition algorithm to its expectation–maximization (EM) version to handle incomplete data with imputation. Extensive simulation studies are conducted to demonstrate MULTIFAC’s ability to (i) approximate underlying signal, (ii) identify shared and unshared structures, and (iii) impute missing data.  The approach yields an interpretable decomposition on multi-way multi-omics data for a study on early-life iron deficiency. }

\keywords{Data integration, multi-way arrays, dimension reduction, tensor decomposition, low-rank factorization, missing data imputation}

\maketitle

\section{Introduction}
In many scientific disciplines, the representation and analysis of high-content data with complex structure has become critical.  Tensors, which extend traditional two-way matrices to multi-dimensional arrays, are often well-suited to capture the rich content inherent in such data. For example, as a motivating application we consider a study on early life iron deficiency in a cohort of infant rhesus monkeys \citep{rao_lock_etal_2023}. Hematology data are collected at multiple developmental time points from a group of monkey, and the resulting data are best represented by a three-way tensor data: $Monkeys \times Age\times Hematology \; Indices$. 

Similar to how a matrix decomposition captures latent structures within two-dimensional data, tensor decomposition techniques can be used to uncover the underlying signals from noise, thereby reducing the dimensionality and enhancing interpretability of multi-way data arrays. One traditional and widely used method is the CANDECOMP/PARAFAC (CP) decomposition \citep{harshman_1970, carroll_chang_1970}. The CP decomposition represents a tensor as a sum of multiple outer products of vectors, each corresponding to a dimension of the tensor. This method can be viewed as an extension of matrix decomposition, such as singular value decomposition (SVD) and principal component analysis (PCA), to a higher-order version that enables analysis of additional ways such as time, tissue types or other dimensions in the dataset. 

Often, multiple high dimensional datasets are collected for a single study that capture different facets of the research subjects. This motivates a large number of integrative data analysis methods that combine multiple datasets linked in some ways. Matrix data can be vertically or horizontally linked, sharing the same features or samples across datasets. A simple example for linked data that share the same samples is multi-omics data,
in which data for multiple molecular platforms (protoeomics, metabolomics, genomics, etc.) are measured for the same sample set.
Under this scenario, there are several methods that decompose each omics dataset into a sum of shared structures and individual structures \citep{lock_jive_2013, gaynanova_li_slide_2019, sjive_2022,argelaguet2018multi}, where the shared structure is represented by a latent factorization (analogous to PCA) with the same score matrix across datasets, and individual structure is defined by each dataset's unique loading and score matrices. Other methods take a similar approach to decompose data with the same features measured across different sample sets \citep{wang_lock_2024,de2021bayesian}.  
Bidimensional Integrative Factorization (BIDIFAC) further extends this framework by simultaneously factorizing bidimensionally linked matrices, enabling the capture of structures shared both vertically and horizontally \citep{park_lock_2019, lock_park_2020}. 

Beyond matrices, there is a small but growing literature on the integrative analysis of multiple data with tensor structure. These scenarios often involve collecting multiple tensor datasets, some of which share one or more tensor dimensions. The coupled matrix and tensor decomposition (CMTF) method is a special case for the integration of a matrix and a tensor, which offers flexibility in handling heterogeneous data but lacks discussion on the choice of tensor ranks \citep{acar_2011, acar_2013}. Further advancements in the field are represented by structured data fusion (SDF) and Bayesian multi-tensor decomposition \citep{sorber_sdf_2015, khan_bayesian_2014}. The SDF approach describes a very general framework to joint fusing factorization of multiple matrices or tensors. The Bayesian multi-tensor factorization method provides a robust framework for decomposing multiple matrices and tensors into shared and individual factors by the spike and slab prior. However, these methods  do not focus on missing data imputation.  Often, tensors are incomplete, and accommodating missing data and imputing missing values are critical for a factorization approach. In the context of multiple linked tensors, a commonly encountered special case is tensor-wise missing, in which data are missing for an tensor for some samples. 

For our motivating application to early life iron deficiency, there is also diffusion tensor imaging (DTI) data available for the monkeys at a single timepoint over multiple brain regions. Thus the data have the form of two tensors, $\textit{Monkeys} \times \textit{Age} \times \times{Hematology Indices}$ and $\textit{Monkeys} \times \textit{Brain Regions}  \times \textit{DTI Parameters}$, that are linked across the same monkeys.  The data have complex missing structure, as the hematology data are incomplete or not collected for all timepoints for some monkeys, and not all monkeys have both tensors (hematology and DTI) observed.   

We propose a novel approach to tensor and multiple tensor factorization, leveraging new results on penalized CP factorization. Specifically, we show that an $L_2$ penalty on the individual factor matrices for a CP factorization is equivalent to an $L_p$ ($p<1$) penalty on the singular values of the tensor approximation, promoting rank sparsity. The $L_2$ penalty is computationally straightforward to optimize, and effectively identifying the tensor ranks and shared and individual signal.   
   
The motivation behind our proposed MULTIFAC method stems from the need to address specific challenges in iron deficiency and other real-world data applications. The purpose is several fold: one goal is dimension reduction, aiming to find an efficient way to represent multiple tensors while uncover the underlying signal. Secondly, compared with applying single factorization to each tensor, our method can reveal shared structure across the data for interpretation. Last but not least, our method can properly address both tensor-wise missing and entry-wise missing problems, providing a flexible and powerful tools for imputation. 


\section{Notation and Preliminaries}
The order, ways or modes of a tensor is the number of dimensions. Throughout this paper, scalars are denoted by lowercase letters ($a$), vectors (one-way tensor) are denoted by boldface lowercase letters (${\mathbf{a}}$), matrices (two-way tensor) are denoted by boldface capital letters ($\mathbf{A}$), and high-order tensors are denoted by Euler script letters ($\bm{\mathscr{X}}$).  For an $N$-way tensor $\eX \in \mathbb{R}^{I_1\times \cdots \times I_N}$, its entry ($i_1, \ldots, i_N$) is denoted by $x_{i_1\ldots i_N}$.

Given two matrices $\cA \in \mathbb{R}^{I_1\times I_3}$ and $\cB \in \mathbb{R}^{I_2\times I_3}$, the Khatri-Rao product of them, denoted as $\cA \odot \cB$, is defined as their “matching columnwise” Kronecker product. The result is a matrix of size $(I_1I_2)\times I_3$ defined by
$$
\mathbf{A} \odot \mathbf{B} = \left[ \mathbf{a}_1 \otimes \mathbf{b}_1 \quad \mathbf{a}_2 \otimes \mathbf{b}_2 \quad \cdots \quad \mathbf{a}_{I_3} \otimes \mathbf{b}_{I_3} \right],
$$
where $\otimes$ denotes the Kronecker product. The Kronecker product of matrices $\cA\in \mathbb{R}^{I_1\times I_2}$ and $\cB\in \mathbb{R}^{I_3\times I_4}$ is a matrix of size $(I_1I_3) \times (I_2I_4)$ defined by

\begin{align*}
    \cA \otimes \cB &=
\begin{bmatrix}
    a_{11} \cB & a_{12} \cB & \cdots & a_{1I_2} \cB \\
    a_{21} \cB & a_{22} \cB & \cdots & a_{2I_2} \cB \\
    \vdots & \vdots & \ddots & \vdots \\
    a_{I_11} \cB & a_{I_12} \cB & \cdots & a_{I_1I_2} \cB
\end{bmatrix}\\
&=
\begin{bmatrix}
    \ba_1 \otimes \bb_1 & \ba_1 \otimes \bb_2 & \ba_1 \otimes \bb_3 & \cdots & \ba_{I_2} \otimes \bb_{I_4-1} & \ba_{I_2} \otimes \bb_{I_4}
\end{bmatrix}.
\end{align*}

The Frobenius norm of a tensor $\eX$ is defined by the square root of the sum of the squares of all the elements
$$
\| \eX \|_F = \sqrt{\sum^{I_1}_{i_1=1} \cdots \sum^{I_N}_{i_N=1} x_{i_1\ldots i_N}^2}.
$$

Given vectors $\ba_1 = [a_{11}, a_{12}, \ldots, a_{1I_1}]^\top, \ba_2 = [a_{21}, a_{22}, \ldots, a_{2I_2}]^\top, \ldots, \mathbf{a}_N = [a_{N1}, a_{N2}, \ldots, a_{NI_N}]^\top$ of length $I_1, \ldots, I_N$ respectively, the outer product
$$
\bm{\mathscr{X}} = \ba_1 \circ \ba_2 \circ \cdots \circ  \ba_N
$$
defines a $N$-way rank-1 tensor of dimensions $I_1\times \cdots \times I_N$ with entries
$$
x_{i_1, \ldots, i_N} = a_{1i_1} a_{2i_2}\cdots a_{Ni_N}, 1\leq i_n \leq I_N.
$$

Given two tensors $\eX$ and $\eY$ of the same dimensions $I_1\times \cdots \times I_N$, the Hadamard product is denoted by $\eX * \eY \in \mathbb{R}^{I_1\times \cdots \times I_N}$ and defined as 
$$
(\eX * \eY)_{i_1i_2\ldots i_N} = x_{i_1i_2\ldots i_N} y_{{i_1i_2\ldots i_N}}.
$$

The $n$-mode matricization or unfolding of tensor $\eX$, denoted by $\cX_{(n)}$, rearranges $\eX$ as a matrix by using the mode-$n$ ``fibers" as the columns of the resulting matrix (see \citet{kolda_tensor_review_2009} for details).


\subsection{Matrix Decomposition and Results}
Here we present some well-established results for a matrix $\cX \in \mathbb{R}^{I_1 \times I_2}$, to lay the groundwork for novel extensions of these results for a tensor. Define the SVD for $\cX$  as:
$$
\cX = \tilde\cA_1 \cD \tilde\cA_2^\top,
$$
where $\tilde \cA_1 \in \mathbb{R}^{I_1 \times R}$ and $\tilde\cA_2 \in \mathbb{R}^{I_2 \times R}$ have orthonormal columns, $\cD \in \mathbb{R}^{R \times R}$ is a diagonal matrix containing the singular values, and $R = \min(I_1, I_2)$ is the rank of matrix. With SVD, one can obtain a low-rank approximation of a matrix $\cX$ by only keeping the largest $k$ singular values and corresponding columns in $\tilde\cA_1$ and $\tilde\cA_2$:
$$
\hat{\cX} = \sum_{r=1}^{k} d_r \tilde\ba_{1r} \tilde\ba_{2r}^\top,
$$
where $d_r$ is the $r$-th largest singular value, and $\tilde\ba_{1r}$ and $\tilde\ba_{2r}$ are the corresponding left and right singular vectors.

The nuclear norm $\|\cX\|_*$ of a matrix $\cX$ is defined as the sum of its singular values:
$$
\|\cX\|_* = \sum_{r=1}^{R} d_r.
$$
Minimizing the nuclear norm can be used to promote low-rank solutions in matrix approximation problems. A popular approach is to penalize squared error loss with the nuclear norm, leading to the following optimization problem:
\begin{equation} \label{svd_nuclear}
    \min_{\hat{\cX}} \frac{1}{2} \|\cX - \hat{\cX}\|_F^2 + \sigma \|\hat{\cX}\|_*.
\end{equation}
The solution to the above optimization problem is given by the soft-thresholding operator on the singular values of $\cX$, as in Proposition \ref{prop:1} \citep{mazumder_2010}. 
\begin{prop} \label{prop:1}
    Let $\cX = \tilde\cA_1 \cD \tilde\cA_2^{\top}$ be the SVD of $\cX$. Then, the optimal solution to problem \eqref{svd_nuclear} $\hat{\cX}$ is given by
\begin{equation}
    \hat\cX = \tilde\cA_1 \hat \cD \tilde\cA_2^{\top},
\end{equation}
where the diagonal elements of $\hat\cD$ are $\hat d_r = \max(d_r - \sigma, 0)$, $r = 1, \ldots, R$.
\end{prop}
By choosing a proper penalty factor $\sigma$, this soft-thresholding property results in a low-rank approximation of $\cX$. We can further rewrite the matrix decomposition as $\cX = \cA_1 \cA_2^{\top}$, where the singular value matrix $\cD$ is absorbed into the factor matrices $\cA_1$ and $\cA_2$. The equivalence of nuclear norm penalization and Frobenius norm ($L_2$) penalization applied on $ \cA_1 $ and $\cA_2$ is established in Proposition \ref{prop:2} \citep{mazumder_2010}.
\begin{prop}\label{prop:2}
    For a matrix $\cX \in \mathbb{R}^{I_1 \times I_2}$, we have
    \begin{equation}\label{nuclear_frob}
    \min_{\hat{\cX}} \|\cX - \hat{\cX}\|_F^2 + 2\sigma \| \hat{\cX}\|_* = \min_{\cA_1:I_1\times r, \cA_1:I_2\times r} \|\cX - \cA_1\cA_2^\top\|_F^2 + \sigma  \left( \|\cA_1\|_F^2 + \|\cA_2\|_F^2 \right),
\end{equation}
where $r = \text{min}(I_1, I_2)$, $ \hat{\cX} = \hat \cA_1\hat \cA_2^\top$ solves the left-hand size of \ref{nuclear_frob} and $\{\hat \cA_1, \hat \cA_2\}$ solves the right-hand size of \ref{nuclear_frob}.
\end{prop}

Proposition \ref{prop:2} shows the ability of a $L_2$ penalty on the factor matrices to find a low-rank representation, which motivates our penalty for tensor decompositions.

\subsection{Single Tensor decomposition}
One of the most commonly used models for tensor decomposition is the CANDECOMP/PARAFAC (CP) decomposition \citep{carroll_chang_1970, harshman_1970}. Analogous to expressing a matrix as a sum of outer products of two vectors, CP decomposition expresses a tensor as a sum of outer product of multiple vectors, i.e., rank-one tensors.  Given a $N$-order tensor $\eX \in \mathbb{R}^{I_1\times \cdots \times I_N}$, $\eX$ may be approximated with $\hat \eX$ that has a rank-$R$ CP decomposition as follows:
\begin{equation}
    \eX \approx \sum_{r=1}^{R} \ba_{1r} \circ \ba_{2r} \circ \cdots \circ  \ba_{Nr} \triangleq \hat \eX,
\end{equation}
where $R$ is a positive integer and $\ba_{nr} \in \mathbb{R}^{I_n}, n= 1,\ldots, N$, are the factor vectors for the $r$-th component. The minimal number of rank-one components required to represent $\eX$ is called the rank of the tensor, denoted by $\text{rank}(\eX)$ \citep{kruskal_1977, hitchcock_1927}. The compact form of the decomposition can be written using factor matrices:
\begin{equation}
    \eX \approx [\![ \cA_1, \cA_2,\ldots, \cA_N]\!],
\end{equation}
where $\cA_n = [\ba_{n1}, \ba_{n2}, \ldots, \ba_{nR}]\in \mathbb{R}^{I_n \times R}, n = 1, 2, \ldots, N$, are the factor matrices containing the corresponding vectors of each rank-one component. Furthermore, letting $\{\tilde \cA_n| n = 1, 2, \ldots, N\}$ represent the factor matrices with columns normalized to norm one and the weights of rank-1 components absorbed into a vector $\bm \lambda =[\lambda_1, \lambda_2, \ldots, \lambda_R]^\top\in \mathbb{R}^R$, we can write:
\begin{equation}
    \eX \approx [\![ \bm\lambda ;\tilde\cA_1, \tilde\cA_2,\ldots, \tilde\cA_N ]\!] = \sum_{r=1}^R \lambda_r \tilde\ba_{1r} \circ \tilde\ba_{2r} \circ \cdots \circ  \tilde\ba_{Nr}.
\end{equation}

To compute the CP decomposition of a tensor, one of the most widely used approaches is the Alternating Least Squares (ALS) algorithm. The CP-ALS algorithm iteratively updates each factor matrix while keeping the others fixed, thereby minimizing the squared Frobenius norm of the residual tensor. The algorithm begins by initializing the factor matrices $\cA_n\in \mathbb{R}^{I_n \times R}$ for $n = 1, \ldots, N$, where $R$ is the prespecified rank of the decomposition. For each iteration, the $n$-th factor matrix $\cA_n$ is updated by solving a linear least squares problem in closed form. This process involves matricizing the tensor along the $n$-th mode and expressing it in terms of the Khatri-Rao product of the remaining factor matrices. The algorithm continues iteratively until convergence is achieved, as determined by the change in fit or the maximum number of iterations. 

\section{Linked Data Factorization: Related Methods}
\subsection{BIDIFAC}
There is an extensive literature on the factorization of linked matrices that share common dimensions. The BIDIFAC approach is one of the methods focusing on joint decomposition of bidimensionally linked matrices \citep{park_lock_2019}. For clarity, consider the example of $K$ matrices $\{\cX_1, \cX_2, \ldots, \cX_K| \cX_k \in \mathbb{R}^{I_0\times I_1^{(k)}}, k = 1,2,\ldots,K\}$ that are horizontally linked, i.e., sharing the same rows (samples). The BIDIFAC method decomposes matrices as
\begin{equation}{\label{eq:2_1}}
    \cX_k = \cA_0^{(\text{share})} \cA_1^{(\text{share}, k)\top} + \cA_0^{(\text{indiv}, k)} \cA_1^{(\text{indiv}, k)\top} + \cE^{(k)} \text{ for } k=1,\ldots, K.
\end{equation}
The common sample loading matrix $\cA_0^{(\text{share})}\in \mathbb{R}^{I_0 \times R_\text{share}}$ represents the shared structures, explaining variability across multiple rows (samples). The score matrices $\cA_1^{(\text{share}, k)} \in \mathbb{R}^{I_1^{(k)} \times R_\text{share}}$ indicate how these loadings are expressed across columns in each dataset. Individual structure is represented by the loading matrices $\cA_0^{(\text{indiv}, k)}\in \mathbb{R}^{I_0 \times R_\text{indiv}^{(k)}}$ and score matrices $\cA_1^{(\text{indiv}, k)} \in \mathbb{R}^{I_1^{(k)} \times R_\text{indiv}^{(k)} }$ specific to each data source. $\cE^{(k)} \in \mathbb{R}^{I_0\times I_1^{(k)}}$ represents the Gaussian error term. $R_\text{share}$ is the rank of shared structures across datasets and $R_\text{indiv}^{(k)}$ is the rank of individual structure for the $k$-th matrix. Model \ref{eq:2_1} can be obtained by solving the following objective function:
\begin{align}
    f(\cA_0^{(\text{share})}, \ & \cA_0^{(\text{indiv}, k)}, \cA_1^{(\text{share}, k)}, \cA_1^{(\text{indiv}, k)}| k =1, \ldots, K) \nonumber\\
    =&\sum_{k=1}^K \| \cX_k -\cA_0^{(\text{share})} \cA_1^{(\text{share}, k)\top} - \cA_0^{(\text{indiv}, k)} \cA_1^{(\text{indiv}, k)\top}  \|_F^2 \nonumber\\
    &+ \sigma \left( \| \cA_0^{(\text{share})} \|_F^2 + \sum_{k=1}^K \| \cA_1^{(\text{share}, k)} \|_F^2\right) + \sum_{k=1}^K \sigma_k\left( \| \cA_0^{(\text{indiv}, k)} \|_F^2 + \|\cA_1^{(\text{indiv}, k)}\|_F^2 \right).
\end{align}

The Frobenius norm penalty on the factor matrices is motivated by its equivalence to the nuclear norm penalty on the corresponding matrix structures when minimizing the objective function. This relationship further motivates the choice of penalty factors $\{\sigma, \sigma_k|k=1,\ldots, K\}$ by utilizing the soft-thresholding property of nuclear norm penalty \citep{mazumder_2010}.


\subsection{Coupled Matrix and Tensor Factorization}
Building on the concept of integrative factorization for multiple matrices, coupled tensor and matrix factorization (CMTF) extends this approach to simultaneously decompose tensors and matrices that are linked in a shared dimension \citep{acar_2011, acar_2013}. This method allows for the extraction of shared latent structures across different data forms. A CMTF problem for a third-order tensor $\eX \in \mathbb{R}^{I_0\times I_1\times I_2}$ and a matrix $\cY \in \mathbb{R}^{I_0\times I_3}$ is typically formulated as 
\begin{equation}
    f(\cA_0, \cA_1, \cA_2, \cA_3) = \|\eX - [\![\cA_0, \cA_1, \cA_2]\!] \|^2 + \|\cY - \cA_0 \cA_3^\top\|^2
\end{equation}
This formula implicitly assumes that all columns of factor matrix $\cA_0$, i.e. $\{\ba_{0r} |r=1,\ldots, R\}$, are fully shared across datasets. To model shared and unshared structure, Acar et al. proposed to impose $L_1$ penalty to promote sparsity\citep{acar_2013}. The modified objective function is shown below:
\begin{align}
    f(\bm\lambda, \bd, \tilde\cA_0, \tilde\cA_1, \tilde\cA_2, \tilde\cA_3) &= \|\eX - [\![ \bm \lambda; \tilde\cA_0, \tilde\cA_1, \tilde\cA_2 ]\!] \|^2 + \|\cY - \tilde\cA_0 \cD \tilde\cA_3^\top\|^2 + \sigma \|\bm\lambda\|_1 + \sigma \|\bd\|_1\\
    &\text{s.t. }\|\ba_{0r}\| = \|\ba_{1r}\| = \|\ba_{2r}\| = \|\ba_{3r}\| = 1 \text{ for } r=1,\ldots, R, \notag
\end{align}
where $\bm \lambda$ and $\bd$ are the corresponding weights of rank-one components in the factorization of tensor and matrix and $\{\tilde\cA_0, \tilde\cA_1, \tilde\cA_2, \tilde\cA_3\}$ are normalized factor matrices with column norms to be 1. By imposing the $L_1$ penalty, the weights are sparsified so that unshared components will have 
$0$ weight in one of the datasets and shared components will have non-zero weight in both datasets.

\subsection{Structured Data Fusion}
The integrative factorization problem in the previous two sections is often referred to as a data fusion problem when analyzing data from multiple sources. Sorber et al. proposed a highly flexible framework, structured data fusion (SDF), to factorize multiple tensors that can be linked in various ways \citep{sorber_sdf_2015}. This general method can handle different types of factorization models, incorporate various penalties, and enable the representation of factors as functions of latent variables. Although SDF is versatile and applicable to a wide range of integrative factorization problems, it does not provide clear guidelines for selecting appropriate penalty parameters or determining tensor rank. To address the optimization challenge, the authors proposed two techniques: quasi-Newton and nonlinear least squares. For practical implementation, a MATLAB package called Tensorlab was developed, serving as the main comparison of our proposed model. 

\section{Proposed Methods}

\subsection{Single Tensor Penalized Factorization}
For the CP decomposition of a single tensor $\eX \in \mathbb{R}^{I_1\times \cdots \times I_N}$, we consider an objective function with $L_2$ penalties on the factor matrices as follows:
\begin{align} \label{eq:sing_tensor}
f = \big\|\eX-[\![\cA_{1},\ldots,\cA_N ]\!] \big\|_F^2 + \sigma \sum_{i=1}^N \big\| \cA_i \big\|_F^2,
\end{align}
where $\sigma$ is the penalty factor. Similar to the ALS algorithm for the unpenalized CP decomposition, we can solve the above objective by iteratively updating each factor matrix in a closed form while keeping the other fixed. The details are outlined in Algorithm \ref{alg:1}.

\begin{algorithm}
\caption{Alternating Least Squares (ALS) for CP Decomposition with $L_2$ Penalty}
\label{alg:1}
\begin{algorithmic}[1]
    \STATE \textbf{Input:} Tensor $\eX$, rank $R$, penalty factor $\sigma$, maximum iterations $T$, tolerance $\epsilon$
    \STATE \textbf{Output:} Factor matrices $\mathbf{A}_1, \ldots, \mathbf{A}_N$
    \STATE Initialize factor matrices $\mathbf{A}_1, \ldots, \mathbf{A}_N$ randomly
    \REPEAT
        \FOR{$i = 1, \ldots, N$}
        \STATE Update the $i$-th factor matrices as
            $$
            \mathbf{A}_i \leftarrow \arg \min_{\mathbf{A}_i} \left\| \cX_{(i)} - \mathbf{A}_i \left( \bigodot_{\substack{j=N \\ j \neq i}}^1 \mathbf{A}_j \right)^T \right\|_F^2 + \sigma \left\| \mathbf{A}_i \right\|_F^2,
            $$
            where $\cX_{(i)}$ is the mode-$i$ matricization of the tensor $\eX$. This update has a closed-form solution similar to the ridge regression framework \citep{hoerl_kennard_1970_ridge}.
        \ENDFOR
    \UNTIL improvement of objective function is less than $\epsilon$ or maximum iteration $T$ is reached.
\end{algorithmic}
\end{algorithm}
The following Theorem \ref{thm:1} extends Proposition~
\ref{prop:2} to the tensor context, demonstrating that a Frobenius norm penalty on the factor matrices in a tensor is equivalent to a penalty on the component weights (analogous to singular values) for the tensor.  

\begin{theorem}\label{thm:1}
For the CP decomposition problem of a $N$-way tensor, a $L_2$ penalty on factor matrices $\cA_i$, $i=1,\ldots,N$ is equivalent to a $L_{2/N}$ penalty on weights $\bm \lambda$, i.e,
$$
\operatorname*{min}\big\|\eX-[\![\cA_{1},\ldots,\cA_N ]\!] \big\|_F^2 + \sigma \sum_{i=1}^N \big\| \cA_i \big\|_F^2 = \operatorname*{min}\big\|\eX-[\![\bm\lambda; \tilde\cA_{1},\ldots,\tilde\cA_N ]\!] \big\|_F^2 +\sigma N \|\bm \lambda\|_{2/N}^{2/N},
$$
where $\{\tilde\cA_{1},\ldots,\tilde\cA_N\}$ are normalized factor matrices with column norms to be 1.
\end{theorem}
For any tensor order $N>2$, $L_{2/N}$ is a sparsity-inducing penalty and thus can shrink rank-1 components to $0$. This result utilizes the established property of the $L_p$ penalty in penalized regression models, where applying an $L_p$ penalty with $p \leq 1$ is known to shrink coefficients to zero \citep{tibshirani_1996_lasso, fu_1998}.  Theorem \ref{thm:2} extends this result to show how the adjusted weights can be calculated from the unpenalized decomposition in a rank-1 approximation problem. 

\begin{theorem}\label{thm:2}
Suppose that $(\hat{\lambda}, \hat{\ba}_1, \hat{\ba}_2, \ldots, \hat{\ba}_N)$ is the solution to the rank-1 approximation problem for a N-order tensor $\eX$.
\begin{equation*}
    \operatorname*{min} \big \| \eX - [\![ \lambda; \ba_1, \ba_2, \ldots, \ba_N]\!] \big \|_F^2
\end{equation*}
Then the solution to the penalized approximation problem
\begin{equation*}
    \operatorname*{min} \big \| \eX - [\![ \lambda; \ba_1, \ba_2, \ldots, \ba_N]\!] \big \|_F^2 + N\sigma \|\lambda\|_{2/N}^{2/N}
\end{equation*}
will be $[\![\hat{\lambda}_p; \hat{\ba}_1, \hat{\ba}_2, \ldots, \hat{\ba}_N]\!]$, where $\hat{\lambda}_p=\operatorname*{argmin}_{\lambda} (\lambda - \hat{\lambda})^2 + N\sigma \lambda^{2/N}$.
\end{theorem}

By setting $N=2$, the result reduces to $\hat{\lambda}_p = \hat{\lambda} - \sigma$, which is consistent with the soft-thresholding result for a matrix in Proposition~\ref{prop:1}.  An analogous result does not hold for rank greater than $1$, i.e., the factor matrices $\hat{\cA}_i$  will not necessarily be proportional to that for the unpenalized decomposition.     

\subsection{Linked Tensor Factorization}
Building upon the foundations of single tensor decomposition and the integrative factorization of matrices, we extend these techniques to the joint analysis of multiple tensors. The concept of ``linked" matrices can be naturally generalized to the tensor setting. We define multiple tensors as ``linked" if each pair shares the same number of dimensions in at least one mode. Specifically, a set of $K$ one-dimensionally linked tensors is defined as multiple tensors of different orders that share a common first mode (without loss of generality, we can refer to the shared mode as the first mode): 
$\{\eX_k:I_0\times I_1^{(k)}\times\cdots\times I_{N_k}^{(k)}|k=1,\ldots,K\}$. The proposed tensor decomposition formula is defined as
\begin{equation}
\begin{aligned}
    \eX_k\approx[\![ \bm \lambda_k; \tilde\cA_0, \tilde\cA_1^{(k)},\ldots,\tilde\cA_{N_k}^{(k)} ]\!], k = 1, \ldots, K.
\end{aligned}
\end{equation}

Instead of modeling shared and individual structures explicitly like BIDIFAC or adding an $L_1$ penalty on component weights like CMTF, we choose to model the shared and individual structures by $L_2$ penalization on the factor matrices as in \eqref{eq:sing_tensor}:
\begin{equation}\label{eq:9}
        f = \sum_{k=1}^K\big\|\eX_k-[\![\cA_0, \cA_1^{(k)},\ldots,\cA_{N_k}^{(k)} ]\!] \big\|_F^2  +\sigma\left(\big\|\cA_0\big\|_F^2 + \sum_{k=1}^K\sum_{i=1}^{N_k} \big\|\cA_i^{(k)}\big\|_F^2\right).
    \end{equation}
This objective is derived from Theorem \ref{thm:3}, which extends the rank sparsity property from the decomposition of a single tensor to the decomposition of multiple tensors. Theorem \ref{thm:3} demonstrates that applying an $L_2$ penalty to the factor matrices induces sparsity by shrinking some of the component weights to zero. In this framework, each tensor is initially assumed to be approximated by the sum of $R$ rank-1 tensors, where $R$ is a pre-specified initial rank. However, due to rank sparsity, the effective rank of each tensor becomes smaller, with each tensor ultimately approximated by the sum of only a subset of these rank-1 tensors. Shared components retain non-zero weights across all datasets, while individual components have non-zero weights exclusively in their respective datasets.

\begin{theorem}\label{thm:3}
The minimization problem for objective function \eqref{eq:9} is equivalent to the following problem with mixed $L_p$ and $L_2$ penalty:
\begin{equation*}\label{12}
\begin{aligned}
    &\operatorname{min} \sum_{k=1}^K\big\|\eX_k-[\![\cA_0, \cA_1^{(k)},\ldots,\cA_{N_k}^{(k)} ]\!] \big\|_F^2  +\sigma\left(\big\|\cA_0\big\|_F^2 + \sum_{k=1}^K\sum_{i=1}^{N_k} \big\|\cA_i^{(k)}\big\|_F^2\right)\\
    =\ &\operatorname{min}\sum_{k=1}^K\big\|\bm \eX_k-[\![\bm \lambda_0* \bm \lambda_{(0)k}; \tilde\cA_0, \tilde\cA_1^{(k)},\ldots,\tilde\cA_{N_k}^{(k)} ]\!]\big\|_F^2 + \sigma\left( \|\bm\lambda_0\|_2^2 + \sum_{k=1}^K N_k\|\bm\lambda_{(0)k} \|_{2/N_k}^{2/N_k}\right),
\end{aligned}
\end{equation*}
where the $r$-th element of $\bm \lambda_0$ is the Frobenius norm for the $r$-th column of shared factor matrix $\cA_0$. Similarly, the $r$-th element of $\bm \lambda_{(0)i}$ is the product of Frobenius norms for the $r$-th column of remaining factor matrices $\cA_{i}^{(k)}$, for $i=1,\ldots, N_k$.
\end{theorem}

An advantage of using an $L_2$ penalty on the factor matrices, compared to an $L_1$ norm on the weights $\bm \lambda$, is that our proposed objective function \eqref{eq:9} remains smooth while still retaining the sparsity-inducing properties of the $L_1$ penalty. This smoothness facilitates straightforward optimization, allowing for simple modifications to Algorithm \ref{alg:1}. With penalty factor $\sigma$ fixed, the ALS algorithm can be easily extended to handle multiple linked tensor decomposition as detailed in Algorithm \ref{alg:2}. Another advantage is that this objective leads to automatic rank selection with a single penalty. We begin with an upper bound on the overall rank $R$ ($\cA_{i}^{(k)}: I_i^{(k) \times R)}$), and the sparsity-inducing property of the penalty leads to lower rank shared and unshared structures as detailed in Figure~\ref{fig:multifac}. 
\begin{figure}[!h]
    \centering
    \subfigure[MULTIFAC Model]{\includegraphics[width=0.7\linewidth]{MULTIFAC.jpg}}

    \subfigure[Rank Sparsity Property]{\includegraphics[width=0.95\linewidth]{Rank_sparsity.jpg}}
    
    \caption{An illustration of MULTIFAC for two linked 3-way tensors. Subfigure (a) demonstrates how MULTIFAC decomposes tensors into shared and individual structures, where the individual components of the two tensors have distinct factor matrices $\cA_0^{(\text{indiv, 1})}$ and $\cA_0^{(\text{indiv, 2})}$. In practice, however, these individual structures are not explicitly modeled; instead, both tensors are assumed to share the same $\cA_0$, as depicted in the left-hand side of subfigure (b). Through penalization, certain columns are shrunk to zero, indicated by white regions, effectively reducing the rank and leading to the emergence of individual structures.}
    \label{fig:multifac}
\end{figure}

\begin{algorithm}[!h]
\caption{Alternating Least Squares (ALS) for Multi-Tensor CP Decomposition with $L_2$ Penalty}
\label{alg:2}
\begin{algorithmic}[1]
    \STATE \textbf{Input:} Tensor $\{\eX_k|k=1,\ldots,K\}$, rank $R$, penalty factor $\sigma$, maximum iterations $T$, tolerance $\epsilon$
    \STATE \textbf{Output:} Factor matrices $\cA_0$, $\{\cA_1^{(k)},\ldots,\cA_{N_k}^{(k)}| k=1, \ldots,K\}$.
    \STATE Initialize factor matrices randomly
    \REPEAT
        \STATE Update shared factor matrices as
        \begin{equation*}
            \cA_0 \leftarrow \arg \min_{\cA_0} \sum_{k=1}^K \left\| \cX_{k(1)} - \cA_0 \left( \bigodot_{j=N_k}^1 \cA_j^{(k)} \right)^T \right\|_F^2 + \sigma \left\| \cA_0 \right\|_F^2,
        \end{equation*}  
        where the update for $\cA_0$ has a closed-form solution similar to ridge regression \citep{hoerl_kennard_1970_ridge}.
        \FOR{$k = 1, \ldots, K$}
            \FOR{$i = 1, \ldots, N_k$}
                \STATE Update individual factor matrices as
                \begin{equation*}
                    \cA_i^{(k)} \leftarrow \arg \min_{\cA_i^{(k)}} \left\| \cX_{k(i+1)} - \cA_i^{(k)} \left( \bigodot_{\substack{j=N_k \\ j \neq i}}^1 \cA_j^{(k)} \odot \cA_0 \right)^T \right\|_F^2 + \sigma \left\| \cA_i^{(k)} \right\|_F^2,
                \end{equation*} 
                where the update for $\cA_i^{(k)}$ has a closed-form solution similar to ridge regression \citep{hoerl_kennard_1970_ridge}.
            \ENDFOR
        \ENDFOR
    \UNTIL improvement of objective function is less than $\epsilon$ or maximum iteration $T$ is reached.
\end{algorithmic}
\end{algorithm}




\subsection{Missing Data  Imputation}\label{sec: GALS}


When dealing with missing data, this algorithm can be extended using an expectation-maximization (EM) approach. During each iteration, the missing entries are updated based on their estimated low-rank approximations. Similar algorithms have been used for imputation of a single tensor with a CP factorization \cite{acar2011scalable}. 
 We consider two types of missing patterns. The first type is tensor-wise missing, where data is missing in an entire sample for one of the $K$ tensors, meaning all entries in a particular mode are absent. The second type is entry-wise missing, where individual entries are randomly missing throughout the tensor.

\begin{algorithm}

\caption{EM-ALS for Multi-Tensor CP Decomposition with $L_2$ Penalty}
\label{alg:3}
    \begin{algorithmic}[1]
        \STATE \textbf{Input:} Tensor $\{\eX_k|k=1,\ldots,K\}$, rank $R$, penalty factor $\sigma$, maximum iterations $T$, tolerance $\epsilon$
        \STATE \textbf{Output:} Factor matrices $\cA_0$, $\{\cA_1^{(k)},\ldots,\cA_{N_k}^{(k)}| k=1, \ldots,K\}$.
        \STATE Initialize factor matrices randomly.
        \STATE Impute entry-wise missing values with current low-rank structure and impute tensor-wise missing values with shared low-rank structure.
        \STATE Update factor matrices as in Step 5 to 10 in Algorithm \ref{alg:2}.
        \STATE Repeat Step 4 and 5 until convergence.
        \STATE Impute entry-wise missing values with current low-rank structure and impute tensor-wise missing values with shared low-rank structure.
    \end{algorithmic}
\end{algorithm}

During the iterative imputation process, the entry-wise missing values are updated using their corresponding low-rank approximations derived from the current estimates. For tensor-wise missing data, only the shared low-rank structures across samples are used for updating. This approach is adopted because other samples do not provide complete information about an entirely missing sample, particularly regarding the scale (i.e., sample loading) of a completely missing sample. However, the shared structures can still offer valuable scale information by leveraging the available data sample from other sources to help impute the shared sample loading. The imputation algorithm is described in Algorithm \ref{alg:3}.

\subsection{Parameter Tuning and Rank Selection}
\label{sec: two_step_estimation}
In this section, we discuss the estimation procedure for both parameter tuning and the selection of the tensor ranks. The proposed framework consists of two cross-validation steps: step 1 determines the tensor ranks using $L_2$ penalization, while step 2 further tunes the penalty parameter $\sigma$ with the ranks fixed. A key summary statistic employed in this process is the relative squared error (RSE), defined as:
$$
\text{RSE}(\hat{\eX},\eS) = \frac{\|\hat\eX - \eS\|_F^2}{\|\eS\|_F^2},
$$
where $\hat{\eX}$ represents the estimated structure and $\eS$ is the true signal $\eX$. In practice the true low-rank signal is unknown, and in what follows we use the RSE for data artificially held-out as missing, that is, $\text{RSE}_{\text{missing}}=\text{RSE}(\hat{X}_\text{missing},X_\text{missing})$.   

Step 1: Determine tensor rank using cross-validation. The first cross-validation step addresses the challenge of determining the ranks of the underlying shared and unshared signals. Starting with a large initial rank, the rank sparsity property is leveraged to guide the selection process. Cross-validation is  performed, in which data are artificially held out as missing and imputed, over different penalty factors $\sigma$.  The performance metric is the average $\text{RSE}_{\text{missing}}$ calculated over the validation sets. We employ the one standard error rule \citep{1se_rule}. This rule selects the most parsimonious model whose performance is within one standard error of the minimum $\text{RSE}_{\text{missing}}$ observed across the grid. The rationale behind this approach is to favor simpler models that are still close in performance to the best model, thereby avoiding overfitting and enhancing generalizability. The ranks (i.e. number of non-zero components) for the shared and unshared structures in the selected model will be fixed for Step 2. 

Step 2: Further tune the penalty parameter $\sigma$. In the second cross-validation step, we fix the tensor ranks as determined in the first step and perform a grid search to fine-tune the penalty parameter $\sigma$. In this step, we aim to identify the optimal $\sigma$ that yields the best imputation performance, thereby ensuring the most accurate factorization. The grid search evaluates a range of penalty values, and the optimal $\sigma$ is selected as the one that minimizes the average $\text{RSE}_{\text{missing}}$  across the validation sets.

There are several important practical considerations when calculating the decomposition for multiple tensors. Previous studies have shown that the ALS algorithm is not guaranteed to converge to a global minimum and can be significantly influenced by the choice of initialization \citep{kolda_tensor_review_2009}. To mitigate this issue, we begin with multiple randomly initialized factor matrices and select the solution that achieves the lowest value of the final unpenalized objective function. Additionally, to stabilize the results, we employ a technique known as tempered regularization, where we start with a low penalty value and gradually increase it to the desired level \citep{lock_park_2020}. This approach facilitates a smoother optimization process toward the optimal solution.

\section{Simulation}
To evaluate the performance of our proposed tensor decomposition algorithm, we conducted simulation studies for both single and multiple tensor decomposition. These simulations assess the accuracy and robustness of the algorithm under various noise conditions and dimensions. The tensor data consist of a true low-rank signal and additive noise. For all simulations, the true signal was generated using factor matrices, where each column was sampled from a standard normal distribution. Noise was simulated from a normal distribution, with variance adjusted to achieve predefined signal-to-noise ratio (SNR) levels. To evaluate robustness under varying noise conditions, the signal-to-noise ratio (SNR) was set at three levels: $1/3$, 1, and 3 across all simulation settings.

\subsection{Single Tensor Decomposition}
\subsubsection{Simulation Setup}
We first evaluate the method's performance on single tensor decomposition before extending the analysis to multiple tensor decomposition. The generated tensor has size $50\times 50\times 50$. The low-rank for the tensor was set to be 5 and the pre-specified rank in estimation under Algorithm!~\ref{alg:2} was set to be 20. Performance metrics were calculated at each step of the selection procedure in section~\ref{sec: two_step_estimation}, including the parsimonious model with selected $\sigma$ value at step 1, unpenalized model with selected rank after step 1, and the ``optimal" model after step 2. We also compared them with the standard unpenalized  CP decomposition model with true rank. 

We considered the decomposition of both complete tensors and tensors with missing data. For the complete tensor scenario, we evaluated the decomposition accuracy of our methods and compared it with the Nonlinear Least Squares (NLS) solver given true rank from the Tensorlab package \citep{sorber_sdf_2015}. In the missing data imputation simulation, we randomly removed 10\% of each tensor and assess the imputation performance. Each simulation setting was repeated 100 times to assess the stability of the results.
\subsubsection{Simulation Results}
Table \ref{table:2.5} presents the decomposition accuracy for single complete tensor decomposition, measured using RSE across different SNR levels. Notably, once the rank was selected in our model, there was no significant difference in decomposition performance across methods. It is worth noticed that our method consistently picked the true rank in step 1 of coss validation across all settings.
Table \ref{table:2.6} summarizes the results for single tensor imputation, measured by RSE comparing imputed value with underlying missing signal. The results indicate that our ``optimal" model consistently outperforms alternative methods, even when the true rank is specified, with the performance gap becoming more pronounced as the SNR increases. 

\begin{table}[!h]
\centering
\caption{Relative square error for single complete tensor decomposition}
\begin{tabular}{|l|ll|ll|ll|}
\hline
 & \multicolumn{2}{l|}{SNR = 1/3} & \multicolumn{2}{l|}{SNR = 1} & \multicolumn{2}{l|}{SNR = 3} \\ \hline
\textbf{Method} & \multicolumn{1}{l|}{Mean} & SD & \multicolumn{1}{l|}{Mean} & SD & \multicolumn{1}{l|}{Mean} & SD \\ \hline
Step 1 & \multicolumn{1}{l|}{0.1588} & 0.008 & \multicolumn{1}{l|}{0.0958} & 0.0056 & \multicolumn{1}{l|}{0.0578} & 0.0037 \\ \hline
Constraint & \multicolumn{1}{l|}{0.134} & 0.0038 & \multicolumn{1}{l|}{0.0769} & 0.0021 & \multicolumn{1}{l|}{0.0443} & 0.0012 \\ \hline
Step 2 & \multicolumn{1}{l|}{0.1336} & 0.0038 & \multicolumn{1}{l|}{0.0768} & 0.0021 & \multicolumn{1}{l|}{0.0443} & 0.0012 \\ \hline
True Rank & \multicolumn{1}{l|}{0.134} & 0.0038 & \multicolumn{1}{l|}{0.0769} & 0.0021 & \multicolumn{1}{l|}{0.0443} & 0.0012 \\ \hline
NLS & \multicolumn{1}{l|}{0.134} & 0.0038 & \multicolumn{1}{l|}{0.0769} & 0.0021 & \multicolumn{1}{l|}{0.0443} & 0.0012 \\ \hline
\end{tabular}
\label{table:2.5}
\end{table}

\begin{table}[!h]
\centering
\caption{Relative square error for single tensor imputation}
\begin{tabular}{|l|ll|ll|ll|}
\hline
 & \multicolumn{2}{l|}{SNR = 1/3} & \multicolumn{2}{l|}{SNR = 1} & \multicolumn{2}{l|}{SNR = 3} \\ \hline
Mehod & \multicolumn{1}{l|}{Mean} & SD & \multicolumn{1}{l|}{Mean} & SD & \multicolumn{1}{l|}{Mean} & SD \\ \hline
Step 1 & \multicolumn{1}{l|}{0.1653} & 0.0067 & \multicolumn{1}{l|}{0.101} & 0.0045 & \multicolumn{1}{l|}{0.0611} & 0.0034 \\ \hline
Constraint & \multicolumn{1}{l|}{0.1419} & 0.0044 & \multicolumn{1}{l|}{0.0901} & 0.0515 & \multicolumn{1}{l|}{0.0729} & 0.0893 \\ \hline
Step 2 & \multicolumn{1}{l|}{0.1405} & 0.0043 & \multicolumn{1}{l|}{0.0811} & 0.0024 & \multicolumn{1}{l|}{0.0468} & 0.0014 \\ \hline
True Rank & \multicolumn{1}{l|}{0.1419} & 0.0044 & \multicolumn{1}{l|}{0.0892} & 0.0462 & \multicolumn{1}{l|}{0.0715} & 0.0851 \\ \hline
\end{tabular}
\label{table:2.6}
\end{table}

\subsection{Multiple Tensor Decomposition}
\subsubsection{Simulation Setup}
In this section, we conducted two sets of simulations to evaluate the performance of the MULTIFAC method in the context of linked tensor factorization. Specifically, we assess (1) the factorization of multiple complete tensors and (2)the imputation of missing values. Each simulation was performed using two different tensor dimension settings:
\begin{itemize}
    \item Tensors with identical dimensions $50 \times 50 \times 50$;
    \item Tensors with different dimensions: the first tensor is $100 \times 100 \times 4$ and the second tensor is $100 \times 40 \times 10 \times 3$,
\end{itemize}
where the first mode of the tensors was assumed to be linked in both settings. The true rank of the tensors was set as 2 for shared structures and 3 for individual structures in each tensor. Each scenario was repeated 50 times.

For the first simulation with complete tensors, we compared the  decomposition accuracy of MULTIFAC and the NLS solver from the Tensorlab package \citep{sorber_sdf_2015}. For NLS, we used two rank settings: the true rank and twice the true rank. The RSEs were computed to evaluate the accuracy in recovering the full signal, shared components, and individual components. In the second simulation, we evaluated the imputation performance of MULTIFAC when tensors contain missing values. For each tensor, 10\% of the entries were randomly set to be missing, comprising 5\% tensor-wise missing entries and 5\% entry-wise missing entries. We reported the RSE for all missing entries, as well as separately for randomly missing entries and tensor-wise missing entries.

\subsubsection{Simulation Results}
Tables \ref{table:2.1} and \ref{table:2.2} present the decomposition accuracy for complete tensors across various SNR levels. The pre-specified rank for MULTIFAC was set to 20, exceeding the true tensor rank. Overall, MULTIFAC consistently outperformed NLS in terms of RSE under all tested conditions, demonstrating greater accuracy in uncovering the underlying signals. When comparing the ability to decompose different structures, MULTIFAC significantly outperformed NLS. Notably, NLS often failed to correctly estimate these structures (SNR $>$ 1), even when the true rank was provided.

Imputation performance is summarized in Tables \ref{table:2.3} and \ref{table:2.4}. The RSE for entry-wise imputation closely matched the RSE for the observed tensor entries. As expected, tensor-wise imputation was less accurate than entry-wise imputation, since only shared structures were used for imputing missing tensors. Nonetheless, with RSE values significantly below 1, it can be concluded that MULTIFAC achieved a reasonable degree of accuracy in recovering tensor-wise missing entries.


\begin{table}[!h]
\caption{Performance metrics for decomposing complete tensors of varying sizes. The superscripts denote the tensor being analyzed (e.g., $\text{RSE}^1_\text{full}$ for Tensor 1) while subscripts identify the structural components evaluated: ``full" for the complete signal, ``share" for shared signals across tensors, and ``indiv" for individual tensor signals. Each cell reports the average RSEs, with the standard deviation of RSEs shown in parentheses.}
\renewcommand{\arraystretch}{1.2}
\centering
\begin{tabular}{|c|c|c|c|c|c|c|c|}
\hline
Method & SNR &$\text{RSE}^1_\text{full}$& $\text{RSE}^2_\text{full}$& $\text{RSE}^1_\text{share}$ & $\text{RSE}^2_\text{share}$ & $\text{RSE}^1_\text{Indiv}$ & $\text{RSE}^2_\text{Indiv}$ \\ \hline
MULTIFAC & \multirow{5}{*}{3} & \begin{tabular}[c]{@{}c@{}}0.113\\ (0.025)\end{tabular} & \begin{tabular}[c]{@{}c@{}}0.046\\ (0.003)\end{tabular} & \begin{tabular}[c]{@{}c@{}}0.162\\ (0.187)\end{tabular} & \begin{tabular}[c]{@{}c@{}}0.112\\ (0.227)\end{tabular} & \begin{tabular}[c]{@{}c@{}}0.156\\ (0.095)\end{tabular} & \begin{tabular}[c]{@{}c@{}}0.116\\ (0.233)\end{tabular} \\ 
NLS with true rank &  & \begin{tabular}[c]{@{}c@{}}0.25\\ (0.145)\end{tabular} & \begin{tabular}[c]{@{}c@{}}0.212\\ (0.127)\end{tabular} & \begin{tabular}[c]{@{}c@{}}1.327\\ (2.371)\end{tabular} & \begin{tabular}[c]{@{}c@{}}0.975\\ (0.524)\end{tabular} & \begin{tabular}[c]{@{}c@{}}1.028\\ (1.573)\end{tabular} & \begin{tabular}[c]{@{}c@{}}0.814\\ (0.409)\end{tabular} \\ 
NLS with double rank &  & \begin{tabular}[c]{@{}c@{}}0.142\\ (0.008)\end{tabular} & \begin{tabular}[c]{@{}c@{}}0.065\\ (0.011)\end{tabular} & \begin{tabular}[c]{@{}c@{}}1.334\\ (1.765)\end{tabular} & \begin{tabular}[c]{@{}c@{}}1.096\\ (0.672)\end{tabular} & \begin{tabular}[c]{@{}c@{}}1.012\\ (1.003)\end{tabular} & \begin{tabular}[c]{@{}c@{}}0.876\\ (0.455)\end{tabular} \\ \hline
MULTIFAC & \multirow{5}{*}{1} & \begin{tabular}[c]{@{}c@{}}0.169\\ (0.02)\end{tabular} & \begin{tabular}[c]{@{}c@{}}0.083\\ (0.02)\end{tabular} & \begin{tabular}[c]{@{}c@{}}0.235\\ (0.224)\end{tabular} & \begin{tabular}[c]{@{}c@{}}0.144\\ (0.213)\end{tabular} & \begin{tabular}[c]{@{}c@{}}0.232\\ (0.143)\end{tabular} & \begin{tabular}[c]{@{}c@{}}0.161\\ (0.229)\end{tabular} \\ 
NLS with true rank &  & \begin{tabular}[c]{@{}c@{}}0.295\\ (0.117)\end{tabular} & \begin{tabular}[c]{@{}c@{}}0.24\\ (0.13)\end{tabular} & \begin{tabular}[c]{@{}c@{}}1.254\\ (1.211)\end{tabular} & \begin{tabular}[c]{@{}c@{}}1.509\\ (3.305)\end{tabular} & \begin{tabular}[c]{@{}c@{}}1.051\\ (1.093)\end{tabular} & \begin{tabular}[c]{@{}c@{}}1.363\\ (2.941)\end{tabular} \\ 
NLS with double rank &  & \begin{tabular}[c]{@{}c@{}}0.257\\ (0.011)\end{tabular} & \begin{tabular}[c]{@{}c@{}}0.12\\ (0.007)\end{tabular} & \begin{tabular}[c]{@{}c@{}}1.209\\ (0.489)\end{tabular} & \begin{tabular}[c]{@{}c@{}}1.068\\ (0.43)\end{tabular} & \begin{tabular}[c]{@{}c@{}}1.005\\ (0.45)\end{tabular} & \begin{tabular}[c]{@{}c@{}}0.927\\ (0.436)\end{tabular} \\ \hline
MULTIFAC & \multirow{5}{*}{1/3} & \begin{tabular}[c]{@{}c@{}}0.274\\ (0.028)\end{tabular} & \begin{tabular}[c]{@{}c@{}}0.151\\ (0.041)\end{tabular} & \begin{tabular}[c]{@{}c@{}}0.379\\ (0.239)\end{tabular} & \begin{tabular}[c]{@{}c@{}}0.293\\ (0.305)\end{tabular} & \begin{tabular}[c]{@{}c@{}}0.392\\ (0.207)\end{tabular} & \begin{tabular}[c]{@{}c@{}}0.31\\ (0.276)\end{tabular} \\ 
NLS with true rank &  & \begin{tabular}[c]{@{}c@{}}0.361\\ (0.094)\end{tabular} & \begin{tabular}[c]{@{}c@{}}0.262\\ (0.127)\end{tabular} & \begin{tabular}[c]{@{}c@{}}1.378\\ (1.584)\end{tabular} & \begin{tabular}[c]{@{}c@{}}0.903\\ (0.632)\end{tabular} & \begin{tabular}[c]{@{}c@{}}1.141\\ (1.168)\end{tabular} & \begin{tabular}[c]{@{}c@{}}0.761\\ (0.5)\end{tabular} \\ 
NLS with double rank &  & \begin{tabular}[c]{@{}c@{}}0.46\\ (0.019)\end{tabular} & \begin{tabular}[c]{@{}c@{}}0.218\\ (0.016)\end{tabular} & \begin{tabular}[c]{@{}c@{}}1.448\\ (1.177)\end{tabular} & \begin{tabular}[c]{@{}c@{}}0.882\\ (0.464)\end{tabular} & \begin{tabular}[c]{@{}c@{}}1.227\\ (0.908)\end{tabular} & \begin{tabular}[c]{@{}c@{}}0.811\\ (0.492)\end{tabular} \\ \hline
\end{tabular}
\label{table:2.1}
\end{table}

\begin{table}[!h]
\caption{Performance metrics for decomposing complete tensors of same sizes. The superscripts denote the tensor being analyzed (e.g., $\text{RSE}^1_\text{full}$ for Tensor 1) while subscripts identify the structural components evaluated: ``full" for the complete signal, ``share" for shared signals across tensors, and ``indiv" for individual tensor signals. Each cell reports the average RSEs, with the standard deviation of RSEs shown in parentheses.}
\renewcommand{\arraystretch}{1.2}
\centering
\begin{tabular}{|c|c|c|c|c|c|c|c|}
\hline
Method & SNR &$\text{RSE}^1_\text{full}$ & $\text{RSE}^2_\text{full}$ & $\text{RSE}^1_\text{share}$ & $\text{RSE}^2_\text{share}$ & $\text{RSE}^1_\text{Indiv}$ & $\text{RSE}^2_\text{Indiv}$ \\ \hline
MULTIFAC & \multirow{5}{*}{3} & \begin{tabular}[c]{@{}c@{}}0.132\\ (0.043)\end{tabular} & \begin{tabular}[c]{@{}c@{}}0.13\\ (0.043)\end{tabular} & \begin{tabular}[c]{@{}c@{}}0.113\\ (0.037)\end{tabular} & \begin{tabular}[c]{@{}c@{}}0.112\\ (0.04)\end{tabular} & \begin{tabular}[c]{@{}c@{}}0.145\\ (0.05)\end{tabular} & \begin{tabular}[c]{@{}c@{}}0.145\\ (0.049)\end{tabular} \\ 

NLS with true rank &  & \begin{tabular}[c]{@{}c@{}}0.079\\ (0.003)\end{tabular} & \begin{tabular}[c]{@{}c@{}}0.074\\ (0.003)\end{tabular} & \begin{tabular}[c]{@{}c@{}}0.95\\ (0.428)\end{tabular} & \begin{tabular}[c]{@{}c@{}}0.915\\ (0.33)\end{tabular} & \begin{tabular}[c]{@{}c@{}}0.8\\ (0.372)\end{tabular} & \begin{tabular}[c]{@{}c@{}}0.758\\ (0.263)\end{tabular} \\ 
NLS with double rank &  & \begin{tabular}[c]{@{}c@{}}0.079\\ (0.003)\end{tabular} & \begin{tabular}[c]{@{}c@{}}0.074\\ (0.003)\end{tabular} & \begin{tabular}[c]{@{}c@{}}0.95\\ (0.428)\end{tabular} & \begin{tabular}[c]{@{}c@{}}0.915\\ (0.33)\end{tabular} & \begin{tabular}[c]{@{}c@{}}0.8\\ (0.372)\end{tabular} & \begin{tabular}[c]{@{}c@{}}0.758\\ (0.263)\end{tabular} \\ \hline
MULTIFAC & \multirow{5}{*}{1} & \begin{tabular}[c]{@{}c@{}}0.147\\ (0.033)\end{tabular} & \begin{tabular}[c]{@{}c@{}}0.147\\ (0.036)\end{tabular} & \begin{tabular}[c]{@{}c@{}}0.169\\ (0.202)\end{tabular} & \begin{tabular}[c]{@{}c@{}}0.163\\ (0.246)\end{tabular} & \begin{tabular}[c]{@{}c@{}}0.19\\ (0.141)\end{tabular} & \begin{tabular}[c]{@{}c@{}}0.18\\ (0.125)\end{tabular} \\ 
NLS with true rank &  & \begin{tabular}[c]{@{}c@{}}0.202\\ (0.152)\end{tabular} & \begin{tabular}[c]{@{}c@{}}0.328\\ (0.139)\end{tabular} & \begin{tabular}[c]{@{}c@{}}1.013\\ (0.65)\end{tabular} & \begin{tabular}[c]{@{}c@{}}0.852\\ (0.55)\end{tabular} & \begin{tabular}[c]{@{}c@{}}0.882\\ (0.58)\end{tabular} & \begin{tabular}[c]{@{}c@{}}0.698\\ (0.676)\end{tabular} \\ 
NLS with double rank &  & \begin{tabular}[c]{@{}c@{}}0.138\\ (0.004)\end{tabular} & \begin{tabular}[c]{@{}c@{}}0.13\\ (0.004)\end{tabular} & \begin{tabular}[c]{@{}c@{}}1.035\\ (0.401)\end{tabular} & \begin{tabular}[c]{@{}c@{}}0.923\\ (0.329)\end{tabular} & \begin{tabular}[c]{@{}c@{}}0.867\\ (0.305)\end{tabular} & \begin{tabular}[c]{@{}c@{}}0.768\\ (0.27)\end{tabular} \\ \hline
MULTIFAC & \multirow{5}{*}{1/3} & \begin{tabular}[c]{@{}c@{}}0.154\\ (0.015)\end{tabular} & \begin{tabular}[c]{@{}c@{}}0.154\\ (0.016)\end{tabular} & \begin{tabular}[c]{@{}c@{}}0.946\\ (0.363)\end{tabular} & \begin{tabular}[c]{@{}c@{}}0.919\\ (0.423)\end{tabular} & \begin{tabular}[c]{@{}c@{}}0.825\\ (0.267)\end{tabular} & \begin{tabular}[c]{@{}c@{}}0.793\\ (0.29)\end{tabular} \\ 
NLS with true rank &  & \begin{tabular}[c]{@{}c@{}}0.277\\ (0.151)\end{tabular} & \begin{tabular}[c]{@{}c@{}}0.381\\ (0.124)\end{tabular} & \begin{tabular}[c]{@{}c@{}}1.026\\ (0.398)\end{tabular} & \begin{tabular}[c]{@{}c@{}}1.558\\ (2.343)\end{tabular} & \begin{tabular}[c]{@{}c@{}}0.881\\ (0.35)\end{tabular} & \begin{tabular}[c]{@{}c@{}}1.326\\ (1.973)\end{tabular} \\ 
NLS with double rank &  & \begin{tabular}[c]{@{}c@{}}0.241\\ (0.006)\end{tabular} & \begin{tabular}[c]{@{}c@{}}0.226\\ (0.008)\end{tabular} & \begin{tabular}[c]{@{}c@{}}1.208\\ (0.539)\end{tabular} & \begin{tabular}[c]{@{}c@{}}1.075\\ (0.385)\end{tabular} & \begin{tabular}[c]{@{}c@{}}1.003\\ (0.389)\end{tabular} & \begin{tabular}[c]{@{}c@{}}0.92\\ (0.359)\end{tabular} \\ \hline
\end{tabular}
\label{table:2.2}
\end{table}


\begin{table}[!h]
\caption{Imputation performance for tensors of varying sizes. The subscript denote the signal subsets evaluated: ``observe" for observed signal, ``missing" for full missing signal, ``entry-wise" for entry-wise missing signal and ``tensor-wise" for tensor-wise missing signal. Each cell reports the average RSE, with the standard deviation of RSE shown in parentheses.}
\centering
\begin{tabular}{|c|c|c|c|c|c|}
\hline
Tensor Size & SNR & $\text{RSE}_\text{observe}$ & $\text{RSE}_\text{missing}$ & $\text{RSE}_\text{entry-wise}$ & $\text{RSE}_\text{tensor-wise}$ \\ \hline
$100 \times 100 \times 4$ & \multirow{2}{*}{3} & 0.126 (0.032) & 0.543 (0.095) & 0.129 (0.033) & 0.262 (0.206) \\  
$100\times 40 \times 10 \times 3$ &  & 0.049 (0.004) & 0.525 (0.117) & 0.049 (0.004) & 0.232 (0.252) \\ \hline
$100 \times 100 \times 4$ & \multirow{2}{*}{1} & 0.174 (0.016) & 0.564 (0.093) & 0.179 (0.017) & 0.327 (0.221) \\  
$100\times 40 \times 10 \times 3$ &  & 0.084 (0.018) & 0.531 (0.112) & 0.086 (0.019) & 0.281 (0.24) \\ \hline
$100 \times 100 \times 4$ & \multirow{2}{*}{1/3} & 0.286 (0.033) & 0.602 (0.086) & 0.292 (0.036) & 0.447 (0.227) \\  
$100\times 40 \times 10 \times 3$ &  & 0.164 (0.051) & 0.561 (0.112) & 0.166 (0.053) & 0.39 (0.232) \\ \hline
\end{tabular}
\label{table:2.3}
\end{table}

\begin{table}[!h]
\caption{Imputation performance for tensors of same sizes. The subscript denote the signal subsets evaluated: ``observe" for observed signal, ``missing" for full missing signal, ``entry-wise" for entry-wise missing signal and ``tensor-wise" for tensor-wise missing signal. Each cell reports the average RSE, with the standard deviation of RSE shown in parentheses.}
\centering
\begin{tabular}{|c|c|c|c|c|c|}
\hline
Tensor Size & SNR & $\text{RSE}_\text{observe}$ & $\text{RSE}_\text{missing}$ & $\text{RSE}_\text{entry-wise}$ & $\text{RSE}_\text{tensor-wise}$  \\ \hline
$50 \times 50 \times 50$ & \multirow{2}{*}{3} & 0.122 (0.053) & 0.593 (0.103) & 0.124 (0.053) & 0.455 (0.315) \\  
$50 \times 50 \times 50$ &  & 0.12 (0.052) & 0.587 (0.1) & 0.122 (0.053) & 0.445 (0.294) \\ \hline
$50 \times 50 \times 50$ & \multirow{2}{*}{1} & 0.15 (0.033) & 0.555 (0.097) & 0.152 (0.033) & 0.284 (0.16) \\  
$50 \times 50 \times 50$ &  & 0.148 (0.03) & 0.556 (0.104) & 0.15 (0.031) & 0.293 (0.187) \\ \hline
$50 \times 50 \times 50$ & \multirow{2}{*}{1/3} & 0.172 (0.03) & 0.552 (0.095) & 0.174 (0.03) & 0.279 (0.121) \\  
$50 \times 50 \times 50$ &  & 0.171 (0.024) & 0.554 (0.101) & 0.173 (0.025) & 0.288 (0.147) \\ \hline
\end{tabular}
\label{table:2.4}
\end{table}

\section{Real Data Application}
\subsection{Sysmex Hematology and MRI data}
We applied our method to analyze the progression of iron deficiency in a cohort of infant monkeys observed from birth to eight months of age. Hematology indices were collected at 2 weeks, and 2, 4, 6, 8 months after birth using a Sysmex hematology analyzer; for more detail see \citet{rao_lock_etal_2023}. Magnetic Resonance Imaging (MRI) data were also collected after eight months, focusing on four diffusion tensor imaging (DTI) parameters—axial diffusivity (AD), fractional anisotropy (FA), mean diffusivity (MD), and radial diffusivity (RD)—across 14 brain regions. Integrative analysis was conducted using the subset of infants with matched identifiers in both the hematology and MRI datasets. 

The final datasets contains 19 infants, among which 8 developed anemia, defined as hemoglobin levels below 10 g/dL during the observation period. The first dataset is a 3-way tensor contains the hematology information $\eX_1 \in \mathbb{R}^{19\times 5 \times 18}$, organized as $\textit{Monkey Infants} \times \textit{Age} \times \textit{Hematology Indices}$. The second tensor dataset contains the MRI data: $\eX_2\in \mathbb{R}^{19\times 4 \times 14}$, organized as $\textit{Monkey Infants} \times \textit{DTI Parameters} \times \textit{Brain Regions}$. While the MRI data were complete, the hematology tensor $\eX_1$ exhibited missing entries at certain time points that we would like to impute. We aim to apply the MULTIFAC method to identify whether there is common structures between $\eX_1$ and $\eX_2$, which could potentially indicate associations between hematological profiles and difference in brain function after the onset of iron deficiency.   

\subsection{Results}
We summarized the proportion of variance explained by various structures and their estimated tensor ranks in Table \ref{table:6_1}. For the analysis based on the common ID subset, 20–40\% of the variability was captured by the shared structures between hematology and MRI data. For each structure, we visualized the top two sample loadings associated with the rank-1 components explaining the largest variance in Figure \ref{fig:2.1}. The anemia status of each sample was indicated by color. The shared structure plots exhibited some patterns associated with anemia development but did not provide clear discrimination. Notably, the individual hematology loadings clearly discriminated anemia status, whereas the individual MRI loadings did not. This is expected, as hematology is a more directly related to anemia status (so any patterns associated with anemia in MRI would also be shared with hematology).     

Interestingly, the sample loadings for the second rank-1 component demonstrated better discrimination of anemia status than those for the first component for both the shared and individual structure. Figures \ref{fig:2.2} and \ref{fig:2.3} present the corresponding loadings for other dimensions. For the second shared component, the 6 and 8 month time-points have the largest loadings for the hematology data; these time-points are closer to the MRI data collection, which occurred at approximately 12-months. The second component for the hematology-specific structure showed the best discrimination  by anemia status, and had the largest loadings at 4 and 6 months; these time points are generally when the effects of anemia are greatest, prior to recovery.

\begin{figure}[H]
    \centering
    \includegraphics[width=\linewidth]{loadings_common_id.png}    
    \caption{Top Sample Loadings in Different Structures}
    \label{fig:2.1}
\end{figure}

\begin{figure}[H]
    \centering
    \includegraphics[width=0.95\linewidth]{shared_loading.png}    
    \caption{Loading plots for the second component of in shared structure}
    \label{fig:2.2}
\end{figure}

\begin{figure}[H]
    \centering
    \includegraphics[width=0.95\linewidth]{indiv_loading.png}    
    \caption{Loading plots for the second component of the individual structure for hematology.}
    \label{fig:2.3}
\end{figure}

\begin{table}[!h]
\caption{Application of MULTIFAC to Sysmex Hematology and MRI data. Each cell reports the proportion of variance explained by different structures, with the estimated rank shown inside parentheses.}
\centering
\begin{tabular}{|l|l|l|l|}
\hline
Dataset, Structure & Total & Shared & Individual\\ \hline
Hematology ($\eX_1$) & 0.757 (13) & 0.244 (6) & 0.451 (7) \\ \hline
MRI ($\eX_2$) & 0.708 (16) & 0.373 (6) & 0.311 (10) \\ \hline
\end{tabular}
\label{table:6_1}
\end{table}



\section{Discussion}
This work introduces a novel framework for tensor decomposition, designed specifically to analyze multiple tensors linked through their first mode. The proposed method effectively distinguishes shared and individual structures across datasets while handling missing data scenarios, such as entry-wise and tensor-wise missing patterns. Through rigorous simulations, the method demonstrated superior performance over existing approaches, consistently achieving lower RSE values across a wide range of conditions. Notably, the penalty framework employed enables rank sparsity, facilitating automatic determination of tensor ranks without compromising model robustness.

While the proposed method offers significant improvements, there are areas requiring further exploration. First, the current grid search approach for selecting penalty factors is computationally intensive, particularly for large datasets. Future efforts could focus on incorporating advanced optimization techniques to expedite the penalty tuning process while maintaining robustness. Additionally, empirical methods for penalty selection could be replaced with theoretically grounded approaches, such as variational Bayesian models that directly treat the penalty as a prior parameter or random tensor theory to derive penalties systematically.  Moreover, the current framework is tailored for tensors linked along their first mode. Extending this model to scenarios where subsets of tensors share multiple modes presents an exciting challenge. Such an extension would enhance the applicability of the method to more complex multi-way data structures but would necessitate the development of more efficient optimization algorithms to manage increased computational complexity.

\section*{Supplementary materials} 
Python code to perform MULTIFAC with examples are available at \url{https://github.com/zhiyu-kang/MULTIFAC}. 

\section*{Acknowledgment}
This work was supported by the NIH National Institute of General Medical Sciences
(NIGMS) grant R01-GM130622. The data used in this study  was funded by grants from the National Institute of Health/Eunice Kennedy Shriver National Institute of Child Health and Development (HD089989) and Sysmex America, Inc., Lincolnshire, Illinois. The funding agencies were not involved in study design, execution, data collection, interpretation and dissemination.


\appendix
\section{Appendix}
\label{appendix:A}
\subsection{Proof of Theorem \ref{thm:1}}
\begin{proof}
Consider the simplest problem where we penalize a rank-1 CP decomposition for a three-way tensor $\eX: m\times n \times k$. Denote the factor matrices as $\ba\in \mathbb{R}^{m}$, $\bb\in \mathbb{R}^{n}$ and $\bc\in \mathbb{R}^{k}$. 
By the relationship between arithmetic mean and geometric mean, we have:
\begin{equation}\label{eq:10}
    \big\|\ba\big\|_F^2 + \big\|\bb\big\|_F^2 + \big\|\bc\big\|_F^2 \geq 3 \left(\big\|\ba\big\|_F  \big\|\bb\big\|_F  \big\|\bc\big\|_F\right)^{2/3}
\end{equation}
Equity is obtained if and only if $\|\ba\|_F = \|\bb\|_F = \|\bc\|_F$.
We claim that: 
\begin{equation}\label{eq:11}
\begin{aligned}
    &\text{min}\big\|\eX-[\![\ba, \bb,\bc]\!] \big\|_F^2  +\sigma\left(\big\|\ba\big\|_F^2 + \big\|\bb\big\|_F^2 + \big\|\bc\big\|_F^2\right)\\
    = \operatorname*{min}&_{\|\ba\|_F = \|\bb\|_F = \|\bc\|_F} \big\|\eX-[\![\ba, \bb,\bc ]\!] \big\|_F^2  + 3\sigma \left(\big\|\ba\big\|_F  \big\|\bb\big\|_F  \big\|\bc\big\|_F\right)^{2/3}
\end{aligned}
\end{equation}
If the solution to the left-hand side of \ref{eq:11} does not satisfy the condition: $\|\ba\|_F = \|\bb\|_F = \|\bc\|_F$, we can rescale the factor matrices such that the condition holds but the Frobenius norm of residual stays unchanged. Thus, this solution is not the argument of minimal according to \ref{eq:10}. By contradiction, the condition must be satisfied and equation \ref{eq:11} will hold.
Hence, the problem is equivalent to the constraint problem:
\begin{equation*}
\begin{aligned}
    \operatorname*{min}&_{\|\ba\|_F = \|\bb\|_F = \|\bc\|_F} \big\|\eX-[\![\ba, \bb,\bc ]\!] \big\|_F^2  +3\sigma \left(\big\|\ba\big\|_F  \big\|\bb\big\|_F  \big\|\bc\big\|_F\right)^{2/3}\\
    &=\text{min} \big\|\eX-\lambda \tilde\ba \circ \tilde\bb \circ \tilde\bc \big\|_F^2  +3\sigma \lambda^{2/3}, 
\end{aligned}
\end{equation*}
where $\tilde\ba={\ba}/{\|\ba\|_F}$, $\tilde\bb={\bb}/{\|\bb\|_F}$ and $\tilde\bc={\bc}/{\|\bc\|_F}$. The $L_{2/3}$ penalty on the factor $\lambda$ may shrink this parameter estimate to 0. 

For a rank $R$ CP decomposition, the result is similar:
\begin{equation*}
\begin{aligned}
    &\text{min}\big\|\eX-[\![\cA, \cB,\cC ]\!] \big\|_F^2  +\sigma\left(\big\|\cA\big\|_F^2 + \big\|\cB\big\|_F^2 + \big\|\cC\big\|_F^2\right)\\
    &=\operatorname*{min}_{\| \ba_i\|_F = \| \bb_i\|_F = \| \bc_i\|_F} \big\|\eX-[\![\cA, \cB,\cC ]\!] \big\|_F^2  +3\sigma \sum_{i=1}^R \left(\big\| \ba_i\big\|_F  \big\| \bb_i\big\|_F  \big\| \bc_i\big\|_F\right)^{2/3}\\
    &=\text{min} \big\|\eX-\sum_{i=1}^R\lambda_i  \tilde\ba_i \circ \tilde\bb_i \circ \tilde\bc_i \big\|_F^2  +3\sigma \sum_{i=1}^R\lambda_i^{2/3},\\
\end{aligned}
\end{equation*}
where $\ba_i$, $\bb_i$ and $\bc_i$ are the ith columns of factor matrices $\cA$, $\cB$ and $\cC$ respectively. We rescale the factor matrices such that each column has unit Frobenius norm. $\tilde\ba_i$, $\tilde\bb_i$ and $\tilde\bc_i$ are the $i$-th columns of the rescaled factor matrices. Same as in the rank-1 case, the $L_{2/3}$ penalty yields rank sparsity. This conclusion can be easily generalized to multi-way tensor factorization. An equivalent form of the $L_2$ penalized problem can be written as a $L_{2/n}$ penalized problem:
$$
\text{min} \big\|\eX-[\![\bm \lambda; \tilde\cA_{1},\ldots,\tilde\cA_n ]\!] \big\|_F^2  +N\sigma \|\bm \lambda\|_{2/N}^{2/N}.
$$
Since $2/n<1$, the penalty term on the factor vector $\bm \lambda$ yields sparsity in $\bm \lambda$, thus it lead to rank sparsity of the tensor.
\end{proof}

\subsection{Proof of Theorem \ref{thm:2}}
\begin{proof}
In order to prove the solutions of factor matrices are the same for two minimization problems, we first simplify the penalized problem by ignoring terms that are unrelated to $(\lambda, \ba_1, \ba_2, \ldots, \ba_N)$:
\begin{equation*}
    \begin{aligned}
        &\text{min }\big \| \eX - [\![ \lambda; \ba_1, \ba_2, \ldots, \ba_N]\!] \big \|_F^2 + N\sigma \|\lambda\|_{2/N}^{2/N}\\
        =\ & \text{min } tr[(\eX_{(1)}-\lambda \ba_1(\ba_N\odot \ba_{N-1}\odot \cdots \odot \ba_2)^\top)^\top(\eX_{(1)}-\lambda \ba_1(\ba_N\odot \ba_{N-1}\odot \cdots \odot \ba_2)^\top)] + N\sigma \lambda^{2/N}\\
        =\ & \text{min } \lambda^2 tr((\ba_N\odot  \cdots \odot \ba_2)\ba_1^\top \ba_1(\ba_N\odot  \cdots \odot \ba_2)^\top)- 2\lambda tr(\eX_{(1)}^\top \ba_1(\ba_N\odot  \cdots \odot \ba_2)^\top)
        + N\sigma \lambda^{2/N}\\
        =\ & \text{min }\lambda^2 tr((\ba_N\odot  \cdots \odot \ba_2)^\top(\ba_N\odot  \cdots \odot \ba_2)\ba_1^\top \ba_1) - 2\lambda tr(\eX_{(1)}^\top \ba_1(\ba_N\odot  \cdots \odot \ba_2)^\top) + N\sigma \lambda^{2/N}\\
        =\ & \text{min }\lambda^2-2\lambda tr(\eX_{(1)}^\top \ba_1(\ba_N\odot  \cdots \odot \ba_2)^\top)+N\sigma \lambda^{2/N}
    \end{aligned}
\end{equation*}
Note that the unpenalized problem can be simplified in the same way as:
\begin{equation*}
    \begin{aligned}
        & \text{min }\big \| \eX - [\![ \lambda; \ba_1, \ba_2, \ldots, \ba_N]\!] \big \|_F^2\\
        =\ & \text{min }tr[(\eX_{(1)}-\lambda \ba_1(\ba_N\odot  \cdots \odot \ba_2)^\top)^\top(\eX_{(1)}-\lambda \ba_1(\ba_N\odot  \cdots \odot \ba_2)^\top)]\\
        =\ & \text{min }\lambda^2 -2\lambda tr(\eX_{(1)}^\top \ba_1(\ba_N\odot  \cdots \odot \ba_2)^\top) 
    \end{aligned}
\end{equation*}
Since both problems have only one term related to $(\ba_1, \ba_2, \ldots, \ba_N)$: $-2\lambda tr(\eX_{(1)}^\top \ba_1(\ba_N\odot  \cdots \odot \ba_2)^\top) $ and $\lambda \geq 0$, their solution to $(\ba_1, \ba_2, \ldots, \ba_N)$ should be the same. When given the solution $(\hat{\ba}_1, \hat{\ba}_2, \ldots, \hat{\ba}_N)$,
$$
\operatorname*{min}_{\lambda}\big \| \eX - [\![ \lambda; \hat{\ba}_1, \hat{\ba}_2, \ldots, \hat{\ba}_N]\!] \big \|_F^2 =\operatorname*{min}_{\lambda} (\lambda-tr(\eX_{(1)}^\top \hat{\ba}_1(\hat{\ba}_N\odot  \cdots \odot \hat{\ba}_2)^\top))^2 - tr(\eX_{(1)}^\top \hat{\ba}_1(\hat{\ba}_N\odot  \cdots \odot \hat{\ba}_2)^\top) ^2,
$$
meaning that the objective function is minimized at $tr(\eX_{(1)}^\top \hat{\ba}_1(\hat{\ba}_N\odot  \cdots \odot \hat{\ba}_2)^\top)= \hat{\lambda}$.
For the penalized problem we can have 
$$
\hat{\lambda}_p = \operatorname*{argmin}_{\lambda} \big \| \eX - [\![ \lambda; \hat{\ba}_1, \hat{\ba}_2, \ldots, \hat{\ba}_N]\!] \big \|_F^2 + N\sigma \|\lambda\|_{2/N}^{2/N} = \operatorname*{argmin}_{\lambda} (\lambda - \hat{\lambda})^2 + N\sigma \lambda^{2/N}.
$$
\end{proof}

\subsection{Proof of Theorem \ref{thm:3}}
\begin{proof}
Denote the $r$-th column of shared factor matrix $\cA_0$ as $\ba_{0r}$ and the the $r$-th column of remaining factor matrices $\cA_{i}^{(k)}$, as $\ba_{ir}^{(k)}$, for $i=1,\ldots, N_k$. Similar to the idea in Theorem 1, we can have 
\begin{align*}
    &\big\|\cA_0\big\|_F^2 + \sum_{k=1}^K\sum_{i=1}^{N_k} \big\|\cA_i^{(k)}\big\|_F^2\\
    = \ & \sum_{r=1}^R \|\ba_{0r}\|_F^2 + \sum_{k=1}^K\sum_{r=1}^R\sum_{i=1}^{N_k} \| \ba_{ir}^{(k)}\|_F^2\\
    \geq\  & \sum_{r=1}^R \|\ba_{0r}\|_F^2 + \sum_{k=1}^K\sum_{r=1}^R N_k \big(\prod_{k=1}^{N_k}\|\ba_{ir}^{(k)}\|_F\big)^{2/N_k}\\
    =\ & \|\bm \lambda_0\|_2^2 + \sum_{k=1}^K N_k\|\bm\lambda_{(0)i} \|_{2/N_k}^{2/N_k}
\end{align*}

Equality holds if and only if the condition $\bm \ba_{ir}^{(1)} = \ba_{ir}^{(2)} = \cdots =\ba_{ir}^{(K)}$ is satisfied for $i=1,\ldots,K$ and $r=1,\ldots,R$. These conditions can be easily satisfied by simply scaling columns of the factor matrices $\ba_{ir}^{(k)}$.
\end{proof}






\documentclass[conference, 9pt]{IEEEtran}

\usepackage[normalem]{ulem}
\usepackage{amsmath}
\usepackage{amsfonts}
\usepackage{hyperref}
\usepackage{graphicx}
\usepackage{cleveref}
\usepackage{amsthm}
\usepackage{algorithm, algpseudocode,subcaption}
\usepackage{xcolor}
\usepackage{tikz}
\usepackage{caption}
\usepackage{listings}
\usepackage{array}
\usepackage{booktabs}
\usepackage{diagbox}
\usepackage{float}
\usepackage{geometry}
\geometry{a4paper,margin=0.5in}
\newtheorem{condition}{Condition}
\newtheorem{claim}{Claim}
\newtheorem{example}{Example} 
\newtheorem{theorem}{Theorem}
\newtheorem{lemma}{Lemma} 
\newtheorem{proposition}{Proposition} 
\newtheorem{remark}{Remark}
\newtheorem{corollary}{Corollary}
\newtheorem{definition}{Definition}
\newtheorem{conjecture}{Conjecture}
\newtheorem{axiom}{Axiom}
\title{Robust Anomaly Detection via Tensor Pseudoskeleton Decomposition}
\author{Bowen Su }
\definecolor{officegreen}{rgb}{0.0, 0.5, 0.0}
\lstset{
    language=Python,
    basicstyle=\ttfamily\small,
    commentstyle=\color{blue},
    keywordstyle=\color{black},
    showstringspaces=false,
    numbers=left,
    numberstyle=\tiny,
    stepnumber=1,
    numbersep=5pt,
}



\begin{document}

\maketitle

\begin{abstract}
Anomaly detection plays a critical role in modern data-driven applications, from identifying fraudulent transactions and safeguarding network infrastructure to monitoring sensor systems for irregular patterns. Traditional approaches—such as distance-, density-, or cluster-based methods, face significant challenges when applied to high-dimensional tensor data, where complex interdependencies across dimensions amplify noise and computational complexity. To address these limitations, this paper leverages Tensor  pseudoskeleton decomposition within a tensor-robust principal component analysis framework to extract low-Tucker-rank structure while isolating sparse anomalies, ensuring robustness to anomaly detection. We establish theoretical analysis of convergence, and estimation error, demonstrating the stability and accuracy of the proposed approach. Numerical experiments on real-world spatiotemporal data from New York City taxi trip records validate the superiority of the proposed method in detecting anomalous urban events compared to existing benchmark methods. The results underscore the potential of Tensor  pseudoskeleton decomposition to enhance anomaly detection for large-scale, high-dimensional data.
\end{abstract}
\section{Introduction}
Anomaly detection is a crucial task in data analysis, with applications spanning various domains such as fraud detection~\cite{motie2024financial}, cybersecurity~\cite{wurzenberger2024analysis}, healthcare monitoring~\cite{kadir2024anomaly}, and sensor network analysis~\cite{tarish2025anomaly}. Anomalies, or outliers, represent data points or patterns that deviate significantly from the expected behavior, often signaling critical events or errors that require immediate attention. Detecting these anomalies, especially within high-dimensional and complex datasets, is challenging due to the sheer volume of data and the underlying noise that can mask unusual patterns.

Traditional anomaly detection techniques, including distance-based~\cite{Angiulli2002}, density-based~\cite{Breunig2000}, and clustering-based methods~\cite{Jiang2003,Hautamaki2004}, have shown some success in identifying anomalies in lower-dimensional datasets.
 However, these approaches often struggle when extended to high-dimensional tensor data, where intricate dependencies exist across multiple dimensions. Tensor data structures are common in fields such as video surveillance, biomedical imaging, and environmental monitoring, where data is naturally organized in multi-way arrays. The increased dimensionality not only complicates the detection of anomalies but also amplifies the computational costs, making scalability a critical concern.

In recent years, tensor decomposition methods have emerged as powerful tools for managing high-dimensional data. By transforming complex data into a lower-dimensional, interpretable form, tensor decompositions facilitate efficient storage, processing, and analysis. Among these methods, Tucker decomposition, a form of higher-order singular value decomposition, is particularly effective at capturing the core structure of tensor data. However, while Tucker decomposition enables significant dimensionality reduction, it remains sensitive to outliers, which can distort the decomposition and lead to unreliable results in anomaly detection.

To address these limitations,  Tensor  pseudoskeleton decomposition offers an alternative approach by selecting representative parts of the data, thereby preserving essential features while reducing redundancy.  Tucker pseudoskeleton decomposition provides a structured decomposition that is both computationally efficient and robust~\cite{hamm2023generalized,cai2021mode}. 

In this paper, we focus on anomaly detection within the tensor robust principal component analysis framework by leveraging a Tucker pseudoskeleton decomposition specifically tailored for high-dimensional datasets~\cite{hamm2023generalized,cai2021mode}. By incorporating sparsity and regularization constraints, our method reduces sensitivity to anomalies, enabling more accurate and resilient detection of unusual patterns. The Tucker  pseudoskeleton decomposition framework combines the strengths of Tucker decomposition’s structural insight with pseudoskeleton’s selective feature extraction while enhancing robustness against outliers~\cite{hamm2023generalized}.


\subsection{Notations and definitions}
In this section, we introduce notation and review foundational properties of Tucker-based tensor decomposition, which will be essential throughout the chapter. Tucker decomposition serves as a powerful tool for capturing the core structure of high-dimensional data, providing both a compact representation and interpretability of multi-dimensional relationships within the data.

To distinguish between different mathematical entities, we adopt the following conventions: calligraphic capital letters (e.g., $\mathcal{T}$) represent tensors, regular uppercase letters (e.g., ${X}$) denote matrices, regular lowercase letters (e.g., ${x}$) indicate vectors or scalars. For submatrices, $[X]_{I,:}$ and $[X]_{:,J}$ refer to the rows and columns of matrix ${X}$ indexed by sets $I$ and $J$, respectively. For tensors, $[\mathcal{T}]_{I_1, \dots, I_n}$ represents a subtensor of $\mathcal{T}$ with index sets $I_k$ along each mode $k$. A specific element in a tensor is accessed by the index notation $[\mathcal{T}]_{i_1, \dots, i_n}$. 

The tensor norm used in this chapter is the Frobenius norm~\cite{kolda2009tensor}, defined for a tensor \(\mathcal{T}\) as:
\begin{equation*}
    \|\mathcal{T}\|_\mathrm{F} = \sqrt{\sum_{i_1, \dots, i_n} [\mathcal{T}]_{i_1, \dots, i_n}^2}.
\end{equation*}
This norm represents the square root of the sum of the squared entries of \(\mathcal{T}\), extending the Frobenius norm from matrices to higher-order tensors.
For matrices, the Moore-Penrose Pseudoinverse is denoted by ${X}^\dagger$. The notation $[d] := \{1, \dots, d\}$ represents the set of natural numbers up to $d$.

\begin{definition}[\textbf{Tensor Matricization/Unfolding}~\cite{kolda2009tensor}]
    An $n$-mode tensor $\mathcal{T}$ can be reshaped into a matrix by unfolding it along each of its $n$ modes. The mode-$k$ unfolding of a tensor $\mathcal{T} \in \mathbb{R}^{d_1 \times \dots \times d_n}$, denoted $\mathcal{T}_{(k)}$, is a matrix of size $\mathbb{R}^{d_k \times \prod_{j \neq k} d_j}$, obtained by arranging all vectors of $\mathcal{T}$ with indices fixed in all modes except the $k$-th. This transformation, $\mathcal{T} \mapsto \mathcal{T}_{(k)}$, is referred to as the mode-$k$ unfolding operator.
\end{definition}

\begin{definition}[\textbf{Mode-$k$ Product}~\cite{kolda2009tensor}]
    Let $\mathcal{T} \in \mathbb{R}^{d_1 \times \dots \times d_n}$ and ${A} \in \mathbb{R}^{J \times d_k}$. The mode-$k$ product of $\mathcal{T}$ with ${A}$, denoted by $\mathcal{Y} = \mathcal{T} \times_k {A}$, is defined element-wise as:
    \begin{equation*}
        [\mathcal{Y}]_{i_1, \dots, i_{k-1}, j, i_{k+1}, \dots, i_n} = \sum_{s=1}^{d_k} [\mathcal{T}]_{i_1, \dots, i_{k-1}, s, i_{k+1}, \dots, i_n} [{A}]_{j, s}.
    \end{equation*}
    Alternatively, this operation can be represented in matrix form as $\mathcal{Y}_{(k)} = {A} \mathcal{T}_{(k)}$. For a sequence of tensor-matrix products across different modes, we use the notation $\mathcal{T} \times_{i=t}^{s} {A}_i$ to indicate the product $\mathcal{T} \times_{t} {A}_{t} \times_{t+1} \dots \times_{s} {A}_{s}$. This operation is referred to as the `tensor-matrix product' throughout the paper.
\end{definition}
\begin{definition}[\textbf{Tucker Rank and Tucker Decomposition}~\cite{kolda2009tensor}]
    The Tucker decomposition of a tensor $\mathcal{T}$ approximates it by expressing it as a product of a core tensor $\mathcal{C}$ and factor matrices ${A}_k$ along each mode:
    \begin{equation*}\label{eqn:Tucker_Decomposition}
        \mathcal{T} \approx \mathcal{C} \times_{i=1}^n {A}_i.
    \end{equation*}
    If the approximation in \eqref{eqn:Tucker_Decomposition} becomes an equality and the core tensor $\mathcal{C} \in \mathbb{R}^{r_1 \times \dots \times r_n}$, this is termed an exact Tucker decomposition of $\mathcal{T}$. The ranks $(r_1, \dots, r_n)$ are known as the Tucker ranks of the tensor $\mathcal{T}$.
\end{definition}

%\begin{remark}
%    Tucker decomposition can be viewed as a generalization of matrix singular value decomposition (SVD) to higher dimensions, preserving essential structure while reducing dimensionality. The HOSVD~\cite{de2000best} is a specific orthogonal form of Tucker decomposition commonly used in applications.
%\end{remark}
In the realm of matrix algebra, the pseudoskeleton decomposition technique is a good alternative to SVD~\cite{Goreinov}. Specifically, this method entails selecting specific columns ${C}$ and rows ${R}$ from a matrix ${X} \in \mathbb{R}^{d_1 \times d_2}$, and constructing a core matrix ${U} = {X}(I, J)$. The matrix ${X}$ is then reconstructed through the product ${C} {U}^\dagger {R}$, under the condition that $\operatorname{rank}({U}) = \operatorname{rank}({X})$. Expanding from matrices to tensors, the initial adaptations of pseudoskeleton decompositions applied a single-mode unfolding to 3-mode tensors~\cite{mahoney2008tensor}. To my best knowledge, the following are recent works on tensor pseudoskeleton decompositions or Tensor CUR Decompostions~\cite{hamm2023generalized,cai2021mode,cai2023robust,ahmadi2022cross,caiafa2010generalizing}. Furthermore, H. Cai, K. Hamm, and etc have presented rigorous theoretical results on the exact tensor pseudoskeleton decomposition \cite{cai2021mode,cai2023robust,hamm2023generalized}. For completeness, we present their work below.
\begin{definition}[Tensor pseudoskeleton decompositions or Tensor CUR Decompostions~\cite{hamm2023generalized,cai2021mode,cai2023robust}]
 Consider a tensor \(\mathcal{A} \in \mathbb{R}^{d_1 \times \cdots \times d_n}\) with Tucker ranks \((r_1, \dots, r_n)\). Suppose that for each mode \(i\), there exists a subset \(I_i \subseteq [d_i]\) such that 
 \[
\mathcal{A} = \mathcal{R} \times_{i=1}^{n} \left( C_i U_i^\dagger \right),
\]
where \(\mathcal{R} = [\mathcal{A}]_{I_1, \dots, I_n}, \quad C_i = [\mathcal{A}_{(i)}]_{:, J_i}, \quad \text{and} \quad U_i = [\mathcal{C}_{(i)}]_{I_i, :}, \quad J_i = \bigotimes_{j \neq i} I_j.
\)
\end{definition}
\begin{theorem}[{~\cite{hamm2023generalized,cai2021mode,cai2023robust}}]\label{thm:  Tucker Decomposition}
    For a tensor $\mathcal{A} \in \mathbb{R}^{d_1 \times \cdots \times d_n}$ with Tucker ranks $(r_1, \dots, r_n)$, consider subsets $I_i \subseteq [d_i]$ and let $J_i = \bigotimes_{j \neq i} I_j$ for each mode $i$. Define $\mathcal{R} = [\mathcal{A}]_{I_1, \dots, I_n}$, ${C}_i = [\mathcal{A}_{(i)}]_{:, J_i}$, and ${U}_i = [\mathcal{C}_{(i)}]_{I_i, :}$. The following conditions are equivalent:
    \begin{enumerate}
        \item $\mathcal{A} = \mathcal{R} \times_{i=1}^{n} ({C}_{i} {U}_i^\dagger)$,
        \item $\operatorname{rank}({U}_i) = r_i$ for all $i$,
        \item $\operatorname{rank}({C}_i) = r_i$ for all $i$, and $\mathcal{R}$ has Tucker rank $(r_1, \dots, r_n)$.
    \end{enumerate}
\end{theorem}
 For those interested in further details, It is recommended to read works~\cite{hamm2023generalized,cai2021mode,ahmadi2022cross,caiafa2010generalizing,cai2023robust,che2022perturbations,saibaba2016hoid}.



\section{Methodology}
We employ Tensor Robust Principal Component Analysis, an extension of classical Robust PCA that can operate directly on multi-dimensional (tensor) data. Unlike conventional low-rank models that assume the entire dataset is low-rank, TRPCA decomposes a given tensor into two distinct components: a low-rank component representing regular patterns and a sparse component isolating anomalies. This decomposition effectively isolates outliers in spatial-temporal data while retaining core structural patterns, providing a more flexible and robust approach to anomaly detection. By handling high-dimensional tensor data, TRPCA is particularly well-suited for scenarios where data is naturally structured as a multi-way array, allowing for the detection of unusual patterns that vary across both space and time.

In this framework, we represent the spatial-temporal data as a tensor $\mathcal{T} \in \mathbb{R}^{d_1 \times d_2 \times \cdots \times d_n}$, where each dimension $d_i$ corresponds to a specific mode of the data. For example, $d_1$ might represent spatial coordinates, $d_2$ temporal intervals, and additional dimensions might capture contextual features or sensor types. The objective is to decompose $\mathcal{T}$ into two components: a low-rank tensor $\mathcal{L}^\star$ that captures the dominant spatial-temporal structure, and a sparse tensor $\mathcal{S}^\star$ representing anomalies or outliers. The decomposition is expressed as:
\begin{equation*}
    \mathcal{T} = \mathcal{L}^\star + \mathcal{S}^\star,
\end{equation*}
where $\mathcal{L}^\star \in \mathbb{R}^{d_1 \times \cdots \times d_n}$ encapsulates the smooth, regular patterns in the data, while $\mathcal{S}^\star \in \mathbb{R}^{d_1 \times \cdots \times d_n}$ captures deviations from these patterns, isolating events that significantly differ from expected behavior. This separation allows for robust anomaly detection, as $\mathcal{S}^\star$ can pinpoint localized irregularities without interference from the regular structure. Mathematically, we formulate the anomaly detection problem as an optimization problem that seeks to minimize the reconstruction error between $\mathcal{T}$ and the sum of $\mathcal{L}$ and $\mathcal{S}$. This is achieved through the following objective:
\begin{equation*}\label{eq:trpca_formulation}
    \begin{split}
        \min_{\mathcal{R}, {C}_{i}, {U}_i, \mathcal{S}} & \quad \|\mathcal{T} - \mathcal{L} - \mathcal{S}\|_\mathrm{F} \\
        \text{subject to} & \quad \mathcal{L}= { \mathcal{R} \times_{i=1}^{n} ({C}_{i} {U}_i^\dagger)}\\
        &\quad\|\mathcal{S}\|_{\infty} \leq \alpha.
    \end{split}
\end{equation*}






\subsection{TRPCA via Tensor 
Pseudoskeleton Decomposition}

\begin{algorithm}[H]
    \caption{TRPCA via Tensor 
Pseudoskeleton Decomposition}
    \label{alg:TCPD}
    \begin{algorithmic}[1]
        \State \textbf{Input: } $\mathcal{T}  \in \mathbb{R}^{d_1 \times \cdots \times d_n}$: observed tensor; 
        $(r_1, \cdots, r_n)$: estimated Tucker rank; 
        $\varepsilon$: targeted precision; 
        $\zeta^{(0)}, \gamma$: thresholding parameters; $\{|I_i|\}_{i=1}^n,\{|J_i|\}_{i=1}^n$: cardinalities for sample indices.
        \State  Uniformly sample the indices $\{I_i\}_{i=1}^n, \{J_i\}_{i=1}^n$ 
        \State \textbf{Initialization:} $\mathcal{L}^{(0)} = 0, \mathcal{S}^{(0)} = 0, k = 0$
        \While {$e^{(k)} > \varepsilon$}
            \State \textcolor{officegreen}{// Step (I): Updating $\mathcal{S}$}
            \State $\zeta^{(k+1)} = \gamma \cdot \zeta^{(k)}$ 
            \State $\mathcal{S}^{(k+1)} = \mathrm{HT}_{\zeta^{(k+1)}}(\mathcal{T} - \mathcal{L}^{(k)})$  
            \State \textcolor{officegreen}{// Step (II): Updating $\mathcal{L}$}
            \State $\mathcal{L}^{(k+1)} = [\mathcal{T} - \mathcal{S}^{(k+1)}]_{I_1, \cdots, I_n}$
            \For{$i = 1, \cdots, n$}
                \State $C_i^{(k+1)} = [(\mathcal{T} - \mathcal{S}^{(k+1)})_{(i)}]_{:, J_i}$ 
                %\State $U_i^{(k+1)} = \mathcal{H}_{r_i}([C_i^{(k+1)}]_{I_i, :})$ 
                \State $[Q,R] = \operatorname{qr}\left([C_i^{(k+1)}]_{I_i, :}^{\top}\right)$

             \State$\mathcal{L}^{(k+1)} = \mathcal{L}^{(k+1)} \times C_i^{(k+1)}[Q]_{:,:r}[R]_{:r,:}^{\dagger}$    
            \EndFor
            
            \State $k = k + 1$ 
        \EndWhile
        \State \textbf{Output: } $\mathcal{L}^{(k+1)}, \mathcal{S}^{(k+1)}$.
    \end{algorithmic}
\end{algorithm}

\subsubsection{Step~(I): Update Sparse Component $\mathcal{S}$} 
\label{sec:updateS}
In this step, we update the sparse component \(\mathcal{S}\) — which captures data outliers — using the technique described in \cite{cai2023robust,netrapalli2014non,cai2019accelerated}. Specifically, we apply an iterative decaying threshold within the hard thresholding operator \(\mathrm{HT}_\zeta\) paired with \(\gamma\), as described in \cite{cai2023robust,CaiR2024}.
The hard thresholding operator $\mathrm{HT}_\zeta$ is defined as follows:

\begin{equation*}
    [\mathrm{HT}_{\zeta}(\mathcal{T})]_{i_1,\cdots,i_n} =
    \begin{cases}
        [\mathcal{T}]_{i_1,\cdots,i_n}, & \quad |[\mathcal{T}]_{i_1,\cdots,i_n}| > \zeta; \\
        0,  & \quad \text{otherwise.}
    \end{cases}
\end{equation*}

This operator $\mathrm{HT}_\zeta$ effectively filters out entries with magnitudes less than or equal to $\zeta$, treating them as negligible. By applying this to the tensor $\mathcal{T}$, only values deemed significant (i.e., values exceeding the threshold) remain in the updated sparse component $\mathcal{S}$, thereby enhancing the sparsity of $\mathcal{S}$.
\subsubsection{Step~(II): Update Low-Tucker-rank Component $\mathcal{L}$}
\label{sec:updateL}
In this step, we aim to update the low-Tucker-rank component \(\mathcal{L}\), which models the structured, low-rank part of the data tensor via tensor pseudoskeleton decomposition. The update process is divided into two key stages: subspace identification and projective reconstruction.
To approximate the low-rank structure along each mode, we begin by extracting the mode-\(i\) fibers from the residual tensor \(\mathcal{T} - \mathcal{S}^{(k)}\), which represents the current estimate of the sparse component subtracted from the observed data tensor. The fibers are assembled into the matrix representation:
\[
C_i^{(k)} \in \mathbb{R}^{d_i \times |J_i|},
\]
where each column of \(C_i^{(k)}\) corresponds to a mode-\(i\) fiber indexed by a subset of indices \(J_i\). We select a subset of mode-\(i\) fibers indexed by \(I_i \subseteq \{1, \dots, d_i\}\) and perform an economy-size QR decomposition on the transposed submatrix formed by these selected fibers:
\begin{equation*}
    \left[C_i^{(k)}\right]_{I_i,:}^\top = Q R,
\end{equation*}
where \(Q \in \mathbb{R}^{|J_i| \times r_i}\) is a matrix with orthonormal columns representing the estimated basis, and \(R \in \mathbb{R}^{r_i \times |I_i|}\) is an upper triangular matrix. The dimension \(r_i\) is the estimated Tucker rank along mode-\(i\). This step yields a low-dimensional orthonormal basis that approximates the column space of the matricized low-rank component along mode-\(i\), i.e., the dominant subspace of \(\mathcal{L}^\star_{(i)}\).
Once the subspace is identified, we project the full set of mode-\(i\) fibers onto this estimated low-rank subspace. This is achieved by updating the mode-\(i\) factor matrix of the Tucker decomposition as follows:
\begin{equation*}
    \mathcal{L}^{(k+1)} \leftarrow \mathcal{L}^{(k+1)} \times_i \left( C_i^{(k)} \left[Q\right]_{:,:r_i} \left[R\right]_{:r_i,:}^\dagger \right).
\end{equation*}
This projection aligns the updated factor matrices along mode-\(i\) with the estimated low-dimensional subspace. Using QR decomposition and projecting onto the selected subspace, the computational complexity for each mode is reduced from the cubic cost \(\mathcal{O}(d_i^3)\) to the more efficient:
\(
\mathcal{O}(d_i r_i^2 + r_i^3),
\)
where \(d_i\) is the dimension along mode-\(i\), and \(r_i\) is the target Tucker rank. This reduction is particularly beneficial when the Tucker rank \(r_i\) is significantly smaller than the mode dimension \(d_i.\)



\section{Theoretical Foundations}\label{sec:theory}

\begin{theorem}\label{thm:subspace}
Let $\mathcal{L}^\star \in \mathbb{R}^{d_1 \times \cdots \times d_n}$ be a rank-$(r_1,\ldots,r_n)$ Tucker tensor with factor matrices $\mathbf{U}_i \in \mathbb{R}^{d_i \times r_i}$ satisfying the $\mu$-incoherence condition:
\begin{equation*}
\max_{1 \leq j \leq d_i} \|\mathbf{U}_i(j,:)\|_2 \leq \sqrt{\frac{\mu r_i}{d_i}}, \quad \forall i \in [n].
\end{equation*}
For any mode $i$ and failure probability $\delta \in (0,1)$, if we sample row indices $I_i \subseteq [d_i]$ with cardinality 
\begin{equation*}
|I_i| \geq c_0 \mu r_i \log^3\left(\frac{\mu r_i}{\delta}\right),
\end{equation*}
then with probability at least $1-\delta$, the sampled factor matrix satisfies
\begin{equation*}
\frac{1}{2}\sqrt{\frac{|I_i|}{d_i}} \leq \sigma_{\min}\left(\mathbf{U}_i(I_i,:)\right) \leq \sigma_{\max}\left(\mathbf{U}_i(I_i,:)\right) \leq \frac{3}{2}\sqrt{\frac{|I_i|}{d_i}},
\end{equation*}
where $c_0 > 0$ is an absolute constant and $\sigma_{\min}(\cdot)$, $\sigma_{\max}(\cdot)$ denote extremal singular values.
\end{theorem}
\begin{proof}
Define the normalized sampling matrix $\mathbf{\Phi}_i = \sqrt{\frac{d_i}{|I_i|}}\mathbf{S}_i$ where $\mathbf{S}_i \in \{0,1\}^{|I_i|\times d_i}$ has exactly one 1 per row. The subsampled matrix becomes:
\[
\widetilde{\mathbf{U}}_i = \mathbf{\Phi}_i\mathbf{U}_i \in \mathbb{R}^{|I_i|\times r_i}.
\]
Applying the matrix Bernstein inequality \cite{tropp2015introduction} to $\mathbf{U}_i\mathbf{U}_i^\top$:
\[
\mathbb{P}\left(\left\|\widetilde{\mathbf{U}}_i\widetilde{\mathbf{U}}_i^\top - \mathbf{I}\right\|_2 \geq t\right) \leq 2r_i \exp\left(-\frac{t^2|I_i|}{C\mu r_i \log d_i}\right).
\]

Setting $t = 1/2$ and solving for $|I_i|$:
\[
|I_i| \geq C\mu r_i \log^3\left(\frac{\mu r_i}{\delta}\right) \implies \frac{1}{2}\mathbf{I} \preceq \widetilde{\mathbf{U}}_i\widetilde{\mathbf{U}}_i^\top \preceq \frac{3}{2}\mathbf{I}.
\]

Notice that
\[
\sigma_{\min}^2(\mathbf{U}_i(I_i,:)) = \frac{d_i}{|I_i|}\sigma_{\min}^2(\widetilde{\mathbf{U}}_i) \geq \frac{d_i}{2|I_i|}.
\]
Similarly for $\sigma_{\max}$. Rearrangement completes the proof.
\end{proof}
\begin{theorem}\label{thm:convergence}
Under the conditions of Theorem \ref{thm:subspace} and assuming $\|\mathcal{S}^\star\|_\infty \leq \frac{\zeta^{(0)}}{2\sqrt{\log d_{\max}}}$, the iterates satisfy:
\begin{equation*}
\|\mathcal{L}^{(k+1)} - \mathcal{L}^\star\|_F \leq \rho\|\mathcal{L}^{(k)} - \mathcal{L}^\star\|_F + C\sqrt{\frac{\log d_{\max}}{|I|}}\|\mathcal{S}^\star\|_\infty,
\end{equation*}
where the contraction factor \[\rho = \max_{1 \leq i \leq n} \left(1 - \frac{\sigma_{\min}^2(\mathbf{U}_i(I_i,:))}{2}\right) < 1\] and $|I| = \min\limits_i |I_i|$.
\end{theorem}

\begin{proof}
Define the errors:
\[
\Delta^{(k)} := \mathcal{L}^{(k)} - \mathcal{L}^\star, \quad \mathcal{E}^{(k)} := \mathcal{S}^{(k)} - \mathcal{S}^\star
\]
The update rule induces coupled dynamics:
\[
\Delta^{(k+1)} = \underbrace{\sum_{i=1}^n (\mathcal{P}_{\mathbf{Q}_i^{(k)}} - \mathcal{P}_{\mathbf{U}_i})\Delta^{(k)}}_{\text{Projection error}} + \underbrace{\mathcal{B}^{(k)}\mathcal{E}^{(k)}}_{\text{Sparsity propagation}}
\]
where $\mathcal{B}^{(k)}$ represents the multi-modal projection of residual errors.
From the hard thresholding operation and incoherence condition:
\begin{align}
\|\mathcal{E}^{(k)}\|_1 &\leq \gamma\|\mathcal{E}^{(k-1)}\|_1 + C_1\|\Delta^{(k)}\|_F \\
&\leq \gamma^k\|\mathcal{E}^{(0)}\|_1 + C_1\sum_{m=0}^{k-1}\gamma^{k-m-1}\|\Delta^{(m)}\|_F
\end{align}
Under the sparsity condition $\|\mathcal{S}^\star\|_\infty \leq \frac{\zeta^{(0)}}{2\sqrt{\log d_{\max}}}$:
\[
\|\mathcal{B}^{(k)}\mathcal{E}^{(k)}\|_F \leq C_2\sqrt{\log d_{\max}}\|\mathcal{S}^\star\|_\infty
\]
Using Wedin's theorem \cite{wedin1972perturbation} and Theorem \ref{thm:subspace}:
\[
\|\mathcal{P}_{\mathbf{Q}_i^{(k)}} - \mathcal{P}_{\mathbf{U}_i}\|_2 \leq C_3\sqrt{\frac{\mu r_i d_i \log d_i}{|I_i|^2}}
\]
Summing over all modes:
\[
\left\|\sum_{i=1}^n (\mathcal{P}_{\mathbf{Q}_i^{(k)}} - \mathcal{P}_{\mathbf{U}_i})\Delta^{(k)}\right\|_F \leq \left(1 - \frac{c}{|I|}\right)\|\Delta^{(k)}\|_F
\]
Combining both components:
\begin{align}
\|\Delta^{(k+1)}\|_F &\leq \left(1 - \frac{c}{|I|}\right)\|\Delta^{(k)}\|_F + C_2\sqrt{\log d_{\max}}\|\mathcal{S}^\star\|_\infty \\
&\leq \rho\|\Delta^{(k)}\|_F + C\sqrt{\frac{\log d_{\max}}{|I|}}\|\mathcal{S}^\star\|_\infty
\end{align}
where $\rho = 1 - \frac{c}{2|I|}$. Solving the recursion completes the proof.
\end{proof}

\begin{lemma}\label{lem:Sparsity}
The projected sparsity term satisfies:
\[
\|\mathcal{B}^{(k)}\mathcal{E}^{(k)}\|_F \leq C\sqrt{\frac{\log d_{\max}}{|I|}}\left(\|\mathcal{E}^{(k)}\|_1 + \|\Delta^{(k)}\|_F\right)
\]
\end{lemma}

\begin{proof}
Decompose the sparsity propagation using the following inequality:
\[
\|\mathcal{B}^{(k)}\mathcal{E}^{(k)}\|_F \leq \|\mathcal{B}^{(k)}\|_F\|\mathcal{E}^{(k)}\|_1
\]
From Theorem \ref{thm:subspace}, the projection operator norm is bounded by:
\[
\|\mathcal{B}^{(k)}\|_F \leq C\sqrt{\frac{\log d_{\max}}{|I|}}
\]
Combining with the threshold error bound completes the proof.
\end{proof}
\begin{theorem}\label{thm:error}
After $K = \mathcal{O}\left(\frac{\log(1/\epsilon)}{\log(1/\rho)}\right)$ iterations, the estimation error decomposes as:
\begin{equation*}
\|\mathcal{L}^{(K)} - \mathcal{L}^\star\|_F \leq \underbrace{C_1\sqrt{\frac{r_{\max}d_{\max}\log d_{\max}}{|I|}}}_{\text{Approximation Error}} + \underbrace{C_2\frac{\|\mathcal{S}^\star\|_\infty}{\sqrt{\log d_{\max}}}}_{\text{Optimization Error}},
\end{equation*}
where $r_{\max} = \max_i r_i$, $d_{\max} = \max_i d_i$, and $C_1, C_2 > 0$ are constants.
\end{theorem}

\begin{proof}
From Theorem \ref{thm:subspace}:
\[
\|\mathcal{L}^{(0)} - \mathcal{L}^\star\|_F \leq C\sqrt{\frac{r_{\max}d_{\max}}{|I|}}.
\]
Applying Theorem \ref{thm:convergence} recursively:
\[
\|\mathcal{L}^{(K)} - \mathcal{L}^\star\|_F \leq \rho^K C\sqrt{\frac{r_{\max}d_{\max}}{|I|}} + \frac{C'\sqrt{\log d_{\max}}}{1-\rho}\|\mathcal{S}^\star\|_\infty.
\]
Setting $\rho^K \leq \sqrt{\frac{\log d_{\max}}{r_{\max}d_{\max}}}$ yields the optimal error decomposition.
\end{proof}

\begin{corollary}[Sample Complexity]\label{cor:sample}
To achieve $\epsilon$-accuracy with $\epsilon < \|\mathcal{S}^\star\|_\infty/\sqrt{\log d_{\max}}$, the required sampling complexity per mode is:
\begin{equation*}
|I_i| \geq C\mu r_i d_i \log^3 d_i \left(\frac{r_{\max}d_{\max}}{\epsilon^2} + \frac{\|\mathcal{S}^\star\|_\infty^2}{\epsilon^2\log d_{\max}}\right).
\end{equation*}
\end{corollary}




\section{Numerical Experiments}
We utilize the NYC yellow taxi trip records from 2018 as a real-world spatiotemporal dataset~\cite{indibi2024spatiotemporal,sofuoglu2022gloss}. This dataset provides a detailed log of each taxi trip, including departure and arrival information (zones and times), the number of passengers, and tip amounts. 

In our experiments, we aggregate the data same as that in ~\cite{indibi2024spatiotemporal} by counting the number of arrivals per zone over hourly intervals. To ensure statistical significance, we restrict our analysis to 81 central zones, which represent high-traffic areas and exclude zones with minimal activity. This selection reduces noise from sparsely populated zones and provides a more robust representation of NYC’s high-demand regions.
With these parameters, we constructed a four-dimensional tensor \( \mathbf{Y} \) with dimensions \( 24 \times 7 \times 53 \times 81 \). The modes of this tensor are defined as follows: the first mode corresponds to the 24 hours of a day; the second mode represents the 7 days of the week; the third mode encompasses the 53 weeks of the year; the fourth mode covers the 81 selected central zones in New York City. Thus, each entry  in the tensor represents the count of taxi arrivals for hour \( h \), day \( d \), week \( w \), and zone \( z \), aggregate over the year. 

We evaluate our anomaly detection approach by identifying the top \(K\%\) of entries with the highest anomaly scores from the extracted sparse tensors, with \(K\) varying across multiple thresholds (0.014, 0.07, 0.14, 0.3, 0.7, 1, 2, and 3). Each top-\(K\%\) subset is then compared to compiled event list to determine how many events are correctly detected. The compiled event list is chosen same as~\cite{sofuoglu2022gloss,indibi2024spatiotemporal}.\Cref{tab:events} compares the number of events detected by our method against five benchmark methods—LR-STSS~\cite{indibi2024spatiotemporal}, LR-TS~\cite{indibi2024spatiotemporal}, LR-SS~\cite{indibi2024spatiotemporal}, and HoRPCA~\cite{li2015low, geng2014high}—across different \(K\%\) thresholds.
 The parameters for our method are set as follows: a maximum of 150 iterations, a tolerance level of \(10^{-7}\), and a Tucker rank of \((26, 6, 4, 10)\). The parameters for the other four methods are adopted from \cite{indibi2024spatiotemporal}. %The RCURC algorithm was developed specifically for a specialized random sampling framework known as the Cross Concentrated Sampling model~\cite{cai2024robust}. Thus, to apply the RCURC algorithm, we first flatten the tensor along its spatial dimension, transforming it into a matrix of size \(81 \times 8904\). Next, we perform robust CCS sampling, where rows and columns are selected based on a uniform observation rate across submatrices and a specified percentage of sampled indices. Finally, the RCURC algorithm is applied to extract the low-rank structure of the flatten matrix. The parameter settings for RCURC are as follows: the maximum number of iterations is fixed at 200, sampling densities (\(\delta\)) range from 0.1 to 0.8, tolerance levels (\(\text{tol}\)) are set to \(1 \times 10^{-5}\), outlier amplification factors (\(c\)) vary from 0.0 to 0.4, and the rank parameter \(r\) is incremented sequentially from 1 to 40. Parameters of all other four methods are same as in~\cite{indibi2024spatiotemporal}.


\begin{table}[H]
\centering
\resizebox{0.5\textwidth}{!}{%
\begin{tabular}{|c|c|c|c|c|c|c|c|c|}
\hline
\%       & 0.014 & 0.07 & 0.14 & 0.3  & 0.7  & 1    & 2    & 3    \\ \hline
Ours    & \textbf{3}     & \textbf{6} & \textbf{10} & \textbf{14} & \textbf{16} & \textbf{18} & \textbf{20} & \textbf{20} \\ \hline
LR-STSS  & \textbf{3} & {4} & {7} & {12} & {15} & {17} & {19} & {19} \\ \hline
LR-TS    & 3     & 4     & 5     & 6     & 13    & 13    & 18    & 19    \\ \hline
LR-SS    & 1     & 1     & 2     & 3     & 5     & 6     & 13    & 16    \\ \hline
%RCURC    & 0     & 0     & 1     & 1     & 5     & 8     & {11}  & {12}  \\ \hline
HoRPCA   & 0     & 0     & 2     & 2     & 2     & 3     & 7     & 10    \\ \hline
\end{tabular}%
}
\caption{Number of detected events among 20 compiled events in NYC for varying top-\(K\%\) of the anomaly scores}
\label{tab:events}
\end{table}
\begin{figure}
    \centering
    \includegraphics[width=1\linewidth]{methods_runtime.png}
    \caption{Running Time}
    \label{fig:time}
\end{figure}
As shown in Table~\ref{tab:events} and \Cref{fig:time}, \Cref{alg:TCPD} not only achieves higher event detection accuracy across various thresholds but also significantly reduces running time compared to LR-STSS, LR-TS, LR-SS, and HoRPCA. This balance of efficiency and effectiveness underscores \Cref{alg:TCPD}’s practical advantages for large-scale or real-time anomaly detection scenarios. This performance affirms the efficacy of our model parameters, including a Tucker rank configuration suited for complex, multi-dimensional datasets.
\section{Conclusion}

In this short paper, we investigate the application of tensor  pseudoskeleton decomposition for anomaly detection in high-traffic areas of New York City. Specifically, we aim to capture temporal and spatial patterns in taxi arrival data. By focusing on central zones with significant activity, the result demonstrates its possibility of  tensor  pseudoskeleton decomposition to remove sparsity and highlight urban regions with high demand. 

\bibliographystyle{IEEEtran}
\bibliography{reference}
\end{document}


\end{document}
\typeout{get arXiv to do 4 passes: Label(s) may have changed. Rerun}