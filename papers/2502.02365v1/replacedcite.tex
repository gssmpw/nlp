\section{Related Work}
\label{sec:related_work}
Tracking the hierarchical mobility of human societies between generations has been a focus of Sociologists and Economists for decades. Traditionally, the way to model a hierarchy is by ranking individual things by some metric. Sociologists tend to focus on occupational ``class" hierarchies____ which are qualitative and subjectively ranked____. One sociological measure of mobility is the ``Log-Multiplicative Layer Effect Model''____ which compares two matrices (called ``layers'' or ``generations'') of class associations by assuming a uniform multiplicative association. This uniform association removes much of the much needed nuance between inter-generational class associations. 

Economists tend to focus on financial income ``bands"____ which are quantifiable and inherently ranked. A widely used mobility measurement used in economics is the Pearson correlation coefficient____
$\beta = r(\ln{Y_c},\ln{Y_p})$
where $r(X,Y)$ is the Pearson correlation between numerical series $X$ and $Y$. This is usually used in conjunction with
$\ln{Y_c} = \alpha + \beta_p\ln{Y_p} + \epsilon_c$
where $Y_c$ is the income of children, and $Y_p$ is the income of parents, $\alpha$ is a constant and $\epsilon_c$ is a fitted constant. 

A more direct approach to analysing hierarchical dynamics was taken in a recent paper____ by observing the dynamics of a variety of rank-ordered lists. The authors found that systems where the top 100 ranks are more ``open" to new inhabitants experience higher hierarchical mobility, as opposed to less open systems where mobility is low. Also, the rest of the list is invariably much less stable than these top 100 ranks.

Temporal networks model the time evolution of pair-wise interactions that are inherent in natural networks. The analysis of such networks is becoming more common____ as the traditional approach of analysing a single static snapshot of networks does not allow for the analysis of network dynamics. Instead, comparing multiple snapshots of a network taken within subsequent time windows allows for us to track the time evolution of important nodes____ and their association with the nodes they connect to____. In this paper we expand on these two ideas by using similar metrics and correlating them between time windows. We compare the outcomes of these correlations across many artificial models and real world networks.

Building artificial networks with particular characteristics using models which apply particular growth rules over time to nodes and edges is a very well trodden field. Of course, there are the two models referenced in the introduction, the Barab\'{a}si-Albert (BA)____ model and the Fortunato____ model which both replicate power-law degree distributions using simple rules. However, complex models are becoming more prevalent, such as models which mimic assortativity in social networks____.

In this paper, we build on the economic hierarchical mobility measurements by removing the need for classes or bands by increasing the resolution to individual people (or nodes). We also incorporate the interactions between people by modelling them as temporal networks and we increase the resolution of the time windows from generations to any ``useful" time-frame. Finally, we use Pearson correlations to measure the time evolution of a network's ``degree hierarchy". The mechanisms for time evolution involve timestamped node and edge additions, and windows of time for which the network is aggregated and analysed in comparison to other time window graphs.

We briefly look at equality in our data through the lens of the Gini coefficient in our results section. Some authors have considered how inequality changes over time by conducting either simulated or real life experiments on networks. In____ the authors create an artificial trading model to study the effect on wealth inequality which demonstrates that the Matthew effect can arise from relatively few experimental assumptions. 
In____ the authors use an experimental economics approach to look at how people redistribute wealth in a social network, reporting initially high inequalities dropping and initially low inequalities rising. It is hard to generalise from their findings to the current work.