\documentclass[10pt]{article}
\usepackage[utf8]{inputenc}
\usepackage[T1]{fontenc}
\usepackage{caption}
\usepackage{graphicx}
\usepackage{biblatex}
\usepackage{xcolor}
\usepackage{colortbl}
\usepackage{tabularx}
\usepackage[left=1.5cm, right=2cm]{geometry}
\newcolumntype{C}[1]{>{\centering\arraybackslash}p{#1}}



\title{Supplementary Information for Measuring social mobility in temporal networks}

\author{Matthew Russell Barnes, Vincenzo Nicosia, Richard G. Clegg}
\bibliography{data_corpus_2023}

\begin{document}
\maketitle
\section{Data Corpus Details}
\label{subsec:data_corpus_intro}
There are many different varieties of systems which can be modelled using temporal networks. Therefore, we have gathered a corpus of 26 network datasets taken from real world systems. The networks vary in size from just over 100 edges to more than one million and all networks are treated as undirected and unweighted with added their links never being removed. All of the networks can be seen in Table~S\ref{tab:datasets} with a description for each. 

We have assigned each dataset a ``collection type" based on where the data was collected. Social data is gathered from a social media site, email list or messaging app; nodes are people and edges are emails, messages or ``follows'' between two people. Co-occurrence data is gathered from books, articles and musical notation; nodes are words or musical notes and the edges are created when two nodes exist next to one another in the source material. Citation data is gathered from literature which cites other literature; nodes are pieces of literature and edges are the citations. Economic data is gathered from places where financial transactions are of interest; nodes are products, buyers and sellers and edges are linking buyers/sellers with their products. Computer data is gathered from computer networks; nodes are computers and edges are interactions between the computers. Transport data is gathered from transportation networks; nodes are geographical places and edges are the transportation links between the places. Finally, contact data is gathered when people interact in close proximity, usually using proximity sensors; nodes are people and edges are created when two people are physically close for a threshold of time. 

We also assign a ``link creation type" based on the dynamics of the networks themselves. Stars consist of all edges connecting to one node, for example when one paper cites others. Individual edges are created when only two nodes interact, for example two characters in a book. Cliques are fully interconnected node clusters, for example everyone who answers a question on StackOverflow is connected together. Bipartite networks consist of two separate node pools which only link with each other, and not within themselves. Finally, spatial networks are a mix of many of the others and so have been separated into their own category. 

\begin{table}[!tb]
    \resizebox{\textwidth}{!}{\begin{tabular}{c|c|c|c|c|c|c}
        & Name & Description & Collection & Link Creation & Nodes & Edges  \\
         \hline \hline
        a & College Messages~\cite{Panzarasa2009PatternsCommunity} & Messages between students on a UC-Irvine message board & Social & Star & 1,899 & 59,835 \\
        \hline
        b & SCOTUS Majority~\cite{Fowler2007NetworkCourt} & Legal citations among majority opinions by SCOTUS & Citation & Star & 34,613 & 202,167 \\
        \hline
        c & Amazon Ratings~\cite{Lim2010DetectingBehaviors} & Amazon user connects to all products they have rated & Economic & Bipartite & 3,376,972 & 5,838,041 \\
        \hline
        d & Citation US Patents~\cite{Hall2001TheTools} & Citation among patents in the United States & Citation & Star & 3,774,362 & 16,516,270 \\
        \hline
        e & Classical Piano~\cite{Park2020NoveltyNetworks} & Transitions of chords in western classical piano music & Co-occurrence & Individual & 144,183 & 585,154 \\
        \hline
        f & Email EU~\cite{Paranjape2017MotifsNetworks} & E-mails between users at an EU research institution & Social & Star & 986 & 332,334 \\
        \hline
        g & Procurement EU~\cite{Wachs2021CorruptionPerspective} & Public EU procurement contracts & Economic & Bipartite & 839,824 & 4,098,711 \\
        \hline
        h & Facebook Wall~\cite{Viswanath2009OnFacebook} & Posts by users on other users' Facebook wall & Social & Star & 46,952 & 876,993 \\
        \hline
        i & Lord Of the Rings~\cite{Clegg2009} & Character co-occurrence in Lord of the Rings Trilogy & Co-occurrence & Individual & 139 & 2,649 \\
        \hline
        j & PhD Exchange~\cite{Taylor2017Eigenvector-basedNetworks} & Exchange of PhD mathematicians between unis in the US & Citation & Star & 230 & 9,584 \\
        \hline
        k & Programming Languages~\cite{Valverde2015PunctuatedLanguages} & Influence relationships among programming languages & Citation & Star & 366 & 762 \\
        \hline
        l & Reuters Terror News~\cite{Corman2006StudyingSystems} & Word co-use in Reuters 9/11 coverage & Co-occurrence & Individual & 13,308 & 148,035 \\
        \hline
        m & Route Views~\cite{Clegg2009} & Route Views internet topology & Computer & Individual & 33,804 & 94,993 \\
        \hline
        n & Reddit Hyperlinks Body~\cite{Kumar2018CommunityWeb} & Subreddit-to-subreddit hyperlinks from body of posts & Social & Clique & 35,776 & 286,561 \\
        \hline
        o & Reddit Hyperlinks Title~\cite{Kumar2018CommunityWeb} & Subreddit-to-subreddit hyperlinks from title of posts & Social & Individual & 54,075 & 571,927 \\
        \hline
        p & Hypertext Conference~\cite{Isella2011WhatsNetworks} & Contacts among attendees of ACM Hypertext 2009 & Contact & Spatial & 113 & 20,818 \\
        \hline
        q & Infectious~\cite{Isella2011WhatsNetworks} & Contacts during Infectious SocioPatterns 2011 event  & Contact & Spatial & 10,972 & 415,912 \\
        \hline
        r & Office~\cite{Genois2015DataLinkers} & Contacts between individuals in an office building & Contact & Spatial & 92 & 9,827 \\
        \hline
        s & AskUbuntu~\cite{Paranjape2017MotifsNetworks} & User answers or comments on questions on AskUbuntu & Social & Clique & 159,316 & 964,437 \\
        \hline
        t & MathOverflow~\cite{Paranjape2017MotifsNetworks} & User answers or comments on questions on MathOverflow  & Social & Clique & 24,818 & 506,550 \\
        \hline
        u & StackOverflow~\cite{Paranjape2017MotifsNetworks} & User answers or comments on questions on StackOverflow  & Social & Clique & 2,601,977 & 63,497,050 \\
        \hline
        v & SuperUser~\cite{Paranjape2017MotifsNetworks} & User answers or comments on questions on SuperUser  & Social & Clique & 194,085 & 1,443,339 \\
        \hline
        x & UCLA AS~\cite{Clegg2009} & UCLA AS level internet topology & Computer & Individual & 38,219 & 226,729 \\
        \hline
        y & US Air Traffic~\cite{Paranjape2017MotifsNetworks} & Flights among all commercial airports in the US & Transport & Individual & 2,278 & 6,390,340 \\
        \hline
        z & Wiki Talk~\cite{Paranjape2017MotifsNetworks} & Wikipedia users editing each other's Talk page  & Social & Individual & 1,140,149 & 7,833,140 \\
        \hline
        A & IETF~\cite{khare2022untangling} & IETF mailing list replies & Social & Clique & 23,792 & 989,740 \\
    \end{tabular}}
        \caption{Dataset corpus used for network creation.}
    \label{tab:datasets}
\end{table}

For each network, we treat each edge event as one time iteration for the network, where all edges that happen simultaneously are grouped into the same edge event. The whole time period is split into five equal time windows, this is to ensure any long periods of inactivity in the network do not leave empty spaces in our plots. Then, we focus on the first window and divide it into two windows of equal length. We build a network snapshot for both of these two sub-windows, and calculate the mobility taxonomy correlations between them. Then, we push the time window forward by $\frac{1}{2}$ of its duration. This is repeated until the time window reaches the end of the whole time period of the data. Therefore, we calculate 9 correlations for each network, each taking into account $\frac{1}{5}$ of the total time period of the network.

In Figure~S\ref{fig:taxonomy_lines}, we plot each network in the corpus individually. This plot is much harder to interpret than Fig. 1 which uses standard deviations to give insight into the ranges of correlation values for each network collection type. However, this plot clearly shows the two low {\it mobility} and {\it neighbour mobility} networks which are referred to in the main paper. The first of these is Citation US Patents~\cite{Hall2001TheTools} which has unusually high {\it philanthropy} correlation suggesting patents citing highly cited patents themselves receive a large amount of patents. The other is Infectious~\cite{Isella2011WhatsNetworks} which has a recurring ``jump'' in each correlation, this is due to a sudden large decrease in the amount of active nodes in the network (the authors have not managed to verify the cause of this decrease). Finally, the large shift in the Lord Of the Rings~\cite{Clegg2009} philanthropy and {\it community} correlations, as referred to in the main paper, can be seen more easily in this plot.

\begin{figure}[!tb]
\centering{\includegraphics[width=1.0\textwidth]{mob_tax.png}
  \caption{All networks' mobility taxonomy correlations in the corpus coloured by data type. The x-axis represents the ordered window numbers starting at the first timestamp of the network. The y-axis shows the correlation coefficient as measured using Pearson correlation.}
\label{fig:taxonomy_lines}
}
\end{figure}

\section{Extra artificial data results}

Table~\ref{tab:models} describes the twelve models used in the artificial data study. The models chosen were taken from the literature combined with models that the authors reasoned might maximise or minimise the taxonomy statistics being investigated. As mentioned in the main text, each model uses the basic structure of the classic preferential attachment model~\cite{Barabasi1999EmergenceNetworks} where one node is introduced at every iteration and three links. The three links are connected to an existing node with probabilities chosen according to certain rules, for example the classic preferential attachment rule is that the probability of connecting to a node is precisely proportional to its degree.


\begin{table}[!tbp]
    \resizebox{\textwidth}{!}{\begin{tabular}{c|c}
        Name & Probability of Receiving Edge \\
         \hline \hline
        Random & Each node is equal \\
        \hline
        Preferential attachment & Proportional to each node's individual degree \\
        \hline
        Preferential neighbour attachment & First, a node is chosen using preferential attachment (PA); then one of its neighbours is chosen using PA \\
        \hline
        Equality & Inversely proportional to each node's individual degree\\
        \hline
        Sum neighbour degree  & Proportional to the sum of the degree of the neighbours of each node \\
        \hline
        Average neighbour degree & Proportional to the average of the degree of the neighbours of each node \\
        \hline
        Inverse average neighbour degree & Inversely proportional to the average of the degree of the neighbours of each node \\
        \hline
        Cluster & Proportional to the clustering coefficient of each node \\
        \hline
        Eigen & Proportional to the eigenvector centrality of each node \\
        \hline 
        Fitness & Proportional to a sample taken from a $\Gamma(x)$ distribution \\
        \hline
        Gamma & Same as above but every node is re-sampled every $m$ iterations \\
        \hline
        Gamma individual & Same as above but only one node is re-sampled \\
    \end{tabular}}
        \caption{Models used for generating the evolutionary dynamics of artificial networks.}
    \label{tab:models}
\end{table}


The figures~\ref{fig:optimiser_runs_max} and ~\ref{fig:optimiser_runs_min} show precisely which models were used in the maximisation and minimisation experiments. Each of the ten runs for each experiment is a separate row. For maximisation we can see that for maximising neighbour mobility and mobility the strategy was almost always philanthropy with occasionally the preferential attachment strategy. It makes sense that the strategy that maximises the correlation between degree in one time slice and the next does the same for a neighbourhood. For maximising philanthropy the sum neighbour degree was usually the most successful with occasionally eigen. Finally maximising community took an alternating strategy with average neighbour degree alternating with eigen and sum neighbour degree.

For minimisation equality and sometimes inverse average neighbour degree minimised philanthropy and community. These two strategies would tend to produce networks where all nodes have a similar degree. Minimising neighbour mobility and mobility itself both took an alternating strategy with a complex mix of strategies involved. 

\begin{figure}[!tb]
\centering{\includegraphics[width=0.8\textwidth]{optimiser_runs_max.png}
  \caption{Visualisation of the evolution models used in the maximisation of mobility taxonomy aspects. The x-axis represents the number of 1,000 iteration-long windows (not including the initialisation of the graph).}
\label{fig:optimiser_runs_max}
}
\end{figure}

\begin{figure}[!tb]
\centering{\includegraphics[width=0.8\textwidth]{optimiser_runs_min.png}
  \caption{Visualisation of the evolution models used in the minimisation of mobility taxonomy aspects. The x-axis represents the number of 1,000 iteration-long windows (not including the initialisation of the graph).}
\label{fig:optimiser_runs_min}
}
\end{figure}


\printbibliography
\end{document}

