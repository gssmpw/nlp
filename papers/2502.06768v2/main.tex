 %%%%%%%% ICML 2025 EXAMPLE LATEX SUBMISSION FILE %%%%%%%%%%%%%%%%%

\documentclass{article}

\usepackage{mdframed}
\usepackage{microtype}
\usepackage{graphicx}
\usepackage{array}
\usepackage{subfigure}
\usepackage{booktabs} % for professional tables
\usepackage{stfloats}
\usepackage{float}
\usepackage{enumitem}
\usepackage{multirow}
\usepackage{adjustbox}

\usepackage{hyperref}

\usepackage{ dsfont }

% Attempt to make hyperref and algorithmic work together better:
\newcommand{\theHalgorithm}{\arabic{algorithm}}
\newcommand{\kulin}[1]{{\color{blue} Kulin: #1}}
\newcommand{\sitan}[1]{{\color{blue} Sitan: #1}}
\newcommand{\E}{\mathbb{E}}
\newcommand{\mask}{0}


\newcounter{boxcounter}
\renewcommand{\theboxcounter}{\arabic{boxcounter}}


% Custom environment for the box



\newcommand{\topk}{Top-\(K\) }
\newcommand{\topkprobdiff}{Top-\(K\) probability margin }
\newcommand{\prior}{p_{\textrm{prior}}}

\usepackage[accepted]{icml2025}

% For theorems and such
\usepackage{amsmath}
\usepackage{amssymb}
\usepackage{mathtools}
\usepackage{amsthm}

% if you use cleveref.
\usepackage[capitalize,noabbrev]{cleveref}

%%%%%%%%%%%%%%%%%%%%%%%%%%%%%%%%
% THEOREMS
%%%%%%%%%%%%%%%%%%%%%%%%%%%%%%%%
\theoremstyle{plain}
\newtheorem{theorem}{Theorem}[section]
\newtheorem{proposition}[theorem]{Proposition}
\newtheorem{lemma}[theorem]{Lemma}
\newtheorem{corollary}[theorem]{Corollary}
\newtheorem{example}[theorem]{Example}
\newtheorem{conjecture}[theorem]{Conjecture}
\theoremstyle{definition}
\newtheorem{definition}[theorem]{Definition}
\newtheorem{assumption}[theorem]{Assumption}

\theoremstyle{remark}

\newcommand{\msg}{\textrm{MS}}


\newcounter{theo}
\renewcommand{\thetheo}{\arabic{theo}}
\newenvironment{theo}[2][]{%
\refstepcounter{theo}%
\ifstrempty{#1}%
{\mdfsetup{%
frametitle={%
\tikz[baseline=(current bounding box.east),outer sep=0pt]
\node[anchor=east,rectangle,fill=blue!20]
{\strut Theorem~\thetheo};}}
}%
{\mdfsetup{%
frametitle={%
\tikz[baseline=(current bounding box.east),outer sep=0pt]
\node[anchor=east,rectangle,fill=blue!20]
{\strut ~#1};}}%
}%
\mdfsetup{innertopmargin=10pt,linecolor=blue!20,%
linewidth=2pt,topline=true,%
frametitleaboveskip=\dimexpr-\ht\strutbox\relax
}
\begin{mdframed}[]\relax%
\label{#2}}{\end{mdframed}}

% Todonotes is useful during development; simply uncomment the next line
%    and comment out the line below the next line to turn off comments
%\usepackage[disable,textsize=tiny]{todonotes}
\usepackage[textsize=tiny]{todonotes}



\newcommand{\loss}{\mathcal L}


% The \icmltitle you define below is probably too long as a header.
% Therefore, a short form for the running title is supplied here:
\icmltitlerunning{Train for the Worst, Plan for the Best: Understanding Token Ordering in Masked Diffusions}

\begin{document}


\twocolumn[
\icmltitle{Train for the Worst, Plan for the Best: \\
Understanding Token Ordering in Masked Diffusions}

\icmlsetsymbol{equal}{*}

\begin{icmlauthorlist}
\icmlauthor{Jaeyeon Kim}{equal,yyy}
\icmlauthor{Kulin Shah}{equal,comp}
\icmlauthor{Vasilis Kontonis}{comp}
\icmlauthor{Sham Kakade}{yyy}
\icmlauthor{Sitan Chen}{yyy}
\end{icmlauthorlist}



\icmlaffiliation{yyy}{Harvard University}
\icmlaffiliation{comp}{University of Texas Austin}

\icmlcorrespondingauthor{Kulin Shah}{kulinshah@utexas.edu}
% \icmlcorrespondingauthor{Firstname2 Lastname2}{first2.last2@www.uk}

% You may provide any keywords that you
% find helpful for describing your paper; these are used to populate
% the "keywords" metadata in the PDF but will not be shown in the document
\icmlkeywords{Machine Learning, ICML}

\vskip 0.3in
]

\printAffiliationsAndNotice{\icmlEqualContribution}

\begin{abstract}
In recent years, masked diffusion models (MDMs) have emerged as a promising alternative approach for generative modeling over discrete domains. Compared to autoregressive models (ARMs), MDMs trade off complexity at training time with flexibility at inference time. At training time, they must learn to solve an exponentially large number of infilling problems, but at inference time, they can decode tokens in essentially arbitrary order. In this work, we closely examine these two competing effects. On the training front, we theoretically and empirically demonstrate that MDMs indeed train on computationally intractable subproblems compared to their autoregressive counterparts. On the inference front, we show that a suitable strategy for adaptively choosing the token decoding order significantly enhances the capabilities of MDMs, allowing them to sidestep hard subproblems. On logic puzzles like Sudoku, we show that adaptive inference can boost solving accuracy in pretrained MDMs from $<7$\% to $\approx 90$\%, even outperforming ARMs with $7\times$ as many parameters and that were explicitly trained via teacher forcing to learn the right order of decoding.
\end{abstract}

\section{Introduction}

While diffusion models~\cite{ho2020denoising,song2021score} are now the dominant approach for generative modeling in continuous domains like image, video, and audio, efforts to extend this methodology to discrete domains like text and proteins~\cite{austin2023structured,lou2024discrete,hoogeboom2021argmax} remain nascent. Among numerous proposals, masked diffusion models (MDMs) \cite{lou2024discrete,sahoo2024simple,shi2025simplified} have emerged as a leading variant, distinguished by a simple and principled objective: to generate samples, learn to reverse a noise process which independently and randomly masks tokens. 


In many applications, such as language modeling, masked diffusion models (MDMs) still underperform compared to autoregressive models (ARMs)~\cite{nie2024scaling,zheng2024maskeddiffusionmodelssecretly}, which instead learn to reverse a noise process that unmasks tokens sequentially from left to right. However, recent studies suggest that MDMs may offer advantages in areas where ARMs fall short, including reasoning \cite{nie2024scaling,kitouni2024factorization}, planning \cite{ye2024beyond}, and infilling \cite{gong2024scaling}. This raises a key question: what are the strengths and limitations of MDMs compared to ARMs, and under what conditions can MDMs be scaled to challenge the dominance of ARMs in discrete generative modeling?

To understand these questions, we turn a microscope to two key competing factors when weighing the merits of MDMs over ARMs:
\begin{itemize}[leftmargin=*,topsep=0pt,itemsep=0pt]
    \item \textbf{Complexity at training time}: By design, the prediction task that MDMs are trained on is more challenging. Whereas ARMs seek to predict the next token given an unmasked prefix, MDMs seek to predict a token conditioned on a set of unmasked tokens in arbitrary positions. 
    \item \textbf{Flexibility at inference time}: On the other hand, the sampling paths taken by an MDM are less rigid. The order in which tokens are decoded at inference time is random instead of fixed to left-to-right. In fact, even more is possible: MDMs can actually be used to decode in \emph{any order}~\cite{zheng2024maskeddiffusionmodelssecretly}.
\end{itemize}
Therefore, we ask:
\begin{center}
    \emph{Are the benefits of inference flexibility for MDMs enough to outweigh the drawbacks of training complexity?}
\end{center}
In this work, we provide dual perspectives on this question.

\textbf{(1) Training for the worst.} \enspace First, we provide theoretical and empirical evidence that the overhead imposed by training complexity quantifiably impacts MDMs' performance. 

We prove that even for simple, benign models of data, there are noise levels at which a large fraction, but not all, of the corresponding subproblems solved by MDMs are computationally intractable. We then show this imbalance in computational complexity across subproblems persists even in real-world text data (Fig.~\ref{fig:scaling_laws}, left).

\paragraph{(2) Planning for the best.} While the above might appear to be bad news for MDMs, in the second part of this paper we answer our guiding question in the affirmative by building upon the observation~\cite{zheng2024maskeddiffusionmodelssecretly} that MDMs which can perfectly solve all masking subproblems can be used to decode in \emph{any} order.

In place of vanilla MDM inference whereby tokens are unmasked in random order, we consider \emph{adaptive} strategies that carefully select which token to unmask next. 
Our key insight is that this adaptivity makes it possible to \emph{sidestep} the hard subproblems from training (Fig.~\ref{fig:main_fig}). In fact, we find that \textbf{even without modifying how MDMs are trained, the resulting models' logits contain enough information to determine the right order in which to unmask.}

Our main empirical result is to show that the performance of MDMs pretrained on logic puzzle data dramatically improves when one goes from vanilla to adaptive inference. For example, on Sudoku puzzles, a simple adaptive strategy (Section~\ref{subsec:effective-design}) improves the accuracy of MDMs from $<7$\% to almost 90\%. Remarkably, this not only outperforms vanilla ARMs, but even bespoke ARMs trained to learn the right decoding order via supervised teacher forcing~\cite{shah2024causal,lehnert2024beyond} (Table~\ref{tab:sudoku-results}).

\paragraph{Organization.} In Section~\ref{sec:2}, we provide preliminaries on MDMs and set notation. In Section~\ref{sec:hardness}, we 
examine MDM training and demonstrate the imbalance in computational intractability across subproblems. In Section~\ref{sec:inference}, we consider adaptive inference in MDMs and investigate its impact on likelihood modeling across various tasks.
 
\section{Masked Diffusion Models (MDM)} \label{sec:2}
In this section, we explain the framework of Masked Diffusion Models \cite{shi2025simplified,sahoo2024simple} and its interpretation as an \emph{order-agnostic learner}. MDMs gradually add noise to the true discrete data and learn the marginal distribution of the induced reverse process. Below, we formulate the forward and reverse processes for MDMs.


Let the distribution $p_{\rm{data}}$ on $\{1,\ldots,m\}^L$ be the data distribution over sequences of length $L$ and with vocabulary $\{1, \ldots, m\}$. We use $\mask$ to denote the ``mask'' token. 

\paragraph{Forward process.} For a given $x_0 \sim p_{\rm{data}}$ and a noise level $t \in [0,1]$, the forward process $x_t \sim q_{t|0}(\cdot \, | \, x_0)$ is a coordinate-independent masking process via
$q_{t|0}(x_t | x_0) = \prod_{i=0}^{L-1} q_{t|0}(x_t^i | x_0^i)$,
where 
\begin{equation*}
    q_{t|0}(x_t^i \mid x_0^i) = \mathrm{Cat}\bigl(\alpha_t \mathbf{e}_{x_0^i} + (1-\alpha_t)\mathbf{e}_{\mask} \bigr)\,,
\end{equation*}
where $\alpha_t$ is the predefined noise schedule satisfying $\alpha_0 \approx 1, \alpha_1 \approx 0$ and $e_{x_0^i} \in \mathbb{R}^{m+1}$ denotes a one-hot vector corresponding to the value of token $x_0^i$. $\mathrm{Cat}(\pi)$ denotes the categorical distribution given by $\pi \in \Delta^{m}$. In other words, for each $i$-th coordinate, $x_t^i$ is masked to the mask token $\mask$ with probability $1-\alpha_t$ and unchanged otherwise.

\begin{figure}[t]
    \centering
    \includegraphics[width=0.47\textwidth]{Figures/figure_1_3.png}
    \vspace{-0.2in}
    \includegraphics[width=0.47\textwidth]{Figures/figure_1_4.png}
    \caption{ 
    (\textbf{Top}) MDM training can be seen as learning multiple masked prediction problems, where some are harder to learn, leading to performance imbalance (Section~\ref{sec:hardness}). 
    (\textbf{Bottom}) During inference, adaptive MDM can avoid difficult problem instances, improving performance (Section~\ref{sec:inference}).}
    \label{fig:main_fig}
\end{figure}


\paragraph{Reverse process.} The reverse process of the above forward process is denoted using $q_{s|t}(x_s | x_t, x_0)$ and is given by $q_{s|t}(x_s | x_t, x_0) = \prod_{i=0}^{L-1} q_{s|t}(x_s^{i} | x_t, x_0)$ for any $s<t$, where
{\small
\begin{equation*}
     q_{s|t}(x_s^i \, \lvert\, x_t, x_0) = \begin{cases} 
       \mathrm{Cat}(\mathbf{e}_{x_t^{i}}) \quad & x_t^i \ne  m \\
     \mathrm{Cat}\left(\frac{1-\alpha_s}{1-\alpha_t}\mathbf{e}_m + \frac{\alpha_s - \alpha_t}{1-\alpha_t}\mathbf{e}_{x_0}\right)
  \quad &x_t^i= m\,.
     \end{cases}
\end{equation*}}The reverse transition probability $q_{s|t}(x_s^i | x_t, x_0)$ is approximated using $g_{\theta}(x_s^i | x_t) \triangleq q_{s|t}(x_s^i \, \lvert\, x_t, x_0 \leftarrow p_{\theta}(x_t, t) )$ where $p_\theta(x_t,t)$ is a denoising network trained to predict the marginal on $x_0$ via an ELBO-based loss. To be precise, $q_{s|t} \left( x_s^i \mid x_t, x_0 \leftarrow p_{\theta}(x_t, t) \right)$ indicates the conditional probability where $p_{\theta}(x_t, t)$ is placed in the position of $x_0$ within $q_{s|t}(x_s^i \mid x_t, x_0)$.
\begin{equation*}
    \mathcal{L}_\theta = \int_{0}^1 \frac{\alpha_t'}{1-\alpha_t} \displaystyle \mathop{\mathbb{E}}_{ \substack{x_0 \sim p_{\rm data} \\ x_t \sim q_{t|0}(\cdot | x_0)}  }  \left[\delta_{x_t,\mask} \mathbf{e}_{x_0}^\intercal \log p_\theta(x_t,t) \right] dt.
\end{equation*}
Here, $\alpha_t'=\frac{d \alpha_t}{dt}$ and $\delta_{x_t,\mask}$ is the indicator function; the summation is computed over coordinates $i$ s.t. $x_t^i = \mask$. In practice, a time-embedding-free architecture for the denoising network, i.e., $p_\theta(x_t, t) = p_\theta(x_t)$, is usually employed as \(x_t\) implicitly contains information about \(t\) via the number of masked tokens.

The reverse sampling process starts from the fully masked sentence $x_1 = (\mask,\ldots,\mask)$. At a given noise level \(t \in (0,1]\), suppose we have a partially masked sequence \(x_t\). For predetermined noise level \(s < t\), we sample $x_s \sim g_\theta(\cdot | x_t)$. This process is repeated recursively from \(t=1\) to \(t=0\).


\subsection{Reformulating the training and inference of MDMs} \label{sec:agnostic_learner}

In this section, we first discuss vanilla order-agnostic training of MDMs and compare it with ``left-to-right" order training of autoregressive models in \cref{sec:vanilla-mdm-training}. Then, we reformulate vanilla MDM inference in \cref{sec:vanilla-mdm-inference} to set the stage for the upcoming discussion.

\subsubsection{Order-agnostic training of MDMs}
\label{sec:vanilla-mdm-training}

Recent works \cite{zheng2024maskeddiffusionmodelssecretly,ou2024absorbing} have observed that the learning problem of MDM is equivalent to a masked language model. Building upon their analysis, we reformulate the loss $\mathcal{L}_\theta$ to show that $\loss_{\theta}$ is a linear combination of the loss for all possible infilling masks. We first define \(x_0[M]\) as a masked sequence, obtained from original sequence $x_0$ where indices in the mask set $M$ (regarded as a subset of $[L]\triangleq\{1,2,\ldots,L\}$) are replaced with mask token $0$.


\begin{proposition} \label{prop:mdm_loss}
Assume $\alpha_0=1$, $\alpha_1 =0$ and denoising network $p_\theta$ is time-embedding free.
Then $ \mathcal{L}_\theta \le -\mathbb{E}_{x_0 \sim p_{\rm data}}[\log p_\theta(x_0)]$ and
\begin{equation} \label{eqn:mdm_loss}
\mathcal{L}_\theta = -\frac{1}{L}\sum_{ M\subseteq [L],i \in M}\frac{1}{\binom{L-1}{|M|-1}} \displaystyle \mathop{\mathbb{E}}_{x_0 \sim p_{\rm data}} [ \log p_\theta(x^i_0 | x_0[M]) ],
\end{equation}
where $|M|$ denote the size of the mask $M$ and \(p_\theta(x_i \mid x_0[M])\) indicates the conditional probability of the \(i\)-th coordinate from \(p_\theta(x_t)\).
\end{proposition}
The proof of the above proposition is given in Appendix~\ref{appenix:mdm-equivalent-loss}. As the MDM loss is a linear combination of the loss for all possible infilling mask $M$, the minimizer of the loss $\loss_{\theta}$ learns to solve \emph{every} masking problem. More formally, for all subsets $M \subseteq \{1, 2, \ldots, L\}$, we have $\arg \min_\theta \log p_\theta(x^i_0 | x_0[M])= p_{\rm{data}} (x^i_0 | x_0[M])$. In other words, the optimal predictor $p_\theta$ is the posterior marginal of the $i$-th token, conditioned on $x_0[M]$ for all masks $M$. The training objective of MDM aims to predict $x_0$ from $x_0[M]$ across all possible masks. Hence, we will refer to the MDM training as \emph{order-agnostic} training. 

On the other hand, Autoregressive Models (ARMs) learn to solve a smaller set of infilling problems ($L$ infilling problems in ARMs as opposed to $\exp(L)$ infilling problems in MDM) by predicting $i^{\textrm{th}}$ token $x^i$ given all previous tokens $x^0,\ldots,x^{i-1}$. This prediction problem is equivalent to predicting $x^i$ by masking at positions $\{i,\ldots, L-1\}$. Therefore, we can write it as
\begin{equation} \label{eqn:ar_loss}
    \log p_\theta(x_0) = \sum_{i=0}^{L-1} \log p_\theta ( x_0^i | x_0 [\{i,\ldots,L-1\}]).
\end{equation}
ARMs are trained to predict tokens sequentially from left to right in all sequences. We refer to this as left-to-right training. In general, one can also consider predicting tokens sequentially under some \emph{fixed, known} permutation of the sequence; we refer to this as \emph{order-aware training}.


\subsubsection{Order-agnostic inference of MDMs}
\label{sec:vanilla-mdm-inference}

The MDM inference can be decomposed into two steps: (a) randomly selecting a set of positions to unmask and (b) assigning token values to each position via the denoising network $p_\theta$. More precisely, we can reformulate the reverse process $x_s \sim g_\theta(\cdot | x_t)$ as follows. 

\begin{theo}[Vanilla MDM inference]{alg:random_sampler_redefine}
\vspace{-0.1in}
\begin{itemize}
    \item[(a)] Sample a set of masked tokens \(\mathcal{S} \subseteq \{i \mid x_t^i = \mask\}\), \(\mathbb{P}(i \in \mathcal{S}) = \frac{\alpha_s-\alpha_t}{1-\alpha_t}\).
    \item[(b)] For each $i \in \mathcal{S}$, sample $x_s^i \sim p_\theta(x^i | x_t)$.
\end{itemize}
\end{theo}
Therefore, the inference in MDM is implemented by randomly selecting $S$ and then filling each token value according to the posterior probability $p_{\theta}(x_s^i | x_t)$.

\section{MDMs train on hard problems}\label{sec:hardness}

\begin{figure*}[t]
    \centering
    \includegraphics[width=0.8\textwidth]{Figures/figure_2_2.png} 
\caption{\textbf{Left: MDMs train on hard problems (Section~\ref{sec:hardness_text})}. x-axis and y-axis correspond to $\log(\text{FLOPs})$ and $-\log p_\theta(x)$, respectively. MDM {\color{blue} (Blue)} is worse than ARM {\color{orange} (Orange)} in likelihood modeling. Most masking problems {\color{purple} (Other lines)} that MDM is trained on are harder than those encountered by ARM, as indicated by small log-likelihoods. \textbf{Right: Task error imbalance (Section~\ref{sec:imbalance_error})}. MDM's performance varies across different tasks. For text data (top right), this is indicated by validation loss. For L\&O-NAE-SAT (bottom right), MDM performs well on the masking problems for observation positions (light region) but struggles with latent positions (dark region).}\label{fig:scaling_laws}
\end{figure*}

In this section, we theoretically and empirically demonstrate that a large portion of masking subproblems $p_\theta(x^i_0 \mid x_0[M])$ can be difficult to learn. For intuition, consider solving a masked prediction problem $p_\theta(x^i\mid x_0 [M ])$ on text data like masking an arbitrary sentence in the middle of a document and predicting the correct word for a specific position in that sentence. It is reasonable that this task should be more complex, even for humans, than left-to-right prediction, and in this section, we place this intuition on a rigorous footing.


In Section~\ref{sec:csp}, we show several examples of simple, non-pathological distributions for which: (1) the masking problems encountered during order-\emph{aware} training are computationally tractable, yet (2) many of the ones encountered during order-agnostic training are computationally intractable. In Section~\ref{sec:hardness_text}, we empirically show that text data also exhibits this gap between the computational complexity of order-aware and order-agnostic training. In Section~\ref{sec:imbalance_error}, we reveal that this discrepancy in computational complexity manifests empirically in \underline{\textbf{performance imbalance across tasks}}: as predicted by the theory, MDMs trained on data from such distributions exhibits small errors on easy subproblems but suffers from large errors on harder ones.

\subsection{Benign distributions with hard masking problems} \label{sec:csp}

We now describe a simple model of data under which we explore the computational complexity of masking problems.

\begin{definition}\label{definition:planted}
    A \emph{latents-and-observations (L\&O) distribution} is a data distribution $p_{\rm data}$ over sequence of length $L$ with alphabet size $m$ (precisely, $p_{\rm data}$ is over $\{0,\ldots,m\}^L$)  is specified by a permutation $\pi$ over indices $\{1, 2, \ldots, L \}$, number of latent tokens $N$, number of observation tokens $P$ such that $N + P = L$, prior distribution $\prior$ of latent variables over $\{1,\ldots,m\}$ and efficiently learnable \emph{observation functions} $\mathcal{O}_1,\ldots,\mathcal{O}_P: \{1,\ldots,m\}^N \to \Delta(\{0,\ldots,m\})$,\footnote{Here \emph{efficiently learnable} is in the standard PAC sense: given polynomially many examples of the form $(z,y)$ where $z\sim \pi^n$ and $y\sim \mathcal{O}_j(z)$, there is an efficient algorithm that can w.h.p. learn to approximate $\mathcal{O}_j$ in expectation over $\pi^n$.}
    \begin{itemize}[topsep=0pt,itemsep=0pt,leftmargin=*]
        \item (\textbf{Latent tokens}) For $i = 1,\ldots,N$, sample $x^{\pi(i)}$ 
        independently from the prior distribution $\prior$ of the latents.
        \item (\textbf{Observation tokens}) For $j = 1,\ldots,P$, sample $x^{\pi(N + j)}$ independently from $\mathcal{O}_j(x^{\pi(1)},\ldots,x^{\pi(N)})$.
    \end{itemize} 
\end{definition}

L\&O distributions contain two types of tokens: (1) \emph{latent tokens} and (2) \emph{bservation tokens}. Intuitively, latent tokens are tokens in the sequence, indexed by $\pi(1), \pi(2), \ldots, \pi(N)$ that serve as ``seeds" that provide randomness in the sequence; the remaining tokens, called observation tokens (indexed by $\pi(N+1), \pi(N+2), \ldots, \pi(N+P)$), are determined as (possibly randomized) functions of the latent tokens via $\mathcal{O}_1,\ldots,\mathcal{O}_P$.




Note that by design, order-aware training, e.g. by permuting the sequence so that $\pi$ becomes the identity permutation and then performing autoregressive training, is computationally tractable: predicting $x^{\pi(i)}$ given $x^{\pi(1)},\ldots,x^{\pi(i-1)}$ is trivial when $i \le N$ as the tokens are independent, and computationally tractable when $i > N$ because $x^{\pi(i)}$ only depends on $x^{\pi(1)},\ldots,x^{\pi(N)}$ and is efficiently learnable by assumption. In contrast, below we will show examples where if one performs order-agnostic training \emph{à la} MDMs, one will run into hard masking problems with high probability.


First note that if the observations $(\mathcal{O}_1,\ldots,\mathcal{O}_P)$ are given by a cryptographic hash function, then the masking problem of predicting $(x^{\pi(1)},\ldots,x^{\pi(L)})$ given $(x^{\pi(N+1)},\ldots,x^{\pi(N+P)})$ is computationally intractable by design because it requires inverting the hash function. While this is a well-known folklore observation regarding the role of token ordering in language modeling, it is not entirely satisfying because this construction is worst-case in nature \--- in real-world data, one rarely trains on sequences given by cryptographic hash functions. Furthermore, it only establishes hardness for a specific masking pattern which need not be encountered in the course of running the reverse process.

We provide several simple instances of L\&O distributions that address these issues: instead of leveraging delicate cryptographic constructions, they are \emph{average-case} in nature and furthermore we can establish hardness for \emph{typical} masking problems encountered along the reverse process. 

In all these examples, the hardness results we establish hold even if the algorithm knows all of the parameters of $p_{\rm data}$ as well as the observation functions $\mathcal{O}_1,\ldots,\mathcal{O}_P$. 

Due to space constraints, here we focus on the following example, deferring two others to Apps.~\ref{app:parity} and~\ref{app:slab}.

\begin{example}[Sparse predicate observations]\label{example:csp}
    Consider the following class of L\&O distributions. Given \emph{arity} $k\ge 2$, fix a \emph{predicate} function $g: \{1,\ldots,m\}^k \to \{0,1\}$. Consider 
    the set of all ordered subsets of $\{1,2,\ldots,N\}$ of size $k$ and set the total number of observation latents $P$ equal to the size of this set (hence $P = N ! / (N-k)! = N(N-1)\cdots(N-k+1)$). To sample a new sequence, we first sample latent tokens $x^{\pi(1)},\ldots,x^{\pi(N)}$ from the prior distribution $\prior$ and an observation latent corresponding to a $k$-sized subset $S$ is given by $g( \{ x^{\pi(i)} \}_{i \in S} )$. In other words, each observation latent corresponds to a $k$-sized subset $S$ of $\{1,2,\ldots,N\}$ and the corresponding observation function $\mathcal{O}_S(x^{\pi(1)}, \ldots, x^{\pi(N)} )$ is given by $g( \{ x^{\pi(i)} \}_{i \in S} )$.
\end{example}


\begin{proposition}\label{prop:csp}
    Let $x$ be a sample from an L\&O distribution $p_{\rm data}$ with sparse predicate observations as defined in Example~\ref{example:csp}, with arity $k$ and predicate $g$ satisfying Assumption~\ref{assume:paramagnetic}, and let $\gamma$ be the probability that $g$ is satisfied by a random assignment from $\{1,\ldots,m\}^k$. Let $D_{\rm KS}$ and $D_{\rm cond}$ be some constants associated with the predicate function $g$ (see Definition~\ref{def:thresholds}). Suppose each token in $x$ is independently masked with probability $\alpha$, and $M$ is the set of indices for the masked tokens. If $1 - \gamma^{-1} D_{\rm KS}/kN^{k-1} \le \alpha \le 1 - \gamma^{-1} D_{\rm cond}/kN^{k-1}$, then under the \emph{1RSB cavity prediction} (see Conjecture~\ref{conj:1rsb}), with probability $\Omega_k(1)$ over the randomness of the masking, no polynomial-time algorithm can solve the resulting subproblem of predicting any of the masked tokens among $x^{\pi(1)},\ldots,x^{\pi(N)}$ given $x[M]$.
\end{proposition}


The complete proof of the proposition is given in \Cref{app:planted_result} but here we provide an overview of techniques used to prove the above result. 

\paragraph{Proof overview.} To understand the proof idea, we consider the case where all the latent tokens are masked and some of the observation tokens are unmasked. In this case, the prediction task reduces to learning to recover the latent tokens that are consistent with the observations. Intuitively, each observation provides some constraints and the task is to recover an assignment that satisfies the constraints. This is reminiscent of \emph{Constraint Satisfaction Problems} (CSPs). Indeed, to show the hardness result, we use the rich theory developed for \emph{planted} CSPs at the intersection of statistical physics and average-case complexity. 

In a planted CSP, there is an unknown randomly sampled vector $y$ of length $N$ and, one is given randomly chosen Boolean constraints %(e.g., $y^7 \wedge y^{8} \wedge y^{3} = 1$) 
which $y$ is promised to satisfy, and the goal is to recover $y$ as best as possible (see Definition~\ref{def:plantedcsp}). Prior works have shown the hardness of efficiently learning to solve the planted CSP problem \cite{krzakala2009hiding, alaoui2024hardness}. We show the hardness of masking problems in L\&O distributions based on these results. Consider the ground truth latent tokens as the random vector $y$ and each observation as a constraint. In this case, the problem of learning to recover the latent tokens from the observation tokens reduces to recovery for the planted CSP.


There are precise predictions for the values of vocabulary size $m$ and the number of observations for which the information-theoretically best possible overlap and the best overlap achievable by any computationally efficient algorithm are different. We show that these predictions directly translate to predictions about when masking problems become computationally intractable:

\begin{figure}
    \centering
    \hspace{-5mm}
    \includegraphics[width=0.8\linewidth]{Figures/bp_fig_1.png}
    \vspace{-0.05in}
    \caption{Overlap achieved by belief propagation initialized at ground truth versus random for planted CSP with $k = 3$, $m = 3$, and $g = \mathrm{NAE}$, for $N = 10000$ and varying choices of average degree $D$. $D_{\rm KS} / K$ can be shown analytically to be $64$, consistent with the phase transition depicted. Plot suggests $D_{\rm cond}/K \approx 50$. By Prop.~\ref{prop:csp} this implies a range of masking fractions at which $\Omega(1)$ fraction of masking problems are computationally hard.}
    \label{fig:csp}
\end{figure}

As a simple example, let us consider sparse predicate observations with $k=2$ and $g(x',x'') = \mathbf{1}[x' \neq x'']$. These can be formally related to the well-studied problem of \emph{planted $m$-coloring}. In the planted $m$-coloring, a random graph of average degree $D$ is sampled consistent with an unknown vertex coloring and the goal is to estimate the coloring as well as possible~\cite{krzakala2009hiding}, as measured by the \emph{overlap} of the output of the algorithm to the ground-truth coloring (see Definition~\ref{def:plantedcsp}). As a corollary of our main result, we show that when all the latent tokens $x^{\pi(1)}, \ldots, x^{\pi(N)}$ are masked and a few unmasked observation tokens provide the information of the form $g(x^{\pi(i)}, x^{\pi(j)}) = \mathbf{1}[ x^{\pi(i)} \neq x^{\pi(j)} ]$ for $i, j \leq N$, then solving the masking problem can be reduced to solving planted coloring. 


For planted $m$-coloring, when $m = 5$ the thresholds in Proposition~\ref{prop:csp} are given by $D_{\rm KS} / 2 = 16$ and $D_{\rm cond} / 2 \approx 13.23$~\cite{krzakala2009hiding} (the factor of $2$ here is simply because the observations correspond to \emph{ordered} subsets of size $2$). For general predicates and arities, there is an established recipe for numerically computing $D_{\rm KS}$ and $D_{\rm cond}$ based on the behavior of the \emph{belief propagation} algorithm (see the discussion in Appendix~\ref{app:planted_result}). As an example, in Fig.~\ref{fig:csp}, we execute this recipe for $m = 3$, $k = 3$, and $g$ given by the Not-All-Equal predicate $\mathrm{NAE}(x',x'',x'') = 1 - \mathbf{1}[x' = x'' = x''']$ to obtain thresholds that can be plugged into Proposition~\ref{prop:csp}.

\paragraph{Additional examples of the hardness.} The above setup can also be generalized to capture \emph{Bayesian constraint satisfaction problems}~\cite{montanari2008estimating,liu2022statistical}, one notable example of which is the stochastic block model~\cite{PhysRevE.84.066106}. There are analogous predictions for the onset of hardness of inference, which can likewise be translated to hardness of masking problems for seemingly benign L\&O distributions.

In Appendix~\ref{app:parity} and~\ref{app:slab}, we give two more examples of L\&O distributions for which order-aware training is tractable yet order-agnostic training of the MDM is computationally hard. First, we consider L\&O distributions whose observations are sparse, noisy parities in the latents and deduce hardness for order-agnostic training from the Sparse Learning Parity with Noise assumption~\cite{alekhnovich2003more}. We then consider L\&O distributions whose observations are \emph{generalized linear models} in the latents, and deduce hardness for a large class of efficient algorithms from existing results on Lipschitz hardness~\cite{alaoui2024hardness} for the symmetric binary perceptron~\cite{aubin2019storage}.


\subsection{Empirical evidence of hardness via likelihoods}
\label{sec:hardness_text}
Recent studies \cite{nie2024scaling, zheng2024maskeddiffusionmodelssecretly} have shown that masked diffusion models (MDMs) underperform compared to autoregressive models (ARMs) on natural text data. In this section, we provide evidence that this performance gap is primarily due to the order-agnostic training of MDMs. Since natural text follows a left-to-right token order, we demonstrate that as training deviates from this order, model performance gradually deteriorates. 

To understand the importance of the order during the training, we use the following setting: Given a permutation $\pi$ of indices $\{0,1, \ldots, L-1 \}$, define a \emph{$\pi$-learner} to be a likelihood model $\log p_{\theta}(x_0)$ given as follows: 
\begin{equation}
\label{eq:pi-learner-likelihood}
    \log p_{\theta}(x_0) = \sum_{i=0}^{L-1} \log p_\theta \bigl( x_0^{\pi(i)} \Big| x_0 [\pi\{i,\ldots,L-1\}] \bigr)\,
\end{equation}
In other words, the $\pi$-learner predicts the token at position $\pi(i)$ given the clean tokens $x_0^{\pi(0)},\ldots, x_0^{\pi(i-1)}$ and masked tokens $x_0^{\pi(i)},\ldots, x_0^{\pi(L-1)}$. If $\pi$ is the identity permutation, this reduces to the standard (left-to-right) autoregressive model. Note that the MDM loss encodes a $\pi$-learner for every permutation $\pi$ because
the MDM loss~\eqref{eqn:mdm_loss} is equivalent to the average loss of those $\pi$-learners over $\pi$ sampled from $\mathrm{Unif}(\mathbb{S}_L)$:

{\small
\begin{align*}
    \mathcal{L}_\theta &=-\mathbb{E}_{\pi,x_0\sim p_{\rm{data}}}\left[\sum_{i=0}^{L-1} \log p_\theta \left( x_0^{\pi(i)} \Big| x_0 [\pi\{i,\ldots,L-1\}] \right) \right]
\end{align*}}

where $\mathbb{S}_L$ denotes the set of all permutations over $\{0, 1, \ldots, L-1\}$ (The proof of the above equivalence is given in \cref{appenix:mdm-equivalent-loss}). Therefore, by measuring the `hardness' of each $\pi$-learner, we can probe differences in hardness between arbitrary masking problems and left-to-right masking problems.

\paragraph{Experimental setup.} We use the Slimpajama dataset \cite{soboleva2023slimpajama} to evaluate the performance of training in different orders. To train a $\pi$-learner, we employ a transformer with causal attention and use permuted data $\pi(x_0)$ as input. By varying $\pi$ while maintaining all other training configurations (e.g., model, optimization), we can use the resulting likelihood (computed using \cref{eq:pi-learner-likelihood}) as a metric to capture the hardness of subproblems solved by the $\pi$-learner.

In our experiments, the sequence length $L$ is approximately $10^3$, so repeating the above for each $\pi$ is infeasible. Instead, we sample $\pi \sim \mathrm{Unif}(\mathbb{S}_L)$ and examine the scaling law of the $\pi$-learner's likelihood. We leverage the codebase from \cite{nie2024scaling}, where the baseline scaling laws of MDM and ARM were introduced. Moreover, given that RoPE has an inductive bias towards left-to-right ordering, we employ a learnable positional embedding layer for all experiments to correct this. Consequently, we also re-run the baseline results, where RoPE was employed. To investigate how the distance between $\pi$ and the identity permutation affects the scaling law, we sample $\pi$ from other distributions interpolating between $\mathrm{Unif}(\mathbb{S}_L)$ and the point mass at the identical permutation. Further experimental details are provided in Appendix~\ref{appendix:exp_detail_text}.


\paragraph{Results.} As shown in Fig.~\ref{fig:scaling_laws}, the scaling law for a $\pi$-learner with uniformly random $\pi$ is worse than that of an ARM. This elucidates the inherent hardness of masking problems \( p_\theta(x_i \mid x_0[M]) \) beyond left-to-right prediction and also explains why MDM, which is trained simultaneously on all $\pi \in \mathbb{S}_L$, is worse than ARM in likelihood modeling. Additionally, as $\pi$ gets closer to the identity permutation, the scaling laws also get closer to ARM ($\pi$-learner-closer and $\pi$-learner-much-closer in Fig.~\ref{fig:scaling_laws}). This also supports the common belief that ARM is a good fit for text data as it inherently follows a \emph{left-to-right} ordering.


That said, it should also be noted that even though MDMs are trained on exponentially more masking problems than ARM ($\Theta(L2^L)$ versus $L$), its performance is not significantly worse than $\pi$-learners. We attribute this to the \emph{blessing of task diversity};  multi-task training can benefit both the optimization dynamics \cite{kim2024task} and validation performance \cite{tripuraneni2022provablem,andreas2016benefit,ruder2017overview} due to positive transfers across tasks.


\subsection{Error is imbalanced across masking problems}
\label{sec:imbalance_error}
In previous sections, we have demonstrated that the hardness of different masking problems \( p_\theta(x^i \mid x_0[M]) \) can vary significantly, potentially hindering the MDM's learning. In this section, we provide empirical evidence that the MDM's final performance exhibits a similar imbalance across subproblems. Details are provided in App.~\ref{appendix:exp_detail_3_3}.

\paragraph{L\&O-NAE-SAT .}
Consider an L\&O distribution with $\pi$ given by the identity permutation and where each observation $\mathcal{O}_j$ is deterministically given by $\mathrm{NAE}(x_{i_1},x_{i_2},x_{i_3}) \triangleq 1 - \mathbf{1}[x_{i_1} = x_{i_2} = x_{i_3}]$ for some randomly chosen (prefixed) triples $(i_1,i_2,i_3) \in[N]$.

For an MDM trained on this distribution, we measure the error it achieves on each task $\log p_\theta(x_0 | x_0[M])$ via $ \mathbb{E}_{x_0} \Bigl \| \log p_\theta(x_0 | x_0[M])-  \log p_{\rm data}(x_0 | x_0[M]) \Bigr\|^2$,
where $p_{\rm data}(x_0 | x_0[M])$ denotes the Bayes-optimal predictor.
Technically, we do not have access to this, so instead we train another MDM for a much larger number of iterations and use this as a proxy. Fig.~\ref{fig:scaling_laws} reveals that prediction tasks for latent positions (light region) exhibit larger errors compared to those for observation positions (dark region). 
 
\paragraph{Text.} 
Here we revisit the text experiment from Section~\ref{sec:hardness_text}. Since we do not have access to the Bayes-optimal predictor, we use the metric
{\small
$
    \mathbb{E}_{x_0 \sim p_{\rm{data}}}\left[\sum_{i=0}^{L-1} \log p_\theta \left( x_0^{\pi(i)} \Big| x_0 [\pi\{i,\ldots,L-1\}] \right) \right]
$}.

This captures the accumulation of error across subproblems $p_\theta \left( x_0^{\pi(i)} \Big| x_0 [\pi\{i,\ldots,L-1\}] \right)$, since $p_\theta(x_0 | x_0[M]) = p_{\rm{data}}(x_0 | x_0[M])$ minimizes this metric. Fig.~\ref{fig:scaling_laws} shows a clear gap between different subproblems.


The theoretical and empirical evidence demonstrates that MDMs perform better in estimating $p_{\theta}(x_0 | x_0[M])$ for some subproblems $M$ than for others. We therefore want to avoid encountering hard subproblems $M$ at inference time. In the next section, we show that while vanilla MDM inference can run into such subproblems, simple modifications at the inference stage can effectively circumvent these issues, resulting in dramatic, \emph{training-free} performance improvements.


\section{MDMs can plan around hard problems} \label{sec:inference}
We previously argued that due to the complex nature of masking subproblems, MDM must perform poorly on certain ones $p_\theta(x^i | x_t)$. Therefore, during vanilla MDM inference,
MDM inevitably encounters such difficult subproblems at Step (b). While this might suggest that we need to fundamentally revisit how MDMs are trained, in this section we show that, surprisingly, simple modifications at the inference stage—\emph{without any further training}—can sidestep these issues and lead to significant performance improvements.

\paragraph{MDM offers multiple sampling paths.}


\begin{table}[t]
    \centering 
    \caption{\textbf{L\&O-NAE-SAT}. Adaptive MDM inference achieves better likelihood matching than vanilla MDM inference. Note that naive guessing leads to $75\%$ accuracy, indicating that vanilla inference performs similar or worse than naive guessing.}
    \vspace{0.1in}
    \begin{tabular}{c c c}
        \toprule
        \textbf{($N,P$)} & \textbf{Vanilla inference} & \textbf{Adaptive inference}\\
        \midrule
        $(25,275)$  & 78.06\%   & 93.76\%  \\
        $(30,270)$  & 75.70\% & 93.54\% \\
        $(40,260)$  & 74.60\%   & 92.21\%  \\
        $(50,250)$  & 67.94\% & 90.01\% \\
        $(100,200)$ & 62.84\% & 88.91\% \\
        \bottomrule
    \end{tabular}
    \label{tab:csp_sampler}
\end{table}

The vanilla MDM inference (Algorithm~\ref{alg:random_sampler_redefine}) aim to align the intermediate distributions with the forward process, as used in continuous diffusion. However, unlike continuous diffusion, the reverse process of MDM allows multiple valid sampling paths (different orders of unmasking the tokens) that match the starting distribution of the forward process of MDM. 

We first show that when we have an ideal MDM that perfectly solves all masking problems, i.e., $p_\theta(x_0^i | x_0[M]) = p_{\rm{data}}(x_0^i | x_0[M])$, then using any sampling path (unmasking the tokens in any order) results in the same distribution. Consider the following sampler: For every step, $S$ is a set with one index selected agnostically (without following any distribution). For any clean sample $x_0$ generated by this sampler, note that $p_\theta(x_0) = \prod_{i=0}^{L-1}  p_\theta \left( x_0^{\pi(i)} \Big| x_0 [\pi\{i,\ldots,L-1\}] \right)$ by chain rule, and this is equal to $\prod_{i=0}^{L-1}  p_{\rm{data}} \left( x_0^{\pi(i)} \Big| x_0 [\pi\{i,\ldots,L-1\}] \right) = p_{\rm{data}}(x_0)$.
Therefore, other choices of $S$, not necessarily following Algorithm~\ref{alg:random_sampler_redefine}, still capture the true likelihood.

\begin{figure}[t]
    \centering
    \includegraphics[width=0.9\linewidth]{Figures/perplexity_vs_sampling_steps_with_entropy_1.1B.png}
    \vspace{-0.1in}
    \caption{\textbf{Generative Perplexity.} We compare the resulting generative perplexity (GenPPL) of adaptive vs. vanilla MDM inference. We employ a pretrained $170$M MDM and LLaMA-$7$B \cite{touvron2023llama2} as inference and evaluation, respectively. Adaptive MDM inference {(\color{blue} Blue)} leads to a substantial reduction in generative perplexity, while maintaining the entropy.}
    \label{fig:genppl}
\end{figure}

In practice, unlike this ideal case, MDM does not perform equally well on all subproblems, as shown in Section~\ref{sec:imbalance_error}. Consequently, different sampling paths result in varying likelihood modeling abilities. Motivated by this observation, we consider \emph{adaptive inference for MDMs}:

\vspace{0.05in}
\begin{theo}[Adaptive MDM inference]{alg:sampling_adaptive}
\vspace{-0.1in}
\begin{itemize}
    \item[(a)] Sample a set of masked tokens $\mathcal{S} = \mathcal{F}\left( \theta, x_t \right) \subseteq \{i \mid x_t^i = \mask\}$.
    \item[(b)] For each $i \in \mathcal{S}$, sample $x_s^i \sim p_\theta(x^i | x_t)$.
\end{itemize}
\end{theo}

Instead of selecting $S$ randomly, adaptive MDM inference leverages an oracle $\mathcal{F}(\theta, x_t)$ to select $S$ strategically to avoid hard masking problems. This naturally raises the question of how to design an effective oracle $\mathcal{F}$.

In the following sections, we demonstrate that adaptive MDM inference with careful choices of $\mathcal{F}$ enhance MDM's likelihood matching ability. In other words, a pretrained MDM, even if it performs poorly on certain hard subproblems, \underline{\textbf{still contains sufficient information to avoid them}} when paired with an effective oracle $\mathcal{F}$.

\subsection{Effective design of ordering oracle}
\label{subsec:effective-design}
We introduce two different oracles, \topk and Top-$K$ probability margin. Intuitively, both strategies are based on the idea that $S$ should be selected based on how ``certain'' the model is about each position. We caution that these strategies should not be confused with notions like nucleus sampling in ARMs~\cite{holtzman2019curious}; the oracles we describe are for selecting the \emph{position} of the next token to decode, rather than the \emph{value}, and thus are only meaningful in the context of MDMs.

\textbf{\topk probability~\cite{zheng2024reparameterized}.} Suppose we want to unmask $K$ positions at time step $t$, i.e., select $|S|=K$. In the \topk strategy, the uncertainty of a position is estimated by the maximum probability assigned to any value in the vocabulary. More precisely, the certainty at position $i$ is $\max_{j \in \{ 0, \ldots, m-1 \} } p_\theta(x^i = j | x_t)$ and \(\mathcal{F}(\theta, x_t) = \text{Top } K \left(\max p_\theta(x^i | x_t) \right)\). 

\topk strategy is a good proxy for many tasks and works well in practice \cite{zheng2024reparameterized,ye2024beyond,wang2024diffusion}. However, this approach can often provide misleading estimates of uncertainty. Consider when an MDM is confused between two token values, thus assigning them almost equal but high probabilities.  In this case, \topk strategy may still choose to unmask this position, despite its uncertainty. To mitigate this issue, we propose the following alternative strategy.

\textbf{Top-$K$ probability margin.} In this strategy, the uncertainty of a position is instead estimated using the absolute difference between the two most probable values at position $i$. More precisely, if $j_1$ and $j_2$ are the two most probable values in vocabulary according to $p_\theta(x^i | x_t)$ in position $i$, the certainty in the position is given by $| p_\theta(x^i = j_1 | x_t) - p_\theta(x^i = j_2 | x_t) |$  and \(\mathcal{F}(\theta, x_t) = \text{Top } K \left(| p_\theta(x^i = j_1 | x_t) - p_\theta(x^i = j_2 | x_t) | \right)\). When multiple values have similar probabilities at a position, \topkprobdiff will provide a better estimate of the uncertainty of a position, and when there is a single best choice of value then \topk and \topkprobdiff work similarly. 

\subsection{Adaptive MDM inference} \label{subsec:adaptive_inference}

In this section, we experimentally validate that adaptive MDM inference helps MDMs avoid hard subproblems, leading to better likelihood matching. We first show our results on L\&O-NAE-SAT and text data, before turning to our primary application to logic puzzles.

\textbf{L\&O-NAE-SAT and text data.} For the L\&O-NAE-SAT distribution defined in \cref{sec:imbalance_error}, we evaluate the effectiveness of adaptive inference by measuring the accuracy in predicting the observation tokens. Table~\ref{tab:csp_sampler} reveals a clear improvement over vanilla inference. For the text dataset, we evaluate using the standard metric of \emph{generative perplexity}, by which likelihood is measured by a large language model. We also compute the entropy of the generated samples to ensure both inference strategies exhibit similar levels of diversity. As shown in Fig.~\ref{fig:genppl}, we observe a substantial decrease in generative perplexity using adaptive inference. We defer further experimental details to Appendix~\ref{appendix:exp_detail_inference}.

\textbf{Logic puzzles.} We consider two different types of logic puzzles: Sudoku and Zebra (Einstein) puzzles. Intuitively, for Sudoku, some empty (masked) cells are significantly easier to predict than others and we want to choose the cells that are easier to predict during the inference. We evaluate the effectiveness of adaptive MDM inference over vanilla MDM inference in selecting such cells. 

To measure the performance of an inference method, we use the percentage of correctly solved puzzles. For both puzzles, we use train and test datasets from \cite{shah2024causal}. For the Sudoku puzzle (Table~\ref{tab:sudoku-results}) we observe that adaptive MDM inference, in particular Top-$K$ probability margin, obtains substantially higher accuracy (89.49\%) compared to vanilla MDM inference (6.88\%). Additionally, \topkprobdiff obtains higher accuracy (89.49\%) than \topk (18.51\%). As mentioned in \cref{subsec:effective-design}, this is because \topkprobdiff more reliably estimates uncertainty when multiple competing values are close in probability at a given position, as is often the case in Sudoku. For the Zebra puzzle, as shown in \Cref{tab:zebra-results}, we observe a consistent result: \topk (98.5\%) and \topkprobdiff (98.3\%) outperform vanilla MDM inference (76.9\%).

\begin{table}[t]
    \centering
    \caption{Comparison of accuracy for solving the Sudoku puzzle.}
    \vspace{0.1in}
    \begin{tabular}{l >{\centering\arraybackslash}p{1.4cm} c}  % Proper center alignment
        \toprule
        \textbf{Method} & \textbf{\# Param} & \textbf{Accuracy} \\
        \midrule
        ARM (w/o ordering) & \multirow{2}{*}{42M} & 9.73\% \\
        ARM (with ordering) &  & 87.18\% \\
        \midrule
        MDM (vanilla) & \multirow{3}{*}{6M} & 6.88\% \\
        MDM (Top-$K$ probability) &  & 18.51\% \\
        MDM (Top-$K$ prob. margin) &  & 89.49\% \\
        \bottomrule
    \end{tabular}
    \label{tab:sudoku-results}
\end{table}

\begin{table}[t]
    \centering
    \caption{Comparison of accuracy for solving the Zebra puzzle.}
    \vspace{0.1in}
    \begin{tabular}{l >{\centering\arraybackslash}p{1.4cm} c}  % Proper center alignment
            \toprule
        \textbf{Method} & \textbf{\# Param} & \textbf{Accuracy} \\
        \midrule
        ARM (w/o ordering) & \multirow{2}{*}{42M} & 80.31 \% \\
        ARM (with ordering) &  & 91.17 \% \\
        \midrule
        MDM (vanilla) & \multirow{3}{*}{19M} & 76.9 \% \\
        MDM (\topk probability) &  & 98.5 \% \\
        MDM (Top-$K$ prob. margin) &  & 98.3 \% \\
        \bottomrule
    \end{tabular}
    \label{tab:zebra-results}
\end{table}

\subsection{Eliciting sequence-dependent reasoning paths using adaptive MDM inference} 
\label{sec:sequence-dependent-tasks}

In this section, we study the effectiveness of adaptive MDM inference in finding the right reasoning/generation order for tasks where every sequence has a different ``natural'' order. To do so, we will compare the performance of adaptive MDM inference to that of ARM on Sudoku and Zebra puzzles. For these puzzles, the natural order of generation is not only different from left-to-right, but it is also sequence-dependent. For such tasks, prior works have shown that ARMs struggle if the information about the order is not provided during the training \cite{shah2024causal, lehnert2024beyond}. Therefore, to obtain a strong baseline, we not only consider an ARM trained without the order information but also consider an ARM trained with the order information for each sequence in the training data. Note that the latter is a much stronger baseline than the former as one can hope to teach the model to figure out the correct order by some form of supervised teacher forcing (as performed in \citet{shah2024causal, lehnert2024beyond}), eliminating the issue of finding the right order in an unsupervised manner. 

We compare ARMs and MDMs for Sudoku in \cref{tab:sudoku-results} and Zebra puzzles in \cref{tab:zebra-results}.\footnote{A prior work \cite{ye2024beyond} reported that a $6$M MDM with \topk inference achieves 100\% accuracy on Sudoku. Given that a 6M MDM with \topk only achieves 18.51\% on our dataset (Table~\ref{tab:sudoku-results}), this suggests that the Sudoku dataset in~\cite{ye2024beyond} is significantly easier than ours.} We observe that for both, Top-$K$ probability margin-based adaptive MDM inference not only outperforms the ARM trained without ordering information, but it \emph{even outperforms the ARM trained with ordering information}! This shows that the \emph{unsupervised} way of finding the correct order and solving such logic puzzles using adaptive MDM inference outperforms the \emph{supervised} way of finding the correct order and solving such puzzles using an ARM, and is significantly less computationally intensive.

\subsection{Easy to hard generalization}

In the previous section we showed that when the training and inference sequences come from the same distribution, order-agnostic training of MDMs combined with adaptive inference can perform very well on logic puzzles. To evaluate if the model has learned the correct way of solving the puzzles and test the robustness of adaptive inference, we also test the MDMs on harder puzzles than the ones from training, for Sudoku. 

We keep the training dataset the same as proposed in \citet{shah2024causal}. \citet{shah2024causal} created this dataset from \citet{david_g__radcliffe_2020} by selecting the puzzles that can be solved using 7 fixed strategies and do not require backtracking-based search. We use the remaining puzzles in \citet{david_g__radcliffe_2020} as our hard dataset. Hence, these puzzles all use a strategy not seen during training and/or backtracking to obtain the correct solution.  

\begin{table}[t]
    \centering
    \caption{Comparison of accuracy for solving the hard Sudokus.}
    \vspace{0.1in}
    \begin{tabular}{l >{\centering\arraybackslash}p{1.4cm} c}
    \toprule
        \textbf{Method} & \textbf{$\#$Param} & \textbf{Accuracy} \\
        \midrule
         ARM (with ordering) & 42M & 32.57 \% \\
         \midrule
         MDM (random) & \multirow{3}{*}{6M} & 3.62 \% \\
         MDM (Top-$K$ probability) & & 9.44 \% \\
         MDM (Top-$K$ prob. margin) &  & 49.88 \% \\
         \bottomrule
    \end{tabular}
    \label{tab:easy-to-hard-sudoku}
\end{table}


We measure the accuracy of MDMs and ARMs on the hard test set and present the results in \cref{tab:easy-to-hard-sudoku}. We see that the Top-$K$ probability margin-based adaptive MDM inference strategy (49.88\%) again significantly outperforms ARMs trained with order information (32.57\%). In particular, although the accuracy drops for both methods due to the more challenging test set, MDMs with adaptive inference appear to be more robust to this distribution shift than ARMs. We believe this is due to the fact that MDMs try to solve a significantly higher number of infilling problems than ARMs ($\exp(L)$ compared to $L$) and therefore are able to extract knowledge about the problem more efficiently than ARMs. 

\section{Conclusion}

In this work, we examined the impact of token ordering on training and inference in MDMs. We provided theoretical and experimental evidence that MDMs train on hard masking problems. We also demonstrated that adaptive inference strategies can be used to sidestep these hard problems. For logic puzzles, we find that this leads to dramatic improvements in performance not just over vanilla MDMs, but even over ARMs trained with teacher forcing to learn the right order of decoding.

An important direction for future work is to explore settings beyond logic puzzles where adaptive inference can help MDMs match or surpass ARMs. For these, it may be crucial to go beyond the relatively simple adaptive strategies like \topk and \topkprobdiff considered here.

\paragraph{Acknowledgements.} JK thanks Kiwhan Song for discussions about MDM training. KS and VK are supported by the NSF AI Institute for Foundations of Machine Learning (IFML). KS thanks Nishanth Dikkala for the initial discussions about the project. SC is supported by the Harvard Dean's Competitive Fund for Promising Scholarship and thanks Brice Huang and Sidhanth Mohanty for enlightening discussions about computational-statistical tradeoffs for planted CSPs. 

\section*{Impact statement}
This paper advances the understanding of discrete diffusion models, contributing to the broader field of Machine Learning. There are many potential societal consequences of our work, none of which we feel must be specifically highlighted here.

\bibliography{main}
\bibliographystyle{icml2025}

\newpage
\appendix
\onecolumn
\subsection{Lloyd-Max Algorithm}
\label{subsec:Lloyd-Max}
For a given quantization bitwidth $B$ and an operand $\bm{X}$, the Lloyd-Max algorithm finds $2^B$ quantization levels $\{\hat{x}_i\}_{i=1}^{2^B}$ such that quantizing $\bm{X}$ by rounding each scalar in $\bm{X}$ to the nearest quantization level minimizes the quantization MSE. 

The algorithm starts with an initial guess of quantization levels and then iteratively computes quantization thresholds $\{\tau_i\}_{i=1}^{2^B-1}$ and updates quantization levels $\{\hat{x}_i\}_{i=1}^{2^B}$. Specifically, at iteration $n$, thresholds are set to the midpoints of the previous iteration's levels:
\begin{align*}
    \tau_i^{(n)}=\frac{\hat{x}_i^{(n-1)}+\hat{x}_{i+1}^{(n-1)}}2 \text{ for } i=1\ldots 2^B-1
\end{align*}
Subsequently, the quantization levels are re-computed as conditional means of the data regions defined by the new thresholds:
\begin{align*}
    \hat{x}_i^{(n)}=\mathbb{E}\left[ \bm{X} \big| \bm{X}\in [\tau_{i-1}^{(n)},\tau_i^{(n)}] \right] \text{ for } i=1\ldots 2^B
\end{align*}
where to satisfy boundary conditions we have $\tau_0=-\infty$ and $\tau_{2^B}=\infty$. The algorithm iterates the above steps until convergence.

Figure \ref{fig:lm_quant} compares the quantization levels of a $7$-bit floating point (E3M3) quantizer (left) to a $7$-bit Lloyd-Max quantizer (right) when quantizing a layer of weights from the GPT3-126M model at a per-tensor granularity. As shown, the Lloyd-Max quantizer achieves substantially lower quantization MSE. Further, Table \ref{tab:FP7_vs_LM7} shows the superior perplexity achieved by Lloyd-Max quantizers for bitwidths of $7$, $6$ and $5$. The difference between the quantizers is clear at 5 bits, where per-tensor FP quantization incurs a drastic and unacceptable increase in perplexity, while Lloyd-Max quantization incurs a much smaller increase. Nevertheless, we note that even the optimal Lloyd-Max quantizer incurs a notable ($\sim 1.5$) increase in perplexity due to the coarse granularity of quantization. 

\begin{figure}[h]
  \centering
  \includegraphics[width=0.7\linewidth]{sections/figures/LM7_FP7.pdf}
  \caption{\small Quantization levels and the corresponding quantization MSE of Floating Point (left) vs Lloyd-Max (right) Quantizers for a layer of weights in the GPT3-126M model.}
  \label{fig:lm_quant}
\end{figure}

\begin{table}[h]\scriptsize
\begin{center}
\caption{\label{tab:FP7_vs_LM7} \small Comparing perplexity (lower is better) achieved by floating point quantizers and Lloyd-Max quantizers on a GPT3-126M model for the Wikitext-103 dataset.}
\begin{tabular}{c|cc|c}
\hline
 \multirow{2}{*}{\textbf{Bitwidth}} & \multicolumn{2}{|c|}{\textbf{Floating-Point Quantizer}} & \textbf{Lloyd-Max Quantizer} \\
 & Best Format & Wikitext-103 Perplexity & Wikitext-103 Perplexity \\
\hline
7 & E3M3 & 18.32 & 18.27 \\
6 & E3M2 & 19.07 & 18.51 \\
5 & E4M0 & 43.89 & 19.71 \\
\hline
\end{tabular}
\end{center}
\end{table}

\subsection{Proof of Local Optimality of LO-BCQ}
\label{subsec:lobcq_opt_proof}
For a given block $\bm{b}_j$, the quantization MSE during LO-BCQ can be empirically evaluated as $\frac{1}{L_b}\lVert \bm{b}_j- \bm{\hat{b}}_j\rVert^2_2$ where $\bm{\hat{b}}_j$ is computed from equation (\ref{eq:clustered_quantization_definition}) as $C_{f(\bm{b}_j)}(\bm{b}_j)$. Further, for a given block cluster $\mathcal{B}_i$, we compute the quantization MSE as $\frac{1}{|\mathcal{B}_{i}|}\sum_{\bm{b} \in \mathcal{B}_{i}} \frac{1}{L_b}\lVert \bm{b}- C_i^{(n)}(\bm{b})\rVert^2_2$. Therefore, at the end of iteration $n$, we evaluate the overall quantization MSE $J^{(n)}$ for a given operand $\bm{X}$ composed of $N_c$ block clusters as:
\begin{align*}
    \label{eq:mse_iter_n}
    J^{(n)} = \frac{1}{N_c} \sum_{i=1}^{N_c} \frac{1}{|\mathcal{B}_{i}^{(n)}|}\sum_{\bm{v} \in \mathcal{B}_{i}^{(n)}} \frac{1}{L_b}\lVert \bm{b}- B_i^{(n)}(\bm{b})\rVert^2_2
\end{align*}

At the end of iteration $n$, the codebooks are updated from $\mathcal{C}^{(n-1)}$ to $\mathcal{C}^{(n)}$. However, the mapping of a given vector $\bm{b}_j$ to quantizers $\mathcal{C}^{(n)}$ remains as  $f^{(n)}(\bm{b}_j)$. At the next iteration, during the vector clustering step, $f^{(n+1)}(\bm{b}_j)$ finds new mapping of $\bm{b}_j$ to updated codebooks $\mathcal{C}^{(n)}$ such that the quantization MSE over the candidate codebooks is minimized. Therefore, we obtain the following result for $\bm{b}_j$:
\begin{align*}
\frac{1}{L_b}\lVert \bm{b}_j - C_{f^{(n+1)}(\bm{b}_j)}^{(n)}(\bm{b}_j)\rVert^2_2 \le \frac{1}{L_b}\lVert \bm{b}_j - C_{f^{(n)}(\bm{b}_j)}^{(n)}(\bm{b}_j)\rVert^2_2
\end{align*}

That is, quantizing $\bm{b}_j$ at the end of the block clustering step of iteration $n+1$ results in lower quantization MSE compared to quantizing at the end of iteration $n$. Since this is true for all $\bm{b} \in \bm{X}$, we assert the following:
\begin{equation}
\begin{split}
\label{eq:mse_ineq_1}
    \tilde{J}^{(n+1)} &= \frac{1}{N_c} \sum_{i=1}^{N_c} \frac{1}{|\mathcal{B}_{i}^{(n+1)}|}\sum_{\bm{b} \in \mathcal{B}_{i}^{(n+1)}} \frac{1}{L_b}\lVert \bm{b} - C_i^{(n)}(b)\rVert^2_2 \le J^{(n)}
\end{split}
\end{equation}
where $\tilde{J}^{(n+1)}$ is the the quantization MSE after the vector clustering step at iteration $n+1$.

Next, during the codebook update step (\ref{eq:quantizers_update}) at iteration $n+1$, the per-cluster codebooks $\mathcal{C}^{(n)}$ are updated to $\mathcal{C}^{(n+1)}$ by invoking the Lloyd-Max algorithm \citep{Lloyd}. We know that for any given value distribution, the Lloyd-Max algorithm minimizes the quantization MSE. Therefore, for a given vector cluster $\mathcal{B}_i$ we obtain the following result:

\begin{equation}
    \frac{1}{|\mathcal{B}_{i}^{(n+1)}|}\sum_{\bm{b} \in \mathcal{B}_{i}^{(n+1)}} \frac{1}{L_b}\lVert \bm{b}- C_i^{(n+1)}(\bm{b})\rVert^2_2 \le \frac{1}{|\mathcal{B}_{i}^{(n+1)}|}\sum_{\bm{b} \in \mathcal{B}_{i}^{(n+1)}} \frac{1}{L_b}\lVert \bm{b}- C_i^{(n)}(\bm{b})\rVert^2_2
\end{equation}

The above equation states that quantizing the given block cluster $\mathcal{B}_i$ after updating the associated codebook from $C_i^{(n)}$ to $C_i^{(n+1)}$ results in lower quantization MSE. Since this is true for all the block clusters, we derive the following result: 
\begin{equation}
\begin{split}
\label{eq:mse_ineq_2}
     J^{(n+1)} &= \frac{1}{N_c} \sum_{i=1}^{N_c} \frac{1}{|\mathcal{B}_{i}^{(n+1)}|}\sum_{\bm{b} \in \mathcal{B}_{i}^{(n+1)}} \frac{1}{L_b}\lVert \bm{b}- C_i^{(n+1)}(\bm{b})\rVert^2_2  \le \tilde{J}^{(n+1)}   
\end{split}
\end{equation}

Following (\ref{eq:mse_ineq_1}) and (\ref{eq:mse_ineq_2}), we find that the quantization MSE is non-increasing for each iteration, that is, $J^{(1)} \ge J^{(2)} \ge J^{(3)} \ge \ldots \ge J^{(M)}$ where $M$ is the maximum number of iterations. 
%Therefore, we can say that if the algorithm converges, then it must be that it has converged to a local minimum. 
\hfill $\blacksquare$


\begin{figure}
    \begin{center}
    \includegraphics[width=0.5\textwidth]{sections//figures/mse_vs_iter.pdf}
    \end{center}
    \caption{\small NMSE vs iterations during LO-BCQ compared to other block quantization proposals}
    \label{fig:nmse_vs_iter}
\end{figure}

Figure \ref{fig:nmse_vs_iter} shows the empirical convergence of LO-BCQ across several block lengths and number of codebooks. Also, the MSE achieved by LO-BCQ is compared to baselines such as MXFP and VSQ. As shown, LO-BCQ converges to a lower MSE than the baselines. Further, we achieve better convergence for larger number of codebooks ($N_c$) and for a smaller block length ($L_b$), both of which increase the bitwidth of BCQ (see Eq \ref{eq:bitwidth_bcq}).


\subsection{Additional Accuracy Results}
%Table \ref{tab:lobcq_config} lists the various LOBCQ configurations and their corresponding bitwidths.
\begin{table}
\setlength{\tabcolsep}{4.75pt}
\begin{center}
\caption{\label{tab:lobcq_config} Various LO-BCQ configurations and their bitwidths.}
\begin{tabular}{|c||c|c|c|c||c|c||c|} 
\hline
 & \multicolumn{4}{|c||}{$L_b=8$} & \multicolumn{2}{|c||}{$L_b=4$} & $L_b=2$ \\
 \hline
 \backslashbox{$L_A$\kern-1em}{\kern-1em$N_c$} & 2 & 4 & 8 & 16 & 2 & 4 & 2 \\
 \hline
 64 & 4.25 & 4.375 & 4.5 & 4.625 & 4.375 & 4.625 & 4.625\\
 \hline
 32 & 4.375 & 4.5 & 4.625& 4.75 & 4.5 & 4.75 & 4.75 \\
 \hline
 16 & 4.625 & 4.75& 4.875 & 5 & 4.75 & 5 & 5 \\
 \hline
\end{tabular}
\end{center}
\end{table}

%\subsection{Perplexity achieved by various LO-BCQ configurations on Wikitext-103 dataset}

\begin{table} \centering
\begin{tabular}{|c||c|c|c|c||c|c||c|} 
\hline
 $L_b \rightarrow$& \multicolumn{4}{c||}{8} & \multicolumn{2}{c||}{4} & 2\\
 \hline
 \backslashbox{$L_A$\kern-1em}{\kern-1em$N_c$} & 2 & 4 & 8 & 16 & 2 & 4 & 2  \\
 %$N_c \rightarrow$ & 2 & 4 & 8 & 16 & 2 & 4 & 2 \\
 \hline
 \hline
 \multicolumn{8}{c}{GPT3-1.3B (FP32 PPL = 9.98)} \\ 
 \hline
 \hline
 64 & 10.40 & 10.23 & 10.17 & 10.15 &  10.28 & 10.18 & 10.19 \\
 \hline
 32 & 10.25 & 10.20 & 10.15 & 10.12 &  10.23 & 10.17 & 10.17 \\
 \hline
 16 & 10.22 & 10.16 & 10.10 & 10.09 &  10.21 & 10.14 & 10.16 \\
 \hline
  \hline
 \multicolumn{8}{c}{GPT3-8B (FP32 PPL = 7.38)} \\ 
 \hline
 \hline
 64 & 7.61 & 7.52 & 7.48 &  7.47 &  7.55 &  7.49 & 7.50 \\
 \hline
 32 & 7.52 & 7.50 & 7.46 &  7.45 &  7.52 &  7.48 & 7.48  \\
 \hline
 16 & 7.51 & 7.48 & 7.44 &  7.44 &  7.51 &  7.49 & 7.47  \\
 \hline
\end{tabular}
\caption{\label{tab:ppl_gpt3_abalation} Wikitext-103 perplexity across GPT3-1.3B and 8B models.}
\end{table}

\begin{table} \centering
\begin{tabular}{|c||c|c|c|c||} 
\hline
 $L_b \rightarrow$& \multicolumn{4}{c||}{8}\\
 \hline
 \backslashbox{$L_A$\kern-1em}{\kern-1em$N_c$} & 2 & 4 & 8 & 16 \\
 %$N_c \rightarrow$ & 2 & 4 & 8 & 16 & 2 & 4 & 2 \\
 \hline
 \hline
 \multicolumn{5}{|c|}{Llama2-7B (FP32 PPL = 5.06)} \\ 
 \hline
 \hline
 64 & 5.31 & 5.26 & 5.19 & 5.18  \\
 \hline
 32 & 5.23 & 5.25 & 5.18 & 5.15  \\
 \hline
 16 & 5.23 & 5.19 & 5.16 & 5.14  \\
 \hline
 \multicolumn{5}{|c|}{Nemotron4-15B (FP32 PPL = 5.87)} \\ 
 \hline
 \hline
 64  & 6.3 & 6.20 & 6.13 & 6.08  \\
 \hline
 32  & 6.24 & 6.12 & 6.07 & 6.03  \\
 \hline
 16  & 6.12 & 6.14 & 6.04 & 6.02  \\
 \hline
 \multicolumn{5}{|c|}{Nemotron4-340B (FP32 PPL = 3.48)} \\ 
 \hline
 \hline
 64 & 3.67 & 3.62 & 3.60 & 3.59 \\
 \hline
 32 & 3.63 & 3.61 & 3.59 & 3.56 \\
 \hline
 16 & 3.61 & 3.58 & 3.57 & 3.55 \\
 \hline
\end{tabular}
\caption{\label{tab:ppl_llama7B_nemo15B} Wikitext-103 perplexity compared to FP32 baseline in Llama2-7B and Nemotron4-15B, 340B models}
\end{table}

%\subsection{Perplexity achieved by various LO-BCQ configurations on MMLU dataset}


\begin{table} \centering
\begin{tabular}{|c||c|c|c|c||c|c|c|c|} 
\hline
 $L_b \rightarrow$& \multicolumn{4}{c||}{8} & \multicolumn{4}{c||}{8}\\
 \hline
 \backslashbox{$L_A$\kern-1em}{\kern-1em$N_c$} & 2 & 4 & 8 & 16 & 2 & 4 & 8 & 16  \\
 %$N_c \rightarrow$ & 2 & 4 & 8 & 16 & 2 & 4 & 2 \\
 \hline
 \hline
 \multicolumn{5}{|c|}{Llama2-7B (FP32 Accuracy = 45.8\%)} & \multicolumn{4}{|c|}{Llama2-70B (FP32 Accuracy = 69.12\%)} \\ 
 \hline
 \hline
 64 & 43.9 & 43.4 & 43.9 & 44.9 & 68.07 & 68.27 & 68.17 & 68.75 \\
 \hline
 32 & 44.5 & 43.8 & 44.9 & 44.5 & 68.37 & 68.51 & 68.35 & 68.27  \\
 \hline
 16 & 43.9 & 42.7 & 44.9 & 45 & 68.12 & 68.77 & 68.31 & 68.59  \\
 \hline
 \hline
 \multicolumn{5}{|c|}{GPT3-22B (FP32 Accuracy = 38.75\%)} & \multicolumn{4}{|c|}{Nemotron4-15B (FP32 Accuracy = 64.3\%)} \\ 
 \hline
 \hline
 64 & 36.71 & 38.85 & 38.13 & 38.92 & 63.17 & 62.36 & 63.72 & 64.09 \\
 \hline
 32 & 37.95 & 38.69 & 39.45 & 38.34 & 64.05 & 62.30 & 63.8 & 64.33  \\
 \hline
 16 & 38.88 & 38.80 & 38.31 & 38.92 & 63.22 & 63.51 & 63.93 & 64.43  \\
 \hline
\end{tabular}
\caption{\label{tab:mmlu_abalation} Accuracy on MMLU dataset across GPT3-22B, Llama2-7B, 70B and Nemotron4-15B models.}
\end{table}


%\subsection{Perplexity achieved by various LO-BCQ configurations on LM evaluation harness}

\begin{table} \centering
\begin{tabular}{|c||c|c|c|c||c|c|c|c|} 
\hline
 $L_b \rightarrow$& \multicolumn{4}{c||}{8} & \multicolumn{4}{c||}{8}\\
 \hline
 \backslashbox{$L_A$\kern-1em}{\kern-1em$N_c$} & 2 & 4 & 8 & 16 & 2 & 4 & 8 & 16  \\
 %$N_c \rightarrow$ & 2 & 4 & 8 & 16 & 2 & 4 & 2 \\
 \hline
 \hline
 \multicolumn{5}{|c|}{Race (FP32 Accuracy = 37.51\%)} & \multicolumn{4}{|c|}{Boolq (FP32 Accuracy = 64.62\%)} \\ 
 \hline
 \hline
 64 & 36.94 & 37.13 & 36.27 & 37.13 & 63.73 & 62.26 & 63.49 & 63.36 \\
 \hline
 32 & 37.03 & 36.36 & 36.08 & 37.03 & 62.54 & 63.51 & 63.49 & 63.55  \\
 \hline
 16 & 37.03 & 37.03 & 36.46 & 37.03 & 61.1 & 63.79 & 63.58 & 63.33  \\
 \hline
 \hline
 \multicolumn{5}{|c|}{Winogrande (FP32 Accuracy = 58.01\%)} & \multicolumn{4}{|c|}{Piqa (FP32 Accuracy = 74.21\%)} \\ 
 \hline
 \hline
 64 & 58.17 & 57.22 & 57.85 & 58.33 & 73.01 & 73.07 & 73.07 & 72.80 \\
 \hline
 32 & 59.12 & 58.09 & 57.85 & 58.41 & 73.01 & 73.94 & 72.74 & 73.18  \\
 \hline
 16 & 57.93 & 58.88 & 57.93 & 58.56 & 73.94 & 72.80 & 73.01 & 73.94  \\
 \hline
\end{tabular}
\caption{\label{tab:mmlu_abalation} Accuracy on LM evaluation harness tasks on GPT3-1.3B model.}
\end{table}

\begin{table} \centering
\begin{tabular}{|c||c|c|c|c||c|c|c|c|} 
\hline
 $L_b \rightarrow$& \multicolumn{4}{c||}{8} & \multicolumn{4}{c||}{8}\\
 \hline
 \backslashbox{$L_A$\kern-1em}{\kern-1em$N_c$} & 2 & 4 & 8 & 16 & 2 & 4 & 8 & 16  \\
 %$N_c \rightarrow$ & 2 & 4 & 8 & 16 & 2 & 4 & 2 \\
 \hline
 \hline
 \multicolumn{5}{|c|}{Race (FP32 Accuracy = 41.34\%)} & \multicolumn{4}{|c|}{Boolq (FP32 Accuracy = 68.32\%)} \\ 
 \hline
 \hline
 64 & 40.48 & 40.10 & 39.43 & 39.90 & 69.20 & 68.41 & 69.45 & 68.56 \\
 \hline
 32 & 39.52 & 39.52 & 40.77 & 39.62 & 68.32 & 67.43 & 68.17 & 69.30  \\
 \hline
 16 & 39.81 & 39.71 & 39.90 & 40.38 & 68.10 & 66.33 & 69.51 & 69.42  \\
 \hline
 \hline
 \multicolumn{5}{|c|}{Winogrande (FP32 Accuracy = 67.88\%)} & \multicolumn{4}{|c|}{Piqa (FP32 Accuracy = 78.78\%)} \\ 
 \hline
 \hline
 64 & 66.85 & 66.61 & 67.72 & 67.88 & 77.31 & 77.42 & 77.75 & 77.64 \\
 \hline
 32 & 67.25 & 67.72 & 67.72 & 67.00 & 77.31 & 77.04 & 77.80 & 77.37  \\
 \hline
 16 & 68.11 & 68.90 & 67.88 & 67.48 & 77.37 & 78.13 & 78.13 & 77.69  \\
 \hline
\end{tabular}
\caption{\label{tab:mmlu_abalation} Accuracy on LM evaluation harness tasks on GPT3-8B model.}
\end{table}

\begin{table} \centering
\begin{tabular}{|c||c|c|c|c||c|c|c|c|} 
\hline
 $L_b \rightarrow$& \multicolumn{4}{c||}{8} & \multicolumn{4}{c||}{8}\\
 \hline
 \backslashbox{$L_A$\kern-1em}{\kern-1em$N_c$} & 2 & 4 & 8 & 16 & 2 & 4 & 8 & 16  \\
 %$N_c \rightarrow$ & 2 & 4 & 8 & 16 & 2 & 4 & 2 \\
 \hline
 \hline
 \multicolumn{5}{|c|}{Race (FP32 Accuracy = 40.67\%)} & \multicolumn{4}{|c|}{Boolq (FP32 Accuracy = 76.54\%)} \\ 
 \hline
 \hline
 64 & 40.48 & 40.10 & 39.43 & 39.90 & 75.41 & 75.11 & 77.09 & 75.66 \\
 \hline
 32 & 39.52 & 39.52 & 40.77 & 39.62 & 76.02 & 76.02 & 75.96 & 75.35  \\
 \hline
 16 & 39.81 & 39.71 & 39.90 & 40.38 & 75.05 & 73.82 & 75.72 & 76.09  \\
 \hline
 \hline
 \multicolumn{5}{|c|}{Winogrande (FP32 Accuracy = 70.64\%)} & \multicolumn{4}{|c|}{Piqa (FP32 Accuracy = 79.16\%)} \\ 
 \hline
 \hline
 64 & 69.14 & 70.17 & 70.17 & 70.56 & 78.24 & 79.00 & 78.62 & 78.73 \\
 \hline
 32 & 70.96 & 69.69 & 71.27 & 69.30 & 78.56 & 79.49 & 79.16 & 78.89  \\
 \hline
 16 & 71.03 & 69.53 & 69.69 & 70.40 & 78.13 & 79.16 & 79.00 & 79.00  \\
 \hline
\end{tabular}
\caption{\label{tab:mmlu_abalation} Accuracy on LM evaluation harness tasks on GPT3-22B model.}
\end{table}

\begin{table} \centering
\begin{tabular}{|c||c|c|c|c||c|c|c|c|} 
\hline
 $L_b \rightarrow$& \multicolumn{4}{c||}{8} & \multicolumn{4}{c||}{8}\\
 \hline
 \backslashbox{$L_A$\kern-1em}{\kern-1em$N_c$} & 2 & 4 & 8 & 16 & 2 & 4 & 8 & 16  \\
 %$N_c \rightarrow$ & 2 & 4 & 8 & 16 & 2 & 4 & 2 \\
 \hline
 \hline
 \multicolumn{5}{|c|}{Race (FP32 Accuracy = 44.4\%)} & \multicolumn{4}{|c|}{Boolq (FP32 Accuracy = 79.29\%)} \\ 
 \hline
 \hline
 64 & 42.49 & 42.51 & 42.58 & 43.45 & 77.58 & 77.37 & 77.43 & 78.1 \\
 \hline
 32 & 43.35 & 42.49 & 43.64 & 43.73 & 77.86 & 75.32 & 77.28 & 77.86  \\
 \hline
 16 & 44.21 & 44.21 & 43.64 & 42.97 & 78.65 & 77 & 76.94 & 77.98  \\
 \hline
 \hline
 \multicolumn{5}{|c|}{Winogrande (FP32 Accuracy = 69.38\%)} & \multicolumn{4}{|c|}{Piqa (FP32 Accuracy = 78.07\%)} \\ 
 \hline
 \hline
 64 & 68.9 & 68.43 & 69.77 & 68.19 & 77.09 & 76.82 & 77.09 & 77.86 \\
 \hline
 32 & 69.38 & 68.51 & 68.82 & 68.90 & 78.07 & 76.71 & 78.07 & 77.86  \\
 \hline
 16 & 69.53 & 67.09 & 69.38 & 68.90 & 77.37 & 77.8 & 77.91 & 77.69  \\
 \hline
\end{tabular}
\caption{\label{tab:mmlu_abalation} Accuracy on LM evaluation harness tasks on Llama2-7B model.}
\end{table}

\begin{table} \centering
\begin{tabular}{|c||c|c|c|c||c|c|c|c|} 
\hline
 $L_b \rightarrow$& \multicolumn{4}{c||}{8} & \multicolumn{4}{c||}{8}\\
 \hline
 \backslashbox{$L_A$\kern-1em}{\kern-1em$N_c$} & 2 & 4 & 8 & 16 & 2 & 4 & 8 & 16  \\
 %$N_c \rightarrow$ & 2 & 4 & 8 & 16 & 2 & 4 & 2 \\
 \hline
 \hline
 \multicolumn{5}{|c|}{Race (FP32 Accuracy = 48.8\%)} & \multicolumn{4}{|c|}{Boolq (FP32 Accuracy = 85.23\%)} \\ 
 \hline
 \hline
 64 & 49.00 & 49.00 & 49.28 & 48.71 & 82.82 & 84.28 & 84.03 & 84.25 \\
 \hline
 32 & 49.57 & 48.52 & 48.33 & 49.28 & 83.85 & 84.46 & 84.31 & 84.93  \\
 \hline
 16 & 49.85 & 49.09 & 49.28 & 48.99 & 85.11 & 84.46 & 84.61 & 83.94  \\
 \hline
 \hline
 \multicolumn{5}{|c|}{Winogrande (FP32 Accuracy = 79.95\%)} & \multicolumn{4}{|c|}{Piqa (FP32 Accuracy = 81.56\%)} \\ 
 \hline
 \hline
 64 & 78.77 & 78.45 & 78.37 & 79.16 & 81.45 & 80.69 & 81.45 & 81.5 \\
 \hline
 32 & 78.45 & 79.01 & 78.69 & 80.66 & 81.56 & 80.58 & 81.18 & 81.34  \\
 \hline
 16 & 79.95 & 79.56 & 79.79 & 79.72 & 81.28 & 81.66 & 81.28 & 80.96  \\
 \hline
\end{tabular}
\caption{\label{tab:mmlu_abalation} Accuracy on LM evaluation harness tasks on Llama2-70B model.}
\end{table}

%\section{MSE Studies}
%\textcolor{red}{TODO}


\subsection{Number Formats and Quantization Method}
\label{subsec:numFormats_quantMethod}
\subsubsection{Integer Format}
An $n$-bit signed integer (INT) is typically represented with a 2s-complement format \citep{yao2022zeroquant,xiao2023smoothquant,dai2021vsq}, where the most significant bit denotes the sign.

\subsubsection{Floating Point Format}
An $n$-bit signed floating point (FP) number $x$ comprises of a 1-bit sign ($x_{\mathrm{sign}}$), $B_m$-bit mantissa ($x_{\mathrm{mant}}$) and $B_e$-bit exponent ($x_{\mathrm{exp}}$) such that $B_m+B_e=n-1$. The associated constant exponent bias ($E_{\mathrm{bias}}$) is computed as $(2^{{B_e}-1}-1)$. We denote this format as $E_{B_e}M_{B_m}$.  

\subsubsection{Quantization Scheme}
\label{subsec:quant_method}
A quantization scheme dictates how a given unquantized tensor is converted to its quantized representation. We consider FP formats for the purpose of illustration. Given an unquantized tensor $\bm{X}$ and an FP format $E_{B_e}M_{B_m}$, we first, we compute the quantization scale factor $s_X$ that maps the maximum absolute value of $\bm{X}$ to the maximum quantization level of the $E_{B_e}M_{B_m}$ format as follows:
\begin{align}
\label{eq:sf}
    s_X = \frac{\mathrm{max}(|\bm{X}|)}{\mathrm{max}(E_{B_e}M_{B_m})}
\end{align}
In the above equation, $|\cdot|$ denotes the absolute value function.

Next, we scale $\bm{X}$ by $s_X$ and quantize it to $\hat{\bm{X}}$ by rounding it to the nearest quantization level of $E_{B_e}M_{B_m}$ as:

\begin{align}
\label{eq:tensor_quant}
    \hat{\bm{X}} = \text{round-to-nearest}\left(\frac{\bm{X}}{s_X}, E_{B_e}M_{B_m}\right)
\end{align}

We perform dynamic max-scaled quantization \citep{wu2020integer}, where the scale factor $s$ for activations is dynamically computed during runtime.

\subsection{Vector Scaled Quantization}
\begin{wrapfigure}{r}{0.35\linewidth}
  \centering
  \includegraphics[width=\linewidth]{sections/figures/vsquant.jpg}
  \caption{\small Vectorwise decomposition for per-vector scaled quantization (VSQ \citep{dai2021vsq}).}
  \label{fig:vsquant}
\end{wrapfigure}
During VSQ \citep{dai2021vsq}, the operand tensors are decomposed into 1D vectors in a hardware friendly manner as shown in Figure \ref{fig:vsquant}. Since the decomposed tensors are used as operands in matrix multiplications during inference, it is beneficial to perform this decomposition along the reduction dimension of the multiplication. The vectorwise quantization is performed similar to tensorwise quantization described in Equations \ref{eq:sf} and \ref{eq:tensor_quant}, where a scale factor $s_v$ is required for each vector $\bm{v}$ that maps the maximum absolute value of that vector to the maximum quantization level. While smaller vector lengths can lead to larger accuracy gains, the associated memory and computational overheads due to the per-vector scale factors increases. To alleviate these overheads, VSQ \citep{dai2021vsq} proposed a second level quantization of the per-vector scale factors to unsigned integers, while MX \citep{rouhani2023shared} quantizes them to integer powers of 2 (denoted as $2^{INT}$).

\subsubsection{MX Format}
The MX format proposed in \citep{rouhani2023microscaling} introduces the concept of sub-block shifting. For every two scalar elements of $b$-bits each, there is a shared exponent bit. The value of this exponent bit is determined through an empirical analysis that targets minimizing quantization MSE. We note that the FP format $E_{1}M_{b}$ is strictly better than MX from an accuracy perspective since it allocates a dedicated exponent bit to each scalar as opposed to sharing it across two scalars. Therefore, we conservatively bound the accuracy of a $b+2$-bit signed MX format with that of a $E_{1}M_{b}$ format in our comparisons. For instance, we use E1M2 format as a proxy for MX4.

\begin{figure}
    \centering
    \includegraphics[width=1\linewidth]{sections//figures/BlockFormats.pdf}
    \caption{\small Comparing LO-BCQ to MX format.}
    \label{fig:block_formats}
\end{figure}

Figure \ref{fig:block_formats} compares our $4$-bit LO-BCQ block format to MX \citep{rouhani2023microscaling}. As shown, both LO-BCQ and MX decompose a given operand tensor into block arrays and each block array into blocks. Similar to MX, we find that per-block quantization ($L_b < L_A$) leads to better accuracy due to increased flexibility. While MX achieves this through per-block $1$-bit micro-scales, we associate a dedicated codebook to each block through a per-block codebook selector. Further, MX quantizes the per-block array scale-factor to E8M0 format without per-tensor scaling. In contrast during LO-BCQ, we find that per-tensor scaling combined with quantization of per-block array scale-factor to E4M3 format results in superior inference accuracy across models. 

\end{document}


