\section{Related Work}
Existing studies propose various error classification methods based on different perspectives, such as linguistic structure, cognitive processes, and communicative impact. \citet{language_two} categorize language errors into four primary taxonomies: 
(1) \textit{Linguistic Category Taxonomy} \citep{linguistic_one,gooficon,linguistic_three} classifies errors based on language components or linguistic constituents;
(2) \textit{Surface Strategy Taxonomy} \citep{surface_one,surface_two,surface_three} focuses on structural modifications made by learners;
(3) \textit{Comparative Taxonomy} \citep{comparative_one,comparative_child,comparative_adult} classifies errors by comparing L2 learner errors to L1 acquisition errors or native language structures;
(4) \textit{Communicative Effect Taxonomy} \citep{communicative_one,communicative_two} classify errors based on their communicative impact. In addition to the above systematic classification taxonomies, other studies \citep{other_two,NUCLE,other_four,enhancing} have introduced other taxonomies, such as the simplification strategy taxonomy \citep{other_one} and the rule-based annotation toolkit \citep{errant,ye-etal-2023-cleme}.
