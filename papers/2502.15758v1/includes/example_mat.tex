\begin{table*}[ht]
\centering
\caption{Examples of quality assessment requirements. The full table is in \autoref{appendix:quality}}
\label{table:sample_reqs}
\begin{tabularx}{\linewidth}{
    p{0.15\linewidth}
    p{0.20\linewidth}
    p{0.23\linewidth}
    p{0.35\linewidth}} %{XXXX}
\toprule
\textbf{Sub-characteristic} & \textbf{Minimal requirement \ckmark} & \textbf{Full requirement \doubleckmark} & \textbf{Reasoning} \\
\midrule
\makecell[l]{Utility:  \\ \textbf{Accuracy}}
 & The ML system outperforms a simple baseline & The ML system outperforms a baseline and its input data are validated & The ML efforts are justified by outperforming a baseline \cite{huyen2022designing}. Input data is validated to avoid problematic deployments \cite{google-data-validation}. \\
\midrule
%\makecell[l]{Utility:  \\ \textbf{Effectiveness}} & Short-term effectiveness is verified with an A/B experiment & Long-term effectiveness is verified by repeating the AB test, if the system is in production for more than 6 months & A/B testing is a reliable way to assess a system's effectiveness \cite{Kohavi-rules-of-thumb, kohavi2022b, bernardi2019150, booking2021personalization} \\
%\midrule
\makecell[l]{Robustness:  \\ \textbf{Resilience}} & At most $30\%$ failed ML pipelines per quarter & At most $10\%$ failed ML pipelines per quarter & Automated ML pipelines should not fail frequently to ensure that the most updated model is available for predictions \cite{resilient-ml}. \\
 \midrule
 & - & ... & - \\
\midrule
\makecell[l]{Responsibility:  \\ \textbf{Ownership}}   & - & There is an appointed team responsible for maintaining the ML system & Ownership ensures that there is always an appointed individual to maintain the system in case of issues \cite{microsoft-ownership} \\
\bottomrule
\end{tabularx}
\end{table*}