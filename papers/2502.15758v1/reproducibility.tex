%\documentclass{article}
%\usepackage[margin=1.1in]{geometry} 
%\usepackage{amssymb}
%\begin{document}



\section*{Reproducibility Checklist}

 This paper:
    \begin{itemize}
        \item Includes a conceptual outline and/or pseudocode description of AI methods introduced: \checkmark
        \item Clearly delineates statements that are opinions, hypothesis, and speculation from objective facts and results: \checkmark
        \item Provides well marked pedagogical references for less-familiar readers to gain background necessary to replicate the paper: \checkmark
        \item Does this paper make theoretical contributions? - No
        \item Does this paper rely on one or more datasets? - No
        \item Does this paper include computational experiments? \checkmark

            \item Any code required for pre-processing data is included in the appendix: \checkmark
            \item All source code required for conducting and analyzing the experiments is included in a code appendix: \checkmark
            \item All source code required for conducting and analyzing the experiments will be made publicly available upon publication of the paper with a license that allows free usage for research purposes: \checkmark
            \item All source code implementing new methods have comments detailing the implementation, with references to the paper where each step comes from: \checkmark
            \item If an algorithm depends on randomness, then the method used for setting seeds is described in a way sufficient to allow replication of results: \checkmark
            \item This paper specifies the computing infrastructure used for running experiments (hardware and software), including GPU/CPU models; amount of memory; operating system; names and versions of relevant software libraries and frameworks: \checkmark
            \item This paper formally describes evaluation metrics used and explains the motivation for choosing these metrics: \checkmark
            \item This paper states the number of algorithm runs used to compute each reported result: \checkmark
            \item Analysis of experiments goes beyond single-dimensional summaries of performance (e.g., average; median) to include measures of variation, confidence, or other distributional information: \checkmark
            \item The significance of any improvement or decrease in performance is judged using appropriate statistical tests (e.g., Wilcoxon signed-rank): \checkmark
            \item This paper lists all final (hyper-)parameters used for each model/ algorithm in the paper’s experiments: \checkmark
            \item This paper states the number and range of values tried per (hyper-)parameter during development of the paper, along with the criterion used for selecting the final parameter setting: \checkmark
        \end{itemize}



%\end{document}
