\section{Related Works}
\label{sec:related_works}

In this section, we present some studies that comprises the state-of-the-art in performance models for Hyperledger Fabric, which is the focus of this paper. Recently, several works have aimed to evaluate the performance metrics, availability, and behavior of blockchain systems deployment infrastructures using models, as we discuss now.

Melo et al.~\cite{melo_computing2022,melo_supercomp2021} developed models to analyze the availability and infrastructure resources provisioning for Hyperledger Fabric and Ethereum blockchain platforms. In these works, Continuous Time Markov Chains (CTMC), Reliability Block Diagrams (RBD), and Stochastic Petri Nets were used for modeling those platforms. The authors concluded that the proposed models could help in the planning of blockchain applications based on both platforms, allowing the identification of infrastructure bottlenecks and the comparison of implementation costs. Cost comparisons were also made in implementing infrastructure for them.

Salle et al.~\cite{la2023joint} focus on modeling Hyperledger Fabric (HLF) under cyberattack conditions, specifically through Petri Net simulations of Sybil attacks. In contrast, our paper emphasizes performance optimization across various configurations, examining not only the impact of different network settings on general performance but also exploring how variations in workload affect throughput and latency. While we do not directly address the influence of malicious transactions, our analysis extends to understanding how changes in transaction arrival rates can significantly alter network performance metrics.

The provided abstract focuses on modeling Hyperledger Fabric (HLF) under cyberattack conditions, specifically through Petri Net simulations of Sybil attacks, aiming to fill a gap in academic literature on HLF’s cybersecurity vulnerabilities. This contrasts sharply with the earlier document's broader evaluation of HLF performance through Stochastic Petri Nets without a cybersecurity focus. The new text introduces a novel joint model that quantitatively assesses the impact of cyberattacks on HLF, a specific and targeted analysis that the original study does not undertake. Whereas the first document emphasizes performance optimization across various configurations, the latest abstract highlights innovative contributions to understanding and mitigating cyber risks in blockchain environments, thus marking a significant shift towards enhancing blockchain resilience against cyber threats.

Other works focused on modeling general performance metrics of the Hyperledger Fabric platform specifically~\cite{wu_acm_ease2022, jiang_springer_p2pna2020, ke_springer_cc2022, wang2023stochastic}. 
Jiang et al.~\cite{jiang_springer_p2pna2020} developed a hierarchical model for Hyperledger Fabric 1.4 transactions process and conducted numerical analyses of throughput, discard rate, and mean response time.
Wu et al.~\cite{wu_acm_ease2022} modeled the transaction processes of the Hyperledger Fabric 2.0 using a queuing model with limited transaction pools. By analyzing this model's two-dimensional continuous-time Markov process, they obtained the throughput, transaction rejection probability, system queue length, and mean response time.
Similarly, Ke and Park~\cite{ke_springer_cc2022} developed a series of queuing models for the performance evaluation with different service rates for endorsing and validation through the metrics of queue length and mean response time.

% block size and block timeout settings
Finally, some performance modeling efforts looked mainly at inferring the metrics throughput and latency of Hyperledger Fabric based on varying arrival rates, block size, and block timeout settings~\cite{xu_ipm2021,sukhwani2018performance, yuan2020performance}. 
Xu et al.~\cite{xu_ipm2021} proposed an analytical model based on equations, and its inferred metrics were compared with simulations. 
%
Yuan et al.~\cite{yuan2020performance} performed a comparison relying on the Generalized Stochastic Petri Nets (GSPN) modeling approach.
In turn, Sukhwani et al.~\cite{sukhwani2018performance} proposed Stochastic Reward Networks (SRN) models that allowed, in addition to performance metrics, to estimate the average queue size of each step in Hyperledger Fabric transaction flow. Other transaction steps were also analyzed, identifying bottlenecks and presenting several possible scenarios. 
The authors concluded that the time to complete the endorsement step is affected by the number of nodes participating and the policies adopted.

Wang et al.~\cite{wang2023stochastic} delves into the stochastic performance analysis of phase decomposition within Hyperledger Fabric, employing a more granular approach to dissect transaction processing phases—execute, order, validate (endorsing, ordering and committing). It uses stochastical modeling to evaluate how resource allocation, such as CPU and network settings, affects transaction latency and throughput performance metrics. The present paper differs from \cite{wang2023stochastic} by providing a generalizable Petri Net to model the entire transaction flow and interactions within Hyperledger Fabric at each transaction stage in the given system, providing a holistic view of the system's performance under various configurations.

This paper proposes an SPN model to estimate the value of important performance characteristics of Hyperledger Fabric. 
Like some works presented above, the model estimates the metrics of \textit{mean response time, throughput}, and \textit{resource utilization}. 
As additional contributions, the model also estimates two new metrics: the \textit{maximum block size rate} and \textit{timeout block rate}, as well as the \textit{discarding probability}, which directly impacts the efficiency and reliability of transaction processing. 
High discarding probability can lead to increased transaction failures and inefficiencies within the blockchain network, affecting overall system performance and user experience. 
The study also presents a calibration strategy for setting the timeout and block size parameters.

Table~\ref{tab:related} shows the main characteristics addressed by each work discussed in this section. 
We differ from all those prior efforts by jointly considering all these characteristics. 
It is worth mentioning that none of the approaches allowed easily identifying the timeout and size parameters for an arrival rate, as well as to identify block formation failures due to timeout.

\begin{table*}[!htp]
\centering
\begin{center}
\caption{Research Works on Performance Modeling and their comparison to the main aspects addressed by this paper.}
\begin{tabular}{@{}lcccccc@{}}
\toprule
\multicolumn{1}{c}{\textbf{Publication}} & \multicolumn{1}{c}{\textbf{MRT}} & \multicolumn{1}{c}{\textbf{Throughput}} & \multicolumn{1}{c}{\textbf{RU}} & \multicolumn{1}{c}{\textbf{MSBR}} & \multicolumn{1}{c}{\textbf{TBR}} & \multicolumn{1}{c}{\textbf{DP}} \\ \midrule
Sukhwani et al. \cite{sukhwani2018performance} & \cmark & \cmark & \cmark & \cmark & \cmark & \xmark \\
Jiang et al. \cite{jiang_springer_p2pna2020} & \cmark & \cmark & \xmark & \xmark & \xmark & \cmark \\
Yuan et al. \cite{yuan2020performance} & \cmark & \cmark & \xmark & \cmark & \cmark & \xmark \\
Xu et al. \cite{xu_ipm2021} & \cmark & \xmark & \xmark& \cmark  & \xmark & \xmark \\
Melo et al. \cite{melo_supercomp2021} & \xmark & \xmark& \cmark& \xmark  & \xmark & \xmark \\
Ke and Park \cite{ke_springer_cc2022} & \cmark & \xmark& \xmark& \xmark  & \xmark & \xmark \\
Wu et al. \cite{wu_acm_ease2022} & \cmark & \xmark& \xmark& \xmark  & \xmark & \xmark \\
Wang et al. \cite{wang2023stochastic} & \cmark & \cmark & \cmark & \cmark  & \cmark & \xmark \\
Salle et al. \cite{la2023joint} & \cmark & \cmark & \cmark & \xmark  & \xmark & \xmark \\
Melo et al. \cite{melo_computing2022} & \xmark & \xmark& \cmark& \xmark  & \xmark & \xmark \\
This paper & \cmark & \cmark& \cmark& \cmark  & \cmark & \cmark \\ \bottomrule
\end{tabular}
\label{tab:related}
\end{center}
\begin{tablenotes}[flushleft]\footnotesize
\item[]Metrics abbreviations: Mean Response Time (MRT), Resource Utilization (RU), Maximum Size Block Rate (MSBR), Timeout Block Rate (TBR), Discarding probability (DP)
 \par
\end{tablenotes}
\end{table*}
%----------------------