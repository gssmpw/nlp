\section{Experiments}
This section presents a comprehensive evaluation of our proposed \model~method. We begin by detailing the evaluation metrics used to assess the quality of the shape models, covering surface sampling accuracy, correspondence quality, and SSM performance. Following this, we describe the diverse anatomical datasets employed in our experiments, highlighting their unique characteristics and challenges. We then introduce the comparison models, including SOTA methods in shape modeling and analysis. 

\subsection{Evaluation Metrics} \label{metrics}
This section outlines the metrics used to evaluate the quality of the shape models. 
\begin{enumerate}
    \item \textit{Surface sampling metrics to assess accuracy of surface representation}
    \setlist[itemize]{leftmargin=0pt}
    \begin{itemize}
        \item \textbf{Chamfer Distance (CD)} measures the average distance between two point sets, calculated bidirectionally: from each point in set \(\C_j\) to its nearest neighbor in set \(\V_j\), and vice versa. This provides a comprehensive measure of dissimilarity between the two point sets.
        \item \textbf{Point-to-Mesh Distance (P2M)} is determined by summing two components: the point-to-mesh face distance and the face-to-point distance. This distance is calculated between the predicted correspondences \(\C_j\) and the mesh defined by vertices \(\V_j\) and edges \(\set{E}_j\).
    \end{itemize}
    \item \textit{Correspondence metrics to assess ability to capture population-level statistics:}
    \begin{itemize}
        \item \textbf{Surface-to-Surface Distance (S2S)} is computed between the original surface mesh and a generated mesh derived from predicted correspondences. To obtain the reconstructed mesh, correspondences are matched to the mean shape, and the warp between the predicted correspondences and the mean particles is applied to the mean mesh to get the reconstructed surface. Smooth reconstruction and low surface-to-surface distance indicate good quality of correspondences.
    \end{itemize}
    \item \textit{SSM Metrics:} evaluate the quality and performance of the shape models, ensuring they accurately represent the shape variations in a population while maintaining the ability to describe new instances and generate valid shapes.
    \setlist[itemize]{leftmargin=0pt}
    \begin{itemize}
    \item \textbf{Compactness} measures how efficiently a shape model represents the variability in a population using as few parameters as possible. It is quantified by the number of principal components (PC) or shape modes needed to explain a certain percentage of the total shape variance. A more compact model requires fewer PCs to capture the same variation. Mathematically, compactness can be defined as the cumulative explained variance of the Mth eigenmode obtained by the model's covariance matrix decomposition.

    \item \textbf{Generalization} assesses how well the shape model can describe shapes that were not part of the training set. It evaluates the model's ability to represent new, unseen instances of the shape class. This is typically measured by the reconstruction error when the model attempts to match new data. A lower reconstruction error indicates better generalization.

    \item \textbf{Specificity} measures the model's ability to generate valid instances of the trained shape class. It quantifies how well the shapes generated by the model resemble those in the training set. This is often calculated as the average distance between randomly sampled model-generated shapes and the nearest shapes in the training set. A lower average distance indicates better specificity. 
    \end{itemize}
\end{enumerate}
\subsection{Dataset}
We employ segmentation datasets (three public and two in-house) comprising five distinct anatomies with varying cohort sizes. All datasets undergo random partitioning into training, validation, and testing sets using an \(80\%/10\%/10\%\) split. Each dataset is described as follows:
% \setlist[itemize]{leftmargin=0pt}
\begin{itemize}
\item \textbf{Femur (56 shapes)}: The femur dataset contains proximal femur bones clipped under the lesser trochanter to focus on the femoral head. Nine of the femurs have the cam-FAI pathology characterized by an abnormal bone growth lesion that causes hip osteoarthritis. 
\item \textbf{Spleen (40 shapes)} \cite{simpson2019large}: The spleen organ dataset provides a limited data scenario with challenging shapes that vary significantly in size and curvature.
\item \textbf{Pancreas (272 shapes)} \cite{simpson2019large}: The pancreas dataset comprises of pancreas organs with tumors of varying sizes from cancer patients, providing complex patient-specific shape variability.
\item \textbf{Liver (834 shapes)} \cite{Ma-2021-AbdomenCT-1K}: The liver dataset provides an organ dataset with nonlinear shape variation.
\item \textbf{Left Atrium (923)}: This dataset includes 3D late gadolinium enhancement (LGE) and stacked cine cardiovascular magnetic resonance (CMR). The dataset comprises 923 anonymized obtained from distinct patients and was manually segmented by cardiovascular medicine experts at the University of Utah Division of Cardiovascular Medicine; the endocardium wall was used to cut off pulmonary veins.
\end{itemize}

\subsection{Comparison Models}
\begin{itemize}
\item \textbf{ShapeWorks (SW)} \cite{cates2017shapeworks} is a SOTA PSM tool that establishes dense correspondences on complete surface representations using a particle-based approach. It is computationally efficient and widely adopted for shape variability analysis. We use its open-source implementation to benchmark our method against standard SSM metrics.
\item \textbf{Deformetrica} \cite{durrleman2014morphometry}, a large deformation diffeomorphic metric mapping (LDDMM) framework, is a well-established method in the medical imaging domain and serves as a baseline for state-of-the-art techniques. Unlike data-driven approaches, LDDMM formulates shape alignment as a pairwise optimization problem. Its objective is to minimize the varifold distance between the target shape surface mesh and the template surface mesh. We use the open-source implementation of Deformetrica, and only the kernel width parameter was altered to gain better results on each dataset; otherwise, we use the default parameters.
\item \textbf{FlowSSM} \cite{ludke2022landmark} operates directly on surface meshes and uses neural networks to parameterize the deformations field between two shapes in a low dimensional latent space and rely on an encoder-free setup. The encoder-free step randomly initializes the latent representations for each sample, and the latent representations are optimized to produce the optimal deformations.
\item \textbf{Mesh2SSM} \cite{iyer2023mesh2ssm} model described in section~\ref{mesh2ssm}
\item \textbf{\model~*} autoencoder version without normalizing flows and simple template update as the average of training correspondences. Abbreviated as M++AE for readability. 
\item \textbf{\model} is the proposed model described in section~\ref{mesh2ssm_plus} and abbreviated as M++Flow for readability. 
\end{itemize}
All models use the same median-shape template mesh to avoid bias. \model-based approaches periodically update the template based on learned statistics. In the case of the M++Flow model, the template is updated using the sampling procedure described in section~\ref{mesh2ssm_plus}. In contrast, for M++AE, the template is updated as the mean correspondence of all the predicted training samples. All SW, Mesh2SSM, and \model~based methods use 1024 correspondence points, whereas Deformetrica and FlowSSM establish vertex-wise correspondences. Note that we attempted to replicate the results of the FUSS \cite{el2024universal} using the code and hyperparameters provided by the authors. However, we were unable to achieve the performance reported in their paper \cite{el2024universal}. To ensure fairness, we have excluded the FUSS model from our comparative analysis.