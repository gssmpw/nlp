\section{Related Work}

The increasing complexity of anatomical shapes and the volume of data in modern medical imaging have rendered manual landmarks/correspondence annotation impractical. This has led to a reliance on computational methods to establish anatomical correspondences. Methods to establish correspondences broadly include (a) registration-based landmark estimation and (b) parametric and non-parametric correspondence optimization. Registration-based methods involve manually annotating landmarks on a reference shape and warping these landmarks to other shapes using registration techniques \cite{paulsen2002building,heitz2005statistical,mcinerney1996deformable}. Parametric methods use fixed geometrical bases to derive correspondences \cite{styner2006framework}, but they often struggle with complex anatomical shapes due to limited flexibility. Non-parametric approaches establish correspondences based on cohort-level variability by optimizing objective functions \cite{cates2008particle,cates2017shapeworks,dai2020statistical}. A notable family of tools employs entropy-based objectives to establish correspondences \cite{oguz2016entropy,cates2007shape,cates2008particle,cates2017shapeworks} across a cohort of shapes while ensuring faithful representation of each shape. Particle-based Shape Modeling (PSM) has emerged as a state-of-the-art method in this category.

However, conventional methods face multiple challenges. Recent advancements in deep learning have significantly improved SSM by enabling the efficient learning of non-linear shape representations. Mesh-based and point cloud-based deep learning methods have emerged as promising alternatives due to their ability to model the structured representation of 3D shapes. Recent advancements in point cloud-based SSM, such as Point2SSM \cite{adams2023point2ssm}, Point2SSM++ \cite{adams2022images}, and other point cloud-based approaches \cite{lang2021dpc,wang2019dynamic,chen2020unsupervised} have demonstrated the potential of using raw point clouds for constructing shape models. These methods offer advantages in terms of reduced computational burden and relaxed input requirements. However, they lack the crucial connectivity information inherent in mesh-based approaches. Mesh-based SSM methods leverage this connectivity to better capture local geometric relationships and preserve fine anatomical details, which is particularly important for accurately modeling complex structures. Therefore, despite the progress in point cloud-based shape modeling, mesh-based approaches remain essential for comprehensive and precise statistical shape modeling in medical applications.

Early mesh-based networks, such as ShapeNet \cite{chang2015shapenet} and 3D-R2N2 \cite{choy20163d}, introduced volumetric grids and multi-view projections for mesh classification and vertex segmentation but faced challenges with high computational costs and resolution limitations. Mesh-specific neural networks, such as GCNN \cite{zhang2019graph,kipf2016semi} and MeshCNN \cite{hanocka2019meshcnn}, extended convolution operations to irregular graph-like structures, enabling effective feature extraction while preserving geometric details. Spectral methods like ChebNet \cite{defferrard2016convolutional} provided further advancements by operating in the spectral domain, and dynamic graph convolution networks (DGCNN) \cite{wang2019dynamic} introduced dynamic edge connections to improve the capture of local geometric relationships. Although ChebNet excels at modeling global structures, DGCNN spatial adaptability makes it particularly suitable for medical SSM applications that require localized and flexible shape modeling. 

Recent methods have addressed the challenges of shape vertex matching using unsupervised learning approaches. ShapeFlow \cite{jiang2020shapeflow} learns a deformation space for 3D shapes with large intra-class variations, using neural networks to parameterize continuous flow fields between a pair of meshes. FlowSSM \cite{ludke2022landmark} extended the idea to model population variation using neural networks to parameterize the deformations field between a cohort of shapes and a template in a low dimensional latent space and rely on an encoder-free setup. However, FlowSSM exhibits sensitivity to template selection and lacks an encoder, requiring re-optimizing latent representations for unseen shapes, increasing computational overhead \cite{iyer2023mesh2ssm}. Atlas-R-ASMG \cite{kalaie2024end} is an end-to-end deep learning generative framework for refinable shape matching and generation using 3D surface mesh data. The framework jointly learns high-quality refinable shape matching and generation while constructing a population-derived atlas model, enabling the generation of diverse and realistic anatomical shapes for virtual populations. However, Atlas-R-ASMG still relies on vertex-level correspondence by employing an attention mechanism in latent space to measure the similarity between local vertex embeddings between the atlas and the shapes. This approach, while effective, may limit the framework's flexibility in handling highly variable anatomical structures or shapes with significant topological differences. Addressing these limitations, Mesh2SSM \cite{iyer2023mesh2ssm} introduces a novel approach by replacing the encoder-free setup of FlowSSM with geodesic features and an EdgeConv-based \cite{wang2019dynamic} mesh autoencoder. This method overcomes the issues its predecessors face by eliminating reliance on vertex-level correspondences. Instead, Mesh2SSM produces a correspondence model with a fixed number of landmarks determined by the initial template. This is typically smaller than the total number of vertices, resulting in a more compact representation. Mesh2SSM also introduces a variational autoencoder \cite{kingma2019introduction} for learning a data-driven template from the predicted correspondences. However, Mesh2SSM cannot ensure predicted correspondences lie on the surface of the mesh, does not include aleatoric uncertainty quantification for predictions, and can have instabilities in training due to the VAE used for analysis. 

Our work builds on recent progress, particularly Mesh2SSM's strengths while addressing its limitations. We propose a method that introduces a probabilistic framework with an end-to-end training strategy, improved sample generation, aleatoric uncertainty estimation, and enhanced correspondence quality across variable anatomical structures.