
\section{Introduction} \label{introduction}
Anatomy evaluation is the systematic quantitative assessment of the human body's form and function, crucial for comprehending the medical condition, diagnosing abnormalities, and guiding medical interventions \cite{singh2020evaluation,dai2020statistical,zhang20243dcmm,quiceno2024statistical}. Statistical shape modeling (SSM) is essential for anatomy evaluation in medical image analysis and computational anatomy. SSM facilitates the discovery of quantitative morphological shape descriptors that comprehensively describe anatomical variations within the context of the population, aiding in diagnosis \cite{khan2022machine,schaufelberger2022radiation}, pathology detection \cite{peiffer2022statistical,sophocleous2022feasibility}, treatment planning \cite{vicory2022statistical}, and enhancing early intervention strategies \cite{merle2019high,mulder2022dynamic,okegbile2022human}.

\begin{figure}
    \centering
    \includegraphics[width=\linewidth]{figures/correspondnces_example_LA.png}
    \caption{Correspondences are sets of ordered points on different shapes representing the same anatomical or geometric feature, thereby establishing a consistent relationship between the shapes. The white highlighted points represent predicted correspondences by \model~in the right superior pulmonary vein (RSPV) antrum region of the left atrium, consistently located across all shapes. Matching colors across samples indicate additional corresponding points. }
    \label{fig:correspondences_example}
    \vspace{-8mm}
\end{figure}

SSM methods have traditionally relied on two primary approaches for representing anatomical structures: \textit{implicit} representations (e.g., deformation fields \cite{durrleman2014morphometry} and level set methods \cite{samson2000level}) and \textit{explicit} representations (e.g., Point Distribution Models, or PDMs, which use ordered set of landmarks or correspondence points). Figure~\ref{fig:correspondences_example} shows examples of PDM for three different anatomies. The points highlighted in white represent these correspondence points, and the matching colors across samples indicate the established correspondences. Traditional SSM methods establish correspondences via (a) pairwise methods that map a shape subject to a pre-defined atlas or template via fixed geometrical bases (SPHARM-PDM \cite{styner2006framework}) or diffeomorphic metric mapping (Deformetrica \cite{durrleman2009statistical}), (b) group-wise approaches that establish the correspondences by considering the variability of the entire shape cohort (Particle-based Shape Modeling \cite{cates2007shape,cates2008particle}). 
However, the effectiveness of SSM in anatomy evaluation critically depends on the quality and robustness of the models, which must accurately capture real-world anatomical variability while handling noise and incomplete data \cite{cerrolaza2019computational}.

Despite their utility, traditional SSM approaches face several significant limitations. They often rely on computationally expensive optimization frameworks and manually tuned parameters, making them non-scalable and inefficient for handling large datasets. Their dependence on linearity assumptions further restricts their ability to model non-linear anatomical variations, leading to limited generalization. Moreover, traditional methods require full recomputation of the model for new samples, rendering them impractical for real-time or large-scale clinical applications. Deep learning-based approaches for SSM have emerged as a promising avenue for streamlining SSM. Deep learning models can learn complex non-linear representations of the shapes, which can be used to construct shape models. Moreover, they can efficiently perform inference on new samples without computation overhead or re-optimization. Unsupervised methods \cite{iyer2023mesh2ssm,adams2023point2ssm,ludke2022landmark,bhalodia2024deepssm,bhalodia2018deepssm,el2024universal,kalaie2024end} have demonstrated the ability to estimate population-level correspondences from meshes, point clouds, and images while eliminating the need to repeat the shape modeling pipeline for unseen samples and maintaining computational efficiency at inference. Specifically, Mesh2SSM \cite{iyer2023mesh2ssm} is an unsupervised deep learning approach that generates statistical shape models directly from surface meshes by deforming a template point cloud to subject-specific meshes. It eliminates the bias arising from template selection by incorporating a VAE \cite{kingma2019introduction} on the learned correspondences that help learn the underlying manifold and enable sampling a population-informed template. 

However, Mesh2SSM has certain limitations. The incorporation of VAE complicates the training process as the end-to-end training can be tricky as VAE requires careful parameter tuning to avoid posterior collapse. Mesh2SSM uses Chamfer distance to train the network such that the predicted correspondences faithfully represent the underlying shape, but this does not guarantee that the predicted particles will lie on the surface of the mesh. Medical datasets pose additional challenges, including data limitations and the need for uncertainty quantification to avoid overconfident predictions and provide clinicians with insights into model limitations. Real-world medical data often contains artifacts like noise and blurred edges, creating ambiguity in estimating correspondence positions, particularly near sharp edges or in noisy regions. This ambiguity necessitates accounting for aleatoric uncertainty, stemming from inherent data variability. Addressing this is essential for reliable outcomes in medical applications where precise modeling is paramount. Mesh2SSM does not account for these scenarios or provide any insights into the confidence of the model predictions. 

Therefore, we introduce \model, an unsupervised learning framework for constructing correspondence-based probabilistic statistical shape models directly from surface meshes. Building upon its predecessor, Mesh2SSM \cite{iyer2023mesh2ssm}, the proposed \model~introduces several key improvements to aid in effective anatomy evaluation. Our contributions are summarized as follows: 
\begin{enumerate}
        \item \textbf{Flow-enhanced latent space for probabilistic shape modeling: }Introduction of a continuous normalizing flow \cite{rezende2015variational,dinh2016density} in the latent space of the mesh autoencoder. This enhancement streamlines the model architecture while endowing the framework with powerful probabilistic capabilities. The normalizing flow enables:
        \begin{itemize}
            \item High-quality sample generation, owing to the invertibility of the transformations of the flows, facilitates a more accurate representation of anatomical variability.
            \item Streamlined data-informed template updates with simplified end-to-end training of a probabilistic network. 
            \item Efficient aleatoric uncertainty estimation for the correspondence prediction task, providing crucial reliability measures in medical applications.
        \end{itemize}
        \item \textbf{Mesh-constrained correspondence prediction:} We introduce a new loss function to encourage predicted correspondences to lie on the mesh surface, thereby improving the quality and anatomical accuracy of the correspondences.
        \item \textbf{Self-supervised training to increase robustness to noisy data:} We introduce vertex masking as an additional form of data augmentation, enhancing the model's robustness to noisy input and improving generalization.
        \item \textbf{Comprehensive evaluation:} We conduct comprehensive experimentation on multiple anatomical datasets, employing a wide range of evaluation metrics to assess model performance thoroughly. Our rigorous benchmarking against state-of-the-art (SOTA) mesh-based models demonstrates the effectiveness and advantages of \model~in medical shape analysis. We also show the utility of \model~by assessing its performance for two downstream tasks. 
\end{enumerate}

Overall, \model~enhances shape descriptor extraction, improves efficiency in correspondence generation, and extends SSM applicability to diverse anatomical structures. These advancements make \model~a robust, scalable solution for clinical applications requiring accurate and interpretable shape models.