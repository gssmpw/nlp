\section{Limitations and Future Work}
The current model assumes that the cohort of shapes is roughly aligned, which may limit its applicability to diverse datasets and clinical scenarios. Addressing this limitation by developing robust alignment algorithms or exploring alignment-free approaches could significantly expand the usability of \model~across various applications. Furthermore, improving the robustness and computational efficiency of mesh feature extraction methods would eliminate the reliance on geodesic distance calculations, which are often computationally intensive. Instead, leveraging alternative representations that preserve the structural knowledge of meshes—such as topological features, intrinsic coordinates, or spectral embeddings—can balance efficiency and fidelity to the mesh's geometric properties. These approaches enable the model to capture critical shape characteristics without the overhead of geodesic distance computation, paving the way for scalable and robust feature extraction in complex datasets.

While the CD effectively ensures that predicted correspondences are near the ground truth surface, it does not guarantee that they lie exactly on it. Our model addresses this limitation by incorporating a surface projection step, which aligns predicted correspondences directly with the input mesh surface. However, CD may face challenges when the input mesh contains missing regions, spurious surfaces, or significant noise, potentially leading to suboptimal or inaccurate results \cite{araslanovdiffcd}.

Alternative approaches, such as Neural Implicit Functions \cite{wang2023neural}, that leverage self-supervised construction of Signed Distance Fields (SDFs) for surface reconstruction accurately define the underlying surface using the zero-level set. Initially developed for point cloud scenarios, Neural Implicit Functions can be adapted to address mesh irregularities effectively. This extension could improve generalization, handle noisy or incomplete data, and achieve superior results, thereby enhancing the robustness and versatility of shape modeling methodologies.

\section{Conclusion}
This study establishes the effectiveness of \model-based approaches in shape modeling for diverse anatomical datasets, highlighting their strengths in accuracy, adaptability, and uncertainty quantification. The proposed methods, M++Flow and M++AE, demonstrate superior or comparable performance to traditional optimization-based techniques such as ShapeWorks across key metrics, including compactness, generalization, specificity, and distance-based evaluations. Notably, the proposed models achieve high fidelity in representing both simple structures, like the femur, and complex geometries, such as the liver, pancreas, and left atrium, and display reduced computational complexity at inference while maintaining robust performance.

The calibration of aleatoric uncertainty proves to be a significant advantage in evaluating model reliability and identifying outliers. The spatial correlation between uncertainty and prediction errors is particularly evident in datasets with complex or irregular geometries, such as the pancreas and liver. The sample-wise uncertainties are particularly powerful in identifying outliers with more prevalent structural anomalies or segmentation ambiguities. Identifying outliers and quantifying prediction confidence has profound implications for clinical applications, enabling more reliable decision-making and improved model interpretability.

The multi-anatomy modeling experiments further validate the capability of \model~in capturing subtle morphological differences across highly similar structures, as demonstrated by the lumbar vertebrae dataset. The results emphasize the model's precision in distinguishing structural variations. Moreover, the modes of variation analysis align well with known population-level anatomical differences, underscoring the utility of \model~in statistical shape analysis and downstream tasks such as disease localization and group difference analysis.

Overall, integrating deep learning with probabilistic modeling in \model~provides a comprehensive framework for shape modeling. The methods are highly adaptable, efficient, and reliable, making them well-suited for diverse biomedical applications. By addressing challenges like correspondence efficiency, uncertainty quantification, and multi-anatomy modeling, \model~sets a new benchmark for shape analysis tools in clinical and research settings. 