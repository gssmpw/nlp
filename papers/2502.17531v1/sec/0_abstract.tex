\begin{abstract}
With the rising popularity of 3D Gaussian splatting and the expanse of applications from rendering to 3D reconstruction, there comes also a need for geometry processing applications directly on this new representation. 
While considering the centers of Gaussians as a point cloud or meshing them is an option that allows to apply existing algorithms, this might ignore information present in the data or be unnecessarily expensive. 
Additionally, Gaussian splatting tends to contain a large number of outliers which do not affect the rendering quality but need to be handled correctly in order not to produce noisy results in geometry processing applications. 
In this work, we propose a formulation to compute the Laplace-Beltrami operator, a widely used tool in geometry processing, directly on Gaussian splatting using the Mahalanobis distance. 
While conceptually similar to a point cloud Laplacian, our experiments show superior accuracy on the point clouds encoded in the Gaussian splatting centers and, additionally, the operator can be used to evaluate the quality of the output during optimization. Our code will be released on github soon.
\end{abstract}
