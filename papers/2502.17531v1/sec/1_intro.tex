\section{Introduction}
\label{sec:intro}

3D Gaussian Splatting (3DGS)~\cite{kerbl3Dgaussians} has recently revolutionized 3D scene representations. By representing complex scenes as a set of 3D Gaussians, it achieves photorealistic results for novel view synthesis while allowing much more efficient training and real-time rendering compared to NeRF based approaches~\cite{barron2022mip}. 
Even though it was originally proposed for rendering applications, it is now also popular as a general 3D representation~\cite{Huang2DGS2024,GaoMeshGaussian2024}. 
However, due to its initial focus on rendering, most 3DGS methods do not lead to an accurate surface representation. 
To overcome this, a series of work has been produced with a focus on surface reconstruction instead~\cite{Huang2DGS2024,chen2023neusgneuralimplicitsurface}. 
One possibility is to pull the 3D Gaussians to the suspected surface and then reconstruct a mesh via Poisson reconstruction~\cite{guedon2023sugar,kazhdan2006poisson}.
Another approach was suggested in~\cite{Huang2DGS2024} by restricting the Gaussians to be 2D which leads to more accurate surface mappings (as the surface of a 3D object is a 2D manifold). 

As the geometric reconstruction accuracy advances, the need for common geometry processing tools directly on this representation rises.
For example, manual editing through deformation energies~\cite{huang2024sc} and semantic segmentation~\cite{cen2023saga} have been proposed for Gaussian splatting.
In this paper we will look at the Laplace-Beltrami operator (LBO), which is often called the "Swiss Army knife" of geometry processing and used in a wide variety of geometry processing applications~\cite{sorkine2005laplacian,ovsjanikov2012functional,weber2024finsler}. 
One core information in the LBO is the connectivity between points which is not explicitly included in Gaussian splatting. 
It is possible to consider the 3DGS as a point cloud by taking the set of Gaussian centers,
however, this ignores a lot of information included in the variances and generally does not work well due to the high density of outliers in Gaussian splatting. 
The outliers stem from the optimization from images, which normally do not depict the inside of objects.
Additionally, low opacity Gaussians do not significantly change the appearance but stay present in the point cloud of centers.
We propose to compute the Laplace-Beltrami operator by using the Mahalanobis distance which can leverage the variance of the Gaussians as an indicator of surface direction. 
To show the performance of the LBO, we experiment on classical geometry processing applications such as distance computation and shape matching.


\paragraph{Contributions. } Our contributions are as follows: 
\begin{enumerate}
    \item The definition of a Laplace-Beltrami operator that can be computed directly on Gaussian splatting and takes into account all information encoded by the variance. 
    \item We show a relationship between the eigenvalues of the LBO and stable geometry of 3DGS, and how they can be used to determine convergence during optimization. 
    \item We experimentally demonstrate the effectiveness of this LBO on a variety of geometry processing applications.
\end{enumerate}
In addition, we will publish a variation of the popular geometry processing dataset~\cite{MB08}, in which we generated renderings and 3D Gaussian splatting reconstructions of all shapes, including correspondences between both, to facilitate further research on geometry processing applications using 3DGS.

