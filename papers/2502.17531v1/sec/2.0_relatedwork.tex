\section{Related Work}
In this section, we show connections to the most relevant related work. A broader introduction into Gaussian splatting can be found in the survey of \cite{chen2024survey3dgaussiansplatting}.

\subsection{Gaussian Splatting}\label{sub:rw:gaussian}

3D Gaussian splatting was introduced in \cite{kerbl3Dgaussians} as an efficient framework for novel-view synthesis. 
It optimizes over a set of Gaussian distributions in 3D space with color and opacity values which can be rendered from new view points. 
While this works exceptionally well and has been applied to many applications~\cite{SplattingAvatar:CVPR2024,yan2024street}, the pipeline is focused on clean-looking rendering results but not clean geometry. 
To overcome this, several methods introduced additional constrains that focus on the geometric accuracy in the optimization. 
For example, the method of \cite{Huang2DGS2024} restricts the variance to a 2D plane (the dimension of the surface). 
Another approach, SuGaR~\cite{guedon2023sugar}, extracts a mesh and realigns the Gaussian splats with the surface of this mesh to obtain a cleaner geometry. 
Gaussian opacity fields~\cite{yu2024gaussian} combine Gaussians with a signed distance function to further regularize the results. 
While this leads to better geometry, the extraction of a mesh in the process is expensive and the extracted mesh might still be noisy. Without the extraction, outlier splats can prevail, and while they tend to not deteriorate the rendering, they can heavily interfere with geometry processing applications.


\subsection{Laplacian-Beltrami Operator}\label{sub:rw:lbo}

\paragraph{Laplacian Operator on Meshes.} 
The Laplace-Beltrami operator (LBO) is the generalization of the second derivative on general manifolds and a popular tool in many geometry processing applications. 
Its discretization, especially on triangular meshes, has been studied extensively and it has been shown that not all properties of the continuous Laplacian can be fulfilled in the discrete case at the same time~\cite{wardetzky2007nofreelunch}. 
While it is possible to use the graph Laplacian~\cite{taubin1995fairsurface} on a mesh by discarding the face information, this fails to take into account all information about the local geometry. 
More advanced mesh LBOs, like the cotan discretization~\cite{pinkallporthier,meyer2003discrete} or intrinsic Delaunay discretization~\cite{bobenko2007simplicial}, provide a more accurate approximation of the continuous behaviour. 
These can also be extended to more complex domains, like n-dimensional data~\cite{crane2019ndim}, general polygonal meshes~\cite{alexa2011polygonal,bunge2020polygon}, or non-manifold meshes~\cite{sharp2020nonmanifold}.
However, all of these depend on explicitly given connectivity information which can guarantee certain properties but does not work for less structured shape representations, like point clouds or Gaussian splatting.

\paragraph{Laplacian Operator on Point Clouds.} 
As the discrete Laplacian operator relies on the definition of the neighborhood that is not explicitly given in the point cloud, it is essential to estimate the neighborhood function in a good way so that it approximate the intrinsic connectivity.
A straight-forward solution would be to triangulate the point cloud, however, this is expensive and often leads to errors on sparse or noisy point clouds. 
Instead, a common solution is to \emph{locally} approximate the surface via its tangent plane and projection of surrounding points onto it, often by a nearest neighbor search~\cite{Belkin2009ConstructingLO}. 
This works well on smooth or flat regions, but struggles around very sharp features. 
The resulting inaccuracies can be diminished by building an operator that is robust to incorrectly found and non-manifold surface connections~\cite{sharp2020nonmanifold}, or by 
employing improvements in the surface estimation for these cases, for example through anisotropic Voronoi diagrams~\cite{Qin2018anisotropic}, or physic dynamics~\cite{petronetto2013meshfree}.
The recent work of \cite{pang2024neurallaplacianoperator3d} avoids direct estimation of the surface by learning the behavior of the LBO on different examples and then generalizing the behavior directly to new point clouds. 
While the centers of a Gaussian splatting do form a point cloud on which the previous methods can be applied, the directional variance at each point provides valuable additional information about the surface. 
In this work we propose a more accurate way to extract the LBO from Gaussian splatting which includes the variances in the surface estimation. 


