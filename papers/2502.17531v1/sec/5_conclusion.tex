\section{Conclusion}
\label{sec:conclusion}

We defined a new Laplace Beltrami operator that can be directly computed on 3D Gaussian splatting by using the Mahalanobis distance on the Gaussian splats using the variance to determine the neighborhood and filter out outlier splats. 
Our experiments show that this performs much better than just computing the point cloud Laplacian on the point cloud of Gaussian centers, and also does not require an extra expensive step like extracting a mesh from the 3D Gaussians. 
Additionally, while an extracted mesh can lead to accurate computations, the mesh might also contain topological noise (e.g. in the form of disconnected components) which distorts the results significantly. 
Our Laplacian allows common geometry processing applications, like shape matching, to be directly performed on 3DGS, and the GS Laplacian can be used during training to determine convergence and improve the quality of the geometry. 
