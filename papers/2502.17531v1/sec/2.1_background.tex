\section{Background}
\label{sec:background}

In this section we introduce the background on Gaussian splatting and the Laplace-Beltrami operator necessary to understand the rest of the paper. 

\subsection{Gaussian Splatting} \label{sub:bg:gaussian}

3D Gaussian splatting~\cite{kerbl3Dgaussians} represents a scene as a collection of 3D Gaussian distributions $\{ (\mu_i, \Sigma_i, \alpha_i, c_i) \}_i$ with mean $\mu_i \in \mathbb{R}^3$, variance $\Sigma_i \in \mathbb{R}^{3 \times 3}$, alpha value $\alpha_i$, and color function $c_i$ in spherical harmonic representation. 
This collection can be easily projected to 2D and rendered by accumulating density along a ray. 
The parameters of each Gaussian splat are optimized to render to a set of training images from different view points. 
Since this optimization is focused on rendering, the 3D Gaussian splats are not necessarily localized on the surface of the objects in the scene. 
There have been efforts to align the 3D Gaussian splats to the surface by regularizing on the SDF~\cite{guedon2023sugar}, the depth and the normal~\cite{yu2024gaussian}, or training 2D Gaussian splatting~\cite{Huang2DGS2024}, that focus on the geometry of the reconstructed mesh.

\subsection{(Discrete) Laplace-Beltrami Operator} \label{sub:bg:lbo}

The Laplace-Beltrami operator (LBO) $L = \text{div}\cdot \nabla$ generalizes the second derivative to general closed compact manifolds. 
The operator and its eigenfunctions and eigenvalues, which are non-trivial solutions to $L \phi_i = \lambda_i \phi_i$, are popular tools in geometry processing (see \cref{sub:rw:lbo}). 
When discretizing the underlying manifold, the LBO also has to be discretized which leads to approximation artifacts \cite{wardetzky2007nofreelunch}. 

\begin{figure}
    \centering
    (a) \begin{tikzpicture}[scale=1.1]
    \coordinate (i) at (0,0);
    \coordinate (A) at (1,0);
    \coordinate (B) at (0.5,0.6);
    \coordinate (C) at (-1,0.5);
    \coordinate (D) at (-1.2,-0.3);
    \coordinate (E) at (0.1,-1);

    \node[label=below:{$v_j$}] (EN) at (E) {};
    \node[label=above:{$v_i$}] (iN) at (i) {};

    \draw (i) -- (A);
    \draw (i) -- (B);
    \draw (i) -- (C);
    \draw (i) -- (D);
    \draw (i) -- (E);

    \draw (A) -- (B);
    \draw (B) -- (C);
    \draw (C) -- (D);
    \draw (D) -- (E);
    \draw (E) -- (A);

    \pic [draw=red, -, "\footnotesize{$\alpha_{ij}$}", angle eccentricity=1.5] {angle = E--D--i};
    \pic [draw=red, -, "\footnotesize{$\beta_{ij}$}", angle eccentricity=1.4] {angle = i--A--E};
       
      
    \draw[fill=black, draw=black] (i) circle (2pt);
    \draw[fill=black, draw=black] (A) circle (2pt);
    \draw[fill=black, draw=black] (B) circle (2pt);
    \draw[fill=black, draw=black] (C) circle (2pt);
    \draw[fill=black, draw=black] (D) circle (2pt);
    \draw[fill=black, draw=black] (E) circle (2pt);
\end{tikzpicture}

    (b) \includegraphics[width=.4\linewidth]{figures/projection.png}
    \caption{Discretization of the Laplace Beltrami operator on a mesh (a) and a point cloud (b). In the mesh the connectivity is directly given and can be used to compute properties like angles. The point cloud Laplacian relies on a approximation of tangent plane at a point on which the neighborhood is projected (blue). }
    \label{fig:cotan}
\end{figure}

The cotan-discretization \cite{pinkallporthier} for triangular meshes is defined as follows:
\begin{align}
    W_{ij} = \begin{cases}
        \frac{1}{2} (\text{cot}\alpha_{ij} + \text{cot} \beta_{ij}), & \text{if } (i,j) \in E \\
        - \sum_{k \in \mathcal{N}(i)} w_{ik}, & \text{if } j=i \\
        0, & \text{otherwise}
    \end{cases}
\end{align}
where $\alpha_{ij}, \beta_{ij}$ are the opposing angles to the edge between vertices $v_i, v_j$ (see \cref{fig:cotan}). 
In combination with the diagonal mass matrix $M$ describing the local weight at each vertex $M_{ii}$ the area of the voronoi cell around vertex $i$, the LBO is $L = M^{-1} W$.
While the default cotan-Laplacian does not fulfill the maximum principle, it will when applied to the intrinsic Delaunay triangulation of the input mesh \cite{bobenko2007simplicial}.
However, this depends on a clean mesh with only triangles and no non-manifoldness (e.g., edges with three triangles attached). 
For meshes of arbitrary topology, \cite{sharp2020nonmanifold} suggested to use the tufted cover, which generates an implicit manifold overlay of a given connectivity, in combination with the intrinsic triangulation. 
Due to its flexibility w.r.t. the connectivity, the tufted Laplacian is also well-suited to be used on point clouds by approximating a local neighborhood and connectivity for each point without the need to reconstruct a full, consistent triangle mesh, which is expensive and prone to be noisy. 
