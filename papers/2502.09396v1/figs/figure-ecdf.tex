\begin{figure}[t!]
    \centering
    \includegraphics[width=1.0\textwidth]{figure-ecdf_ssd.pdf}
    \caption{Empirical cumulative distribution functions for proportion of cases ($y$-axis) vs.\ value of different metrics ($x$-axis). Top row shows vulnerability to \textit{all} structural metrics, where red lines correspond to hypothesis 2. Bottom row shows vulnerability to \textit{any} of the structural metrics, where green lines relate to hypothesis 1. Green lines are when the structural risk metric is 0 (low risk) and red lines when risk is 1 (high risk) with solid lines for all cases, and dashed lines with markers showing only those cases where the reported MIA metric is statistically significant with $>$95\% confidence. In an ideal case, the green lines would rise immediately to 1 at the origin; that is, all models with no structural risk would have zero MIA risk. By contrast, ideally the red lines (structural risk is present) would begin low then rise only if significant MIA metrics were observed. If all observed results were statistically significant then the paired lines (i.e., with/without cross markers) would be identical.}%
    \label{fig:ecdf_metrics_ssd}
\end{figure}
