\begin{figure*}[t]
    \centering   \includegraphics[width=0.98\textwidth]{figs/IG-gain-double.pdf} 
    \vspace{-13pt} 
     % 动态调整图形和标题之间的间距
  % \ifdim\baselineskip>0.1in
  %   \vspace*{-\baselineskip} % 清除默认间距
  %   \vspace{0.1in}           % 设置固定间距为 0.1 英寸
\caption{
% \alex{This figure looks good. 1. Make the y axis title bold. 
% 2.I do not understand x axis. for "identification" it should be tokenizer into multiple sub tokens like "identif" or others?. 
% But why only have one point?
% If it split into multiple tokens, it is best to make the figure height smaller or present the word in x-axis rather than rotated.
% }
\textbf{Token-Level Information Gain} (IG) in sentence ``\textit{So \myred{transference} is a \myred{therapeut}ic \myred{device} but also \myred{something} that has always \myred{operated} in \myred{creation}.}". The red dashed line masks of 50\% the most frequent tokens based on training set statistics. 
This strategy preserves tokens with higher information gain, while eliminating statistically prevalent but low-value elements, thus enhancing semantic density.
% IG calculation is in Appendix.
}
 \label{fig:info_gain}
 \vspace{-12pt}   
\end{figure*}

% corpus-level frequency distribution