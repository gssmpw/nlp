\begin{figure*}[!t]
    \centering
    \includegraphics[width=0.9\textwidth]{figs/oview.pdf} 
    \small
  \vspace{-5pt}
  % 动态调整图形和标题之间的间距
  % \ifdim\baselineskip>0.1in
    % \vspace*{-\baselineskip} % 清除默认间距
    % \vspace{0.1in}           % 设置固定间距为 0.1 英寸
    
    \caption{$_{\!}$\textbf{Overview of \textsc{Vist}}. 
    \textsc{Vist} addresses the context window limitation of LLMs by converting long text into compact visual representations by a lightweight visual encoder.
    These features are then integrated into LLM via cross-attention, enabling efficient processing of extended contexts.
   To prioritize informative content, \textsc{Vist} employs \textbf{Frequency-based Masking} on text token embeddings, suppressing high-frequency but low-information tokens \raisebox{-0.1em}{\tikz \filldraw[myblue, rounded corners=1pt] (0,0) rectangle (0.8em,0.8em);} (\eg, ``the" and ``with").
    Such refined embeddings guide the Resampler in extracting critical semantics from the images.
    This dual mechanism—context expansion through visual abstraction and probability-informed visual semantic enrichment—facilitates robust representations and opens new possibilities for long-context compression.
     }
    \label{fig:overview}
\vspace{-10pt}
\end{figure*}

% ($\hat{\bm{F}}_\text{v}$ and $\hat{\bm{F}}_\text{a}$)