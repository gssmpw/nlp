\section{Related Work}
\subsection{Children's Social Online Game}
Social games typically provide a gaming experience that involves more than one player, emphasizing interactions between players and spectators. Social game interactions can include voice or text-based chat, non-verbal communication like pointing and touching, message boards, and features such as combat and trading \cite{DeKort,Emmerich,GONCALVES2023107851}. Although there are various definitions of social games, one review paper identified five common ways that research characterizes social gaming: 1) non-solitary play, (2) design intent, (3) the interactions it promotes, (4) the resulting social outcomes, and (5) the inherent social nature of gaming. Social games often challenge players in ways that promote competition and collaboration, requiring active communication both within and outside the game environment to achieve success \cite{Depping2018,depping2017cooperation,Depping2016,nardi2006strangers}.

\rr{Social online games are multiplayer games where players interact with each other through computer mediated communication and collaborative or competitive gameplay \cite{domahidi2014dwell, domahidi2018longitudinal}.} These games are immensely popular among children and adolescents, who make up the majority of users. For example, Minecraft has become the best-selling game of all time, with over 180 million monthly active players \cite{minecraftStates}. Similarly, Roblox, another prominent social and Metaverse gaming platform, has experienced dramatic growth, reaching over 70 million daily active users worldwide \cite{robloxStats}. In 2024, Roblox reported over 32 million daily active users under the age of 13, up from 28.2 million in the previous year for the same age group \cite{robloxStats-children}. These platforms have become central to the Metaverse experience for Gen Z (born from 1995 to 2009) and Gen Alpha (born from 2010 to 2024) children.

Playing digital social games is crucial for children's social well-being and the development of social-emotional skills \cite{parkash2022utilizing,Granic2014-aj,przybylski2010motivational}. Young people often play with peers, discuss games, and use gaming as a way to expand their social interactions and relationships, thereby increasing social closeness \cite{lenhart2008teens,Orleans2000, olson2010children, depping2017cooperation}. Research highlights the cognitive, motivational, emotional, educational, health, and social benefits of gameplay \cite{Granic2014-aj}. For instance, for children on the autism spectrum, games can serve as powerful tools for social and meta-cognitive learning \cite{Bartoli2013, Hiniker2013}, with Minecraft, in particular, empowering autistic youth to modify the game to better support their emotional and social needs \cite{Ringland2016}. Co-playing with parents and other children can also serve as a bonding opportunity, with studies finding that increased gaming time with siblings leads to greater affection \cite{coyne2016super,COYNE2011160}.

Despite the recognized benefits of social games, parental opinions are mixed. Concerns include children spending excessive time on video games \cite{mcclain2022parents}, games interfering with family time or sleep \cite{Freed2020}, and exposure to violent or sexual content \cite{kutner2008parents}. From the perspective of young players, while they acknowledge the risks associated with social gaming—such as playing with a mean-spirited stranger\cite{Carter2020}—many view these social features as integral to their enjoyment of the game \cite{livingstone2021playful}. However, the value of sibling play in social gaming remains understudied \cite{kaitlin2024}.

In our study, we allowed parents and children to determine their interpretation of social gaming and interviewed children who play Minecraft, Roblox, Fortnite, Genshin Impact, and Rocket League. Our focus was on exploring children’s perspectives on avatar creation and the motivations and meanings behind their choices in these social games.

\subsection{Avatars in \rr{Social Online Game}}
Avatars aren't merely digital puppets; they become sites of 'subjectification,' where individuals are both shaped by and actively negotiate systems of governance. Avatar customization choices are influenced by platform affordances, social norms, and community expectations, but users also creatively express themselves through these constraints, sometimes even subverting them \cite{bardzell2014lonely}. In social online games, players can create and customize avatars to serve both as their visual representation and the primary means of interacting with the virtual world and other players \cite{szolin2022gaming, castronova2003theory, mazlan2012students}. An avatar can be any element that represents a user, from human-like characters to non-human representations such as cars, animals, or even icons and text \cite{boberg2008designing}, and most avatars combine elements of both reality and imagination \cite{mazlan2012students}. Social online games often offer extensive customization options, allowing users to modify various aspects of their avatars, such as hairstyles, skins (the textures that are placed on avatars, e.g. in Fortnite), outfits, accessories, and non-human features. When creating avatars, players often balance their self-image with how they want to be perceived by others \cite{szolin2022gaming}. Avatars are central to social online games, enabling players to express their identities, preferences, and personalities in a virtual environment. The ability to customize avatars has been shown to influence players’ sense of presence \cite{bailey2009avatar}, playtime \cite{castronova2003theory}, cognitive and emotional processing \cite{bailey2009avatar}, emotional arousal \cite{chung2008avatar},  and overall enjoyment \cite{turkay2010enjoyment, bailey2009avatar, Birk2016, trepte2010avatar}. 

Factors such as the capacity to identify with a person or a character motivates players to engage in avatar creation \cite{foster2008games}. The identification process allows players to experiment with different aspects of their personalities and strategically select traits they wish to project through their avatars \cite{waggoner2007passage, triberti2017changing, wood2020me}. \rr{For example, research shows gifted adolescents may identify as ``gamers,'' and features for customizing avatars offer them opportunities for exploring multiple identities, experimenting with different roles, and navigating social interactions in a low-risk environment \cite{wood2020me}. Studies about avatars in Virtual Reality (VR) show that while some users experiment with different identities through their avatars, many prioritize presenting a consistent self that resembles their physical self. Aesthetics play a significant role in initial social interactions, with visually appealing avatars more likely to draw positive attention \cite{freeman2021body}.} In addition, identifying with avatar can even influence players' behavior outside the game \cite{yee2007proteus}, making them more susceptible to persuasive messages \cite{moyer2011identification}. 





Prior research Research shows people’s avatars in online games both reflect and contribute to aspects of their identity formation, including aspects of age, gender, race/ethnicity, and socio-economic status \cite{triberti2017changing}. For example, some researches find young people use avatars to explore gender by simply presenting as a different gender or developing gender-diverse identities in preparation for coming out as transgender or non-binary in real life \cite{crowe2014click, morgan2020role, hussain2008gender}. In addition, another research notes as adolescents get older, they create more detailed representations of themselves, including features associated with changes during adolescence and puberty, showing avatars are reflections of players real-world change. \rr{The avatar-user relationship is not monolithic but complex and dynamic, varying across different virtual worlds and contexts.  The relationship between a user and their avatar(s) evolves over time, reflecting changing uses, purposes, and expectations \cite{de2012my}.} 

\rr{Compared to adults, children’s avatar creation is often more focused on playfulness and creativity. Children are likely to choose avatars based on fun and imaginative aspects, often opting for characters that are colorful, whimsical, or fantastical \cite{kriglstein2013study}. This difference in approach can be attributed to cognitive development stages; children are still forming their self-concept and may not yet engage in the same level of self-reflection as adults \cite{principe2013children}. In addition, children do not exhibit the same level of concern for gender representation in avatars as adults. Their choices are often influenced by the characters they admire from media or their peers, leading to a more fluid approach to gender in avatar selection \cite{kriglstein2013study}. }

Game design can affect how players create avatars. For instance, Crowe \& Watts \cite{crowe2014click} found that avatar customization options for female characters in a game were often stereotypical and highly sexualized, featuring exaggerated body shapes and revealing clothing. The option limitation influences players variably, while some teenage girls expressed frustration with this limited representation, others appreciated the ability to make their characters appear conventionally attractive. Similarly, another study shows there are limited customization options for avatars that do not align with the default racial characteristics often associated with White players \cite{fields2012navigating}. 

\rr{Monetization strategies in social online games also play a crucial role in shaping avatar customization behaviors. Micro-transactions—small in-game purchases made with real-world money or virtual currency—enable players to acquire digital goods, upgrades, or content that enhance their avatars \cite{gibson2022relationship}. These transactions range from purely cosmetic items, such as skins that alter the appearance of characters or items, to more functional enhancements that can impact gameplay. Research has demonstrated that monetization mechanisms exploit various cognitive biases to drive repeat purchasing. For example, loot boxes, pay-to-win models, and limited-time offers leverage the sunk-cost fallacy and scarcity bias to encourage players to spend more frequently \cite{james2022between, petrovskaya2021predatory, gonzalez2024mediating}. The frequency and nature of these micro-transactions have been identified as key indicators of problematic gaming and gambling behaviors \cite{gibson2022relationship}.}

Specifically relating to avatar customization, micro-transactions foster the development of virtual shopping behaviors among children aged 8-12. Hota and Derbaix \cite{hota2016real} observed that children engage in purchasing accessories to personalize their avatars, with motivations varying by gender. Boys are often driven by the desire for functional advantages related to power and progress within the game, while girls are more motivated by the need to enhance their social status and aesthetic appeal. \rr{These digital items are inherently alluring and socially significant, compelling children to invest in paid options to improve their social standing and identity representation in online spaces \cite{james2022between}. Moreover, predatory monetization strategies exacerbate these behaviors by obscuring real-world costs and creating barriers to spending transparency. In-game virtual currencies often make it difficult for players, especially children, to understand the true monetary value of their purchases \cite{james2022between, petrovskaya2021predatory}. These strategies frequently result in bundled sales, where players accumulate surplus virtual currency that cannot be easily spent, thereby encouraging further purchases to utilize these leftover funds \cite{report}.}




Our study builds upon this literature by focusing on children's perceptions and motivations in avatar creation and how game design and monetization influences children's choices. While previous research has explored the role of avatars in games mostly in the view of adults, we contribute to this field by examining how children perceive, create, and customize avatars, and how game design and monetization strategies shape these processes. By interviewing children, we provide insights into their needs and motivations in avatar-making within digital social games.