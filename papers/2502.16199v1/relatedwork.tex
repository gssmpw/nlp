\section{Related Work}
\label{sec:Background}
% This section discusses three aspects of related work: wireless key generation, compressed sensing, and LoRa security.
\subsection{Large language models}
% Recent advancements in LLMs have dramatically advanced artificial intelligence and natural language processing. LLMs such as OpenAI's GPT-3 and Meta's LLaMA 2~\cite{touvron2023llama} have shown exceptional capabilities in generating coherent and contextually relevant narratives, handling complex tasks such as multilingual translation, query responses, and code generation.
% Historically, the development of neural language models (NLMs)\cite{arisoy2012deep} and early LLMs like GPT-2\cite{radford2019language} and BERT~\cite{devlin2018bert} 
% set foundational milestones. 
% The development of these models has progressively featured enhanced complexity and capabilities, as exemplified by PaLM~\cite{chowdhery2023palm} and GPT-4.
% Zero-shot generalization has significantly enhanced the utility of LLMs as assistants, prompting the development of methods aimed at aligning LLMs with human preferences and instructions. 
% In this study, we demonstrate that the zero-shot generalization capabilities of LLMs and their preference towards compressible patterns are not limited to language understanding but can also be effectively applied to transportation sensor data.
Recent breakthroughs in large language models (LLMs) have significantly propelled advancements in artificial intelligence and natural language processing. Models like OpenAI's GPT-3 and Meta's LLaMA 2~\cite{touvron2023llama} have demonstrated remarkable proficiency in producing contextually appropriate and coherent text, excelling in tasks such as multilingual translation, question answering, and code generation.
The evolution of neural language models (NLMs)\cite{arisoy2012deep}, alongside earlier LLMs like GPT-2\cite{radford2019language} and BERT~\cite{devlin2018bert}, laid the groundwork for these advancements. Over time, language models have grown in sophistication and capability, as seen with more recent models like PaLM~\cite{chowdhery2023palm} and GPT-4~\cite{achiam2023gpt}.
% A key factor in enhancing the versatility of LLMs is their ability to generalize in zero-shot scenarios, driving the development of techniques to align these models with human intentions and preferences.

\subsection{Wireless Key Generation} 

% Wireless key generation has received considerable attention over the past decades. In the literature, a large volume of systems have been proposed for different wireless technologies, such as Wi-Fi~\cite{xi2016instant, liu2013fast}, Zigbee~\cite{Aonowireless2005}, and Bluetooth~\cite{premnath2014secret}. In these studies, researchers have used a variety of physical-layer features, including Channel State Information (CSI) \cite{xi2016instant, liu2013fast}, RSSI \cite{yang2022vehicle, jiayaoipsn}, and phase \cite{wang2011fast}. For example, TDS \cite{xi2016instant} exploited Wi-Fi CSI as channel characteristics to generate keys for mobile devices. 
% To enhance the channel reciprocity of CSI, Liu \textit{et al.}~\cite{liu2013fast} leveraged channel response in multiple Orthogonal Frequency-Division Multiplexing (OFDM) subcarriers, coupled with a Channel Gain Complement (CGC) scheme for key generation. 
Wireless key generation~\cite{zhang2016key} has attracted significant interest over the past few decades. Numerous systems have been developed across various wireless technologies, including Wi-Fi~\cite{xi2016instant, liu2013fast}, Zigbee~\cite{Aonowireless2005}, LoRa~\cite{zhang2018channel, ruotsalainen2019experimental, xulorakey, jiayaoipsn, yang2022vehicle}, and Bluetooth~\cite{premnath2014secret}. Researchers have explored diverse physical-layer attributes, such as Channel State Information (CSI)\cite{xi2016instant, liu2013fast}, Received Signal Strength Indicator (RSSI)\cite{yang2022vehicle, jiayaoipsn}, and phase~\cite{wang2011fast}.
For instance, TDS~\cite{xi2016instant} utilized Wi-Fi CSI to derive cryptographic keys for mobile devices by leveraging channel characteristics. To further improve CSI reciprocity, Liu et al.~\cite{liu2013fast} employed channel responses from multiple Orthogonal Frequency-Division Multiplexing (OFDM) subcarriers. This approach was complemented by a Channel Gain Complement (CGC) scheme to facilitate more reliable key generation.

\subsection{LLMs for Time Series Analysis}
The application of LLMs to time series tasks has garnered increasing attention from researchers. For instance, TIME-LLM repurposed an existing LLM for time series forecasting while maintaining the original architecture of the language model~\cite{jin2023time}. 
% This approach reformatted time series data into textual prototypes, allowing for alignment between textual and time series representations.
Similarly, LLMTIME~\cite{gruver2024large} leveraged pretrained LLMs for continuous time series prediction by encoding numerical values as text and generating forecasts through text-based completions. However, this method primarily focuses on the temporal dimension, overlooking spatial dependencies in the data.
To address spatial considerations, GATGPT integrated a graph attention network with GPT to perform spatio-temporal imputation~\cite{chen2023gatgpt}. 
Furthermore, researchers have explored the use of LLMs for interpreting sensor data to facilitate real-world sensing~\cite{ji2024hargpt,yang2024you,yang2024transcompressor}.

% \subsection{LLMs for Time Series}

% Several researchers have investigated the potential of using LLMs for time series applications. 
% For example, TIME-LLM adapted an existing LLM for time series forecasting while preserving the original language model structure~\cite{jin2023time}. The model reprogrammed the input time series as text prototypes, facilitating alignment between text and time series modalities.
% LLMTIME~\cite{gruver2024large} harnessed pretrained LLMs for continuous time series forecasting by representing numerical values in textual format and generating extrapolations through text completions. This model, however, only addresses the temporal dimension of the data, neglecting spatial aspects.
% GATGPT combined the graph attention network with GPT for spatial-temporal imputation~\cite{chen2023gatgpt}, enhancing the LLM's ability to understand spatial dependencies, though it somewhat neglects the temporal aspects. Besides, some researchers propose to use LLMs to understand sensor data to sense the physical world~\cite{ji2024hargpt,yang2024you,yang2024transcompressor}. 
% However, these methods, designed specifically for time series forecasting and understanding, are not suitable for transportation sensor data compression and reconstruction. Therefore, we propose \SystemName—a system that can effectively reconstruct compressed sensor data by leveraging the extensive knowledge base of LLMs.



% LoRa is an emerging wireless communication technology designed specifically for long-range and low-power communications. The low data rate and long airtime feature of LoRa bring new research challenges. To address these challenges, several key generation systems for LoRa networks have been proposed in recent years. For example, LoRa-Key~\cite{xulorakey} is the first RSSI-based key generation method for LoRa. In their follow-up research~\cite{jiayaoipsn}, a variant RSSI feature, register RSSI, is exploited, which can provide finer granularity of channel sampling for key generation. 
% Recently, Yang \textit{et al.}~\cite{yang2022vehicle} utilized the mean value of adjacent rRSSI (arRSSI) of LoRa signals for key establishment on Internet of Vehicles (IoV) scenarios. 
% However, the performance of these systems is limited because of the use of coarse-grained channel characteristics and inefficient quantization mechanism. Our work differs from existing studies in two aspects. First, we propose a novel LoRa-specific channel characteristics that can provide fine-grained channel state information. Second, we propose a novel PCS-based key delivery method instead of using quantization-based key generation methods.  

% \textbf{Compressed Sensing.} 
% Compressed sensing is a signal processing technique used to reconstruct signals efficiently by finding solutions for underdetermined linear systems. In addition to data compression, it can also  be applied in IoT security schemes \cite{dautov2013establishing, xue2017kryptein}. For example, 
% H2B~\cite{lin2019h2b} used a compressed sensing-based reconciliation method to correct key mismatches due to the low SNR of the heartbeat interval signals. Additionally, Kryptein~\cite{xue2017kryptein} proposed a compressed sensing-based encryption method to enable secure data queries for cloud-enabled IoT systems. Dautov \textit{et al.}~\cite{dautov2013establishing} explored the feasibility of constructing secure compressed sensing matrices based on wireless physical-layer security.
% These existing works apply compressed sensing to different IoT security scenarios, which inspires this paper to pioneer the use of a compressed sensing framework for wireless key generation. Unlike existing studies that used standard compressed sensing techniques for IoT security, our work uses the perturbed version of compressed sensing for the physical-layer key generation, which is more practical in real-world applications.


% \textbf{LoRa Security.}
% With massive deployments of LoRa networks, the security issues have attracted significant efforts~\cite{li2022lora, sun2022recent, hou2022cloaklora,hou2021jamming}, with an emphasis on key management, authentication, and physical-layer key generation. 
% For key management, existing works mainly focus on the derivation, distribution, and destruction process of application layer keys, including DevNonce~\cite{tomasin2017security} and NwkSKey~\cite{kim2017dual}. 
% Additionally, researchers have exploited different features of LoRa signals resulted from hardware imperfections for device authentication purpose, such as carrier frequency offset~\cite{wang2020slora}, amplitude-phase~\cite{al2021deeplora}, and signal spectrogram~\cite{shen2021radio}. 
% However, such methods generally suffer from heavy computation cost and poor scalability. 
% Thus, key generation has emerged as a promising solution for secure wireless communication due to its high energy efficiency and prior-free requirements compared with the aforementioned schemes. In this paper, we propose a novel secret key generation scheme to provide robust and lightweight key establishment for LoRa networks.

\begin{figure*}
    \centering
    \includegraphics[width=5.5in]{figure/overview_llmkey.pdf}
    \caption{System workflow.}
    \label{fig:workflow}
    \vspace{-0.3in}
\end{figure*}