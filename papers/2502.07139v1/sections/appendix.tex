

\section{Prompt Details}\label{sec:prompt}

We present the detailed prompt templates used for preprocessing temporal point process datasets. The templates consist of system prompts and event-specific formats.

\textbf{System prompt templates:} there are two variants of system templates used in the preprocessing where the original number prompt is used in~\Cref{ssec:ablation}

\begin{table}[h]
\small
\begin{tabular}{|p{0.95\textwidth}|}
\hline
\textbf{Byte-token Prompt:} \\
$<|im\_start|>$system\\
textual representation of an event sequence denoted by event times in float Byte-tokens (each number as 4 byte tokens) along with textual event types\\
INFO: \{sequence\_info\}\\
$<|im\_end|>$\\
$<|im\_start|>$sequence \\
\hline
\textbf{Original Number Prompt:} \\
$<|im\_start|>$system\\
textual representation of an event sequence denoted by event times in float numbers along with textual event types\\
INFO: \{sequence\_info\}\\
$<|im\_end|>$\\
$<|im\_start|>$sequence \\
\hline
\end{tabular}
\end{table}
where \{sequence\_info\} refers to dataset-specific information

\begin{table}[h]
\small
\begin{tabular}{|l|p{0.7\textwidth}|}
\hline
\textbf{Dataset} & \textbf{Information} \\
\hline
\textit{StackOverflow} & This sequence is a sequence of badges awarded to a user in StackOverflow. There are 22 event types. \\
\hline
\textit{Retweet} & This sequence is a sequence of retweets of a tweet. There are 3 event types. \\
\hline
\textit{Taobao} & This sequence is a sequence of clicks on a product in Taobao. There are 20 event types. \\
\hline
\textit{Taxi} & This sequence tracks taxi pick-up and drop-off events. There are 10 event types. \\
\hline
\textit{Amazon Review} & This sequence is a product review event from an Amazon user where event type is product category. \\
\hline
\end{tabular}
\label{tab:dataset_info}
\end{table}
After the system prompt,
individual events are formatted using the following template:

\begin{table}[h]
\small
\begin{tabular}{|p{0.95\textwidth}|}
\hline
$<|start\_of\_event|>$\\
$<|type\_prefix|>$\{event\_type\}$<|description\_prefix|>$\{event\_description\}$<|time\_prefix|>$\{event\_time\}\\
$<|end\_of\_event|>$ \\
\hline
\end{tabular}
\end{table}
where \{event\_type\}, \{event\_description\} and \{event\_time\} are textual content of an event in a temporal point process.

\textbf{Special tokens:} we present all special tokens added to the Qwen2.5 tokenizer vocabulary and used in our work to structure the prompts in~\Cref{tab:special_tokens}.

\begin{table}[h]
\small
\centering
\begin{tabular}{|l|p{0.65\textwidth}|}
\hline
\textbf{Special Token} & \textbf{Description} \\
\hline
$<|start\_of\_event|>$ & Tokens for marking the start of an event \\
$<|end\_of\_event|>$ & Tokens for marking the end of an event \\
\hline
$<|type\_prefix|>$ & Prefix token for event type \\
$<|description\_prefix|>$ & Prefix token for event description \\
$<|time\_prefix|>$ & Prefix token for event timestamp \\
\hline
$<|type\_prediction|>$ & Task token for type inference \\
$<|description\_prediction|>$ & Task token for description inference \\
$<|time\_prediction|>$ & Task token for time inference \\
\hline
$<|byte\_0|>$ to $<|byte\_255|>$ & Byte tokens for representing event time intervals as float32 numbers \\
\hline
\end{tabular}
\caption{Special tokens used in prompt templates}
\label{tab:special_tokens}
\end{table}



\textbf{Event sequence samples:} we provide two samples of generated event sequences from \textit{Amazon Review} dataset in~\Cref{tab:event_sample} and \textit{StackOverflow} in~\Cref{tab:event_sample_2}, respectively. These are used in training stage 1. We also show a sample of prompt-response pair from \textit{Amazon Review} used in next-event fine-tuning in~\Cref{tab:prompt_response}. We note that $<|im\_start|>$ and $<|im\_end|>$ are built-in special tokens from the Qwen2.5 tokenizer.

\begin{table}[htb]
\small
\begin{tabular}{|p{0.95\textwidth}|}
\hline
\vspace{0.1cm}
\textbf{Prompt:} \\
$<|im\_start|>$system\\
textual representation of an event sequence denoted by event times in float Byte-tokens (each number as 4 byte tokens) along with textual event types\\
INFO: This sequence is a product review event from an Amazon user where event type is product category\\
$<|im\_end|>$\\
$<|im\_start|>$sequence\\
$<|start\_of\_event|>$\\
$<|type\_prefix|>$Luggage \& Travel Gear\\
$<|description\_prefix|>$Great bag..Scooby in pink!\\
$<|time\_prefix|>$$<|byte\_0|>$$<|byte\_0|>$$<|byte\_0|>$$<|byte\_0|>$\\
$<|end\_of\_event|>$\\
$<|start\_of\_event|>$\\
$<|type\_prefix|>$Children Shoes\\
$<|description\_prefix|>$Twinkle toes is the best!\\
$<|time\_prefix|>$$<|byte\_68|>$$<|byte\_42|>$$<|byte\_0|>$$<|byte\_0|>$\\
$<|end\_of\_event|>$\\
$<|start\_of\_event|>$\\
$<|type\_prefix|>$Women Jewelry\\
$<|description\_prefix|>$Pretty earrings.\\
$<|time\_prefix|>$$<|byte\_66|>$$<|byte\_140|>$$<|byte\_0|>$$<|byte\_0|>$\\
$<|end\_of\_event|>$\\
$<|start\_of\_event|>$\\
$<|type\_prefix|>$Men Uniforms, Work \& Safety\\
$<|description\_prefix|>$Nice, work shirt.\\
$<|time\_prefix|>$$<|byte\_66|>$$<|byte\_156|>$$<|byte\_0|>$$<|byte\_0|>$\\
$<|end\_of\_event|>$\\
\hline
\end{tabular}
\caption{Event sequence sample from \textit{Amazon Review}}\label{tab:event_sample}
\end{table}

\begin{table}[h]
\small
\begin{tabular}{|p{0.95\textwidth}|}
\hline
\vspace{0.1cm}
\textbf{Prompt:} \\
$<|im\_start|>$system \\
textual representation of an event sequence denoted by event times in float Byte-tokens (each number as 4 byte tokens) along with textual event types \\
INFO: This sequence is a sequence of badges awarded to a user in \textit{StackOverflow}. There are 22 event types. \\
$<|im\_end|>$ \\
$<|im\_start|>$sequence \\
$<|start\_of\_event|>$ \\
$<|type\_prefix|>$5 \\
$<|time\_prefix|>$$<|byte\_0|>$$<|byte\_0|>$$<|byte\_0|>$$<|byte\_0|>$ \\
$<|end\_of\_event|>$ \\
$<|start\_of\_event|>$ \\
$<|type\_prefix|>$13 \\
$<|time\_prefix|>$$<|byte\_58|>$$<|byte\_240|>$$<|byte\_0|>$$<|byte\_0|>$ \\
$<|end\_of\_event|>$ \\
$<|start\_of\_event|>$ \\
$<|type\_prefix|>$3 \\
$<|time\_prefix|>$$<|byte\_64|>$$<|byte\_134|>$$<|byte\_225|>$$<|byte\_0|>$ \\
$<|end\_of\_event|>$ \\
$<|start\_of\_event|>$ \\
$<|type\_prefix|>$3 \\
$<|time\_prefix|>$$<|byte\_62|>$$<|byte\_222|>$$<|byte\_48|>$$<|byte\_0|>$ \\
$<|end\_of\_event|>$ \\
$<|start\_of\_event|>$ \\
$<|type\_prefix|>$2 \\
$<|time\_prefix|>$$<|byte\_61|>$$<|byte\_82|>$$<|byte\_128|>$$<|byte\_0|>$ \\
$<|end\_of\_event|>$ \\
$<|start\_of\_event|>$ \\
$<|type\_prefix|>$3 \\
$<|time\_prefix|>$$<|byte\_64|>$$<|byte\_109|>$$<|byte\_212|>$$<|byte\_0|>$ \\
$<|end\_of\_event|>$ \\
$<|start\_of\_event|>$ \\
$<|type\_prefix|>$3 \\
$<|time\_prefix|>$$<|byte\_62|>$$<|byte\_142|>$$<|byte\_64|>$$<|byte\_0|>$ \\
$<|end\_of\_event|>$ \\
$<|start\_of\_event|>$ \\
$<|type\_prefix|>$5 \\
$<|time\_prefix|>$$<|byte\_63|>$$<|byte\_50|>$$<|byte\_24|>$$<|byte\_0|>$ \\
$<|end\_of\_event|>$ \\
$<|start\_of\_event|>$ \\
$<|type\_prefix|>$13 \\
$<|time\_prefix|>$$<|byte\_60|>$$<|byte\_26|>$$<|byte\_0|>$$<|byte\_0|>$ \\
$<|end\_of\_event|>$ \\
$<|start\_of\_event|>$ \\
$<|type\_prefix|>$7 \\
$<|time\_prefix|>$$<|byte\_63|>$$<|byte\_96|>$$<|byte\_232|>$$<|byte\_0|>$ \\
$<|end\_of\_event|>$ \\
$<|start\_of\_event|>$ \\
$<|type\_prefix|>$7 \\
$<|time\_prefix|>$$<|byte\_0|>$$<|byte\_0|>$$<|byte\_0|>$$<|byte\_0|>$ \\
$<|end\_of\_event|>$ \\
$<|start\_of\_event|>$ \\
$<|type\_prefix|>$17 \\
$<|time\_prefix|>$$<|byte\_57|>$$<|byte\_0|>$$<|byte\_0|>$$<|byte\_0|>$ \\
$<|end\_of\_event|>$ \\
$<|start\_of\_event|>$ \\
$<|type\_prefix|>$17 \\
$<|time\_prefix|>$$<|byte\_0|>$$<|byte\_0|>$$<|byte\_0|>$$<|byte\_0|>$ \\
$<|end\_of\_event|>$ \\
\hline
\end{tabular}
\caption{Event sequence sample from \textit{StackOverflow}}\label{tab:event_sample_2}
\end{table}

\begin{table}[h]
\small
\begin{tabular}{|p{0.95\textwidth}|}
\hline
\vspace{0.1cm}
\textbf{Prompt:} \\
$<|im\_start|>$system \\
textual representation of an event sequence denoted by event times in float Byte-tokens (each number as 4 byte tokens) along with textual event types \\
INFO: This sequence is a product review event from an Amazon user where event type is product category \\
$<|im\_end|>$ \\
$<|im\_start|>$sequence \\
$<|start\_of\_event|>$$<|type\_prefix|>$Men Surf, Skate \& Street$<|description\_prefix|>$I am a long-time fan of Reef sandals \\
$<|time\_prefix|>$$<|byte\_0|>$$<|byte\_0|>$$<|byte\_0|>$$<|byte\_0|>$ \\
$<|end\_of\_event|>$ \\
$<|start\_of\_event|>$$<|type\_prefix|>$Men Shoes$<|description\_prefix|>$I own 8 pair of Allen Edmonds and I like them all.  They are very comfortable \\
$<|time\_prefix|>$$<|byte\_67|>$$<|byte\_18|>$$<|byte\_0|>$$<|byte\_0|>$ \\
$<|end\_of\_event|>$ \\
$<|start\_of\_event|>$$<|type\_prefix|>$Shoe, Jewelry \& Watch Accessories$<|description\_prefix|>$Easy to use \\
$<|time\_prefix|>$$<|byte\_65|>$$<|byte\_144|>$$<|byte\_0|>$$<|byte\_0|>$ \\
$<|end\_of\_event|>$ \\
$<|start\_of\_event|>$$<|type\_prefix|>$Men Shoes$<|description\_prefix|>$Great Shoe \\
$<|time\_prefix|>$$<|byte\_65|>$$<|byte\_224|>$$<|byte\_0|>$$<|byte\_0|>$ \\
$<|end\_of\_event|>$ \\
$<|start\_of\_event|>$ \\
$<|time\_prediction|>$ \\
\hline
\vspace{0.1cm}
\textbf{Response:} \\
$<|byte\_61|>$$<|byte\_130|>$$<|byte\_13|>$$<|byte\_139|>$ \\
\hline
\end{tabular}
\caption{Prompt-response pair sample from \textit{Amazon Review}}\label{tab:prompt_response}
\end{table}


\section{Datasets}\label{ap:datasets}
We provide extra details about the splitting and statistics of the datasets here:
\begin{itemize}
    \item For the conventional TPP datasets, including \textit{Retweet}, \textit{Stackoverflow}, \textit{Taobao} and \textit{Taxi}, we obtain the processed datasets from \url{http://bit.ly/40seop9} following the splitting setups in~\citep{xue2023easytpp}.
    \item \textit{Amazon Review}:The data is chronologically split into training (before 2015-08-01), validation (2015-08-01 to 2016-02-01), and test (after 2016-02-01) sets following setups from~\citet{shi2024language}.
\end{itemize}


\section{Experimental Setups}\label{sssec:hyper}

\textbf{Hyperparameters:} we use similar hyperparameter settings for each training stage shown in~\Cref{tab:hyperparams}. All stages use mixed-precision training with bfloat16, evaluate and save checkpoints at the end of each epoch, and keep only the best performing model. Stage 2 specifically uses accuracy as the metric for selecting the best model.
\begin{table}[htp]
\centering
\begin{tabular}{lccc}
\toprule
\textbf{Hyperparameter} & \textbf{Stage 1} & \textbf{Stage 2} & \textbf{Stage 3} \\
\midrule
Learning Rate & 1e-4 & 1e-4 & 1e-4 \\
Batch Size & 4 & 4 & 32 \\
Weight Decay & 0.01 & 0.01 & 0.01 \\
Number of Epochs & 5 & 5 & 5 \\
Gradient Accumulation Steps & 4 & 4 & 4 \\
\bottomrule
\end{tabular}
\caption{Training hyperparameters for different stages}
\label{tab:hyperparams}
\end{table}

