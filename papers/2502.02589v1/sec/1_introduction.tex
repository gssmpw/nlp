\section{Introduction}
Recent advancements in multi-modal foundation models have been largely driven by the availability of large-scale paired text-image datasets. These datasets, often collected via web crawling with basic filtering techniques~\cite{schuhmann2021laion400m,schuhmann2022laion5b,gadre2024datacomp}, contain low-quality, web-sourced captions that lack depth and accuracy. In contrast, human-annotated caption datasets, such as COCO-caption~\cite{chen2015coco_caption}, offer higher-quality descriptions but are limited in scale and tend to be concise, with an average caption length of 10 words. To overcome the limitations of short captions, the research community has leveraged vision-language models (VLMs)~\cite{liu2024llavanext,li2024recaption,li2024DenseFusion,chen2023sharegpt4v,team2023gemini} to generate detailed synthetic captions. While these machine-generated captions improve visual understanding~\cite{li2024DenseFusion,chen2023sharegpt4v} and generation tasks~\cite{li2024recaption}, they remain inferior to high-quality, human-verified annotations~\cite{Onoe2024docci}.






Addressing this challenge requires balancing scalability and annotation quality, as generating detailed and accurate image descriptions at scale remains labor-intensive~\cite{garg2024imageinwords,Onoe2024docci}. In this paper, we introduce an efficient annotation approach that combines dense mask annotations with commercial VLMs~\cite{chen2023sharegpt4v} to produce high-quality image captions. Our goal is to minimize human effort while generating rich, structured descriptions. 

To achieve this, we base our work on the COCO-caption dataset~\cite{chen2015coco_caption} due to its widespread use and diverse image content.
We revisit the COCO-caption dataset to provide more detailed and comprehensive caption annotations.
Our approach involves creating holistic captions synthesized from region-based dense captions that describe distinct areas within each image. Specifically, we build on recent COCONut panoptic segmentation annotations~\cite{deng2024coconut} to generate a new set of detailed captions by: (a) annotating each segmentation region with a VLM-generated draft, carefully refined through human corrections, and (b) summarizing these region captions into a comprehensive image caption while preserving the grounding correspondence between image masks and object references. This enables a novel task that integrates panoptic segmentation with grounded captioning.
Our structured annotation process ensures that the captions are both \textit{complete}, covering the majority of objects in each image, and \textit{grounded}, with precise segmentation masks.


The final dataset, named \textbf{COCONut-PanCap}, is designed for a wide range of vision-language applications, combining \textbf{Pan}optic segmentation and grounded \textbf{Cap}tioning. It comprises 118K image-text pairs for training, with an average caption length of 203 words, as well as an additional 25K image-text pairs, with an average caption length of 233 words for validation. We demonstrate that COCONut-PanCap significantly boosts the performance of both VLM and text-to-image generation models at the instruction tuning and fine-tuning stages, outperforming recent detailed caption datasets~\cite{Onoe2024docci}. This highlights the potential of our grounding-based captions for both vision-language understanding and image generation tasks.


Our contributions are summarized as follows: 

\begin{itemize}
    \item We propose a caption annotation pipeline leveraging panoptic segmentation to create a high-quality, detailed caption dataset comprising \textbf{143K} annotated images. The resulting annotations are comprehensive, accurate, and include grounding masks, making this dataset substantially larger than recent detailed caption datasets.
   
    \item Our \textbf{COCONut-PanCap} dataset facilitates a new challenging task combining \textbf{P}anoptic segmentation and \textbf{G}rounded \textbf{C}aptioning (\textbf{PGC}). We establish evaluation metrics and settings for this PGC task and benchmark several recent methods to assess performance on this novel challenge. 
    \item We validate the utility of our proposed dataset across various fine-grained Image-to-Text (I2T) and Text-to-Image (T2I) tasks, including detailed caption generation, PGC, visual question answering (VQA), referring segmentation, and text-conditioned image generation. Experimental results show that our dataset significantly enhances model performance across all these tasks.
\end{itemize}







