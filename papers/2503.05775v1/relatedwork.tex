\section{Related Works }
\label{sec:Literature}
The quality and completeness of datasets are essential for effective analysis, especially in time series applications where missing values can obscure key patterns, such as seasonality, and reduce predictive accuracy. Seasonality, characterized by periodic fluctuations over time (e.g., daily or weekly patterns), can critically impact model performance by introducing systematic variations that imputation methods must account for. However, many imputation methods do not incorporate seasonality, relying only on remaining observed data, which simplifies modeling but reduces accuracy. 

For small gaps, basic interpolation methods are commonly used, fitting smooth curves between known data points to estimate missing values. While simple, these methods fail to capture temporal dependencies and can lead to biased results. Other approaches include single-imputation techniques, such as Hot Deck, Cold Deck, and \ac{em}, which replace each missing value with a single estimate but may not reduce bias effectively \cite{DONDERS20061087}. In contrast, \ac{mi} techniques offer advantages by providing information on how missing data impacts parameter estimates \cite{DONDERS20061087}. Advanced methods, such as regression-based imputation, \ac{som} \cite{JUNNINEN20042895}, and \ac{knn}, have proven more effective, particularly in datasets where temporal relationships are significant. \ac{knn}, for instance, fills missing values by identifying the k-closest patterns around the missing data point, utilizing local similarity to improve imputation accuracy \cite{Tarsitano2011}.

Incorporating seasonality into imputation methods can markedly improve accuracy. Techniques such as seasonal adjustment with Kalman filters and linear interpolation on seasonally decomposed data, available in tools like the forecast and zoo R-packages, have proven useful \cite{article2}. Additionally, \ac{sarima} has been applied to seasonal time series, although it struggles with consecutive missing values \cite{article}. More advanced approaches include neural network-based methods, such as \ac{mlp} \cite{atmos14020355} and \ac{lstm} networks, as well as hybrid neural models \cite{Bandara_2021}. Pattern-based methods, like the \ac{tkcm} algorithm, have also been used to handle missing data \cite{article3}. However, many of these methods are limited to single-seasonal patterns and are not well-suited for multiple seasonalities.

All imputation validation approaches rely on ground truth data to evaluate accuracy. Typically, studies create artificial gaps in complete datasets, then fill these gaps and compare the imputed values to the known originals. This approach enables traditional metrics like \ac{rmse} and \ac{mae} to measure a model’s ability to accurately reconstruct missing data. However, this validation strategy assumes that ground truth data is available—an assumption that often does not hold in real-world applications, especially in dynamic environments such as \ac{5g} networks.

This paper addresses this gap by introducing two statistical metrics to evaluate imputation methods without relying on ground truth. These metrics assess how closely the distributions of imputed data align with those of the original data, providing a way to evaluate imputation effectiveness based on internal structure and data consistency. By adapting these metrics, we offer a set of tools that complements existing validation approaches and extends evaluation capabilities to real-world applications where complete datasets are not available.