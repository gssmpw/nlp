\documentclass[conference]{IEEEtran}
\IEEEoverridecommandlockouts
% The preceding line is only needed to identify funding in the first footnote. If that is unneeded, please comment it out.
\usepackage{cite}
\usepackage{amsmath,amssymb,amsfonts}
\usepackage{algorithmic}
\usepackage{graphicx}
\usepackage{textcomp}
\usepackage{xcolor}
\usepackage{balance}
\usepackage{csquotes}
\usepackage{soul}
\usepackage{url}
\usepackage{tabularx}
\usepackage{booktabs} % For better horizontal rules
\usepackage{enumitem}
\usepackage{fontawesome}
\usepackage{array} % For advanced column formatting
\usepackage{multirow} % For advanced column formatting
\usepackage{amsmath}
\usepackage{amssymb}
\usepackage{hyperref}

\newcommand{\dyadicunit}{\vec{e}_i \otimes \vec{e}_j}

\newcolumntype{P}[1]{>{\centering\arraybackslash}p{#1}}

\def\BibTeX{{\rm B\kern-.05em{\sc i\kern-.025em b}\kern-.08em
    T\kern-.1667em\lower.7ex\hbox{E}\kern-.125emX}}

\def\eg{\emph{e.g., }} \def\Eg{\emph{e.g., }}
\def\ie{\emph{i.e., }} \def\Ie{\emph{I.e., }}

\newcommand{\todo}[1]{\noindent \textcolor{orange}{#1}}
\newcommand{\cmh}[1]{\textcolor{blue}{#1}}
\newcommand{\vah}[1]{\textcolor{purple}{#1}}

\newcommand{\orangeLightbulb}{\textcolor[rgb]{0, 0.5, 0}{\faLightbulbO}\phantom{.}}
\newcommand{\darkBlueAnchor}{\textcolor[rgb]{0, 0, 0.5}{\faAnchor}\phantom{.}}
\newcommand{\darkGreenLink}{\textcolor[rgb]{0.9, 0.7, 0.2}{\faLink}\phantom{.}}


\newcommand{\pquotes}[1]{\textcolor[gray]{0.35}{\textit{#1}}}

\makeatletter
\newcommand{\linebreakand}{%
  \end{@IEEEauthorhalign}
  \hfill\mbox{}\par
  \mbox{}\hfill\begin{@IEEEauthorhalign}
}
\makeatother

\begin{document}

\title{The Design of On-Body Robots for Older Adults
\thanks{
\textbf{Authorship Statement.} Conceptualization, Methodology, Validation, Writing - Review \& Editing (all authors), Investigation (VNA, CJ, JL), Formal analysis, Data curation (VNA, CJ), Visualization (VNA, CJ, CH), Writing-Original Draft, Project Administration (VNA), Resources (CH, HP), Supervision (CH, AO, HP, GG), Funding (CH). \textbf{AI Use.} Text edited with LLM; output checked for correctness by authors.}
}

% Designing On-Body Robot Interactions \\with Older Adults\\
% Designing On-Body Human-Robot Interaction

% \thanks{Identify applicable funding agency here. If none, delete this.}

\author{\IEEEauthorblockN{Victor Nikhil Antony}
\IEEEauthorblockA{
\textit{Johns Hopkins University}\\
Baltimore, MD, USA \\
vantony1@jhu.edu}
\and
\IEEEauthorblockN{Clara Jeon}
\IEEEauthorblockA{
\textit{Johns Hopkins University}\\
Baltimore, MD, USA \\
cjeon6@jh.edu}
\and
\IEEEauthorblockN{Jiasheng Li}
\IEEEauthorblockA{
\textit{University of Maryland}\\
College Park, MD, USA \\
jsli@umd.edu}
\and
\IEEEauthorblockN{Ge Gao}
\IEEEauthorblockA{
\textit{University of Maryland}\\
College Park, MD, USA \\
gegao@umd.edu}
\linebreakand
\IEEEauthorblockN{Huaishu Peng}
\IEEEauthorblockA{
\textit{University of Maryland}\\
College Park, MD, USA \\
huaishu@umd.edu}
\and
\IEEEauthorblockN{Anastasia K. Ostrowski}
\IEEEauthorblockA{
\textit{Purdue University}\\
West Lafayette, IN, USA \\
akostrow@purdue.edu}
\and
\IEEEauthorblockN{Chien-Ming Huang}
\IEEEauthorblockA{
\textit{Johns Hopkins University}\\
Baltimore, MD, USA \\
chienming.huang@jhu.edu}
}

% \author{Victor Nikhil Antony$^{1}$, Clara Jeon$^{1}$, Jiasheng Li$^{2}$, Ge Gao$^{2}$, \\ Huaishu Peng$^{2}$, Anastasia K. Ostrowski$^{3}$, and Chien-Ming Huang$^{1}$
% \thanks{$^{1}$Johns Hopkins University, Baltimore, Maryland, USA {\tt\small \{vantony1, cjeon6, chienming.huang\}@jhu.edu}. $^{2}$University of Maryland, College Park, Maryland, USA {\tt\small \{jsli, gegao, huaishu\}@umd.edu}. $^{3}$Purdue University, West Lafayette, IN, USA {\tt\small akostrow@purdue.edu}. \textbf{Authorship Statement.} Conceptualization, Methodology, Validation, Writing - Review \& Editing (all authors), Investigation (VNA, CJ, JL), Formal analysis, Data curation (VNA, CJ), Visualization (VNA, CJ, CH), Writing-Original Draft, Project Administration (VNA), Resources (CH, HP), Supervision (CH, AO, HP, GG), Funding (CH). \textbf{AI Use.} Text edited with LLM; output checked for correctness by authors.}
% \thanks{\textbf{CRediT author statement}. VNA: Conceptualization, Methodology, Validation, Formal analysis, Investigation, Data curation, Writing, Visualization, Project Administration. CJ: Conceptualization, Methodology, Validation, Formal analysis, Investigation, Visualization, Writing-Review\&Editing. JS: Conceptualization, Methodology, Investigation, Writing-Review\&Editing. GG, HP, AKO: Conceptualization, Methodology, Validation, Formal analysis, Writing-Review\&Editing, CMH: Conceptualization, Methodology, Validation, Writing-Review\&editing, Visualization, Supervision, Funding acquisition.}}

\maketitle


% \author{\IEEEauthorblockN{Victor Nikhil Antony}
% \IEEEauthorblockA{
% \textit{Johns Hopkins University}\\
% Baltimore, MD, USA \\
% vantony1@jhu.edu}
% \and
% \IEEEauthorblockN{Clara Jeon}
% \IEEEauthorblockA{
% \textit{Johns Hopkins University}\\
% Baltimore, MD, USA \\
% cjeon6@jh.edu}
% \and
% \IEEEauthorblockN{Jiasheng Li}
% \IEEEauthorblockA{
% \textit{University of Maryland}\\
% College Park, MD, USA \\
% jsli@umd.edu}
% \and
% \IEEEauthorblockN{Ge Gao}
% \IEEEauthorblockA{
% \textit{University of Maryland}\\
% College Park, MD, USA \\
% gegao@umd.edu}
% \and
% \IEEEauthorblockN{Huaishu Peng}
% \IEEEauthorblockA{
% \textit{University of Maryland}\\
% College Park, MD, USA \\
% huaishu@umd.edu}
% \and
% \IEEEauthorblockN{Anastasia K. Ostrowski}
% \IEEEauthorblockA{
% \textit{Purdue University}\\
% West Lafayette, IN, USA \\
% akostrow@purdue.edu}
% \and
% \IEEEauthorblockN{Chien-Ming Huang}
% \IEEEauthorblockA{
% \textit{Johns Hopkins University}\\
% Baltimore, MD, USA \\
% chienming.huang@jhu.edu}
% }

\begin{abstract}
Wearable technology has significantly improved the quality of life for older adults, and the emergence of on-body, movable robots presents new opportunities to further enhance well-being. Yet, the interaction design for these robots remains under-explored, particularly from the perspective of older adults. We present findings from a two-phase co-design process involving 13 older adults to uncover design principles for on-body robots for this population. We identify a rich spectrum of potential applications and characterize a design space to inform how on-body robots should be built for older adults. Our findings highlight the importance of considering factors like co-presence, embodiment, and multi-modal communication. Our work offers design insights to facilitate the integration of on-body robots into daily life and underscores the value of involving older adults in the co-design process to promote usability and acceptance of emerging wearable robotic technologies.
\end{abstract}

\begin{IEEEkeywords}
on-body robots, wearable robots, co-design, older adults, human-robot interaction
\end{IEEEkeywords}

\section{Introduction}

Wearable technology, such as smartwatches, glucose monitors, and hearing aids, has played a vital role in improving the quality of life for older adults by offering benefits like enhancing socio-emotional and cognitive functions, reducing depression, and promoting self-awareness and behavior change \cite{mulrow1990quality, litchman2017real, chung2023community}. Emerging wearables like on-skin sensors \cite{kim2011epidermalelectronics, weigel2015iskin}, implanted interfaces \cite{kao2015nailo, holz2012implanted}, and digital fabrics \cite{poupyrev2016project} are unlocking new capabilities in sensing, interaction, and expression \cite{vega2014beauty}. As wearable technology continues to miniaturize and integrate closer with the human body, wearables are poised to play a pivotal role in enabling graceful aging \cite{orlov2012technology, fournier2020designing}.

%\cmh{i think it's too strong; i would say something like present unique opportunities to; never mind you use similar language in a later para}
Wearable robotic systems with their proximity to the user's body can enable new interaction paradigms and unique opportunities for empowering older adults. To date, wearable robots for older adults have primarily been in the form of exoskeletons, which are designed to enhance mobility and improve gait \cite{kong2006design, jayaraman2022modular, jung2019older}. A distinct class of wearable robots—referred to in this work as \textit{on-body robots}—differs from exoskeletons in two key aspects: their compact form factor, and their ability to move around the body \cite{sathya2022calico, dementyev2017skinbot, dementyev2016rovables}. On-Body, movable robots such as \textit{Calico}\cite{sathya2022calico}, \textit{SkinBot}\cite{weigel2015iskin}, and \textit{Rovables}\cite{dementyev2016rovables} have existed for several years, yet many open questions remain regarding their interaction design for the aging populations: What roles could these robots play in older adults' lives? How would they communicate effectively with users? How can their presence be made comfortable and seamless?

\begin{figure}
    \centering
    \includegraphics[width=\linewidth]{figures/teaser-draft2.pdf}
    \caption{We engaged older adults as co-designers of on-body robots using \textit{Calico}\cite{sathya2022calico} as a design probe to ground the design process.}
    \label{fig:teaser}
    \vspace{-1em}
\end{figure}


Proximity to the human body signifies a deep sense of comfort and trust, typically reserved for only the most essential and intimate entities \cite{hall1990hidden}. The ability of on-body robots to move across the body, combined with their physical closeness, presents unique opportunities to empower older adults by building on the benefits of existing wearable technologies. While this closeness offers potential for fostering rich human-robot relationships, it also demands a high level of trust and reliance for successful adoption. Moreover, this proximity requires interactions so seamlessly integrated into the human experience they are perceived as an extension of the body. For on-body robots to be successfully integrated into the lives of older adults, the design of on-body interactions must be grounded in principles that reflect the specific needs and preferences of this population. Leveraging co-design as a methodology can uncover these principles by actively involving stakeholders (\ie older adults) in the design process \cite{muller1993participatory, rogers2022maximizing, antony2023codesign}, enhancing the usability and adoption of on-body robots.

In this work, we engage older adults as designers of on-body robots through a dual-phase co-design process. First, we explore the use cases and design of on-body robots broadly (\textit{divergence}), followed by application-focused design workshops to understand the finer-grained interaction needs of these systems (\textit{convergence}). Through our co-design work, we make the following contributions:
\begin{enumerate}[leftmargin=*]
    \item An initial framework of the design space for on-body robots based on design insights gathered from and interactions designed by older adults to guide future research.
    \item Reflections on involving older adults in designing on-body robots, drawing from our design process learnings.
\end{enumerate}

% \item We present use cases for on-body robots envisioned by older adults, highlighting their potential to augment the quality of life for older adults.

\section{Related Works}

\subsection{Wearable Robots}

Wearable robots vary widely in form and function. Exoskeletons, a prominent class, focus on mobility assistance and rehabilitation, targeting specific areas such as shoulders or hands \cite{oneill2017exoshoulder, gao2023portable, in2015exo}, or providing broader support to regions such as the lower limbs \cite{huo2014exolower, asbeck2014stronger}. Other wearable robots offer physical augmentation, such as a third thumb \cite{zhou2019novel, shafti2021playing} or arm \cite{muehlhaus2023need, vatsal2017wearing}, delivering active assistance while remaining fixed in place. Moreover, stationary robots perched on the shoulder have been studied as wearable companions \cite{tsumaki201220, jiang2019survey}.

% Wearable robots encompass a broad range of systems in terms of form and function. Exoskeletons, a prominent class, are primarily designed for mobility assistance and rehabilitation, targeting specific areas such as the shoulder \cite{oneill2017exoshoulder, gao2023portable} or hand \cite{in2015exo}, or providing support to larger regions like the lower limbs \cite{huo2014exolower, asbeck2014stronger}. While these systems enhance muscle actuation, they are not actively actuated themselves. Other wearable robots provide physical augmentations, such as a third thumb \cite{zhou2019novel, shafti2021playing} or third arm \cite{muehlhaus2023need, vatsal2017wearing}, offering active assistance while remaining fixed in place. Stationary humanoid robots perched on the shoulder have also been explored as wearable companion robots \cite{tsumaki201220, jiang2019survey}.


A newer class of wearable robots consists of on-body, locomotive systems, capable of moving across the body, offering greater flexibility in interaction. These robots can move via direct skin contact \cite{dementyev2017skinbot}, climb on clothing \cite{dementyev2016rovables, birkmeyer2011clash}, or travel along tracks embedded in garments \cite{sathya2022calico}. This mobility sets them apart by enabling dynamic, on-body interactions and unlocking new possibilities for user engagement. While the design needs of exoskeletons have been explored \cite{jung2019older}, effective interaction paradigms for movable, on-body robots remains unexplored. Developing these paradigms is essential if on-body robots would be developed to support older adults.

\subsection{Designing with Older Adults}

Co-design is an effective methodology for understanding the design needs of special populations, such as older adults, by leveraging their lived experiences \cite{rogers2022maximizing, alves2015social, ostrowski2021personal}. This flexible approach allows stakeholders to act as users, testers, informants, partners, or co-researchers \cite{alves2021children, nanavati2023design}. Its versatility spans contexts ranging from assistive robots for aging in place \cite{lee2018reframing, alves2015social} and robots for dementia or depression \cite{moharana2019robots, lee2017steps}, to challenges like promoting physical activity \cite{antony2023codesign}, designing fitness apps \cite{harrington2018designing}, and improving wearable tech adoption \cite{nevay2015role}.

The flexibility of co-design extends to the diverse range of activities it supports, such as sketching \cite{lee2017steps}, story-boarding \cite{bjorling2019participatory}, mind-mapping \cite{antony2023codesign}, prototyping \cite{bjorling2019participatory, lee2017steps}, worksheets \cite{axelsson2021social} and role-playing \cite{bjorling2019participatory}. This adaptability allows for the selection of design exercises that foster both divergent and convergent design thinking \cite{ostrowski2021long}. In addition to design exercises, low-fidelity design probes can further inspire design thinking and offer new insights for future technologies \cite{hutchinson2003technology, gough2021co}. To effectively engage older adults as designers of on-body robots, we employ a co-design approach that incorporates various design activities to foster both convergent and divergent thinking, with a design probe used to anchor the design process. 


% \todo{talk about design probe: }
% %https://dl.acm.org/doi/pdf/10.1145/3520495.3520513

\section{Design Process}
We employed a two-phase co-design process to explore the design space of on-body robots. The first phase involved exploratory workshops that encouraged open-ended, broad exploration of on-body robots (\textit{divergence}). The second phase consisted of workshops, where participants engaged in application-driven design exercises (\textit{convergence}) (see Fig. \ref{fig:process}).


\begin{figure*}[h]
\centering
\includegraphics[width=\textwidth]{figures/design-process.pdf}
\caption{Our two-phased design process consisted of three exploratory workshops for divergent ideation of use cases for on-body robots and application-focused workshops for convergent, detailed interaction design for three domains: \textit{massage, physical therapy} and \textit{walking}.}
\label{fig:process}
\end{figure*}

\textbf{Design Probe.} We used \textit{Calico} \cite{sathya2022calico}, a small robot that moves around the body on flexible 3D-printed tracks embedded in clothing and communicate via LEDs strips, as our design probe. Calico’s simple setup and Wizard-of-Oz interface enabled us to introduce the on-body robot concept to participants, providing a tangible foundation for design \cite{ostrowski2021long}.

\textbf{Participants.} We recruited 13 independently living older adults to engage in the co-design of on-body robots. Each participant is assigned a pseudonym instead of depersonalized IDs (\eg p1) \cite{lee2018reframing} (see Table \ref{tab:participants_short}). The workshops were approved by our institutional review board and compensated at 15USD/hr. The sole inclusion criterion was age of 65 or older.

\begin{table}[h]
\centering
\caption{Overview of Older Adult Participant Demographics}
\label{tab:participants_short}
\resizebox{\columnwidth}{!}{%
\begin{tabular}{p{1.0cm}|p{0.8cm}|p{0.5cm}|p{1.2cm}|p{1.65cm}|p{1.8cm}} 
    \hline
    \textbf{Pseudo} & \textbf{Gender} & \textbf{Age} & \textbf{Ethnicity} & \textbf{Lives} & \textbf{Workshops} \\ 
    \hline
    \hline
    Rachel     & Female & 66 & Caucasian & Alone & EW \#2, AW \#2\\ 
    \hline
    Sylvia     & Female & 70 & Caucasian & with Partner & AW \#2\\ 
    \hline
    Leona      & Female & 81 & Caucasian & Alone & EW \#1, AW \#3\\ 
    \hline
    Raymond    & Male & 76 & Caucasian & Alone & EW \#1, AW \#2\\ 
    \hline
    Melanie    & Female & 76 & Caucasian & Alone & AW \#3\\ 
    \hline
    Martin     & Male & 75 & Caucasian & with Partner & AW \#3\\ 
    \hline
    Randall    & Male & 73 & Caucasian & with Partner & EW \#1, AW \#1\\ 
    \hline
    Norman     & Male & 69 & Caucasian & with Partner & AW \#1\\ 
    \hline
    Camille    & Male & 76  & Caucasian & Alone & EW \#2, AW \#1\\ 
    \hline
    Elizabeth    & Female & 82 & Caucasian & Alone& EW \#2\\ 
    \hline
    Lauren     & Female & 74 & African-American & in Community & EW \#3\\ 
    \hline
    Margaret   & Female & 82 & Caucasian & in Community & EW \#3\\ 
    \hline
    Cindy   & Female & 84 & Caucasian & in Community & EW \#3\\ 
    \hline
\end{tabular}}
\end{table}

% \begin{figure*}[t]
% \centering
% \includegraphics[width=\textwidth]{figures/co-design-process.jpg}
% \caption{Visualization of the two-phase design process.}
% \label{fig:results-barriers-draft}
% \end{figure*}

\subsection{Phase 1: Exploratory Workshops}\label{process:ews}
We conducted three exploratory workshops to generate a broad spectrum of potential applications for on-body robots and to gather insights into the perceived benefits and barriers to adoption of these systems grounded in older adults' lived experiences. Each workshop lasted 70 to 90 minutes.

%\cmh{example pic}.

\textbf{Grounding.} Each session began with an introductory video about \textit{Calico}\footnote{link to video: \url{https://youtu.be/R1Mcj5uil6Q}}, followed by a demonstration of the design probe on a sleeve embedded with \textit{Calico}'s track system (see Fig. \ref{fig:teaser}). Interested participants wore the sleeve, and the robot was tele-operated along the track to allow them to experience the haptic feedback and on-body interaction firsthand. This hands-on introduction served as a foundation for the rest of the workshop. Then, we facilitated an open discussion to gather participants' initial impressions about on-body robots.

\textbf{Brainstorming.} Next, we facilitated an open-ended, collaborative map-making session to explore the design space for on-body robots aimed at enhancing the well-being of older adults. Participants first brainstormed potential use cases, writing their ideas on post-it notes and adding them to a shared mind map \cite{antony2023codesign} with images of common spaces for older adults such as park, bedroom \cite{chudyk2015destinations} to invoke imagination for usage. We then asked them to reflect on the potential benefits and challenges of adopting on-body robots, recording these insights on additional post-it notes and contributing to a second mind map. This process encouraged ongoing discussion, with participants organizing and visualizing emerging ideas collectively \cite{lee2017collaborative}.

\textbf{Preferences.} Finally, each participant selected the application ideas they found most promising, either for themselves or for their friends and family. For each selected idea, they shared key points explaining their choice, the factors that would influence their decision to use such a system, and any further thoughts on the robot's embodiment.



\subsection{Phase 1 Outcomes} \label{process:midpoint}
Phase 1 of our design process generated a diverse range of potential applications for on-body robots. Older adults envisioned use cases such as supporting activities of daily living (\eg personal hygiene, navigation), delivering targeted therapies like acupuncture, enhancing recreational activities (\eg dance), serving as wearable jewelry, mitigating fall risks, and enabling health monitoring and diagnostics. To explore this design space, we organized Phase 1 ideas along key dimensions—user movement, social context, and duration of use—highlighting differences and supporting systematic exploration. We also accounted for participants’ enthusiasm, for certain ideas (\eg walking). Building on this foundation, Phase 2 focused on three applications from Phase 1: Massage (recreation), Physical Therapy (rehabilitation), and Walking (daily living). These activities represented diverse aspects of older adults’ lives and varied in interaction styles, such as movement, duration, and social context, supporting a comprehensive exploration of the design space for on-body robots.

% \textit{Phase 1} of our design process generated a diverse range of potential applications for on-body robots. Older adults envisioned various use cases, including supporting activities of daily living (\eg maintaining personal hygiene, navigation), delivering targeted medical therapies such as acupuncture, enhancing recreational activities like dance, serving as wearable jewelry, mitigating fall risks, and enabling continuous health monitoring and diagnostics. In \textit{Phase 2}, we facilitated three application-focused workshops, each centered on \textit{Massage} (a recreational activity), \textit{Physical Therapy} (a prescribed rehabilitation activity), and \textit{Walking} (an activity of daily living) respectively. These applications were chosen by the research team to explore different aspects of aging-in-place, with interactions varying across movement, duration of use, and social context. Our selection aimed at providing a comprehensive view of the design space for on-body robots. 

% \begin{figure}[h]
% \centering
% \includegraphics[width=\columnwidth]{figures/onbody-use-cases.pdf}
% \caption{Use Cases for On-Body Robots as Imagined by Older Adults}
% \label{fig:results-use-cases}
% \end{figure}

\subsection{Phase 2: Application-Focused Workshops} \label{process:dfws}

The application-focused workshops aimed to develop concrete designs for on-body robots tailored to a specific application area, providing a grounded perspective on interaction design of on-body robots. Each workshop took about 3 hours.

\textbf{Grounding.} We invited interested participants to try on the design probe showcasing locomotive and communicative capabilities. We, then, facilitated a collaborative map-making activity to explore the benefits of the application area, independent of the robot, to inform subsequent design activities.

\textbf{Contextualization.} Participants then imagined contexts for using on-body robots in the given applications, considering key factors like location, presence of others, and time of use. These ideas were then shared and discussed within the group.

\textbf{Interaction Flow.} Using free-form worksheets, participants brainstormed potential interaction designs for on-body robots in their chosen context, encouraging open-ended thinking. This was followed by three experience flow timelines\footnote{adapted from \href{https://tinyurl.com/jvrjr3nf}{experience-based co-design toolkit}}, focusing on the finer grained design---\textit{start, during,} and \textit{end}---of the interaction. A supplementary sheet outlining potential sensing and actuation capabilities, derived from the exploratory workshops, was provided to inspire the designs without limiting creativity. After completing each timeline, participants shared and discussed their envisioned designs with the group (see Supplementary Materials for worksheets).


\textbf{Bodystorming.} After a brief recess, participants engaged in a \textit{``bodystorming''} session, physically enacting their envisioned interactions with a 3D printed replica of the design probe. This activity helped anchor their designs in the practicalities of on-body interaction, offering valuable insights into the feasibility and user experience of their concepts \cite{segura2016bodystorming}.

\textbf{Personality and Embodiment.} Participants next designed the robot’s personality and embodiment using a set of worksheets \cite{axelsson2021social, cauchard2016emotion}. However, these worksheets were excluded from our analysis, as participants, fatigued at this late stage of the workshop, did not engage with them effectively.

%Using a set of design worksheets, they brainstormed potential character traits, grouped these traits to create a holistic personality, and further developed it by mapping out specific characteristics. Participants were also encouraged to name their robot, adding a layer of personalization to their designs.

% \textbf{Embodiment.} The workshop concluded with participants reflecting on and refining the physical form of their on-body robot concept using an embodiment design worksheet.

% \cmh{need to have an overview paragraph or fig to illustrate the process; the process involved many steps to mentally follow and remember}

\begin{figure*}[t]
\centering
\includegraphics[width=\textwidth]{figures/storyboard2.pdf}
\caption{Storyboard visualizing key components of the co-designed interaction of an on-body robot as a \textit{PT Coach}.}
\label{fig:storyboard-annotated}
\end{figure*}

\subsection{Data Analysis}

We transcribed workshop audio, digitized worksheets and mind maps, and extracted bodystorming session videos. Two researchers independently coded the data using a codebook\footnote{codebook and phase 1 data are provided in the \href{https://github.com/intuitivecomputing/Publications/blob/7f8b69042dbe6a1bd04598c66356058112780ecc/2025/HRI/Supplementary_2025_HRI_Antony_OnBody.pdf}{Supplementary Materials}} developed after an initial data review, resolving disagreements through discussion. Thematic analysis \cite{AMEEThematicAnalysis} revealed key ideas (\eg soft exterior, heart rate detection), which were grouped into higher-level concepts—design scopes and factors—through iterative team discussions. Finally, we mapped interactions between these concepts (units and anchors) to structure the initial design space for on-body robots.

% We transcribed all the audio collected during the workshops, digitized the worksheets and mind-maps and extracted the videos from the bodystorming sections. Two researchers independently coded all the collected data and met to review the data, resolving any disagreements through discussion and consensus. We conducted thematic analysis on this processed data to identify key concepts and ideas that emerged \cite{AMEEThematicAnalysis}.

% We transcribed workshop audio, digitized worksheets and mind maps, and extracted videos from the bodystorming sessions. Two researchers independently coded the data using a codebook created after an initial pass of the data and resolved disagreements through discussion and consensus. Through thematic analysis \cite{AMEEThematicAnalysis}, we identified emergent key ideas (e.g., soft exterior, heart rate detection). These ideas were then organized into higher-level concepts—design scopes and factors—through iterative discussions with the research team. Finally, we mapped the interactions between these concepts (units and anchors) to structure our initial design space for on-body robots.



% \cmh{how are the themes you identified related to the following sections?}
% \todo{how did we choose the three application areas.}
\section{Co-Designed Interactions of On-Body Robots}

In the three application-focused workshops, participants designed on-body robots as \textit{Walking Sentinel}, \textit{Pet-Like Walking Companion}, \textit{Expert Masseuse} and \textit{Gamified PT Coach}. To highlight the diversity of interactions and roles envisioned, we present two co-designed applications from the Walking and Physical Therapy workshops.

% \begin{figure*}[h]
% \centering
% \includegraphics[width=\textwidth]{figures/walking-storyboard.png}
% \caption{Storyboard visualizing the Co-Designed On-Body Robots as Fall Risk Mitigators.}
% \label{fig:results-barriers-draft}
% \end{figure*}

\subsection{On-Body Robots as Fall Risk Mitigators}

Participants emphasized the critical impact of falls on older adults' quality of life. \textit{Melanie (F/76)} illustrated this sharing \pquotes{``Outside, inside, my head is down because I'm scared to death that I'm going to hit uneven pavement.''}. 

%Participants explored how on-body robots could mitigate fall risks.

%through continuous feedback and real-time monitoring. 

% \begin{displayquote}
% \textbf{Melanie} (F/76):
% \textit{``Outside, inside, my head is down because I'm scared to death that I'm going to hit uneven pavement.''}
% \end{displayquote}

% To address fall risks, participants conceptualized on-body robots as vigilant sentinels, offering active feedback to promote safe walking gaits. Specifically, they proposed that these robots should deliver regular cues—such as sound, vibration, or continuous motion—in a consistent rhythm like a gong or the ebb and flow of ocean waves to encourage users to lift their feet while walking.

To address fall risks, participants conceptualized on-body robots as vigilant sentinels, providing active feedback to promote safe walking gaits. By delivering rhythmic cues—such as sound, vibration, or motion—similar to a gong or ocean waves, robots would prompt users to lift their feet while walking.

Participants also emphasized the importance of these robots detecting and responding to consistently poor gait patterns and environmental hazards (\eg curbs, overgrown roots). They envisioned the robots alerting users to imminent risks, with feedback that intensifies based on the proximity and severity of the danger, much like the escalating beeps of a car's backup sensor. Additionally, participants suggested that the robot could deploy countermeasures, such as activating lights in low-visibility conditions (\eg cloudy days, nighttime trips to the bathroom), further reducing the risk of falls.

This sentinel robot was envisioned to be worn and function continuously, seamlessly transitioning between active and passive modes based on the context (\eg user state, location).

Participants imagined wearing the robot near their ankles and, during bodystorming sessions, explored integrating it into footwear. The design was proposed to be utilitarian and discreet, ensuring minimal visibility---much like modern hearing aids---to mitigate any social stigma tied to its use.

\subsection{On-Body Robots As Physical Therapy Coach} 

Physical therapy (PT) was identified as a key component of healing holistically. To encourage older adults to consistently engage with PT protocols, participants envisioned on-body robots facilitating gamified physical therapy sessions. They emphasized making the experience enjoyable and feel as if \pquotes{``you are playing a game with your robot''} (see Fig. \ref{fig:storyboard-annotated})

%(\textbf{Sylvia}(F/70))

To keep users engaged, participants imagined the on-body robot sensing the user’s state (\eg energy levels, mood) to provide optimal motivation; the robot would deliver dopamine-inducing, casino-style feedback throughout the session. This feedback was imagined in various modalities, including verbal (\eg \textit{``Good Job!''}), acoustic (\eg \textit{ding,ding,ding}), visual (\eg slot-machine like rainbow colors), and physical cues (\eg rotating on an axis). Additionally, the robot would track the user’s progress and incorporate it into its nudging behaviors to ensure adherence to the therapy regimen.

Participants cited overexertion and improper execution of exercises as major barriers to PT progress. As a solution, they envisioned the on-body robot monitoring the user's range of motion and exercise intensity, offering corrective feedback and adjusting the PT protocol as necessary. The robot was imagined to move across the body, providing feedback specific to the part being exercised at each stage. Initially, participants envisioned the robot providing visual feedback to guide exercises, but during bodystorming, they adapted this to a haptic-driven system after realizing that certain body poses prevented them from visually accessing the robot. 

%\cmh{ref to fig 3 somewhere in this section?}

% In addition to exercises, many forms of physical therapy include complementary treatments such as heat, cold, or electrical stimulation to stretch muscles or aid recovery. Participants imagined the robot administering these forms of stimulation as part of an integrated therapy protocol.

% Finally, participants wanted the robot's embodiment to be \textit{``adorable''} and \textit{``friendly''} suggesting designs such as a \textit{``cute little bug''} or a \textit{``cup of coffee''} to foster a positive emotional connection with the user.


\section{Characterizing the Design Space}

\begin{figure*}[t]
\centering
\includegraphics[width=\textwidth]{figures/design-space-draft-v2.pdf}
\caption{We characterize an initial two-level design space for on-body robots consisting of design scopes and interconnected design factors.}
\label{fig:results-barriers-draft}
\end{figure*}

% The possibilities envisioned in the exploratory workshops, combined with the interactions from the application-focused workshops, underscores the nuances of designing on-body robots for older adults. We characterize a design space that organizes and distills our findings from the workshops into a two-level design space consisting of core concepts, design anchors, design units and design principles. The concepts on Level 1, namely \textit{context}, \textit{human}, and \textit{application}, act as \darkBlueAnchor design anchors establishing boundaries for the exploration of the design of on-body robots based around on the concepts in Level 2, \textit{robot} and \textit{communication}. The boundaries established by sets of design anchors represent high-level \orangeLightbulb design principles that our design workshops yielded. Moreover, we present \darkGreenLink design units that depict how concepts on Level 2 influence each other. 

The possibilities envisioned in the exploratory workshops, combined with insights from the application-focused workshops, highlight the complexities of designing on-body robots for older adults. We characterize a two-level design space, composed of core concepts---design \textit{scopes} and design \textit{factors}---connected by design anchors and design units that synthesizes and organizes our workshop findings (see Fig. \ref{fig:results-barriers-draft}). The concepts on Level 1, namely \textit{context}, \textit{human}, and \textit{application}, scope the design space with the  design anchors establishing boundaries for the exploration of the key design factors of on-body robots around on the concepts in Level 2, \textit{robot} and \textit{communication}. The boundaries, defined by the design anchors, represent key  design principles derived from our workshops.  Design units illustrate the Level 2 design factors should be considered jointly as they interact closely.

\subsection{Design Scope: Context}
On-Body robots were envisioned to be used in diverse social contexts, defined by two key aspects: \textit{co-presence} (\ie presence of others) and \textit{location of use}.

\textbf{Co-presence.}
Participants imagined various actors being present during the use of on-body robots, including family members such as life partners (\eg spouses), children, and grandchildren; acquaintances such as sexual partners and friends; professionals like doctors, lawyers, physical therapists, occupational therapists, and caregivers; and even pets.
% , such as cats, dogs, and parrots. 
Participants also considered the impact of entities such as handbags, walking sticks, bikes on the interaction with the on-body robot.

\textbf{Location of use.} The envisioned usage locations for on-body robots ranged from fully public spaces like gyms, group classes, and beaches to semi-private areas such as hospital clinics and care facilities, and completely private spaces like bedrooms. The novelty of these robots can lead to heightened attention, especially in public contexts, where they may provoke strong reactions. For example, \textit{Cindy (F/84)} expressed how encountering an on-body robot in a public setting could challenge social norms, stating, \pquotes{``If we saw [\textbf{Calico}] today… in the city, [I would go] `what the hell?\dots what is that?'.''}

%\cmh{here you meant the five design factors under ``robot''?}

The context anchors downstream design decisions such as  maintaining privacy of communication content (\eg information about user's blood sugar level),  handling interactions between the robot  and the co-present entity (\eg cats startled by the robot) and  fitting form to social norms.

%\cmh{just check -- you meant embodiement?}

 \textbf{DP1:} \textit{Adapt to Social Norms.} On-Body robots, being novel and noticeable, may cause discomfort or embarrassment for older adults in social settings. To mitigate this, designs should be discreet, aesthetically pleasing, and context-aware. Subtle cues like gentle vibrations can maintain privacy in public, while private settings allow more expressive communication. Interaction with co-present social actors, such as the user's pets or caregivers, can help further normalize use. Social norms will evolve with wider adoption however early designs should align with current norms to facilitate acceptance.

 %On-body robots, being novel and noticeable, may cause discomfort for older adults in social settings. To mitigate this, designs should be discreet, aesthetically pleasing, and context-aware. Subtle cues like gentle vibrations can maintain privacy in public, while private settings allow more expressive communication. Natural interaction with co-present actors, like pets or caregivers, can help normalize use. Early designs should align with current social norms to foster acceptance as adoption grows.


\subsection{Design Scope: Human}

The unique proximity of on-body robots to the human body underscores the need to consider human factors in their design.

\textbf{Kinesiology.} A key consideration is the body's movement and posture in a usage context. Activity levels, ranging from sedentary to high-velocity movement, significantly influence the design space. The other aspect to consider is the posture of the human; in certain positions, the robot may not be reachable or be viewable to the user thus rendering certain modes of communication un-viable (\eg the \textit{Walking Sentinel} worn near the ankle, visual and verbal communication would fail).

%A key consideration is the body’s movement and posture in a usage context. Activity levels, from sedentary to high-velocity, shape the design space. Posture also matters; in some positions, the robot may be unreachable or out of view, making certain communication modes unviable (\eg the \textit{Walking Sentinel} near the ankle, where visual and verbal cues fail).

Moreover, in certain postures, the user may unintentionally sit or lie on the robot, potentially causing discomfort unless the embodiment is designed with these scenarios in mind. \textit{Rachel (F/66)} raised a similar concern when thinking about the robot’s interaction in daily life, particularly in a sedentary setting: \pquotes{``I think of, of our friend [\textbf{Raymond}]… He spends a lot of time sitting in his recliner… How would this work in that kind of sitting environment? Do I just sit on the track?''}  Kinesiology anchors the robot’s embodiment, topology and communication modalities to the human form. 


\textbf{Senses.} Designing communication pathways for on-body robots requires consideration of \textit{sensory factors}, such as variations in hearing, vision, and other sensory experiences.  Multiple communication modes are necessary to ensure accessibility and inclusivity, enabling effective interaction. \textit{Raymond (M/76)} highlighted this need, noting, \pquotes{``In terms of aging eyes, losing eye sight\dots There has to be more than one way of communicating with you ...''}.%[because] you may be deaf, you may be blind.

\textbf{Cognition.} Certain \textit{cognitive factors}, \eg dementia or phobias like arachnophobia, may make on-body robots unsettling for some individuals. \textit{Rachel (F/66)} highlighted this concern, stating, \pquotes{``Something crawling on your body is a terrifying concept for somebody who’s genuinely aged now and may have less mobility \dots has some aspects of dementia where they’re not quite certain what’s happening around them.''} 


 \textbf{DP2:} \textit{Practical Interaction Design.} Physical comfort is crucial for on-body robots due to their proximity to the body. These robots should be lightweight, soft, and adaptable to the user’s body, minimizing discomfort or injury. Designs must account for human kinesiology, avoiding interference with natural movement and being sensitive to areas prone to pain or discomfort. Robots should adjust behavior---such as pressure, movement, or positioning---based on physical cues. To prevent sensory overload, designers must consider the frequency and modality of feedback, ensuring it is gentle and non-intrusive.

 %Physical comfort is crucial for on-body robots due to their proximity to the body. They should be lightweight, soft, and adaptable, minimizing discomfort or injury. Designs must respect human kinesiology, avoiding interference with natural movement and areas prone to pain. Robots should adjust behavior—pressure, movement, or positioning—based on physical cues. To avoid sensory overload, feedback frequency and modality must remain gentle and non-intrusive.


\subsection{Design Scope: Application} 
The application the on-body robot facilitates for its user is central to its design; the application informs the role the robot assumes and the temporal aspects of its usage. 

\textbf{Role.} Participants envisioned various roles for on-body robots, including \textit{sentinel}, \textit{companion}, \textit{coach}, and \textit{tool}. Each of these roles influences the user’s level of reliance and trust, thereby defining the nature of the relationship and usage. For instance, \textit{Melanie (F/76)} highlighted how a deep reliance can form with the on-body robot, stating, \pquotes{``If it [on-body robot] is your lifeline, literally, then it’s kind of like [\textbf{Leona’s}] [SOS bracelet]. I would suspect you don’t ever turn that off.''}

\textbf{Usage.} Temporal interaction factors significantly shape the human-robot relationship. These include when and how often interactions occur, the duration of each interaction, and the long-term presence of the robot in user’s life. \textit{Melanie (F/76)} expressed the importance of how interactions begin and end: \pquotes{“If you do actually form a partnership or feel something about it, it would be nice if it wasn’t just… too abrupt, you know, like, hi, bye, not just turning and walking out the door.”} 

% \cmh{not sure which link in fig 4 this sentence is referring to, but maybe it is okay; i guess you are not necessarily single out specific links in the design space?}. 

 The general interaction is shaped by the robot’s role and application, informing its design to establish clear utility and identity  \textbf{DP3:} \textit{Clear Utility and Cohesive Identity.} On-Body robots must justify their use by leveraging proximity and mobility to enable functions beyond those of stationary devices. As \textit{Lauren (F/74)} noted, ``\pquotes{I'd be waiting for not the calico [version] one o one… my watch does all of these things but it doesn't move… I don’t want it yet… while it looks like that.}'' A clear utility, coupled with a cohesive agency, will help build trust and create more natural, enjoyable interactions, ultimately making the robot a meaningful part of daily life.


%The interaction is shaped by the robot’s role and application, informing its design to establish clear utility and identity. \textbf{DP3:} \textit{Clear Utility and Cohesive Identity.} On-body robots must justify their use by leveraging proximity and mobility for functions beyond stationary devices. As \textit{Lauren (F/74)} noted, ``\pquotes{I’d be waiting for not the calico [version] one o one… my watch does all of these things but it doesn’t move… I don’t want it yet… while it looks like that.}’’ Clear utility and cohesive agency build trust and foster natural, enjoyable interactions, making the robot a meaningful part of daily life.

\subsection{Design Factors: Robot}

\textbf{Embodiment.} The on-body nature of these robots makes their embodiment a crucial factor for user comfort and adoption, particularly for long-term use. Both the sensory aspects (\eg weight, softness) and visual elements (\eg colors, visibility) play a key role in shaping the user’s experience. Participants emphasized the embodiment should avoid signaling disability or medical conditions, which could discourage use. 

Participants emphasized the influence of the visual design and suggested enhancing “cuteness” while minimizing “weird” or “goofy” features to improve the robot’s appeal. Moreover, participants suggested aligning the robot's design with current fashion trends to promote usage, particularly in public spaces; \textit{Raymond (M/76)} remarked, \pquotes{“It might be more socially acceptable if it was something that is seen as an adornment instead of a monitor.”} Customizing the robot’s appearance, such as offering different colors or characters, could further encourage users to embrace the technology. As \textit{Randall (M/73)} noted, \pquotes{``You’d want an assortment of covers… you could put a ladybug or a little dog\dots whatever you want.''}

In terms of sensory factors, softness was repeatedly desired, as it could facilitate safer interactions for users with more sedentary lifestyles or in contexts like sleep aids. For instance, participants imagined the \textit{Expert Masseuse} robot having a soft exterior, allowing users to comfortably fall asleep during a session. Weight, on the other hand, was seen as a versatile feature. While lighter robots were preferred for designs like the \textit{Walking Sentinel} to avoid fatigue, heavier robots were considered beneficial for specific functions, such as adding resistance for exercise with the \textit{Gamified Coach} or applying additional pressure for massages with the \textit{Expert Masseuse}.  The embodiment has clear implications for robot topology. 

%\cmh{maybe introduce this unit after topology?}.

Participants explored the functional potential of robot embodiment. For instance, \textit{Randall (M/73)} envisioned a soft exterior serving as an airbag to reduce injury risks during falls or bumps. Beyond safety, a robot’s embodiment should align with its intended role and interaction style. The \textit{Walking Companion}, for example, was imagined to resemble a pet, reflecting the need for its design to match its relational and functional purpose. Embodiment can also enable communication modalities, as demonstrated by the arrow-shaped \textit{Gamified Coach} providing haptic and visual feedback (Fig. \ref{fig:storyboard-annotated}).
\textbf{Topology.} Participants envisioned a range of locations for the robot to be on body from their feet to their shoulders; from being worn on clothing such as a bolero or a sock to direct on-skin context. It is also important to consider the number of robots that need to be on a person at the same time for a given role. For instance, there might need to be one \textit{Walking Sentinel} for each leg, and multiple robots may be monitoring and providing feedback for different parts of the body for the \textit{Gamified Coach}. The location on body and the number of on-body robots inform the communication modality and must be considered to avoid any discomfort for the user.


\textbf{Perception.} On-Body robots were envisioned to monitor the user and the environment to inform their behaviors. In terms of environmental perception, participants imagined on-body robots perceiving factors such as location, potential hazards (\eg curbs, overhanging branches), weather conditions, visibility, temperature, and humidity. Accurate ego-location on the body was highlighted as a crucial feature, as it would guide both social and functional behaviors. For example, participants wanted the robot to adjust its behavior around sensitive areas (\eg neck), ensuring user comfort and safety.

Regarding the user, participants expected on-body robots to sense and interpret a range of human indicators such as emotions (\eg nervousness, anxiety), bio-statistics (\eg heart rate, oxygen levels, body temperature), physical states (\eg odor, muscle tension, movement), and goal progress.  The perception capabilities enable the necessary autonomy and the communication content necessary for the robot's role.

\textbf{Autonomy.} Leveraging their world and user model, on-body robots were expected to plan and execute actions towards their goals. Participants imagined the robot using a combination of explicit information-seeking and implicit sensing of real-time signals to adjust its plans to maintain trust and reliability.

Two primary classes of behaviors emerged during the workshops: reactive and proactive behaviors. Reactive behavior occurs in response to specific triggers, such as sensor recognition, or changes in the user’s body or environment. In contrast, proactive behaviors happen regularly as a method of checking in, requesting user input, or prompting the user as part of the robot’s functionality (\eg reminding the user to lift their foot or maintain correct posture). Tuning the frequency of behaviors based on the robot’s role, the severity of the event, and the user’s state and preferences were considered critical. 

Importantly, the degree of user input---whether frequent or occasional---would depend on the robot’s role and application. Moreover, participants envisioned shared autonomy paradigms where the robot's action would be influenced by inputs from external actors (\eg therapist) or would share information with caregivers (\eg medical professionals). However, participants emphasized that the user must always retain ultimate control. \textit{Norman (M/69)} underscored the importance of this dynamic, noting, \pquotes{``If it’s not good, [the user] asks the robot to stop and the robot stops immediately… then [the robot] looks at what’s going on and asks the person, ‘OK, I’m stopped. Do you want me to change the protocol or do you want me to just call it quits? You’re in control.’”} Robot responsiveness and user control was seen as vital for fostering usage and ensuring the robot’s actions remain aligned with the user’s preferences.

\textbf{Adoption.} Participants suggested including an onboarding process, featuring guided tutorials, to support initial adoption and lower the interaction learning curve. Participants also emphasized the need for convenience in daily tasks like charging, cleaning, and storing the robot. Designing the robot with easy-to-clean or washable materials is essential for maintaining hygiene, especially for a device in constant contact with the body. Participants noted that older adults often misplace small devices, so features like a ``find my robot'' function or the robot autonomously returning to its docking station were suggested to help mitigate this issue. Additionally, extended battery life were desired to enhance convenience by enabling more widespread usage. All \textit{design scopes} influence but do not impose clear constraints on the design for adoption.

\subsection{Design Factors: Communication}

\textbf{Modality.} Participants envisioned communication with on-body robots using verbal and non-verbal methods. They favored natural, conversational verbal interactions over commands. Non-verbal modes included gestures (e.g., two fingers touching), physical actions (e.g., tapping, pressing or slapping the robot), and remote controls for specific scenarios.

Participants imagined a wide range of robot-to-user communication methods, taking advantage of the robot’s proximity to the body and its mobility. \textit{Physical movement} in meaningful patterns or navigating to specific body areas was proposed to convey intent, affect, or information. \textit{Visual feedback}, such as blinking or color-changing LEDs, was another suggested option; however, designing on-robots to convey more complex information via visual feedback was thought to be challenging.

\textit{Haptic feedback}, both kinesthetic (\eg tugs, pinches) and vibro-tactile (\eg buzzes), was widely discussed. \textit{Raymond (M/76)} noted its importance for visually impaired users, explaining, \pquotes{“It could literally be in the shape of an arrow… you could feel it if it actually moved…”} Participants also raised the idea of \textit{olfactory output}, where the robot could emit essential oils or burn incense to provide feedback through scent, opening up novel sensory communication avenues.

For \textit{verbal feedback}, thoughtful voice design characterized as ‘cute,’ and ‘lovely’ was emphasized; moreover, communicating via hearing aids was suggested as a way to improve communication for users with hearing impairments or in noisy environments. \textit{Non-verbal acoustic cues}, like “ding ding ding,” were also imagined for certain contexts.

The close proximity of the robot to the body opens up new opportunities for communication beyond traditional verbal interaction. To ensure clear and intuitive communication, participants stressed the importance of building on familiar conventions, such as the colors of traffic lights, to create easily understandable patterns of feedback underscoring  the need for the modality to support the content of communication.

Participants designed multi-modal affect communication for on-body robots, particularly in social roles. For example, the \textit{Walking Companion} was imagined to light up and use acoustic cues to express excitement before a walk, while the \textit{Gamified Coach} would signal the end of a session by performing a playful dance, including spinning in place. Leveraging multi-sensory experiences could help establish the robot’s character and foster deeper emotional connections with users, ultimately reinforcing adoption and continued usage \cite{duffy2003anthropomorphism}. However, participants noted the importance of balancing these communication pathways to prevent sensory overload to ensure the robot integrates smoothly into the user’s daily life.

\textbf{Communication Content.}
Two types of content for communication emerged: supportive and informative. Supportive content provides encouragement or rewards, offering “dopamine” feedback to motivate the user or celebrate successful events. Informative content, on the other hand, includes data from the robot’s sensors—such as alerts to potential dangers—or information crucial to the user, like reminders, explanations of robot actions, or corrections (\eg adjusting the user’s posture or addressing discomfort causing robot action).

\section{Using the Design Space}

Our design space serves as an initial framework for research and practical development in the field of on-body robots, particularly for older adults. It can be leveraged in two distinct ways: first, to identify and investigate open research questions, and second, to guide the creation of on-body robot prototypes. Our design space aims to enable an iterative exploration of on-body robots as a practical HRI paradigm for older adults.


\subsection{Open-Questions for On-Body Robots}

The design space helps uncover several open research questions critical to advancing on-body robots as a technology. At the core of these questions are the design factors articulated in Level 2 of the design space: robot (embodiment, topology, perception, autonomy, and adoption) and communication (content and modality). For instance, \textit{How can on-body robots use olfactory output to communicate? How to account for the fragility of older adults' skins? What are the trade-offs between using a single on-body robot and multiple robots distributed across the body?
} These questions highlight the need for deeper exploration into how each concept can be practically implemented thereby refining this initial design space.

Understanding the interplay within design units presents another layer of inquiry. For example, \textit{How to design embodiments suitable for multiple on-body robots without overwhelming users?} There are also broader questions on how to practically achieve the presented design principles for different populations and settings. These open questions can guide future work to enhance on-body robots’ real-world viability and refine the design space through situated co-design \cite{stegner2023situated}.

\subsection{Building On-Body Robot Prototypes}

Our design space offers a structured approach for creating functional prototypes of on-body robots facilitating further exploration of this HRI paradigm. The process begins by defining the design scopes at Level 1, which covers the target population (\eg older adults, blind or visually impaired (BVI) individuals), the robot’s application (\eg navigation, acupuncture), and the expected context of use (\eg clinics, park). These foundational decisions establish clear boundaries and constraints for the robot’s design, ensuring that it aligns with user needs and environmental factors. For instance, while designing for BVI people, visual modalities cannot be used.%, and vibrations may be a more effective communication modality in public than sounds.

Next, the design factors---such as the robot’s embodiment, communication modalities, and autonomy---can be explored broadly within these established boundaries. For instance, acoustics and robot movement can be evaluated as communication modality for BVI individuals. The nascent nature of on-body robots requires significant exploration with functional prototypes, and iterative testing with users to garner better understanding of this design space. For example, different embodiments can be built and evaluated for their fit with social norms, while communication pathways may be adapted to include affordances tailored to certain populations.

Using our design space, researchers and designers can find a set of feasible prototypes for exploring relationships between design elements and broader open questions in this novel HRI space (see Supplementary Materials for our visual guide).


% \section{Discussion}
% Our design space helps highlight the key nuances for the design of on-body, locomotive robots grounded in perspectives of older adults. Building effective on-body robots requires careful consideration of how these dimensions interact and align to produce a cohesive and meaningful interaction experience over time to enable actual adoption. Moreover, situating on-body, movable robots in the wider wearable ecosystem is important to understand how this HRI paradigm will evolve and evaluate the value of the design space beyond our class of robots. 


% \subsection{Synergy of Wearable Technology and On-Body Robots}

% % The convergence of wearable technologies—such as smart textiles, on-skin interfaces, and on-body robots—offers opportunities to bridge the gap between computation and the human body, enabling more seamless integration and richer interactions.

% Participants envisioned on-body robots providing comprehensive health tracking and diagnostics, suggesting that the sleeve itself could function as a sensor. Interfacing on-body robots with smart textiles can provide access to rich sensing capabilities \cite{poupyrev2016project} supporting sophisticated tasks. For example, in a hospital setting, cloth-climbing robots like Rovables \cite{dementyev2016rovables} could navigate smart garments to perform tasks such as administering medication or performing heat/cold therapy thereby enhancing patient care through timely interventions.

% Beyond sensing, on-skin interfaces \cite{kao2015nailo, weigel2015iskin} can enable intuitive communication with on-body robots, especially in hard-to-reach areas or in situations requiring privacy. For example, if a robot is performing massage therapy on the back, an on-skin interface on the forearm could enable a new communication pathway overcoming the visual and tactile limitations. The interaction between smart textiles and on-body robots could also enables symbiotic functionalities; Smart garments could provide power, data connectivity, and structural pathways, reducing the need for bulky batteries on the robot and improving overall user experience.

% \subsection{Translating Learnings To Exoskeletons}

% The design space we have identified for on-body, movable robots offers valuable insights that can help inform the broader field of wearable robotic systems. A key insight for on-body robots is the importance of aesthetic and customizable designs in reducing the stigma often associated with wearable robots. Previous efforts have focused on making exoskeletons softer, lighter, and less obtrusive \cite{jung2019older}. Building on this, our design space highlights the role of fashion trends and personal expression in promoting adoption. By allowing personalization, wearable robots can shift from being purely tools to becoming extensions of personal identity.

% Additionally, exosuits and exoskeletons are increasingly being used for both physical assistance and rehabilitation. The communication modalities identified in our design space—such as haptic feedback, visual cues, and non-verbal auditory signals—can be integrated into exoskeletons to enhance user engagement and interaction intuitiveness. These sensory channels provide real-time, context-aware feedback, making training regimens more engaging and improving the overall user experience. Involving older adults in development of technology is essential to achieve real-world applicability and improved outcomes \cite{vines2015age}. 

\section{Reflections on Co-Designing On-Body Robots}
The sensitive and safety-critical applications envisioned for on-body robots (\eg fall prevention) highlight the importance of involving end-users as design partners to ensure usability and adoption. Our design workshops provided key insights for effectively engaging older adults as co-designers.

\subsubsection{Lived Experiences with On-Body Robots} The novelty of on-body robots underscores the need to introduce these concepts in an experiential and digestible manner to engage participants as effective design partners \cite{ostrowski2021personal, ostrowski2021long,lee2017steps,lee2018reframing}. Hands-on demonstration of our design probe jump-started participants’ design thinking and helped demystify on-body robots. We also observed that participants who engaged in both exploratory and application-focused workshops were more comfortable imagining interaction paradigms compared to those who only participated in the later. Introducing on-body robots with a design probe, with time to reflect between sessions, encouraged more active and creative engagement \cite{mahmood2024our}. Thus, a multi-stage design process, with shorter, focused workshops may enable deeper involvement from older adults.

\subsubsection{Structure in Design} 
Utilizing the experience flow worksheets introduced a malleable structure into the design activity, making it less overwhelming for older adults to engage with the design process. Moreover, conducting a free-form activity, followed by more structured experience flows, helped engage participants’ creativity while simultaneously making the design process more approachable—particularly given the novelty and complexity of on-body robots.

\subsubsection{Bodystorming On-Body Interactions} Bodystorming played a pivotal role in our design process by allowing participants to identify subtle, grounded design considerations \cite{oulasvirta2003understanding, schleicher2010bodystorming, stegner2023situated} and engage more deeply with the physical aspects of the interaction \cite{segura2016bodystorming}. For example, \textit{Raymond} bodystormed the \textit{Physical Therapy} scenario from the perspective of a blind person, highlighting the importance of multimodal communication. For the \textit{Massage} scenario, bodystorming prompted participants to consider how the robot would adapt to different body areas, sparking discussions on custom 3D-printed form factors tailored to individual users and specific therapies.

% \cmh{be consistent: Body-storming or Body-Storming or Bodystorming throughout; same for On-body vs On-Body. The reason I keep making this point is that this shows the care you as an author gives to the work you present to the community -- makes sense will do!}


\section{Limitations and Future Work}
Our design process uncovered promising applications and provided valuable insights into the design space of on-body robots. However, to gain a deeper, more grounded understanding of this interaction paradigm, future work should implement the proposed interactions for on-body robots and engage older adults in evaluation processes to further refine the design space. Additional workshops using on-body robots beyond Calico \cite{sathya2022calico} could provide further insights. The co-design partners involved in this study are not fully representative of the diverse population of older adults, who vary widely in physical and cognitive abilities. Future research should involve a broader spectrum of older adults in the design process to explore the appropriateness and specific design needs of on-body robots for different subgroups within the aging population.

% \cmh{calico only represents one type of on-body robots}

\section*{Acknowledgment}
This work was supported in part by the JHU Malone Center for Engineering in Healthcare.

% The preferred spelling of the word ``acknowledgment'' in America is without 
% an ``e'' after the ``g''. Avoid the stilted expression ``one of us (R. B. 
% G.) thanks $\ldots$''. Instead, try ``R. B. G. thanks$\ldots$''. Put sponsor 
% acknowledgments in the unnumbered footnote on the first page.

% \section*{Acknowledgment}
% Anonymized for blind review.


\clearpage
\balance

\bibliographystyle{IEEEtran}
% \todo{DOUBLE CHECK REFERENCES STYLE}
\bibliography{references}

\end{document}

%Understanding the interplay between these dimensions is essential for designing robots that offer compelling, intuitive interactions. However, these dimensions do not exist in isolation; they interact in complex ways that significantly impact the overall design. 

% \subsection{Multi-Party Interactions}

% The context of on-body robot use can invite interactions from individuals beyond the primary user, especially with common usage. How these interactions are handled depends on the robot's role. For a pet-like companion robot, simply acknowledging external interaction might be sufficient. In contrast, in scenarios like \textit{Massage}, where participants imagined the robot being used with sexual partners, the robot may need to actively engage with multiple individuals, adjusting its communication and behavior accordingly. The robot may need to recognize and respond appropriately to all parties, facilitating meaningful and coordinated interactions. These examples illustrate how the robot's role and context can create new functional requirements. Changes in one design dimension can have significant ripple effects across other areas, impacting the overall design. Accounting for this interplay is essential to ensuring a cohesive and engaging interaction experience.

%\subsection{The Wider Wearable Technology Ecosystem}

% Our exploratory workshops uncovered a diverse array of potential applications for on-body robots, capitalizing on their unique ability to operate directly on the body and move across it. 

% Through our exploration of these applications, coupled with targeted design workshops focused on walking, massage, and physical therapy, we identified a set of key dimensions that are critical to informing the design of on-body robots for older adults.

%On-body robots were envisioned to be used in a broad spectrum of social contexts, highlighting the diverse environments in which these robots could operate. We identified two key aspects of social context: \textit{co-presence}, referring to the presence of other individuals or entities, and the \textit{location of use}.

% The envisioned usage locations were equally varied, ranging from fully public spaces like gyms, group classes and beaches to semi-private areas such as hospital clinics and care facilities, and completely private spaces like bedrooms. The novelty of on-body robots can lead to heightened attention that would need to be considered; The social norms associated with the location of usage need to be considered for several downstream design decisions.

% \begin{displayquote}
% \textbf{Cindy} (F/84):
% \textit{``If we saw [\textbf{Calico}] today ... in the city, you know, [I would go] 'what the hell?... like, what is that?'.'' }
% \end{displayquote}

% Moreover, in certain postures the user may inadvertently be sitting or lying on top of the robot which may cause pain unless the embodiment is designed accordingly.  

% \begin{displayquote}
% \textbf{Rachel} (F/??):
% \textit{``I think of, of our friend [\textbf{Raymond}]...He spends a lot of time sitting in his recliner...How would this work in that kind of sitting environment? Do I just sit on the track?''}
% \end{displayquote}

% Factors such as the robot’s embodiment and the mode of communication must be tailored to accommodate varying level of physical activity as this would influence the robot's location on the body and effective communication modalities.

% \textit{Sensory impairments} of the user, such as whether they are deaf or hard of hearing, blind, or have other sensory conditions, must be carefully considered when designing communication pathways. Having multiple modes of communication may help ensure accessibility and inclusivity thus ensuring effective interaction.

% \begin{displayquote}
% \textbf{Raymond} (M/??):
% \textit{``In terms of aging eyes, losing eye sight...There has to be more than one way of communicating with you [because] you may be deaf, you may be blind.''}
% \end{displayquote}

% \begin{displayquote}
% \textbf{Rachel} (F/??):
% \textit{``Something crawling on your body is a terrifying concept for somebody who's genuinely aged now and may have less mobility and... some aspects of dementia where they're not quite certain what's happening around them at all times. This would be frightening.''}
% \end{displayquote}

% The relationship between an on-body robot and the user is a key design consideration, shaped by both the robot's role and the temporal aspects of its use.

% Participants imagined various roles for on-body robots to assume: the robot as a protective \textit{sentinel}, the robot as a \textit{companion}, the robot as a \textit{coach}, and the robot as a \textit{service}. Each of these roles influences the level of reliance and trust the user places in the robot, which in turn defines their relationship. These variations have significant implications for the robot's behavior, interaction style, and overall design.

% \begin{displayquote}
% \textbf{Melanie} (F/??):
% \textit{``It [on-body robot] is your lifeline, literally, then it's kind of like Leona's [SOS bracelet]. I would suspect you don't ever turn that off.''}
% \end{displayquote}

% Temporal factors also play a crucial role in shaping the human-robot relationship. Key considerations include when interactions occur, how often they take place, the duration of each interaction, and the overall timespan of the robot’s presence in the user's life. 


% The temporal aspects of the interaction coupled with the on-body robot's role define the nature of the human-robot relationship and must be considered when designing the robot's behaviors and interaction patterns. For instance, the design of the initiation and conclusion of each interaction is heavily dependent on the relational role.

% \begin{displayquote}
% \textbf{Melanie} (F/??):
% \textit{``if you're using this and you do actually form a partnership or feel something about it, it would be nice if it wasn't just...too abrupt, you know, like, hi, bye, not just turning and walking out the door.''}
% \end{displayquote}

 % The on-body nature of these robots makes their form factor and embodiment particularly critical for ensuring adoption and user comfort, especially for long-term usage. Both the sensory features (\eg weight, softness) and visual aspects (\eg colors, visibility) of the robot's embodiment are important. Given the safety-critical roles these robots may play, participants emphasized the importance of design the embodiment in a way that avoids signaling disability or medical condition.



% , while the \textit{Gamified Coach} would require an embodiment that could support casino-style audio-visual feedback, enhancing the motivational aspect participants desired.

% The robot's visual design, emphasizing features that enhance ``cuteness'' while minimizing elements that might be perceived as ``weird'' or ``goofy,'' plays a crucial role in shaping perception. Participants also envisioned these robots as fashion accessories; by leveraging trends and the appeal of new technology in the robot's design, adoption, especially in public spaces, could be facilitated. Moreover, adapting the embodiment to the user's preferences (\eg colors, character) may further support adoption.

% \begin{displayquote}
% \textbf{Randall} (M/??):
% \textit{``You'd want an assortment of covers ...you could put a ladybug or you could put a little dog or a cat or whatever you want. Then, it would reinforce the fashion idea''}

% \textbf{Raymond} (M/??): 
% \textit{``It might be more socially acceptable if it was something that is seen as a adornment instead of a monitor''}
% \end{displayquote}

% Sensory factors, such as the weight, softness, flatness, that influence the sensory experience for the robot would determine comfort and influence usage. Softness was the most frequently desired feature with implications in enabling safer interactions for those with more sedentary lifestyles or applications like sleep aids as the \textit{Expert Masseuse} was imagined to be or when the robot is always on-body as the \textit{Walking Sentinel} where the user may fall asleep with the robot on. With regards to weight, participants desired light robots for designs such as \textit{Walking Sentinel} but also suggested that weight may be used to enable the application such as augmenting the exercise facilitate by the \textit{Gamified Coach} or providing extra pressure for the massage by the \textit{Expert Masseuse}.



% The embodiment of the robot was also imagined to have functional uses with a soft exterior forming a shield acting as an airbag to limit damage from a fall. Lastly, it is important to design the embodiment of the robot to fit its role and its communication modalities. For instance, the \textit{Walking Sentinel} embodiment should look like a pet and the the \textit{Gamified Coach} should enable the casino-style audio-visual reward that participants envisioned.

% \textbf{Capabilities.} On-body robots were envisioned to monitor both the user and the environment, allowing them to plan, execute and adapt behaviors in response to dynamic conditions. 

% Robots that are worn on the body can take various forms. Exoskeletons are a key class of such robots that are designed primiary for mobility assistance and rehabilitation; these robots can be restricted to a single body part such as the shoulder \cite{oneill2017exoshoulder}, or hand \cite{in2015exo} or provide support to a large area such as the lower limb \cite{huo2014exolower, asbeck2014stronger}. These robot help the user in actuating their muscles but are not actuated themselves.

% Wearable robots can act as physical augmentations by taking form of a third-thumb \cite{zhou2019novel} or third-arm \cite{muehlhaus2023need, vatsal2017wearing} where they are fixed to one location but are actuated themselves. 

% Affective feedback from wearable robotic systems has also been explored through on-body tactile \cite{chen2018wearable} and social  robot \cite{tsumaki201220}.

% A relatively newer form of on-body robots is those that can locomote across the body either through direct on-skin contact \cite{dementyev2017skinbot}, by climbing on clothing \cite{dementyev2016rovables, birkmeyer2011clash} or by moving on tracks embedded on clothing \cite{sathya2022calico}.



% Types of wearables robots and their applications

% Exoskeletons 
%     wearable robots for assistance and rehabilitation
%         ExoGlove \cite{in2015exo}
%         Shoulder Exo-Skeleton \cite{oneill2017exoshoulder}
%         Lower Limb Exo-Skeleton \cite{huo2014exolower, asbeck2014stronger}

% Wearable robots as physical augmentations
%     third-thumb \cite{zhou2019novel}
%     third-arm \cite{muehlhaus2023need, vatsal2017wearing}
        

% Affective Robots
%     Affective Tactile Robots \cite{chen2018wearable}
%     Social robot \cite{tsumaki201220}
    
% Movable Robots
%     Rovables \cite{dementyev2016rovables}
%     Calico \cite{sathya2022calico}
%     SkinBot \cite{dementyev2017skinbot}
%     Rubbot \cite{chen2013rubbot}
%     CLASH \cite{birkmeyer2011clash}

% Applications of wearable robots

% \cmh{need to introduce calico first and provide ref }

% Calico represents a unique class of on-body robots which are distinct from exo-skeletons which are the primary focus of on-body robotics so far \cmh{rewrite: two which}. Calico may be described as an actuated sensor that is capable of locomotion across the body creating a unique design challenge for human-robot interaction. The three key axis that distinguish Calico are proximity to the user, size and ability for locomotion \cmh{elaborate on the three axis}. 

% Participants engaged with a free-form worksheet designed to prompt them to imagine the interaction design of an on-body robot within their chosen context. This activity encouraged broad, unrestricted thinking about how on-body robots could interact with users.

% Participants then focused on designing the finer details of on-body robot interactions using experience flow timelines. They worked with three worksheets: one for the start of the interaction, one for the core interaction, and one for its conclusion. These worksheets were chosen to ensure a holistic approach, addressing key aspects of human-robot interaction. After completing each worksheet, participants shared and discussed their envisioned designs with the group. To inspire their design thinking during the experience flow activity without restricting creativity, we provided a supplementary sheet outlining potential sensing and actuation capabilities of on-body robots, derived from the exploratory workshops. 


% The robot will be on their ankle and provides cyclic, repeating feedback \cmh{can we visualize this? difficult to picture it in my head} to keep reminding user to continue lifting their feet while walking. The feedback was imagined to be continous and pleasant like a gong or a wave of an ocean. Moreover, robot would provide a warnings when environmental dangers (\eg curbs, overgrows roots) and physical dangers (\eg consistently bad gait) is detected. This kind of robot would be worn always is will never switch off instead cycling through an active and passive mode based on the level of activity.

% \textit{Designing a walking sentinel.} 

% Given the robot is aimed to be worn on the ankle, and used constantly as protective sentinel, how can we design a utilitarian, less-visible embodiment? Do we need one robot for each leg?

% agency: does robot share information with caregivers?

% intelligence: what human and environmental variables beyond gait, body stats would the robot need to be monitoring given its role?

% how does the user communicate with the robot?

% how frequent should the feedback be? What modalities would be appropriate for communicating the corrective and informative content given their frequencies and content severity?

% The temporal factors of usage also influence the relationship encompassing several factors: when interactions occur, the frequency and duration of each interaction, and the overall timespan of the on-body robot’s use.

% Interactions with an on-body robot may happen according to a schedule (\eg pre-set therapy schedule), when needed (\eg during pain, at nightime bathroom visit), when desired (\eg for relaxation), or be happening continuously. The overall time-span of usage is largely application-dependent, ranging from short-term (\eg during a group class) to medium-term (\eg for posture retraining or physical therapy), long-term (\eg for home exercise), or even life-long (\eg for fall risk mitigation). 

%The duration of each interaction can range from a few seconds to an hour, with frequency varying from every few seconds to perhaps every other day. 


% The on-body position of these robots provides them with intimate access to the user, enabling them to perceive implicit user states and intentions. This proximity not only allows the robots to effectively respond to the user's needs in a timely manner, but also carries that expectation with it.

% \begin{displayquote}
% \textbf{Rachel} (F/??):
% \textit{``it might initiate small rests when fatigue is noticed, it might say, 'boy, that was a good session, but we're not going to do the rest of it'. Kind of stopping about. Ok. This is enough. I mean, that would be the reason you would concede to wear a device, right? Is that it's going to do something that you can't do yourself.''}
% \end{displayquote}
% Embodiment adorable, friendly character | cup of coffee 


% "it fun is a really important, potentially valuable component."

% move to different parts that are being worked on 

% visual communication --> haptic communication 

% While designing the \textit{Gamified Coach} participants had originally a largely visual communication modality however adapted it to be haptically driven during body-storming since they noticed that during certain body poses, they couldn't see the robot. Moreover, \textit{Raymond} bodystormed the interaction from the perspective of a blind person and further underscored the need for multiple means of communication.

% Participants imagined the on-body robot to facilitate a gamified exercise session to motivate them while provide active monitoring and feedback to ensure correct form and regulate the range of motion.

% \textit{Proposed Scenario:} Participants imagined the on-body robot facilitating a gamified physical therapy session on a set schedule with scoring inspired by the audio-visual effects from casinos; They envisioned the robot moving to different locations in the body and providing haptic feedback in form of pulls and tugs to move part of the body to where the exercise requires within the safe range of motion. between every set, the robot would provide rests and also provide 'dopamine' feedback in form of casino-machine-style feedback. 

% \textbf{Design Questions.} 
% Co-Presense: Can the robot leverage the social aspects of exercising to motivate the user more? multi-party interactions?

% Embodiment: Given its role as a coach, usage in private, and its need to provide motivational gamified feedback, what would the ideal embodiment look like? Do we need multiple robots on body to enable this application? Can the weight of the robot be used to augment the exercise?

% Location on Body: where all does the robot move? if there are multiple robot, how do they sync navigation?

% Robot Agency: will the robot learn changes in human capability and update the exercise protocol? will the robot nudge users?

% Communication: how does the robot communicate supportive, informative and corrective content in the context of the gamified interaction? how does the robot 

% \begin{displayquote}
% \textbf{Raymond + Rachel} (F,M/??):
% \textit{``
% It would be nice if we could say, see you Tuesday, it's that it was so great to work with you today. See you, whatever the next session, you know, uh, can start anytime between 12 and two.''}
% \end{displayquote}

% \textbf{Expert Masseuse. }

% \text{Proposed Scenario:} Participant imagined using the robot as a service when desired to facilitate pain relief; Participant proposed a range of autonomy for the robot from the user specifying all the different factors of the massage (\eg intensity, location, duration) to the robot automatically deciding this based on programmed protocol and body sensor input; regardless of the autonomy, user feedback should be elicited periodically. Participants envisioned appendages for the robot 3D printed to fit each person and the specific kind of therapy that was desired; The robot would use heated and cooled pads to relax the user and also use aromatic smells. Moreover, Robot should be soft and flat to avoid damage and discomfort if used as a sleeping aid.

% \cmh{need to think about connection and transitions -- design process to the application to characterizing design space; not flowing well.}

% \textbf{Experience Flow Timelines.} Using the experience flow worksheets and splitting the interaction design into three clear sections and incorporate some structure into the design activity enabled the design activity to be less daunting and enabled older adults to clearly specify their ideas. By first doing and free flow activity and then doing a relatively more structured activity allowed us to engage the participants' creativity while making the design process more approachable especially given the daunting novelty of on-body robots. 


% \subsection{Social Capabilities (kb)}

% Expressing affect 

% Perceiving human emotions

% \subsection{social norms}
% Additionally, participants reflected on how these robots might integrate into and potentially alter existing social norms, an important factor to consider in their design and deployment.

% \subsection{personal preferences}
% Lastly, individual preferences play a significant role in the human dimension. Factors such as the desired level of agency, preferred communication style, and interaction modalities may vary widely among users. Where feasible, users should be given control over the robot’s behavior, allowing for customization to suit personal needs and comfort levels.

% As the distance between humans and robots continues to diminish, the concept of on-body robots is emerging as a potential reality. 

% This approach has been used to design robots to engage older adults as  physical activity 

% Participatory design, also known as Co-Design, is a method from human-centered design that empowers users during a collaborative design process, leading to meaningful, approachable, and joyful experiences. It serves to include the end user in the earlier stages of creating the experience rather than prior methods of including the user to test an experience after its creation. The involvement of end-users in the design process serves a crucial factor that often gets overlooked--the need to walk a mile in another person's shoes.

% Older Adults are a portion of the population that often gets overlooked. Due to the negative stereotypes of older adults, technology for this population tends to focus more on care--sometimes in excess. Many technologies designed for Older Adults tend to ignore the heterogeneity of this group and employ selective exclusion--and in some cases upperage limits. Other studies that have co-designed with older adults have had positive results with these adults feeling heard and as if they are "partners." Further 


% Participatory Design with Older Adults differs from the general population in that sensory differences--in visual, audio, and haptics--must be accounted for. Examples include increased font size and larger collaborative maps. These shortcomings must be accounted for in order to have an inclusive Participatory Design process.

% Older Adults can be involved in Participatory Design as many roles: user, tester, informant, design partner, co-researcher, and protagonist. Here are the role details:
% - Older Adults as users....
% - Older Adults as testers....
% - Older Adults as informants....
% - Older Adults as partners....
% - Older Adults as co-researchers....
% - Older Adults as protagonists....

% With Calico, Older Adults were involved in Participatory Design as different roles depending on the different workshops in order to gain a deeper understanding of their needs and how Calico could support them.

% \subsection{Technology Use by Older Adults}
% Robots and Older Adults
% How are on-body robots novel?

% Robots have been proposed to support older adults' aging in place by, for instance, by providing physical assistance, engaging them in physical activity, cognitive games and assisting in activities of daily living.

%GOAL: positioning our class of robots: on-body, locomotive systems

%  


% \textit{Design Questions.} Positioning the proposed scenario in the design space, yields a key set of design questions to be considered during implementation.

% With respect to the public social context, would the robot need to support multi-party interactions like a pet does? 

% As a robot role as a motivating companion, does the robot ever nudge the user to go on a walk like a pet would?  

% What environmental and human factors does the robot have to perceive and react to enable its goal? how does it avoid moving to sensitive areas of the torso? 

% communication: how does the robot emulate the 'excitement' of a pet? how would the robot communicate information regarding the weather and the user's body state while conforming to its role? what communication modalities does the location on the torso enable? how often does the robot provide and seek feedback?

% How would the visual and sensory features of the embodiment fit its role as a pet-like companion while enabling communication of supportive (\ie excitement) and informative (\ie weather, blood sugar) information?

% The design space we have identified for on-body, movable robots may also help inform at least some aspects of the design of wearable robotic systems in general.  

% For instance, the sensory aspects of exo-skeletons and exo-suits has already been considered with efforts to developer softer, lighter, less-obstrusive systems \cite{jung2019older}; our design space would have implications for the visual design and customizability of the exoskeleton systems to support adoption by leveraging the power of fashion trend as a means of undermining stigma with usage. 

% Exosuits and exoskeletons are not merely used for physical assistance, they may also used as tools for rehabilitation or have additional roles as a senitel and in this case the communication pathways our space exposes could help enable more intutive usage. 

% For additional robotic limbs, our design space enables positioning their 

% physical augmentations

% wearables in general

% Smart textiles and on-skin interfaces are poised to further close the proximity gap between computation and the human body. As on-body, locomotive robots are developed, interfacing with smart textiles can provide them access to rich sensing capabilities \cite{poupyrev2016project} that can power their roles and utilty. Participants imagined on-body robots providing comprehensive health tracking and diagnostic services, and enabling interventions. One can imagine a cloth climbing on-body robot \cite{dementyev2016rovables} co-existing on a hospital smart garment that enable the robot to execute actions. 

% Beyond sensing, on-skin interfaces \cite{kao2015nailo, weigel2015iskin} can enable easier communcation from the user to the robot with on-body robots when they are in locations that are hard to reach to interact with. 




% For some key roles and applications of on-body robots with older adults, the on-body robot was imagined to be a socially intelligent agent; this requires capabilities of affect communication. Beyond traditional methods of robotic affect communication, the proximity and on-body nature of these robots can enable richer modes of affect communication. For instance, tactile feedback 

% Moreover, the physical movement itself can be used to in meaningul manners to communicate affect. 

% Overall, leveraging the multiple communication modalities available to on-body robots can enable

% it is essential to design a meaningful on-boarding process that familiarizes the user with the robot, and eases them into the interaction.  Moreover, there are factors around the daily usage such as: charging, storing and finding the robot that need to be carefully investigated and designed for to enable long-term usage. Ultimately being a wearable device, questions were raised about how to clean on-body robots to be sanitary. \cmh{battery charging?} TAM (technology adoption model)

% \subsection{Temporal Context}

% The temporal context of usage encompasses the following factors: when the interaction happens, frequency and duration of each interaction and the overall timespan of usage of the on-body application.

% interactions with an on-body robot may occur on a schedule, when needed (in pain), when desired, or in perpetuity.

% the duration of each interaction can be short as a few seconds to an hour. the frequency of interaction similarly may happen every few seconds to perhaps every other day.

% overall timespan of usage of the on-body robot is largely application dependent can similarly be of short-term (for e.g. at a group class), medium-term (for e.g. posture retraining, physical therapy), long-term (for e.g. home excerise) or life-long (e.g. fall risk mitigation)

% DA Walking: Always wearing with active/passive mode


% This context particularly informs the design of the start and end of each interaction.

% \subsection{Relational Role}
% Mapping application to robot role: as a part of you, as a service/tool, as a companion.

% DA Walking: Pet-like walking companion 

% DA: Physical therapy: partner, ..., ..., 

% DA Walking: as a part of you

% DA Massage: as a service or tool

% The nature of the application and the temporal context heavily influence the type of relational role that the on-body plays; this role should be considered during the 

% \subsection{Communication}

% The communication types that emerged can be classified into two classes: user to robot and robot to user. 

% \textbf{User to Robot:} Participants envisioned communicating verbally to the robot as well as through non-verbal modalities such as gestures, pressing buttons on the robot and tapping. Moreover, in certain scenarios, participants envisioned communicating via a remote control or smartphone app.

% \textbf{Robot to Calico}

% Several types of communication modalities were imagined, interestingly leveraging the potential form factors and locomotive abilities of on-body robots.

% Physical movement in semantically meaningful patterns or certain locations on the body was proposed a to communicate intent.  

% Haptic modalities were also explored in forms of vibrations and pinches. 

% Apart from verbal communication, visual communication leveraging blinking or changing colors of LEDS was explored as well.

% Olfactory output with the robot excreting essential oils or burning incense was proposed as an interesting avenue of feedback.  

% \textbf{Interfacing w/ exiting tech}

% Interfacing with blood sugar monitors for body state communication.

% Interfacing with hearing aids for verbal communications.

% Interfacing with smartphone apps to communicate.

% What form of communication on-body robots not only depend on the application and the user, but also on the context. They ought to be able to use a variety of communication modalities and switch based on context.



% \subsection{Feedback}
% Two frequencies of feedback emerged: 

% Reactive feedback as response to a sensor/model recognition or robot action/body/environmental state change.

% Cyclic feedback that happens regularly as a method of checking in or reminding.

% DA Walking: Cyclic feedback (gong-like)

% These feedback types can happen to either recieve or give feedback.

% Three types of feedback emerged: 

% Corrective Feedback in form of fixing gait, fixing posture, ... 

% Supportive Feedback in form of encouragement, rewards ... 'Dopamine' feedback that happens as reaction to a successful event to reward the user. 

% Informative Feedback in form of reminders, explanation of actions .. 

% \subsection{Levels of Autonomy and Intelligence (kb)}
% Perception and modelling of environment
%     environmental state (tripping hazards, weather, visibility, location) 
%     placement of the robot with relation to the body.
% Perception and modelling of human-related factors
%     user physical state (health stats, muscle tension, balance, range of motion, olfactory)
%     user mental state (resistance to activity start)
    

% Planning actions to interact with environment and human(s)
%     largely autonomous operation were desired with some wanting more control
    
% Executing plans under physical and social constraints

% \subsection{Sensors}

% The suite of sensors available on the on-body robot is a crucial factor in determining the type of communication and modelling that are possible and the very nature of applications that can be enabled. 

% Vision, Sound, Touch, Olfactory, Temperature and Humidity sensors came up in our workshops. Moreover, with the virtue of being an on-body robot, ability to sense body state and variable was desired such as: Skin Temperature, Sweat, Pulse, Muscle Tension, Heart rate, Blood Sugar. 

% \subsection{Embodiment}
% Wanted cute embodiment, however, for application specific design wanted utilitarian.
%     -- appliaction dependent
% Usage in public was an interesting dimension

% Would rather it not be an indicator of disability

% application as jewelry

% \section{Design Principles For On-Body Robots}

% Building on the insights from our co-design workshops with older adults, we propose the following overarching design principles for on-body robots. These principles aim to guide designers in creating robots that are not only functional but also seamlessly integrate into the lives of older adults

% \subsubsection{Adapt To Social Norms}

% On-body robots are a novel concept that can attract attention, potentially causing discomfort or embarrassment for older adults in social settings. To mitigate this, apart from to designing these robots with aesthetically pleasing or discreet embodiment's, the robot's communication methods should be context-aware—utilizing subtle cues like gentle vibrations in public to avoid disclosing private information or drawing unwanted attention while retaining more expressive modes of communication for private settings. Additionally, these robots could be designed to handle interests from other social actors to blend in more easily with social norms and fast-track normalization of the concept. While more widespread adoption of on-body robots may eventually influence social norms, initial designs should consider prioritizing conformity to existing expectations to facilitate acceptance and integration.

% \subsubsection{Ensuring Physical Comfort}

% The intimate, on-body nature of these robots makes physical comfort a vital design requirement. On-body robots should be designed with materials and forms that are lightweight, soft, and compliant to avoid causing discomfort or injury; Materials and forms that conform to the body can enhance comfort and make the robot feel like a natural extension rather than an intrusive device. 

% These robots should consider the user’s kinesiology, avoiding interference with natural movements and being sensitive to areas that may be prone to discomfort or pain. Robots should have the ability to detect and respond to the user’s physical cues—adjusting pressure, movement, or position to maintain comfort. 

% Additionally, avoiding sensory overload is crucial. Abrupt robot behaviors or over-stimulation through constant sensory feedback can startle users, particularly those who may have heightened sensitivity. Thoughtful consideration of communication frequency and modality can help maintain a comfortable and non-disruptive user experience.

% \subsubsection{Integration Into Daily Life}

% For on-body robots to be embraced by older adults, they must integrate effortlessly into daily routines. This seamless integration involves not only practical aspects like charging, cleaning, and maintenance but also requires intuitive interaction between the user and the robot. To enable intutive Communication, robots should accommodate sensory variations among older adults to use multi-modal communication pathways embedded with appropriate affordances.

% Considering the co-presence of other entities—such as partners, pets, or assistive devices like walking sticks—is necessary to ensure the robot fits into older adults' lives rather than requiring significant lifestyle changes to enable usage. Durability and resilience against unintended interactions can further enhance integration into the user’s life.


% \subsubsection{Providing Clear Utility}

% For on-body robots to become a common paradigm, they must offer clear and significant utility, especially when compared to existing, less intrusive technologies. Participants highlighted the necessity of perceiving tangible benefits to motivate the transition to this new technology. On-body robots possess unique advantages due to their proximity to the body and locomotive capabilities. These features enable functionalities that are not feasible with stationary devices or conventional wearables.

% Designers should leverage these capabilities to address unmet needs or enhance existing solutions meaningfully. For example, real-time health monitoring with immediate feedback can provide a level of support not available through other means. However, the current technological maturity of on-body robots is relatively nascent as Lauren(F/74) noted saying \textit{``I'd be waiting for not the calico one o one. I want to wait for four o something... my watch does all of these things but it doesn't move\dots I don't want it yet... while it looks like that.''} Careful consideration of the various design dimensions—such as context, human factors, robot characteristics, and communication methods—is crucial to create intuitive and comfortable experiences that validate the robot’s usage.  \todo{findability and visibility}


% \subsubsection{Creating a Cohesive Robot Agency}

% The robot’s autonomy, communication style, and physical appearance should collectively convey a coherent identity that resonates with the user. This includes designing appropriate levels of autonomy—allowing the robot to act independently when beneficial but always keeping the user in control. Communication should be consistent in tone and style, matching the robot’s role (e.g., supportive coach, friendly companion). Embodiment features should reinforce this identity, using design elements that evoke the desired emotional response. A cohesive agency helps build trust and familiarity, making interactions more natural and enjoyable.


% \subsection{On-Body Robot as a Pet-like Walking Companion}
% For design an on-body robot as pet-like walking companion as briefly discussed in the \textit{Walking} application-focused workshop. Participants imagined using this robot as a way of accountability to maintain regular walking habit and as a means of monitoring which define the application level.
 
% \begin{displayquote}
% \textbf{Leona} (F/??):
% \textit{``if you state it, 'ok *sigh*, let's take a walk'. Then [the robot] it gets ready. Ok, let's go! like a dog all excited and there's no backing out.''}
% \end{displayquote}


% \subsubsection{Level 1 Variables}

% The \textit{human} element, focusing on kinesiology, plays a key role in shaping the design. In this case, the older adult engages primarily in walking, with occasional breaks, such as resting in a park. There are no significant sensory or cognitive impairments that would hinder interaction with the robot. The robot’s \textit{application} centers on accountability and health monitoring, ensuring it acts as a reliable walking companion for the user’s daily routine over the long term.

% The \textit{context} of use includes public settings, such as community spaces and parks, where the user may encounter strangers or walk with friends. These social environments influence the robot’s design, ensuring it remains subtle and avoids drawing unnecessary attention.

% \subsubsection{Level 1 to Level 2 Design Anchors}

% The design choices informed by the Level 1 variables are crucial to the robot's \textit{embodiment}, \textit{topology}, \textit{perception}, \textit{autonomy}, \textit{communication content}, and \textit{modality}.

% The robot’s embodiment must consider the user’s kinesiology. To avoid fatigue during walks, the robot is designed to be lightweight and unobtrusive. As a companion, it takes on an active role, similar to a pet, offering customization to match the user’s preferences. Given its use in public spaces, the robot avoids flashy designs that might attract unwanted attention.

% In terms of topology, the robot's placement on the body is carefully considered. It is positioned within the user’s line of sight but does not restrict movement, ensuring comfort during walks and smooth social interactions. The robot must not interfere with the user’s natural motion or their ability to engage with others.

% Perception plays a key role in the robot’s ability to monitor the user. It tracks health indicators such as heart rate, blood sugar, energy levels, and emotional state, while also remaining aware of environmental factors like weather conditions, uneven terrain, and low-light situations.

% The robot’s autonomy is designed to be proactive yet respectful of the user’s autonomy. For instance, the robot may suggest eating if blood sugar levels are low, but always allows the user to make the final decision. In public settings, it behaves in a way that respects social norms, balancing proactive suggestions with the user’s comfort and minimizing interruptions.

% The communication content is tailored to provide subtle and encouraging feedback. It offers gentle nudeges to promote healthier walking habits while ensuring discretion in public to avoid causing embarrassment or drawing attention. Non-verbal communication methods, such as physical nudges or expressive behaviors (e.g., appearing excited or satisfied), allow the robot to convey messages in a subtle manner, appropriate for public settings without loud noises or overt displays.

% \subsubsection{Level 2 Interactions}

% The interaction between Level 2 elements enhances the robot’s overall functionality. For example, its \textit{perception} informs its \textit{autonomy} by using sensor data to make contextually relevant suggestions, such as recommending a snack if the user’s blood sugar is low. Similarly, the robot’s \textit{embodiment} influences its \textit{communication modality}. A pet-like robot without a voice may rely on non-verbal cues, such as movement or light patterns, to communicate effectively with the user.

% Moreover, if the design involves multiple robots (for instance, for enhanced balance monitoring), their \textit{topology} must enable coordination between them, ensuring a seamless and coherent interaction with the user. The collective function of these robots, such as inflating to cushion a fall, directly influences their embodiment and how they are perceived by the user.


% \textit{Proposed Scenario:} The user has robot on a sweater/tee-shirt they are wearing and says ``lets take a walk'' and the robot sense that and ``gets excited'' (plays a short tune or melody). The robot tells the user information about the weather (temperature, humidity, wind) and suggests that they eat something if their blood sugar is lower.

% Level 1:
% Human --- Kinesiology, walking might also sit down in park for a break

% no sensory and cognitive factors

% Application---Role: accountability and monitoring companion, usage is daily long-term

% Context- co-presence, stranger walkers, sometimes friends, location is public (community, park etc)


% Level 2:

% embodiment: human (kin) not too heavy, should have a active presence, role---> companion pet should look like it (customizable), context---> shouldn't be flashy

% Topology: human(kins) ---> not on the back cause walking. 
% pet like companion ----> not restrictive just at eyeline. 
% context---> not restrictive 

% perception: monitor ---> human health variables (heartbeat, blood sugar, energy levels, excitement), environmental (weather forecast, environmental factors (uneven surfaces, dark areas etc), 

% Autonomy: accountability ---> can be proactive and take charge, context since it is basic walking ---> user always has the final say. frequency: role: accountability ---> proactive, always reactive

% content: role---> corrective actions, promotion on better walking etc, motivation through nudges (implicit)


% modality: role ---> physical nudges, implicit encouragement, look sad to convey disappointment, look excited to motivate, look satisfied as positive feedback, 
% context---> if public, only non-verbal, confidential silent signals, no loud alarms to attract attention or cause embarrassment


% level 2 to level 2
% perception---> autonomy, reaction based on sensor data

% embodiment----> no voice for animal looking bots, vocal and non-verbal cues

% topology (number of robots) ---> modality: multiple robots for walking ---> should communicate among each other and in a coherent manner.  

% topology ---> embodiment, multiple robots inflate to prevent fall






% This information enables us to define enable 
