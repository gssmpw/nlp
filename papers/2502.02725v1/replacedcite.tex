\section{Related Works}
\subsection{Wearable Robots}

Wearable robots vary widely in form and function. Exoskeletons, a prominent class, focus on mobility assistance and rehabilitation, targeting specific areas such as shoulders or hands ____, or providing broader support to regions such as the lower limbs ____. Other wearable robots offer physical augmentation, such as a third thumb ____ or arm ____, delivering active assistance while remaining fixed in place. Moreover, stationary robots perched on the shoulder have been studied as wearable companions ____.

% Wearable robots encompass a broad range of systems in terms of form and function. Exoskeletons, a prominent class, are primarily designed for mobility assistance and rehabilitation, targeting specific areas such as the shoulder ____ or hand ____, or providing support to larger regions like the lower limbs ____. While these systems enhance muscle actuation, they are not actively actuated themselves. Other wearable robots provide physical augmentations, such as a third thumb ____ or third arm ____, offering active assistance while remaining fixed in place. Stationary humanoid robots perched on the shoulder have also been explored as wearable companion robots ____.


A newer class of wearable robots consists of on-body, locomotive systems, capable of moving across the body, offering greater flexibility in interaction. These robots can move via direct skin contact ____, climb on clothing ____, or travel along tracks embedded in garments ____. This mobility sets them apart by enabling dynamic, on-body interactions and unlocking new possibilities for user engagement. While the design needs of exoskeletons have been explored ____, effective interaction paradigms for movable, on-body robots remains unexplored. Developing these paradigms is essential if on-body robots would be developed to support older adults.

\subsection{Designing with Older Adults}

Co-design is an effective methodology for understanding the design needs of special populations, such as older adults, by leveraging their lived experiences ____. This flexible approach allows stakeholders to act as users, testers, informants, partners, or co-researchers ____. Its versatility spans contexts ranging from assistive robots for aging in place ____ and robots for dementia or depression ____, to challenges like promoting physical activity ____, designing fitness apps ____, and improving wearable tech adoption ____.

The flexibility of co-design extends to the diverse range of activities it supports, such as sketching ____, story-boarding ____, mind-mapping ____, prototyping ____, worksheets ____ and role-playing ____. This adaptability allows for the selection of design exercises that foster both divergent and convergent design thinking ____. In addition to design exercises, low-fidelity design probes can further inspire design thinking and offer new insights for future technologies ____. To effectively engage older adults as designers of on-body robots, we employ a co-design approach that incorporates various design activities to foster both convergent and divergent thinking, with a design probe used to anchor the design process. 


% \todo{talk about design probe: }
% %https://dl.acm.org/doi/pdf/10.1145/3520495.3520513