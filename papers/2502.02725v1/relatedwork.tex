\section{Related Works}
\subsection{Wearable Robots}

Wearable robots vary widely in form and function. Exoskeletons, a prominent class, focus on mobility assistance and rehabilitation, targeting specific areas such as shoulders or hands \cite{oneill2017exoshoulder, gao2023portable, in2015exo}, or providing broader support to regions such as the lower limbs \cite{huo2014exolower, asbeck2014stronger}. Other wearable robots offer physical augmentation, such as a third thumb \cite{zhou2019novel, shafti2021playing} or arm \cite{muehlhaus2023need, vatsal2017wearing}, delivering active assistance while remaining fixed in place. Moreover, stationary robots perched on the shoulder have been studied as wearable companions \cite{tsumaki201220, jiang2019survey}.

% Wearable robots encompass a broad range of systems in terms of form and function. Exoskeletons, a prominent class, are primarily designed for mobility assistance and rehabilitation, targeting specific areas such as the shoulder \cite{oneill2017exoshoulder, gao2023portable} or hand \cite{in2015exo}, or providing support to larger regions like the lower limbs \cite{huo2014exolower, asbeck2014stronger}. While these systems enhance muscle actuation, they are not actively actuated themselves. Other wearable robots provide physical augmentations, such as a third thumb \cite{zhou2019novel, shafti2021playing} or third arm \cite{muehlhaus2023need, vatsal2017wearing}, offering active assistance while remaining fixed in place. Stationary humanoid robots perched on the shoulder have also been explored as wearable companion robots \cite{tsumaki201220, jiang2019survey}.


A newer class of wearable robots consists of on-body, locomotive systems, capable of moving across the body, offering greater flexibility in interaction. These robots can move via direct skin contact \cite{dementyev2017skinbot}, climb on clothing \cite{dementyev2016rovables, birkmeyer2011clash}, or travel along tracks embedded in garments \cite{sathya2022calico}. This mobility sets them apart by enabling dynamic, on-body interactions and unlocking new possibilities for user engagement. While the design needs of exoskeletons have been explored \cite{jung2019older}, effective interaction paradigms for movable, on-body robots remains unexplored. Developing these paradigms is essential if on-body robots would be developed to support older adults.

\subsection{Designing with Older Adults}

Co-design is an effective methodology for understanding the design needs of special populations, such as older adults, by leveraging their lived experiences \cite{rogers2022maximizing, alves2015social, ostrowski2021personal}. This flexible approach allows stakeholders to act as users, testers, informants, partners, or co-researchers \cite{alves2021children, nanavati2023design}. Its versatility spans contexts ranging from assistive robots for aging in place \cite{lee2018reframing, alves2015social} and robots for dementia or depression \cite{moharana2019robots, lee2017steps}, to challenges like promoting physical activity \cite{antony2023codesign}, designing fitness apps \cite{harrington2018designing}, and improving wearable tech adoption \cite{nevay2015role}.

The flexibility of co-design extends to the diverse range of activities it supports, such as sketching \cite{lee2017steps}, story-boarding \cite{bjorling2019participatory}, mind-mapping \cite{antony2023codesign}, prototyping \cite{bjorling2019participatory, lee2017steps}, worksheets \cite{axelsson2021social} and role-playing \cite{bjorling2019participatory}. This adaptability allows for the selection of design exercises that foster both divergent and convergent design thinking \cite{ostrowski2021long}. In addition to design exercises, low-fidelity design probes can further inspire design thinking and offer new insights for future technologies \cite{hutchinson2003technology, gough2021co}. To effectively engage older adults as designers of on-body robots, we employ a co-design approach that incorporates various design activities to foster both convergent and divergent thinking, with a design probe used to anchor the design process. 


% \todo{talk about design probe: }
% %https://dl.acm.org/doi/pdf/10.1145/3520495.3520513