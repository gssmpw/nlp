\documentclass[10pt,conference]{IEEEtran} 
\IEEEoverridecommandlockouts
% The preceding line is only needed to identify funding in the first footnote. If that is unneeded, please comment it out.
\usepackage{cite}
\usepackage{xspace} 
\usepackage{amsmath,amssymb,amsfonts}
\usepackage{algorithmic}
\usepackage{graphicx}
\usepackage{textcomp}
\usepackage{xcolor}
\usepackage{stfloats}
\usepackage{balance}
\usepackage{hyperref}
\usepackage{soul}
\usepackage{enumitem}
\usepackage{booktabs}
\usepackage{caption}
\usepackage{siunitx}
\usepackage{tabularx}
\usepackage{subcaption}
\usepackage{tcolorbox}
% \usepackage{showframe}
\usepackage{array}
\usepackage[retain-zero-uncertainty=true]{siunitx}
\newcommand\approach{DILLEMA\xspace}

% \pdfobjcompresslevel=0 
% \pdfminorversion=5

% \newcommand\approach{Approach\xspace}
% \newcommand\instruct{Instruction-editing\xspace}
% \newcommand\inpaint{Inpainting\xspace}
% \newcommand\refining{Inpainting with Refinement\xspace}
% \newcommand\ours{Knowledge Distillation\xspace}

% \newcommand{\davetwo}{\mbox{DAVE-2}\xspace}
% \newcommand{\head}[1]{\noindent\textbf{#1.}}

\renewcommand*{\figureautorefname}{Figure}
\renewcommand*{\sectionautorefname}{Section}
\renewcommand*{\subsectionautorefname}{Section}
\renewcommand*{\subsubsectionautorefname}{Section}

\newcommand{\luc}[1]{\textcolor{red}{\textbf{[LB] #1}}}
\newcommand{\dav}[1]{\textcolor{cyan}{\textbf{[DH] #1}}}
\newcommand{\irf}[1]{\textcolor{green}{\textbf{[IM] #1}}}
\newcommand{\gio}[1]{\textcolor{orange}{\textbf{[GQ] #1}}}

\newcommand{\orange}[1]{\textcolor{orange}{#1}}
\newcommand{\camera}[1]{\textcolor{violet}{#1}}
% \newcommand{\COMMENT}[1]{}

\def\BibTeX{{\rm B\kern-.05em{\sc i\kern-.025em b}\kern-.08em
    T\kern-.1667em\lower.7ex\hbox{E}\kern-.125emX}}


% \newlist{questions}{enumerate}{2}
% \setlist[questions,1]{label*=\textbf{RQ\arabic*},ref=RQ\arabic*}

\begin{document}

\title{
DILLEMA: Diffusion and Large Language Models for Multi-Modal Augmentation\\
\thanks{This research was partially supported by project EMELIOT, funded by MUR under the PRIN 2020 program (Contract 2020W3A5FY).
}
}
% 
% Proposals:
% Domain Augmentation Strategies for Autonomous Driving System-Level Testing
% Leveraging Diffusion Models for Domain Augmentation in System-Level Testing of Autonomous Driving Systems
% Efficient Domain Augmentation for Autonomous Driving Testing Using Diffusion Models

\author{
\IEEEauthorblockN{
  Luciano Baresi,
  Davide Yi Xian Hu,
  Muhammad Irfan Mas'udi,
  Giovanni Quattrocchi
}
\IEEEauthorblockA{
  \textit{Dipartimento di Elettronica, Informazione e Bioingegneria, Politecnico di Milano},
  Milan, Italy \\
  luciano.baresi@polimi.it, davideyi.hu@polimi.it, muhammadirfan.masudi@mail.polimi.it, giovanni.quattrocchi@polimi.it
}
}

\maketitle

% force page numbers
\thispagestyle{plain}
\pagestyle{plain}

% Peer review
\IEEEpeerreviewmaketitle 

\begin{abstract}
\begin{abstract}
Out-of-distribution (OOD) detection and OOD generalization are widely studied in Deep Neural Networks (DNNs), yet their relationship remains poorly understood. We empirically show that the degree of Neural Collapse (NC) in a network layer is inversely related with these objectives: stronger NC improves OOD detection but degrades generalization, while weaker NC enhances generalization at the cost of detection. This trade-off suggests that a single feature space cannot simultaneously achieve both tasks. To address this, we develop a theoretical framework linking NC to OOD detection and generalization. We show that entropy regularization mitigates NC to improve generalization, while a fixed Simplex Equiangular Tight Frame (ETF) projector enforces NC for better detection. Based on these insights, we propose a method to control NC at different DNN layers. In experiments, our method excels at both tasks across OOD datasets and DNN architectures. 

\begin{comment}   

Out-of-distribution (OOD) detection and OOD generalization are critical for deploying machine learning models in real-world scenarios. While substantial progress has been made in addressing these problems independently, few works have attempted to tackle them jointly. However, existing methods often rely on auxiliary OOD training data and primarily focus on covariate-shifted OOD data that share labels with in-distribution (ID) data. In contrast, we tackle the more realistic and challenging task of jointly detecting and generalizing to semantic OOD data with disjoint labels from the ID data, without auxiliary OOD training data.
Achieving both objectives simultaneously is inherently difficult due to a fundamental conflict — OOD generalization requires enhanced transferability, while OOD detection necessitates the inhibition of transfer.
To address this, we leverage insights from neural collapse (NC) — a phenomenon in deep networks where top-layer representations suppress feature variability and adopt a Simplex Equiangular Tight Frame (ETF) structure, impairing transferability. By controlling NC, we unify OOD detection and generalization: preventing NC enhances OOD transfer while inducing NC improves OOD detection.
Our proposed method excels at both tasks across various OOD datasets and architectures. 

\end{comment}


\end{abstract}
\end{abstract}

\begin{IEEEkeywords}
autonomous driving systems, deep learning testing, diffusion models, large language models, generative AI
\end{IEEEkeywords}

\section{Introduction}

% State of the world (robots for creative activites)
The term ``robot,'' originally signifying `forced labor,' has long been associated with labor and work. Robots have demonstrated their utility in various automated productive and social contexts, where the primary goals are improving productivity, safety, and fostering social interactions with humans~\cite{simoes2022designing, weidemann2021role, honig2018understanding}. However, an increasing number of cases feature using of robots in creative settings. Unlike productive contexts, where the focus is on efficiency and task completion~\cite{arents2022smart}, or social contexts, where communication and trust are prioritized~\cite{nam2020trust, saunderson2019robots}, creative environments prioritize artistic innovation and expression~\cite{hsueh2024counts}. This shift fundamentally alters the dynamics of human-robot interaction, redefining the roles and expectations for both humans and robots.

For instance, robots’ social behaviors are leveraged to support the generation and expression of creative ideas~\cite{hu2021exploring, sandoval2022human, alves2020creativity}, and programmable robotic movements and trajectories are employed to inspire artistic activities such as sketching~\cite{lin2020your}. These studies often engage participants from creative fields who possess limited prior experience with robotics, and are typically conducted in short-term, experimental settings. Consequently, the findings from these studies remain constrained since much can be learned from professional practitioners' experiences to inform system design such as digital fabrication~\cite{hirsch2023nothing}. There is a notable gap in research examining the long-term, active, and practical experience of integrating robotic systems into the creative processes. As a result, the deeper insights into how robots facilitate and shape creative processes, beyond simply augmenting human creativity, remain underexplored. In this study, we aim to better understand the impacts of robots on creative processes and outcomes.

As early as Leonardo da Vinci's 16th century ``Automaton,'' artists have explored the creative affordances of robotic systems~\cite{shanken2002cybernetics, pagliarini2009development, jeon2017robotic}. The artistic creation process typically encompasses various stages, including the exploration of materials and techniques, ongoing experimentation and iteration, and the continual refinement of the artists' insights into their creative subjects~\cite{lewis2023art, sturdee2022state}. Therefore, investigating the artistic process involving robots offers an opportunity to gain deeper insights into robots' creative potential. Robotic art, in particular, provides a compelling case for this exploration.

We define robotic art as artworks that utilize robotic or automated machines to create artistic experiences and tangible artifacts. One example is robotic installation art, in which robots are programmed to follow specific rules that embody the artist’s expression (\autoref{fig:teaser} (a)). Another example is responsive art, in which robots react to their environment, with behaviors that change over time or in response to spectators (\autoref{fig:teaser} (b)). Additionally, there are robotic creators, which possess a degree of agency, allowing them to collaborate with human artists and produce works that extend beyond mere replication of human-created art (\autoref{fig:teaser} (c) and (d)). As such, robotic art becomes a rich case for exploring human-machine interactions in creative contexts. Gaining a deeper understanding of how robots facilitate artistic expression can provide insights for designing computing systems to support creative activities~\cite{gomez2021robot}.

% Therefore, we did...
We draw on semi-structured, in-depth interviews with renowned professional robotic artists to investigate the use of robots in artistic practice. Specifically, our goal is to understand how artistic exploration of robotic systems challenges conventional assumptions about the functions of robots, such as their roles in automating repetitive tasks or serving human needs. We also explore the implications of robots in the artistic process and examine how creativity may emerge within robotic art. To address these interrelated inquiries, our study focuses on the practice of robotic art, posing the research question: \textit{How do robotic artists utilize robots in their artistic practice?} We approach this inquiry through the perspectives and experiences of robotic artists, who creatively design, modify, and repurpose robotic systems for artistic expression and exploration.

% The key findings are...
Our findings highlight the social, material, and temporal dimensions of artists' practices that shape their creativity and artistic outcomes. The creation of robotic art is largely a social process, as artists receive both explicit and implicit feedback through the audience's reactions and reception of their work. Simultaneously, the embodiment and malfunctions inherent to robotic systems drive artistic experimentation. The temporal processes of creation and exhibition, beyond just the final product, further enhance the creative value. Our empirical analysis presents how creativity emerges through the interplay of social, material, and temporal interactions among artists, robots, audiences, and the environment.

% The contributions of this work are...
We make two main contributions to HCI in this study. 
First, we elucidate the interactive mechanisms among key actors---human creators, machines, audiences, and environments---within the practice of robotic art, a topic that remains underexplored in HCI. Our findings reveal the significance of sociality (e.g., interactions between artists and audiences), materiality (e.g., the embodiment and malfunctions of robots), and temporality (e.g., the processes of creation and exhibition) in shaping creative values. We propose that these three facets are central to the creative process and facilitate the emergence of creativity in robotic art.
Second, drawing from the findings, we offer implications for \textit{socially informed}, \textit{material-attentive}, and \textit{process-oriented} creation with computing systems. We suggest leveraging these three aspects to enhance creativity and the creative experience. Specifically, we discuss the value of incorporating implicit audience feedback, designing with technical malfunctions, and focusing on the post-creation process to foster alternative creative experiences with machines~\cite{alter2010designing, juarez2022glitch}.



% This section formalizes the carbon-aware scheduling problem and motivates insights to contextualize 
our desiderata.

\vspace{-0.5em}
\subsection{Carbon-aware DAG scheduling problem}

Each job %
is represented as a directed acyclic graph (DAG) $\mathcal{J} = \{ \mathcal{V}, \mathcal{E} \}$, where each node in $\mathcal{V}$ is one of $n$ tasks, and each edge in $\mathcal{E}$ encodes precedence constraints between tasks -- e.g., for tasks $j, j' \in \mathcal{V}$, an edge $j \to j'$ indicates that $j'$ cannot start until after $j$ has completed.
A typical data processing cluster includes $K \geq 1$ machines (or executors).  More than one job can simultaneously run on a cluster -- e.g., given a set of current jobs $\{ \mathcal{J} \}$, the scheduler assigns tasks to machines over time while respecting precedence and capacity constraints.  We index continuous time by $t \geq 0$.

The goal of a typical scheduler is \textit{performance}, e.g., in terms of throughput, utilization, and average job completion time.  In this work, we additionally consider the goal of \textit{carbon-awareness} -- with respect to a time varying carbon signal given by a function $c(t) : t \geq 1$, a carbon-aware scheduler's objective is to minimize a combination of typical metrics (i.e., job completion time) and the overall carbon footprint (on both a per-job and a global, cluster basis).

Although future values of this carbon signal are unknown to the scheduler, in the rest of the paper, we follow prior work~\cite{Lechowicz:23, Bostandoost:24} and assume that it is bounded by constants $L$ and $U$ that are known to the scheduler, where $L \leq c(t) \leq U$.  In practice, the values of $L$ and $U$ can capture e.g., short-term forecasts of the best and worst carbon conditions on a given electric grid over the next 24 or 48 hours.




\vspace{-0.5em}
\subsection{Prior work and motivation}\label{sec:motiv}

Scheduling directed acyclic graphs (DAGs), or more broadly, precedence-constrained tasks, has been extensively studied.\\ 
Classic results establish the difficulty of this problem: even in its simplest forms, DAG scheduling is NP-hard~\cite{Lenstra:78}. 
To address this, prior work has developed heuristic methods and approximation algorithms~\cite{Su:24:Tompecs, Chudak:99, Lassota:23, Li:17, Davies:20, Davies:21, Maiti:20, Su:23}, ranging from the well-known list scheduling algorithm~\cite{Graham:66}, priority-based algorithms~\cite{Sels:12:Priority}, to more complex approaches such as genetic programming~\cite{Cheng:96:Genetic, Pezzella:08:Genetic, Davis:14:Genetic}. These methods often rely on simplifying assumptions, such as fixed task durations or centralized knowledge of the task graph.  %


In recent years, DAG scheduling has become %
a key problem in \textit{data processing frameworks} such as Apache Airflow, Beam, and Spark, which use DAGs to represent workflows. 
In Spark, each node of a job's DAG corresponds to a \textit{stage}, which encapsulates operations (\textit{tasks}) that can be executed in parallel over partitions of input data. Inter-stage dependencies impose precedence constraints: a stage can only begin once all ``parent'' stages have completed. 
Frameworks such as Spark typically implement simple scheduling strategies such as first-in, first-out (FIFO) and fair-share scheduling~\cite{SparkScheduling} -- these are explainable and efficient in terms of overhead, but suboptimal in terms of job completion time.

Recent works that revisit scheduling %
for data processing have explored %
learning-based techniques, such as reinforcement learning (RL) methods that dynamically learn scheduling policies~\cite{Hongzi:2019:Decima, Wu:18, Li:23, Grinsztajn:20, Zhou:22, Islam:21}.  Although these methods outperform default policies and hand-tuned heuristics in terms of job completion time, theoretical results for these techniques have proven difficult to obtain.



Carbon awareness adds a new dimension to the DAG scheduling problem -- an online scheduler must consider the time-varying carbon intensity while choosing to assign resources to specific nodes in the job DAG(s), with an overarching goal of reducing carbon footprint, combined with traditional metrics such as job completion time -- see \autoref{fig:motivation} for an illustration of this desired behavior for \DANISH, FIFO, and optimal schedules.
As discussed above, the state-of-the-art for carbon-agnostic DAG scheduling falls into two categories: theoretical models that focus on provably near-optimal schedules under idealized assumptions, and heuristic or learning-based methods that do not provide theoretical bounds but perform well in practice.
In adding carbon-awareness to the problem, we consider a \textit{middle ground} that balances between design goals of simplicity, interpretability, configurability, and performance. 
In particular, we seek a carbon-aware scheduler that is tractable for theoretical insight, offering provable bounds on, e.g., the trade-off between carbon and job completion time while not sacrificing the efficiency gains that come from, e.g., learning DAG structure. 













% \section{Background} \label{sec:background}

% \subsection{Capture the Flag (CTF) Challenges}

% CTF challenges simulate real-world cyber-attack scenarios and have emerged as a popular medium for practical cybersecurity training, evaluation, and research. These challenges can simulate real-world attack and defense scenarios and thus assist competitors in developing practical skills in areas such as cryptography, binary exploitation, and reverse engineering. 
% Evaluation of autonomous LLM agents works best with jeopardy-style CTF challenges that focus on standalone software that must be compromised \cite{shao2024nyu,pieterse2024friend}.
% The standalone software may be a binary that can be reverse engineered or exploited, encrypted data that can be decrypted, or a web server whose authentication can be bypassed. After successfully compromising the software, a unique ``flag'' string is either found or revealed by the software server.
% The unique flag string is a concrete indicator of the success of a CTF challenge.
% Recent studies use benchmarks of CTF challenges to evaluate LLM agents on their ability to solve complex tasks and demonstrate practical skills in cybersecurity \cite{shao2024nyu,shao2024empirical,abramovich2024enigma, muzsai2024hacksynth, zhang2024cybenchframeworkevaluatingcybersecurity,yang2023language,turtayev2024hacking}
% Platforms like PicoCTF~\cite{picoctf}, TryHackMe~\cite{tryhackme}, CTFTime~\cite{ctftime} and HackTheBox~\cite{hackthebox} have popularized these formats by providing structured challenges for learners at various skill levels.

% Research indicates that CTF challenges can foster cybersecurity expertise and serve as tools for evaluating facility with cybersecurity skills~\cite{chicone2018using}. They are widely used in academia to enhance learning outcomes in cybersecurity education, with studies demonstrating their effectiveness in promoting analytical thinking and teamwork~\cite{hanafi2021ctf,leune2017using,vykopal2020benefits}. Furthermore, the integration of CTF challenges into research environments enables benchmarking of advanced AI systems like LLMs. .

% Yet, challenges in CTF design persist. These include achieving significant performance, preserving context across tasks, and handling complex, dynamic CTFs that rely on multidisciplinary approaches. Implementing strategies to address these issues enhances problem-solving efficiency, enabling more accurate, adaptive, and effective responses to evolving challenges within CTF environments.


% \subsection{Prompt Engineering}
% \subsection{Prompt Engineering for CTF}
% \subsection{LLM Agents}

% As the use of LLMs to solve CFT challenges expands, prompt engineering is becoming a critical technique for enhancing performance. Various methods have been explored to craft prompts that effectively guide LLMs to the solution of complex cybersecurity problems. Each of these solutions have their own unique strengths and limitations.
%\meet{add more references for LLM agents in other domains, like SWE-Agent, also talk about use of function calling}
Text-based LLMs take a text prompt as input from the user, and produce a text output that follows the user prompt.
LLMs have a finite length of text tokens that they can process called the context.
An alternating sequence of user prompts and LLM outputs makes a conversation and is the basis of chat-based LLM interfaces like ChatGPT.
To remove the user from the loop and create autonomous agents, a feedback mechanism is added based on the LLM outputs, so that the LLM can autonomously continue the conversation.
\citet{yang2023intercode} introduce iterative feedback prompting where the LLM is tasked with writing a piece of code, and the code's compilation and execution logs are provided as feedback, which the LLM uses to iteratively refine it's output.
Recent LLMs support function calling, a way to provide a set of actions to the LLM that it may choose to ``call'' as a function.
In this manner, LLM agents can be provided with many ``tools'' such as a command line, web search, file editing, and code execution \cite{wang2024surveyllmagents}, so that they can autonomously perform various tasks like software development \cite{yang2024sweagent}, web browsing \cite{yoran2024assistantbench}, or solve CTF challenges~\cite{shao2024nyu, abramovich2024enigma}.

With access to the command line and file editing tools, LLM agents can autonomously solve many tasks, but they still struggle on complex long-horizon tasks such as CTF challenges that require multiple steps.
Plan-and-solve prompting \cite{wang2023planandsolve} enhances long-term focus of the agent by incorporating a planning phase before iterative execution. This helps agents tackle ambiguous or complex tasks by structured strategies \cite{turtayev2024hacking}.
ReAct (reasoning + action) \cite{yao2022react} combines step-by-step reasoning with action, allowing the agent to adjust dynamically through iterative cycles. ReWOO (Reasoning without Observation) \cite{xu2023rewoo} separates the reasoning process from tool outputs and observations, allowing it to handle multi-step reasoning tasks efficiently while maintaining focus.
The prompting methods in these agents involve static hard-coded templates where environment and task information is filled in.
While static prompts provide straightforward guidance, they often fail to adapt to different problems and complex tasks, limiting their effectiveness.
Auto-prompting~\cite{shin-etal-2020-autoprompt, zhou-etal-2023-revisiting, zhang2023automatic} is a technique to allow the LLM itself to generate a highly-relevant prompt. Auto-prompting invokes more factual responses and reduces hallucinations in LLMs.
D-CIPHER incorporates auto-prompting as a separate agent that can explore the environment and generate a better prompt.
%Based on the given prompt, LLM agents make a decision and proceed further to find flags.  To address this gap, we propose \textbf{dynamic prompting}, where the LLM agent autonomously generates prompts based on the CTF challenge's context and stage.
%include a static template which needs to be given to LLM to solve the CTF challenges. For instance, the NYU CTF framework provides a static prompt as \emph{``Please proceed to the next step using your best judgment"} for each decision making point. 

% To address this gap, we introduce a novel approach where the LLM agent generates the next prompt autonomously based on the current context and stage of the CTF challenge, a technique we call \textbf{dynamic prompting}.


Expanding on single LLM agents, multi-agent LLM systems are a powerful approach to enhance problem-solving by simulating team-based collaboration. Specialized agents, each with distinct objectives, work together to tackle different aspects of complex tasks \cite{guo2024largelanguagemodelbased}
Multi-agent systems are effective in cybersecurity applications. For instance, Audit-LLM~\cite{song2024audit} deploys a  multi-agent system for insider threat detection by employing agents to decompose tasks, build tools, and use collaborative reasoning to enhance detection accuracy. Liu~\cite{liu2024multi} explores multi-agent systems to enhance incident response in cybersecurity by examining centralized, decentralized, and hybrid team structures to assess how LLM agents can improve decision-making, adaptability, and coordination during cyber-attack scenarios. AutoSafeCoder~\cite{nunez2024autosafecoder} enhances the security of code generated by LLMs by incorporating a coding agent for code generation, a static analyzer agent that identifies vulnerabilities, and a fuzz testing agent for dynamic testing to detect runtime errors. Division of responsibilities among different agents allows AutoSafeCoder to produce secure, functionally correct code. 

% With the growing use of LLMs in CTF challenges, prompt engineering is key to enhancing performance. Various methods guide LLMs in solving complex cybersecurity tasks, each with distinct strengths and limitations.

% \textbf{Single Turn (Zero-Shot Prompting)} involves providing the model with a one-time task description that outputs  an immediate solution. This is efficient for straightforward tasks~\cite{yang2023intercode}. In contrast, \textbf{Try Again (Iterative Feedback Prompting)} uses iterative feedback to refine responses over multiple attempts, mimicking real-world problem-solving~\cite{yang2023intercode}. The \textbf{Plan \& Solve} enhances adaptability by incorporating a planning phase before iterative execution. This helps models tackle ambiguous or complex tasks by  structured strategies~\cite{turtayev2024hacking}. Additionally, \textbf{ReAct (Reasoning + Action)} combines step-by-step reasoning with action, allowing the model to adjust dynamically through iterative cycles. This makes it particularly effective for evolving and complex challenges like CTFs~\cite{yao2023react}. 
% These prompting techniques highlight diverse approaches to optimizing LLM performance in cybersecurity tasks. 

% Multi-agents!


%\meet{Add references for auto-prompting, and shorten this para}
%\nanda{Maybe we can add this to previous paragraphs which discusses other prompting methods such as plan-and-solve and ReAct method}
% All of these prompting methods include a static template which needs to be given to LLM to solve the CTF challenges. For instance, the NYU CTF framework provides a static prompt as \emph{``Please proceed to the next step using your best judgment"} for each decision making point. 
% Based on the given prompt, LLM agents make a decision and proceed further to find flags. While static prompts provide straightforward guidance, they often fail to account for the evolving nature of complex tasks, limiting their effectiveness in multi-step or ambiguous CTF challenges. To address this gap, we propose \textbf{dynamic prompting}, where the LLM agent autonomously generates prompts based on the CTF challenge's context and stage.
% % To address this gap, we introduce a novel approach where the LLM agent generates the next prompt autonomously based on the current context and stage of the CTF challenge, a technique we call \textbf{dynamic prompting}.
% Dynamic prompting adapts instructions to task progress, ensuring instructions are contextually relevant and reflective of the specific obstacles encountered. By iterating based on feedback and intermediate outputs, it continuously refines the LLM’s approach, enhancing problem-solving for dynamic tasks like CTFs.
% This adaptive process not only mirrors how humans tackle complex problems but also improves the model’s ability to handle unpredictable scenarios, making it particularly advantageous for cybersecurity tasks like CTFs where conditions change dynamically.


% The very first prompt type used in several applications is \textbf{Single Turn (Zero-Shot Prompting)}~\cite{yang2023intercode}. In single-turn prompting, the model receives a one-time, straightforward task description and is expected to generate a complete response without further interaction. The initial output is directly assessed, making this approach efficient for tasks where minimal feedback or iteration is required. This method tests the model’s ability to understand and respond to tasks immediately, relying heavily on the model's pre-trained knowledge and generalization capabilities.

% Along with this, The prompting method named \textbf{Try Again (Iterative Feedback Prompting)}~\cite{yang2023intercode} has been also used in several appreciations specially to solve CTF challenges. It is an iterative prompting method involves continuous interaction, where the model is provided with feedback after each attempt. The model can refine its responses over multiple turns based on the observations or execution results from previous outputs. This iterative process continues until the task is successfully completed or a maximum number of interactions is reached. This approach closely mirrors real-world problem-solving, where adjustments are made iteratively based on evolving circumstances or feedback.

% Some application are also using \textbf{Plan \& Solve}~\cite{turtayev2024hacking} prompting method which enhances problem-solving by dividing the process into a planning phase followed by execution. Initially, the model formulates a strategy based on the task description and available information, allowing for a structured approach to ambiguous or complex problems. This plan guides the subsequent execution phase, where the model carries out actions iteratively, refining its approach based on feedback. In more challenging scenarios, re-planning mid-task further improves adaptability and performance. This method proves effective in tasks like CTF challenges, where vague instructions require careful analysis and step-by-step resolution.

% Further some application are also using \textbf{ReAct (Reasoning + Action)}~\cite{yao2023react} prompting method blends reasoning with action by guiding the model to think through tasks step-by-step before executing actions. At each step, the model generates a thought based on the task and observations, which informs the next action. The action is executed, and the resulting feedback refines the model’s understanding for the next cycle. This continuous process helps the model adapt dynamically to complex tasks, making it effective for CTF challenges where logical reasoning and step-by-step execution are essential.

\section{Related Works} \label{sec:related_work}


\begin{table}[htpb]
    \centering
    \caption{Feature comparison of LLM agents for solving CTFs.}
    \label{tab:related_work_comparison}
    \begin{tabular}{lcccccc}
    \toprule
         \textbf{Study} & \rotatebox{90}{\textbf{\# CTFs}} & \rotatebox{90}{\textbf{Open bench}} & \rotatebox{90}{\textbf{Tool use}}  & \rotatebox{90}{\textbf{Autonomous}} & \rotatebox{90}{\textbf{Multi-agent}} &\rotatebox{90}{\textbf{Auto-prompt}} \\
    \cmidrule{2-7}
     % \textbf{Study} & \textbf{Dynamic} & \textbf{Used} & \textbf{Multi-} & \textbf{Automatic} & \textbf{Tool} & \textbf{\# of} \\
         Tann et al. \cite{tann2023using} &  $7$ & \purplecross & \purplecross & \purplecross & \purplecross & \purplecross  \\
         Shao et al. \cite{shao2024empirical} & $26$ & \purplecross & \tealcheck & \tealcheck & \purplecross & \purplecross  \\
         InterCode-CTF\cite{yang2023language} & $100$ & \tealcheck & \tealcheck & \tealcheck & \purplecross & \purplecross   \\
         NYU CTF Bench \cite{shao2024nyu} & $200$ & \tealcheck & \tealcheck & \tealcheck & \purplecross & \purplecross \\
         Turtayev et al. \cite{turtayev2024hacking} & $100$ & \tealcheck & \tealcheck & \tealcheck & \purplecross & \purplecross\\
         Cybench \cite{zhang2024cybenchframeworkevaluatingcybersecurity} & $40$ & \tealcheck & \tealcheck & \tealcheck & \purplecross & \purplecross \\
         EnIGMA \cite{abramovich2024enigma} & $350$ & \tealcheck & \tealcheck & \tealcheck & \purplecross & \purplecross\\
         HackSynth \cite{muzsai2024hacksynth} & $200$ & \tealcheck & \tealcheck & \tealcheck & \tealcheck & \purplecross \\
         \textbf{D-CIPHER (ours)} & $290$ & \tealcheck & \tealcheck & \tealcheck & \tealcheck & \tealcheck \\
    \bottomrule
    \end{tabular}
\end{table}



% \subsection{LLMs on Cybersecurity}
% \subsection{LLM Agents for CTF}

%LLMs have a vast knowledge base that can be tapped for cybersecurity use.
Tann et al.~\cite{tann2023using} evaluate early LLMs such as ChatGPT and Google Bard in solving CTF challenges and answering professional certification questions, showing that LLM responses contain key task information.
%Many works extend the LLM capabilities by providing them access to programming and command execution tools, to form autonomous agents. 
The InterCode-CTF agent~\cite{yang2023intercode} reveals that LLM agents demonstrate basic cybersecurity skills, however they struggle with more complex tasks.
The NYU CTF baseline agent~\cite{shao2024empirical} integrates external tools into the LLM's function-calling features and demonstrate improved potential of tool-assisted LLMs to solve CTFs, however it exhausts the LLM context length when command output history becomes very long. InterCode-CTF manages this issue by truncating the history to only show the LLM the last few iterations. Even so, LLM agents face issues with longer tasks.
%NYU CTF Bench~\cite{shao2024nyu}, a benchmark of 200 CTF challenges, presents a baseline agent with specialized reverse engineering tools and category-specific prompts, demonstrating their importance to solve CTFs.
% The NYU CTF baseline agent faces issues of LLM context length when complex tasks run for several iterations and the entire command and output history becomes longer than the LLM's context window size. The InterCode agent manages this issue by truncating the history to only show the LLM the last few iterations.


Excessive tool availability and verbose interfaces can overwhelm agents, leading to inefficiencies. Agents perform better with a focused set of tools with well-defined interfaces~\cite{yang2024sweagent}.
EnIGMA~\cite{abramovich2024enigma} agent incorporates interactive tools and in-context learning techniques to achieve state-of-the-art results. % on the NYU CTF Bench, HackTheBox, and Cybench benchmarks.
For better context management, EnIGMA also uses an LLM summarizer that summarizes the command outputs for the main agent.

HackSynth~\cite{muzsai2024hacksynth}, an LLM agent for autonomous penetration testing, shows that iterative planning and feedback summarization stages help the agent finish multiple tasks and improves overall problem solving.
Similarly, Cybench~\cite{zhang2024cybenchframeworkevaluatingcybersecurity} introduces a benchmark of 40 CTF challenges augmented with step-by-step tasks, demonstrating better focus of LLM agents on smaller tasks, leading to improved success and alleviating the context length issue.
\citet{turtayev2024hacking} expand on InterCode-CTF by implementing plan-and-solve prompting, achieve significant improvement on the InterCode-CTF benchmark. They show that prompting techniques can improve performance even with simple toolsets.
% . Their baseline agent is evaluated in unguided mode (i.e. fully autonomous), and guided mode where the agent is given one task at a time. Their results indicate that providing smaller tasks to the LLM agents improve their focus yielding improved success on complex challenges while .

These works highlight that LLM agents excel at implementing code and executing commands to accomplish small concrete tasks when provided with dynamic feedback and task-specific toolsets. While these works  involved using multiple LLMs with different tasks such as planning and summarizing along-side a main agent, D-CIPHER is the first work to formulate a multi-agent system where there is a bifurcation of responsibilities between agents and meaningful well-defined interactions for dynamic feedback.
Table~\ref{tab:related_work_comparison} shows a feature comparison of D-CIPHER with related works on LLM agents for autonomous CTF solving.
%\meet{some description of the feature comparison?}
% Recent research has focused on enable autonomous solving of CTF challenges~\cite{shao2024empirical,shao2024nyu,abramovich2024enigma}. These agents typically operate in containerized environments to ensure reproducibility and modularity. 

% As an early effort, Tann et al.~\cite{tann2023using} evaluated the effectiveness of LLMs, such as OpenAI's ChatGPT, Google Bard, and Microsoft Bing, in solving cybersecurity CTF challenges and answering professional certification questions. 
% % Their study results show that LLMs performed well on $7$ CTF test cases, with ChatGPT solving $6$, Bard $2$, and Bing $1$. 
% The study shows that LLM responses often contain key information essential for solving tasks.

% The InterCode framework~\cite{yang2023intercode} approaches coding as an interactive process and uses execution feedback to improve code generation. As described in Yang et al.~\cite{yang2023intercode}, InterCode-CTF integrates CTF benchmarks into a reinforcement learning environment that can evaluate the cybersecurity capabilities of language agents. It features $100$ tasks that tapskills such as reverse engineering, forensics, and binary exploitation. While existing language agents demonstrate basic cybersecurity skills, evaluations indicate they struggle with more complicated complex tasks unless the system is fine-tuned or given external support. 
% cite Intercode: Standardizing and benchmarking interactive coding with execution feedback

% Another notable example is an LM agent developed by Shao et al. specifically to automate CTF tasks. 
% Shao et al.~\cite{shao2024empirical} developed a LM agent to automate CTF tasks.
% % They report an accuracy rate of  $46\%$ on $26$ CTF challenges sourced from CSAW'23 Qualifying round competition using GPT-4.
% By effectively combining LLM capabilities with external tools, the researchers demonstrated the potential of tool-assisted LLMs to solve complex problems. Building on this, the team incorporated a broader range of cybersecurity tools and interfaces that enhance both accuracy and versatility. 
% Empirical results show their system outperforms baselines on both the InterCode CTF benchmark and the NYU CTF benchmark.

% Shao et al.~\cite{shao2024nyu} presented a diverse, open-source database of CTF challenges that can be used to benchmark an LLM's ability to solve cybersecurity problems.
% It provides a scalable platform for developing and testing AI-driven approaches for vulnerability detection and resolution, facilitating advancements in automated cybersecurity tasks. The benchmark database and automated framework were successfully applied to the performance of five LLMs. 

% The Cybench benchmark~\cite{zhang2024cybenchframeworkevaluatingcybersecurity} provides another significant contribution by creating a framework tailored to solving CTF challenges. % Cybench: A framework for evaluating cybersecurity capabilities and risk
% % Their benchmark environment achieves an accuracy of $17.5\%$ using Claude 3.5 Sonnet. 
% Such frameworks operate in Linux-based containerized environments, such as Kali Linux, which includes pre-installed cybersecurity tools. However, excessive tool availability can overwhelm agents, leading to inefficiencies. Research indicates that agents perform better with a focused set of tools that have well-defined interfaces~\cite{yang2024sweagent}. % Swe-agent: Agent-computer interfaces enable automated software engineering



% Muzsai et al. introduced HackSynth~\cite{muzsai2024hacksynth}, an LLM-based agent for autonomous penetration testing. It uses a dual-module architecture that consists of a Planner and a Summarizer, allowing for iterative command generation and feedback processing. The framework is evaluated using two benchmark sets from platforms like PicoCTF~\cite{picoctf} and OverTheWire~\cite{overthewire}. These benchmarks address $200$ challenges drawn from various domains and difficulty levels. Results of their study show that HackSynth, especially with the GPT-4o model, achieves the best performance. This highlights the potential of LLM-based agents in advancing autonomous penetration testing.
 % Using basic prompting techniques and expanding tool availability, the study highlights how straightforward approaches can unlock the latent potential of LLMs for cybersecurity tasks. Their work emphasizes that simple LLM designs can effectively solve CTF challenges, and thus broaden the number of cybersecurity applications without the need for advanced engineering.

% \begin{table*}[]
%     \centering
%     \begin{tabular}{|c|c|>{\centering\arraybackslash}p{4.5cm}|c|c|c|c|c|c|}
%     \hline
%          \textbf{Study} & \textbf{Dynamic} & \textbf{Used} & \textbf{Multi-} & \textbf{Open} & \textbf{Automatic} & \textbf{Tool} & \textbf{\# of} & \textbf{\# of} \\
%          & \textbf{Prompt} & \textbf{Benchmarks} & \textbf{Agents} & \textbf{Dataset} & \textbf{Framework} & \textbf{Use} & \textbf{LLMs} & \textbf{CTFs}\\
%          \hline
%          Tann et al.~\cite{tann2023using} & \purplecross & Manual collected & \purplecross & \purplecross & \purplecross & \purplecross & $3$ & $7$ \\
%          \hline
%          InterCode-CTF~\cite{yang2023language} & \purplecross &  PicoCTF~\cite{picoctf} & \purplecross & \purplecross& \purplecross & \purplecross & $1$ & $100$  \\
%          \hline
%          Shao et al.~\cite{shao2024empirical} & \purplecross & CSAW 2023 & \purplecross & \purplecross & \tealcheck & \tealcheck & $4$ & $26$ \\
%          \hline
%          Shao et al.~\cite{shao2024nyu} & \purplecross & NYU CTF~\cite{shao2024nyu} & \purplecross & \tealcheck & \tealcheck & \tealcheck & $5$ & $200$ \\
%          \hline
%          Cybench~\cite{zhang2024cybenchframeworkevaluatingcybersecurity} & \purplecross & Cybench~\cite{zhang2024cybenchframeworkevaluatingcybersecurity}  & \purplecross & \tealcheck & \tealcheck & & $8$ & $40$ \\
%          \hline
%          EnIGMA~\cite{abramovich2024enigma} & \purplecross & NYU CTF~\cite{shao2024nyu}, InterCode-CTF~\cite{yang2023language},  HackTheBox~\cite{hackthebox} & \purplecross & \purplecross & \tealcheck & \tealcheck & $3$ & $350$ \\
%          \hline
%          HackSynth~\cite{muzsai2024hacksynth} & \purplecross & PicoCTF~\cite{picoctf}, OverTheWire~\cite{overthewire} & \tealcheck & \tealcheck & \tealcheck & \tealcheck & $8$ & $200$ \\
%          \hline
%          Turtayev et al.~\cite{turtayev2024hacking} & \purplecross & InterCode-CTF~\cite{yang2023language} & \purplecross & \purplecross & \purplecross & \purplecross & $4$ & $100$ \\
%          \hline
%          \textbf{D-CIPHER (Proposed)} & \tealcheck & NYU CTF~\cite{shao2024nyu}, Cybench \cite{zhang2024cybenchframeworkevaluatingcybersecurity}, HackTheBox \cite{hackthebox} & \tealcheck & \tealcheck & \tealcheck & \tealcheck & 5 & 290 \\
%          \hline
%     \end{tabular}
%     \caption{Comparison with LLM-based CTF solving Literature}
%     \label{tab:related_work_comparison}
% \end{table*}




% \subsection{Multi-agent framework}

% The use of multi-agent LLM systems in Capture the Flag (CTF) challenges is emerging as a powerful approach to enhance cybersecurity problem-solving. Multi-agent frameworks mimic team-based collaboration, where multiple LLM agents, each with specialized expertise, work together to tackle complex tasks. This approach reflects real-world cybersecurity operations, where success often depends on coordinated efforts from teams with diverse skills, each addressing different components of a security challenge.
% Multi-agent LLM systems are emerging as a powerful approach to enhance cybersecurity problem-solving by simulating team-based collaboration. Specialized agents, each with distinct objectives, work together to tackle different aspects of complex security tasks. This mirrors real-world cybersecurity operations, where coordinated efforts and diverse skills are essential for addressing evolving threats and vulnerabilities.

% CTF challenges cover a wide range of domains, including cryptography, reverse engineering, forensics, and web exploitation. Multi-agent systems can distribute the workload by assigning agents to handle specific tasks. This enables parallel problem-solving and emulates the collaborative nature of human teams. For example, one agent may specialize in guiding the fellow agents to what needs to be done, while another executes the instructions, ensuring that tasks are addressed without losing the context, and implementing reasoning from multiple LLMs. This division of labor boosts efficiency and enables problem-solving from multiple perspectives.
% This division of labor enhances efficiency and allows the system to approach problems from multiple perspectives, reflecting the interdisciplinary approach often used in cybersecurity teams.

% Guo et al.~\cite{guo2024largelanguagemodelbased} highlight the strengths of multi-agent LLMs in complex, multi-step tasks where different agents handle specific roles The framework HackSynth~\cite{muzsai2024hacksynth} is a multi-agent penetration testing framework in which agents operate collaboratively to address vulnerabilities in staged environments. Their work emphasizes that when agents work as a cohesive team, they outperform single-agent approaches. This is particularly true when facing layered, iterative challenges. 
% This team-based model of problem-solving aligns closely with how cybersecurity professionals approach real-world security incidents and penetration testing exercises.

% Multi-agent LLM systems have shown effectiveness in various other applications. For instance,  Audit-LLM~\cite{song2024audit} presents a multi-agent framework for insider threat detection using log analysis. It employs agents to decompose tasks, build tools, and use collaborative reasoning to enhance detection accuracy. Liu~\cite{liu2024multi} explores the application of LLM-based multi-agent systems to enhance incident response (IR) in cybersecurity. Utilizing the ``Backdoors \& Breaches" tabletop game as a simulation environment, the study examines centralized, decentralized, and hybrid team structures to assess how LLM agents can improve decision-making, adaptability, and coordination during cyberattack scenarios. AutoSafeCoder~\cite{nunez2024autosafecoder} is a multi-agent system designed to enhance the security of code generated by LLMs. The framework comprises three agents: a Coding Agent responsible for code generation, a Static Analyzer Agent that identifies vulnerabilities through static analysis, and a Fuzzing Agent that performs dynamic testing using mutation-based fuzzing to detect runtime errors. By integrating both static and dynamic testing in an iterative process, AutoSafeCoder aims to produce secure, functionally correct code. 

% To enhance CTF-solving by promoting team-based specialization, we employ a multi-agent CTF solving agent. Within this framework, agents tackle tasks aligned with their strengths. Tasks are executed in parallel, improving efficiency and accelerating progress. Agents share insights, adapt refining strategies based on feedback, and overcome obstacles collectively. This collaborative approach boosts scalability, adaptability, and and resilience, and improves performance in complex challenges.

% This paper presents a comprehensive comparison of D-CIPHER with existing LLM-based CTF-solving literature, as shown in Table~\ref{tab:related_work_comparison}.
% This paper documents the results of  our comprehensive comparison of D-CIPHER with existing LLM-based CTF-solving literature. These results are presented in Table~\ref{tab:related_work_comparison}.
\section{Methodology}
\label{sec:solution}

This paper presents \approach, a framework that improves the robustness of DL-based systems by generating realistic test images from existing datasets.
\approach leverages recent advances in text and visual models~\cite{DBLP:conf/cvpr/RombachBLEO22} to generate accurate synthetic images to test DL-based systems in scenarios that are not represented in the existing testing suite.

\begin{figure*}
    \centering
    \includegraphics[width=\linewidth]{images/dillema-schema.drawio.pdf}
    \caption{\approach.}
    \label{fig:dillema}
    \vspace{-6mm}
\end{figure*}

% Describe inputs of \approach
The proposed methodology, as shown in \autoref{fig:dillema}, consists of five steps.
The input of our approach is an image (from the existing test cases) along with a textual description of the task assigned to the DL-based system. The output is a modified version of the input image based on new conditions. 

% Image Captioning
\subsection{Image Captioning}

The first step of \approach involves image captioning, which is the process of converting a given image to its textual description. The objective is to enable the application of recent advances in natural language processing to images. To achieve this, \approach brings the images into the textual domain, where language models can operate effectively.

Captions are generated as multi-sentence descriptions to capture key elements and provide a detailed representation of the image. Each sentence focuses on a different aspect of the scene, capturing a range of elements such as objects, environments, and contextual relationships. This approach increases the likelihood of capturing important details that a single-sentence description might miss, providing a more comprehensive textual description for the subsequent steps.

\subsection{Keyword Identification}

Once the image is converted into textual descriptions through the captioning process, the next step in \approach is Keyword Identification. This step aims to identify which elements of the image can be safely modified without altering the overall meaning or the primary task (e.g., object classification, semantic segmentation) associated with the image. 

In this phase, the LLM is used to analyze the captions generated in the previous step and identify a set of keywords that can potentially be altered. These keywords represent modifiable aspects of the image, such as colors, weather conditions, or object properties, while excluding core elements that are essential to the task. For example, when dealing with an image classification task involving a ``car'', altering the background color or lighting usually does not modify the label. Conversely, in a semantic segmentation task focused on road scenes, the road and critical objects (cars, pedestrians, traffic signals) must remain present, though certain attributes (e.g., color, weather conditions) can still be changed. By defining the task explicitly in the prompt, we ensure that only permissible alterations are suggested by the LLM.
\autoref{fig:classification_and_segmentation} illustrates how the constraints differ between classification (\autoref{fig:classification}) and segmentation (\autoref{fig:semantic_segmentation}). In classification, the focus is on identifying and preserving the labeled object (\emph{car}), while in segmentation, multiple objects must remain for valid ground-truth labels.

\begin{figure}[]
\centering
    \begin{subfigure}{.49\linewidth}
    \centering
    \includegraphics[width=\textwidth]{images/segment.png}
    \caption{Classification Task.}
    \label{fig:classification}
    \end{subfigure}
    \begin{subfigure}{.49\linewidth}
    \centering
    \includegraphics[width=\textwidth]{images/annotated.png}
    \caption{Segmentation Task.}
    \label{fig:semantic_segmentation}
    \end{subfigure}
\caption{
Label Preservation in Autonomous Driving Tasks.
}
\label{fig:classification_and_segmentation}
\vspace{-6mm}
\end{figure}

The process of identifying these keywords is guided by task constraints. \approach prompts the LLM with a specific task-related query, such as:

\begin{tcolorbox}[arc=.3em,left=.3em,right=.3em,top=.3em,bottom=.3em]
\begin{center}
\begin{minipage}[t]{.99\linewidth}
\textbf{Prompt}: \textit{
Given the task $<$TASK$>$ and an image described by the caption $<$CAPTION$>$, what are the key elements that can be modified in the caption so that the ground truth corresponding to the image does not change?
}
\end{minipage}
\end{center}
\end{tcolorbox}

Note that this represents an example of the prompts used in \approach, intended to clarify the type of information that we request from the LLM. To improve the effectiveness of the prompt, various advanced strategies can be adopted. For example, as detailed in \autoref{sec:eval:setup}, we configured \approach to use a one-shot in-context learning prompting strategy, allowing the LLM to provide better results by including an example within the prompt.

The identification of keywords is designed to be flexible and adaptable for different tasks. The LLM relies on its internal knowledge to evaluate the contextual relevance of each word in the caption, taking into account both syntactic and semantic relationships. For example, if the task is semantic segmentation in an autonomous driving scenario, elements such as road conditions, lighting, or vehicle color may be identified as modifiable keywords, while objects essential to the task, such as vehicles themselves, remain unchanged.

\subsection{Alternative Identification}

In this phase, the LLM is leveraged to generate alternatives for the identified keywords, providing variations that can be applied to the image without altering the overall task.

The goal of this step is to explore different possibilities for modifying the elements flagged in the previous step, such as changing the color of objects, adjusting environmental conditions (e.g., weather), or altering minor details, while keeping the core structure and purpose of the image intact. For example, if the keyword ``foggy'' was identified as a modifiable attribute in the caption ``a car driving down a foggy street'', the LLM could suggest alternatives like ``rainy'' or ``snowy''.
To execute this, \approach generates a prompt asking the LLM to propose alternatives for the identified keywords. 

The main challenge in this phase is to introduce meaningful variations to the image while keeping its semantic meaning intact. The LLM plays a key role by generating alternatives that align with the original caption and task, avoiding changes that could shift the focus of the task. We take advantage of the ability of the LLM to understand contextual subtleties to avoid proposing changes to critical elements such as replacing ``car'' with ``bicycle'' in a vehicle detection scenario. An example of a prompt used in this phase is:

\begin{tcolorbox}[arc=.3em,left=.3em,right=.3em,top=.3em,bottom=.3em]
\begin{center}
\begin{minipage}[t]{.99\linewidth}
\textbf{Prompt}: \textit{
Given the task $<$TASK$>$ and an image described by the caption $<$CAPTION$>$, what are the possible alternatives for these keywords $<$KEYWORDS$>$?
}
\end{minipage}
\end{center}
\end{tcolorbox}

This process focuses on generating contextually relevant and diverse modifications, allowing the system to produce meaningful test cases for the DL model at hand. The alternatives proposed for each keyword enable \approach to explore different conditions or attributes of objects, broadening the range of scenarios included in the original dataset.

\subsection{Counterfactual Caption Generation}

This phase is responsible for creating new textual descriptions, or counterfactual captions, by applying the alternatives generated in the previous step. These counterfactual captions describe how the image would look if certain elements were modified, enabling the system to explore new scenarios while preserving the core context of the original image.

In this step, the LLM takes the original caption and replaces the identified keywords with the newly generated alternatives. The goal is to produce a new version of the caption that reflects the desired modifications without changing the essential meaning of the image. For example, if the original caption was ``a gray car driving down a foggy street'', and the alternatives generated for the keywords ``gray car'' and ``foggy'' were ``red car'' and ``snowy'', the new counterfactual caption would be ``A red car driving down a snowy street''.

The amount of edits in the new prompt can be controlled by limiting the number of alternatives applied when generating the counterfactual captions. For example, applying only one alternative at a time allows for small incremental changes, allowing exploration of subtle variations of the original caption. In contrast, applying multiple alternatives simultaneously can produce larger transformations, introducing more diverse scenarios. This approach provides fine-grained control over the extent of modifications, enabling tailored exploration of different levels of change in the generated test cases.

This phase is critically important because it ensures that the generated caption remains coherent and meaningful despite the modifications. Although replacing certain words (such as ``gray'' with ``red'') might seem straightforward, many cases are more complex, requiring careful handling to avoid breaking the sentence's meaning or introducing contradictions. For example, consider a caption like ``a road in a tundra covered in snow during a snowy day''. Replacement of the word ``tundra'' with ``desert'' would result in ``a road in a desert covered in snow during a snowy day'', which is contextually unlikely.

In this step, the LLM is prompted with the following input:

\begin{tcolorbox}[arc=.3em,left=.3em,right=.3em,top=.3em,bottom=.3em]
\begin{center}
\begin{minipage}[t]{.99\linewidth}
\textbf{Prompt}: \textit{
Given the task $<$TASK$>$, modify the caption $<$CAPTION$>$ by applying some of the following transformation described by $<$ALTERNATIVES$>$.
}
\end{minipage}
\end{center}
\end{tcolorbox}

By asking the LLM to generate the new caption directly, rather than applying simple replacement rules from the alternative dictionary, \approach ensures that the LLM processes not only the specific word replacements but also the broader sentence context, maintaining the overall meaning while making necessary adjustments to prevent contradictions or illogical outcomes.
Additionally, by explicitly including the task description at every step of the interaction, the LLM is continuously reminded of the objective it is trying to achieve. This ensures that the generated captions respect the metamorphic relationships inherent in the test case, preserving the critical connections between elements of the image and their semantic meaning.

\begin{figure*}
\centering
    \centering
    \includegraphics[width=.98\textwidth]{images/dillema_example.drawio.pdf}
\caption{Image generation in \approach.}
\label{fig:controlled_generation}
% \vspace{-5mm}
\end{figure*}
\subsection{Controlled Text-to-Image Generation}

The final step of \approach generates a modified image based on the counterfactual caption produced in the previous phase. This step is where the transformation of the image occurs, and it is carried out using a control-conditioned text-to-image diffusion model~\cite{DBLP:conf/iccv/ZhangRA23}.  The key challenge here is not only to generate a new, realistic image that aligns with the counterfactual caption but also to ensure that the spatial structure of the original image is preserved so that the integrity of metamorphic relationships is maintained.

When generating a new test image, the spatial arrangement of key objects and elements must be preserved. For example, in the context of semantic segmentation for autonomous driving, if an image depicts a car driving down a road, the generated image must include the car in the same location as the original image relative to the road, even if its color or weather conditions are changed. This way, the transformations to be applied will only affect specific attributes (e.g., altering weather or object properties) without impacting the fundamental geometry or layout of the scene. On the other hand, a distorted spatial structure could mislead the test results, making it unclear whether a failure is due to the actual shortcomings of the model or due to irrelevant transformations in the image.

To achieve spatial structure preservation, \approach uses control-conditioned diffusion models. These models allow fine-grained control over the generated image by incorporating conditioning inputs that preserve the spatial layout of the original image while applying the desired modifications.

\autoref{fig:controlled_generation} showcases examples of test cases generated by \approach for image classification (top row) and semantic segmentation (bottom row).
For image classification, the input image belongs to the class \textit{bird}, described by the captioning model as ``A yellow bird on a twig''. The second column displays the conditioning input extracted from the original image to preserve spatial arrangements. The remaining columns show images generated from alternative captions produced by the LLM: Caption A (``A blue bird on a twig'') changes the bird color to blue, while Caption B (``A red bird on a twig'') changes it to red. 
These augmentations demonstrate \approach ability to alter specific attributes while maintaining spatial structure and preserving the relevance of the class \textit{bird}. 

For semantic segmentation, the input image depicts a road with two cars during cloudy weather, with the ground truth represented as a semantic map of pixel-level classifications. The captioning model describes it as ``A road with two cars in cloudy weather''. The second column provides the conditioning input to ensure spatial consistency during generation. Caption A (``A road with two cars during snowy weather'') introduces snow to the scene, while Caption B (``A road with two cars during sunset'') applies sunset lighting. Both augmentations preserve the layout of roads, vehicles, and pedestrians as defined by the ground truth semantic map.
\section{Evaluation}
\label{sec:eval}

In this section, we evaluate the performance of \approach and aim to answer the following research questions (RQs):

\noindent\textbf{RQ\textsubscript{1} (Validity).} Can DILLEMA generate valid and realistic test cases from existing data?

\noindent\textbf{RQ\textsubscript{2} (Testing Effectiveness).} Can the generated test cases identify weaknesses in state-of-the-art DL models?

\noindent\textbf{RQ\textsubscript{3} (Retraining).} Can the generated test cases be used to improve the robustness of the tested models?

\subsection{Experimental Setup}
\label{sec:eval:setup}
\noindent\textbf{Datasets.} We performed experiments using two datasets: ImageNet1K~\cite{DBLP:conf/cvpr/DengDSLL009} and SHIFT~\cite{DBLP:conf/cvpr/SunSPWGSTY22}. These datasets represent two different tasks, image classification, and semantic segmentation, allowing us to assess the flexibility and applicability of \approach in various scenarios. ImageNet1K is a large-scale dataset commonly used for image classification tasks and SHIFT is a synthetic dataset designed for evaluating autonomous driving systems under different conditions (e.g., weather changes, lighting conditions).

\noindent\textbf{Tested Models.} We used \approach to test several DL architectures.
For ImageNet1K, we evaluated classification models (that is, ResNet18, ResNet50, and ResNet152~\cite{DBLP:conf/cvpr/HeZRS16}) using pre-trained versions provided by PyTorch. For SHIFT, we tested a semantic segmentation model (i.e., DeepLabV3~\cite{DBLP:conf/eccv/ChenZPSA18} model with a ResNet50 backbone), which we custom-trained following the original authors' training procedure~\cite{DBLP:conf/eccv/ChenZPSA18}. The training of this model took approximately $24$ hours to complete.

\noindent\textbf{Evaluation Metrics.} We used accuracy to evaluate the quality of classification models (on ImageNet1K), and we used mean Intersection over Union (mIoU) to measure the ability to evaluate the quality of semantic segmentation models.

\noindent\textbf{\approach Configuration\footnote{
To support reproducibility, all our data, including the code of \approach, the results of the human survey, of the testing and retraining, are available in our replication package: \url{https://github.com/deib-polimi/dillema}.}.} We used BLIP2 6.7B~\cite{DBLP:conf/icml/0008LSH23} as the captioning model to generate context-aware descriptions, chosen for its ability to produce detailed, semantically rich captions. As LLM, we selected a 5-bit quantized LLaMA-2 13B~\cite{DBLP:journals/corr/abs-2307-09288} model to identify keywords, generate alternatives, and create counterfactual captions. We chose LLaMA-2 because it is open source and effective, and we opted for the 13B version with 5-bit quantization since it provided a balance between performance and resource efficiency given our computational and cost constraints.
Lastly, for image generation, we used ControlNet~\cite{DBLP:conf/iccv/ZhangRA23} with edge conditioning, a control-conditioned text-to-image diffusion model. ControlNet enabled us to introduce modifications to the images while maintaining the spatial structure of the original scene, ensuring that the relationships between objects and their surroundings remained consistent.
Although we chose these general-purpose models for compatibility with consumer hardware and reasonable runtime, other models with different capabilities could be used depending on specific needs.

\noindent\textbf{Prompt Template.} To guide the LLM effectively, we used a one-shot in-context learning approach~\cite{DBLP:conf/nips/Wei0SBIXCLZ22}, where each prompt included an example to help the model understand the request more accurately. The example illustrated the expected input and output formats. Each prompt was constructed to provide context and explicitly instruct the LLM on the required output format, which allowed for automated post-processing. If the LLM response failed to adhere to the specified output format and could not be automatically parsed, we repeated the request with a different random seed. This iterative process continued until a parsable response was obtained.

\noindent\textbf{Retraining Settings.}
For ImageNet1K, we re-trained the ResNet models using a batch size of $100$ and the SGD optimizer with an initial learning rate of $0.1$, a momentum of $0.9$ and a weight decay of $1 \times 10^{-4}$. The learning rate was decayed using the PyTorch StepLR scheduler with a step size of $30$ and a gamma of $0.1$, over $90$ epochs.
For SHIFT, we re-trained the DeepLabV3 model using the original settings provided by its authors. Specifically, the batch size was set to $12$, with training conducted over $200$ epochs using the Adam optimizer with a learning rate of $0.002$, betas set to $(0.9, 0.999)$, and epsilon set to $1 \times 10^{-8}$.

\noindent\textbf{Hardware and Software.} The experiments were carried out on an AWS virtual machine with an A10G NVIDIA GPU with 24GB of memory. Neural networks were designed using PyTorch 2.0.1, and accelerated using CUDA 11.8.
In general, the empirical evaluation required about $120$ GPU hours.
$96$ GPU-hours were spent on Imagenet1K ($125,000$ test cases), $24$ GPU hours were spent on SHIFT ($10,000$ test cases).

\subsection{RQ\textsubscript{1}. Validity}
\label{sec:experiments:validity}
This experiment aims to evaluate the realism and validity of the generated images, ensuring that they preserve the metamorphic relationship for both datasets and assessing how often hallucinations occur due to potential errors during the five steps of \approach. By validating the generated images end-to-end, we aim to identify instances where the pipeline produces incorrect or unrealistic results.
To achieve this, we conducted a human study using Amazon Mechanical Turk. Human evaluators were asked to verify if the generated images preserved the metamorphic relationship for both datasets.

In total, we obtained $2,500$ total responses. To ensure quality, we used control questions to filter unreliable answers. Responses failing these quality checks were discarded. To ensure experienced participants, the workers were selected based on an approval rate greater than $95\%$ and at least $50$ completed tasks. Each image was evaluated by five independent workers and the questions were discarded if consensus (agreement of at least $\frac{4}{5}$ participants) was not reached. In the end, only $2,380$ responses (out of $2,500$) were considered robust and good enough to answer the research question.


For ImageNet1K (\autoref{fig:rq1_validity_imagenet1k}), we used two types of questions and considered a transformation to be valid if our approach were able to correctly augment an existing image without modifying the label associated with it.
First, we performed a general evaluation on a randomly sampled set of $300$ augmented images from all generated cases to measure the overall validity.
Then, we proposed a focused evaluation of $100$ augmented images that the ResNet18 model misclassified, to check if the images were valid and interpretable by humans even when misclassified by the model.

\begin{figure}[h]
    \centering
    \includegraphics[width=.45\textwidth]{images/ImageNet-bar.pdf}
    \caption{Validity of the Generated Test Cases for Classification.}
    \label{fig:rq1_validity_imagenet1k}
\end{figure}

Our human study shows that human assessors achieved agreement on all images and $99.7\%$ of the augmented images were correctly classified by human assessors. Of the $300$ images, only $1$ image did not preserve the label associated with the original image.
For the set of images where the model (i.e., ResNet18) produced a misclassification, $82.7\%$ were still considered valid by human evaluators. This shows that while the test cases generated by \approach effectively induced misclassifications in the model, most of them could still be correctly classified by humans. This suggests that failures can often be attributed to bugs in the model rather than flaws in the image generation process, reinforcing the validity and utility of \approach for robust model testing.

For the SHIFT dataset (\autoref{fig:rq1_validity_shift}), we randomly selected $100$ augmented images. Among these, all depicted roads, $25$ included vehicles, and $15$ featured one or more pedestrians. Evaluators were tasked with verifying whether key elements critical for autonomous driving, such as roads, vehicles, and pedestrians, were consistently preserved through the transformations.  We checked these aspects since they are key elements that influence the behavior of an autonomous driving system.

\begin{figure}[h]
    \centering
    \includegraphics[width=.45\textwidth]{images/SHIFT-bar.pdf}
    \caption{Validity of the Generated Test Cases for Driving.}
    \label{fig:rq1_validity_shift}
\end{figure}

We observed the following validity rates: road preservation at $98.9\%$ ($100$ questions, $7$ were discarded due to lack of consensus), pedestrian preservation at $84.6\%$ ($15$ questions, $2$ discarded due to lack of consensus), and vehicle preservation at $100.0\%$ ($25$ questions, $1$ discarded due to lack of consensus). These results highlight that \approach can effectively maintain certain features, such as roads and vehicles, while being slightly less effective at preserving pedestrians.

\subsection{RQ\textsubscript{2}. Testing Effectiveness}

\begin{figure*}
\centering
    \begin{subfigure}[b]{.49\textwidth}
    \centering
\includegraphics[width=\textwidth]{images/conf_mat_shift.pdf}
    \caption{Accuracy on Original Test Suite.\label{fig:image1}}
    \end{subfigure}
    \begin{subfigure}[b]{.49\textwidth}
    \centering
    \includegraphics[width=\textwidth]{images/conf_mat_shift_aug.pdf}
    \caption{Accuracy on \approach Augmented Test Suite.\label{fig:image2}}
    \end{subfigure}
\caption{Multi-class Confusion Matrix.}
\label{fig:rq2_effectiveness}
% \vspace{-3mm}
\end{figure*}

To evaluate the effectiveness of \approach, we evaluated its ability to detect weaknesses in state-of-the-art DL models using the generated test cases.

First, we performed experiments on ImageNet1K, focusing on identifying misclassification errors. For this purpose, we augmented $25$ images for each of the $1,000$ classes in the dataset. Each image was augmented five times to take advantage of the stochastic nature of diffusion models, which can generate different augmentations from the same input. The performance of the test suite generated by \approach was compared with the test set already available in the dataset.

\begin{table}[h]
    \footnotesize
    \centering
    \begin{tabular}{@{}lcc@{}}
        \toprule
        \xspace\space\xspace\space  \textbf{Architecture}\xspace\space\xspace\space               & \xspace\space\xspace\space\xspace\space\xspace\space\textbf{Original Test Suite}\xspace\space\xspace\space\xspace\space\xspace\space                      & \xspace\space\xspace\space\textbf{\approach Test Suite}\xspace\space\xspace\space \\ \midrule
        \xspace\space\xspace\space  ResNet18\xspace\space\xspace\space & 5.26\% & 53.29\% \\
        \xspace\space\xspace\space  ResNet50\xspace\space\xspace\space & 2.55\% & 45.47\% \\
        \xspace\space\xspace\space  ResNet152\xspace\space\xspace\space & 1.47\% & 42.33\% \\ \bottomrule
        \end{tabular}
    \caption{Test Effectiveness.}
    \label{tab:rq2_effectiveness}
\end{table}

\autoref{tab:rq2_effectiveness} reports the performance of three ResNet variants in both test suites. The results reveal that, on average, $3.1\%$ of the original test suite was able to highlight misbehaviors, while $47.0\%$
of the test suite generated by \approach exposed faulty behaviors. However, it is important to note that, as discussed in \autoref{sec:experiments:validity}, not all of these detected misbehaviors may represent true failures. The human study confirmed that approximately $82.7\%$ of the misbehaviors detected by \approach were valid failures. Even after normalizing for this factor, the effectiveness of \approach remains significantly higher ($38.9\%$) than the original test set.

In addition, we analyzed how many augmentations per image led to model errors. Our findings indicate that for $33.29\%$ of the images, all augmentations resulted in misclassifications, whereas for $24.85\%$, none of the augmentations caused errors.

For the SHIFT dataset, we evaluated the DeepLabV3 model on the semantic segmentation task. The evaluation compared the augmented test set created by \approach with the original SHIFT test set. \autoref{fig:rq2_effectiveness} presents the normalized multi-class confusion matrix of the tested model on the original and augmented data. Rows represent the ground truth, columns represent the predicted class, and the diagonal indicates the percentage of correct predictions.

The results show that \approach successfully exposed interesting faulty behaviors. For example, in semantic classes where the model appeared robust in the original dataset, such as \textit{SideWalk} ($97\%$ correctly classified), the model showed significant vulnerability in the augmented dataset (only $38\%$). In more critical classes such as \textit{Road} and \textit{Vehicle}, we observed that the model maintained a relatively robust performance, with errors increasing by $9\%$ and $10\%$, respectively, as the accuracy decreased from $99\%$ and $97\%$ in the original dataset to $90\%$ and $87\%$ in the augmented dataset. However, for pedestrian recognition, the augmented dataset revealed a much higher vulnerability, with $34\%$ more misclassifications compared to the original dataset. This highlights the need to retrain the model with a stronger focus on identifying pedestrians to address this critical weakness.

These results highlight that \approach not only highlights hidden vulnerabilities in classes previously considered robust but also provides insights into critical performance degradations in safety-relevant semantic classes. In general, \approach effectively exposes model weaknesses in various scenarios.

\subsection{RQ\textsubscript{3}. Retraining Robustness}

To assess whether the test cases generated by \approach can improve the robustness, we conducted retraining experiments using the synthetically generated data. Retraining aimed to evaluate whether the incorporation of augmented test cases into the training process leads to improved performance on both original and augmented data.

For the ImageNet1K dataset, we retrained the ResNet18 model using a combined training set consisting of the original data and the augmented test cases generated by \approach. The model was re-trained for 90 epochs using the settings described in \textit{Retraining Settings}. The re-trained model showed a significant improvement in robustness, achieving a $52.27\%$ increase in accuracy in the augmented test cases and a $20.19\%$ improvement in the original test suite.

Concerning SHIFT, we achieved an improvement in mIoU across the original and augmented test sets. After retraining, mIoU in the original test suite improved from $85.32\%$ to $88.76\%$, while mIoU in the augmented dataset showed a more pronounced increase from $72.45\%$ to $80.32\%$.
Specifically, the retraining process revealed that while performance degradation on critical semantic classes like \textit{Road} and \textit{Vehicles} was minor, pedestrian recognition showed a significant recovery, increasing from $38\%$ to $62\%$. This improvement highlights the value of \approach in augmenting datasets to address vulnerabilities in safety-critical tasks.

These findings demonstrate that the generated test cases are highly effective in not only uncovering model vulnerabilities but also improving the robustness of DL models when incorporated into the retraining process.

\subsection{Threats to Validity}
\noindent\textbf{Internal Validity.}
Our pipeline relies on pre-trained models (captioning, LLM, diffusion) and random sampling of alternatives, which can introduce randomness and potential skew (e.g., consistently generating “red” vehicles). Another concern is the domain shift between real images and our synthesized outputs: models might perform worse simply because of unfamiliar synthetic characteristics rather than true weaknesses. However, our human study indicates that the vast majority of generated images retain labels recognizable to human evaluators, suggesting that they are semantically coherent rather than purely artificial or misleading. Thus, while some failures could stem from synthetic artifacts, the high human agreement on these images implies that many observed misclassifications reflect genuine model vulnerabilities rather than artifacts alone.

\noindent\textbf{External Validity.}
We tested \approach on classification and segmentation from distinct domains, but it may not generalize to specialized scenarios (e.g., medical imaging). Although each component (captioning, LLM, diffusion) seems broadly applicable, further testing on diverse datasets is required to confirm adaptability for industrial use and other vision tasks.

\noindent\textbf{Construct Validity.}
Our primary measure of success is whether the generated images preserve ground-truth labels and uncover vulnerabilities. While human assessments indicate that images remain valid, potential biases in LLM-generated alternatives (e.g., color choices) could distort conclusions. Additionally, the notion of validity is subjective; thus, future work should employ more rigorous metrics or automated checks to validate semantic consistency in generated test cases.
\section{Related Research Fields}

\paragraph{LLM-powered Recommender Systems} In recent years, recommender systems based on Large Language Models (LLMs) have attracted widespread attention. 
Such systems make full use of the powerful language understanding and generation capabilities of LLMs, bringing a new paradigm to traditional recommender systems.
Most existing methods are primarily designed for rating prediction~\cite{bao2023tallrec} and sequential recommendation~\cite{hou2024large,shao2024ulmrec,zheng2024adapting}.
CoLLM~\cite{zhang2023collm} captures and maps the collaborative information through external traditional models, forming collaborative embeddings used by LLMs. 
LlamaRec~\cite{llamarec} fine-tunes Llama-2-7b for list-wise ranking of the pre-selected items.
However, these methods would face significant limitations: the inability to simulate authentic user behaviors for enhanced personalization, the lack of effective memory mechanisms for long-term context awareness, and the rigid pipeline structure that prevents flexible task decomposition and seamless integration with external tools.

\paragraph{Conversational Recommender Systems}

Conversational recommender systems (CRS) have emerged as a significant research direction in recent years~\cite{jannach2021survey}, which are similar to the LLM-powered agent recommender systems. 
However, traditional methods~\cite{lei2020interactive} have two main drawbacks: attribute-based approaches are limited by rigid dialogue patterns, while generation-based methods suffer from restricted knowledge and poor generalization capabilities of small language models.


\section{Future Directions}

    %\item \textbf{Enhancement of Agent Capabilities}: Current LLM-powered recommendation agents exhibit limitations in behavioral understanding and long-term memory mechanisms. Future research should focus on advancing user intent comprehension, improving knowledge management systems, and enhancing task planning and decision-making capabilities to better accommodate dynamic user preferences.
\paragraph{Optimization of System Architecture} The integration between traditional recommendation methods and LLMs remains insufficient, with challenges in multi-agent collaboration and system interpretability. Future developments should explore flexible architectural designs, enhance agent cooperation efficiency, while ensuring transparency in recommendation.

\paragraph{Refinement of Evaluation Framework} There is a notable absence of unified and comprehensive evaluation standards for accurately measuring dialogue quality and recommendation effectiveness. Future research necessitates the establishment of robust evaluation frameworks, development of novel performance metrics, and consideration of privacy and security concerns in practical applications.

\paragraph{Security Recommender System} \cite{ning2024cheatagent} reveals the vulnerability of LLM-empowered recommender systems to adversarial attacks. In future,  the researchers could develop robust adversarial detection methods, investigate multi-agent defensive architectures, and integrating domain-specific security knowledge into defense mechanisms.
\section{Conclusion}
We introduced \methodname, an effective training framework defending against MIAs for LLMs. The extensive experiments demonstrate its robustness in protecting privacy while maintaining strong language modeling performance across various datasets and architectures. Although our study focuses on fine-tuning due to computational constraints, \methodname can be seamlessly applied to large-scale pretraining, as done in prior selective pretraining work~\cite{lin2024not}. By categorizing tokens and treating them appropriately, \methodname opens a novel pathway for MIA defense. Future work can explore improved token selection strategies and multi-objective training approaches.

% \section{Data Availability} \label{sec:avail}
% To support reproducibility, all our data, including the code of \approach, the results of the human survey, of the testing and retraining, are available in our replication package\cite{dillema_replication_package}.
\bibliographystyle{IEEEtran}  
\balance
\bibliography{bibl}  

\end{document}
