\documentclass[10pt,conference]{IEEEtran} 
\IEEEoverridecommandlockouts
% The preceding line is only needed to identify funding in the first footnote. If that is unneeded, please comment it out.
\usepackage{cite}
\usepackage{xspace} 
\usepackage{amsmath,amssymb,amsfonts}
\usepackage{algorithmic}
\usepackage{graphicx}
\usepackage{textcomp}
\usepackage{xcolor}
\usepackage{stfloats}
\usepackage{balance}
\usepackage{hyperref}
\usepackage{soul}
\usepackage{enumitem}
\usepackage{booktabs}
\usepackage{caption}
\usepackage{siunitx}
\usepackage{tabularx}
\usepackage{subcaption}
\usepackage{tcolorbox}
% \usepackage{showframe}
\usepackage{array}
\usepackage[retain-zero-uncertainty=true]{siunitx}
\newcommand\approach{DILLEMA\xspace}

% \pdfobjcompresslevel=0 
% \pdfminorversion=5

% \newcommand\approach{Approach\xspace}
% \newcommand\instruct{Instruction-editing\xspace}
% \newcommand\inpaint{Inpainting\xspace}
% \newcommand\refining{Inpainting with Refinement\xspace}
% \newcommand\ours{Knowledge Distillation\xspace}

% \newcommand{\davetwo}{\mbox{DAVE-2}\xspace}
% \newcommand{\head}[1]{\noindent\textbf{#1.}}

\renewcommand*{\figureautorefname}{Figure}
\renewcommand*{\sectionautorefname}{Section}
\renewcommand*{\subsectionautorefname}{Section}
\renewcommand*{\subsubsectionautorefname}{Section}

\newcommand{\luc}[1]{\textcolor{red}{\textbf{[LB] #1}}}
\newcommand{\dav}[1]{\textcolor{cyan}{\textbf{[DH] #1}}}
\newcommand{\irf}[1]{\textcolor{green}{\textbf{[IM] #1}}}
\newcommand{\gio}[1]{\textcolor{orange}{\textbf{[GQ] #1}}}

\newcommand{\orange}[1]{\textcolor{orange}{#1}}
\newcommand{\camera}[1]{\textcolor{violet}{#1}}
% \newcommand{\COMMENT}[1]{}

\def\BibTeX{{\rm B\kern-.05em{\sc i\kern-.025em b}\kern-.08em
    T\kern-.1667em\lower.7ex\hbox{E}\kern-.125emX}}


% \newlist{questions}{enumerate}{2}
% \setlist[questions,1]{label*=\textbf{RQ\arabic*},ref=RQ\arabic*}

\begin{document}

\title{
DILLEMA: Diffusion and Large Language Models for Multi-Modal Augmentation\\
\thanks{This research was partially supported by project EMELIOT, funded by MUR under the PRIN 2020 program (Contract 2020W3A5FY).
}
}
% 
% Proposals:
% Domain Augmentation Strategies for Autonomous Driving System-Level Testing
% Leveraging Diffusion Models for Domain Augmentation in System-Level Testing of Autonomous Driving Systems
% Efficient Domain Augmentation for Autonomous Driving Testing Using Diffusion Models

\author{
\IEEEauthorblockN{
  Luciano Baresi,
  Davide Yi Xian Hu,
  Muhammad Irfan Mas'udi,
  Giovanni Quattrocchi
}
\IEEEauthorblockA{
  \textit{Dipartimento di Elettronica, Informazione e Bioingegneria, Politecnico di Milano},
  Milan, Italy \\
  luciano.baresi@polimi.it, davideyi.hu@polimi.it, muhammadirfan.masudi@mail.polimi.it, giovanni.quattrocchi@polimi.it
}
}

\maketitle

% force page numbers
\thispagestyle{plain}
\pagestyle{plain}

% Peer review
\IEEEpeerreviewmaketitle 

\begin{abstract}
Large language model (LLM)-based agents have shown promise in tackling complex tasks by interacting dynamically with the environment. 
Existing work primarily focuses on behavior cloning from expert demonstrations and preference learning through exploratory trajectory sampling. However, these methods often struggle in long-horizon tasks, where suboptimal actions accumulate step by step, causing agents to deviate from correct task trajectories.
To address this, we highlight the importance of \textit{timely calibration} and the need to automatically construct calibration trajectories for training agents. We propose \textbf{S}tep-Level \textbf{T}raj\textbf{e}ctory \textbf{Ca}libration (\textbf{\model}), a novel framework for LLM agent learning. 
Specifically, \model identifies suboptimal actions through a step-level reward comparison during exploration. It constructs calibrated trajectories using LLM-driven reflection, enabling agents to learn from improved decision-making processes. These calibrated trajectories, together with successful trajectory data, are utilized for reinforced training.
Extensive experiments demonstrate that \model significantly outperforms existing methods. Further analysis highlights that step-level calibration enables agents to complete tasks with greater robustness. 
Our code and data are available at \url{https://github.com/WangHanLinHenry/STeCa}.
\end{abstract}

\begin{IEEEkeywords}
autonomous driving systems, deep learning testing, diffusion models, large language models, generative AI
\end{IEEEkeywords}

\section{Introduction}

Despite the remarkable capabilities of large language models (LLMs)~\cite{DBLP:conf/emnlp/QinZ0CYY23,DBLP:journals/corr/abs-2307-09288}, they often inevitably exhibit hallucinations due to incorrect or outdated knowledge embedded in their parameters~\cite{DBLP:journals/corr/abs-2309-01219, DBLP:journals/corr/abs-2302-12813, DBLP:journals/csur/JiLFYSXIBMF23}.
Given the significant time and expense required to retrain LLMs, there has been growing interest in \emph{model editing} (a.k.a., \emph{knowledge editing})~\cite{DBLP:conf/iclr/SinitsinPPPB20, DBLP:journals/corr/abs-2012-00363, DBLP:conf/acl/DaiDHSCW22, DBLP:conf/icml/MitchellLBMF22, DBLP:conf/nips/MengBAB22, DBLP:conf/iclr/MengSABB23, DBLP:conf/emnlp/YaoWT0LDC023, DBLP:conf/emnlp/ZhongWMPC23, DBLP:conf/icml/MaL0G24, DBLP:journals/corr/abs-2401-04700}, 
which aims to update the knowledge of LLMs cost-effectively.
Some existing methods of model editing achieve this by modifying model parameters, which can be generally divided into two categories~\cite{DBLP:journals/corr/abs-2308-07269, DBLP:conf/emnlp/YaoWT0LDC023}.
Specifically, one type is based on \emph{Meta-Learning}~\cite{DBLP:conf/emnlp/CaoAT21, DBLP:conf/acl/DaiDHSCW22}, while the other is based on \emph{Locate-then-Edit}~\cite{DBLP:conf/acl/DaiDHSCW22, DBLP:conf/nips/MengBAB22, DBLP:conf/iclr/MengSABB23}. This paper primarily focuses on the latter.

\begin{figure}[t]
  \centering
  \includegraphics[width=0.48\textwidth]{figures/demonstration.pdf}
  \vspace{-4mm}
  \caption{(a) Comparison of regular model editing and EAC. EAC compresses the editing information into the dimensions where the editing anchors are located. Here, we utilize the gradients generated during training and the magnitude of the updated knowledge vector to identify anchors. (b) Comparison of general downstream task performance before editing, after regular editing, and after constrained editing by EAC.}
  \vspace{-3mm}
  \label{demo}
\end{figure}

\emph{Sequential} model editing~\cite{DBLP:conf/emnlp/YaoWT0LDC023} can expedite the continual learning of LLMs where a series of consecutive edits are conducted.
This is very important in real-world scenarios because new knowledge continually appears, requiring the model to retain previous knowledge while conducting new edits. 
Some studies have experimentally revealed that in sequential editing, existing methods lead to a decrease in the general abilities of the model across downstream tasks~\cite{DBLP:journals/corr/abs-2401-04700, DBLP:conf/acl/GuptaRA24, DBLP:conf/acl/Yang0MLYC24, DBLP:conf/acl/HuC00024}. 
Besides, \citet{ma2024perturbation} have performed a theoretical analysis to elucidate the bottleneck of the general abilities during sequential editing.
However, previous work has not introduced an effective method that maintains editing performance while preserving general abilities in sequential editing.
This impacts model scalability and presents major challenges for continuous learning in LLMs.

In this paper, a statistical analysis is first conducted to help understand how the model is affected during sequential editing using two popular editing methods, including ROME~\cite{DBLP:conf/nips/MengBAB22} and MEMIT~\cite{DBLP:conf/iclr/MengSABB23}.
Matrix norms, particularly the L1 norm, have been shown to be effective indicators of matrix properties such as sparsity, stability, and conditioning, as evidenced by several theoretical works~\cite{kahan2013tutorial}. In our analysis of matrix norms, we observe significant deviations in the parameter matrix after sequential editing.
Besides, the semantic differences between the facts before and after editing are also visualized, and we find that the differences become larger as the deviation of the parameter matrix after editing increases.
Therefore, we assume that each edit during sequential editing not only updates the editing fact as expected but also unintentionally introduces non-trivial noise that can cause the edited model to deviate from its original semantics space.
Furthermore, the accumulation of non-trivial noise can amplify the negative impact on the general abilities of LLMs.

Inspired by these findings, a framework termed \textbf{E}diting \textbf{A}nchor \textbf{C}ompression (EAC) is proposed to constrain the deviation of the parameter matrix during sequential editing by reducing the norm of the update matrix at each step. 
As shown in Figure~\ref{demo}, EAC first selects a subset of dimension with a high product of gradient and magnitude values, namely editing anchors, that are considered crucial for encoding the new relation through a weighted gradient saliency map.
Retraining is then performed on the dimensions where these important editing anchors are located, effectively compressing the editing information.
By compressing information only in certain dimensions and leaving other dimensions unmodified, the deviation of the parameter matrix after editing is constrained. 
To further regulate changes in the L1 norm of the edited matrix to constrain the deviation, we incorporate a scored elastic net ~\cite{zou2005regularization} into the retraining process, optimizing the previously selected editing anchors.

To validate the effectiveness of the proposed EAC, experiments of applying EAC to \textbf{two popular editing methods} including ROME and MEMIT are conducted.
In addition, \textbf{three LLMs of varying sizes} including GPT2-XL~\cite{radford2019language}, LLaMA-3 (8B)~\cite{llama3} and LLaMA-2 (13B)~\cite{DBLP:journals/corr/abs-2307-09288} and \textbf{four representative tasks} including 
natural language inference~\cite{DBLP:conf/mlcw/DaganGM05}, 
summarization~\cite{gliwa-etal-2019-samsum},
open-domain question-answering~\cite{DBLP:journals/tacl/KwiatkowskiPRCP19},  
and sentiment analysis~\cite{DBLP:conf/emnlp/SocherPWCMNP13} are selected to extensively demonstrate the impact of model editing on the general abilities of LLMs. 
Experimental results demonstrate that in sequential editing, EAC can effectively preserve over 70\% of the general abilities of the model across downstream tasks and better retain the edited knowledge.

In summary, our contributions to this paper are three-fold:
(1) This paper statistically elucidates how deviations in the parameter matrix after editing are responsible for the decreased general abilities of the model across downstream tasks after sequential editing.
(2) A framework termed EAC is proposed, which ultimately aims to constrain the deviation of the parameter matrix after editing by compressing the editing information into editing anchors. 
(3) It is discovered that on models like GPT2-XL and LLaMA-3 (8B), EAC significantly preserves over 70\% of the general abilities across downstream tasks and retains the edited knowledge better.
% \section{Preliminaries}
\label{sec:preli}
In this paper, we focus on undirected and unweighted simple {\cheng graphs} without self-loops and parallel edges. {\cheng For ease of presentation, we focus on  graphs without vertex labels, but} 
% We note that 
our methods \eat{{\cheng to be introduced}} can be easily adapted to vertex-labeled graphs. Consider two graphs $Q=(V_Q,E_Q)$ and $G=(V_G,E_G)$\eat{, with vertex sets $V_Q$ and $V_G$ and edge sets $E_Q$ and $E_G$}. For simplicity, we let $u$ and $v$ (and their primed or index variants) denote a vertex in $Q$ and $G$ respectively. Given a vertex set $X\subseteq V_Q$, we use $Q[X]$ to denote the subgraph of $Q$ induced by $X$, i.e., $Q[X]=(X,\{(u,u')\in E_Q \mid u, u'\in X\})$. All subgraphs in this paper are induced subgraphs. We let $q = (V_q, E_q)$ denote an arbitrary induced subgraph of $Q$. 
%
Given $u\in V_Q$, we denote by $N(u,V_Q)$ (resp. $\overline{N}(u,V_Q)$) the set of neighbours (resp. non-neighbours) of $u$ in $Q[V_Q]$.  \eat{We have the symmetric definitions for a vertex set $Y$ and each vertex $v$ in $G$.} \laks{We use a similar notation for neighbours and non-neighbours of vertices in $G$.}  

We  review the definition of graph isomorphism for  simple graphs without labels.

\begin{definition}[Graph isomorphism~\cite{mcgregor1982backtrack}]
    \label{def:GS}
    $Q$ is said to be isomorphic to $G$ if and only if there exists a \textbf{bijection} $\phi: V_Q\rightarrow V_G$ such that
    \begin{equation}
    \label{eq:isomorphic}
        \forall u, u' \in V_Q: (u,u')\in E_Q \iff (\phi(u),\phi(u'))\in E_G.
    \end{equation}
\end{definition}

\laks{Note that} two isomorphic graphs are structurally equivalent, {\chengC and thus we have} $|V_Q|=|V_G|$ and $|E_Q|=|E_G|$. We next review induced subgraph isomorphism for  graphs. 

\begin{definition}[Induced subgraph isomorphism~\cite{mcgregor1982backtrack}]
    $Q$ is said to be (induced) subgraph isomorphic to $G$ if and only if there exists an \textbf{injection} $\phi: V_Q\rightarrow V_G$ such that
    \begin{equation}
    %\label{eq:isomorphic}
        \forall u, u' \in V_Q: (u,u')\in E_Q {\YuiR \iff} (\phi(u),\phi(u'))\in E_G. %\Longrightarrow 
    \end{equation}
    %that satisfies Equation~(\ref{eq:isomorphic}).
\end{definition}

Notice that  induced subgraph isomorphism is a special case of  graph isomorphism. The injection mapping $\phi: V_Q\rightarrow V_G$ is also known as \emph{embedding} of $Q$ into $G$, {\chengC and thus we have} $|V_Q|\leq |V_G|$ and $|E_Q|\leq |E_G|$. The subgraph matching problem aims to find all embeddings of a small query graph $Q$ in a large data graph $G$. The common induced subgraph is defined as follows. 

\begin{definition}[Common induced subgraph~\cite{mcgregor1982backtrack}]
    \label{def:CIS}
    A common subgraph of $Q$ and $G$, {\cheng denoted by $\langle q,g,\phi \rangle$, is defined as} a \eat{set of vertex pairs} \laks{triple   consisting of an induced subgraph $q$ of $Q$, an induced subgraph $g$ of $G$, and a bijection $\phi: V_q\rightarrow V_g$, such that $q$ is isomorphic to $g$ under   $\phi$.} %{\cheng Formally, we have}
    %\begin{equation}
    %    \langle q,g,\phi \rangle :=\{\langle u,\phi(u) \rangle \mid \forall u\in V_q\}.
   % \end{equation}
\end{definition}

%A common subgraph $\langle q,g,\phi \rangle$ is said to be contained in (or a subgraph of) another common subgraph $\langle q',g',\phi' \rangle$ if and only if $\{\langle u,\phi(u) \rangle \mid \forall u\in V_q\}\subseteq \{\langle u,\phi'(u) \rangle \mid \forall u\in V_{q'}\}$.
By the size of a common subgraph $\langle q, g,\phi\rangle$ we mean the number of {\YuiR vertices in $q$ or $g$}. %vertex pairs. 
%
Clearly, the size of a common subgraph is at most $\min\{|V_Q|,|V_G|\}$. 
%
{\YuiR Sometimes, for  ease of presentation, we represent a common subgraph $\langle q,g,\phi \rangle$ by a set of vertex pairs $\{\langle u,\phi(u) \rangle \mid  u\in V_q\}$.}

\begin{example}
Consider the input graphs in Figure~\ref{fig:Input_graph}. The graphs $q := Q[u_1,u_2,u_3,u_4,u_7]$ and  $g := G[v_3,v_4,v_5,v_6,v_7]$ form a common subgraph with size 5 under the bijection $\phi:=\{u_1\!\rightarrow\! v_5,u_2\!\rightarrow\! v_6, u_3\!\rightarrow\! v_4, u_4\!\rightarrow\! v_3, u_7\!\rightarrow\! v_7\}$.%, which can be represented by a set of vertex pairs, i.e., $\{\langle u_1,v_5 \rangle,\langle u_2,v_6 \rangle,\langle u_3,v_4\rangle,\langle u_4,v_3\rangle,\langle u_7,v_7\rangle\}$.
\end{example}
%
We are ready to formulate the problem studied in this paper.


\begin{problem}[Maximum Common Subgraph~\cite{lewis1983michael}]
    Given two graphs $Q$ and $G$, the Maximum Common Subgraph (MCS) problem aims to find the maximum common subgraph {\cheng of $Q$ and $G$}, i.e., a common subgraph  with the largest number of vertices.
\end{problem}

%We use MaxCS as a shorthand of the maximum common subgraph thoughout this paper.
\eat{If {\chengB we further require} the size of found maximum common subgraph {\cheng to be} at least $|V_Q|$, then the MCS problem would reduce to  subgraph matching (i.e., {\cheng it finds} one embedding of $Q$ in $G$). Therefore, the MCS problem is a natural generalization of the subgraph matching problem. } 
\laks{Note that the MCS problem is a generalization of subgraph matching: there is a MCS between $Q$ and $G$ whose size is $|Q|$ iff $Q$ is isomorphic to a subgraph of $G$. It is well known that  MCS  is NP-hard~\cite{lewis1983michael} and is hard to approximate, i.e., {\YuiR there is no $r$-approximate PTIME algorithm for the problem for any $r\geq 1$~\cite{kann1992approximability}, \laks{unless P=NP}.} %$O(n^{\epsilon})$-approximate PTIME algorithm for the problem for any $\epsilon>0$, where $n$ is the input instance size~\cite{kann1992approximability}. 
} 

\eat{ 
\smallskip
\noindent\textbf{Hardness.} We remark that the problem of finding the maximum common subgraph is NP-hard~\cite{lewis1983michael}. Besides, 
% it is shown to be 
{\cheng the problem is}
hard to approximate, {\cheng e.g.,} it does not admit any $O(n^{\epsilon})$-approximate algorithm that runs in polynomial time (unless P=NP), where $\epsilon>0$ and $n$ is the size of an input instance~\cite{kann1992approximability}.  
} 

\begin{figure}[]
		\subfigure[\textsf{Input graph $Q$}]{
			\includegraphics[width=2.7cm]{figure/Input_graph_Q.pdf}
		}
        \hspace{0.4in}
		\subfigure[\textsf{Input graph $G$}]{
			\includegraphics[width=2.7cm]{figure/Input_graph_G.pdf}
		}
  \vspace{-0.2in}
	\caption{Input graphs used throughout the paper}
	\label{fig:Input_graph}
 \vspace{-0.2in}
\end{figure}
% \section{Background}
\label{sec:background}


\subsection{Code Review Automation}
Code review is a widely adopted practice among software developers where a reviewer examines changes submitted in a pull request \cite{hong2022commentfinder, ben2024improving, siow2020core}. If the pull request is not approved, the reviewer must describe the issues or improvements required, providing constructive feedback and identifying potential issues. This step involves review commment generation, which play a key role in the review process by generating review comments for a given code difference. These comments can be descriptive, offering detailed explanations of the issues, or actionable, suggesting specific solutions to address the problems identified \cite{ben2024improving}.


Various approaches have been explored to automate the code review comments process  \cite{tufano2023automating, tufano2024code, yang2024survey}. 
Early efforts centered on knowledge-based systems, which are designed to detect common issues in code. Although these traditional tools provide some support to programmers, they often fall short in addressing complex scenarios encountered during code reviews \cite{dehaerne2022code}. More recently, with advancements in deep learning, researchers have shifted their focus toward using large-language models to enhance the effectiveness of code issue detection and code review comment generation.

\subsection{Knowledge-based Code Review Comments Automation}

Knowledge-based systems (KBS) are software applications designed to emulate human expertise in specific domains by using a collection of rules, logic, and expert knowledge. KBS often consist of facts, rules, an explanation facility, and knowledge acquisition. In the context of software development, these systems are used to analyze the source code, identifying issues such as coding standard violations, bugs, and inefficiencies~\cite{singh2017evaluating, delaitre2015evaluating, ayewah2008using, habchi2018adopting}. By applying a vast set of predefined rules and best practices, they provide automated feedback and recommendations to developers. Tools such as FindBugs \cite{findBugs}, PMD \cite{pmd}, Checkstyle \cite{checkstyle}, and SonarQube \cite{sonarqube} are prominent examples of knowledge-based systems in code analysis, often referred to as static analyzers. These tools have been utilized since the early 1960s, initially to optimize compiler operations, and have since expanded to include debugging tools and software development frameworks \cite{stefanovic2020static, beller2016analyzing}.



\subsection{LLMs-based Code Review Comments Automation}
As the field of machine learning in software engineering evolves, researchers are increasingly leveraging machine learning (ML) and deep learning (DL) techniques to automate the generation of review comments \cite{li2022automating, tufano2022using, balachandran2013reducing, siow2020core, li2022auger, hong2022commentfinder}. Large language models (LLMs) are large-scale Transformer models, which are distinguished by their large number of parameters and extensive pre-training on diverse datasets.  Recently, LLMs have made substantial progress and have been applied across a broad spectrum of domains. Within the software engineering field, LLMs can be categorized into two main types: unified language models and code-specific models, each serving distinct purposes \cite{lu2023llama}.

Code-specific LLMs, such as CodeGen \cite{nijkamp2022codegen}, StarCoder \cite{li2023starcoder} and CodeLlama \cite{roziere2023code} are optimized to excel in code comprehension, code generation, and other programming-related tasks. These specialized models are increasingly utilized in code review activities to detect potential issues, suggest improvements, and automate review comments \cite{yang2024survey, lu2023llama}. 




\subsection{Retrieval-Augmented Generation}
Retrieval-Augmented Generation (RAG) is a general paradigm that enhances LLMs outputs by including relevant information retrieved from external databases into the input prompt \cite{gao2023retrieval}. Traditional LLMs generate responses based solely on the extensive data used in pre-training, which can result in limitations, especially when it comes to domain-specific, time-sensitive, or highly specialized information. RAG addresses these limitations by dynamically retrieving pertinent external knowledge, expanding the model's informational scope and allowing it to generate responses that are more accurate, up-to-date, and contextually relevant \cite{arslan2024business}. 

To build an effective end-to-end RAG pipeline, the system must first establish a comprehensive knowledge base. It requires a retrieval model that captures the semantic meaning of presented data, ensuring relevant information is retrieved. Finally, a capable LLM integrates this retrieved knowledge to generate accurate and coherent results \cite{ibtasham2024towards}.




\subsection{LLM as a Judge Mechanism}

LLM evaluators, often referred to as LLM-as-a-Judge, have gained significant attention due to their ability to align closely with human evaluators' judgments \cite{zhu2023judgelm, shi2024judging}. Their adaptability and scalability make them highly suitable for handling an increasing volume of evaluative tasks. 

Recent studies have shown that certain LLMs, such as Llama-3 70B and GPT-4 Turbo, exhibit strong alignment with human evaluators, making them promising candidates for automated judgment tasks \cite{thakur2024judging}

To enable such evaluations, a proper benchmarking system should be set up with specific components: \emph{prompt design}, which clearly instructs the LLM to evaluate based on a given metric, such as accuracy, relevance, or coherence; \emph{response presentation}, guiding the LLM to present its verdicts in a structured format; and \emph{scoring}, enabling the LLM to assign a score according to a predefined scale \cite{ibtasham2024towards}. Additionally, this evaluation system can be enriched with the ability to explain reasoning behind verdicts, which is a significant advantage of LLM-based evaluation \cite{zheng2023judging}. The LLM can outline the criteria it used to reach its judgment, offering deeper insights into its decision-making process.





\section{Methodology}
\label{sec:solution}

This paper presents \approach, a framework that improves the robustness of DL-based systems by generating realistic test images from existing datasets.
\approach leverages recent advances in text and visual models~\cite{DBLP:conf/cvpr/RombachBLEO22} to generate accurate synthetic images to test DL-based systems in scenarios that are not represented in the existing testing suite.

\begin{figure*}
    \centering
    \includegraphics[width=\linewidth]{images/dillema-schema.drawio.pdf}
    \caption{\approach.}
    \label{fig:dillema}
    \vspace{-6mm}
\end{figure*}

% Describe inputs of \approach
The proposed methodology, as shown in \autoref{fig:dillema}, consists of five steps.
The input of our approach is an image (from the existing test cases) along with a textual description of the task assigned to the DL-based system. The output is a modified version of the input image based on new conditions. 

% Image Captioning
\subsection{Image Captioning}

The first step of \approach involves image captioning, which is the process of converting a given image to its textual description. The objective is to enable the application of recent advances in natural language processing to images. To achieve this, \approach brings the images into the textual domain, where language models can operate effectively.

Captions are generated as multi-sentence descriptions to capture key elements and provide a detailed representation of the image. Each sentence focuses on a different aspect of the scene, capturing a range of elements such as objects, environments, and contextual relationships. This approach increases the likelihood of capturing important details that a single-sentence description might miss, providing a more comprehensive textual description for the subsequent steps.

\subsection{Keyword Identification}

Once the image is converted into textual descriptions through the captioning process, the next step in \approach is Keyword Identification. This step aims to identify which elements of the image can be safely modified without altering the overall meaning or the primary task (e.g., object classification, semantic segmentation) associated with the image. 

In this phase, the LLM is used to analyze the captions generated in the previous step and identify a set of keywords that can potentially be altered. These keywords represent modifiable aspects of the image, such as colors, weather conditions, or object properties, while excluding core elements that are essential to the task. For example, when dealing with an image classification task involving a ``car'', altering the background color or lighting usually does not modify the label. Conversely, in a semantic segmentation task focused on road scenes, the road and critical objects (cars, pedestrians, traffic signals) must remain present, though certain attributes (e.g., color, weather conditions) can still be changed. By defining the task explicitly in the prompt, we ensure that only permissible alterations are suggested by the LLM.
\autoref{fig:classification_and_segmentation} illustrates how the constraints differ between classification (\autoref{fig:classification}) and segmentation (\autoref{fig:semantic_segmentation}). In classification, the focus is on identifying and preserving the labeled object (\emph{car}), while in segmentation, multiple objects must remain for valid ground-truth labels.

\begin{figure}[]
\centering
    \begin{subfigure}{.49\linewidth}
    \centering
    \includegraphics[width=\textwidth]{images/segment.png}
    \caption{Classification Task.}
    \label{fig:classification}
    \end{subfigure}
    \begin{subfigure}{.49\linewidth}
    \centering
    \includegraphics[width=\textwidth]{images/annotated.png}
    \caption{Segmentation Task.}
    \label{fig:semantic_segmentation}
    \end{subfigure}
\caption{
Label Preservation in Autonomous Driving Tasks.
}
\label{fig:classification_and_segmentation}
\vspace{-6mm}
\end{figure}

The process of identifying these keywords is guided by task constraints. \approach prompts the LLM with a specific task-related query, such as:

\begin{tcolorbox}[arc=.3em,left=.3em,right=.3em,top=.3em,bottom=.3em]
\begin{center}
\begin{minipage}[t]{.99\linewidth}
\textbf{Prompt}: \textit{
Given the task $<$TASK$>$ and an image described by the caption $<$CAPTION$>$, what are the key elements that can be modified in the caption so that the ground truth corresponding to the image does not change?
}
\end{minipage}
\end{center}
\end{tcolorbox}

Note that this represents an example of the prompts used in \approach, intended to clarify the type of information that we request from the LLM. To improve the effectiveness of the prompt, various advanced strategies can be adopted. For example, as detailed in \autoref{sec:eval:setup}, we configured \approach to use a one-shot in-context learning prompting strategy, allowing the LLM to provide better results by including an example within the prompt.

The identification of keywords is designed to be flexible and adaptable for different tasks. The LLM relies on its internal knowledge to evaluate the contextual relevance of each word in the caption, taking into account both syntactic and semantic relationships. For example, if the task is semantic segmentation in an autonomous driving scenario, elements such as road conditions, lighting, or vehicle color may be identified as modifiable keywords, while objects essential to the task, such as vehicles themselves, remain unchanged.

\subsection{Alternative Identification}

In this phase, the LLM is leveraged to generate alternatives for the identified keywords, providing variations that can be applied to the image without altering the overall task.

The goal of this step is to explore different possibilities for modifying the elements flagged in the previous step, such as changing the color of objects, adjusting environmental conditions (e.g., weather), or altering minor details, while keeping the core structure and purpose of the image intact. For example, if the keyword ``foggy'' was identified as a modifiable attribute in the caption ``a car driving down a foggy street'', the LLM could suggest alternatives like ``rainy'' or ``snowy''.
To execute this, \approach generates a prompt asking the LLM to propose alternatives for the identified keywords. 

The main challenge in this phase is to introduce meaningful variations to the image while keeping its semantic meaning intact. The LLM plays a key role by generating alternatives that align with the original caption and task, avoiding changes that could shift the focus of the task. We take advantage of the ability of the LLM to understand contextual subtleties to avoid proposing changes to critical elements such as replacing ``car'' with ``bicycle'' in a vehicle detection scenario. An example of a prompt used in this phase is:

\begin{tcolorbox}[arc=.3em,left=.3em,right=.3em,top=.3em,bottom=.3em]
\begin{center}
\begin{minipage}[t]{.99\linewidth}
\textbf{Prompt}: \textit{
Given the task $<$TASK$>$ and an image described by the caption $<$CAPTION$>$, what are the possible alternatives for these keywords $<$KEYWORDS$>$?
}
\end{minipage}
\end{center}
\end{tcolorbox}

This process focuses on generating contextually relevant and diverse modifications, allowing the system to produce meaningful test cases for the DL model at hand. The alternatives proposed for each keyword enable \approach to explore different conditions or attributes of objects, broadening the range of scenarios included in the original dataset.

\subsection{Counterfactual Caption Generation}

This phase is responsible for creating new textual descriptions, or counterfactual captions, by applying the alternatives generated in the previous step. These counterfactual captions describe how the image would look if certain elements were modified, enabling the system to explore new scenarios while preserving the core context of the original image.

In this step, the LLM takes the original caption and replaces the identified keywords with the newly generated alternatives. The goal is to produce a new version of the caption that reflects the desired modifications without changing the essential meaning of the image. For example, if the original caption was ``a gray car driving down a foggy street'', and the alternatives generated for the keywords ``gray car'' and ``foggy'' were ``red car'' and ``snowy'', the new counterfactual caption would be ``A red car driving down a snowy street''.

The amount of edits in the new prompt can be controlled by limiting the number of alternatives applied when generating the counterfactual captions. For example, applying only one alternative at a time allows for small incremental changes, allowing exploration of subtle variations of the original caption. In contrast, applying multiple alternatives simultaneously can produce larger transformations, introducing more diverse scenarios. This approach provides fine-grained control over the extent of modifications, enabling tailored exploration of different levels of change in the generated test cases.

This phase is critically important because it ensures that the generated caption remains coherent and meaningful despite the modifications. Although replacing certain words (such as ``gray'' with ``red'') might seem straightforward, many cases are more complex, requiring careful handling to avoid breaking the sentence's meaning or introducing contradictions. For example, consider a caption like ``a road in a tundra covered in snow during a snowy day''. Replacement of the word ``tundra'' with ``desert'' would result in ``a road in a desert covered in snow during a snowy day'', which is contextually unlikely.

In this step, the LLM is prompted with the following input:

\begin{tcolorbox}[arc=.3em,left=.3em,right=.3em,top=.3em,bottom=.3em]
\begin{center}
\begin{minipage}[t]{.99\linewidth}
\textbf{Prompt}: \textit{
Given the task $<$TASK$>$, modify the caption $<$CAPTION$>$ by applying some of the following transformation described by $<$ALTERNATIVES$>$.
}
\end{minipage}
\end{center}
\end{tcolorbox}

By asking the LLM to generate the new caption directly, rather than applying simple replacement rules from the alternative dictionary, \approach ensures that the LLM processes not only the specific word replacements but also the broader sentence context, maintaining the overall meaning while making necessary adjustments to prevent contradictions or illogical outcomes.
Additionally, by explicitly including the task description at every step of the interaction, the LLM is continuously reminded of the objective it is trying to achieve. This ensures that the generated captions respect the metamorphic relationships inherent in the test case, preserving the critical connections between elements of the image and their semantic meaning.

\begin{figure*}
\centering
    \centering
    \includegraphics[width=.98\textwidth]{images/dillema_example.drawio.pdf}
\caption{Image generation in \approach.}
\label{fig:controlled_generation}
% \vspace{-5mm}
\end{figure*}
\subsection{Controlled Text-to-Image Generation}

The final step of \approach generates a modified image based on the counterfactual caption produced in the previous phase. This step is where the transformation of the image occurs, and it is carried out using a control-conditioned text-to-image diffusion model~\cite{DBLP:conf/iccv/ZhangRA23}.  The key challenge here is not only to generate a new, realistic image that aligns with the counterfactual caption but also to ensure that the spatial structure of the original image is preserved so that the integrity of metamorphic relationships is maintained.

When generating a new test image, the spatial arrangement of key objects and elements must be preserved. For example, in the context of semantic segmentation for autonomous driving, if an image depicts a car driving down a road, the generated image must include the car in the same location as the original image relative to the road, even if its color or weather conditions are changed. This way, the transformations to be applied will only affect specific attributes (e.g., altering weather or object properties) without impacting the fundamental geometry or layout of the scene. On the other hand, a distorted spatial structure could mislead the test results, making it unclear whether a failure is due to the actual shortcomings of the model or due to irrelevant transformations in the image.

To achieve spatial structure preservation, \approach uses control-conditioned diffusion models. These models allow fine-grained control over the generated image by incorporating conditioning inputs that preserve the spatial layout of the original image while applying the desired modifications.

\autoref{fig:controlled_generation} showcases examples of test cases generated by \approach for image classification (top row) and semantic segmentation (bottom row).
For image classification, the input image belongs to the class \textit{bird}, described by the captioning model as ``A yellow bird on a twig''. The second column displays the conditioning input extracted from the original image to preserve spatial arrangements. The remaining columns show images generated from alternative captions produced by the LLM: Caption A (``A blue bird on a twig'') changes the bird color to blue, while Caption B (``A red bird on a twig'') changes it to red. 
These augmentations demonstrate \approach ability to alter specific attributes while maintaining spatial structure and preserving the relevance of the class \textit{bird}. 

For semantic segmentation, the input image depicts a road with two cars during cloudy weather, with the ground truth represented as a semantic map of pixel-level classifications. The captioning model describes it as ``A road with two cars in cloudy weather''. The second column provides the conditioning input to ensure spatial consistency during generation. Caption A (``A road with two cars during snowy weather'') introduces snow to the scene, while Caption B (``A road with two cars during sunset'') applies sunset lighting. Both augmentations preserve the layout of roads, vehicles, and pedestrians as defined by the ground truth semantic map.
\section{Evaluation}
\label{sec:eval}

In this section, we evaluate the performance of \approach and aim to answer the following research questions (RQs):

\noindent\textbf{RQ\textsubscript{1} (Validity).} Can DILLEMA generate valid and realistic test cases from existing data?

\noindent\textbf{RQ\textsubscript{2} (Testing Effectiveness).} Can the generated test cases identify weaknesses in state-of-the-art DL models?

\noindent\textbf{RQ\textsubscript{3} (Retraining).} Can the generated test cases be used to improve the robustness of the tested models?

\subsection{Experimental Setup}
\label{sec:eval:setup}
\noindent\textbf{Datasets.} We performed experiments using two datasets: ImageNet1K~\cite{DBLP:conf/cvpr/DengDSLL009} and SHIFT~\cite{DBLP:conf/cvpr/SunSPWGSTY22}. These datasets represent two different tasks, image classification, and semantic segmentation, allowing us to assess the flexibility and applicability of \approach in various scenarios. ImageNet1K is a large-scale dataset commonly used for image classification tasks and SHIFT is a synthetic dataset designed for evaluating autonomous driving systems under different conditions (e.g., weather changes, lighting conditions).

\noindent\textbf{Tested Models.} We used \approach to test several DL architectures.
For ImageNet1K, we evaluated classification models (that is, ResNet18, ResNet50, and ResNet152~\cite{DBLP:conf/cvpr/HeZRS16}) using pre-trained versions provided by PyTorch. For SHIFT, we tested a semantic segmentation model (i.e., DeepLabV3~\cite{DBLP:conf/eccv/ChenZPSA18} model with a ResNet50 backbone), which we custom-trained following the original authors' training procedure~\cite{DBLP:conf/eccv/ChenZPSA18}. The training of this model took approximately $24$ hours to complete.

\noindent\textbf{Evaluation Metrics.} We used accuracy to evaluate the quality of classification models (on ImageNet1K), and we used mean Intersection over Union (mIoU) to measure the ability to evaluate the quality of semantic segmentation models.

\noindent\textbf{\approach Configuration\footnote{
To support reproducibility, all our data, including the code of \approach, the results of the human survey, of the testing and retraining, are available in our replication package: \url{https://github.com/deib-polimi/dillema}.}.} We used BLIP2 6.7B~\cite{DBLP:conf/icml/0008LSH23} as the captioning model to generate context-aware descriptions, chosen for its ability to produce detailed, semantically rich captions. As LLM, we selected a 5-bit quantized LLaMA-2 13B~\cite{DBLP:journals/corr/abs-2307-09288} model to identify keywords, generate alternatives, and create counterfactual captions. We chose LLaMA-2 because it is open source and effective, and we opted for the 13B version with 5-bit quantization since it provided a balance between performance and resource efficiency given our computational and cost constraints.
Lastly, for image generation, we used ControlNet~\cite{DBLP:conf/iccv/ZhangRA23} with edge conditioning, a control-conditioned text-to-image diffusion model. ControlNet enabled us to introduce modifications to the images while maintaining the spatial structure of the original scene, ensuring that the relationships between objects and their surroundings remained consistent.
Although we chose these general-purpose models for compatibility with consumer hardware and reasonable runtime, other models with different capabilities could be used depending on specific needs.

\noindent\textbf{Prompt Template.} To guide the LLM effectively, we used a one-shot in-context learning approach~\cite{DBLP:conf/nips/Wei0SBIXCLZ22}, where each prompt included an example to help the model understand the request more accurately. The example illustrated the expected input and output formats. Each prompt was constructed to provide context and explicitly instruct the LLM on the required output format, which allowed for automated post-processing. If the LLM response failed to adhere to the specified output format and could not be automatically parsed, we repeated the request with a different random seed. This iterative process continued until a parsable response was obtained.

\noindent\textbf{Retraining Settings.}
For ImageNet1K, we re-trained the ResNet models using a batch size of $100$ and the SGD optimizer with an initial learning rate of $0.1$, a momentum of $0.9$ and a weight decay of $1 \times 10^{-4}$. The learning rate was decayed using the PyTorch StepLR scheduler with a step size of $30$ and a gamma of $0.1$, over $90$ epochs.
For SHIFT, we re-trained the DeepLabV3 model using the original settings provided by its authors. Specifically, the batch size was set to $12$, with training conducted over $200$ epochs using the Adam optimizer with a learning rate of $0.002$, betas set to $(0.9, 0.999)$, and epsilon set to $1 \times 10^{-8}$.

\noindent\textbf{Hardware and Software.} The experiments were carried out on an AWS virtual machine with an A10G NVIDIA GPU with 24GB of memory. Neural networks were designed using PyTorch 2.0.1, and accelerated using CUDA 11.8.
In general, the empirical evaluation required about $120$ GPU hours.
$96$ GPU-hours were spent on Imagenet1K ($125,000$ test cases), $24$ GPU hours were spent on SHIFT ($10,000$ test cases).

\subsection{RQ\textsubscript{1}. Validity}
\label{sec:experiments:validity}
This experiment aims to evaluate the realism and validity of the generated images, ensuring that they preserve the metamorphic relationship for both datasets and assessing how often hallucinations occur due to potential errors during the five steps of \approach. By validating the generated images end-to-end, we aim to identify instances where the pipeline produces incorrect or unrealistic results.
To achieve this, we conducted a human study using Amazon Mechanical Turk. Human evaluators were asked to verify if the generated images preserved the metamorphic relationship for both datasets.

In total, we obtained $2,500$ total responses. To ensure quality, we used control questions to filter unreliable answers. Responses failing these quality checks were discarded. To ensure experienced participants, the workers were selected based on an approval rate greater than $95\%$ and at least $50$ completed tasks. Each image was evaluated by five independent workers and the questions were discarded if consensus (agreement of at least $\frac{4}{5}$ participants) was not reached. In the end, only $2,380$ responses (out of $2,500$) were considered robust and good enough to answer the research question.


For ImageNet1K (\autoref{fig:rq1_validity_imagenet1k}), we used two types of questions and considered a transformation to be valid if our approach were able to correctly augment an existing image without modifying the label associated with it.
First, we performed a general evaluation on a randomly sampled set of $300$ augmented images from all generated cases to measure the overall validity.
Then, we proposed a focused evaluation of $100$ augmented images that the ResNet18 model misclassified, to check if the images were valid and interpretable by humans even when misclassified by the model.

\begin{figure}[h]
    \centering
    \includegraphics[width=.45\textwidth]{images/ImageNet-bar.pdf}
    \caption{Validity of the Generated Test Cases for Classification.}
    \label{fig:rq1_validity_imagenet1k}
\end{figure}

Our human study shows that human assessors achieved agreement on all images and $99.7\%$ of the augmented images were correctly classified by human assessors. Of the $300$ images, only $1$ image did not preserve the label associated with the original image.
For the set of images where the model (i.e., ResNet18) produced a misclassification, $82.7\%$ were still considered valid by human evaluators. This shows that while the test cases generated by \approach effectively induced misclassifications in the model, most of them could still be correctly classified by humans. This suggests that failures can often be attributed to bugs in the model rather than flaws in the image generation process, reinforcing the validity and utility of \approach for robust model testing.

For the SHIFT dataset (\autoref{fig:rq1_validity_shift}), we randomly selected $100$ augmented images. Among these, all depicted roads, $25$ included vehicles, and $15$ featured one or more pedestrians. Evaluators were tasked with verifying whether key elements critical for autonomous driving, such as roads, vehicles, and pedestrians, were consistently preserved through the transformations.  We checked these aspects since they are key elements that influence the behavior of an autonomous driving system.

\begin{figure}[h]
    \centering
    \includegraphics[width=.45\textwidth]{images/SHIFT-bar.pdf}
    \caption{Validity of the Generated Test Cases for Driving.}
    \label{fig:rq1_validity_shift}
\end{figure}

We observed the following validity rates: road preservation at $98.9\%$ ($100$ questions, $7$ were discarded due to lack of consensus), pedestrian preservation at $84.6\%$ ($15$ questions, $2$ discarded due to lack of consensus), and vehicle preservation at $100.0\%$ ($25$ questions, $1$ discarded due to lack of consensus). These results highlight that \approach can effectively maintain certain features, such as roads and vehicles, while being slightly less effective at preserving pedestrians.

\subsection{RQ\textsubscript{2}. Testing Effectiveness}

\begin{figure*}
\centering
    \begin{subfigure}[b]{.49\textwidth}
    \centering
\includegraphics[width=\textwidth]{images/conf_mat_shift.pdf}
    \caption{Accuracy on Original Test Suite.\label{fig:image1}}
    \end{subfigure}
    \begin{subfigure}[b]{.49\textwidth}
    \centering
    \includegraphics[width=\textwidth]{images/conf_mat_shift_aug.pdf}
    \caption{Accuracy on \approach Augmented Test Suite.\label{fig:image2}}
    \end{subfigure}
\caption{Multi-class Confusion Matrix.}
\label{fig:rq2_effectiveness}
% \vspace{-3mm}
\end{figure*}

To evaluate the effectiveness of \approach, we evaluated its ability to detect weaknesses in state-of-the-art DL models using the generated test cases.

First, we performed experiments on ImageNet1K, focusing on identifying misclassification errors. For this purpose, we augmented $25$ images for each of the $1,000$ classes in the dataset. Each image was augmented five times to take advantage of the stochastic nature of diffusion models, which can generate different augmentations from the same input. The performance of the test suite generated by \approach was compared with the test set already available in the dataset.

\begin{table}[h]
    \footnotesize
    \centering
    \begin{tabular}{@{}lcc@{}}
        \toprule
        \xspace\space\xspace\space  \textbf{Architecture}\xspace\space\xspace\space               & \xspace\space\xspace\space\xspace\space\xspace\space\textbf{Original Test Suite}\xspace\space\xspace\space\xspace\space\xspace\space                      & \xspace\space\xspace\space\textbf{\approach Test Suite}\xspace\space\xspace\space \\ \midrule
        \xspace\space\xspace\space  ResNet18\xspace\space\xspace\space & 5.26\% & 53.29\% \\
        \xspace\space\xspace\space  ResNet50\xspace\space\xspace\space & 2.55\% & 45.47\% \\
        \xspace\space\xspace\space  ResNet152\xspace\space\xspace\space & 1.47\% & 42.33\% \\ \bottomrule
        \end{tabular}
    \caption{Test Effectiveness.}
    \label{tab:rq2_effectiveness}
\end{table}

\autoref{tab:rq2_effectiveness} reports the performance of three ResNet variants in both test suites. The results reveal that, on average, $3.1\%$ of the original test suite was able to highlight misbehaviors, while $47.0\%$
of the test suite generated by \approach exposed faulty behaviors. However, it is important to note that, as discussed in \autoref{sec:experiments:validity}, not all of these detected misbehaviors may represent true failures. The human study confirmed that approximately $82.7\%$ of the misbehaviors detected by \approach were valid failures. Even after normalizing for this factor, the effectiveness of \approach remains significantly higher ($38.9\%$) than the original test set.

In addition, we analyzed how many augmentations per image led to model errors. Our findings indicate that for $33.29\%$ of the images, all augmentations resulted in misclassifications, whereas for $24.85\%$, none of the augmentations caused errors.

For the SHIFT dataset, we evaluated the DeepLabV3 model on the semantic segmentation task. The evaluation compared the augmented test set created by \approach with the original SHIFT test set. \autoref{fig:rq2_effectiveness} presents the normalized multi-class confusion matrix of the tested model on the original and augmented data. Rows represent the ground truth, columns represent the predicted class, and the diagonal indicates the percentage of correct predictions.

The results show that \approach successfully exposed interesting faulty behaviors. For example, in semantic classes where the model appeared robust in the original dataset, such as \textit{SideWalk} ($97\%$ correctly classified), the model showed significant vulnerability in the augmented dataset (only $38\%$). In more critical classes such as \textit{Road} and \textit{Vehicle}, we observed that the model maintained a relatively robust performance, with errors increasing by $9\%$ and $10\%$, respectively, as the accuracy decreased from $99\%$ and $97\%$ in the original dataset to $90\%$ and $87\%$ in the augmented dataset. However, for pedestrian recognition, the augmented dataset revealed a much higher vulnerability, with $34\%$ more misclassifications compared to the original dataset. This highlights the need to retrain the model with a stronger focus on identifying pedestrians to address this critical weakness.

These results highlight that \approach not only highlights hidden vulnerabilities in classes previously considered robust but also provides insights into critical performance degradations in safety-relevant semantic classes. In general, \approach effectively exposes model weaknesses in various scenarios.

\subsection{RQ\textsubscript{3}. Retraining Robustness}

To assess whether the test cases generated by \approach can improve the robustness, we conducted retraining experiments using the synthetically generated data. Retraining aimed to evaluate whether the incorporation of augmented test cases into the training process leads to improved performance on both original and augmented data.

For the ImageNet1K dataset, we retrained the ResNet18 model using a combined training set consisting of the original data and the augmented test cases generated by \approach. The model was re-trained for 90 epochs using the settings described in \textit{Retraining Settings}. The re-trained model showed a significant improvement in robustness, achieving a $52.27\%$ increase in accuracy in the augmented test cases and a $20.19\%$ improvement in the original test suite.

Concerning SHIFT, we achieved an improvement in mIoU across the original and augmented test sets. After retraining, mIoU in the original test suite improved from $85.32\%$ to $88.76\%$, while mIoU in the augmented dataset showed a more pronounced increase from $72.45\%$ to $80.32\%$.
Specifically, the retraining process revealed that while performance degradation on critical semantic classes like \textit{Road} and \textit{Vehicles} was minor, pedestrian recognition showed a significant recovery, increasing from $38\%$ to $62\%$. This improvement highlights the value of \approach in augmenting datasets to address vulnerabilities in safety-critical tasks.

These findings demonstrate that the generated test cases are highly effective in not only uncovering model vulnerabilities but also improving the robustness of DL models when incorporated into the retraining process.

\subsection{Threats to Validity}
\noindent\textbf{Internal Validity.}
Our pipeline relies on pre-trained models (captioning, LLM, diffusion) and random sampling of alternatives, which can introduce randomness and potential skew (e.g., consistently generating “red” vehicles). Another concern is the domain shift between real images and our synthesized outputs: models might perform worse simply because of unfamiliar synthetic characteristics rather than true weaknesses. However, our human study indicates that the vast majority of generated images retain labels recognizable to human evaluators, suggesting that they are semantically coherent rather than purely artificial or misleading. Thus, while some failures could stem from synthetic artifacts, the high human agreement on these images implies that many observed misclassifications reflect genuine model vulnerabilities rather than artifacts alone.

\noindent\textbf{External Validity.}
We tested \approach on classification and segmentation from distinct domains, but it may not generalize to specialized scenarios (e.g., medical imaging). Although each component (captioning, LLM, diffusion) seems broadly applicable, further testing on diverse datasets is required to confirm adaptability for industrial use and other vision tasks.

\noindent\textbf{Construct Validity.}
Our primary measure of success is whether the generated images preserve ground-truth labels and uncover vulnerabilities. While human assessments indicate that images remain valid, potential biases in LLM-generated alternatives (e.g., color choices) could distort conclusions. Additionally, the notion of validity is subjective; thus, future work should employ more rigorous metrics or automated checks to validate semantic consistency in generated test cases.
\vspace{-5pt}
\section{Related Works}\label{sec:related}
\vspace{-3pt}
\subsection{Network Alignment}
Traditional network alignment methods are often built upon alignment consistency principles.
IsoRank~\cite{isorank} conducts random walk on the product graph to achieve topological consistency. FINAL~\cite{final} interprets IsoRank as an optimization problem and introduces consistency at attribute level to handle attributed network alignment.
% MOANA~\cite{moana} aligns networks at multiple granularities to achieve better scalability.
% However, the consistency assumption only considers node relationships within a local neighborhood while ignoring global graph geometry~\cite{parrot}. 
Another line of works \cite{li2022unsupervised,wang2023networked,yan2022dissecting} learn informative node embeddings in a unified space to infer alignment.
REGAL~\cite{regal} conducts matrix factorization on cross-network similarity matrix for node embedding learning.
DANA~\cite{dana} learns domain-invariant embeddings for network alignment via adversarial learning.
% NetTrans~\cite{nettrans} aligns networks based on nonlinear network transformation.
BRIGHT~\cite{bright} bridges the consistency and embedding-based alignment methods, and NeXtAlign~\cite{nextalign} further balances between the alignment consistency and disparity by crafting the sampling strategies.
WL-Align~\cite{wlalign} utilizes cross-network Weisfeiler-Lehman relabeling to learn proximity-preserving embeddings. 
% Nevertheless, the success of embedding-based methods relies heavily on hand-crafted sampling strategies and are sensitive to graph noises.
More related works on network alignment are reviewed in~\cite{du2021new}.

\vspace{-3pt}
\subsection{Optimal Transport on Graphs}
OT has recently gained increasing attention in graph mining and network alignment, whose effectiveness often depends on the pre-defined cost function restricted to specific graphs.
% The key idea is to represent graphs as distributions over the node sets and minimize the total transportation distance based on cost functions defined over the two distributions.
% However, the effectiveness of most OT-based alignment methods depend largely on the pre-defined cost function restricted to specific graphs.
For example, \cite{got, walign, fgot,yan2024trainable} represent graphs as distributions of filtered graph signals, focusing on one specific graph property, while other cost designs are mostly based on node attributes~\cite{got2} or graph structures~\cite{goat}. PARROT~\cite{parrot} integrates various graph properties and consistency principles via a linear combination, but requires arduous parameter tuning.
More recent works combine both embedding and OT-based alignment methods.
% to supervise embedding learning for better cost design.
GOT~\cite{got2} adopts a deep model to encode transport cost. GWL~\cite{gwl} learns graph matching and node embeddings jointly in a GW learning framework. SLOTAlign~\cite{slotalign} utilizes a parameter-free GNN model to encode the GW distance between two graph distributions.
CombAlign~\cite{combalign} further proposes to combine the embeddings and OT-based alignment via an ensemble framework.

\vspace{-3pt}
\subsection{Graph Representation Learning}
Representation learning gained increasing attention in analyzing complex systems with applications in trustworthy ML~\cite{liu2024aim,yoo2024ensuring,fu2023privacy,bao2024adarc}, drug discovery \cite{wei2022impact,wu2023risk,zhang2024clinical,sui2024cancer} and recommender systems~\cite{liu2024collaborative,zeng2024interformer,wei2024towards}.
Early approaches \cite{perozzi2014deepwalk,grover2016node2vec} utilize random walks and process graphs as sequences by a skip-gram model. \cite{Hamilton2017Inductive,zeng2019graphsaint} sample fixed-size neighbors for better scalability to large graphs. More recent studies~\cite{huang2018adaptive,yan2024reconciling} focus on adaptive and unified sampling strategies that benefit various graphs.
Based on these strategies, graph contrastive learning\cite{velivckovic2018deep,jing2022coin,jing2024sterling,zheng2024pyg,Sun2020InfoGraph} learns node embeddings by pulling similar nodes together while pushing dissimilar ones apart.

\section{Conclusion }
This paper introduces the Latent Radiance Field (LRF), which to our knowledge, is the first work to construct radiance field representations directly in the 2D latent space for 3D reconstruction. We present a novel framework for incorporating 3D awareness into 2D representation learning, featuring a correspondence-aware autoencoding method and a VAE-Radiance Field (VAE-RF) alignment strategy to bridge the domain gap between the 2D latent space and the natural 3D space, thereby significantly enhancing the visual quality of our LRF.
Future work will focus on incorporating our method with more compact 3D representations, efficient NVS, few-shot NVS in latent space, as well as exploring its application with potential 3D latent diffusion models.


% \section{Data Availability} \label{sec:avail}
% To support reproducibility, all our data, including the code of \approach, the results of the human survey, of the testing and retraining, are available in our replication package\cite{dillema_replication_package}.
\bibliographystyle{IEEEtran}  
\balance
\bibliography{bibl}  

\end{document}
