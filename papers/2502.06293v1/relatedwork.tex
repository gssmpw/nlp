\section{Related Work}
\label{sec:related-work}
Rust and its usage have been the focus of much research in
the last decade.
Notably, RustBelt~\cite{JKD18} provides formal verification of Rust's
type system and ownership model.
Evans et al.~\cite{EvansCS20} shows that while only 30\% of Rust
libraries explicitly use unsafe code, over half contain unsafe
operations somewhere in their dependency chain that bypass Rust's
static checks.
Astraukas et al.~\cite{AstrauskasMP0S20} shows that unsafe code
appears frequently, particularly for language interoperability, though
its unsafe code tends to be straightforward and well-contained.


Dynamic partial order reduction (DPOR) techniques have been
successfully applied to model check concurrent programs across
different languages and
contexts~\cite{AbdullaAJS14,FlanaganG05,AbdullaAAJLS15}. For Rust
specifically, several tools target concurrency bug detection, though
with different approaches and tradeoffs compared to ours.

Miri~\cite{miri} is part of the Rust compiler project and is
advertised as ``an undefined behavior detection tool for Rust''. It
works as an interpreter of Rust's mid-level IR.
Miri is particularly effective at detecting undefined behaviours, but
its inherently dynamic approach means that it cannot systematically
analyse all potential interleavings of concurrent programs.
Loom~\cite{loom} is an adaptation of CDSChecker~\cite{NorrisD16}. It
explores interleavings of concurrent code under
the C11 memory model using partial-order reduction techniques.
Shuttle~\cite{awshuttle} is another concurrent code testing tool for
Rust, developed at AWS. It is similar to (and inspired by) Loom but
aims at better scalability at the cost of soundness. Shuttle uses a
randomised scheduler~\cite{BurckhardtKMN10} to guarantee good
coverage.
Lockbud~\cite{Lockbud,QinCLZWSZ24} is a static bug detector for
blocking bugs and potential atomicity violations in Rust. It
identifies potential atomicity violations by detecting syntactical
patterns commonly found in atomicity violation bugs. While this
pattern-based approach can identify potential issues, \rustmc's
systematic exploration of thread interleavings allows it to
definitively verify atomicity properties through assert statements,
detecting actual violations rather than potential ones.
Other work~\cite{CRUST_TomanPT15,LiWSL22_FFI_checker} explores 
verification techniques for Rust's FFI dependencies, but 
focuses primarily on memory management bugs rather than concurrency
issues.

\genmc{} has evolved significantly since its introduction in 2019, adding
support for locks~\cite{lock_handling_Kokologiannakis19}, 
barriers~\cite{barrier_aware_model_checking_Kokologiannakis21} and symmetry 
reduction~\cite{SPORE_2024_Kokologiannakis}.
Its latest development, MIXER, enables verification of programs with
mixed-size accesses at the cost of some additional
overhead~\cite{kokolgiannakis_mixed_size}. Integrating MIXER into
\rustmc would expand its capabilities to handle more use cases, but 
at the time of writing MIXER is yet to be integrated into \genmc's master
branch. As discussed in~\S\ref{sec:llvm_intrinsics}, we intend to
explore a hybrid strategy that combines our efficient heuristics with MIXER to
support a broader range of Rust programs while preserving verification
performance.
Independently, Sato et al.~\cite{SatoMKT24} have proposed an
alternative technique to support mixed-size access in \genmc, but
their implementation is not publicly available at the time of writing.