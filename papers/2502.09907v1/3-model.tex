\section{Model}\label{sec:model}

\paragraph{Notation.} We associate the Borel sigma algebra with the set $[0,1]$, and denote it by $\borel$. We use $\lambda(\cdot)$ to represent the Lebesgue measure.  For any set $A$, $\Delta \left( A \right)$ denotes the set of probability measures on $A$. For every $a \in A$, we denote by $\delta_a$ the Dirac distribution  which puts all mass at $a$. Given a space of probability measures $\Delta(A)$, we refer to the topology induced by the weak convergence on that space as the ``weak topology''. Unless otherwise specified, product spaces are endowed with the product sigma algebra and the product topology. For a cumulative distribution function (CDF) $F:[0,1] \to [0,1]$, we use $F^-$ to denote its generalized inverse, i.e., $F^-(t) \coloneqq \inf\{x \mid F(x) \geq t\}$. For $A,B \subset \mathbb{R}$, we say that $A \preccurlyeq B$ if and only if $\sup A \leq \inf B$. We use $\{f(X) \mid X \sim \mathcal D\}$ to denote the distribution of $f(X)$ when $X \sim \mathcal D$.

Consider a buyer participating in a sealed-bid first-price auction for a single indivisible good. Let $v$ denote the value the buyer derives from winning the good. We assume that $v \in [0,1]$ and it is distributed according to the distribution $F \in \Delta([0,1])$; we use $F$ to denote both the CDF and the measure it defines. Given a value ${v \sim F}$, the buyer selects a (potentially random) bid $b \sim s(v)$, where ${s: [0,1] \longrightarrow \Delta([0,1])}$ is the buyer's bidding strategy that maps value $v$ to the bid distribution $s(v)$. For the sake of completeness and rigor, we only consider bidding strategies for which there exists a Markov kernel\footnote{$\kappa: \borel \times [0,1] \to [0,1]$ is called a \emph{Markov kernel} if (i) for every fixed $B \in \borel$, $v \mapsto \kappa(B,v)$ is measurable; (ii) for every fixed $v \in [0,1]$, $\kappa(\cdot, v)$ is a probability measure on $([0,1], \borel)$.} ${\kappa_s:\borel \times [0,1] \to [0,1]}$ such that $\Prob_{b \sim s(v)}(B) \coloneqq \kappa_s(B, v)$ for all $B \in \borel$; let $\Scal$ denote the set of all such bidding strategies.
We use $J_{s,F}$ to denote the joint distribution of value-bid pairs $(v,b)$ under the strategy $s$, i.e., ${J_{s,F}(A \times B) = \E_{v \sim F}[\kappa_s(B,v) \cdot \mathbbm{1}(v \in A)]}$ for all $A,B \in \borel$. Additionally, we define $P_{s,F} \in \Delta([0,1])$ to be the bid distribution induced by the bidding strategy~$s$ and the value distribution $F$, i.e., $P_{s,F}(B) = \E_{v\sim F}[\kappa_s(B, v)]$ for all $B \in \borel$.

Simultaneously, the competitors submit their own bids $\{b_i\}_i \subset [0,1]$. We use $h \coloneqq \max_i b_i$ to denote the \emph{highest competing bid}, and $H \in \Delta([0,1])$ to denote its distribution (represented by its cumulative distribution function). We assume that $H$ is independent of the value-bid joint distribution $J_{s,F}$. In sealed-bid auctions, the independence follows directly from the independent-private-values assumption that prevails in much of the work on first-price auctions (e.g., see \citealt{krishna2009auction, milgrom2004putting, balseiro2023contextual, feng2021convergence}). In practice, it holds in scenarios where correlation in values is caused by a publicly-observable context, which yields independence upon conditioning, i.e., the values are independent for any fixed context. For example, in online advertising, buyers' values depend on  user-specific features, which are communicated to the buyers (or their autobidders) before bids are solicited, and act as the context for the auction. As buyers can specify different bids for different user segments, via separate campaigns if necessary, their values are often independent for each segment. Crucially, it allows us to endow the buyer with independent private information, which can be used to determine her bid but cannot be exploited by others. The ``amount'' of such information turns out to have a significant impact on performance.

Once the bids are submitted, the allocation and payment are decided according to a first-price auctions: the good is allocated to the buyer if and only if she is the highest bidder, i.e., $b \geq h$, in which case the buyer pays the auctioneer her bid $b$. If the product is not allocated to the buyer, i.e., $b < h$, the buyer does not pay anything. We posit that the buyer has a quasi-linear utility, i.e, for value $v$, an associated bid $b$, and a highest competing bid $h$, the utility of the buyer is
\begin{equation*}
    u(b,h ; v) := \left( v - b \right) \cdot \mathbbm{1} \left\{ b \geq h \right\}.
\end{equation*}

Given a bidding strategy $s$, which maps each value $v \in [0,1]$ to a distribution of bids $s(v) \in \Delta([0,1])$, and a distribution of highest competing bids $H$, the expected utility of the buyer is
\begin{equation*}
\U[F]{s, H}\ \coloneqq\   \mathbb{E}_{(v, h) \sim F \times H}  \left[ \mathbb{E}_{b \sim s(v)} \left[ u(b,h; v) \right] \right]\,.
\end{equation*}
We abuse notation slightly and use $\U[F]{s, h}$ to denote $\U[F]{s,\delta_h}$.


\noindent \textbf{Objective.} The buyer aims to select a strategy $s \in \Scal$ which maximizes her expected utility $\U[F]{s, H}$. However, as is often the case in practice, the buyer does not know the distribution of the highest competing bids $H$. In light of this uncertainty about $H$, it is natural to design bidding strategies that guarantee strong performance simultaneously against \emph{all} potential highest bid distributions $H \in \Delta([0,1])$, which is our aim in this work. This motivates us to measure the performance of a bidding strategy using regret, which is defined as
\begin{equation}
    \label{eq:regret}
    \reg_F(s,H)\ \coloneqq\  \sup_{s' \in \Scal}\ \U[F]{s',H} - \U[F]{s,H} \,.
\end{equation}

The regret $\reg_F(s, H)$ quantifies the sub-optimality of employing a given bidding strategy $s$ against the highest competing bid distribution $H$. It is defined as the difference between the utility achieved by an oracle, who knows the distribution $H$ and selects the optimal bidding strategy, and the utility obtained by our chosen bidding strategy. As the distribution $H$ is unknown and unavailable while designing $s$, we take the robust-optimization approach and aim to minimize this sub-optimality uniformly over all highest competing bid distributions $H$, i.e., we aim to minimize the \emph{worst-case regret} $\mathrm{WReg}_F(s) \coloneqq \sup_H \reg_F(s,H)$. Formally, our goal is to characterize bidding strategies which minimize worst-case regret:
\begin{equation}
\label{eq:minimax_strategy_problem}
  \inf_{s \in \Scal}\ \mathrm{WReg}_F(s)\ =\  \inf_{s \in \Scal}\ \sup_{H \in \Delta([0,1])} \reg_F(s, H)\,.
\end{equation}
When the problem \eqref{eq:minimax_strategy_problem} admits a minimizer $s^*$, we refer to it as a \emph{minimax-optimal bidding strategy}. 


We conclude with a brief discussion of the model. First, note that our definition of utility assumes ties are broken in favor of the buyer under consideration. We make this choice purely for notational convenience and it is without loss of generality: the minimax-optimal bidding strategies we design under this assumption continue to be minimax optimal for all possible tie-breaking rules, as we show in \Cref{appendix:tie-breaking}. Intuitively, this is because our minimax-optimal strategies induce absolutely continuous bid distributions, thereby making ties a zero-probability event.

Next, note that our definition of regret in \eqref{eq:regret} is at the ex-ante stage, i.e., regret is measured in expectation over the private value of the buyer. This choice is motivated by online display advertising, where a buyer (advertiser) typically participates in thousands of first-price auctions as a part of their ad campaign. Therefore, standard concentration arguments apply, and any strategy which does well in expectation ends up performing well cumulatively across the large number of auctions. Similar reasoning has motivated prior works on budget management in auctions to use expected utility as the objective and study budget constraints which hold in expectation. For example, see \citet{gummadi2012repeated, abhishek2013optimal, balseiro2021budget, balseiro2023contextual} for models of single-shot auctions with in-expectation constraints and objectives. Thus, bidding strategies with low worst-case ex-ante regret would yield good performance over the entire campaign. Especially in uncertain market conditions and high-volatility periods that make it impossible to use machine-learning techniques to learn good strategies, either due to a dearth of data or rapid changes in the market that render past data obsolete. Our work offers a robust alternative to learning-based methods: minimax-optimal bidding strategies come with regret guarantees which hold regardless of how the market behaves, and hold for each auction individually.

Furthermore, our model treats the value distribution $F$ as a parameter and allows it to take arbitrary values. One particular value it can take is $\delta_v$, which corresponds to the case where the value is deterministic and equal to $v$. Thus, our model can also capture the interim regret minimization problem, where the value $v$ of the buyer is fixed and known, and she wishes to minimize worst-case regret over all possible highest-competing bid distributions. In other words, our model is more general that the one which measures regret at the interim stage. This added generality is crucial because it endows the buyer with independent private information that can be used by the buyer but not the competition. Such information is common in real-life auctions due to idiosyncratic preferences of the participants. The amount of variation in the value distribution is a measure of the ``amount'' of private information, and understanding its impact on regret is one of our primary contributions. However, the more general definition of regret comes with a cost: it makes our analysis significantly more challenging because our problem is parameterized by an infinite-dimensional value distribution $F \in \Delta([0,1])$, instead of the one-dimensional  value $v$.





