\begin{abstract}
    First-price auctions are one of the most popular mechanisms for selling goods and services, with applications ranging from display advertising to timber sales. Unlike their close cousin, the second-price auction, first-price auctions do not admit a dominant strategy. Instead, each buyer must design a bidding strategy that maps values to bids---a task that is often challenging due to the lack of prior knowledge about competing bids. To address this challenge, we conduct a principled analysis of prior-independent bidding strategies for first-price auctions using worst-case regret as the performance measure. First, we develop a technique to evaluate the worst-case regret for (almost) any given value distribution and bidding strategy, reducing the complex task of ascertaining the worst-case competing-bid distribution  to a simple line search. Next, building on our evaluation technique, we minimize worst-case regret and characterize a minimax-optimal bidding strategy for \emph{every value distribution}. We achieve it by explicitly constructing a bidding strategy as a solution to an ordinary differential equation, and by proving its optimality for the intricate infinite-dimensional minimax problem underlying worst-case regret minimization. Our construction provides a systematic and computationally-tractable procedure for deriving minimax-optimal bidding strategies. When the value distribution is continuous, it yields a deterministic strategy that maps each value to a single bid. We also show that our minimax strategy significantly outperforms the uniform-bid-shading strategies advanced by prior work.
    Importantly, our result allows us to precisely quantify, through minimax regret, the performance loss due to a lack of knowledge about competing bids. We leverage this to analyze the impact of the value distribution on the performance loss, and find that it decreases as the buyer's values become more dispersed.
\end{abstract}

