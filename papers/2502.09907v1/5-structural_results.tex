



\section{Impact of the Value Distribution}\label{sec:value-dist-impact}

Beyond characterizing a minimax-optimal bidding strategy, \Cref{thm:main-result} also quantifies the buyer’s minimax regret, providing a valuable metric for assessing the impact of informational deficiency in bidding environments. In this section, we show that our result facilitates comparisons of worst-case regret across different value distributions $F$, allowing buyers to ascertain whether acquiring information about competing bids could substantially improve their performance.

We first characterize the value distribution which maximizes the minimax regret, i.e., the one for which the buyer is suffering the most from the information deficiency. 
For $\rho \in [1,\infty]$, let $\mathcal{F}_\rho$ denote the set of distributions $D$ on $[0,1]$ such that $D(A) \leq \rho \cdot \lambda(A)$ for all $A \in \borel$. Here, $\mathcal F_\infty$ is simply the set $\Delta([0,1])$ of all distributions on $[0,1]$. Intuitively, $\mathcal F_\rho$ is the set of distributions whose ``density'' is bounded above by $\rho$.
Our goal is to solve, for every $\rho \in [1,\infty]$, the following optimization problem:
\begin{equation*}
    \sup_{F \in \mathcal{F}_{\rho}}\ \inf_{s \in \Scal}\ \sup_{H \in \Delta([0,1])}\ \reg_F(s,H)\,.
\end{equation*}
We note that this max-min-max problem introduces an additional maximization layer over the infinite-dimensional space of value distributions, further complicating the already challenging minimax problem solved in \Cref{sec:minimax}. Nevertheless, our next result shows that the problem remains solvable, allowing us to characterize the worst-case value distribution for every $\rho \in [1,\infty].$ 
\begin{theorem}\label{thm:worst-value-dist}
	For every $\rho \in [1,\infty]$, we have
	\begin{align*}
		\unif(1 - \tfrac{1}{\rho}, 1)\ \in\ \argmax_{F \in \mathcal F_\rho}\ \inf_{s \in \Scal}\ \sup_{H \in \Delta([0,1])}\ \reg_F(s,H)\,,
	\end{align*}
	where $\unif(1 - \tfrac{1}{\infty},1) \coloneqq \delta_1$.
\end{theorem}
\Cref{thm:worst-value-dist} shows that, for every $\rho \in [1,\infty]$, the value distribution that maximizes the buyer’s minimax regret is the one that concentrates mass as much as possible among all distributions in $\mathcal{F}_{\rho}$. Indeed, the regret-maximizing distribution $F_{\rho} = \unif(1 - \tfrac{1}{\rho}, 1)$ satisfies $F_{\rho}(A) = \rho \cdot \lambda(A)$ for every measurable set $A$ included in the support of $F_{\rho}$. 

\Cref{fig:regret_unif} plots the regret of our minimax-optimal strategy when the value distribution follows a $\unif(a,1)$ for $a \in [0,1)$. With $\rho = 1/(1 - a)$, the worst-case regret at $a$ reflects the largest regret that is incurred by our minimax-optimal strategy across all distributions in $\mathcal F_\rho$. We see that the worst-case regret improves significantly as the value distribution becomes less concentrated. In particular, the minimax-optimal regret increases by over 100\%  when $a$ shits from $0$, representing the uniform distribution, to $1$, representing the completely concentrated $\delta_1$ distribution.
\begin{figure}[h!]
    \centering
    \begin{tikzpicture}[scale = 0.65]
    \begin{axis}[
        width=10cm,
        height=10cm,
        xmin=0,xmax=1,
        ymin=0,ymax=0.6,
        table/col sep=comma,
        xlabel={$a$},
        ylabel={Worst-case regret},
        grid=both,
        legend pos=north west,
        %legend style ={font ={\footnotesize}}
    ]

    
    \addplot [blue,  line width=0.7mm, ,mark=square,mark options={scale=.1}] table[x=a,y={regret}] {Data/regrets_uniform_a.csv};
    \addlegendentry{Minimax strategy}


    \addplot [violet,  line width=0.7mm, ,mark=square,mark options={scale=.1}] table[x={rho},y={best_performance}] {Data/best_alpha_and_performance_uniform_dist.csv};
    \addlegendentry{Optimal uniform-bid-shading}
    
    \addplot [black, line width = 0.7mm] {0.25*x + 0.25};  
   \addlegendentry{$\alpha = 0.5$}
      \addlegendimage{ultra thick,black}
    \end{axis}
    \end{tikzpicture}
    \caption{\textbf{Impact of the value distribution on performance.} We report the worst-case regret of various bidding strategies as a function of $a$, when the value distributions is of the form $\unif(a,1)$. The black curve corresponds to the uniform-bid-shading strategy with $\alpha = 0.5$, the violet curve to the strategy which uses the optimal uniform-bid-shading strategy for each $a$, and the blue curve to the minimax-optimal strategy.} 
    \label{fig:regret_unif}
\end{figure}

\Cref{fig:regret_unif} also quantifies the benefit one can derive from accounting for the value distribution, instead of treating each value in isolation. To see this, consider a tale of two advertisers (buyers), called Alice and Bob, who wish to launch an online ad campaign. Each of them is going to participate in a large number of first-price auctions independently of each other. Both lack information about the competition, and wish to develop a bidding strategy to minimize worst-case regret. Additionally, they have a preference for deterministic strategies due to their simplicity and reproducibility. Moreover, they share the same value distribution $F$. Although they face the same problem, they diverge in their solution approaches. Alice accounts for the fact that her values are random and distributed according to $F$, and thus elects to use our minimax-optimal strategy $\s[Q^*]$. Bob, on the other hand, ignores the value distribution and treats each value in isolation. Therefore, he uses the result of \citet{kasberger2023robust} and selects the deterministic minimax-optimal bid $b = 0.5 \cdot v$ for each value $v$, i.e., he uses the uniform-bid-shading strategy with $\alpha = 0.5$. Unlike Alice, Bob is (implicitly) making the pessimistic assumption that the competition has knowledge of his value $v$, and can customize their bids using that information. \Cref{fig:regret_unif} depicts the outcome of this tale with a performance comparison of the two strategies. The minimax-optimal strategy of Alice consistently and significantly outperforms the uniform-bid-shading strategy of Bob. For example, when the values are uniformly distributed ($a = 0$), the worst-case regret of the uniform-bid-shading strategy with $\alpha = 0.5$ is 66\% higher than the worst-case regret of the minimax-optimal strategy. The primary driver of this gap in performance is the ability of the minimax-optimal strategy to account for the value distribution $F$. In particular, the greater the variation in values, the greater the information asymmetry between the buyer and the competition. The minimax-optimal strategy exploits this asymmetry and leverages the inherent variation in values to hedge against bad outcomes. If one ignores the value distribution, as Bob did, it leads to an overly-pessimistic strategy which assumes that private information, which is not even available to the competition, can be used to hurt the buyer. This prevents it from leveraging the inherent variation in values, and leads to consistently sub-optimal performance across all distributions.

Finally, our minimax-optimal strategy also outperforms uniform-bid-shading strategies as a class. In particular, it does consistently better than the optimal uniform-bid-shading strategy which accounts for the value distribution, improving performance across all values of $a$. The improvement comes in spite of the unfavorable choice of distribution: \Cref{thm:worst-value-dist} implies that $\unif(a,1)$ leads to the largest minimax-optimal regret among all value distributions in $\mathcal{F}_{\frac{1}{1-a}}$. 


