\section{Alternative Tie-Breaking Rules}\label{appendix:tie-breaking}

In our model (\Cref{sec:model}), we assumed that the utility is given by
\begin{align*}
    u(b,h;v) = (v - b) \mathbbm{1}(b \geq h)\,.
\end{align*}

This assumed that ties are broken in favor of the buyer under consideration. Here, we show that our results continue to hold for all other tie-breaking rules. Consider an arbitrary tie-breaking rule which yields an expected utility of $\pi(v,h)$ whenever the buyer bids $h$ and ties with the highest competing bid. Here, $\pi(v,h) \leq (v-h)$ because $v-h$ is the utility obtained when the tie is broken completely in favor of the buyer. The utility under this tie-breaking rule is given by
\begin{align*}
    u^\pi(b,h;v) = (v - b) \mathbbm{1}(b > h) + \pi(v,b) \mathbbm{1}(b = h)\,,
\end{align*}
and it satisfies $u^\pi(b,h;v) \leq u(b,h;v)$ for all $b,h,v \in [0,1]$.

We start by showing that the tie-breaking rules does not affect the optimal utility which can be attained against any highest-competing-bid distribution.

\begin{lemma}\label{lemma:tie-breaking-bechmark}
    For every highest-competing-bid distribution $H \in \Delta([0,1])$, we have
    \begin{align*}
        \sup_{s' \in \Scal}\ \E_{(v,h) \sim F \times H}\left[ \E_{b \sim s'(v)}[u^\pi(b,h;v)] \right] = \sup_{s' \in \Scal}\ \E_{(v,h) \sim F \times H}\left[ \E_{b \sim s'(v)}[u(b,h;v)] \right] = \sup_{s' \in \Scal} \U[F]{s', H}
    \end{align*}
\end{lemma}
\begin{proof}
    Since $u^\pi(b,h;v) \leq u(b,h;v)$, we have
    \begin{align*}
        \sup_{s' \in \Scal}\ \E_{(v,h) \sim F \times H}\left[ \E_{b \sim s'(v)}[u^\pi(b,h;v)] \right] \leq \sup_{s' \in \Scal}\ \E_{(v,h) \sim F \times H}\left[ \E_{b \sim s'(v)}[u(b,h;v)] \right]\,.
    \end{align*}
    For contradiction, suppose the inequality is strict. Then, there exists $\epsilon > 0$ and a strategy $s \in \Scal$ such that
    \begin{align*}
        \E_{(v,h) \sim F \times H}\left[ \E_{b \sim s'(v)}[u^\pi(b,h;v)] \right] < \E_{(v,h) \sim F \times H}\left[ \E_{b \sim s(v)}[u(b,h;v)] \right]\ -\ \epsilon \qquad \forall\ s' \in \Scal\, \tag{\#}
    \end{align*}
    As bidding higher that the value is never beneficial, we can assume without loss of generality that we always have $b \leq v$ when $b \sim s(v)$. 
    
     Now, define the strategy $s'$ to be the one which always bids $\epsilon$ more than the bidding strategy $s$ whenever possible, i.e., $s'(v)$ is the distribution of $\min\{b+\epsilon, 1\}$ when $b \sim s(v)$.  Observe that, for all $b,h,v \in [0,1]$ such that $b \leq v$, we have 
    \begin{align*}
        u^\pi(b + \epsilon,h;v) \geq (v - b - \epsilon) \mathbbm{1}(\min\{b + \epsilon, 1\} > h) \geq (v - b) \mathbbm{1}(b \geq h)\ -\ \epsilon\,.
    \end{align*}
    Therefore, we get
    \begin{align*}
        \E_{(v,h) \sim F \times H}\left[ \E_{b \sim s'(v)}[u^\pi(b,h;v)] \right] &= \E_{(v,h) \sim F \times H}\left[ \E_{b \sim s(v)}[u^\pi(b+\epsilon,h;v)] \right]\\
        &\geq \E_{(v,h) \sim F \times H}\left[ \E_{b \sim s(v)}[u(b,h;v)] \right]\ -\ \epsilon\,.
    \end{align*}
    This yields the desired contradiction to $(\#)$.
\end{proof}

Next, we show that the tie-breaking rule does not impact utility under any bidding strategy which induces a continuous bid distribution.

\begin{lemma}\label{lemma:tie-breaking-cont-strat}
    Every bidding strategy $s \in \Scal$ which induces an absolutely continuous bid distribution $P_{s,F}$ satisfies
    \begin{align*}
        \E_{(v,h) \sim F \times H}\left[ \E_{b \sim s(v)}[u^\pi(b,h;v)] \right] = \E_{(v,h) \sim F \times H}\left[ \E_{b \sim s(v)}[u(b,h;v)] \right] = \U[F]{s,F} \qquad \forall\ H \in \Delta([0,1])\,.
    \end{align*}
\end{lemma}
\begin{proof}
    We show that ties are a zero probability event in these conditions. Note that, for any highest competing bid $h$, the absolute continuity of $P_{s,F}$ implies
    \begin{align*}
        \Prob_{v \sim F, b \sim s(v)}(b = h) = \Prob_{b \sim P_{s,F}}(b = h) = 0\,.
    \end{align*}
    Therefore, for any fixed $h$, we have $u^\pi(b,h;v) = u(b,h;v)$ almost surely for $v \sim F, b \sim s(v)$. The lemma follows as a consequence.
\end{proof}

Combining the two lemmas allows us to establish the minimax optimality of our strategy $\s[Q^*]$ for all tie-breaking rules. Before stating the result, we introduce one new piece of notation: define the worst-case regret under the alternative tie-breaking rule $\pi$ as follows
\begin{align*}
    \mathrm{WReg}^\pi_F(s) \coloneqq \sup_{H \in \Delta([0,1]}\ \sup_{s' \in \Scal}\ \E_{(v,h) \sim F \times H}\left[ \E_{b \sim s'(v)}[u^\pi(b,h;v)] \right] - \E_{(v,h) \sim F \times H}\left[ \E_{b \sim s(v)}[u^\pi(b,h;v)] \right]\,.
\end{align*}
Now, \Cref{lemma:tie-breaking-bechmark} and $u^\pi(b,h;v) \leq u(b,h;v)$ imply
\begin{align}\label{eq:tie-breaking-wreg-ineq}
    \mathrm{WReg}^\pi_F(s) \geq \mathrm{WReg}_F(s) \qquad \forall\ s \in \Scal\,.
\end{align}
The following proposition shows that they attain the same optimal worst-case regret, and do so with our minimax-optimal strategy.

\begin{proposition}\label{prop:tie-breaking-wreg}
    For every value distribution $F$ and tie-breaking rule $\pi$, we have
    \begin{align*}
        \mathrm{WReg}^\pi_F(\s[Q^*])\ =\ \inf_{s \in \Scal}\  \mathrm{WReg}^\pi_F(s)\ =\  \inf_{s \in \Scal}\  \mathrm{WReg}_F(s)\ =\ \mathrm{WReg}_F(\s[Q^*])\,,
    \end{align*}
    where $\s[Q^*]$ is the minimax-optimal bidding strategy described in \Cref{thm:main-result}.
\end{proposition}
\begin{proof}
    In light of \eqref{eq:tie-breaking-wreg-ineq} and \Cref{cor:partial-info-saddle}, it suffices to show that
    \begin{align*}
        \mathrm{WReg}^\pi_F(\s[Q^*])\ =\ \mathrm{WReg}_F(\s[Q^*])\,.
    \end{align*}
    Recall that \Cref{lemma:quantile-based-bidding} implies the absolute continuity of bidding strategies based on quantile-based bidding strategies. $\s[Q^*]$ is one such strategy, and therefore it induces an absolutely continuous bid distributions. Therefore, \Cref{lemma:tie-breaking-cont-strat} applies and we get
    \begin{align*}
        \mathrm{WReg}^\pi_F(\s[Q^*])\ &=\ \sup_{H \in \Delta([0,1]}\ \sup_{s' \in \Scal}\ \E_{(v,h) \sim F \times H}\left[ \E_{b \sim s'(v)}[u^\pi(b,h;v)] \right] - \E_{(v,h) \sim F \times H}\left[ \E_{b \sim \s[Q^*](v)}[u^\pi(b,h;v)] \right]\\
        &=\ \sup_{H \in \Delta([0,1]}\  \sup_{s' \in \Scal} \U[F]{s', H} - \U[F]{\s[Q^*], H}\\
        &=\ \mathrm{WReg}_F(\s[Q^*]) \qedhere
    \end{align*}
\end{proof}

Thus, we have shown that $\s[Q^*]$ is a minimax-optimal bidding strategy for all tie-breaking rules $\pi$, as desired.



\section{$\mathbf{F(0) = 0}$ is Without Loss of Generality}\label{appendix:atom-at-0}

We  assumed that the value distribution $F$ satisfied $F(0) = 0$ in \Cref{sec:minimax}. Here, we establish our claim that this assumption is without loss of generality. In particular, we show that \Cref{thm:main-result} can be also used to characterize the minimax-optimal strategy, and its regret, even for value distributions which do not satisfy $F(0) = 0$. Note that, in the scenario where $v = 0$ with probability 1, always bidding zero is minimax optimal and the minimax regret is zero. Therefore, we only consider value distributions which produce a positive value with positive probability.


\begin{proposition}\label{prop:atom-at-0}
	Consider an arbitrary value distribution $\tilde F \in \Delta([0,1])$ with $\tilde F(0) = a < 1$, and let $F$ be the distribution of the value $v \sim F$ conditioned on it being non-zero, i.e., $F(t) = (\tilde F(t) - a)/(1-a)$ for all $t \in [0,1]$. Let $\s[Q^*]$ be the minimax strategy corresponding to the value distribution $F$, as given in Theorem~\ref{thm:main-result}. Then, the bidding strategy $\tilde \s_{Q^*}$ defined as
	\begin{align*}
		\tilde s_{Q^*}(v) = \begin{cases}
			\delta_0 &\text{if } v = 0\\
			\s[Q^*](v) &\text{if } v \in (0,1]\\
		\end{cases}
	\end{align*}
	is a minimax-optimal bidding strategy and satisfies
	\begin{align*}
		\sup_{H \in \Delta([0,1])} \reg_{\tilde F}(\tilde s_{Q^*}, H) = \inf_{s \in \Scal} \sup_{H \in \Delta([0,1])} \reg_{\tilde F}(s, H) = (1 - a) \cdot \sup_{H \in \Delta([0,1])} \reg_F(\s[Q^*], H)\,.
	\end{align*}
\end{proposition}

\begin{proof}
	
	Let $\Scal_0 \subset \Scal$ be the set of strategies which always bid $0$ when the value is $0$.
	
	Observe that $u(0,h;0) = 0$ for all $h \in [0,1]$. Therefore, for any strategy $s \in \Scal_0$ and any highest-competing-bid distribution $H \in \Delta([0,1])$, we have
	\begin{align*}
		\U[\tilde F]{s, H} &= \mathbb{E}_{(v, h) \sim \tilde F \times H}  \left[ \mathbb{E}_{b \sim s(v)} \left[ u(b,h;v) \right] \right]\\
		&= \mathbb{E}_{(v, h) \sim \tilde F \times H}  \left[ \mathbb{E}_{b \sim s(v)} \left[ u(b,h;v) \right] \cdot \mathbbm{1}(v > 0) \right]\\
		&= \mathbb{E}_{h \sim H}\left[ \E_{v \sim \tilde F}  [ \mathbb{E}_{b \sim s(v)} \left[ u(b,h;v) \right] \mid v > 0] \cdot \Prob(v > 0) \right]\\
		&= (1 - a) \cdot \mathbb{E}_{h \sim H}\left[ \E_{v \sim \tilde F}  [ \mathbb{E}_{b \sim s(v)} \left[ u(b,h;v) \right] \mid v > 0]] \right]\\
		&= (1 - a) \cdot \mathbb{E}_{h \sim H}\left[ \E_{v \sim F} [ \mathbb{E}_{b \sim s(v)} \left[ u(b,h;v) \right] \right]\\
		&= (1 - a) \cdot \U[F]{s, H}\,.
	\end{align*}
	
	Since 0 is the optimal bid for value 0 regardless of the highest-competing-bid distribution, we must have
	\begin{align*}
		\sup_{s' \in \Scal} \U[\tilde F]{s',H} = \sup_{s' \in \Scal_0} \U[\tilde F]{s', H} = (1 - a) \cdot \sup_{s' \in \Scal_0} \U[F]{s', H} = (1 - a) \cdot \sup_{s' \in \Scal} \U[F]{s',H}\,.
	\end{align*}
	
	Therefore, for every $H$ and $s \in \Scal_0$, we have
	\begin{align*}
		\reg_{\tilde F}(s, H) = \sup_{s' \in \Scal} \U[\tilde F]{s',H} - \U[\tilde F]{s,H} = (1 -a) \cdot \left\{ \sup_{s' \in \Scal} \U[F]{s',H} - \U[F]{s,H} \right\} = (1 -a)\cdot \reg_F(s, H).
	\end{align*}
	Taking a supremum over $H$ yields
	\begin{align*}
		\sup_{H \in \Delta([0,1])} \reg_{\tilde F}(s, H) = (1 -a)\cdot \sup_{H \in \Delta([0,1])} \reg_F(s, H) \qquad \forall s \in \Scal_0\,.
	\end{align*}
	
	Now, for any strategy $s \in \Scal$, if we define $\tilde s \in \Scal_0$ as $\tilde s(0) = 0$ with probability 1 and $\tilde s(v) = s(v)$ for all $v > 0$, then $\U[\tilde F]{s, H} \leq \U[\tilde F]{\tilde s, H}$ because zero is the optimal bid for the value $v = 0$ against all $h \sim H$, and any other bid can only do worse. Therefore, we have
	\begin{align*}
		\inf_{s \in \Scal} \sup_{H \in \Delta([0,1])} \reg_{\tilde F}(s, H) = \inf_{s \in \Scal_0} \sup_{H \in \Delta([0,1])} \reg_{\tilde F}(s, H) = (1 - a) \cdot \inf_{s \in \Scal_0} \sup_{H \in \Delta([0,1])} \reg_{F}(s, H)\,.
	\end{align*}
	\Cref{cor:partial-info-saddle} implies that $\s[Q^*]$ is an optimal strategy for the minimax problem given in the last term. As a consequence, we have
	\begin{align*}
		\inf_{s \in \Scal} \sup_{H \in \Delta([0,1])} \reg_{\tilde F}(s, H) = (1 - a) \cdot \sup_{H \in \Delta([0,1])} \reg_{F}(\s[Q^*], H)\,.
	\end{align*}
	Finally, the definitions of $\s[Q^*]$ and $\tilde s_{Q^*}$ imply $\U[F]{\tilde s_{Q^*}, H} = \U[F]{s_{Q^*}, H}$ for all $H \in \Delta([0,1])$. Hence, we get
	\begin{align*}
		\sup_{H \in \Delta([0,1])} \reg_{\tilde F}(\tilde s_{Q^*}, H) = (1 -a) \cdot \sup_{H \in \Delta([0,1])} \reg_{F}(\tilde s_{Q^*}, H)
		= (1 -a) \cdot \sup_{H \in \Delta([0,1])} \reg_{F}(s_{Q^*}, H)\,,
	\end{align*}
	which completes the proof.
\end{proof}




























\section{Uniform-bid-shading}\label{sec:uniform-bid-shading}


In this section, we provide a stronger characterization of the worst-case regret for uniform-bid-shading strategies under additional assumptions on the value distribution. Namely, we prove the following result.
\begin{proposition}\label{prop:uniform-bid-shading}
	Consider a value distribution $F$ with a density $f:[0,1] \to [0,1]$ such that the map $t \mapsto t\cdot f(t)$ is non-decreasing. Then, for any shading factor $\alpha \in [0,1]$,
	\begin{align*}
		\sup_{H \in \Delta([0,1])}\ \reg_F(s_\alpha, H)\ =\ \max\left\{\alpha \cdot \E_{v \sim F}[v],\ \E_{v\sim F}\left[(v - \alpha \cdot F^-(1))^+ \right]\right\}\,.
	\end{align*}
\end{proposition}
We note that the assumption on the value distribution holds for any distribution $\unif(1 - \tfrac{1}{\rho}, 1)$, with $\rho >0$. This family of value distributions is particularly significant, as it arises as the worst-case scenario when evaluating the impact of informational asymmetries in bidding environments (see \Cref{sec:value-dist-impact}).


\begin{proof}[\textbf{Proof of \Cref{prop:uniform-bid-shading}}]
	As $F$ has a density, it is absolutely continuous, and as a consequence, so is the distribution of bids $\alpha \cdot v$ under $s_\alpha$. Therefore, \Cref{thm:evaluation} applies, and we get
	\begin{align*}
		\sup_{H \in \Delta([0,1])}\ \reg_F(s_\alpha, H)\ =\ \E_{v \sim F}[(v - h) \cdot \mathbbm{1}(v \geq h)]\ - \U[F]{s_\alpha, \delta_h} \,.
	\end{align*}
	
	Consider the mapping $r : h 
     \mapsto \E_{v \sim F}[(v - h) \cdot \mathbbm{1}(v \geq h)]\ - \U[F]{s_\alpha, \delta_h}$. For every $h \in [0, \alpha \cdot F^{-}(1)]$, we have
	\begin{align*}
	 r(h) \ &=\ \E_{v \sim F}[(v - h) \mathbbm{1}(v \geq h)]\ -  \E_{v \sim F}[(v - \alpha \cdot v) \mathbbm{1}(\alpha \cdot v \geq h)]\\
		&=\ \int_h^1 (v - h) \cdot f(v) dv\ -\ \int_{h/\alpha}^1 (1 - \alpha) \cdot v \cdot f(v)  dv\\
		&=\ (1 - h)F(1) - (h-h)F(h)\ - \int_h^1 F(v)dv\ -\ \int_{h/\alpha}^1 (1 - \alpha) \cdot v \cdot f(v)  dv\\
		&=\ \int_h^1(1 - F(v))dv\ - \int_{h/\alpha}^1 (1 - \alpha) \cdot v \cdot f(v)  dv\,.
	\end{align*}
	Therefore, we get
	\begin{align*}
		r'(h)\ =\ (1 - \alpha) \cdot \frac{h}{\alpha} \cdot f\left( \frac{h}{\alpha}\right)\ -\ (1 - F(h))\,.
	\end{align*}
	As $t \mapsto t \cdot f(t)$ is assumed to be non-decreasing, and $t \mapsto 1 - F(t)$ is non-increasing for all distributions, we get that $r'(\cdot)$ is non-decreasing for $h \in [0, \alpha \cdot F^-(1)]$. Or equivalently, $r(\cdot)$ is a convex function for $h \in [0, \alpha \cdot F^-(1)]$. Moreover, $r(\cdot)$ is also continuous. Therefore, Bauer Maximum Principle (see 7.69 of \citealt{aliprantis2006infinite}) applies and we get
	\begin{align*}
		\sup_{h \in [0, \alpha \cdot F^-(1)]}\ r(h) \ =\ \max\{r(0), r(\alpha \cdot F^-(1))\}\,.
	\end{align*}
	Since $s_\alpha(v) \in [0,\alpha \cdot F^-(1)]$ for all values $v \in [0,1]$, we have for all $h > \alpha \cdot F^-(1)$ that,
	\begin{align*}
		r(h) \ &= \ \E_{v \sim F}[(v - h) \cdot \mathbbm{1}(v \geq h)]\ - \U[F]{s_\alpha, \delta_h}\\
        &=\ \E_{v \sim F}[(v - h) \cdot \mathbbm{1}(v \geq h)]\\
        &\geq\ \E_{v \sim F}[(v - \alpha \cdot F^{-}(1)) \cdot \mathbbm{1}(v \geq \alpha \cdot F^{-}(1))]\\
        &= \ \E_{v \sim F}[(v - \alpha \cdot F^{-}(1)) \cdot \mathbbm{1}(v \geq \alpha \cdot F^{-}(1))] \ - \ \U[F]{s_\alpha, \alpha \cdot F^-(1)} \\ 
        &=\ r \left( \alpha \cdot F^{-}(1) \right) \,.
	\end{align*}
	Therefore, $r(h) \leq r(\alpha \cdot F^-(1))$ for all $h > \alpha \cdot F^-(1)$. Altogether, we get
	\begin{align*}
		\sup_{H \in \Delta([0,1])}\ \reg_F(s_\alpha, H)\ =\ \sup_{h \in [0,\alpha \cdot F^-(1)]}\ r(h) \ =\ \max\{  r(0), r(\alpha \cdot F^-(1))\}\,.
	\end{align*}
	
	Finally, observe that
	\begin{align*}
		r(0)\ =\ \E_{v \sim F}[v] - \E_{v \sim F}[v - \alpha \cdot v]\ =\ \alpha \cdot \E_{v \sim F}[v]\,,
	\end{align*}
	and
	\begin{align*}
		r( \alpha \cdot F^-(1))\ =\ \E_{v \sim F}[(v - \alpha \cdot F^-(1)) \mathbbm{1}(v \geq \alpha\cdot F^-(1))] - 0\ =\ \E_{v\sim F}\left[(v - \alpha \cdot F^-(1))^+ \right]\,.\ &\qedhere
	\end{align*}
\end{proof}
