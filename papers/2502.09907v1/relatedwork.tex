\section{Related Work}
Starting from the seminal work of \citet{vickrey1961counterspeculation}, first-price auctions have received significant attention in the literature. Most of this work has focused on the equilibrium analysis of multi-buyer interactions. In contrast, our focus is on developing bidding strategies for an individual buyer which are robust to the behavior of the competition, without regard for how the competition arrives at that behavior. Therefore, we do not review the vast literature on the traditional equilibrium analysis and refer the reader to standard texts~\citep{krishna2009auction, milgrom2004putting}.

The closest works to ours are the recent ones of \citet{kasberger2023robust} and \citet{qu2024double}. \citet{kasberger2023robust} study robust bidding in first-price auctions, with the goal of providing practical guidance for real-life auctions. They provide comprehensive evidence on the need to go beyond traditional analyses in the form of surveys, laboratory data and empirical analysis. On the technical front, they consider a buyer with a fixed value who is uncertain about the bids of others, and construct deterministic bids which achieve low worst-case regret with respect to the uncertainty in competing bids. Beyond modeling uncertainty directly in the highest competing bid, they also consider models with higher order information about how the competing bids are generated. We focus on uncertainty in the highest competing bids and extend their results along two dimensions: (i) we allow for randomized bidding strategies; (ii) we allow the value to be random, measure regret in expectation over this randomness, and characterize minimax-optimal strategy for every value distribution. Both extensions create significant challenges by replacing finite-dimensional optimizations with infinite-dimensional ones. \citet{qu2024double} adopts a distributionally-robust approach. They optimize a single bid to maximize worst-case expected utility over value and highest-competing-bid distributions lying within a Kullback-Leibler ball around empirical estimates. In contrast, we assume that buyer knows her own value and can alter the bid based on the value, which results in an action space consisting of bidding strategies instead of a single bid. Moreover, we do not restrict the highest-competing-bid to lie some known neighborhood, and we use regret as our metric, which, unlike absolute utility, is not linear in the highest-competing-bid distribution.

Another line of work develops learning algorithms for a buyer participating in repeated first-price auctions, either in stochastic settings \citep{han2020optimal,balseiro2022contextual,badanidiyuru2023learning,schneider2024optimal} or adversarial ones \citep{han2020learning,zhang2022leveraging,kumar2024strategically}. In these works, the benchmark for regret is the optimal fixed strategy in hindsight. Although we do not explicitly model repeated auctions or learning dynamics, our results imply a regret guarantee against the optimal \emph{sequence of strategies} in hindsight without making any assumptions on the environment. Our minimax-strategy serves as a natural choice for settings with a high degree of uncertainty and churn that make learning impossible. It can also be used to warm-start learning algorithms to improve performance early on. Finally, our work contributes to the literature on decision-making under uncertainty via distributionally-robust regret. This approach provides structural insights into robust decision-making when faced with limited information, and has been applied in various contexts, such as robust pricing \citep{bergemann2011robust}, inventory management \citep{perakis2008regret}, auction design \citep{anunrojwong2022robustness, anunrojwong2023robust}, and bidding \citep{kasberger2023robust}.