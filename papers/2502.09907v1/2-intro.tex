\section{Introduction}\label{sec:intro}
First-price auctions are a dominant mechanism across various markets, playing a central role in how goods and services are allocated. This time-tested mechanism has gained renewed interest following its recent adoption by the display advertising industry, replacing second-price auction. Unlike second-price auctions, where truthfully bidding one's value is a dominant strategy, i.e., it is optimal regardless of the competitor’s actions, first-price auctions require bidders to engage in far more complex strategic considerations. Thus, this change has brought to the forefront fundamental questions about how buyers should navigate these strategic intricacies to maximize their utility. Our work takes the perspective of such a buyer, and develops a principled approach to answering these questions.


The primary challenge a buyer faces in first-price auctions is that the outcome of her bid depends heavily on the competing bids she will encounter. Thus, a buyer must anticipate how others will bid in order to determine her own bid, often without knowing the competitors' values or strategies. Traditional game-theoretic approaches to the design of bidding strategies rely on strong assumptions like common knowledge and equilibrium behavior, which fail to hold in real-life settings~\citep{kasberger2023robust}. This leaves open the practical problem of coming up with good strategies despite only having limited knowledge about the competition. To address this problem, we take a prior-independent approach to the design of bidding strategies for first-price auctions. Rather than making assumptions about the behavior of competing buyers, we seek to develop a principled methodology for evaluating, comparing, and designing bidding strategies that perform well across all possibilities. Our approach is inspired by distributional robust paradigms, where the objective is to ensure strong bidding performance under minimal distributional assumptions. By doing so, we aim to provide a framework that is both theoretically sound and practically relevant for buyers who must make bidding decisions in complex and uncertain market conditions.


\subsection{Main Contributions}

We consider a buyer participating in a first-price auction who wishes to design a bidding strategy---a mapping from her private value to a (potentially random) bid---that maximizes her expected utility. 
We adopt a prior-independent approach which does not make any assumptions on the competing bids, and instead aims to optimize performance uniformly against all possible competing bids. To jointly evaluate the performance across all possibilities, we use the standard metric of worst-case regret \citep{savage1951theory}. In particular, for any distribution of the highest competing bid, we define regret to be the difference between the expected utility achievable by the oracle who knows this distribution and the one generated by the strategy of interest. We focus on the highest competing bid as it is a sufficient statistic that completely determines the utility generated by any strategy. The worst-case regret of a strategy then is the maximum regret it incurs across all possible highest-competing-bid distributions. It captures the loss incurred by the strategy due to a lack of knowledge about competing bids, which is a core concern in first-price auctions. A small worst-case regret implies that the strategy does not incur a large loss due this lack of information no matter how the competing bids are determined, thereby circumventing the challenging task of accurately understanding and predicting the behavior of competitors.


\textbf{Performance evaluation.} Before minimizing worst-case regret, one must tackle the task of evaluating it for a given bidding strategy. The infinite-dimensional nature of bidding strategies and highest-competing-bid distributions makes this task challenging. These difficulties are further exacerbated by the fact that our benchmark---utility achievable by the oracle who knows the highest-competing-bid distributions---is itself the value of an infinite-dimensional optimization problem, making even the task of characterizing regret for a fixed highest-competing-bid distribution difficult. In \Cref{thm:evaluation}, we show that the problem of evaluating worst-case regret can be considerably simplified for a wide class of strategies and value distributions: the dimension of the underlying optimization problem can be reduced from infinity to one. Firstly, from a computational perspective, it reduces an a priori intractable problem to a remarkably simple one that can be solved with a line search. Secondly, this result has an insightful economic interpretation: we show that the worst-case highest-competing-bid distribution is always a deterministic one. In other words, when designing strategies with small worst-case regret, one can focus on deterministic highest competing bids and ignore their potential for random variation. We leverage \Cref{thm:evaluation} to compare different uniform-bid-shading strategies, an important class of practical bidding strategies that bid $\alpha \cdot v$ when the buyer's value is $v$, for some fixed $\alpha \in [0,1]$. In particular, we show that the uniform-bid-shading strategy with $\alpha = 0.5$, which emerges in \citet{kasberger2023robust} when the buyer robustly optimizes the worst-case regret for each value in isolation, can be considerably improved upon by accounting for the distribution of values and choosing the optimal $\alpha$ for it. For instance, when the value distribution is uniform, we show that the best choice of $\alpha$ is $0.38$, and this choice yields a worst-case regret that is $20\%$ lower than the common choice of $0.5$. 


\begin{figure}[h!]
    \centering
    \begin{tikzpicture}[scale = 0.65]
            \begin{axis}[
        width=10cm,
        height=10cm,
        xmin=0,xmax=1,
        ymin=0,ymax=1,
        table/col sep=comma,
        xlabel={$v$},
        ylabel={},
        grid=both,
        legend pos=south east,
        %legend style ={font ={\footnotesize}}
    ]

    \addplot [blue, line width = 0.7mm] {x}; 
    \addlegendentry{Value CDF}

    \addplot [violet,  line width=0.7mm, select coords between index={1}{79}] table[x={Qs_alpha=1.0},y={x_alpha=1.0}] {Data/plot_bidding_strategy_beta.csv};
    \addlegendentry{Bid CDF}
    
    \addplot [Circle-Latex, very thick,  black] coordinates { (0.28,0.795) (0.35,0.86)};

    \draw[dashed, thick, black] (axis cs:0.29,0.29) -- (axis cs:0.29,1);
    \draw[dotted, very thick, black] (axis cs:0.29,0.8) -- (axis cs:0.8,0.8);
    

    
    \end{axis}
    \end{tikzpicture}
    \caption{\textbf{ODE which constructs a minimax-optimal bidding strategy for the uniform value distribution.} If one starts at (0,0) and moves with a slope equal to the ratio of the dashed line to dotten line, then the resulting curve will trace out the CDF of bids under a minimax-optimal bidding strategy. The strategy is then to simply bid the corresponding quantile for every value, i.e., bid $b$ for value $v$ if and only if the quantiles of $b$ and $v$ are equal under the bid and value CDFs respectively.}
    \label{fig:intro_ode}
\end{figure}


\textbf{Minimax-optimal bidding strategy.} Having developed an efficient method for evaluating worst-case regret, we next turn towards optimizing it. We provide a complete characterization of minimax-optimal regret, which is the smallest-possible worst-case regret achievable by any bidding strategy, and do so for \emph{every} value distribution. Our characterization takes the form of an explicit construction of a saddle point for the underlying minimax optimization problem using ordinary differential equations (ODEs). On the technical front, this requires multiple advances. First, we leverage our result on performance evaluation to simplify the benchmark from the optimal utility achievable with knowledge of the highest-competing-bid distribution to that achievable with knowledge of its realization. Moreover, we simplify the space of bidding strategies via a reformulation that assigns a deterministic bid to each quantile of the value distribution instead of a random bid to each value. Then, we consider the first-order optimality conditions of this reformulation, and analyze the resulting ordinary differential equations. Our primary technical contributions pertain to the analysis of these ODEs and addressing the associated challenges: (i)~the worst-case-regret minimization problem is parameterized by the value distribution of the buyer and so are the ODEs; (ii)~discontinuities in the value distribution manifest as discontinuities in the ODEs; (iii)~the ODEs have ill-conditioned denominators prone to divergence and they are not even well-defined everywhere. To establish our main result (\Cref{thm:main-result}), we navigate these hurdles to characterize the saddle point as a solution to these ODEs. It yields an efficient procedure for constructing minimax-optimal bidding strategies for arbitrary value distributions; \Cref{fig:intro_ode} illustrates it for the uniform value distribution. Once the minimax-optimal strategy has been constructed, we show that the corresponding value of optimal regret is given by a simple integral. Altogether, our characterization provides an efficient technique for solving the intricate infinite-dimensional minimax optimization problem that arises in the prior-independent setting.

\begin{figure}[h!]
    \centering
\begin{tikzpicture}[scale = 0.65]
    \begin{axis}[
        width=10cm,
        height=10cm,
        xmin=0,xmax=1,
        ymin=0,ymax=0.5,
        table/col sep=comma,
        xlabel={$a$},
        ylabel={Worst-case Regret},
        grid=both,
        legend pos=north west,
        %legend style ={font ={\footnotesize}}
    ]

    \addplot [blue,  line width=0.7mm, ,mark=square,mark options={scale=.1}] table[x=a,y={regret}] {Data/regrets_uniform_a.csv};
    \addlegendentry{Minimax strategy}
    \addplot [black, line width = 0.7mm] {0.25*x + 0.25};  % Plots the line y = 2x + 1
    %\draw[black,line width = 0.7mm, domain=0:200, smooth, variable=\x, black] plot ({\x}, {2.5*\x + 150});
   \addlegendentry{$b(v) = 0.5 \cdot v$}
      \addlegendimage{ultra thick,black}
    \end{axis}
    \end{tikzpicture}
    \caption{\textbf{Summary of our insights.} For every $a$, the blue curve represents the largest worst-case regret incurred by our minimax-optimal bidding strategy across all value distributions with a density bounded above $\frac{1}{1-a}$. The black curve corresponds to the worst-case regret of the bidding strategy from \citet{kasberger2023robust} which bids $0.5 \cdot v$ for every value $v$, when the value distribution is a uniform on $[a,1]$.}
    \label{fig:intro_impact_value}
\end{figure}


\textbf{Structural Insights.} The procedure we develop for constructing minimax-optimal bidding strategies delivers benefits beyond computational tractability. It allows us to glean structural insights about the drivers of performance loss, as measured by regret. First, when the value distribution is continuous, our construction yields a \emph{deterministic} minimax-optimal bidding strategy. In contrast, deterministic strategies are sub-optimal when the value is deterministic. In this case, the problem is equivalent to robust pricing \citep{bergemann2011robust}, inheriting the associated sub-optimality of deterministic strategies. Since any distribution can be perturbed ever so slightly to arrive at a continuous one, our result implies the optimality of deterministic strategies for a large class of value distributions. In particular, even though deterministic strategies are sub-optimal when the buyer's value is deterministic and known with certainty, they immediately jump to optimality in the presence of even an infinitesimal amount of (continuous) random noise in the value estimate. Furthermore, we characterize the impact of the value distribution on minimax regret. Specifically, we show in \Cref{thm:worst-value-dist} that the uniform distribution on $[1 - \tfrac{1}{\rho}, 1]$ yields the highest minimax-optimal regret among all value distributions with a density bounded above by $\rho \geq 1$. This allows us to evaluate minimax regret as a function of how concentrated the values are: we find that a greater dispersion in values leads to lower minimax-optimal regret; see \Cref{fig:intro_impact_value}. Taken together, these insights indicate that even a small amount of variation in values is often sufficient to render deterministic strategies optimal, and the performance of the optimal policy improves with the amount of variation. Hence, the presence of random private information in the form of values, which is known to the buyer but not the competition, obviates the need to hedge bids with randomization. Moreover, the optimal achievable performance improves with the magnitude of randomness.




\subsection{Related Work}

Starting from the seminal work of \citet{vickrey1961counterspeculation}, first-price auctions have received significant attention in the literature. Most of this work has focused on the equilibrium analysis of multi-buyer interactions. In contrast, our focus is on developing bidding strategies for an individual buyer which are robust to the behavior of the competition, without regard for how the competition arrives at that behavior. Therefore, we do not review the vast literature on the traditional equilibrium analysis and refer the reader to standard texts~\citep{krishna2009auction, milgrom2004putting}.

The closest works to ours are the recent ones of \citet{kasberger2023robust} and \citet{qu2024double}. \citet{kasberger2023robust} study robust bidding in first-price auctions, with the goal of providing practical guidance for real-life auctions. They provide comprehensive evidence on the need to go beyond traditional analyses in the form of surveys, laboratory data and empirical analysis. On the technical front, they consider a buyer with a fixed value who is uncertain about the bids of others, and construct deterministic bids which achieve low worst-case regret with respect to the uncertainty in competing bids. Beyond modeling uncertainty directly in the highest competing bid, they also consider models with higher order information about how the competing bids are generated. We focus on uncertainty in the highest competing bids and extend their results along two dimensions: (i) we allow for randomized bidding strategies; (ii) we allow the value to be random, measure regret in expectation over this randomness, and characterize minimax-optimal strategy for every value distribution. Both extensions create significant challenges by replacing finite-dimensional optimizations with infinite-dimensional ones. \citet{qu2024double} adopts a distributionally-robust approach. They optimize a single bid to maximize worst-case expected utility over value and highest-competing-bid distributions lying within a Kullback-Leibler ball around empirical estimates. In contrast, we assume that buyer knows her own value and can alter the bid based on the value, which results in an action space consisting of bidding strategies instead of a single bid. Moreover, we do not restrict the highest-competing-bid to lie some known neighborhood, and we use regret as our metric, which, unlike absolute utility, is not linear in the highest-competing-bid distribution.

Another line of work develops learning algorithms for a buyer participating in repeated first-price auctions, either in stochastic settings \citep{han2020optimal,balseiro2022contextual,badanidiyuru2023learning,schneider2024optimal} or adversarial ones \citep{han2020learning,zhang2022leveraging,kumar2024strategically}. In these works, the benchmark for regret is the optimal fixed strategy in hindsight. Although we do not explicitly model repeated auctions or learning dynamics, our results imply a regret guarantee against the optimal \emph{sequence of strategies} in hindsight without making any assumptions on the environment. Our minimax-strategy serves as a natural choice for settings with a high degree of uncertainty and churn that make learning impossible. It can also be used to warm-start learning algorithms to improve performance early on. Finally, our work contributes to the literature on decision-making under uncertainty via distributionally-robust regret. This approach provides structural insights into robust decision-making when faced with limited information, and has been applied in various contexts, such as robust pricing \citep{bergemann2011robust}, inventory management \citep{perakis2008regret}, auction design \citep{anunrojwong2022robustness, anunrojwong2023robust}, and bidding \citep{kasberger2023robust}. 


