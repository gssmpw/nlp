

\section{Minimax-Optimal Bidding Strategy}\label{sec:minimax}

Directly computing a minimax-optimal strategy by solving \eqref{eq:minimax_strategy_problem} is inherently challenging. Beyond the challenges discussed in \Cref{sec:strat_evaluation} for evaluating the worst-case regret  $\mathrm{WReg}_F(s)$ of a fixed strategy $s$, we now face the additional challenge of minimizing worst-case regret over randomized bidding strategies $s \in \Scal$. Each such strategy defines an uncountable family of distributions $\{s(v) \in \Delta([0,1]) \mid v \in [0,1] \}$ over bids, further complicating the problem. 
In order to address the complexities arising from these nested infinite-dimensional optimization problems, we proceed in multiple steps, each aimed at simplifying the overall minimax optimization to make it tractable. 

In \Cref{sec:full-info}, we build on the reduction established in \Cref{thm:evaluation} to reformulate the worst-case regret problem as an alternative maximization problem. This new formulation leverages a full-information benchmark with a significantly simpler utility expression than the one used in \eqref{eq:regret}.
This result allows us to considerably simplify the inner maximization problem of \eqref{eq:minimax_strategy_problem}, by simplifying both the space of distributions of highest competing bids and the expression of the regret objective.
 In \Cref{sec:quantile-strat}, we introduce the class of quantile-based bidding strategies which specify a single bid for each \emph{quantile}, instead of a distribution of bids for each value. The resulting re-parametrization significantly simplifies the space of bidding strategies 
 and allows us to represent bidding strategies using a single function from $[0,1]$ to itself---a stark contrast to the uncountable family of bid distributions required to specify a general bidding strategy.
In \Cref{sec:saddle_point}, we leverage this simplification to construct a quantile-based bidding strategy $Q^*$ and a highest-competing-bid distribution $H^*$ which form a candidate saddle-point for our problem. The construction is based on ordinary differential equations that capture appropriate first-order-optimality conditions.
In \Cref{sec:main-result}, we establish our main result (\Cref{thm:main-result}) by proving  that our candidate $(Q^*,H^*)$ does in fact form a saddle-point for \eqref{eq:minimax_strategy_problem}. 

Throughout this section, we make the assumption that the value distribution satisfies $F(0) = 0$, i.e., the probability of the value being zero is zero. It is made purely for ease of exposition, and all our results readily extended to the general case, as we show in \Cref{appendix:atom-at-0}.

\subsection{Reduction to the Full-Information Benchmark}\label{sec:full-info}

Recall that the regret $\reg_F(s,H)$ of a bidding strategy $s$ against the highest-competing-bid distribution $H$ measures its utility loss against the benchmark $\sup_{s'} \U[F]{s',H}$. This benchmark is based on partial information and captures the optimal utility that can be attained with knowledge of the distribution $H$, but not the realization of the highest competing bid $h$. One can alternatively consider the stronger full-information benchmark that also has knowledge of the highest competing bid $h$ that was realized. In particular, for a given highest-competing-bid distribution $H$, the \emph{full-information benchmark} is defined as
\begin{align*}
	\O_F(H)\ \coloneqq\ \E_{h \sim H}\left[ \max_{s' \in \Scal}\  \U[F]{s', h} \right].
\end{align*}
 Observe that, if the highest competing bid $h$ is known, then the utility-maximizing strategy is to simply bid $h$ whenever the value $v$ is greater than or equal to it. Thus, the full-information benchmark can be rewritten as
\begin{align*}
	\O_F(H)\ =\ \E_{h \sim H}\left[ \max_{s' \in \Scal}\  \U[F]{s',h} \right] = \E_{(v,h) \sim F \times H}\left[ (v-h) \cdot \mathbbm{1}(v \geq h) \right]\,.
\end{align*}
Based on the full-information benchmark, one can define an alternative notion of regret
\begin{align*}
	\rr[F]{s,H}\ \coloneqq \O_F(H)\ -\ \U[F]{s,H}\,. 
\end{align*}

With a slight abuse of notation, we use $\O_F(h)$ and $\rr[F]{s,h}$ to represent $\O_F(\delta_h)$ and $\rr[F]{s,\delta_h}$ respectively, where $\delta_h$ is the Dirac distribution. The following result follows from \Cref{thm:evaluation} and show that for every bidding strategy $s$ which induces an absolutely continuous bid distribution $P_{s,F}$, the worst-case regret is identical under both partial-information and full-information benchmarks. %Moreover, it shows that this worst-case regret is incurred against a deterministic highest competing bid $h$.
\begin{corollary}\label{thm:full-info}
	For any value distribution $F \in \Delta([0,1])$ and any bidding strategy $s \in \Scal$ which induces an absolutely continuous bid distribution $P_{s,F}$, we have
	\begin{align*}
		\sup_{H \in \Delta([0,1])} \reg_F(s,H)\ =\ \sup_{h \in [0,1]}\ \rr[F]{s, h}\ =\ \sup_{H \in \Delta([0,1])}\ \rr[F]{s, H}\,.
	\end{align*}
\end{corollary}
\Cref{thm:full-info} directly follows from \Cref{thm:evaluation} by noting that, for every $h \in [0,1]$, we have 
\begin{equation*}
    \rr[F]{s,h} = \E_{v \sim F}\left[ (v-h) \cdot \mathbbm{1}(v \geq h) \right]-\ \U[F]{s,\delta_h}.
\end{equation*}
Moreover, we remark that for every $H \in \Delta([0,1])$ we have that $\rr[F]{s,H} = \E_{h \sim H}[\rr[F]{s,h}]$. Therefore, $\sup_{h \in [0,1]}\ \rr[F]{s, h}\ =\ \sup_{H \in \Delta([0,1])}\ \rr[F]{s, H}$, which establishes the second equality.

\Cref{thm:full-info} has a powerful interpretation, as it implies that the robust value of the information associated with the knowledge of the \emph{distribution} of the highest competing bid is equal to the one associated with knowledge of its \emph{realization}.
In what follows, we use \Cref{thm:full-info} to reduce the problem of finding a minimax-optimal bidding strategy for the partial-information benchmark defined in \eqref{eq:minimax_strategy_problem} to that of finding one for of the simpler full-information benchmark. Namely, we now aim at solving
\begin{equation}
    \label{eq:minimax_full_info}
    \inf_{s \in \mathcal{S}} \sup_{h \in [0,1]} \rr[F]{s,h}.
\end{equation}
We note that this reduction simplified not only the optimization space in the inner supremum but also the objective itself. 

\subsection{Quantile-Based Strategies}\label{sec:quantile-strat}

Next, we further simplify the minimax problem \eqref{eq:minimax_full_info} by reducing the strategy space of the buyer from the set $\Scal$ of arbitrary randomized strategies $s: [0,1] \to \Delta([0,1])$ to simpler \emph{quantile-based strategies}, with the latter being parameterized by a single function $Q:[0,1] \to [0,1]$. The primary goal of this section is to describe this reformulation and build intuition for it. Accordingly, we first provide informal arguments to motivate our reformulation, before formally defining quantile-based bidding strategies in \Cref{def:quantile-based-bidding-strategy}.



Note that, for any fixed highest-competing-bid distribution $H$, the regret-minimization problem
\begin{align*}
	\inf_{s \in \Scal}\ \rr[F]{s,H}\ =\ \inf_{s \in \Scal}\ \O_F(H)\ -\ \U[F]{s,H}\ =\ \O_F(H)\ -\ \sup_{s \in \Scal}\ \U[F]{s,H}
\end{align*}
boils down to the utility-maximization problem $\sup_{s \in \Scal}\ \U[F]{s,H}$ for the buyer. And therefore, we can leverage the following intuitive observation about utility-maximizing strategies in first-price auctions: higher values result in higher optimal bids. In other words, the optimal-bid correspondence
\begin{equation*}
    v \mapsto {\mathbb S(v) \coloneqq \argmax_{b \in [0,1]} \ (v-b)\cdot H(b)}
\end{equation*}
is increasing in the value $v$, i.e., $\mathbb S(v_1) \succcurlyeq \mathbb S(v_2)$ for all $v_1 \geq v_2$. 
The proof of this monotonicity result is a standard application of monotone comparative statics and various forms of it have been proven in prior work (see, for example, \citealt{kumar2024strategically}). We do not directly use it in our proof, but instead treat it as a guiding principle for our reformulation.

Let $\ell(\mathbb S(v)) = \sup \mathbb S(v) - \inf \mathbb S(v)$ denote the ``length'' of the set of optimal bids for value $v$, and observe that $\sum_{v \in [0,1]} \ell(\mathbb S(v)) \leq 1$. As the sum of any uncountable collection of positive integers necessarily diverges to infinity, we must have that the set $\{ v \mid \ell(\mathbb S(v))> 0\}$ is at most countable, and the remaining values have a unique optimal bid. If $F$ had no atoms, the probability of $v \sim F$ taking one of these countably-many values would be 0, and we could focus on deterministic bidding strategies for the remaining values. However, when $F$ contains atoms, we do not have this luxury. The following example illustrates this fact by showin that prior-independent pricing~\citep{bergemann2011robust} is a special case of our problem with a deterministic value.
\begin{example}\label{example:atom-at-1}
	Suppose the value is deterministic and equal to $1$. As there is only one value, the bidding strategy is simply the distribution of bids  $s(1)$ for value 1. For the highest-competing-bid distribution $H$, the expected regret is given by
    \begin{align*}
        \rr[F]{s, H}\ =\ \E_{h \sim H}[1 - h]\ -\ \E_{h \sim H, b \sim s(1)}\left[ (1 - b) \mathbbm{1}(b \geq h)\right]\,. 
    \end{align*}
    To see the equivalence to prior-independent pricing, define $\nu = 1-h$ to be the willingness-to-pay (WTP) and $p = 1 - b$ to be the price. Then, regret can be re-written as
    \begin{align*}
         \rr[F]{s, H}\ =\ \E_{(1- \nu) \sim H}[\nu]\ -\ \E_{(1 -\nu) \sim H, (1 -p) \sim s(1)}\left[ p \mathbbm{1}(\nu \geq p)\right]\,.
    \end{align*}
    The right-hand side is exactly the expected regret for the pricing problem when the WTP $\nu$ is distributed according to $(1 - \nu) \sim H$ and the price $p$ is distributed according to $(1 - p) \sim s(1)$. Thus, computing the minimax regret over $s$ and $H$ is equivalent to computing the minimax regret over all randomized pricing policies and WTP distributions. Hence, these two problems are equivalent and the sub-optimality of deterministic pricing policies established in \citet{bergemann2011robust} implies the sub-optimality of deterministic bidding strategies in this example. 
\end{example}

Fortunately (and surprisingly), it turns out that there are no other cases requiring randomization: we can focus on bidding strategies which bid randomly only for atoms of $F$, and bid deterministically for all other values. This is where we make our second crucial observation: quantiles provide a natural method to specify bids which are random for atoms of $F$, but deterministic for all other values. In particular, it is well known that $v = F^-(y)$ is distributed according to $F$ when the quantile $y \in [0,1]$ is uniformly distributed. Importantly, there are infinitely many quantiles $y \in [0,1]$ which result in $v = F(y)$ whenever $v$ is an atom, and there is exactly one such quantile for all other values; exactly as desired. Thus, if we specify a bid for each quantile $y \in [0,1]$, instead of each value, we naturally end up with a bidding strategy which is random for atoms and deterministic otherwise.

\begin{definition}\label{def:quantile-based-bidding-strategy}
	We refer to any strictly-increasing absolutely-continuous function ${Q: [0,1] \to [0,1]}$ as a quantile-based bidding strategy, and use $\mathcal Q$ to refer to the set of all such functions. For every $Q \in \mathcal Q$, we define the corresponding bidding strategy $\s[Q] \in \Scal$ as follows:
	\begin{align*}
		\s[Q](v)\ = \begin{cases}
			\delta_{Q(F(v))} &\text{if } \lambda(Y(v)) = F(\{v\}) = 0\\
			\{Q(y)\mid y \sim \unif(Y(v))\} &\text{if } \lambda(Y(v)) = F(\{v\}) >  0
		\end{cases}
	\end{align*}
	where $Y(v) \coloneqq \{y \in [0,1] \mid F^-(y) = v\}$ is the interval of quantiles which correspond to the value $v$.
\end{definition}

\begin{remark}\label{remark:Y-interval}
	Note that $Y(v)$ is the pre-image of $\{v\}$ for the non-decreasing function ${F^-: [0,1] \to [0,1]}$, and as a consequence it is an interval in $[0,1]$. Also, if $y \sim \unif(0,1)$, then $F^-(y) \sim F$~\citep{embrechts2013note}. Therefore, $\lambda(Y(v)) = F(\{v\})$, as noted in \Cref{def:quantile-based-bidding-strategy}.
\end{remark}

\begin{remark}\label{remark:kernel}
	$\kappa_{\s[Q]}$ is a Markov kernel and $\s[Q] \in \Scal$. See \Cref{appendix:minimax-remark} for details. 
\end{remark}


Hereafter, we focus our search for minimax-optimal policies to quantile-based bidding strategies $\mathcal Q$. In particular, we analyze the following regret-minimization problem:
\begin{equation*}
    \inf_{Q \in \mathcal{Q}}\ \sup_{h \in [0,1]}\ \rr[F]{\s[Q],h}
\end{equation*}




 Since the highest possible bid under $\s[Q]$ is $Q(1)$, setting $h > Q(1)$ only hurts the benchmark and not the utility of the buyer. In other words $\rr[F]{\s[Q],h} \leq \rr[F]{\s[Q],Q(1)}$ for all $h > Q(1)$. Thus, we can focus on $h \in [0,Q(1)]$. Alternatively, we can focus on $h = Q(y)$ for some $y \in[0,1]$. In this case, the benchmark can be written as
\begin{align*}
	\mathcal{O}_{F}(h)\ =\ \E_{v \sim F}\left[\mathcal(v-h)\mathbbm{1}(v \geq h) \right]\ =\ \int_{h}^1 (1 - F(t)) dt\ =\ \int_{Q(y)}^1 (1 - F(t)) dt
\end{align*}
and the expected utility of $\s[Q]$ can be written as
\begin{align*}
	\U[F]{\s[Q], h}\ =\ \E_{v \sim F}[ \E_{b \sim \s[Q](v)} [(v - b) \mathbbm{1}(b \geq h)]]\,.
\end{align*}

\begin{lemma}\label{lemma:quantile-based-bidding}
	For every quantile-based bidding strategy $Q \in \mathcal{Q}$ and highest competing bid $h = Q(y)$ with $y \in [0,1]$, we have
	\begin{align*}
		\U[F]{\s[Q], h}\ =\ \int_y^1 (F^-(t) - Q(t))dt\,.
	\end{align*}
    Moreover, the distribution of bids induced by $\s[Q]$ (and $F$) is $P_{\s[Q], F} = \{Q(t) \mid t \sim \unif(0,1)\}$.
\end{lemma}

We note that the joint distribution over values and bids is not a product distribution in general as, for any practical bidding strategy, the bid submitted by the buyer depends on the value. 
\Cref{lemma:quantile-based-bidding} allows us to express the expected utility with respect to the intricate joint distribution as a much simpler expectation over quantiles. In particular, the right-hand side is the expected utility under the distribution which ``couples'' the values $v = F^-(t)$ and the bids $Q(t)$ by generating them from the same uniform distribution $t \sim \unif(0,1)$. 


\Cref{lemma:quantile-based-bidding} allows us to reduce the minimax problem to the following simple form
\begin{equation}
\label{eq:simpler-saddle}
	\inf_{Q \in \mathcal{Q}}\ \sup_{y \in [0,1]}\ \int_{Q(y)}^1 (1 - F(t)) dt\ -\ \int_y^1 (F^-(t) - Q(t)) dt \,.
\end{equation}
%Before getting back to the proof where we leverage this reduction, we note a few of its important properties. 
Observe that restricting our attention to quantile-based strategies considerably simplifies the buyer's strategy space. Quantile-based bidding strategies can be specified with a single monotonic function $Q:[0,1] \to [0,1]$, whereas general bidding strategies $s: [0,1] \to \Delta([0,1])$ require an uncountable collection of monotonic functions, namely the CDFs of the bid distributions $\{s(v)\}_v$ for each value $v$. %This is a drastic reduction, and will prove helpful in our proofs. 
Furthermore, re-expressing the regret as a function of the quantile-based bidding strategy $Q$ allows us to obtain a formulation that is convex in $Q$ for any fixed $y$. Indeed, the function
\begin{align*}
	Q \longmapsto \int_{Q(y)}^1 (1 - F(t)) dt\ -\ \int_y^1 (F^-(t) - Q(t)) dt
\end{align*}
is convex, and taking a supremum over $y$ preserves this convexity. As such, the outer minimization problem \eqref{eq:simpler-saddle} corresponds to the minimization of a convex functional. In particular, local minima are also global minimax for convex functions, and the former can be found using first-order conditions---we leverage this observation in \Cref{sec:saddle_point} to construct a candidate minimax-optimal quantile-based bidding strategy.



\subsection{Saddle Point}\label{sec:saddle_point}


In light of the reformulation of the previous section, we now focus on solving the simplified minimax problem \eqref{eq:simpler-saddle}.
We do so by explicitly constructing a quantile-based bidding strategy $Q^*$ and a distribution of highest competing bids $H^*$ such that the pair $(Q^*, H^*)$ forms a saddle point, i.e.,
\begin{align*}
	\rr[F]{\s[Q^*], H^*} \leq \rr[F]{s, H^*} \quad \forall\ s \in \mathcal{S} \quad \text{and} \quad \rr[F]{\s[Q^*], H^*} \geq \rr[F]{\s[Q^*],H} \quad \forall\ H \in \Delta([0,1])\,.
\end{align*}

Intuitively, the minimax problem in \eqref{eq:simpler-saddle} is a convex-concave saddle-point problem and we would like to leverage this convex structure. However, as $\mathcal Q$ was defined to be the set of all strictly-increasing absolutely-continuous functions, it is not compact under standard topologies. This makes a direct application of results from infinite-dimensional convex optimization difficult. 
But all is not lost, the convex structure of the problem lends itself to a constructive proof using first-order optimality conditions. We leverage this fact to construct a candidate quantile-based bidding strategy $Q^*$ in \Cref{sec:construct_Q} and a candidate distribution of highest competing bids $H^*$ in \Cref{sec:construct_H}. 

\subsubsection{Candidate quantile-based bidding strategy}
\label{sec:construct_Q}

Fix a quantile-based bidding strategy $Q$ and consider the inner-maximization problem over quantiles $y \in [0,1]$ in \eqref{eq:simpler-saddle}:
\begin{align}
\label{eq:inner_max}
	\sup_{y \in [0,1]}\quad \int_{Q(y)}^1 (1 - F(t)) dt\ -\ \int_y^1 (F^-(t) - Q(t)) dt \,.
\end{align}

Looking forward, we would like all $y \in [0,1]$ to be optimal and have the same objective value. If all $y \in [0,1]$ have the same objective value, the derivative with respect to $y$ must be 0, which yields the following first-order condition:
\begin{align}\label{eq:first-ode}
	- (1 - F(Q(y))) \cdot Q'(y) \ +\ (F^-(y) - Q(y)) = 0 &&\forall\ y \in [0,1]\,.
\end{align}

This ODE will form a crucial part of our constructive characterization of the saddle point $(Q^*, H^*)$. 
Our next result establishes that it admits a well-behaved solution.

\begin{lemma}\label{lemma:ode-existence}
	For every value distribution $F$, there exists an absolutely continuous function\\ ${Q^*: [0,1] \to [0,1]}$ such that the following conditions hold:
	\begin{enumerate}
		\item First-order optimality: There exists a measurable set $A \subseteq [0,1]$ with measure $\lambda(A) = 1$ such that $Q^*$ is differentiable and $ Q^{*'}(y) \ =\ (F^-(y) - Q^*(y))/(1 - F(Q^*(y)))$  for all $y \in A$.
		\item Invertibility: $Q^*$ is strictly increasing, i.e., $Q^*(y_1) > Q^*(y_2)$ for all $y_1 > y_2$.
		\item Strict Boundedness: $0 < Q^*(y) < F^-(y)$ for all non-zero $y \in (0,1]$ and $Q^*(0) = 0$.
	\end{enumerate}
\end{lemma}
\Cref{lemma:ode-existence} shows the existence of a quantile-based bidding strategy $Q^*$ which satisfies the first order condition \eqref{eq:first-ode} almost surely; it will serve as our candidate optimal quantile-based bidding strategy. 
%The proof of \Cref{lemma:ode-existence} is ripe with technical challenges and 
\Cref{lemma:ode-existence} is one of our primary technical results; we provide an informal sketch of it here. 
Consider the ODE in \eqref{eq:first-ode} written in standard form,
\begin{align}
    \label{eq:main-first-ode-rewrite}
    x'(t) = g(t,x(t)) \quad \text{where}\quad g(t,x) \coloneqq \frac{F^-(t) - x}{1 - F(x)}\,, && \forall \,(t,x) 
    \in [0,1] \times [0,F^{-}(1))\,,
\end{align}
with the initial condition $x(0) = 0$. The denominator of $g$ raises the first challenge: as $x$ approaches $F^-(1)$, the denominator diverges to infinity and it is undefined beyond that point.
To the best of our knowledge, no existence theorems directly apply to such a setting. To address it, we leverage our first intuitive observation: whenever $x(t)$ approaches the $F^-(t)$ curve, the numerator of $g$ goes to zero and reduces its rate of approach $x'(t)$. This motivates us to conjecture the existence of a solution with $x(0) =0$ and $x(t) < F^-(t)$ for all $t > 0$. To prove it, we use a constant $\alpha > 0$ to modify $g$ and bound it from above:
\begin{equation*}
    x'(t) = g_\alpha(t,x(t)) \quad \text{where}\quad g_\alpha(t, x) = \frac{F^-(t) - x}{1 - F(\min\{x, \alpha\})}.
\end{equation*}
Observe that if, for some constant $\alpha > 0$, we are able to show the existence of a solution for this modified ODE with the property that $x(t) \leq \alpha$ for all $t \in [0,1]$, then such a solution would also be a solution to \eqref{eq:main-first-ode-rewrite}. We first establish the existence of a solution for this modified ODE. We remark that this result does not follow from standard existence theorems. Not only is the function $g_{\alpha}$ not guaranteed to be continuous, thereby ruling out the standard Peano's existence theorem, but even its restriction $x \mapsto g_\alpha(t,x)$ may be discontinuous for \emph{every} $t \in [0,1]$, thereby prima facie also ruling out Caratheodary's existence theorem. To get around this difficulty, we employ the existence theorem by \citet{biles1997caratheodory} and \citet{biles2000solvability} for discontinuous ODE which satisfy quasi-semicontinuity. It is a generalization of Caratheodary's theorem which relies on weaker continuity conditions that are satisfied by our use case.

With a solution $x_\alpha(t)$ in hand, we move onto the task of bounding it with $\alpha$. In fact we prove the stronger condition that $x_\alpha(t) < F^-(t)$ for all $t > 0$. To do so, we introduce the auxiliary ODE:
\begin{align*}
    x'(t) = \bar g(t,x(t)) \quad \text{where}\quad \bar g(t,x) \coloneqq \frac{F^-(t) \cdot (1 -x)}{1 - F(x)}\,, && \forall \,(t,x) 
    \in [0,1] \times [0,F^{-}(1))\,.
\end{align*}

Crucially, note that (i) this ODE is separable, which allows us to explicitly construct a solution $\bar x(t)$ for it, and (ii) it satisfies $\bar g(t, x) \geq g(t,x)$, which allows us to use its solution to bound the solution of the modified ODE, i.e., $x_\alpha(t) \leq \bar x(t)$. Therefore, to prove $x(t) < F^-(t)$, it suffices to show that $\bar x(t) < F^-(t)$. Intuitively, to see why this is the case, write the ODE in separable form
\begin{align*}
    \frac{1 - F(x)}{1 -x} \cdot dx\ =\ F^-(t) \cdot dt\,.
\end{align*}
Therefore, if $\bar x(z) =  F^-(z)$ for some $z > 0$, then
\begin{align*}
    \int_0^{F^-(z)} (1 - F(x)) dx\ <\ \int_0^{F^-(z)} \frac{1 - F(x)}{1 - x} dx\ =\ \int_0^{\bar x(z)} \frac{1 - F(x)}{1 - x} dx\ =\  \int_0^z F^-(t) dt\,.
\end{align*}
Integration by parts reveals the left-most term to be larger than $\E_F[v \mathbbm{1}(v \leq F^-(z))]$, which is no smaller than the right-most term. Thus, we have a contradiction and $\bar x(z) <  F^-(z)$ for all $z > 0$, as required to establish \Cref{lemma:ode-existence}. The full proof has to contend with subtle technical nuances which we have skipped here; refer to Appendix~\ref{appendix:minimax-lemmas} for details.





Next, we show that the quantile-based bidding strategy $Q^*$ described in Lemma~\ref{lemma:ode-existence} does indeed imply that all $y \in [0,1]$ are optimal solutions for the inner-maximization problem \eqref{eq:inner_max}. 




\begin{lemma}\label{lemma:saddle-max-prob}
	Consider the quantile-based bidding strategy $Q^*$ defined in Lemma~\ref{lemma:ode-existence}. Then, for all $y \in [0,1]$, we have
	\begin{align*}
		\int_{Q^*(y)}^1 (1 - F(t)) dt\ -\ \int_y^1 (F^-(t) - Q^*(t)) dt\ =\ \int_0^1 Q^*(t) dt\,.
	\end{align*} 
\end{lemma}
\Cref{lemma:saddle-max-prob} concludes our construction of the candidate quantile-based bidding strategy $Q^*$. It shows that the bidding strategy defined in \Cref{lemma:ode-existence} makes nature indifferent between any highest competing bid in the interval $[0,Q^*(1)]$. As a consequence, any highest-competing-bid distribution on $[0,Q^*(1)]$ is a best-response against $Q^*$. To conclude our saddle-point construction, we construct one such particular highest-competing-bid distribution $H^*$ which in turn admits our quantile-based bidding strategy $Q^*$ as a best response.

\subsubsection{Candidate highest-competing-bid distribution}\label{sec:construct_H}

We start our construction of a candidate highest-competing-bid distribution $H^*$ by looking at the first-order optimality conditions which arise when the
buyer chooses the optimal bid $b$ for each value $v$. In contrast to the previous section, where we encountered a single first-order optimality condition, here we face an uncountable family of such conditions, one for each value, which we combine into a single ordinary differential equation (ODE) via $Q^*$.
 
Consider an absolutely continuous highest-competing-bid distribution $H$. For a fixed value $v \in [0,1]$, the expected utility of the buyer as a function of the bid $b$ satisfies
\begin{align*}
	u(b \mid v,H)\ \coloneqq\ \E_{h \sim H}[(v - b) \cdot \mathbbm{1} \{ b \geq h\}]\ =\ (v-b) \cdot H(b)\,.
\end{align*}

If $b^*(v) \in \argmax_{b \in [0,1]} u(b \mid v,H)$ is an optimal bid for value $v$, and $u(\cdot \mid v,H)$ is differentiable at $b^*(v)$, then it must satisfy
\begin{align*}
	u'(b^*(v) \mid v,H)\ =\ 0 \quad \iff \quad (v- b^*(v)) \cdot H'(b^*(v))\ -\ H(b^*(v))\ =\ 0\,.
\end{align*}
As we are seeking a distribution $H$ for which our quantile-based bidding strategy $Q^*$ is optimal, the bid $b^*(v) = Q^*(y)$ must satisfy the first-order optimality condition whenever $v = F^-(y)$. This yields the ordinary differential equation
\begin{align}\label{eq:second-ode}
	(F^-(y) - Q^*(y)) \cdot H'(Q^*(y))\ -\ H(Q^*(y))\ =\ 0\,.
\end{align}
Unlike the previous ODE \eqref{eq:first-ode}, the ODE in \eqref{eq:second-ode} is separable. We leverage this fact to characterize a solution in the following lemma. 
\begin{lemma}\label{lemma:second-ode}
	Consider the quantile-based bidding strategy $Q^*$ defined in Lemma~\ref{lemma:ode-existence}, and define\\ $H^*:[0,Q^*(1)] \to [0,1]$ as
	\begin{align*}
		H^*(x)\ \coloneqq\ \exp\left( - \int_{(Q^*)^{-1}(x)}^1 \frac{1}{1 - F(Q^*(t))} dt \right)\,.
	\end{align*}
	Then, $G \coloneqq H^* \circ Q^*$ is absolutely continuous and satisfies
	\begin{align*}
		G'(y)\ =\ G(y) \cdot \frac{1}{1 - F(Q^*(y))}\ =\ G(y) \cdot \frac{(Q^*)'(y)}{F^-(y) - Q^*(y)}\,
	\end{align*}
	almost surely for $y \in [0,1]$.
\end{lemma}
\Cref{lemma:second-ode} constructs a solution to ODE \eqref{eq:second-ode}. Our construction is obtained by solving ODE \eqref{eq:second-ode} with the initial condition $H(Q^*(1)) = 1$, and moving backwards towards $0$. The choice of this initial condition is designed to obtain a distribution $H^*$ which is supported on $[0,Q^*(1)]$. We also note that, in \Cref{lemma:second-ode}, we express the ODE in terms of $G$, as opposed to $H$. This characterization is more convenient because defining $H^*$ involves the function $(Q^*)^{-1}$, for which we have not established absolute continuity, nor has its derivative been characterized. 
We next show that the characterization of $G$ and $G'$ obtained in \Cref{lemma:second-ode} is sufficient to prove that the quantile-based bidding strategy $Q^*$ is optimal against the candidate highest-competing-bid distribution $H^*$.

\begin{lemma}\label{lemma:bid-strat-optimality}
	Consider the quantile-based bidding strategy from \Cref{lemma:ode-existence} and the highest-competing-bid distribution $H^*$ from \Cref{lemma:second-ode}; let $G = H^* \circ Q^*$. Then, for every value $v \in [0,1]$, 
    we have that
	\begin{align*}
		\{Q^*(y) \mid y \in[0,1], \, F^-(y) = v\}\ \subseteq\ \argmax_{b \in [0,1]}\ u(b \mid v, H^*)\,.
	\end{align*} 
\end{lemma}
For every value $v \in [0,1]$, \Cref{lemma:bid-strat-optimality} establishes that \emph{every} bid in the support of $\s[Q^*](v)$---which is the bid distribution for value $v$ prescribed by the quantile-based bidding strategy $Q^*$---is optimal against the highest-competing-bid distribution $H^*$ defined in \Cref{lemma:second-ode}. In other words, having shown that any distribution on $[0,Q^*(1)]$ produces the same regret against $Q^*$, we have now constructed a distribution $H^*$ on $[0,Q^*(1)]$ which admits $Q^*$ as a best response; we are now ready to establish our saddle-point result.




\subsection{Putting it All Together}\label{sec:main-result}

With all of the ingredients assembled in the previous sections, we can now combine everything together to prove our main result: characterization of a saddle point for the problem of minimizing worst-case regret.

\begin{theorem}\label{thm:main-result}
	 For every value distribution $F \in \Delta([0,1])$ with $F(0) = 0$, there exists
	 \begin{itemize}
	 	\item a quantile-based bidding strategy $Q^*: [0,1] \to [0,1]$ which is a solution to the initial value problem
	 		\begin{align}
	 			x'(t)\ =\ \frac{F^-(t) - x(t)}{1 - F(x(t))}; \qquad x(0) = 0\; \label{eq:ODE_theorem}
	 		\end{align}
	 		and satisfies the properties outlined in \Cref{lemma:ode-existence}; and
	 	\item a distribution $H^*$ of highest competing bids defined for every $x \in [0,Q^*(1)] \subseteq [0,1]$ as
	 		\begin{align*}
				H^*(x)\ \coloneqq\ \exp\left( - \int_{(Q^*)^{-1}(x)}^1 \frac{1}{1 - F(Q^*(t))} dt \right)\,,
			\end{align*}
	 \end{itemize}
	 such that the pair $(\s[Q^*], H^*)$, where $\s[Q^*]$ is the bidding strategy corresponding to $Q^*$ (see \Cref{def:quantile-based-bidding-strategy}), satisfies,
     \begin{equation*}
         \inf_{s \in \Scal}\ \sup_{H \in \Delta([0,1])} \rr[F]{s, H} = \rr[F]{\s[Q^*], H^*} = \int_0^1 Q^*(t) dt.
     \end{equation*}
\end{theorem}

\Cref{thm:main-result} and \Cref{thm:full-info} together yield a saddle point for regret against the original partial-information benchmark, as the following corollary notes. And, in turn, this fact immediately implies the minimax optimality of the bidding strategy $\s[Q^*]$.
\begin{corollary}\label{cor:partial-info-saddle}
	The pair $(s_{Q^*}, \mathcal H^*)$, where $\mathcal{H}^*$ is the distribution of $\delta_h$ when $h \sim H^*$, satisfies
    \begin{align*}
        \inf_{s \in \Scal}\ \sup_{H \in \Delta([0,1])}\ \reg_F(s, H)\ =\ \reg_F(\s[Q^*], \mathcal H^*)\ =\   \int_0^1 Q^*(t)dt\,,
    \end{align*}
    with $\reg_F(s, \mathcal H^*) \coloneqq \E_{H \sim \mathcal{H^*}}[\reg_F(s,H)]$. In particular, $\s[Q^*]$ is a minimax-optimal bidding strategy.
\end{corollary}




\Cref{thm:main-result} establishes our main result. It characterizes the minimax-optimal bidding strategy $\s[Q^*]$ for the full-information benchmark, which we then extend to the partial-information benchmark in \Cref{thm:full-info}. Crucially, \Cref{thm:main-result} provides an efficient procedure for the construction of $Q^*$---and consequently the minimax-optimal bidding strategy $\s[Q^*]$---as a solution to the ordinary differential equation \eqref{eq:ODE_theorem}. The corresponding minimax-optimal regret can then be determined through a simple integral evaluation. This framework drastically reduces the complexity faced by the buyer: instead of contending with an intricate infinite-dimensional minimax optimization problem, she only needs to solve an ordinary differential equation. In addition to computational tractability, our framework offers a wide range of structural insights about minimax-optimal bidding in first-price auctions. For starters, our construction of $\s[Q^*]$ yields a \emph{deterministic} minimax-optimal bidding strategy for continuous value distributions. This fact follows directly from \Cref{def:quantile-based-bidding-strategy} because $F(\{v\}) = 0$ holds for all $v \in [0,1]$ when $F$ has a continuous CDF.

\begin{corollary}\label{cor:deterministic-opt-strat}
    For any value distribution $F$ with a continuous CDF, the minimax-optimal bidding strategy $\s[Q^*]$ is deterministic.
\end{corollary}
When contrasted with \Cref{example:atom-at-1}, \Cref{cor:deterministic-opt-strat} reveals an important insight: even infinitesimal variations in values can make deterministic strategies minimax-optimal. While \Cref{example:atom-at-1} demonstrates the strict sub-optimality of deterministic strategies for distributions with atoms, these can be approximated
\footnote{The notion of approximation can be made precise with the Prokhorov or Wasserstein metrics.} 
arbitrarily well by continuous distributions, all of which admit deterministic minimax-optimal strategies. Thus, any value distribution can be perturbed ever so slightly to yield another one which admits a deterministic minimax-optimal bidding strategy.
   

