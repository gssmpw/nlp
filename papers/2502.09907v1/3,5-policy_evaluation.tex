\section{Performance Evaluation}\label{sec:strat_evaluation}

A crucial first step in the design of good bidding strategies is the ability to evaluate their performance on the metric of interest, which for us is the worst-case regret criterion defined in \Cref{sec:model}. However, computing the worst-case regret $\mathrm{WReg}_F(s)$ for a bidding strategy is challenging because it requires one to solve an infinite-dimensional maximization problem over the set of highest-competing-bid distributions $H$. Furthermore, the benchmark $\sup_{s' \in \Scal} \U[F]{s',H}$ against which we measure regret is itself the value of an optimization problem over $\Scal$—the space of all bidding strategies which map values to distributions. Consequently, $\reg_F(s,H)$ is itself the value of an infinite-dimensional optimization problem. These challenges are further exacerbated by the fact that all these optimization problems are parameterized by the value distribution $F$, which can be an arbitrary element in the space of all distributions $\Delta([0,1])$.


Our first main result reduces this complicated infinite-dimensional optimization to a simple one-dimensional optimization. It does so for the large class of bidding strategies comprised of all $s \in \mathcal{S}$ which induce a continuous bid distribution $P_{s,F}$.

\begin{theorem}\label{thm:evaluation}
	For any value distribution $F \in \Delta([0,1])$ and any bidding strategy $s \in \Scal$ which induces an absolutely continuous bid distribution $P_{s,F}$, we have
    \begin{align*}
		\mathrm{WReg}_F(s)= \sup_{H \in \Delta([0,1])} \reg_F(s,H)\ =\ \sup_{h\in [0,1]} \reg_F(s,\delta_h)= \sup_{h\in [0,1]}\E_{v \sim F}\left[ (v-h) \cdot \mathbbm{1}(v \geq h) \right]-\ \U[F]{s,h} \,.
    \end{align*}
    \normalsize
\end{theorem}

\Cref{thm:evaluation} is a critical result that characterizes the worst-case regret $\mathrm{WReg}_F(s)$ of a strategy as the output of a one-dimensional optimization over deterministic highest competing bids.
It implies that, when evaluating the worst-case performance over all highest-competing-bid distributions $H$, the buyer may restrict attention to deterministic highest competing bids, i.e., distributions of the form  $H = \delta_h$. Importantly, deterministic highest competing bids drastically simplify the evaluation of the benchmark $\sup_{s' \in \Scal} \U[F]{s',\delta_h}$: the optimal strategy for the benchmark is to bid $h$ whenever the value is higher than $h$. This yields the simplified expression $\E_{v \sim F}\left[ (v-h) \cdot \mathbbm{1}(v \geq h) \right]$ for the benchmark in \Cref{thm:evaluation}. Crucially, the simplification of \Cref{thm:evaluation} holds for all value distributions $F$ and bidding strategies, so long as they induce an absolutely continuous bid distribution $P_{s,F}$. In which case, the worst-case regret can be computed via a simple line search.


The proof of \Cref{thm:evaluation} follows from Bauer Maximum Principle (see 7.69 of \citealt{aliprantis2006infinite}). However, the application of such a result is not straightforward in our setting as the function $H \mapsto \reg_F(s,H)$ is not continuous on $\Delta([0,1])$ for the weak topology. Indeed, for a fixed bid $b$ and value $v > b$, note that the utility function is not continuous as a function of the highest competing bid $h$. As such, $\reg_F(s,H)$ is the pointwise supremum over an infinite dimensional space (over the bidding strategies chosen by the oracle) of a function which is discontinuous. To get around this challenge, we first use the Portmonteau lemma to establish the upper semi-continuity of the expected utility, and then use a version of Berge's maximum theorem for upper semi-continuous functions to conclude that $\reg_F(s,H)$, which is the pointwise supremum of such functions, is also upper semi-continuous.

We conclude this section by demonstrating the benefits of \Cref{thm:evaluation} through the evaluation of \emph{uniform-bid-shading strategies}, which are bidding strategies that uniformly scale each value $v$ by the same constant $\alpha \in [0,1]$ to determine the corresponding bid $\alpha \cdot v$. These strategies naturally arise in online ad auctions, they have been documented in laboratory experiments~\cite{bajari2005structural, filiz2007auctions}, and have been the popular choice for ``simple'' strategies in prior work~\citep{fikioris2024liquid, gaitonde2023budget}. Crucially, if one were to treat each value in isolation and minimize regret one value at a time, as \citet{kasberger2023robust} did, then the optimal deterministic strategy ends up being the uniform-bid-shading policy $v \mapsto 0.5 \cdot v$ with $\alpha = 0.5$. Note that separately evaluating worst-case regret for each value $v$ assumes that the choice of the highest competing bid distribution $H$ can depend on the value, which is tantamount to assuming that the buyer's value is not private information. This begs the following questions: Do uniform-bid-shading strategies work well if one accounts for private information in the form of a non-deterministic value distribution $F$? If so, what should be the shading factor $\alpha$? 


\begin{figure}[h!]
    \centering
    \subfigure[Worst-case regret]{
    \begin{tikzpicture}[scale = 0.65]
    \begin{axis}[
        width=10cm,
        height=10cm,
        xmin=0,xmax=50,
        ymin=0.15,ymax=0.3,
        table/col sep=comma,
        xlabel={$\rho$},
        ylabel={Worst-case regret},
        grid=both,
        legend pos=south east,
        %legend style ={font ={\footnotesize}}
    ]


    \addplot [red,  line width=0.7mm, ,mark=square,mark options={scale=.1}] table[x=beta,y={alpha = 0.4004}] {Data/worst_case_regret_all_betas.csv};
    \addlegendentry{$\alpha = 0.4$}

    \addplot [gray,  line width=0.7mm, ,mark=square,mark options={scale=.1}] table[x=beta,y={alpha = 0.4496}] {Data/worst_case_regret_all_betas.csv};
    \addlegendentry{$\alpha = 0.45$}

    \addplot [black,  line width=0.7mm, ,mark=square,mark options={scale=.1}] table[x=beta,y={alpha = 0.5}] {Data/worst_case_regret_all_betas.csv};
    \addlegendentry{$\alpha = 0.5$}

    \addplot [violet,  line width=0.7mm, ,mark=square,mark options={scale=.1}] table[x=beta,y={best_performance}] {Data/best_alpha_and_performance_beta_dist.csv};
    \addlegendentry{Best alpha}

    
    \end{axis}
    \end{tikzpicture}
    \label{fig:regret_bid_shade}
    }
    \subfigure[Best choice of $\alpha$]{
    \begin{tikzpicture}[scale = 0.65]
    \begin{axis}[
        width=10cm,
        height=10cm,
        xmin=0,xmax=50,
        ymin=0.3,ymax=0.5,
        table/col sep=comma,
        xlabel={$\rho$},
        ylabel={Optimal choice of $\alpha$},
        grid=both,
        legend pos=south east,
        %legend style ={font ={\footnotesize}}
    ]


    \addplot [violet,  line width=0.7mm, ,mark=square,mark options={scale=.1}] table[x=beta,y={best_alpha}] {Data/best_alpha_and_performance_beta_dist.csv};
    %\addlegendentry{Choice of optima}

    \end{axis}
    \end{tikzpicture}
    \label{fig:best_alpha}
    }
    \caption{\textbf{Performance of uniform-bid-shading strategies for Beta$(\rho,\rho)$ value distributions.} In (a), we report the worst-case regret of different uniform-bid-shading strategies $s_{\alpha}$ as a function of $\rho$, when the value distributions is of the form Beta$(\rho,\rho)$. We also report the worst-case regret for the uniform-bid-shade strategy which uses the best $\alpha$ for each $\rho$ (see violet curve). In (b), we plot the best choice of $\alpha$ for each value of $\rho$.}
    \label{fig:alpha_bid_shade}
\end{figure}


To answer these questions, we leverage \Cref{thm:evaluation} to evaluate the worst-case regret of uniform-bid-shading strategies across various value distributions, where each evaluation reduces to a simple line search over $h \in [0,1]$. In \Cref{fig:alpha_bid_shade}, we plot the worst-case regret of uniform-bid-shading strategies when the value distribution follows a $\mathrm{Beta}(\rho,\rho)$ distribution for $\rho > 0$. This provides a large class of value distributions for comparison, each with a mean of 0.5 but different levels of spread in values. \Cref{fig:alpha_bid_shade} highlights that, when the competing bids can only depend on the value distribution but not the realized value, the optimal choice of $\alpha$ depends heavily on the buyer's value distribution. When $\rho$ is very high, or very low, the choice $\alpha =0.5$ suggested by \cite{kasberger2023robust} performs strongly among uniform-bid-shading strategies. This aligns with their findings, as the value distribution converges to the point mass at $0.5$ for high $\rho$ and a Bernoulli distribution (which is essentially a point mass at $1$ for our problem), when $\rho$ is low. However, when the value distribution is more dispersed---for instance when $\rho = 1$ (corresponding to the uniform distribution)---worst-case regret can be significantly reduced by adopting the uniform-bid-shading strategy with $\alpha = 0.4$. In fact, this choice results in a worst-case regret that is $20\%$ lower than than that of $\alpha = 0.5$.  This improvement highlights the importance of considering the entire value distribution when designing bidding strategies, particularly in practical settings where the buyer's realized value is private and not observed by competitors.


All in all, \Cref{fig:alpha_bid_shade} demonstrates that \Cref{thm:evaluation} provides a powerful framework for comparing bidding strategies on the basis of worst-case regret. It equips the buyer with a principled method for selecting the strategy best suited to their value distribution. However, this approach is limited to evaluating and selecting among a predefined set of strategies, and leaves open the question: can we go beyond simple bidding strategies and characterize the optimal one that minimizes worst-case regret? In the next section, we answer this question in the affirmative by providing an efficient procedure for constructing minimax-optimal bidding strategies for arbitrary value distributions.