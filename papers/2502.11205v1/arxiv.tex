\documentclass[]{nature_mod}
\usepackage{amsmath,amsfonts}

%% make you have the nature.cls and naturemag.bst files where
%% LaTeX can find them
%\bibliographystyle{naturemag}
\usepackage{txfonts}
% \usepackage{soul, color}
%\usepackage{amsmath}
\usepackage[english]{}
\usepackage[normalem]{ulem}
\usepackage{xr}
\usepackage{graphicx}
\usepackage{tocloft}
\usepackage{rotating}
\usepackage{booktabs}
\usepackage[table]{xcolor}
\usepackage{multirow}
\usepackage{adjustbox}
\usepackage{verbatim}
\usepackage{amsmath,amssymb}
\usepackage{algorithm}
\usepackage{algpseudocode}
%\usepackage[table]{xcolor}
%\usepackage[footnotesize]{caption2}
%\definecolor{tableShade}{HTML}{F1F5FA} 
\usepackage[hidelinks]{hyperref}
\usepackage[numbers]{natbib}
%\definecolor{MyRed2}{rgb}{0.7,0.2,0.2}
%\definecolor{MyBlue}{rgb}{0.2,0.2,0.7}
%\definecolor{MyGreen}{rgb}{0.2,0.7,0.2}
% \usepackage{setspace}
% \singlespacing

\definecolor{tableShade}{HTML}{F1F5FA} 
\definecolor{MyGray}{rgb}{0.95,0.95,0.95}

% \newcommand{\ra}[1]{
% \renewcommand{\arraystretch}{#1}}
% \newcommand{\rr}{\raggedright}

\makeatletter %% <- make @ usable in command sequences
\newcount\SOUL@minus
\makeatother  %% <- revert @

%\title{Urban Flooding Triggers Multiplex Failures in Transportation Networks }

\title{Deep Contrastive Learning for Feature Alignment: Insights from Housing-Household Relationship Inference}

\author{Xiao Qian$^{1}$, Shangjia Dong$^{2}$ \& Rachel Davidson$^{3}$}

\begin{document}

\maketitle

\begin{affiliations}
 \item Graduate Research Assistant, Department of Civil and Environmental Engineering, University of Delaware, Newark, DE 19716. USA.
 \item Corresponding author, Assistant Professor, Department of Civil and Environmental Engineering, University of Delaware, Newark, DE 19716. USA. (sjdong@udel.edu)
 \item Professor, Department of Civil and Environmental Engineering, University of Delaware, Newark, DE 19716. USA.
\end{affiliations}

\begin{abstract}
\textbf{Abstract} Housing and household characteristics are key determinants of social and economic well-being, yet our understanding of their interrelationships remains limited. This study addresses this knowledge gap by developing a deep contrastive learning (DCL) model to infer housing-household relationships using the American Community Survey (ACS) Public Use Microdata Sample (PUMS). More broadly, the proposed model is suitable for a class of problems where the goal is to learn joint relationships between two distinct entities without explicitly labeled ground truth data. Our proposed dual-encoder DCL approach leverages co-occurrence patterns in PUMS and introduces a bisect K-means clustering method to overcome the absence of ground truth labels. The dual-encoder DCL architecture is designed to handle the semantic differences between housing (building) and household (people) features while mitigating noise introduced by clustering. To validate the model, we generate a synthetic ground truth dataset and conduct comprehensive evaluations. The model further demonstrates its superior performance in capturing housing-household relationships in Delaware compared to state-of-the-art methods. A transferability test in North Carolina confirms its generalizability across diverse sociodemographic and geographic contexts. Finally, the post-hoc explainable AI analysis using SHAP values reveals that tenure status and mortgage information play a more significant role in housing-household matching than traditionally emphasized factors such as the number of persons and rooms.
\end{abstract}

\section{Introduction}

Housing represents one of the most significant assets individuals own, serving as the foundation for people's daily activities and broader urban dynamics. Researchers have created synthetic populations \cite{farooq2013simulation, borysov2019generate, qian2024deep, kotelnikov2023tabddpm} to enable assessments at the household and individual levels. Understanding the characteristics of households living in various types of housing units is critical for capturing the intricacies of urban life. However, our current understanding of the relationship between housing and households remains limited. This knowledge gap hinders the creation of a comprehensive joint housing-household inventory that accurately reflects household-level disaster impacts. Consequently, it also impedes the development of effective disaster intervention policies, such as targeted relief and recovery efforts.
While previous studies have sought to develop integrated housing and household inventories \cite{rosenheim2021integration, ye2024enhancing, harada2017projecting}, they primarily emphasize population density, building capacity, or sociodemographic distribution. They often neglect the explicit linkage between households and housing units, preventing us from identifying which households are more likely to reside in specific types of housing. This missing linkage limits our ability to construct a truly integrated inventory for understanding households' interaction with the built environment and assessing disaster impacts on different populations. To address this knowledge gap, this research aims to explicitly model the housing-household relationship, enabling a more accurate analysis of housing occupancy patterns and their implications for disaster resilience and urban planning.

The American Community Survey (ACS) Public Use Microdata Sample (PUMS) \cite{us_census_acs}, hereafter referred to as microdata, serves as a valuable resource for analyzing the relationship between households and the housing units they occupy. However, due to privacy considerations and other constraints, ACS releases only a limited 1-5\% sample of the actual population within Public Use Microdata Areas (PUMAs). Household characteristics primarily include sociodemographic information such as income, age, education, and race, while housing unit attributes describe features like the number of bedrooms, mortgage or rent payments, and property value. Figure \ref{fig:desriptive_stats} illustrates the microdata structure and some descriptive statistics, showing, for example, how households of 5 individuals can reside in units with a range of numbers of bedrooms and property values. Similarly, households with incomes $\geq \$140$k may occupy diverse types of housing units. When considering multiple sociodemographic factors—such as household size alongside income—the distribution of suitable housing options shifts. This relationship is complex and cannot be captured solely through linear models or simple rules, highlighting the intricate patterns between household demographics and housing characteristics.

\begin{figure}[!ht]
    \centering
    \includegraphics[width=\linewidth]{eda.pdf}
    \caption{Microdata table illustration and descriptive statistics}
    \label{fig:desriptive_stats}
\end{figure}

Modeling the joint housing-household relationship is inherently complex, requiring advancements in data processing techniques and feature alignment methods. These challenges are not unique to housing-household relationship modeling; rather, they represent a broader class of tabular data integration problems, where ground truth linkages between two distinct datasets are missing, and only a limited number of positive co-occurring instances are available, such as pairing user profiles with product listings \cite{mcauley2015inferring}, matching patient records with medical treatments \cite{hripcsak2013next}, or aligning job applicants with job openings \cite{malinowski2006matching}. This research not only generates new knowledge on housing-household relationships but also drives broader innovations in information fusion techniques. 

To illustrate the technical advances in a concrete and relatable manner, we use housing-household relationship learning as a representative case of joint tabular data relationship modeling. However, it is important to note that the proposed methods extend beyond this specific application and can be widely applied to similar challenges in other domains.

\textbf{Challenges} Modeling housing-household relationships presents several key challenges due to inherent data limitations.
\begin{enumerate}
\vspace{-4pt}
    \item \textit{Data with positive pairs only}: microdata only provides household sociodemographics and housing unit features that co-occur within individual records. We lack explicitly labeled “negative” data points that indicate which types of households are unlikely to reside in specific types of housing. Unobserved pairs are not necessarily negative, as these may exist in reality but are simply absent from the sample. The absence of explicitly labeled negative data calls for innovative data engineering and learning techniques to ensure accurate and reliable relationship modeling.  \vspace{-6pt}  
    \item \textit{Many-to-many matching possibility}: each housing unit could be suitable for multiple households, but the microdata only records a single pairing. An alternative is to group households and housing units into different clusters and extract the many-to-many pairing information. While this broader perspective allows us to account for various housing options, it also introduces potential false positives (or noise in computer science terms) during clustering, as it implicitly assumes households may live in a variety of housing types, even if this relationship is not explicitly recorded in the microdata. Addressing this requires a clustering method capable of fine-grained grouping and a robust and noise-aware learning approach. \vspace{-6pt}  
    \item \textit{Binary matching instead of matching degree}: the co-occurrence of housing unit and household in a microdata record is a binary indicator of a household's residence choice, but the selected unit may not necessarily represent the household’s ideal living preference. Housing-household matching, therefore, requires a more nuanced approach. Ideally, we aim to provide a degree of match and rank housing-household feature alignment rather than assuming all possible units are equal likely matches. This calls for a modeling approach that can learn from diverse co-occurrences and assign a matching score to each housing-household pair.  \vspace{-6pt} 
    \item \textit{Matching subjects consist of two distinct entities}: household sociodemographics and housing unit features, representing people and buildings, respectively. Applying a standard feature alignment pipeline that assumes they share the same feature space would result in inaccurate matching. Addressing this challenge requires us to develop a specialized feature-learning pipeline that captures the unique yet interconnected nature of housing-household relationships. \vspace{-6pt}  
    \item \textit{Lack of a labeled test set}: the microdata provides only a single observed match for each household and its corresponding housing unit, and we lack a labeled dataset with negative housing-household pairs. Additionally, inferred or clustered pairs—where each household could theoretically match with multiple housing units—are plausible but not definitive. This necessitates a comprehensive and accurately labeled test set to effectively evaluate model performance.
\end{enumerate}

\textbf{Contribution} In this research, we present a deep contrastive learning (DCL) approach to model the relationship between housing unit features and household attributes. The main technical contributions of this work are as follows:
\begin{itemize}
\vspace{-4pt}
    \item We utilized the co-occurrence of housing and household data as a pretext task to address the challenge of limited positive pair data and lack of negative pair data (Challenge 1). \vspace{-6pt}  
    \item We introduced a clustering method (Section \ref{sec:bisect-cluster}) to categorize households and housing units into distinct types (Challenge 2), facilitating feature alignment between households and housing units. This grouping approach also enhances the pool of positive and negative pairs available for modeling training (Challenge 1). \vspace{-6pt}  
    \item We developed a DCL framework with dual encoders (Section \ref{sec:model-architecture}) to address the challenges of noisy data introduced by clustering (Challenge 2) and the distinct feature spaces of housing and households (Challenge 4). This framework also enables nuanced feature alignment, capturing degrees of match rather than binary outcomes (Challenge 3). Additionally, we implemented a sigmoid contrastive learning loss to account for the many-to-many relationships between households and housing units (Challenge 2). \vspace{-6pt}  
    \item To ensure robust and accurate model evaluation in the absence of a fully labeled dataset (Challenge 5), we created a synthetic ground truth to validate model performance and compare it to the state-of-the-art models prior to applying it to the microdata for learning the housing-household relationship. (Section \ref{sec:theoretical_performance}). 
\end{itemize}


\section{Literature Review}
Recognizing the importance of a joint housing and household inventory for disaster management and urban planning, several researchers have contributed to the housing-household joining effort. Harada and Murata \cite{harada2017projecting} used geospatial data to allocate households to housing units based on building specifications, such as building types and locations. Rosenheim et al. \cite{rosenheim2021integration} developed a more granular joint synthetic household and housing inventory by expanding and linking different datasets through random or rule-based processes for assigning households to housing units. Ye et al. \cite{ye2024enhancing} integrated LiDAR (Light Detection and Ranging) data, Points of Interest (POI), and quadratic programming techniques to create a population-building inventory. 

These studies illustrate the progression of joint household and housing unit modeling, evolving from random allocations to data-driven approaches. However, they also reveal a persistent challenge and research gap: the lack of housing-household relationships that capture the interaction between household characteristics and housing unit features, which is critical for advancing accurate disaster impact research. As noted by Aerts et al. \cite{aerts2018integrating}, individual and family characteristics play a significant role in decision-making processes, influencing housing choices. Ignoring these correlations undermines the accuracy and validity of subsequent research. Thus, developing methods that can accurately model housing-household relationships is essential. However, matching tabular information such as household characteristics and housing unit features faces many data and methodological challenges, as detailed earlier. 

Capturing relationships among heterogeneous datasets, commonly termed schema matching \cite{nachouki2008multi, johnston2008web}, is an active research area. Current table-matching methods largely focus on identifying content similarity between tables. For instance, the Valentine system \cite{koutras2021valentine} is an open-source experimental suite with a user-friendly graphical user interface (GUI) that is designed for large-scale schema matching. Structure-aware Bidirectional Encoder Representations from Transformers (StruBERT) \cite{trabelsi2022strubert} introduces structure- and context-aware features by integrating structural and textual information, leveraging deep contextualized language models like BERT to capture semantic similarities between tables. Similarly, Schema Matching Using Generative Tags and Hybrid Features (SMUTF) \cite{zhang2024smutf} approach combines rule-based feature engineering, pre-trained language models, and generative large language models to perform large-scale schema matching, excelling in cross-domain tasks. While these methods have advanced table-matching techniques, they mainly focus on matching similar content (e.g., Merging duplicate records in databases when two entries share overlapping attributes), whereas feature alignment involves interpreting and transforming different data types (e.g., household attributes and housing unit attributes) to reveal meaningful relationships, which is the core challenge of our research.

Recent advances in information fusion have demonstrated remarkable progress in integrating heterogeneous data sources for enhanced predictive capabilities. In biomedical applications, deep learning-based fusion methods have shown effectiveness in combining medical imaging, biomarkers, and clinical data to improve diagnostic accuracy and treatment planning \cite{zitnik2019machine, duan2024deep, zhao2024review}. In cancer research, fusion methods have successfully integrated imaging and omics data to enhance prognosis prediction and treatment response assessment \cite{lu2024privacy}.
Urban computing has similarly benefited from cross-domain data fusion techniques, particularly in integrating geographical, traffic, social media, and environmental data \cite{zou2025deep}.

While existing information fusion approaches primarily focus on aggregating heterogeneous data sources for enhanced prediction \cite{mai2023learning}, our method takes a distinct approach by learning the inherent alignment patterns from co-occurring data pairs. We leverage the co-occurrence relationships in housing-household microdata to learn a generalizable feature alignment model. This approach enables us to extend beyond the training dataset and match previously unseen housing-household pairs, providing a valuable tool for researchers in social sciences, urban computing, and disaster management. The learned feature alignment model effectively captures the complex interplay between housing and household characteristics, facilitating more accurate housing allocation analysis and disaster impact assessment.

Recent advancements in feature alignment have leveraged developments in learning-based methods from computer vision and multimodal data fusion, often referred to as Cross-modal Retrieval Methods \cite{cao2022image}. For example, the CLIP (Contrastive Language-Image Pre-training) method \cite{radford2021learning} pioneered a transformative approach by training on large-scale datasets of image-text pairs, addressing limitations in labeled data for visual recognition tasks. Building on this, ALIGN (A Large-scale ImaGe and Noisy-text embedding) \cite{jia2021scaling} scaled training data to billions of image-text pairs, achieving significant improvements in zero-shot classification. BLIP (Bootstrapping Language-Image Pre-training) \cite{li2022blip} introduced a unified framework for vision-language pre-training, incorporating captioning and filtering techniques to manage noisy web-crawled data. Other notable work includes ALBEF (Align Before Fuse) \cite{li2021align}, which used contrastive loss to align image and text representations, and UNITER (Universal Image-Text Representation) \cite{chen2020uniter}, which proposed a unified architecture for diverse vision-language tasks. These methodologies have significantly advanced cross-modal retrieval and representation learning. 

While cross-modal retrieval methods hold great promise, they are not directly suited to our objectives. These methods focus on matching data from different modalities that describe the same subject, such as pairing an image of a cat with its textual description. In contrast, our goal is to match inherently distinct features within tabular data—specifically, housing unit attributes and household characteristics. While these features pertain to different domains, one describing physical structures and the other human subjects, they share latent connections. This distinction highlights unique challenges and opportunities. While we can draw inspiration from the contrastive learning techniques employed in cross-modal retrieval, these methods must be adapted to align two different features within a single modality (tabular data). Our approach aims to capture the complex, multi-dimensional relationships between housing and household features, bridging a gap that traditional cross-modal techniques fail to address.

Given the data challenges we face, self-supervised learning with pretext tasks, where models learn from auxiliary tasks without requiring labeled data, presents a promising alternative. Pretext tasks allow models to learn from unlabeled data by solving contrived objectives that reveal inherent data patterns, with the learned representations transferable to downstream tasks. For example, BERT’s Next Sentence Prediction (NSP) task trains models to predict whether one sentence logically follows another, enabling an understanding of how two segments of discourse are connected to each other, either logically or structurally \cite{devlin2018bert}. However, RoBERTa \cite{liu2019roberta} showed that removing NSP and instead training on longer, contiguous text sequences improves downstream performance. In tabular data feature alignment, SubTab introduces pretext tasks like Reconstruction, Contrastive Learning, and Feature Vector Distance, which effectively capture tabular representations \cite{ucar2021subtab}. 

While these methods focus on general-purpose representations, our approach is tailored specifically to housing-household matching relationships. First, we use a task-specific architecture: our CLIP-inspired model processes household and housing features separately, simplifying the computational complexity from $O(N^2)$ to $O(N)$ for matching tasks. Second, we formulate a task-specific objective: we employ a single contrastive learning task optimized for housing-household alignment rather than multiple generic pretext tasks. This specialized design results in superior performance, robustness, and computational efficiency, enabling our model to address the unique challenges of housing-household matching more effectively than existing general-purpose methods, as shown in Section \ref{sec:results}. 

\section{Methodology}

We frame the learning of joint relationships between housing units and households as a feature alignment problem, with the goal of mapping them in a shared space so the compatible match can be pulled closer and incompatible ones will be distanced. We represent household-level features as $h_i \in H$, and housing unit-level features are defined as $u_j \in U$, where $H$ and $U$ are sub-tables in the microdata, capturing household and housing unit information, respectively, and $i$ and $j$ denote the row indices within the table. When $i=j$, we say that $h_i$ and $u_j$ co-occur and form a housing-household pair. 

The goal is to train an alignment function, $f_\theta(h_i, u_j)$, which maps housing unit features and household sociodemographics from their distinct spaces to a shared feature space, producing an alignment score that quantifies the degree of housing and household feature matching. At an ideal level, a well-aligned housing-household pair should exhibit relationships reflective of real-world correlations. For instance, high-income households are more likely to be matched with high-value housing units, while larger households may correlate with units offering additional vehicles or greater space.

However, these relationships are complex, displaying multivariate and nonlinear dependencies that are not easily captured by simple rules or linear models. To address this complexity, we propose a deep learning-based approach to model these high-dimensional nonlinear relationships, learning sophisticated mappings between household and housing unit features to effectively characterize their intricate dependencies. 

\subsection{Housing-Household co-occurrence in microdata}
In supervised learning, models are trained by minimizing the error between their predictions and ground-truth labels. However, in this case, we only have data on a subset of households with verified housing-household pairings, lacking a fully labeled dataset with both positive (matching) and negative (non-matching) pairs. When labeled data is limited, such as in this study, where only positive pairs are available in the microdata, self-supervised learning with pretext tasks provides an effective solution \cite{jing2020self}. This approach eliminates the need for an explicitly labeled dataset \cite{liu2021self}, as the labels are derived from the data itself such as data co-occurrence. Through the pretext task, the model learns representations that can then be transferred to downstream tasks, such as classification or regression.

In our study, we treat household sociodemographics and housing unit features that co-occur in the same microdata entry as positive sample pairs. The goal of our pretext task is to match housing units and households based on their co-occurrence in the microdata, which forms our pretext task. This task makes use of the matching information within the microdata and provides a learning objective that aligns closely with our main goal: to learn a relationship that enables accurate matching between compatible housing units and households. Figure \ref{fig:co-occurring} visualizes this concept using the ACS microdata. Features from the same record in ACS microdata are treated as matches, while features from different records are treated as non-matches.

\begin{figure}[!ht]
  \centering
  \includegraphics[width=0.99\textwidth]{co-occur.pdf}
  \caption{Illustration of the co-occurring pretext task using the ACS dataset.}
  \label{fig:co-occurring}
\end{figure}

Using co-occurrence as a pretext task presents several key challenges. First, due to the lack of explicit negative labels, we need to generate negative pairs for modeling. We assume that households and housing units from different rows in the microdata are incompatible, hereafter referred to as pseudo-labels \cite{jing2020self}. While the generated negative samples facilitate the training, it also introduces potential false negatives. Because the microdata only represents a sample of the entire population, households and housing units deemed "incompatible" within the sample may actually be compatible pairs in the broader population. This assumption introduces noise that requires a robust model capable of learning the true relationship between households and housing units despite these false negatives. Co-occurrence in the microdata is a binary indicator of a household's residence choice, but not all suitable housing units are equally appealing. The housing-household feature alignment task should assess the degree of matching. Third, the model must generalize beyond the microdata to match unseen households and housing units accurately. 

To address these challenges, our approach includes strategies to mitigate bias and noise in the co-occurrence-based pretext task. Specifically, in Section \ref{sec:bisect-cluster}, we introduce a noise-aware preprocessing procedure that minimizes data noise impact. Sections \ref{sec:model-architecture} details our use of a contrastive learning framework and loss function, which improve the model’s capacity for robust representation learning amidst noise. 

Figure \ref{fig:pipeline} shows our proposed pipeline. The pipeline first preprocesses the microdata, and then feeds it into a dual-encoder neural network architecture. This architecture comprises separate encoders for housing unit characteristics and household sociodemographic features, which enables domain-specific learning for each feature type. During training, the two networks work collaboratively to project household and housing unit features into a shared latent space. In this space, semantically similar housing-household pairs are positioned closer together, allowing the model to capture complex, non-linear relationships and efficiently match household and housing features.

\begin{figure}[!ht]
  \centering
  \includegraphics[width=\textwidth]{Pipeline.pdf}
  \caption{Joint Housing-household relationship learning pipeline}
  \label{fig:pipeline}
\end{figure}

\subsection{Bisecting K-Means clustering of households and housing units}\label{sec:bisect-cluster}

The co-occurrence between individual housing units and household demographics in microdata provides a preliminary understanding of the housing-household relationship. However, microdata does not fully capture the complexity, as a single household could theoretically live in various housing units, but ultimately can only occupy one at a time. Similarly, a housing unit might be suitable for multiple households. To address this, a more useful approach is to explore what types of households are best suited for specific types of housing units.

One basic approach to classify households would be to categorize based on all possible demographic and housing attribute combinations. For instance, income or age could be divided into discrete ranges, and their combinations enumerated. However, this method is inherently arbitrary and unlikely to represent an optimal grouping of households and housing units. More critically, such simplistic combinatorial approaches treat each category as a discrete and independent entity, failing to capture relationships or distances between different categorical items. For example, a household with an income of \$49,000 would be considered entirely distinct from one with an income of \$51,000 if they fall into separate income brackets. In reality, their socioeconomic statuses are very close when income is viewed as a continuous variable. This approach fragments the feature space into isolated categorical combinations, creating "islands" that hinder the establishment of meaningful relationships between groups.

Instead, we employ a bisecting K-Means clustering approach. The process begins with feature standardization—numerical features such as household income and housing value are normalized using z-scores, while categorical features like education level and housing type are one-hot encoded. This ensures features of different scales are weighted equally in the clustering. 

Next, we apply the Bisecting K-Means method to cluster households. Unlike traditional K-Means, which partitions the dataset into k clusters in one step, Bisecting K-Means takes a hierarchical, divisive approach. It iteratively splits clusters with the largest sum of squared errors (SSE) into two sub-clusters, continuing this process until the desired number of clusters is reached. This approach provides more refined groupings, allowing us to better capture variations in housing unit and household compatibility.

\begin{figure}[!ht]
  \centering
  \includegraphics[width=0.99\textwidth]{cluster.pdf}
    \caption{Illustration of the motivation and process of bisecting K-Means clustering.}
  \label{fig:cluster}
\end{figure}

In microdata, housing-household relationships are initially represented through simple one-to-one co-occurrence, as illustrated in Figure \ref{fig:cluster}(a), where each housing unit is matched with a corresponding household (e.g., ${U_1-H_1, U_2-H_2, U_3-H_3}$). However, this straightforward matching approach is overly restrictive and fails to account for similarities between housing units and households. As shown in Figure \ref{fig:cluster}(b), if two housing units (e.g., $U_1$ and $U_2$) are highly similar in the feature space, they should also be compatible with similar households. For instance, if $U_1$ matches with $H_1$, then $U_2$ should be considered a potential match for $H_1$. To systematically implement this insight, we propose a cluster-based co-occurring strategy (Figure \ref{fig:cluster}(c)). Housing units and households are clustered separately based on their feature similarities. By identifying clusters (e.g., $C_1^U$ for housing units and $C_2^H$ for households), we expand the matching relationships beyond the original co-occurrence pairs. For example, when $U_1$ and $U_2$ belong to the same cluster $C_1^U$, and $H_2$ and $H_3$ belong to cluster $C_2^H$, additional pairs such as $U_1-H_2$, $U_1-H_3$, and $U_2-H_3$ can be derived alongside the original pair $U_2-H_2$. These expanded housing-household pairs serve as augmented labels for model training, enabling the model to learn more generalized and robust matching patterns that extend beyond the limited scope of direct co-occurrence.


\begin{figure}[!ht]
  \centering
  \includegraphics[width=\textwidth]{false_negative_and_false_positive.pdf}
  \caption{Impact of clustering granularity on label matching quality}
  \label{fig:clustering_error}
\end{figure}

While clustering helps capture nuanced relationships between different types of households and housing units and generates additional data for model training, it can also introduce errors. 
In the ideal scenario depicted in Figure \ref{fig:clustering_error}(a), a ground truth dataset assigns each housing-household pair a matching score represented by different shades of color, with positive samples highlighted in red and negative samples in blue. However, microdata co-occurrence, as illustrated in Figure \ref{fig:clustering_error}(b), only records observed matches along the diagonal, leaving many potentially valid matches unaccounted for. Treating all non-co-occurring pairs as negatives risks introducing false negatives, as some unobserved pairs may still be valid matches outside of microdata.

Clustering addresses this limitation by generating new housing-household pairs, capturing some of the missing valid matches. Yet, clustering creates a new challenge, as it may introduce false positives by creating housing-household pairs that are not truly compatible. The balance between true and false positives depends on the chosen cluster size. A moderate cluster size (Figure \ref{fig:clustering_error}(c)) reduces false negatives by grouping similar samples together, keeping false positives to a manageable level. Conversely, overly aggressive clustering (Figure \ref{fig:clustering_error}(d)) significantly lowers false negatives but introduces many false positives by forcing dissimilar samples into the same cluster. The impact of cluster size on data quality and model performance is critical. We investigate the impact of cluster size through sensitivity analysis in Section \ref{sec:sensitivity}.


\subsection{Deep contrastive learning (DCL) model architecture}\label{sec:model-architecture}

Joining housing units and households can be formulated as a feature alignment problem. However, due to the lack of an explicitly labeled dataset (with both positive and negative pairs), a self-supervised learning approach—contrastive learning—becomes an effective solution for this task. Contrastive learning is well-suited for learning discriminative features that can draw related housing-household pairs closer while pushing unrelated pairs farther apart. Additionally, this approach is robust to the class imbalance inherent in our dataset, where only a limited subset of co-occurring housing and household records are captured through clustering, while most other housing-household pairs are considered negative (i.e., pseudo-labels). Contrastive learning has also proven to excel in representation learning and matching for unseen data, a critical advantage since our constructed training set cannot fully capture the numerous housing-household combinations. 

Common feature alignment tasks, such as image-text pairing in models like CLIP \cite{radford2021learning} and ALIGN \cite{jia2021scaling}, involve matching images of a subject with corresponding text descriptions. In these tasks, although the data is in different modalities (e.g., images and text), both describe the same underlying subject. In our case, however, household sociodemographics describe people’s characteristics, while housing unit features reflect aspects of their living environment, making them semantically and numerically different. Therefore, using a single encoder to process both household and housing data would be ineffective. Instead, we adopt a dual encoder structure, as illustrated in Figure \ref{fig:DL_constrastive_model}. 

This architecture offers several advantages. First, it enables separate feature embeddings for household and housing attributes, allowing each encoder to capture domain-specific representations. Additionally, it supports more efficient and scalable feature matching. By processing housing and household features through two separate encoders (Figure \ref{fig:DL_constrastive_model}) and projecting them into a shared space, our approach avoids the computationally intensive task of computing pairwise similarities across all housing-household combinations, which scales quadratically with sample size. Instead, it reduces complexity to linear, significantly improving scalability.

\begin{figure}[!ht]
  \centering
  \includegraphics[width=\textwidth]{model.pdf}
  \caption{DCL model architecture}
  \label{fig:DL_constrastive_model}
\end{figure}

Each encoder consists of a series of custom Multi-Layer Perceptron (MLP) layers that progressively transform the input features, capturing increasingly abstract representations at each stage. Each MLP layer contains an input layer normalization, a linear transformation, a GELU activation, a second linear transformation, and a dropout. After encoding, the housing and household features are projected into a shared latent space, where a dot product operation computes their similarity. The resulting value indicates the likelihood of a match.

The pseudo-labels we generated—including assumed positive pairs from clustering and assumed negative pairs from outside clusters—introduce noise into the contrastive learning environment. When the model encounters these noisy labels, it adjusts its parameters to fit potentially inaccurate signals, leading to frequent and large parameter fluctuations that can compromise model stability and generalization. To address this, we incorporate momentum distillation \cite{chen2020simple, li2021align}, a training technique that uses a moving average to update model parameters. This approach ensures that model weights evolve gradually, helping to prevent abrupt fluctuations during training with noisy data.

The loss function plays a crucial role in accurately learning housing-household relationships. While the conventional SoftMax-based contrastive loss, typically implemented as InfoNCE (Information Noise-Contrastive Estimation), is effective in single-positive tasks like image-text pairing, it falls short for our many-to-many matching task, where multiple valid matches may exist for each household or housing unit. In this case, InfoNCE’s global normalization does not apply to our context as it assumes a single positive match per query. Additionally, its batch-wide normalization and numerical stabilization techniques (maximum logit subtraction) create significant computational overhead, which poses efficiency challenges for large-scale housing datasets \cite{zhai2023sigmoid}. 

The sigmoid loss function addresses these challenges and has shown effectiveness in multi-modal contrastive learning \cite{zhai2023sigmoid}. First, its formulation inherently supports the many-to-many nature of housing-household pairings. Second, it removes the need for batch-wide normalization as required in InfoNCE, allowing for more efficient processing of large housing datasets. Moreover, the sigmoid function yields more stable gradients for both positive and negative samples, which is critical for learning robust representations in noisy and label-scarce settings. Therefore, we adopt the sigmoid-based loss function, defined as:

\begin{equation} \mathcal{L}_{ij} = \log\frac{1}{1+e^{z_{ij}(-t \cdot \mathbf{h}_i\cdot\mathbf{u}_j+b)}} \end{equation}

\begin{equation} \mathcal{L} = -\frac{1}{|\mathcal{B}|}\sum_{i=1}^{|\mathcal{B}|}\sum_{j=1}^{|\mathcal{B}|}\mathcal{L}_{ij} \end{equation}

where $\mathbf{h}_i$ and $\mathbf{u}_j$ represent the embeddings of the $i$-th household and $j$-th housing unit, respectively. $z_{ij}$ is the binary indicator representing whether they form a positive housing-household pair based on the training set. $t$ is the learnable temperature that scales the dot product similarity, enabling the model to adjust its sensitivity to variations in housing unit and household features—a critical factor given the diversity within the microdata. $b$ serves as a dynamic threshold for determining housing-household matches, helping to address the class imbalance in the training set. 


\section{Performance Evaluation Metrics}\label{sec:performance-evaluation}


We assess the learned housing-household relationship by using it to match housing units with households and measure the accuracy of these matches. To capture various dimensions of matching performance, we employ multiple evaluation metrics. 

\textbf{Average Precision (AP)} \cite{davis2006relationship} is used to evaluate the quality of housing-household matching predictions, especially the extent to which the model ranks the positive ones higher as desired. Our proposed DCL pipeline generates a matching score for each housing unit-household pair. Depending on the threshold selected (above which a pair is classified as a positive match and below which it is considered a negative match), different matching results are produced. By plotting precision against recall across varying thresholds, we generate a Precision-Recall (PR) curve. AP is computed as the area under this curve, ranging from 0 to 1, defined as:
\begin{equation}
  AP = \sum_{n} (R_n - R_{n-1}) P_n
\end{equation}
where $P_n$ and $R_n$ are the precision and recall at the $n$-th threshold, respectively. A higher AP indicates superior model performance, reflecting an optimal balance between precision and recall across the entire range of thresholds.

\textbf{Normalized Discounted Cumulative Gain (NDCG)} \cite{jarvelin2002cumulated} is adopted to assess how well the model prioritizes the most suitable housing units for each household. In this problem, while multiple housing units may be suitable for a household, they vary in compatibility. The model predicts a ranked list of matching housing units for each household, with scores indicating the strength of each match. NDCG evaluates whether the highest-ranked units are truly the best matches, effectively capturing such "degree of matching". A high NDCG score indicates the model effectively captures the relationship between household and housing unit features, prioritizing the most relevant matches. The calculation of NDCG at position $k$ is defined as:

\begin{equation}
NDCG@k = \frac{DCG@k}{IDCG@k} = \frac{\sum_{i=1}^k \frac{y_{true}[\pi_i]}{\log_2(i + 1)}}{\sum_{i=1}^k \frac{y_{true}^*[i]}{\log_2(i + 1)}} 
\end{equation}
where $\pi = \text{argsort}(y_{score})[::-1]$ (descending order indices). $\pi_i$ is the original index at position $i$ after sorting the predictions in descending order, $y_{score}$ represents the predicted matching scores from the model. $y_{true}^* = \text{sort}(y_{true}$, descending=True) (sorted ground truth). $y_{true}$ represents the actual relevance scores (1 for suitable matches, 0 for unsuitable ones), and $y_{true}^*$ represents sorted ground truth relevance scores in descending order. $k$ refers to the number of top positions to evaluate. The numerator ($NDCG@k$) measures the discounted cumulative gain based on the ranking induced by model predictions. The denominator ($IDCG@k$) computes the ideal discounted cumulative gain from the best possible ranking. The logarithmic discount factor $\log_2(i + 1)$ penalizes relevance scores at lower ranks. Finally, the ratio between DCG and IDCG normalizes the score to $[0,1]$, where 1 indicates perfect ranking and 0 indicates the worst possible ranking.

We compare our approach with state-of-the-art tabular data feature alignment methods, SubTab \cite{ucar2021subtab}, to demonstrate the superior performance of our DCL framework in modeling housing-household relationships. SubTab is a self-supervised learning framework designed to generate meaningful representations for tabular data by dividing features into subsets and reconstructing the original data. To adapt SubTab for our matching task, we made several key modifications to align the method with our data's characteristics. First, we applied one-hot encoding to preprocess categorical features, enabling the model to handle the diverse categorical variables in the dataset. Second, instead of randomly dividing features as in the original SubTab, we partitioned the data into household and housing unit sub-tables. Third, to account for differences in feature set sizes, we padded the sub-tables to ensure uniform feature dimensions. After training the encoder-decoder architecture, we extracted feature representations from the encoders for both households and housing units. Using the outer product of these representations, we computed the logits for the matching relationships, following the same process as in our proposed method.

\section{Performance Evaluation of DCL in a Controlled Experiment} \label{sec:theoretical_performance}

We designed our DCL framework to be dataset-agnostic, making it applicable to any matching tasks between two sets of features. Ideally, this method should be evaluated using a ground truth dataset, where both positive and negative matches, along with their matching degrees, are explicitly known. However, in real-world scenarios, such a dataset is often unavailable, posing a limitation to direct validation. Many real-world tabular datasets only capture positive co-occurrences. For instance, which individuals live in which households, or which individuals hold which job types. These observed pairings are often limited due to the sample size and may not represent the most optimal matches. For example, a household residing in a particular housing unit does not necessarily indicate that it is the most compatible option, only a feasible one. Additionally, datasets rarely provide explicit non-matching information, such as which individuals should not be living in a given house. To overcome this limitation, we construct a synthetic ground truth dataset where all matching relationships, including both positive and negative cases, are fully known, enabling a more rigorous evaluation of our method’s performance.

\subsection{Synthetic ground truth generation}
\label{sec:synthetic_data}


A synthetic ground truth can be constructed by interlinking two datasets through predefined non-linear relationships. Given a table $\mathbf{X}$ with $N$ records, where each record $\mathbf{x}_i = [x_{i1}, x_{i2}, ..., x_{iM}]$ consists of both categorical and numerical features, a corresponding table $\mathbf{Y}$ can be generated for controlled experiments by applying the following non-linear transformations:
\begin{equation}
  \mathbf{y}_i = f(\mathbf{x}_{i1}, \mathbf{x}_{i2}, ..., \mathbf{x}_{iM}) = f(\alpha_1\mathbf{x}_{i1} + \alpha_2\mathbf{x}_{i2} + ... + \alpha_M\mathbf{x}_{iM})
\end{equation}
where $\mathbf{y}_i$ represents the target feature vector for the $i$-th record, and $\mathbf{x}_{ij}$ denotes the $j$-th feature of the $i$-th record in the input table $ \mathbf{X}$. The coefficient $\alpha_j$ controls the contribution of each feature. The function $f_k(\cdot)$ then applies a non-linear transformation to the weighted sum of these powered features, generating the $k$-th target feature. This formulation constructs complex yet controlled relationships between the two tables, mirroring real-world scenarios where features across datasets are interconnected through intricate non-linear dependencies. 

Specifically, in this experiment, the first table includes the categorical feature $c_i \in {1,2,3,4,5}$, and numerical features $n_{i1}$ and $ n_{i2} \sim \mathcal{U}(0,1)$, drawn from a uniform distribution. The second table consists of three features derived through the following non-linear transformations:

\begin{equation}
\begin{aligned}
y_{i1} &= \sin(\pi(0.5n_{i1} + 0.3n_{i2} + 0.2c_i/5)) \\
y_{i2} &= \exp(0.4n_{i1} + 0.4n_{i2} + 0.2c_i/5) \\
y_{i3} &= \tanh(0.3n_{i1} + 0.3n_{i2} + 0.4c_i/5)
\end{aligned}
\end{equation}

These features first go through weighted, non-linear operations and then apply different non-linear transformations, including trigonometric, exponential, and hyperbolic functions. To ensure a balanced contribution from all features, the categorical variable $c_i$ is normalized by its maximum value of 5. This design preserves interpretability by distinctly separating the weighted non-linear combination from the subsequent transformations while still maintaining complex relationships.

We generate 6,400 unique samples using stratified sampling to ensure a balanced representation across all categorical values. The dataset is then divided into 5,120 samples (80\%) for training, 1,024 (16\%) for validation, and 256 (4\%) for testing. To prevent data leakage and ensure a robust evaluation, we enforce strict partitioning, ensuring that no sample appears in multiple splits while maintaining a balanced categorical feature distribution across all subsets. This synthetic ground truth dataset enables precise performance evaluation by offering known relationships, allowing for a systematic assessment of model capabilities under controlled complexity, free from noise and missing values.



\subsection{Performance evaluation of joint variable relationship learning}

We applied the proposed DCL model to learn the constructed complex joint relationships in the synthetic ground truth dataset. Performance evaluation in Table \ref{tab:synthetic_results} using Average Precision (AP) and Normalized Discounted Cumulative Gain (NDCG) demonstrates the superior effectiveness of our DCL model compared to the state-of-the-art SubTab model. 

\begin{table}[!ht]
  \centering
  \caption{Variable joining performance comparison on synthetic ground truth}
  \label{tab:synthetic_results}
  \begin{tabular}{@{}lcc@{}}
    \toprule
    \textbf{Model} & \textbf{Average Precision (AP)} & \textbf{NDCG} \\
    \midrule
    DCL (Ours) & \textbf{0.8690} & \textbf{0.9791} \\
    SubTab & 0.3044 & 0.8493 \\
    \bottomrule
  \end{tabular}
\end{table}

The DCL model achieved an AP of 0.8690, substantially outperforming SubTab’s 0.3044. The significant improvement in AP indicates that our model more accurately distinguishes between positive and negative matching pairs while maintaining higher precision across varying recall thresholds. Regarding ranking quality, our model attained an NDCG score of 0.9791, compared to SubTab’s 0.8493. This higher NDCG score underscores the DCL model’s ability to rank truly matching pairs more effectively, which is particularly important in applications such as housing-household matching, where ranking quality directly impacts decision-making. 

The superior performance of DCL demonstrates that the contrastive learning framework effectively captures meaningful representations, leading to more accurate matching between related entries. The performance gap between DCL and SubTab can be attributed to its dual-encoder architecture (Figure \ref{fig:DL_constrastive_model}). Unlike SubTab, which relies on a single encoder to reconstruct and represent the entire table, DCL employs a dual-encoder approach to generate distinct representations for each table, resulting in a substantial performance boost. Its strong results on the synthetic ground truth dataset validate its effectiveness, assuring its applicability to real-world scenarios where joint relationships are only partially observed, such as housing-household matching.

\section{Case Study in Delaware and North Carolina}\label{sec:results}

Building on DCL’s strong performance with synthetic ground truth, we applied it to learn the joint housing-household relationship. Delaware was used to evaluate the model’s accuracy in matching housing units with households, while North Carolina served as a test site to assess its transferability. The Delaware microdata consists of 18,641 housing-household co-occurrence records, while North Carolina’s dataset contains 198,037 entries.

\subsection{Microdata for housing-household relationship modeling and testing}

The microdata includes a wide range of variables. For this study, we selected 11 key variables each from the household and housing unit data (Table \ref{tab:selected_variables}), focusing on those assumed to be most relevant for understanding their relationship in our applications. We included tenure attributes for both housing units and households, as households can be renters or owners, and housing units can be rental or owned properties. This selection also enhances the model’s test set. Additional variables can be included as needed to better align with specific user needs and application goals.

\begin{table}[!ht]
  % \centering
  \caption{Selected variables for housing-household relationship modeling}
  % \small
  \resizebox{\textwidth}{!}{%
  \begin{tabular}{ll|ll}
\hline
\multicolumn{2}{c|}{\textbf{Housing unit}}          & \multicolumn{2}{c}{\textbf{Household}}                                \\ \hline
\textbf{Variable Name} & \textbf{Description}       & \textbf{Variable Name} & \textbf{Description}                         \\ \hline
ACR                    & Lot size                   & NP                     & Number of persons                            \\
BDSP                   & Number of bedrooms         & GRNTP                  & Gross rent                                   \\
BLD                    & Units in structure         & GRPIP                  & Gross rent as percentage of household income \\
MRGP                   & First mortgage payment     & HHL                    & Household language                           \\
RMSP                   & Number of rooms            & HHLDRAGEP              & Age of the householder                       \\
RNTP                   & Monthly rent               & HHLDRRAC1P             & Race of the householder                      \\
TEN\_U                 & Tenure                     & TEN\_H                 & Tenure                            \\
VALP                   & Property value             & HUPAC                  & Household presence and age of children       \\
VEH                    & Vehicles available         & R65                    & Presence of persons 65 years and over        \\
YRBLT                  & When structure first built & SCHL                   & Highest education attainment                 \\
TAXAMT                 & Property taxes             & DIS                    & Number of disabilities                       \\ 
                       &                            & HINCP                    & Household income \\ \hline
\end{tabular}
}
\label{tab:selected_variables}
\end{table}



Our dual-encoder contrastive learning model is designed to adapt to noisy training data. However, it is impractical to test the model on the enhanced microdata, from which the training data is derived, as clustering introduces invalid positive and negative household-housing pairs. To ensure reliable evaluation, we revert to the original microdata for model testing.

However, the lack of ground truth negative pairs in the original microdata limits our ability to fully assess the model’s discriminative capacity. Without negative test data, the model may develop a bias toward positive predictions, leading to inflated performance scores and preventing an accurate evaluation of false positive rates, precision, and recall.

Many existing studies address this challenge by relying on expert input to create their own test sets. For instance, Radford et al. \cite{radford2021learning} manually curated datasets for zero-shot evaluation instead of relying on potentially noisy, web-crawled data. Jia et al. \cite{jia2021scaling} annotated image classification datasets to ensure evaluation reliability. However, manually defining negative housing-household pairs is more complex than image-text labeling, as real-world exceptions are common. For example, a 20-year-old could be a college professor, or a household earning under \$75k might live in a million-dollar home due to inheritance. Given these complexities, we can only apply fundamental rules in housing-household relationships to curate negative pairs for testing.

In this research, we use tenure matching as the primary criterion for identifying false matches. Specifically, we structure the microdata so that tenure acts as a shared attribute between household and housing unit features, where a household can be either a renter or owner, and a housing unit can be rented or owned. If a renter is paired with an owned property, we classify this as a negative housing-household pair. Using the household and housing records in the microdata, we apply this rule to create a set of negative pairs based on mismatched tenure statuses.

We acknowledge that tenure mismatch $\{\mathtt{S}\}$ is only a subset of the constraints $\{\mathtt{C}\}$ defining negative housing-household relationships. However, this rule serves as a practical starting point, especially given the limited understanding of housing-household relationships, which is also the motivation behind this research. To ensure this rule-based subset serves as an effective test set of the model’s performance, we avoid labeling or sampling negative pairs based on tenure status when generating the training data, preventing this tenure constraint leak into the training process. We hope this subset will demonstrate the model's capability, and as more insights into housing-household relationships emerge, we can further refine the test set.

\subsection{Housing-household joining performance in Delaware}\label{sec:hu_matching}

We first tested the model using Delaware's microdata. Households in Delaware microdata were grouped into 3,000 clusters, with an average of 5 households per cluster. Figure \ref{fig:pr_curves}(a) presents the precision-recall curves comparing our DCL approach to the SubTab method. The curves show the superior performance of our method, which maintains high precision (>0.95) across a wide range of recall values. SubTab's precision diminishes rapidly as recall increases, indicating less reliable matching predictions. Overall, our method achieves an AP of 98.33\%, substantially outperforming SubTab's 38.62\%. This result highlights our method's superior ability to reliably and accurately identify positive housing-household matches. Our approach also excels in the NDCG metric (Figure \ref{fig:pr_curves}(b)), achieving 99.76\% compared to SubTab's 88.52\%. This highlights our model's effectiveness in ranking the most compatible housing unit for a given household at the top when calculating matching scores.

\begin{figure}[!ht]
  \centering
  \includegraphics[width=0.99\textwidth]{PR_Curve.pdf}
  \caption{Model performance}
  \label{fig:pr_curves}
\end{figure}

The superior performance of our method can be attributed to its tailored architectural design. Unlike SubTab, which is optimized for learning general representations across entire tables, our model is specifically designed to capture representations that determine whether a given household and housing unit form a valid match. This targeted approach enables our DCL model to effectively capture the intricate nuances of housing-household relationships.

\begin{figure}[!ht]
  \centering
  \includegraphics[width=\textwidth]{visualize_matching_scores.pdf}
  \caption{Housing-household matching illustration}
  \label{fig:matching_scores}
\end{figure}

Due to the high dimensionality of the learned housing-household relationship, it is impractical to summarize a rule that can describe which households are likely to match with specific housing units. Instead, we visually present the matching results for a randomly selected set of housing units and households (Figure \ref{fig:matching_scores}(a) and (b)). Figure \ref{fig:matching_scores}(c) illustrates cluster-based housing-household pairs. Households 2 and 3 are grouped into the same cluster as they are both high-income households with mortgages. Correspondingly, their associated housing units, 2 and 3, are also clustered together since they are mortgaged properties with multiple bedrooms. Due to the clustering, Household 2 (3) is also paired with Housing unit 3 (2). Figure \ref{fig:matching_scores}(d) shows our DCL model’s matched and unmatched housing-household pairs. Unlike traditional binary classification, our model assigns a continuous score to each pair, indicating their degree of fitness. This scoring approach provides richer insights compared to binary labels, enabling the model to capture subtle differences in compatibility between households and housing units. Moreover, our model generates a broader set of potential matches beyond those observed in the microdata. As previously discussed, microdata is a sampled dataset that does not capture all possible housing-household pairs. Additionally, the housing unit a household resides in is not necessarily the only viable option. By expanding the scope of possible matches, our model better reflects the broader range of housing choices available to households.


\subsection{Sensitivity analysis of cluster size}
\label{sec:sensitivity}

Clustering is a key step in our data preprocessing, making the choice of the number of clusters a critical factor for our model's performance. To evaluate the effect of cluster quantity on learning outcomes, we performed a sensitivity analysis. Table \ref{tab:cluster_impact} summarizes the model's performance across varying numbers of clusters. The results show that as the number of clusters increases, the model's performance initially improves, but beyond a certain threshold, it begins to decline. This indicates that a finer-grained clustering can better capture and enhance the learning of the housing-household relationship. Notably, even without clustering preprocessing (treating each housing unit and household as its own singleton cluster), our model maintained strong performance. 

\begin{table}[!ht]
  \centering
  \caption{Impact of Different Cluster Numbers on the Performance of Our Method}
  \label{tab:cluster_impact}
  \begin{tabular}{@{}lccccc@{}}
  \toprule
                           & \textbf{15,220} & \textbf{3,000} & \textbf{2,000} & \textbf{1,000} & \textbf{100} \\
                           & \textbf{Clusters} & \textbf{Clusters} & \textbf{Clusters} & \textbf{Clusters} & \textbf{Clusters} \\
                          \textbf{Medium number of} & & & & & \\ 
                          \textbf{households per cluster} & \textbf{Singleton} & 5 & 8 & 16 & 158 \\ \midrule
  \textbf{Average Precision (AP)} & 83.99\%             & \textbf{98.33\%}       & 97.45\%                & 61.68\%                & 43.97\%               \\
  \textbf{NDCG}              & 97.76\%             & \textbf{99.76\%}       & 99.65\%                & 92.48\%                & 89.65\%               \\ \bottomrule
  \end{tabular}
\end{table}

This phenomenon can be attributed to the inherent noise in the self-supervised training dataset. Because microdata only covers a sample (1\%-5\%) of the total population, without clustering, many potential positive housing-household pairs are incorrectly labeled as negative, resulting in a high rate of false negatives. Conversely, using large-scale clustering causes most housing-household pairs to be labeled as positive, significantly increasing false positives. As cluster size grows, the false positive rate escalates rapidly (see Figure \ref{fig:clustering_error}), leading to diminished model performance. By comparing these extreme conditions, we observe that the model handles fine-grained clustering more effectively than coarse clustering. Therefore, we adopt a strategy of small cluster sizes, with each cluster containing approximately 4–8 samples. In our experiment, 3000 clusters yield an excellent performance. This approach minimizes false negatives while controlling the false positive rate, striking a critical balance that preserves high model performance.

\subsection{Transferability test in North Carolina}

To assess our model's generalization capability across different geographic regions, we conducted transfer learning experiments in North Carolina (NC). The NC microdata contains 161,135 households, and based on insights from the Delaware sensitivity analysis, we maintained 3,000 clusters for the NC study, resulting in a cluster size of 53 households.

\begin{figure}[!ht]
  \centering
  \includegraphics[width=0.99\textwidth]{PR_Curve_transfer.pdf}
  \caption{Transferability test in North Carolina (NC)}
  \label{fig:transferability}
\end{figure}

The results (Figure \ref{fig:transferability}) show an outstanding transfer learning performance. Using the model trained on North Carolina microdata, we achieved an Average Precision of 97.92\% and NDCG of 99.80\%. This highlights the robustness of our proposed data preprocessing and DCL pipeline, suggesting that our approach effectively captures key features of housing-household relationships across varying sociodemographic and built environment contexts.

\subsection{Influential factors of housing-household joining}

To gain insights into how our deep learning model makes housing-household matching decisions, we utilize a post-hoc explainable AI technique \cite{gilpin2018explaining}. This approach enhances the interpretability of our trained DCL model without altering its structure or training process. The approach is applied post-training to provide clarity on the model's outputs and the reasoning behind its decisions. Such techniques are particularly valuable for "black-box" models like our deep contrastive model, which, despite their high accuracy, are inherently difficult to interpret. Ensuring interpretability is critical not only for validating the model but also for practical applications in housing policy, where understanding the reasoning behind matching decisions is as important as the decisions themselves.

\begin{figure}[!ht]
  \centering
  \includegraphics[width=0.99\textwidth]{shap.pdf}
  \caption{SHAP analysis of feature importance in housing-household matching}
  \label{fig:shap_summary}
\end{figure}


In this study, we utilized an XGBoost model as a post-hoc method to analyze feature contributions within our housing-household unit matching framework. To quantify the impact of each feature on the model's predictions, we employed SHAP (SHapley Additive exPlanations) values, which are widely recognized for their consistency and accuracy in interpreting complex models \cite{lundberg2018consistent}. Figure \ref{fig:shap_summary}(a) displays a beeswarm plot of SHAP values, illustrating the distribution and influence of individual feature values on the model’s predictions. Additionally, Figure \ref{fig:shap_summary}(b) summarizes the mean absolute SHAP values, providing a clear ranking of each feature’s overall importance.

The analysis reveals a distinct hierarchy in feature importance, with tenure-related attributes (HH\_TEN: 2.2141, HU\_TEN: 0.8683) and financial indicators (HU\_MRGP: 1.3854) playing dominant roles in the matching decisions. The significant variation in SHAP values for these key features (e.g., 2.4913 for HH\_TEN) with both positive and negative effects on predictions underscores the complexity of the housing-household matching process, which is highly context-dependent and influenced by the specific values of these features and their interactions with other variables.

More important, the analysis also reveals that physical attributes of housing units, such as the number of bedrooms (HU\_BDSP: 0.0749) and rooms (HU\_RMSP: 0.0640), have relatively modest importance compared to financial factors. This finding extends beyond existing housing-household matching research, which primarily focuses on building capacity and geospatial density, uncovering new key drivers in the alignment between households and housing units.


\section{Conclusion}
This paper presents a deep contrastive learning (DCL) pipeline (Figure \ref{fig:pipeline}) to establish a high-dimensional statistical relationship between housing units and households, addressing a critical gap in disaster impact analysis. The modeling of this relationship is challenging due to several factors: the absence of labeled datasets with embedded matching degrees, limited positive housing-household pairs with binary outcomes, a lack of information on negative housing-household pairings, the multi-modality issue arising from sociodemographic features describing humans versus structural features describing housing units, and the missing many-to-many pairing information between housing and households.

To tackle these challenges, we leverage the natural housing-household pairing available in microdata, employing co-occurrence as a pretext task for modeling the housing-household relationship (Figure \ref{fig:co-occurring}). To address data limitations, we propose a Bisecting K-Means clustering technique to generate essential training data (Figure \ref{fig:cluster} and \ref{fig:clustering_error}). Furthermore, in the absence of a labeled dataset, we implement a self-supervised learning approach—deep contrastive learning (DCL)—with a sigmoid contrastive loss function (Figure \ref{fig:DL_constrastive_model}). This technique enables the model to learn the many-to-many compatibility between housing units and households, producing a matching score that quantifies the degree of alignment between them.

To rigorously evaluate the performance of the proposed DCL variable joining framework, we constructed a synthetic ground truth with a known non-linear relationship between two tables. The controlled experiment demonstrates that our DCL framework can more effectively capture variable relationships compared to the state-of-the-art SubTab method (Table \ref{tab:synthetic_results}). This superior performance confirms its suitability for applying to microdata in housing-household relationship modeling.

The empirical experimental results in Delaware demonstrate that our approach outperforms state-of-the-art models across multiple metrics, including Average Precision (AP) and Normalized Discounted Cumulative Gain (NDCG), confirming its superior ability to identify and rank compatible housing-household pairs (Figure \ref{fig:pr_curves}). The model not only accurately predicts the matching pairs documented in the microdata but also generates additional potential matches (Figure \ref{fig:matching_scores}). Our analysis shows that the model performs most robustly and reliably when the cluster size is small (Table \ref{tab:cluster_impact}). To further validate the approach, we tested the pipeline in North Carolina, where it delivered similarly excellent performance (Figure \ref{fig:transferability}). To enhance the explainability of the proposed DCL model, we developed a post-hoc analysis using an XGBoost model to identify key factors influencing the housing-household matching process (Figure \ref{fig:shap_summary}). The results reveal that the tenure status of both the housing unit and household, along with the housing unit's mortgage information, are the most significant factors in mapping their relationships.

Despite the advancements achieved in this research, several limitations remain. First, the ACS dataset used is relatively small, representing only a 5\% sample of the actual population and housing units within Public Use Microdata Areas. This limitation may lead to gaps in geographic coverage in certain states, potentially introducing biases that affect the method's alignment with real-world scenarios. Future research could address this issue by integrating data from more diverse sources to improve dataset representativeness. Additionally, while our method for generating negative samples in the test set is innovative, it captures only a subset of the true negative instances. Future studies could investigate more sophisticated approaches to constructing a comprehensive test set, such as developing complex housing-household unmatching rules or engaging human experts to create a large-scale validation dataset. This enhancement would allow for a more robust evaluation of the model's performance.

This study represents a significant leap in housing-household matching by integrating innovative data preprocessing techniques and the DCL method. By moving beyond traditional binary allocation approaches, our method provides a nuanced degree of matching that transcends the basic physical and distributional constraints such as building capacity, population density, and sociodemographic marginal distributions. These advancements offer a more accurate representation of household distributions within the built environment, enhancing our understanding of urban dynamics and social interactions. The improved accuracy has broad implications, supporting more informed urban planning decisions, strengthening disaster response efforts, and deepening our understanding of societal structures within urban environments.


%%%%%%%%%%%%%%%%%%%%%%%%%%%%%%%%%%%%
% ACKNOWLEDGEMENTS
%%%%%%%%%%%%%%%%%%%%%%%%%%%%%%%%%%%%
\section*{Acknowledgments}
Shangjia Dong would like to acknowledge funding support from the National Science Foundation \#2443784. Rachel Davison would like to acknowledge funding support from the National Science Foundation \#2209190. Any opinions, conclusions, and recommendations expressed in this research are those of the authors and do not necessarily reflect the view of the funding agencies. The authors would also like to thank the editor and the anonymous reviewers for their constructive comments and valuable insights to improve the quality of the article. 

\bibliographystyle{plainnat}
% \documentclass[11pt,reqno]{amsart}
\documentclass[conference]{IEEEtran}
% \pdfoutput=1
%%%%%%%%%%%%%%%%%%%%%%%%%%%%%%%%%%%%%%%%%%%%%%%%%%%%%%%
%%%%%%%%%%%%%%%    theorems %%%%%%%%%%%%%%%%%%%%%%%%%%%
%%%%%%%%%%%%%%%%%%%%%%%%%%%%%%%%%%%%%%%%%%%%%%%%%%%%%%%
% \usepackage{mdframed}
\usepackage{kantlipsum}

%%%%%%%%%%%%%%%%%%%%%%%%%%%%%%%%%%%%%%%%%%%%%%%%%%%%%%%
%%%%%%%%%%%%%%%    theorems %%%%%%%%%%%%%%%%%%%%%%%%%%%
%%%%%%%%%%%%%%%%%%%%%%%%%%%%%%%%%%%%%%%%%%%%%%%%%%%%%%%
\theoremstyle{plain}
\newtheorem{theorem}{Theorem}[section]
\newtheorem{proposition}[theorem]{Proposition}
\newtheorem{lemma}[theorem]{Lemma}
\newtheorem{example}[theorem]{Example}
\newtheorem{corollary}[theorem]{Corollary}
\theoremstyle{definition}
\newtheorem{definition}[theorem]{Definition}
\newtheorem{assumption}[theorem]{Assumption}
\theoremstyle{remark}
\newtheorem{remark}[theorem]{Remark}


% \titleformat{\subsection}[runin]% runin puts it in the same paragraph
%        {\normalfont\bfseries}% formatting commands to apply to the whole heading
%        {\thesubsection}% the label and number
%        {0.5em}% space between label/number and subsection title
%        {}% formatting commands applied just to subsection title
%        [.]% punctuation or other commands following subsection title


%%%%%%%%%%%%%%%%%%%%%%%%%%%%%%%%%%%%%%%%%%%%%%%%%%%%%%%
%%%%%%%%%%%%%%%  mathematical notations%%%%%%%%%%%%%%%%
% \usepackage[english]{babel}
% \usepackage[utf8]{inputenc}
% \usepackage[T1]{fontenc}

%% Figures, tables and lists
\usepackage[dvipsnames]{xcolor}
\usepackage{paralist}
\usepackage{graphicx}
\usepackage{subcaption}
\usepackage{longtable} 
\usepackage{multirow}
\usepackage{listings}
\usepackage{makecell}
\usepackage{array}
\usepackage{float}
\usepackage{dsfont}
\usepackage{rotating}
\usepackage{booktabs}
\usepackage{enumerate}
\usepackage{tikz}
\usepackage{pgf}
\usepackage{enumitem}
\usepackage{lipsum} % for generating filler text
\usepackage{titlesec}

%% Math
% \usepackage{amssymb, amsthm,bbm}
\usepackage{mathtools}
\usepackage{mathrsfs}
%% References and author info 
\mathtoolsset{showonlyrefs}
\usepackage{natbib}
\usepackage{authblk}
\usepackage{todonotes}
\usepackage{xr-hyper}


%%%%%%%%%%%%%%%%%%%%%%%%%%%%%%%%%%%%%%%%%%%%%%%%%%%%%%%
\newcommand{\R}{\mathbb R}
\newcommand{\EE}{\mathbb{E}}

\DeclareMathOperator{\Tr}{Tr}
\DeclareMathOperator*{\argmin}{argmin}
\DeclareMathOperator*{\argmax}{argmax}

\newcommand{\bs}[1]{\ensuremath{\boldsymbol{#1}}}
\newcommand{\mc}{\mathcal}
\newcommand{\opt}{^\star}


\newcommand{\diff}{\textnormal{d}}


\def \iid {\stackrel{\textnormal{i.i.d.}}{\sim}}
\def \iidtext {\textnormal{i.i.d.}}





%%%%%%%%%%%%%%%%%%%%%%%%%%%%%%%%%%%%%%%%%%%%%%%%%%%%%%%
%%%%%%%%%%%%%%%%%%%%% colors     %%%%%%%%%%%%%%%%%%%%%%
%%%%%%%%%%%%%%%%%%%%%%%%%%%%%%%%%%%%%%%%%%%%%%%%%%%%%%%
\definecolor{myblue}{rgb}{.8, .8, 1}
\definecolor{mathblue}{rgb}{0.2472, 0.24, 0.6} % mathematica's Color[1, 1--3]
\definecolor{mathred}{rgb}{0.6, 0.24, 0.442893}
\definecolor{mathyellow}{rgb}{0.6, 0.547014, 0.24}


% May add more in future.






\usepackage{enumitem,diagbox}
\usepackage{stfloats}
\newcommand{\nnote}[1]{{\highlightname{Nolan}{#1}{neworange}}}
\newcommand{\snote}[1]{{\highlightname{AB}{#1}{newred}}}
\renewcommand\thesection{\arabic{section}} 
\renewcommand\thesubsectiondis{\thesection.\arabic{subsection}}
\renewcommand\thesubsubsectiondis{\thesubsectiondis.\alph{subsubsection}}
\renewcommand\theparagraphdis{\arabic{paragraph}.}
\setlength{\abovedisplayskip}{3.5pt}
\setlength{\belowdisplayskip}{3.5pt}
\usepackage{cite}
%\setlength{\topsep}{0pt plus3pt minus0.5pt}
%\newcommand{\nnote}[1]{}

\newcommand{\bfsl}{\bfseries\slshape}
\newcommand{\bfit}{\bfseries\itshape}
\newcommand{\sfsl}{\sffamily\slshape}
\newcommand{\dfn}{\sffamily\slshape\small}

\newcommand\nnfootnote[1]{%
   \begin{NoHyper}
    \renewcommand\thefootnote{}\footnote{#1}%
    \addtocounter{footnote}{-1}%
   \end{NoHyper}
}
\makeatletter
\newcommand\footnoteref[1]{\protected@xdef\@thefnmark{\ref{#1}}\@footnotemark}
\makeatother
%\addtolength{\skip\footins}{-.05in}
\pagestyle{plain}

\usepackage{balance}

\title{Classical and quantum Coxeter codes:\\ Extending the Reed--Muller family}
\author{%
   \IEEEauthorblockN{{\sc Nolan J. Coble}  \qquad    {\sc Alexander Barg}}
    \IEEEauthorblockA{University of Maryland, College Park, USA}
 }
\date{}

\setstretch{1.}

\begin{document}
\maketitle

\begin{abstract}
We introduce a class of binary linear codes that generalizes the Reed--Muller family by replacing the group $\ZZ_2^m$ with an arbitrary finite Coxeter group. Similar to the Reed--Muller codes, this class is closed under duality and has rate determined by a Gaussian distribution. We also construct quantum CSS codes arising from the Coxeter codes, which admit transversal logical operators outside of the Clifford group.
\vspace*{-.17in}
\end{abstract}

\nnfootnote{N.C. was partially supported by NSF grant DMS-2231533. A.B. was supported in part by NSF grant CCF-2330909.}



% \newpage
% \tableofcontents % uncomment to include table of contents page


% \printlen[10][cm]{\linewidth}


%%%%%%%%%%%%%%%%%%%%%%%%%%%%%%%%%%%%%%%%%%%%%%%%
%%%%%%%%%%%%%%%%%%%%%%%%%%%%%%%%%%%%%%%%%%%%%%%%
%%%%%%%%%%%%%%%%%%%%%%%%%%%%%%%%%%%%%%%%%%%%%%%%

\section{Introduction}
Reed--Muller (RM) codes form a classic family studied for its interesting algebraic and combinatorial properties \cite{MS77,Assmus98} as well as from the perspective of information transmission \cite{YeAbbe2020,abbe2023reed}. 
They achieve Shannon capacity of the basic binary channel models such as channels with independent erasures or flip errors
\cite{Kudekar2015ReedMullerCA,AbbeSandon2023}. They also give rise to a large family of quantum codes \cite{Steane1999} with well-understood logical operators \cite{kubica2015universal,campbell_magic-state_2012,rengaswamy2020optimality,barg2024geometric}. This motivated us to look into possible extensions of the RM
code family, viewing them as codes in the Coxeter complex of the group $\ZZ_2^m$. The starting point of this
research is a realization than an RM code $RM(r,m)$ is spanned by (the indicator vectors of) the $(m-r)$-dimensional faces
of the $m$-dimensional Boolean cube, i.e., the Cayley graph of the group $\ZZ_2^m$. Once we adopt this description, the next step is to replace $\ZZ_2^m$ with an arbitrary (finite) Coxeter group, $W$. 
Coxeter groups naturally give rise to Cayley graphs, which are $m$-dimensional polytopes whose faces are
themselves defined through Coxeter subgroups. These polytopes and their suitable generalizations are often studied in combinatorial group theory \cite{BB05,AB08}. We define a Coxeter code of order $r$ as a
binary linear code obtained as an $\FF_2$-linear span of the set of $(m-r)$-dimensional faces. We show that the duality relation $RM(r,m)^\bot=RM(m-r-1,m)$ extends to all Coxeter codes. We also find the dimension of the codes
in terms of the $W$-polynomial of the group, whose components are given by Eulerian numbers associated to $W$ \cite{petersen2015eulerian,BB05}. Codes arising from Coxeter systems, such as the one from the permutation group, exhibit dependence of the rate on the parameters $m,r$ similar to that of RM codes; in particular, the asymptotic behavior of the rate parallels that of RM codes.

One of the motivations to study Coxeter codes is derived from our earlier work \cite{barg2024geometric}, which 
explored the structure of quantum RM codes and their transversal logical gates in terms of the faces of the cubical complex (cosets of $\ZZ_2^m$). In \cref{sec:quantum} we extend some of the results of \cite{barg2024geometric} to Coxeter codes.


\vspace*{.05in}
\noindent{\bfit 1.1. Reed-Muller codes.} Let $\FF\coloneqq\FF_2$ be the binary field and let $S_m=\br{e_1,\dots,e_m}$ be the standard basis of the $m$-dimensional cube $\ZZ_2^m$.
A \emph{standard $\ell$-cube} is a subgroup $\langle J \rangle\leq\ZZ_2^m$ spanned by a subset $J\subseteq S_m, |J|=\ell$. An \emph{$\ell$-cube} is a shift of a standard $\ell$-cube, i.e., a set $x+\langle J \rangle,$ where $x\in\ZZ_2^m$. 
%
\begin{definition}\label{def: RM}
    For $m\ge 2, r\in \{-1,0,\dots,m\}$ let 
    $$
    H_i:=\{x+\langle J \rangle\mid  x\in\ZZ_2^m, J\subseteq S, |J|=i\}.
    $$
The \emph{order-$r$ Reed-Muller code} $RM(r,m)$ is the $\FF$-linear subspace of $\FF^{2^m}$ spanned by the indicators of the $(m-r)$-subcubes, $RM(r,m)=\standard{\1_A, A\in H_{m-r}}$ \cite{barg2024geometric}.\footnote{\cite{barg2024geometric} may be not the first place to define RM codes in this way, although we are not aware of earlier references.} Note that $H_{m+1}=\emptyset$ and $RM(-1,m)=0^{2^m}$.
\end{definition}
%(\!\!\cite{barg2024geometric} may be not the first place to define RM codes in this way, although we are not aware of earlier references).
Other definitions of Reed-Muller codes rely on evaluations of polynomials of $m$ variables \cite{MS77} or the group algebra formalism \cite{willems2021codes}. We mention the second of these because the generalization of RM codes we consider is based on the perspective of combinatorial group theory, for which \cref{def: RM} is particularly
well suited.

% \subsection{Group algebra codes} Let $G$ be a finite group and let $\FF$ be a finite field. The group algebra $\FF G$ is an $\FF$-vector space $\{a=\sum_{g\in G} a_g g\mid a_g\in \FF\}$ and a {\em group code} is a (right) ideal in $\FF G$. Group codes were introduced by Berman \cite{berman1967theory} and MacWilliams \cite{macwilliams1970binary}; see \cite{willems2021codes} for a recent overview of the literature on them. Standard examples of group codes include cyclic codes and binary RM codes, obtained from cyclic groups and $\ZZ_2^m$, respectively. Another example often studied in the literature is codes obtained from the dihedral group $D_{2n}$ \cite{VD21,sales2024codes}. 
% While we consider code vectors as elements of the group algebra, our approach departs from these references by adopting a combinatorial geometry perspective rather than focusing on ideals in group algebras.



\section{Coxeter systems and codes}


\noindent
{\bfit 2.1 Coxeter systems.} Before we introduce the Coxeter code family (\cref{def: Coxeter codes}), we will prepare the combinatorial background, listing several facts about Coxeter systems in
the form and level of generality suitable for our needs. A more general presentation of finite Coxeter systems appears in  comprehensive references \cite{AB08,BB05}.

\begin{definition}
    Let $S\coloneqq\br{s_1,\dots,s_m}$ be a set of $m<\infty$ letters and consider the group, $W$, given by the presentation
    $$
        W\coloneqq \left\langle S\Bigmid (s_i s_j)^{M(i,j)}=1 \right\rangle,
    $$
    where %$M(i,j)=\text{ord}(s_is_j),$ 
    $M(i,i)=1$ and $M(i,j)=M(j,i)\in \ZZ_{\geq 2}$. %By convention, $M(i,j)=\infty$ means that there is no relation between between $s_i$ and $s_j$.
    We say that $W$ is a \emph{Coxeter group} and that the pair $(W,S)$ is a \emph{Coxeter system}. The cardinality $\abs{S}=m$  is called the \emph{rank} of $(W,S)$. Throughout, we will assume that $W$ is a finite group (finite Coxeter groups, a.k.a. finite reflection groups, are completely classified \cite[App.A.1]{BB05}).
\end{definition}

\begin{definition}[\sc Standard subgroups and cosets]
    For a fixed Coxeter system, $(W,S)$, and a subset $J\subseteq S$ of generators, the subgroup $\langle J\rangle\leq W$ is called a \emph{standard subgroup of 
    $W$}, and the \emph{type} of $\langle J\rangle$ is $J$. In particular, $(\standard{J},J)$ is a Coxeter system in its own right. % We will denote standard subgroups of type $J$ by $\standard{J}\coloneqq\langle J\rangle$. 
    A \emph{standard (left) coset} of $W$ is any coset of the form $R\coloneqq \sigma\standard{J}$ for $\sigma\in W$, $J\subseteq S$, with $J$ referred to as the \emph{type} of the coset. The \emph{rank} of $R=\sigma\standard{J}$ is $\rank (R)\coloneqq \abs{J}$.
    The collection of all standard cosets is denoted by $\Sigma\coloneqq \br{\sigma\standard{J}\mid \sigma\in W,\; J\subseteq S}$.
\end{definition}


\begin{definition}[\sc Cayley graphs]
    The \emph{(right) Cayely graph} of a finite Coxeter system, $(W,S)$, is a graph $\mcG(W,S)$ whose vertices are indexed by elements of $W$, and for $r,t\in W$, there is an edge between them whenever $t=rs_i$ for some $s_i\in S$. Since each $s_i$ is an involution, $\mcG(W,S)$ is an undirected graph and each vertex of $\mcG(W,S)$ is incident to precisely $m$ edges, one for each generator $s_i\in S$.
\end{definition}

\begin{remark}
    The Cayley graph $\mcG(W,S)$ of any Coxeter system of rank-$m$ is a polytope in the $m$-dimensional space, with the $i$-dimensional faces corresponding to the rank-$i$ standard cosets of $(W,S)$. For example, $\mcG(\ZZ_2^m,S_m)$ is simply an $m$-dimensional hypercube. For the symmetric group on $m+1$ elements, $A_m\coloneqq(\mathrm{Sym}(m+1),S)$, generated by adjacent transpositions $S\coloneqq\br{(i\;\;\, i+1)\mid i\in[m]}$, $\mcG(A_m)$ is an $m$-dimensional \emph{permutohedron}. See \cref{fig: permutohedron}.\hfill$\lhd$
\end{remark}



\begin{figure}[t]
    \centering
    \includegraphics[width=.8\linewidth]{images/A3_descents.pdf}
    \caption{The Cayley graph $\mcG(A_3)$ for the symmetric group on 4 letters is a 3-dimensional polytope called a permutohedron. The dark gray vertex is the identity element of the group, and the three colored edges indicate right multiplication by a pairwise swap, $(i\;\;i+1)$. The vertices are labeled with the descent number of the corresponding group element (\cref{def:descents}).\vspace{-0.5em}}
    \label{fig: permutohedron}
\end{figure}


\noindent{\bfit 2.2. Coxeter codes.}
We will now use the structure of Coxeter systems and standard cosets to build a family of linear codes that generalizes the RM family. Throughout, we assume that $(W,S)$ is a finite Coxeter system of rank $m$ and we denote the binary field by $\FF\coloneqq\FF_2$ and $n\coloneqq\abs{W}$. Consider the group algebra $\FF W$, which is an $n$-dimensional vector space over $\FF$ whose elements are of the form $v=\sum_{w\in W} c_w w$, for $c_w\in\FF$. We can view $\FF W$ as a vector space whose basis vectors are indexed by vertices of the Cayley graph $\mcG(W,S)$. We will not make explicit use of $\mcG(W,S)$, though it is a useful picture to keep in mind.
By abuse of notation we will consider each standard coset $R\coloneqq\sigma\standard{J}$ as an element of $\FF W$ by setting
$
    R\coloneqq \sum_{w\in R} 1\cdot w
$
and conflating subsets and their indicators.

\begin{figure}[t]
    \centering
    \includegraphics[width=.8\linewidth]{images/A3_codeword.pdf}
    \caption{The codes $\coxeter{A_3}{S}{1}\subset \coxeter{A_3}{S}{2}$ are generated by the faces and edges 
    of the permutohedron $\mcG(A_3)$, respectively. The  bit assignment shown in the figure represents the codeword in $\coxeter{A_3}{S}{1}$ generated by the colored hexagonal and square faces. The same codeword within
    the code $\coxeter{A_3}{S}{2}$ is equivalently generated by the three solid red edges.\vspace{-0.15em}}
    \label{fig: codes}
\end{figure}

% \vspace{1em}
\begin{definition}\label{def: Coxeter codes}
    Given $r\in\br{-1,\dots,m}$, the \emph{order-$r$ Coxeter code of type $(W,S)$}, $\Coxeter{r}$, is defined to be the $\FF$-linear span
    of all rank-$(m-r)$ standard cosets in $\Sigma$,
    $$
        \Coxeter{r} \coloneqq \Bigg\{\sum_{\substack{R\in \Sigma,\\ \rank(R)=m-r}} c_R R\Biggmid c_R\in\FF\Bigg\}.
    $$
    See \cref{fig: codes} for an illustration.
\end{definition}

\begin{remark}
    \hspace{0em}
    \begin{itemize}[leftmargin=*]
        \item The elementary Abelian 2-group $\ZZ_2^m$ with its standard generating set $S\coloneqq \br{e_i}_{i\in[m]}$ is a finite Coxeter system of rank $m$. As remarked above, the order-$r$ Coxeter code of type $(\ZZ_2^m,S)$ is, in fact, the code
        $RM(r,m)$.
        % Coxeter codes are thus a broad class of error-correcting codes which generalize the classic Reed--Muller family.
        \item For every Coxeter system the code $\Coxeter{-1}=0^{\abs{W}}$ is the trivial $\abs{W}$-bit code (given by an empty generating set), the code $\Coxeter{0}$ is the $\abs{W}$-bit repetition code, the code $\Coxeter{m-1}$ is the $\abs{W}$-bit single parity-check code and the code $\Coxeter{m}=\FF W$ is the entire vector space $\FF^{|W|}$. 
        \item The collection $\Sigma$ is invariant under the left action of $W$, so Coxeter codes are ideals in the group algebra $\FF W$. \hfill$\lhd$
    \end{itemize}
\end{remark}

% \vspace{1em}

We prove in Section~\hyperlink{sec: basis}{3.1} that some well-known structural results about the RM family extend to \emph{any} Coxeter code. First, Coxeter codes are a nested family of codes:
\begin{theorem}\label{thm: nested}
    For integers $q\leq r\leq m$, the order-$q$ Coxeter code of type $(W,S)$ is contained in the order-$r$ code:
    $$
        \Coxeter{q}\subseteq\Coxeter{r}.
    $$
\end{theorem}
The intuition for \cref{thm: nested} is that any coset $\sigma\standard{J}$ with $\abs{J}>m-r$ can be partitioned into $\abs{S}/\abs{J}$ cosets $\sigma_i\standard{J'}$ where $J'\subseteq J$ is any choice of $\abs{J'}=m-r$ elements in $J$. 

Like RM codes, Coxeter codes are also closed under duality:
\vspace{-1em}
\begin{theorem}\label{thm: Coxeter duality}
    The dual of the order-$r$ Coxeter code of type $(W,S)$ is the corresponding order-$(m-r-1)$ Coxeter code:
    $$
        \Coxeter{r}^\perp = \Coxeter{m-r-1}.
    $$
\end{theorem}

\begin{figure}[t!]
    \centering
    \includegraphics[width=.8\linewidth]{images/A3_extensions.pdf}
    \caption{The solid blue hexagon and the dashed red edge adjacent to the vertex $w_1$ represent the extension $R_{w_1}$ and reverse extension $\overline{R}_{w_1}$, respectively, of $w_1$. The solid blue edge and the dashed red square adjacent to the vertex $w_2$ represent the extension $R_{w_2}$ and reverse extension $\overline{R}_{w_2}$, respectively, of $w_2$. For $i\in\br{1,2}$, $w_i$ is the unique element of $R_{w_i}$ closest to the identity (dark vertex) and the unique element of $\overline R_{w_i}$ farthest from the identity (\cref{lem: unique shortest longest}).\vspace{-0.6em}
    }    \label{fig: extensions1}
\end{figure}

\section{Code parameters}

% \noindent{\bfit 3.1 Combinatorics of Coxeter systems.}
Coxeter systems carry a natural \emph{length function}, $\ell\coloneqq W\rightarrow \NN$, where the length of an element, $w$, is the smallest number of elements from $S$ needed to generate $w$. That is, $\ell(w)$ is the smallest $\ell'$ for which there is a decomposition $w=s_{i_1} s_{i_2}\dots s_{i_{\ell'}}$, where each $i_j\in [m]$, and \emph{any} such decomposition of $w$ must contain at least $\ell'$ elements. Even though $l(w)$ is well defined, there usually are multiple ways of writing $w$ as a word of length $l(w)$. The length function satisfies the following natural properties:
\begin{fact}[\cite{BB05}, Prop.1.4.2]
    The length function satisfies:
    \begin{enumerate}
        \item $\ell(e)=0$,
        \item $\ell(w^{-1})=\ell(w)$ for all $w\in W$,
        \item $\ell(w_1 w_2) \leq \ell(w_1)+\ell(w_2)$ for all $w_1,w_2\in W$, and
        \item $\ell(ws)=\ell(w)\pm 1$ for all $w\in W$, $s\in S$.
    \end{enumerate}
    In particular, property (4) implies that multiplication of an element in $W$ by a generator necessarily changes the length of the element.
\end{fact}


%\vspace*{.05in}

\begin{figure}[t!]
    \centering
    \includegraphics[width=.8\linewidth]{images/torus_extensions.pdf}
    \caption{Consider the Coxeter system $A_2\times A_2$ given by the direct product of two copies of the symmetric group on $3$ letters. The solid blue strip and the dashed red edge adjacent to the vertex $w_1$ represent the extension $R_{w_1}$ and reverse extension $\overline{R}_{w_1}$, respectively, of $w_1$. The solid blue square and the dashed red square adjacent to the vertex $w_2$ represent the extension $R_{w_2}$ and reverse extension $\overline{R}_{w_2}$, respectively, of $w_2$.}
    \label{fig: extensions2}
\end{figure}
\begin{definition}\label{def:descents}
    For $w\in W$, the subset of generators $D(w)\subseteq S$, defined as
    \begin{align*}
        D(w)\coloneqq \br{s\in S\bigmid \ell(ws)<\ell(w)},
    \end{align*}
    is called the (right) \emph{descent set} of $w$. The value $d(w)\coloneqq\abs{D(w)}$ is called the (right) \emph{descent number} of $w$. The $W$-Eulerian numbers \cite{petersen2015eulerian}, \cite[Sec.7.2]{BB05}, denoted $\euler{W}{i}$, correspond to numbers of elements in $W$ with particular descent numbers,
    \begin{equation*}
        \euler{W}{i}:=|\{{w\in W\Bigmid d(w)=i}\}|,
    \end{equation*}
    and satisfy the so-called {\em Dehn--Sommerville equations}
    \begin{equation*}
        \euler{W}{i} = \euler{W}{m-i},\tag{\hypertarget{eq:DS}{DS}}
    \end{equation*}
    for all $i\in\br{0,\dots,m}$. We note that the above definitions are dependent on the choice of the generating set $S$, but we suppress this dependence in the notations for simplicity, as is standard. 
\end{definition}
\begin{remark}
    If $W=\ZZ_2^m$ then $\euler Wi=\binom mi$. If $(W,S)=A_m$ then $\euler W i$ is the classic Eulerian number, i.e., the count of permutations in $W$ with $i$ descents \cite[p.6]{petersen2015eulerian}. See \cref{sec: computing Eulerian numbers} for expressions computing $W$-Eulerian numbers for reducible and irreducible Coxeter systems. \hfill$\lhd$
\end{remark}
\vspace{-0.2em}

The dimension of a Coxeter code is given by the sum of $W$-Eulerian numbers:
\vspace{-0.3em}
\begin{theorem}\label{thm: dimension of order r}
    The dimension of the order-$r$ Coxeter code of type $(W,S)$ is given by
    \begin{equation}\label{eq:dimension}
        \dim \Coxeter{r} = \sum_{i=0}^r \euler{W}{i}.
    \end{equation}
\end{theorem}
    For the RM case when $W=\ZZ_2^m$, this recovers the standard formula $\dim RM(r,m)=\sum_{i=0}^r\binom mi$. 
 We prove \cref{thm: dimension of order r} by constructing a basis of $\Coxeter{r}$.

\begin{definition}
    For $w\in W$, the coset $R_w\coloneqq w\standard{S\setminus D(w)}$ is called the \emph{extension} of $w$ in $W$. The coset $\overline R_w\coloneqq w\standard{D(w)}$ is called the \emph{reverse extension} of $w$ in $W$. Note that $\rank(R_w) = m-d(w)$ and $\rank (\overline R_w)=d(w)$. See \cref{fig: extensions1,fig: extensions2}.
\end{definition}



\begin{definition}
The set of all extensions (reverse extensions) in $W$ is denoted by $\mcB$ ($\overline\mcB$). For $i\in\br{0,\dots,m}$, let $\overline \mcB_i$ and $\overline{\mcB_i}$ denote the subset of extensions and reverse extensions of rank equal to $i$, which are, in turn, given by
    \begin{align*}
        \mcB_i=\br{R_w \bigmid d(w)=m-i},\;\;
        \overline{\mcB}_i =\br{\overline{R}_w \bigmid d(w)=i}.
    \end{align*}
    By the Dehn--Sommerville equations, we have
    \begin{align*}
        \abs{\mcB_{m-i}} = \abs{\overline\mcB_{m-i}}
        = \abs{\mcB_{i}} = \abs{\overline\mcB_{i}} =\euler{W}{i}.
    \end{align*}
\end{definition}



\vspace*{.03in}
\begin{lemma}\label{lem: unique shortest longest}
    For $w\in W$, $w$ is the unique shortest (resp. longest) element of its extension (resp. reverse extension). 
\end{lemma}
\begin{proof}
   Let $w'\in \standard{D(w)}, w'\ne e$. Proposition 2.17 of \cite{AB08} states that $\ell (w'w)=\ell(w)-\ell(w')>0$,
proving the claim for the reverse extension. This proposition also 
    implies that $\ell(w)>\ell(ww')$ for all $w'\in\standard{D(w)}\setminus\br{e}$. Further, \cite[Prop.2.20]{AB08} states that the minimal element $w_1\in w\standard{S\setminus D(w)}$ is uniquely characterized by the property $\ell(w_1s)=\ell(w_1)+1$ for all $s\in S\setminus D(w)$, which is satisfied by $w$ by construction of $D(w)$. \end{proof}

\begin{lemma}\label{lem: lemma for independence}
    Let $w\in W$. If $U\subseteq W\setminus\br{w}$ is a subset satisfying $\ell(u)\geq\ell(w)$ for all $u\in U$, then $w\notin R_u$ for any $u\in U$.
\end{lemma}
\begin{proof}
    Suppose $w\in R_u$ for some $u\in U$. As $w\neq u$, \cref{lem: unique shortest longest} implies that $\ell(w)>\ell(u)$, contradicting the assumption on $U$.
\end{proof}



Lastly, the following two simple results will be crucial in proving duality.

\begin{fact}\label{fact: even order}
    A non-trivial, finite Coxeter group has even order.
\end{fact}
\begin{proof}
   As the order of any $s\in S\neq\emptyset$ is 2, the result holds by Lagrange's theorem.
\end{proof}
As the intersection of two cosets is either empty, or a coset of the intersection of the component subgroups, for two standard cosets we have the following:
\begin{lemma}\label{lem: even overlap}
    Let $\sigma_1\standard{J_1}$ and $\sigma_2\standard{J_2}$ be two standard cosets. If $\abs{J_1}+\abs{J_2}>m$ then $\abs{\sigma_1\standard{J_1}\cap \sigma_2\standard{J_2}}$ is even.
\end{lemma}
\begin{proof}
    The result is true if the cosets have trivial overlap. Otherwise, $\sigma_1\standard{J_1}\cap \sigma_2\standard{J_2} = \sigma\standard{J_1\cap J_2}$ for some $\sigma \in W$. As $\abs{J_1}+\abs{J_2}>m$ but $\abs{J_1},\abs{J_2}\leq m$, the intersection $J_1\cap J_2$ is non-empty. Thus $\abs{\sigma_1\standard{J_1}\cap \sigma_2\standard{J_2}}=\abs{\standard{J_1\cap J_2}}$, which is even by \cref{fact: even order}.
\end{proof}

\vspace*{.03in}
\noindent{\bfit \hypertarget{sec: basis}{3.1.} A basis for the code. }%\label{sec: structural}
For $r\in\br{-1,\dots,m}$ consider the collection of extensions with rank at least $m-r$, $\mcB_{\geq m-r}\coloneqq\bigcup_{i\geq m-r}\mcB_i$. For the RM case when $W=\ZZ_2^m$, this collection is precisely the evaluations of monomials in $m$ variables with degree at most $r$, i.e., the standard RM basis. We will prove that $\mcB_{\geq m-r}$ is a basis for the order-$r$ Coxeter code of type $(W,S)$, from which \cref{thm: nested,thm: dimension of order r,thm: Coxeter duality} will all follow. 

\begin{lemma}\label{lem: independence}
    The collection $\mcB$ is linearly independent.
\end{lemma}
\begin{proof}
    Suppose for contradiction that $\sum c_u R_u = 0$ is a non-trivial relation on $\mcB$. As $W$ is finite, there must exist a $w\in W$, $c_w\neq 0$, whose length is minimal among the elements with non-zero coefficients. Denoting the set $U\coloneqq\br{u\in W\mid c_u\neq 0,\ u\neq w}$, this means that $\ell(w)\leq\ell(u)$ for all $u\in U$. By the linear relation we further have that \begin{equation}\label{eq: relation}
        R_w=\sum_{u\in U}R_u.
    \end{equation}
    We can apply \cref{lem: lemma for independence} to the set $U$, which implies that the element $w$ does not appear on the RHS of \cref{eq: relation}. However, $w$ clearly appears on the LHS of \cref{eq: relation}, 
    making this equality impossible.
\end{proof}

This, of course, implies that the $\mcB_{\geq m-r}$ are linearly independent, as well. It also implies that $\abs{\mcB} = \abs{W} = \sum_{i=0}^m\euler{W}{i}$.

We now show that the span of $\mcB_{\geq m-r}$ satisfies the desired duality structure. 
As we are treating standard cosets $R_1,R_2$ as elements of $\FF W$, we can consider their dot product $R_1\cdot R_2 = \abs{R_1\cap R_2}\pmod{2}$.

\begin{lemma}\label{lem: basis duality}
    For each $r\in\br{-1,\dots,m}$ we have
    \begin{equation*}
        \Span \mcB_{\geq m-r} \subseteq \left(\Span \mcB_{\geq r+1}\right)^\perp.
    \end{equation*}
\end{lemma}
\begin{proof}
    We must show that for each $R_1\in \mcB_{\geq m-r}$ and each $R_2\in \mcB_{\geq r+1}$ we have $R_1\cdot R_2=0$. By construction, the rank of $R_1$ is at least $r_1\geq m-r$ and similarly the rank of $R_2$ is at least $r_2\geq r+1$. As $r_1+r_2 > m$, the result holds by \cref{lem: even overlap}.
\end{proof}




Using the symmetry of $W$-Eulerian numbers given by the DS equations, these two spaces are, in fact, equal:
\begin{lemma}\label{lem: dual spans}
    For each $r\in\br{-1,\dots,m}$ we have
    \begin{equation*}
        \Span \mcB_{\geq m-r} =\left(\Span \mcB_{\geq r+1}\right)^\perp.
    \end{equation*}
\end{lemma}
\begin{proof}
    By \cref{lem: basis duality} and the fact that $\dim C+\dim C^\perp = n$ for all length-$n$ linear codes, we simply must show that $\dim (\Span \mcB_{\geq m-r}) +\dim(\Span \mcB_{\geq r+1} )= \abs{W}$. Using (\DS) and the linear independence of $\mcB_{\geq m-r}$, we have
    \begin{align*}
        \dim (\Span \mcB_{\geq m-r}) &= \sum_{i=m-r}^m \euler{W}{i} %\since{\DS}
        \stackrel{\rm (\DS)}=\sum_{i=0}^{r} \euler{W}{i} %\sum_{i=m-r}^m \euler{W}{m-i},\\
 %       &= \sum_{i=0}^{r} \euler{W}{i},
    \end{align*}
    where we have reindexed the summation to start from 0. Thus,
    \begin{align*}
        \dim (\Span \mcB_{\geq m-r}) +\dim(\Span \mcB_{\geq r+1})%\hspace{4em}\\
        %= \sum_{i=0}^{r} \euler{W}{i} +\sum_{i=r+1}^{m} \euler{W}{i} 
        = \abs{W},
    \end{align*}
    as desired.
\end{proof}

Finally, we have the following:
\begin{lemma}\label{lem: extensions form a basis}
    For $r\in\br{-1,\dots,m}$, $\mcB_{\geq m-r}$ is a basis for the order-$r$ Coxeter code of type $(W,S)$:
    \begin{equation}\label{eq: basis for Coxeter code}
        \Coxeter{r} = \Span \mcB_{\geq m-r}.
    \end{equation}
\end{lemma}
\begin{proof}
    ($\supseteq$) Recall that $\Coxeter{r}$ is the span of \emph{all} standard cosets with rank exactly equal to $m-r$. Consider an $R_w\in \mcB_{\geq m-r}$, which by definition is equal to $R_w=w\standard{S\setminus D(w)}$. Let $J\subseteq S\setminus D(w)$ be any subset of $\abs{J}=m-r$ elements of $S\setminus D(w)$, which must exist since $\rank (R_w)\geq m-r$. As the cosets of $\standard{J}$ in $\standard{S\setminus D(w)}$, denoted by $\standard{S\setminus D(w)}/\standard{J}$, form a partition of $\standard{S\setminus D(w)}$, we have that
    $$
        R_w = \sum_{R\in \standard{S\setminus D(w)}/\standard{J}} R,
    $$
    where each $R$ has rank $(m-r)$  by construction.

    ($\subseteq$) Let $R$ be a standard coset with rank $(m-r)$. Consider an arbitrary standard coset $R'$ with rank at least $r+1$. Since $\rank(R)+\rank (R')>m$, by \cref{lem: even overlap} we have that $R\cdot R'=0$. In particular, this applies to all $R'\in\mcB_{r+1}$, so we necessarily have that $R\in (\Span \mcB_{r+1})^\perp$ which equals $\Span\mcB_{m-r}$ by \cref{lem: dual spans}.
\end{proof}
\begin{proof}[Proofs of \cref{thm: nested,thm: dimension of order r,thm: Coxeter duality}]
    \cref{thm: nested} is a trivial consequence of \cref{lem: extensions form a basis}. \cref{thm: Coxeter duality} holds by \cref{lem: extensions form a basis,lem: dual spans}. \cref{thm: dimension of order r} holds by \cref{lem: extensions form a basis} and the definition of $W$-Eulerian numbers. See also the proof of \cref{lem: dual spans}.
\end{proof}

\begin{proposition}
    The results in this section hold if $\mcB$ is replaced with $\overline{\mcB}$. In particular, $\Coxeter{r} = \Span \overline\mcB_{\geq m-r}$.
\end{proposition}

% \begin{lemma}\label{lem: reverse extensions form a basis}
%     For $r\in\br{-1,\dots,m}$, $\overline\mcB_{\geq m-r}$ is a basis for the order-$r$ Coxeter code of type $(W,S)$:
%     \begin{equation}
        
%     \end{equation}
% \end{lemma}


\vspace{0.5em}
\noindent {\bfit 3.2. Rate of the codes.}
The rate of the Reed-Muller code $RM(r,m)$ equals $2^{-m}\sum_{k=0}^r\binom mi$. By the standard asymptotic arguments,
it changes from near zero to near one when $r$ crosses $m/2$, and is about $1/2$ if $r=\lfloor m/2\rfloor$, with more
precise information derived from the standard Gaussian distribution. Here we argue that largely the same behavior extends
to many Coxeter codes. We address the three infinite series of groups in the Coxeter-Dynkin classification, namely
$A_m$ (the symmetric group on $m+1$ elements), $B_m$ (the hyperoctahedral group of order $2^mm!$), and $D_m$ (the generalized dihedral group of order $2^{m-1}m!$). 
The dimension of the code $\C_{W}(r)$ is given in \eqref{eq:dimension}, from which the rate is found to be
     $$
  R(\C_W(r))=\frac 1{|W|}\sum_{i=0}^r \euler W i.
   $$
There are no closed-form expressions for any of the three cases (for that matter, there is no such expression even for RM codes),
but asymptotic analysis of Eulerian numbers of types $A,B,D$ has been addressed in many places in the literature, with \cite{HCD19}
being the most comprehensive source. We combine several results from \cite{HCD19} into the following theorem:
\begin{theorem}\label{thm: Gaussian} Suppose that $(W,S)$ is one of the irreducible Coxeter families $A_m,B_m$, or $D_m$. Then the code rate 
$R(\Coxeter{r})$ is asymptotically normal, namely, 
$$\frac{R(\Coxeter{\lfloor x\rfloor})-m/2}{m/12}\longrightarrow \frac 1{\sqrt{2\pi}}\int_{-\infty}^x {e^{-t^2/2}}dt$$
as $m\to\infty$.
\end{theorem}
    

This implies that for $R(\C_{W}(r))$ not to tend to 0 or 1 as $m\to\infty$, the quantity
$r/m$ should be separated from 0 and 1. Moreover, the variance $\text{Var}(X_r)=\frac m{12}$ implies that concentration around the mean is sharper for these Coxeter codes than for RM codes where it is controlled by the binomial distribution with variance $\frac m4$. Lastly, we note that the product structure of the $W$-polynomials of Coxeter groups implies that the rate of any infinite family
of Coxeter codes, including the ones constructed from reducible systems (\cref{sec: computing Eulerian numbers}), exhibits a behavior similar to \cref{thm: Gaussian}.
\begin{table}[t!]
    \centering
    \begin{tabular}{|l|c|c|c|c|c|}
    \hline
 %       Group\textbackslash $r$   & 1 & 2  \\ \hline\hline
 \diagbox[width=\dimexpr .6\textwidth/8+2\tabcolsep\relax, height=.55cm]{ $W$ }{$r$}  & 1 & 2  \\ \hline\hline
        $A_3$  & $[24, 12, 4]$ & $[24, 23, 2]$  \\ \hline
        $A_4$  & $[120, 27, 12]$ & $[120, 93, 4]$   \\ \hline
        % $I_2(n)$     & $[2n , 2n-1, 2]$   & $[2n , 2n, 1]$        \\ \hline
        % $I_2(n)\times I_2(n)$ & $[4n^2 , 1, 4n^2 ]$  & $[4n^2  , 4n-3, 4n]$   & $[4n^2  , 4n^2-4n+3, 4]$   & $[4n^2 , 4n^2-1, 2]$  & $[4n^2 , 4n^2, 1]$  \\ \hline
        $I_2(3)^2$   & $[36  , 9, 12]$   & $[36  , 27, 4]$       \\ \hline
        $I_2(4)^2$   & $[64  , 13, 16]$   & $[64  , 51, 4]$       \\ \hline
        $B_3$  & $[48, 24, 4]$ & $[48, 47, 2]$   \\ \hline
        $A_3\times A_1$  & $[48, 13, 8]$ & $[48, 35, 4]$  \\ \hline
        $B_3\times A_1$  & $[96, 25, 8]$ & $[96, 71, 4]$  \\ \hline
    \end{tabular}
    \vspace{0.5em}
    \caption{Parameters for various Coxeter codes.}
    \label{tab: parameters}
\end{table}

\vspace{.03in}\noindent {\bfit 3.3. Distance of the codes.} Given that $\Coxeter{r}$ is generated by standard cosets of rank $m-r$, there is a trivial upper bound on the code distance given by the \emph{smallest} such coset. We conjecture that this bound is, in fact, tight:
\begin{conjecture}\label{conj: distance}
    The distance of $\Coxeter{r}$ is given by 
    \begin{equation}
        \mathrm{dist}(\Coxeter{r})=\min_{J\subseteq S, \abs{J}=m-r} \abs{\standard{J}}.
    \end{equation}
\end{conjecture}
This conjecture is known to be true for RM codes and the family of Coxeter codes given by the dihedral groups, $I_2(n)$, for all $n\geq 2$. We have further verified it by computer for all nontrivial Coxeter codes of length at most $120$ (some of them are listed in \cref{tab: parameters}).


\section{Quantum codes from Coxeter groups}\label{sec:quantum}
We denote by $[[n,k]]$ the parameters of a qubit stabilizer code that encodes $k$ logical qubits into $n$ physical qubits. Given binary $[n,k_i]$ codes $C_i$, $i\in\br{1,2}$, such that $C_1^\perp\subseteq C_2$ there is an $[[n,k_1+k_2-n]]$ stabilizer code, known as the CSS code associated to $C_1$, $C_2$, denoted by $\CSS(C_1,C_2)$. The codes $C_1^\perp$ and $C_2^\perp$ represent the $X$ and $Z$ stabilizers of $\CSS(C_1,C_2)$, respectively. That is, denoting $X^x\coloneqq \bigotimes_{i\in[n]}X^{x_i}$ and $Z^z\coloneqq \bigotimes_{i\in[n]}Z^{z_i}$ where $X$ and $Z$ are the Pauli matrices, the operators
  \begin{equation}\label{eq: XZ}
    \br{X^x, Z^z\Bigmid x\in C_1^\perp, z\in C_2^\perp},
 \end{equation}
commute and have a joint $+1$ eigenspace in $\CC^{2^n}$ of dimension $2^{k_1+k_2-n}$. The codes $C_1$ and $C_2$ likewise represent the space of logical $Z$ and $X$ Pauli operators, respectively.

Let $(W,S)$ be a finite Coxeter system of rank $m\geq 1$.
For $-1\leq q\leq r\leq m$, \cref{thm: nested} implies that $\Coxeter{q}\subseteq\Coxeter{r}$, and so we immediately construct a quantum code using Coxeter codes:
\begin{definition}[Quantum Coxeter code]\label{def: Quantum Coxeter}
     The \emph{order-$(q,r)$ quantum Coxeter code of type $(W,S)$}, $\QCoxeter{q,r}$, is defined to be the CSS code $$\QCoxeter{q,r}\coloneqq\CSS(\Coxeter{m-q-1},\Coxeter{r})$$ with parameters $[[n=\abs{W}, \kappa=\sum_{i=q+1}^r\euler{W}{i}]]$.
\end{definition}
Consider $n=\abs{W}$ physical qubits indexed by the elements of $W$. For a subset $A\subseteq W$ let $X_{A}$ denote the $n$-qubit Pauli operator acting as $X$ on the qubits in $A$ and $\eye$ elsewhere, and analogously for $Z_{w\standard{J}}$.
\begin{lemma}\label{lem: Quantum Coxeter}
    Given $q,r\in\br{-1,\dots,m}$, $q\leq r$, consider the collections of rank-$(m-q)$ and rank-$(r+1)$ standard cosets in $(W,S)$: 
    \begin{align*}
        \Sigma_{m-q}&\coloneqq\br{{w\standard{J}}\bigmid w\in W,J\subseteq S, \abs{J} = m-q},\\
        \Sigma_{r+1}&\coloneqq\br{{w\standard{J}}\bigmid \sigma\in W,J\subseteq S, \abs{J} = r+1}.
    \end{align*}
    The operators $\br{X_{R_1},Z_{R_2}\mid R_1\in\Sigma_{m-q}, R_2\in\Sigma_{r+1}}$ form a (redundant) generating set for the stabilizers of $\QCoxeter{q,r}$.\footnote{As a simple example, consider the dihedral group $I_2(n)$ whose Cayley graph is a $2n$-cycle. Then $\QCoxeter{0,1}$ is the Iceberg code generated by global $X^{\otimes 2n}$ and $Z^{\otimes 2n}$ stabilizers.}
\end{lemma}
\cref{lem: Quantum Coxeter} is a simple consequence of the definition of classical Coxeter codes and their duality structure given in \cref{thm: Coxeter duality}. 


In prior work \cite{barg2024geometric}, we utilized the geometric and combinatorial structure of the group $\ZZ_2^m$ with its standard generating set to study transversal logical operators in higher levels of the Clifford hierarchy of the quantum RM family, $QRM_m(q,r) = \Qcoxeter{\ZZ_2^m}{}{q,r}$. For instance, the exact nature of the logic implemented by certain transversal operators acting on a standard coset depends only on the rank of the coset. This result holds in the case of arbitrary quantum Coxeter codes.
\begin{claim}\label{lem: validity} Let $\QCoxeter{q,r}$ be the quantum Coxeter code and let $R$ be a standard coset.
    For the single-qubit operator 
    $$Z(k)\coloneqq\ketbra{0}+e^{i\frac{\pi}{2^k}}\ketbra{1},$$
    \begin{enumerate}[leftmargin=*]
        \item If $\rank(R) \leq q+kr$ then applying $Z(k)$ to the qubits in $R$ does not preserve the code space.
        \item If $q+kr+1\leq \rank(R)\leq (k+1)r$ then applying $Z(k)$ to the qubits in $R$ implements a non-trivial logical operation the code space.
        \item If $\rank(R)\geq (k+1)r+1$ then applying $Z(k)$ to the qubits in $R$  implements a logical identity on the code space.
    \end{enumerate}
\end{claim}
The proof of \cref{lem: validity} is identical to the proof of Theorem 6.2 in \cite{barg2024geometric}, which relied only on the Coxeter group structure of $\ZZ_2^m$. A natural future direction, following the main results of \cite{barg2024geometric}, is to give a combinatorial description of the logical circuit implemented by a $Z(k)_R$ operator when $q+kr+1\leq \rank(R)\leq (k+1)r$. A necessary first step would be to construct a so-called ``symplectic basis'' for $\QCoxeter{q,r}$. In a few cases--- including the QRM family--- the collections of forward and reverse extensions satisfy the symplectic condition. At the same time, in many cases this fails to be true, including some small quantum Coxeter codes. Examples of groups for which the symplectic condition fails, include $A_3$, the symmetric group on 4 letters generated by pairwise swaps, and $I_2(4)$, the dihedral group of order 8 generated by two reflections meeting at a 45{\textdegree}  angle.

The codes $\QCoxeter{0,1}$ for the Coxeter systems $A_3$, $B_3$, and $H_3$ appear in \cite{Vasmer2022} as examples of 3D ball codes. The authors of \cite{Vasmer2022} note that a global transversal $T$ operator is a non-trivial logical operator for these codes; this is also a consequence of our \cref{lem: validity}.\footnote{\cite{Vasmer2022} technically considers a \emph{signed} version of transversal $T$ which acts as $T$ on half of the qubits and $T^\dagger$ on the remaining qubits.}

\section{Conclusion and outlook}
We have introduced a broad family of binary codes which generalizes and shares several features with the classic Reed--Muller family. It is natural to wonder what other properties of RM codes are shared with the Coxeter code family beyond our conjectured value of the distance. For instance, the codewords of minimum weight in RM codes are given by flats in the affine geometry; is there a characterization of the minimum weight codewords for arbitrary Coxeter codes, as well?
One can also ask what the equivalent notion of a \emph{projective} RM code is in the case of Coxeter codes. Switching to a probabilistic view of the Coxeter codes, one could also study their capacity-achieving \cite{Kudekar2015ReedMullerCA,AbbeSandon2023} and local-testability properties \cite{blum1990self,alon2005testing}.

Instead of considering the Cayley graphs of Coxeter groups we could alternatively consider their \emph{dual} polytopes, which are guaranteed to be simplicial, i.e., their facets are $(m-1)$-simplices. In this view, the order-$r$ Coxeter code would have bits indexed by the facets and parity checks given by the incidence vectors of $(m-r-2)$-simplices in the polytope. In a certain sense, Coxeter codes are related to codes on simplicial complexes where local codes are placed on simplices of a given dimension, e.g., some locally-testable codes based on high-dimensional expanders \cite{dinur2023new}.

In this simplicial view, finite Coxeter systems form a subclass of objects known as {\em spherical buildings} \cite{AB08}. Every
such object is assembled of multiple pieces, called {\em apartments}, each of which is isomorphic to a fixed
Coxeter system. There are two suitable generalizations of Coxeter codes to buildings: one where the code
is generated by subbuildings within the apartments, and another where the code is generated by simplices of a given dimension. In the example of the building associated to the Fano plane, these families happen to be dual to each other. Perhaps this is a general feature of codes on arbitrary buildings; we leave this direction to future work.





\section{Computing \texorpdfstring{$W$}{W}-Eulerian numbers}\label{sec: computing Eulerian numbers}
To find the code dimension \eqref{eq:dimension}, it is useful to have explict expressions for the $W$-Eulerian numbers. For the irreducible families of Coxeter groups, they appear in many references, e.g., \cite{petersen2015eulerian,hyatt2016recurrences,BRENTI1994417}. We give these expressions in our notation, along with an expression to compute the $W$-Eulerian numbers for direct products of Coxeter groups.

For every finite Coxeter system $(W,S)$ of rank $m$, the 0-th and $m$-th $W$-Eulerian numbers equal 1, $\euler{W}{0}=\euler{W}{m}=1$.

\vspace{0.5em}
\noindent {\bfit Type A.} \cite[A008292]{oeis} The $A_n$-\emph{Eulerian numbers} can be computed by the recurrence relation
\begin{equation*}
    \euler{A_n}{k} = (n-k+1)\euler{A_{n-1}}{k-1} + (k+1)\euler{A_{n-1}}{k}.
\end{equation*}


\vspace{0.5em}
\noindent {\bfit Type B.} \cite[A060187]{oeis} The $B_n$-Eulerian numbers can be computed by the recurrence relation
\begin{equation*}
    \euler{B_n}{k} = (2n-2k+1)\euler{B_{n-1}}{k-1} + (2k+1)\euler{B_{n-1}}{k}.
\end{equation*}


\vspace{0.5em}
\noindent {\bfit Type D.} \cite[A066094]{oeis} The $D_n$-Eulerian numbers can be computed from the $A_n$- and $B_n$-Eulerian numbers via
\begin{equation*}
    \euler{D_n}{k} = \euler{B_n}{k}-n 2^{n-1}\euler{A_{n-2}}{k-1}.
\end{equation*}


\vspace{0.5em}
\noindent {\bfit Dihedral group.}  The non-trivial $I_2(n)$-Eulerian number is given by $$\euler{I_2(n)}{1} = 2n-2.$$

\vspace{0.5em}
\noindent {\bfit Exceptional types.} See \cref{tab: exceptional}.

\begin{table*}[t!]
    \centering
    \begin{tabular}{|l|c|c|c|c|c|c|c|c|c|}
    \hline
 %       Group\textbackslash $r$   & 1 & 2  \\ \hline\hline
 \diagbox[width=\dimexpr .6\textwidth/8+2\tabcolsep\relax, height=.55cm]{ $W$ }{$k$} &0 & 1& 2& 3& 4& 5& 6& 7& 8  \\ \hline\hline
        $E_6$  &1& 1272 &12183 & 24928&12183 &1272 &1 & & \\ \hline
        $E_7$  & 1& 17635 & 309969 &1123915  & 1123915 & 309969 & 17635&1  &   \\ \hline
        $E_8$  & 1& 881752& 28336348& 169022824&300247750  & 169022824& 28336348& 881752&    1  \\ \hline
        $F_4$  &1& 236& 678 & 236&1 & & & &       \\ \hline
        $H_3$  &1& 59& 59& 1& & & & &    \\ \hline
        $H_4$  &1& 2636& 9126& 2636& 1& & & &   \\ \hline
    \end{tabular}
    \vspace{0.5em}
    \caption{$W$-Eulerian numbers for the groups of exceptional type \cite{petersen2015eulerian}.}
    \label{tab: exceptional}
\end{table*}

\vspace{0.5em}
\noindent {\bfit Reducible systems.}
Let $(W_1,S_1)$ and $(W_2,S_2)$ be finite Coxeter systems of ranks $m_1$ and $m_2$, respectively. Their direct product $(W,S)\coloneqq (W_1,S_1)\times (W_2,S_2)$ is a finite Coxeter system of rank $m_1+m_2$ where $$S\coloneqq \br{ (s_1,e_{W_2})\mid s_1\in S_1}\cup \br{ (e_{W_1},s_2)\mid s_2\in S_2}.$$

The $W$-Eulerian numbers can be computed from the component $W_1$- and $W_2$-Eulerian numbers by the equation
\begin{equation*}
    \euler{W}{k} = \sum_{i+j=k}\euler{W_1}{i}\euler{W_2}{j},
\end{equation*}
which can be proven by noting that the $W$-polynomial for a direct product of Coxeter groups is the product of the component $W$-polynomials \cite[p.202]{BB05}.
% For a more general direct product, $(W,S)\coloneqq\prod_{\ell=1}^n(W_\ell,S_\ell)$, this yields
% \begin{equation*}
%     \euler{W}{k} = \sum_{\sum_{p=1}^n i_p =k} \left(\prod_{\ell=1}^n \euler{W_\ell}{i_p}\right).
% \end{equation*}

% \newpage
\vspace{0.5em}
{
\balance
\bibliographystyle{IEEEtranS}
\bibliography{ref}
}




\end{document}

\label{sec:refs}

\end{document}