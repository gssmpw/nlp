% This must be in the first 5 lines to tell arXiv to use pdfLaTeX, which is strongly recommended.
\pdfoutput=1
% In particular, the hyperref package requires pdfLaTeX in order to break URLs across lines.

\documentclass[11pt]{article}

% Remove the "review" option to generate the final version.
% \usepackage[review]{ACL2023}
\usepackage[]{ACL2023}

% Standard package includes
\usepackage{times}
\usepackage{latexsym}

% For proper rendering and hyphenation of words containing Latin characters (including in bib files)
\usepackage[T1]{fontenc}
% For Vietnamese characters
% \usepackage[T5]{fontenc}
% See https://www.latex-project.org/help/documentation/encguide.pdf for other character sets

% This assumes your files are encoded as UTF8
\usepackage[utf8]{inputenc}

% This is not strictly necessary, and may be commented out.
% However, it will improve the layout of the manuscript,
% and will typically save some space.
\usepackage{microtype}

% This is also not strictly necessary, and may be commented out.
% However, it will improve the aesthetics of text in
% the typewriter font.
\usepackage{inconsolata}
% \usepackage[ruled,linesnumbered]{algorithm2e}
\usepackage{multicol}
\usepackage{booktabs}
\usepackage{colortbl}
\usepackage[fleqn]{amsmath}
\usepackage{multirow}
\usepackage[normalem]{ulem}
\useunder{\uline}{\ul}{}
\usepackage{hyperref}
\usepackage{graphicx}
\usepackage{subfigure}
\usepackage{fixltx2e}
\usepackage{xcolor}
\usepackage{algorithm}
\usepackage{algorithmic}
\usepackage{setspace}
\usepackage{amsfonts}
\usepackage[most]{tcolorbox}
\usepackage{afterpage}
\newtheorem{definition}{Definition}

\newcommand{\ie}{\textit{i.e.}}
\newcommand{\eg}{\textit{e.g.}}
\newcommand{\ours}{\texttt{mmE5}}
\newcommand{\todo}[1]{\{\textcolor{blue}{\textbf{TODO}}\}}


\title{\ours{}: Improving Multimodal Multilingual Embeddings via \\ High-quality Synthetic Data}

\author{Haonan Chen$^{1}$\thanks{$^{*}$Work done during Haonan’s internship at MSR Asia. Prof. Zhicheng Dou is the corresponding author.}, Liang Wang$^2$, Nan Yang$^2$, Yutao Zhu$^1$ \\ \textbf{Ziliang Zhao$^1$, Furu Wei$^2$, Zhicheng Dou$^{1}$} \\
        $^1$Gaoling School of Artificial Intelligence, Renmin University of China \\ 
        $^2$Microsoft Corporation \\ 
        \texttt{\{hnchen,dou\}@ruc.edu.cn} \\
        \texttt{\{wangliang,nanya,fuwei\}@microsoft.com} \\
        \url{https://github.com/haon-chen/mmE5} \\
}

\newtcolorbox[list inside=prompt]{prompt}[1][]{
    colbacktitle=black!60,
    coltitle=white,
    fontupper=\footnotesize,
    boxsep=5pt,
    left=0pt,
    right=-1pt,
    top=0pt,
    bottom=0pt,
    boxrule=1pt,
    width=\textwidth,
    #1,
}
\begin{document}
\maketitle

\begin{abstract}

Multimodal embedding models have gained significant attention for their ability to map data from different modalities, such as text and images, into a unified representation space. 
However, the limited labeled multimodal data often hinders embedding performance. 
Recent approaches have leveraged data synthesis to address this problem, yet the quality of synthetic data remains a critical bottleneck. 
In this work, we identify three criteria for high-quality synthetic multimodal data. First, \textbf{broad scope} ensures that the generated data covers diverse tasks and modalities, making it applicable to various downstream scenarios. 
Second, \textbf{robust cross-modal alignment} makes different modalities semantically consistent. 
Third, \textbf{high fidelity} ensures that the synthetic data maintains realistic details to enhance its reliability.
Guided by these principles, we synthesize datasets that: (1) cover a wide range of tasks, modality combinations, and languages, (2) are generated via a deep thinking process within a single pass of a multimodal large language model, and (3) incorporate real-world images with accurate and relevant texts, ensuring fidelity through self-evaluation and refinement.
Leveraging these high-quality synthetic and labeled datasets, we train a \textbf{m}ultimodal \textbf{m}ultilingual \textbf{E5} model \ours{}. 
Extensive experiments demonstrate that \ours{} achieves state-of-the-art performance on the MMEB Benchmark and superior multilingual performance on the XTD benchmark.
Our codes, datasets and models are released in \url{https://github.com/haon-chen/mmE5}.


\end{abstract}
\section{Introduction}

Video generation has garnered significant attention owing to its transformative potential across a wide range of applications, such media content creation~\citep{polyak2024movie}, advertising~\citep{zhang2024virbo,bacher2021advert}, video games~\citep{yang2024playable,valevski2024diffusion, oasis2024}, and world model simulators~\citep{ha2018world, videoworldsimulators2024, agarwal2025cosmos}. Benefiting from advanced generative algorithms~\citep{goodfellow2014generative, ho2020denoising, liu2023flow, lipman2023flow}, scalable model architectures~\citep{vaswani2017attention, peebles2023scalable}, vast amounts of internet-sourced data~\citep{chen2024panda, nan2024openvid, ju2024miradata}, and ongoing expansion of computing capabilities~\citep{nvidia2022h100, nvidia2023dgxgh200, nvidia2024h200nvl}, remarkable advancements have been achieved in the field of video generation~\citep{ho2022video, ho2022imagen, singer2023makeavideo, blattmann2023align, videoworldsimulators2024, kuaishou2024klingai, yang2024cogvideox, jin2024pyramidal, polyak2024movie, kong2024hunyuanvideo, ji2024prompt}.


In this work, we present \textbf{\ours}, a family of rectified flow~\citep{lipman2023flow, liu2023flow} transformer models designed for joint image and video generation, establishing a pathway toward industry-grade performance. This report centers on four key components: data curation, model architecture design, flow formulation, and training infrastructure optimization—each rigorously refined to meet the demands of high-quality, large-scale video generation.


\begin{figure}[ht]
    \centering
    \begin{subfigure}[b]{0.82\linewidth}
        \centering
        \includegraphics[width=\linewidth]{figures/t2i_1024.pdf}
        \caption{Text-to-Image Samples}\label{fig:main-demo-t2i}
    \end{subfigure}
    \vfill
    \begin{subfigure}[b]{0.82\linewidth}
        \centering
        \includegraphics[width=\linewidth]{figures/t2v_samples.pdf}
        \caption{Text-to-Video Samples}\label{fig:main-demo-t2v}
    \end{subfigure}
\caption{\textbf{Generated samples from \ours.} Key components are highlighted in \textcolor{red}{\textbf{RED}}.}\label{fig:main-demo}
\end{figure}


First, we present a comprehensive data processing pipeline designed to construct large-scale, high-quality image and video-text datasets. The pipeline integrates multiple advanced techniques, including video and image filtering based on aesthetic scores, OCR-driven content analysis, and subjective evaluations, to ensure exceptional visual and contextual quality. Furthermore, we employ multimodal large language models~(MLLMs)~\citep{yuan2025tarsier2} to generate dense and contextually aligned captions, which are subsequently refined using an additional large language model~(LLM)~\citep{yang2024qwen2} to enhance their accuracy, fluency, and descriptive richness. As a result, we have curated a robust training dataset comprising approximately 36M video-text pairs and 160M image-text pairs, which are proven sufficient for training industry-level generative models.

Secondly, we take a pioneering step by applying rectified flow formulation~\citep{lipman2023flow} for joint image and video generation, implemented through the \ours model family, which comprises Transformer architectures with 2B and 8B parameters. At its core, the \ours framework employs a 3D joint image-video variational autoencoder (VAE) to compress image and video inputs into a shared latent space, facilitating unified representation. This shared latent space is coupled with a full-attention~\citep{vaswani2017attention} mechanism, enabling seamless joint training of image and video. This architecture delivers high-quality, coherent outputs across both images and videos, establishing a unified framework for visual generation tasks.


Furthermore, to support the training of \ours at scale, we have developed a robust infrastructure tailored for large-scale model training. Our approach incorporates advanced parallelism strategies~\citep{jacobs2023deepspeed, pytorch_fsdp} to manage memory efficiently during long-context training. Additionally, we employ ByteCheckpoint~\citep{wan2024bytecheckpoint} for high-performance checkpointing and integrate fault-tolerant mechanisms from MegaScale~\citep{jiang2024megascale} to ensure stability and scalability across large GPU clusters. These optimizations enable \ours to handle the computational and data challenges of generative modeling with exceptional efficiency and reliability.


We evaluate \ours on both text-to-image and text-to-video benchmarks to highlight its competitive advantages. For text-to-image generation, \ours-T2I demonstrates strong performance across multiple benchmarks, including T2I-CompBench~\citep{huang2023t2i-compbench}, GenEval~\citep{ghosh2024geneval}, and DPG-Bench~\citep{hu2024ella_dbgbench}, excelling in both visual quality and text-image alignment. In text-to-video benchmarks, \ours-T2V achieves state-of-the-art performance on the UCF-101~\citep{ucf101} zero-shot generation task. Additionally, \ours-T2V attains an impressive score of \textbf{84.85} on VBench~\citep{huang2024vbench}, securing the top position on the leaderboard (as of 2025-01-25) and surpassing several leading commercial text-to-video models. Qualitative results, illustrated in \Cref{fig:main-demo}, further demonstrate the superior quality of the generated media samples. These findings underscore \ours's effectiveness in multi-modal generation and its potential as a high-performing solution for both research and commercial applications.
\section{Related Work}

\subsection{Large 3D Reconstruction Models}
Recently, generalized feed-forward models for 3D reconstruction from sparse input views have garnered considerable attention due to their applicability in heavily under-constrained scenarios. The Large Reconstruction Model (LRM)~\cite{hong2023lrm} uses a transformer-based encoder-decoder pipeline to infer a NeRF reconstruction from just a single image. Newer iterations have shifted the focus towards generating 3D Gaussian representations from four input images~\cite{tang2025lgm, xu2024grm, zhang2025gslrm, charatan2024pixelsplat, chen2025mvsplat, liu2025mvsgaussian}, showing remarkable novel view synthesis results. The paradigm of transformer-based sparse 3D reconstruction has also successfully been applied to lifting monocular videos to 4D~\cite{ren2024l4gm}. \\
Yet, none of the existing works in the domain have studied the use-case of inferring \textit{animatable} 3D representations from sparse input images, which is the focus of our work. To this end, we build on top of the Large Gaussian Reconstruction Model (GRM)~\cite{xu2024grm}.

\subsection{3D-aware Portrait Animation}
A different line of work focuses on animating portraits in a 3D-aware manner.
MegaPortraits~\cite{drobyshev2022megaportraits} builds a 3D Volume given a source and driving image, and renders the animated source actor via orthographic projection with subsequent 2D neural rendering.
3D morphable models (3DMMs)~\cite{blanz19993dmm} are extensively used to obtain more interpretable control over the portrait animation. For example, StyleRig~\cite{tewari2020stylerig} demonstrates how a 3DMM can be used to control the data generated from a pre-trained StyleGAN~\cite{karras2019stylegan} network. ROME~\cite{khakhulin2022rome} predicts vertex offsets and texture of a FLAME~\cite{li2017flame} mesh from the input image.
A TriPlane representation is inferred and animated via FLAME~\cite{li2017flame} in multiple methods like Portrait4D~\cite{deng2024portrait4d}, Portrait4D-v2~\cite{deng2024portrait4dv2}, and GPAvatar~\cite{chu2024gpavatar}.
Others, such as VOODOO 3D~\cite{tran2024voodoo3d} and VOODOO XP~\cite{tran2024voodooxp}, learn their own expression encoder to drive the source person in a more detailed manner. \\
All of the aforementioned methods require nothing more than a single image of a person to animate it. This allows them to train on large monocular video datasets to infer a very generic motion prior that even translates to paintings or cartoon characters. However, due to their task formulation, these methods mostly focus on image synthesis from a frontal camera, often trading 3D consistency for better image quality by using 2D screen-space neural renderers. In contrast, our work aims to produce a truthful and complete 3D avatar representation from the input images that can be viewed from any angle.  

\subsection{Photo-realistic 3D Face Models}
The increasing availability of large-scale multi-view face datasets~\cite{kirschstein2023nersemble, ava256, pan2024renderme360, yang2020facescape} has enabled building photo-realistic 3D face models that learn a detailed prior over both geometry and appearance of human faces. HeadNeRF~\cite{hong2022headnerf} conditions a Neural Radiance Field (NeRF)~\cite{mildenhall2021nerf} on identity, expression, albedo, and illumination codes. VRMM~\cite{yang2024vrmm} builds a high-quality and relightable 3D face model using volumetric primitives~\cite{lombardi2021mvp}. One2Avatar~\cite{yu2024one2avatar} extends a 3DMM by anchoring a radiance field to its surface. More recently, GPHM~\cite{xu2025gphm} and HeadGAP~\cite{zheng2024headgap} have adopted 3D Gaussians to build a photo-realistic 3D face model. \\
Photo-realistic 3D face models learn a powerful prior over human facial appearance and geometry, which can be fitted to a single or multiple images of a person, effectively inferring a 3D head avatar. However, the fitting procedure itself is non-trivial and often requires expensive test-time optimization, impeding casual use-cases on consumer-grade devices. While this limitation may be circumvented by learning a generalized encoder that maps images into the 3D face model's latent space, another fundamental limitation remains. Even with more multi-view face datasets being published, the number of available training subjects rarely exceeds the thousands, making it hard to truly learn the full distibution of human facial appearance. Instead, our approach avoids generalizing over the identity axis by conditioning on some images of a person, and only generalizes over the expression axis for which plenty of data is available. 

A similar motivation has inspired recent work on codec avatars where a generalized network infers an animatable 3D representation given a registered mesh of a person~\cite{cao2022authentic, li2024uravatar}.
The resulting avatars exhibit excellent quality at the cost of several minutes of video capture per subject and expensive test-time optimization.
For example, URAvatar~\cite{li2024uravatar} finetunes their network on the given video recording for 3 hours on 8 A100 GPUs, making inference on consumer-grade devices impossible. In contrast, our approach directly regresses the final 3D head avatar from just four input images without the need for expensive test-time fine-tuning.


\section{RoleMRC}
\label{sec:method}

In this section, we build RoleMRC. Figure\,\ref{fig:method} illustrates the overall pipeline of RoleMRC from top to bottom, which is divided into three steps.

\subsection{A Meta-pool of 10k Role Profiles}
\label{sec:meta_pool}
We first collect a meta-pool of 10k role profile using two open-source datasets, with Step 1 and 2.

\paragraph{Step 1: Persona Sampling.} We randomly sample 10.5k one-sentence demographic persona description from PersonaHub\,\cite{ge2024scaling}, such as ``\emph{A local business owner interested in economic trends}'', as shown at the top of Figure\,\ref{fig:method}. 

\paragraph{Step 2: Role Profile Standardization.} Next, we use a well-crafted prompt with gpt-4o\,\cite{gpt4o} to expand each sampled persona into a complete role profile, in reference to the 1-shot standardized example. Illustrated in the middle of Figure\,\ref{fig:method}, we require a standardized role profile consisting of seven components: \emph{Role Name and Brief Description}, \emph{Specific Abilities and Skills}, \emph{Speech Style}, \emph{Personality Characteristics}, \emph{Past Experience and Background}, \emph{Ability and Knowledge Boundaries} and \emph{Speech Examples}. %Setting standard specifications helps convert the generated role profiles into formatted records, which is beneficial for the post quality control. 
Standardizing these profiles ensures structured formatting, simplifying quality control. 
After manual checking and format filtering, we remove 333 invalid responses from gpt-4o, resulting in 10.2k final role profiles. We report complete persona-to-profile standardization prompt and structure tree of final role profiles in Appendix\,\ref{sec:app_prompt_1} and \,\ref{sec:app_tree}, respectively.

Machine Reading Comprehension (MRC) is one of the core tasks for LLMs to interact with human users. Consequently, we choose to synthesize fine-grained role-playing instruction-following data based on MRC. We first generate a retrieval pool containing 808.7k MRC data from the MSMARCO training set\,\cite{bajaj2016ms}. By leveraging SFR-Embedding\,\cite{SFR-embedding-2}, we perform an inner product search to identify the most relevant and least relevant MRC triplets (Passages, Question, Answer) for each role profile. For example, the middle part of Figure\,\ref{fig:method} shows that for the role \emph{Jessica Thompson, a resilient local business owner}, the most relevant question is about \emph{the skill of resiliency}, while the least relevant question is \emph{converting Fahrenheit to Celsius}. After review, we categorise the most relevant MRC triplet as within a role's knowledge boundary, and the least relevant MRC triplet as beyond their expertise.

\begin{figure}[t]
    \centering
    \includegraphics[width=1.0\linewidth]{figures/step3.png}
    \caption{The strategy of gradually synthesizing finer role-playing instructions in step 3 of Figure\,\ref{fig:method}.}
    \vspace{-1.0em}
    \label{fig:step3}
\end{figure}

\subsection{38k Role-playing Instructions}
Based on the role profiles, we then adopt \textbf{Step 3: Multi-stage Dialogue Synthesis} to generate 38k role-playing instructions, progressively increasing granularity across three categories %including three types with gradually finer granularity 
(Figure\,\ref{fig:step3}):
%\begin{itemize}
%[leftmargin=*,noitemsep,topsep=0pt]

\noindent \textbf{\underline{Free Chats.}} The simplest dialogues, free chats, are synthesized at first. Here, we ask gpt-4o to simulate and generate multi-turn open-domain conversations between the role and an imagined user based on the standardized role profile. When synthesizing the conversation, we additionally consider two factors: the \textbf{initial speaker} in the starting round of the conversation, and whether the role's speech has \textbf{a narration wrapped in brackets} at the beginning (e.g., \emph{(Aiden reviews the network logs, his eyes narrowing as he spots unusual activity) I found it!}). The narration refers to a short, vivid description of the role's speaking state from an omniscient perspective, which further strengthens the sense of role's depth and has been adopted in some role-playing datasets\,\cite{tu2024charactereval}. 

As shown on the left side of Figure\,\ref{fig:step3}, based on the aforementioned two factors, we synthesize four variations of Free Chats. In particular, when  narration is omitted, we deleted all the 
narration content in the speech examples from the role profile; %and for the case that 
when narration is allowed, we retain the narration content, and also add instructions to allow appropriate insertion of narration in the task prompt of gpt-4o. It worth to note that, in narration-allowed dialogues, not every response of the role has narration inserted to prevent overfitting. All categories of data in RoleMRC incorporate narration insertion and follow similar control mechanisms. The following sections will omit further details on narration.

\noindent \textbf{\underline{On-scene MRC Dialogues.}} The synthesis of on-scene MRC dialogues can be divided into two parts. The first part is similar to the free chats. As shown by the {\color{lightgreen}{green round rectangle}} in the upper part of Figure\,\ref{fig:step3}, we ask gpt-4o to synthesize a conversation (lower left corner of Figure\,\ref{fig:step3}) between the role and the user focusing on relevant passages. This part of the synthesis and the Free Chats share the entire meta-pool, so each consisting of 5k dialogues.

The remaining part forms eight types of single-turn role-playing Question Answering (QA). In the middle of Figure\,\ref{fig:step3}, we randomly select a group of roles and examined the most relevant MRCs they matched: if the question in the MRC is answerable, then the ground truth answer is stylized to match the role profile; otherwise, a seed script of ``unanswerable'' is randomly selected then stylized. The above process generates four groups of 1k data from type ``[1]'' to type``[4]''. According to the middle right side of Figure\,\ref{fig:step3}, we also select a group of roles and ensure that the least relevant MRCs they matched contain answerable QA pairs. Since the most irrelevant MRCs are outside the knowledge boundary of the roles, the role-playing responses to these questions are ``out-of-mind'' refusal or ``let-me-try'' attempt, thus synthesizing four groups of 1k data, from type ``[5]'' to type ``[8]''.

\noindent \textbf{\underline{Ruled Chats.}} We construct Ruled Chats by extending On-scene MRC Dialogues in categories ``[1]'' to ``[8]'' with incorporated three additional rules, as shown in the right bottom corner of Figure\,\ref{fig:step3}. For the \textbf{multi-turn rules}, we apply them to the four unanswerable scenarios ``[3]'', ``[4]'', ``[5]'', and ``[6]'', adding a user prompt that  forces the role to answer. Among them, data ``[3]'' and ``[4]'' maintain refusal since the questions in MRC are unanswerable; while ``[5]'' and ``[6]'' are transformed into attempts to answer despite knowledge limitations. For the \textbf{nested formatting rules}, we add new formatting instructions to the four categories of data ``[1]'', ``[2]'', ``[3]'', and ``[4]'', such as requiring emojis,  capitalization, specific punctuation marks, and controlling the total number of words, then modify the previous replies accordingly. For the last \textbf{prioritized rules}, we apply them to subsets ``[1]'' and ``[2]'' that contain normal stylized answers, inserting a  global refusal directive from the system, and thus creating a conflict between system instructions and the role's ability boundary.
%\end{itemize}

\begin{table}[t]
\resizebox{\columnwidth}{!}{%
  \begin{tabular}{c|c|c|c|c|c}
    \toprule
    & & \textbf{S*} & \textbf{P*} & \textbf{\#Turns} & \textbf{\#Words} \\ 
    \midrule
    \multirow{13.5}{*}{\textbf{RoleMRC}} 
    & \multicolumn{5}{c|}{\textbf{Free Chats}} \\ 
    \cmidrule(lr){2-6}
    & Chats & 5k & / & 9.47 & 38.62 \\ 
    \cmidrule(lr){2-6}
    & \multicolumn{5}{c|}{\textbf{On-scene MRC Dialogues}} \\ 
    \cmidrule(lr){2-6} 
    & On-scene Chats & 5k & / & 9.2 & 43.18 \\
    & Answer & 2k & 2k & 1 & 39.45 \\ 
    & No Answer & 2k & 2k & 1 & 47.09 \\ 
    & Refusal & 2k & 2k & 1 & 48.41 \\ 
    & Attempt & 2k & 2k & 1 & 47.92 \\ 
    \cmidrule(lr){2-6}
    & \multicolumn{5}{c|}{\textbf{Ruled Chats}} \\ 
    \cmidrule(lr){2-6}
    & Multi-turn & 2k & 2k & 2 & 42.47 \\ 
    & Nested & 1.6k & 1.6k & 1 & 46.17 \\ 
    & Prioritized & 2.4k & 2.4k & 1 & 42.65 \\ 
    \midrule
    & \textbf{Total} & 24k & 14k & 3.5 & 40.6 \\ 
    \midrule
    \multirow{3}{*}{\textbf{-mix}} 
    & RoleBench & 16k & / & 1 & 23.95 \\ 
    & RLHFlow & 40k & / & 1.39 & 111.79 \\ 
    & UltraFeedback & / & 14k & 1 & 199.28 \\ 
    \midrule
    & \textbf{Total} & 80k & 28k & 2 & 67.1 \\ 
    \bottomrule
  \end{tabular}}
  \vspace{-2mm}
  \caption{Statistics of RoleMRC. In particular, the column names S*, P*, \#Turns, and \#Words, stands for size of single-label data, size of pair-label data, average turns, and average number of words per reply, respectively. RoleMRC-mix expands RoleMRC by adding existing role-playing data.}
 \vspace{-3mm}
  \label{tab:roleMRC}
\end{table}

\subsection{Integration and Mix-up}
All the seed scripts and prioritized rules used for constructing On-scene Dialogues and Ruled Chats are reported in Appendix\,\ref{sec:app_scripts}. These raw responses are logically valid manual answers that remain unaffected by the roles' speaking styles, making them suitable as negative labels to contrast with the stylized answers. Thanks to these meticulous seed texts, we obtain high-quality synthetic data with stable output from gpt-4o. After integration, as shown in Table\,\ref{tab:roleMRC}, the final RoleMRC contains 24k single-label data for Supervised Fine-Tuning (SFT) and 14k pair-label data for Human Preference Optimization (HPO)\,\cite{ouyang2022training,rafailov2023direct,sampo,hong2024reference}. Considering that fine-tuning LLMs with relatively fixed data formats may lead to catastrophic forgetting\,\cite{kirkpatrick2017overcoming}, we create RoleMRC-mix as a robust version by incorporating external role-playing data (RoleBench\,\cite{wang2023rolellm}) and general instructions (RLHFlow\,\cite{dong2024rlhf}, UltraFeedback\,\cite{cui2023ultrafeedback}).

\section{Experiments}

\subsection{Setups}
\subsubsection{Implementation Details}
We apply our FDS method to two types of 3DGS: 
the original 3DGS, and 2DGS~\citep{huang20242d}. 
%
The number of iterations in our optimization 
process is 35,000.
We follow the default training configuration 
and apply our FDS method after 15,000 iterations,
then we add normal consistency loss for both
3DGS and 2DGS after 25000 iterations.
%
The weight for FDS, $\lambda_{fds}$, is set to 0.015,
the $\sigma$ is set to 23,
and the weight for normal consistency is set to 0.15
for all experiments. 
We removed the depth distortion loss in 2DGS 
because we found that it degrades its results in indoor scenes.
%
The Gaussian point cloud is initialized using Colmap
for all datasets.
%
%
We tested the impact of 
using Sea Raft~\citep{wang2025sea} and 
Raft\citep{teed2020raft} on FDS performance.
%
Due to the blurriness of the ScanNet dataset, 
additional prior constraints are required.
Thus, we incorporate normal prior supervision 
on the rendered normals 
in ScanNet (V2) dataset by default.
The normal prior is predicted by the Stable Normal 
model~\citep{ye2024stablenormal}
across all types of 3DGS.
%
The entire framework is implemented in 
PyTorch~\citep{paszke2019pytorch}, 
and all experiments are conducted on 
a single NVIDIA 4090D GPU.

\begin{figure}[t] \centering
    \makebox[0.16\textwidth]{\scriptsize Input}
    \makebox[0.16\textwidth]{\scriptsize 3DGS}
    \makebox[0.16\textwidth]{\scriptsize 2DGS}
    \makebox[0.16\textwidth]{\scriptsize 3DGS + FDS}
    \makebox[0.16\textwidth]{\scriptsize 2DGS + FDS}
    \makebox[0.16\textwidth]{\scriptsize GT (Depth)}

    \includegraphics[width=0.16\textwidth]{figure/fig3_img/compare3/gt_rgb/frame_00522.jpg}
    \includegraphics[width=0.16\textwidth]{figure/fig3_img/compare3/3DGS/frame_00522.jpg}
    \includegraphics[width=0.16\textwidth]{figure/fig3_img/compare3/2DGS/frame_00522.jpg}
    \includegraphics[width=0.16\textwidth]{figure/fig3_img/compare3/3DGS+FDS/frame_00522.jpg}
    \includegraphics[width=0.16\textwidth]{figure/fig3_img/compare3/2DGS+FDS/frame_00522.jpg}
    \includegraphics[width=0.16\textwidth]{figure/fig3_img/compare3/gt_depth/frame_00522.jpg} \\

    % \includegraphics[width=0.16\textwidth]{figure/fig3_img/compare1/gt_rgb/frame_00137.jpg}
    % \includegraphics[width=0.16\textwidth]{figure/fig3_img/compare1/3DGS/frame_00137.jpg}
    % \includegraphics[width=0.16\textwidth]{figure/fig3_img/compare1/2DGS/frame_00137.jpg}
    % \includegraphics[width=0.16\textwidth]{figure/fig3_img/compare1/3DGS+FDS/frame_00137.jpg}
    % \includegraphics[width=0.16\textwidth]{figure/fig3_img/compare1/2DGS+FDS/frame_00137.jpg}
    % \includegraphics[width=0.16\textwidth]{figure/fig3_img/compare1/gt_depth/frame_00137.jpg} \\

     \includegraphics[width=0.16\textwidth]{figure/fig3_img/compare2/gt_rgb/frame_00262.jpg}
    \includegraphics[width=0.16\textwidth]{figure/fig3_img/compare2/3DGS/frame_00262.jpg}
    \includegraphics[width=0.16\textwidth]{figure/fig3_img/compare2/2DGS/frame_00262.jpg}
    \includegraphics[width=0.16\textwidth]{figure/fig3_img/compare2/3DGS+FDS/frame_00262.jpg}
    \includegraphics[width=0.16\textwidth]{figure/fig3_img/compare2/2DGS+FDS/frame_00262.jpg}
    \includegraphics[width=0.16\textwidth]{figure/fig3_img/compare2/gt_depth/frame_00262.jpg} \\

    \includegraphics[width=0.16\textwidth]{figure/fig3_img/compare4/gt_rgb/frame00000.png}
    \includegraphics[width=0.16\textwidth]{figure/fig3_img/compare4/3DGS/frame00000.png}
    \includegraphics[width=0.16\textwidth]{figure/fig3_img/compare4/2DGS/frame00000.png}
    \includegraphics[width=0.16\textwidth]{figure/fig3_img/compare4/3DGS+FDS/frame00000.png}
    \includegraphics[width=0.16\textwidth]{figure/fig3_img/compare4/2DGS+FDS/frame00000.png}
    \includegraphics[width=0.16\textwidth]{figure/fig3_img/compare4/gt_depth/frame00000.png} \\

    \includegraphics[width=0.16\textwidth]{figure/fig3_img/compare5/gt_rgb/frame00080.png}
    \includegraphics[width=0.16\textwidth]{figure/fig3_img/compare5/3DGS/frame00080.png}
    \includegraphics[width=0.16\textwidth]{figure/fig3_img/compare5/2DGS/frame00080.png}
    \includegraphics[width=0.16\textwidth]{figure/fig3_img/compare5/3DGS+FDS/frame00080.png}
    \includegraphics[width=0.16\textwidth]{figure/fig3_img/compare5/2DGS+FDS/frame00080.png}
    \includegraphics[width=0.16\textwidth]{figure/fig3_img/compare5/gt_depth/frame00080.png} \\



    \caption{\textbf{Comparison of depth reconstruction on Mushroom and ScanNet datasets.} The original
    3DGS or 2DGS model equipped with FDS can remove unwanted floaters and reconstruct
    geometry more preciously.}
    \label{fig:compare}
\end{figure}


\subsubsection{Datasets and Metrics}

We evaluate our method for 3D reconstruction 
and novel view synthesis tasks on
\textbf{Mushroom}~\citep{ren2024mushroom},
\textbf{ScanNet (v2)}~\citep{dai2017scannet}, and 
\textbf{Replica}~\citep{replica19arxiv}
datasets,
which feature challenging indoor scenes with both 
sparse and dense image sampling.
%
The Mushroom dataset is an indoor dataset 
with sparse image sampling and two distinct 
camera trajectories. 
%
We train our model on the training split of 
the long capture sequence and evaluate 
novel view synthesis on the test split 
of the long capture sequences.
%
Five scenes are selected to evaluate our FDS, 
including "coffee room", "honka", "kokko", 
"sauna", and "vr room". 
%
ScanNet(V2)~\citep{dai2017scannet}  consists of 1,613 indoor scenes
with annotated camera poses and depth maps. 
%
We select 5 scenes from the ScanNet (V2) dataset, 
uniformly sampling one-tenth of the views,
following the approach in ~\citep{guo2022manhattan}.
To further improve the geometry rendering quality of 3DGS, 
%
Replica~\citep{replica19arxiv} contains small-scale 
real-world indoor scans. 
We evaluate our FDS on five scenes from 
Replica: office0, office1, office2, office3 and office4,
selecting one-tenth of the views for training.
%
The results for Replica are provided in the 
supplementary materials.
To evaluate the rendering quality and geometry 
of 3DGS, we report PSNR, SSIM, and LPIPS for 
rendering quality, along with Absolute Relative Distance 
(Abs Rel) as a depth quality metrics.
%
Additionally, for mesh evaluation, 
we use metrics including Accuracy, Completion, 
Chamfer-L1 distance, Normal Consistency, 
and F-scores.




\subsection{Results}
\subsubsection{Depth rendering and novel view synthesis}
The comparison results on Mushroom and 
ScanNet are presented in \tabref{tab:mushroom} 
and \tabref{tab:scannet}, respectively. 
%
Due to the sparsity of sampling 
in the Mushroom dataset,
challenges are posed for both GOF~\citep{yu2024gaussian} 
and PGSR~\citep{chen2024pgsr}, 
leading to their relative poor performance 
on the Mushroom dataset.
%
Our approach achieves the best performance 
with the FDS method applied during the training process.
The FDS significantly enhances the 
geometric quality of 3DGS on the Mushroom dataset, 
improving the "abs rel" metric by more than 50\%.
%
We found that Sea Raft~\citep{wang2025sea}
outperforms Raft~\citep{teed2020raft} on FDS, 
indicating that a better optical flow model 
can lead to more significant improvements.
%
Additionally, the render quality of RGB 
images shows a slight improvement, 
by 0.58 in 3DGS and 0.50 in 2DGS, 
benefiting from the incorporation of cross-view consistency in FDS. 
%
On the Mushroom
dataset, adding the FDS loss increases 
the training time by half an hour, which maintains the same
level as baseline.
%
Similarly, our method shows a notable improvement on the ScanNet dataset as well using Sea Raft~\citep{wang2025sea} Model. The "abs rel" metric in 2DGS is improved nearly 50\%. This demonstrates the robustness and effectiveness of the FDS method across different datasets.
%


% \begin{wraptable}{r}{0.6\linewidth} \centering
% \caption{\textbf{Ablation study on geometry priors.}} 
%         \label{tab:analysis_prior}
%         \resizebox{\textwidth}{!}{
\begin{tabular}{c| c c c c c | c c c c}

    \hline
     Method &  Acc$\downarrow$ & Comp $\downarrow$ & C-L1 $\downarrow$ & NC $\uparrow$ & F-Score $\uparrow$ &  Abs Rel $\downarrow$ &  PSNR $\uparrow$  & SSIM  $\uparrow$ & LPIPS $\downarrow$ \\ \hline
    2DGS&   0.1078&  0.0850&  0.0964&  0.7835&  0.5170&  0.1002&  23.56&  0.8166& 0.2730\\
    2DGS+Depth&   0.0862&  0.0702&  0.0782&  0.8153&  0.5965&  0.0672&  23.92&  0.8227& 0.2619 \\
    2DGS+MVDepth&   0.2065&  0.0917&  0.1491&  0.7832&  0.3178&  0.0792&  23.74&  0.8193& 0.2692 \\
    2DGS+Normal&   0.0939&  0.0637&  0.0788&  \textbf{0.8359}&  0.5782&  0.0768&  23.78&  0.8197& 0.2676 \\
    2DGS+FDS &  \textbf{0.0615} & \textbf{ 0.0534}& \textbf{0.0574}& 0.8151& \textbf{0.6974}&  \textbf{0.0561}&  \textbf{24.06}&  \textbf{0.8271}&\textbf{0.2610} \\ \hline
    2DGS+Depth+FDS &  0.0561 &  0.0519& 0.0540& 0.8295& 0.7282&  0.0454&  \textbf{24.22}& \textbf{0.8291}&\textbf{0.2570} \\
    2DGS+Normal+FDS &  \textbf{0.0529} & \textbf{ 0.0450}& \textbf{0.0490}& \textbf{0.8477}& \textbf{0.7430}&  \textbf{0.0443}&  24.10&  0.8283& 0.2590 \\
    2DGS+Depth+Normal &  0.0695 & 0.0513& 0.0604& 0.8540&0.6723&  0.0523&  24.09&  0.8264&0.2575\\ \hline
    2DGS+Depth+Normal+FDS &  \textbf{0.0506} & \textbf{0.0423}& \textbf{0.0464}& \textbf{0.8598}&\textbf{0.7613}&  \textbf{0.0403}&  \textbf{24.22}& 
    \textbf{0.8300}&\textbf{0.0403}\\
    
\bottomrule
\end{tabular}
}
% \end{wraptable}



The qualitative comparisons on the Mushroom and ScanNet dataset 
are illustrated in \figref{fig:compare}. 
%
%
As seen in the first row of \figref{fig:compare}, 
both the original 3DGS and 2DGS suffer from overfitting, 
leading to corrupted geometry generation. 
%
Our FDS effectively mitigates this issue by 
supervising the matching relationship between 
the input and sampled views, 
helping to recover the geometry.
%
FDS also improves the refinement of geometric details, 
as shown in other rows. 
By incorporating the matching prior through FDS, 
the quality of the rendered depth is significantly improved.
%

\begin{table}[t] \centering
\begin{minipage}[t]{0.96\linewidth}
        \captionof{table}{\textbf{3D Reconstruction 
        and novel view synthesis results on Mushroom dataset. * 
        Represents that FDS uses the Raft model.
        }}
        \label{tab:mushroom}
        \resizebox{\textwidth}{!}{
\begin{tabular}{c| c c c c c | c c c c c}
    \hline
     Method &  Acc$\downarrow$ & Comp $\downarrow$ & C-L1 $\downarrow$ & NC $\uparrow$ & F-Score $\uparrow$ &  Abs Rel $\downarrow$ &  PSNR $\uparrow$  & SSIM  $\uparrow$ & LPIPS $\downarrow$ & Time  $\downarrow$ \\ \hline

    % DN-splatter &   &  &  &  &  &  &  &  & \\
    GOF &  0.1812 & 0.1093 & 0.1453 & 0.6292 & 0.3665 & 0.2380  & 21.37  &  0.7762  & 0.3132  & $\approx$1.4h\\ 
    PGSR &  0.0971 & 0.1420 & 0.1196 & 0.7193 & 0.5105 & 0.1723  & 22.13  & 0.7773  & 0.2918  & $\approx$1.2h \\ \hline
    3DGS &   0.1167 &  0.1033&  0.1100&  0.7954&  0.3739&  0.1214&  24.18&  0.8392& 0.2511 &$\approx$0.8h \\
    3DGS + FDS$^*$ & 0.0569  & 0.0676 & 0.0623 & 0.8105 & 0.6573 & 0.0603 & 24.72  & 0.8489 & 0.2379 &$\approx$1.3h \\
    3DGS + FDS & \textbf{0.0527}  & \textbf{0.0565} & \textbf{0.0546} & \textbf{0.8178} & \textbf{0.6958} & \textbf{0.0568} & \textbf{24.76}  & \textbf{0.8486} & \textbf{0.2381} &$\approx$1.3h \\ \hline
    2DGS&   0.1078&  0.0850&  0.0964&  0.7835&  0.5170&  0.1002&  23.56&  0.8166& 0.2730 &$\approx$0.8h\\
    2DGS + FDS$^*$ &  0.0689 &  0.0646& 0.0667& 0.8042& 0.6582& 0.0589& 23.98&  0.8255&0.2621 &$\approx$1.3h\\
    2DGS + FDS &  \textbf{0.0615} & \textbf{ 0.0534}& \textbf{0.0574}& \textbf{0.8151}& \textbf{0.6974}&  \textbf{0.0561}&  \textbf{24.06}&  \textbf{0.8271}&\textbf{0.2610} &$\approx$1.3h \\ \hline
\end{tabular}
}
\end{minipage}\hfill
\end{table}

\begin{table}[t] \centering
\begin{minipage}[t]{0.96\linewidth}
        \captionof{table}{\textbf{3D Reconstruction 
        and novel view synthesis results on ScanNet dataset.}}
        \label{tab:scannet}
        \resizebox{\textwidth}{!}{
\begin{tabular}{c| c c c c c | c c c c }
    \hline
     Method &  Acc $\downarrow$ & Comp $\downarrow$ & C-L1 $\downarrow$ & NC $\uparrow$ & F-Score $\uparrow$ &  Abs Rel $\downarrow$ &  PSNR $\uparrow$  & SSIM  $\uparrow$ & LPIPS $\downarrow$ \\ \hline
    GOF & 1.8671  & 0.0805 & 0.9738 & 0.5622 & 0.2526 & 0.1597  & 21.55  & 0.7575  & 0.3881 \\
    PGSR &  0.2928 & 0.5103 & 0.4015 & 0.5567 & 0.1926 & 0.1661  & 21.71 & 0.7699  & 0.3899 \\ \hline

    3DGS &  0.4867 & 0.1211 & 0.3039 & 0.7342& 0.3059 & 0.1227 & 22.19& 0.7837 & 0.3907\\
    3DGS + FDS &  \textbf{0.2458} & \textbf{0.0787} & \textbf{0.1622} & \textbf{0.7831} & 
    \textbf{0.4482} & \textbf{0.0573} & \textbf{22.83} & \textbf{0.7911} & \textbf{0.3826} \\ \hline
    2DGS &  0.2658 & 0.0845 & 0.1752 & 0.7504& 0.4464 & 0.0831 & 22.59& 0.7881 & 0.3854\\
    2DGS + FDS &  \textbf{0.1457} & \textbf{0.0679} & \textbf{0.1068} & \textbf{0.7883} & 
    \textbf{0.5459} & \textbf{0.0432} & \textbf{22.91} & \textbf{0.7928} & \textbf{0.3800} \\ \hline
\end{tabular}
}
\end{minipage}\hfill
\end{table}


\begin{table}[t] \centering
\begin{minipage}[t]{0.96\linewidth}
        \captionof{table}{\textbf{Ablation study on geometry priors.}}
        \label{tab:analysis_prior}
        \resizebox{\textwidth}{!}{
\begin{tabular}{c| c c c c c | c c c c}

    \hline
     Method &  Acc$\downarrow$ & Comp $\downarrow$ & C-L1 $\downarrow$ & NC $\uparrow$ & F-Score $\uparrow$ &  Abs Rel $\downarrow$ &  PSNR $\uparrow$  & SSIM  $\uparrow$ & LPIPS $\downarrow$ \\ \hline
    2DGS&   0.1078&  0.0850&  0.0964&  0.7835&  0.5170&  0.1002&  23.56&  0.8166& 0.2730\\
    2DGS+Depth&   0.0862&  0.0702&  0.0782&  0.8153&  0.5965&  0.0672&  23.92&  0.8227& 0.2619 \\
    2DGS+MVDepth&   0.2065&  0.0917&  0.1491&  0.7832&  0.3178&  0.0792&  23.74&  0.8193& 0.2692 \\
    2DGS+Normal&   0.0939&  0.0637&  0.0788&  \textbf{0.8359}&  0.5782&  0.0768&  23.78&  0.8197& 0.2676 \\
    2DGS+FDS &  \textbf{0.0615} & \textbf{ 0.0534}& \textbf{0.0574}& 0.8151& \textbf{0.6974}&  \textbf{0.0561}&  \textbf{24.06}&  \textbf{0.8271}&\textbf{0.2610} \\ \hline
    2DGS+Depth+FDS &  0.0561 &  0.0519& 0.0540& 0.8295& 0.7282&  0.0454&  \textbf{24.22}& \textbf{0.8291}&\textbf{0.2570} \\
    2DGS+Normal+FDS &  \textbf{0.0529} & \textbf{ 0.0450}& \textbf{0.0490}& \textbf{0.8477}& \textbf{0.7430}&  \textbf{0.0443}&  24.10&  0.8283& 0.2590 \\
    2DGS+Depth+Normal &  0.0695 & 0.0513& 0.0604& 0.8540&0.6723&  0.0523&  24.09&  0.8264&0.2575\\ \hline
    2DGS+Depth+Normal+FDS &  \textbf{0.0506} & \textbf{0.0423}& \textbf{0.0464}& \textbf{0.8598}&\textbf{0.7613}&  \textbf{0.0403}&  \textbf{24.22}& 
    \textbf{0.8300}&\textbf{0.0403}\\
    
\bottomrule
\end{tabular}
}
\end{minipage}\hfill
\end{table}




\subsubsection{Mesh extraction}
To further demonstrate the improvement in geometry quality, 
we applied methods used in ~\citep{turkulainen2024dnsplatter} 
to extract meshes from the input views of optimized 3DGS. 
The comparison results are presented  
in \tabref{tab:mushroom}. 
With the integration of FDS, the mesh quality is significantly enhanced compared to the baseline, featuring fewer floaters and more well-defined shapes.
 %
% Following the incorporation of FDS, the reconstruction 
% results exhibit fewer floaters and more well-defined 
% shapes in the meshes. 
% Visualized comparisons
% are provided in the supplementary material.

% \begin{figure}[t] \centering
%     \makebox[0.19\textwidth]{\scriptsize GT}
%     \makebox[0.19\textwidth]{\scriptsize 3DGS}
%     \makebox[0.19\textwidth]{\scriptsize 3DGS+FDS}
%     \makebox[0.19\textwidth]{\scriptsize 2DGS}
%     \makebox[0.19\textwidth]{\scriptsize 2DGS+FDS} \\

%     \includegraphics[width=0.19\textwidth]{figure/fig4_img/compare1/gt02.png}
%     \includegraphics[width=0.19\textwidth]{figure/fig4_img/compare1/baseline06.png}
%     \includegraphics[width=0.19\textwidth]{figure/fig4_img/compare1/baseline_fds05.png}
%     \includegraphics[width=0.19\textwidth]{figure/fig4_img/compare1/2dgs04.png}
%     \includegraphics[width=0.19\textwidth]{figure/fig4_img/compare1/2dgs_fds03.png} \\

%     \includegraphics[width=0.19\textwidth]{figure/fig4_img/compare2/gt00.png}
%     \includegraphics[width=0.19\textwidth]{figure/fig4_img/compare2/baseline02.png}
%     \includegraphics[width=0.19\textwidth]{figure/fig4_img/compare2/baseline_fds01.png}
%     \includegraphics[width=0.19\textwidth]{figure/fig4_img/compare2/2dgs04.png}
%     \includegraphics[width=0.19\textwidth]{figure/fig4_img/compare2/2dgs_fds03.png} \\
      
%     \includegraphics[width=0.19\textwidth]{figure/fig4_img/compare3/gt05.png}
%     \includegraphics[width=0.19\textwidth]{figure/fig4_img/compare3/3dgs03.png}
%     \includegraphics[width=0.19\textwidth]{figure/fig4_img/compare3/3dgs_fds04.png}
%     \includegraphics[width=0.19\textwidth]{figure/fig4_img/compare3/2dgs02.png}
%     \includegraphics[width=0.19\textwidth]{figure/fig4_img/compare3/2dgs_fds01.png} \\

%     \caption{\textbf{Qualitative comparison of extracted mesh 
%     on Mushroom and ScanNet datasets.}}
%     \label{fig:mesh}
% \end{figure}












\subsection{Ablation study}


\textbf{Ablation study on geometry priors:} 
To highlight the advantage of incorporating matching priors, 
we incorporated various types of priors generated by different 
models into 2DGS. These include a monocular depth estimation
model (Depth Anything v2)~\citep{yang2024depth}, a two-view depth estimation 
model (Unimatch)~\citep{xu2023unifying}, 
and a monocular normal estimation model (DSINE)~\citep{bae2024rethinking}.
We adapt the scale and shift-invariant loss in Midas~\citep{birkl2023midas} for
monocular depth supervision and L1 loss for two-view depth supervison.
%
We use Sea Raft~\citep{wang2025sea} as our default optical flow model.
%
The comparison results on Mushroom dataset 
are shown in ~\tabref{tab:analysis_prior}.
We observe that the normal prior provides accurate shape information, 
enhancing the geometric quality of the radiance field. 
%
% In contrast, the monocular depth prior slightly increases 
% the 'Abs Rel' due to its ambiguous scale and inaccurate depth ordering.
% Moreover, the performance of monocular depth estimation 
% in the sauna scene is particularly poor, 
% primarily due to the presence of numerous reflective 
% surfaces and textureless walls, which limits the accuracy of monocular depth estimation.
%
The multi-view depth prior, hindered by the limited feature overlap 
between input views, fails to offer reliable geometric 
information. We test average "Abs Rel" of multi-view depth prior
, and the result is 0.19, which performs worse than the "Abs Rel" results 
rendered by original 2DGS.
From the results, it can be seen that depth order information provided by monocular depth improves
reconstruction accuracy. Meanwhile, our FDS achieves the best performance among all the priors, 
and by integrating all
three components, we obtained the optimal results.
%
%
\begin{figure}[t] \centering
    \makebox[0.16\textwidth]{\scriptsize RF (16000 iters)}
    \makebox[0.16\textwidth]{\scriptsize RF* (20000 iters)}
    \makebox[0.16\textwidth]{\scriptsize RF (20000 iters)  }
    \makebox[0.16\textwidth]{\scriptsize PF (16000 iters)}
    \makebox[0.16\textwidth]{\scriptsize PF (20000 iters)}


    % \includegraphics[width=0.16\textwidth]{figure/fig5_img/compare1/16000.png}
    % \includegraphics[width=0.16\textwidth]{figure/fig5_img/compare1/20000_wo_flow_loss.png}
    % \includegraphics[width=0.16\textwidth]{figure/fig5_img/compare1/20000.png}
    % \includegraphics[width=0.16\textwidth]{figure/fig5_img/compare1/16000_prior.png}
    % \includegraphics[width=0.16\textwidth]{figure/fig5_img/compare1/20000_prior.png}\\

    % \includegraphics[width=0.16\textwidth]{figure/fig5_img/compare2/16000.png}
    % \includegraphics[width=0.16\textwidth]{figure/fig5_img/compare2/20000_wo_flow_loss.png}
    % \includegraphics[width=0.16\textwidth]{figure/fig5_img/compare2/20000.png}
    % \includegraphics[width=0.16\textwidth]{figure/fig5_img/compare2/16000_prior.png}
    % \includegraphics[width=0.16\textwidth]{figure/fig5_img/compare2/20000_prior.png}\\

    \includegraphics[width=0.16\textwidth]{figure/fig5_img/compare3/16000.png}
    \includegraphics[width=0.16\textwidth]{figure/fig5_img/compare3/20000_wo_flow_loss.png}
    \includegraphics[width=0.16\textwidth]{figure/fig5_img/compare3/20000.png}
    \includegraphics[width=0.16\textwidth]{figure/fig5_img/compare3/16000_prior.png}
    \includegraphics[width=0.16\textwidth]{figure/fig5_img/compare3/20000_prior.png}\\
    
    \includegraphics[width=0.16\textwidth]{figure/fig5_img/compare4/16000.png}
    \includegraphics[width=0.16\textwidth]{figure/fig5_img/compare4/20000_wo_flow_loss.png}
    \includegraphics[width=0.16\textwidth]{figure/fig5_img/compare4/20000.png}
    \includegraphics[width=0.16\textwidth]{figure/fig5_img/compare4/16000_prior.png}
    \includegraphics[width=0.16\textwidth]{figure/fig5_img/compare4/20000_prior.png}\\

    \includegraphics[width=0.30\textwidth]{figure/fig5_img/bar.png}

    \caption{\textbf{The error map of Radiance Flow and Prior Flow.} RF: Radiance Flow, PF: Prior Flow, * means that there is no FDS loss supervision during optimization.}
    \label{fig:error_map}
\end{figure}




\textbf{Ablation study on FDS: }
In this section, we present the design of our FDS 
method through an ablation study on the 
Mushroom dataset to validate its effectiveness.
%
The optional configurations of FDS are outlined in ~\tabref{tab:ablation_fds}.
Our base model is the 2DGS equipped with FDS,
and its results are shown 
in the first row. The goal of this analysis 
is to evaluate the impact 
of various strategies on FDS sampling and loss design.
%
We observe that when we 
replace $I_i$ in \eqref{equ:mflow} with $C_i$, 
as shown in the second row, the geometric quality 
of 2DGS deteriorates. Using $I_i$ instead of $C_i$ 
help us to remove the floaters in $\bm{C^s}$, which are also 
remained in $\bm{C^i}$.
We also experiment with modifying the FDS loss. For example, 
in the third row, we use the neighbor 
input view as the sampling view, and replace the 
render result of neighbor view with ground truth image of its input view.
%
Due to the significant movement between images, the Prior Flow fails to accurately 
match the pixel between them, leading to a further degradation in geometric quality.
%
Finally, we attempt to fix the sampling view 
and found that this severely damaged the geometric quality, 
indicating that random sampling is essential for the stability 
of the mean error in the Prior flow.



\begin{table}[t] \centering

\begin{minipage}[t]{1.0\linewidth}
        \captionof{table}{\textbf{Ablation study on FDS strategies.}}
        \label{tab:ablation_fds}
        \resizebox{\textwidth}{!}{
\begin{tabular}{c|c|c|c|c|c|c|c}
    \hline
    \multicolumn{2}{c|}{$\mathcal{M}_{\theta}(X, \bm{C^s})$} & \multicolumn{3}{c|}{Loss} & \multicolumn{3}{c}{Metric}  \\
    \hline
    $X=C^i$ & $X=I^i$  & Input view & Sampled view     & Fixed Sampled view        & Abs Rel $\downarrow$ & F-score $\uparrow$ & NC $\uparrow$ \\
    \hline
    & \ding{51} &     &\ding{51}    &    &    \textbf{0.0561}        &  \textbf{0.6974}         & \textbf{0.8151}\\
    \hline
     \ding{51} &           &     &\ding{51}    &    &    0.0839        &  0.6242         &0.8030\\
     &  \ding{51} &   \ding{51}  &    &    &    0.0877       & 0.6091        & 0.7614 \\
      &  \ding{51} &    &    & \ding{51}    &    0.0724           & 0.6312          & 0.8015 \\
\bottomrule
\end{tabular}
}
\end{minipage}
\end{table}




\begin{figure}[htbp] \centering
    \makebox[0.22\textwidth]{}
    \makebox[0.22\textwidth]{}
    \makebox[0.22\textwidth]{}
    \makebox[0.22\textwidth]{}
    \\

    \includegraphics[width=0.22\textwidth]{figure/fig6_img/l1/rgb/frame00096.png}
    \includegraphics[width=0.22\textwidth]{figure/fig6_img/l1/render_rgb/frame00096.png}
    \includegraphics[width=0.22\textwidth]{figure/fig6_img/l1/render_depth/frame00096.png}
    \includegraphics[width=0.22\textwidth]{figure/fig6_img/l1/depth/frame00096.png}

    % \includegraphics[width=0.22\textwidth]{figure/fig6_img/l2/rgb/frame00112.png}
    % \includegraphics[width=0.22\textwidth]{figure/fig6_img/l2/render_rgb/frame00112.png}
    % \includegraphics[width=0.22\textwidth]{figure/fig6_img/l2/render_depth/frame00112.png}
    % \includegraphics[width=0.22\textwidth]{figure/fig6_img/l2/depth/frame00112.png}

    \caption{\textbf{Limitation of FDS.} }
    \label{fig:limitation}
\end{figure}


% \begin{figure}[t] \centering
%     \makebox[0.48\textwidth]{}
%     \makebox[0.48\textwidth]{}
%     \\
%     \includegraphics[width=0.48\textwidth]{figure/loss_Ignatius.pdf}
%     \includegraphics[width=0.48\textwidth]{figure/loss_family.pdf}
%     \caption{\textbf{Comparison the photometric error of Radiance Flow and Prior Flow:} 
%     We add FDS method after 2k iteration during training.
%     The results show
%     that:  1) The Prior Flow is more precise and 
%     robust than Radiance Flow during the radiance 
%     optimization; 2) After adding the FDS loss 
%     which utilize Prior 
%     flow to supervise the Radiance Flow at 2k iterations, 
%     both flow are more accurate, which lead to
%     a mutually reinforcing effects.(TODO fix it)} 
%     \label{fig:flowcompare}
% \end{figure}






\textbf{Interpretive Experiments: }
To demonstrate the mutual refinement of two flows in our FDS, 
For each view, we sample the unobserved 
views multiple times to compute the mean error 
of both Radiance Flow and Prior Flow. 
We use Raft~\citep{teed2020raft} as our default optical flow model
for visualization.
The ground truth flow is calculated based on 
~\eref{equ:flow_pose} and ~\eref{equ:flow} 
utilizing ground truth depth in dataset.
We introduce our FDS loss after 16000 iterations during 
optimization of 2DGS.
The error maps are shown in ~\figref{fig:error_map}.
Our analysis reveals that Radiance Flow tends to 
exhibit significant geometric errors, 
whereas Prior Flow can more accurately estimate the geometry,
effectively disregarding errors introduced by floating Gaussian points. 

%





\subsection{Limitation and further work}

Firstly, our FDS faces challenges in scenes with 
significant lighting variations between different 
views, as shown in the lamp of first row in ~\figref{fig:limitation}. 
%
Incorporating exposure compensation into FDS could help address this issue. 
%
 Additionally, our method struggles with 
 reflective surfaces and motion blur,
 leading to incorrect matching. 
 %
 In the future, we plan to explore the potential 
 of FDS in monocular video reconstruction tasks, 
 using only a single input image at each time step.
 


\section{Conclusions}
In this paper, we propose Flow Distillation Sampling (FDS), which
leverages the matching prior between input views and 
sampled unobserved views from the pretrained optical flow model, to improve the geometry quality
of Gaussian radiance field. 
Our method can be applied to different approaches (3DGS and 2DGS) to enhance the geometric rendering quality of the corresponding neural radiance fields.
We apply our method to the 3DGS-based framework, 
and the geometry is enhanced on the Mushroom, ScanNet, and Replica datasets.

\section*{Acknowledgements} This work was supported by 
National Key R\&D Program of China (2023YFB3209702), 
the National Natural Science Foundation of 
China (62441204, 62472213), and Gusu 
Innovation \& Entrepreneurship Leading Talents Program (ZXL2024361)
\section{Conclusion}
We introduce a novel approach, \algo, to reduce human feedback requirements in preference-based reinforcement learning by leveraging vision-language models. While VLMs encode rich world knowledge, their direct application as reward models is hindered by alignment issues and noisy predictions. To address this, we develop a synergistic framework where limited human feedback is used to adapt VLMs, improving their reliability in preference labeling. Further, we incorporate a selective sampling strategy to mitigate noise and prioritize informative human annotations.

Our experiments demonstrate that this method significantly improves feedback efficiency, achieving comparable or superior task performance with up to 50\% fewer human annotations. Moreover, we show that an adapted VLM can generalize across similar tasks, further reducing the need for new human feedback by 75\%. These results highlight the potential of integrating VLMs into preference-based RL, offering a scalable solution to reducing human supervision while maintaining high task success rates. 

\section*{Impact Statement}
This work advances embodied AI by significantly reducing the human feedback required for training agents. This reduction is particularly valuable in robotic applications where obtaining human demonstrations and feedback is challenging or impractical, such as assistive robotic arms for individuals with mobility impairments. By minimizing the feedback requirements, our approach enables users to more efficiently customize and teach new skills to robotic agents based on their specific needs and preferences. The broader impact of this work extends to healthcare, assistive technology, and human-robot interaction. One possible risk is that the bias from human feedback can propagate to the VLM and subsequently to the policy. This can be mitigated by personalization of agents in case of household application or standardization of feedback for industrial applications. 

% This must be in the first 5 lines to tell arXiv to use pdfLaTeX, which is strongly recommended.
\pdfoutput=1
% In particular, the hyperref package requires pdfLaTeX in order to break URLs across lines.

\documentclass[11pt]{article}

% Remove the "review" option to generate the final version.
\usepackage[]{ACL2023}

% Standard package includes
\usepackage{times}
\usepackage{latexsym}
\usepackage{graphicx}
\usepackage{amsfonts}
\usepackage{booktabs}
\usepackage{multirow}
\usepackage{makecell}
\usepackage{amsmath}
\usepackage{subfigure}

% For proper rendering and hyphenation of words containing Latin characters (including in bib files)
\usepackage[T1]{fontenc}
% For Vietnamese characters
% \usepackage[T5]{fontenc}
% See https://www.latex-project.org/help/documentation/encguide.pdf for other character sets

% This assumes your files are encoded as UTF8
\usepackage[utf8]{inputenc}

% This is not strictly necessary, and may be commented out.
% However, it will improve the layout of the manuscript,
% and will typically save some space.
\usepackage{microtype}

% This is also not strictly necessary, and may be commented out.
% However, it will improve the aesthetics of text in
% the typewriter font.
\usepackage{inconsolata}

% If the title and author information does not fit in the area allocated, uncomment the following
%
%\setlength\titlebox{<dim>}
%
% and set <dim> to something 5cm or larger.

\title{Back Attention: Understanding and Enhancing \\ Multi-Hop Reasoning in Large Language Models}

% Author information can be set in various styles:
% For several authors from the same institution:
% \author{Author 1 \and ... \and Author n \\
%         Address line \\ ... \\ Address line}
% if the names do not fit well on one line use
%         Author 1 \\ {\bf Author 2} \\ ... \\ {\bf Author n} \\
% For authors from different institutions:
% \author{Author 1 \\ Address line \\  ... \\ Address line
%         \And  ... \And
%         Author n \\ Address line \\ ... \\ Address line}
% To start a seperate ``row'' of authors use \AND, as in
 \author{Zeping Yu \\ University of Manchester          \And
         Yonatan Belinkov \\ Technion - IIT, Israel  \And
         Sophia Ananiadou \\ University of Manchester}

\begin{document}
\maketitle
\begin{abstract}
We investigate how large language models perform latent multi-hop reasoning in prompts like ``Wolfgang Amadeus Mozart's mother's spouse is''. To analyze this process, we introduce logit flow, an interpretability method that traces how logits propagate across layers and positions toward the final prediction. Using logit flow, we identify four distinct stages in single-hop knowledge prediction: (A) entity subject enrichment, (B) entity attribute extraction, (C) relation subject enrichment, and (D) relation attribute extraction. Extending this analysis to multi-hop reasoning, we find that failures often stem from the relation attribute extraction stage, where conflicting logits reduce prediction accuracy. To address this, we propose back attention, a novel mechanism that enables lower layers to leverage higher-layer hidden states from different positions during attention computation. With back attention, a 1-layer transformer achieves the performance of a 2-layer transformer. Applied to four LLMs, back attention improves accuracy on five reasoning datasets, demonstrating its effectiveness in enhancing latent multi-hop reasoning ability.
\end{abstract}

\section{Introduction}
Enhancing the multi-hop reasoning capabilities of large language models (LLMs) has become a central research focus in recent studies \cite{openaio1,qi2024mutual,snell2024scaling,luo2024improve}. A widely used approach, chain-of-thought (COT) reasoning \cite{wei2022chain}, improves accuracy by explicitly articulating intermediate reasoning steps. Many studies have expanded on this idea by generating explicit reasoning chains to further enhance performance \cite{zhou2022least,creswell2022selection,shum2023automatic,yao2024tree}. However, these methods often require substantial computational resources due to multiple inference steps or extensive sampling, leading to high costs and deployment challenges, particularly in large-scale or resource-constrained scenarios.

Therefore, enhancing the ability of latent multi-hop reasoning is crucial for reducing the cost. For example, predicting ``Wolfgang Amadeus Mozart's mother's spouse is'' -> ``Leopold'' demonstrates a model’s ability to internally retrieve and integrate relevant knowledge. Recent studies have investigated the mechanisms underlying latent multi-hop reasoning. Given two hops <e1, r1, e2> and <e2, r2, e3>, where ``e'' represents an ``entity'' and ``r'' a ``relation'', \citet{yang2024large} observe that LLMs can sometimes successfully predict queries like ``The r2 of the r1 of e1 is'' \texttt{->} ``e3'' by latently identifying the bridge entity ``e2''. However, \citet{biran2024hopping} find that the accuracy of latent multi-hop reasoning remains low, even when both individual hops are correct. They hypothesize that the low accuracy arises because factual knowledge is primarily stored in the early layers. If the first hop is resolved too late, the later layers may fail to encode the knowledge for subsequent reasoning steps.

Although latent multi-hop reasoning has been explored, its underlying mechanism remains unclear. First, previous studies primarily focus on the format ``The r2 of the r1 of e1 is''. In this format, the e1 position and the last position inherently obtain the information of r1 and r2, making it unsurprising that information flows between them. A more complex format, ``e1's r1's r2 is'', introduces additional challenges. Due to the autoregressive nature of decoder-only LLMs, earlier positions cannot access later tokens, hindering relational knowledge propagation and leading to lower accuracy than ``The r2 of the r1 of e1 is'' prompts. Second, several studies have shown that the higher attention and feed-forward network (FFN) layers also store knowledge \cite{geva2023dissecting,yu2024neuron}, challenging the prevailing hypothesis about multi-hop reasoning mechanisms. Last, how to leverage interpretability insights to enhance reasoning remains uncertain. Previous studies \cite{sakarvadia2023memory, li2024understanding} rely on model editing methods, which may cause potential risks \cite{gu2024model, gupta2024model}.

\begin{figure}[thb]
  \centering
  \includegraphics[width=0.75\columnwidth]{figure1_new.pdf}
  \caption{Four stages in single-hop knowledge prediction. At entity position: (A) entity subject enrichment by FFN neurons; (B) entity attribute extraction by attention neurons. At relation and last positions: (C) relation subject enrichments by FFN neurons; (D) relation attribute extraction by attention neurons and FFN neurons.}
\vspace{-10pt}
\end{figure}

In this study, we focus on addressing these challenges. First, we propose an innovative interpretability analysis method named ``logit flow'', which analyzes how logits propagate across different layers and positions toward the final prediction on neuron-level. We use logit flow and activation patching \cite{wang2022interpretability} to analyze the mechanism of single-hop knowledge prediction. We examine prompts such as ``e1's r1 is'' \texttt{->} ``e2'', where e1 represents an entity (e.g. Mozart), r1 represents a relation (e.g. mother), and e2 is the correct answer (e.g. Maria), which is also an entity. We find four main stages, as shown in Figure 1: (A) entity subject enrichment by FFN neurons at e1 position, (B) entity attribute extraction by attention neurons at e1 position, (C) relation subject enrichment by FFN neurons at r1 and last positions, and (D) relation attribute extraction by attention neurons and FFN neurons at r1 and last positions. The first two stages align with \citet{geva2023dissecting}, where entity-related features are enriched and extracted (``e1'' \texttt{->} ``e1 features''). Our analysis further reveals that the last two stages integrate these enriched entity features with the relation, facilitating the prediction of the final token (``e1 features \& r1'' \texttt{->} ``e2'').

Next, we use logit flow and activation patching to analyze correct cases and false cases in two-hop reasoning queries like ``e1's r1's r2 is'', where the correct answer is ``e3'' and the false answer is ``e2''. In false cases, the relation attribute extraction stage strongly captures r1 position's high layer information. Since this attribution occurs at a later stage than when the model encodes ``e2'' -> ``e2 features'' and ``e2 features \& r2'' -> ``e3'', it reinforces e2 more than e3, ultimately reducing two-hop reasoning accuracy. Based on the interpretability findings, we propose an innovative method named ``back attention'' to enhance the multi-hop ability, which allows lower layers to capture higher hidden states. When trained from scratch on arithmetic tasks, a 1-layer transformer with back attention achieves the accuracy of a 2-layer transformer. When applied to four LLMs, back attention boosts accuracy across five reasoning datasets, highlighting its effectiveness in improving multi-hop reasoning ability.

Overall, our contributions are as follow:

a) We introduce logit flow, an innovative interpretability method that traces how logits propagate across layers and positions. We demonstrate its effectiveness in both single-hop and multi-hop reasoning. Specifically, for single-hop knowledge prediction, we identify four key stages: entity subject enrichment, entity attribute extraction, relation subject enrichment, and relation attribute extraction.

b) We apply logit flow to analyze both correct and incorrect multi-hop reasoning cases. Our findings reveal that failures often stem from the relation attribute extraction stage, where conflicting logits disrupt accurate predictions.

c) We propose back attention, a novel technique that enhances feature capture in lower layers by integrating higher-level information. This method is effective both for training from scratch and for adapting pretrained LLMs.

\section{Experimental Settings}
In Section 3 and 4, we use the TwoHop reasoning dataset \cite{biran2024hopping}. Each data instance contains two hops like <e1, r1, e2> and <e2, r2, e3>, where e1, e2, e3 are entities and r1, r2 are relations. For instance, <Wolfgang Amadeus Mozart, mother, Maria Anna Mozart> and <Maria Anna Mozart, spouse, Leopold Mozart> represent two such hops.

We formulate prompts for first-hop, second-hop, and two-hop queries as ``e1's r1 is'', ``e2's r2 is'', and ``e1's r1's r2 is'', respectively. Following \citet{biran2024hopping}, we remove shortcut cases \cite{ju2024investigating} and retain the instances where both the first-hop and two-hop predictions are correct. Then we exclude ⟨e1, e2, e3⟩ triplets appearing fewer than 30 times, ensuring that the model has sufficient exposure to the retained knowledge types. To prevent excessive data duplication, we limit the number of cases where the correct answer e3 appears more than five times. In Section 3, we analyze 889 cases where the first-hop, second-hop, and two-hop queries are all answered correctly. In Section 4, we focus on 568 cases where e1, e2, and e3 are all human entities. This set includes both correct and incorrect two-hop reasoning cases, enabling a broader evaluation of multi-hop reasoning by comparing successful and failed cases.

\section{Mechanism of Single-Hop Prediction}
In Section 3.1, we introduce the background. In Section 3.2, we introduce the proposed interpretability method ``logit flow''. In Section 3.3, we utilize logit flow method and identify the four stages in single-hop knowledge prediction.

\subsection{Background}

\paragraph{Residual Stream.} To better understand how logit flow captures information propagation in decoder-only LLMs, we first introduce the residual stream \cite{elhage2021mathematical}. Given an input sentence $X=[t_1, t_2, ..., t_T]$ with $T$ tokens, the model processes it through residual connections, ultimately producing the probability distribution $y$ over $B$ tokens in vocabulary $V$ for the next token prediction. Each token $t_i$ at position $i$ is transformed into a word embedding $h_i^0 \in \mathbb{R}^{d}$ by the embedding matrix $E \in \mathbb{R}^{B \times d}$. Next, the word embeddings are taken as the $0th$ layer input and transformed by $L+1$ transformer layers ($0th-Lth$). The output of layer $l$ is the sum of the layer input, the attention layer output $A_i^l$ and the FFN layer output $F_i^l$:
\begin{equation}
h_i^l = h_i^{l-1} + A_i^{l} + F_i^{l}
\end{equation}
The probability distribution $y$ is computed by multiplying $h_T^L$ (the final layer $L$ output at the last position $T$) and the unembedding matrix $E_u \in \mathbb{R}^{B \times d}$.
\begin{equation}
y = softmax(E_u \, h_T^{L})
\end{equation}
The attention layer output $A_i^l$ can be regarded as the sum of vectors on different heads and positions:

\begin{equation}
A_i^l = \sum_{j=1}^H \sum_{p=1}^T \alpha_{i,j,p}^l W^o_{j,l} (W^v_{j,l} h_p^{l-1})
\end{equation}
\begin{equation}
\alpha_{i,j,p}^l = softmax(W^q_{j,l} h_i^{l-1} \cdot W^k_{j,l} h_p^{l-1})
\end{equation}
where $H$ is the head number and $\alpha$ is the attention score. $W^q$, $W^k$, $W^v$, $W^o$ are the query, key, value and output matrices in each attention head.

\paragraph{FFN and attention neurons.} 
Based on the computation of FFN output (Eq.5), \citet{geva2020transformer} find that the FFN output is a weighted sum of neurons, where each neuron's contribution is determined by its learned weights and input interactions:
\begin{equation}
F_i^l = W_{fc2}^l\sigma (W_{fc1}^l (h_i^{l-1}+A_i^l))
\end{equation}
\begin{equation}
F_i^l = \sum_{k=1}^N {m_{i,k}^l fc2_{k}^l}
\end{equation}
\begin{equation}
m_{i,k}^l = \sigma (fc1_k^l \cdot (h_i^{l-1}+A_i^l))
\end{equation}
Here, $fc2_k^l$ is the $kth$ column of the second MLP $W_{fc2}^l \in \mathbb{R}^{d \times N}$. Its coefficient score $m$ is computed by the inner product between the residual output and $fc1_k^l$ (the $kth$ row of the first MLP $W_{fc1}^l \in \mathbb{R}^{N \times d}$). Similarly, in attention mechanisms, neuron activations are influenced by key-value transformations \cite{yu2024neuron}. These activations shape how information is stored and propagated through layers, ultimately influencing the model’s predictions:
\begin{equation}
A_i^l = \sum_{j=1}^H \sum_{p=1}^T \sum_{e=1}^{d/H} \alpha_{i,j,p}^l \beta_{j,p,e}^l wo_{j,e}^l
\end{equation}
\begin{equation}
\beta_{j,p,e}^l = wv_{j,e}^l \cdot h_p^{l-1}
\end{equation}
Here, $wo_{j,e}^l$ is the $eth$ column of $W^o_{j,l}$, whose coefficient score $\alpha\beta$ is computed by the inner product between the layer input $h_p^{l-1}$ and $wv_{j,e}^l$ (the $eth$ row of $W^v_{j,l}$), combined with the attention score $\alpha$. 

In this study, we define: 1) A \textbf{subvalue} as the column of the second MLP ($fc2$ in FFN and $wo$ in the attention head). 2) A \textbf{subkey} as the row of the first MLP ($fc1$ in FFN and $wv$ in the attention head). 3) A \textbf{neuron} as the product of the coefficient score and the subvalue (Eq. 6 and Eq. 8).

\subsection{Logit Flow: Tracing the Logits on Different Layers and Positions}

\paragraph{Identifying important neurons in deep layers.} Many studies \cite{dar2022analyzing,geva2022transformer,wang2022interpretability,katz2023visit,yu2024neuron,nikankin2024arithmetic} find that the layer-level and neuron-level vectors in deep layers store logits related to final predictions. When we say a vector stores logits about $s$, we mean that multiplying this vector with the unembedding matrix results in a high log probability for $s$, where the probability of a vector is obtained by multiplying this vector with the unembedding matrix (replacing $h^L_T$ with this vector in Eq.2) \cite{nostalgebraist2020}.

The final vector $h_T^L$ stores large logits about the prediction $s$. The logit increase, $log(p(s|h_T^L))-log(p(s|h_T^0))$, can be decomposed into contributions from $L \times N$ FFN neurons and $L \times H \times T \times d/H$ attention neurons. To identify the neurons in deep layers, we use the log probability increase \cite{yu2024neuron} as importance score:
\begin{equation}
Imp(v^l) = log(p(s|v^l+h^{l-1})) - log(p(s|h^{l-1}))
\end{equation}
If the importance score $Imp(v^l)$ of a neuron $v^l$ is large, it indicates that adding this neuron on its layer input $h^{l-1}$ significantly enhances the log probability of the final prediction $s$. 

\paragraph{Identifying important neurons in shallow layers.} Although shallow neurons typically do not store logits directly related to the final prediction, they can contribute by amplifying the coefficient scores of deeper neurons. For instance, in Eq.9, $\beta$ is computed by the inner product between the attention subkey $wv$ and the layer input $h^{l-1}$, where the layer input is the sum of the neurons from previous layers in the residual stream at this position. 

To analyze this effect, we compute the inner product between the subkey of the 300 most important attention neurons and each preceding FFN neuron, weighting the result by the importance score of the attention neuron. This approach allows us to identify the most influential shallow FFN neurons. If a shallow FFN neuron has a high summed inner product score, it indicates that this neuron activates multiple important attention neurons, thereby indirectly increasing the logits of the final prediction. Unlike previous studies \cite{yu2024neuron}, we retain the inner product of each FFN neuron at every position, rather than summing the scores across all positions. This method enables us to analyze which specific positions and layers contribute the most to activating attention neurons.

\paragraph{Logit flow: an interpretability method for analyzing the logits in different positions and layers.} After identifying the deep FFN and attention neurons that store the final logits, we compute and visualize the sum of their importance scores across different layers and positions. A large score in a specific layer or position indicates that it stores crucial information related to the final prediction. Additionally, we compute and illustrate the weighted sum of inner products of FFN neurons at each layer and position, revealing which layers and positions play a significant role in activating important attention neurons. This approach allows us to distinguish the layers and positions that contribute to predictions both directly and indirectly.

\subsection{Four Stages in Single-Hop Prediction} We utilize logit flow to analyze 889 first-hop queries (``e1's r1 is'' -> ``e2''). We compute the average scores across all cases using LLama2-7B \cite{touvron2023llama2}. If an entity or relation consists of multiple BPE tokens, we sum the scores of these tokens across their respective positions in each layer. The average scores on each layer and position are illustrated in Figure 2. In this and all subsequent logit flow visualizations, the horizontal axis represents the layers, while the vertical axis represents the positions. Darker colors indicate higher logits at a specific position and layer.

\begin{figure}[thb]
  \centering
  \includegraphics[width=0.8\columnwidth]{firsthop.pdf}
  \caption{Results of logit flow: ``e1's r1 is'' \texttt{->} ``e2''}
\vspace{-10pt}
\end{figure}

The attention neurons storing logits are distributed across the e1, r1, and last positions, with the layers at e1 being lower than those at r1 and the last position. Similarly, FFN neurons with large inner products are also concentrated at e1, r1, and the last positions, but they generally appear just before the average layers of the attention neurons. The stages at entity position align with the layer-level conclusions in \citet{geva2023dissecting}, where FFN features are activated by the entity's word embeddings and subsequently processed by attention layers.

Additionally, we find that subject enrichment and attribute extraction occur not only at entity position but also at relation and last positions. Due to the autoregressive nature of decoder-only LLMs, the mechanisms at the entity position and r1/last positions differ. At entity position, lower-layer FFN and attention neurons encode knowledge about ``e1 -> e1 features''. In contrast, at the relation and last positions, deeper FFN and attention neurons store knowledge of ``e1 features \& r1 -> e2''. For example, consider ``Mozart's mother is -> Maria'' and ``Mozart's father is -> Leopold''. The hidden states at the position of ``Mozart's'' are identical in both cases, meaning these positions cannot directly determine whether the final prediction is ``Maria'' or ``Leopold''. Instead, at the entity position, lower layers extract Mozart's features containing both ``Maria'' and ``Leopold''. At the relation and last positions, deeper layers refine this information, encoding ``Mozart's features \& mother -> Maria'' and ``Mozart's features \& father -> Leopold'', which enables the model to generate the correct prediction. To verify this, we compute the average logit difference of each layer's hidden state between the correct answer (e.g. Maria) and the conflicting answer (e.g. Leopold) at entity, relation and last positions across all correct human->human cases. The results align with our analysis, detailed in Appendix A. The entity position cannot distinguish the correct answer and the conflicting answer, while the relation and last positions' logit difference start to increase after the entity attribute extraction stage.

We also analyze the logit flow of 889 second-hop cases ``e2's r2 is'' \texttt{->} ``e3'', detailed in Appendix B. Similar to the first-hop results, we observe the same four stages in the second-hop predictions, further validating the single-hop prediction mechanism. In addition, we utilize the activation patching \cite{wang2022interpretability} method to analyze the layer-level information flow, as presented in Appendix C, also observing the importance in entity, relation and last positions. Compared to the layer-level approach, our method provides a neuron-level perspective on information flow, offering a more granular and detailed understanding.

\section{Mechanism of Two-Hop Prediction}
\citet{biran2024hopping} find that the two-hop accuracy remains low, even when both the first-hop and second-hop queries are correct. In this section, we investigate the cause of this phenomenon. We focus on the prompt like ``e1's r1's r2 is'', where the correct answer is ``e3''. We use the logit flow method to analyze the 889 correct two-hop queries, as shown in appendix D. We find that the importance of attention neurons at relation positions is significantly lower than that in single-hop queries. Based on this observation, we hypothesize that the model may incorrectly predict the entity corresponding to ``e1's r1'' or ``e1's r2'' instead of ``e3''. This interference could lead the model to favor intermediate entities over the correct final answer, ultimately reducing the accuracy of two-hop reasoning. 

To verify this, we analyze 568 human->human->human cases with the prompt ``e1's r1's r2 is'' and the correct answer ``e3'' in Llama2-7B, where e1, e2, e3 are all human entities. We compare the ranking of the correct answer ``e3'' against two conflicting answers: ``e1's r1'' and ``e1's r2''. For example, for ``Mozart's mother's spouse is'', the correct answer is ``Leopold'', and the conflicting answers are ``Maria'' (Mozart's mother) and ``Constanze'' (Mozart's spouse). Among 568 cases, 52.3\% correctly predict ``e3'', 42.4\% predict ``e2'' (the answer of ``e1's r1''), and 5.3\% predict the answer of ``e1's r2''. This indicates that the conflicting entities can cause the accuracy decrease. 

\begin{figure}[htb]
  \centering
  \includegraphics[width=0.88\columnwidth]{twohop-human.pdf}
  \caption{Results of logit flow on correct and false human->human->human cases in Llama2-7B.}
\vspace{-10pt}
\end{figure}

To further investigate this phenomenon, we use the logit flow method to compare correct cases (where the predicted answer is ``e3'') with false cases (where the predicted answer is ``e2''), as shown in Figure 3. We observe that in the false cases, the influence at the r1 position is significantly stronger. The results of activation patching (Appendix E) and Llama3.1-8B \& Llama3.2-3B (Appendix F) reveal a similar trend. This finding appears counterintuitive—why does the model predict the wrong answer when it relies more heavily on the features at the r1 position?

A closer look at the single-hop analysis provides an explanation. In the case of ``e1's r1 is'', the high layers at the r1 position store logits related to ``e2''. Due to the autoregressive nature of decoder-only LLMs, the hidden states at r1 position remain the same in both ``e1's r1 is'' and ``e1's r1's r2 is''. Consequently, when the high-layer information at the r1 position is extracted in ``e1's r1's r2 is'', it inadvertently reinforces the probability of ``e2'', leading to lower accuracy in two-hop reasoning.

This phenomenon can also be understood through the four stages of knowledge storage. In the single-hop analysis (Figure 2), the knowledge of ``e1 -> e1 features'' and ``e2 -> e2 features'' is stored in lower layers (layers 7–20), whereas the knowledge of ``e1 features \& r1 -> e2'' and ``e2 features \& r2 -> e3'' is stored in deeper layers (layers 20–31). In two-hop false cases (Figure 3), when the features at r1 positions, which are related to e2, are extracted at layer 28, they only activate the ``e2 features \& r2 -> e3'' parameters in layers 28–31. Although this process does enhance the probability of e3, it amplifies the probability of e2 even more. This imbalance leads to the model predicting e2 instead of e3, resulting in lower accuracy for two-hop reasoning. From this perspective, our results partially align with the "hopping too late" hypothesis \cite{biran2024hopping}. However, our findings reveal a key difference: while some parameters encoding "e2 \& r2 -> e3" are still activated, their contribution is weaker compared to the direct influence of ``e2''.

\section{Back Attention: Letting Lower Layers Capture Higher-Layer features}
Based on the single-hop mechanism, if we can restore the r1 position's deep layer features back to later positions' shallow layers, the parameters storing ``e2 -> e2 features'' and ``e2 features \& r2 -> e3'' can be activated, thereby strengthening the competitiveness of the correct answer. Motivated by this, we propose an innovative technique, ``back attention'', to allow the lower layers capture higher features. The computations of the original attention output $A$ and the back attention output $B$ are shown in Eq. 11-12. In the original attention computation, the query, key, and value vectors are computed by the hidden states $\mathbf{h}$ on the same layer:
\begin{equation}
\begin{small}
\text{A} = \text{Softmax} \left( \frac{\mathbf{h} \mathbf{W}^q (\mathbf{h} \mathbf{W}^k)^\top}{\sqrt{d'}} \right) (\mathbf{h} \mathbf{W}^v) \mathbf{W}^{o}.
\end{small}
\end{equation}
In contrast, back attention modifies this mechanism by computing queries from a lower source layer $\mathbf{hs}$ while obtaining keys and values from a target layer $\mathbf{ht}$, which are the hidden states on a higher layer or the stack of all higher layers' hidden states. This adjustment allows a lower layer to capture richer representations stored in higher layers:
\begin{equation}
\begin{small}
\text{B} = \text{Softmax} \left( \frac{\mathbf{hs} \mathbf{W}^q_{B} (\mathbf{ht} \mathbf{W}^k_{B})^\top}{\sqrt{d'}} \right) (\mathbf{ht} \mathbf{W}^v_{B}) \mathbf{W}^{o}_{B}.
\end{small}
\end{equation}

\begin{figure}[thb]
  \centering
  \includegraphics[width=0.75\columnwidth]{backattention.pdf}
  \caption{Back attention on a 1-layer transformer.}
\vspace{-10pt}
\end{figure}

Figure 4 illustrates how back attention is integrated into a single-layer transformer. Back attention occurs after the original inference pass, during which the hidden states of all layers and positions are calculated. The query vector is computed from the 0th layer input ($\mathbf{hs}$), while the key and value vectors are computed from the 0th layer output ($\mathbf{ht}$). Then the back attention output $B$ is added back onto the 0th layer input, and recompute the forward pass again. Notably, Figure 4 only shows the back attention computation at the last position, while similar computation happens at all positions on the 0th layer input. Back attention restores high-layer features at different positions using the back attention scores. If the back attention score is 1.0 at r1 position and 0.0 at other positions, it means that the r1 position's 0th layer output is added at the last position's 0th layer input. The results when adding back attention in training and fine-tuning stages are as follows.

\paragraph{Training from scratch: back attention enhances the ability of 1-layer transformer.} We conduct experiments on a 2-digit addition arithmetic dataset. In each training and testing set, there are 12,150 single-sum cases (``a+b=''), and 6,188 double-sum cases (``c+d+e=''), where ``a'', ``b'', ``c'', ``d'', and ``e'' are integers ranging from 0 to 99. The model needs to ``memorize'' the single-sum cases and ``learn'' the double-sum patterns. We utilize the Llama tokenizer, representing each digit as a separate token (e.g., 12 is tokenized as [``1'', ``2'']), ensuring that each token appears sufficiently during training. We compare the results of a 1-layer transformer, a 2-layer transformer, and a 1-layer transformer with back attention. In all models, the dimension is 440 for attention/FFN layers, and 160 for back attention. We use the AdamW optimizer \cite{loshchilov2017decoupled} with a learning rate of 0.0001, a batch size of 64, and a maximum of 500 epochs. 

The accuracy of 1-layer transformer, 1-layer transformer with attention, and 2-layer transformer are 83.8\%, 93.8\%, and 92.5\%, respectively. The details of loss and accuracy are shown in Appendix G. The 2-layer transformer and the 1-layer transformer with back attention converge faster than the 1-layer transformer. Notably, the 1-layer transformer with back attention requires only 56.7\% of the parameters of the 2-layer transformer. Therefore, incorporating back attention during the training stage can significantly enhance the model's performance while reducing parameter requirements.

\begin{figure}[thb]
  \centering
  \includegraphics[width=0.99\columnwidth]{layeracc.pdf}
  \caption{Test accuracy of back attention on each layer.}
\vspace{-10pt}
\end{figure}

\paragraph{Adding back attention in pre-trained LLMs: back attention increases the reasoning accuracy.} Back attention can also be integrated into a pre-trained LLM, using all higher-layer states to compute the keys and values. For instance, if back attention is added to the 6th layer, the keys and values are calculated using the layer outputs of all positions from the 6th layer to the final layer. We add back attention on each layer in Llama-7B \cite{touvron2023llama}, fine-tuning on the double-sum arithmetic cases (the same in the previous section). Figure 5 shows the accuracy when fine-tuning back attention on each layer (freezing LLM parameters), where the original accuracy is 67.1\%. The accuracy across the 0-5 layers exhibits significant fluctuation. Adding back attention to the 6th layer achieves a peak accuracy of 93.2\%, followed by a steady decline compared with higher layers.

\begin{table}[htb]
\centering
\begin{small}
%\begin{sc}
\begin{tabular}{cccccc}
\toprule
  & 1DC & SVAMP & MA & TwoHop & SQA \\
\midrule
Llama3 & 72.7 & 55.7 & 21.1 & 11.5 & 65.1 \\
+backattn & 97.0 & 69.3 & 88.9 & 47.8 & 86.2 \\
\midrule
Llama3.1 & 74.6 & 56.0 & 30.0 & 8.8 & 65.4 \\
+backattn & 98.5 & 70.7 & 86.2 & 42.7 & 87.0 \\
\midrule
Llama3.2 & 49.3 & 44.3 & 15.0 & 6.5 & 62.0 \\
+backattn & 92.9 & 62.0 & 52.8 & 37.0 & 86.3 \\
\midrule
Mistral & 51.9 & 63.0 & 26.1 & 8.8 & 71.5 \\
+backattn & 87.4 & 71.7 & 47.2 & 40.1 & 87.8 \\
\bottomrule
\end{tabular}
%\end{sc}
\end{small}
\caption{Accuracy (\%) on 5 datasets before/after adding back attention on 6th layer in four LLMs.}
\vspace{-10pt}
\end{table}

Then we do experiments on 5 arithmetic and reasoning datasets 1-Digit-Composite (1DC) \cite{brown2020language}, SVAMP \cite{patel2021nlp}, MultiArith (MA) \cite{roy2016solving}, TwoHop \cite{biran2024hopping}, and StrategyQA (SQA) \cite{geva2021did}. We fine-tune back attention on the 6th layer in Llama3-8B \cite{meta2024introducing}, Llama3.1-8B \cite{dubey2024llama}, Llama3.2-3B \cite{meta2024llama}, and Mistral-7B \cite{jiang2023mistral}. The accuracy is shown in Table 1. On all LLMs, adding back attention achieves a significant accuracy increase. 

\begin{figure}[thb]
  \centering
  \includegraphics[width=0.9\columnwidth]{casestudy.pdf}
  \caption{Back attention scores at all positions and higher layers when adding on the 6th layer.}
\vspace{-10pt}
\end{figure}

To evaluate whether back attention functions as intended, we analyze the case ``Mozart's mother's spouse is'' \texttt{->} ``Leopold'' in TwoHop dataset and visualize the back attention scores (darker larger) in Figure 6. Back attention effectively learns to recover ``mother'' position's 27-30 layers' hidden states into the last position's 6th layer. This visualization proves that back attention successfully propagates high-layer information from important positions to lower layers, enabling the model to better utilize knowledge for accurate predictions.

\paragraph{Advantages of back attention.} First, back attention can be incorporated during fine-tuning, enabling flexible enhancement of a powerful pre-trained LLM without retraining the model from scratch. Second, the back attention parameters are remarkably lightweight compared to those of pre-trained LLMs. For instance, the back attention parameters account for just 0.002\% of LLama3's 8 billion parameters. Third, back attention exhibits substantial potential, increasing the average accuracy from 46.9\% to 77.0\% in Llama3.1-8B, even when applied to a single layer. Finally, the visualization of back attention scores (Figure 6) serves as an interpretability tool, offering insights into which positions are most critical for a given task, thereby improving our understanding of the mechanisms.

\section{Related Work}
\subsection{Multi-Hop Reasoning in LLMs}
Improving the reasoning ability of LLMs has become a key focus of recent research \cite{lightman2023let,huang2023large,li2024chain,wang2024chain}. \citet{wei2022chain} use chain-of-thought to enhance the reasoning ability by articulating intermediate steps. \citet{fu2022complexity} propose complexity-based prompting, showing that selecting and generating reasoning chains with higher complexity significantly improves reasoning accuracy. \citet{wang2022self} combine chain-of-thought with the self-consistency decoding strategy, achieving significant improvements by sampling diverse reasoning paths and selecting the most consistent answer. \citet{chen2024self} propose self-play fine-tuning, which enhances LLMs' reasoning abilities by refining their outputs through self-generated data, thereby reducing reliance on human-annotated datasets. \citet{brown2024large} propose scaling inference compute by increasing the number of generated samples, demonstrating significant improvements across tasks like coding and math. \citet{hao2023reasoning,yao2024tree} use tree-based methods to improve the performance.

\subsection{Mechanistic Interpretability}
Mechanistic interpretability \cite{Chris2022} aims to reverse engineer the internal mechanisms of LLMs. Logit lens \cite{nostalgebraist2020} is a widely used method \cite{dar2022analyzing,katz2023visit,yu2024interpreting} to analyze the information of hidden states, by multiplying the vectors with the unembedding matrix. A commonly used localization method is causal mediation analysis \cite{vig2020investigating,meng2022locating,stolfo2023mechanistic,geva2023dissecting}, whose core idea is to compute the change of the output when modifying a hidden state. Another types of studies focus on constructing the circuit in the model \cite{olsson2022context,zhang2023towards,gould2023successor,hanna2024does,wang2022interpretability}. Due to the superposition phenomenon \cite{elhage2022toy,scherlis2022polysemanticity,bricken2023towards}, sparse auto-encoder (SAE) is useful for interpreting the features \cite{gao2024scaling,templeton2024scaling,cunningham2023sparse}. A useful characteristic is the residual stream \cite{elhage2021mathematical}, revealing that the final embedding can be represented as the sum of layer outputs. Furthermore, \citet{geva2020transformer,geva2022transformer} find that the FFN output is the weighted sum of FFN neurons. \citet{yu2024neuron} find that the attention head outputs can also be regarded as the weighted sum of attention neurons. 

While previous neuron-level studies primarily focus on ``localization''—identifying which neurons are important—they often lack a deeper ``analysis'' of how these neurons influence predictions. By applying our logit flow method, we gain a clearer understanding of how neurons are activated and contribute to the final prediction.

\section{Conclusion}
We investigate the mechanisms of latent multi-hop reasoning in LLMs and identify key factors affecting the accuracy. Through our interpretability method logit flow, we uncover four distinct stages in single-hop knowledge prediction: entity subject enrichment, entity attribute extraction, relation subject enrichment, and relation attribute extraction. Analyzing two-hop queries, we find that failures often arise in the relation attribute extraction stage, where conflicting logits lower prediction accuracy. To address this, we propose back attention, a novel method that enables lower layers to access higher-layer hidden states, effectively restoring important features. Back attention significantly enhances reasoning performance, allowing a 1-layer transformer to match the accuracy of a 2-layer transformer. When applied to pre-trained LLMs, it improves accuracy across five datasets and four models, demonstrating its effectiveness in multi-hop reasoning. Overall, our analysis provides new insights and introduces a powerful approach for improving reasoning accuracy in LLMs.

\clearpage
\section{Limitations}
In this study, the interpretability analysis primarily focuses on single-hop and two-hop knowledge queries, which represent specific reasoning scenarios. While these cases provide valuable insights, it is important to acknowledge that other types of reasoning tasks might involve different mechanisms not captured in our analysis. Despite these constraints, the observed performance improvements across a variety of reasoning tasks and LLMs suggest that the proposed back attention method and the derived insights possess a degree of general applicability. Further investigations will be needed to validate these findings on more diverse reasoning tasks and refine the interpretability framework for broader applicability.

In this work, back attention is applied to only a single layer, where it has demonstrated promising results. Nevertheless, back attention can also be extended to two or more layers, potentially yielding even greater improvements. We view the success of the single-layer application as a foundational step, paving the way for future research aimed at exploring and optimizing back attention in more complex and multi-layer configurations.

\bibliography{custom}
\bibliographystyle{acl_natbib}

\clearpage
\appendix

\section{Logit Difference at Different Positions}
\begin{figure}[thb]
  \centering
  \includegraphics[width=0.88\columnwidth]{logitdiff.pdf}
  \caption{Logit difference at entity, relation and last positions on human->human cases in Llama2-7B. The logit difference is small at entity position, but large on relation and last positions' deep layers.}
\vspace{-10pt}
\end{figure}

We compute the average logit difference at entity, relation and last positions across all correct human -> human cases, shown in Figure 7. Take ``Mozart's mother is -> Maria'' as an example. We compute the logit difference between ``Maria'' and ``Leopold'' (Mozart's father). At the entity position, the logit difference is small on all layers. At the relation and last positions, the logit difference increases sharply after the entity subject enrichment and entity attribute extraction stages (layers 19–20). This indicates that the entity position primarily extracts general features of ``Mozart'', including information relevant to both ``Maria'' and ``Leopold''. In contrast, the deeper layers at the relation and last positions encode specific knowledge, such as ``Mozart's features \& mother -> Maria'' and ``Mozart's features \& father -> Leopold'', which ultimately differentiate the correct prediction.

\section{Results of Logit Flow on Second-Hop Queries in Llama2-7B}
\begin{figure}[thb]
  \centering
  \includegraphics[width=0.8\columnwidth]{secondhop.pdf}
  \caption{Results of logit flow on second-hop queries ``e2's r2 is'' \texttt{->} ``e3'' in Llama2-7B. There are four similar stages with the first-hop queries: (A) entity subject enrichment, (B) entity attribute extraction, (C) relation subject enrichment, and (D) relation subject extraction.}
\vspace{-10pt}
\end{figure}

The results of logit flow on second-hop queries ``e2's r2 is'' \texttt{->} ``e3'' are shown in Figure 8. There are also four stages existing in the second-hop queries, similar to those in the first-hop queries (Figure 2).

\section{Results of Activation Patching on Single-Hop Queries in Llama2-7B}
The results of activation patching on single-hop queries are shown in Figure 9, using the pyvene \cite{wu2024pyvene} and NNsight \cite{fiotto2024nnsight} libraries. Compared to the logit flow results (Figure 2), the entity and last positions exhibit higher importance, while the relation position appears less significant. This difference arises because activation patching aggregates the importance of both FFN and attention modules into a single visualization. In contrast, the logit flow method distinguishes and separately visualizes the importance of FFN and attention neurons, offering a more granular, neuron-level understanding of the information flow.

\begin{figure}[thb]
  \centering
  \includegraphics[width=0.88\columnwidth]{activationpatching.pdf}
  \caption{Results of activation patching on single-hop queries in Llama2-7B. Similar to logit flow (but not as obvious as logit flow), there is also importance on r1 position's high layers.}
\vspace{-10pt}
\end{figure}

\section{Results of Logit Flow on Two-Hop Queries in Llama2-7B}
\begin{figure}[thb]
  \centering
  \includegraphics[width=0.8\columnwidth]{twohop.pdf}
  \caption{Results of logit flow on two-hop queries ``e1's r1's r2 is'' \texttt{->} ``e3''. The importance of relation positions (r1 and r2) is lower than single-hop queries.}
\vspace{-10pt}
\end{figure}

The results of logit flow on the two-hop queries ``e1's r1's r2 is'' \texttt{->} ``e3'' are shown in Figure 10. Compared to the logit flow results on single-hop queries (Figure 2), the importance of relation positions is significantly lower. This suggests that e1's features at the e1 position are primarily extracted into the last position, potentially activating the parameters associated with ``e1's r1'', ``e1's r2'', and ``e1's r1's r2''. This motivates our exploration between the correct and false human->human->human cases in Section 4.

\section{Results of Activation Patching on Correct and False Two-Hop Queries in Llama2-7B}

The results of activation patching on correct and false human->human->human cases in Llama2-7B are shown in Figure 11. Compared with the correct cases, the false cases show a much clearer influence at r1 position's high layers. This trend is similar to the findings of logit flow method (Figure 3), indicating that the r1 position's high features increase the probability of ``e2'', thereby reducing the accuracy of two-hop reasoning.

\begin{figure}[thb]
  \centering
  \includegraphics[width=0.88\columnwidth]{twohop-human-activationpatching.pdf}
  \caption{Results of activation patching on correct and false human->human->human cases in Llama2-7B. The importance of r1 position is 1.66\% in correct cases and 5.43\% in false cases.}
\vspace{-10pt}
\end{figure}

\section{Results of Logit Flow and Activation Patching on Correct and False Two-Hop Queries in Llama3.1-8B and Llama3.2-3B}

\begin{figure}[thb]
  \centering
  \includegraphics[width=0.88\columnwidth]{twohop-human-llama3.pdf}
  \caption{Results of logit flow on correct and false human->human->human cases in Llama3.1-8B. The importance of r1 position is 6.38\% in correct cases and 32.18\% in false cases.}
\vspace{-10pt}
\end{figure}

\begin{figure}[thb]
  \centering
  \includegraphics[width=0.88\columnwidth]{twohop-human-llama3-activationpatching.pdf}
  \caption{Results of activation patching on correct and false human->human->human cases in Llama3.1-8B. The importance of r1 position is 4.98\% in correct cases and 18.00\% in false cases.}
\vspace{-10pt}
\end{figure}

\begin{figure}[thb]
  \centering
  \includegraphics[width=0.88\columnwidth]{twohop-human-llama3.2.pdf}
  \caption{Results of logit flow on correct and false human->human->human cases in Llama3.2-3B. The importance of r1 position is 17.50\% in correct cases and 40.36\% in false cases.}
\vspace{-10pt}
\end{figure}

\begin{figure}[thb]
  \centering
  \includegraphics[width=0.88\columnwidth]{twohop-human-llama3.2-activationpatching.pdf}
  \caption{Results of activation patching on correct and false human->human->human cases in Llama3.2-3B. The importance of r1 position is 11.23\% in correct cases and 21.52\% in false cases.}
\vspace{-10pt}
\end{figure}

The comparison of correct and false human->human->human cases in Llama3.1-8B are shown in Figure 12 (results of logit flow) and Figure 13 (results of activation patching). Similar results of Llama3.2-3B are shown in Figure 14 (results of logit flow) and Figure 15 (results of activation patching). In both methods and models, the impact of r1 position's high layers in the false cases are larger than that in the correct cases. These results show similar trends with the results of Llama2-7B.

\section{Loss and Accuracy of Back Attention on 1-Layer Transformer}

The loss and accuracy of 1-layer transformer, 1-layer transformer with back attention, and 2-layer transformer are shown in Figure 16. The performance of 1-layer transformer with 2-layer transformer is similar, much better than that of 1-layer transformer.

\begin{figure}[thb]
  \centering
  \includegraphics[width=0.97\columnwidth]{arithmetic-experiment.pdf}
  \caption{Loss (left) and accuracy (right) on arithmetic dataset of 1-layer transformer, 1-layer transformer with back attention, and 2-layer transformer.}
\vspace{-10pt}
\end{figure}


\end{document}


\appendix

\subsection{Lloyd-Max Algorithm}
\label{subsec:Lloyd-Max}
For a given quantization bitwidth $B$ and an operand $\bm{X}$, the Lloyd-Max algorithm finds $2^B$ quantization levels $\{\hat{x}_i\}_{i=1}^{2^B}$ such that quantizing $\bm{X}$ by rounding each scalar in $\bm{X}$ to the nearest quantization level minimizes the quantization MSE. 

The algorithm starts with an initial guess of quantization levels and then iteratively computes quantization thresholds $\{\tau_i\}_{i=1}^{2^B-1}$ and updates quantization levels $\{\hat{x}_i\}_{i=1}^{2^B}$. Specifically, at iteration $n$, thresholds are set to the midpoints of the previous iteration's levels:
\begin{align*}
    \tau_i^{(n)}=\frac{\hat{x}_i^{(n-1)}+\hat{x}_{i+1}^{(n-1)}}2 \text{ for } i=1\ldots 2^B-1
\end{align*}
Subsequently, the quantization levels are re-computed as conditional means of the data regions defined by the new thresholds:
\begin{align*}
    \hat{x}_i^{(n)}=\mathbb{E}\left[ \bm{X} \big| \bm{X}\in [\tau_{i-1}^{(n)},\tau_i^{(n)}] \right] \text{ for } i=1\ldots 2^B
\end{align*}
where to satisfy boundary conditions we have $\tau_0=-\infty$ and $\tau_{2^B}=\infty$. The algorithm iterates the above steps until convergence.

Figure \ref{fig:lm_quant} compares the quantization levels of a $7$-bit floating point (E3M3) quantizer (left) to a $7$-bit Lloyd-Max quantizer (right) when quantizing a layer of weights from the GPT3-126M model at a per-tensor granularity. As shown, the Lloyd-Max quantizer achieves substantially lower quantization MSE. Further, Table \ref{tab:FP7_vs_LM7} shows the superior perplexity achieved by Lloyd-Max quantizers for bitwidths of $7$, $6$ and $5$. The difference between the quantizers is clear at 5 bits, where per-tensor FP quantization incurs a drastic and unacceptable increase in perplexity, while Lloyd-Max quantization incurs a much smaller increase. Nevertheless, we note that even the optimal Lloyd-Max quantizer incurs a notable ($\sim 1.5$) increase in perplexity due to the coarse granularity of quantization. 

\begin{figure}[h]
  \centering
  \includegraphics[width=0.7\linewidth]{sections/figures/LM7_FP7.pdf}
  \caption{\small Quantization levels and the corresponding quantization MSE of Floating Point (left) vs Lloyd-Max (right) Quantizers for a layer of weights in the GPT3-126M model.}
  \label{fig:lm_quant}
\end{figure}

\begin{table}[h]\scriptsize
\begin{center}
\caption{\label{tab:FP7_vs_LM7} \small Comparing perplexity (lower is better) achieved by floating point quantizers and Lloyd-Max quantizers on a GPT3-126M model for the Wikitext-103 dataset.}
\begin{tabular}{c|cc|c}
\hline
 \multirow{2}{*}{\textbf{Bitwidth}} & \multicolumn{2}{|c|}{\textbf{Floating-Point Quantizer}} & \textbf{Lloyd-Max Quantizer} \\
 & Best Format & Wikitext-103 Perplexity & Wikitext-103 Perplexity \\
\hline
7 & E3M3 & 18.32 & 18.27 \\
6 & E3M2 & 19.07 & 18.51 \\
5 & E4M0 & 43.89 & 19.71 \\
\hline
\end{tabular}
\end{center}
\end{table}

\subsection{Proof of Local Optimality of LO-BCQ}
\label{subsec:lobcq_opt_proof}
For a given block $\bm{b}_j$, the quantization MSE during LO-BCQ can be empirically evaluated as $\frac{1}{L_b}\lVert \bm{b}_j- \bm{\hat{b}}_j\rVert^2_2$ where $\bm{\hat{b}}_j$ is computed from equation (\ref{eq:clustered_quantization_definition}) as $C_{f(\bm{b}_j)}(\bm{b}_j)$. Further, for a given block cluster $\mathcal{B}_i$, we compute the quantization MSE as $\frac{1}{|\mathcal{B}_{i}|}\sum_{\bm{b} \in \mathcal{B}_{i}} \frac{1}{L_b}\lVert \bm{b}- C_i^{(n)}(\bm{b})\rVert^2_2$. Therefore, at the end of iteration $n$, we evaluate the overall quantization MSE $J^{(n)}$ for a given operand $\bm{X}$ composed of $N_c$ block clusters as:
\begin{align*}
    \label{eq:mse_iter_n}
    J^{(n)} = \frac{1}{N_c} \sum_{i=1}^{N_c} \frac{1}{|\mathcal{B}_{i}^{(n)}|}\sum_{\bm{v} \in \mathcal{B}_{i}^{(n)}} \frac{1}{L_b}\lVert \bm{b}- B_i^{(n)}(\bm{b})\rVert^2_2
\end{align*}

At the end of iteration $n$, the codebooks are updated from $\mathcal{C}^{(n-1)}$ to $\mathcal{C}^{(n)}$. However, the mapping of a given vector $\bm{b}_j$ to quantizers $\mathcal{C}^{(n)}$ remains as  $f^{(n)}(\bm{b}_j)$. At the next iteration, during the vector clustering step, $f^{(n+1)}(\bm{b}_j)$ finds new mapping of $\bm{b}_j$ to updated codebooks $\mathcal{C}^{(n)}$ such that the quantization MSE over the candidate codebooks is minimized. Therefore, we obtain the following result for $\bm{b}_j$:
\begin{align*}
\frac{1}{L_b}\lVert \bm{b}_j - C_{f^{(n+1)}(\bm{b}_j)}^{(n)}(\bm{b}_j)\rVert^2_2 \le \frac{1}{L_b}\lVert \bm{b}_j - C_{f^{(n)}(\bm{b}_j)}^{(n)}(\bm{b}_j)\rVert^2_2
\end{align*}

That is, quantizing $\bm{b}_j$ at the end of the block clustering step of iteration $n+1$ results in lower quantization MSE compared to quantizing at the end of iteration $n$. Since this is true for all $\bm{b} \in \bm{X}$, we assert the following:
\begin{equation}
\begin{split}
\label{eq:mse_ineq_1}
    \tilde{J}^{(n+1)} &= \frac{1}{N_c} \sum_{i=1}^{N_c} \frac{1}{|\mathcal{B}_{i}^{(n+1)}|}\sum_{\bm{b} \in \mathcal{B}_{i}^{(n+1)}} \frac{1}{L_b}\lVert \bm{b} - C_i^{(n)}(b)\rVert^2_2 \le J^{(n)}
\end{split}
\end{equation}
where $\tilde{J}^{(n+1)}$ is the the quantization MSE after the vector clustering step at iteration $n+1$.

Next, during the codebook update step (\ref{eq:quantizers_update}) at iteration $n+1$, the per-cluster codebooks $\mathcal{C}^{(n)}$ are updated to $\mathcal{C}^{(n+1)}$ by invoking the Lloyd-Max algorithm \citep{Lloyd}. We know that for any given value distribution, the Lloyd-Max algorithm minimizes the quantization MSE. Therefore, for a given vector cluster $\mathcal{B}_i$ we obtain the following result:

\begin{equation}
    \frac{1}{|\mathcal{B}_{i}^{(n+1)}|}\sum_{\bm{b} \in \mathcal{B}_{i}^{(n+1)}} \frac{1}{L_b}\lVert \bm{b}- C_i^{(n+1)}(\bm{b})\rVert^2_2 \le \frac{1}{|\mathcal{B}_{i}^{(n+1)}|}\sum_{\bm{b} \in \mathcal{B}_{i}^{(n+1)}} \frac{1}{L_b}\lVert \bm{b}- C_i^{(n)}(\bm{b})\rVert^2_2
\end{equation}

The above equation states that quantizing the given block cluster $\mathcal{B}_i$ after updating the associated codebook from $C_i^{(n)}$ to $C_i^{(n+1)}$ results in lower quantization MSE. Since this is true for all the block clusters, we derive the following result: 
\begin{equation}
\begin{split}
\label{eq:mse_ineq_2}
     J^{(n+1)} &= \frac{1}{N_c} \sum_{i=1}^{N_c} \frac{1}{|\mathcal{B}_{i}^{(n+1)}|}\sum_{\bm{b} \in \mathcal{B}_{i}^{(n+1)}} \frac{1}{L_b}\lVert \bm{b}- C_i^{(n+1)}(\bm{b})\rVert^2_2  \le \tilde{J}^{(n+1)}   
\end{split}
\end{equation}

Following (\ref{eq:mse_ineq_1}) and (\ref{eq:mse_ineq_2}), we find that the quantization MSE is non-increasing for each iteration, that is, $J^{(1)} \ge J^{(2)} \ge J^{(3)} \ge \ldots \ge J^{(M)}$ where $M$ is the maximum number of iterations. 
%Therefore, we can say that if the algorithm converges, then it must be that it has converged to a local minimum. 
\hfill $\blacksquare$


\begin{figure}
    \begin{center}
    \includegraphics[width=0.5\textwidth]{sections//figures/mse_vs_iter.pdf}
    \end{center}
    \caption{\small NMSE vs iterations during LO-BCQ compared to other block quantization proposals}
    \label{fig:nmse_vs_iter}
\end{figure}

Figure \ref{fig:nmse_vs_iter} shows the empirical convergence of LO-BCQ across several block lengths and number of codebooks. Also, the MSE achieved by LO-BCQ is compared to baselines such as MXFP and VSQ. As shown, LO-BCQ converges to a lower MSE than the baselines. Further, we achieve better convergence for larger number of codebooks ($N_c$) and for a smaller block length ($L_b$), both of which increase the bitwidth of BCQ (see Eq \ref{eq:bitwidth_bcq}).


\subsection{Additional Accuracy Results}
%Table \ref{tab:lobcq_config} lists the various LOBCQ configurations and their corresponding bitwidths.
\begin{table}
\setlength{\tabcolsep}{4.75pt}
\begin{center}
\caption{\label{tab:lobcq_config} Various LO-BCQ configurations and their bitwidths.}
\begin{tabular}{|c||c|c|c|c||c|c||c|} 
\hline
 & \multicolumn{4}{|c||}{$L_b=8$} & \multicolumn{2}{|c||}{$L_b=4$} & $L_b=2$ \\
 \hline
 \backslashbox{$L_A$\kern-1em}{\kern-1em$N_c$} & 2 & 4 & 8 & 16 & 2 & 4 & 2 \\
 \hline
 64 & 4.25 & 4.375 & 4.5 & 4.625 & 4.375 & 4.625 & 4.625\\
 \hline
 32 & 4.375 & 4.5 & 4.625& 4.75 & 4.5 & 4.75 & 4.75 \\
 \hline
 16 & 4.625 & 4.75& 4.875 & 5 & 4.75 & 5 & 5 \\
 \hline
\end{tabular}
\end{center}
\end{table}

%\subsection{Perplexity achieved by various LO-BCQ configurations on Wikitext-103 dataset}

\begin{table} \centering
\begin{tabular}{|c||c|c|c|c||c|c||c|} 
\hline
 $L_b \rightarrow$& \multicolumn{4}{c||}{8} & \multicolumn{2}{c||}{4} & 2\\
 \hline
 \backslashbox{$L_A$\kern-1em}{\kern-1em$N_c$} & 2 & 4 & 8 & 16 & 2 & 4 & 2  \\
 %$N_c \rightarrow$ & 2 & 4 & 8 & 16 & 2 & 4 & 2 \\
 \hline
 \hline
 \multicolumn{8}{c}{GPT3-1.3B (FP32 PPL = 9.98)} \\ 
 \hline
 \hline
 64 & 10.40 & 10.23 & 10.17 & 10.15 &  10.28 & 10.18 & 10.19 \\
 \hline
 32 & 10.25 & 10.20 & 10.15 & 10.12 &  10.23 & 10.17 & 10.17 \\
 \hline
 16 & 10.22 & 10.16 & 10.10 & 10.09 &  10.21 & 10.14 & 10.16 \\
 \hline
  \hline
 \multicolumn{8}{c}{GPT3-8B (FP32 PPL = 7.38)} \\ 
 \hline
 \hline
 64 & 7.61 & 7.52 & 7.48 &  7.47 &  7.55 &  7.49 & 7.50 \\
 \hline
 32 & 7.52 & 7.50 & 7.46 &  7.45 &  7.52 &  7.48 & 7.48  \\
 \hline
 16 & 7.51 & 7.48 & 7.44 &  7.44 &  7.51 &  7.49 & 7.47  \\
 \hline
\end{tabular}
\caption{\label{tab:ppl_gpt3_abalation} Wikitext-103 perplexity across GPT3-1.3B and 8B models.}
\end{table}

\begin{table} \centering
\begin{tabular}{|c||c|c|c|c||} 
\hline
 $L_b \rightarrow$& \multicolumn{4}{c||}{8}\\
 \hline
 \backslashbox{$L_A$\kern-1em}{\kern-1em$N_c$} & 2 & 4 & 8 & 16 \\
 %$N_c \rightarrow$ & 2 & 4 & 8 & 16 & 2 & 4 & 2 \\
 \hline
 \hline
 \multicolumn{5}{|c|}{Llama2-7B (FP32 PPL = 5.06)} \\ 
 \hline
 \hline
 64 & 5.31 & 5.26 & 5.19 & 5.18  \\
 \hline
 32 & 5.23 & 5.25 & 5.18 & 5.15  \\
 \hline
 16 & 5.23 & 5.19 & 5.16 & 5.14  \\
 \hline
 \multicolumn{5}{|c|}{Nemotron4-15B (FP32 PPL = 5.87)} \\ 
 \hline
 \hline
 64  & 6.3 & 6.20 & 6.13 & 6.08  \\
 \hline
 32  & 6.24 & 6.12 & 6.07 & 6.03  \\
 \hline
 16  & 6.12 & 6.14 & 6.04 & 6.02  \\
 \hline
 \multicolumn{5}{|c|}{Nemotron4-340B (FP32 PPL = 3.48)} \\ 
 \hline
 \hline
 64 & 3.67 & 3.62 & 3.60 & 3.59 \\
 \hline
 32 & 3.63 & 3.61 & 3.59 & 3.56 \\
 \hline
 16 & 3.61 & 3.58 & 3.57 & 3.55 \\
 \hline
\end{tabular}
\caption{\label{tab:ppl_llama7B_nemo15B} Wikitext-103 perplexity compared to FP32 baseline in Llama2-7B and Nemotron4-15B, 340B models}
\end{table}

%\subsection{Perplexity achieved by various LO-BCQ configurations on MMLU dataset}


\begin{table} \centering
\begin{tabular}{|c||c|c|c|c||c|c|c|c|} 
\hline
 $L_b \rightarrow$& \multicolumn{4}{c||}{8} & \multicolumn{4}{c||}{8}\\
 \hline
 \backslashbox{$L_A$\kern-1em}{\kern-1em$N_c$} & 2 & 4 & 8 & 16 & 2 & 4 & 8 & 16  \\
 %$N_c \rightarrow$ & 2 & 4 & 8 & 16 & 2 & 4 & 2 \\
 \hline
 \hline
 \multicolumn{5}{|c|}{Llama2-7B (FP32 Accuracy = 45.8\%)} & \multicolumn{4}{|c|}{Llama2-70B (FP32 Accuracy = 69.12\%)} \\ 
 \hline
 \hline
 64 & 43.9 & 43.4 & 43.9 & 44.9 & 68.07 & 68.27 & 68.17 & 68.75 \\
 \hline
 32 & 44.5 & 43.8 & 44.9 & 44.5 & 68.37 & 68.51 & 68.35 & 68.27  \\
 \hline
 16 & 43.9 & 42.7 & 44.9 & 45 & 68.12 & 68.77 & 68.31 & 68.59  \\
 \hline
 \hline
 \multicolumn{5}{|c|}{GPT3-22B (FP32 Accuracy = 38.75\%)} & \multicolumn{4}{|c|}{Nemotron4-15B (FP32 Accuracy = 64.3\%)} \\ 
 \hline
 \hline
 64 & 36.71 & 38.85 & 38.13 & 38.92 & 63.17 & 62.36 & 63.72 & 64.09 \\
 \hline
 32 & 37.95 & 38.69 & 39.45 & 38.34 & 64.05 & 62.30 & 63.8 & 64.33  \\
 \hline
 16 & 38.88 & 38.80 & 38.31 & 38.92 & 63.22 & 63.51 & 63.93 & 64.43  \\
 \hline
\end{tabular}
\caption{\label{tab:mmlu_abalation} Accuracy on MMLU dataset across GPT3-22B, Llama2-7B, 70B and Nemotron4-15B models.}
\end{table}


%\subsection{Perplexity achieved by various LO-BCQ configurations on LM evaluation harness}

\begin{table} \centering
\begin{tabular}{|c||c|c|c|c||c|c|c|c|} 
\hline
 $L_b \rightarrow$& \multicolumn{4}{c||}{8} & \multicolumn{4}{c||}{8}\\
 \hline
 \backslashbox{$L_A$\kern-1em}{\kern-1em$N_c$} & 2 & 4 & 8 & 16 & 2 & 4 & 8 & 16  \\
 %$N_c \rightarrow$ & 2 & 4 & 8 & 16 & 2 & 4 & 2 \\
 \hline
 \hline
 \multicolumn{5}{|c|}{Race (FP32 Accuracy = 37.51\%)} & \multicolumn{4}{|c|}{Boolq (FP32 Accuracy = 64.62\%)} \\ 
 \hline
 \hline
 64 & 36.94 & 37.13 & 36.27 & 37.13 & 63.73 & 62.26 & 63.49 & 63.36 \\
 \hline
 32 & 37.03 & 36.36 & 36.08 & 37.03 & 62.54 & 63.51 & 63.49 & 63.55  \\
 \hline
 16 & 37.03 & 37.03 & 36.46 & 37.03 & 61.1 & 63.79 & 63.58 & 63.33  \\
 \hline
 \hline
 \multicolumn{5}{|c|}{Winogrande (FP32 Accuracy = 58.01\%)} & \multicolumn{4}{|c|}{Piqa (FP32 Accuracy = 74.21\%)} \\ 
 \hline
 \hline
 64 & 58.17 & 57.22 & 57.85 & 58.33 & 73.01 & 73.07 & 73.07 & 72.80 \\
 \hline
 32 & 59.12 & 58.09 & 57.85 & 58.41 & 73.01 & 73.94 & 72.74 & 73.18  \\
 \hline
 16 & 57.93 & 58.88 & 57.93 & 58.56 & 73.94 & 72.80 & 73.01 & 73.94  \\
 \hline
\end{tabular}
\caption{\label{tab:mmlu_abalation} Accuracy on LM evaluation harness tasks on GPT3-1.3B model.}
\end{table}

\begin{table} \centering
\begin{tabular}{|c||c|c|c|c||c|c|c|c|} 
\hline
 $L_b \rightarrow$& \multicolumn{4}{c||}{8} & \multicolumn{4}{c||}{8}\\
 \hline
 \backslashbox{$L_A$\kern-1em}{\kern-1em$N_c$} & 2 & 4 & 8 & 16 & 2 & 4 & 8 & 16  \\
 %$N_c \rightarrow$ & 2 & 4 & 8 & 16 & 2 & 4 & 2 \\
 \hline
 \hline
 \multicolumn{5}{|c|}{Race (FP32 Accuracy = 41.34\%)} & \multicolumn{4}{|c|}{Boolq (FP32 Accuracy = 68.32\%)} \\ 
 \hline
 \hline
 64 & 40.48 & 40.10 & 39.43 & 39.90 & 69.20 & 68.41 & 69.45 & 68.56 \\
 \hline
 32 & 39.52 & 39.52 & 40.77 & 39.62 & 68.32 & 67.43 & 68.17 & 69.30  \\
 \hline
 16 & 39.81 & 39.71 & 39.90 & 40.38 & 68.10 & 66.33 & 69.51 & 69.42  \\
 \hline
 \hline
 \multicolumn{5}{|c|}{Winogrande (FP32 Accuracy = 67.88\%)} & \multicolumn{4}{|c|}{Piqa (FP32 Accuracy = 78.78\%)} \\ 
 \hline
 \hline
 64 & 66.85 & 66.61 & 67.72 & 67.88 & 77.31 & 77.42 & 77.75 & 77.64 \\
 \hline
 32 & 67.25 & 67.72 & 67.72 & 67.00 & 77.31 & 77.04 & 77.80 & 77.37  \\
 \hline
 16 & 68.11 & 68.90 & 67.88 & 67.48 & 77.37 & 78.13 & 78.13 & 77.69  \\
 \hline
\end{tabular}
\caption{\label{tab:mmlu_abalation} Accuracy on LM evaluation harness tasks on GPT3-8B model.}
\end{table}

\begin{table} \centering
\begin{tabular}{|c||c|c|c|c||c|c|c|c|} 
\hline
 $L_b \rightarrow$& \multicolumn{4}{c||}{8} & \multicolumn{4}{c||}{8}\\
 \hline
 \backslashbox{$L_A$\kern-1em}{\kern-1em$N_c$} & 2 & 4 & 8 & 16 & 2 & 4 & 8 & 16  \\
 %$N_c \rightarrow$ & 2 & 4 & 8 & 16 & 2 & 4 & 2 \\
 \hline
 \hline
 \multicolumn{5}{|c|}{Race (FP32 Accuracy = 40.67\%)} & \multicolumn{4}{|c|}{Boolq (FP32 Accuracy = 76.54\%)} \\ 
 \hline
 \hline
 64 & 40.48 & 40.10 & 39.43 & 39.90 & 75.41 & 75.11 & 77.09 & 75.66 \\
 \hline
 32 & 39.52 & 39.52 & 40.77 & 39.62 & 76.02 & 76.02 & 75.96 & 75.35  \\
 \hline
 16 & 39.81 & 39.71 & 39.90 & 40.38 & 75.05 & 73.82 & 75.72 & 76.09  \\
 \hline
 \hline
 \multicolumn{5}{|c|}{Winogrande (FP32 Accuracy = 70.64\%)} & \multicolumn{4}{|c|}{Piqa (FP32 Accuracy = 79.16\%)} \\ 
 \hline
 \hline
 64 & 69.14 & 70.17 & 70.17 & 70.56 & 78.24 & 79.00 & 78.62 & 78.73 \\
 \hline
 32 & 70.96 & 69.69 & 71.27 & 69.30 & 78.56 & 79.49 & 79.16 & 78.89  \\
 \hline
 16 & 71.03 & 69.53 & 69.69 & 70.40 & 78.13 & 79.16 & 79.00 & 79.00  \\
 \hline
\end{tabular}
\caption{\label{tab:mmlu_abalation} Accuracy on LM evaluation harness tasks on GPT3-22B model.}
\end{table}

\begin{table} \centering
\begin{tabular}{|c||c|c|c|c||c|c|c|c|} 
\hline
 $L_b \rightarrow$& \multicolumn{4}{c||}{8} & \multicolumn{4}{c||}{8}\\
 \hline
 \backslashbox{$L_A$\kern-1em}{\kern-1em$N_c$} & 2 & 4 & 8 & 16 & 2 & 4 & 8 & 16  \\
 %$N_c \rightarrow$ & 2 & 4 & 8 & 16 & 2 & 4 & 2 \\
 \hline
 \hline
 \multicolumn{5}{|c|}{Race (FP32 Accuracy = 44.4\%)} & \multicolumn{4}{|c|}{Boolq (FP32 Accuracy = 79.29\%)} \\ 
 \hline
 \hline
 64 & 42.49 & 42.51 & 42.58 & 43.45 & 77.58 & 77.37 & 77.43 & 78.1 \\
 \hline
 32 & 43.35 & 42.49 & 43.64 & 43.73 & 77.86 & 75.32 & 77.28 & 77.86  \\
 \hline
 16 & 44.21 & 44.21 & 43.64 & 42.97 & 78.65 & 77 & 76.94 & 77.98  \\
 \hline
 \hline
 \multicolumn{5}{|c|}{Winogrande (FP32 Accuracy = 69.38\%)} & \multicolumn{4}{|c|}{Piqa (FP32 Accuracy = 78.07\%)} \\ 
 \hline
 \hline
 64 & 68.9 & 68.43 & 69.77 & 68.19 & 77.09 & 76.82 & 77.09 & 77.86 \\
 \hline
 32 & 69.38 & 68.51 & 68.82 & 68.90 & 78.07 & 76.71 & 78.07 & 77.86  \\
 \hline
 16 & 69.53 & 67.09 & 69.38 & 68.90 & 77.37 & 77.8 & 77.91 & 77.69  \\
 \hline
\end{tabular}
\caption{\label{tab:mmlu_abalation} Accuracy on LM evaluation harness tasks on Llama2-7B model.}
\end{table}

\begin{table} \centering
\begin{tabular}{|c||c|c|c|c||c|c|c|c|} 
\hline
 $L_b \rightarrow$& \multicolumn{4}{c||}{8} & \multicolumn{4}{c||}{8}\\
 \hline
 \backslashbox{$L_A$\kern-1em}{\kern-1em$N_c$} & 2 & 4 & 8 & 16 & 2 & 4 & 8 & 16  \\
 %$N_c \rightarrow$ & 2 & 4 & 8 & 16 & 2 & 4 & 2 \\
 \hline
 \hline
 \multicolumn{5}{|c|}{Race (FP32 Accuracy = 48.8\%)} & \multicolumn{4}{|c|}{Boolq (FP32 Accuracy = 85.23\%)} \\ 
 \hline
 \hline
 64 & 49.00 & 49.00 & 49.28 & 48.71 & 82.82 & 84.28 & 84.03 & 84.25 \\
 \hline
 32 & 49.57 & 48.52 & 48.33 & 49.28 & 83.85 & 84.46 & 84.31 & 84.93  \\
 \hline
 16 & 49.85 & 49.09 & 49.28 & 48.99 & 85.11 & 84.46 & 84.61 & 83.94  \\
 \hline
 \hline
 \multicolumn{5}{|c|}{Winogrande (FP32 Accuracy = 79.95\%)} & \multicolumn{4}{|c|}{Piqa (FP32 Accuracy = 81.56\%)} \\ 
 \hline
 \hline
 64 & 78.77 & 78.45 & 78.37 & 79.16 & 81.45 & 80.69 & 81.45 & 81.5 \\
 \hline
 32 & 78.45 & 79.01 & 78.69 & 80.66 & 81.56 & 80.58 & 81.18 & 81.34  \\
 \hline
 16 & 79.95 & 79.56 & 79.79 & 79.72 & 81.28 & 81.66 & 81.28 & 80.96  \\
 \hline
\end{tabular}
\caption{\label{tab:mmlu_abalation} Accuracy on LM evaluation harness tasks on Llama2-70B model.}
\end{table}

%\section{MSE Studies}
%\textcolor{red}{TODO}


\subsection{Number Formats and Quantization Method}
\label{subsec:numFormats_quantMethod}
\subsubsection{Integer Format}
An $n$-bit signed integer (INT) is typically represented with a 2s-complement format \citep{yao2022zeroquant,xiao2023smoothquant,dai2021vsq}, where the most significant bit denotes the sign.

\subsubsection{Floating Point Format}
An $n$-bit signed floating point (FP) number $x$ comprises of a 1-bit sign ($x_{\mathrm{sign}}$), $B_m$-bit mantissa ($x_{\mathrm{mant}}$) and $B_e$-bit exponent ($x_{\mathrm{exp}}$) such that $B_m+B_e=n-1$. The associated constant exponent bias ($E_{\mathrm{bias}}$) is computed as $(2^{{B_e}-1}-1)$. We denote this format as $E_{B_e}M_{B_m}$.  

\subsubsection{Quantization Scheme}
\label{subsec:quant_method}
A quantization scheme dictates how a given unquantized tensor is converted to its quantized representation. We consider FP formats for the purpose of illustration. Given an unquantized tensor $\bm{X}$ and an FP format $E_{B_e}M_{B_m}$, we first, we compute the quantization scale factor $s_X$ that maps the maximum absolute value of $\bm{X}$ to the maximum quantization level of the $E_{B_e}M_{B_m}$ format as follows:
\begin{align}
\label{eq:sf}
    s_X = \frac{\mathrm{max}(|\bm{X}|)}{\mathrm{max}(E_{B_e}M_{B_m})}
\end{align}
In the above equation, $|\cdot|$ denotes the absolute value function.

Next, we scale $\bm{X}$ by $s_X$ and quantize it to $\hat{\bm{X}}$ by rounding it to the nearest quantization level of $E_{B_e}M_{B_m}$ as:

\begin{align}
\label{eq:tensor_quant}
    \hat{\bm{X}} = \text{round-to-nearest}\left(\frac{\bm{X}}{s_X}, E_{B_e}M_{B_m}\right)
\end{align}

We perform dynamic max-scaled quantization \citep{wu2020integer}, where the scale factor $s$ for activations is dynamically computed during runtime.

\subsection{Vector Scaled Quantization}
\begin{wrapfigure}{r}{0.35\linewidth}
  \centering
  \includegraphics[width=\linewidth]{sections/figures/vsquant.jpg}
  \caption{\small Vectorwise decomposition for per-vector scaled quantization (VSQ \citep{dai2021vsq}).}
  \label{fig:vsquant}
\end{wrapfigure}
During VSQ \citep{dai2021vsq}, the operand tensors are decomposed into 1D vectors in a hardware friendly manner as shown in Figure \ref{fig:vsquant}. Since the decomposed tensors are used as operands in matrix multiplications during inference, it is beneficial to perform this decomposition along the reduction dimension of the multiplication. The vectorwise quantization is performed similar to tensorwise quantization described in Equations \ref{eq:sf} and \ref{eq:tensor_quant}, where a scale factor $s_v$ is required for each vector $\bm{v}$ that maps the maximum absolute value of that vector to the maximum quantization level. While smaller vector lengths can lead to larger accuracy gains, the associated memory and computational overheads due to the per-vector scale factors increases. To alleviate these overheads, VSQ \citep{dai2021vsq} proposed a second level quantization of the per-vector scale factors to unsigned integers, while MX \citep{rouhani2023shared} quantizes them to integer powers of 2 (denoted as $2^{INT}$).

\subsubsection{MX Format}
The MX format proposed in \citep{rouhani2023microscaling} introduces the concept of sub-block shifting. For every two scalar elements of $b$-bits each, there is a shared exponent bit. The value of this exponent bit is determined through an empirical analysis that targets minimizing quantization MSE. We note that the FP format $E_{1}M_{b}$ is strictly better than MX from an accuracy perspective since it allocates a dedicated exponent bit to each scalar as opposed to sharing it across two scalars. Therefore, we conservatively bound the accuracy of a $b+2$-bit signed MX format with that of a $E_{1}M_{b}$ format in our comparisons. For instance, we use E1M2 format as a proxy for MX4.

\begin{figure}
    \centering
    \includegraphics[width=1\linewidth]{sections//figures/BlockFormats.pdf}
    \caption{\small Comparing LO-BCQ to MX format.}
    \label{fig:block_formats}
\end{figure}

Figure \ref{fig:block_formats} compares our $4$-bit LO-BCQ block format to MX \citep{rouhani2023microscaling}. As shown, both LO-BCQ and MX decompose a given operand tensor into block arrays and each block array into blocks. Similar to MX, we find that per-block quantization ($L_b < L_A$) leads to better accuracy due to increased flexibility. While MX achieves this through per-block $1$-bit micro-scales, we associate a dedicated codebook to each block through a per-block codebook selector. Further, MX quantizes the per-block array scale-factor to E8M0 format without per-tensor scaling. In contrast during LO-BCQ, we find that per-tensor scaling combined with quantization of per-block array scale-factor to E4M3 format results in superior inference accuracy across models. 


\end{document}
