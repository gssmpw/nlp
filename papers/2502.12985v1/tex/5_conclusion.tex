% !TEX root = ../top.tex
% !TEX spellcheck = en-US

\section{Limitations}
\label{sec:limitations}

The main limitation of \PSDF{} is its reliance on part labels during training. For our intended application, computer-aided design, this is not a major issue because the decomposition of shapes into parts is known \textit{a priori}. Thus \PSDF{} is particularly relevant in scenarios where objects are naturally created with part-aware structures. While this limits \PSDF{}’s applicability to datasets without predefined decompositions—such as most of those available online—it better matches with practical engineering requirements. Nonetheless, if generalization to unlabeled data is desired, \PSDF{} could be combined with co-segmentation methods to generate pseudo-labels for its training. 
%As a proof of concept, we provide an example of such an approach in the supplementary material. \NT{TODO: keep last sentence only if I manage it by the deadline.}


%With regard to evaluation, our most recent part-based baseline DAE-NET is from 2023.  However, as It is unsupervised, the true comparison is against  BAE-NET and PQ-NET that are older. We were not able to compare more recent techniques~cite{Deng22b,Hertz22,Li24c} because no code is publicly available. This is regrettable. However, it is worth noting that these primarily focus on generation or manipulation rather than providing a framework for consistent optimization of the model. In other words, no recent technique we know of is geared towards the wide range of problems we address in this paper using a {\it single} representation. This is why we also compared to the more recent \VecSet{}~\cite{Zhang23d}, even though it is not part-based and therefore not suitable for use in our context. 


\section{Conclusion}
\label{sec:conclusion}

In this work, we introduced \PSDF{}, a modular and supervised approach specifically designed for composite shape representation. \PSDF{} enables flexible, part-based shape modeling, supporting independent part manipulation and optimization while maintaining overall shape coherence. Our method leverages a simple architecture, achieving strong performance across tasks such as shape reconstruction, manipulation and generation.

Experimental results demonstrate that \PSDF{} consistently outperforms baseline methods in part-level accuracy, and can be adapted to a wide range of tasks, highlighting its effectiveness as a robust shape prior for composite structures. This flexibility makes \PSDF{} particularly suited for applications in fields like engineering design, where precise control over individual components is essential for customization and optimization.

While \PSDF{} achieves promising results, future work will  explore enhancing part interactions to further support applications involving highly dynamic shapes or complex inter-part dependencies. Furthermore, developing more advanced topological supervision for each part should also be investigated to help prevent topologically inconsistent predictions, further improving its the robustness and applicability.
