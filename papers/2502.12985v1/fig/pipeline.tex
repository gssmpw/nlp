% !TEX root = ../top.tex
% !TEX spellcheck = en-US


\begin{figure*}[t]
	\centering
%	\includegraphics[width=\textwidth]{fig/pipeline}
	\includegraphics[width=\textwidth]{fig/pipeline2}
	\vspace{-7mm}
	\caption{\textbf{\PSDF{} pipeline.} (a) Our model's core is a part auto-decoder $f_\theta$ that takes as input part latents $\bz_p$ and poses expressed in terms of a quaternion $\bq_p$, translation $\bt_p$, and scale $\bs_p$, along with the query position $\bx\in\real^3$. It outputs signed distances $\hat{s}_p$ for all parts at the queried position, which may be combined into the global signed distance. (b) A secondary model may be used based on the task at hand, such as encoders to map a given modality, \eg, point clouds, to part latents and poses, or a diffusion model to generate them from noise.}
	\label{fig:method}
	\vspace{-3mm}
\end{figure*}