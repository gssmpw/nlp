%%
%% This is file `sample-sigconf-authordraft.tex',
%% generated with the docstrip utility.
%%
%% The original source files were:
%%
%% samples.dtx  (with options: `all,proceedings,bibtex,authordraft')
%% 
%% IMPORTANT NOTICE:
%% 
%% For the copyright see the source file.
%% 
%% Any modified versions of this file must be renamed
%% with new filenames distinct from sample-sigconf-authordraft.tex.
%% 
%% For distribution of the original source see the terms
%% for copying and modification in the file samples.dtx.
%% 
%% This generated file may be distributed as long as the
%% original source files, as listed above, are part of the
%% same distribution. (The sources need not necessarily be
%% in the same archive or directory.)
%%
%%
%% Commands for TeXCount
%TC:macro \cite [option:text,text]
%TC:macro \citep [option:text,text]
%TC:macro \citet [option:text,text]
%TC:envir table 0 1
%TC:envir table* 0 1
%TC:envir tabular [ignore] word
%TC:envir displaymath 0 word
%TC:envir math 0 word
%TC:envir comment 0 0
%%
%%
%% The first command in your LaTeX source must be the \documentclass
%% command.
%%
%% For submission and review of your manuscript please change the
%% command to \documentclass[manuscript, screen, review]{acmart}.
%%
%% When submitting camera ready or to TAPS, please change the command
%% to \documentclass[sigconf]{acmart} or whichever template is required
%% for your publication.
%%
%%
\documentclass[sigconf]{acmart} % for submission

\usepackage{makecell}
\usepackage{enumitem}
\usepackage{multirow}
\usepackage{subfigure}
\usepackage{graphicx}
\usepackage{subfigure}
\usepackage{caption}
\usepackage{subcaption}
\usepackage{tabularx}
\usepackage{xcolor}



% Redaction and commenting macros
\newif\ifredact
\newif\ifcomment

% Set switches
\redactfalse  % Change to \redacttrue to test redaction
\commenttrue  % Change to \commentfalse to test no comments

\newcommand{\redact}[2]{\ifredact #1\else #2\fi}
\newcommand{\dquote}[1]{\textit{``#1''}}

\ifcomment
  \newcommand{\missing}[1]{\textcolor{red}{~#1}}
  \newcommand{\ken}[1]{\textcolor{magenta}{~Kenny: #1}}
  \newcommand{\wei}[1]{\textcolor{orange}{~Weiyan: #1}}
  \newcommand{\discuss}[1]{\textcolor{purple}{Used in discussion: #1}}
\else
  \newcommand{\missing}[1]{}
  \newcommand{\ken}[1]{}
  \newcommand{\wei}[1]{}
  \newcommand{\discuss}[1]{}
\fi


%%
%% \BibTeX command to typeset BibTeX logo in the docs
\AtBeginDocument{%
  \providecommand\BibTeX{{%
    Bib\TeX}}}

%% Rights management information.  This information is sent to you
%% when you complete the rights form.  These commands have SAMPLE
%% values in them; it is your responsibility as an author to replace
%% the commands and values with those provided to you when you
%% complete the rights form.
% \copyrightyear{2025}
% \acmYear{2025}
% \setcopyright{rightsretained}
% \acmConference[CHI EA '25]{Extended Abstracts of the CHI Conference on Human Factors in Computing Systems}{April 26-May 1, 2025}{Yokohama, Japan}
% \acmBooktitle{Extended Abstracts of the CHI Conference on Human Factors in Computing Systems (CHI EA '25), April 26-May 1, 2025, Yokohama, Japan}\acmDOI{10.1145/3706599.3720215}
% \acmISBN{979-8-4007-1395-8/2025/04}
%%  Uncomment \acmBooktitle if the title of the proceedings is different
%%  from ``Proceedings of ...''!
%%
%%\acmBooktitle{Woodstock '18: ACM Symposium on Neural Gaze Detection,
%%  June 03--05, 2018, Woodstock, NY}


%%
%% Submission ID.
%% Use this when submitting an article to a sponsored event. You'll
%% receive a unique submission ID from the organizers
%% of the event, and this ID should be used as the parameter to this command.
%%\acmSubmissionID{123-A56-BU3}

%%
%% For managing citations, it is recommended to use bibliography
%% files in BibTeX format.
%%
%% You can then either use BibTeX with the ACM-Reference-Format style,
%% or BibLaTeX with the acmnumeric or acmauthoryear sytles, that include
%% support for advanced citation of software artefact from the
%% biblatex-software package, also separately available on CTAN.
%%
%% Look at the sample-*-biblatex.tex files for templates showcasing
%% the biblatex styles.
%%

%%
%% The majority of ACM publications use numbered citations and
%% references.  The command \citestyle{authoryear} switches to the
%% "author year" style.
%%
%% If you are preparing content for an event
%% sponsored by ACM SIGGRAPH, you must use the "author year" style of
%% citations and references.
%% Uncommenting
%% the next command will enable that style.
%%\citestyle{acmauthoryear}


%%
%% end of the preamble, start of the body of the document source.
\begin{document}
\settopmatter{printacmref=false} % Removes citation information below abstract
\renewcommand\footnotetextcopyrightpermission[1]{} % Removes copyright footnote
\pagestyle{plain} % Removes running headers

\definecolor{Plum}{rgb}{0.914, 0.008, 0.980}
\definecolor{NavyBlue}{rgb}{0.137, 0.596, 0.969}


%%
%% The "title" command has an optional parameter,
%% allowing the author to define a "short title" to be used in page headers.
\title{Towards Multimodal Large-Language Models for Parent-Child Interaction: A Focus on Joint Attention}

%%
%% The "author" command and its associated commands are used to define
%% the authors and their affiliations.
%% Of note is the shared affiliation of the first two authors, and the
%% "authornote" and "authornotemark" commands
%% used to denote shared contribution to the research.
\author{Weiyan Shi}
\email{weiyanshi6@gmail.com}
\orcid{0009-0001-6035-9678}
\affiliation{
  \institution{Singapore University of Technology and Design}
  \country{Singapore}
}

\author{Viet Hai Le}
\email{lehaivin03@gmail.com}
\orcid{0009-0007-3040-6625}
\affiliation{
  \institution{Singapore University of Technology and Design}
  \country{Singapore}
}

\author{Kenny Tsu Wei Choo}
\email{kennytwchoo@gmail.com}
\orcid{0000-0003-3845-9143}
\affiliation{
  \institution{Singapore University of Technology and Design}
  \country{Singapore}
}
\authornote{Corresponding author. This is a preprint of the paper accepted at CHI 2025 Late Breaking Work. The final version will be available in the ACM Digital Library.}

%%
%% By default, the full list of authors will be used in the page
%% headers. Often, this list is too long, and will overlap
%% other information printed in the page headers. This command allows
%% the author to define a more concise list
%% of authors' names for this purpose.
\renewcommand{\shortauthors}{Shi et al.}

%%
%% The abstract is a short summary of the work to be presented in the
%% article.
\begin{abstract}


The choice of representation for geographic location significantly impacts the accuracy of models for a broad range of geospatial tasks, including fine-grained species classification, population density estimation, and biome classification. Recent works like SatCLIP and GeoCLIP learn such representations by contrastively aligning geolocation with co-located images. While these methods work exceptionally well, in this paper, we posit that the current training strategies fail to fully capture the important visual features. We provide an information theoretic perspective on why the resulting embeddings from these methods discard crucial visual information that is important for many downstream tasks. To solve this problem, we propose a novel retrieval-augmented strategy called RANGE. We build our method on the intuition that the visual features of a location can be estimated by combining the visual features from multiple similar-looking locations. We evaluate our method across a wide variety of tasks. Our results show that RANGE outperforms the existing state-of-the-art models with significant margins in most tasks. We show gains of up to 13.1\% on classification tasks and 0.145 $R^2$ on regression tasks. All our code and models will be made available at: \href{https://github.com/mvrl/RANGE}{https://github.com/mvrl/RANGE}.

\end{abstract}



%%
%% The code below is generated by the tool at http://dl.acm.org/ccs.cfm.
%% Please copy and paste the code instead of the example below.
%%
\begin{CCSXML}
<ccs2012>
   <concept>
       <concept_id>10003120.10003130.10011762</concept_id>
       <concept_desc>Human-centered computing~Empirical studies in collaborative and social computing</concept_desc>
       <concept_significance>500</concept_significance>
       </concept>
   <concept>
       <concept_id>10010147.10010178.10010179</concept_id>
       <concept_desc>Computing methodologies~Natural language processing</concept_desc>
       <concept_significance>500</concept_significance>
       </concept>
   <concept>
       <concept_id>10010147.10010178.10010224</concept_id>
       <concept_desc>Computing methodologies~Computer vision</concept_desc>
       <concept_significance>500</concept_significance>
       </concept>
 </ccs2012>
\end{CCSXML}

\ccsdesc[500]{Human-centered computing~Empirical studies in collaborative and social computing}
\ccsdesc[500]{Computing methodologies~Natural language processing}
\ccsdesc[500]{Computing methodologies~Computer vision}
%%
%% Keywords. The author(s) should pick words that accurately describe
%% the work being presented. Separate the keywords with commas.
\keywords{Parent-Child Joint attention, Multimodal Large Language Models, Eye Contact, Temporal Understanding}
%% A "teaser" image appears between the author and affiliation
%% information and the body of the document, and typically spans the
%% page.
% \begin{teaserfigure}
%   \includegraphics[width=\textwidth]{sampleteaser}
%   \caption{Seattle Mariners at Spring Training, 2010.}
%   \Description{Enjoying the baseball game from the third-base
%   seats. Ichiro Suzuki preparing to bat.}
%   \label{fig:teaser}
% \end{teaserfigure}

% \received{20 February 2007}
% \received[revised]{12 March 2009}
% \received[accepted]{5 June 2009}

%%
%% This command processes the author and affiliation and title
%% information and builds the first part of the formatted document.
\maketitle

\section{Introduction}

Video generation has garnered significant attention owing to its transformative potential across a wide range of applications, such media content creation~\citep{polyak2024movie}, advertising~\citep{zhang2024virbo,bacher2021advert}, video games~\citep{yang2024playable,valevski2024diffusion, oasis2024}, and world model simulators~\citep{ha2018world, videoworldsimulators2024, agarwal2025cosmos}. Benefiting from advanced generative algorithms~\citep{goodfellow2014generative, ho2020denoising, liu2023flow, lipman2023flow}, scalable model architectures~\citep{vaswani2017attention, peebles2023scalable}, vast amounts of internet-sourced data~\citep{chen2024panda, nan2024openvid, ju2024miradata}, and ongoing expansion of computing capabilities~\citep{nvidia2022h100, nvidia2023dgxgh200, nvidia2024h200nvl}, remarkable advancements have been achieved in the field of video generation~\citep{ho2022video, ho2022imagen, singer2023makeavideo, blattmann2023align, videoworldsimulators2024, kuaishou2024klingai, yang2024cogvideox, jin2024pyramidal, polyak2024movie, kong2024hunyuanvideo, ji2024prompt}.


In this work, we present \textbf{\ours}, a family of rectified flow~\citep{lipman2023flow, liu2023flow} transformer models designed for joint image and video generation, establishing a pathway toward industry-grade performance. This report centers on four key components: data curation, model architecture design, flow formulation, and training infrastructure optimization—each rigorously refined to meet the demands of high-quality, large-scale video generation.


\begin{figure}[ht]
    \centering
    \begin{subfigure}[b]{0.82\linewidth}
        \centering
        \includegraphics[width=\linewidth]{figures/t2i_1024.pdf}
        \caption{Text-to-Image Samples}\label{fig:main-demo-t2i}
    \end{subfigure}
    \vfill
    \begin{subfigure}[b]{0.82\linewidth}
        \centering
        \includegraphics[width=\linewidth]{figures/t2v_samples.pdf}
        \caption{Text-to-Video Samples}\label{fig:main-demo-t2v}
    \end{subfigure}
\caption{\textbf{Generated samples from \ours.} Key components are highlighted in \textcolor{red}{\textbf{RED}}.}\label{fig:main-demo}
\end{figure}


First, we present a comprehensive data processing pipeline designed to construct large-scale, high-quality image and video-text datasets. The pipeline integrates multiple advanced techniques, including video and image filtering based on aesthetic scores, OCR-driven content analysis, and subjective evaluations, to ensure exceptional visual and contextual quality. Furthermore, we employ multimodal large language models~(MLLMs)~\citep{yuan2025tarsier2} to generate dense and contextually aligned captions, which are subsequently refined using an additional large language model~(LLM)~\citep{yang2024qwen2} to enhance their accuracy, fluency, and descriptive richness. As a result, we have curated a robust training dataset comprising approximately 36M video-text pairs and 160M image-text pairs, which are proven sufficient for training industry-level generative models.

Secondly, we take a pioneering step by applying rectified flow formulation~\citep{lipman2023flow} for joint image and video generation, implemented through the \ours model family, which comprises Transformer architectures with 2B and 8B parameters. At its core, the \ours framework employs a 3D joint image-video variational autoencoder (VAE) to compress image and video inputs into a shared latent space, facilitating unified representation. This shared latent space is coupled with a full-attention~\citep{vaswani2017attention} mechanism, enabling seamless joint training of image and video. This architecture delivers high-quality, coherent outputs across both images and videos, establishing a unified framework for visual generation tasks.


Furthermore, to support the training of \ours at scale, we have developed a robust infrastructure tailored for large-scale model training. Our approach incorporates advanced parallelism strategies~\citep{jacobs2023deepspeed, pytorch_fsdp} to manage memory efficiently during long-context training. Additionally, we employ ByteCheckpoint~\citep{wan2024bytecheckpoint} for high-performance checkpointing and integrate fault-tolerant mechanisms from MegaScale~\citep{jiang2024megascale} to ensure stability and scalability across large GPU clusters. These optimizations enable \ours to handle the computational and data challenges of generative modeling with exceptional efficiency and reliability.


We evaluate \ours on both text-to-image and text-to-video benchmarks to highlight its competitive advantages. For text-to-image generation, \ours-T2I demonstrates strong performance across multiple benchmarks, including T2I-CompBench~\citep{huang2023t2i-compbench}, GenEval~\citep{ghosh2024geneval}, and DPG-Bench~\citep{hu2024ella_dbgbench}, excelling in both visual quality and text-image alignment. In text-to-video benchmarks, \ours-T2V achieves state-of-the-art performance on the UCF-101~\citep{ucf101} zero-shot generation task. Additionally, \ours-T2V attains an impressive score of \textbf{84.85} on VBench~\citep{huang2024vbench}, securing the top position on the leaderboard (as of 2025-01-25) and surpassing several leading commercial text-to-video models. Qualitative results, illustrated in \Cref{fig:main-demo}, further demonstrate the superior quality of the generated media samples. These findings underscore \ours's effectiveness in multi-modal generation and its potential as a high-performing solution for both research and commercial applications.
\section{Related Work}

\subsection{Large 3D Reconstruction Models}
Recently, generalized feed-forward models for 3D reconstruction from sparse input views have garnered considerable attention due to their applicability in heavily under-constrained scenarios. The Large Reconstruction Model (LRM)~\cite{hong2023lrm} uses a transformer-based encoder-decoder pipeline to infer a NeRF reconstruction from just a single image. Newer iterations have shifted the focus towards generating 3D Gaussian representations from four input images~\cite{tang2025lgm, xu2024grm, zhang2025gslrm, charatan2024pixelsplat, chen2025mvsplat, liu2025mvsgaussian}, showing remarkable novel view synthesis results. The paradigm of transformer-based sparse 3D reconstruction has also successfully been applied to lifting monocular videos to 4D~\cite{ren2024l4gm}. \\
Yet, none of the existing works in the domain have studied the use-case of inferring \textit{animatable} 3D representations from sparse input images, which is the focus of our work. To this end, we build on top of the Large Gaussian Reconstruction Model (GRM)~\cite{xu2024grm}.

\subsection{3D-aware Portrait Animation}
A different line of work focuses on animating portraits in a 3D-aware manner.
MegaPortraits~\cite{drobyshev2022megaportraits} builds a 3D Volume given a source and driving image, and renders the animated source actor via orthographic projection with subsequent 2D neural rendering.
3D morphable models (3DMMs)~\cite{blanz19993dmm} are extensively used to obtain more interpretable control over the portrait animation. For example, StyleRig~\cite{tewari2020stylerig} demonstrates how a 3DMM can be used to control the data generated from a pre-trained StyleGAN~\cite{karras2019stylegan} network. ROME~\cite{khakhulin2022rome} predicts vertex offsets and texture of a FLAME~\cite{li2017flame} mesh from the input image.
A TriPlane representation is inferred and animated via FLAME~\cite{li2017flame} in multiple methods like Portrait4D~\cite{deng2024portrait4d}, Portrait4D-v2~\cite{deng2024portrait4dv2}, and GPAvatar~\cite{chu2024gpavatar}.
Others, such as VOODOO 3D~\cite{tran2024voodoo3d} and VOODOO XP~\cite{tran2024voodooxp}, learn their own expression encoder to drive the source person in a more detailed manner. \\
All of the aforementioned methods require nothing more than a single image of a person to animate it. This allows them to train on large monocular video datasets to infer a very generic motion prior that even translates to paintings or cartoon characters. However, due to their task formulation, these methods mostly focus on image synthesis from a frontal camera, often trading 3D consistency for better image quality by using 2D screen-space neural renderers. In contrast, our work aims to produce a truthful and complete 3D avatar representation from the input images that can be viewed from any angle.  

\subsection{Photo-realistic 3D Face Models}
The increasing availability of large-scale multi-view face datasets~\cite{kirschstein2023nersemble, ava256, pan2024renderme360, yang2020facescape} has enabled building photo-realistic 3D face models that learn a detailed prior over both geometry and appearance of human faces. HeadNeRF~\cite{hong2022headnerf} conditions a Neural Radiance Field (NeRF)~\cite{mildenhall2021nerf} on identity, expression, albedo, and illumination codes. VRMM~\cite{yang2024vrmm} builds a high-quality and relightable 3D face model using volumetric primitives~\cite{lombardi2021mvp}. One2Avatar~\cite{yu2024one2avatar} extends a 3DMM by anchoring a radiance field to its surface. More recently, GPHM~\cite{xu2025gphm} and HeadGAP~\cite{zheng2024headgap} have adopted 3D Gaussians to build a photo-realistic 3D face model. \\
Photo-realistic 3D face models learn a powerful prior over human facial appearance and geometry, which can be fitted to a single or multiple images of a person, effectively inferring a 3D head avatar. However, the fitting procedure itself is non-trivial and often requires expensive test-time optimization, impeding casual use-cases on consumer-grade devices. While this limitation may be circumvented by learning a generalized encoder that maps images into the 3D face model's latent space, another fundamental limitation remains. Even with more multi-view face datasets being published, the number of available training subjects rarely exceeds the thousands, making it hard to truly learn the full distibution of human facial appearance. Instead, our approach avoids generalizing over the identity axis by conditioning on some images of a person, and only generalizes over the expression axis for which plenty of data is available. 

A similar motivation has inspired recent work on codec avatars where a generalized network infers an animatable 3D representation given a registered mesh of a person~\cite{cao2022authentic, li2024uravatar}.
The resulting avatars exhibit excellent quality at the cost of several minutes of video capture per subject and expensive test-time optimization.
For example, URAvatar~\cite{li2024uravatar} finetunes their network on the given video recording for 3 hours on 8 A100 GPUs, making inference on consumer-grade devices impossible. In contrast, our approach directly regresses the final 3D head avatar from just four input images without the need for expensive test-time fine-tuning.


\begin{table*}[htb!]
\centering
\caption{Comparison of annotation between SLPs: \textbf{SLP} (annotator ID), \textbf{Total} (number of segments), \textbf{Strong (\%)} and \textbf{Poor (\%)} (counts and percentages of strong/poor joint attention), \textbf{Avg. (s)} (average segment duration), \textbf{Avg. Strong (s)}, and \textbf{Avg. Poor (s)} (average durations of strong and poor segments). The \textbf{Intersection} row shows metrics for mutually agreed segments.}
\begin{tabular}{p{0.12\textwidth} p{0.12\textwidth} p{0.12\textwidth} p{0.12\textwidth} p{0.12\textwidth} p{0.12\textwidth} p{0.12\textwidth}} 
\toprule
\textbf{SLP} & \textbf{Total} & \textbf{Strong (\%)} & \textbf{Poor (\%)} & \textbf{Avg. (s)} & \textbf{Avg. Strong (s)} & \textbf{Avg. Poor (s)} \\ 
\midrule
S1           & 96             & 89 (92.71)           & 7 (7.29)           & 5.61               & 5.24                      & 10.30                   \\
S2           & 72             & 56 (77.78)           & 16 (22.22)         & 14.51              & 14.85                     & 13.33                   \\
\midrule
Intersection & 62             & 58 (93.55)           & 4 (6.45)           & 3.88               & 3.65                      & 10.73                   \\
\bottomrule
\end{tabular}
\label{tab:segment_metrics}
\end{table*}
\section{Video Data Collection and Preparation}
\subsection{Video Selection Method}
We collected videos from YouTube using targeted keywords \dquote{parent-child interaction} to create a dataset for analysing parent-child interactions, carefully selecting videos that met quality and relevance criteria. Each video had to feature one child and one adult as the primary subjects. While most videos required active interactions between the adult and the child, we also included cases where the adult's role was limited to accompanying a very young child for support without directly engaging in the interaction. Videos involving multiple children or adults actively participating were excluded to maintain the focus on dyadic dynamics.

We applied stringent selection criteria to ensure the videos were suitably focused towards analysing multimodal signals. The scenes needed to be static with minimal camera movement, and the videos had to provide clear views of both the child's and adult's faces, enabling precise analysis of gaze direction and expressions. High-quality audio and visual clarity were also essential for accurate verbal and non-verbal communication observations.

Additionally, the dataset was curated to include children across a broad age range to capture a variety of developmental stages. We prioritised videos that showcased diverse and meaningful interactions, such as language learning activities, skill-building tasks, or natural daily exchanges. All textual elements, including subtitles and transitions, were removed to preserve the raw multimodal content.

\subsection{Dataset Composition and Analysis}
We curated a final dataset comprising 26 parent-child interaction videos that satisfied our quality and relevance criteria. These videos are of dialogue and interaction in fixed settings, covering age ranges between 0-8 years (age information obtained from the descriptions provided on the video websites). The majority of videos focus on children aged 4–6 years (n=16), with fewer videos in the 2–4 (n=5) and 0–2 (n=4) age groups. Only one video includes children aged 6–8, concentrating on younger age groups where parent-child interactions are more actively studied.

The duration of the videos ranges from brief clips of around 0.5 to 12 minutes. Most videos are relatively short: 13 videos are less than 1 minute, and 11 fall between 1 and 5 minutes. Only two videos extend beyond 5 minutes, one categorised as 5–10 minutes and the other over 10 minutes. This distribution highlights the concise nature of the interactions and tasks typically observed in parent-child settings.

The video content in the dataset can be categorised into three main categories. The first category is behaviour guidance and skill modelling (n=10), where parents demonstrate specific techniques (e.g., PRIDE skills, \dquote{Big Ignore} techniques) to guide children's behaviour and improve interaction quality. The second category focuses on language and cognitive development (n=10), showcasing how parent-child interactions foster language learning and cognitive skills through tasks or experiments, such as discussing specific topics or engaging in Piaget's cognitive experiments. The third category involves daily life skills and interaction (n=6), presenting natural exchanges in structured settings, such as reviewing reward charts or ending special playtime, emphasising practical life skills and building strong parent-child relationships.

% \begin{figure}[htbp]
%     \begin{minipage}[t]{0.32\linewidth}
%         \centering
%         \includegraphics[width=\textwidth]{figures/age_distribution.png}
%         \caption{Age Distribution}
%         \label{fig:age_distribution}
%     \end{minipage}%
%     \begin{minipage}[t]{0.32\linewidth}
%         \centering
%         \includegraphics[width=\textwidth]{figures/duration_distribution.png}
%         \caption{Duration Distribution}
%         \label{fig:duration_distribution}
%     \end{minipage}
%     \begin{minipage}[t]{0.32\linewidth}
%         \centering
%         \includegraphics[width=\textwidth]{figures/classification_distribution.png}
%         \caption{Classification Distribution}
%         \label{fig:classification_distribution}
%     \end{minipage}
% \end{figure}

\section{Speech-Language Pathologists Annotation Study}
\subsection{Annotation Process}
To evaluate the performance of MLLMs in identifying strong or poor joint attention segments, we recruited two SLPs (both female) with extensive experience to provide expert annotations. They each have 7–9 years of experience and training in programs such as DIR Floortime~\cite{dir_floortime}, It Takes Two to Talk~\cite{pepper2004talk}, and More Than Words~\cite{sussman1999words}. While a third SLP was recruited, the session was incomplete and the results were not included in this paper. Our studies complied with ethical regulations approved by our institution's ethics review board.


% The first expert holds a bachelor's degree in Speech Pathology and has 9 years of experience working with children. She is trained in DIR Floortime and the Hanen Programs (It Takes Two to Talk and More Than Words) and is skilled in various communication and sensory intervention techniques.  

% The second expert has a bachelor's degree in Speech Pathology and 7 years of experience in pediatric speech therapy. She is certified in DIR Floortime and the Hanen More Than Words program, focusing on enhancing communication and social skills in children with developmental delays.  

The study was conducted in person over one month. Each session lasted approximately 2 hours and was held in a private meeting room to ensure focus and comfort. The labelling was carried out using our custom-built annotation system. Videos can be played, paused, or navigated frame-by-frame, while the interactive timeline enables users to select segments and label them as \dquote{strong} or \dquote{poor}. The system exports the results in CSV format, including each labelled segment's start and end times, along with corresponding indices. Additionally, a panel allows experts to add notes during the labelling process.

% \begin{table}[htb!]
% \centering
% \caption{Comparison of Annotation Across SLPs: \textbf{SLP} (annotator ID), \textbf{Total} (number of segments), \textbf{Strong (\%)} and \textbf{Poor (\%)} (counts and percentages of strong/poor joint attention), \textbf{Avg. (s)} (average segment duration), \textbf{Avg. Strong (s)}, and \textbf{Avg. Poor (s)} (average durations of strong and poor segments). The \textbf{Intersection} row shows metrics for mutually agreed segments.}
% \begin{tabular*}{\textwidth}{@{\extracolsep{\fill}}lcccccc}
% \toprule
% \textbf{SLP} & \textbf{Total} & \textbf{Strong (\%)} & \textbf{Poor (\%)} & \textbf{Avg. (s)} & \textbf{Avg. Strong (s)} & \textbf{Avg. Poor (s)} \\ \midrule
% S1        & 96          & 89 (92.71)           & 7 (7.29)            & 5.61              & 5.24                    & 10.30                 \\
% S2        & 72          & 56 (77.78)           & 16 (22.22)          & 14.51             & 14.85                   & 13.33                 \\\midrule
% Intersection      & 62            & 58 (93.55)              & 4 (6.45)              & 3.88                             & 3.65 &       10.73    \\       
% \bottomrule
% \end{tabular*}
% \label{tab:segment_metrics}
% \end{table}


\subsection{Annotation Outcomes Analysis}
We conducted full annotation interviews with SLP 1 (\textit{S1}) and SLP 2 (\textit{S2}). Through these interviews, a shared criterion for determining strong joint attention was established: it is characterized by the child maintaining consistent eye contact and \textit{active engagement} with the parent during the interaction. 
% Due to time constraints, Expert 3 was unable to participate in the interviews, and the results are based solely on the annotations provided by \textit{E1} and \textit{E2}.

Table \ref{tab:segment_metrics} summarizes the annotation metrics provided by the two SLPs. \textit{S1} annotated a total of 96 segments. Among these, \textit{92.71\%} were identified as strong segments, with an average segment duration of \textit{5.61 s}. Strong segments were notably shorter, averaging \textit{5.24 s}, while poor segments were longer at \textit{10.30 s}. \textit{S2} annotated 72 segments. Among these, \textit{77.78\%} were classified as strong, and \textit{22.22\%} as poor. The average duration for strong and poor segments was \textit{14.85 s} and \textit{13.33 s}, respectively, reflecting S2's longer segments in annotations.

To ensure fairness and consistency, we calculated the intersection of annotations where SLPs agreed. This resulted in \textit{62} segments, with \textit{93.55\%} identified as strong and \textit{6.45\%} as poor. The average duration for strong and poor segments in the intersection was \textit{3.65 s} and \textit{10.73 s}, respectively, while the overall average duration was \textit{3.88 s}.

\section{MLLM Comparative Study}
\subsection{Experimental Setup and Evaluation Metrics}
% Recent advances in MLLMs have demonstrated significant potential in general-purpose video understanding. However, as noted in~\cite{liu2024bench}, state-of-the-art models designed for video-level understanding face challenges in handling fine-grained tasks, such as temporal video grounding.

To test the performance of MLLMs on our parent-child joint-attention detection task, a fine-grained temporal video grounding challenge, we selected three state-of-the-art MLLMs to evaluate: GPT-4o-2024-08-06\footnote{\url{https://openai.com/index/hello-gpt-4o/}}, Gemini 1.5 Flash\footnote{\url{https://deepmind.google/technologies/gemini/flash/}}, and Video-ChatGPT~\cite{maaz2023video}.

The Gemini 1.5 model directly supports audio and video processing, allowing us to input the entire video seamlessly. In contrast, Video-ChatGPT can only process videos without audio, so we provided the full video without additional preprocessing. GPT-4o, however, does not natively support direct video processing. To address this, we extracted video frames and converted them into a Base64-encoded array of images. 
We also used WhisperX~\cite{bain2023whisperx} to transcribe the videos' audio and manually enriched these transcriptions with notes for greater informativeness. 
For instance, we included annotations such as \dquote{the child is laughing} and descriptions of background sounds, which helped compensate for the unclear speech often observed in younger children. These transcriptions were used as text input, serving as a substitute for audio input for both Video-ChatGPT and GPT-4o.

We designed Instruction Templates (see Table~\ref{tab:instructions}) inspired by Liu et al.~\cite{liu2024bench} and based on our interviews with SLPs. 
Our interviews identified key criteria for determining joint attention, such as consistent eye contact and full engagement between the child and parent.

\begin{table*}[!htb]
\small
  \caption{Instruction templates in our task. \textcolor{orange}{<time>} denotes the timestamp representation, e.g., "23.6s" \textcolor{NavyBlue}{<description>} denotes the brief summary of the segment focused on the child's joint attention behaviour. \textcolor{Plum}{<label>} denotes the child joint attention quality of the segment.}
  \label{tab:instructions}
  \Description{}
  \begin{tabular}{p{.5\linewidth} p{.4\linewidth}}
    \toprule
    \textbf{Instruction} &  \textbf{Example Response}\\
    \midrule
    You are given a video about parent-child interactions. Watch the video carefully and your task is to:

    1. Identify the key moments where the child's joint attention occurs. Child joint attention is defined as the child making eye contact and being fully engaged with the parent through interaction.
    
    2. Specify the timestamps for when each moment starts and ends.

    3. Classify the quality of the child's joint attention into "Strong" and "Poor".

    For example:
    
    - Timestamp: 23.6s - 26.8s
    
    - Description: The child looks at his mother as she gives him instructions.
    
    - Quality: Strong
    &
    Timestamp: \textcolor{orange}{<time>} - \textcolor{orange}{<time>} and \textcolor{orange}{<time>} - \textcolor{orange}{<time>}
    
    Description: \textcolor{NavyBlue}{<description>}
    
    Quality: \textcolor{Plum}{<label>}
    \\
  \bottomrule
\end{tabular}
\end{table*}

\subsubsection{Evaluation Method of MLLMs Output Quality and Content Objectivity}

To assess the objectivity and quality of the models' responses, we define four metrics building on key aspects emphasized by SLPs in their annotations:

\begin{itemize}
    \item \textbf{Time Sensitivity}: Proportion of responses that include \textcolor{orange}{<time>} - \textcolor{orange}{<time>}.
    \item \textbf{Description Accuracy}: Proportion of accurate \textcolor{NavyBlue}{<description>}s within the specified \textcolor{orange}{<time>} - \textcolor{orange}{<time>} in the responses.
    \item \textbf{Eye-Contact Sensitivity}: Proportion of accurate \textcolor{NavyBlue}{<description>} that include eye contact information.
    \item \textbf{Eye-Contact Accuracy}: Proportion of accurate \textcolor{NavyBlue}{<description>} that include accurate eye contact information.
\end{itemize}

We reviewed the responses against the original videos. For each response,  we verified the inclusion of time information, descriptions' accuracy, and eye-contact events' presence and correctness. Annotations were assigned based on these criteria, ensuring a reliable benchmark for comparing the models' performance across the four evaluation metrics.

\subsubsection{Evaluation Method of MLLMs Temporal Grounding in Joint Attention Segments}

To evaluate the temporal understanding capabilities of MLLMs in detecting strong and poor joint attention segments, we used mean Intersection over Union (mIoU) and Recall at IoU thresholds (R@m). These metrics assess the alignment quality between predicted and ground truth time segments. IoU measures the overlap between the predicted and ground truth time segments, defined as the ratio of the intersection to the union of these two time intervals. A higher IoU indicates better alignment, and mIoU represents the average IoU across all segments. R@m evaluates the proportion of ground truth segments with at least one predicted segment with sufficient overlap based on predefined IoU thresholds.

The evaluation process used the intersection of annotations from the two SLPs as the ground truth. Each model outputted its predicted timestamps for strong and poor joint attention segments. Using these predictions, we computed the metrics as follows:
\begin{itemize}
    \item For \textbf{mIoU}, we calculated the average overlap between the predicted and ground truth timestamps across all segments, ensuring that both temporal precision and alignment quality were considered.
    \item For \textbf{R@m}, we assessed how many ground truth segments had at least one predicted segment with sufficient overlap (defined by IoU thresholds of 0.3, 0.5, and 0.7).
\end{itemize}

\subsection{Performance Analysis and Findings}
We conducted experiments on three models. For GPT-4o and Gemini-1.5-Flash, we utilized their respective APIs. The model was deployed and executed locally on a Ubuntu 22.04 LTS server equipped with two NVIDIA RTX 3090 GPUs for Video-ChatGPT. 

\subsubsection{Evaluation Result of MLLMs response Quality and Content Objectivity}

Table~\ref{tab:model-evaluation} presents the evaluation of three models—GPT-4o, Gemini-1.5-Flash, and Video-ChatGPT—across four metrics.
For \textit{Time Sensitivity}, both GPT-4o and Gemini-1.5-Flash achieved perfect scores (\textit{100\%}), indicating their consistent inclusion of time information in the responses. 
In contrast, Video-ChatGPT performed poorly, with only a small proportion of responses with time information (\textit{36.36\%}).
GPT-4o demonstrated the strongest performance (\textit{90.17\%}) in \textit{Description Accuracy}, accurately describing tasks and scenes within the specified time ranges. 
Gemini-1.5-Flash and Video-ChatGPT performed worse, scoring (\textit{50.53\%}) and (\textit{35.71\%}), respectively. 
Gemini-1.5-Flash frequently misinterpreted interactions, often assuming that the child and parent had eye contact. At the same time, Video-ChatGPT exhibited significant scene misrecognition, such as mistaking a boy for both a boy and a girl.
For \textit{Eye-Contact Sensitivity}, GPT-4o showed limited inclusion of eye-contact information (\textit{23.64\%}), while Gemini-1.5-Flash demonstrated a much higher sensitivity (\textit{92.12\%}), actively identifying eye-contact cues. Video-ChatGPT failed to provide any eye-contact information in its responses.
For \textit{Eye-Contact Accuracy}, GPT-4o achieves the best performance (\textit{62.82\%}), avoiding overestimations of eye contact. Gemini-1.5-Flash has a lower accuracy (\textit{43.92\%}) due to frequent overestimation, assuming eye contact occurs more often than it does. Video-ChatGPT does not output eye-contact information, making it unsuitable for tasks requiring such data.

\begin{table*}[htb!]
\centering
\caption{Evaluation of MLLMs Output Quality and Content Objectivity Across Four Metrics: Time Sensitivity (proportion of responses including time information), Description Accuracy (accuracy of descriptions within specified time ranges), Eye-Contact Sensitivity (proportion of responses including eye contact information), and Eye-Contact Accuracy (accuracy of identified eye-contact information).}
\label{tab:model-evaluation}
\begin{tabular}{p{0.15\textwidth} p{0.18\textwidth} p{0.18\textwidth} p{0.18\textwidth} p{0.18\textwidth}}     
    \toprule
    \textbf{Model} & \textbf{Time Sens.} & \textbf{Description Acc.} & \textbf{Eye-Contact Sens.} & \textbf{Eye-Contact Acc.} \\
    \midrule
    GPT-4o & 100\% & 90.17\% & 23.64\% & 62.82\% \\
    Gemini-1.5-Flash & 100\% & 50.53\% & 92.12\% & 43.92\% \\
    Video-ChatGPT & 36.36\% & 35.71\% & 0\% & 0\% \\
    \bottomrule
\end{tabular}
\end{table*}

While GPT-4o and Gemini-1.5-Flash perform well in Time Sensitivity and Description Accuracy, all models show weaknesses in Eye-Contact Sensitivity and Accuracy. Since Video-ChatGPT performed poorly in time sensitivity, only GPT-4o and Gemini-1.5-Flash were considered for the temporal grounding evaluation in the following section.

\begin{table*}[htb!]
\centering
\caption{Comparison of Temporal Grounding Metrics between Gemini-1.5-Flash and GPT-4o for Strong, Poor, and Overall Joint Attention Segments. The metrics include Recall at IoU thresholds (R@0.3, R@0.5, and R@0.7) and mean Intersection over Union (mIoU). R@0.3, R@0.5, and R@0.7 represent the proportion of ground truth segments with at least one predicted segment with sufficient overlap at IoU thresholds of 0.3, 0.5, and 0.7, respectively. mIoU measures the average alignment quality between predicted and ground truth segments. "Strong" refers to the model's performance on segments with strong joint attention, "Poor" refers to segments with poor joint attention, and "Overall" reflects performance across all segments.}
\begin{tabular}{p{0.25\textwidth} p{0.15\textwidth} p{0.15\textwidth} p{0.15\textwidth} p{0.15\textwidth}}     \toprule
\textbf{Model}               & \textbf{R@0.3 (\%)} & \textbf{R@0.5 (\%)} & \textbf{R@0.7 (\%)} & \textbf{mIoU (\%)} \\ \midrule
GPT-4o (Strong)              & 2.01                 & 0.57                 & 0.57                 & 2.19               \\
Gemini-1.5-Flash (Strong)    & 1.72                 & 0.69                 & 0.52                 & 1.39               \\
GPT-4o (Poor)                & 0.00                 & 0.00                 & 0.00                 & 3.30               \\
Gemini-1.5-Flash (Poor)      & 0.00                 & 0.00                 & 0.00                 & 2.28               \\
\midrule
GPT-4o (Overall)             & 1.77                 & 0.51                 & 0.51                 & 2.29               \\
Gemini-1.5-Flash (Overall)   & 1.17                 & 0.47                 & 0.35                 & 1.78               \\
\bottomrule
\end{tabular}
\label{tab:metrics_comparison}
\end{table*}


\subsubsection{Evaluation Result of MLLMs Temporal Grounding in Joint Attention Segments}

Table \ref{tab:metrics_comparison} highlights that GPT-4o outperforms Gemini-1.5-Flash on strong joint attention segments, achieving a higher mIoU (2.19\% vs. 1.39\%) and a slightly better R@0.3 (2.01\% vs. 1.72\%). This suggests GPT-4o aligns more effectively with ground truth for strong segments, likely due to its broader contextual understanding and reduced reliance on explicit eye-contact information.

% \begin{table*}[htb!]
% \centering
% \caption{Comparison of Temporal Grounding Metrics between Gemini-1.5-Flash and GPT-4o for Strong, Poor, and Overall Joint Attention Segments. The metrics include Recall at IoU thresholds (R@0.3, R@0.5, and R@0.7) and mean Intersection over Union (mIoU). R@0.3, R@0.5, and R@0.7 represent the proportion of ground truth segments that have at least one predicted segment with sufficient overlap at IoU thresholds of 0.3, 0.5, and 0.7, respectively. mIoU measures the average alignment quality between predicted and ground truth segments. "Strong" refers to the model's performance on segments with strong joint attention, "Poor" refers to segments with poor joint attention, and "Overall" reflects performance across all segments.}
% \begin{tabular}{p{0.15\textwidth} p{0.18\textwidth} p{0.18\textwidth} p{0.18\textwidth} p{0.18\textwidth}}     \toprule
% \textbf{Model}               & \textbf{R@0.3 (\%)} & \textbf{R@0.5 (\%)} & \textbf{R@0.7 (\%)} & \textbf{mIoU (\%)} \\ \midrule
% GPT-4o (Strong)              & 2.01                 & 0.57                 & 0.57                 & 2.19               \\
% Gemini-1.5-Flash (Strong)    & 1.72                 & 0.69                 & 0.52                 & 1.39               \\
% GPT-4o (Poor)                & 0.00                 & 0.00                 & 0.00                 & 3.30               \\
% Gemini-1.5-Flash (Poor)      & 0.00                 & 0.00                 & 0.00                 & 2.28               \\
% \midrule
% GPT-4o (Overall)             & 1.77                 & 0.51                 & 0.51                 & 2.29               \\
% Gemini-1.5-Flash (Overall)   & 1.17                 & 0.47                 & 0.35                 & 1.78               \\
% \bottomrule
% \end{tabular}
% \label{tab:metrics_comparison}
% \end{table*}

For poor joint attention segments, GPT-4o also achieves a higher mIoU (3.30\% vs. 2.28\%), despite both models reporting zero Recall across all IoU thresholds (R@0.3, R@0.5, R@0.7). This discrepancy occurs because mIoU captures partial overlaps between predicted and ground truth regions, even if the overlaps are insufficient to meet recall thresholds. In the poor category, models may predict larger or misaligned regions, contributing to mIoU but failing to exceed the recall thresholds.

Overall, GPT-4o demonstrates superior performance, with a higher mIoU (2.29\% vs. 1.78\%) and slightly better recall values across all categories. However, this low value indicates limited alignment between predicted and ground truth regions. The result reflects the challenges of temporal grounding in joint attention tasks, where precise boundary prediction is difficult.

Despite these results, the findings highlight limitations in current MLLMs for direct understanding of joint attention. As noted earlier, deficiencies in processing eye-contact information significantly reduce accuracy, underscoring the need for improved multimodal integration.
This paper presents a planning approach for effective and efficient joint motion generation for manipulators to cover a surface, aiming to minimize specific joint space costs.

\textit{Limitations} -- Our work has several limitations that suggest potential directions for future research. First, our method uses a heuristic to accelerate the traditional Joint-GTSP approach. While we provide empirical evidence of its efficiency in producing high-quality solutions, we cannot guarantee consistent performance in all scenarios.
Second, our bi-level hierarchical method reduces the size of GTSP. Future research could extend it to multiple levels to further improve performance, though this may produce misleading guide paths.
Third, we observe that both Joint-GTSP and H-Joint-GTSP tend to generate paths with frequent turns, a pattern also observed in the motions of prior work \cite{kaljaca2020coverage, zhang2024jpmdp}.  Future work should explore strategies to balance joint movements with other objectives such as motion smoothness.

\footnotetext{Visualization tool: \url{https://github.com/uwgraphics/MotionComparator}}
\textit{Implications} -- The hierarchical approach presented in this work enables effective and efficient coverage path planning for robot manipulators. 
This approach is beneficial to applications that require dexterous surface coverage, such as sanding, polishing, wiping, and sensor scanning. 


\paragraph{Summary}
Our findings provide significant insights into the influence of correctness, explanations, and refinement on evaluation accuracy and user trust in AI-based planners. 
In particular, the findings are three-fold: 
(1) The \textbf{correctness} of the generated plans is the most significant factor that impacts the evaluation accuracy and user trust in the planners. As the PDDL solver is more capable of generating correct plans, it achieves the highest evaluation accuracy and trust. 
(2) The \textbf{explanation} component of the LLM planner improves evaluation accuracy, as LLM+Expl achieves higher accuracy than LLM alone. Despite this improvement, LLM+Expl minimally impacts user trust. However, alternative explanation methods may influence user trust differently from the manually generated explanations used in our approach.
% On the other hand, explanations may help refine the trust of the planner to a more appropriate level by indicating planner shortcomings.
(3) The \textbf{refinement} procedure in the LLM planner does not lead to a significant improvement in evaluation accuracy; however, it exhibits a positive influence on user trust that may indicate an overtrust in some situations.
% This finding is aligned with prior works showing that iterative refinements based on user feedback would increase user trust~\cite{kunkel2019let, sebo2019don}.
Finally, the propensity-to-trust analysis identifies correctness as the primary determinant of user trust, whereas explanations provided limited improvement in scenarios where the planner's accuracy is diminished.

% In conclusion, our results indicate that the planner's correctness is the dominant factor for both evaluation accuracy and user trust. Therefore, selecting high-quality training data and optimizing the training procedure of AI-based planners to improve planning correctness is the top priority. Once the AI planner achieves a similar correctness level to traditional graph-search planners, strengthening its capability to explain and refine plans will further improve user trust compared to traditional planners.

\paragraph{Future Research} Future steps in this research include expanding user studies with larger sample sizes to improve generalizability and including additional planning problems per session for a more comprehensive evaluation. Next, we will explore alternative methods for generating plan explanations beyond manual creation to identify approaches that more effectively enhance user trust. 
Additionally, we will examine user trust by employing multiple LLM-based planners with varying levels of planning accuracy to better understand the interplay between planning correctness and user trust. 
Furthermore, we aim to enable real-time user-planner interaction, allowing users to provide feedback and refine plans collaboratively, thereby fostering a more dynamic and user-centric planning process.


\begin{acks}
This research is supported by the National Research Foundation, Singapore, under its AI Singapore Programme (AISG Award No. AISG2-TC-2022-007).
\end{acks}



%%
%% The next two lines define the bibliography style to be used, and
%% the bibliography file.
\bibliographystyle{ACM-Reference-Format}
\bibliography{main}


%%
%% If your work has an appendix, this is the place to put it.
\appendix

\end{document}
\end{document}
\endinput
%%
%% End of file `sample-sigconf-authordraft.tex'.
