%%
%% This is file `sample-sigconf-authordraft.tex',
%% generated with the docstrip utility.
%%
%% The original source files were:
%%
%% samples.dtx  (with options: `all,proceedings,bibtex,authordraft')
%% 
%% IMPORTANT NOTICE:
%% 
%% For the copyright see the source file.
%% 
%% Any modified versions of this file must be renamed
%% with new filenames distinct from sample-sigconf-authordraft.tex.
%% 
%% For distribution of the original source see the terms
%% for copying and modification in the file samples.dtx.
%% 
%% This generated file may be distributed as long as the
%% original source files, as listed above, are part of the
%% same distribution. (The sources need not necessarily be
%% in the same archive or directory.)
%%
%%
%% Commands for TeXCount
%TC:macro \cite [option:text,text]
%TC:macro \citep [option:text,text]
%TC:macro \citet [option:text,text]
%TC:envir table 0 1
%TC:envir table* 0 1
%TC:envir tabular [ignore] word
%TC:envir displaymath 0 word
%TC:envir math 0 word
%TC:envir comment 0 0
%%
%%
%% The first command in your LaTeX source must be the \documentclass
%% command.
%%
%% For submission and review of your manuscript please change the
%% command to \documentclass[manuscript, screen, review]{acmart}.
%%
%% When submitting camera ready or to TAPS, please change the command
%% to \documentclass[sigconf]{acmart} or whichever template is required
%% for your publication.
%%
%%
\documentclass[sigconf]{acmart} % for submission

\usepackage{makecell}
\usepackage{enumitem}
\usepackage{multirow}
\usepackage{subfigure}
\usepackage{graphicx}
\usepackage{subfigure}
\usepackage{caption}
\usepackage{subcaption}
\usepackage{tabularx}
\usepackage{xcolor}



% Redaction and commenting macros
\newif\ifredact
\newif\ifcomment

% Set switches
\redactfalse  % Change to \redacttrue to test redaction
\commenttrue  % Change to \commentfalse to test no comments

\newcommand{\redact}[2]{\ifredact #1\else #2\fi}
\newcommand{\dquote}[1]{\textit{``#1''}}

\ifcomment
  \newcommand{\missing}[1]{\textcolor{red}{~#1}}
  \newcommand{\ken}[1]{\textcolor{magenta}{~Kenny: #1}}
  \newcommand{\wei}[1]{\textcolor{orange}{~Weiyan: #1}}
  \newcommand{\discuss}[1]{\textcolor{purple}{Used in discussion: #1}}
\else
  \newcommand{\missing}[1]{}
  \newcommand{\ken}[1]{}
  \newcommand{\wei}[1]{}
  \newcommand{\discuss}[1]{}
\fi


%%
%% \BibTeX command to typeset BibTeX logo in the docs
\AtBeginDocument{%
  \providecommand\BibTeX{{%
    Bib\TeX}}}

%% Rights management information.  This information is sent to you
%% when you complete the rights form.  These commands have SAMPLE
%% values in them; it is your responsibility as an author to replace
%% the commands and values with those provided to you when you
%% complete the rights form.
% \copyrightyear{2025}
% \acmYear{2025}
% \setcopyright{rightsretained}
% \acmConference[CHI EA '25]{Extended Abstracts of the CHI Conference on Human Factors in Computing Systems}{April 26-May 1, 2025}{Yokohama, Japan}
% \acmBooktitle{Extended Abstracts of the CHI Conference on Human Factors in Computing Systems (CHI EA '25), April 26-May 1, 2025, Yokohama, Japan}\acmDOI{10.1145/3706599.3720215}
% \acmISBN{979-8-4007-1395-8/2025/04}
%%  Uncomment \acmBooktitle if the title of the proceedings is different
%%  from ``Proceedings of ...''!
%%
%%\acmBooktitle{Woodstock '18: ACM Symposium on Neural Gaze Detection,
%%  June 03--05, 2018, Woodstock, NY}


%%
%% Submission ID.
%% Use this when submitting an article to a sponsored event. You'll
%% receive a unique submission ID from the organizers
%% of the event, and this ID should be used as the parameter to this command.
%%\acmSubmissionID{123-A56-BU3}

%%
%% For managing citations, it is recommended to use bibliography
%% files in BibTeX format.
%%
%% You can then either use BibTeX with the ACM-Reference-Format style,
%% or BibLaTeX with the acmnumeric or acmauthoryear sytles, that include
%% support for advanced citation of software artefact from the
%% biblatex-software package, also separately available on CTAN.
%%
%% Look at the sample-*-biblatex.tex files for templates showcasing
%% the biblatex styles.
%%

%%
%% The majority of ACM publications use numbered citations and
%% references.  The command \citestyle{authoryear} switches to the
%% "author year" style.
%%
%% If you are preparing content for an event
%% sponsored by ACM SIGGRAPH, you must use the "author year" style of
%% citations and references.
%% Uncommenting
%% the next command will enable that style.
%%\citestyle{acmauthoryear}


%%
%% end of the preamble, start of the body of the document source.
\begin{document}
\settopmatter{printacmref=false} % Removes citation information below abstract
\renewcommand\footnotetextcopyrightpermission[1]{} % Removes copyright footnote
\pagestyle{plain} % Removes running headers

\definecolor{Plum}{rgb}{0.914, 0.008, 0.980}
\definecolor{NavyBlue}{rgb}{0.137, 0.596, 0.969}


%%
%% The "title" command has an optional parameter,
%% allowing the author to define a "short title" to be used in page headers.
\title{Towards Multimodal Large-Language Models for Parent-Child Interaction: A Focus on Joint Attention}

%%
%% The "author" command and its associated commands are used to define
%% the authors and their affiliations.
%% Of note is the shared affiliation of the first two authors, and the
%% "authornote" and "authornotemark" commands
%% used to denote shared contribution to the research.
\author{Weiyan Shi}
\email{weiyanshi6@gmail.com}
\orcid{0009-0001-6035-9678}
\affiliation{
  \institution{Singapore University of Technology and Design}
  \country{Singapore}
}

\author{Viet Hai Le}
\email{lehaivin03@gmail.com}
\orcid{0009-0007-3040-6625}
\affiliation{
  \institution{Singapore University of Technology and Design}
  \country{Singapore}
}

\author{Kenny Tsu Wei Choo}
\email{kennytwchoo@gmail.com}
\orcid{0000-0003-3845-9143}
\affiliation{
  \institution{Singapore University of Technology and Design}
  \country{Singapore}
}
\authornote{Corresponding author. This is a preprint of the paper accepted at CHI 2025 Late Breaking Work. The final version will be available in the ACM Digital Library.}

%%
%% By default, the full list of authors will be used in the page
%% headers. Often, this list is too long, and will overlap
%% other information printed in the page headers. This command allows
%% the author to define a more concise list
%% of authors' names for this purpose.
\renewcommand{\shortauthors}{Shi et al.}

%%
%% The abstract is a short summary of the work to be presented in the
%% article.
\begin{abstract}

% Recent works to jointly reconstruct 3D human and object from a single RGB image, are mostly model-based, that fail to capture the fine details of the clothed human body and object surface. In this paper, we introduce ReCHOR, a novel, model-free, first-method to produce realistic clothed human-object reconstructions from a monocular view. This is extremely challenging due to human-object occlusions, diverse interactions and depth ambiguity, as it needs to infer both 3D spatial awareness and high resolution details. Our core idea is based on estimating neural implicit representations for human and object respectively by an attention-based neural implicit model that attends to pixel-aligned features from both the global human-object image for spatial awareness and  the local separate view of human and object images for high quality details. Additionally, the network is conditioned on semantic features from an initial estimated human-object pose prior and a generative diffusion model that inpaints occluded regions, thus enabling the retrieval of details from them.
% We also propose a synthetic dataset with rendered scenes of diverse, inter-occluded 3D human and object scans, to train our network. We evaluate our method on the synthetic and real world BEHAVE dataset. Our experiments show that our method outperforms the SOTA in achieving realistic clothed human-object reconstructions.
Recent approaches to jointly reconstruct 3D humans and objects from a single RGB image represent 3D shapes with template-based or coarse models, which fail to capture details of loose clothing on human bodies. In this paper, we introduce a novel implicit approach for jointly reconstructing realistic 3D clothed humans and objects from a monocular view. For the first time, we model both the human and the object with an implicit representation, allowing to capture more realistic details such as clothing. This task is extremely challenging due to human-object occlusions and the lack of 3D information in 2D images, often leading to poor detail reconstruction and depth ambiguity. To address these problems, we propose a novel attention-based neural implicit model that leverages image pixel alignment from both the input human-object image for a global understanding of the human-object scene and from local separate views of the human and object images to improve realism with, for example, clothing details. Additionally, the network is conditioned on semantic features derived from an estimated human-object pose prior, which provides 3D spatial information about the shared space of humans and objects. To handle human occlusion caused by objects, we use a generative diffusion model that inpaints the occluded regions, recovering otherwise lost details. For training and evaluation, we introduce a synthetic dataset featuring rendered scenes of inter-occluded 3D human scans and diverse objects. Extensive evaluation on both synthetic and real-world datasets demonstrates the superior quality of the proposed human-object reconstructions over competitive methods.
\end{abstract}

%%
%% The code below is generated by the tool at http://dl.acm.org/ccs.cfm.
%% Please copy and paste the code instead of the example below.
%%
\begin{CCSXML}
<ccs2012>
   <concept>
       <concept_id>10003120.10003130.10011762</concept_id>
       <concept_desc>Human-centered computing~Empirical studies in collaborative and social computing</concept_desc>
       <concept_significance>500</concept_significance>
       </concept>
   <concept>
       <concept_id>10010147.10010178.10010179</concept_id>
       <concept_desc>Computing methodologies~Natural language processing</concept_desc>
       <concept_significance>500</concept_significance>
       </concept>
   <concept>
       <concept_id>10010147.10010178.10010224</concept_id>
       <concept_desc>Computing methodologies~Computer vision</concept_desc>
       <concept_significance>500</concept_significance>
       </concept>
 </ccs2012>
\end{CCSXML}

\ccsdesc[500]{Human-centered computing~Empirical studies in collaborative and social computing}
\ccsdesc[500]{Computing methodologies~Natural language processing}
\ccsdesc[500]{Computing methodologies~Computer vision}
%%
%% Keywords. The author(s) should pick words that accurately describe
%% the work being presented. Separate the keywords with commas.
\keywords{Parent-Child Joint attention, Multimodal Large Language Models, Eye Contact, Temporal Understanding}
%% A "teaser" image appears between the author and affiliation
%% information and the body of the document, and typically spans the
%% page.
% \begin{teaserfigure}
%   \includegraphics[width=\textwidth]{sampleteaser}
%   \caption{Seattle Mariners at Spring Training, 2010.}
%   \Description{Enjoying the baseball game from the third-base
%   seats. Ichiro Suzuki preparing to bat.}
%   \label{fig:teaser}
% \end{teaserfigure}

% \received{20 February 2007}
% \received[revised]{12 March 2009}
% \received[accepted]{5 June 2009}

%%
%% This command processes the author and affiliation and title
%% information and builds the first part of the formatted document.
\maketitle

\section{Introduction}\label{sec:intro}

In computational finance, Monte Carlo simulations are used extensively to estimate the expected value of financial payoffs based on the solution of stochastic differential equations (SDEs) which model the evolution of stock prices, interest rates, exchange rates and other quantities \cite{glasserman04}.  Monte Carlo methods are very general and flexible, but for high accuracy it requires generating a large number of costly SDE path approximations, which has motivated research into a number of variance reduction or, equivalently, cost reduction techniques. One such method is
Multilevel Monte Carlo (MLMC), which was proposed in \cite{GILES2008} and was adapted for various applications that are summarised in \cite{Giles_overview17} and successfully combined with other methods such as quasi-Monte Carlo methods. The main idea of MLMC is to approximate the payoff using different time stepping resolutions when numerically solving the underlying SDE and to generate an optimal number of samples on each level, such that the overall computational cost is minimised subject to the desired bound on the variance. %, such that the total computational cost is minimised. 
The computational savings come from the fact that most samples are computed on the coarser levels and hence are less expensive while only a few samples from the finest levels are required \cite{GILES2008}.


Among the directions in which the computational cost 
of MLMC methods could further be reduced, an important avenue is the use of lower precision calculations, especially for the first Monte Carlo levels where the targeted accuracy is relatively low. 
 An overview of the research on mixed precision for the standard Monte Carlo (MC) framework is provided in \cite{ChowMixedPrecisionStandardMC} but only a few references study the potential of low precision computation in the MLMC framework \cite{Rounding_error_oliver}. To the best of our knowledge, the only MLMC framework with customised precision in the literature is \cite{brugger2014mixed}, but they use a uniform precision for all operations on each Monte Carlo level instead of optimising 
 the precision of each intermediary variable to reduce as much as possible the cost of path generation.
 
An important motivation for an MLMC framework with variable precision would be performing the low precision computations on reconfigurable hardware devices such as Field Programmable Gate Arrays (FPGAs). FPGAs contain customizable logic blocks and connectors that make it easy to adapt the digital circuit architecture for a specific application, leading to a highly parallel and optimised implementation. Therefore they are successfully exploited in applications that require high speed and have high computational workload, such as signal processing \cite{woods2008fpga}, and real time applications like high frequency trading \cite{HFT1,HFT2}. That is why a number of previous works in hardware architecture design implemented the MLMC algorithm to price financial options using FPGAs as accelerators, which resulted in improved speed and power efficiency compared to full CPU architectures \cite{Schryver2013AMM}. The paper \cite{lindsey2016domain} also proposed 
a Domain Specific Language to automate the configuration of FPGAs for this specific application. However, only \cite{brugger2014mixed} proposed a heuristic to reduce the precision in calculations.

In addition, all aforementioned works considered that the random number generation (RNG) is performed in single or double precision. Yet in most cases an important portion of the workload in the overall MLMC simulation comes from the RNG and in \cite{brugger2014mixed} this limited the total computational savings.
To reduce the cost of MLMC simulations in particular those based on the Geometric Brownian Motion (GBM), \cite{approximateICDF_Oliver, NestedOliver} have proposed to use approximate random numbers that are generated by applying an approximation of the inverse CDF to uniform random numbers. In \cite{NestedOliver}, the authors proposed a way to integrate these lower precision random variables into a \textit{nested} MLMC framework and completed a numerical analysis to bound the resulting error at each MC level by a product of the time step and the error in the random number approximation. The same authors show in \cite{approximateICDF_Oliver} that using approximate random variables reduces the cost of path generation by a factor 7.


In this paper we propose a nested MLMC framework that combines the use of approximate random normal variables and lower precision calculations to reduce the computational cost of MLMC even further than \cite{brugger2014mixed,NestedOliver}. We illustrate the efficiency of our framework in Matlab, after making several assumptions on the cost of operations and size of the errors that we carefully justify. We focus on the case of GBM and use the approximate RNG methods presented in \cite{approximateICDF_Oliver} as well as a new slightly modified method that combines CDF inversion and the central limit theorem. To choose the precision of the variables in the low precision path generation, we introduce a novel method to optimise the bit-widths. This optimisation is performed before the main path generation loop is executed and is based on a linear model of the payoff error  
due to rounding when computing in low precision. The error model relies on algorithmic differentiation in a similar manner to \cite{unifying-bwoptim,bitwidth-AD,ADAPT}. The bit-width optimisation procedure can be performed off-line, so this stage can be excluded from the on-line time complexity of our framework. The user specified desired accuracy is then enforced by calculating on-line the number of samples that need to be generated.

In terms of hardware design, we suggest implementing the low precision path generation on FPGAs and the full-precision ones on a CPU or GPU. 
The FPGA offers enough flexibility to define a separate bit-width for every variable in the low precision path generation, and can be reconfigured periodically to update the bit-widths when the market parameters have changed considerably. 


The paper is organized as follows : \Cref{sec:MLMC} introduces MLMC and nested MLMC to make clear the estimator that is implemented in our framework. Then in \Cref{sec:RNG} we detail the methods that could be used to obtain approximate random normally distributed numbers very cheaply for the low precision path generation. In \Cref{sec:error_model} and \Cref{sec:costModel} we propose an error model and a cost model (resp.) that we then use to formulate the optimisation problem that is solved to obtain the optimal bit-widths of fixed point variables in \Cref{sec:optimisation}. Finally we summarise our results and future directions in \Cref{sec:conclusion}.



\section{Related Work}
\label{sec:related_work}

The original investigation \cite{gibson1979ecological} on the relationship between visual perception and human action defines \emph{affordance} as the opportunities for interaction with the surrounding environment. Behavioral studies on regular and cognitively impaired persons have shown evidence that perception results in both visual and motor signals in the human brain. An extended study \cite{anderson2002attentional} shows that visual attention to the spatial characteristics of the perceived objects initiates automatic motor signals for different actions. In computer vision, human affordance learning involves novel pose prediction such that the estimated pose represents a valid human action within the scene context. The task is fundamental to many problems requiring robust semantic reasoning about the environment, such as human motion synthesis \cite{wang2021scene} and scene-aware human pose generation \cite{wang2017binge, roy2016multi, zhang2022inpaint, yao2023scene}.

Earlier methods of affordance learning have explored knowledge mining \cite{zhu2014reasoning} and multimodal feature cues \cite{roy2016multi} to address the problem. In \cite{zhu2014reasoning}, the authors use a Markov Logic Network for constructing a knowledge base by extracting several object attributes from different image and metadata sources, which can perform various downstream visual inference tasks without any additional classifier, including zero-shot affordance prediction. In \cite{roy2016multi}, the authors use depth map, surface normals, and segmentation map as multimodal cues to train a multi-scale convolutional neural network (CNN) for scene-level semantic label assignment associated with specific human actions. In \cite{do2018affordancenet}, the authors design a multi-branch end-to-end CNN with two separate pathways for object detection and affordance label assignment to achieve high real-time inference throughput. Researchers \cite{chuang2018learning} have also explored socially imposed constraints for affordance learning. In \cite{chuang2018learning}, the authors propose a graph neural network (GNN) to propagate contextual scene information from egocentric views for action-object affordance reasoning.

Probabilistic modeling of scene-aware human motion generation also involves semantic reasoning of human interaction with the environment. Initial works on human motion synthesis have taken different architectural approaches, such as sequence-to-sequence models \cite{barsoum2018hp}, generative adversarial networks (GAN) \cite{barsoum2018hp, cai2018deep, yang2018pose}, graph convolutional networks (GCN) \cite{yan2019convolutional}, and variational autoencoders (VAE) \cite{guo2020action2motion}. However, these methods have mostly ignored the role of environmental semantics. Due to potential uncertainty in human motion, in a recent approach \cite{wang2021scene}, the authors address such motion synthesis with a GAN conditioned on scene attributes and motion trajectory to predict probable body pose dynamics.

One key challenge of human affordance generation in 2D scenes is the lack of large-scale datasets with rich pose annotations. In \cite{wang2017binge}, the authors compile the only public dataset of annotated human body poses in complex 2D indoor scenes by extracting frames from sitcom videos. Aiming to generate a contextually valid human affordance at a user-defined location, the authors propose sampling the scale and deformation parameters for an existing human pose template using a VAE conditioned on the localized image patches as scene context. In \cite{zhang2022inpaint}, the authors introduce a two-stage GAN architecture for achieving a similar goal by estimating the affine bounding box parameters to localize a probable human in the scene and then generating a potential body pose at that location. The method uses the input scene, corresponding depth, and segmentation maps as semantic guidance. In \cite{yao2023scene}, the authors propose a transformer-based approach with knowledge distillation for generating human affordances in 2D indoor scenes.


\begin{table*}[htb!]
\centering
\caption{Comparison of annotation between SLPs: \textbf{SLP} (annotator ID), \textbf{Total} (number of segments), \textbf{Strong (\%)} and \textbf{Poor (\%)} (counts and percentages of strong/poor joint attention), \textbf{Avg. (s)} (average segment duration), \textbf{Avg. Strong (s)}, and \textbf{Avg. Poor (s)} (average durations of strong and poor segments). The \textbf{Intersection} row shows metrics for mutually agreed segments.}
\begin{tabular}{p{0.12\textwidth} p{0.12\textwidth} p{0.12\textwidth} p{0.12\textwidth} p{0.12\textwidth} p{0.12\textwidth} p{0.12\textwidth}} 
\toprule
\textbf{SLP} & \textbf{Total} & \textbf{Strong (\%)} & \textbf{Poor (\%)} & \textbf{Avg. (s)} & \textbf{Avg. Strong (s)} & \textbf{Avg. Poor (s)} \\ 
\midrule
S1           & 96             & 89 (92.71)           & 7 (7.29)           & 5.61               & 5.24                      & 10.30                   \\
S2           & 72             & 56 (77.78)           & 16 (22.22)         & 14.51              & 14.85                     & 13.33                   \\
\midrule
Intersection & 62             & 58 (93.55)           & 4 (6.45)           & 3.88               & 3.65                      & 10.73                   \\
\bottomrule
\end{tabular}
\label{tab:segment_metrics}
\end{table*}
\section{Video Data Collection and Preparation}
\subsection{Video Selection Method}
We collected videos from YouTube using targeted keywords \dquote{parent-child interaction} to create a dataset for analysing parent-child interactions, carefully selecting videos that met quality and relevance criteria. Each video had to feature one child and one adult as the primary subjects. While most videos required active interactions between the adult and the child, we also included cases where the adult's role was limited to accompanying a very young child for support without directly engaging in the interaction. Videos involving multiple children or adults actively participating were excluded to maintain the focus on dyadic dynamics.

We applied stringent selection criteria to ensure the videos were suitably focused towards analysing multimodal signals. The scenes needed to be static with minimal camera movement, and the videos had to provide clear views of both the child's and adult's faces, enabling precise analysis of gaze direction and expressions. High-quality audio and visual clarity were also essential for accurate verbal and non-verbal communication observations.

Additionally, the dataset was curated to include children across a broad age range to capture a variety of developmental stages. We prioritised videos that showcased diverse and meaningful interactions, such as language learning activities, skill-building tasks, or natural daily exchanges. All textual elements, including subtitles and transitions, were removed to preserve the raw multimodal content.

\subsection{Dataset Composition and Analysis}
We curated a final dataset comprising 26 parent-child interaction videos that satisfied our quality and relevance criteria. These videos are of dialogue and interaction in fixed settings, covering age ranges between 0-8 years (age information obtained from the descriptions provided on the video websites). The majority of videos focus on children aged 4–6 years (n=16), with fewer videos in the 2–4 (n=5) and 0–2 (n=4) age groups. Only one video includes children aged 6–8, concentrating on younger age groups where parent-child interactions are more actively studied.

The duration of the videos ranges from brief clips of around 0.5 to 12 minutes. Most videos are relatively short: 13 videos are less than 1 minute, and 11 fall between 1 and 5 minutes. Only two videos extend beyond 5 minutes, one categorised as 5–10 minutes and the other over 10 minutes. This distribution highlights the concise nature of the interactions and tasks typically observed in parent-child settings.

The video content in the dataset can be categorised into three main categories. The first category is behaviour guidance and skill modelling (n=10), where parents demonstrate specific techniques (e.g., PRIDE skills, \dquote{Big Ignore} techniques) to guide children's behaviour and improve interaction quality. The second category focuses on language and cognitive development (n=10), showcasing how parent-child interactions foster language learning and cognitive skills through tasks or experiments, such as discussing specific topics or engaging in Piaget's cognitive experiments. The third category involves daily life skills and interaction (n=6), presenting natural exchanges in structured settings, such as reviewing reward charts or ending special playtime, emphasising practical life skills and building strong parent-child relationships.

% \begin{figure}[htbp]
%     \begin{minipage}[t]{0.32\linewidth}
%         \centering
%         \includegraphics[width=\textwidth]{figures/age_distribution.png}
%         \caption{Age Distribution}
%         \label{fig:age_distribution}
%     \end{minipage}%
%     \begin{minipage}[t]{0.32\linewidth}
%         \centering
%         \includegraphics[width=\textwidth]{figures/duration_distribution.png}
%         \caption{Duration Distribution}
%         \label{fig:duration_distribution}
%     \end{minipage}
%     \begin{minipage}[t]{0.32\linewidth}
%         \centering
%         \includegraphics[width=\textwidth]{figures/classification_distribution.png}
%         \caption{Classification Distribution}
%         \label{fig:classification_distribution}
%     \end{minipage}
% \end{figure}

\section{Speech-Language Pathologists Annotation Study}
\subsection{Annotation Process}
To evaluate the performance of MLLMs in identifying strong or poor joint attention segments, we recruited two SLPs (both female) with extensive experience to provide expert annotations. They each have 7–9 years of experience and training in programs such as DIR Floortime~\cite{dir_floortime}, It Takes Two to Talk~\cite{pepper2004talk}, and More Than Words~\cite{sussman1999words}. While a third SLP was recruited, the session was incomplete and the results were not included in this paper. Our studies complied with ethical regulations approved by our institution's ethics review board.


% The first expert holds a bachelor's degree in Speech Pathology and has 9 years of experience working with children. She is trained in DIR Floortime and the Hanen Programs (It Takes Two to Talk and More Than Words) and is skilled in various communication and sensory intervention techniques.  

% The second expert has a bachelor's degree in Speech Pathology and 7 years of experience in pediatric speech therapy. She is certified in DIR Floortime and the Hanen More Than Words program, focusing on enhancing communication and social skills in children with developmental delays.  

The study was conducted in person over one month. Each session lasted approximately 2 hours and was held in a private meeting room to ensure focus and comfort. The labelling was carried out using our custom-built annotation system. Videos can be played, paused, or navigated frame-by-frame, while the interactive timeline enables users to select segments and label them as \dquote{strong} or \dquote{poor}. The system exports the results in CSV format, including each labelled segment's start and end times, along with corresponding indices. Additionally, a panel allows experts to add notes during the labelling process.

% \begin{table}[htb!]
% \centering
% \caption{Comparison of Annotation Across SLPs: \textbf{SLP} (annotator ID), \textbf{Total} (number of segments), \textbf{Strong (\%)} and \textbf{Poor (\%)} (counts and percentages of strong/poor joint attention), \textbf{Avg. (s)} (average segment duration), \textbf{Avg. Strong (s)}, and \textbf{Avg. Poor (s)} (average durations of strong and poor segments). The \textbf{Intersection} row shows metrics for mutually agreed segments.}
% \begin{tabular*}{\textwidth}{@{\extracolsep{\fill}}lcccccc}
% \toprule
% \textbf{SLP} & \textbf{Total} & \textbf{Strong (\%)} & \textbf{Poor (\%)} & \textbf{Avg. (s)} & \textbf{Avg. Strong (s)} & \textbf{Avg. Poor (s)} \\ \midrule
% S1        & 96          & 89 (92.71)           & 7 (7.29)            & 5.61              & 5.24                    & 10.30                 \\
% S2        & 72          & 56 (77.78)           & 16 (22.22)          & 14.51             & 14.85                   & 13.33                 \\\midrule
% Intersection      & 62            & 58 (93.55)              & 4 (6.45)              & 3.88                             & 3.65 &       10.73    \\       
% \bottomrule
% \end{tabular*}
% \label{tab:segment_metrics}
% \end{table}


\subsection{Annotation Outcomes Analysis}
We conducted full annotation interviews with SLP 1 (\textit{S1}) and SLP 2 (\textit{S2}). Through these interviews, a shared criterion for determining strong joint attention was established: it is characterized by the child maintaining consistent eye contact and \textit{active engagement} with the parent during the interaction. 
% Due to time constraints, Expert 3 was unable to participate in the interviews, and the results are based solely on the annotations provided by \textit{E1} and \textit{E2}.

Table \ref{tab:segment_metrics} summarizes the annotation metrics provided by the two SLPs. \textit{S1} annotated a total of 96 segments. Among these, \textit{92.71\%} were identified as strong segments, with an average segment duration of \textit{5.61 s}. Strong segments were notably shorter, averaging \textit{5.24 s}, while poor segments were longer at \textit{10.30 s}. \textit{S2} annotated 72 segments. Among these, \textit{77.78\%} were classified as strong, and \textit{22.22\%} as poor. The average duration for strong and poor segments was \textit{14.85 s} and \textit{13.33 s}, respectively, reflecting S2's longer segments in annotations.

To ensure fairness and consistency, we calculated the intersection of annotations where SLPs agreed. This resulted in \textit{62} segments, with \textit{93.55\%} identified as strong and \textit{6.45\%} as poor. The average duration for strong and poor segments in the intersection was \textit{3.65 s} and \textit{10.73 s}, respectively, while the overall average duration was \textit{3.88 s}.

\section{MLLM Comparative Study}
\subsection{Experimental Setup and Evaluation Metrics}
% Recent advances in MLLMs have demonstrated significant potential in general-purpose video understanding. However, as noted in~\cite{liu2024bench}, state-of-the-art models designed for video-level understanding face challenges in handling fine-grained tasks, such as temporal video grounding.

To test the performance of MLLMs on our parent-child joint-attention detection task, a fine-grained temporal video grounding challenge, we selected three state-of-the-art MLLMs to evaluate: GPT-4o-2024-08-06\footnote{\url{https://openai.com/index/hello-gpt-4o/}}, Gemini 1.5 Flash\footnote{\url{https://deepmind.google/technologies/gemini/flash/}}, and Video-ChatGPT~\cite{maaz2023video}.

The Gemini 1.5 model directly supports audio and video processing, allowing us to input the entire video seamlessly. In contrast, Video-ChatGPT can only process videos without audio, so we provided the full video without additional preprocessing. GPT-4o, however, does not natively support direct video processing. To address this, we extracted video frames and converted them into a Base64-encoded array of images. 
We also used WhisperX~\cite{bain2023whisperx} to transcribe the videos' audio and manually enriched these transcriptions with notes for greater informativeness. 
For instance, we included annotations such as \dquote{the child is laughing} and descriptions of background sounds, which helped compensate for the unclear speech often observed in younger children. These transcriptions were used as text input, serving as a substitute for audio input for both Video-ChatGPT and GPT-4o.

We designed Instruction Templates (see Table~\ref{tab:instructions}) inspired by Liu et al.~\cite{liu2024bench} and based on our interviews with SLPs. 
Our interviews identified key criteria for determining joint attention, such as consistent eye contact and full engagement between the child and parent.

\begin{table*}[!htb]
\small
  \caption{Instruction templates in our task. \textcolor{orange}{<time>} denotes the timestamp representation, e.g., "23.6s" \textcolor{NavyBlue}{<description>} denotes the brief summary of the segment focused on the child's joint attention behaviour. \textcolor{Plum}{<label>} denotes the child joint attention quality of the segment.}
  \label{tab:instructions}
  \Description{}
  \begin{tabular}{p{.5\linewidth} p{.4\linewidth}}
    \toprule
    \textbf{Instruction} &  \textbf{Example Response}\\
    \midrule
    You are given a video about parent-child interactions. Watch the video carefully and your task is to:

    1. Identify the key moments where the child's joint attention occurs. Child joint attention is defined as the child making eye contact and being fully engaged with the parent through interaction.
    
    2. Specify the timestamps for when each moment starts and ends.

    3. Classify the quality of the child's joint attention into "Strong" and "Poor".

    For example:
    
    - Timestamp: 23.6s - 26.8s
    
    - Description: The child looks at his mother as she gives him instructions.
    
    - Quality: Strong
    &
    Timestamp: \textcolor{orange}{<time>} - \textcolor{orange}{<time>} and \textcolor{orange}{<time>} - \textcolor{orange}{<time>}
    
    Description: \textcolor{NavyBlue}{<description>}
    
    Quality: \textcolor{Plum}{<label>}
    \\
  \bottomrule
\end{tabular}
\end{table*}

\subsubsection{Evaluation Method of MLLMs Output Quality and Content Objectivity}

To assess the objectivity and quality of the models' responses, we define four metrics building on key aspects emphasized by SLPs in their annotations:

\begin{itemize}
    \item \textbf{Time Sensitivity}: Proportion of responses that include \textcolor{orange}{<time>} - \textcolor{orange}{<time>}.
    \item \textbf{Description Accuracy}: Proportion of accurate \textcolor{NavyBlue}{<description>}s within the specified \textcolor{orange}{<time>} - \textcolor{orange}{<time>} in the responses.
    \item \textbf{Eye-Contact Sensitivity}: Proportion of accurate \textcolor{NavyBlue}{<description>} that include eye contact information.
    \item \textbf{Eye-Contact Accuracy}: Proportion of accurate \textcolor{NavyBlue}{<description>} that include accurate eye contact information.
\end{itemize}

We reviewed the responses against the original videos. For each response,  we verified the inclusion of time information, descriptions' accuracy, and eye-contact events' presence and correctness. Annotations were assigned based on these criteria, ensuring a reliable benchmark for comparing the models' performance across the four evaluation metrics.

\subsubsection{Evaluation Method of MLLMs Temporal Grounding in Joint Attention Segments}

To evaluate the temporal understanding capabilities of MLLMs in detecting strong and poor joint attention segments, we used mean Intersection over Union (mIoU) and Recall at IoU thresholds (R@m). These metrics assess the alignment quality between predicted and ground truth time segments. IoU measures the overlap between the predicted and ground truth time segments, defined as the ratio of the intersection to the union of these two time intervals. A higher IoU indicates better alignment, and mIoU represents the average IoU across all segments. R@m evaluates the proportion of ground truth segments with at least one predicted segment with sufficient overlap based on predefined IoU thresholds.

The evaluation process used the intersection of annotations from the two SLPs as the ground truth. Each model outputted its predicted timestamps for strong and poor joint attention segments. Using these predictions, we computed the metrics as follows:
\begin{itemize}
    \item For \textbf{mIoU}, we calculated the average overlap between the predicted and ground truth timestamps across all segments, ensuring that both temporal precision and alignment quality were considered.
    \item For \textbf{R@m}, we assessed how many ground truth segments had at least one predicted segment with sufficient overlap (defined by IoU thresholds of 0.3, 0.5, and 0.7).
\end{itemize}

\subsection{Performance Analysis and Findings}
We conducted experiments on three models. For GPT-4o and Gemini-1.5-Flash, we utilized their respective APIs. The model was deployed and executed locally on a Ubuntu 22.04 LTS server equipped with two NVIDIA RTX 3090 GPUs for Video-ChatGPT. 

\subsubsection{Evaluation Result of MLLMs response Quality and Content Objectivity}

Table~\ref{tab:model-evaluation} presents the evaluation of three models—GPT-4o, Gemini-1.5-Flash, and Video-ChatGPT—across four metrics.
For \textit{Time Sensitivity}, both GPT-4o and Gemini-1.5-Flash achieved perfect scores (\textit{100\%}), indicating their consistent inclusion of time information in the responses. 
In contrast, Video-ChatGPT performed poorly, with only a small proportion of responses with time information (\textit{36.36\%}).
GPT-4o demonstrated the strongest performance (\textit{90.17\%}) in \textit{Description Accuracy}, accurately describing tasks and scenes within the specified time ranges. 
Gemini-1.5-Flash and Video-ChatGPT performed worse, scoring (\textit{50.53\%}) and (\textit{35.71\%}), respectively. 
Gemini-1.5-Flash frequently misinterpreted interactions, often assuming that the child and parent had eye contact. At the same time, Video-ChatGPT exhibited significant scene misrecognition, such as mistaking a boy for both a boy and a girl.
For \textit{Eye-Contact Sensitivity}, GPT-4o showed limited inclusion of eye-contact information (\textit{23.64\%}), while Gemini-1.5-Flash demonstrated a much higher sensitivity (\textit{92.12\%}), actively identifying eye-contact cues. Video-ChatGPT failed to provide any eye-contact information in its responses.
For \textit{Eye-Contact Accuracy}, GPT-4o achieves the best performance (\textit{62.82\%}), avoiding overestimations of eye contact. Gemini-1.5-Flash has a lower accuracy (\textit{43.92\%}) due to frequent overestimation, assuming eye contact occurs more often than it does. Video-ChatGPT does not output eye-contact information, making it unsuitable for tasks requiring such data.

\begin{table*}[htb!]
\centering
\caption{Evaluation of MLLMs Output Quality and Content Objectivity Across Four Metrics: Time Sensitivity (proportion of responses including time information), Description Accuracy (accuracy of descriptions within specified time ranges), Eye-Contact Sensitivity (proportion of responses including eye contact information), and Eye-Contact Accuracy (accuracy of identified eye-contact information).}
\label{tab:model-evaluation}
\begin{tabular}{p{0.15\textwidth} p{0.18\textwidth} p{0.18\textwidth} p{0.18\textwidth} p{0.18\textwidth}}     
    \toprule
    \textbf{Model} & \textbf{Time Sens.} & \textbf{Description Acc.} & \textbf{Eye-Contact Sens.} & \textbf{Eye-Contact Acc.} \\
    \midrule
    GPT-4o & 100\% & 90.17\% & 23.64\% & 62.82\% \\
    Gemini-1.5-Flash & 100\% & 50.53\% & 92.12\% & 43.92\% \\
    Video-ChatGPT & 36.36\% & 35.71\% & 0\% & 0\% \\
    \bottomrule
\end{tabular}
\end{table*}

While GPT-4o and Gemini-1.5-Flash perform well in Time Sensitivity and Description Accuracy, all models show weaknesses in Eye-Contact Sensitivity and Accuracy. Since Video-ChatGPT performed poorly in time sensitivity, only GPT-4o and Gemini-1.5-Flash were considered for the temporal grounding evaluation in the following section.

\begin{table*}[htb!]
\centering
\caption{Comparison of Temporal Grounding Metrics between Gemini-1.5-Flash and GPT-4o for Strong, Poor, and Overall Joint Attention Segments. The metrics include Recall at IoU thresholds (R@0.3, R@0.5, and R@0.7) and mean Intersection over Union (mIoU). R@0.3, R@0.5, and R@0.7 represent the proportion of ground truth segments with at least one predicted segment with sufficient overlap at IoU thresholds of 0.3, 0.5, and 0.7, respectively. mIoU measures the average alignment quality between predicted and ground truth segments. "Strong" refers to the model's performance on segments with strong joint attention, "Poor" refers to segments with poor joint attention, and "Overall" reflects performance across all segments.}
\begin{tabular}{p{0.25\textwidth} p{0.15\textwidth} p{0.15\textwidth} p{0.15\textwidth} p{0.15\textwidth}}     \toprule
\textbf{Model}               & \textbf{R@0.3 (\%)} & \textbf{R@0.5 (\%)} & \textbf{R@0.7 (\%)} & \textbf{mIoU (\%)} \\ \midrule
GPT-4o (Strong)              & 2.01                 & 0.57                 & 0.57                 & 2.19               \\
Gemini-1.5-Flash (Strong)    & 1.72                 & 0.69                 & 0.52                 & 1.39               \\
GPT-4o (Poor)                & 0.00                 & 0.00                 & 0.00                 & 3.30               \\
Gemini-1.5-Flash (Poor)      & 0.00                 & 0.00                 & 0.00                 & 2.28               \\
\midrule
GPT-4o (Overall)             & 1.77                 & 0.51                 & 0.51                 & 2.29               \\
Gemini-1.5-Flash (Overall)   & 1.17                 & 0.47                 & 0.35                 & 1.78               \\
\bottomrule
\end{tabular}
\label{tab:metrics_comparison}
\end{table*}


\subsubsection{Evaluation Result of MLLMs Temporal Grounding in Joint Attention Segments}

Table \ref{tab:metrics_comparison} highlights that GPT-4o outperforms Gemini-1.5-Flash on strong joint attention segments, achieving a higher mIoU (2.19\% vs. 1.39\%) and a slightly better R@0.3 (2.01\% vs. 1.72\%). This suggests GPT-4o aligns more effectively with ground truth for strong segments, likely due to its broader contextual understanding and reduced reliance on explicit eye-contact information.

% \begin{table*}[htb!]
% \centering
% \caption{Comparison of Temporal Grounding Metrics between Gemini-1.5-Flash and GPT-4o for Strong, Poor, and Overall Joint Attention Segments. The metrics include Recall at IoU thresholds (R@0.3, R@0.5, and R@0.7) and mean Intersection over Union (mIoU). R@0.3, R@0.5, and R@0.7 represent the proportion of ground truth segments that have at least one predicted segment with sufficient overlap at IoU thresholds of 0.3, 0.5, and 0.7, respectively. mIoU measures the average alignment quality between predicted and ground truth segments. "Strong" refers to the model's performance on segments with strong joint attention, "Poor" refers to segments with poor joint attention, and "Overall" reflects performance across all segments.}
% \begin{tabular}{p{0.15\textwidth} p{0.18\textwidth} p{0.18\textwidth} p{0.18\textwidth} p{0.18\textwidth}}     \toprule
% \textbf{Model}               & \textbf{R@0.3 (\%)} & \textbf{R@0.5 (\%)} & \textbf{R@0.7 (\%)} & \textbf{mIoU (\%)} \\ \midrule
% GPT-4o (Strong)              & 2.01                 & 0.57                 & 0.57                 & 2.19               \\
% Gemini-1.5-Flash (Strong)    & 1.72                 & 0.69                 & 0.52                 & 1.39               \\
% GPT-4o (Poor)                & 0.00                 & 0.00                 & 0.00                 & 3.30               \\
% Gemini-1.5-Flash (Poor)      & 0.00                 & 0.00                 & 0.00                 & 2.28               \\
% \midrule
% GPT-4o (Overall)             & 1.77                 & 0.51                 & 0.51                 & 2.29               \\
% Gemini-1.5-Flash (Overall)   & 1.17                 & 0.47                 & 0.35                 & 1.78               \\
% \bottomrule
% \end{tabular}
% \label{tab:metrics_comparison}
% \end{table*}

For poor joint attention segments, GPT-4o also achieves a higher mIoU (3.30\% vs. 2.28\%), despite both models reporting zero Recall across all IoU thresholds (R@0.3, R@0.5, R@0.7). This discrepancy occurs because mIoU captures partial overlaps between predicted and ground truth regions, even if the overlaps are insufficient to meet recall thresholds. In the poor category, models may predict larger or misaligned regions, contributing to mIoU but failing to exceed the recall thresholds.

Overall, GPT-4o demonstrates superior performance, with a higher mIoU (2.29\% vs. 1.78\%) and slightly better recall values across all categories. However, this low value indicates limited alignment between predicted and ground truth regions. The result reflects the challenges of temporal grounding in joint attention tasks, where precise boundary prediction is difficult.

Despite these results, the findings highlight limitations in current MLLMs for direct understanding of joint attention. As noted earlier, deficiencies in processing eye-contact information significantly reduce accuracy, underscoring the need for improved multimodal integration.
This paper presents a planning approach for effective and efficient joint motion generation for manipulators to cover a surface, aiming to minimize specific joint space costs.

\textit{Limitations} -- Our work has several limitations that suggest potential directions for future research. First, our method uses a heuristic to accelerate the traditional Joint-GTSP approach. While we provide empirical evidence of its efficiency in producing high-quality solutions, we cannot guarantee consistent performance in all scenarios.
Second, our bi-level hierarchical method reduces the size of GTSP. Future research could extend it to multiple levels to further improve performance, though this may produce misleading guide paths.
Third, we observe that both Joint-GTSP and H-Joint-GTSP tend to generate paths with frequent turns, a pattern also observed in the motions of prior work \cite{kaljaca2020coverage, zhang2024jpmdp}.  Future work should explore strategies to balance joint movements with other objectives such as motion smoothness.

\footnotetext{Visualization tool: \url{https://github.com/uwgraphics/MotionComparator}}
\textit{Implications} -- The hierarchical approach presented in this work enables effective and efficient coverage path planning for robot manipulators. 
This approach is beneficial to applications that require dexterous surface coverage, such as sanding, polishing, wiping, and sensor scanning. 


\section{Concluding Remarks}
In this paper, we proposed a novel approach utilizing multimodal LLMs to generate gesture-aware speech recognition transcripts for patients with language disorders. Our framework integrates verbal speech and iconic gestures, enabling the generation of enriched transcripts that capture the latent meaning conveyed through both modalities. Through extensive experimentation, we demonstrated that the proposed method effectively contextualizes incomplete or disfluent speech by incorporating gesture information, leading to more accurate and meaningful representations of the speaker's intent. These findings highlight the potential of our approach to significantly contribute to the field of speech and language therapy, offering innovative tools that can enhance the quality of life for individuals with language disorders by facilitating better communication and assessment methods.

\subsection{Ethical Statement} 
Our dataset was obtained from AphasiaBank with the approval of the Institutional Review Board (IRB) and adheres to the data sharing guidelines set by TalkBank\footnote{https://talkbank.org/share/ethics.html}. This includes complying with the Ground Rules for all TalkBank databases, which are based on the American Psychological Association Code of Ethics~\cite{american2002ethical}.

\subsection{Limitation \& Future Work} 
%This study represents a preliminary investigation into using multimodal LLMs to generate gesture-aware speech recognition transcripts. 
While the results are promising, we recognize several limitations and outline our plans to extend this work further.

One primary limitation is the absence of a definitive ground truth for quantitative evaluation. Since our model generates transcripts by synthesizing speech and gesture data from scratch, traditional benchmarks, such as comparisons with standard speech recognition outputs, are insufficient. Moreover, existing original transcripts lack gesture annotations, making direct comparisons challenging. In future work, we aim to address this gap by collaborating with certified pathologists to conduct qualitative assessments, such as A-B preference tests, to evaluate the effectiveness of gesture-enriched transcripts in accurately conveying the speaker's intentions.

To support quantitative evaluations, we plan to develop novel metrics that assess transcript quality, including grammar accuracy, semantic consistency, and the integration of multimodal information. Such metrics will provide a more objective basis for assessing our model's performance and facilitate comparisons with other multimodal and unimodal approaches.

Another limitation of this study is its focus on structured gestures from a specific task, the Peanut Butter Sandwich Task. While this task offers a controlled context for testing our approach, it does not encompass the diversity of gestures and communication patterns seen in everyday scenarios. As part of our future work, we plan to expand the scope of our model to include tasks such as the Cinderella Story Recall Task~\cite{bird1996cinderella}, which involves unstructured and complex narrative gestures. This expansion will allow us to evaluate the adaptability and robustness of our model in handling varied linguistic and gestural contexts.

In summary, while this study establishes a strong foundation for gesture-aware speech recognition, we aim to refine and extend our methods through collaborative qualitative evaluations, the development of robust quantitative metrics, and broader task applications. These efforts will ensure that our approach continues to evolve, ultimately contributing to more effective communication tools and interventions for individuals with language disorders.





\begin{acks}
This research is supported by the National Research Foundation, Singapore, under its AI Singapore Programme (AISG Award No. AISG2-TC-2022-007).
\end{acks}



%%
%% The next two lines define the bibliography style to be used, and
%% the bibliography file.
\bibliographystyle{ACM-Reference-Format}
\bibliography{main}


%%
%% If your work has an appendix, this is the place to put it.
\appendix

\end{document}
\end{document}
\endinput
%%
%% End of file `sample-sigconf-authordraft.tex'.
