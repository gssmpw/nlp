\section{CONCLUSION}
Joint attention plays a vital role in early language development, serving as a cornerstone for effective parent-child interaction. In this study, we analysed 26 parent-child interaction videos, with annotations from two SLPs identifying segments of strong and poor joint attention. 

We tested the capabilities of MLLMs in understanding these interactions. Our findings reveal that current MLLMs face limitations due to the absence of explicit eye-contact information, which significantly impacts their ability to comprehend joint attention.







