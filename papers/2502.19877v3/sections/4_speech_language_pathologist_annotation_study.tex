\section{Speech-Language Pathologists Annotation Study}
\subsection{Annotation Process}
To evaluate the performance of MLLMs in identifying strong or poor joint attention segments, we recruited two SLPs (both female) with extensive experience to provide expert annotations. They each have 7–9 years of experience and training in programs such as DIR Floortime~\cite{dir_floortime}, It Takes Two to Talk~\cite{pepper2004talk}, and More Than Words~\cite{sussman1999words}. While a third SLP was recruited, the session was incomplete and the results were not included in this paper. Our studies complied with ethical regulations approved by our institution's ethics review board.


% The first expert holds a bachelor's degree in Speech Pathology and has 9 years of experience working with children. She is trained in DIR Floortime and the Hanen Programs (It Takes Two to Talk and More Than Words) and is skilled in various communication and sensory intervention techniques.  

% The second expert has a bachelor's degree in Speech Pathology and 7 years of experience in pediatric speech therapy. She is certified in DIR Floortime and the Hanen More Than Words program, focusing on enhancing communication and social skills in children with developmental delays.  

The study was conducted in person over one month. Each session lasted approximately 2 hours and was held in a private meeting room to ensure focus and comfort. The labelling was carried out using our custom-built annotation system. Videos can be played, paused, or navigated frame-by-frame, while the interactive timeline enables users to select segments and label them as \dquote{strong} or \dquote{poor}. The system exports the results in CSV format, including each labelled segment's start and end times, along with corresponding indices. Additionally, a panel allows experts to add notes during the labelling process.

% \begin{table}[htb!]
% \centering
% \caption{Comparison of Annotation Across SLPs: \textbf{SLP} (annotator ID), \textbf{Total} (number of segments), \textbf{Strong (\%)} and \textbf{Poor (\%)} (counts and percentages of strong/poor joint attention), \textbf{Avg. (s)} (average segment duration), \textbf{Avg. Strong (s)}, and \textbf{Avg. Poor (s)} (average durations of strong and poor segments). The \textbf{Intersection} row shows metrics for mutually agreed segments.}
% \begin{tabular*}{\textwidth}{@{\extracolsep{\fill}}lcccccc}
% \toprule
% \textbf{SLP} & \textbf{Total} & \textbf{Strong (\%)} & \textbf{Poor (\%)} & \textbf{Avg. (s)} & \textbf{Avg. Strong (s)} & \textbf{Avg. Poor (s)} \\ \midrule
% S1        & 96          & 89 (92.71)           & 7 (7.29)            & 5.61              & 5.24                    & 10.30                 \\
% S2        & 72          & 56 (77.78)           & 16 (22.22)          & 14.51             & 14.85                   & 13.33                 \\\midrule
% Intersection      & 62            & 58 (93.55)              & 4 (6.45)              & 3.88                             & 3.65 &       10.73    \\       
% \bottomrule
% \end{tabular*}
% \label{tab:segment_metrics}
% \end{table}


\subsection{Annotation Outcomes Analysis}
We conducted full annotation interviews with SLP 1 (\textit{S1}) and SLP 2 (\textit{S2}). Through these interviews, a shared criterion for determining strong joint attention was established: it is characterized by the child maintaining consistent eye contact and \textit{active engagement} with the parent during the interaction. 
% Due to time constraints, Expert 3 was unable to participate in the interviews, and the results are based solely on the annotations provided by \textit{E1} and \textit{E2}.

Table \ref{tab:segment_metrics} summarizes the annotation metrics provided by the two SLPs. \textit{S1} annotated a total of 96 segments. Among these, \textit{92.71\%} were identified as strong segments, with an average segment duration of \textit{5.61 s}. Strong segments were notably shorter, averaging \textit{5.24 s}, while poor segments were longer at \textit{10.30 s}. \textit{S2} annotated 72 segments. Among these, \textit{77.78\%} were classified as strong, and \textit{22.22\%} as poor. The average duration for strong and poor segments was \textit{14.85 s} and \textit{13.33 s}, respectively, reflecting S2's longer segments in annotations.

To ensure fairness and consistency, we calculated the intersection of annotations where SLPs agreed. This resulted in \textit{62} segments, with \textit{93.55\%} identified as strong and \textit{6.45\%} as poor. The average duration for strong and poor segments in the intersection was \textit{3.65 s} and \textit{10.73 s}, respectively, while the overall average duration was \textit{3.88 s}.
