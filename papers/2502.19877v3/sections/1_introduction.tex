\section{INTRODUCTION}
In parent-child interaction, joint attention, play, and imitation are crucial for promoting child speech-language development~\cite{charman2000testing}. Joint attention, in particular, plays a foundational role in fostering communication and language skills. Two individuals simultaneously focus on the same object or event, sharing their experience through eye contact, gestures, or verbal expressions~\cite{hanen_joint_attention}. Such mechanisms facilitate shared focus and enable parents and infants to achieve the social coordination necessary for language learning~\cite{baldwin2014understanding}.

While some technologies~\cite{hwang2014talkbetter,song2016talklime} have been developed to enhance joint attention, there is little work~\cite{kwon2022captivate} on detecting joint attention in parent-child interaction. With the rise of multimodal large language models (MLLMs), which have demonstrated strengths in processing conversations and analysing multimodal content~\cite{lu2024gpt}, these models offer promising tools to identify joint attention segments in parent-child videos.

To bridge this gap, our study explores the use of MLLMs to understand joint attention in parent-child interactions. Using our selection criteria, we collected a final dataset of 26 publicly sourced online videos showcasing parent-child interactions. Collaborating with two professional speech-language pathologists (SLPs), we annotated the videos with strong and poor joint attention segments. By comparing the MLLMs' outputs with expert annotations, we identified the limitations of joint attention detection in current MLLMs. 

Our contributions are as follows:

\begin{itemize}
    \item We collected and labelled the first-of-its-kind dataset of parent-child interaction videos. This was labelled for joint attention together with professional SLPs.
    \item We tested the capabilities of MLLMs in understanding joint attention within these interactions and identified the limitations of current MLLMs' understanding of joint attention for speech-language therapy.
\end{itemize}