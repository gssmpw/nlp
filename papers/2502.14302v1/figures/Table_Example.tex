\begin{table*}[htbp]
\centering
\small
\caption{Examples from the MedHallu Dataset.}
\label{tab:dataset-samples}
\begin{tabular}{|p{3cm}|p{2cm}|p{5cm}|p{3cm}|p{1cm}|}
\hline
\textbf{Question} & \textbf{Category of Hallucination} & \textbf{Ground Truth} & \textbf{Hallucinated Example} & \textbf{Difficulty} \\
\hline
Do mitochondria play a role in remodelling lace plant leaves during programmed cell death? 
& Mechanism and Pathway Misattribution 
& This is the first report of mitochondria and chloroplasts moving on transvacuolar strands to form a ring structure surrounding the nucleus during developmental PCD. Also, for the first time, we have shown the feasibility for the use of CsA in a whole plant system. Overall, our findings implicate the mitochondria as playing a critical and early role in developmentally regulated PCD in the lace plant. 
& Mitochondria regulate the formation of perforations in lace plant leaves through the modulation of calcium channels and by activating specific proteases that degrade cellular structures, facilitating the PCD process.
& Medium \\
\hline
Differentiation of nonalcoholic from alcoholic steatohepatitis: are routine laboratory markers useful?
& Misinterpretation of \#Question\#
& Higher MCVs and AST/ALT ratios in ASH reflect the severity of underlying liver disease and do not differentiate NASH from ASH. Instead, these biomarkers might prove useful in guiding selection of patients for liver biopsy and in targeting therapy.
& Routine laboratory markers, particularly mean corpuscular volume (MCV) and AST/ALT ratios, significantly aid in differentiating various types of nonalcoholic steatohepatitis (NASH).
& Hard \\
\hline
Amblyopia: is visual loss permanent?
& Incomplete Information
& Older people with a history of amblyopia who develop visual loss in the previously normal eye can experience recovery of visual function in the amblyopic eye over a period of time. This recovery in visual function occurs in the wake of visual loss in the fellow eye and the improvement appears to be sustained.
& Visual loss due to amblyopia is permanent unless treated with early intervention during childhood.
& Hard \\
\hline
\end{tabular}
\end{table*}