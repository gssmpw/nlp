\documentclass[letterpaper,twocolumn,10pt]{article}
\usepackage{usenix2019_v3}

\usepackage{tikz}
\usepackage{amsmath}

\usepackage{graphicx}
\usepackage{xcolor}
\usepackage{comment}
\usepackage{subcaption}
\usepackage[T1]{fontenc}
\usepackage{booktabs}
\usepackage{makecell}

\newcommand*\emptycirc[1][1ex]{\tikz\draw (0,0) circle (#1);} 
\newcommand*\halfcirc[1][1ex]{%
  \begin{tikzpicture}
  \draw[fill] (0,0)-- (90:#1) arc (90:270:#1) -- cycle ;
  \draw (0,0) circle (#1);
  \end{tikzpicture}}
\newcommand*\fullcirc[1][1ex]{\tikz\fill (0,0) circle (#1);} 
\newcommand*\circled[1]{\tikz[baseline=(char.base)]{
            \node[shape=circle,draw,inner sep=0.05pt] (char) {#1};}}

\newcommand\fatima[1]{\textcolor{orange}{(F: #1)}}
\newcommand\adeel[1]{\textcolor{olive}{(A: #1)}}

\microtypecontext{spacing=nonfrench}
\usepackage[switch]{lineno}

%-------------------------------------------------------------------------------
\begin{document}
%-------------------------------------------------------------------------------
\date{}

\title{\Large \bf SoK: State of the time: On Trustworthiness of Digital Clocks}

\author{
{\rm Adeel Nasrullah}\\
University of Massachusetts Amherst
\and
{\rm Fatima M. Anwar}\\
University of Massachusetts Amherst
}

\maketitle

%-------------------------------------------------------------------------------
\begin{abstract}
%-------------------------------------------------------------------------------
Despite the critical role of timing infrastructure in enabling essential services—from public key infrastructure and smart grids to autonomous navigation and high-frequency trading—modern timing stacks remain highly vulnerable to malicious attacks. These threats emerge due to several reasons, including inadequate security mechanisms, the timing architecture's unique vulnerability to delays, and implementation issues. In this paper, we aim to obtain a holistic understanding of the issues that make the timing stacks vulnerable to adversarial manipulations, what the challenges are in securing them, and what solutions can be borrowed from the research community to address them. To this end, we perform a systematic analysis of the security vulnerabilities of the timing stack. In doing so, we discover new attack surfaces, i.e., \textit{physical timing components and on-device timekeeping}, which are often overlooked by existing research that predominantly studies the security of time synchronization protocols. We also show that the emerging trusted timing architectures are flawed \& risk compromising wider system security, and propose an alternative design using \emph{hardware-software co-design}.
\end{abstract}

\documentclass[../main.tex]{subfiles}
\graphicspath{{../images/}}
\makeatletter
\def\input@path{{../images/}}
\makeatother
\begin{document}
\section{Introduction}
\begin{figure}
\centering
\begin{tikzpicture}
\node[inner sep=0pt] (ws) at (0, 0) {
\includegraphics[height=.4\textwidth, trim={10cm 0 10cm 0},clip]{world_space.png}};
\node[inner sep=0pt] (cs) at (6,0) {\includegraphics[height=.4\textwidth, trim={10cm 1cm 10cm 4cm},clip]{conf_space.png}};
\end{tikzpicture}
\vspace{-5pt}
\label{fig:pbrm_intro}
\caption{\textbf{Left}: Shows world space obstacles as grey spheres. Robots start and goal configuration is colored red and green, respectively. Configurations along the computed path are colored transparent blue. \textbf{Right:} Mapped world space scenario to configuration space. Obstacle region is the grey mesh. Red spheres are collision-free regions computed by the neural SCDF. The optimized shortest path in the convex corridor is the blue curve.}
\vspace{-25pt}
\end{figure}
Motion planning is the problem of finding a collision-free trajectory that connects a given start and goal configuration. The planning takes place in the configuration space of the robot. For single body robots, like mobile robots or drones, the configuration space and the world space are usually the same. This simplifies the planning, since explicit obstacle representations are available which enables geometrical tools like separating hyperplanes, smallest distance to obstacles etc., to be used when designing motion planning algorithms. For multi-body robots like manipulators, the situation is completely different. The world space obstacles are usually mapped to non-convex regions, and to make the problem even harder, the mapping is usually not known. Forming explicit representations of the obstacle region in the configuration space is usually too expensive or intractable. Despite all of this, sampling based planners are used with great success, which mainly is due to their use of implicit representations of the obstacle region. The basic idea is to construct a graph in the configuration space that covers and connects the collision-free region. From this graph, a path can be extracted that connects a given start and goal configuration. The approach is computationally expensive, since the graph is constructed with the smallest geometrical building block available, points, which represents a collision-check. Furthermore, the extracted paths from the graph are non-smooth and jagged due to the stochastic nature of the approach. This adds an additional post-processing step to the process, where the paths are shortcutted and smoothened, before the path can be used for tracking. Clearly a lot of time is invested to form this graph and produce smooth paths. Thus, if the obstacles start to move, then all of this work is done in no use, since all points that make up this graph need to be re-verified, which is simply too time consuming to be done in real time.
\\\\
In this work, we want to address the existing drawbacks of the sampling based planners. Our main contribution is an improved motion planner where each vertex in the graph covers a collision-free region in the form of a sphere instead of a point and where the edges are formed with neighboring intersecting spheres. This representation has the advantage of instead of returning piecewise linear paths, returning a sequence of overlapping spheres, i.e. a convex corridor, that connects a given start and goal configuration, illustrated in Figure \ref{fig:pbrm_intro}. This convex corridor allows us to use convex optimization to produce smooth trajectories, instead of computationally expensive post-processing methods. The representation further allows us to estimate the coverage of the collision-free space, which gives us awareness and feedback in the offline roadmap construction phase. Finally, our representation is simple to adapt to moving obstacles, simply requery for the new radii and recheck for intersections. 
\\\\
The spherical collision-free regions are formed using a signed distance function (SDF), which is a function that returns the smallest distance from an arbitrary point to the boundary of an obstacle. As the name implies, the distance is signed, thus if the point is inside the obstacle it is negative otherwise positive. If the distance is positive, a sphere with radius equal to the distance is guaranteed to cover a collision-free region. Using an SDF in motion planning is not new, but what is novel about our approach is that we express the distance in the configuration space instead of the world space and by doing so allows us to form these convex collision-free regions. We refer to the resulting SDF as a signed configuration distance function (SCDF). Computing an SCDF analytically is non-trivial, our approach is therefore to parameterize the SCDF with a deep neural network and learn the mapping by supervised learning. Our resulting neural SCDF can compute distances for different parameter values of obstacle shapes and we also show how multiple distances can be combined, thus making our approach flexible.
\section{Related work}
Motion planning algorithms can roughly be divided into three families, grid-based, sampling based and optimization based methods. Grid-based methods (GBM) discretize the planning space from which a graph is then compiled. A standard search method is A$^\star$ \citep{a_star}, which is classified as an \textit{informed} search method, since it employs a heuristic function to speed up the search. A$^\star$ guarantees to return an optimal path at the level of discretization used. GBMs usually discretize the planning space by a regular lattice and this limits the GBMs to problems with low dimensionality due to the curse of dimensionality. Thus, GBMs are usually limited to single-body robots where the degrees of freedom (DOF) are low. To overcome the inherent scaling problem with the GBMs, stochastic methods are usually used for multi-body robots. These methods are termed as sampling-based methods (SBM) and core members within this family are the rapidly-exploring random trees (RRT) \citep{rrt} and the probabilistic roadmap (PRM) \citep{prm}. RRT grows a tree from the start configuration and explores the collision-free region in a rapid way until it is able to connect to the goal region. RRT is usually improved by bi-directional planning \citep{rrt_connect}, i.e. an additional tree is grown from the goal configuration and the trees are tested for connection after any tree has been expanded. RRT is a single-query method, thus it searches for a path from scratch each time it is queried. Contrary to this, PRM is a multi-query method, which solves for multiple queries without starting from scratch. PRM does this by creating a roadmap (graph) that covers the collision-free space as an offline step. The graph is then used to solve for multiple queries. PRMs are used in cases where the environment does not change since the extra offline step is too computationally costly and needs to be re-done if the environment is changed. In our work, we address this inherent issue by using a different roadmap representation. Our vertices in the graph cover a collision-free region in the form of spheres and we form the edges by checking for intersecting spheres. If something in the environment changes, we recompute the spheres radii and recheck the intersections, without relying on collision detection. We use a trained neural network to compute the sphere radius, therefore querying for the radius can be done fast, hence our representation enables the PRM for dynamic environments.
\\\\
In the recent decades, optimization based methods (OBM) \citep{chomp, schulman, itomp, stomp} have been introduced as an alternative to SBM for multi-body robots. Like the SBM, the OBMs scale well to higher dimensional problems and produce smoother motion. It is common to use a SDF in the optimization since it is a smooth function, thus enabling gradient-based methods. However, the standard way of expressing the SDF is in world space. The distance therefore needs to be mapped to the configuration space by the forward kinematics. This mapping makes the optimization problem a non-linear program (NLP), which is computationally expensive to solve. Recently, a different approach has been proposed. In \cite{mp_gcs} motion planning is formulated as a convex optimization problem by using the graph of convex sets framework \citep{gcs}. The underlying idea is to decompose the collision-free space into intersecting convex sets from which a convex optimization problem is formulated. In cases where an explicit representation of the obstacles in the configuration space exists, like for single-body robots, creating collision-free convex regions can be done fast \citep{iris}. For multi-body robots, this is non-trivial. Existing work does this successfully \citep{iris_nlp, iris_c} by an optimization based approach, but the methods are still too time consuming to be used in the presence of moving obstacles. Our approach is instead to use deep learning to learn an SDF expressed in the configuration space. With this, we can query for shortest distances to the collision boundary, which allows us to expand spherical regions which are collision-free. Our approach is fast and therefore enables our suggested roadmap planner to be used in dynamic environments.
\\\\
Recent research has focused on learning collision detection \citep{fk_kernel_distance, diffco, graphdistnet} by predicting the signed distance between the robot links and the surrounding obstacles in the world space. The learned SDF is used in trajectory optimization but since the distance is expressed in the world space, the problem becomes an NLP and therefore takes a long time to solve. We take a novel approach and suggest to instead express the signed distance in the configuration space. This allows us to improve the PRM at the same time as it enables convex optimization for trajectory optimization, which runs faster and is more reliable than NLP solvers. In \cite{cspf} a learned signed distance function in the configuration space is proposed similar to our approach. However, their approach is restricted to point cloud representations, while we propose to represent the obstacles as parameterized geometric shapes, e.g. spheres. Furthermore, we also show how to use our learned SCDF to improve an existing roadmap planner.
\section{Problem formulation}
A robot is located in the world space, $\W \subset \R^3 $. The unique location of the robot is given by its configuration $\q \in \C$, where $\C$ is the configuration space. The set of points covered by the robots bodies at a certain configuration is expressed as $\B(\q) \subset \W$. The robot is surrounded by $\NrObst$ obstacles $\O = \bigcup_{i=1}^{\NrObst} \O_i$, where  $\O_i \subset \W$. The representation of the obstacle in the configuration space is the set $\C\O_i = \{\q \in \C \: |\: \B(\q) \cap \O_i \neq \emptyset \}$. The obstacle space is formed as $\Co = \bigcup_{i=1}^{\NrObst} \C \O_i$. The complement is referred to as the free space, $\Cf = \C \setminus \Co$. The path planning problem is a tuple, ($\Cf$, $\qStart$, $\qGoal$), where we want to connect a query pair, consisting of a start, $\qStart$, and goal configuration, $\qGoal$, with a geometric path, $\q(s): [0, 1] \mapsto \Cf$, such that $\q(0)=\qStart$ and $\q(1)=\qGoal$, or report correctly when such a path does not exist.
\end{document}

\section{Basic Background: Supervised Learning and the PAC Model}
\label{sec:background}

At this point almost everyone has heard of machine learning (ML). Anyone likely to stumble upon this article will have also heard of its most influential special case, supervised learning, and those theoretically inclined will also be familiar with the PAC model. Nonetheless, I will set the stage by  recapping the basics.

\subsection{Basics of Supervised Learning}%Let's set the stage in any case

\emph{Supervised Learning} is the task of ``coming up'' with a function $f: \X \to \Y$ to ``explain'' or ``fit'' a sequence of input/output examples   $(x_1,y_1), \ldots, (x_n,y_n)$, with $x_i \in \X$ and $y_i \in \Y$.  Here $\X$ is a \emph{data domain} consisting of \emph{datapoints} $x \in \X$, $\Y$ is a \emph{label set} consisting of \emph{labels} $y \in \Y$, and the sequence $(x_1,y_1),\ldots,(x_n,y_n)$ is the \emph{training data} consisting of \emph{labeled examples (a.k.a. samples)}~$(x_i,y_i)$.  I~will refer to the chosen function $f$ as a \emph{predictor}, and to $n$ as the \emph{sample size}. A \emph{learning algorithm} takes as input training data, and outputs (some representation of) a predictor $f \in \Y^\X$.\footnote{Note that this describes the usual \emph{batch}, a.k.a.~\emph{offline}, setting of supervised learning. I do not discuss other paradigms such as online or active learning in this article.} 



Success in supervised learning is defined as \emph{generalization} to  future examples: For a typical \emph{test example}  $(x_{\tst},y_{\tst})$, the predicted label $y'_{\tst}=f(x_{\tst})$ should ``equal'' $y_{\tst}$, perhaps approximately. We usually assume the test example is drawn from the same  ``source'' as the training data  --- commonly, i.i.d.~from the same distribution. The quality of the prediction is quantified by $\ell(y'_{\tst},y_{\tst})$, where $\ell:~\Y~\times~\Y \to \RR_{\geq 0}$ is a \emph{loss function} chosen as part of the problem definition. Common loss functions include the 0-1 loss $\ell_{0-1}(y',y) = [y' \neq y]$ for \emph{classification} problems,\footnote{The notation $[P]$ denotes $1$ when predicate $P$ is true, and denotes $0$ when $P$ is false.} as well as the absolute loss $|y'-y|$ or squared loss $(y'-y)^2$ for \emph{regression problems} featuring $\Y  \sse \RR$.

Nontrivial generalization properties are typically only possible if one assumes something about the data.\footnote{The need for such an assumption is formalized by the  \emph{no free lunch theorems} of supervised learning \cite{wolpert_connection_1992,wolpert_lack_1996,schaffer_conservation_1994}.} The Bayesian approach to  machine learning, common in many applications, assumes some parametric form for the distribution generating the data, and postulates a prior on the parameters. This is not the approach I will take in this article. Instead, I will focus on the frequentist --- and some would say ``worst-case'' or ``adversarial'' ---  approach that is common in the computational learning theory community, embodied by the PAC model. Here we assume that the (training and test) data can be explained, perhaps approximately, by a function in some ``simple enough to learn'' class of functions $\H \sse \Y^\X$, often called the \emph{hypotheses}. Equivalently, we  seek a predictor which explains the unseen data roughly  as well as the best hypothesis $h^* \in \H$, whether or not we assume that $h^*$ itself provides a perfect explanation.



 \paragraph{Common Algorithmic Templates.} Perhaps the best known general-purpose supervised learning algorithm is \emph{empirical risk minimization (ERM)}, which chooses as its predictor a hypothesis $f \in \H$ minimizing $\frac{1}{n} \sum_{i=1}^n \ell(f(x_i),y_i)$ --- a quantity called the \emph{training error}, \emph{empirical error}, or \emph{empirical risk} of $f$. %\footnote{When multiple hypotheses minimize the empirical risk, we assume ERM breaks ties arbitrarily.}
A common template for generalizing ERM involves adding a \emph{regularization term} $\psi(f)$ to the  objective function, typically chosen to measure some notion of ``hypothesis complexity.'' An algorithm instantiating this template is known as a \emph{structural risk minimizer (SRM)}, and chooses as its predictor the hypothesis $f \in \H$ minimizing the \emph{structural risk} $\frac{1}{n} \sum_{i=1}^n \ell(f(x_i),y_i) + \psi(f)$. Other well-known algorithms, such as gradient descent and its variations,  can frequently be interpreted as approximate implementations of ERM or SRM.


\paragraph{Proper vs Improper Learning.} A learning algorithm is said to be \emph{proper} if its predictor $f$ is always chosen from the hypothesis class, i.e., $f \in \H$, otherwise it is said to be \emph{improper}. ERM  is an example of a proper learning algorithm, as are SRM algorithms of the form described above.  In the \emph{proper regime} of learning, algorithms are required to be proper. This article will be concerned with the more flexible \emph{improper regime} (a.k.a \emph{representation-independent learning}), where no such constraint is placed on the learner. In other words, all we care about is predictive power at test time, rather than any insights derived from the functional form or representation of the predictor~itself.


\subsection{The PAC Model}
A standard mathematical setup for evaluation of supervised learning algorithms, at least in the theoretical computer science community, is Valiant's \emph{Probably Approximately Correct (PAC) model} of learning (see e.g.~\cite{kearns_introduction_1994,mohri_foundations_2018}). Here, we assume there is an unknown distribution $\D$ on $\X \times \Y$ from which training and test data are  drawn.  Specifically, the labeled datapoints of the training set  $(x_1,y_1), \ldots, (x_n,y_n)$, as well as the test data  $(x_\tst,y_\tst)$, are i.i.d.~from $\D$. Often it is assumed that $\D$ lies in some class of distributions of interest. The \emph{true expected loss}, or simply \emph{loss}, of a predictor $f: \X \to \Y$ is the expected loss it incurs on draws from $\D$, written $L_\D(f) = \Ex_{(x,y) \sim \D} \ell(f(x),y)$.


There are two main ``settings'' in PAC learning. The  \emph{realizable setting} only requires that the data be perfectly explained by some hypothesis in $\H$. More generally, the \emph{agnostic setting} makes no assumption relating the data to the hypotheses, but shifts the goalposts as necessary to allow nontrivial guarantees: the expected loss at test time is evaluated only ``relative'' to that of the best hypothesis $h^* \in \H$. There are other settings which make more nuanced assumptions, such as $\D$ being of a particular parametric form or its support living in some (unknown) lower-dimensional space, etc. I will mostly discuss the realizable and agnostic settings in this article, those being the simplest and most studied from a theoretical perspective. %TODO:We will briefly discuss other settings in Section ??

The PAC model demands high probability guarantees of learners, in the worst case over distributions of interest. Consider first the realizable setting, where $\D$ is such that $\min_{h \in \H} L_{\D}(h) = 0$. A PAC learner has \emph{error} $\epsilon=\epsilon(n)$ and \emph{confidence} $\delta=\delta(n)$ if, when training data consists of $n$ i.i.d~samples from a realizable distribution $\D$, it produces a predictor $f$  satisfying $L_\D(f) \leq \epsilon$ with probability at least $1-\delta$. In the agnostic setting, where $\D$ can be arbitrary, we require $L_\D(f) - \min_{h \in \H} L_\D(h) \leq \epsilon$ with probability $1-\delta$.

In both the realizable and agnostic settings, we look for PAC learners with small $\epsilon$ and $\delta$ as a function of the sample size $n$. An equivalent perspective looks at the sample complexity $m(\epsilon,\delta)$, which is the minimum sample size which guarantees error  at most $\epsilon$ with probability at least $1-\delta$. We say a problem is \emph{PAC learnable} if its PAC sample complexity is finite whenever $\epsilon,\delta > 0$.

For most PAC learning problems, learnability and sample complexity are characterized in terms of a  ``dimension'' of the hypothesis class. Most prominently this is the \emph{VC dimension} for binary classification, the \emph{fat shattering dimension} for agnostic regression, and the \emph{DS dimension} for multiclass classification (see \cite{anthony_neural_1999,daniely_optimal_2014,brukhim_characterization_2022}). Treatment of these is beyond the scope of this article. The unfamiliar reader need not worry, however,  as dimensions will feature only tangentially in our~discussion.




%\paragraph{Learning settings: Realizable, Agnostic, etc.} In learning theory, evaluating a supervised learning algorithm requires specifying a data model and an objective. We will leave the details of the data model flexible for now, to allow for both the PAC model and the adversarial transductive model. Nonetheless we will describe two variations, which we call ``settings'', which cut across different models. The  \emph{realizable setting}  requires only that the data be perfectly explained by some hypothesis $h \in \H$ --- i.e., there exists a hypothesis which is guaranteed to suffer a loss of $0$ on training and test data. The performance of the learning algorithm is its expected loss at test time for some ``worst case'' realizable instance. More generally, the \emph{agnostic setting} makes no assumption relating the data to the hypotheses, but shifts the goalposts as necessary to allow nontrivial guarantees: the expected loss at test time is evaluated only ``relative'' to that of the best hypothesis $h^* \in \H$, again for some ``worst case'' instance. There are other settings which make more nuanced assumptions about the data, such as it is drawn from a distribution of a particular parametric form, or that it lives in some (unknown) lower-dimensional space, etc. We will mostly discuss the realizable and agnostic settings, those being the simplest and most studied from a theoretical perspective.




%%% Local Variables:
%%% mode: latex
%%% TeX-master: "learning_matching"
%%% End:

\begin{figure*}[t]
\begin{center}
\includegraphics[width=.85\linewidth]{fig_overview_v3.pdf}
\end{center}
\caption{
FastAtlas Overview: In each frame, we compute charts spanning fully or partially visible triangles (a), determine texture space bounding boxes for the visible portions of the view-space projections of each chart, and tightly pack these boxes into atlases (b, here $2K \times 2K$). We simultaneously bijectively parameterize and shade the charts into their atlas boxes, obtaining high quality texture space shading (c), and use this shading to render the shaded frames (d).}
\label{fig:overview}
\label{fig:alg_overview}
\end{figure*}

\section{Overview}
\label{sec:overview}
Our work has two core contributions: a real-time, GPU-based algorithm for tight packing of general parameterized charts into compact atlases; and a real-time TSS method that
utilizes this packing.  

\paragraph*{FastAtlas Packing.}
FastAtlas runs entirely on the GPU as a series of compute shaders. It takes the bounding boxes of parameterized charts as input, and packs them into an atlas (Fig~\ref{fig:overview}b, Sec.~\ref{sec:pack}). As such, the only input it requires are the dimensions of the bounding boxes.
Its outputs are deterministic; identical input charts are packed into identical atlases. This is critical for TSS and similar applications, as it ensures that consecutive frames taken from the same camera view have the same shading. Even minute shading differences across such frames can cause sampling jitter, leading to undesirable flicker \cite{baker2012rock}. 
While prior methods such as \cite{mueller2018shading,hladky2019tessellated,hladky2021snakebinning,Neff2022MSA} cap the dimensions of the charts that can be packed as-is for a given atlas size, and scale down all charts that exceed these dimensions, we scale all charts by the same factor, and do so only when strictly necessary to achieve packing success (Figs~\ref{fig:atlas},~\ref{fig:sas_issues}). 

\paragraph*{TSS using FastAtlas.}
Our end-to-end TSS atlas generation method combines the packing method above with a novel approach for computing seamless per-frame charts. 
We define our charts as the connected components of the visible surfaces in each frame (Fig.~\ref{fig:overview}a), and efficiently compute them using a parallel union-find algorithm (Sec.~\ref{sec:visible}). Since the boundaries of these charts coincide with the contours of the rendered surface, they are {\em invisible} to the viewer. This approach 
eliminates the artifacts caused by shading discontinuities along visible seams (Fig.~\ref{fig:seams}). 

\begin{parWithWrapFigure}
\begin{wrapfigure}{l}{.27\columnwidth}%
\includegraphics[width=\linewidth]{fig_inset_view_plane.pdf}%
\end{wrapfigure}
We bijectively parametrize the {\em visible portions} of our charts by projecting them to view space (inset). This maps a constant number of texels to each pixel in the final rendered output, evenly distributing residual undersampling error across all image pixels. While conceptually straightforward, efficiently parameterizing charts containing partially visible triangles using viewspace projection is non-trivial, as the visible portions may no longer be triangular (e.g. green triangle in the inset); applying naive projection to triangles with vertices behind the camera may produce ill-posed results. Clipping triangles before projection is both computationally expensive and significantly complicates downstream operations. We avoid explicit clipping by observing that all that is required for atlas packing is the dimensions of, potentially conservative, bounding boxes of these projected visible portions. We compute such bounding boxes without explicit chart clipping by adapting a conservative screen coverage estimator \shortcite{Blinn:CalculatingScreenCoverage} (Sec.~\ref{sec:box}). We then pack the computed boxes using FastAtlas. 
\end{parWithWrapFigure}

Finally, we shade the visible portion of each chart into its corresponding atlas bounding box (Fig~\ref{fig:overview}c). 
The resulting texture is then used during rasterization as a standard texture map (Fig. ~\ref{fig:overview}d). 
Our framework is compatible with all existing approaches for texture space shading, including forward shading, raytraced illumination, or deferred shading in texture space \cite{baker:2016}. In the examples shown, we use the standard forward shading based rendering pipeline included in the G3D Innovation Engine \cite{G3D17}, a commercial grade renderer.

\section{Hardware Issues} \label{sec:hardware}
This section delves into vulnerabilities of the timing hardware, highlighting how such weaknesses can be exploited to alter a system's perception of time.

\subsection{Physical Side Channels} 
The physical components of the timing stack are susceptible to side channel attacks such as fault injection and manipulation of the system's thermal characteristics. To exploit these vulnerabilities, the adversary must have \textit{physical access} to the target system.

\noindent\textbf{\texttt{I01.} Laser Based Attacks on Crystal Oscillators.} Quartz crystal oscillators, integral to systems-on-chip (SoCs), are susceptible to optical laser attacks. Research by Kohei et al.~\cite{redshift} demonstrates a direct correlation between an oscillator's frequency and power of the laser incident upon it. 
They used this exploit to extract cryptographic keys embedded in the hardware. However, an adversary can use the same mechanism to alter the frequency of the oscillator that drives the hardware counters and timer registers on the SoC (\textit{time warping} attack --  \textbf{\texttt{A2}}). Fundamentally, this attack is possible as a result of crystal frequency sensitivity to ambient temperature (a laser incident on the crystal raises its temperature). Other time sources (e.g. atomic clocks) are also susceptible to environmental factors (temperature~\cite{nasa-atomic-clock}) and could be exploited by malicious adversaries albeit using a different attack mechanism than described here.

\noindent\textbf{\texttt{I02.} Computational Faults against Local Clock.} Computational faults resulting from under-volting a processor core affect both x86~\cite{hardware-plundervolt, hardware-V0ltpawn, hardware-voltage-pillager} and ARM~\cite{clock-sync-fault-presence, hardware-volt-jockey}; two of the most popular processor architectures. Such faults can corrupt instruction execution results (or even skip instructions altogether) and can be exploited to launch attacks against the \textit{local clocks} maintained by the system software. As described in section~\ref{subsec:local-clocks} (Time Keeping), system tick updates to the \textit{local clock} compute a new timestamp. Under-volting a processor core during this system tick update may introduce faults and random errors in the calculated timestamp (uncertainty in local time \textbf{\texttt{A3}}). For a successful attack, the adversary must predict when these computation are going to take place. An attacker using a physical interface for the attack~\cite{hardware-voltage-pillager}, can do so by monitoring the timer interrupt pin on the SoC. Finally, note that these computational faults can also be induced using other methods such as laser fault injection and voltage glitches using physical probes~\cite{hardware-plundervolt, hardware-V0ltpawn}~\footnote{Research has demonstrated that the under-volting attacks are also possible through a software interface only~\cite{hardware-voltage-pillager} and a remote adversary with \textit{privileged access} can also launch this attack.}.

\subsection{Design Limitations}
The trade-off between security and performance often de-prioritizes the former in system design, potentially exposing timing mechanisms to exploitation due to design oversights or flawed assumptions. Exploiting these design flaws often require the adversary to have \textit{privileged access} to the system software.

\noindent\textbf{\texttt{I03.} Exploiting Energy Management Mechanisms.} The prevalent Dynamic Voltage and Frequency Scaling (DVFS) mechanisms in CPS for energy efficiency inadvertently introduces a vulnerability. Typically, a dedicated system software module oversees energy management and controls the DVFS interface. However, any software with escalated privileges can access this interface. This also includes an attacker who gained escalated privileges by exploiting system software bugs (table~\ref{tab:cve-stats}). Such an adversary uses the DVFS interface to alter the system frequency without alerting the \textit{local clock}. Being unaware of the change, system software relies on an outdated clock frequency to convert time from clock cycles to wall clock time~\footnote{As detailed in section~\ref{subsec:local-clocks}, system software uses clock frequency value to translate hardware time measurements from cycle count to seconds}. This attack introduces \textit{time warping} (\textbf{\texttt{A2}}) to the \textit{local clock} and the extent of this warping can be \textit{precisely controlled} by the \textit{stealthy} adversary.

\noindent\textbf{\texttt{I04.} Re-configurable timing counters.} 
System software uses hardware counters such as Intel's TSC~\cite{intel-tsc} and ARM's CNTVCT~\cite{arm-cntvct} for updating the \textit{local clock} (section~\ref{subsec:local-clocks}--Time Keeping). In modern systems, these counters are write-protected and does not allow the operating system to manipulate them directly. However, with the implementation of virtualization extensions, both architectures introduce an offset register~\cite{arm-cntvoff, intel-tsc-offset} shown in figure~\ref{fig:software-stime-stacks}. This register, originally designed to allow virtualization software to emulate counters for multiple guests, is writable. A malicious agent with \textit{privileged execution} access can exploit this offset register to manipulate the system's time view and induce \textit{time travel} (\textbf{\texttt{A1}}) in the local clock. Older systems such Intel processors designed before $2011$~\cite{intel-variant} provide writable counters and are even more susceptible to adversarial attacks as it allows the adversary to launch without having to rely on virtualization extensions which may or may not be enabled by default.
\section{Software Manager and Interface}
\label{sec:software}

The software interface of \name{} wraps around both the Xilinx XRT environment and the compression functionalities, to provide a familiar and transparent interface to acceleration on both compressed and uncompressed data.

\subsection{Programming Interface}

The \name{} software interface provides a familiar, XRT-style wrapper around accelerated kernel calls.
Let's consider a kernel \texttt{accel} with two device-side buffers ``\texttt{a}'' and ``\texttt{b}'' as parameters, where ``\texttt{a}'' is a genome data source.
If ``\texttt{a}'' comes from a \name{}-managed compressed source, subsections of the file can be transmitted as part of the kernel execution simply by setting \texttt{a.source = \name{}.compressed\_file(1); a.offset = N; a.size = M;}, and then passing it to the kernel via \texttt{accel.kernel(a,b)}.

The current version of the \name{} decompressor only supports stream sources, meaning transmitted buffers are decompressed immediately and provided to the kernel over a FIFO interface.
If the kernel requires random access into a buffer, it must copy the decompressed data to a separate in-memory buffer.
We are working on building an on-device index for compressed data to overcome this limitation.


\subsection{Index Structure}

One of the most important features of compressed data management is random access.
Downstream processing, such as graph construction based on reads, requires each read to be accessed separately, and the fundamental workload of reference-based alignment requires random access capabilities into the reference.

\name{} achieves this feature using a B+tree data structure, as hinted at in Figure~\ref{fig:overall}.
During compression, a B+tree is constructed with the file-internal offset as the key.
The unit element of insertion and lookup is a chunk of compressed data, which shares a single 32-bit header.
While the size of a decompressed chunk can be up to 128 bytes large in our prototype, we discovered that the random access requirement is typically coarser than this, in the unit of long reads.


\subsection{Reference Genome Encoding}

The software environment also must store the reference genome used for compression, because the compression process is split between hardware and software, as described in Section~\ref{sec:compression_arch}.
As described in Section~\ref{sec:compression_arch}, the hardware portion of compression is primarily responsible for calculating the hash function.
On the other hand, the software must perform a lookup into the cuckoo hash table, which is too large to store comfortably in the accelerator, and discover the correct cuckoo hash slot (if any) by comparing the target k-mer against up to four k-mer substrings sampled from the reference genome according to the cuckoo hash lookup.

Because \name{} stores the reference in a compact 2-bit encoded format, comparing k-mers can cause performance overhead due to sub-byte addressing.
For example, if a k-mer starts from offset 7, with two bit encoding, the 7th base starts in bit 6 of byte 2.
Comparing k-mers in this setting requires repeated shifting operations, which can have performance overhead.
To overcome this, \name{} stores four copies of the reference, each shifted from the original by 2 to 6 bits.
During compression, one of the copies are chosen according to the ``offset mod 4'' value, and then fast \texttt{memcmp} can be used since the string would be aligned along byte boundaries.

Both alternatives to storing four binary copies: Storing byte-alined ASCII files, and performing shifts on the fly, showed performance degradation by 4$\times$ on average.


\section{Network Issues} \label{sec:network-issues}
This section delves into the network layer's vulnerabilities, pivotal for synchronizing time across digital systems. Such synchronization is vital for applications ranging from digital payments to industrial automation. Yet, it faces threats from \textit{attackers controlling network devices} (on-path attacker) or \textit{possessing privileged access to a victim's local network stack} (off-path attacker).

\subsection{Limited Use of Authentication Mechanisms} Cryptography techniques, used by protocols like NTP~\cite{ntpv4-rfc} and PTP~\cite{ptp-std-doc}, play a critical role in ensuring data integrity and origin authentication of the time-sync traffic, thwarting man-in-the-middle (MITM) attacks. Yet, several issues persist regarding the adoption of these methods making time-sync protocols vulnerable to attacks.


\noindent\textbf{\texttt{I10.} False packet injection.} A MITM adversary can impersonate a genuine time server and send false time-sync packets to the target. These attacks may result from weak assumptions underlying the authentication mechanism adopted by the time-sync protocol. For instance, the reliance of NTP's broadcast mode authentication protocol TESLA~\cite{tesla-cryptography} (also used by PTP~\cite{ptp-std-doc}) on loosely synchronized devices creates a circular dependency between authentication and time-sync~\cite{ntp-replay-drop-attack}, rendering the former useless. Moreover, infiltration of malicious servers in the pool of legitimate time servers is  a genuine concern~\cite{shark-ntp-pool, devil-time-origin}. It is because cryptographic authentication only protects against a MITM attacker and the malicious servers render it ineffective. This allows Kwon et. al., to use a handful of malicious time servers, injected to the NTP pool~\cite{ntpd-pool-project}, to disrupt time-sync clients spread over entire countries~\cite{shark-ntp-pool}. Despite their shortcomings, authentication techniques make packet injection attacks harder. However, the adoption of these mechanisms is not universal. For instance, Huygens~\cite{huygens}, RBS~\cite{Elson2003RBS}, FTSP~\cite{ftsp-2004}, TPSN~\cite{tpsn-2003} do not implement any origin authentication mechanisms and have no protection against packet injection. The severity of the issue is evident from the fact that \textit{RBS, FTSP and TPSN} are among the most cited protocols for time-sync in sensor networks. In contrast, secure time synchronization protocols such as the one introduced by Ganeriwal et. al.~\cite{net-sync-wsn-sec-prot} has received an order of magnitude fewer citations (see table~\ref{tab:time-sync-wsn-citations}). Packet injection is one of the most potent attacks against time-sync protocols and could be used to induce \textit{time travel, warping or just increased uncertainty} (\textbf{\texttt{A1-3}}) in the victim's view of time.

\begin{table}
\scriptsize
\centering
\begin{tabular}{ | c | c | c | c | }
 \hline
  \textbf{Protocol} & \textbf{Authentication} & \textbf{Date Published} & \textbf{Citations} \\
 \hline
 \hline
  RBS~\cite{Elson2003RBS}  & \textit{No} & $2003$ & $3927$   \\
 \hline
   TPSN~\cite{tpsn-2003}  & \textit{No} & $2003$ & $3206$   \\
 \hline
   FTSP~\cite{ftsp-2004}  & \textit{No} & $2004$ & $3052$   \\
 \hline
 \hline
   Secure Time-Sync~\cite{Elson2003RBS}  & \textit{Yes} & $2005$ & $278$   \\
 \hline
\end{tabular}
\caption{One of the earliest time-sync protocols proposed for wireless sensor networks (WSNs). The protocols (RBS, TPSN and FTSP) that do not incorporate authentication mechanisms have received an order of magnitude more citations than the protocol (STS) that make use of cryptography mechanisms. \textit{Source: Google Scholars as of Jan 22, 2024.}}
\label{tab:time-sync-wsn-citations}
\end{table}

\noindent\textbf{\texttt{I11.} Packet modification.} Correct implementation of authentication protocols prevents false packet injection but may not prevent against packet modification. This is best exemplified by PTP, which makes use of authentication~\cite{ptp-std-doc} to protects the PTP packets except the correction field of the packet header. This field allows each network node to update correction field with the packet processing delay. PTP uses this information to achieve better time-sync accuracy by eliminating the variable network delays~\cite{net-sync-ptp-covert-channel}. However, a MITM attacker (on-path or off-path) can add incorrect information to this field and manipulate the PTP client. Jacobs et, al., use this channel to introduce \textit{significant offsets} (\textbf{\texttt{A3}}) to the victim device while \textit{avoiding detection}. They could also induce the victim device to change its clock frequency (\textbf{\textit{A2}}), resulting in an even larger time deviation from the time server~\cite{net-sync-ptp-covert-channel}. We note that this attack is not PTP specific, and any time-sync protocol seeking network delay information may be subject to this attack. Finally, we also note that this technique is less sophisticated as it does not require by-passing authentication requires.

\noindent\textbf{\texttt{I12.} Packet replay.} Authentication issues discussed in $I10$ can also result in replay attacks. In this attack, the adversary repeatedly sends one or a sequence of pre-recorded time server packets to the victim. Packet replay attacks have been successfully demonstrated against NTP broadcast mode~\cite{ntp-replay-drop-attack}. Malhotra et. al. exploited limitations in existing NTP client implementations to keep the victim stuck at a single point in time (\textbf{\texttt{A1}}). They point out that the one-way nature of the time-sync traffic (NTP broadcast mode) enables this attack. It implies that other one-way time synchronization protocols such as RBS~\cite{Elson2003RBS} may also be susceptible to this attack.

\noindent\textbf{\texttt{I13.} Spoofing Wireless Timing Signals.} Time-sync protocols like GPS, ROCS~\cite{ROCS-FM-Beacons}, Syntonizor~\cite{Syntonizor-AC-powerlines} and WizSync~\cite{WizSync-Wifi-Beacons} work using a periodic wireless timing signal that is transmitted directly from the timing source(s) to the clients i.e. over a single hop. These protocols lack authentication mechanisms allowing adversaries to spoof timing signals. This attack is the equivalent to packet manipulation attack on packet exchange based protocols (NTP~\cite{nts-rfc}, PTP~\cite{ptp-std-doc}, FTSP~\cite{ftsp-2004} etc.). Similar to the packet manipulation attacks, an external adversary mimics a trusted timing source but transmits incorrect timing information. It does so by generating a powerful spoof signal, using antenna(s), that can overpower the legitimate signal. Such an attacker often stays \textit{stealthy} while introducing uncertainty in the victim's local clocks~\cite{gps-spoofing-fundamentals} (\textbf{\texttt{A3}}). Satellite based global positing systems (GPS) is a typical target of this attack~\cite{gps-spoofing-21}. However, other time-sync protocols in this category (e.g., ROCS,~\cite{ROCS-FM-Beacons}, WizSync~\cite{WizSync-Wifi-Beacons} and Syntonizor~\cite{Syntonizor-AC-powerlines} etc.) haven't seen significant spoofing attacks due to their limited application. Nevertheless, signal spoofing remains a viable attack option for a motivated adversary.

\subsection{Availability Issues}
Beyond modifying timing packets, time-sync is also affected by just delaying the transmission of the timing information (as discussed in section~\ref{subsec:case-studies}). An adversary may leverage this observation and use unpredictable delays to add errors to the time-sync process or it  may outright block time-sync traffic headed towards the victim. 

\noindent\textbf{\texttt{I14.} Packet delay.} Time synchronization protocols determine the time offset between the server and the client by exchanging packets over the network. These network packets experience delays causing uncertainty in the exchanged timing information and the corresponding offset calculations (see section~\ref{subsec:case-studies}). Time-sync protocols rely heavily on precise network delay measurements to remove this uncertainty in the offset estimations. NTP~\cite{ntpv4-rfc} solves this challenge by measuring round trip times (RTTs) and computes network delay as half of the RTT, assuming symmetric delays~\cite{rfc1305}. On the other hand PTP measures the network delays by mandating each processing node to update the PTP packets with its resident delay (see $I11$). While effective under normal network conditions, these delay estimation mechanisms are not robust to adversarial delays. A malicious network node may introduce additional network delays~\footnote{In case of NTP, the server-bound and client-bound packets are delayed by different duration while for PTP the adversary would not update PTP packets with its resident delay} to degrade the synchronization performance. For instance, Annesi et. al. show that delay attacks against PTP can induce errors of several milliseconds, accumulating over time to even larger values under a sustained attacks~\cite{ptp-futile-encryption} (\textbf{\texttt{A2}}). However, vulnerability to delay attacks extend beyond NTP and PTP; virtually all time-sync protocols are susceptible to these attacks.

\noindent\textbf{\texttt{I15.} Packet drop.} 
Intercepting and dropping time-sync packets is a simple yet effective MiTM attack that desynchronizes the victim device from its time server. Facing this \textit{denial-of-service} attack, the victim solely relies on its \textit{local clock} which diverges away from the server time (\textbf{\texttt{A3}}) dictated by the stability of the victim's time source. For low-end systems using inexpensive quartz crystals, the time difference may accumulate to several minutes per day. In contrast, devices using more stable oven-controlled quartz oscillators may experience deviations of only a few seconds in the same period. However, despite its effectiveness, the victim can deduce potential instances of this attack, with relative ease, from sudden unavailability of time-server.

\noindent\textbf{\texttt{I16.} Blocking Wireless Timing Signals.} For single-hop wireless time synchronization (GPS~\cite{gps-spoofing-fundamentals}, ROCS~\cite{ROCS-FM-Beacons}, WizSync~\cite{WizSync-Wifi-Beacons} etc.), denial of service attack takes the form of blocking the wireless timing signal. An adversary achieves this by generating high powered noise in the frequency band used by the wireless timing signal. It requires physical proximity to the target and signal transmission equipment, raising the cost of this attack. Nevertheless, GPS signal blocking techniques have been studied extensively~\cite{gps-jamming-overview} due to ubiquitous use of GPS by defense and civil infrastructure. In principle, other single-hop wireless protocols such as Syntonizor~\cite{Syntonizor-AC-powerlines} and ROC~\cite{ROCS-FM-Beacons} are also vulnerable to these attacks, even though no such attack against them is known.

\subsection{Implementation Issues} In addition to the the communication medium, the end-points of this channel i.e. the applications implementing the time-sync protocol themselves represent an attack surface.

\noindent\textbf{\texttt{I17.} Untrusted time synchronization software.} Applications implementing time-sync protocols may harbor security vulnerabilities of their own. For instance, CVE database lists 98 vulnerabilities, discovered over the years, in the NTP application developed by \textit{NTP.org}~\cite{ntp-cve-details}. This application is used by both the time-sync clients and servers,~\footnote{It is recommended for servers joining the NTP pool project~\cite{ntpd-pool-project}.} and can be exploited by an adversary with access to \textit{privileged execution} on the victim device or \textit{a network connection to the NTP application}. An attack exploiting client side application vulnerability would only affect a single machine, however, the server side exploit would affect time alignment at all of its clients. Further, these attacks may cause the target applications to crash pausing the time-sync service or may just degrade time-sync performance (\textbf{\texttt{A3}}) over longer periods. It is worth pointing out time-sync applications executing in the privileged context present an even bigger risk, as any vulnerability in them could compromise the system beyond time-sync service.
\section{Defenses Against Timing Attacks}
This section examines defense mechanisms aimed at mitigating the vulnerabilities faced by the timing stack's three layers. Table~\ref{tab:defense-examples} provides representative examples of this work and the extent to which it addresses timing stack issues. In this analysis, we focus exclusively on system-based solutions for safeguarding timing architectures, deferring the discussion of theory-based approaches for the next section.

\begin{table*}[t]
\footnotesize
\centering
\begin{tabular}{ p{4.75cm}  p{1cm}  p{1.25cm}  p{1.25cm}  p{1.25cm}  p{1.5cm}  p{1.5cm}  p{1.5cm}  }
 \multicolumn{1}{c}{} & \multicolumn{2}{c}{Hardware Issues} & \multicolumn{2}{c}{Software Issues} & \multicolumn{3}{c}{Network Issues}\\
 \cmidrule(lr){2-3} \cmidrule(lr){4-5} \cmidrule(lr){6-8}
    & \textit{Side Ch.} & \textit{Flaw. Des.} & \textit{Soft. Bugs} & \textit{TEE Issues} & \textit{Auth. Issues} & \textit{Avail. Issues} & \textit{Impl. Issues} \\
 \hline
 Wei et. al.~\cite{lfi-ro-based} & \halfcirc & \emptycirc & \emptycirc & \emptycirc & \emptycirc & \emptycirc & \emptycirc \\
 Arm Generic Timer~\cite{arm-generic-timer} & \emptycirc & \halfcirc & \emptycirc & \emptycirc & \emptycirc & \emptycirc & \emptycirc \\
 TPM Counters~\cite{ftpm} & \emptycirc & \fullcirc & \emptycirc & \emptycirc & \emptycirc & \emptycirc & \emptycirc \\
 \hline
 Timeseal~\cite{time-stack-timeseal} & \emptycirc & \emptycirc & \halfcirc & \halfcirc & \emptycirc & \emptycirc & \emptycirc \\
 T3E~\cite{trusted-time-t3e} & \emptycirc & \emptycirc & \halfcirc & \halfcirc & \emptycirc & \emptycirc & \emptycirc \\
Scone~\cite{sandbox-scone} & \emptycirc & \emptycirc & \emptycirc & \halfcirc & \emptycirc & \emptycirc & \emptycirc \\
SeCloak~\cite{sandbox-secloak} & \emptycirc & \emptycirc & \emptycirc & \halfcirc & \emptycirc & \emptycirc & \emptycirc \\
\hline
Cryptographic Communications~\cite{net-sync-ptp-sec, ntpv4-rfc} & \emptycirc & \emptycirc & \emptycirc & \emptycirc & \halfcirc & \emptycirc & \emptycirc \\
Chronos~\cite{net-sync-chronos} & \emptycirc & \emptycirc & \emptycirc & \emptycirc & \halfcirc & \fullcirc & \emptycirc \\
Semperfi et. al.~\cite{gps-anti-spoofing-semperfi} & \emptycirc & \emptycirc & \emptycirc & \emptycirc & \halfcirc & \fullcirc & \emptycirc \\
\hline
\end{tabular}
\caption{Examples of representative papers that propose mitigation for timing stack issues (\textit{Ixx}). 
A full circle indicates the research paper mitigates all issues in the category, a half circle indicates that some of the issues in a category are addressed and an empty circle signifies the lack of proposed mitigation for the given category. Note that implementation issues ($I17$) are not addressed by system-based approach as they arise from errors in execution of this approach itself.}
\label{tab:defense-examples}
\end{table*}

\subsection{Securing the Hardware}
We begin by highlighting design solutions that address issues arising from the physical side channel and the design limitations of the timing circuitry.

\noindent\textbf{\texttt{D01.} Laser Fault Injection Countermeasures.} In response to the rising threat of laser fault injection (LFI), a significant body of research has focused on detection techniques~\cite{lfi-multi-spot, lfi-tdc, lfi-ro-based}. These efforts concentrate on identifying lasers incident on the SoC and provide countermeasures against laser-based computational faults ($I02$). However, these methods may not detect laser-based attacks on crystal oscillators which are external to the SoC. Nevertheless, they offer a promising starting point for developing countermeasures against laser attacks on crystal oscillators ($I01$). For instance, He et al. introduced a ring oscillator-based watchdog that analyzes clock signal irregularities to detect LFI attacks~\cite{lfi-ro-based}. Such solutions can be further developed to detect attacks on the external oscillators by analyzing the analog clock signal generated by them.

\noindent\textbf{\texttt{D02.} Monotonic and Fixed Frequency Counters.} Many contemporary System-on-Chips (SoCs) incorporate specialized counters that are monotonic i.e. consistently counting in a single direction and immune to resets. Further, the frequency of these counters remains independent of the DVFS mechanism, thwarting any attempt by a privileged adversary to exploit energy management interfaces for attacks on the timing stack ($I03$). ARM's generic timer~\cite{arm-generic-timer} and Intel's TSC are notable examples~\footnote{when virtualization extensions are disabled.}~\cite{intel-tsc}. Monotonic counters, uninfluenced by DVFS, are also prevalent in Trusted Platform Modules (TPMs) embedded in modern SoCs\cite{ftpm}. However, TPMs, situated outside processor chips, suffer from substantial access latency, constraining their utility~\cite{time-stack-timeseal}. These fixed rate monotonic counters present a substantial advance towards fixing hardware design issues that could enable timing attacks. However, despite these advancements, the challenge of counter manipulation still persists on legacy hardware and modern systems that incorporate virtualization extensions (see $I04$). 

\subsection{Software Defenses}
This subsection delves into solutions designed to protect the integrity of \textit{local clock}'s data. 

\noindent\textbf{\texttt{D03.} Trusted Timing Services.} Development efforts have been concentrated on integrating trusted timing services within TEEs to counter vulnerabilities in system software and device drivers ($I05, I06$). For instance, ARM's TrustZone provides a privileged TEE with direct access to a secure counter and timer~\cite{arm-generic-timer}, enabling TEE software to maintain a \textit{trusted local clock}. Intel SGX, a user-space TEE, benefits from solutions like Timeseal~\cite{time-stack-timeseal}, which secures the enclave's trusted timing API against delay attacks, and T3E~\cite{trusted-time-t3e}, utilizing secure TPM counters to provide trusted timing within the SGX enclave. These approaches for constructing a trusted timing stack are not confined to Intel and ARM-based TEEs but are extensible to other TEE architectures. Despite their significance, these trusted timing solutions exhibit several notable limitations: (i) the application code requiring trusted time must execute within the TEE's isolated environments, which enlarges the Trusted Computing Base (TCB), thereby increasing the security risk to the TEE software\footnote{TCB refers to the code and data that reside inside a TEE.}; and (ii) crucially, these trusted timing solutions do not incorporate secure time-synchronization services.

\noindent\textbf{\texttt{D04.} Trusted I/O.} Researchers have proposed several solutions to enable user-space TEEs' (Figure~\ref{fig:userspace-tee}) direct access to I/O devices ($I08$). One such effort, Aurora~\cite{sandbox-aurora}, allows Intel SGX to access high-resolution counters in the hardware, among other peripherals. SGX enclave's access to these counters can improve trusted time services such as Timeseal~\cite{time-stack-timeseal} when integrated with it. Other initiatives like Scone~\cite{sandbox-scone} and SGXIO~\cite{sgxio} provide secure network I/O to the SGX enclave, essential for synchronizing trusted time stacks inside TEEs with network time. However, they only safeguard the integrity and confidentiality of network packets, falling short of preventing delay attacks by untrusted system software.

\subsection{Secure Time Synchronization}
Time synchronization security has garnered significant focus within timing security research. Efforts in this domain have concentrated on enhancing various aspects of time-sync protocols to fortify them against adversarial actions.

\noindent\textbf{\texttt{D05.} Cryptographic Communications.} Time-sync protocols can enhance their security against packet manipulation attacks ($I10$, $I11$) by utilizing authentication and encryption, provided they adhere to the following conditions: (i) cryptographic functions should fully protect network packets~\cite{net-sync-ptp-covert-channel}, and (ii) their operation should not require time synchronization as a prerequisite~\cite{ntp-replay-drop-attack}. For example, the authentication and encryption mechanisms introduced by the NTS standard (RFC8915~\cite{nts-rfc}) for NTP's client-server mode fulfill both criteria. Conversely, PTP's authentication mechanisms fail to meet these conditions because: (i) the correction field in the PTP packet header remains unprotected by authentication, and (ii) the recommended authentication protocol, TESLA, requires devices to be loosely synchronized. Despite these shortcomings, such security measures constitute the primary defense against network adversaries and should be implemented by time-sync protocols to either partially or completely mitigate packet manipulation attacks. It is important to note that packet manipulation by malicious time servers~\cite{shark-ntp-pool} remains feasible and necessitates further defense mechanisms.

\noindent\textbf{\texttt{D06.} Multipath Time Transfer.} Delay attacks $I14$ cannot be mitigated using cryptographic mechanisms and must be prevented using other techniques. Using multiple paths for synchronization between two systems is one strategy to mitigate these attacks. This approach forces the attacker to identify and introduce delay along all the time-sync paths which is a significantly more challenging task. Mizrahi et al. propose a game-theoretic model for such a multipath time synchronization scheme~\cite{multi-path-game-theory}. Similarly, Chronos~\cite{net-sync-chronos} employs multi-path synchronization by querying reference time from various NTP servers. This approach is not only effective in mitigating delay attacks $I14$ but it also mitigates packet drop $I15$ attacks. However, the multi-path approach is ineffective against a malicious device that functions as bottleneck on the path between the client and the time server(s).

\noindent\textbf{\texttt{D07.} Algorithmic Updates.} Recent research has focused on algorithmic approaches for mitigating delay ($I14$) and replay ($I12$) attacks. For instance, Fatima et al.\cite{net-sync-feedforward} present a feedforward clock model, for PTP, along with an algorithm proficient in detecting delay-free packets. They remove rest of the packets that were potentially delayed or replayed by an adversary before estimating time-sync parameters (offset and skew). Likewise, Chronos\cite{net-sync-chronos} reinforces the multi-path time-sync approach using a byzantine fault tolerance-based algorithm for the random selection of time servers in each synchronization round. Adopting these algorithmic updates will enhance time-sync protocol's resiliency against delay attacks, but it does not completely prevent performance degradation~\cite{net-sync-feedforward}.

\noindent\textbf{\texttt{D08.} GPS Anti-jamming \& spoofing.} The prevalence of GPS jamming ($I16$) and spoofing ($I13$) attacks against critical navigation systems have motivated a large number of studies. These works have demonstrated the ability to acquire weak GPS signals amidst jamming, allowing for an average positioning error of 16m~\cite{gps-anti-jamming-post-correlation, gps-anti-jamming-post-wavelet}. Addressing GPS spoofing, Khalajmehrabadi et al. present detection techniques that estimate the extent of the spoofing signal, empowering devices to take necessary mitigation measures~\cite{gps-anti-spoofing-tsarm}. Building on this work, Lee et al. introduced techniques to prevent GPS spoofing for static receivers~\cite{gps-anti-spoofing-static}. And Semperfi et. al.~\cite{gps-anti-spoofing-semperfi} developed anti-spoofing technique for mobile GPS receivers, such as UAVs, enhancing robustness for diverse navigation systems. Location and time information in GPS signals is tightly coupled. It means that these techniques mitigate both location and time-sync errors under adversarial conditions.
\section{Theoretical Tools for Securing Time}
In this section, we discuss research work that employs theoretical tools to secure the timing stack. In this context, theoretical tools are employed for three distinct tasks: i) establish properties of a system model, ii) proving correctness of a system design i.e. it aligns with the stated goals and iii) verify software implementation of a given time-stack component. 

\subsection{Hardware}
The timing vulnerabilities in the hardware layer primarily result from either physical side channels ($I01$, $I02$) or design limitations ($I03$, $I04$).
Traditional formal verification tools cannot address these issues because mitigating them requires either the addition of new components or changing the existing designs. However, these new designs may be evaluated using standard formal verification tools~\cite{formal-verification-intel}.

\subsection{Software}
Timekeeping functionalities, integrated within broader system software like operating systems or hypervisors, are prone to inherent software vulnerabilities ($I05, I06, I07$). Although formal verification tools offer a means to analyze these systems for potential flaws, the large code bases and complex interaction among various subsystems of the system software make it infeasible to verify them. Despite these challenges, advancements in formal verification techniques have enabled the verification of specific OS components~\cite{formal-verification-eBPF} and hypervisors~\cite{formal-verification-hypervisor-arm, formal-verification-hypervisor-memory, formal-verification-kvm}. Employing these verification tools to assess the security aspects of system software can diminish the timekeeping software's vulnerability exposure ($I05, I06, I07$). Further, timing subsystems may also contain vulnerabilities originating from incorrect implementations. To enhance their security, it is crucial to apply the latest formal verification tools to verify the correctness of timing subsystems including trusted timing services such as Timeseal~\cite{time-stack-timeseal} and T3E~\cite{trusted-time-t3e}. As far as our knowledge extends, applying formal verification tools to timekeeping software remains an open area of research.

\subsection{Network} \label{subsec:network-theory}
The use of theoretical tool to establish trust in the timing stacks have almost exclusively focused on its network component i.e. time synchronization. This literature has focused on following lines of work:

\noindent\textbf{\texttt{T01.} Establishing Secure Time Synchronization Requirements.} 
Theorem proving tools have been used to establish requirements for secure clock synchronization. Narula et al.\cite{net-sync-gps-sec-transfer} constructed formal models for one-way and two-way time transfer, assuming a line-of-sight link between the systems and a threat model with a MITM adversary. They present proves for i) one-way time transfer's inherent susceptibility to delay attacks (as discussed in $I12$) and ii) essential requirements for a two-way secure time-synchronization protocol. Building upon this work, they study two-way time synchronization over a multi-hop network where systems at the both ends implement cryptography~\cite{net-sync-gps-sec-sync}. They put forward the prerequisites for a secure clock synchronization algorithm applicable to protocols like PTP~\cite{ptp-std-doc}. Among other requirements, they show that the timing packets must travel along the shortest path between the server and the client to completely prevent delay attacks. This has an important implication that \textit{delay attacks, over the network, cannot be prevented entirely if they do not guarantee shortest path traversal.} This is indeed the case of today's internet that employs TCP/IP stack for networking.

\begin{table*}[t]
\footnotesize
\centering
\begin{tabular}{p{2 cm} p{3.5cm}  p{0.5cm}  p{0.5cm}  p{0.5cm}  p{0.5cm}  p{0.5cm}  p{0.5cm}  p{0.5cm} p{0.5cm}  p{0.5cm}  p{0.5cm}  p{0.5cm}  p{0.5cm} }
 \multicolumn{1}{c}{} & \multicolumn{1}{c}{} & \multicolumn{8}{c}{Systems Approach} & \multicolumn{4}{c}{Theoretical Approach} \\
 \cmidrule(lr){3-10} \cmidrule(lr){11-14}
    & & \texttt{D01} & \texttt{D02} & \texttt{D03} & \texttt{D04} & \texttt{D05} & \texttt{D06} & \texttt{D07} & \texttt{D08} & \texttt{T01} & \texttt{T02} & \texttt{T03} & \texttt{T04} \\
 \hline
  Hardware Issues & Physical Side Channels & \halfcirc & \emptycirc & \emptycirc & \emptycirc & \emptycirc & \emptycirc & \emptycirc & \emptycirc & \emptycirc & \emptycirc & \emptycirc & \emptycirc \\
& Design Limitations & \emptycirc & \halfcirc & \emptycirc & \emptycirc & \emptycirc & \emptycirc & \emptycirc & \emptycirc & \emptycirc & \emptycirc & \emptycirc & \emptycirc \\
\hline
Software Issues & System Software Bugs & \emptycirc & \emptycirc & \halfcirc & \emptycirc & \emptycirc & \emptycirc & \emptycirc & \emptycirc & \emptycirc & \emptycirc & \emptycirc & \emptycirc \\
& TEE Limitations & \emptycirc & \emptycirc & \halfcirc & \fullcirc & \emptycirc & \emptycirc & \emptycirc & \emptycirc & \emptycirc & \emptycirc & \emptycirc & \emptycirc \\
\hline
 & Limited Crypto Adoption & \emptycirc & \emptycirc & \emptycirc & \emptycirc & \halfcirc & \emptycirc & \emptycirc & \halfcirc & \fullcirc & \halfcirc & \emptycirc & \emptycirc \\
Network Issues & Availability Issues & \emptycirc & \emptycirc & \emptycirc & \emptycirc & \emptycirc & \halfcirc & \halfcirc & \halfcirc & \fullcirc & \emptycirc & \emptycirc & \halfcirc \\
& Implementation Issues & \emptycirc & \emptycirc & \emptycirc & \emptycirc & \emptycirc & \emptycirc & \emptycirc & \emptycirc & \emptycirc & \emptycirc & \halfcirc & \emptycirc \\
\hline
\end{tabular}
\caption{Research contributions towards mitigating timing stack issues utilize both \textit{system-based} and \textit{theoretical} approaches. A full circle indicates that the defense technique mitigates all issues within the category, a half circle suggests that some of the issues in a category are addressed, and an empty circle signifies the given defense's lack of mitigation for issues in the specified category. While system-based defenses tackle attack surfaces across all three layers of the timing stack, theoretical solutions predominantly concentrate on the network component.}
\label{tab:system-v-defense}
\end{table*}

\noindent\textbf{\texttt{T02.} Proving Correctness of the Time-Sync Protocols.} Formal verification tools have been used to prove the correctness of fault-tolerant clock sync protocols. For instance, Schwier et. al.~\cite{theory-MechanicalVO} used protocol verification system (PVS) to verify a generalized time-sync protocol's correctness based on conditions established by Schneider~\cite{theory-schneider-conditons} for byzantine faults. Improving on this work, Barsotti et al., ~\cite{theory-schneider, theory-deductivetools} used deductive tools to prove correctness of fault-tolerant clock synchronization algorithms proposed by Lamport-Melliar\cite{clock-sync-fault-presence} and Lundlies-Lynch~\cite{clock-sync-fault-tolerant}. This research offers a promising direction for formal verification of time-sync protocols dealing with malicious faults.

Recent efforts regarding secure time synchronization using mathematical analysis have shifted focus to wireless sensor networks~\cite{theory-attack-resilient-pulse-coupled, theory-self-stablizing}. Most of these works assume a MITM attack model where few nodes in the network are compromised ($I10-I15$). Wang et al.\cite{theory-attack-resilient-pulse-coupled} introduced an attack-resilient pulse-coupled synchronization scheme for wireless sensor networks, deriving necessary conditions and analytically proving that it guarantees secure synchronization in the presence of a single malicious node. Another work by Hoepman et. al.\cite{theory-self-stablizing} presented a self-stabilizing clock synchronization algorithm, for wireless sensor networks. Their design is secure against pulse delay attacks by malicious nodes and they providing proofs for the correctness of their random beacon scheduling algorithm. These works demonstrate the potential of using theoretical tools for verify time-sync protocol designs.

\noindent\textbf{\texttt{T03.} Verification of Protocol Implementations.} Formal verification techniques can be used to verify the implementations of time-sync protocols ($I17$). This is demonstrated by Luca et al., who performed the automated verification of the gossip time-sync protocol's~\cite{model-checking-gossip} implementation using parameterized model checking. However, gossip is rather a simple protocol, and the methods employed for its verification do not readily extend to more complex protocols such as NTP, PTP etc~\cite{net-sync-openchallenges}. However, despite its limitations, partial verification of time-sync implementations using the existing formal verification methods can yield important security insights. In one such instance, Dieter et al.\cite{theory-nts-specs} performed (partial) formal verification of Network Time Security (NTS -- RFC8915) specifications~\cite{nts-rfc} and discovered two vulnerabilities in the analyzed version~\cite{theory-nts-formal-analysis}, which are currently being addressed. It shows that future research on enabling complete verification of widely used time-sync protocols would greatly contribute to their security.

\noindent\textbf{\texttt{T04.} Mitigating Attacks on Time Synchronization.} Beyond analyzing complete protocol design, mathematical tools have also been used to study specific time-sync attacks. This research primarily focuses on mitigating delay attacks against time-sync protocols by a network adversary. For instance, Mizrahi et al. proposed a multi-path time synchronization scheme designed to resist delay attacks by a man-in-the-middle attacker. They leverage game theory to provide proofs for the delay resiliency of their design~\cite{multi-path-game-theory}. Similarly, Anto et al. propose modifications to PTP aimed at mitigating delay attacks and formally verified the correctness of their proposed updates~\cite{theory-formal-attack-1588}. Likewise, Moussa et al. proposed extensions to the PTP protocol and formally proved the correctness of these protocol extensions~\cite{theory-smart-grids, theory-ptp-extension}. However, the proposed extensions are domain specific as they rely on redundant master clocks on power grid substations mandated by IEC 61850. Another work by Lisova et al., took a game-theoretic approach, modeling the interaction of a man-in-the-middle attacker introducing asymmetric delays to PTP packets and a network inspection system collecting clock offset information. Their work uses a game-theoretic tool to predict attacker strategies and develop mitigation mechanisms accordingly~\cite{theory-game-theory-1588}. While delay attacks have been a dominant subject of this research, defenses against other attacks would equally benefit from the use of theoretical tools.
This work identifies signal collapse as a critical bottleneck in one-shot neural network pruning. Performance loss in pruned networks is due to \textbf{signal collapse} in addition to the removal of critical parameters. We propose \textbf{REFLOW} (\textbf{Re}storing \textbf{F}low of \textbf{Low}-variance signals), a simple yet effective method that mitigates signal collapse without computationally expensive weight updates. By focusing on signal preservation, REFLOW highlights the importance of mitigating signal collapse in sparse networks and enables magnitude pruning to match or surpass state-of-the-art one-shot pruning methods such as CHITA, CBS, and WF.

REFLOW consistently achieves state-of-the-art accuracy across diverse architectures, restoring ResNeXt-101 from under 4.1\% to 78.9\% top-1 accuracy at 80\% sparsity on ImageNet. Its lightweight design makes it a practical solution for both research and deployment, delivering high-quality sparse models without the overhead of traditional approaches. These findings challenge the traditional emphasis on weight selection strategies and underscore the critical role of signal propagation for achieving high-quality sparse networks in the context of one-shot pruning.



\section*{Conclusion}
This paper aims to enhance our understanding of the computational complexity of computing various Shapley value variants. We found that for various ML models --- including decision trees, regression tree ensembles, weighted automata, and linear regression --- both local and global interventional and baseline SHAP can be computed in polynomial time under HMM modeled distributions. This extends popular algorithms, such as TreeSHAP, beyond their empirical distributional scope. We also establish strict complexity gaps between the various SHAP variants (baseline, interventional, and conditional) and prove the intractability of computing SHAP for tree ensembles and neural networks in simplified scenarios. Overall, we present SHAP as a versatile framework whose complexity depends on four key factors: \begin{inparaenum}[(i)] \item model type, \item SHAP variant, \item distribution modeling approach, \item and local vs. global explanations\end{inparaenum}. We believe this perspective provides deeper insight into the computational complexity of SHAP, paving the way for future work.




%We believe that our framework provides a more intricate understanding of SHAP computation complexity across different models, distributions, and variants, paving the way for further research.

Our work opens promising directions for future research. First, expanding our computational analysis to other SHAP-related metrics, such as asymmetric SHAP~\citep{frye20} and SAGE~\citep{covert2020understanding}, would be valuable. Additionally, we aim to explore more expressive distribution classes and relaxed assumptions beyond those in Section \ref{sec:tractable} while maintaining tractable SHAP computation. Finally, when exact computation is intractable (Section \ref{sec:intractable}), investigating the approximability of SHAP metrics through approximation and parameterized complexity theory~\citep{downey2012parameterized} is an important direction.

%Our work opens several promising avenues for future research on the computational properties of explainable AI methods, with a particular focus on SHAP. First, it would be interesting to broaden the computational analysis conducted in this work to include other popular SHAP-related metrics in the literature, such as asymmetric SHAP \cite{frye20} and SAGE \cite{covert2020understanding}. Also, in the future, we aim to explore more expressive distribution classes and relaxed distributional assumptions—extending beyond those examined in Section \ref{sec:tractable} —that still yield tractable SHAP computation. Finally, when exact computation proves intractable (Section \ref{sec:intractable}), it is worthwhile to theoretically investigate the question of the approximability of computing the SHAP metrics across various configurations, through the lens of approximation and parametrized complexity theory \cite{arora2009computational}.

%This paper aims to deepen our understanding of the computational complexity involved in obtaining different Shapley value variants. We found that for a variety of ML models, including decision trees, tree ensembles for regression, weighted automata, and linear regression models — computing both local and global interventional and baseline SHAP can be done in polynomial time when distributions are modeled by HMMs. This extends the distributional scope of popular algorithms like TreeSHAP, which is limited to empirical distributions. Additionally, we demonstrate a strict complexity gap between SHAP variants, showing that interventional and baseline SHAP can be strictly easier to compute than conditional SHAP. Despite these positive results, we uncovered intractability for various SHAP variants in neural networks and tree ensembles. Finally, we provided generalized complexity relations across SHAP variants. We believe that our framework offers a deeper understanding of the complexity involved in computing SHAP across various variants, models, distributions, as well as in both local and global computations, laying the groundwork for future research.

\bibliographystyle{plain}
\bibliography{main}
%%%%%%%%%%%%%%%%%%%%%%%%%%%%%%%%%%%%%%%%%%%%%%%%%%%%%%%%%%%%%%%%%%%%%%%%%%%%%%%%
\end{document}
%%%%%%%%%%%%%%%%%%%%%%%%%%%%%%%%%%%%%%%%%%%%%%%%%%%%%%%%%%%%%%%%%%%%%%%%%%%%%%%%