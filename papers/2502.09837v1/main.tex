\documentclass[letterpaper,twocolumn,10pt]{article}
\usepackage{usenix2019_v3}

\usepackage{tikz}
\usepackage{amsmath}

\usepackage{graphicx}
\usepackage{xcolor}
\usepackage{comment}
\usepackage{subcaption}
\usepackage[T1]{fontenc}
\usepackage{booktabs}
\usepackage{makecell}

\newcommand*\emptycirc[1][1ex]{\tikz\draw (0,0) circle (#1);} 
\newcommand*\halfcirc[1][1ex]{%
  \begin{tikzpicture}
  \draw[fill] (0,0)-- (90:#1) arc (90:270:#1) -- cycle ;
  \draw (0,0) circle (#1);
  \end{tikzpicture}}
\newcommand*\fullcirc[1][1ex]{\tikz\fill (0,0) circle (#1);} 
\newcommand*\circled[1]{\tikz[baseline=(char.base)]{
            \node[shape=circle,draw,inner sep=0.05pt] (char) {#1};}}

\newcommand\fatima[1]{\textcolor{orange}{(F: #1)}}
\newcommand\adeel[1]{\textcolor{olive}{(A: #1)}}

\microtypecontext{spacing=nonfrench}
\usepackage[switch]{lineno}

%-------------------------------------------------------------------------------
\begin{document}
%-------------------------------------------------------------------------------
\date{}

\title{\Large \bf SoK: State of the time: On Trustworthiness of Digital Clocks}

\author{
{\rm Adeel Nasrullah}\\
University of Massachusetts Amherst
\and
{\rm Fatima M. Anwar}\\
University of Massachusetts Amherst
}

\maketitle

%-------------------------------------------------------------------------------
\begin{abstract}
%-------------------------------------------------------------------------------
Despite the critical role of timing infrastructure in enabling essential services—from public key infrastructure and smart grids to autonomous navigation and high-frequency trading—modern timing stacks remain highly vulnerable to malicious attacks. These threats emerge due to several reasons, including inadequate security mechanisms, the timing architecture's unique vulnerability to delays, and implementation issues. In this paper, we aim to obtain a holistic understanding of the issues that make the timing stacks vulnerable to adversarial manipulations, what the challenges are in securing them, and what solutions can be borrowed from the research community to address them. To this end, we perform a systematic analysis of the security vulnerabilities of the timing stack. In doing so, we discover new attack surfaces, i.e., \textit{physical timing components and on-device timekeeping}, which are often overlooked by existing research that predominantly studies the security of time synchronization protocols. We also show that the emerging trusted timing architectures are flawed \& risk compromising wider system security, and propose an alternative design using \emph{hardware-software co-design}.
\end{abstract}

\section{Introduction}
\label{sec:introduction}
The business processes of organizations are experiencing ever-increasing complexity due to the large amount of data, high number of users, and high-tech devices involved \cite{martin2021pmopportunitieschallenges, beerepoot2023biggestbpmproblems}. This complexity may cause business processes to deviate from normal control flow due to unforeseen and disruptive anomalies \cite{adams2023proceddsriftdetection}. These control-flow anomalies manifest as unknown, skipped, and wrongly-ordered activities in the traces of event logs monitored from the execution of business processes \cite{ko2023adsystematicreview}. For the sake of clarity, let us consider an illustrative example of such anomalies. Figure \ref{FP_ANOMALIES} shows a so-called event log footprint, which captures the control flow relations of four activities of a hypothetical event log. In particular, this footprint captures the control-flow relations between activities \texttt{a}, \texttt{b}, \texttt{c} and \texttt{d}. These are the causal ($\rightarrow$) relation, concurrent ($\parallel$) relation, and other ($\#$) relations such as exclusivity or non-local dependency \cite{aalst2022pmhandbook}. In addition, on the right are six traces, of which five exhibit skipped, wrongly-ordered and unknown control-flow anomalies. For example, $\langle$\texttt{a b d}$\rangle$ has a skipped activity, which is \texttt{c}. Because of this skipped activity, the control-flow relation \texttt{b}$\,\#\,$\texttt{d} is violated, since \texttt{d} directly follows \texttt{b} in the anomalous trace.
\begin{figure}[!t]
\centering
\includegraphics[width=0.9\columnwidth]{images/FP_ANOMALIES.png}
\caption{An example event log footprint with six traces, of which five exhibit control-flow anomalies.}
\label{FP_ANOMALIES}
\end{figure}

\subsection{Control-flow anomaly detection}
Control-flow anomaly detection techniques aim to characterize the normal control flow from event logs and verify whether these deviations occur in new event logs \cite{ko2023adsystematicreview}. To develop control-flow anomaly detection techniques, \revision{process mining} has seen widespread adoption owing to process discovery and \revision{conformance checking}. On the one hand, process discovery is a set of algorithms that encode control-flow relations as a set of model elements and constraints according to a given modeling formalism \cite{aalst2022pmhandbook}; hereafter, we refer to the Petri net, a widespread modeling formalism. On the other hand, \revision{conformance checking} is an explainable set of algorithms that allows linking any deviations with the reference Petri net and providing the fitness measure, namely a measure of how much the Petri net fits the new event log \cite{aalst2022pmhandbook}. Many control-flow anomaly detection techniques based on \revision{conformance checking} (hereafter, \revision{conformance checking}-based techniques) use the fitness measure to determine whether an event log is anomalous \cite{bezerra2009pmad, bezerra2013adlogspais, myers2018icsadpm, pecchia2020applicationfailuresanalysispm}. 

The scientific literature also includes many \revision{conformance checking}-independent techniques for control-flow anomaly detection that combine specific types of trace encodings with machine/deep learning \cite{ko2023adsystematicreview, tavares2023pmtraceencoding}. Whereas these techniques are very effective, their explainability is challenging due to both the type of trace encoding employed and the machine/deep learning model used \cite{rawal2022trustworthyaiadvances,li2023explainablead}. Hence, in the following, we focus on the shortcomings of \revision{conformance checking}-based techniques to investigate whether it is possible to support the development of competitive control-flow anomaly detection techniques while maintaining the explainable nature of \revision{conformance checking}.
\begin{figure}[!t]
\centering
\includegraphics[width=\columnwidth]{images/HIGH_LEVEL_VIEW.png}
\caption{A high-level view of the proposed framework for combining \revision{process mining}-based feature extraction with dimensionality reduction for control-flow anomaly detection.}
\label{HIGH_LEVEL_VIEW}
\end{figure}

\subsection{Shortcomings of \revision{conformance checking}-based techniques}
Unfortunately, the detection effectiveness of \revision{conformance checking}-based techniques is affected by noisy data and low-quality Petri nets, which may be due to human errors in the modeling process or representational bias of process discovery algorithms \cite{bezerra2013adlogspais, pecchia2020applicationfailuresanalysispm, aalst2016pm}. Specifically, on the one hand, noisy data may introduce infrequent and deceptive control-flow relations that may result in inconsistent fitness measures, whereas, on the other hand, checking event logs against a low-quality Petri net could lead to an unreliable distribution of fitness measures. Nonetheless, such Petri nets can still be used as references to obtain insightful information for \revision{process mining}-based feature extraction, supporting the development of competitive and explainable \revision{conformance checking}-based techniques for control-flow anomaly detection despite the problems above. For example, a few works outline that token-based \revision{conformance checking} can be used for \revision{process mining}-based feature extraction to build tabular data and develop effective \revision{conformance checking}-based techniques for control-flow anomaly detection \cite{singh2022lapmsh, debenedictis2023dtadiiot}. However, to the best of our knowledge, the scientific literature lacks a structured proposal for \revision{process mining}-based feature extraction using the state-of-the-art \revision{conformance checking} variant, namely alignment-based \revision{conformance checking}.

\subsection{Contributions}
We propose a novel \revision{process mining}-based feature extraction approach with alignment-based \revision{conformance checking}. This variant aligns the deviating control flow with a reference Petri net; the resulting alignment can be inspected to extract additional statistics such as the number of times a given activity caused mismatches \cite{aalst2022pmhandbook}. We integrate this approach into a flexible and explainable framework for developing techniques for control-flow anomaly detection. The framework combines \revision{process mining}-based feature extraction and dimensionality reduction to handle high-dimensional feature sets, achieve detection effectiveness, and support explainability. Notably, in addition to our proposed \revision{process mining}-based feature extraction approach, the framework allows employing other approaches, enabling a fair comparison of multiple \revision{conformance checking}-based and \revision{conformance checking}-independent techniques for control-flow anomaly detection. Figure \ref{HIGH_LEVEL_VIEW} shows a high-level view of the framework. Business processes are monitored, and event logs obtained from the database of information systems. Subsequently, \revision{process mining}-based feature extraction is applied to these event logs and tabular data input to dimensionality reduction to identify control-flow anomalies. We apply several \revision{conformance checking}-based and \revision{conformance checking}-independent framework techniques to publicly available datasets, simulated data of a case study from railways, and real-world data of a case study from healthcare. We show that the framework techniques implementing our approach outperform the baseline \revision{conformance checking}-based techniques while maintaining the explainable nature of \revision{conformance checking}.

In summary, the contributions of this paper are as follows.
\begin{itemize}
    \item{
        A novel \revision{process mining}-based feature extraction approach to support the development of competitive and explainable \revision{conformance checking}-based techniques for control-flow anomaly detection.
    }
    \item{
        A flexible and explainable framework for developing techniques for control-flow anomaly detection using \revision{process mining}-based feature extraction and dimensionality reduction.
    }
    \item{
        Application to synthetic and real-world datasets of several \revision{conformance checking}-based and \revision{conformance checking}-independent framework techniques, evaluating their detection effectiveness and explainability.
    }
\end{itemize}

The rest of the paper is organized as follows.
\begin{itemize}
    \item Section \ref{sec:related_work} reviews the existing techniques for control-flow anomaly detection, categorizing them into \revision{conformance checking}-based and \revision{conformance checking}-independent techniques.
    \item Section \ref{sec:abccfe} provides the preliminaries of \revision{process mining} to establish the notation used throughout the paper, and delves into the details of the proposed \revision{process mining}-based feature extraction approach with alignment-based \revision{conformance checking}.
    \item Section \ref{sec:framework} describes the framework for developing \revision{conformance checking}-based and \revision{conformance checking}-independent techniques for control-flow anomaly detection that combine \revision{process mining}-based feature extraction and dimensionality reduction.
    \item Section \ref{sec:evaluation} presents the experiments conducted with multiple framework and baseline techniques using data from publicly available datasets and case studies.
    \item Section \ref{sec:conclusions} draws the conclusions and presents future work.
\end{itemize}
\section{Background}\label{sec:backgrnd}

\subsection{Cold Start Latency and Mitigation Techniques}

Traditional FaaS platforms mitigate cold starts through snapshotting, lightweight virtualization, and warm-state management. Snapshot-based methods like \textbf{REAP} and \textbf{Catalyzer} reduce initialization time by preloading or restoring container states but require significant memory and I/O resources, limiting scalability~\cite{dong_catalyzer_2020, ustiugov_benchmarking_2021}. Lightweight virtualization solutions, such as \textbf{Firecracker} microVMs, achieve fast startup times with strong isolation but depend on robust infrastructure, making them less adaptable to fluctuating workloads~\cite{agache_firecracker_2020}. Warm-state management techniques like \textbf{Faa\$T}~\cite{romero_faa_2021} and \textbf{Kraken}~\cite{vivek_kraken_2021} keep frequently invoked containers ready, balancing readiness and cost efficiency under predictable workloads but incurring overhead when demand is erratic~\cite{romero_faa_2021, vivek_kraken_2021}. While these methods perform well in resource-rich cloud environments, their resource intensity challenges applicability in edge settings.

\subsubsection{Edge FaaS Perspective}

In edge environments, cold start mitigation emphasizes lightweight designs, resource sharing, and hybrid task distribution. Lightweight execution environments like unikernels~\cite{edward_sock_2018} and \textbf{Firecracker}~\cite{agache_firecracker_2020}, as used by \textbf{TinyFaaS}~\cite{pfandzelter_tinyfaas_2020}, minimize resource usage and initialization delays but require careful orchestration to avoid resource contention. Function co-location, demonstrated by \textbf{Photons}~\cite{v_dukic_photons_2020}, reduces redundant initializations by sharing runtime resources among related functions, though this complicates isolation in multi-tenant setups~\cite{v_dukic_photons_2020}. Hybrid offloading frameworks like \textbf{GeoFaaS}~\cite{malekabbasi_geofaas_2024} balance edge-cloud workloads by offloading latency-tolerant tasks to the cloud and reserving edge resources for real-time operations, requiring reliable connectivity and efficient task management. These edge-specific strategies address cold starts effectively but introduce challenges in scalability and orchestration.

\subsection{Predictive Scaling and Caching Techniques}

Efficient resource allocation is vital for maintaining low latency and high availability in serverless platforms. Predictive scaling and caching techniques dynamically provision resources and reduce cold start latency by leveraging workload prediction and state retention.
Traditional FaaS platforms use predictive scaling and caching to optimize resources, employing techniques (OFC, FaasCache) to reduce cold starts. However, these methods rely on centralized orchestration and workload predictability, limiting their effectiveness in dynamic, resource-constrained edge environments.



\subsubsection{Edge FaaS Perspective}

Edge FaaS platforms adapt predictive scaling and caching techniques to constrain resources and heterogeneous environments. \textbf{EDGE-Cache}~\cite{kim_delay-aware_2022} uses traffic profiling to selectively retain high-priority functions, reducing memory overhead while maintaining readiness for frequent requests. Hybrid frameworks like \textbf{GeoFaaS}~\cite{malekabbasi_geofaas_2024} implement distributed caching to balance resources between edge and cloud nodes, enabling low-latency processing for critical tasks while offloading less critical workloads. Machine learning methods, such as clustering-based workload predictors~\cite{gao_machine_2020} and GRU-based models~\cite{guo_applying_2018}, enhance resource provisioning in edge systems by efficiently forecasting workload spikes. These innovations effectively address cold start challenges in edge environments, though their dependency on accurate predictions and robust orchestration poses scalability challenges.

\subsection{Decentralized Orchestration, Function Placement, and Scheduling}

Efficient orchestration in serverless platforms involves workload distribution, resource optimization, and performance assurance. While traditional FaaS platforms rely on centralized control, edge environments require decentralized and adaptive strategies to address unique challenges such as resource constraints and heterogeneous hardware.



\subsubsection{Edge FaaS Perspective}

Edge FaaS platforms adopt decentralized and adaptive orchestration frameworks to meet the demands of resource-constrained environments. Systems like \textbf{Wukong} distribute scheduling across edge nodes, enhancing data locality and scalability while reducing network latency. Lightweight frameworks such as \textbf{OpenWhisk Lite}~\cite{kravchenko_kpavelopenwhisk-light_2024} optimize resource allocation by decentralizing scheduling policies, minimizing cold starts and latency in edge setups~\cite{benjamin_wukong_2020}. Hybrid solutions like \textbf{OpenFaaS}~\cite{noauthor_openfaasfaas_2024} and \textbf{EdgeMatrix}~\cite{shen_edgematrix_2023} combine edge-cloud orchestration to balance resource utilization, retaining latency-sensitive functions at the edge while offloading non-critical workloads to the cloud. While these approaches improve flexibility, they face challenges in maintaining coordination and ensuring consistent performance across distributed nodes.


\section{Overview}

\revision{In this section, we first explain the foundational concept of Hausdorff distance-based penetration depth algorithms, which are essential for understanding our method (Sec.~\ref{sec:preliminary}).
We then provide a brief overview of our proposed RT-based penetration depth algorithm (Sec.~\ref{subsec:algo_overview}).}



\section{Preliminaries }
\label{sec:Preliminaries}

% Before we introduce our method, we first overview the important basics of 3D dynamic human modeling with Gaussian splatting. Then, we discuss the diffusion-based 3d generation techniques, and how they can be applied to human modeling.
% \ZY{I stopp here. TBC.}
% \subsection{Dynamic human modeling with Gaussian splatting}
\subsection{3D Gaussian Splatting}
3D Gaussian splatting~\cite{kerbl3Dgaussians} is an explicit scene representation that allows high-quality real-time rendering. The given scene is represented by a set of static 3D Gaussians, which are parameterized as follows: Gaussian center $x\in {\mathbb{R}^3}$, color $c\in {\mathbb{R}^3}$, opacity $\alpha\in {\mathbb{R}}$, spatial rotation in the form of quaternion $q\in {\mathbb{R}^4}$, and scaling factor $s\in {\mathbb{R}^3}$. Given these properties, the rendering process is represented as:
\begin{equation}
  I = Splatting(x, c, s, \alpha, q, r),
  \label{eq:splattingGA}
\end{equation}
where $I$ is the rendered image, $r$ is a set of query rays crossing the scene, and $Splatting(\cdot)$ is a differentiable rendering process. We refer readers to Kerbl et al.'s paper~\cite{kerbl3Dgaussians} for the details of Gaussian splatting. 



% \ZY{I would suggest move this part to the method part.}
% GaissianAvatar is a dynamic human generation model based on Gaussian splitting. Given a sequence of RGB images, this method utilizes fitted SMPLs and sampled points on its surface to obtain a pose-dependent feature map by a pose encoder. The pose-dependent features and a geometry feature are fed in a Gaussian decoder, which is employed to establish a functional mapping from the underlying geometry of the human form to diverse attributes of 3D Gaussians on the canonical surfaces. The parameter prediction process is articulated as follows:
% \begin{equation}
%   (\Delta x,c,s)=G_{\theta}(S+P),
%   \label{eq:gaussiandecoder}
% \end{equation}
%  where $G_{\theta}$ represents the Gaussian decoder, and $(S+P)$ is the multiplication of geometry feature S and pose feature P. Instead of optimizing all attributes of Gaussian, this decoder predicts 3D positional offset $\Delta{x} \in {\mathbb{R}^3}$, color $c\in\mathbb{R}^3$, and 3D scaling factor $ s\in\mathbb{R}^3$. To enhance geometry reconstruction accuracy, the opacity $\alpha$ and 3D rotation $q$ are set to fixed values of $1$ and $(1,0,0,0)$ respectively.
 
%  To render the canonical avatar in observation space, we seamlessly combine the Linear Blend Skinning function with the Gaussian Splatting~\cite{kerbl3Dgaussians} rendering process: 
% \begin{equation}
%   I_{\theta}=Splatting(x_o,Q,d),
%   \label{eq:splatting}
% \end{equation}
% \begin{equation}
%   x_o = T_{lbs}(x_c,p,w),
%   \label{eq:LBS}
% \end{equation}
% where $I_{\theta}$ represents the final rendered image, and the canonical Gaussian position $x_c$ is the sum of the initial position $x$ and the predicted offset $\Delta x$. The LBS function $T_{lbs}$ applies the SMPL skeleton pose $p$ and blending weights $w$ to deform $x_c$ into observation space as $x_o$. $Q$ denotes the remaining attributes of the Gaussians. With the rendering process, they can now reposition these canonical 3D Gaussians into the observation space.



\subsection{Score Distillation Sampling}
Score Distillation Sampling (SDS)~\cite{poole2022dreamfusion} builds a bridge between diffusion models and 3D representations. In SDS, the noised input is denoised in one time-step, and the difference between added noise and predicted noise is considered SDS loss, expressed as:

% \begin{equation}
%   \mathcal{L}_{SDS}(I_{\Phi}) \triangleq E_{t,\epsilon}[w(t)(\epsilon_{\phi}(z_t,y,t)-\epsilon)\frac{\partial I_{\Phi}}{\partial\Phi}],
%   \label{eq:SDSObserv}
% \end{equation}
\begin{equation}
    \mathcal{L}_{\text{SDS}}(I_{\Phi}) \triangleq \mathbb{E}_{t,\epsilon} \left[ w(t) \left( \epsilon_{\phi}(z_t, y, t) - \epsilon \right) \frac{\partial I_{\Phi}}{\partial \Phi} \right],
  \label{eq:SDSObservGA}
\end{equation}
where the input $I_{\Phi}$ represents a rendered image from a 3D representation, such as 3D Gaussians, with optimizable parameters $\Phi$. $\epsilon_{\phi}$ corresponds to the predicted noise of diffusion networks, which is produced by incorporating the noise image $z_t$ as input and conditioning it with a text or image $y$ at timestep $t$. The noise image $z_t$ is derived by introducing noise $\epsilon$ into $I_{\Phi}$ at timestep $t$. The loss is weighted by the diffusion scheduler $w(t)$. 
% \vspace{-3mm}

\subsection{Overview of the RTPD Algorithm}\label{subsec:algo_overview}
Fig.~\ref{fig:Overview} presents an overview of our RTPD algorithm.
It is grounded in the Hausdorff distance-based penetration depth calculation method (Sec.~\ref{sec:preliminary}).
%, similar to that of Tang et al.~\shortcite{SIG09HIST}.
The process consists of two primary phases: penetration surface extraction and Hausdorff distance calculation.
We leverage the RTX platform's capabilities to accelerate both of these steps.

\begin{figure*}[t]
    \centering
    \includegraphics[width=0.8\textwidth]{Image/overview.pdf}
    \caption{The overview of RT-based penetration depth calculation algorithm overview}
    \label{fig:Overview}
\end{figure*}

The penetration surface extraction phase focuses on identifying the overlapped region between two objects.
\revision{The penetration surface is defined as a set of polygons from one object, where at least one of its vertices lies within the other object. 
Note that in our work, we focus on triangles rather than general polygons, as they are processed most efficiently on the RTX platform.}
To facilitate this extraction, we introduce a ray-tracing-based \revision{Point-in-Polyhedron} test (RT-PIP), significantly accelerated through the use of RT cores (Sec.~\ref{sec:RT-PIP}).
This test capitalizes on the ray-surface intersection capabilities of the RTX platform.
%
Initially, a Geometry Acceleration Structure (GAS) is generated for each object, as required by the RTX platform.
The RT-PIP module takes the GAS of one object (e.g., $GAS_{A}$) and the point set of the other object (e.g., $P_{B}$).
It outputs a set of points (e.g., $P_{\partial B}$) representing the penetration region, indicating their location inside the opposing object.
Subsequently, a penetration surface (e.g., $\partial B$) is constructed using this point set (e.g., $P_{\partial B}$) (Sec.~\ref{subsec:surfaceGen}).
%
The generated penetration surfaces (e.g., $\partial A$ and $\partial B$) are then forwarded to the next step. 

The Hausdorff distance calculation phase utilizes the ray-surface intersection test of the RTX platform (Sec.~\ref{sec:RT-Hausdorff}) to compute the Hausdorff distance between two objects.
We introduce a novel Ray-Tracing-based Hausdorff DISTance algorithm, RT-HDIST.
It begins by generating GAS for the two penetration surfaces, $P_{\partial A}$ and $P_{\partial B}$, derived from the preceding step.
RT-HDIST processes the GAS of a penetration surface (e.g., $GAS_{\partial A}$) alongside the point set of the other penetration surface (e.g., $P_{\partial B}$) to compute the penetration depth between them.
The algorithm operates bidirectionally, considering both directions ($\partial A \to \partial B$ and $\partial B \to \partial A$).
The final penetration depth between the two objects, A and B, is determined by selecting the larger value from these two directional computations.

%In the Hausdorff distance calculation step, we compute the Hausdorff distance between given two objects using a ray-surface-intersection test. (Sec.~\ref{sec:RT-Hausdorff}) Initially, we construct the GAS for both $\partial A$ and $\partial B$ to utilize the RT-core effectively. The RT-based Hausdorff distance algorithms then determine the Hausdorff distance by processing the GAS of one object (e.g. $GAS_{\partial A}$) and set of the vertices of the other (e.g. $P_{\partial B}$). Following the Hausdorff distance definition (Eq.~\ref{equation:hausdorff_definition}), we compute the Hausdorff distance to both directions ($\partial A \to \partial B$) and ($\partial B \to \partial A$). As a result, the bigger one is the final Hausdorff distance, and also it is the penetration depth between input object $A$ and $B$.


%the proposed RT-based penetration depth calculation pipeline.
%Our proposed methods adopt Tang's Hausdorff-based penetration depth methods~\cite{SIG09HIST}. The pipeline is divided into the penetration surface extraction step and the Hausdorff distance calculation between the penetration surface steps. However, since Tang's approach is not suitable for the RT platform in detail, we modified and applied it with appropriate methods.

%The penetration surface extraction step is extracting overlapped surfaces on other objects. To utilize the RT core, we use the ray-intersection-based PIP(Point-In-Polygon) algorithms instead of collision detection between two objects which Tang et al.~\cite{SIG09HIST} used. (Sec.~\ref{sec:RT-PIP})
%RT core-based PIP test uses a ray-surface intersection test. For purpose this, we generate the GAS(Geometry Acceleration Structure) for each object. RT core-based PIP test takes the GAS of one object (e.g. $GAS_{A}$) and a set of vertex of another one (e.g. $P_{B}$). Then this computes the penetrated vertex set of another one (e.g. $P_{\partial B}$). To calculate the Hausdorff distance, these vertex sets change to objects constructed by penetrated surface (e.g. $\partial B$). Finally, the two generated overlapped surface objects $\partial A$ and $\partial B$ are used in the Hausdorff distance calculation step.
\section{Experimental Results}

Hardware
\section{Software Manager and Interface}
\label{sec:software}

The software interface of \name{} wraps around both the Xilinx XRT environment and the compression functionalities, to provide a familiar and transparent interface to acceleration on both compressed and uncompressed data.

\subsection{Programming Interface}

The \name{} software interface provides a familiar, XRT-style wrapper around accelerated kernel calls.
Let's consider a kernel \texttt{accel} with two device-side buffers ``\texttt{a}'' and ``\texttt{b}'' as parameters, where ``\texttt{a}'' is a genome data source.
If ``\texttt{a}'' comes from a \name{}-managed compressed source, subsections of the file can be transmitted as part of the kernel execution simply by setting \texttt{a.source = \name{}.compressed\_file(1); a.offset = N; a.size = M;}, and then passing it to the kernel via \texttt{accel.kernel(a,b)}.

The current version of the \name{} decompressor only supports stream sources, meaning transmitted buffers are decompressed immediately and provided to the kernel over a FIFO interface.
If the kernel requires random access into a buffer, it must copy the decompressed data to a separate in-memory buffer.
We are working on building an on-device index for compressed data to overcome this limitation.


\subsection{Index Structure}

One of the most important features of compressed data management is random access.
Downstream processing, such as graph construction based on reads, requires each read to be accessed separately, and the fundamental workload of reference-based alignment requires random access capabilities into the reference.

\name{} achieves this feature using a B+tree data structure, as hinted at in Figure~\ref{fig:overall}.
During compression, a B+tree is constructed with the file-internal offset as the key.
The unit element of insertion and lookup is a chunk of compressed data, which shares a single 32-bit header.
While the size of a decompressed chunk can be up to 128 bytes large in our prototype, we discovered that the random access requirement is typically coarser than this, in the unit of long reads.


\subsection{Reference Genome Encoding}

The software environment also must store the reference genome used for compression, because the compression process is split between hardware and software, as described in Section~\ref{sec:compression_arch}.
As described in Section~\ref{sec:compression_arch}, the hardware portion of compression is primarily responsible for calculating the hash function.
On the other hand, the software must perform a lookup into the cuckoo hash table, which is too large to store comfortably in the accelerator, and discover the correct cuckoo hash slot (if any) by comparing the target k-mer against up to four k-mer substrings sampled from the reference genome according to the cuckoo hash lookup.

Because \name{} stores the reference in a compact 2-bit encoded format, comparing k-mers can cause performance overhead due to sub-byte addressing.
For example, if a k-mer starts from offset 7, with two bit encoding, the 7th base starts in bit 6 of byte 2.
Comparing k-mers in this setting requires repeated shifting operations, which can have performance overhead.
To overcome this, \name{} stores four copies of the reference, each shifted from the original by 2 to 6 bits.
During compression, one of the copies are chosen according to the ``offset mod 4'' value, and then fast \texttt{memcmp} can be used since the string would be aligned along byte boundaries.

Both alternatives to storing four binary copies: Storing byte-alined ASCII files, and performing shifts on the fly, showed performance degradation by 4$\times$ on average.


\section{Network Issues} \label{sec:network-issues}
This section delves into the network layer's vulnerabilities, pivotal for synchronizing time across digital systems. Such synchronization is vital for applications ranging from digital payments to industrial automation. Yet, it faces threats from \textit{attackers controlling network devices} (on-path attacker) or \textit{possessing privileged access to a victim's local network stack} (off-path attacker).

\subsection{Limited Use of Authentication Mechanisms} Cryptography techniques, used by protocols like NTP~\cite{ntpv4-rfc} and PTP~\cite{ptp-std-doc}, play a critical role in ensuring data integrity and origin authentication of the time-sync traffic, thwarting man-in-the-middle (MITM) attacks. Yet, several issues persist regarding the adoption of these methods making time-sync protocols vulnerable to attacks.


\noindent\textbf{\texttt{I10.} False packet injection.} A MITM adversary can impersonate a genuine time server and send false time-sync packets to the target. These attacks may result from weak assumptions underlying the authentication mechanism adopted by the time-sync protocol. For instance, the reliance of NTP's broadcast mode authentication protocol TESLA~\cite{tesla-cryptography} (also used by PTP~\cite{ptp-std-doc}) on loosely synchronized devices creates a circular dependency between authentication and time-sync~\cite{ntp-replay-drop-attack}, rendering the former useless. Moreover, infiltration of malicious servers in the pool of legitimate time servers is  a genuine concern~\cite{shark-ntp-pool, devil-time-origin}. It is because cryptographic authentication only protects against a MITM attacker and the malicious servers render it ineffective. This allows Kwon et. al., to use a handful of malicious time servers, injected to the NTP pool~\cite{ntpd-pool-project}, to disrupt time-sync clients spread over entire countries~\cite{shark-ntp-pool}. Despite their shortcomings, authentication techniques make packet injection attacks harder. However, the adoption of these mechanisms is not universal. For instance, Huygens~\cite{huygens}, RBS~\cite{Elson2003RBS}, FTSP~\cite{ftsp-2004}, TPSN~\cite{tpsn-2003} do not implement any origin authentication mechanisms and have no protection against packet injection. The severity of the issue is evident from the fact that \textit{RBS, FTSP and TPSN} are among the most cited protocols for time-sync in sensor networks. In contrast, secure time synchronization protocols such as the one introduced by Ganeriwal et. al.~\cite{net-sync-wsn-sec-prot} has received an order of magnitude fewer citations (see table~\ref{tab:time-sync-wsn-citations}). Packet injection is one of the most potent attacks against time-sync protocols and could be used to induce \textit{time travel, warping or just increased uncertainty} (\textbf{\texttt{A1-3}}) in the victim's view of time.

\begin{table}
\scriptsize
\centering
\begin{tabular}{ | c | c | c | c | }
 \hline
  \textbf{Protocol} & \textbf{Authentication} & \textbf{Date Published} & \textbf{Citations} \\
 \hline
 \hline
  RBS~\cite{Elson2003RBS}  & \textit{No} & $2003$ & $3927$   \\
 \hline
   TPSN~\cite{tpsn-2003}  & \textit{No} & $2003$ & $3206$   \\
 \hline
   FTSP~\cite{ftsp-2004}  & \textit{No} & $2004$ & $3052$   \\
 \hline
 \hline
   Secure Time-Sync~\cite{Elson2003RBS}  & \textit{Yes} & $2005$ & $278$   \\
 \hline
\end{tabular}
\caption{One of the earliest time-sync protocols proposed for wireless sensor networks (WSNs). The protocols (RBS, TPSN and FTSP) that do not incorporate authentication mechanisms have received an order of magnitude more citations than the protocol (STS) that make use of cryptography mechanisms. \textit{Source: Google Scholars as of Jan 22, 2024.}}
\label{tab:time-sync-wsn-citations}
\end{table}

\noindent\textbf{\texttt{I11.} Packet modification.} Correct implementation of authentication protocols prevents false packet injection but may not prevent against packet modification. This is best exemplified by PTP, which makes use of authentication~\cite{ptp-std-doc} to protects the PTP packets except the correction field of the packet header. This field allows each network node to update correction field with the packet processing delay. PTP uses this information to achieve better time-sync accuracy by eliminating the variable network delays~\cite{net-sync-ptp-covert-channel}. However, a MITM attacker (on-path or off-path) can add incorrect information to this field and manipulate the PTP client. Jacobs et, al., use this channel to introduce \textit{significant offsets} (\textbf{\texttt{A3}}) to the victim device while \textit{avoiding detection}. They could also induce the victim device to change its clock frequency (\textbf{\textit{A2}}), resulting in an even larger time deviation from the time server~\cite{net-sync-ptp-covert-channel}. We note that this attack is not PTP specific, and any time-sync protocol seeking network delay information may be subject to this attack. Finally, we also note that this technique is less sophisticated as it does not require by-passing authentication requires.

\noindent\textbf{\texttt{I12.} Packet replay.} Authentication issues discussed in $I10$ can also result in replay attacks. In this attack, the adversary repeatedly sends one or a sequence of pre-recorded time server packets to the victim. Packet replay attacks have been successfully demonstrated against NTP broadcast mode~\cite{ntp-replay-drop-attack}. Malhotra et. al. exploited limitations in existing NTP client implementations to keep the victim stuck at a single point in time (\textbf{\texttt{A1}}). They point out that the one-way nature of the time-sync traffic (NTP broadcast mode) enables this attack. It implies that other one-way time synchronization protocols such as RBS~\cite{Elson2003RBS} may also be susceptible to this attack.

\noindent\textbf{\texttt{I13.} Spoofing Wireless Timing Signals.} Time-sync protocols like GPS, ROCS~\cite{ROCS-FM-Beacons}, Syntonizor~\cite{Syntonizor-AC-powerlines} and WizSync~\cite{WizSync-Wifi-Beacons} work using a periodic wireless timing signal that is transmitted directly from the timing source(s) to the clients i.e. over a single hop. These protocols lack authentication mechanisms allowing adversaries to spoof timing signals. This attack is the equivalent to packet manipulation attack on packet exchange based protocols (NTP~\cite{nts-rfc}, PTP~\cite{ptp-std-doc}, FTSP~\cite{ftsp-2004} etc.). Similar to the packet manipulation attacks, an external adversary mimics a trusted timing source but transmits incorrect timing information. It does so by generating a powerful spoof signal, using antenna(s), that can overpower the legitimate signal. Such an attacker often stays \textit{stealthy} while introducing uncertainty in the victim's local clocks~\cite{gps-spoofing-fundamentals} (\textbf{\texttt{A3}}). Satellite based global positing systems (GPS) is a typical target of this attack~\cite{gps-spoofing-21}. However, other time-sync protocols in this category (e.g., ROCS,~\cite{ROCS-FM-Beacons}, WizSync~\cite{WizSync-Wifi-Beacons} and Syntonizor~\cite{Syntonizor-AC-powerlines} etc.) haven't seen significant spoofing attacks due to their limited application. Nevertheless, signal spoofing remains a viable attack option for a motivated adversary.

\subsection{Availability Issues}
Beyond modifying timing packets, time-sync is also affected by just delaying the transmission of the timing information (as discussed in section~\ref{subsec:case-studies}). An adversary may leverage this observation and use unpredictable delays to add errors to the time-sync process or it  may outright block time-sync traffic headed towards the victim. 

\noindent\textbf{\texttt{I14.} Packet delay.} Time synchronization protocols determine the time offset between the server and the client by exchanging packets over the network. These network packets experience delays causing uncertainty in the exchanged timing information and the corresponding offset calculations (see section~\ref{subsec:case-studies}). Time-sync protocols rely heavily on precise network delay measurements to remove this uncertainty in the offset estimations. NTP~\cite{ntpv4-rfc} solves this challenge by measuring round trip times (RTTs) and computes network delay as half of the RTT, assuming symmetric delays~\cite{rfc1305}. On the other hand PTP measures the network delays by mandating each processing node to update the PTP packets with its resident delay (see $I11$). While effective under normal network conditions, these delay estimation mechanisms are not robust to adversarial delays. A malicious network node may introduce additional network delays~\footnote{In case of NTP, the server-bound and client-bound packets are delayed by different duration while for PTP the adversary would not update PTP packets with its resident delay} to degrade the synchronization performance. For instance, Annesi et. al. show that delay attacks against PTP can induce errors of several milliseconds, accumulating over time to even larger values under a sustained attacks~\cite{ptp-futile-encryption} (\textbf{\texttt{A2}}). However, vulnerability to delay attacks extend beyond NTP and PTP; virtually all time-sync protocols are susceptible to these attacks.

\noindent\textbf{\texttt{I15.} Packet drop.} 
Intercepting and dropping time-sync packets is a simple yet effective MiTM attack that desynchronizes the victim device from its time server. Facing this \textit{denial-of-service} attack, the victim solely relies on its \textit{local clock} which diverges away from the server time (\textbf{\texttt{A3}}) dictated by the stability of the victim's time source. For low-end systems using inexpensive quartz crystals, the time difference may accumulate to several minutes per day. In contrast, devices using more stable oven-controlled quartz oscillators may experience deviations of only a few seconds in the same period. However, despite its effectiveness, the victim can deduce potential instances of this attack, with relative ease, from sudden unavailability of time-server.

\noindent\textbf{\texttt{I16.} Blocking Wireless Timing Signals.} For single-hop wireless time synchronization (GPS~\cite{gps-spoofing-fundamentals}, ROCS~\cite{ROCS-FM-Beacons}, WizSync~\cite{WizSync-Wifi-Beacons} etc.), denial of service attack takes the form of blocking the wireless timing signal. An adversary achieves this by generating high powered noise in the frequency band used by the wireless timing signal. It requires physical proximity to the target and signal transmission equipment, raising the cost of this attack. Nevertheless, GPS signal blocking techniques have been studied extensively~\cite{gps-jamming-overview} due to ubiquitous use of GPS by defense and civil infrastructure. In principle, other single-hop wireless protocols such as Syntonizor~\cite{Syntonizor-AC-powerlines} and ROC~\cite{ROCS-FM-Beacons} are also vulnerable to these attacks, even though no such attack against them is known.

\subsection{Implementation Issues} In addition to the the communication medium, the end-points of this channel i.e. the applications implementing the time-sync protocol themselves represent an attack surface.

\noindent\textbf{\texttt{I17.} Untrusted time synchronization software.} Applications implementing time-sync protocols may harbor security vulnerabilities of their own. For instance, CVE database lists 98 vulnerabilities, discovered over the years, in the NTP application developed by \textit{NTP.org}~\cite{ntp-cve-details}. This application is used by both the time-sync clients and servers,~\footnote{It is recommended for servers joining the NTP pool project~\cite{ntpd-pool-project}.} and can be exploited by an adversary with access to \textit{privileged execution} on the victim device or \textit{a network connection to the NTP application}. An attack exploiting client side application vulnerability would only affect a single machine, however, the server side exploit would affect time alignment at all of its clients. Further, these attacks may cause the target applications to crash pausing the time-sync service or may just degrade time-sync performance (\textbf{\texttt{A3}}) over longer periods. It is worth pointing out time-sync applications executing in the privileged context present an even bigger risk, as any vulnerability in them could compromise the system beyond time-sync service.
\begin{table*}[htbp]
\caption{Evaluation results of different defense methods on adversarial robustness. }
 \resizebox*{\linewidth}{!}{
\begin{tabular}{c|l|llllllll}
\toprule
 Datasets & Attacks & $R^{A}_{18}\mathop{-}\limits^{s}R^{V}_{18}$ & $R^{A}_{18}\mathop{-}\limits^{c}R^{V}_{18}$ & $R^{A}_{18}\mathop{-}\limits^{s}R^{V}_{34}$ & $R^{A}_{18}\mathop{-}\limits^{c}R^{V}_{34}$ & $R^{A}_{34}\mathop{-}\limits^{s}R^{V}_{18}$ & $R^{A}_{34}\mathop{-}\limits^{c}R^{V}_{18}$ & $R^{A}_{34}\mathop{-}\limits^{s}R^{V}_{34}$ & $R^{A}_{34}\mathop{-}\limits^{c}R^{V}_{34}$\\
 \hline
  \multirow{5}{*}{Music}
     & Ori. &  ~~~~~82.1& ~~~~~84.3 & ~~~~~85.0 &  ~~~~~85.0 &  ~~~~~84.2 &  ~~~~~84.7 & ~~~~~87.9&~~~~~89.2\\
     &Adv.Train &  ~~~~~82.1& ~~~~~84.3 & ~~~~~85.0 &  ~~~~~85.0 &  ~~~~~84.2 &  ~~~~~84.7 & ~~~~~87.9&~~~~~89.2\\
     & DCFL&  ~~~~~38.7& ~~~~~38.2 & ~~~~~40.8 &  ~~~~~44.1 &  ~~~~~39.1 &  ~~~~~40.3& ~~~~~47.2&~~~~~48.5\\
     & Mixup& ~~~~~33.9& ~~~~~32.7 & ~~~~~35.2 &  ~~~~~34.1 &  ~~~~~32.6 &  ~~~~~33.0 & ~~~~~30.8&~~~~~29.6\\
      &Ours&  ~~~~~13.4& ~~~~~15.7 & ~~~~~12.2 &  ~~~~~11.9 &  ~~~~~11.6&  ~~~~~12.0&  ~~~~~12.3& ~~~~~12.1\\
     \hdashline
     \multirow{5}{*}{K-S}
     & Ori. & ~~~~~70.8 & ~~~~~72.5&~~~~~73.7 &  ~~~~~74.2 &~~~~~73.3 &  ~~~~~73.9 &~~~~~78.3 &  ~~~~~79.0 \\
      &Adv.Train &  ~~~~~82.1& ~~~~~84.3 & ~~~~~85.0 &  ~~~~~85.0 &  ~~~~~84.2 &  ~~~~~84.7 & ~~~~~87.9&~~~~~89.2\\
     & DCFL&  ~~~~~43.9& ~~~~~44.6 & ~~~~~42.5 &  ~~~~~41.3 &  ~~~~~43.8 &  ~~~~~42.0& ~~~~~40.1&~~~~~39.9\\
     & Mixup& ~~~~~35.4& ~~~~~36.1 & ~~~~~34.2 &  ~~~~~34.9 &  ~~~~~35.3 &  ~~~~~35.8 & ~~~~~33.2&~~~~~34.0\\
      & Ours & ~~~~~15.5& ~~~~~15.5 & ~~~~~14.3 &  ~~~~~15.7 &  ~~~~~14.7 &  ~~~~~14.9 & ~~~~~13.6&~~~~~13.8\\
     \bottomrule
\end{tabular}}
\label{tab:defense}
\end{table*}
\section{Theoretical Tools for Securing Time}
In this section, we discuss research work that employs theoretical tools to secure the timing stack. In this context, theoretical tools are employed for three distinct tasks: i) establish properties of a system model, ii) proving correctness of a system design i.e. it aligns with the stated goals and iii) verify software implementation of a given time-stack component. 

\subsection{Hardware}
The timing vulnerabilities in the hardware layer primarily result from either physical side channels ($I01$, $I02$) or design limitations ($I03$, $I04$).
Traditional formal verification tools cannot address these issues because mitigating them requires either the addition of new components or changing the existing designs. However, these new designs may be evaluated using standard formal verification tools~\cite{formal-verification-intel}.

\subsection{Software}
Timekeeping functionalities, integrated within broader system software like operating systems or hypervisors, are prone to inherent software vulnerabilities ($I05, I06, I07$). Although formal verification tools offer a means to analyze these systems for potential flaws, the large code bases and complex interaction among various subsystems of the system software make it infeasible to verify them. Despite these challenges, advancements in formal verification techniques have enabled the verification of specific OS components~\cite{formal-verification-eBPF} and hypervisors~\cite{formal-verification-hypervisor-arm, formal-verification-hypervisor-memory, formal-verification-kvm}. Employing these verification tools to assess the security aspects of system software can diminish the timekeeping software's vulnerability exposure ($I05, I06, I07$). Further, timing subsystems may also contain vulnerabilities originating from incorrect implementations. To enhance their security, it is crucial to apply the latest formal verification tools to verify the correctness of timing subsystems including trusted timing services such as Timeseal~\cite{time-stack-timeseal} and T3E~\cite{trusted-time-t3e}. As far as our knowledge extends, applying formal verification tools to timekeeping software remains an open area of research.

\subsection{Network} \label{subsec:network-theory}
The use of theoretical tool to establish trust in the timing stacks have almost exclusively focused on its network component i.e. time synchronization. This literature has focused on following lines of work:

\noindent\textbf{\texttt{T01.} Establishing Secure Time Synchronization Requirements.} 
Theorem proving tools have been used to establish requirements for secure clock synchronization. Narula et al.\cite{net-sync-gps-sec-transfer} constructed formal models for one-way and two-way time transfer, assuming a line-of-sight link between the systems and a threat model with a MITM adversary. They present proves for i) one-way time transfer's inherent susceptibility to delay attacks (as discussed in $I12$) and ii) essential requirements for a two-way secure time-synchronization protocol. Building upon this work, they study two-way time synchronization over a multi-hop network where systems at the both ends implement cryptography~\cite{net-sync-gps-sec-sync}. They put forward the prerequisites for a secure clock synchronization algorithm applicable to protocols like PTP~\cite{ptp-std-doc}. Among other requirements, they show that the timing packets must travel along the shortest path between the server and the client to completely prevent delay attacks. This has an important implication that \textit{delay attacks, over the network, cannot be prevented entirely if they do not guarantee shortest path traversal.} This is indeed the case of today's internet that employs TCP/IP stack for networking.

\begin{table*}[t]
\footnotesize
\centering
\begin{tabular}{p{2 cm} p{3.5cm}  p{0.5cm}  p{0.5cm}  p{0.5cm}  p{0.5cm}  p{0.5cm}  p{0.5cm}  p{0.5cm} p{0.5cm}  p{0.5cm}  p{0.5cm}  p{0.5cm}  p{0.5cm} }
 \multicolumn{1}{c}{} & \multicolumn{1}{c}{} & \multicolumn{8}{c}{Systems Approach} & \multicolumn{4}{c}{Theoretical Approach} \\
 \cmidrule(lr){3-10} \cmidrule(lr){11-14}
    & & \texttt{D01} & \texttt{D02} & \texttt{D03} & \texttt{D04} & \texttt{D05} & \texttt{D06} & \texttt{D07} & \texttt{D08} & \texttt{T01} & \texttt{T02} & \texttt{T03} & \texttt{T04} \\
 \hline
  Hardware Issues & Physical Side Channels & \halfcirc & \emptycirc & \emptycirc & \emptycirc & \emptycirc & \emptycirc & \emptycirc & \emptycirc & \emptycirc & \emptycirc & \emptycirc & \emptycirc \\
& Design Limitations & \emptycirc & \halfcirc & \emptycirc & \emptycirc & \emptycirc & \emptycirc & \emptycirc & \emptycirc & \emptycirc & \emptycirc & \emptycirc & \emptycirc \\
\hline
Software Issues & System Software Bugs & \emptycirc & \emptycirc & \halfcirc & \emptycirc & \emptycirc & \emptycirc & \emptycirc & \emptycirc & \emptycirc & \emptycirc & \emptycirc & \emptycirc \\
& TEE Limitations & \emptycirc & \emptycirc & \halfcirc & \fullcirc & \emptycirc & \emptycirc & \emptycirc & \emptycirc & \emptycirc & \emptycirc & \emptycirc & \emptycirc \\
\hline
 & Limited Crypto Adoption & \emptycirc & \emptycirc & \emptycirc & \emptycirc & \halfcirc & \emptycirc & \emptycirc & \halfcirc & \fullcirc & \halfcirc & \emptycirc & \emptycirc \\
Network Issues & Availability Issues & \emptycirc & \emptycirc & \emptycirc & \emptycirc & \emptycirc & \halfcirc & \halfcirc & \halfcirc & \fullcirc & \emptycirc & \emptycirc & \halfcirc \\
& Implementation Issues & \emptycirc & \emptycirc & \emptycirc & \emptycirc & \emptycirc & \emptycirc & \emptycirc & \emptycirc & \emptycirc & \emptycirc & \halfcirc & \emptycirc \\
\hline
\end{tabular}
\caption{Research contributions towards mitigating timing stack issues utilize both \textit{system-based} and \textit{theoretical} approaches. A full circle indicates that the defense technique mitigates all issues within the category, a half circle suggests that some of the issues in a category are addressed, and an empty circle signifies the given defense's lack of mitigation for issues in the specified category. While system-based defenses tackle attack surfaces across all three layers of the timing stack, theoretical solutions predominantly concentrate on the network component.}
\label{tab:system-v-defense}
\end{table*}

\noindent\textbf{\texttt{T02.} Proving Correctness of the Time-Sync Protocols.} Formal verification tools have been used to prove the correctness of fault-tolerant clock sync protocols. For instance, Schwier et. al.~\cite{theory-MechanicalVO} used protocol verification system (PVS) to verify a generalized time-sync protocol's correctness based on conditions established by Schneider~\cite{theory-schneider-conditons} for byzantine faults. Improving on this work, Barsotti et al., ~\cite{theory-schneider, theory-deductivetools} used deductive tools to prove correctness of fault-tolerant clock synchronization algorithms proposed by Lamport-Melliar\cite{clock-sync-fault-presence} and Lundlies-Lynch~\cite{clock-sync-fault-tolerant}. This research offers a promising direction for formal verification of time-sync protocols dealing with malicious faults.

Recent efforts regarding secure time synchronization using mathematical analysis have shifted focus to wireless sensor networks~\cite{theory-attack-resilient-pulse-coupled, theory-self-stablizing}. Most of these works assume a MITM attack model where few nodes in the network are compromised ($I10-I15$). Wang et al.\cite{theory-attack-resilient-pulse-coupled} introduced an attack-resilient pulse-coupled synchronization scheme for wireless sensor networks, deriving necessary conditions and analytically proving that it guarantees secure synchronization in the presence of a single malicious node. Another work by Hoepman et. al.\cite{theory-self-stablizing} presented a self-stabilizing clock synchronization algorithm, for wireless sensor networks. Their design is secure against pulse delay attacks by malicious nodes and they providing proofs for the correctness of their random beacon scheduling algorithm. These works demonstrate the potential of using theoretical tools for verify time-sync protocol designs.

\noindent\textbf{\texttt{T03.} Verification of Protocol Implementations.} Formal verification techniques can be used to verify the implementations of time-sync protocols ($I17$). This is demonstrated by Luca et al., who performed the automated verification of the gossip time-sync protocol's~\cite{model-checking-gossip} implementation using parameterized model checking. However, gossip is rather a simple protocol, and the methods employed for its verification do not readily extend to more complex protocols such as NTP, PTP etc~\cite{net-sync-openchallenges}. However, despite its limitations, partial verification of time-sync implementations using the existing formal verification methods can yield important security insights. In one such instance, Dieter et al.\cite{theory-nts-specs} performed (partial) formal verification of Network Time Security (NTS -- RFC8915) specifications~\cite{nts-rfc} and discovered two vulnerabilities in the analyzed version~\cite{theory-nts-formal-analysis}, which are currently being addressed. It shows that future research on enabling complete verification of widely used time-sync protocols would greatly contribute to their security.

\noindent\textbf{\texttt{T04.} Mitigating Attacks on Time Synchronization.} Beyond analyzing complete protocol design, mathematical tools have also been used to study specific time-sync attacks. This research primarily focuses on mitigating delay attacks against time-sync protocols by a network adversary. For instance, Mizrahi et al. proposed a multi-path time synchronization scheme designed to resist delay attacks by a man-in-the-middle attacker. They leverage game theory to provide proofs for the delay resiliency of their design~\cite{multi-path-game-theory}. Similarly, Anto et al. propose modifications to PTP aimed at mitigating delay attacks and formally verified the correctness of their proposed updates~\cite{theory-formal-attack-1588}. Likewise, Moussa et al. proposed extensions to the PTP protocol and formally proved the correctness of these protocol extensions~\cite{theory-smart-grids, theory-ptp-extension}. However, the proposed extensions are domain specific as they rely on redundant master clocks on power grid substations mandated by IEC 61850. Another work by Lisova et al., took a game-theoretic approach, modeling the interaction of a man-in-the-middle attacker introducing asymmetric delays to PTP packets and a network inspection system collecting clock offset information. Their work uses a game-theoretic tool to predict attacker strategies and develop mitigation mechanisms accordingly~\cite{theory-game-theory-1588}. While delay attacks have been a dominant subject of this research, defenses against other attacks would equally benefit from the use of theoretical tools.
\section{Discussion of Assumptions}\label{sec:discussion}
In this paper, we have made several assumptions for the sake of clarity and simplicity. In this section, we discuss the rationale behind these assumptions, the extent to which these assumptions hold in practice, and the consequences for our protocol when these assumptions hold.

\subsection{Assumptions on the Demand}

There are two simplifying assumptions we make about the demand. First, we assume the demand at any time is relatively small compared to the channel capacities. Second, we take the demand to be constant over time. We elaborate upon both these points below.

\paragraph{Small demands} The assumption that demands are small relative to channel capacities is made precise in \eqref{eq:large_capacity_assumption}. This assumption simplifies two major aspects of our protocol. First, it largely removes congestion from consideration. In \eqref{eq:primal_problem}, there is no constraint ensuring that total flow in both directions stays below capacity--this is always met. Consequently, there is no Lagrange multiplier for congestion and no congestion pricing; only imbalance penalties apply. In contrast, protocols in \cite{sivaraman2020high, varma2021throughput, wang2024fence} include congestion fees due to explicit congestion constraints. Second, the bound \eqref{eq:large_capacity_assumption} ensures that as long as channels remain balanced, the network can always meet demand, no matter how the demand is routed. Since channels can rebalance when necessary, they never drop transactions. This allows prices and flows to adjust as per the equations in \eqref{eq:algorithm}, which makes it easier to prove the protocol's convergence guarantees. This also preserves the key property that a channel's price remains proportional to net money flow through it.

In practice, payment channel networks are used most often for micro-payments, for which on-chain transactions are prohibitively expensive; large transactions typically take place directly on the blockchain. For example, according to \cite{river2023lightning}, the average channel capacity is roughly $0.1$ BTC ($5,000$ BTC distributed over $50,000$ channels), while the average transaction amount is less than $0.0004$ BTC ($44.7k$ satoshis). Thus, the small demand assumption is not too unrealistic. Additionally, the occasional large transaction can be treated as a sequence of smaller transactions by breaking it into packets and executing each packet serially (as done by \cite{sivaraman2020high}).
Lastly, a good path discovery process that favors large capacity channels over small capacity ones can help ensure that the bound in \eqref{eq:large_capacity_assumption} holds.

\paragraph{Constant demands} 
In this work, we assume that any transacting pair of nodes have a steady transaction demand between them (see Section \ref{sec:transaction_requests}). Making this assumption is necessary to obtain the kind of guarantees that we have presented in this paper. Unless the demand is steady, it is unreasonable to expect that the flows converge to a steady value. Weaker assumptions on the demand lead to weaker guarantees. For example, with the more general setting of stochastic, but i.i.d. demand between any two nodes, \cite{varma2021throughput} shows that the channel queue lengths are bounded in expectation. If the demand can be arbitrary, then it is very hard to get any meaningful performance guarantees; \cite{wang2024fence} shows that even for a single bidirectional channel, the competitive ratio is infinite. Indeed, because a PCN is a decentralized system and decisions must be made based on local information alone, it is difficult for the network to find the optimal detailed balance flow at every time step with a time-varying demand.  With a steady demand, the network can discover the optimal flows in a reasonably short time, as our work shows.

We view the constant demand assumption as an approximation for a more general demand process that could be piece-wise constant, stochastic, or both (see simulations in Figure \ref{fig:five_nodes_variable_demand}).
We believe it should be possible to merge ideas from our work and \cite{varma2021throughput} to provide guarantees in a setting with random demands with arbitrary means. We leave this for future work. In addition, our work suggests that a reasonable method of handling stochastic demands is to queue the transaction requests \textit{at the source node} itself. This queuing action should be viewed in conjunction with flow-control. Indeed, a temporarily high unidirectional demand would raise prices for the sender, incentivizing the sender to stop sending the transactions. If the sender queues the transactions, they can send them later when prices drop. This form of queuing does not require any overhaul of the basic PCN infrastructure and is therefore simpler to implement than per-channel queues as suggested by \cite{sivaraman2020high} and \cite{varma2021throughput}.

\subsection{The Incentive of Channels}
The actions of the channels as prescribed by the DEBT control protocol can be summarized as follows. Channels adjust their prices in proportion to the net flow through them. They rebalance themselves whenever necessary and execute any transaction request that has been made of them. We discuss both these aspects below.

\paragraph{On Prices}
In this work, the exclusive role of channel prices is to ensure that the flows through each channel remains balanced. In practice, it would be important to include other components in a channel's price/fee as well: a congestion price  and an incentive price. The congestion price, as suggested by \cite{varma2021throughput}, would depend on the total flow of transactions through the channel, and would incentivize nodes to balance the load over different paths. The incentive price, which is commonly used in practice \cite{river2023lightning}, is necessary to provide channels with an incentive to serve as an intermediary for different channels. In practice, we expect both these components to be smaller than the imbalance price. Consequently, we expect the behavior of our protocol to be similar to our theoretical results even with these additional prices.

A key aspect of our protocol is that channel fees are allowed to be negative. Although the original Lightning network whitepaper \cite{poon2016bitcoin} suggests that negative channel prices may be a good solution to promote rebalancing, the idea of negative prices in not very popular in the literature. To our knowledge, the only prior work with this feature is \cite{varma2021throughput}. Indeed, in papers such as \cite{van2021merchant} and \cite{wang2024fence}, the price function is explicitly modified such that the channel price is never negative. The results of our paper show the benefits of negative prices. For one, in steady state, equal flows in both directions ensure that a channel doesn't loose any money (the other price components mentioned above ensure that the channel will only gain money). More importantly, negative prices are important to ensure that the protocol selectively stifles acyclic flows while allowing circulations to flow. Indeed, in the example of Section \ref{sec:flow_control_example}, the flows between nodes $A$ and $C$ are left on only because the large positive price over one channel is canceled by the corresponding negative price over the other channel, leading to a net zero price.

Lastly, observe that in the DEBT control protocol, the price charged by a channel does not depend on its capacity. This is a natural consequence of the price being the Lagrange multiplier for the net-zero flow constraint, which also does not depend on the channel capacity. In contrast, in many other works, the imbalance price is normalized by the channel capacity \cite{ren2018optimal, lin2020funds, wang2024fence}; this is shown to work well in practice. The rationale for such a price structure is explained well in \cite{wang2024fence}, where this fee is derived with the aim of always maintaining some balance (liquidity) at each end of every channel. This is a reasonable aim if a channel is to never rebalance itself; the experiments of the aforementioned papers are conducted in such a regime. In this work, however, we allow the channels to rebalance themselves a few times in order to settle on a detailed balance flow. This is because our focus is on the long-term steady state performance of the protocol. This difference in perspective also shows up in how the price depends on the channel imbalance. \cite{lin2020funds} and \cite{wang2024fence} advocate for strictly convex prices whereas this work and \cite{varma2021throughput} propose linear prices.

\paragraph{On Rebalancing} 
Recall that the DEBT control protocol ensures that the flows in the network converge to a detailed balance flow, which can be sustained perpetually without any rebalancing. However, during the transient phase (before convergence), channels may have to perform on-chain rebalancing a few times. Since rebalancing is an expensive operation, it is worthwhile discussing methods by which channels can reduce the extent of rebalancing. One option for the channels to reduce the extent of rebalancing is to increase their capacity; however, this comes at the cost of locking in more capital. Each channel can decide for itself the optimum amount of capital to lock in. Another option, which we discuss in Section \ref{sec:five_node}, is for channels to increase the rate $\gamma$ at which they adjust prices. 

Ultimately, whether or not it is beneficial for a channel to rebalance depends on the time-horizon under consideration. Our protocol is based on the assumption that the demand remains steady for a long period of time. If this is indeed the case, it would be worthwhile for a channel to rebalance itself as it can make up this cost through the incentive fees gained from the flow of transactions through it in steady state. If a channel chooses not to rebalance itself, however, there is a risk of being trapped in a deadlock, which is suboptimal for not only the nodes but also the channel.

\section{Conclusion}
This work presents DEBT control: a protocol for payment channel networks that uses source routing and flow control based on channel prices. The protocol is derived by posing a network utility maximization problem and analyzing its dual minimization. It is shown that under steady demands, the protocol guides the network to an optimal, sustainable point. Simulations show its robustness to demand variations. The work demonstrates that simple protocols with strong theoretical guarantees are possible for PCNs and we hope it inspires further theoretical research in this direction.
\section{Conclusion}
In this work, we propose a simple yet effective approach, called SMILE, for graph few-shot learning with fewer tasks. Specifically, we introduce a novel dual-level mixup strategy, including within-task and across-task mixup, for enriching the diversity of nodes within each task and the diversity of tasks. Also, we incorporate the degree-based prior information to learn expressive node embeddings. Theoretically, we prove that SMILE effectively enhances the model's generalization performance. Empirically, we conduct extensive experiments on multiple benchmarks and the results suggest that SMILE significantly outperforms other baselines, including both in-domain and cross-domain few-shot settings.

\bibliographystyle{plain}
\bibliography{main}
%%%%%%%%%%%%%%%%%%%%%%%%%%%%%%%%%%%%%%%%%%%%%%%%%%%%%%%%%%%%%%%%%%%%%%%%%%%%%%%%
\end{document}
%%%%%%%%%%%%%%%%%%%%%%%%%%%%%%%%%%%%%%%%%%%%%%%%%%%%%%%%%%%%%%%%%%%%%%%%%%%%%%%%