\section{Background} \label{background}
In this section, we provide a brief introduction to the timing stacks in contemporary digital systems. Additionally, we describe the types of timing attacks and present case studies demonstrating attacks on the timing stack.

\subsection{Timing Stack Basics}\label{subsec:local-clocks}
\noindent\textbf{Measuring Time.}
The foundation of any timing stack is its timing source, which emits a recurring signal at fixed intervals. This \textit{clock signal} is essential for digital systems to measure time accurately. Figure~\ref{fig:time-stack-example} illustrates a typical Cyber-Physical System (CPS) timing stack, where the time source is integrated into the physical hardware, accompanied by \textit{time measurement} components such as counters and timer registers. The time source, often a quartz crystal, produces an analog signal that feeds into these components. The counter records the number of \textit{clock periods (cycles)} since the system was powered on, whereas the timer registers are designed to initiate an interrupt after a predetermined number of cycles. System software utilizes these elements to maintain \textit{time} and distribute it to user applications.
\begin{figure}[h]
        \centering           
        \includegraphics[width=0.5\columnwidth]{figures/time-stack-updated.pdf}
        \caption{Time stack in a modern CPS}
        \label{fig:time-stack-example}
\end{figure}

\noindent\textbf{Keeping Time.} Hardware counters and timers measure time in clock cycles, a unit that varies across systems due to differences in time source frequencies. Furthermore, this measurement begins at the system's power-up, an arbitrary starting point. Thus, time obtained from the hardware counters and timers is non-standardized. System software establishes a \textit{local clock} by converting these clock cycles into standard wall clock time using the clock signal frequency and network-derived current time (Figure~\ref{fig:time-stack-example}). This local clock is updated at regular intervals using a recurring timer interrupt called \textit{system tick}. When a timestamp is needed between two consecutive system ticks, the system software reads the current value of the processor \textit{counter} to determine the time elapsed since the last \textit{system tick} and adds it to the time recorded at the last tick to compute the current time~\cite{linux-hrtimers}. Similarly, it also maintains a \textit{software timer} providing a standardized interface to the user applications.

\noindent\textbf{Time Synchronization.} Time source frequency variation, influenced by environmental factors like temperature, causes \textit{local clocks} to drift from the actual time~\cite{graham-clock-sync}. Time synchronization services correct this by estimating the local clock's deviation from a network-provided reference clock, using packet exchanges and integrity checks to securely adjust the local clock (Figure~\ref{fig:time-stack-example}). Time-sync protocols fall into two categories: (1) \textit{Two-way Time-Sync} utilizes bidirectional message exchange to compute offset and skew~\footnote{Offset and skew refer to the baseline time difference between two clocks and the rate at which this difference grows, respectively.}. These parameters are utilized to align local clocks with an external reference (Figure~\ref{fig:two-way}).(2)  \textit{One-way Time-Sync} relies on one-way broadcasts from servers to clients, as used by GPS~\cite{gps-spoofing-fundamentals} and some sensor network protocols~\cite{Elson2003RBS} (Figure~\ref{fig:one-way}).

\begin{figure}[t]
    \centering
    \begin{subfigure}{0.28\columnwidth}
        \centering   
     \includegraphics[width=\linewidth]{figures/two-way-transfer.pdf}
        \caption{}
        \label{fig:two-way}
    \end{subfigure}
    \hspace{2em}
    \begin{subfigure}{0.35\columnwidth}
        \centering 
    \includegraphics[width=\linewidth]{figures/one-way-transfer.pdf}
        \caption{}
        \label{fig:one-way}
    \end{subfigure}
    \caption{(a) Two-way time synchronization. (b) One-way time synchronization.}
    \label{fig:sync-paradigms}
\end{figure}

\subsection{Temporal Manipulations}\label{subsec:attack-types}
Malicious adversaries may seek to attack the timing stacks of cyber-physical systems, just as they target other components of the digital infrastructure. Such attacks aim to violate one or more of an ideal time stack's properties: 
(\textbf{\texttt{P1}}) the \textit{local clock} is monotonic, i.e., time always moves forward, (\textbf{\texttt{P2}}) it maintains a constant frequency, i.e., moves at a fixed rate, (\textbf{\texttt{P3}}) its frequency matches that of a reference clock, and (\textbf{\texttt{P4}}) it provides time relative to the epoch\footnote{A fixed date and time (Jan 1, 1970) used as a reference from which a system measures time.} also used by the reference clock. Figure~\ref{fig:attack-types} illustrates different forms of timing attacks that result from the violation of these four properties.

\noindent\textbf{(\texttt{A1}) Time Travel:} The local clock travels back or forward in time, violating \textbf{\texttt{P1}} and \textbf{\texttt{P4}}.

\noindent\textbf{(\texttt{A2}) Time Warping:} The local clock moves slower or faster relative to the reference, distorting the system's perception of time. This violates \textbf{\texttt{P3}} and, consequently, the discrepancy between local and the reference clock grows over time.

\noindent\textbf{(\texttt{A3}) Increased Uncertainty:} This attack targets the precision of the local clock (\textbf{\texttt{P2}}) while generally maintaining \textbf{\texttt{P3}} in the long-term. It reduces the effective temporal resolution of the local clock; for example, a clock that should provide time accurate to a millisecond is now only reliable to the second.

\begin{figure}[ht]
    \small
    \centering
    \begin{subfigure}{0.48\columnwidth}
        \centering
        \includegraphics[width=\columnwidth]{figures/timing-attack-types.pdf}
        \caption{}
        \label{fig:attack-types}
    \end{subfigure}
    \hspace{0.2em}
    \begin{subfigure}{0.48\columnwidth}
        \centering
        \includegraphics[width=\columnwidth]{figures/delay-attack.pdf}
        \caption{}
        \label{fig:delay-attack}
    \end{subfigure}
    \caption{(a) \textbf{\texttt{T}} represents the system time under normal conditions. \textbf{\texttt{A1}}, \textbf{\texttt{A2}}, and \textbf{\texttt{A3}} illustrate the time under time travel, warping, and uncertainty attacks, respectively. (b) Agent \textbf{\texttt{U}} obtains its time from another agent, \textbf{\texttt{T}}, assuming instantaneous time transfer. After time $t_2$, an attacker takes control of agent $T$ and delays the time transfer by a duration $0 \leq \delta t \leq \infty$.} 
    \label{fig:timing-attack-types}
\end{figure}

\subsection{Timing Attacks and CPS}\label{subsec:case-studies}
We demonstrate the vulnerability of CPS to timing attacks through various case studies, illustrating how adversaries exploit timing stack weaknesses to launch the attacks, described above.

\noindent\textbf{\texttt{C1.} No Trace Industrial Sabotage.} Industrial control systems are highly sensitive to temporal uncertainties~\cite{hardware-butterfly, hardware-polyrythem} that could be exploited by adversaries. An attacker can create temporal uncertainties just by exposing the control unit's quartz crystal (\textit{time source}) to lasers~\cite{redshift}. It would vary the crystal's frequency substantially, inducing \textit{time warping} (\textbf{\texttt{A2}}) to the control unit's \textit{local clock} and causing unstable behavior. An insider with \textit{physical access} to industrial equipment could launch such an attack without leaving any digital traces. 

\noindent\textbf{\texttt{C2.} Zero Knowledge Attack on AV-perception.} Autonomous vehicles (AV) utilize deep learning based multi-modal perception, relying on tightly synchronized sensor data~\cite{chen2019selectfusion}. Lack of synchronized inputs can destabilize these systems with potentially fatal consequences~\cite{hardware-chronos-slam-attack}. An attacker could introduce such de-synchronization by following these steps: i) maintain multiple copies of the \textit{local clock} on the victim AV, each moving at a different pace (\textbf{\texttt{A2}}) i.e. asynchronous clocks. ii) It then presents a different clock to each sensor subsystem disrupting their mutual synchronization and consequently causes the AV perception system to malfunction. This attack requires \textit{privileged access} to the AV system software, and identification of sensor subsystems. The former could be enabled by the privilege escalation vulnerabilities, that are discovered routinely (table~\ref{tab:cve-stats}), in the commodity system software. And the sensor subsystems are identified using the fact that they request repeated timestamps with a fixed interval between consecutive requests. Finally, we argue that this exploit lowers the cost of attacking AV perception because, in contrast to traditional attacks~\cite{hallyburton2022security, av-spoofing-attack}, it does not require any machine learning knowledge on part of the attacker nor does it require physical proximity to the victim.

\noindent\textbf{\texttt{C3.} Database Performance Degradation.} The consistency of database systems hinges on precise timekeeping~\cite{huygens}. To elaborate further, consider a database whose \textit{local clock} has an uncertainty of $\Delta t$ seconds. It receives two write requests at times $t_1$ and $t_2$, respectively, where $t_1-t_2 < \Delta t$. Due to its local clock's uncertainty, the database cannot determine the order in which two requests were issued. If it completes the two transactions, it risks losing data by committing write requests in wrong order. To avoid such issues many databases such as Google's spanner choose to process transaction only if they are temporally spaced by at least  $\Delta t$ (uncertainty in the \textit{local clock}'s time)~\cite{google-spanner}. An adversary with the \textit{privileged access} to the system software can manipulate the \textit{local clock} to increase uncertainty $\Delta t$ in its time \textbf{\texttt{A3}}. This will increase the wait times between consecutive transactions and severely degrade the database performance. In contrast, if a database does not wait out these timing uncertainties, such an attack would result in database inconsistencies.

\noindent\textbf{\texttt{C4.} Manipulating Smart Contract Systems.} Smart contracts require secure time synchronization between the client (the offeree) and the server (the offerer) devices. It establishes the order of the events such as contract offering, modifications and signing. This ordering is indispensable for these events' validity in the case of a litigation~\cite{smart-contract-tabellion}. Consider a scenario where an offeree may agree to unfavourable terms to secure the contract by out-competing rivals. However, before signing the contract, the offeree changes its device's system time to the past when the unfavorable clauses were yet to be added (\textbf{\texttt{A1}}). They can get away with this, if the offerer, having trust in the contract system, does not notice this discrepancy. In the case of a dispute, it enables the offeree to make a plausible case that they did not agree to contentious terms causing financial losses to the offerer. It is important to note that, if the smart contract system gets its time from the network, the offeree can still rewind the clock back, albeit using a more sophisticated attack (see section~\ref{sec:network-issues}). 

\noindent\textbf{\texttt{C5.} GPS manipulation example.} GPS is indispensable to the navigation systems in aviation, maritime trade, and public transport, among others. Its adversarial manipulation could lead to significant financial and human costs. Adversarial attacks on GPS exploit its dependence on signals broadcast by satellites with tightly synchronized clocks. GPS uses the relative delay in the reception of these signals to estimate its location and clock offset relative to the satellite clocks. The adversary captures and replays a legitimate satellite signal with a delay~\cite{gps-spoofing-fundamentals} (\textbf{\texttt{A3}}) to cause uncertainty in the location and time perceived by the receiver. Furthermore, a single antenna is sufficient for launching this attack against a victim, as shown by Tippenhauer et al.,~\cite{gps-spoofing-fundamentals}. The only constraint is that the attacker's signal power, as received by the victim, should be higher than the legitimate GPS signals. Alternatively, an adversary with \textit{physical access} to the GPS receiver could remove the antenna and attach a low-cost device that generates a fake GPS signal to the antenna input, achieving the same results.

It is important to note here that all timing stacks, including GPS based time-sync, are uniquely sensitive to \textit{delay} in the information transfer. To elaborate further, consider Figure~\ref{fig:delay-attack} that shows two agents $U$ and $T$, where the former requests time from the latter. Under non-malicious settings, $U$ has the same view of time as $T$ assuming instantaneous request and time transfer. However, an adversary that manages to intercept the time transfer could delay it by an arbitrary time $\delta t$. In this case, the time as seen by $U$ has an error of $\delta t$ with respect to the reference time maintained by $T$. \textit{Delays associated with time transfer affect the integrity of the information being transferred}. 

To conclude, we note that above case studies underscore the multifaceted nature of timing attacks that can be launched not just by a network based adversary (as discussed in section~\ref{sec:introduction}) but also by exploiting \textit{physical side channels} (\textbf{\texttt{C1,C5}}) and \textit{system software vulnerabilities} (\textbf{\texttt{C2-4}}).

\begin{table}
\scriptsize
\centering
\begin{tabular}{| p{2cm} | p{0.8cm} | p{0.9cm} | p{0.9cm} | p{0.85cm} | p{0.5cm} |}
 \cline{1-6}
\multicolumn{1}{|c|}{} & \multicolumn{1}{c|}{} & \multicolumn{4}{c|}{\textbf{Attacker Characteristics}}  \\
 \cline{3-6}
\textbf{\makecell[l]{Case Study}} & \textbf{\makecell[l]{Attack\\Type}} & \textbf{\makecell[l]{Access\\Type}} & \textbf{\makecell[l]{Control}} & \textbf{\makecell[l]{Stealth}} & \textbf{\makecell[l]{DoS}} \\
 \hline
 \makecell[l]{Industrial Sabotage} & \textbf{\texttt{\makecell[l]{A2}}} & \makecell[l]{Physical}  & \makecell[l]{\emptycirc} & \makecell[l]{\fullcirc} & \makecell[l]{\halfcirc} \\
 \hline
 \makecell[l]{Attack on\\AV-perception} & \textbf{\texttt{\makecell[l]{A2}}} & \makecell[l]{Privileged\\Execution}  & \makecell[l]{\emptycirc} & \makecell[l]{\halfcirc} & \makecell[l]{\fullcirc}  \\
 \hline
 \makecell[l]{Database\\Degradation} & \textbf{\texttt{\makecell[l]{A3}}} & \makecell[l]{Network\\Device} & \makecell[l]{\fullcirc} & \makecell[l]{\fullcirc} & \makecell[l]{\emptycirc}  \\
 \hline
\makecell[l]{Compromised Smart\\Contract Systems} & \textbf{\texttt{\makecell[l]{A1}}} & \makecell[l]{Physical/\\Priv. Exec.} & \makecell[l]{\fullcirc} & \makecell[l]{\fullcirc} & \makecell[l]{\emptycirc}  \\
 \hline
 \makecell[l]{GPS Manipulation} & \textbf{\texttt{\makecell[l]{A3}}} & \makecell[l]{Network/\\Physical} & \makecell[l]{\fullcirc} & \makecell[l]{\halfcirc} & \makecell[l]{\emptycirc}  \\
 \hline
\end{tabular}
\caption{Capabilities of adversaries from our case studies: full, half and empty circle indicates adversary has the particular capability, has it partially and no capability, respectively.}
\label{tab:case-study-attack-requirements}
\end{table}

\begin{table*}
\scriptsize
\centering
\begin{tabular}{| p{3.5cm} | p{2.75cm} | p{1.95cm} | p{2cm} | p{2.4cm} | p{1.55cm} |}
 \cline{1-6}
\multicolumn{1}{|c|}{} & \multicolumn{3}{c|}{\textbf{Timing Stacks}} & \multicolumn{1}{c|}{} & \multicolumn{1}{c|}{}  \\
 \cline{2-4}
\textbf{\makecell[l]{Applications}} & \textbf{\makecell[l]{Time Source}} & \textbf{\makecell[l]{Platform Software}} & \textbf{\makecell[l]{Sync. Protocols}} & \textbf{\makecell[l]{Comm. Tech./Medium}} & \textbf{\makecell[l]{Network Scale}} \\
 \hline
 \makecell[l]{GPS Satellites} & \makecell[l]{Atomic Clocks} & -  & \makecell[l]{Laser Ranging} & \makecell[l]{Microwaves} & \makecell[l]{Global} \\
 \hline
 \makecell[l]{Astronomical Telescopes} & \makecell[l]{Atomic Clocks} & -  & - & - & -  \\
 \hline
 \makecell[l]{Underwater Surveillance Networks} & \makecell[l]{Chip Scale Atomic Clocks} & - & - & \makecell[l]{Wired} & \makecell[l]{Continent}  \\
 \hline
\makecell[l]{Underwater Sensor Networks} & \makecell[l]{Quartz Crystals} & \makecell[l]{RTOS} & \makecell[l]{\{PCDE/MM\}-Sync} & \makecell[l]{Acoustic} & \makecell[l]{several Km}  \\
 \hline
\makecell[l]{Cellular Networks} & \makecell[l]{Quartz Crystals (TCXO)} & \makecell[l]{GPOS}  & \makecell[l]{PTP, NTP} & \makecell[l]{Ethernet} & \makecell[l]{City Scale}  \\
 \hline
 \makecell[l]{Data Centers} & \makecell[l]{Quartz  Crystals (TCXO)} & \makecell[l]{GPOS, Hypervisor}  & \makecell[l]{PTP,  NTP, Huygens} & \makecell[l]{Ethernet} & \makecell[l]{$~100$ meters}  \\
 \hline
  \makecell[l]{Smart Homes} & \makecell[l]{Quartz Crystals} & \makecell[l]{RTOS}  & \makecell[l]{FTSP, TPSN} & \makecell[l]{BLE, ZigBee} & \makecell[l]{$<100$m}  \\
 \hline
  \makecell[l]{Wireless Body Area Networks} & \makecell[l]{Quartz Crystals} & \makecell[l]{RTOS}  & \makecell[l]{NTP, FTSP} & \makecell[l]{BLE} & \makecell[l]{$\sim 1$m}  \\
 \hline
\end{tabular}
\caption{Time stack composition used by various CPS. They are composed of different technologies, however, they each have a time source, an operating system that manages this time source and a time synchronization protocol.} 
\label{tab:time-stack-diversity}
\end{table*}