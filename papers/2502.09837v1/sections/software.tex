\section{Software Issues} 
This section examines the security vulnerabilities of the software layer of CPS, that could be exploited to attack their timing stack. These vulnerabilities are exploited by an adversary with \textit{privileged access} to the system. 

\begin{table}
\scriptsize
\centering
\begin{tabular}{
 | c | c | c | c | }
 \hline
  \textbf{System} & \textbf{Application Domain} & \textbf{Total} & \textbf{Critical} \\
 \hline
 \hline
Xen  & Virtualization & $173$ & $74$   \\
 \hline
 \hline
 Linux Kernel & Virtualization, Desktop, Embedded & $1198$ & $549$   \\
 \hline
 Android  & Embedded & $4574$ & $1971$  \\
 \hline
 iOS & Embedded & $1557$ & $969$   \\
 \hline
 FreeRTOS  & Low End Embedded & $14$ & $4$   \\
 \hline
 \hline
  Nvidia Tegra  & TEE & $22$ & $17$   \\
 \hline
  Linaro OPTEE  & TEE & $50$ & $7$   \\
 \hline
\end{tabular}
\caption{List of vulnerabilities discovered in system software used by different application domains~\cite{cve-details}.}
\label{tab:cve-stats}
\end{table}

\subsection{System Software Bugs} The vulnerabilities of the system software often lead to a non-privileged adversary to gain escalated privileges~\cite{sanitizing-for-security}. With escalated privileges, the adversary also gains ability to attack the \textit{local clock}.

\noindent\textbf{\texttt{I05.} Privilege Escalation Attacks.} System software provides interfaces to the time-sync protocol (network layer) to update the \textit{local clock}~\cite{linux-adjtime} and align it with the network time. While this capability is crucial for time synchronization, it can also be exploited by an adversary with elevated privileges to manipulate system's time. Such an attacker is free to launch \textit{time travel} (\textbf{\texttt{A1}}), \textit{time warping} (\textbf{\texttt{A2}}) or \textit{random error} (\textbf{\texttt{A3}}) attacks using this interface. This attack is fundamentally enabled by the privilege escalation vulnerabilities that are prevalent in commodity systems. Table~\ref{tab:cve-stats} provides a list of vulnerabilities discovered in system software of various platforms between 2018 and 2023, each representing a potential threat to the system's timing stack. Note that the attacker does not need to find a new vulnerability in the victim's software. It can also exploit privilege escalation vulnerabilities discovered by others but not yet patched by the system administrator.

\noindent\textbf{\texttt{I06.} Untrusted device drivers.} On most systems, device drivers execute with privileged access, sharing the same context as the operating system (figure~\ref{fig:software-stime-stacks}). Drivers installed from untrusted sources may contain malicious code~\cite{sok-attacks-on-software-supply-chains} that executes with elevated privileges and may launch attacks against the timing stack. It can do so via one of the following mechanisms: i) alter the \textit{timing counters} ($I04$) used by the \textit{local clock}. When timing counters are protected, ii) it would locate the system clock data by scanning physical memory (leveraging its privileged position). Once located, it can manipulate the \textit{local clock}. If the physical memory scan is infeasible, iii) the malicious driver may register an interrupt and configure it to trigger frequently. It seeks to intercept and delay system tick updates for the \textit{local clock}. The first two of these mechanisms allow the adversary \textit{precise control} to launch any of the attacks (\textbf{\texttt{A1-3}}) described in section~\ref{subsec:attack-types}. While, the last mechanism offers \textit{less control} and is likely to result in \textit{increased timing error} (\textbf{\texttt{A3}}).

\begin{figure}[h]
    \centering
    \includegraphics[scale=0.12]{figures/software-stack.pdf}
    \caption{Attack surfaces: a) general purpose platforms, b) virtualized environments and c) with a unprivileged TEE.}
    \label{fig:software-stime-stacks}
\end{figure}

\noindent\textbf{\texttt{I07.} Virtual timer interrupts in the cloud.} Hypervisor is the most privileged software in virtualized environments and enables hardware resource sharing among multiple operating systems (guests). As shown in the figure~\ref{fig:software-stime-stacks}(b), it provides virtual instances of the physical hardware components to each guest. These emulated components also include hardware counters and timer interrupts~\cite{time-stack-hyp-paratick}. Hypervisor, just like an OS, contains bugs (see Xen in table~\ref{tab:cve-stats}) and may get compromised by a malicious agent. Such a hypervisor may manipulate the emulated hardware to alter guests' view of time. For instance, it can launch \textit{time warping attack} (\textbf{\texttt{A2}}) by delaying the timer interrupt and slow down the counter, or vice versa, without notifying a guest. Beyond \textit{warping guests' time}, the hypervisor can also manipulate time by shifting it backward or forward (see $I04$). While the specific mechanisms may vary across platforms, the inherent design of virtualized platforms enables a malicious hypervisor to launch timing attacks.

\subsection{TEE Limitations}\label{subsec:tee-limitation}
Hardware-based TEEs are designed under a strong threat model, treating the system software (e.g., OS, hypervisor etc.) as potentially malicious. These TEE designs follow two paradigms: i) user-space TEEs, exemplified by Intel SGX~\cite{intel-sgx-explained}, and ii) privileged TEEs, like ARM Trustzone~\cite{sok-trustzone-cves} (see Figure~\ref{fig:tee-paradigms}). Both designs, however, aim to secure sensitive code and data. Enabling a secure time stack within these enclaves have been a challenge; especially in the case of the former due to the limited hardware access~\cite{time-stack-abouttime}. 

\noindent\textbf{\texttt{I08.} TEE Design Limitations.}
User-space TEEs (HETEE, Fidelious, HIX, Intel SGX) protect the user application's code and data via cryptography~\cite{sok-hardware-tee}. However, they fall short of securing the time stack due to limited hardware access. The TEE software's access to system's hardware resources such as interrupts, timers and network devices is mediated by the untrusted OS (see figure~\ref{fig:software-stime-stacks}c). A compromised OS can intercept and manipulate the TEE software's access to the timing resources. For instance, timing API $sgx\_get\_trusted\_time$ provided by Intel SGX, a user-space TEE, is vulnerable to delay attacks by a compromised OS~\cite{time-stack-timeseal}. These attacks are launched by directly exploiting lack of direct TEE access to the hardware counter causing increased uncertainty in SGX time (\textbf{\texttt{A3}}). Newer SGX iterations mitigate this by enabling direct $TSC$ (hardware counter) access by the TEE software. However, this still does not enable a secure time stack as the $TSC$ is not fully secure and is vulnerable to the compromised OS (see $I04$). Beyond local time, the lack of TEE direct access to network card also prevents it from obtaining trusted time through the network. The TEE's network traffic is handled by an untrusted network driver (see figure~\ref{fig:software-stime-stacks}c) which can add arbitrarily delay timing packets and induce uncertainty in the timing information (\textbf{\texttt{A3}}) received by the TEE.

\noindent\textbf{\texttt{I09.} Compromised TEE Software.}
Privileged TEEs such as ARM Trustzone does have access to a secure counter and timer, allowing the TEE software to maintain secure \textit{local clock}. Unfortunately, this secure clock may still get exposed to adversaries because of the TEE software vulnerabilities which are discovered regularly~\cite{sok-sgx-fail, sok-trustzone-cves} (see Nvidia Tegra \& Linaroo OPTEE in table~\ref{tab:cve-stats}). Many TEE vulnerabilities allow adversary to manipulate code inside the TEE~\cite{boomerang-trustzone} and put the adversary in control of the previously isolated resources of the TEE including its \textit{local clock}. Such an attacker can \textit{hide} itself with relative ease and launch any of the attacks \textbf{\textit{A1-3}} discussed in the section~\ref{subsec:attack-types}. It is important to note that if the TEE software is compromised, an application can no longer trust any timing stack on the system as it already does not trust timing stack maintained by the untrusted OS. 

\begin{figure}[t]
    \small
    \centering
    \begin{subfigure}[t]{0.225\columnwidth}
        \centering
        \includegraphics[width=\columnwidth]{figures/privileged-tee.pdf}
        \caption{}
        \label{fig:privileged-tee}
    \end{subfigure}
    \hspace{20pt}
    \begin{subfigure}[t]{0.215\columnwidth}
        \centering
        \includegraphics[width=\columnwidth]{figures/userspace-tee.pdf}
        \caption{}
        \label{fig:userspace-tee}
    \end{subfigure}
    \caption{a) Privileged TEE design; trusted software has direct (secure) access to hardware resources. b) Un-privileged TEE design; cannot securely access any hardware resource except memory.}
    \label{fig:tee-paradigms}
\end{figure}