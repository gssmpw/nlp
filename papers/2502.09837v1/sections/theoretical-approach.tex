\section{Theoretical Tools for Securing Time}
In this section, we discuss research work that employs theoretical tools to secure the timing stack. In this context, theoretical tools are employed for three distinct tasks: i) establish properties of a system model, ii) proving correctness of a system design i.e. it aligns with the stated goals and iii) verify software implementation of a given time-stack component. 

\subsection{Hardware}
The timing vulnerabilities in the hardware layer primarily result from either physical side channels ($I01$, $I02$) or design limitations ($I03$, $I04$).
Traditional formal verification tools cannot address these issues because mitigating them requires either the addition of new components or changing the existing designs. However, these new designs may be evaluated using standard formal verification tools~\cite{formal-verification-intel}.

\subsection{Software}
Timekeeping functionalities, integrated within broader system software like operating systems or hypervisors, are prone to inherent software vulnerabilities ($I05, I06, I07$). Although formal verification tools offer a means to analyze these systems for potential flaws, the large code bases and complex interaction among various subsystems of the system software make it infeasible to verify them. Despite these challenges, advancements in formal verification techniques have enabled the verification of specific OS components~\cite{formal-verification-eBPF} and hypervisors~\cite{formal-verification-hypervisor-arm, formal-verification-hypervisor-memory, formal-verification-kvm}. Employing these verification tools to assess the security aspects of system software can diminish the timekeeping software's vulnerability exposure ($I05, I06, I07$). Further, timing subsystems may also contain vulnerabilities originating from incorrect implementations. To enhance their security, it is crucial to apply the latest formal verification tools to verify the correctness of timing subsystems including trusted timing services such as Timeseal~\cite{time-stack-timeseal} and T3E~\cite{trusted-time-t3e}. As far as our knowledge extends, applying formal verification tools to timekeeping software remains an open area of research.

\subsection{Network} \label{subsec:network-theory}
The use of theoretical tool to establish trust in the timing stacks have almost exclusively focused on its network component i.e. time synchronization. This literature has focused on following lines of work:

\noindent\textbf{\texttt{T01.} Establishing Secure Time Synchronization Requirements.} 
Theorem proving tools have been used to establish requirements for secure clock synchronization. Narula et al.\cite{net-sync-gps-sec-transfer} constructed formal models for one-way and two-way time transfer, assuming a line-of-sight link between the systems and a threat model with a MITM adversary. They present proves for i) one-way time transfer's inherent susceptibility to delay attacks (as discussed in $I12$) and ii) essential requirements for a two-way secure time-synchronization protocol. Building upon this work, they study two-way time synchronization over a multi-hop network where systems at the both ends implement cryptography~\cite{net-sync-gps-sec-sync}. They put forward the prerequisites for a secure clock synchronization algorithm applicable to protocols like PTP~\cite{ptp-std-doc}. Among other requirements, they show that the timing packets must travel along the shortest path between the server and the client to completely prevent delay attacks. This has an important implication that \textit{delay attacks, over the network, cannot be prevented entirely if they do not guarantee shortest path traversal.} This is indeed the case of today's internet that employs TCP/IP stack for networking.

\begin{table*}[t]
\footnotesize
\centering
\begin{tabular}{p{2 cm} p{3.5cm}  p{0.5cm}  p{0.5cm}  p{0.5cm}  p{0.5cm}  p{0.5cm}  p{0.5cm}  p{0.5cm} p{0.5cm}  p{0.5cm}  p{0.5cm}  p{0.5cm}  p{0.5cm} }
 \multicolumn{1}{c}{} & \multicolumn{1}{c}{} & \multicolumn{8}{c}{Systems Approach} & \multicolumn{4}{c}{Theoretical Approach} \\
 \cmidrule(lr){3-10} \cmidrule(lr){11-14}
    & & \texttt{D01} & \texttt{D02} & \texttt{D03} & \texttt{D04} & \texttt{D05} & \texttt{D06} & \texttt{D07} & \texttt{D08} & \texttt{T01} & \texttt{T02} & \texttt{T03} & \texttt{T04} \\
 \hline
  Hardware Issues & Physical Side Channels & \halfcirc & \emptycirc & \emptycirc & \emptycirc & \emptycirc & \emptycirc & \emptycirc & \emptycirc & \emptycirc & \emptycirc & \emptycirc & \emptycirc \\
& Design Limitations & \emptycirc & \halfcirc & \emptycirc & \emptycirc & \emptycirc & \emptycirc & \emptycirc & \emptycirc & \emptycirc & \emptycirc & \emptycirc & \emptycirc \\
\hline
Software Issues & System Software Bugs & \emptycirc & \emptycirc & \halfcirc & \emptycirc & \emptycirc & \emptycirc & \emptycirc & \emptycirc & \emptycirc & \emptycirc & \emptycirc & \emptycirc \\
& TEE Limitations & \emptycirc & \emptycirc & \halfcirc & \fullcirc & \emptycirc & \emptycirc & \emptycirc & \emptycirc & \emptycirc & \emptycirc & \emptycirc & \emptycirc \\
\hline
 & Limited Crypto Adoption & \emptycirc & \emptycirc & \emptycirc & \emptycirc & \halfcirc & \emptycirc & \emptycirc & \halfcirc & \fullcirc & \halfcirc & \emptycirc & \emptycirc \\
Network Issues & Availability Issues & \emptycirc & \emptycirc & \emptycirc & \emptycirc & \emptycirc & \halfcirc & \halfcirc & \halfcirc & \fullcirc & \emptycirc & \emptycirc & \halfcirc \\
& Implementation Issues & \emptycirc & \emptycirc & \emptycirc & \emptycirc & \emptycirc & \emptycirc & \emptycirc & \emptycirc & \emptycirc & \emptycirc & \halfcirc & \emptycirc \\
\hline
\end{tabular}
\caption{Research contributions towards mitigating timing stack issues utilize both \textit{system-based} and \textit{theoretical} approaches. A full circle indicates that the defense technique mitigates all issues within the category, a half circle suggests that some of the issues in a category are addressed, and an empty circle signifies the given defense's lack of mitigation for issues in the specified category. While system-based defenses tackle attack surfaces across all three layers of the timing stack, theoretical solutions predominantly concentrate on the network component.}
\label{tab:system-v-defense}
\end{table*}

\noindent\textbf{\texttt{T02.} Proving Correctness of the Time-Sync Protocols.} Formal verification tools have been used to prove the correctness of fault-tolerant clock sync protocols. For instance, Schwier et. al.~\cite{theory-MechanicalVO} used protocol verification system (PVS) to verify a generalized time-sync protocol's correctness based on conditions established by Schneider~\cite{theory-schneider-conditons} for byzantine faults. Improving on this work, Barsotti et al., ~\cite{theory-schneider, theory-deductivetools} used deductive tools to prove correctness of fault-tolerant clock synchronization algorithms proposed by Lamport-Melliar\cite{clock-sync-fault-presence} and Lundlies-Lynch~\cite{clock-sync-fault-tolerant}. This research offers a promising direction for formal verification of time-sync protocols dealing with malicious faults.

Recent efforts regarding secure time synchronization using mathematical analysis have shifted focus to wireless sensor networks~\cite{theory-attack-resilient-pulse-coupled, theory-self-stablizing}. Most of these works assume a MITM attack model where few nodes in the network are compromised ($I10-I15$). Wang et al.\cite{theory-attack-resilient-pulse-coupled} introduced an attack-resilient pulse-coupled synchronization scheme for wireless sensor networks, deriving necessary conditions and analytically proving that it guarantees secure synchronization in the presence of a single malicious node. Another work by Hoepman et. al.\cite{theory-self-stablizing} presented a self-stabilizing clock synchronization algorithm, for wireless sensor networks. Their design is secure against pulse delay attacks by malicious nodes and they providing proofs for the correctness of their random beacon scheduling algorithm. These works demonstrate the potential of using theoretical tools for verify time-sync protocol designs.

\noindent\textbf{\texttt{T03.} Verification of Protocol Implementations.} Formal verification techniques can be used to verify the implementations of time-sync protocols ($I17$). This is demonstrated by Luca et al., who performed the automated verification of the gossip time-sync protocol's~\cite{model-checking-gossip} implementation using parameterized model checking. However, gossip is rather a simple protocol, and the methods employed for its verification do not readily extend to more complex protocols such as NTP, PTP etc~\cite{net-sync-openchallenges}. However, despite its limitations, partial verification of time-sync implementations using the existing formal verification methods can yield important security insights. In one such instance, Dieter et al.\cite{theory-nts-specs} performed (partial) formal verification of Network Time Security (NTS -- RFC8915) specifications~\cite{nts-rfc} and discovered two vulnerabilities in the analyzed version~\cite{theory-nts-formal-analysis}, which are currently being addressed. It shows that future research on enabling complete verification of widely used time-sync protocols would greatly contribute to their security.

\noindent\textbf{\texttt{T04.} Mitigating Attacks on Time Synchronization.} Beyond analyzing complete protocol design, mathematical tools have also been used to study specific time-sync attacks. This research primarily focuses on mitigating delay attacks against time-sync protocols by a network adversary. For instance, Mizrahi et al. proposed a multi-path time synchronization scheme designed to resist delay attacks by a man-in-the-middle attacker. They leverage game theory to provide proofs for the delay resiliency of their design~\cite{multi-path-game-theory}. Similarly, Anto et al. propose modifications to PTP aimed at mitigating delay attacks and formally verified the correctness of their proposed updates~\cite{theory-formal-attack-1588}. Likewise, Moussa et al. proposed extensions to the PTP protocol and formally proved the correctness of these protocol extensions~\cite{theory-smart-grids, theory-ptp-extension}. However, the proposed extensions are domain specific as they rely on redundant master clocks on power grid substations mandated by IEC 61850. Another work by Lisova et al., took a game-theoretic approach, modeling the interaction of a man-in-the-middle attacker introducing asymmetric delays to PTP packets and a network inspection system collecting clock offset information. Their work uses a game-theoretic tool to predict attacker strategies and develop mitigation mechanisms accordingly~\cite{theory-game-theory-1588}. While delay attacks have been a dominant subject of this research, defenses against other attacks would equally benefit from the use of theoretical tools.