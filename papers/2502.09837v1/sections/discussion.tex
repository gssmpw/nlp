\section{Building a secure timing infrastructure}
Our evaluation of the existing time security research shows a patchwork of solutions aimed at mitigating one or few security issues. We advocate for a fundamentally different paradigm and put forward recommendations for designing an integrated trusted time stack.

\noindent\textbf{System-wide Trusted Timing Services.} We embrace \emph{hardware-software co-design} based approach for building an integrated trusted time stack. At the heart of this new time stack is a fixed frequency monotonic counter $\mathbf{C}$, which is unaffected by DVFS and immune to writes by the software. Access to this counter and its frequency must be an atomic operation and available system-wide irrespective of the software's privilege level, trust and virtualization status. System-wide availability allows each application or other software component to maintain its own \textit{local clock} in an untrusted environment ($I03-09$). More importantly, this design also mitigates $I09$ in contrast to existing solutions~\cite{time-stack-timeseal, trusted-time-t3e}. The second component to this design is secure time-synchronization, which should be achieved by a dedicated co-processor outside the control of untrusted software. TimeCard~\cite{time-card} by OCP Time Appliance Project~\cite{ocp-tap} offers inspiration for this design. While their design does not consider time-sync security in particular, there are no fundamental barriers in implementing it. Finally, such module must communicate the time-sync parameters to the applications securely. Again, we recommend introducing extra registers (with system-wide availability) on the SoC that are updated by the time-sync co-processor but cannot be written to by any processor controlled by the untrusted software. 

It is important to remember that providing unrestricted access to high resolution time to untrusted software is a double edged sword~\cite{browser-timing-side-channel} because despite immense benefits, it can also enable side channel attacks. Hence, we recommend enabling system firmware (e.g., SMM mode on x86, secure monitor on ARM, etc.) to control the resolution and access permissions to the counter $\mathbf{C}$. These firmwares afford better protection by dint of their small TCBs that can be formally analyzed. It allows us to provide flexibility to the system designers, who can configure trusted time stack to meet their needs, while ensuring reasonable amount of trust in time.

\noindent\textbf{Complimenting System Design with Theoretical Tools.} We recommend using theoretical tools in conjunction with the system-based approaches for verifying trusted time stack design and its implementations. They have made important contributions towards a secure time stack such as working out requirements for secure time-sync~\cite{net-sync-gps-sec-sync}, verifying implementations of time-sync protocols~\cite{model-checking-gossip}, proving the security guarantees offered by Chronos~\cite{net-sync-chronos} and identifying issues in the implementations of NTS~\cite{theory-nts-formal-analysis}. Yet, the use of these tools to improve time stack security has been rather limited as shown in table~\ref{tab:system-v-defense}. This is, in part, due to the lack of automated tools for time-sync protocol verification as pointed out by Swen et al~\cite{net-sync-openchallenges}. Development of an automated tool for verifying models with time and clock abstractions represents a key research challenge.

\noindent\textbf{Delay Attacks.} Delay attacks represent one of the most challenging problem for the timing stack. As discussed in $T01$, solving this problem for two-way time-sync requires network packets to traverse the shortest network path~\cite{net-sync-gps-sec-sync}, in addition to other mechanisms. An important implication is that these attacks cannot be avoided completely. Nevertheless, it is possible to deal with them as we discussed in $D06 \& D07$. However, these solutions either mitigate against limited adversaries ($D06$) or result in performance degradation ($D07$). Future research must investigate improving timing stack's resiliency to delay attacks with lower trade-offs than offered by the state-of-the-art.