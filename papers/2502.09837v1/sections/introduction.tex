\section{Introduction}\label{sec:introduction}
An accurate perception of time is indispensable in modern digital ecosystems. Critical applications such as Public Key Infrastructure (PKI)\cite{time-stack-abouttime}, streaming authentication protocols such as TESLA~\cite{tesla-cryptography}, smart grid operations\cite{intro-tech-report-smart-grid, intro-smart-grid-pmu-attack}, autonomous vehicle perception systems\cite{hardware-chronos-slam-attack}, and high-frequency trading\cite{intro-high-frequency-trading} depend on precise timing for their proper functionality. Further, the absence of accurate time synchronization endangers \textit{Time Difference of Arrival} (TDoA)-based applications, which are vital for electronic warfare\cite{intro-electronic-warfare}, vehicular triangulation\cite{intro-vehicle}, sonar operations in marine environments\cite{intro-sonar}, environmental conservation\cite{intro-gunshot-localization}, and indoor localization\cite{intro-indoor-localization}. Disrupting these cyber-physical systems (CPS), by exploiting their timing stacks, poses a serious threat to society.

Despite the pivotal role of timing infrastructure, it remains susceptible to adversarial manipulations. NTP~\cite{ntpv4-rfc}, the predominant time-sync protocol on the internet, could face continental-scale disruptions due to a few malicious servers within its pool~\cite{shark-ntp-pool}. PTP~\cite{ptp-std-doc}, essential for high-precision synchronization, is vulnerable to frequency manipulations by compromised nodes despite using dual authentication~\cite{net-sync-ptp-covert-channel}. GPS, integral to critical infrastructure, can be easily spoofed~\cite{gps-spoofing-fundamentals}. Furthermore, recently developed high-precision time-sync protocols, e.g., Huygens~\cite{huygens} and Sundial~\cite{sundial}, as well as the IoT framework \textit{Matter}'s timing stack~\cite{matter}, do not adequately address their security. The prevalence of security vulnerabilities in existing and emerging time stacks warrants a systematic examination of their issues.

    
\iffalse
\begin{figure*}[ht]
    \small
    \centering
    \begin{subfigure}{0.34\textwidth}
        \centering    
        \includegraphics[scale=0.15]{figures/time-stack-updated.pdf}
        \caption{}
        \label{fig:time-stack-example}
    \end{subfigure}
    \begin{subfigure}{0.32\textwidth}
        \centering   
        \includegraphics[scale=0.15]{figures/two-way-transfer.pdf}
        \caption{}
        \label{fig:two-way}
    \end{subfigure}
    \begin{subfigure}{0.32\textwidth}
        \centering 
        \includegraphics[scale=0.15]{figures/one-way-transfer.pdf}
        \caption{}
        \label{fig:one-way}
    \end{subfigure}
    \caption{a) Time stack in a modern CPS. b) Two-way time synchronization. c) One-way time synchronization.}
    \label{fig:sync-paradigms}
\end{figure*}
\fi

In this paper, we present the first systematization of knowledge for time security in a typical CPS. While existing research on this topic has made important contributions, it has been limited to examining single protocols (e.g., NTP~\cite{ntp-replay-drop-attack, shark-ntp-pool, theory-nts-specs}, PTP~\cite{ptp-futile-encryption, net-sync-ptp-sec, net-sync-ptp-covert-channel}, and GPS~\cite{gps-spoofing-fundamentals, gps-anti-jamming-post-correlation, gps-anti-jamming-post-wavelet, gps-anti-spoofing-static} etc.) or specific types of attacks (e.g., delay attacks~\cite{ptp-futile-encryption, multi-path-game-theory}). In contrast, we propose the idea of a timing framework and provide a holistic view of time stack security; we analyze the vulnerabilities of \textit{physical timing components} (e.g., quartz crystals), \textit{software-based clocks}, and the \textit{time-sync protocols}. Utilizing this timing framework, we highlight existing and emerging attack surfaces, gain insights into the extent and scope of proposed countermeasures, and identify open research problems.

Leveraging our framework, we discover previously uncharted attack surfaces threatening the timing stack of CPS, and present case studies that underscore the gravity of these threats. Through our study, we demonstrate that these attacks surfaces, constituting \textit{side-channel attacks on physical timing components}, and \textit{risks posed by privileged software to the integrity of system time}, are insufficiently addressed by the current body of timing security research. This oversight often results in conflating timing stack security with mere time synchronization issues, predominantly attributing threats to \textit{network-based attackers} (e.g., DNS cache poisoning on NTP~\cite{shark-ntp-pool}, delay attacks on PTP~\cite{ptp-futile-encryption}, and GPS spoofing~\cite{intro-attacks-critical}). While some recent research has focused on securing time within Trusted Execution Environments (TEEs) to guard against privileged software adversary, our analysis reveals that this area is fraught with unresolved challenges. Our work fills this gap through a comprehensive systematization of security vulnerabilities across the timing stack's hardware, software, and network dimensions.

We also show that research methodologies used for building secure timing services can be classified into two categories: \textit{system-based} approach, which utilizes design mechanisms like NTP's message authentication~\cite{ntpv4-rfc} to deter attacks, and \textit{theoretical} approach, applying mathematical tools (e.g., theorem proving, model checking, game theory) for the security analysis of timing protocols. Despite their significant contributions to timing stack security, very few works successfully utilize both strategies together. Their limited interaction is exemplified by the formal analysis of NTS~\footnote{NTS is a secure version of NTP.} by Tiechel et al.~\cite{theory-nts-formal-analysis}, which falls short of analyzing full specifications due to the analysis tool's inability to model time and clocks. Nevertheless, we evaluate the two approaches to determine their coverage of the timing stack's security issues and contrast their contributions.

Moreover, we identify a significant risk to CPS security posed by state-of-the-art trusted timing services like Timeseal~\cite{time-stack-timeseal} and T3E~\cite{trusted-time-t3e}. These solutions provide TEE~\footnote{Trusted Execution Environment.}-confined trusted timestamps to user applications, requiring them to execute inside the TEE. Given that such applications may come from untrusted sources, they pose a threat to sensitive code and data inside the TEE. As a result, these solutions inadvertently increase system-wide security risks. Drawing from our timing stack security analysis, we provide recommendations for an alternative design that aligns with the goal of overall system security. Additionally, we highlight future research directions to advance this domain.

In summary, we make the following contributions: (1) We conduct the first systematic study of timing stack security in light of the proposed timing framework. (2) We identify previously overlooked attack surfaces through case studies. (3) We propose an intuitive taxonomy for categorizing timing stack vulnerabilities and analyze relevant literature. (4) We categorize timing stack solutions into system-based and theoretical, highlighting their unique contributions and limitations. (5) Finally, we identify open research challenges related to the timing stack's security.

