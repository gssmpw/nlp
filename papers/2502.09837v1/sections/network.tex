\section{Network Issues} \label{sec:network-issues}
This section delves into the network layer's vulnerabilities, pivotal for synchronizing time across digital systems. Such synchronization is vital for applications ranging from digital payments to industrial automation. Yet, it faces threats from \textit{attackers controlling network devices} (on-path attacker) or \textit{possessing privileged access to a victim's local network stack} (off-path attacker).

\subsection{Limited Use of Authentication Mechanisms} Cryptography techniques, used by protocols like NTP~\cite{ntpv4-rfc} and PTP~\cite{ptp-std-doc}, play a critical role in ensuring data integrity and origin authentication of the time-sync traffic, thwarting man-in-the-middle (MITM) attacks. Yet, several issues persist regarding the adoption of these methods making time-sync protocols vulnerable to attacks.


\noindent\textbf{\texttt{I10.} False packet injection.} A MITM adversary can impersonate a genuine time server and send false time-sync packets to the target. These attacks may result from weak assumptions underlying the authentication mechanism adopted by the time-sync protocol. For instance, the reliance of NTP's broadcast mode authentication protocol TESLA~\cite{tesla-cryptography} (also used by PTP~\cite{ptp-std-doc}) on loosely synchronized devices creates a circular dependency between authentication and time-sync~\cite{ntp-replay-drop-attack}, rendering the former useless. Moreover, infiltration of malicious servers in the pool of legitimate time servers is  a genuine concern~\cite{shark-ntp-pool, devil-time-origin}. It is because cryptographic authentication only protects against a MITM attacker and the malicious servers render it ineffective. This allows Kwon et. al., to use a handful of malicious time servers, injected to the NTP pool~\cite{ntpd-pool-project}, to disrupt time-sync clients spread over entire countries~\cite{shark-ntp-pool}. Despite their shortcomings, authentication techniques make packet injection attacks harder. However, the adoption of these mechanisms is not universal. For instance, Huygens~\cite{huygens}, RBS~\cite{Elson2003RBS}, FTSP~\cite{ftsp-2004}, TPSN~\cite{tpsn-2003} do not implement any origin authentication mechanisms and have no protection against packet injection. The severity of the issue is evident from the fact that \textit{RBS, FTSP and TPSN} are among the most cited protocols for time-sync in sensor networks. In contrast, secure time synchronization protocols such as the one introduced by Ganeriwal et. al.~\cite{net-sync-wsn-sec-prot} has received an order of magnitude fewer citations (see table~\ref{tab:time-sync-wsn-citations}). Packet injection is one of the most potent attacks against time-sync protocols and could be used to induce \textit{time travel, warping or just increased uncertainty} (\textbf{\texttt{A1-3}}) in the victim's view of time.

\begin{table}
\scriptsize
\centering
\begin{tabular}{ | c | c | c | c | }
 \hline
  \textbf{Protocol} & \textbf{Authentication} & \textbf{Date Published} & \textbf{Citations} \\
 \hline
 \hline
  RBS~\cite{Elson2003RBS}  & \textit{No} & $2003$ & $3927$   \\
 \hline
   TPSN~\cite{tpsn-2003}  & \textit{No} & $2003$ & $3206$   \\
 \hline
   FTSP~\cite{ftsp-2004}  & \textit{No} & $2004$ & $3052$   \\
 \hline
 \hline
   Secure Time-Sync~\cite{Elson2003RBS}  & \textit{Yes} & $2005$ & $278$   \\
 \hline
\end{tabular}
\caption{One of the earliest time-sync protocols proposed for wireless sensor networks (WSNs). The protocols (RBS, TPSN and FTSP) that do not incorporate authentication mechanisms have received an order of magnitude more citations than the protocol (STS) that make use of cryptography mechanisms. \textit{Source: Google Scholars as of Jan 22, 2024.}}
\label{tab:time-sync-wsn-citations}
\end{table}

\noindent\textbf{\texttt{I11.} Packet modification.} Correct implementation of authentication protocols prevents false packet injection but may not prevent against packet modification. This is best exemplified by PTP, which makes use of authentication~\cite{ptp-std-doc} to protects the PTP packets except the correction field of the packet header. This field allows each network node to update correction field with the packet processing delay. PTP uses this information to achieve better time-sync accuracy by eliminating the variable network delays~\cite{net-sync-ptp-covert-channel}. However, a MITM attacker (on-path or off-path) can add incorrect information to this field and manipulate the PTP client. Jacobs et, al., use this channel to introduce \textit{significant offsets} (\textbf{\texttt{A3}}) to the victim device while \textit{avoiding detection}. They could also induce the victim device to change its clock frequency (\textbf{\textit{A2}}), resulting in an even larger time deviation from the time server~\cite{net-sync-ptp-covert-channel}. We note that this attack is not PTP specific, and any time-sync protocol seeking network delay information may be subject to this attack. Finally, we also note that this technique is less sophisticated as it does not require by-passing authentication requires.

\noindent\textbf{\texttt{I12.} Packet replay.} Authentication issues discussed in $I10$ can also result in replay attacks. In this attack, the adversary repeatedly sends one or a sequence of pre-recorded time server packets to the victim. Packet replay attacks have been successfully demonstrated against NTP broadcast mode~\cite{ntp-replay-drop-attack}. Malhotra et. al. exploited limitations in existing NTP client implementations to keep the victim stuck at a single point in time (\textbf{\texttt{A1}}). They point out that the one-way nature of the time-sync traffic (NTP broadcast mode) enables this attack. It implies that other one-way time synchronization protocols such as RBS~\cite{Elson2003RBS} may also be susceptible to this attack.

\noindent\textbf{\texttt{I13.} Spoofing Wireless Timing Signals.} Time-sync protocols like GPS, ROCS~\cite{ROCS-FM-Beacons}, Syntonizor~\cite{Syntonizor-AC-powerlines} and WizSync~\cite{WizSync-Wifi-Beacons} work using a periodic wireless timing signal that is transmitted directly from the timing source(s) to the clients i.e. over a single hop. These protocols lack authentication mechanisms allowing adversaries to spoof timing signals. This attack is the equivalent to packet manipulation attack on packet exchange based protocols (NTP~\cite{nts-rfc}, PTP~\cite{ptp-std-doc}, FTSP~\cite{ftsp-2004} etc.). Similar to the packet manipulation attacks, an external adversary mimics a trusted timing source but transmits incorrect timing information. It does so by generating a powerful spoof signal, using antenna(s), that can overpower the legitimate signal. Such an attacker often stays \textit{stealthy} while introducing uncertainty in the victim's local clocks~\cite{gps-spoofing-fundamentals} (\textbf{\texttt{A3}}). Satellite based global positing systems (GPS) is a typical target of this attack~\cite{gps-spoofing-21}. However, other time-sync protocols in this category (e.g., ROCS,~\cite{ROCS-FM-Beacons}, WizSync~\cite{WizSync-Wifi-Beacons} and Syntonizor~\cite{Syntonizor-AC-powerlines} etc.) haven't seen significant spoofing attacks due to their limited application. Nevertheless, signal spoofing remains a viable attack option for a motivated adversary.

\subsection{Availability Issues}
Beyond modifying timing packets, time-sync is also affected by just delaying the transmission of the timing information (as discussed in section~\ref{subsec:case-studies}). An adversary may leverage this observation and use unpredictable delays to add errors to the time-sync process or it  may outright block time-sync traffic headed towards the victim. 

\noindent\textbf{\texttt{I14.} Packet delay.} Time synchronization protocols determine the time offset between the server and the client by exchanging packets over the network. These network packets experience delays causing uncertainty in the exchanged timing information and the corresponding offset calculations (see section~\ref{subsec:case-studies}). Time-sync protocols rely heavily on precise network delay measurements to remove this uncertainty in the offset estimations. NTP~\cite{ntpv4-rfc} solves this challenge by measuring round trip times (RTTs) and computes network delay as half of the RTT, assuming symmetric delays~\cite{rfc1305}. On the other hand PTP measures the network delays by mandating each processing node to update the PTP packets with its resident delay (see $I11$). While effective under normal network conditions, these delay estimation mechanisms are not robust to adversarial delays. A malicious network node may introduce additional network delays~\footnote{In case of NTP, the server-bound and client-bound packets are delayed by different duration while for PTP the adversary would not update PTP packets with its resident delay} to degrade the synchronization performance. For instance, Annesi et. al. show that delay attacks against PTP can induce errors of several milliseconds, accumulating over time to even larger values under a sustained attacks~\cite{ptp-futile-encryption} (\textbf{\texttt{A2}}). However, vulnerability to delay attacks extend beyond NTP and PTP; virtually all time-sync protocols are susceptible to these attacks.

\noindent\textbf{\texttt{I15.} Packet drop.} 
Intercepting and dropping time-sync packets is a simple yet effective MiTM attack that desynchronizes the victim device from its time server. Facing this \textit{denial-of-service} attack, the victim solely relies on its \textit{local clock} which diverges away from the server time (\textbf{\texttt{A3}}) dictated by the stability of the victim's time source. For low-end systems using inexpensive quartz crystals, the time difference may accumulate to several minutes per day. In contrast, devices using more stable oven-controlled quartz oscillators may experience deviations of only a few seconds in the same period. However, despite its effectiveness, the victim can deduce potential instances of this attack, with relative ease, from sudden unavailability of time-server.

\noindent\textbf{\texttt{I16.} Blocking Wireless Timing Signals.} For single-hop wireless time synchronization (GPS~\cite{gps-spoofing-fundamentals}, ROCS~\cite{ROCS-FM-Beacons}, WizSync~\cite{WizSync-Wifi-Beacons} etc.), denial of service attack takes the form of blocking the wireless timing signal. An adversary achieves this by generating high powered noise in the frequency band used by the wireless timing signal. It requires physical proximity to the target and signal transmission equipment, raising the cost of this attack. Nevertheless, GPS signal blocking techniques have been studied extensively~\cite{gps-jamming-overview} due to ubiquitous use of GPS by defense and civil infrastructure. In principle, other single-hop wireless protocols such as Syntonizor~\cite{Syntonizor-AC-powerlines} and ROC~\cite{ROCS-FM-Beacons} are also vulnerable to these attacks, even though no such attack against them is known.

\subsection{Implementation Issues} In addition to the the communication medium, the end-points of this channel i.e. the applications implementing the time-sync protocol themselves represent an attack surface.

\noindent\textbf{\texttt{I17.} Untrusted time synchronization software.} Applications implementing time-sync protocols may harbor security vulnerabilities of their own. For instance, CVE database lists 98 vulnerabilities, discovered over the years, in the NTP application developed by \textit{NTP.org}~\cite{ntp-cve-details}. This application is used by both the time-sync clients and servers,~\footnote{It is recommended for servers joining the NTP pool project~\cite{ntpd-pool-project}.} and can be exploited by an adversary with access to \textit{privileged execution} on the victim device or \textit{a network connection to the NTP application}. An attack exploiting client side application vulnerability would only affect a single machine, however, the server side exploit would affect time alignment at all of its clients. Further, these attacks may cause the target applications to crash pausing the time-sync service or may just degrade time-sync performance (\textbf{\texttt{A3}}) over longer periods. It is worth pointing out time-sync applications executing in the privileged context present an even bigger risk, as any vulnerability in them could compromise the system beyond time-sync service.