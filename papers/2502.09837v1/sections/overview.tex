\section{Systematization}\label{overview}

Conducting an exhaustive security analysis of the timing stack in modern CPS presents numerous challenges. The research literature has yet to fully assess the impact of adversarial manipulations on the timekeeping and measurement components of the timing stack (Figure~\ref{fig:time-stack-example}). This gap necessitates a comprehensive review of broader CPS security concerns and their connections to timing stack integrity. While existing studies often employ limited threat models focusing on specific timing security aspects, our approach leverages a broad threat model for a systematic examination. It requires us to evaluate security solutions designed for narrower threat models against a broader one. Additionally, threat models in current research may include caveats that complicate our analysis, such as Chronos~\cite{net-sync-chronos}, which proposes a secure multi-path NTP solution but overlooks vulnerabilities introduced by intermediary devices like home routers. Moreover, security enhancements for protocols can inadvertently introduce new vulnerabilities; for instance, Chronos' DNS attack mitigation strategy may inadvertently simplify these attacks~\cite{pitfalls-chronos}. This underscores the importance of meticulous analysis to ensure no legitimate threats are overlooked. Next, we describe the adversary model underpinning our study, outline our work's scope, and present the systematization of framework employed to cope with these challenges.

\noindent\textbf{Threat model.} We adopt a adversary model that comprehensively analyzes threats to the timing stack. We assume adversary's manipulation of the victim's perception of time to either be its primary objective or a means to undermine other system functionalities. The adversary may have following types of access to the victim: i) \textit{physical device}, ii) remote \textit{privileged code execution} and iii) \textit{control of a network device} on the path between the timing server and the client. Secondary goals of such adversary may include: i) launching a \textit{control}led attack e.g., controlling the extent of timing uncertainty introduced at the victim, ii) staying \textit{stealthy} in order to launch sustained attacks, or iii) rendering the timing service unusable (\textit{denial of service}--DoS attack). In Table~\ref{tab:case-study-attack-requirements}, we use our threat model to characterize adversaries from the above case studies (section~\ref{subsec:case-studies}).

\noindent\textbf{Analyzed Time Stacks.} Timing stacks exhibit significant diversity across application domains and with respect to underlying technologies (see table~\ref{tab:time-stack-diversity}). For high-precision requirements, GPS satellites employ atomic clocks as their \textit{time source}, whereas applications with relaxed timing precision, such as wireless sensor networks, might utilize quartz crystals. Similarly, \textit{local clock} on desktop computers, programmable logic controllers (PLCs) and cloud servers are maintained by a general purpose OS, a real-time OS and a hypervisor, respectively. And for distributed applications, time synchronization protocol vary with the network's size and topology. NTP~\cite{ntpv4-rfc} is the default time-sync protocol in wide area networks (WANs) using TCP/IP networking stack. Local Area Networks, such as data centers, may employ PTP~\cite{ptp-std-doc} for high accuracy synchronization, and resource constrained sensor networks may utilize RBS~\cite{Elson2003RBS}. Our work presents a unified framework for analyzing the security of these and other timing stacks, irrespective of the application domain or the adopted technologies.

\begin{figure}[tb]
    \small
    \centering
    \includegraphics[angle=270,width=0.5\columnwidth]{figures/time-stack-compromised.pdf}
    \caption{Various components of a typical CPS and their interactions that may be exploited by an adversary to attack its time stack (in red).}
    \label{fig:time-stack-attack-surface}
\end{figure}

\noindent\textbf{{Systematization Framework.}}
To analyze diverse timing stacks, we identify three layers common to all timing stacks, depicted in Figure~\ref{fig:time-stack-attack-surface}. 1) The \textbf{hardware layer} which contains the primary timing source for the system and the time measurement circuitry i.e. counter and timers. An adversary with \textit{physical access} to the target system may leverage physical side channels or hardware design limitations to manipulate this time measurement infrastructure. 2) The \textbf{software layer} maintains standardized clock and timer abstractions. Often these abstractions, implemented by the system software (OS, Hypervisor etc.), are vulnerable to attacks by \textit{adversaries that gain control of the system by exploiting software bugs}. 3) The \textbf{network layer} is responsible for aligning system time to an external reference. It is susceptible to manipulation by \textit{external adversaries having control of a device on the network path between the victim and the time server}.

We use this systematization framework to offer an exhaustive evaluation of security vulnerabilities within each layer of the timing stack in the forthcoming sections.