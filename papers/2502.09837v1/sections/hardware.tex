\section{Hardware Issues} \label{sec:hardware}
This section delves into vulnerabilities of the timing hardware, highlighting how such weaknesses can be exploited to alter a system's perception of time.

\subsection{Physical Side Channels} 
The physical components of the timing stack are susceptible to side channel attacks such as fault injection and manipulation of the system's thermal characteristics. To exploit these vulnerabilities, the adversary must have \textit{physical access} to the target system.

\noindent\textbf{\texttt{I01.} Laser Based Attacks on Crystal Oscillators.} Quartz crystal oscillators, integral to systems-on-chip (SoCs), are susceptible to optical laser attacks. Research by Kohei et al.~\cite{redshift} demonstrates a direct correlation between an oscillator's frequency and power of the laser incident upon it. 
They used this exploit to extract cryptographic keys embedded in the hardware. However, an adversary can use the same mechanism to alter the frequency of the oscillator that drives the hardware counters and timer registers on the SoC (\textit{time warping} attack --  \textbf{\texttt{A2}}). Fundamentally, this attack is possible as a result of crystal frequency sensitivity to ambient temperature (a laser incident on the crystal raises its temperature). Other time sources (e.g. atomic clocks) are also susceptible to environmental factors (temperature~\cite{nasa-atomic-clock}) and could be exploited by malicious adversaries albeit using a different attack mechanism than described here.

\noindent\textbf{\texttt{I02.} Computational Faults against Local Clock.} Computational faults resulting from under-volting a processor core affect both x86~\cite{hardware-plundervolt, hardware-V0ltpawn, hardware-voltage-pillager} and ARM~\cite{clock-sync-fault-presence, hardware-volt-jockey}; two of the most popular processor architectures. Such faults can corrupt instruction execution results (or even skip instructions altogether) and can be exploited to launch attacks against the \textit{local clocks} maintained by the system software. As described in section~\ref{subsec:local-clocks} (Time Keeping), system tick updates to the \textit{local clock} compute a new timestamp. Under-volting a processor core during this system tick update may introduce faults and random errors in the calculated timestamp (uncertainty in local time \textbf{\texttt{A3}}). For a successful attack, the adversary must predict when these computation are going to take place. An attacker using a physical interface for the attack~\cite{hardware-voltage-pillager}, can do so by monitoring the timer interrupt pin on the SoC. Finally, note that these computational faults can also be induced using other methods such as laser fault injection and voltage glitches using physical probes~\cite{hardware-plundervolt, hardware-V0ltpawn}~\footnote{Research has demonstrated that the under-volting attacks are also possible through a software interface only~\cite{hardware-voltage-pillager} and a remote adversary with \textit{privileged access} can also launch this attack.}.

\subsection{Design Limitations}
The trade-off between security and performance often de-prioritizes the former in system design, potentially exposing timing mechanisms to exploitation due to design oversights or flawed assumptions. Exploiting these design flaws often require the adversary to have \textit{privileged access} to the system software.

\noindent\textbf{\texttt{I03.} Exploiting Energy Management Mechanisms.} The prevalent Dynamic Voltage and Frequency Scaling (DVFS) mechanisms in CPS for energy efficiency inadvertently introduces a vulnerability. Typically, a dedicated system software module oversees energy management and controls the DVFS interface. However, any software with escalated privileges can access this interface. This also includes an attacker who gained escalated privileges by exploiting system software bugs (table~\ref{tab:cve-stats}). Such an adversary uses the DVFS interface to alter the system frequency without alerting the \textit{local clock}. Being unaware of the change, system software relies on an outdated clock frequency to convert time from clock cycles to wall clock time~\footnote{As detailed in section~\ref{subsec:local-clocks}, system software uses clock frequency value to translate hardware time measurements from cycle count to seconds}. This attack introduces \textit{time warping} (\textbf{\texttt{A2}}) to the \textit{local clock} and the extent of this warping can be \textit{precisely controlled} by the \textit{stealthy} adversary.

\noindent\textbf{\texttt{I04.} Re-configurable timing counters.} 
System software uses hardware counters such as Intel's TSC~\cite{intel-tsc} and ARM's CNTVCT~\cite{arm-cntvct} for updating the \textit{local clock} (section~\ref{subsec:local-clocks}--Time Keeping). In modern systems, these counters are write-protected and does not allow the operating system to manipulate them directly. However, with the implementation of virtualization extensions, both architectures introduce an offset register~\cite{arm-cntvoff, intel-tsc-offset} shown in figure~\ref{fig:software-stime-stacks}. This register, originally designed to allow virtualization software to emulate counters for multiple guests, is writable. A malicious agent with \textit{privileged execution} access can exploit this offset register to manipulate the system's time view and induce \textit{time travel} (\textbf{\texttt{A1}}) in the local clock. Older systems such Intel processors designed before $2011$~\cite{intel-variant} provide writable counters and are even more susceptible to adversarial attacks as it allows the adversary to launch without having to rely on virtualization extensions which may or may not be enabled by default.