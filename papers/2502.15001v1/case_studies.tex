Together with medical doctors from Eurotransplant and ETKAC, we selected three topics for case studies in which the ETKidney simulator could help quantify the impact of alternative kidney allocation rules. The selected topics are HLA-BDR matching, introduction of a sliding scale based on the vPRA, and candidate-donor age matching. To limit the effects of transient effects, we extend the simulation period for these case studies from 01-01-2016 to 01-01-2024. We simulate all policy alternatives 20 times, and use traditional hypothesis testing to assess whether alternative policies significantly change the outcomes compared to the current policy. To increase power of these tests, we use common random numbers \cite[~p.588]{lawSimulationModelingAnalysis2015} as a variance reduction technique. Consequently, outcomes can be compared with pairwise t-tests.

\subsection{Case study 1: emphasizing HLA matching on B and DR}
\label{sec:application_dr_matching}
The ETKAS point system has placed equal emphasis on HLA-A, -B and -DR matching since the initiation of ETKAS in 1996, despite common acceptance that HLA-DR mismatches are more deleterious to graft survival than HLA-A and HLA-B mismatches \cite{vereerstraetenExperienceWujciakOpelzAllocation1998,Roberts2004}. Internal analyses on registry data from Eurotransplant suggest that mismatches on the B and DR locus are more strongly associated with graft loss than mismatches on the A locus. This motivated us to simulate policies which emphasize matching on the DR and B loci relative to the A locus. %The kidney allocation policy in the United Kingdom emphasizes B and DR-matching \cite{watsonOverviewEvolutionUK2020}, while priority is given to DR-matching in the United States \cite{Roberts2004}. 
\par 
%For quantification of the association between HLA match quality and death-censored graft loss, we considered all transplantations where kidneys were allocated through ETKAS between 01-01-2000 and 01-01-2019. We excluded pediatric candidates, candidates with the HU status, and candidates who obtained a transplantation through non-standard allocation, leaving $n=31,851$ ETKAS transplantations. For each locus, we estimate the hazard ratios of 1 and 2 mismatches and death-censored graft loss with a Cox proportional hazards model. The Cox models are stratified by match quality on the two other loci (00, 01, 11, 12, 21, or 22), the candidate country of listing, and the era of transplantation (2000-2003, 2004-2007, 2008-2011, 2012-2015, 2016-2019). Cox models additionally adjusted for donor variables (DCD, age, hypertension, sex, impaired renal function, death cause group), candidate variables (age, sex, BMI, diagnosis group, time on dialysis, vPRA, type of dialysis), and transplantation variables (cold ischemia time). Death censored graft loss is used as the outcome, with administrative censoring 5 or 10 years after transplantation. We used such administrative censoring, because 5- and 10-year follow-up rates are known only for approximately two thirds and one-thirds of kidney-only transplantations, respectively. We use spline terms to adjust for continuous variables. Multiple imputation is used to impute missing data with final estimates pooled using Rubin's rules.
%\par 
%Table \ref{tab:hla_graft_loss_long} shows the obtained estimates of the adjusted hazard ratios per locus. These results suggest that two mismatches at each locus are all significantly associated with increased 5 and 10 year graft loss rates, but most strongly for 2 mismatches at the B and DR loci.
%\par  
%\begin{table}[htbp]
%	\centering
%	\caption{Estimated hazard ratios for the number of mismatches on HLA-A (broad), HLA-B (broad), and HLA-DR loci (split). Missing data was completed $M$=10 times, with final estimates pooled using Rubin's rules.}
%	
\begin{tabular}{lllll}
	\toprule
	\multicolumn{1}{c}{ } & \multicolumn{2}{c}{5-year death-censored graft loss} & \multicolumn{2}{c}{10-year death-censored graft loss} \\
	\cmidrule(l{3pt}r{3pt}){2-3} \cmidrule(l{3pt}r{3pt}){4-5}
	Locus & 1 mismatch & 2 mismatches & 1 mismatch & 2 mismatches\\
	\midrule
	A & 1.07 [0.98-1.16] & \textbf{1.15} [1.01-1.31] & \textbf{1.08} [1.01-1.16] & \textbf{1.15} [1.02-1.28]\\
	B & 1.06 [0.96-1.17] & \textbf{1.28} [1.13-1.44] & 1.08 [0.99-1.18] & \textbf{1.27} [1.15-1.41]\\
	DR & \textbf{1.12} [1.03-1.22] & \textbf{1.35} [1.18-1.55] & \textbf{1.13} [1.05-1.22] & \textbf{1.33} [1.18-1.49]\\
	\bottomrule
\end{tabular}
%	\label{tab:hla_graft_loss_long}
%\end{table}
%\par
The current ETKAS point system awards 400 points for HLA matching, and penalizes HLA mismatches on the A, B and DR loci with 66.7 points per mismatch. We assess the impact of three alternative HLA matching policies, which all continue to award up to 400 points for candidate-donor HLA match quality but which shift weight from the A locus to the DR and B locus (see Table \ref{tab:alternative_hla_policies}). The first policy is referred to as the $\text{B} + 2\text{DR}$ policy, because it gives no weight to the A locus, maintains the same weight for the B locus (-66.7 points), and doubles the weight on the DR locus (-133.3 points). The second policy, referred to as the $0.5\text{A} + \text{B} + 1.5\text{DR}$ policy, shifts only half of the weight placed on the A locus to the DR locus. The final policy, referred to as $1.5\texttt{B} + 1.5\texttt{DR}$, penalizes mismatches at the B and DR loci both with 100 points per mismatch.
\begin{table}[h!]
	\centering
	\caption{Policy alternatives evaluated for HLA-ABDR matching}
		\begin{tabular}{@{}cccc@{}}
		\toprule
		Policy & $\beta_{\text{MMB}_{HLA_A}}$ &
		$\beta_{\text{MMB}_{HLA_B}}$ & $\beta_{\text{MMS}_{HLA_{DR}}}$ \\
		\midrule
		Current & -66.7     & -66.7     & -66.7\\
		B + 2DR & 0     & -66.7     & -133.3 \\
		$0.5$A + B + $1.5$DR & -33.3 & -66.7     & -100 \\
		$1.5$B + $1.5$DR & 0     & -100     & -100\\
		\bottomrule
	\end{tabular}

	\label{tab:alternative_hla_policies}
\end{table}
\par
Simulation results for these three alternative policies are summarized in Table \ref{tab:results_dr_matching}. When grouping mismatches by a definition of HLA-ABDR match quality used for allocation in the United Kingdom, one can see that the policy alternatives reduces the number of transplantations with 2 DR or 3 or more BDR mismatches by 25 to 39\%, and increase the number of transplantations with 1 B or 1 DR mismatch by 26 to 49\%. Thus, the policies indeed succeed in improving match quality on the B and DR locus.
\par 
However, results also show that there are unintended consequences of this policy: the total number of ABDR mismatches at transplantation increases with all policies, and there are 5 to 12\% fewer transplantations in candidates who are homozygous at the B and/or DR loci, who are already disadvantaged in the current ETKAS system. Results such as those presented in Table \ref{tab:results_dr_matching} can facilitate discussions by ETKAC on whether the improved BDR match quality is worth the increase in total mismatches and reduced access to transplantation for homozygotes. 

\begin{table}[h!]
	\centering
	
	\caption{Simulated change in number of transplantations under alternative HLA-ABDR matching policies. Numbers displayed are the averages difference over 20 simulations. Pairwise t-tests were used to test whether changes in outcomes were statistically significant. mm: mismatches}
	\setlength{\tabcolsep}{5pt}
	
\begin{tabular}{llccc}
\toprule
\multicolumn{2}{c}{ } & \multicolumn{3}{c}{Change in number of transplantations compared to current policy} \\
\cmidrule(l{3pt}r{3pt}){3-5}
  & Current & B + 2DR & 0.5A + B + 1.5DR  & 1.5B + 1.5DR\\
\midrule
\addlinespace[0.3em]
\multicolumn{5}{l}{\textbf{ABDR mismatch count}}\\
\hspace{.5em}0 & 1925 & -9 & -1 & -6\\
\hspace{.5em}1 & 952 & -181*** & -55*** & -99***\\
\hspace{.5em}2 & 3855 & -634*** & -252*** & -509***\\
\hspace{.5em}3 & 6037 & -358*** & -76*** & -224***\\
\hspace{.5em}4 & 3235 & \Plus 837*** & \Plus 282*** & \Plus 642***\\
\hspace{.5em}5 & 730 & \Plus 340*** & \Plus 104*** & \Plus 191***\\
\hspace{.5em}6 & 78 & \Plus 2 & 0 & \Plus 7*\\
\addlinespace[0.3em]
\multicolumn{5}{l}{\textbf{ABDR match quality}}\\
\hspace{.5em}000 & 1925 & -9 & -1 & -6\\
\hspace{.5em}*00, *10, *01 & 3264 & \Plus 1364*** & \Plus 850*** & \Plus 1585***\\
\hspace{.5em}*20, *11 & 7234 & \Plus 209*** & \Plus 240*** & \Plus 131***\\
\hspace{.5em}**2, *21 & 4388 & -1567*** & -1087*** & -1708***\\
\addlinespace[0.3em]
\multicolumn{5}{l}{\textbf{candidate homozygosity on B and DR}}\\
\hspace{.5em}B and DR & 473 & -39*** & -17*** & -30***\\
\hspace{.5em}DR & 1662 & -207*** & -104*** & -94***\\
\hspace{.5em}B & 1052 & \Plus 31*** & \Plus 14** & -19***\\
\hspace{.5em}none & 13625 & \Plus 212*** & \Plus 110*** & \Plus 144***\\
\bottomrule
\end{tabular}

	\label{tab:results_dr_matching}
	\\[.1cm]
	\footnotesize{$\qquad\qquad^{*}p<0.05; \ ^{**}p<0.01; \  ^{***}p<0.001$\hfill}
\end{table}
\par


\FloatBarrier

\subsection{Case study 2: a sliding scale for the vPRA}
\label{sec:application_vpra}
Candidates with unacceptable antigens indirectly receive priority in ETKAS through mismatch probability points (MMPPs), see Section \ref{subsection:etkas}. Despite this form of compensation, studies on Eurotransplant registry data have shown that immunized candidates face longer waiting times than non-immunized candidates in ETKAS \cite{ziemannUnacceptableHumanLeucocyte2017, zecherImpactSensitizationWaiting2022a, deferranteImmunizedPatientsFace2023}. In this case study, we assess whether this disparity can be alleviated by awarding points directly for the vPRA.
\par 
For this, we modify the ETKAS point system in two ways. The first modification is that we directly award points for the vPRA using a \textit{sliding scale}. Such a sliding scale for the vPRA has been a part of kidney allocation in the United States since 2014 \cite{stewartSmoothingItOut2012}. This sliding scale is parameterized by a weight, $\beta_{\texttt{vPRA}}$, which determines the maximum number of points awarded for the vPRA, and a base $b$, which controls the steepness of the sliding scale. %Specifically, we add a term to the ETKAS point system given by $\beta_{\texttt{vPRA}}\cdot f({\texttt{vPRA}})$. In this,  $f(\texttt{vPRA})$ is the exponential function given by $\frac{b^\texttt{vPRA} - 1}{b - 1}$. 
The second modification is that we no longer directly award points for the vPRA via the mismatch probability. Instead, we replace mismatch probability points by \textit{\enquote{HLA mismatch probability points}} (HMPPs) (Section \ref{subsection:mmp}). These HMPPs are calculated as $\texttt{HMPP} = \Big[1-f_{\leq1mm}\Big]^{1000}$, where $f_{\leq1mm}$ is the 1-ABDR HLA mismatch frequency (see Section \ref{subsection:mmp}). For calculation of HMPPs, we do not use the candidate blood group, which is motivated by the fact that ETKAS allocation has been blood group identical since 2010.
\par
In discussing this case study, representatives of ETKAC reached consensus that the aim of a sliding scale should be that a candidate's chance to be transplanted through ETKAS should not decline up to a vPRA of 85\%. Above this vPRA, candidates could have access to the AM program, or should consider removing unacceptable antigens in case they do not meet AM criteria. We used the ETKidney simulator to simulate ETKAS for different combinations of weights $\beta_{\texttt{vPRA}}$ and bases $b$, and quantify the association between vPRA and the relative transplantation rate using Cox proportional hazards model on simulated outcomes. For this, the same model specification is used as in a previous analysis of the association between vPRA and the transplantation rate \cite{deferranteImmunizedPatientsFace2023}. In Figure \ref{fig:vpraeffect} the estimated relation between vPRA and the transplantation rate is shown for several sliding scales. From this figure, the most acceptable option would be a sliding scale with a base of 5 and weight of 133, with which the relative transplantation rate no longer decays until a vPRA of 85\%.
\begin{figure}[h]
	\centering
	\includegraphics[width=0.8\linewidth]{figures/Fig4}
	\caption{Relations between the relative transplantation rate and vPRA in ETKAS, estimated on ETKidney simulator outcomes. These relations were estimated with a Cox proportional hazards model, adjusting for the vPRA using spline terms. }
	\label{fig:vpraeffect}
\end{figure}


\subsection{Case study 3: candidate-donor age matching}
\label{sec:application_age_matching}
Consensus in the transplantation literature is that kidneys procured from young donors should preferentially be transplanted in young candidates \cite{waiserAgeMatchingRenal2000, pippiasYoungDeceasedDonor2020, vanittersumIncreasedRiskGraft2017, coemansCompetingRisksModel2024, keithEffectDonorRecipient2004b}. While allocation systems in France and the United Kingdom prioritize such candidate-donor age matching \cite{audryNewFrenchKidney2022a, watsonOverviewEvolutionUK2020}, the \WOP\ does not award points based on candidate or donor age (apart from pediatric points). In this case study, we (i) quantify the associations of candidate and donor age with graft and patient survival retrospectively on Eurotransplant registry data using cause-specific hazard models, and (ii) simulate and evaluate two age matching policies for ETKAS.
\par
To quantify the relation between donor or candidate age and post-transplant survival, we consider all patients transplanted with a kidney through ETKAS or ESP between 2004 and 2019. We exclude candidates with the HU status and candidates without any follow-up information ($n = 7,458$), leaving $n = 36,576$ transplantations. We fit cause-specific Cox proportional hazards models on these transplantations for (i) graft loss and (ii) death with a functioning graft. We censored both time-to-event variables ten years after transplantation, because completeness of follow-up data for more than 10 years is poor in the Eurotransplant registry. Besides donor and candidate age, we adjust for donor characteristics (heart or non-heartbeating donation, hypertension, last creatinine, death cause, diabetes, malignancy), candidate characteristics (dialysis time), and match characteristics (zero mismatch, number of mismatches per locus for the A, B, and DR loci, match geography). To allow for non-linear relations between continuous variables and the hazard rate, we adjust for spline transformations of the continuous variables. The estimated relations between donor / candidate age and patient / graft survival are shown in Figure \ref{fig:competingriskmodel}. These results are qualitatively similar to results by Coemans et al. \cite{coemansCompetingRisksModel2024}, who report that the hazard rate of graft loss decays linearly with recipient age while it increases quadratically with donor age, and that the mortality hazard rate increases quadratically with candidate age. 
\par
\begin{figure}[h]
	\centering
	\includegraphics[width=\linewidth]{figures/Fig5}
	\caption{Estimated relations between graft loss and death with a functioning graft. Note that the hazard ratio for death with functioning graft is shown on the logarithmic scale.}
	\label{fig:competingriskmodel}
\end{figure}
\par
We simulate two candidate-donor age matching policies, which were both inspired by the 2015 French kidney allocation policy \cite{audryNewFrenchKidney2022a}. Like ETKAS, the French policy awards points for candidate waiting time, HLA match, the candidate's likelihood to be favorably matched with a kidney, and the geographic distance between the donor and candidate. However, in France an \enquote{age filter} is applied to the total number of points, with candidates ranked based on the filtered number of points. For instance, the French age filter is 0\% for a candidate who is 20 or more years older than the donor, which means that such candidates receive 0\% of the total points for ranking. This French age filter is asymmetrical, with allocation of kidneys from a young donor in an older patient discouraged more strongly than allocation of a kidney from an older donor to a young candidate.
\par
We evaluate two scenarios for such an asymmetrical age filter for ETKAS (see Figure \ref{fig:baseagefiltermuted}). Both filters give a candidate 100\% of their \WO\ points in case the difference in candidate age and donor age is 5 years or less. The \enquote{strict} filter (blue) is similar to the French age filter in that it gives almost no points in case the candidate is much older than the donor. The \enquote{muted} filter (orange)  gives a larger fraction of the total number of \WO\ points, which we anticipated to be more acceptable for Eurotransplant because it maintains a better balance in the international exchange of kidneys.
\begin{figure}[h]
	\centering
	\includegraphics[width=0.85\linewidth]{figures/Fig6}
	\caption{Evaluated age filters for ETKAS.}
	\label{fig:baseagefiltermuted}
\end{figure}
\FloatBarrier
Simulated outcomes for the muted and strict age matching policies are compared to the current policy in Table \ref{tab:results_age_matching}. The Table shows that the muted and strict policies increase age-matched transplantations (maximum 5 year difference in donor and candidate age) by 60\% and 138\%, respectively. An unintended consequence is that such age matching leads to reduced HLA match quality, with a 12 and 33\% increase in level 4 mismatched kidney transplantations (2 DR, or 3 or 4 B+DR mismatches). In terms of match geography (third panel), Table \ref{tab:results_age_matching} shows that the muted policy modestly increases international sharing (+5\%) and inter-regional sharing (+7\%), while the strict policy leads to a 37\% increase in international transplantations. Simulation results also show that the strict age filter also increases the number of extended or rescue transplantations.
\par 
To assess whether improvements in candidate-donor age matching outweigh the unintended consequences, we use the earlier mentioned cause-specific hazard models to predict the probability of death with a functioning graft ten years after transplantation for all simulated transplantations. For this, we use a competing risk approach \cite{wreedeMstatePackageAnalysis2011}. The expected number of events ten years after transplantation is visualized in Figure \ref{fig:posttxpeventsagematching} for the current policy (green), the muted age filter (orange), and the strict age filter (blue). These results suggest that the muted and strict age filter could reduce the numbers of deaths with a functioning graft 10 years after transplantation by 11\% and 18\%, respectively.

\begin{table}[h!]
	\centering
	\caption{Simulated change in the number of transplantations in ETKAS with the age matching policies. Numbers displayed are the averages difference over 20 simulations. Pairwise t-tests were used to test whether changes in outcomes were statistically significant. mm: mismatches.}
	
\begin{tabular}{llcc}
\toprule
\multicolumn{2}{c}{ } & \multicolumn{2}{c}{\multirow[b]{1.5}{*}{\makecell{Change in number of transplantations\\compared to current policy}}} \\
& & & \vspace{-0em} \\
\cmidrule(l{3pt}r{3pt}){3-4}
  & current & muted age filter & strict age filter\\
\midrule
\addlinespace[0.3em]
\multicolumn{4}{l}{\textbf{age difference}}\\
\hspace{1em}candidate 35+ years older & 708 & -497$^{***}$ & -557$^{***}$\\
\hspace{1em}candidate 15-34 years older & 3166 & -1742$^{***}$ & -2549$^{***}$\\
\hspace{1em}candidate 6-14 years older & 3256 & +434$^{***}$ & -1756$^{***}$\\
\hspace{1em}max 5 year difference & 4810 & +2919$^{***}$ & +6640$^{***}$\\
\hspace{1em}candidate 6-14 years younger & 2624 & -151$^{***}$ & -230$^{***}$\\
\hspace{1em}candidate 15-34 years younger & 2064 & -873$^{***}$ & -1386$^{***}$\\
\hspace{1em}candidate 35+ years younger & 184 & -86$^{***}$ & -155$^{***}$\\
\addlinespace[0.3em]
\multicolumn{4}{l}{\textbf{HLA match quality}}\\
\hspace{1em}level 1 (0 ABDR mm) & 1922 & -8 & -40$^{***}$\\
\hspace{1em}level 2 (at most 1 BDR mm) & 3250 & -380$^{***}$ & -902$^{***}$\\
\hspace{1em}level 3 (2B or 1DR+1B mm) & 7224 & -156$^{***}$ & -496$^{***}$\\
\hspace{1em}level 4 (2DR or 3\Plus BDR mm) & 4417 & +547$^{***}$ & +1446$^{***}$\\
\addlinespace[0.3em]
\multicolumn{4}{l}{\textbf{match geography}}\\
\hspace{1em}local/regional & 11744 & -297$^{***}$ & -1514$^{***}$\\
\hspace{1em}national & 1796 & +121$^{***}$ & +306$^{***}$\\
\hspace{1em}international & 3272 & +179$^{***}$ & +1215$^{***}$\\
\addlinespace[0.3em]
\multicolumn{4}{l}{\textbf{type of allocation}}\\
\hspace{1em}standard allocation & 14487 & +50 & -160*\\
\hspace{1em}non-standard & 2325 & -47 & +167$^*$\\
\bottomrule
\end{tabular}
\\[.1cm]
\footnotesize{$\qquad\qquad^{*}p<0.05; \ ^{**}p<0.01; \  ^{***}p<0.001$\hfill}
	\label{tab:results_age_matching}
\end{table}


\begin{figure}[h]
	\centering
	\includegraphics[width=0.6\linewidth]{figures/Fig7}
	\caption{Expected number of post-transplant events 10 years after transplantation, predicted based on candidate, donor, and transplantation characteristics with competing risk models.}
	\label{fig:posttxpeventsagematching}
\end{figure}


\FloatBarrier