\section{Verification and validation}
\label{sec:validation}
This section describes verification and validation efforts taken to ensure that the ETKidney simulator closely mimics ETKAS and ESP. Under model verification, we understand efforts taken to \textit{\enquote{ensure that the computer program of the computerized model and its implementation are correct}} \cite{sargent2020}. Under model validation, we understand efforts taken to assess that the model \textit{\enquote{possesses a satisfactory range of accuracy consistent with the intended application of the model}} \cite{sargent2020}. 

\subsection{Verification of the ETKidney simulator}
\label{section:verification}
We built unit tests to ensure that the behavior of the ETKidney's simulator modules is in agreement with their intended behavior. For instance, unit tests were constructed to check whether HLA match qualities returned by the HLA system module matched HLA match qualities of actual ETKAS match lists. Unit tests were also used to ascertain that the HLA system module returned the correct mismatch probabilities and vPRAs. %To ensure that the correct serological definitions are used by the HLA system module, antigen definitions were exported directly from the match tables published by the ETRL. %(see \href{https://etrl.eurotransplant.org/resources/hla-a/}{https://etrl.eurotransplant.org/resources/hla-a/}). 

\par
Simulation of the graft offering process and of post-transplant survival is based on statistical models, which were estimated in R. Unit tests were constructed to ensure that predicted probabilities in the ETKidney simulator matched predicted probabilities in R for offer acceptance decisions and for post-transplant survival.

\subsection{Validation of the ETKidney simulator}
\label{section:validation}
To ensure face validity of the model, medical doctors from Eurotransplant were actively involved in the development and conceptual design of the simulator. We also had meetings with ETKAC and the ETRL, and we presented the model at the Eurotransplant Annual Meeting to collect feedback on the conceptual model from other major stakeholders, such as medical doctors and transplantation coordinators from the transplantation centers.
\par 
We use input-output validation to assess how closely the model can approximate outcomes of ETKAS and ESP. For this, we simulate kidney allocation between 01-04-2021 and 01-01-2024 under the actual allocation rules used within this simulation window. For the donor input stream, we used all 4,326 donors with kidneys transplanted through ETKAS or ESP in the simulation period. For the candidate input stream, we used all 9,589 candidates who were on the waiting list on 01-04-2021, and all 14,333 candidates who were activated on the waiting list in the simulation period. To enable accurate simulation of the ETKAS balances, international transplantations in the AM program or via combined transplantations were exported from the Eurotransplant database. In simulations, we schedule donors, candidate status updates, and balance update events on the dates these were actually reported to Eurotransplant. Our input-output validation exercise thus keeps the inputs as close as possible to reality, and assesses whether outputs of the ETKidney simulator are comparable to the actual outputs of ETKAS and ESP.
\par
Important is that outputs of the simulator depend on several stochastic processes (offer acceptance behavior, re-listing, and non-standard allocation). To give insight into the resulting variability in simulator outputs, we simulate ETKAS and ESP allocation 200 times over the simulation window and report \enquote{95\%-interquantile ranges} for relevant summary statistics. These 95\%-IQRs are obtained by simulating allocation 200 times and reporting the 2.5th and 97.5th percentiles of simulation outputs. For each of these 200 simulation runs, we use a different set of imputed status trajectories (see Section \ref{subsection:input_streams}). We say that the ETKidney simulator is \textit{well-calibrated} for a quantity of interest if the actually observed summary statistic falls within the 95\%-IQR of the 200 simulations. We do not test statistically whether the mean outcome over the simulations is different from the actually observed outcome, because such tests can always be made statistically significant by increasing the number of simulation runs.

\subsubsection*{Results of input-output validation}
\label{subsection:val_wl_outc}
Table \ref{tab:waitlist_validation} reports input-output validation results for outcomes on the kidney waiting list. The ETKidney simulator is well-calibrated for almost all summary statistics: the total number of transplantations, the number of dual kidney transplantations, the number of ETKAS / ESP transplantations, the number of re-listings, and the number of waiting list deaths per country in all countries. We only observe miscalibration for the number of waiting list deaths in Hungary (-11\%) and the active waiting list size at simulation termination (+1.8\% too many candidates have an active waitlist status).
\begin{table}[h]
	\caption{Input-output validation of waiting list outcomes between 01-04-2021 to 01-01-2024. For simulations, the numbers shown are averages and 95\%-IQR of outcomes over 200 simulations. Ranges are displayed in bold if the simulator is not well-calibrated, i.e. if the actual statistic does not fall within the 95\%-IQR.}
	\input{tables/validation/2025-01-09/summary_waitlist_results}
	\label{tab:waitlist_validation}
\end{table}
\par
The left side of Table \ref{tab:transplant_validation} reports input-output validation results for ETKAS transplantations. The simulator is well-calibrated for the number of transplantations placed by allocation mechanism (standard or non-standard), transplantations by candidate age group, and transplantations in repeat transplantation candidates. The simulator is also well-calibrated for the number of transplantations by HLA match quality, with only the number of 0-mismatched transplantations overestimated ($\Plus$5\%). The number of transplantations in candidates with vPRAs exceeding 95\% is underestimated (-17\%). Such miscalibration seems to have been the result of the introduction of the virtual crossmatch in January 2023 (see Supplementary Table \ref{tab:vpra_vxm}), potentially because the number of positive recipient center crossmatches has decreased after introduction of the virtual crossmatch \cite{Heidt2024}. The simulator is also well-calibrated for the number of transplantations per country, with only a slight overestimation observed in Croatia ($\Plus 3\%)$ and a slight underestimation observed in Hungary (-2\%). Geographical sharing within ETKAS is underestimated, with 5\% extra local or regional transplantations, 11\% fewer interregional and 11\% fewer international transplantations.
\par
The right side of Table \ref{tab:transplant_validation} shows validation results for ESP transplantations. The simulator is again well-calibrated for most relevant outcomes: the number of transplantations in primary and repeat kidney transplantation candidates, and the number of transplantations by HLA match quality and immunization status. The simulator does overestimate the number of kidneys transplanted by allocation mechanism in ESP, with on average 23\% too many kidneys placed via non-standard allocation. An apparent consequence of this is that the number of kidneys allocated to candidates aged below 65 is overestimated ($\Plus$28\%), particularly in Belgium, the Netherlands, and Slovenia (see Supplementary Table \ref{tab:esp}) where centers appear to be reluctant to transplant a candidate aged under 65 with an ESP donor. Finally, the simulator is well-calibrated for the number of transplantations by recipient country and match geography, with the only exception Germany where 2\% too many ESP kidneys are transplanted.
\par
Overall, the ETKidney simulator appears to be well-calibrated for most outcomes of ETKAS and ESP allocation. The results of this input-output validation exercise were discussed with medical doctors from Eurotransplant and ETKAC, who deemed differences small enough to make the simulator useful for determining the impact of alternative kidney allocation policies. We illustrate this with case studies in the next section. 
\begin{table}[h]
	\caption{Validation of the number of transplantations between 01-04-2021 and 01-01-2024. For simulations, shown numbers are averages and 95\%-IQRs over 200 simulations. 
		Statistics are displayed in bold if the actual statistic does not fall within the 95\%-IQR.}
	\input{tables/validation/2025-01-09/summary_transplants.tex}
	\label{tab:transplant_validation}
\end{table}
\FloatBarrier

%\subsubsection*{Discussion of input-output validation results}
%Our input-output validation exercise thus shows that the ETKidney simulator is able to closely approximate most outcomes of ETKAS and ESP allocation. One outcome on which the simulator is not well-calibrated is the number of transplantations in candidates aged below 65 through ESP. Supplementary Table \ref{tab:esp} shows the number of such transplantations by country of listing, and shows that the simulator overestimates the incidence of such transplantations in Belgium (on average 14 vs. 3 in reality), the Netherlands (18 vs. 1) and Slovenia (5 vs. 1). This suggests that there are country differences in willingness to transplant ESP donors in candidates aged below 65, which are not captured by the used graft offer acceptance models.
%\par
%Another important area in which the simulator is miscalibrated for ETKAS is the number of transplantations in immunized candidates with vPRAs exceeding 95\%.  A likely explanation for this is that the virtual crossmatch has reduced the incidence of positive recipient center crossmatches \cite{Heidt2024}. Such crossmatches were considered as an offer decline in developing the graft offer acceptance models, which may explain why the models overestimate decline rates in immunized candidates after introduction of the virtual crossmatch.
%\par
%The simulator also appears slightly miscalibrated in a few other areas. Such areas of miscalibration were discussed with medical doctors from Eurotransplant and ETKAC, who deemed differences small enough to make the simulator useful for determining the impact of alternative kidney allocation policies. We illustrate this with case studies in the next section. 
\FloatBarrier


%\subsubsection*{Post-transplant outcomes}


%In the previous subsections, we reported validation results for the ETKidney simulator. One important finding of this validation exercise is that relatively too few ESP-aged donors (-4.5\%) are placed through the ESP program, and instead allocated through extended / rescue allocation. This may explain why the simulator appears to be poorly calibrated for many summary statistics for the ESP program. A second important finding is that the ETKidney simulator tends to overestimate the number of transplantations in repeat transplantation candidates and in immunized candidates. A potential explanation for this finding could be that re-listed / immunized candidates are more likely to have a positive crossmatch, even after accounting for the vPRA. Calibration of the ETKidney simulator could potentially be improved by using a logistic model to predict the chance of having a positive crossmatch based on donor and candidate characteristics. We did not pursue this for the ETKidney simulator, since preliminary data shows that repeat transplantation / immunized candidates are no longer more likely to have positive crossmatches in Eurotransplant kidney allocation after introduction of the virtual crossmatch in January 2023.
