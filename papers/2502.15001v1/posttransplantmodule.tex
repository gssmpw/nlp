
\bibliographystyleAC{myama}

\newcommand\Time{{\hspace{-.25pt}\intercal}}


\subsection{Post-transplant survival models \& relistings}
This section discusses how post-transplant survival and listing for a repeat transplantation is simulated based on patient and donor characteristics in the ETKidney simulator. For post-transplant survival, parametric survival models were used. The Kaplan-Meier estimator was used to determine the time-to-relisting relative to simulated post-transplant survival times. Both models were estimated on transplantations in the Eurotransplant region between 01-01-2011 and 01-01-2021. 

\subsubsection{Patient failure or repeat transplantation as the outcome}
Outcomes of interest after kidney transplantation are (i) patient survival, (ii) repeat transplantation, (iii) graft failure, and (iv) return-to-dialysis. For the prediction of post-transplant survival in the ETKidney simulator, we have chosen to define the event of interest as a kidney re-transplantation or patient death, whichever event occurs first. We have chosen not to model time-to-graft failure since centers in Eurotransplant do not work with standardized definitions of graft failure, and return-to-dialysis is not explicitly reported to Eurotransplant. For simulations, we assume that a patient would die on their event time, unless they are re-transplanted before the event materializes.

\subsubsection{Simulating patient failure date}
\label{subsection:sim_time_to_event}
Weibull accelerated failure time regression are used to simulate a time-to-event for patient death/re-transplantation. For this, we used the R \texttt{survival} library. The Weibull distribution in this library is parametrized with a shape parameter $k$ and scale parameter $\lambda = \beta^{\Time} x$, where $x$ are relevant patient and donor characteristics. The survival function for the Weibull distribution is given by:
$$S(t|x) = \exp\Bigg(-\Big(\frac{t}{\beta^{\Time}x}\Big)^{k}\Bigg).$$
After obtaining estimates for $\beta$ and $k$ based on historic data, we can simulate time-to-events for patient-donor pair $i$ by inverse transform sampling from this distribution. Specifically, we can draw a random number $u \sim \texttt{unif}(0,1)$ and simulate a patient's time-to-event as 
$$t = \hat{\mathbf{\beta}}^\Time \mathbf{x_i} \Big(-\log(u)^{\frac{1}{k}}\Big).$$
\par
For the default model supplied with the ETKidney simulator, we adjusted for donor characteristics (DBD/DCD, age, hypertension, last creatinine, diabetes, death cause, malignancy), patient characteristics (age, on dialysis, years on dialysis, repeat transplantation), and transplantation characteristics (year of transplantation, cross-country, standard or non-standard allocation, and the number of HLA-A, -B and -DR mismatches). We included shape parameters for the candidate's country of listing.

\subsubsection{Simulating time-to-relisting}
\label{subsection:sim_time_to_relist}
Most candidates who experience an (early) post-transplant event would list for a repeat transplantation. Such a time-to-relisting $r$ logically has to occur before a candidate's time-to-event $t$. To simulate relistings in the ETKidney simulator, we estimated the proportion of individuals that have relisted with Kaplan-Meier using $r/t$ as the time scale. In this, we stratify the curves by the candidate's time-to-event $t$ ($<$180 days, 180d-$<$1y, 1y-$<$2y, 2y-5y, 5y or more) and age at transplantation (0-17, 18-39, 40-49, 50-54, 55-59, 60-64, 65-69, 70-74, 75+). A relisting time $r$ can then be simulated by inverse transform sampling from the time-to-event $t$ and age-at-transplantation specific curve, i.e. (i) sample a random $u \sim \text{unif}(0,1)$, (ii) choose the first $s$ such that $\mathbb{P}[R_i/T_i > s] \geq u$, and (iii) calculating the time-to-relisting as $s*t$.
\par
In case no such $t$ exists, the patient will not relist and therefore be a post-transplant death. Figure \ref{fig:posttxp_surv} shows estimates of the empirical distribution of $R_i/T_i$, stratified by groups of time-to-event. The figure clearly shows that the fraction of patients dying after transplantation without relisting depends on the failure time and the candidate's age. For instance, most pediatric candidates (0-17) are highly likely to list for repeat transplantation if they have an event within 5 years after listing, whereas very few older candidates (65+) list for repeat transplantation.
\begin{figure}[h]
	\centering
	\includegraphics[width=\linewidth]{figures/sFig1.pdf}
	\caption{Kaplan-Meier curves used to simulate the time-to-relisting $R_i$, estimated on non-HU re-registrations between 2012 and 2020.}
	\label{fig:posttxp_surv}
\end{figure}
\FloatBarrier
\subsubsection{Constructing a synthetic re-registration}
\label{subsection:synth_registrations}
Subsections \ref{subsection:sim_time_to_event} and \ref{subsection:sim_time_to_relist} discussed how a time-to-event $t$ and time-to-relist $r$ are simulated in the ETKidney simulator at transplantation based on patient and donor characteristics. This is not sufficient for simulation on what would happen to this patient after transplantation in case the candidate lists for a repeat transplantation, because it is not known how the candidate's status would evolve when this candidate lists for a repeat transplantation. Here, we describe how synthetic re-listings are constructed in the ETKidney simulator by combining candidate information from the kidney recipient with status updates from an actual listing for a repeat transplantation.
\par 
Actual listings for a repeat transplantation have a known time-to-relisting $r_k$ and time-to-event $t_k = d_k + r_k$, where $d_k$ was potentially imputed using the status imputation procedure detailed in appendix \ref{app:imp_proc}. To find suitable status updates for a synthetic re-listing for candidate $i$, we can thus look for a relisting $k$ with $(r_k, t_k)$ similar to $(r_i, t_i)$. In practice, we could want to constrain matches further by also requiring $i$ and $k$ to match on other characteristics. The post-transplant module includes code to require such matches.
\par
Specifically, the post-transplant module by default finds a re-registration $k$ by:
\begin{enumerate}[noitemsep]
	\item matching $i$ and $k$ on candidate country of listing, and matching $i$ and $k$ on whether the candidate re-listed within 1 year of transplantation because such candidates are eligible for returned dialysis time (see \cite{manualKidney}). We also restrict matches based on continuous variables: the difference in age is at most 20 years, the difference in $r$ at most 2 years, the difference in $t$ is at most 1 year, and the difference in accrued dialysis time at most 3 years.	
	\item From matching registrations, selecting the $m=5$ re-registrations with the closest Mahalanobis distance between ($R_i$, $T_i$) and ($R_k$, $T_k$). 
	\item Sampling a random re-registration from the $m$ re-registrations.
\end{enumerate}
A synthetic re-registration is then constructed by combining patient attributes from patient $i$ with status updates from patient $k$. The post-transplant module only copies over urgency statuses. Other patient statuses are not copied over (allocation profiles, disease group updates, HLA updates, unacceptable antigens, and program choices), because these are patient-specific. For dialysis time, the initial dialysis time at re-listing is copied over from the synthetic re-listing, but updates to the dialysis time are not copied over. Unacceptable antigens are simulated based on the mismatched donor HLAs (see main text). Within simulations, synthetic re-listings thus do not change unacceptables, disease groups, allocation profiles, or their HLA.

\bibliographyAE{refs}

