This appendix describes the eligibility, filtering and ranking criteria of ETKAS and ESP rules introduced in March 2021, and includes examples of ETKAS and ESP match lists.

\FloatBarrier
\subsection{Eligibility, filtering and ranking criteria in ETKAS}
\label{subsection:criteria_etkas}
Eligibility criteria determine whether a candidate can appear on the unfiltered match list. The eligibility criteria used in ETKAS are:
\begin{enumerate}[noitemsep]
	\item the candidate must have the same blood group as the donor,\footnote{Since 2010. Before 2010, transplantations across blood groups were allowed in case the donor and candidate had 0 HLA-ABDR mismatches. This was abandoned because it increased waiting times for blood group O candidates, who can only accept kidneys from blood group O donors.},
	\item the candidate should not have the non-transplantable (NT) status, which is a status used by centers to indicate that their candidate is (temporarily) unavailable for transplantation,
	\item the candidate's HLA typing at the A, B, and DR loci must be known, and the candidate is not allowed to have reported unacceptable antigens against the HLAs present in the donor's HLA typing,
	\item an antibody screening for the candidate must have been reported to Eurotransplant within the last 180 days,
	\item in Germany, candidates above age 65 must have chosen to not be included in ESP\footnote{ETKAS and ESP have been mutually exclusive in Germany since 2010},
	\item the candidate is also not allowed to have an active status in the AM program.
\end{enumerate}
Filtering criteria determine whether Eurotransplant will contact the transplantation center in standard allocation to offer the kidney for transplantation of a named candidate. For ETKAS, used filtering criteria are:
\begin{enumerate}[noitemsep]
	\item the allocation profile, which is used by centers to specify that their candidates does not want to accept certain donor characteristics. Selectable donor characteristics are (i) a minimum and/or maximum donor age, (ii) virology of the donor, such as whether the donor is hepatitis C positive, (iii) whether the donor is a non-heartbeating donor, (iv) extended donor criteria, which include whether the donor has had a malignancy, sepsis, meningitis, euthanasia, or history of drug abuse, and whether the kidney donor is procured from a domino transplantation.
	\item HLA mismatch criteria. Centers can use these to specify that they do not want to consider offers of a certain candidate-donor HLA match qualities. For instance, most centers exclude offers with 6 mismatches in total, and many centers continue to exclude offers with 2B or 2DR mismatches which were used as minimal match criteria in Eurotransplant from 1988 to 1996 \cite{demeesterNewEurotransplantKidney1998}.
\end{enumerate}
\par
An example of a \textit{filtered} ETKAS match list for a Belgian donor is shown in Table \ref{tab:example_etkas_list}. The ranking of candidates on this list is determined by three tiers and the Wuijciak-Opelz point system. The three ETKAS tiers are the zero mismatch tier, the pediatric tier, and the non-zero mismatch-tier \cite{manualKidney}. Candidates in higher tiers have absolute priority over candidates in lower tiers, regardless of their point score. The zero-mismatch tier (0MM-tier) consists of candidates with 0 HLA-ABDR mismatches with the donor. Within the 0MM-tier, subtiers exist to prioritize candidates who are homozygous at the HLA-ABDR loci in case the donor is fully homozygous.\footnote{These subtiers are based on the \textit{homozygosity level} of the candidate. For instance, a candidate with homozygosity on one of the loci would have priority over candidates heterozygous for all loci, regardless of their point score.} The pediatric tier is used to prioritize the allocation of kidneys procured from pediatric donors to pediatric candidates. All other candidates are included in the non-zero mismatch tier ($>$0MM-tier). The Wujciak-Opelz point system is explained in main text.
%An example of a filtered ETKAS match list for a Belgian adult donor with two kidneys available is shown in Table \ref{tab:example_etkas_list}. The left kidney of this donor was accepted by a German candidate who appeared on rank 1 because they had a zero HLA mismatch with the donor. Other candidates had mismatches at the HLA-A, -B, and -DR loci, and were ranked in descending order of their point score. The right kidney of this donor was turned down by several candidates, until finally being accepted at rank 14. Most of these kidney offers were turned down because of donor / organ quality reasons.
\par
We point out that many candidates who met the eligibility criteria for this kidney did not appear on the match list because of filtering criteria. For instance, Table \ref{tab:example_etkas_list} shows that the right donor kidney was accepted by the candidate ranked 14th on the filtered match list. In case filtering criteria were not applied, this candidate would have appeared on the 67th position.
\par
\begin{table}[h]
	\hspace*{-.1\linewidth}
	\resizebox{1.2\linewidth}{!}{
		\centering
		
\begin{tabular}{ccccccccccccc}
	\toprule
	\multicolumn{6}{c}{ } & \multicolumn{6}{c}{match point components} & \multicolumn{1}{c}{ } \\
	\cmidrule(l{3pt}r{3pt}){7-12}
	\makecell{tier} & \makecell{listing\\country} & \makecell{ABDR\\match\\quality} & \makecell{time on\\dialysis\\(years)} & \makecell{rank} & \makecell{total\\points} & dialysis & \makecell{HLA\\match} & \makecell{pediatric\\bonus} & \makecell{balance} & \makecell{distance} & \makecell{MMP} & \makecell{accepted}\\
	\midrule
	0MM & Germany & 000 & 9.0 & 1 & 722 & 298 & 400 & 0 & 0 & 0 & 24 & LKi\\
	\cmidrule{1-13}
	& Croatia & 111 & 7.5 & 2 & 1343 & 249 & 400 & 100 & 550 & 0 & 44 & -\\
	\cmidrule{2-13}
	&  & 111 & 4.7 & 3 & 1300 & 155 & 200 & 0 & 550 & 300 & 95 & -\\
	\cmidrule{3-13}
	&  & 111 & 3.0 & 4 & 1219 & 100 & 200 & 0 & 550 & 300 & 69 & -\\
	\cmidrule{3-13}
	& \multirow{-3}{*}{\centering\arraybackslash Belgium} & 202 & 2.7 & 5 & 1156 & 90 & 133 & 0 & 550 & 300 & 83 & -\\
	\cmidrule{2-13}
	& Hungary & 001 & 0.0 & 6 & 1147 & 0 & 667 & 100 & 370 & 0 & 10 & -\\
	\cmidrule{2-13}
	&  & 101 & 0.2 & 7 & 1143 & 7 & 267 & 0 & 550 & 300 & 19 & -\\
	\cmidrule{3-13}
	&  & 102 & 0.0 & 8 & 1130 & 0 & 200 & 0 & 550 & 300 & 80 & -\\
	\cmidrule{3-13}
	&  & 101 & 5.2 & 9 & 1113 & 172 & 267 & 0 & 550 & 100 & 24 & -\\
	\cmidrule{3-13}
	&  & 111 & 0.1 & 10 & 1097 & 5 & 200 & 0 & 550 & 300 & 42 & -\\
	\cmidrule{3-13}
	&  & 111 & 0.0 & 11 & 1066 & 0 & 200 & 0 & 550 & 300 & 16 & -\\
	\cmidrule{3-13}
	&  & 110 & 1.8 & 12 & 1059 & 60 & 267 & 0 & 550 & 100 & 82 & -\\
	\cmidrule{3-13}
	&  & 202 & 0.7 & 13 & 1053 & 24 & 133 & 0 & 550 & 300 & 46 & -\\
	\cmidrule{3-13}
	\multirow{-13}{*}[1.5\dimexpr\aboverulesep+\belowrulesep+\cmidrulewidth]{\centering\arraybackslash $>$0MM} & \multirow{-8}{*}{\centering\arraybackslash Belgium} & 110 & 1.4 & 14 & 1049 & 47 & 267 & 0 & 550 & 100 & 85 & RKi\\
	\bottomrule
\end{tabular}

	}
	
	\caption{Example of a filtered ETKAS match list for a blood group A donor reported in Belgium. The donor's left kidney was accepted on rank 1, the right kidney on rank 14. One candidate has 0 HLA-ABDR mismatches with the donor, and is ranked on in the 0MM-tier. Within tiers, priority is determined by the Wujciak-Opelz point system. Only candidates registered in the same region as the donor receive 300 distance points in Belgium.
	}
	\label{tab:example_etkas_list}
\end{table}

\FloatBarrier
\newpage
\subsubsection{Eligibility, filtering and prioritization in the ESP program}
\label{subsection:esp_program}
Eligibility criteria used in ESP are:
\begin{enumerate}[noitemsep]
	\item candidates must have an active waiting list status,
	\item the candidate must be aged over 65, or have specified that they want to receive offers through extended ESP allocation in case they are aged under 64,
	\item the candidate has to be blood group identical with the donor,
	\item the candidate must have reported a valid HLA typing, and a valid antibody screening within the last 180 days,
	\item candidates who report unacceptable antigens are only eligible for ESP if the donor HLA is known, and no unacceptable antigens are present in the donor HLA,
	\item candidates cannot have an active AM status.
\end{enumerate}
International candidates are eligible for offers via ESP since March 2021. 
\par
Filtering criteria used for ESP are:
\begin{enumerate}[noitemsep]
	\item the candidate is aged below 65,
	\item the candidate is a German candidate who has chosen for the ETKAS program,
	\item the donor has to be compatible with the candidate's allocation profile.
\end{enumerate}
We point out that HLA mismatch criteria are not used as a filtering criterion in ESP.
\par
Tiers are used to rank candidates in ESP. These tiers are based on the candidate's geographical proximity to the donor, and they differ per ET member country. For instance, in Germany ESP offers are first made to candidates who are located in the same ESP subregion as the donor, and then to candidates located in the same region, while in the Netherlands all candidates appear in a single tier (see the ET manual \cite{manualKidney}). In March 2021, subtiers were introduced in ESP for candidates with a High Urgency (HU) status and the Kidney After Other Organ (KAOO) status. Within tiers, candidates are ranked by the number of days they have waited on dialysis.% A proposal to prioritize HLA-DR matching in ESP is currently pending approval by national competent authorities.
\par 
Table \ref{tab:example_esp_match_list} shows an example of a filtered ESP match list for a donor reported from Germany. The right and left kidney of this donor were accepted at rank 2 and 11, respectively. Both offers were accepted by candidates located in the ESP subregion consisting of Stuttgart, Tübingen, Mannheim, and Heidelberg (the highest tier for this donor). The order of candidates within a tier is based on the time spent on dialysis.

\begin{table}[h]
	\resizebox{\linewidth}{!}{
		\centering
		
\begin{tabular}{ccccccc}
	\toprule
	listing country & donor region & listing center & time on dialysis (days) & rank & total points & accepted\\
	\midrule
	&  & Stuttgart & 1143 & 1 & 1143 & -\\
	\cmidrule{3-7}
	&  & Tübingen & 964 & 2 & 964 & RKi\\
	\cmidrule{3-7}
	&  & Heidelberg & 890 & 3 & 890 & -\\
	\cmidrule{3-7}
	&  & Tübingen & 871 & 4 & 871 & -\\
	\cmidrule{3-7}
	&  & Stuttgart & 867 & 5 & 867 & -\\
	\cmidrule{3-7}
	&  & Tübingen & 855 & 6 & 855 & -\\
	\cmidrule{3-7}
	&  &  & 715 & 7 & 715 & -\\
	\cmidrule{4-7}
	&  & \multirow{-2}{*}{\centering\arraybackslash Heidelberg} & 714 & 8 & 714 & -\\
	\cmidrule{3-7}
	&  & Tübingen & 596 & 9 & 596 & -\\
	\cmidrule{3-7}
	&  &  & 423 & 10 & 423 & -\\
	\cmidrule{4-7}
	\multirow{-11}{*}[4\dimexpr\aboverulesep+\belowrulesep+\cmidrulewidth]{\centering\arraybackslash Germany} & \multirow{-11}{*}[4\dimexpr\aboverulesep+\belowrulesep+\cmidrulewidth]{\centering\arraybackslash \makecell{Baden-\\Württemberg}} & \multirow{-2}{*}{\centering\arraybackslash Mannheim} & 419 & 11 & 419 & LKi\\
	\bottomrule
\end{tabular}

	}
	\caption{Example of an ESP match list for a blood type O donor reported in Baden-Württemberg. ESP donors are only offered to candidates located in vicinity of the donor, in this case Stuttgart, Tübingen, Mannheim, and Heidelberg. Candidates are currently solely ranked by their accrued dialysis time.
	}
	\label{tab:example_esp_match_list}
\end{table}

\FloatBarrier
\subsubsection{Deviation from the standard allocation procedure}
\label{subsection:rescue}
To avoid kidney nonuse, Eurotransplant deviates under specific conditions from standard allocation with extended or rescue allocation. These conditions include:
\begin{itemize}[noitemsep]
	\itemsep0em
	\item when a kidney has been turned down for donor or quality reasons by five different centers,
	\item when an ESP donor has not been allocated five hours after it was procured,
	\item when all candidates on the filtered ESP match list have received an offer (since March 2021),
	\item when loss of a transplantable graft is anticipated because of the cancellation of a planned transplantation procedure.
\end{itemize}
In general, Eurotransplant first tries to place the kidney via extended allocation, and resorts to rescue allocation only if loss of a transplantable kidney is anticipated.
\par 
Extended allocation was implemented in December 2013, and in this procedure centers located in proximity of the kidney are contacted in parallel by phone for the kidney offer. Via an online application, these centers can see which candidates meet the eligibility criteria for the kidney. Centers can then select two candidates for transplantation from this list, and the candidate who would have achieved the highest rank on the original match list receives the kidney offer. In general, Eurotransplant first tries to place the kidney via extended allocation, and resorts to rescue allocation only if loss of a transplantable kidney is anticipated.
\par
In case extended allocation is unsuccessful, or in case of an imminent loss of a transplantable kidney, Eurotransplant starts the rescue allocation procedure. In this procedure, at least three centers located in proximity of the donor are contacted by phone. The first center to propose a candidate for transplantation to Eurotransplant receives the offer. This candidate does not have to meet ETKAS eligibility criteria. 