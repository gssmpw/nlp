\begin{algorithm}[ht]
% \footnotesize
\caption{Meta-algorithm for private partition selection.\\
\selectalgo$(\sets, \eps, \delta, \Delta_0, \weightalgo, h)$
 \newline  {\bf Input: } User sets $\sets=\{(u, S_u)\}_{u\in [n]}$, privacy parameters ($\eps$, $\delta$), degree cap $\Delta_0$, weighting algorithm \weightalgo, upper bound on the novel $\ell_\infty$ sensitivity $h: \mathbb{N} \rightarrow \mathbb{R}$
 \newline  {\bf Output: } Subset of the union of user sets $U \subseteq \U = \cup_{u=1}^n S_u$, noisy weight vector $\tilde{w}_{ext}$
}
\begin{algorithmic}[1]
\label{alg:generic}
    \STATE Set $\sigma$ for the Gaussian Mechanism (\cref{prop:gaussian}) with $(\eps, \delta/2)$-DP and $\Delta_2 = 1$. \COMMENT{Set noise}
    \STATE Set $\rho \gets \max_{t \in [\Delta_0]} h(t) + \sigma \Phi^{-1}\left(\left(1 - \frac{\delta}{2}\right)^{1/t}\right)$ \COMMENT{Set threshold}
    \FORALL{$u \in [n]$} 
        \IF{$|S_u| \geq \Delta_0$} 
            \STATE Randomly subsample $S_u$ to $\Delta_0$ items. \COMMENT{Cap user degrees.}
        \ENDIF
    \ENDFOR
    \STATE $w \gets \weightalgo(\sets)$ \COMMENT{Weights on items in $\U$}
    \STATE $\tilde{w}(i) \gets w(i) + \mathcal{N}(0, \sigma^2 I)$ \COMMENT{Add noise}
    \STATE $U \gets \{i \in \U : \tilde{w}(i) \geq \rho\}$ \COMMENT{Apply threshold}
    \STATE $\tilde{w}_{ext}(i) \gets 
        \begin{cases}
        \tilde{w}(i) \; &\text{ if } i \in \U \\
        \mathcal{N}(0, \sigma^2 I) \; &\text{ if } i \in \Sigma \setminus \U
        \end{cases}$
        \\ \COMMENT{The $i \in \Sigma \setminus \U$ coordinates exist only for privacy analysis (we only ever query this vector on $i \in \U$)}
    \RETURN $U, \tilde{w}_{ext}$ 
\end{algorithmic}
\end{algorithm}


In this section, we formalize the weighting-based meta-algorithm used in prior solutions to the differentially private partition selection problem~\cite{korolova2009releasing, gopi2020dpunion, carvalho2022incorporatingitem, swanberg2023dpsips}. Our algorithm \ouralgo{} also falls within this high-level approach with a novel weighting algorithm that is both adaptive and scalable to massive data.
We alter the presentation of the algorithm from prior work in a subtle, but important, way by having the algorithm release a noisy weight vector $\tilde{w}_{ext}$ in addition to the normal set of items $U$.
This allows us to develop a two-round version of our algorithm \ouralgotworounds{} which queries noisy weights from the first round to give improved performance in the second round, leading to signficant empirical benefit.

The weight and threshold meta-algorithm is given in \cref{alg:generic}. Input is a set of user sets $\sets=\{(u, S_u)\}_{u \in [n]}$, privacy parameters $\eps$ and $\delta$, a maximum degree cap $\Delta_0$, and a weighting algorithm \weightalgo{} (which can itself take some optional input parameters), and a function $h: \mathbb{N} \rightarrow \mathbb{R}$ which describes the sensitivity of \weightalgo{}.

First, each user's set is randomly subsampled so that the size of each resulting set is at most $\Delta_0$ (the necessity of this step will be further explicated).
Then, the \weightalgo{} takes in the cardinality-capped sets and produces a set of weights over all items in the union.
Independent Gaussian noise with standard deviation $\sigma$ is added to each coordinate of the weights, and items with weight above a certain threshold $\rho$ are output.
By construction, this algorithm will only ever output items which belong to the true union,  $U \subseteq \U$, with the size of the output depending on the number of items with noised weight above the threshold.

For the sake of analysis (and not the implementation of the algorithm), we diverge from prior work to return a vector $\tilde{w}_{ext}$ of noisy weights over the entire universe $\Sigma$.
This vector is implicitly used in the proof of privacy for releasing the set of items $U$, but it is never materialized as $|\Sigma|$ is unbounded. 
%As a result, this vector will not be computed in the implementation of the algorithm. 
Within our algorithms, we will ensure that we only ever query entries of this vector which belong to $\U$, so we only ever have to materialize those entries.
Note, however, that it would \emph{not} be private to release $\tilde{w}$ as the output of a final algorithm as the domain of that vector is exactly the true union of the users' sets.

The privacy of this algorithm depends on certain ``sensitivity'' properties of \weightalgo{} as well as our choice of $\sigma$ and $\rho$. Consider any pair of neighboring inputs $\sets$ and $\sets' = \sets \cup \{(v, S_v)\}$, let $\U$ and $\U'$ be the corresponding unions, and let $w$ and $w'$ be the item weights assigned by \weightalgo{} on the two inputs, respectively.

\begin{definition}\label{def:l2-sensitivity}
The \emph{$\ell_2$ sensitivity} of a weighting algorithm is defined as the smallest value $\Delta_2$ such that,
\begin{equation*}
    \Delta_2 \geq \sqrt{\sum_{i \in \U} \left(w'(i) - w(i)\right)^2 + \sum_{i \in \U' \setminus \U} w'(i)^2}.
\end{equation*}
\end{definition}

Given bounded $\ell_2$ sensitivity, choosing the scale of noise $\sigma$ appropriately for the Gaussian mechanism in~\cref{prop:gaussian} ensures that outputting the noised weights on items in $\U$ satisfies $\left(\eps, \frac{\delta}{2} \right)$-DP.
So if we knew $\U$, then the output of the algorithm after thresholding would be private via post-processing.

However, knowledge of the union $\U$ is exactly the problem we want to solve. The challenge is that there may be items in $\U'$ which do not appear in $\U$. Let $T = \U' \setminus \U$ be these ``novel'' items with $t = |T|$. As long as the probability that any of these items are output by the algorithm is at most $\frac{\delta}{2}$, $(\eps, \delta)$-DP will be maintained. Consider a single item $i \in T$ which has zero probability of being output by a weight and threshold algorithm run on $\sets$ but is given some weight $w'(i)$ when \weightalgo{} is run on $\sets'$. The item will be output only if after adding the Gaussian noise with standard deviation $\sigma$, the noised weight exceeds $\rho$. The probability that any item in $T$ is output follows from a union bound. In order to union bound only over finitely many events, we rely on the fact that $t \leq \Delta_0$; this is why the cardinalities must be capped.
This motivates the second important sensitivity measure of \weightalgo{}.

\begin{definition}\label{def:linf-sensitivity}
The \emph{novel $\ell_\infty$ sensitivity} of a weighting algorithm is parameterized by the number $t = |T|$ of items which are unique to the new user,  and is defined as the smallest value $\Delta_\infty(t)$ such that for all possible inputs $\{S_u\}_{u=1}^n$ and new user sets $S_v$,
\begin{equation*}
    \Delta_\infty(t) \geq \max_{i \in T} w'(i).
\end{equation*}
\end{definition}
Then, the calculation of $\rho$ to obtain $(\eps, \delta)$-DP is obtained based on the novel $\ell_\infty$ sensitivity, $\delta$, $\sigma$, and $\Delta_0$. This is formalized in the following theorem.

\begin{theorem}\label{thm:generic-privacy}
Let $\sets, (\eps, \delta), \Delta_0, \weightalgo{}, h$  be inputs to \cref{alg:generic}. 
If \weightalgo{} has bounded $\ell_2$ and novel $\ell_\infty$ sensitivities
\[
    \Delta_2 \leq 1 \; \text{ and } \; \Delta_\infty(t) \leq h(t),
\]
then releasing $U, \tilde{w}_{ext}$ satisfies $(\eps, \delta)$-DP.
\end{theorem}

\begin{proof}\label{proof:generic-privacy}
Let $w_{ext}: \Sigma \rightarrow \mathbb{R}$ be an extension of the weight vector $w$ returned by \weightalgo{} where
\begin{equation*}
w_{ext}(i) = 
\begin{cases}
    w(i) \; &\text{ if } i \in \U \\
    0 \; &\text{ if } i \in \Sigma \setminus \U
\end{cases}.
\end{equation*}
Note that $\tilde{w}_{ext}$ is exactly the result of applying the Gaussian Mechanism to $w_{ext}$.
Furthermore, the $\ell_2$ sensitivity (\cref{def:l2-sensitivity}) of computing $w_{ext}$ is the same as that of $w$ as items outside of $\U'$ do not contribute to the sensitivity as they are $0$ regardless of whether the input is $\sets$ or $\sets'$.
By the choice of $\sigma$ according to \cref{prop:gaussian} with $\Delta_2 \leq 1$, releasing $\tilde{w}_{ext}$ is $\p{\eps, \frac{\delta}{2}}$-DP.

The privacy of releasing $U$ depends on our choice of the threshold $\rho$.
We will first show that the probability that any of $t$ i.i.d.\ draws from a Gaussian random variable $\mathcal{N}(0, \sigma^2)$ exceeds $\sigma \Phi^{-1}\p{\p{1 - \frac{\delta}{2}}^{1/t}}$ is exactly $\frac{\delta}{2}$.
Let $A$ be the bad event, $Y \sim \mathcal{N}(0, \sigma^2)$, and $Z \sim \mathcal{N}(0, 1)$:

\begin{align*}
   \Pr(A) &= 1 - \Pr\p{Y \leq \sigma \Phi^{-1}\p{\p{1 - \frac{\delta}{2}}^{1/t}}}^t \\
   &= 1 - \Pr\p{Z \leq \Phi^{-1}\p{\p{1 - \frac{\delta}{2}}^{1/t}}}^t \\
   &= 1 - \Phi\p{\Phi^{-1}\p{\p{1 - \frac{\delta}{2}}^{1/t}}}^t \\
   &= 1 - \p{1 - \frac{\delta}{2}}^{t/t} \\
   &= \frac{\delta}{2}.
\end{align*}

By the condition that $h(t)$ is an upper bound on $\Delta(t)$, the choice of $\rho$ implies that, no matter how many items are novel (unique to the new user in a neighboring dataset), the probability that any of them belong to $U$ is at most $\frac{\delta}{2}$.
Conditioned on the release of $\tilde{w}_{ext}$, releasing $U$ is $\p{0, \frac{\delta}{2}}$-DP.
By basic composition, the overall release is $(\eps, \delta)$-DP, as required.
\end{proof}

The uniform weighting strategy, which we refer to as \basicalgo{} (see pseudocode in \cref{alg:basic}), satisfies $\Delta_2 \leq 1$ and $\Delta_\infty(t) \leq \frac{1}{\sqrt{t}}$~\cite{gopi2020dpunion}.