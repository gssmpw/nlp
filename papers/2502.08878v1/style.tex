\usepackage[utf8]{inputenc}
\usepackage{palatino}
\usepackage{fullpage}
\usepackage[backref=page]{hyperref}
\hypersetup{
    unicode=false,          % non-Latin characters in Acrobat’s bookmarks
    colorlinks=true,        % false: boxed links; true: colored links
    linkcolor=blue,         % color of internal links (change box color with linkbordercolor)
    citecolor=purple,       % color of links to bibliography
    filecolor=magenta,      % color of file links
    urlcolor=cyan           % color of external links
}
\renewcommand\backrefxxx[3]{%
	\hyperlink{page.#1}{$\uparrow$#1}%
}
\usepackage{algorithm}
\usepackage[noend]{algorithmic}
\usepackage{booktabs}       % professional-quality tables
\usepackage{nicefrac}       % compact symbols for 1/2, etc.
\usepackage{microtype}      % microtypography
\usepackage{xcolor}         % colors
\usepackage{color}
\usepackage{mathtools}
\usepackage{amsfonts}
\usepackage{amsthm}
\usepackage{amssymb}
\usepackage{url}
\usepackage[capitalize,noabbrev,nameinlink]{cleveref}
\usepackage{multicol, multirow}
\usepackage{xfrac}
\renewcommand{\algorithmiccomment}[1]{\bgroup\hfill$\rhd$~\footnotesize{\textcolor{violet}{#1}}\egroup}
\usepackage[group-separator={,}, group-minimum-digits=4]{siunitx}
\usepackage[font=small]{caption}
\usepackage{enumitem}

\newcommand{\RETURN}{\STATE \textbf{return} }

%%%%%%%%%%%%%%%%%%%%%%%%%%%%%%%%
\theoremstyle{plain}
\newtheorem{theorem}{Theorem}[section]
\newtheorem{proposition}[theorem]{Proposition}
\newtheorem{lemma}[theorem]{Lemma}
\newtheorem{corollary}[theorem]{Corollary}
\theoremstyle{definition}
\newtheorem{definition}[theorem]{Definition}
\newtheorem{assumption}[theorem]{Assumption}
\theoremstyle{remark}
\newtheorem{remark}[theorem]{Remark}

\newcommand{\eps}{\varepsilon}
\newcommand{\ouralgo}{\texttt{MAD}}
\newcommand{\ouralgolong}{\texttt{MaxAdaptiveDegree}}
\newcommand{\ouralgotworounds}{\texttt{MAD2R}}
\newcommand{\ouralgotworoundslong}{\texttt{MaxAdaptiveDegreeTwoRounds}}
\newcommand{\userweights}{\texttt{UserWeights}}
%\newcommand{\userweights}{\texttt{GetUserWeights}}
\newcommand{\selectalgo}{\texttt{WeightAndThreshold}}
% Since we use \weightalgo to represent the generic algorithm. I suggest calling it ALG. 
%\newcommand{\weightalgo}{\texttt{Weight}}
\newcommand{\weightalgo}{\texttt{ALG}}
\newcommand{\basicalgo}{\texttt{Basic}}
\newcommand{\sets}{\mathcal{S}}
\newcommand{\dmin}{d_{min}}
\newcommand{\dmax}{d_{max}}
\newcommand{\bmin}{b_{min}}
\newcommand{\bmax}{b_{max}}
\newcommand{\dadapt}{d_{adapt}}
\newcommand{\U}{\mathcal{U}}
\newcommand{\pluseq}{\mathrel{+}=}
\newcommand\err[1]{{\scriptstyle (\pm #1)}}

\newcommand{\DeclareAutoPairedDelimiter}[3]{%
  \expandafter\DeclarePairedDelimiter\csname Auto\string#1\endcsname{#2}{#3}%
  \begingroup\edef\x{\endgroup
    \noexpand\DeclareRobustCommand{\noexpand#1}{%
      \expandafter\noexpand\csname Auto\string#1\endcsname*}}%
  \x}
\DeclareAutoPairedDelimiter\p{\lparen}{\rparen}
\DeclareAutoPairedDelimiter\br{\lbrack}{\rbrack}
\DeclareAutoPairedDelimiter\bc{\lbrace}{\rbrace}
\DeclareAutoPairedDelimiter\angle{\langle}{\rangle}
\DeclareAutoPairedDelimiter\ceil{\lceil}{\rceil}
\DeclareAutoPairedDelimiter\floor{\lfloor}{\rfloor}
\DeclareAutoPairedDelimiter\abs{\lvert}{\rvert}   % For absolute value
\DeclareAutoPairedDelimiter\norm{\lVert}{\rVert}  % For norm (double bars)

\newif\iftodos
\todosfalse


\iftodos
\newcommand\todo[1]{\textcolor{red}{TODO: #1}}
\newcommand\morteza[1]{\textcolor{olive}{Morteza: #1}}
\newcommand\mortezaEdit[1]{\textcolor{olive}{#1}}
\newcommand\justin[1]{\textcolor{blue}{Justin: #1}}
\newcommand\ale[1]{\textcolor{cyan}{Ale: #1}}
\newcommand\vincent[1]{\textcolor{pink}{V: #1}}
\else 
\newcommand\todo[1]{}
\newcommand\morteza[1]{}
\newcommand\mortezaEdit[1]{}
\newcommand\justin[1]{}
\newcommand\ale[1]{}
\newcommand\vincent[1]{}
\fi 
