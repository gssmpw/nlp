
\section{Results}

We first present the results obtained with the smartphone and then the results obtained with the SBC. 

\subsection{Smartphone-Cloudflare}

We considered the following values for the size $s$ of the payload in requests and replies: 400, 1024, and 2048 kB. The period between consecutive requests was set to 20 s. 



%The first one is hosted in Milan, while the second is located in Marseille. Both are reasonably close to the location where the client was placed (Pisa, Italy). Cloudflare specifies that any request exceeding the first one is handled by the nearest facility. The two facilities are subjected to different network conditions since they are located in different countries and are placed at different distances from the client. Hence, we decided to divide the results into two different sets, one per facility. Finally, we discarded the results of the experiments during which the facility was subject to changes. It is important to notice that the facility handling a request cannot be selected by the experimenters, and it is instead automatically chosen by Cloudflare. We preferred to remove the results of these experiments to have reasonably stable network conditions during the whole 1-hour experiments. 

For each 1-hour experiment, we processed the metrics collected by the power analyzer as follows. First, we divided the results of the experiments into 10 slots of equal size. Thus, each slot contains 6 minutes of measurements. Next, for each of these slots, we calculated the average consumption. Figure \ref{fig:Energy_1024} shows a 1-hour experiment, where each curve corresponds to the energy needed by a different version of the HTTP protocol. Points represent the average energy consumption during the considered 6-minute slot. Even though the behavior of the client application does not change during the experiment, some fluctuations can be observed in the amount of needed energy. Such fluctuations are due to changing network conditions and the possibly different load on the edge infrastructure. 


\begin{figure}[t]
    \centering
    \includegraphics[width=0.49\textwidth]{Figures/ultimi/1024K/Power.pdf}
    
    \caption{The energy consumption of the smartphone when interacting with the serverless edge infrastructure; payload size ($s$) is equal to 1024 kB.}
    \label{fig:Energy_1024}
\end{figure}   

\begin{comment} 
 \begin{figure*}[!h]
    \centering
    
    \subfloat[Cloudflare facility code MXP (Milan)]{
        \includegraphics[width=0.4\textwidth]{Figures/Results/Anova1/MXP_e-p.png}
        \label{fig:anova1_1024_proto_milan}
    }
    \hspace{1cm}
    \subfloat[Cloudflare facility code MRS (Marseille)]{
        \includegraphics[width=0.4\textwidth]{Figures/Results/Anova1/MRS_e-p_merged.png}
        \label{fig:anova1_1024_proto_marseille}
    }
    \caption{The impact of the HTTP version on the energy consumption of the smartphones. The results refers to the set of experiments using a payload of 1024000 characters. For each boxplot, the red mark indicates the median value, the upper and the lower borders show the 25th and 75th percentiles, and finally the whiskers confine all the values in the [$q_3 + 1.5 \cdot IQR$, $q_1 - 1.5 \cdot IQR$] interval.}
    \label{fig:anova1_1024_proto}
\end{figure*}
\end{comment}

\begin{table}[]
    \centering
                \begin{tabular}{rrr}
                    \toprule
                     s  & p-value (Version) & p-value (ToD)\\
                     \hline
                     400K  & 0.0025    & 0.0202    \\
                     1024K & 0         & 0.6517    \\
                     2048K & 0.0027    & 0.0072    \\
                     \bottomrule
                \end{tabular}
    \caption{p values for the different experiments}
    \label{tab:pvalues}
\end{table}

\begin{figure*}[!h]
    \centering
        \subfloat[400K, MXP]{
            \includegraphics[width=0.4\textwidth]{Figures/ultimi/400K/MXP_anovan_e-ps.pdf}
        }
        \subfloat[1024K, MRS]{
            \includegraphics[width=0.4\textwidth]{Figures/ultimi/1024K/MRS_anovan_e-ps.pdf}
        }
        \\
        \subfloat[1024K, MXP]{
            \includegraphics[width=0.4\textwidth]{Figures/ultimi/1024K/MXP_anovan_e-ps.pdf}
            \label{subfig:1024}
        }
        \subfloat[2048K, MXP]{
            \includegraphics[width=0.4\textwidth]{Figures/ultimi/2048K/MXP_anovan_e-ps.pdf}
        }
        
        \caption{ANOVAN, protocol Version and ToD are the explanatory variables, power consumption is the dependent variable.}
        \label{fig:nexus-cloudflare-anovan}
\end{figure*}

Then we performed an  analysis of variance (ANOVAN) where we considered the Time of the Day (ToD) and the version of the protocol as explanatory variables and the power consumption as the dependent variable.  
%Figure \ref{fig:nexus-cloudflare-anovan} shows the comparison for different values of $s$. Results are grouped by the protocol version (1, 2, and 3 to respectively indicate HTTP 1.1, HTTP 2, and HTTP 3) and ToD. Results with the same ToD value have been obtained at the same time of the day. The ToD values are different for the considered payload $s$ values because, as mentioned, the facility cannot be selected by us and it can be changed during an experiment, forcing us to discard many runs. In summary, the analysis has been restricted to the ToD values for which we were able to collect an entire hour of data for all the three versions of HTTP.  

Table \ref{tab:pvalues} shows the p-value resulting from the ANOVAN analysis. The version of the protocol is always significant in explaining the different average values of the energy consumption. This is also visible, for instance, in Figure~\ref{subfig:1024}, where for the three ToD values (1, 2, 6), the average power consumption of HTTP 3 is higher than the other two versions. Overall, the HTTP 3 needs XXX-YYY\% more energy than the previous versions of the protocol, with the largest additional energy needed when the payload is larger. 

%As can be seen, the energy consumption metrics are less variable when the requests are processed in Milan. In fact, the boxplots in Figure \ref{fig:anova1_1024_proto_milan} are shorter than those in Figure \ref{fig:anova1_1024_proto_marseille}.  
%Additionally, when HTTP 1.1 and HTTP2 protocols are involved the median value is smaller when the 


\begin{figure*}[!h]
    \centering
        \subfloat[raspberry, 1024K, MXP]{
            \includegraphics[width=0.4\textwidth]{Figures/ultimi/raspberry_1024K/MXP_anovan_e-ps.pdf}
        }
        \\
    
        \subfloat[raspberry, 2048K, MXP]{
            \includegraphics[width=0.4\textwidth]{Figures/ultimi/raspberry_2048K/MXP_anovan_e-ps.pdf}
        }
        \subfloat[raspberry, 4096K, MXP]{
            \includegraphics[width=0.4\textwidth]{Figures/ultimi/raspberry_4096K/MXP_anovan_e-ps.pdf}
        }
        
        \caption{ANOVAN, protocol Version and ToD are the explanatory variables, power consumption is the dependent variable.}
        \label{fig:raspberry-cloudflare-anovan}
\end{figure*}


\begin{figure*}[!h]
    \centering
        \subfloat[XXXDACONTROLLAREXXXsmartphone]{
            \includegraphics[width=0.49\textwidth]{Figures/ultimi/anocova/smartphone/no_outlier.pdf}
        }
        \subfloat[XXXDARIMUOVEREXXXraspberry zero]{
            \includegraphics[width=0.49\textwidth]{Figures/ultimi/anocova/pi0Manual/no_outlier.pdf}
        }
        \\
        \subfloat[XXXDARIMUOVEREXXXraspberry Pi3 with automaticToolbox and outlier]{
            \includegraphics[width=0.49\textwidth]{Figures/ultimi/anocova/Pi3Automated/no_filtering.pdf}
        }\subfloat[XXXDARIMUOVEREXXXraspberry Pi3 with automaticToolbox and without outlier]{
            \includegraphics[width=0.49\textwidth]{Figures/ultimi/anocova/Pi3Automated/no_outlier.pdf}
        }
        \label{fig:raspberry_smartphone-cloudflare-anocova}
\end{figure*}

\begin{figure*}[!h]
    \centering
        \subfloat[raspberry- milano]{
            \includegraphics[width=0.48\textwidth]{results/GCS/Anocova-raspberryPi3/milano.png}
        }
        \subfloat[raspberry- londra]{
            \includegraphics[width=0.48\textwidth]{results/GCS/Anocova-raspberryPi3/londra.png}
        }
        \\
        \subfloat[raspberry- carolina]{
            \includegraphics[width=0.48\textwidth]{results/GCS/Anocova-raspberryPi3/carolina.png}
        }
        \subfloat[raspberry- melbourne]{
            \includegraphics[width=0.48\textwidth]{results/GCS/Anocova-raspberryPi3/melbourne}
        }
        \caption{Esperimenti raspberryPi3, automation toolbox, ultima versioni, giorni 03/01 + 09/01}
        \label{fig:anocova-GCS}
\end{figure*}
\begin{figure*}[!h]
    \centering
        \subfloat[raspberriPI3 - MXP]{
            \includegraphics[width=0.48\textwidth]{results/cloudflare/Anocova-RaspberryPi3/cloudflare_MXP.png}
        }
        
        \caption{Esperimenti raspberryPi3, automation toolbox, ultima versioni, giorni 10/01 - 11/01}
        \label{fig:anocova-GCS}
\end{figure*}
\begin{comment}
\begin{figure*}[!h]
    \centering
        \subfloat[smartphone]{
            \includegraphics[width=0.4\textwidth]{Figures/ultimi/anocova/smartphone/no_outlier_multcompare.pdf}
        }
        \subfloat[raspberry]{
            \includegraphics[width=0.4\textwidth]{Figures/ultimi/anocova/pi0Manual/no_outlier_multcompare.pdf}
        }
        \\
        \subfloat[raspberry Pi3 with automaticToolbox and outlier]{
            \includegraphics[width=0.4\textwidth]{Figures/ultimi/anocova/Pi3Automated/no_filtering_multcompare.pdf}
        }
        \subfloat[raspberry Pi3 with automaticToolbox and without outlier]{
            \includegraphics[width=0.4\textwidth]{Figures/ultimi/anocova/Pi3Automated/no_outlier_multcompare.pdf}
        }
        \caption{}
        \label{fig:raspberry_smartphone-cloudflare-anocova}
\end{figure*}
\end{comment}




\begin{figure}[!h]
    \centering
    \subfloat[Cloudflare facility code MRS (Marseille)]{
        \includegraphics[width=0.4\textwidth]{Figures/Results/Anova1/anova1_0_es.PNG}
    }
    \\
    
    

    \subfloat[Cloudflare facility code MXP (Milan)]{
       \includegraphics[width=0.4\textwidth]{Figures/Results/Anova1/anova1_1_es.PNG}
    }
    
    \caption{The impact of the slot on the energy consumption of the smartphones. The results refers to the set of experiments using a payload of 1024000 characters.}
    \label{fig:anova1_1024_slot}
\end{figure}




