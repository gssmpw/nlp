
\section{Experimental Setup - Chiara}
\label{sec:setup}

\begin{figure}[!tb]
    \centering
    \includegraphics[width=0.4\textwidth]{Figures/clientserver_interaction.pdf}
    \caption{The communication schema between a client application running on an Android smartphne and an edge server.}
    \label{fig:client-serverInteraction}
\end{figure}

In this study, we evaluate the energy consumed by a device to interact with a edge server using multiple versions of the HTTP protocol, in particular HTTP 1.1, 2, and 3.


The device operates according to a client-server paradigm, issuing periodic requests to an edge server, as illustrated in Figure~\ref{fig:client-serverInteraction}. At the beginning of each period, the client sends a HTTP request. The server then sends a HTTP reply as soon as possible. When the HTTP reply is entirely received, the client waits until the start of the next period. In our experiments we used two types of devices, an Android Nexus 5 smartphone with Android 6 OS, and a Raspberry Pi XXX device with XXX OS. To measure the power consumption on the devices, we used a Qoitech Otii external power analyzer~\cite{otii}, capable of collecting current, voltage, and power consumption metrics. The power analyzer also acts as a power supply for the connected device. This can be particularly challenging in the smartphone case. To connect the power analyzer to the smartphone we had to make some modifications to the smartphone. We removed the battery from the smartphone, and we separated the cells from the battery control circuit. Then, we soldered two connectors directly on the circuitry. These connectors are used for connecting the smartphone and the power monitor. To compensate for any voltage drop at the input of the smartphone, we soldered 7 ceramic capacitors of 100 $\mu$F each in parallel as close as possible to the circuit. Finally, we reconnected the modified circuit to the smartphone motherboard.


We measured the energy consumption of a client application interacting with an edge server using the HTTP protocol. The client application was executed on an Android smartphone on a periodical basis. 
The client-server communication schema is illustrated in Figure \ref{fig:client-serverInteraction}. 
At the beginning of each period, the client sends to the server an HTTP POST request using $s$ bytes of payload. The payload of the first request is generated completely randomly, while only the first 40 bytes of the following request are rewritten. This allows limiting the use of computational resources in the presence of large payloads as only a few bytes are generated at each period. At the same time, it guarantees a different payload to each request, inhibiting any caching mechanisms that can be present along the path.
The server starts processing incoming requests as soon as possible. To limit the server-side processing time, the request payload is used also for the response without any modification.



\begin{figure}[!tb]
    \centering
    \includegraphics[width=0.48\textwidth]{Figures/setup.pdf}
    \caption{The setup used during the experimental phase.}
    \label{fig:experimetalSetup}
\end{figure}
We performed a set of experiments considering different versions of HTTP and different payload sizes ($s$). 
The client application was executed on a Nexus 5 with Android 6 OS. The smartphone was connected to the LTE network of a well-known Italian operator. 
To measure the consumption of the client we used an external power analyzer capable of collecting current, voltage, and power consumption metrics. Additionally, the selected appliance works also as a power supply for the smartphone. 
To connect the power monitor and the smartphone we have made some modifications to the smartphone. Precisely, we removed the battery from the smartphone, and we separated the cells from the battery control circuit. Then, we soldered directly on the circuitry two connectors. Basically, the connectors are used for connecting the smartphone and the power monitor. 
To compensate for any voltage drop at the input of the smartphone, we soldered in parallel as close as possible to the circuit 7 ceramic capacitors of 100 $\mu$F each. 
Finally, the modified circuit was reconnected to the smartphone motherboard.  
Instead, the server was executed using the Cloudflare worker service, which allows serverless functions to be run in one of the 270 Cloudflare data centers~\cite{cloudflare-worker}. Cloudflare does not allow selecting the facility. The latter is automatically chosen on the basis of a proximity criterion, thus it should be the closest one~\cite{clouflare-facility}.

In order to assess the energy consumption of our application, it is necessary to consider the impact that third-party applications might have on the consumption of the entire device. For this purpose, before the beginning of the experimental phase, we performed a factory reset of the device. After that, we removed all the pre-installed apps. Unfortunately, there are a small number of apps that could not be removed. For them, we disabled the possibility of using the network when in the background. 
\\**************\\
Instead, during the second round of experiments, we studied the smartphone when communicating with an object storage service.
This service, named Google Cloud Storage (GCS), allows users to retrieve files from buckets hosted on Google servers via the HTTP protocol.
Basically, the client application periodically sends an HTTP GET request on a regular basis, downloading a given resource from the target bucket. To prevent any potential caching mechanism, at each step, the client asks for a different resource. To assess the impact of the latency, we set up three distinct buckets located in Milan (edge), South Carolina (cloud), and Melbourne (far cloud).
\\**************\\

