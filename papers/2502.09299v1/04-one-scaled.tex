\section{Constant-factor approximation for general instances}\label{sec:bounded-stretch}

The key advantage of $2$-scaled instances is the absence of cut vertices, which simplifies the maintenance of connectivity during the reconfiguration.
Therefore, the main challenge in reconfiguring general instances lies in managing cut vertices.

Most parts of our previous method already work independent of the configuration scale.
Thus, the only modification required concerns \cref{lem:move-down}, as the polyomino may become disconnected while moving \comps{} that are not $2$-scaled.
To address this, \cref{lem:one-scaled-move-down} introduces an alternative strategy for translating \comps{} by a single unit while maintaining connectivity throughout the process.
The key observation is that two auxiliary tiles can be used to preserve local connectivity around the cut vertices that need to be moved.
The use of auxiliary tiles to preserve connectivity is also exploited in other models~\cite{akitaya2021universal,michail2019transformation}.

To simplify the arguments, we assume that the robot is capable of holding up to two auxiliary tiles, but will later explain how to modify the strategy to work with no more than a single auxiliary tile.

\begin{lemma}
	\label{lem:one-scaled-move-down}
    Let the robot hold two auxiliary tiles.
    Given a \comp{}~$F$ on a polyomino~$P$,
	we can efficiently compute a schedule of makespan $\BigO(\size{F})$ to translate~$F$ in the target direction by one unit.
\end{lemma}

Before proving this, we give a high-level description of the approach. 
Without loss of generality, we translate $F$ in southern direction.
We decompose $F$ into maximal vertical \emph{strips} of unit width and maximal horizontal \emph{corridors} of unit height, that we collectively refer to as the \emph{elements} of $F$, as visualized in~\cref{fig:one-scaled-decomposition}.
In particular, strips that only consist of a single tile are grouped with adjacent single tiles (if they exist) and become maximal horizontal corridors.
\begin{figure}[htb]
	\begin{subfigure}[t]{0.33\columnwidth}
		\centering%
		\includegraphics[page=1]{decomposition-one-scaled-short}%
	\end{subfigure}%
	\begin{subfigure}[t]{0.33\columnwidth}
		\centering%
		\includegraphics[page=2]{decomposition-one-scaled-short}%
	\end{subfigure}%
	\begin{subfigure}[t]{0.33\columnwidth}
		\centering%
		\includegraphics[page=3]{decomposition-one-scaled-short}%
	\end{subfigure}%
	\caption{%
		A decomposition of a polyomino (left) into gray strips and cyan corridors~(center).
		From that, we create an adjacency graph $G$ and (by leaving out dotted edges) a spanning tree~$T$~(right).
	}
	\label{fig:one-scaled-decomposition}
\end{figure}
We will show that we are able to translate both types of elements one unit to the south, possibly with the support of auxiliary tiles, while retaining connectivity within the respective elements.
Because of the cut vertices, the most challenging part are corridors that consist of at least two tiles.
\Cref{fig:move-down-corridor} illustrates how two auxiliary tiles can be used to translate a corridor one unit south.
\begin{figure}[htb]
	\centering%
	\includegraphics[page=1]{move-down-one-scaled-conn}%
	\caption{The steps of moving a corridor of width 4 south, using two auxiliary tiles.}
	\label{fig:move-down-corridor}
\end{figure}

Strips can be translated south without the use of auxiliary tiles.
Because translating a single element may cause disconnection to adjacent elements, we apply a recursive strategy that first handles the elements \emph{blocking} our translation.
We then move our current element, and finally process all other adjacent elements.

\begin{proof}[Proof of~\cref{lem:one-scaled-move-down}]
	We refer to the aforementioned decomposition of a polyomino into strips and corridors.
	In this construction, each corridor is adjacent to up to two strips and each strip has a height of at least~2.
	Furthermore, all adjacency is either to the west or east side, but never vertical.
	Note that because $F$ is connected, the graph $G = (V, E)$ of adjacent elements is connected as well.

	We now compute an arbitrary spanning tree $T$ from $G$; and for the remainder of the proof, adjacency of elements is considered to be in $T$.
	For two adjacent elements $A, B \in V$, we say that $A$ \emph{blocks} $B$, if translating all tiles in $B$ south by one unit would result in all tiles of $A$ losing adjacency to $B$.
	An example for an element $A$ blocking and element $B$ is illustrated in \cref{fig:one-scaled-decomposition}: if we translate~$B$ south, we disconnect $A$ from the polyomino.
	Note that $A$ blocks $B$ exactly if the northernmost tile in $B$ has the same height as the southernmost tile in $A$.
	As at least one of $A$ or $B$ is a strip (with height at least~2), $A$ and $B$ cannot block one another simultaneously.
	Because we translate every tile south by only a single unit, we do not need to update~$T$,
	which means that if $A$ blocks $B$, then once $A$ has been moved south one step, $B$ can be moved south as well without $A$ and $B$ losing connectivity.
	If $A$ and $B$ are not blocking one another, they can be moved south in any order.
	
	We next look at how to move an element $A$ south in the case that $A$ is not blocked by any adjacent element.
	We use two auxiliary tiles which the robot holds at the beginning and end of the operation.
	If $A$ is a strip or a single-tile corridor, we place one tile south of $A$ and then remove the northernmost tile.
	If $A$ is a corridor of two or more tiles, we place the two auxiliary tiles right below the two westernmost tiles of $A$.
	We then move each tile by two steps to the east and one position south, starting from the east.
	The last two tiles become new auxiliary tiles; see \cref{fig:move-down-corridor}.
	Clearly, moving $A$ south has makespan $\BigO(\size{A})$.
		
	Finally, the robot has to traverse $T$ such that every element is moved south only after its adjacent blocking elements.
	This can be done recursively, starting at any element in $T$ in the following way.
	At the current element $A$:
	
	\begin{enumerate}
		\item Recurse on all adjacent elements that block $A$ and are not moved south yet.
		\item Move $A$ south.
		\item Recurse on the remaining unvisited elements adjacent to $A$.
	\end{enumerate}
	
	Because $T$ is connected, every element is visited with this strategy.
	By ordering the adjacent elements, each recursion on an element $A$ requires only $\BigO(\size{A})$ additional movements to move towards the corresponding elements.
	
	It remains to show that no element is recursed on more than once.
	Assume that some element $B$ is recursed on twice.
	As there are no cycles in $T$ and all unvisited or blocking neighbors are recursed on in each invocation, $B$ can only be recursed on for a second time by some element $A$ which was previously recursed on by $B$.
	This second invocation stems from the first step of handling $A$ because the third step excludes visited elements, which means that $B$ blocks $A$, and $B$ is not moved south.
	Thus, the first invocation on $B$ did not yet reach the second step.
	In other words, the recursive call on $A$ resulted from $A$ blocking $B$.
	This is a contradiction, as $A$ and $B$ cannot mutually block one another.
\end{proof}

It remains to argue how two auxiliary tiles can be obtained and emulated in our model.
By doing so, we obtain the following.

\begin{theorem}
	\label{thm:reconfig-general}
	There exists a constant $c$ such that for any pair of configurations $C_s,C_t\in \configs(n)$ with disjoint bounding boxes, we can efficiently compute a schedule for $C_s\rightrightarrows C_t$ with weighted makespan at most $c\cdot \OPT$.
\end{theorem}

\begin{proof}
	The idea is to execute \cref{lem:one-scaled-move-down} instead of \cref{lem:move-down} every time it is required in \cref{thm:reconfig-2-scaled}.
	To this end, we explain the required modifications for \cref{lem:turn-to-histo}; the same adaptations also work for $H_s \rightrightarrows H_s'$.
	Before we translate the first \comp{} $F$, we move two additional tiles to $F$.
	These tiles can be any tiles from $C_s$ that are not necessary for connectivity (e.g., any leaf from a spanning tree of the dual graph of $C_s$).
	Obtaining these auxiliary tiles requires $\BigO(n)$ movements.
	Next, we need to sequentially move both tiles from one \comp{} to the next whenever we continue our walk on the configuration.
	We can bound these steps with the length of the walk, thus requiring~$\BigO(n)$ movements in sum.

	Now, we only have to argue that \cref{lem:one-scaled-move-down} still holds if the robot initially carries a single tile and must not carry more than one tile at a time:
	Whenever the robot finishes translating the current element, it only picks up one of the two auxiliary tiles and moves it to the next unvisited element.
	This tile is placed in an adjacent position to the target direction of this element (which must be free).
	The robot then moves back and picks up the other auxiliary tile.
	As this extra movement requires~$\BigO(\size{F})$ additional moves in total, the makespan of \cref{lem:one-scaled-move-down} is still in~$\BigO(\size{F})$.

	With these adaptations, we can apply \cref{lem:one-scaled-move-down} instead of \cref{lem:move-down} in \cref{thm:reconfig-2-scaled}.
	Moreover, the makespan of \cref{lem:turn-to-histo} still lies in $\BigO(n + \lowerboundOf{C_s,H_s})$.
	By that, the analysis of the overall makespan is identical to \cref{thm:reconfig-2-scaled}.
	The only difference is that schedules created by \cref{lem:one-scaled-move-down} do not necessarily have optimal carry distance, which also applies to the overall schedule.
\end{proof}
