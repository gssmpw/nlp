\section{Discussion}\label{sec:discussion}

In this paper, we presented progress on the reconfiguration problem for tile-based structures (i.e., polyominoes) within abstract material-robot systems.
In particular, we showed that the problem is \NP-complete for any weighting between moving with or without carrying a tile.

Complementary to this negative result, we developed an algorithm to reconfigure two polyominoes into one another in the case that both configurations are contained in disjoint bounding boxes.
The computed schedules are within a constant factor of the optimal reconfiguration schedule.
It is easy to see that our approach can also be used to construct a polyomino, rather than reconfigure one into another.
Instead of deconstructing a start configuration to generate building material, we can assume that tiles are located within a ``depot'' from which they can be picked up.
Performing the second half of the algorithm works as before and builds the target configuration out of tiles from the depot.

Several open questions remain.
A natural open problem is to adapt the approach to instances in which the bounding boxes of the configurations intersect, i.e., they overlap, or are nested.
Furthermore, it seems plausible that our methods can be generalized to be performed by many robots in parallel.
Much more intricate is the question on whether a fully distributed approach is possible.