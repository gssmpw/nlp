\subsection{Phase (II): Reconfigure a histogram into a histogram}\label{subsec:histogram-to-histogram}
\label{sec:hist-to-hist}

It remains to show how to transform one histogram into the other.
By the assumption of the existence of a horizontal bisector between the bounding boxes of~$C_s$ and~$C_t$, the histogram $H_s$ is north-facing, whereas $H_t$ is south-facing.
The bounding box of $C_s$ is vertically extended to share exactly one $y$-coordinate with the bounding box of $C_t$, and this is where both histogram bases are placed; see~\cref{fig:hist-to-hist} for illustration.
Note that the histograms may not yet overlap.
However, by \cref{cor:translation}, the tiles in $H_s$ can be moved toward $H_t$ in asymptotically optimal makespan until both histogram bases share a tile.

\begin{lemma}
    \label{lem:hist-to-hist}
    Let $H_s$ be a north-facing and $H_t$ a south-facing histogram that share at least one base tile.
    We can efficiently compute a schedule of makespan $\BigO(n + \lowerboundOf{H_s,H_t})$ for $H_s \rightrightarrows H_t$ with optimal carry distance.
\end{lemma}

To prove \cref{lem:hist-to-hist}, we use a simple observation about \emph{crossing paths}, i.e., paths that share a vertex in the integer grid.
We denote by $L_1(s,t)$ the length of a shortest path between the vertices $s$ and $t$.
\begin{observation}
    \label{obs:swapping}
    If any two shortest paths between tile positions $s_i$ and $t_i$ with $i\in \{1,2\}$ cross, then  $L_1(s_1,t_1) + L_1(s_2,t_2) \leq L_1(s_1,t_2) + L_1(s_2,t_1)$ holds.
\end{observation}
This follows from splitting both paths at the crossing vertex and applying triangle~inequality.

\begin{proof}[Proof of \cref{lem:hist-to-hist}]
    We first describe how the schedule $S$ can be constructed and
    then argue that the matching induced by $S$ is an MWPM in~$\bipGraph{H_s, H_t}$.
    We conclude the proof by showing that the schedule is of size $\BigO(n + \lowerboundOf{H_s,H_t})$.

    We write $B_s$ and $B_t$ for the set of all base positions in $H_s$ and $H_t$, respectively, and denote the union of all base positions by $B \coloneqq B_s \cup B_t$.
    We assume that all robot movements between positions in $H_s$ and $H_t$ are realized along a path that moves vertically until $B$ is reached, continues horizontally on $B$ and then moves vertically to the target position.
    These are shortest paths by construction.

    We denote the westernmost and easternmost position in $B$ by $b_W$ and $b_E$, respectively.
    Without loss of generality, we assume $b_E \in B_t$, because this condition can always be reached by either mirroring the instance horizontally or swapping $B_s$ and $B_t$ and applying the resulting schedule in reverse.

    First consider $b_W \in B_s$.
    We order all tiles in both $H_s \setminus H_t$ and $H_t \setminus H_s$ from west to east and north to south, see the left part of \cref{fig:hist-to-hist-greedy}.
    Based on this ordering, $S$ is created by iteratively moving the first remaining tile in $H_s \setminus H_t$ to the first remaining position in $H_t \setminus H_s$.
    This way, all moves can be realized on shortest paths over the bases.

    If $b_W \in B_t \setminus B_s$, this is not possible as the westernmost position in $H_t$ may not yet be reachable.
    Therefore, we iteratively extend $B_s$ to the west until the western part of $B_t$ is constructed.
    The tiles are taken from $H_s$ in the same order as before, see the right part of \cref{fig:hist-to-hist-greedy}.
    Once we have moved a tile to $b_W$, we can continue as above with $b_W \in B_s$.

	\begin{figure}[htb]
		\begin{subfigure}[t]{0.5\columnwidth}
			\centering%
			\includegraphics[page=1]{hist-to-hist-greedy-more-annotations}%
		\end{subfigure}%
		\begin{subfigure}[t]{0.5\columnwidth}
			\centering%
			\includegraphics[page=2]{hist-to-hist-greedy-more-annotations}%
		\end{subfigure}%
		\caption{%
            Position orderings in $H_s\setminus H_t$ and $H_t\setminus H_s$.
            Left: The main case of $b_W \in B_s$.
            Right: Additional steps are required for $b_W \in B_t \setminus B_s$.
            Tiles are moved according to the ordering, as indicated by the arrows.
        }
		\label{fig:hist-to-hist-greedy}
	\end{figure}

    Let $M$ be the matching induced by $S$. We now give an iterative argument why $M$ has the same sum of distances as an arbitrary MWPM $\minMatching$ with $\weight{\minMatching} = \lowerboundOf{H_s,H_t}$.
    First consider $b_W \in B_s$.
    Given the ordering from above, we write $s_i$ and $t_i$ for the $i$th tile position in $H_s \setminus H_t$ and $H_t \setminus H_s$, respectively.
    If $M \neq \minMatching$, there exists a minimal index~$i$ such that $(s_i, t_i) \in M \setminus \minMatching$.
    In $\minMatching$, $s_i$ and $t_i$ are instead matched to other positions $s_j$ and~$t_k$ with~$j, k > i$, i.e., $(s_i, t_k),(s_j, t_i)\in \minMatching$.
    Since $j, k > i$, the positions $s_j$ and $t_k$ do not lie to the west of $s_i$ and $t_i$, respectively.
    Now note that the paths for $(s_i, t_k)$ and $(s_j, t_i)$ always cross:
    Let $b_s$ and $b_t$ be the positions in $B$ that have same $x$-coordinates as $s_i$ and $t_i$, respectively.
    If $s_i$ lies to the east of $t_i$ (case~i), the path from $s_j$ to $t_i$ crosses~$b_s$, and so does any path that starts in $s_i$.
    If $s_i$ does not lie to the east of $t_i$ (case~ii), the path from $s_i$ to $t_k$ crosses $b_t$, and so does any path that ends in $t_i$.
    We illustrate both cases in \cref{fig:hist-to-hist-crossing}.

	\begin{figure}[htb]
		\begin{subfigure}[t]{0.5\columnwidth}
			\centering%
			\includegraphics[page=1]{hist-to-hist-crossing.pdf}%
		\end{subfigure}%
		\begin{subfigure}[t]{0.5\columnwidth}
			\centering%
			\includegraphics[page=2]{hist-to-hist-crossing.pdf}%
		\end{subfigure}%
		\caption{%
            The paths for $(s_i, t_k)$ and $(s_j, t_i)$ cross in $b_s$ or $b_t$.
            Left: $s_i$ lies to the east of $t_i$ (case~i).
            Right: $s_i$ does not lie to the east of $t_i$ (case~ii).
        }
		\label{fig:hist-to-hist-crossing}
	\end{figure}

    As the paths cross, replacing $(s_i, t_k), (s_j, t_i)$ with $(s_i, t_i), (s_j, t_k)$ in $\minMatching$ creates a new matching $\minMatching'$ with $\weight{\minMatching'} \leq \weight{\minMatching}$ due to \cref{obs:swapping}.
    By repeatedly applying this argument, we can turn $\minMatching$ into $M$ and note that $\weight{M} \leq \weight{\minMatching}$ holds.
    If $b_W \in B_t \setminus B_s$, case~(i) applies to all tiles moved to extend $B_s$ in western direction, and we can construct $M$ from $\minMatching$ analogously.
    Thus, the matching induced by $S$ is an MWPM.

    Since the carry distance $\cDist{S}$ is exactly $\lowerboundOf{H_s,H_t}$, it remains to bound the empty distance $\eDist{S}$ with $\BigO(n + \lowerboundOf{H_s,H_t})$.
    To this end, we combine all paths from one drop-off position back to its pickup position (in sum $\lowerboundOf{H_s,H_t}$) with all paths between consecutive pickup locations (in sum $\BigO(n)$ because the locations are ordered).
    Since the robot does not actually return to the pickup location, these paths of total length $\BigO(n + \lowerboundOf{H_s,H_t})$ are an upper bound for $\eDist{S}$, and we get $\makesp{S} \in \BigO(n + \lowerboundOf{H_s,H_t})$.
\end{proof}

\subsection{Correctness of the algorithm}
\label{sec:makespan}
In the previous sections, we presented schedules for each phase of the overall algorithm.
We~will now leverage these insights to prove the main result of this section, restated here.

\thmReconfigTwoScaled*
\begin{proof}
    By \cref{lem:turn-to-histo,lem:hist-to-hist}, the makespan of the three phases is bounded by~$\BigO(n + \lowerboundOf{C_s,H_s})$, $\BigO(n + \lowerboundOf{H_s,H_t})$, and $\BigO(n + \lowerboundOf{H_t,C_t})$, respectively, which we now bound by $\BigO(\lowerboundOf{C_s,C_t})$, proving asymptotic optimality for $C_s \rightrightarrows C_t$.

    Clearly, $n \in \BigO(\lowerboundOf{C_s,C_t})$, as each of the~$n$ tiles has to be moved due to $C_s\cap C_t=\varnothing$.
    We prove a tight bound on the remaining terms, i.e.,
    \begin{equation}
        \label{eq:lower_bound_sum}
        \lowerboundOf{C_s,H_s} + \lowerboundOf{H_s,H_t} + \lowerboundOf{H_t,C_t} = \lowerboundOf{C_s,C_t}.
    \end{equation}

    In Phase (I), tiles are moved exclusively towards the bounding box of~$C_t$ along shortest paths to obtain $H_s$, so $\lowerboundOf{C_s,C_t} = \lowerboundOf{C_s,H_s} + \lowerboundOf{H_s,C_t}$.
    The same applies to the reverse of Phase (III), i.e., $\lowerboundOf{C_t,H_s} = \lowerboundOf{C_t,H_t} + \lowerboundOf{H_t,H_s}$.
    Due to symmetry of the lower bound, \cref{eq:lower_bound_sum} immediately follows.
    The total makespan is thus in $\BigO(\lowerboundOf{C_s,C_t})=\BigO(\OPT)$.
\end{proof}

\Cref{lem:turn-to-histo,lem:hist-to-hist} establish schedules that minimize the carry distance and \cref{eq:lower_bound_sum} ensures that the combined paths remain shortest possible.
Thus, the provided schedule has optimal carry distance.
In~particular, we obtain an optimal schedule when $\lambda = 0$, which corresponds to the case where the robot incurs no cost for movement when not carrying~a~tile.

\begin{corollary}
    For any pair of $2$-scaled configurations $C_s,C_t\in \configs(n)$ with disjoint bounding boxes and $\lambda=0$, we can efficiently compute an optimal schedule for ${C_s\rightrightarrows C_t}$.
\end{corollary}
