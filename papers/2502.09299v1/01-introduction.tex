\section{Introduction}\label{sec:introduction}
Building and modifying structures consisting of many basic components is an
important objective, both in fundamental theory and in a spectrum of practical settings.
Transforming such structures with the help of autonomous robots is
particularly relevant in very large~\cite{2020-Coordinating_IWOCA} and
very small dimensions~\cite{Yangsheng:five_mm_robot}
that are hard to access for direct human manipulation, e.g.,
in extraterrestrial space~\cite{BENLARBI20213598,jenett2017design} or in microscopic environments~\cite{bfh+-cgur-17}.
This gives rise to the natural algorithmic problem of rearranging a
given start configuration of (potentially very many) \emph{passive} objects
by a relatively small set of \emph{active} agents to a desired target configuration.
Performing such reconfiguration at scale faces a number of critical issues,
including (i) the cost and weight of materials, (ii) the potentially disastrous
accumulation of errors, (iii) the development of simple yet resilient agents to carry
out the active role, and (iv) achieving overall feasibility and efficiency.

In recent years, significant advances have been made to deal with these
challenges. On the macroscopic side, ultra-light and scalable composite lattice
materials~\cite{Cheung1219,ultralight24,gregg2018ultra}, address the first
issue, making it possible to construct modular, reconfigurable
structures with platforms such as NASA's \BILLE and \ARMADAS~\cite{jenett2016meso,jenett2017bille,ultralight24}; the underlying self-adjusting lattice
also resolves the issues (ii) of accuracy and error correction,
allowing it to focus on discrete, combinatorial structures,
consisting of regular tiles (in two dimensions) or voxels (in three dimensions).
A further step has been the development of (iii) simple autonomous
robots~\cite{9836082,jenett2019material}
that can be used to carry out complex tasks,
as shown in~\cref{fig:intro-bille}: The robot can move on the tile arrangement, remove
individual tiles and physically relocate them to a new position by walking on
the remaining configuration, which needs to remain connected at all times.
This is mirrored on the microscopic scales, where the development of
microbots~\cite{zhang2019robotic} and even nanobots~\cite{santos2022light}
promises fundamentally new approaches to configuring objects
and mechanisms, such as forming specific configurations or gathering
in designated locations for tasks such as targeted drug
delivery~\cite{2020-Targeted_ICRA,2022-gather_IROS}, again based on
discrete, lattice-based structures.

\begin{figure}
    \centering
    \includegraphics[scale=0.25]{bille_walk_01-02}
    \caption{A simple \BILLE robot that can move on a configuration of digital, light-weight
material and relocate individual voxels for overall reconfiguration. Photo adapted from~\cite{jenett2019material}.}
    \label{fig:intro-bille}
\end{figure}

In this paper, we deal with the algorithmic challenge (iv): How can we use such a robot to transform a given start configuration
into a desired goal arrangement, as quickly as possible?

\subsection{Our contributions}\label{subsec:our-results}

We investigate the problem of finding minimum cost reconfiguration schedules for a single active
robot operating on a (potentially large) number of tiles.
In particular, we obtain the following results for the \probName problem.
\begin{enumerate}[(1)]
	\item {We present a generalized version of the problem, parameterized by weighted costs for moving with or without tiles, and show that this is \NP-complete.}
	\item {On the other hand, we give a polynomial-time constant-factor approximation algorithm for the case of disjoint start and target bounding boxes.
	Our~approach yields optimal carry distance for $2$-scaled instances.}
\end{enumerate}

For precise model and problem descriptions, we refer to~\cref{subsec:preliminaries}.


\subsection{Related work}\label{subsec:related-work}

Garcia et al.~\cite{single-bille-reconfig-IROS,cooperative-bille-reconfig-ICRA} showed that computing optimal schedules for reconfiguring polyominoes with \BILLE bots, see~\cite{jenett2017bille} and \cref{fig:intro-bille}, is in general \NP-complete.
For tackling the problem, they designed heuristic approaches exploiting rapidly exploring random trees~(RRT), and a time-dependent variant of the A$^{*}$ search algorithm.
Furthermore, they used target swapping heuristics to reduce the overall traveled distance in case of multiple \BILLE bots.

A different context for reconfiguration arises from
programmable matter~\cite{gmyr2018recognition,gmyr2020forming,hinnenthal2024efficient}.
Here, even finite automata are capable of building bounding boxes of tiles
around polyominoes, as well as scale and rotate them while maintaining
connectivity at all times~\cite{fekete2022connected,NiesReconfig}.

When considering active matter, i.e., the arrangement is composed of self-moving objects (or agents), multiple different models exist~\cite{almethen2020pushing,almethen2022efficient,connor2025transformation,michail2019transformation}.
For example, in the \emph{sliding cube model} (or the \emph{sliding square model} in two dimensions), first introduced by Fitch et al.~\cite{FitchBR03,FitchBR05} in 2003, agents are allowed to make two different moves, namely, sliding along other agents, or performing a convex transition, i.e., moving around a convex corner.
Akitaya~et~al.~\cite{AkitayaDKKPSSUW22} show that universal sequential reconfiguration in two dimensions is possible, even while maintaining connectivity of all intermediate configurations, but minimizing the makespan of a schedule is \NP-complete.
Recently, Abel et al.~\cite{AbelAKKS24} and Kostitsyna~et~al.~\cite{KostitsynaOPPSS24} independently give similar results for the three dimensional setting.
Most recently, Akitaya~et~al.~\cite{parallel-sliding-squares} show that it is \NP-complete to decide whether there exists a schedule of makespan~$1$ in the parallel sliding square model and they provide a worst-case optimal algorithm.

In a slightly more relaxed model, Fekete et al.~\cite{FeketeKKRS23-journal-connected,FeketeKRS022-journal-labeled-connected} show that parallel connected reconfiguration of a swarm of (labeled) agents is \NP-complete, even for deciding whether there is a schedule of makespan $2$.
Complementary, they present algorithms for computing constant stretch schedules, i.e., the ratio between the makespan of a schedule and a natural lower bound (the maximum minimum distance between an individual start and target position) is bounded by a constant.


\subsection{Preliminaries}
\label{subsec:preliminaries}

For the following, we refer to~\cref{fig:preliminaries}.
We are given a fixed set of~$n$ indistinguishable square \emph{tile} modules located at discrete, unique positions $(x,y)$ in the infinite integer grid~$\mathbb{Z}^2$.
If their positions induce a connected subgraph of the grid, we say that the tiles form a connected \emph{configuration} or \emph{polyomino}.
Let~$\configs(n)$ refer to the set of all polyominoes of $n$ tiles.

\begin{figure}[htb]
	\begin{subfigure}[t]{\columnwidth}
		\centering%
		\includegraphics[page=1]{intro-moves-colors}%
		\caption{Left to right: The cardinal directions, a robot without and with payload, a tile, and tiles that only exist in either the start or the target configuration, respectively.}
		\label{fig:legend}
	\end{subfigure}%
	\vspace{2em}
	\begin{subfigure}[t]{\columnwidth}
		\centering%
		\includegraphics[page=2]{intro-moves-colors}%
		\caption{A valid schedule to reconfigure the instance to the left.
		The robot can move on tiles, in cardinal directions.
		Also, it can pick up and drop off adjacent tiles.}
		\label{fig:example-schedule}
	\end{subfigure}%
	\caption{A brief overview of legal operations in our model.}
	\label{fig:preliminaries}
\end{figure}

Consider a robot that occupies a single tile at any given time.
At each discrete time step, the robot can either move to an adjacent tile, pick up a tile from an adjacent position (if it is not already carrying one), or place a tile at an adjacent unoccupied position (if it is carrying a tile).
Tiles may only be picked up if the resulting configuration is valid, i.e., the operation preserves connectivity.

We use cardinal directions to navigate;
the unit vectors $(1,0)$ and $(0,1)$ correspond to \emph{east} and \emph{north}, respectively.
Naturally, their opposing vectors extend \emph{west} and \emph{south}.

A \emph{schedule} $S$ is a finite, connectivity-preserving sequence of operations to be performed by the robot.
As the robot's motion is restricted to movements on the polyomino, we refer to distances it travels as \emph{geodesic} distances.
Let~$\cDist{S}$ denote the \emph{carry distance}, which is the sum of geodesic distances between consecutive pickups and drop-offs in $S$.
This represents the total distance the robot travels while carrying a tile, with an additional unit of distance added each time the robot either picks up or places a tile.
Similarly, the \emph{empty distance} $\eDist{S}$ is the sum of geodesic distances between drop-offs and pickups in~$S$.
We~say that~$S$ is a schedule for ${C_s\rightrightarrows C_t}$ exactly if it transforms $C_s$ into~$C_t$.

\subparagraph*{Problem statement.}
We consider \probName:
Given two connected configurations $C_s,C_t\in\configs(n)$ and a rational weight factor $\lambda\in[0,1]$,
determine a schedule $S$ for $C_s\rightrightarrows C_t$ that minimizes the \emph{weighted makespan}
$\makesp{S} \coloneqq \lambda\cdot\eDist{S} + \cDist{S}$.
We refer to the minimum weighted makespan for any given instance as \OPT.
