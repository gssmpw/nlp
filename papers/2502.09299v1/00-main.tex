\documentclass[a4paper,UKenglish,cleveref,thm-restate]{lipics-v2021}

\usepackage{complexity}
\usepackage{nicefrac}
\usepackage{mathtools}
\usepackage{amsfonts}
\usepackage{thmtools}
\usepackage{xspace}
\newcommand{\R}{\mathbb{R}}
\newcommand{\Sym}{\mathbb{S}}
\newcommand{\Lam}{\bm{\Lambda}}
\newcommand{\Rext}{\mathbf{R} \cup \{ \infty \}}
\newcommand{\dd}{\,\mathrm{d}}
\newcommand{\ip}[2]{\langle #1, #2 \rangle}
\newcommand{\norm}[1]{\| #1 \|}
\newcommand{\logdet}[1]{\log \det( #1 )}

% bb
\newcommand{\bbE}{\mathbb{E}}

%bold
\newcommand{\bxi}{\mathbf{\xi}}
\newcommand{\btau}{\mathbf{\tau}}
\newcommand{\bs}{\mathbf{s}}
\newcommand{\bz}{\mathbf{z}}
\newcommand{\bX}{\mathbf{X}}
\newcommand{\bx}{\mathbf{x}}
\newcommand{\ba}{\mathbf{a}}
\newcommand{\bc}{\mathbf{c}}
\newcommand{\bh}{\mathbf{h}}
\newcommand{\bw}{\mathbf{w}}
\newcommand{\bg}{\mathbf{g}}
\newcommand{\bp}{\mathbf{p}}
\newcommand{\bq}{\mathbf{q}}
\newcommand{\by}{\mathbf{y}}
\newcommand{\bl}{\bm{\lambda}}
\newcommand{\be}{\bm{\varepsilon}}
\newcommand{\bt}{\bm{\theta}}
\newcommand{\bmu}{\bm{\mu}}
\newcommand{\bsigma}{\bm{\sigma}}
\newcommand{\bnu}{\bm{\nu}}
\newcommand{\bphi}{\bm{\phi}}
\newcommand{\T}{\mathrm{T}}

\newcommand{\bmm}{\mathbf{m}}
\newcommand{\bS}{\mathbf{S}}
\newcommand{\bH}{\mathbf{H}}
\newcommand{\bV}{\mathbf{V}}
\newcommand{\bA}{\mathbf{A}}
\newcommand{\bD}{\mathbf{D}}


\newcommand{\ind}{\mathbf{i}}
\newcommand{\bE}{\mathbf{E}}
\newcommand{\bu}{\mathbf{u}}
\newcommand{\bG}{\mathbf{\Gamma}}
\newcommand{\bW}{\mathbf{W}}

%cal
\newcommand{\cL}{\mathcal{L}}
\newcommand{\cJ}{\mathcal{J}}
\newcommand{\cC}{\mathcal{C}}
\newcommand{\cD}{\mathcal{D}}
\newcommand{\cH}{\mathcal{H}}
\newcommand{\cX}{\mathcal{X}}
\newcommand{\cY}{\mathcal{Y}}
\newcommand{\cK}{\mathcal{K}}
\newcommand{\cP}{\mathcal{P}}
\newcommand{\PM}{\mathcal{P}}
\newcommand{\cM}{\mathcal{M}}
\newcommand{\cN}{\mathcal{N}}
\newcommand{\cQ}{\mathcal{Q}}
\newcommand{\cF}{\mathcal{F}}
\newcommand{\bbS}{\mathbb{S}}

% names
\newcommand{\dist}{\mathrm{dist}}
\newcommand{\epi}{\mathrm{epi}}

% other
\newcommand{\W}{\mathrm{W}}
\newcommand{\V}{\mathrm{V}}

% vector/matrix notation
\newcommand{\vc}[1]{\bm{#1}}
\newcommand{\matr}[1]{\mathbf{#1}}

\newcommand{\Ent}{\mathrm{H}}

%operators
%operators
\DeclareMathOperator{\Div}{Div}
\DeclareMathOperator{\dom}{dom}
\DeclareMathOperator{\ran}{ran}
\DeclareMathOperator{\conv}{conv}
\DeclareMathOperator{\relint}{relint}
\DeclareMathOperator{\inter}{int}
\DeclareMathOperator{\boundary}{bdry}
\DeclareMathOperator{\moreau}{M}
\DeclareMathOperator{\hull}{H}
\DeclareMathOperator{\prox}{P}
\DeclareMathOperator{\proj}{proj}
\DeclareMathOperator{\tr}{tr}
\DeclareMathOperator{\diag}{diag}
\DeclareMathOperator{\Id}{I}
\DeclareMathOperator{\cl}{cl}
\DeclareMathOperator*{\argmin}{arg\,min}
\DeclareMathOperator*{\argmax}{arg\,max}
\DeclareMathOperator*{\limin}{lim~inf}
\DeclareMathOperator*{\mean}{mean}

\DeclareMathOperator{\KLop}{KL}
\newcommand{\myKL}{\mathbb{D}_{\KLop}\infdivx}
\newcommand{\myKLs}{{\mathbb{D}_{\KLop}}^*\infdivx}
\newcommand{\Breg}[1]{\mathbb{B}_{#1}\infdivx}

%\renewcommand{\algorithmicrequire}{\textbf{Input:}}
%\renewcommand{\algorithmicensure}{\textbf{Iterate:}}



\usepackage[subrefformat=simple,labelformat=simple]{subcaption}
\renewcommand\thesubfigure{(\alph{subfigure})}

\graphicspath{{./figures/}}

\bibliographystyle{plainurl}

\hideLIPIcs
\nolinenumbers

\title{Moving Matter:
Efficient Reconfiguration of Tile Arrangements by a Single Active Robot}
\titlerunning{Efficient Reconfiguration of Tile Arrangements by a Single Active Robot}

\author{Aaron T. Becker}{Electrical Engineering, University of Houston, Texas, USA}{atbecker@uh.edu}{https://orcid.org/0000-0001-7614-6282}{}
\author{Sándor P. Fekete}{Computer Science, TU Braunschweig, Germany}{s.fekete@tu-bs.de}{https://orcid.org/0000-0002-9062-4241}{}
\author{Jonas Friemel}{Electrical Engineering and Computer Science, Bochum University of Applied Sciences, Germany}{jonas.friemel@hs-bochum.de}{https://orcid.org/0009-0009-6270-4779}{}
\author{Ramin Kosfeld}{Computer Science, TU Braunschweig, Germany}{kosfeld@ibr.cs.tu-bs.de}{https://orcid.org/0000-0002-1081-2454}{}
\author{Peter Kramer}{Computer Science, TU Braunschweig, Germany}{kramer@ibr.cs.tu-bs.de}{https://orcid.org/0000-0001-9635-5890}{}
\author{Harm Kube}{Computer Science, TU Berlin, Germany}{h.kube@campus.tu-berlin.de}{https://orcid.org/0009-0001-5072-7908}{}
\author{Christian Rieck}{Discrete Mathematics, University of Kassel, Germany}{christian.rieck@mathematik.uni-kassel.de}{https://orcid.org/0000-0003-0846-5163}{}
\author{Christian Scheffer}{Electrical Engineering and Computer Science, Bochum University of Applied Sciences, Germany}{christian.scheffer@hs-bochum.de}{https://orcid.org/0000-0002-3471-2706}{}
\author{Arne Schmidt}{Computer Science, TU Braunschweig, Germany}{aschmidt@ibr.cs.tu-bs.de}{https://orcid.org/0000-0001-8950-3963}{}

\authorrunning{Becker, Fekete, Friemel, Kosfeld, Kramer, Kube, Rieck, Scheffer, Schmidt}

\Copyright{Aaron T. Becker, Sándor P. Fekete, Jonas Friemel, Ramin Kosfeld, Peter Kramer, Harm Kube, Christian Rieck, Christian Scheffer, Arne Schmidt}

\ccsdesc{Theory of computation~Computational geometry}

\keywords{Programmable matter, Reconfiguration, Polyominoes, Assembly, Path planning, Approximation, NP-completeness}

\funding{Work from TU Braunschweig and HS Bochum was partially supported by the German Research Foundation (DFG), project ``Space~Ants'', FE~407/22-1 and SCHE~1931/4-1. Work from the University of Kassel was partially supported by DFG grant 522790373.}

\widowpenalty10000
\clubpenalty10000

\begin{document}

    \maketitle

    \begin{abstract}
        We consider the problem of reconfiguring a two-dimensional connected grid arrangement of passive building blocks
        from a start configuration to a goal configuration, using a single active robot that can move on the tiles,
        remove individual tiles from a given location and physically move them to a new
        position by walking on the remaining configuration. The objective is to
        determine a reconfiguration schedule that minimizes the overall
        makespan, while ensuring that the tile configuration remains connected.
        We provide both negative and positive results.
        (1) We present a generalized version of the problem, parameterized by weighted
        costs for moving with or without tiles, and show that this is \NP-complete.
        (2) We give a polynomial-time constant-factor approximation
        algorithm for the case of disjoint start and target bounding boxes. In addition, our approach yields optimal carry distance for 2-scaled instances.
    \end{abstract}

    
% humans are sensitive to the way information is presented.

% introduce framing as the way we address framing. say something about political views and how information is represented.

% in this paper we explore if models show similar sensitivity.

% why is it important/interesting.



% thought - it would be interesting to test it on real world data, but it would be hard to test humans because they come already biased about real world stuff, so we tested artificial.


% LLMs have recently been shown to mimic cognitive biases, typically associated with human behavior~\citep{ malberg2024comprehensive, itzhak-etal-2024-instructed}. This resemblance has significant implications for how we perceive these models and what we can expect from them in real-world interactions and decisionmaking~\citep{eigner2024determinants, echterhoff-etal-2024-cognitive}.

The \textit{framing effect} is a well-known cognitive phenomenon, where different presentations of the same underlying facts affect human perception towards them~\citep{tversky1981framing}.
For example, presenting an economic policy as only creating 50,000 new jobs, versus also reporting that it would cost 2B USD, can dramatically shift public opinion~\cite{sniderman2004structure}. 
%%%%%%%% 图1:  %%%%%%%%%%%%%%%%
\begin{figure}[t]
    \centering
    \includegraphics[width=\columnwidth]{Figs/01.pdf}
    \caption{Performance comparison (Top-1 Acc (\%)) under various open-vocabulary evaluation settings where the video learners except for CLIP are tuned on Kinetics-400~\cite{k400} with frozen text encoders. The satisfying in-context generalizability on UCF101~\cite{UCF101} (a) can be severely affected by static bias when evaluating on out-of-context SCUBA-UCF101~\cite{li2023mitigating} (b) by replacing the video background with other images.}
    \label{fig:teaser}
\end{figure}


Previous research has shown that LLMs exhibit various cognitive biases, including the framing effect~\cite{lore2024strategic,shaikh2024cbeval,malberg2024comprehensive,echterhoff-etal-2024-cognitive}. However, these either rely on synthetic datasets or evaluate LLMs on different data from what humans were tested on. In addition, comparisons between models and humans typically treat human performance as a baseline rather than comparing patterns in human behavior. 
% \gabis{looks good! what do we mean by ``most studies'' or ``rarely'' can we remove those? or we want to say that we don't know of previous work doing both at the same time?}\gili{yeah the main point is that some work has done each separated, but not all of it together. how about now?}

In this work, we evaluate LLMs on real-world data. Rather than measuring model performance in terms of accuracy, we analyze how closely their responses align with human annotations. Furthermore, while previous studies have examined the effect of framing on decision making, we extend this analysis to sentiment analysis, as sentiment perception plays a key explanatory role in decision-making \cite{lerner2015emotion}. 
%Based on this, we argue that examining sentiment shifts in response to reframing can provide deeper insights into the framing effect. \gabis{I don't understand this last claim. Maybe remove and just say we extend to sentiment analysis?}

% Understanding how LLMs respond to framing is crucial, as they are increasingly integrated into real-world applications~\citep{gan2024application, hurlin2024fairness}.
% In some applications, e.g., in virtual companions, framing can be harnessed to produce human-like behavior leading to better engagement.
% In contrast, in other applications, such as financial or legal advice, mitigating the effect of framing can lead to less biased decisions.
% In both cases, a better understanding of the framing effect on LLMs can help develop strategies to mitigate its negative impacts,
% while utilizing its positive aspects. \gabis{$\leftarrow$ reading this again, maybe this isn't the right place for this paragraph. Consider putting in the conclusion? I think that after we said that people have worked on it, we don't necessarily need this here and will shorten the long intro}


% If framing can influence their outputs, this could have significant societal effects,
% from spreading biases in automated decision-making~\citep{ghasemaghaei2024understanding} to reducing public trust in AI-generated content~\citep{afroogh2024trust}. 
% However, framing is not inherently negative -- understanding how it affects LLM outputs can offer valuable insights into both human and machine cognition.
% By systematically investigating the framing effect,


%It is therefore crucial to systematically investigate the framing effect, to better understand and mitigate its impact. \gabis{This paragraph is important - I think that right now it's saying that we don't want models to be influenced by framing (since we want to mitigate its impact, right?) When we talked I think we had a more nuanced position?}




To better understand the framing effect in LLMs in comparison to human behavior,
we introduce the \name{} dataset (Section~\ref{sec:data}), comprising 1,000 statements, constructed through a three-step process, as shown in Figure~\ref{fig:fig1}.
First, we collect a set of real-world statements that express a clear negative or positive sentiment (e.g., ``I won the highest prize'').
%as exemplified in Figure~\ref{fig:fig1} -- ``I won the highest prize'' positive base statement. (2) next,
Second, we \emph{reframe} the text by adding a prefix or suffix with an opposite sentiment (e.g., ``I won the highest prize, \emph{although I lost all my friends on the way}'').
Finally, we collect human annotations by asking different participants
if they consider the reframed statement to be overall positive or negative.
% \gabist{This allows us to quantify the extent of \textit{sentiment shifts}, which is defined as labeling the sentiment aligning with the opposite framing, rather then the base sentiment -- e.g., voting ``negative'' for the statement ``I won the highest prize, although I lost all my friends on the way'', as it aligns with the opposite framing sentiment.}
We choose to annotate Amazon reviews, where sentiment is more robust, compared to e.g., the news domain which introduces confounding variables such as prior political leaning~\cite{druckman2004political}.


%While the implications of framing on sensitive and controversial topics like politics or economics are highly relevant to real-world applications, testing these subjects in a controlled setting is challenging. Such topics can introduce confounding variables, as annotators might rely on their personal beliefs or emotions rather than focusing solely on the framing, particularly when the content is emotionally charged~\cite{druckman2004political}. To balance real-world relevance with experimental reliability, we chose to focus on statements derived from Amazon reviews. These are naturally occurring, sentiment-rich texts that are less likely to trigger strong preexisting biases or emotional reactions. For instance, a review like ``The book was engaging'' can be framed negatively without invoking specific cultural or political associations. 

 In Section~\ref{sec:results}, we evaluate eight state-of-the-art LLMs
 % including \gpt{}~\cite{openai2024gpt4osystemcard}, \llama{}~\cite{dubey2024llama}, \mistral{}~\cite{jiang2023mistral}, \mixtral{}~\cite{mistral2023mixtral}, and \gemma{}~\cite{team2024gemma}, 
on the \name{} dataset and compare them against human annotations. We find  that LLMs are influenced by framing, somewhat similar to human behavior. All models show a \emph{strong} correlation ($r>0.57$) with human behavior.
%All models show a correlation with human responses of more than $0.55$ in Pearson's $r$ \gabis{@Gili check how people report this?}.
Moreover, we find that both humans and LLMs are more influenced by positive reframing rather than negative reframing. We also find that larger models tend to be more correlated with human behavior. Interestingly, \gpt{} shows the lowest correlation with human behavior. This raises questions about how architectural or training differences might influence susceptibility to framing. 
%\gabis{this last finding about \gpt{} stands in opposition to the start of the statement, right? Even though it's probably one of the largest models, it doesn't correlate with humans? If so, better to state this explicitly}

This work contributes to understanding the parallels between LLM and human cognition, offering insights into how cognitive mechanisms such as the framing effect emerge in LLMs.\footnote{\name{} data available at \url{https://huggingface.co/datasets/gililior/WildFrame}\\Code: ~\url{https://github.com/SLAB-NLP/WildFrame-Eval}}

%\gabist{It also raises fundamental philosophical and practical questions -- should LLMs aim to emulate human-like behavior, even when such behavior is susceptible to harmful cognitive biases? or should they strive to deviate from human tendencies to avoid reproducing these pitfalls?}\gabis{$\leftarrow$ also following Itay's comment, maybe this is better in the dicsussion, since we don't address these questions in the paper.} %\gabis{This last statement brings the nuance back, so I think it contradicts the previous parapgraph where we talked about ``mitigating'' the effect of framing. Also, I think it would be nice to discuss this a bit more in depth, maybe in the discussion section.}






    \section{Computational complexity of the problem}\label{sec:computational-complexity}

In this section, we investigate the computational complexity of the decision variant of the generalized reconfiguration problem.
In particular, we prove that the problem is \NP-complete for any rational factor $\lambda\in[0,1]$.
This generalizes a result by Garcia et al.~\cite{cooperative-bille-reconfig-ICRA}, who demonstrate that the problem is \NP-complete for the special case of $\lambda=1$.

\begin{theorem}
	\label{thm:weighted-hardness}
	\probName is \NP-complete for any rational $\lambda$.
\end{theorem}

By considering the discrete steps of a schedule individually, we can confirm the moves' validity and certify a schedule's total cost.
It follows that the \probName problem is in \NP{} for any rational~$\lambda\in[0,1]$.
We distinguish between two cases of (1)~$\lambda \in (0,1]$ and (2) $\lambda = 0$ and give separate reductions from two \NP-complete problems.

For (1), we reduce from the \textsc{Hamiltonian path} problem in grid graphs~\cite{ItaiPS82}, which asks whether there exists a path in a given grid graph that visits each vertex exactly once.
We modify a construction previously used by Garcia~et~al.~\cite{cooperative-bille-reconfig-ICRA}, where the high-level idea is an expansion of the grid graph, placing small reconfiguration tasks at every vertex of the original graph.
An ``ideal'' schedule visits each vertex exactly once.
\begin{lemma}
	\label{lem:hardness-lambda-not-zero}
	\probName is \NP-complete for $\lambda\in (0,1]$.
\end{lemma}

\begin{proof}
	We reduce from the \textsc{Hamiltonian Path} problem in grid graphs~\cite{ItaiPS82}, which asks us to decide whether there exists a path in a given graph that visits each vertex exactly once.
	We use a modified variant of the construction described by Garcia~et~al.~\cite{cooperative-bille-reconfig-ICRA}, defining our start and target configurations based on the gadgets from~\cref{fig:hardness-bridges} as follows.

	\begin{figure}[htb]
		\begin{subfigure}[t]{0.27\textwidth-0.5em}
			\centering%
			\includegraphics[page=2]{hardness-bridges}%
			\caption{The vertex gadget.}%
			\label{fig:hardness-bridges-vertex}%
		\end{subfigure}%
		\hfill%
		\begin{subfigure}[t]{0.73\textwidth-0.5em}
			\centering%
			\includegraphics[page=1]{hardness-bridges}%
			\caption{An edge gadget, note the blue and red paths.}%
			\label{fig:hardness-bridges-edge}%
		\end{subfigure}%
		\caption{Our construction forces the robot to visit each vertex gadget at least once, traversing edge gadgets along either the red or the blue path.}%
		\label{fig:hardness-bridges}
	\end{figure}
	Let $k\gg \nicefrac{1}{\lambda}$.
	For each vertex of a given grid graph $G_{\mathit{in}}$, we create a copy of a $7\times 7$ \emph{vertex gadget} that represents a constant-size, locally solvable task, see~\cref{fig:hardness-bridges-vertex}.
	We connect all adjacent vertex gadgets with copies of a $(k+1)\times (4k+3)$ \emph{edge gadget}, as depicted in~\cref{fig:hardness-bridges-edge}.
	While we reuse the exact vertex gadget from~\cite{cooperative-bille-reconfig-ICRA}, our edge gadget is a little more complex.
	A robot traveling through the gadget has two options, indicated by red and blue paths in~\cref{fig:hardness-bridges-edge}.

	The blue path offers a simple, walkable option of $6k+2$ empty travel units, while the red path is only available if the robot performs $2$ carry steps to construct and deconstruct a unit length bridge at the center, temporarily creating a walkable path of $4k+2$ units.
	This is cheaper exactly if $\lambda(6k+2)>\lambda(4k+2)+2$,
	i.e., if $k > \nicefrac{1}{\lambda}$.
	Since $\lambda$ is a rational number, we can pick a value $k$ that is polynomial in $\lambda$ and permits a construction in which taking the red path is always cheaper.
	The blue path thus serves solely for the purpose of providing connectivity and will never be taken by the robot.

	An optimal schedule will therefore select a minimal number of edge gadgets to cross, always taking the red path.
	As vertex gadgets always require the same number of moves to solve, we simply capture the total cost for a single vertex gadget as $t_g$.
	For a graph $G_{\mathit{in}}$ with~$m$ vertices, an optimal schedule has makespan
	$m\cdot t_g + (m-1)\cdot (4k\lambda+2\lambda+2)$
	exactly if the underlying graph is Hamiltonian, and greater otherwise.

	Given a connected grid graph, such a schedule never builds bridges of length greater than one:
	Due to their pairwise distance, this will never be cheaper than crossing an edge gadget along the red path, see also~\cite{cooperative-bille-reconfig-ICRA}.
\end{proof}

This has the following implication.

\begin{corollary}
	Given two configurations $C_s,C_t\in\configs(n)$ and an integer $k\in\mathbb{N}$, it is \NP-complete to decide whether there exists an optimal schedule $C_s\rightrightarrows C_t$ with at most $k$ pickup (and at most $k$ drop-off) operations.
\end{corollary}

The reduction for (2) is significantly more involved: As $\lambda = 0$, the robot is effectively allowed to ``teleport'' across the configuration.
We reduce from \textsc{Planar Monotone~3Sat}~\cite{dbk-obspp-10}, utilizing and adapting the reduction given by Akitaya~et~al.~\cite{AkitayaDKKPSSUW22} for the sequential sliding square problem.
\begin{lemma}
	\label{lem:hardness-lambda-zero}
	\probName is \NP-complete for $\lambda = 0$.
\end{lemma}
We reduce from \textsc{Planar Monotone 3Sat}, an \NP-complete variant of the \textsc{Sat} problem, which asks whether a given Boolean formula $\varphi$ in conjunctive normal form is satisfiable, given the following properties:
First, each clause consists of at most~$3$~literals, all either positive or negative.
Second, the clause-variable incidence graph $G_\varphi$ admits a plane drawing in which variables are mapped to the $x$-axis, and all positive (resp., negative) clauses are located in the same half-plane, such that edges do not cross the $x$-axis, see~\cref{fig:hardness-teleportation-graph}.

\begin{figure}[htb]
	\centering%
	\includegraphics[page=1]{hardness-teleportation2x}%
	\caption{A rectilinear embedding of the clause-variable incidence graph of the formula ${\varphi= (x_1\lor x_3\lor x_4)\land (x_1\lor x_3)\land (\overline{x_2}\lor\overline{x_3}\lor\overline{x_4})}$.}%
	\label{fig:hardness-teleportation-graph}%
\end{figure}

We assume, without loss of generality, that each clause contains exactly three literals.
Otherwise, we can extend a shorter clause with a redundant copy of one of the existing literals, e.g., $(x_1\lor x_4)$ becomes $(x_1\lor x_4\lor x_4)$ in~\cref{fig:hardness-teleportation-graph}.

Our reduction maps from an instance $\varphi$ of \textsc{Planar Monotone 3Sat} to an instance~$\mathcal{I}_\varphi$ of \probName such that the minimal feasible makespan~$\mathcal{I}_\varphi$ is determined by whether $\varphi$ is satisfiable.
Recall that, due to $\lambda=0$, we only account for carry distance.
Consider an embedding of the clause-variable incidence graph  $G_\varphi$ as above, where $C$ and $V$ refer to the $m$ clauses and $n$ variables of $\varphi$, respectively.
We construct $\mathcal{I}_\varphi$ as follows.

A \emph{variable gadget} is placed on the $x$-axis for each $x_i\in V$, with gadgets connected along the axis in a straight line.
Intuitively, the variable gadget asks the robot to move a specific tile west along one of two feasible paths.
\begin{figure}[htb]
	\centering%
	\includegraphics[page=2]{hardness-teleportation2x}%
	\caption{The variable gadget. The middle segments are repeated to match the incident clauses. Note the positive and negative paths (blue and red).}%
	\label{fig:hardness-teleportation-variable}%
\end{figure}
These paths are highlighted in red and blue in~\cref{fig:hardness-teleportation-variable}, each of length exactly $9(\delta(x_i)+1)$, where $\delta(x_i)$ refers to the degree of $x_i$ in the clause-variable incidence graph $G_\varphi$.
Both paths require the robot to place its payload into the highlighted gaps to minimize the makespan, stepping over it before picking up again.

Connected to these variable gadgets are \emph{clause gadgets}, exactly one per clause, as illustrated in~\cref{fig:hardness-teleportation-clause}.
\begin{figure}[h!tb]%
	\centering%
	\includegraphics[page=3]{hardness-teleportation2x}%
	\caption{A clause gadget and its attachment to a variable.}%
	\label{fig:hardness-teleportation-clause}%
\end{figure}%
Each clause gadget is composed of a large curve that resembles an upside-down ``L'' attached to the $x$-maximal variable gadget of its literals, with three vertical \emph{prongs} extending vertically towards variables.
The prongs terminate at positions diagonally adjacent to the variable gadget of a corresponding literal.
The tile at the $90^\circ$ bend in said ``L''-shape must be moved by one unit to the south and west, respectively.
Due to the connectivity constraint, this cannot be realized without first establishing a secondary connection between the prongs and one of the variable gadgets.

\begin{proof}[Proof of~\cref{lem:hardness-lambda-zero}]
	The following refers to the aforementioned construction.

	Consider a Boolean formula $\varphi$ for the \textsc{Planar Monotone 3Sat} with $n$ variables and $m$ clauses.
	Recall that each clause can be assumed to contain exactly three literals, i.e, $\sum_{i=1}^{n}\delta(x_i) = 3m$.
	We show that $\varphi$ is satisfiable exactly if there exists a schedule for the constructed instance $\mathcal{I}_\varphi = (C_\varphi, C_\varphi')$ of the reconfiguration problem that achieves a weighted makespan of
	\begin{equation}
		\label{eq:sat-hardness-makespan}
		2m + \sum_{i=1}^{n}9(\delta(x_i)+1)\quad =\quad 2m + 9(3m+n)\quad =\quad 29m+9n.
	\end{equation}
	\begin{claim}
		\label{clm:hardness-teleportation-positive}
		There exists a schedule for $\mathcal{I}_\varphi$ with weighted makespan $29m+9n$ if the Boolean formula $\varphi$ is satisfiable.
	\end{claim}
	\begin{claimproof}
		Consider a satisfying assignment $\alpha(\varphi)$ for the Boolean formula $\varphi$ and let~$\alpha(x_i) \in\{{x_i},\overline{x_i}\}$ represent the literal of $x_i$ with a positive Boolean value, i.e., the literal such that $\alpha(x_i)=\texttt{true}$.
		Using only $\alpha(\varphi)$ and the two paths illustrated in~\cref{fig:hardness-teleportation-variable}, we can construct a schedule for $\mathcal{I}_\varphi$ as follows.

		We follow the path corresponding to~$\alpha(x_i)$, picking up the tile and placing it at the first unoccupied position.
		If this position is adjacent to a clause gadget's prong, this corresponds to the pickup/drop-off sequence $(a)$ in \cref{fig:hardness-teleportation-variable-example}.
		\begin{figure}[htb]
			\centering%
			\includegraphics[page=4]{hardness-teleportation2x}%
			\caption{The variable gadget for $x_3$ as seen in~\cref{fig:hardness-teleportation-graph} and a schedule with makespan $29=9(\delta(x_3)+1) + 2$ that solves an adjacent clause gadget.}%
			\label{fig:hardness-teleportation-variable-example}%
		\end{figure}
		With the marked cell occupied, there then exists a cycle containing the tile that is to be moved in the clause gadget, allowing us to solve the clause gadget in just two steps, marked $(b)$ in \cref{fig:hardness-teleportation-variable-example}.
		We can repeat this process, moving the tile to the next unoccupied position on its path until it reaches its destination.

		This pattern takes makespan exactly $9(\delta(x_i)+1)$ per variable gadget; $9$ for each incident clause and $9$ as the base cost of traveling north and south.

		As $\alpha(\varphi)$ is a satisfying assignment for $\varphi$, every clause of $\varphi$ contains at least one literal $v$ such that $\alpha(v)=\texttt{true}$.
		It follows that every clause gadget can be solved in exactly $2$ steps by the above method, resulting in the exact weighted makespan outlined in~\cref{eq:sat-hardness-makespan}, i.e., $29m+9n$ for $n$ variables and $m$ clauses.
	\end{claimproof}
	\begin{claim}
		There does not exist a schedule for $\mathcal{I}_\varphi$ with weighted makespan $29m+9n$ if the Boolean formula $\varphi$ is not satisfiable.
	\end{claim}
	\begin{claimproof}
		It is easy to observe that there is no schedule that solves a variable gadget for $x_i$ in makespan less than ${9(\delta(x_i)+1)}$:
		Building a bridge structure horizontally along the variable gadget takes a quadratic number of moves in the variable gadget's width, and shorter bridges fail to shortcut the north-south distance that the robot would otherwise cover while carrying the tile.
		Furthermore, we can confine tiles to their respective variable gadgets by spacing them appropriately.

		Even if $\varphi$ is not satisfiable, each variable gadget can clearly be solved in a makespan of ${9(\delta(x_i)+1)}$.
		However, this no longer guarantees the existence of a cycle in every clause gadget at some point in time, leaving us to solve at least one clause gadget in a more expensive manner.
		In particular, we incur an extra cost of at least $4$ units per unsatisfied clause, as a cycle containing the cyan tile must be created in each clause gadget before the tile can be safely picked up.

		Assuming that $m'\in (1,m]$ is the minimum number of clauses that remain unsatisfied in $\varphi$ for any assignment of $x_i$, we obtain a total makespan of
		\begin{equation}
			\label{eq:hardness-teleportation-greedy}
			2m+4m'+\sum_{i=1}^{n}9(\delta(x_i)+1)\quad =\quad 4m'+29m+9n\quad >\quad 29m+9n.
		\end{equation}
		We conclude that a weighted makespan of $29m+9n$ is not achievable and must be exceeded by at least four units if the Boolean formula $\varphi$ is not satisfiable.
	\end{claimproof}
	This concludes our proof of \NP-completeness for $\lambda=0$.
\end{proof}

Our central complexity result \cref{thm:weighted-hardness} is simply the union of \cref{lem:hardness-lambda-not-zero,lem:hardness-lambda-zero}.


    \section{Constant-factor approximation for 2-scaled instances}\label{sec:bounded-approx}

We now turn to a special case of the optimization problem
in which the configurations have \emph{disjoint bounding boxes}, i.e., there exists an axis-parallel bisector that separates the configurations.
Without loss of generality, let this bisector be horizontal such that the target configuration lies south.
We present an algorithm that computes schedules of makespan at most $c\cdot \OPT$ for some fixed $c\geq 1$.

For the remainder of this section, we additionally impose the constraint that both the start and target configurations are \emph{$2$\nobreakdash-scaled}, i.e., they consist of $2 \times 2$-squares of tiles
aligned with a $2\times 2$ integer grid.
In the subsequent section, we extend our result to non-scaled~configurations.

\begin{restatable}{theorem}{thmReconfigTwoScaled}
    \label{thm:reconfig-2-scaled}
    There exists a constant $c$
    such that for any pair of $2$-scaled configurations $C_s,C_t\in \configs(n)$ with disjoint bounding boxes, we can efficiently compute a schedule for $C_s\rightrightarrows C_t$ with weighted makespan at most $c\cdot \OPT$.
\end{restatable}

Our algorithm utilizes a type of intermediate configuration called \emph{histogram}.
A~histogram consists of a \emph{base} strip of unit height and (multiple) orthogonal unit width \emph{columns} attached to its base.
The direction of its columns determines the orientation of a histogram, e.g., the histogram $H_s$ in \cref{fig:hist-to-hist} is \emph{north-facing}.

\begin{figure}[htb]
	\centering%
	\includegraphics[page=1]{hist-to-hist-overview}%
	\caption{
		An example for a start and target configuration $C_s, C_t$, the intermediate histograms $H_s, H_t$ with a shared baseline, and the horizontally translated $H_s'$ that shares a tile with $H_t$.
		If $H_s$ and $H_t$ overlap, we set $H_s' \coloneqq H_s$.
	}
	\label{fig:hist-to-hist}
\end{figure}

We proceed in three phases, see~\cref{fig:hist-to-hist} for an illustration.
\begin{description}
     \item[Phase (I).] {%
        Transform the configuration $C_s$ into a north-facing histogram $H_s$.
    }
    \item[Phase (II).] {%
        Translate $H_s$ to overlap with the target bounding box and transform it into a south-facing histogram $H_t$ contained in the bounding box~of~$C_t$.
    }
    \item[Phase (III).] Finally, apply Phase (I) in reverse to obtain $C_t$ from $H_t$.
\end{description}
We reduce this to two subroutines:
Transforming any $2$\nobreakdash-scaled configuration into a $2$\nobreakdash-scaled histogram and converting any two such histograms into one another.

\subsection{Preliminaries for the algorithm}
\label{subsec:preliminaries-algorithm}
Before investigating the algorithm in more detail, we give some useful definitions.
For two configurations $C_s, C_t \in \configs(n)$, the weighted bipartite graph $\bipGraph{C_s, C_t}=({C_s \cup C_t}, {C_s \times C_t}, L_1)$ assigns each edge a weight corresponding to the $L_1$-distance between its end points.

A \emph{perfect matching} $M$ in $\bipGraph{C_s, C_t}$ is a subset of edges such that every vertex is incident to exactly one of them;
its weight $\weight{M}$ is defined as the sum of its edge weights.
By definition, there exists at least one such matching in $\bipGraph{C_s, C_t}$.
Furthermore, a \emph{minimum weight perfect matching}~(MWPM) is a perfect matching $M$ of minimum weight $\lowerboundOf{C_s,C_t}=\weight{M}$, which is a natural lower bound on the necessary carry distance, and thus \OPT.

Let $S$ be any schedule for ${C_s\rightrightarrows C_t}$.
Then $S$ induces a perfect matching in~$\bipGraph{C_s, C_t}$, as it moves every tile of~$C_s$ to a distinct position of $C_t$.
We say that~$S$ has \emph{optimal carry distance} exactly if $\cDist{S}=\lowerboundOf{C_s,C_t}$.

\subsection{Phase (I): Transform a configuration into a histogram}
\label{subsec:building-a-histogram}

We proceed by constructing a histogram from an arbitrary $2$-scaled configuration by moving tiles strictly in one cardinal direction.

\begin{lemma}
    \label{lem:turn-to-histo}
    Let $C_s$ be a $2$-scaled polyomino and let $H_s$ be a histogram that can be created from $C_s$ by moving tiles in only one cardinal direction.
    We can efficiently compute a schedule of makespan $\BigO(n + \lowerboundOf{C_s,H_s})$ for $C_s \rightrightarrows H_s$ with optimal carry distance.
\end{lemma}

To achieve this, we iteratively move sets of tiles in the target direction by two units, until the histogram is constructed.
We give a high-level explanation of our approach by example of a north-facing histogram, as depicted in~\cref{fig:move-down}.

\begin{figure}[htb]
    \centering%
    \includegraphics[page=1]{create-histogram-shortened}%
    \caption{Visualizing \cref{lem:turn-to-histo}.
    Left: A walk across all tiles (red), the set $H$ (gray) and \comps{} (green).
    Right: Based on the walk, the \comps{} are iteratively moved south to reach a histogram shape.}
    \label{fig:move-down}
\end{figure}
Let $P$ be any intermediate $2$-scaled polyomino obtained by moving tiles south  while realizing ${C_s \rightrightarrows H_s}$.
Let $H$ be the set of maximal vertical strips of tiles that contain a base tile in $H_s$, i.e., all tiles that do not need to be moved further~south.
We define the \emph{\comps{}} of $P$ as the set of connected components in $P \setminus H$.
By definition, these exist exactly if $P$ is not equivalent to $H_s$, and once a tile becomes part of $H$, it is not moved again until $H_s$ is obtained.

We compute a \emph{walk} that covers the polyomino, i.e., a path that is allowed to traverse edges multiple times and visits each vertex at least once, by depth-first traversal on an arbitrary spanning tree of $P$.
The robot then continuously follows this walk, iteratively moving \comps south and updating its path accordingly:
Whenever it enters a \comp $F$ of $P$, we perform a subroutine (as we will describe and prove in~\cref{lem:move-down}) with makespan $\BigO(\size{F})$ that moves $F$ south by two units.
Adjusting the walk afterward may increase its length by $\BigO(\size{F})$ units per \mbox{\comp}.

Upon completion of this algorithm, we can bound both the total time spent performing the subroutine and the additional cost incurred by extending the walk by $\BigO(\lowerboundOf{C_s,H_s})$.
Taking into account the initial length of the walk, the resulting makespan is $\BigO(n+\lowerboundOf{C_s,H_s})$.

Before proving this result, we describe and prove the crucial subroutine for the translation of \comps, that can be stated as follows.

\begin{lemma}
    \label{lem:move-down}
    Given a \comp{} $F$ of a $2$-scaled polyomino $P$, we can efficiently compute a schedule of makespan $\BigO(\size{F})$ to translate~$F$ in the target direction by two units.
\end{lemma}

\begin{proof}
    Without loss of generality, we assume that the target direction is south
    and show how to translate $F$ south by one unit without losing connectivity.
    We then apply this~routine~twice.

    We follow a bounded-length walk across $F$
    that visits exclusively tiles with a tile neighbor in northern direction.
    Such a walk can be computed by depth-first traversal of $F$.
    Whenever the robot enters a maximal vertical strip of $F$ for the first time, it picks up the northernmost tile and places it at the first unoccupied position to its south.

    Clearly, the sum of movements on vertical strips for carrying tiles and returning to the pre-pickup position can be bounded with $\BigO(\size{F})$, as each strip is handled exactly once.
    This bound also holds for the length of the walk.
\end{proof}

By applying \cref{lem:move-down} on the whole polyomino $P$ instead of just a \comp, we can translate $P$
in any direction with asymptotically optimal makespan.

\begin{corollary}
    \label{cor:translation}
    Any $2$-scaled polyomino can be translated by $k$ units in any cardinal direction by a schedule of weighted makespan $\BigO(n\cdot k)$.
\end{corollary}

With this, we are equipped to prove the main result for the first phase.

\begin{proof}[Proof of~\cref{lem:turn-to-histo}]
	Without loss of generality, we assume that the target direction is south.

	We start by identifying a walk $W$ on $C_s$ that passes every edge at most twice, by depth-first traversal of a spanning tree of $C_s$.
	To construct our schedule, we now simply move the robot along~$W$.
	On this walk, we denote the $i$th encountered \comp{} with $F_i$ as follows.
	Whenever the robot enters a position~$t$ that belongs to a \comp{}~$F_i$, we move~$F_i$ south by two positions (refer to~\cref{lem:move-down}) and modify both~$t$ and~$W$ in accordance with the south movement that was just performed.
	The robot stops on the new position for~$t$ and we denote the next \comp{} by~$F_{i+1}$.
	We keep applying \cref{lem:move-down} until~$t$ does not belong to a \comp{}, from where on the robot continues on the modified walk~$W$.
	We~illustrate this approach in \cref{fig:move-down}.

	While it is clear that repeatedly moving all \comps{} south eventually creates $H_s$, it remains to show that this happens within $\BigO(n + \lowerboundOf{C_s,H_s})$ moves.
	To this end, we set $\sumcomps{} = \sum \size{F_i}$.
	We first show that the schedule has a makespan in $\BigO(n + \sumcomps{})$ and then argue that $\sumcomps{} \in \Theta(\lowerboundOf{C_s, H_s})$.
	Clearly, all applications of \cref{lem:move-down} result in a makespan of $\BigO(\sumcomps{})$, and since $\size{W} \in \BigO(n)$ we can also account for the number of steps on the initial walk~$W$.
	However, additional steps may be required:
	Whenever only \emph{parts} of $W$ are moved south by two positions, the distance between the involved consecutive positions in $W$ increases by two, see \cref{fig:walk-enlargement}.
	We account for these additional movements as follows.

	\begin{figure}[htb]
		\centering%
		\includegraphics[page=2]{walk-enlargement-1-scaled}%
		\caption{%
			When all tiles in a \comp{}~$F$ are moved south, the walk~$W$ on the polyomino may become longer.
		}
		\label{fig:walk-enlargement}
	\end{figure}

	For any \comp{}~$F_i$, there are at most $2\size{F_i}$ edges between $F_i$ and the rest of the configuration.
	Each transfer increases the movement cost by two and each edge is traversed at most twice, increasing the length of $W$ by at most $8\size{F_i}$ movements for every $F_i$.
	Consequently, $\BigO(n + \sumcomps{})$ also accounts for these movements.

	It remains to argue that $\sumcomps{} \in \Theta(\lowerboundOf{C_s, H_s})$.
	Clearly, as the schedule for moving $F_i$ two units south induces a matching with a weight in $\Theta(\size{F_i})$, combining all these schedules induces a matching with a weight in $\Theta(\sumcomps{})$.
	We next notice that all schedules for $C_s\rightrightarrows H_s$ that move tiles exclusively south have the same weight because the sum of edge weights in the induced matchings is merely the difference between the sums over all $y$\nobreakdash-positions in~$C_s$ and~$H_s$, respectively.
	Finally, all these matchings have minimal weight, i.e., $\lowerboundOf{C_s, H_s}$:
	If there exists an MWPM between $C_s$ and $H_s$ with crossing paths, these crossings can be removed by \cref{obs:swapping} without increasing the matching's weight.

	By that, $\sumcomps{} \in \Theta(\lowerboundOf{C_s, H_s})$ holds and the bound of $\BigO(n + \lowerboundOf{C_s,H_s})$ results.
	Also, since \cref{lem:move-down} moves all tiles exclusively south, the schedule has optimal carry distance.
\end{proof}
\subsection{Phase (II): Reconfigure a histogram into a histogram}\label{subsec:histogram-to-histogram}
\label{sec:hist-to-hist}

It remains to show how to transform one histogram into the other.
By the assumption of the existence of a horizontal bisector between the bounding boxes of~$C_s$ and~$C_t$, the histogram $H_s$ is north-facing, whereas $H_t$ is south-facing.
The bounding box of $C_s$ is vertically extended to share exactly one $y$-coordinate with the bounding box of $C_t$, and this is where both histogram bases are placed; see~\cref{fig:hist-to-hist} for illustration.
Note that the histograms may not yet overlap.
However, by \cref{cor:translation}, the tiles in $H_s$ can be moved toward $H_t$ in asymptotically optimal makespan until both histogram bases share a tile.

\begin{lemma}
    \label{lem:hist-to-hist}
    Let $H_s$ be a north-facing and $H_t$ a south-facing histogram that share at least one base tile.
    We can efficiently compute a schedule of makespan $\BigO(n + \lowerboundOf{H_s,H_t})$ for $H_s \rightrightarrows H_t$ with optimal carry distance.
\end{lemma}

To prove \cref{lem:hist-to-hist}, we use a simple observation about \emph{crossing paths}, i.e., paths that share a vertex in the integer grid.
We denote by $L_1(s,t)$ the length of a shortest path between the vertices $s$ and $t$.
\begin{observation}
    \label{obs:swapping}
    If any two shortest paths between tile positions $s_i$ and $t_i$ with $i\in \{1,2\}$ cross, then  $L_1(s_1,t_1) + L_1(s_2,t_2) \leq L_1(s_1,t_2) + L_1(s_2,t_1)$ holds.
\end{observation}
This follows from splitting both paths at the crossing vertex and applying triangle~inequality.

\begin{proof}[Proof of \cref{lem:hist-to-hist}]
    We first describe how the schedule $S$ can be constructed and
    then argue that the matching induced by $S$ is an MWPM in~$\bipGraph{H_s, H_t}$.
    We conclude the proof by showing that the schedule is of size $\BigO(n + \lowerboundOf{H_s,H_t})$.

    We write $B_s$ and $B_t$ for the set of all base positions in $H_s$ and $H_t$, respectively, and denote the union of all base positions by $B \coloneqq B_s \cup B_t$.
    We assume that all robot movements between positions in $H_s$ and $H_t$ are realized along a path that moves vertically until $B$ is reached, continues horizontally on $B$ and then moves vertically to the target position.
    These are shortest paths by construction.

    We denote the westernmost and easternmost position in $B$ by $b_W$ and $b_E$, respectively.
    Without loss of generality, we assume $b_E \in B_t$, because this condition can always be reached by either mirroring the instance horizontally or swapping $B_s$ and $B_t$ and applying the resulting schedule in reverse.

    First consider $b_W \in B_s$.
    We order all tiles in both $H_s \setminus H_t$ and $H_t \setminus H_s$ from west to east and north to south, see the left part of \cref{fig:hist-to-hist-greedy}.
    Based on this ordering, $S$ is created by iteratively moving the first remaining tile in $H_s \setminus H_t$ to the first remaining position in $H_t \setminus H_s$.
    This way, all moves can be realized on shortest paths over the bases.

    If $b_W \in B_t \setminus B_s$, this is not possible as the westernmost position in $H_t$ may not yet be reachable.
    Therefore, we iteratively extend $B_s$ to the west until the western part of $B_t$ is constructed.
    The tiles are taken from $H_s$ in the same order as before, see the right part of \cref{fig:hist-to-hist-greedy}.
    Once we have moved a tile to $b_W$, we can continue as above with $b_W \in B_s$.

	\begin{figure}[htb]
		\begin{subfigure}[t]{0.5\columnwidth}
			\centering%
			\includegraphics[page=1]{hist-to-hist-greedy-more-annotations}%
		\end{subfigure}%
		\begin{subfigure}[t]{0.5\columnwidth}
			\centering%
			\includegraphics[page=2]{hist-to-hist-greedy-more-annotations}%
		\end{subfigure}%
		\caption{%
            Position orderings in $H_s\setminus H_t$ and $H_t\setminus H_s$.
            Left: The main case of $b_W \in B_s$.
            Right: Additional steps are required for $b_W \in B_t \setminus B_s$.
            Tiles are moved according to the ordering, as indicated by the arrows.
        }
		\label{fig:hist-to-hist-greedy}
	\end{figure}

    Let $M$ be the matching induced by $S$. We now give an iterative argument why $M$ has the same sum of distances as an arbitrary MWPM $\minMatching$ with $\weight{\minMatching} = \lowerboundOf{H_s,H_t}$.
    First consider $b_W \in B_s$.
    Given the ordering from above, we write $s_i$ and $t_i$ for the $i$th tile position in $H_s \setminus H_t$ and $H_t \setminus H_s$, respectively.
    If $M \neq \minMatching$, there exists a minimal index~$i$ such that $(s_i, t_i) \in M \setminus \minMatching$.
    In $\minMatching$, $s_i$ and $t_i$ are instead matched to other positions $s_j$ and~$t_k$ with~$j, k > i$, i.e., $(s_i, t_k),(s_j, t_i)\in \minMatching$.
    Since $j, k > i$, the positions $s_j$ and $t_k$ do not lie to the west of $s_i$ and $t_i$, respectively.
    Now note that the paths for $(s_i, t_k)$ and $(s_j, t_i)$ always cross:
    Let $b_s$ and $b_t$ be the positions in $B$ that have same $x$-coordinates as $s_i$ and $t_i$, respectively.
    If $s_i$ lies to the east of $t_i$ (case~i), the path from $s_j$ to $t_i$ crosses~$b_s$, and so does any path that starts in $s_i$.
    If $s_i$ does not lie to the east of $t_i$ (case~ii), the path from $s_i$ to $t_k$ crosses $b_t$, and so does any path that ends in $t_i$.
    We illustrate both cases in \cref{fig:hist-to-hist-crossing}.

	\begin{figure}[htb]
		\begin{subfigure}[t]{0.5\columnwidth}
			\centering%
			\includegraphics[page=1]{hist-to-hist-crossing.pdf}%
		\end{subfigure}%
		\begin{subfigure}[t]{0.5\columnwidth}
			\centering%
			\includegraphics[page=2]{hist-to-hist-crossing.pdf}%
		\end{subfigure}%
		\caption{%
            The paths for $(s_i, t_k)$ and $(s_j, t_i)$ cross in $b_s$ or $b_t$.
            Left: $s_i$ lies to the east of $t_i$ (case~i).
            Right: $s_i$ does not lie to the east of $t_i$ (case~ii).
        }
		\label{fig:hist-to-hist-crossing}
	\end{figure}

    As the paths cross, replacing $(s_i, t_k), (s_j, t_i)$ with $(s_i, t_i), (s_j, t_k)$ in $\minMatching$ creates a new matching $\minMatching'$ with $\weight{\minMatching'} \leq \weight{\minMatching}$ due to \cref{obs:swapping}.
    By repeatedly applying this argument, we can turn $\minMatching$ into $M$ and note that $\weight{M} \leq \weight{\minMatching}$ holds.
    If $b_W \in B_t \setminus B_s$, case~(i) applies to all tiles moved to extend $B_s$ in western direction, and we can construct $M$ from $\minMatching$ analogously.
    Thus, the matching induced by $S$ is an MWPM.

    Since the carry distance $\cDist{S}$ is exactly $\lowerboundOf{H_s,H_t}$, it remains to bound the empty distance $\eDist{S}$ with $\BigO(n + \lowerboundOf{H_s,H_t})$.
    To this end, we combine all paths from one drop-off position back to its pickup position (in sum $\lowerboundOf{H_s,H_t}$) with all paths between consecutive pickup locations (in sum $\BigO(n)$ because the locations are ordered).
    Since the robot does not actually return to the pickup location, these paths of total length $\BigO(n + \lowerboundOf{H_s,H_t})$ are an upper bound for $\eDist{S}$, and we get $\makesp{S} \in \BigO(n + \lowerboundOf{H_s,H_t})$.
\end{proof}

\subsection{Correctness of the algorithm}
\label{sec:makespan}
In the previous sections, we presented schedules for each phase of the overall algorithm.
We~will now leverage these insights to prove the main result of this section, restated here.

\thmReconfigTwoScaled*
\begin{proof}
    By \cref{lem:turn-to-histo,lem:hist-to-hist}, the makespan of the three phases is bounded by~$\BigO(n + \lowerboundOf{C_s,H_s})$, $\BigO(n + \lowerboundOf{H_s,H_t})$, and $\BigO(n + \lowerboundOf{H_t,C_t})$, respectively, which we now bound by $\BigO(\lowerboundOf{C_s,C_t})$, proving asymptotic optimality for $C_s \rightrightarrows C_t$.

    Clearly, $n \in \BigO(\lowerboundOf{C_s,C_t})$, as each of the~$n$ tiles has to be moved due to $C_s\cap C_t=\varnothing$.
    We prove a tight bound on the remaining terms, i.e.,
    \begin{equation}
        \label{eq:lower_bound_sum}
        \lowerboundOf{C_s,H_s} + \lowerboundOf{H_s,H_t} + \lowerboundOf{H_t,C_t} = \lowerboundOf{C_s,C_t}.
    \end{equation}

    In Phase (I), tiles are moved exclusively towards the bounding box of~$C_t$ along shortest paths to obtain $H_s$, so $\lowerboundOf{C_s,C_t} = \lowerboundOf{C_s,H_s} + \lowerboundOf{H_s,C_t}$.
    The same applies to the reverse of Phase (III), i.e., $\lowerboundOf{C_t,H_s} = \lowerboundOf{C_t,H_t} + \lowerboundOf{H_t,H_s}$.
    Due to symmetry of the lower bound, \cref{eq:lower_bound_sum} immediately follows.
    The total makespan is thus in $\BigO(\lowerboundOf{C_s,C_t})=\BigO(\OPT)$.
\end{proof}

\Cref{lem:turn-to-histo,lem:hist-to-hist} establish schedules that minimize the carry distance and \cref{eq:lower_bound_sum} ensures that the combined paths remain shortest possible.
Thus, the provided schedule has optimal carry distance.
In~particular, we obtain an optimal schedule when $\lambda = 0$, which corresponds to the case where the robot incurs no cost for movement when not carrying~a~tile.

\begin{corollary}
    For any pair of $2$-scaled configurations $C_s,C_t\in \configs(n)$ with disjoint bounding boxes and $\lambda=0$, we can efficiently compute an optimal schedule for ${C_s\rightrightarrows C_t}$.
\end{corollary}


    \section{Constant-factor approximation for general instances}\label{sec:bounded-stretch}

The key advantage of $2$-scaled instances is the absence of cut vertices, which simplifies the maintenance of connectivity during the reconfiguration.
Therefore, the main challenge in reconfiguring general instances lies in managing cut vertices.

Most parts of our previous method already work independent of the configuration scale.
Thus, the only modification required concerns \cref{lem:move-down}, as the polyomino may become disconnected while moving \comps{} that are not $2$-scaled.
To address this, \cref{lem:one-scaled-move-down} introduces an alternative strategy for translating \comps{} by a single unit while maintaining connectivity throughout the process.
The key observation is that two auxiliary tiles can be used to preserve local connectivity around the cut vertices that need to be moved.
The use of auxiliary tiles to preserve connectivity is also exploited in other models~\cite{akitaya2021universal,michail2019transformation}.

To simplify the arguments, we assume that the robot is capable of holding up to two auxiliary tiles, but will later explain how to modify the strategy to work with no more than a single auxiliary tile.

\begin{lemma}
	\label{lem:one-scaled-move-down}
    Let the robot hold two auxiliary tiles.
    Given a \comp{}~$F$ on a polyomino~$P$,
	we can efficiently compute a schedule of makespan $\BigO(\size{F})$ to translate~$F$ in the target direction by one unit.
\end{lemma}

Before proving this, we give a high-level description of the approach. 
Without loss of generality, we translate $F$ in southern direction.
We decompose $F$ into maximal vertical \emph{strips} of unit width and maximal horizontal \emph{corridors} of unit height, that we collectively refer to as the \emph{elements} of $F$, as visualized in~\cref{fig:one-scaled-decomposition}.
In particular, strips that only consist of a single tile are grouped with adjacent single tiles (if they exist) and become maximal horizontal corridors.
\begin{figure}[htb]
	\begin{subfigure}[t]{0.33\columnwidth}
		\centering%
		\includegraphics[page=1]{decomposition-one-scaled-short}%
	\end{subfigure}%
	\begin{subfigure}[t]{0.33\columnwidth}
		\centering%
		\includegraphics[page=2]{decomposition-one-scaled-short}%
	\end{subfigure}%
	\begin{subfigure}[t]{0.33\columnwidth}
		\centering%
		\includegraphics[page=3]{decomposition-one-scaled-short}%
	\end{subfigure}%
	\caption{%
		A decomposition of a polyomino (left) into gray strips and cyan corridors~(center).
		From that, we create an adjacency graph $G$ and (by leaving out dotted edges) a spanning tree~$T$~(right).
	}
	\label{fig:one-scaled-decomposition}
\end{figure}
We will show that we are able to translate both types of elements one unit to the south, possibly with the support of auxiliary tiles, while retaining connectivity within the respective elements.
Because of the cut vertices, the most challenging part are corridors that consist of at least two tiles.
\Cref{fig:move-down-corridor} illustrates how two auxiliary tiles can be used to translate a corridor one unit south.
\begin{figure}[htb]
	\centering%
	\includegraphics[page=1]{move-down-one-scaled-conn}%
	\caption{The steps of moving a corridor of width 4 south, using two auxiliary tiles.}
	\label{fig:move-down-corridor}
\end{figure}

Strips can be translated south without the use of auxiliary tiles.
Because translating a single element may cause disconnection to adjacent elements, we apply a recursive strategy that first handles the elements \emph{blocking} our translation.
We then move our current element, and finally process all other adjacent elements.

\begin{proof}[Proof of~\cref{lem:one-scaled-move-down}]
	We refer to the aforementioned decomposition of a polyomino into strips and corridors.
	In this construction, each corridor is adjacent to up to two strips and each strip has a height of at least~2.
	Furthermore, all adjacency is either to the west or east side, but never vertical.
	Note that because $F$ is connected, the graph $G = (V, E)$ of adjacent elements is connected as well.

	We now compute an arbitrary spanning tree $T$ from $G$; and for the remainder of the proof, adjacency of elements is considered to be in $T$.
	For two adjacent elements $A, B \in V$, we say that $A$ \emph{blocks} $B$, if translating all tiles in $B$ south by one unit would result in all tiles of $A$ losing adjacency to $B$.
	An example for an element $A$ blocking and element $B$ is illustrated in \cref{fig:one-scaled-decomposition}: if we translate~$B$ south, we disconnect $A$ from the polyomino.
	Note that $A$ blocks $B$ exactly if the northernmost tile in $B$ has the same height as the southernmost tile in $A$.
	As at least one of $A$ or $B$ is a strip (with height at least~2), $A$ and $B$ cannot block one another simultaneously.
	Because we translate every tile south by only a single unit, we do not need to update~$T$,
	which means that if $A$ blocks $B$, then once $A$ has been moved south one step, $B$ can be moved south as well without $A$ and $B$ losing connectivity.
	If $A$ and $B$ are not blocking one another, they can be moved south in any order.
	
	We next look at how to move an element $A$ south in the case that $A$ is not blocked by any adjacent element.
	We use two auxiliary tiles which the robot holds at the beginning and end of the operation.
	If $A$ is a strip or a single-tile corridor, we place one tile south of $A$ and then remove the northernmost tile.
	If $A$ is a corridor of two or more tiles, we place the two auxiliary tiles right below the two westernmost tiles of $A$.
	We then move each tile by two steps to the east and one position south, starting from the east.
	The last two tiles become new auxiliary tiles; see \cref{fig:move-down-corridor}.
	Clearly, moving $A$ south has makespan $\BigO(\size{A})$.
		
	Finally, the robot has to traverse $T$ such that every element is moved south only after its adjacent blocking elements.
	This can be done recursively, starting at any element in $T$ in the following way.
	At the current element $A$:
	
	\begin{enumerate}
		\item Recurse on all adjacent elements that block $A$ and are not moved south yet.
		\item Move $A$ south.
		\item Recurse on the remaining unvisited elements adjacent to $A$.
	\end{enumerate}
	
	Because $T$ is connected, every element is visited with this strategy.
	By ordering the adjacent elements, each recursion on an element $A$ requires only $\BigO(\size{A})$ additional movements to move towards the corresponding elements.
	
	It remains to show that no element is recursed on more than once.
	Assume that some element $B$ is recursed on twice.
	As there are no cycles in $T$ and all unvisited or blocking neighbors are recursed on in each invocation, $B$ can only be recursed on for a second time by some element $A$ which was previously recursed on by $B$.
	This second invocation stems from the first step of handling $A$ because the third step excludes visited elements, which means that $B$ blocks $A$, and $B$ is not moved south.
	Thus, the first invocation on $B$ did not yet reach the second step.
	In other words, the recursive call on $A$ resulted from $A$ blocking $B$.
	This is a contradiction, as $A$ and $B$ cannot mutually block one another.
\end{proof}

It remains to argue how two auxiliary tiles can be obtained and emulated in our model.
By doing so, we obtain the following.

\begin{theorem}
	\label{thm:reconfig-general}
	There exists a constant $c$ such that for any pair of configurations $C_s,C_t\in \configs(n)$ with disjoint bounding boxes, we can efficiently compute a schedule for $C_s\rightrightarrows C_t$ with weighted makespan at most $c\cdot \OPT$.
\end{theorem}

\begin{proof}
	The idea is to execute \cref{lem:one-scaled-move-down} instead of \cref{lem:move-down} every time it is required in \cref{thm:reconfig-2-scaled}.
	To this end, we explain the required modifications for \cref{lem:turn-to-histo}; the same adaptations also work for $H_s \rightrightarrows H_s'$.
	Before we translate the first \comp{} $F$, we move two additional tiles to $F$.
	These tiles can be any tiles from $C_s$ that are not necessary for connectivity (e.g., any leaf from a spanning tree of the dual graph of $C_s$).
	Obtaining these auxiliary tiles requires $\BigO(n)$ movements.
	Next, we need to sequentially move both tiles from one \comp{} to the next whenever we continue our walk on the configuration.
	We can bound these steps with the length of the walk, thus requiring~$\BigO(n)$ movements in sum.

	Now, we only have to argue that \cref{lem:one-scaled-move-down} still holds if the robot initially carries a single tile and must not carry more than one tile at a time:
	Whenever the robot finishes translating the current element, it only picks up one of the two auxiliary tiles and moves it to the next unvisited element.
	This tile is placed in an adjacent position to the target direction of this element (which must be free).
	The robot then moves back and picks up the other auxiliary tile.
	As this extra movement requires~$\BigO(\size{F})$ additional moves in total, the makespan of \cref{lem:one-scaled-move-down} is still in~$\BigO(\size{F})$.

	With these adaptations, we can apply \cref{lem:one-scaled-move-down} instead of \cref{lem:move-down} in \cref{thm:reconfig-2-scaled}.
	Moreover, the makespan of \cref{lem:turn-to-histo} still lies in $\BigO(n + \lowerboundOf{C_s,H_s})$.
	By that, the analysis of the overall makespan is identical to \cref{thm:reconfig-2-scaled}.
	The only difference is that schedules created by \cref{lem:one-scaled-move-down} do not necessarily have optimal carry distance, which also applies to the overall schedule.
\end{proof}

    \section{Conclusion}
\label{sec:conclusion}
We demonstrate the \chat{} system to interactively build AI pipelines using \sys{} and \archytas{}.
Although our demo did not extensively cover physical optimization aspects, more details can be found in~\cite{palimpzestCIDR}.
The \chat{} interface offers a convenient tool for data practitioners to build complex data processing pipelines with little effort and a soft learning curve.
Our vision is that on the one hand, the future of data engineering will include more and more sophisticated frameworks to build complex applications that mix LLMs and traditional data processing.
On the other hand, tools like \chat{} will assist developers and make it easier to adopt new technologies and programming paradigms.
    \bibliography{bibliography}
\end{document}
