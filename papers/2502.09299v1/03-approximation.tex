\section{Constant-factor approximation for 2-scaled instances}\label{sec:bounded-approx}

We now turn to a special case of the optimization problem
in which the configurations have \emph{disjoint bounding boxes}, i.e., there exists an axis-parallel bisector that separates the configurations.
Without loss of generality, let this bisector be horizontal such that the target configuration lies south.
We present an algorithm that computes schedules of makespan at most $c\cdot \OPT$ for some fixed $c\geq 1$.

For the remainder of this section, we additionally impose the constraint that both the start and target configurations are \emph{$2$\nobreakdash-scaled}, i.e., they consist of $2 \times 2$-squares of tiles
aligned with a $2\times 2$ integer grid.
In the subsequent section, we extend our result to non-scaled~configurations.

\begin{restatable}{theorem}{thmReconfigTwoScaled}
    \label{thm:reconfig-2-scaled}
    There exists a constant $c$
    such that for any pair of $2$-scaled configurations $C_s,C_t\in \configs(n)$ with disjoint bounding boxes, we can efficiently compute a schedule for $C_s\rightrightarrows C_t$ with weighted makespan at most $c\cdot \OPT$.
\end{restatable}

Our algorithm utilizes a type of intermediate configuration called \emph{histogram}.
A~histogram consists of a \emph{base} strip of unit height and (multiple) orthogonal unit width \emph{columns} attached to its base.
The direction of its columns determines the orientation of a histogram, e.g., the histogram $H_s$ in \cref{fig:hist-to-hist} is \emph{north-facing}.

\begin{figure}[htb]
	\centering%
	\includegraphics[page=1]{hist-to-hist-overview}%
	\caption{
		An example for a start and target configuration $C_s, C_t$, the intermediate histograms $H_s, H_t$ with a shared baseline, and the horizontally translated $H_s'$ that shares a tile with $H_t$.
		If $H_s$ and $H_t$ overlap, we set $H_s' \coloneqq H_s$.
	}
	\label{fig:hist-to-hist}
\end{figure}

We proceed in three phases, see~\cref{fig:hist-to-hist} for an illustration.
\begin{description}
     \item[Phase (I).] {%
        Transform the configuration $C_s$ into a north-facing histogram $H_s$.
    }
    \item[Phase (II).] {%
        Translate $H_s$ to overlap with the target bounding box and transform it into a south-facing histogram $H_t$ contained in the bounding box~of~$C_t$.
    }
    \item[Phase (III).] Finally, apply Phase (I) in reverse to obtain $C_t$ from $H_t$.
\end{description}
We reduce this to two subroutines:
Transforming any $2$\nobreakdash-scaled configuration into a $2$\nobreakdash-scaled histogram and converting any two such histograms into one another.

\subsection{Preliminaries for the algorithm}
\label{subsec:preliminaries-algorithm}
Before investigating the algorithm in more detail, we give some useful definitions.
For two configurations $C_s, C_t \in \configs(n)$, the weighted bipartite graph $\bipGraph{C_s, C_t}=({C_s \cup C_t}, {C_s \times C_t}, L_1)$ assigns each edge a weight corresponding to the $L_1$-distance between its end points.

A \emph{perfect matching} $M$ in $\bipGraph{C_s, C_t}$ is a subset of edges such that every vertex is incident to exactly one of them;
its weight $\weight{M}$ is defined as the sum of its edge weights.
By definition, there exists at least one such matching in $\bipGraph{C_s, C_t}$.
Furthermore, a \emph{minimum weight perfect matching}~(MWPM) is a perfect matching $M$ of minimum weight $\lowerboundOf{C_s,C_t}=\weight{M}$, which is a natural lower bound on the necessary carry distance, and thus \OPT.

Let $S$ be any schedule for ${C_s\rightrightarrows C_t}$.
Then $S$ induces a perfect matching in~$\bipGraph{C_s, C_t}$, as it moves every tile of~$C_s$ to a distinct position of $C_t$.
We say that~$S$ has \emph{optimal carry distance} exactly if $\cDist{S}=\lowerboundOf{C_s,C_t}$.

\subsection{Phase (I): Transform a configuration into a histogram}
\label{subsec:building-a-histogram}

We proceed by constructing a histogram from an arbitrary $2$-scaled configuration by moving tiles strictly in one cardinal direction.

\begin{lemma}
    \label{lem:turn-to-histo}
    Let $C_s$ be a $2$-scaled polyomino and let $H_s$ be a histogram that can be created from $C_s$ by moving tiles in only one cardinal direction.
    We can efficiently compute a schedule of makespan $\BigO(n + \lowerboundOf{C_s,H_s})$ for $C_s \rightrightarrows H_s$ with optimal carry distance.
\end{lemma}

To achieve this, we iteratively move sets of tiles in the target direction by two units, until the histogram is constructed.
We give a high-level explanation of our approach by example of a north-facing histogram, as depicted in~\cref{fig:move-down}.

\begin{figure}[htb]
    \centering%
    \includegraphics[page=1]{create-histogram-shortened}%
    \caption{Visualizing \cref{lem:turn-to-histo}.
    Left: A walk across all tiles (red), the set $H$ (gray) and \comps{} (green).
    Right: Based on the walk, the \comps{} are iteratively moved south to reach a histogram shape.}
    \label{fig:move-down}
\end{figure}
Let $P$ be any intermediate $2$-scaled polyomino obtained by moving tiles south  while realizing ${C_s \rightrightarrows H_s}$.
Let $H$ be the set of maximal vertical strips of tiles that contain a base tile in $H_s$, i.e., all tiles that do not need to be moved further~south.
We define the \emph{\comps{}} of $P$ as the set of connected components in $P \setminus H$.
By definition, these exist exactly if $P$ is not equivalent to $H_s$, and once a tile becomes part of $H$, it is not moved again until $H_s$ is obtained.

We compute a \emph{walk} that covers the polyomino, i.e., a path that is allowed to traverse edges multiple times and visits each vertex at least once, by depth-first traversal on an arbitrary spanning tree of $P$.
The robot then continuously follows this walk, iteratively moving \comps south and updating its path accordingly:
Whenever it enters a \comp $F$ of $P$, we perform a subroutine (as we will describe and prove in~\cref{lem:move-down}) with makespan $\BigO(\size{F})$ that moves $F$ south by two units.
Adjusting the walk afterward may increase its length by $\BigO(\size{F})$ units per \mbox{\comp}.

Upon completion of this algorithm, we can bound both the total time spent performing the subroutine and the additional cost incurred by extending the walk by $\BigO(\lowerboundOf{C_s,H_s})$.
Taking into account the initial length of the walk, the resulting makespan is $\BigO(n+\lowerboundOf{C_s,H_s})$.

Before proving this result, we describe and prove the crucial subroutine for the translation of \comps, that can be stated as follows.

\begin{lemma}
    \label{lem:move-down}
    Given a \comp{} $F$ of a $2$-scaled polyomino $P$, we can efficiently compute a schedule of makespan $\BigO(\size{F})$ to translate~$F$ in the target direction by two units.
\end{lemma}

\begin{proof}
    Without loss of generality, we assume that the target direction is south
    and show how to translate $F$ south by one unit without losing connectivity.
    We then apply this~routine~twice.

    We follow a bounded-length walk across $F$
    that visits exclusively tiles with a tile neighbor in northern direction.
    Such a walk can be computed by depth-first traversal of $F$.
    Whenever the robot enters a maximal vertical strip of $F$ for the first time, it picks up the northernmost tile and places it at the first unoccupied position to its south.

    Clearly, the sum of movements on vertical strips for carrying tiles and returning to the pre-pickup position can be bounded with $\BigO(\size{F})$, as each strip is handled exactly once.
    This bound also holds for the length of the walk.
\end{proof}

By applying \cref{lem:move-down} on the whole polyomino $P$ instead of just a \comp, we can translate $P$
in any direction with asymptotically optimal makespan.

\begin{corollary}
    \label{cor:translation}
    Any $2$-scaled polyomino can be translated by $k$ units in any cardinal direction by a schedule of weighted makespan $\BigO(n\cdot k)$.
\end{corollary}

With this, we are equipped to prove the main result for the first phase.

\begin{proof}[Proof of~\cref{lem:turn-to-histo}]
	Without loss of generality, we assume that the target direction is south.

	We start by identifying a walk $W$ on $C_s$ that passes every edge at most twice, by depth-first traversal of a spanning tree of $C_s$.
	To construct our schedule, we now simply move the robot along~$W$.
	On this walk, we denote the $i$th encountered \comp{} with $F_i$ as follows.
	Whenever the robot enters a position~$t$ that belongs to a \comp{}~$F_i$, we move~$F_i$ south by two positions (refer to~\cref{lem:move-down}) and modify both~$t$ and~$W$ in accordance with the south movement that was just performed.
	The robot stops on the new position for~$t$ and we denote the next \comp{} by~$F_{i+1}$.
	We keep applying \cref{lem:move-down} until~$t$ does not belong to a \comp{}, from where on the robot continues on the modified walk~$W$.
	We~illustrate this approach in \cref{fig:move-down}.

	While it is clear that repeatedly moving all \comps{} south eventually creates $H_s$, it remains to show that this happens within $\BigO(n + \lowerboundOf{C_s,H_s})$ moves.
	To this end, we set $\sumcomps{} = \sum \size{F_i}$.
	We first show that the schedule has a makespan in $\BigO(n + \sumcomps{})$ and then argue that $\sumcomps{} \in \Theta(\lowerboundOf{C_s, H_s})$.
	Clearly, all applications of \cref{lem:move-down} result in a makespan of $\BigO(\sumcomps{})$, and since $\size{W} \in \BigO(n)$ we can also account for the number of steps on the initial walk~$W$.
	However, additional steps may be required:
	Whenever only \emph{parts} of $W$ are moved south by two positions, the distance between the involved consecutive positions in $W$ increases by two, see \cref{fig:walk-enlargement}.
	We account for these additional movements as follows.

	\begin{figure}[htb]
		\centering%
		\includegraphics[page=2]{walk-enlargement-1-scaled}%
		\caption{%
			When all tiles in a \comp{}~$F$ are moved south, the walk~$W$ on the polyomino may become longer.
		}
		\label{fig:walk-enlargement}
	\end{figure}

	For any \comp{}~$F_i$, there are at most $2\size{F_i}$ edges between $F_i$ and the rest of the configuration.
	Each transfer increases the movement cost by two and each edge is traversed at most twice, increasing the length of $W$ by at most $8\size{F_i}$ movements for every $F_i$.
	Consequently, $\BigO(n + \sumcomps{})$ also accounts for these movements.

	It remains to argue that $\sumcomps{} \in \Theta(\lowerboundOf{C_s, H_s})$.
	Clearly, as the schedule for moving $F_i$ two units south induces a matching with a weight in $\Theta(\size{F_i})$, combining all these schedules induces a matching with a weight in $\Theta(\sumcomps{})$.
	We next notice that all schedules for $C_s\rightrightarrows H_s$ that move tiles exclusively south have the same weight because the sum of edge weights in the induced matchings is merely the difference between the sums over all $y$\nobreakdash-positions in~$C_s$ and~$H_s$, respectively.
	Finally, all these matchings have minimal weight, i.e., $\lowerboundOf{C_s, H_s}$:
	If there exists an MWPM between $C_s$ and $H_s$ with crossing paths, these crossings can be removed by \cref{obs:swapping} without increasing the matching's weight.

	By that, $\sumcomps{} \in \Theta(\lowerboundOf{C_s, H_s})$ holds and the bound of $\BigO(n + \lowerboundOf{C_s,H_s})$ results.
	Also, since \cref{lem:move-down} moves all tiles exclusively south, the schedule has optimal carry distance.
\end{proof}
\subsection{Phase (II): Reconfigure a histogram into a histogram}\label{subsec:histogram-to-histogram}
\label{sec:hist-to-hist}

It remains to show how to transform one histogram into the other.
By the assumption of the existence of a horizontal bisector between the bounding boxes of~$C_s$ and~$C_t$, the histogram $H_s$ is north-facing, whereas $H_t$ is south-facing.
The bounding box of $C_s$ is vertically extended to share exactly one $y$-coordinate with the bounding box of $C_t$, and this is where both histogram bases are placed; see~\cref{fig:hist-to-hist} for illustration.
Note that the histograms may not yet overlap.
However, by \cref{cor:translation}, the tiles in $H_s$ can be moved toward $H_t$ in asymptotically optimal makespan until both histogram bases share a tile.

\begin{lemma}
    \label{lem:hist-to-hist}
    Let $H_s$ be a north-facing and $H_t$ a south-facing histogram that share at least one base tile.
    We can efficiently compute a schedule of makespan $\BigO(n + \lowerboundOf{H_s,H_t})$ for $H_s \rightrightarrows H_t$ with optimal carry distance.
\end{lemma}

To prove \cref{lem:hist-to-hist}, we use a simple observation about \emph{crossing paths}, i.e., paths that share a vertex in the integer grid.
We denote by $L_1(s,t)$ the length of a shortest path between the vertices $s$ and $t$.
\begin{observation}
    \label{obs:swapping}
    If any two shortest paths between tile positions $s_i$ and $t_i$ with $i\in \{1,2\}$ cross, then  $L_1(s_1,t_1) + L_1(s_2,t_2) \leq L_1(s_1,t_2) + L_1(s_2,t_1)$ holds.
\end{observation}
This follows from splitting both paths at the crossing vertex and applying triangle~inequality.

\begin{proof}[Proof of \cref{lem:hist-to-hist}]
    We first describe how the schedule $S$ can be constructed and
    then argue that the matching induced by $S$ is an MWPM in~$\bipGraph{H_s, H_t}$.
    We conclude the proof by showing that the schedule is of size $\BigO(n + \lowerboundOf{H_s,H_t})$.

    We write $B_s$ and $B_t$ for the set of all base positions in $H_s$ and $H_t$, respectively, and denote the union of all base positions by $B \coloneqq B_s \cup B_t$.
    We assume that all robot movements between positions in $H_s$ and $H_t$ are realized along a path that moves vertically until $B$ is reached, continues horizontally on $B$ and then moves vertically to the target position.
    These are shortest paths by construction.

    We denote the westernmost and easternmost position in $B$ by $b_W$ and $b_E$, respectively.
    Without loss of generality, we assume $b_E \in B_t$, because this condition can always be reached by either mirroring the instance horizontally or swapping $B_s$ and $B_t$ and applying the resulting schedule in reverse.

    First consider $b_W \in B_s$.
    We order all tiles in both $H_s \setminus H_t$ and $H_t \setminus H_s$ from west to east and north to south, see the left part of \cref{fig:hist-to-hist-greedy}.
    Based on this ordering, $S$ is created by iteratively moving the first remaining tile in $H_s \setminus H_t$ to the first remaining position in $H_t \setminus H_s$.
    This way, all moves can be realized on shortest paths over the bases.

    If $b_W \in B_t \setminus B_s$, this is not possible as the westernmost position in $H_t$ may not yet be reachable.
    Therefore, we iteratively extend $B_s$ to the west until the western part of $B_t$ is constructed.
    The tiles are taken from $H_s$ in the same order as before, see the right part of \cref{fig:hist-to-hist-greedy}.
    Once we have moved a tile to $b_W$, we can continue as above with $b_W \in B_s$.

	\begin{figure}[htb]
		\begin{subfigure}[t]{0.5\columnwidth}
			\centering%
			\includegraphics[page=1]{hist-to-hist-greedy-more-annotations}%
		\end{subfigure}%
		\begin{subfigure}[t]{0.5\columnwidth}
			\centering%
			\includegraphics[page=2]{hist-to-hist-greedy-more-annotations}%
		\end{subfigure}%
		\caption{%
            Position orderings in $H_s\setminus H_t$ and $H_t\setminus H_s$.
            Left: The main case of $b_W \in B_s$.
            Right: Additional steps are required for $b_W \in B_t \setminus B_s$.
            Tiles are moved according to the ordering, as indicated by the arrows.
        }
		\label{fig:hist-to-hist-greedy}
	\end{figure}

    Let $M$ be the matching induced by $S$. We now give an iterative argument why $M$ has the same sum of distances as an arbitrary MWPM $\minMatching$ with $\weight{\minMatching} = \lowerboundOf{H_s,H_t}$.
    First consider $b_W \in B_s$.
    Given the ordering from above, we write $s_i$ and $t_i$ for the $i$th tile position in $H_s \setminus H_t$ and $H_t \setminus H_s$, respectively.
    If $M \neq \minMatching$, there exists a minimal index~$i$ such that $(s_i, t_i) \in M \setminus \minMatching$.
    In $\minMatching$, $s_i$ and $t_i$ are instead matched to other positions $s_j$ and~$t_k$ with~$j, k > i$, i.e., $(s_i, t_k),(s_j, t_i)\in \minMatching$.
    Since $j, k > i$, the positions $s_j$ and $t_k$ do not lie to the west of $s_i$ and $t_i$, respectively.
    Now note that the paths for $(s_i, t_k)$ and $(s_j, t_i)$ always cross:
    Let $b_s$ and $b_t$ be the positions in $B$ that have same $x$-coordinates as $s_i$ and $t_i$, respectively.
    If $s_i$ lies to the east of $t_i$ (case~i), the path from $s_j$ to $t_i$ crosses~$b_s$, and so does any path that starts in $s_i$.
    If $s_i$ does not lie to the east of $t_i$ (case~ii), the path from $s_i$ to $t_k$ crosses $b_t$, and so does any path that ends in $t_i$.
    We illustrate both cases in \cref{fig:hist-to-hist-crossing}.

	\begin{figure}[htb]
		\begin{subfigure}[t]{0.5\columnwidth}
			\centering%
			\includegraphics[page=1]{hist-to-hist-crossing.pdf}%
		\end{subfigure}%
		\begin{subfigure}[t]{0.5\columnwidth}
			\centering%
			\includegraphics[page=2]{hist-to-hist-crossing.pdf}%
		\end{subfigure}%
		\caption{%
            The paths for $(s_i, t_k)$ and $(s_j, t_i)$ cross in $b_s$ or $b_t$.
            Left: $s_i$ lies to the east of $t_i$ (case~i).
            Right: $s_i$ does not lie to the east of $t_i$ (case~ii).
        }
		\label{fig:hist-to-hist-crossing}
	\end{figure}

    As the paths cross, replacing $(s_i, t_k), (s_j, t_i)$ with $(s_i, t_i), (s_j, t_k)$ in $\minMatching$ creates a new matching $\minMatching'$ with $\weight{\minMatching'} \leq \weight{\minMatching}$ due to \cref{obs:swapping}.
    By repeatedly applying this argument, we can turn $\minMatching$ into $M$ and note that $\weight{M} \leq \weight{\minMatching}$ holds.
    If $b_W \in B_t \setminus B_s$, case~(i) applies to all tiles moved to extend $B_s$ in western direction, and we can construct $M$ from $\minMatching$ analogously.
    Thus, the matching induced by $S$ is an MWPM.

    Since the carry distance $\cDist{S}$ is exactly $\lowerboundOf{H_s,H_t}$, it remains to bound the empty distance $\eDist{S}$ with $\BigO(n + \lowerboundOf{H_s,H_t})$.
    To this end, we combine all paths from one drop-off position back to its pickup position (in sum $\lowerboundOf{H_s,H_t}$) with all paths between consecutive pickup locations (in sum $\BigO(n)$ because the locations are ordered).
    Since the robot does not actually return to the pickup location, these paths of total length $\BigO(n + \lowerboundOf{H_s,H_t})$ are an upper bound for $\eDist{S}$, and we get $\makesp{S} \in \BigO(n + \lowerboundOf{H_s,H_t})$.
\end{proof}

\subsection{Correctness of the algorithm}
\label{sec:makespan}
In the previous sections, we presented schedules for each phase of the overall algorithm.
We~will now leverage these insights to prove the main result of this section, restated here.

\thmReconfigTwoScaled*
\begin{proof}
    By \cref{lem:turn-to-histo,lem:hist-to-hist}, the makespan of the three phases is bounded by~$\BigO(n + \lowerboundOf{C_s,H_s})$, $\BigO(n + \lowerboundOf{H_s,H_t})$, and $\BigO(n + \lowerboundOf{H_t,C_t})$, respectively, which we now bound by $\BigO(\lowerboundOf{C_s,C_t})$, proving asymptotic optimality for $C_s \rightrightarrows C_t$.

    Clearly, $n \in \BigO(\lowerboundOf{C_s,C_t})$, as each of the~$n$ tiles has to be moved due to $C_s\cap C_t=\varnothing$.
    We prove a tight bound on the remaining terms, i.e.,
    \begin{equation}
        \label{eq:lower_bound_sum}
        \lowerboundOf{C_s,H_s} + \lowerboundOf{H_s,H_t} + \lowerboundOf{H_t,C_t} = \lowerboundOf{C_s,C_t}.
    \end{equation}

    In Phase (I), tiles are moved exclusively towards the bounding box of~$C_t$ along shortest paths to obtain $H_s$, so $\lowerboundOf{C_s,C_t} = \lowerboundOf{C_s,H_s} + \lowerboundOf{H_s,C_t}$.
    The same applies to the reverse of Phase (III), i.e., $\lowerboundOf{C_t,H_s} = \lowerboundOf{C_t,H_t} + \lowerboundOf{H_t,H_s}$.
    Due to symmetry of the lower bound, \cref{eq:lower_bound_sum} immediately follows.
    The total makespan is thus in $\BigO(\lowerboundOf{C_s,C_t})=\BigO(\OPT)$.
\end{proof}

\Cref{lem:turn-to-histo,lem:hist-to-hist} establish schedules that minimize the carry distance and \cref{eq:lower_bound_sum} ensures that the combined paths remain shortest possible.
Thus, the provided schedule has optimal carry distance.
In~particular, we obtain an optimal schedule when $\lambda = 0$, which corresponds to the case where the robot incurs no cost for movement when not carrying~a~tile.

\begin{corollary}
    For any pair of $2$-scaled configurations $C_s,C_t\in \configs(n)$ with disjoint bounding boxes and $\lambda=0$, we can efficiently compute an optimal schedule for ${C_s\rightrightarrows C_t}$.
\end{corollary}

