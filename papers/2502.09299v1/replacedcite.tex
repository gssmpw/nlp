\section{Related work}
\label{subsec:related-work}

Garcia et al.____ showed that computing optimal schedules for reconfiguring polyominoes with \BILLE bots, see____ and \cref{fig:intro-bille}, is in general \NP-complete.
For tackling the problem, they designed heuristic approaches exploiting rapidly exploring random trees~(RRT), and a time-dependent variant of the A$^{*}$ search algorithm.
Furthermore, they used target swapping heuristics to reduce the overall traveled distance in case of multiple \BILLE bots.

A different context for reconfiguration arises from
programmable matter____.
Here, even finite automata are capable of building bounding boxes of tiles
around polyominoes, as well as scale and rotate them while maintaining
connectivity at all times____.

When considering active matter, i.e., the arrangement is composed of self-moving objects (or agents), multiple different models exist____.
For example, in the \emph{sliding cube model} (or the \emph{sliding square model} in two dimensions), first introduced by Fitch et al.____ in 2003, agents are allowed to make two different moves, namely, sliding along other agents, or performing a convex transition, i.e., moving around a convex corner.
Akitaya~et~al.____ show that universal sequential reconfiguration in two dimensions is possible, even while maintaining connectivity of all intermediate configurations, but minimizing the makespan of a schedule is \NP-complete.
Recently, Abel et al.____ and Kostitsyna~et~al.____ independently give similar results for the three dimensional setting.
Most recently, Akitaya~et~al.____ show that it is \NP-complete to decide whether there exists a schedule of makespan~$1$ in the parallel sliding square model and they provide a worst-case optimal algorithm.

In a slightly more relaxed model, Fekete et al.____ show that parallel connected reconfiguration of a swarm of (labeled) agents is \NP-complete, even for deciding whether there is a schedule of makespan $2$.
Complementary, they present algorithms for computing constant stretch schedules, i.e., the ratio between the makespan of a schedule and a natural lower bound (the maximum minimum distance between an individual start and target position) is bounded by a constant.