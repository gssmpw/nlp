\section{Computational complexity of the problem}\label{sec:computational-complexity}

In this section, we investigate the computational complexity of the decision variant of the generalized reconfiguration problem.
In particular, we prove that the problem is \NP-complete for any rational factor $\lambda\in[0,1]$.
This generalizes a result by Garcia et al.~\cite{cooperative-bille-reconfig-ICRA}, who demonstrate that the problem is \NP-complete for the special case of $\lambda=1$.

\begin{theorem}
	\label{thm:weighted-hardness}
	\probName is \NP-complete for any rational $\lambda$.
\end{theorem}

By considering the discrete steps of a schedule individually, we can confirm the moves' validity and certify a schedule's total cost.
It follows that the \probName problem is in \NP{} for any rational~$\lambda\in[0,1]$.
We distinguish between two cases of (1)~$\lambda \in (0,1]$ and (2) $\lambda = 0$ and give separate reductions from two \NP-complete problems.

For (1), we reduce from the \textsc{Hamiltonian path} problem in grid graphs~\cite{ItaiPS82}, which asks whether there exists a path in a given grid graph that visits each vertex exactly once.
We modify a construction previously used by Garcia~et~al.~\cite{cooperative-bille-reconfig-ICRA}, where the high-level idea is an expansion of the grid graph, placing small reconfiguration tasks at every vertex of the original graph.
An ``ideal'' schedule visits each vertex exactly once.
\begin{lemma}
	\label{lem:hardness-lambda-not-zero}
	\probName is \NP-complete for $\lambda\in (0,1]$.
\end{lemma}

\begin{proof}
	We reduce from the \textsc{Hamiltonian Path} problem in grid graphs~\cite{ItaiPS82}, which asks us to decide whether there exists a path in a given graph that visits each vertex exactly once.
	We use a modified variant of the construction described by Garcia~et~al.~\cite{cooperative-bille-reconfig-ICRA}, defining our start and target configurations based on the gadgets from~\cref{fig:hardness-bridges} as follows.

	\begin{figure}[htb]
		\begin{subfigure}[t]{0.27\textwidth-0.5em}
			\centering%
			\includegraphics[page=2]{hardness-bridges}%
			\caption{The vertex gadget.}%
			\label{fig:hardness-bridges-vertex}%
		\end{subfigure}%
		\hfill%
		\begin{subfigure}[t]{0.73\textwidth-0.5em}
			\centering%
			\includegraphics[page=1]{hardness-bridges}%
			\caption{An edge gadget, note the blue and red paths.}%
			\label{fig:hardness-bridges-edge}%
		\end{subfigure}%
		\caption{Our construction forces the robot to visit each vertex gadget at least once, traversing edge gadgets along either the red or the blue path.}%
		\label{fig:hardness-bridges}
	\end{figure}
	Let $k\gg \nicefrac{1}{\lambda}$.
	For each vertex of a given grid graph $G_{\mathit{in}}$, we create a copy of a $7\times 7$ \emph{vertex gadget} that represents a constant-size, locally solvable task, see~\cref{fig:hardness-bridges-vertex}.
	We connect all adjacent vertex gadgets with copies of a $(k+1)\times (4k+3)$ \emph{edge gadget}, as depicted in~\cref{fig:hardness-bridges-edge}.
	While we reuse the exact vertex gadget from~\cite{cooperative-bille-reconfig-ICRA}, our edge gadget is a little more complex.
	A robot traveling through the gadget has two options, indicated by red and blue paths in~\cref{fig:hardness-bridges-edge}.

	The blue path offers a simple, walkable option of $6k+2$ empty travel units, while the red path is only available if the robot performs $2$ carry steps to construct and deconstruct a unit length bridge at the center, temporarily creating a walkable path of $4k+2$ units.
	This is cheaper exactly if $\lambda(6k+2)>\lambda(4k+2)+2$,
	i.e., if $k > \nicefrac{1}{\lambda}$.
	Since $\lambda$ is a rational number, we can pick a value $k$ that is polynomial in $\lambda$ and permits a construction in which taking the red path is always cheaper.
	The blue path thus serves solely for the purpose of providing connectivity and will never be taken by the robot.

	An optimal schedule will therefore select a minimal number of edge gadgets to cross, always taking the red path.
	As vertex gadgets always require the same number of moves to solve, we simply capture the total cost for a single vertex gadget as $t_g$.
	For a graph $G_{\mathit{in}}$ with~$m$ vertices, an optimal schedule has makespan
	$m\cdot t_g + (m-1)\cdot (4k\lambda+2\lambda+2)$
	exactly if the underlying graph is Hamiltonian, and greater otherwise.

	Given a connected grid graph, such a schedule never builds bridges of length greater than one:
	Due to their pairwise distance, this will never be cheaper than crossing an edge gadget along the red path, see also~\cite{cooperative-bille-reconfig-ICRA}.
\end{proof}

This has the following implication.

\begin{corollary}
	Given two configurations $C_s,C_t\in\configs(n)$ and an integer $k\in\mathbb{N}$, it is \NP-complete to decide whether there exists an optimal schedule $C_s\rightrightarrows C_t$ with at most $k$ pickup (and at most $k$ drop-off) operations.
\end{corollary}

The reduction for (2) is significantly more involved: As $\lambda = 0$, the robot is effectively allowed to ``teleport'' across the configuration.
We reduce from \textsc{Planar Monotone~3Sat}~\cite{dbk-obspp-10}, utilizing and adapting the reduction given by Akitaya~et~al.~\cite{AkitayaDKKPSSUW22} for the sequential sliding square problem.
\begin{lemma}
	\label{lem:hardness-lambda-zero}
	\probName is \NP-complete for $\lambda = 0$.
\end{lemma}
We reduce from \textsc{Planar Monotone 3Sat}, an \NP-complete variant of the \textsc{Sat} problem, which asks whether a given Boolean formula $\varphi$ in conjunctive normal form is satisfiable, given the following properties:
First, each clause consists of at most~$3$~literals, all either positive or negative.
Second, the clause-variable incidence graph $G_\varphi$ admits a plane drawing in which variables are mapped to the $x$-axis, and all positive (resp., negative) clauses are located in the same half-plane, such that edges do not cross the $x$-axis, see~\cref{fig:hardness-teleportation-graph}.

\begin{figure}[htb]
	\centering%
	\includegraphics[page=1]{hardness-teleportation2x}%
	\caption{A rectilinear embedding of the clause-variable incidence graph of the formula ${\varphi= (x_1\lor x_3\lor x_4)\land (x_1\lor x_3)\land (\overline{x_2}\lor\overline{x_3}\lor\overline{x_4})}$.}%
	\label{fig:hardness-teleportation-graph}%
\end{figure}

We assume, without loss of generality, that each clause contains exactly three literals.
Otherwise, we can extend a shorter clause with a redundant copy of one of the existing literals, e.g., $(x_1\lor x_4)$ becomes $(x_1\lor x_4\lor x_4)$ in~\cref{fig:hardness-teleportation-graph}.

Our reduction maps from an instance $\varphi$ of \textsc{Planar Monotone 3Sat} to an instance~$\mathcal{I}_\varphi$ of \probName such that the minimal feasible makespan~$\mathcal{I}_\varphi$ is determined by whether $\varphi$ is satisfiable.
Recall that, due to $\lambda=0$, we only account for carry distance.
Consider an embedding of the clause-variable incidence graph  $G_\varphi$ as above, where $C$ and $V$ refer to the $m$ clauses and $n$ variables of $\varphi$, respectively.
We construct $\mathcal{I}_\varphi$ as follows.

A \emph{variable gadget} is placed on the $x$-axis for each $x_i\in V$, with gadgets connected along the axis in a straight line.
Intuitively, the variable gadget asks the robot to move a specific tile west along one of two feasible paths.
\begin{figure}[htb]
	\centering%
	\includegraphics[page=2]{hardness-teleportation2x}%
	\caption{The variable gadget. The middle segments are repeated to match the incident clauses. Note the positive and negative paths (blue and red).}%
	\label{fig:hardness-teleportation-variable}%
\end{figure}
These paths are highlighted in red and blue in~\cref{fig:hardness-teleportation-variable}, each of length exactly $9(\delta(x_i)+1)$, where $\delta(x_i)$ refers to the degree of $x_i$ in the clause-variable incidence graph $G_\varphi$.
Both paths require the robot to place its payload into the highlighted gaps to minimize the makespan, stepping over it before picking up again.

Connected to these variable gadgets are \emph{clause gadgets}, exactly one per clause, as illustrated in~\cref{fig:hardness-teleportation-clause}.
\begin{figure}[h!tb]%
	\centering%
	\includegraphics[page=3]{hardness-teleportation2x}%
	\caption{A clause gadget and its attachment to a variable.}%
	\label{fig:hardness-teleportation-clause}%
\end{figure}%
Each clause gadget is composed of a large curve that resembles an upside-down ``L'' attached to the $x$-maximal variable gadget of its literals, with three vertical \emph{prongs} extending vertically towards variables.
The prongs terminate at positions diagonally adjacent to the variable gadget of a corresponding literal.
The tile at the $90^\circ$ bend in said ``L''-shape must be moved by one unit to the south and west, respectively.
Due to the connectivity constraint, this cannot be realized without first establishing a secondary connection between the prongs and one of the variable gadgets.

\begin{proof}[Proof of~\cref{lem:hardness-lambda-zero}]
	The following refers to the aforementioned construction.

	Consider a Boolean formula $\varphi$ for the \textsc{Planar Monotone 3Sat} with $n$ variables and $m$ clauses.
	Recall that each clause can be assumed to contain exactly three literals, i.e, $\sum_{i=1}^{n}\delta(x_i) = 3m$.
	We show that $\varphi$ is satisfiable exactly if there exists a schedule for the constructed instance $\mathcal{I}_\varphi = (C_\varphi, C_\varphi')$ of the reconfiguration problem that achieves a weighted makespan of
	\begin{equation}
		\label{eq:sat-hardness-makespan}
		2m + \sum_{i=1}^{n}9(\delta(x_i)+1)\quad =\quad 2m + 9(3m+n)\quad =\quad 29m+9n.
	\end{equation}
	\begin{claim}
		\label{clm:hardness-teleportation-positive}
		There exists a schedule for $\mathcal{I}_\varphi$ with weighted makespan $29m+9n$ if the Boolean formula $\varphi$ is satisfiable.
	\end{claim}
	\begin{claimproof}
		Consider a satisfying assignment $\alpha(\varphi)$ for the Boolean formula $\varphi$ and let~$\alpha(x_i) \in\{{x_i},\overline{x_i}\}$ represent the literal of $x_i$ with a positive Boolean value, i.e., the literal such that $\alpha(x_i)=\texttt{true}$.
		Using only $\alpha(\varphi)$ and the two paths illustrated in~\cref{fig:hardness-teleportation-variable}, we can construct a schedule for $\mathcal{I}_\varphi$ as follows.

		We follow the path corresponding to~$\alpha(x_i)$, picking up the tile and placing it at the first unoccupied position.
		If this position is adjacent to a clause gadget's prong, this corresponds to the pickup/drop-off sequence $(a)$ in \cref{fig:hardness-teleportation-variable-example}.
		\begin{figure}[htb]
			\centering%
			\includegraphics[page=4]{hardness-teleportation2x}%
			\caption{The variable gadget for $x_3$ as seen in~\cref{fig:hardness-teleportation-graph} and a schedule with makespan $29=9(\delta(x_3)+1) + 2$ that solves an adjacent clause gadget.}%
			\label{fig:hardness-teleportation-variable-example}%
		\end{figure}
		With the marked cell occupied, there then exists a cycle containing the tile that is to be moved in the clause gadget, allowing us to solve the clause gadget in just two steps, marked $(b)$ in \cref{fig:hardness-teleportation-variable-example}.
		We can repeat this process, moving the tile to the next unoccupied position on its path until it reaches its destination.

		This pattern takes makespan exactly $9(\delta(x_i)+1)$ per variable gadget; $9$ for each incident clause and $9$ as the base cost of traveling north and south.

		As $\alpha(\varphi)$ is a satisfying assignment for $\varphi$, every clause of $\varphi$ contains at least one literal $v$ such that $\alpha(v)=\texttt{true}$.
		It follows that every clause gadget can be solved in exactly $2$ steps by the above method, resulting in the exact weighted makespan outlined in~\cref{eq:sat-hardness-makespan}, i.e., $29m+9n$ for $n$ variables and $m$ clauses.
	\end{claimproof}
	\begin{claim}
		There does not exist a schedule for $\mathcal{I}_\varphi$ with weighted makespan $29m+9n$ if the Boolean formula $\varphi$ is not satisfiable.
	\end{claim}
	\begin{claimproof}
		It is easy to observe that there is no schedule that solves a variable gadget for $x_i$ in makespan less than ${9(\delta(x_i)+1)}$:
		Building a bridge structure horizontally along the variable gadget takes a quadratic number of moves in the variable gadget's width, and shorter bridges fail to shortcut the north-south distance that the robot would otherwise cover while carrying the tile.
		Furthermore, we can confine tiles to their respective variable gadgets by spacing them appropriately.

		Even if $\varphi$ is not satisfiable, each variable gadget can clearly be solved in a makespan of ${9(\delta(x_i)+1)}$.
		However, this no longer guarantees the existence of a cycle in every clause gadget at some point in time, leaving us to solve at least one clause gadget in a more expensive manner.
		In particular, we incur an extra cost of at least $4$ units per unsatisfied clause, as a cycle containing the cyan tile must be created in each clause gadget before the tile can be safely picked up.

		Assuming that $m'\in (1,m]$ is the minimum number of clauses that remain unsatisfied in $\varphi$ for any assignment of $x_i$, we obtain a total makespan of
		\begin{equation}
			\label{eq:hardness-teleportation-greedy}
			2m+4m'+\sum_{i=1}^{n}9(\delta(x_i)+1)\quad =\quad 4m'+29m+9n\quad >\quad 29m+9n.
		\end{equation}
		We conclude that a weighted makespan of $29m+9n$ is not achievable and must be exceeded by at least four units if the Boolean formula $\varphi$ is not satisfiable.
	\end{claimproof}
	This concludes our proof of \NP-completeness for $\lambda=0$.
\end{proof}

Our central complexity result \cref{thm:weighted-hardness} is simply the union of \cref{lem:hardness-lambda-not-zero,lem:hardness-lambda-zero}.

