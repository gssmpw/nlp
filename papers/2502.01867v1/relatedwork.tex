\section{Related Work}
Our paper is primarily related to four streams of literature: CTR prediction, item's cold start, multi armed bandit (MAB), position based model (PBM).

\textit{CTR prediction.} Click-through rate (CTR) prediction is one of the most central tasks in online advertising systems. Recent deep learning-based models that exploit feature embedding and high-order data have shown dramatic successes in CTR prediction. These approaches include Deep Interest Network (DIN)\cite{zhou2018deep}, Deep\&Cross Network  (DCN) \cite{wang2017deep}, Wide\&Deep \cite{cheng2016wide}, and DeepFM \cite{guo2017deepfm}. These models have shown dramatic successes in CTR prediction by automatically learning latent feature representations and complex interactions between features in different ways. However, these models work poorly on cold-start ads with new IDs, whose embeddings are not well learned yet.

\textit{Cold start.} The majority of existing solutions to the item cold-start problem rely on a content-based approach, leveraging the characteristics of new items to discover analogous user preferences and recommending these items accordingly. Various techniques \cite{agarwal2009regression, koren2008factorization} have been developed to adapt matrix factorization (MF) methods for cold-start scenarios by incorporating item-specific attributes, such as item's descriptions and contents, into the model. These extensions generate vector embeddings that can be compared with user representations from a lookup table to provide personalized recommendations. As a complementary approach to collaborative filtering, content-based filtering \cite{lops2011content} offers a distinct strategy for recommending new items. By analyzing the attributes and characteristics of each item and comparing them with those of items previously interacted with by the user, this method can identify relevant similarities and provide personalized recommendations. The integration of collaborative filtering and content-based filtering techniques has given rise to hybrid systems \cite{wei2016collaborative}, which are designed to enhance the accuracy and reliability of click-through rate (CTR) predictions. By combining the strengths of both methods, these hybrid systems can provide more accurate and robust predictions. This synergy also enables them to effectively address cold-start challenge. Furthermore, researchers  \cite{vartak2017meta, zheng2021cold} have employed meta-learning approaches \cite{vilalta2002perspective} that enable the exploration of prior knowledge across various tasks and facilitate the development of reasonable initial parameters for cold items. The paper \cite{panda2022approaches} undertakes a comprehensive systematic literature review of research efforts between 2010 and 2021 to address click-through rate prediction and cold start problems. The review synthesizes a diverse range of approach-driven strategies, including deep learning, matrix factorization, hybrid methods, and innovative techniques in collaborative filtering and content-based algorithms, providing a better understanding of the state-of-the-art solutions.

\textit{MAB.} In contrast to existing approaches, our method operates within a multi-armed bandit (MAB) framework \cite{auer2002finite, slivkins2019introduction}, which enables incremental feedback without requiring additional data or using any neural network architecture. The success of MAB algorithms relies heavily on their ability to balance exploration and exploitation in their action selection policies. If an agent prioritizes exploration, it may overlook valuable insights by randomly choosing new actions without considering the knowledge gained from previous steps. Conversely, if an agent focuses solely on exploitation, it will favor short-term rewards over long-term benefits, neglecting optimal solutions.

To address this critical trade-off, researchers have developed a range of MAB algorithms with unique properties and strengths. Some of the most well-established methods include $\epsilon$-Greedy \cite{auer2002finite}, which strikes a balance between exploration and exploitation by selecting actions randomly with probability $\epsilon$. Upper Confidence Bounds (UCB) \cite{auer2002using} uses confidence bounds to determine which action to take, balancing exploration and exploitation based on the uncertainty associated with each option. Thompson Sampling (TS) \cite{chapelle2011empirical} is a probabilistic approach that takes into account both prior knowledge and observed data to make informed decisions.

\textit{PBM} Research in web search relies on understanding user behavior. For example, \cite{joachims2017accurately} demonstrated that users consistently favor top-ranked items over lower-ranked ones on a search engine results page (SERP), with a clear preference for the first item listed. The study \cite{chuklin2022click} presents different basic click models that capture different assumptions about searchers' interaction behavior on a web search engine result page, including the PBM (positional-based model). The PBM posits that users click an item only if they have viewed the item and drawn by its relevance. This model is also grounded in the intuitive notion that the probability of a user examining an item depends heavily on its position on a SERP, typically decreasing with rank.

Our work is closely related to several recent studies \cite{zhou2023bandit, lagree2016multiple, feng2023improved} on online advertising auctions. Specifically,  \cite{feng2023improved} addresses CTR estimation for single-slot Vickrey-Clarke-Groves auctions, while we extend this approach to consider multiple slots. Although \cite{lagree2016multiple, zhou2023bandit} also consider scenarios with multiple slots, they utilize a position-based model(PBM) and neglect the complexities associated with pay-per-click auctions. Our research aims to address these gaps by developing \textbf{AuctionUCB-PBM}, a UCB-like algorithm under MAB setting for PBM, specifically tailored to auction pay-per-click systems.