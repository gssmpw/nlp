\section{Related Works}
In 2022, over 14 thousand complaints were made against the companies due to their inferior commitment to resolving performance. Subsequently, SEBI tried to resolve 2838 complaints through platform score in March  2023\cite{Complaints}.
The intricate nature of complaints can lead to ambiguity in representing reviews \cite{jenkins1979levels,radford2019language}. At times, the indication of a complaint is explicit, clearly identifying the party at fault. However, in other instances, the complaint's nature is implicit, lacking explicit blame attribution \cite{reiter2023global}. 

\begin{table}[h]
\centering
\setlength{\tabcolsep}{1pt}
% \renewcommand{\arraystretch}{1}
\caption{Comparison of complaint datasets and their associated labels}
\label{datset-comp}
% \resizebox{0.40\textwidth}{!}{
\scalebox{0.57}{

\begin{tabular}{c|c|ccccc|c}
\hline
\multirow{2}{*}{\textbf{Domains}} & \multirow{2}{*}{\textbf{Datasets}} & \multicolumn{5}{c|}{\textbf{Labels}} & \textbf{Multimodality} \\ \cline{3-8} 
  && \multicolumn{1}{c|}{\textbf{Complaint}} & \multicolumn{1}{c|}{\textbf{Severity}} & \multicolumn{1}{c|}{\textbf{Sentiment}} & \multicolumn{1}{c|}{\textbf{Cause}} & \textbf{Aspect} & \textbf{Text/Image/Video} \\ \hline
\multirow{4}{*}{Others}& Complaints \cite{DBLP:conf/acl/Preotiuc-Pietro19a} & \multicolumn{1}{c|}{\checkmark}& \multicolumn{1}{c|}{$\times$}& \multicolumn{1}{c|}{$\times$}& \multicolumn{1}{c|}{$\times$}  & $\times$  & Text  \\ \cline{2-8} 
  & Complaints Severity\cite{jin2021modeling}& \multicolumn{1}{c|}{\checkmark}& \multicolumn{1}{c|}{\checkmark}& \multicolumn{1}{c|}{$\times$}& \multicolumn{1}{c|}{$\times$}  & $\times$  & Text  \\ \cline{2-8} 
  & Complaint ESS \cite{singh2022adversarial}  & \multicolumn{1}{c|}{\checkmark}& \multicolumn{1}{c|}{\checkmark}& \multicolumn{1}{c|}{\checkmark}& \multicolumn{1}{c|}{\checkmark}  & $\times$  & Text  \\ \cline{2-8} 
  & CESAMARD \cite{singh2023knowing} & \multicolumn{1}{c|}{\checkmark}& \multicolumn{1}{c|}{\checkmark}& \multicolumn{1}{c|}{\checkmark}& \multicolumn{1}{c|}{\checkmark}  & \checkmark  & Text+Image\\ \hline
\multirow{2}{*}{Finance}  & X-FinCORP \cite{das2023let}  & \multicolumn{1}{c|}{\checkmark}& \multicolumn{1}{c|}{\checkmark}& \multicolumn{1}{c|}{\checkmark}& \multicolumn{1}{c|}{\checkmark}  & $\times$  & Text  \\ \cline{2-8} 
  & \textit{MulComp}(Proposed)  & \multicolumn{1}{c|}{\checkmark}& \multicolumn{1}{c|}{\checkmark}& \multicolumn{1}{c|}{\checkmark}& \multicolumn{1}{c|}{\checkmark}  & \checkmark  & Video  \\ \hline
\end{tabular}
}
\end{table}

In the field of pragmatics, the work by \cite{olshtain198710} pioneered the classification of complaints into five distinct types: a) Beyond Reproach, b) Explicit Complaint, c) Statement of Disapproval, d) Warning, and e) Allegation. However, Trosborg et al.'s foundational study \cite{trosborg2011interlanguage} identified four primary levels of complaint severity: a) Implicit Reproach Absent, b) Disapproval, c) Accusation, and d) Blame. Gradually, Transformers-based Complaint categorization based on their severity criteria was developed by  \cite{jin2021modeling}. Nonetheless, companies must categorize financial complaints according to their severity and offer comprehensive explanations \cite{10379488}. Subsequently, in 2023, the influence of generative AI was illustrated in the domain of financial unimodal complaint identification, incorporating the provision of causal explanations \cite{das2023let}. A research endeavour was showcased focusing on binary complaint classification in a multimodal information setting, devoid of considering the financial aspect context \cite{singh2022sentiment}. The research presented in \cite{poria2018meld,saha2021towards} has connected vision and language within the associated field of polarity and emotion recognition. The recent multimodal predictive classifier, CMA-CLIP \cite{liu2021cma}, has garnered attention for its ability to predict attributes for predefined classes. However, it faces limitations in handling unseen data. In contrast, the generative model developed by Roy et al.\cite{roy2021attribute} approaches the classification task as a generation task. 
In multilabel classification, the co-occurrence of labels can be effectively modeled using probabilistic graph models \cite{li2016conditional}. However, to address the computational burden of these models, neural network-based solutions have gained prevalence. For instance, Wang et al. \cite{wang2016cnn} used recurrent networks to encode labels into embedding vectors for label correlation modeling, while Lin et al. \cite{lin2018nextvlad} utilized a context-gating strategy to integrate label re-ranking into the network architecture. Additionally, attention mechanisms have been leveraged to model label relationships.\\
\textbf{Our research solution standing:}
Past research explored task-specific layers for complaint detection through fine-tuning pre-trained language models and different attention mechanisms to model multiple-label representations. As of our present understanding, an ultimate comprehensive tool that offers a generalized solution for aspect-based financial multimodal complaints, along with an accompanying dataset, remains absent. We aim to bridge the research gap by presenting our novel model and dataset.\\