\documentclass[conference]{IEEEtran}
\IEEEoverridecommandlockouts
% The preceding line is only needed to identify funding in the first footnote. If that is unneeded, please comment it out.
\usepackage{cite}
\usepackage{amsmath,amssymb,amsfonts}
\usepackage{algorithmic}
\usepackage{graphicx}
\usepackage{textcomp}
\usepackage{xcolor}
\usepackage{url}
\usepackage[numbers]{natbib}
\usepackage{hyperref}
\usepackage{comment}
\usepackage{placeins}
\usepackage{float}
\usepackage{enumerate}

\def\BibTeX{{\rm B\kern-.05em{\sc i\kern-.025em b}\kern-.08em
    T\kern-.1667em\lower.7ex\hbox{E}\kern-.125emX}}
\begin{document}

\title{Statistical data analysis for Tourism in Poland %and Bat Species 
in R Programming Environment\\
%{\footnotesize \textsuperscript{*}Note: Sub-titles are not captured in Xplore and should not be used}\thanks{Identify applicable funding agency here. If none, delete this.}
}

\author{\IEEEauthorblockN{%1\textsuperscript{st} 
Saad Ahmed Jamal}
\IEEEauthorblockA{\textit{MED-IIFA} \\
%\textit{name of organization (of Aff.)}\\
Universidade de Evora, Portugal \\
% https://orcid.org/0000-0002-4256-0298
saad.jamal@uevora.pt}
%\and
%\IEEEauthorblockN{2\textsuperscript{nd} Given Name Surname}
%\IEEEauthorblockA{\textit{dept. name of organization (of Aff.)} \\
%\textit{name of organization (of Aff.)}\\
%City, Country \\
%email address or ORCID}
%\and
%\IEEEauthorblockN{3\textsuperscript{rd} Given Name Surname}
%\IEEEauthorblockA{\textit{dept. name of organization (of Aff.)} \\
%\textit{name of organization (of Aff.)}\\
%City, Country \\
%email address or ORCID}
}

\maketitle

% \begin{abstract}
% This research utilises R Programming Language for Statistical Data Analysis for two different case studies.
% In particular it focuses of methods of visualization, multivariate statistics and hypothesis testing.  
% Spending patterns, correlation and association among the variables were analysed for the tourism data of Poland.
% The same approached was used to analyse was the data of bat's species. The results proved to have significant relationships among a few variables. However, ANOVA could not be applied as the assumptions proven to be not true in both cases. 
% A description on programming R environment and applied libraries.
% The document contributes in 
% gaining various statistical analysis methods in R environment. The developed code is available on through Gihub repository \url{https://github.com/SaadAhmedJamal/DataAnalysis_RProgEnv}
% \end{abstract}



\begin{abstract}
This study utilises the R programming language for statistical data analysis %across two distinct case studies
to understand Tourism dynamics in Poland. %, categorical relationships among the variables.
It focuses on methods for data visualisation, multivariate statistics, and hypothesis testing. To investigate the expenditure behavior of tourist, spending patterns, correlations, and associations among variables were analysed in the dataset. %to understand expenditure behavior in Poland
%to investigate the expenditure behavior of tourist in Poland. %The same approach was applied to bat species data to investigate morphological variations. 
The results revealed a significant relationship between accommodation type and the purpose of trip,  showing that the purpose of a trip impacts the selection of accommodation.
%
A strong correlation was observed between organizer expenditure and private expenditure, indicating that individual spending are more when the spending on organizing the trip are higher.
% 
However, no significant difference was observed in total expenditure across different accommodation types and purpose of the trip  
revealing that travelers tend to spend similar amounts regardless of their reason for travel or choice of accommodation.  
%
Although significant relationships were observed among certain variables, ANOVA could not be applied because the dataset was not able to hold on the normality assumption. In future, 
the dataset can be explored further to find more meaningful insights.
%Additionally, 
\begin{comment}
This research provides an overview of the R programming environment and  %The study contributes to the understanding of %various
%statistical analysis methods in R. 
it contributes to the comprehension of statistical analysis methodologies and libraries in R. 
\end{comment}
The developed code is available on GitHub: \url{https://github.com/SaadAhmedJamal/DataAnalysis_RProgEnv}.
\end{abstract}


\begin{IEEEkeywords}
R\_programming, analysis, ANOVA 
\end{IEEEkeywords}

\section{Introduction}

R is a high-level programming language widely used in statistical computing, data analysis, and geospatial applications. It was developed by \citet{0_ihaka1996r} at the University of Auckland and has since become a leading tool for statistical modeling and visualisation. The latest stable version, R ‘4.4.1,’ used in this study, is freely available at \url{https://cran.r-project.org/}. Its extensive package ecosystem allows researchers to perform efficient data wrangling, hypothesis testing, and visualization \citep{wickham2016getting, 4_wickham2017package}.  
There was a debate if R is any better and whether is showed be used for teaching purposes  \citep{1_gomes2018teaching}. A comprehensive learning tutorials were developed which made learning in R easier \citep{1b_de2018teaching}. With time, R has evolved be as important as Python in the field of Data Science.
R for statistical computing and data visualisation due to its flexibility and robust package ecosystem \citep{0_ihaka1996r}. The ability to perform exploratory data analysis (EDA), hypothesis testing, and regression modeling makes R an essential tool for empirical research \citep{wickham2016getting}.

Statistical data analysis plays a crucial role in tourism research and ecological studies, where identifying patterns, trends, and relationships in datasets can drive better decision-making \citep{ding2018analysis, davis2023writing}. For instance, in tourism analytics, understanding expenditure patterns, accommodation choices, and trip purposes enables businesses and policymakers to optimize tourism management strategies. Similarly, in ecological research, morphometric analyses—such as assessing the relationship between forearm length and body weight in bats—help scientists study species adaptation and environmental impacts \citep{5_harrell2019package}. Further these assessment can be used in regression and predictive analysis %\citep{} 
\citep{nusrat2024multiclassdepressiondetection}.
%Statistical analysis plays a crucial role in tourism and ecological research, providing data-driven insights into expenditure trends, behavioral patterns, and biological adaptations. Researchers commonly use 


Tourism data analysis has been widely applied in economic studies to assess travel expenditure, seasonal trends, and consumer behavior \citep{ding2018analysis}. Statistical models %such as correlation analysis, ANOVA, and regression models 
provide insights into the factors influencing tourist spending patterns, allowing policymakers and businesses to optimize tourism planning and resource allocation.
%In tourism research, various statistical methods have been applied to analyze spending behaviors and travel patterns. 
\citet{ding2018analysis} examined domestic tourism consumption in China and found that economic indicators such as GDP and disposable income significantly influenced tourism expenditure. Statistical models such as ANOVA and t-tests have been widely used to assess variations among different categories of travelers, such as business versus leisure tourists \citep{milenkovski2019statistical}. Additionally, multivariate statistical techniques, including time-series models and clustering algorithms, have been employed to predict seasonal fluctuations in travel demand, assisting policymakers in optimizing tourism strategies.

Ecological research has extensively utilized statistical methods to analyse morphometric traits and their ecological significance. Studies indicate that biometric measurements, such as forearm length and body weight in bats, serve as key indicators of species adaptation and health \citep{davis2023writing}. Non-parametric statistical methods %, such as the Wilcoxon rank-sum test, 
are particularly useful when normality assumptions are violated. \citet{davis2023writing} conducted a comparative analysis of statistical methodologies in ecological research, emphasising the importance of selecting appropriate methods based on data distribution. Furthermore, geometric morphometric techniques implemented in R have facilitated the quantification of shape variations and evolutionary patterns in ecological studies \citep{adams2013geomorph}. The existing literature highlights the effectiveness of statistical techniques in tourism and ecological research.


The primary goal of this study is to analyse the Tourism Dataset, which contains information on Polish household trips. The analysis aims to explore relationships between key variables, expenditure trends, and behavioral patterns in trip planning. Specifically, this study investigates:
\begin{enumerate}[i]
    \item The relationship between trip purpose, accommodation type, and nights spent on total expenditure.
    \item Hypothesis testing methodologies, including t-tests and ANOVA, to assess differences between groups.
    \item Graphical visualisations to represent data distributions and spending trends.
\end{enumerate}



Using hypothesis testing and data visualisation techniques, this study aims to contribute to the growing body of research on data-driven decision-making in tourism and consumer behavior analysis.
The integration of R-based data visualisation, hypothesis testing, and multivariate statistical methods has enhanced the reproducibility and accuracy of findings in the field. This study builds upon %previous research by 
applying inferential statistical methods to analyse tourism expenditure and ecological morphometrics, contributing to the growing field of data-driven decision-making.

\begin{comment}
    
\section{Literature Review}

Statistical analysis plays a crucial role in tourism and ecological research, providing data-driven insights into expenditure trends, behavioral patterns, and biological adaptations. Researchers commonly use R for statistical computing and data visualization due to its flexibility and robust package ecosystem \citep{0_ihaka1996r}. The ability to perform exploratory data analysis (EDA), hypothesis testing, and regression modeling makes R an essential tool for empirical research \citep{wickham2016getting}.

In tourism research, various statistical methods have been applied to analyze spending behaviors and travel patterns. \citet{ding2018analysis} examined domestic tourism consumption in China and found that economic indicators such as GDP and disposable income significantly influenced tourism expenditure. Statistical models such as ANOVA and t-tests have been widely used to assess variations among different categories of travelers, such as business versus leisure tourists \citep{milenkovski2019statistical}. Additionally, multivariate statistical techniques, including time-series models and clustering algorithms, have been employed to predict seasonal fluctuations in travel demand, assisting policymakers in optimizing tourism strategies.

Ecological research has extensively utilized statistical methods to analyze morphometric traits and their ecological significance. Studies indicate that biometric measurements, such as forearm length and body weight in bats, serve as key indicators of species adaptation and health \citep{davis2023writing}. Non-parametric statistical methods, such as the Wilcoxon rank-sum test, are particularly useful when normality assumptions are violated. \citet{davis2023writing} conducted a comparative analysis of statistical methodologies in ecological research, emphasizing the importance of selecting appropriate methods based on data distribution. Furthermore, geometric morphometric techniques implemented in R have facilitated the quantification of shape variations and evolutionary patterns in ecological studies \citep{adams2013geomorph}.

The existing literature highlights the effectiveness of statistical techniques in tourism and ecological research. The integration of R-based data visualization, hypothesis testing, and multivariate statistical methods has enhanced the reproducibility and accuracy of findings in these fields. This study builds upon previous research by applying inferential statistical methods to analyze tourism expenditure and ecological morphometrics, contributing to the growing field of data-driven decision-making.

\end{comment}























\begin{comment}
R is a high-level programming language widely used in statistical computing, data analysis, and geospatial applications. It was developed by \citet{1_ihaka1996r} at the University of Auckland. The latest stable version, R ‘4.4.1’ was used, is available for free download at \url{https://cran.r-project.org/}.

The primary goal 
is to analyze the Tourism Dataset, which contains information on Polish household trips. The analysis aims to explore relationships between key variables, understand patterns in expenditure, and test hypotheses about spending behavior in relation to trip purposes, accommodation types, and nights spent.
This analysis uses statistical methods such as descriptive statistics, correlation analysis, hypothesis testing (e.g., t-tests, ANOVA), and graphical visualizations. These methods provide insights into how different factors contribute to total expenditure and whether differences exist among specific groups.
The  objectives were as follows:
\begin{itemize}

\item Characterizing the dataset and its variables descriptively.
\item Examining relationships between variables.
\item Conducting hypothesis tests and interpreting confidence intervals.
\item Presenting key findings and proposing areas for further research.
\end{itemize}
\end{comment}





\section{Methodology}

%A stepwise appraoch was used for the analysis.
A systematic, step-by-step approach was used for the analysis in R Studio.
Figure \ref{figMethod} shows the methodological flowchart for statistical modeling. 

\subsection{Packages Used}

Several R packages were utilised in this study for data handling, visualisation, and statistical analysis:

\begin{itemize}
    \item \texttt{readxl}: Used for reading Excel files (.xls and .xlsx) into R, allowing direct import of spreadsheet data without requiring external dependencies \citep{2_wickham2019package}. Unlike \texttt{read.csv()}, which handles CSV files, \texttt{readxl} supports native Excel formats.
    \item \texttt{dplyr}: A package for efficient data manipulation and transformation, part of the Tidyverse collection. It provides functions for filtering, summarizing, and modifying data frames \citep{3_yarberry2021dplyr}.
    \item \texttt{tidyverse}: A collection of R packages designed for data science, including \texttt{ggplot2}, \texttt{dplyr}, \texttt{tidyr}, \texttt{readr}, \texttt{tibble}, \texttt{stringr}, and \texttt{purrr} \citep{4_wickham2017package}. This suite simplifies data wrangling and visualisation.
    \item \texttt{ggplot2}: The most widely used package for data visualisation in R. It is based on the grammar of graphics, enabling the creation of layered plots \citep{wickham2016getting}.
    \item \texttt{reshape2}: Provides tools to reshape datasets between wide and long formats, making them more suitable for different types of statistical analyses \citep{wickham2007reshaping}.
    \item \texttt{Hmisc}: A package for statistical modeling and data analysis, offering functions for descriptive statistics, imputation, and regression modeling \citep{5_harrell2019package}.
\end{itemize}






% Packages (Libraries) Used:

% The readxl package is used for reading Excel files (.xls and .xlsx) into R \citep{2_wickham2019package}.
% Unlike read.csv(), which is used for CSV files, readxl directly handles Excel spreadsheets without requiring external dependencies.
% dplyr is for data manipulation and transformation.
% It is part of the tidyverse collection and provides functions to work efficiently with dataframes \citep{3_yarberry2021dplyr}.
% tidyverse is a collection of R packages designed for data science and visualization \cite{4_wickham2017package}.
% It includes packages like ggplot2, dplyr, tidyr, readr, tibble, stringr, and purrr, among others.
% ggplot2 is the most popular package for data visualization in R.
% It follows the grammar of graphics approach, allowing users to layer components (data, aesthetics, geometries, etc.).
% reshape2 is used for reshaping data between wide and long formats.
% Hmisc is a package for data analysis and statistical modeling which includes functions for data manipulation, descriptive statistics, imputation, and advanced regression modeling  \citep{5_harrell2019package}.

% \begin{enumerate}
%     \item readxl
%     \item dplyr
%     \item tidyverse
%     \item ggplot2
%     \item reshape2
%     \item hmisc
% \end{enumerate}


\subsection{Dataset Description}

%\subsubsection{Tourism Dataset}


The dataset used in this study was collected from a sample survey of Polish households over a specific period. It is an open-source data available online \citep{Cellini2013-ul}. It consists of numerical and categorical variables, making it suitable for both descriptive and inferential statistical analyses. Table \ref{tab0} provides an overview of the dataset variables.

\begin{table}[H]
\caption{Dataset Variables and  Description}
\begin{center}
\begin{tabular}{|p{2.5cm}|p{6cm}|}
\hline 
\textbf{Variable} & \textbf{Description} \\
\hline
Year & Year in which the trip occurred. \\
PT (Purpose of Trip) & The reason for travel, such as leisure, business, or other. \\
AT (Accommodation Type) & Type of accommodation used during the trip (e.g., private, business). \\
NS (Nights Spent) & Total number of nights spent during the trip. \\
TE (Total Expenditure) & Total monetary spending associated with the trip. \\
EC (Expenditure Categories) & Breakdown of expenses into accommodation, restaurants, and cafés. \\
\hline
\end{tabular}
\label{tab0}
\end{center}
\end{table}







% The Tourism Dataset used in this study was collected from a sample survey of Polish households over a specific period. The dataset, as detailed in Table \ref{tab0}, is well-structured and contains both numerical and categorical variables, making it suitable for descriptive and inferential statistical analyses. 

% \begin{table}[H]
% \caption{Bats data description}
% \begin{center}
% \begin{tabular}{|p{0.5cm}|p{7.5cm}|}
% \hline 
% \textbf{Var}&{\textbf{Description}} \\
% \hline
% Year & Year in which the trip occurred \\
% PT & Purpose of Trip: The reason for travel, such as leisure, business, or other.\\
% AT & Accommodation Type: Type of accommodation used during the trip (e.g., private, business).\\
% NS & Nights Spent: The total number of nights spent during the trip.\\
% TE & Total Expenditure: The monetary spending associated with the trip.\\
% EC & Expenditure Categories: Expenses for accommodation, restaurants, and cafés.\\
% \hline
% \end{tabular}
% \label{tab0}
% \end{center}
% \end{table}


\begin{comment}
    
\subsubsection{Bat Morphometric Dataset}

This dataset contains records of individual (tagged) bats, including their morphological measurements. The data was collected to study the relationship between body size and weight across different sexes and species. By analysing these variables, it is also possible to assess body condition, which is commonly defined as the ratio of body weight to forearm length.

\begin{table}[H]
\caption{Bat Morphometric Dataset Description}
\begin{center}
\begin{tabular}{|p{2.5cm}|p{6cm}|}
\hline
\textbf{Variable} & \textbf{Description} \\
\hline
especie & Bat species (two species included in the dataset). \\
sexo & Gender of the individual (male or female). \\
idade & Age category (juvenile or adult). \\
ab (Forearm Length) & Forearm length in millimeters, used as a morphological measurement of body size. \\
peso (Weight) & Individual's body weight in grams. \\
NA (Missing Value Indicator) & Encodes missing values in the dataset. \\
\hline
\end{tabular}
\label{tab1}
\end{center}
\end{table}




\end{comment}







% The dataset contains information on captures of individual (tagged) bats and morphological
% measurements. They were collected with the aim of studying the relationship between size
% and/or weight of individuals of different sexes and/or species. Using this data it is also
% possible to assess body condition (body weight divided by forearm length) which may be
% another response variable to study



% \begin{table}[htbp]
% \caption{Bats data description}
% \begin{center}
% \begin{tabular}{|c|c|}
% \hline
% \textbf{Variables}&{\textbf{Description}} \\
% \hline
% especie &Bat species (two species)\\
% sexo& Gender (male or female)\\
% idade &Age: juvenile (juv) or adult\\
% ab & 
% Forearm length (mm) %- morphological measurement to assess the body size 
% \\
% peso &individual’s weight (g)\\
% NA &encodes absence of value\\
% \hline

% \end{tabular}
% \label{tab1}
% \end{center}
% \end{table}

%Python is a high-level programming language that has large applications in geospatial and big data processing domains. It was developed by Guido Van Rossum from the University of Amsterdam in 1991 and the latest version is Python 3.10.6 (2022) which is available on \url{https://www.python.org/downloads/}. 

% R or Dataset section can be moved to methodology
% R 



\begin{figure*}[htbp]
\centerline{\includegraphics[width=\textwidth]{DataAnalysis_FlowChart_2.png}}
\caption{Methodological Flowchart.}
\label{figMethod}
\end{figure*}

% Write 2-3 lines for each:
\subsection{Descriptive Statistics}

To summarize the dataset, measures of central tendency and dispersion were calculated, including the mean, median, and standard deviation.

The \textit{mean} represents the average value, calculated as the sum of all values divided by the number of observations. While it provides a useful summary of the data, it is highly sensitive to outliers.

The \textit{median} is the middle value of the dataset when arranged in ascending order. Unlike the mean, the median is not affected by outliers, making it a better measure of central tendency when the data distribution is skewed.

The \textit{standard deviation} measures the dispersion of values around the mean. A higher standard deviation indicates greater variability in the dataset. It is calculated as the square root of the variance, given by Equation \ref{eq0}:

\begin{equation}{\label{eq0}}
    s = \sqrt{\frac{\sum_{i=1}^{N} (x_i - \overline{x})^2}{N-1} }
\end{equation}

where \( s \) is the standard deviation, \( x_i \) represents individual observations, \( \overline{x} \) is the mean, and \( N \) is the total number of observations.


% Mean, Median and Standard Deviation were used to calculate statistical description of the dataset.
% Mean sum of all values divided by the number of observations. It takes into account every value in the dataset however, it is sensitive to outliers.
% Median is the middle value. It is not effected by outliers. Hence, it provides better measure of central tendency when data is skewed.
% Standard deviation in statistics is a measure of the amount of variation of the values of a variable about its mean. It is given by the square root of the variance. The formula for standard deviation is given by Equation \ref{eq0}.
% \begin{equation}{\label{eq0}}
%     s = \sqrt{\frac{\sum_{i=1}^N (x_i - \overline{x})^2}{N-1} }
% \end{equation}



% \subsection{Visualisations and Plots}


% Scatterplot 
% is used examine relationship between two continuous variables. It is used to identify clusters, trends, and outliers. It is preliminary step before calculating Pearson correlation coefficient so to insure that the Pearson coefficient represent true value. 

% Histogram  
% is a type of bar graph that represents the distribution of numerical data. It divides data into bins (intervals) and counts how many values fall into each bin. Unlike a bar chart, which is used for categorical data, a histogram is used for continuous numerical data.
% It is used to understand the distribution of a dataset and to identify skewness (left, right, or symmetric).
% A bell shaped histogram shows a normally distributed data,

% Boxplot
% (also called a box-and-whisker plot) is a graphical representation of the distribution of a dataset. It displays the minimum, first quartile (Q1), median (Q2), third quartile (Q3), and maximum values, along with potential outliers. Boxplots are useful for detecting skewness, spread, central tendency and outliers in data.


% \subsection{Hypothesis Testing}


% To test the assumptions of ANOVA, normality and homogeneity test were applied. The Shapiro Wilk and Levene Test were used for the purposes respectively.

% The Shapiro-Wilk test 
% is a statistical test used to check whether a given dataset follows a normal distribution. It is commonly used in statistical analysis to determine if the assumption of normality is met before applying parametric tests like ANOVA, t-tests, or regression analysis.
% If p-value is less than the threshold than the data does not follow a normal distribution and vice versa.
% %0.05 Data follows a normal distribution (fail to reject the null hypothesis).p-value < 0.05 → Data does not follow a normal distribution (reject the null hypothesis).
% It works better for small datasets. For big datasets Kolmogorov-Smirnov test is more useful.

% The Levene's test 
% is a statistical test used to assess the homogeneity of variances across groups. In simpler terms, it checks whether different groups in your data have the same variance (i.e., whether the spread or dispersion of values is similar across groups). This is an important assumption in many statistical tests like ANOVA and t-tests, which require the variances between the groups being compared to be roughly equal.
% If p-value is less than the threshold then the groups are significantly different, and the assumption of equal variances is violated.


% P-value for significance threshold was used as given in Equation \ref{eq1}
% \begin{equation}
% p < 0.05\label{eq1}
% \end{equation}


% Tuckey 
% Honestly Significant Difference (HSD) Test is a post-hoc statistical test used to determine which specific group means are significantly different after conducting an ANOVA (Analysis of Variance) test. It controls the Type I error rate while making multiple comparisons between group means.
% It is used to determine which pairs of means differ significantly and 
% when comparing more than two groups to avoid multiple testing issues. If the p-value is less than the threshold, the difference between the groups is statistically significant.


% Fischer Test
% Fisher's Exact Test is a statistical test used to determine whether there is a significant association between two categorical variables in a 2x2 contingency table. It is particularly useful when sample sizes are small and when the Chi-Square test is not appropriate due to low expected frequencies.
% p-value < 0.05. There is a significant association between the two categorical variables. Otherwise, there is no significant association between the variables.


% ANOVA
% (Analysis of Variance) is a statistical method used to compare the means of three or more groups to determine if there is a significant difference between them \citep{ML_muller2024anova}. It helps to answer
% which group means differ significantly from each other.
% ANOVA is used when the data have three or more independent groups, to test for differences in means across these groups and when data is normally distributed.

% The Wilcoxon test 
% is a non-parametric statistical test used to compare two related or paired groups to assess whether their population mean ranks differ. It is often used as an alternative to the paired t-test when the data doesn't meet the assumptions of normality (i.e., the data is not normally distributed).
% Wilcoxon - Mann-Whitney U Test
% is used to compare two independent samples, similar to the two-sample t-test but without assuming normality. It tests whether one group tends to have larger values than the other, and it's based on comparing the ranks of the two groups \citep{ML2_thakkar2025continuous}.
% If p-value is less than the threshold it means that the two groups are significantly different in terms of their distribution or medians and vice versa.

% \FloatBarrier
% \section{Results and Analysis}

% \subsection{Tourism Data}

% \subsubsection{Descriptive Statistics}

% A box plot, as shown in Fig. \ref{figbox1}, was used to observe the variables collectively showing the median and the outliers among variables.
% \begin{figure}[htbp]
% \centerline{\includegraphics[width=0.5\textwidth]{BoxPlot_1.png}}
% \caption{Box plot showing spread of numeric variables}
% \label{figbox1}
% \end{figure}
% A scatterplot was generated to analyze the relationship between Nights Spent and Total Expenditure. The data points cluster around specific ranges, with no clear upward or downward trend, indicating that expenditure is not directly proportional to the number of nights spent.
% %Histogram (Figure 2):
% \begin{figure}[htbp]
% \centerline{\includegraphics[width=0.5\textwidth]{Histogram1_RPlot.png}}
% \caption{showing (a)  distribution of Accommodation type (b) distribution of Total Expenditure }
% \label{figbox1}
% \end{figure}
% The histogram of Total Expenditure reveals a skewed distribution, with the majority of spending concentrated in lower ranges. This suggests that most trips were modest in terms of expenditure, with only a few high-spending outliers.
% Non-Graphical Representations
% Summary statistics provide insights into central tendencies and variability in the dataset:
% - Mean Total Expenditure: 104.86 units.
% - Median Total Expenditure: 100 units.
% - Standard Deviation: 16.8 units.
% The correlation matrix showed weak relationships among numerical variables. For example, the correlation between Nights Spent and Total Expenditure was nearly negligible, suggesting that additional factors may influence expenditure patterns.


% \subsubsection{Relationships}
% Correlation Analysis was used to find the relationships among the variables.
% Assumptions for pearson correlation coefficient: 
% 1-Level of Measurement: Both variables should be measured on a continuous scale.
% 2-Related Pairs: Each observation must include pairs of values for the two variables.
% 3-Absence of Outliers: The data should not contain outliers in either variable.
% 4-Linearity: The relationship between the variables should be linear. Important.

% Otherwise use other correlation coefficients such as spearsmen correlation
% The correlation matrix was computed to evaluate linear relationships among numerical variables. 
% Figure \ref{} is a scatter plot which shows a good linearity between Organizer Expenditure and Private Expenditure. Hence, correlation was checked among the two variables. 
% \begin{figure}[H]
% \centerline{\includegraphics[width=0.5\textwidth]{ScatterPlot1_Rplot.png}}
% \caption{Scatterplot showing association between Private and Organiser Expenditure}
% \label{figbox1}
% \end{figure}
% A pearson correlation coefficient of 0.5789 was found which shows a strong correlation between the two variables.
% Also, a weak correlations were found, particularly between Nights Spent and Total Expenditure. This indicates that spending is not strongly linked to the duration of stay.
% Categorical Relationships
% A Fisher’s Exact Test was used to assess associations between Accommodation Type and Purpose of Trip. The test produced a p-value of 1e-05, indicating a significant relationship. For instance, business travelers were more likely to stay in business accommodations, while leisure travelers preferred private accommodations.

% Visual Analysis
% - The scatterplot supports the weak correlation observed between Nights Spent and Total Expenditure.
% - The histogram emphasizes the concentration of expenditures within a specific range, further illustrating the skewness of the distribution.

% 3.3 Hypothesis Testing
% T-Test (Comparison of Two Means)
% A Two Sample T-Test was conducted to compare Total Expenditure for Business and Private accommodation types:
% - t-statistic = 0.899, p-value = 0.369.
% - Confidence Interval: [-0.60, 1.62].
% - Conclusion: There is no statistically significant difference in spending between business and private accommodations.
% ANOVA (Comparison Across Groups)

% There are 2 important Assumptions of ANOVA:
% - Normality: The data within each group should be normally distributed.
% - Homogeneity: The variance of the data within each group should be equal.
% For Normality Check:
% - first check the histogram, scatter plot and box plot (figures in appendix)
% - Then applied the Shapiro-Wilk test.
% The Shapiro-Wilk test, was as follows:
% W = 0.95, p-value < 2.2e-16
% Since the pvalue is less than 0.05. Null hypothesis (data is normally distributed) is rejected. Hence, the data is not normally distributed.
% Shapiro-Wilk test is well suitable for smaller dataset therefore to double check, Also Kolmogorov-Smirnov test was used which also indicated that the data is not normally distributed. Alternative hypothesis: two-sided. 
% D = 0.062742, p-value < 2.2e-16
% Similarly, other expenditure types were also check using Shapiro-Wilk and Kolmogorov-Smirnov test. Each expenditure type has no normal distribution.

% For Homogeneity Check:

% Levene Test: to check variances across groups are equal.

% A levene performed to test  whether Total Expenditure differed across various Trip Purposes: Levene's Test result: p value= 0.4996 is greater than 0.05. Hence, the groups are homogeneous.



% The below mentioned paragraph and conclusion is not significant as ANOVA test assumption of Normality was not upheld in this case.
% An ANOVA was performed to test whether Total Expenditure differed across various Trip Purposes: F-statistic = 0.407, p-value = 0.944. Conclusion: There are no significant differences in spending based on trip purposes, such as business, leisure, or other reasons.
% Hence, ANOVA cannot be performed an alternative Mann- Whitney test was performed as follows:

% The Mann-Whitney U test (also called the Wilcoxon rank-sum test) is a non-parametric test used to compare two independent groups when the normality assumption is violated.
% A wilcoxon test was used to test Total Expenditure across the two types of Accomodation (Private and Buisness). Following were the results:
% W = 2802744, p-value = 0.4868
% Since p-value is greater than 0.05. Hence, null hypothesis was rejected. Conclusion: No significant difference between the two groups was found.

% Fisher’s Exact Test
% This test evaluated the association between Accommodation Type and Purpose of Trip:
% p-value = 1e-05.
% Conclusion: A strong association exists between these two categorical variables, suggesting that the type of accommodation chosen is influenced by the purpose of the trip.





% \subsection{Bats Data}


% First the database was loaded and Exploraty Data Analysis (EDA) was carried out.
% batdata.csv was a semi-colon delineated file. The database has a single column for all data. That was first converted into seperate columns by using semi colon (;) as sepeartor. 
% The batdata file had 109 observations (rows) and 5 variables (columns). 


% All variable columns had type character (chr). Even though ab and peso had numerical floating values. The frequency distribution was observed for the database in R Studio. $<NA>$ entries or values were found in idade and peso which encoded for the absence of value.

% (a) Miniopterus schreibersii 
% (b) Myotis myotis



% The dataset comprised captured records of bats, \%individually tagged bats, 
% along with their morphological measurements. It was gathered to investigate the relationship between body size and/or weight across different sexes and species. Additionally, this dataset allows for the evaluation of body condition, calculated as the ratio of body weight to forearm length, which serves as another potential response variable for analysis.
% By looping through each entity within the database, 6 NA values were found.
% The following R programming code was manually developed for finding NA values:
% The rows with NA values were removed to clean the database of missing values. 103 observations were left which were further analysed.


% The scatter plot shows clear difference between species in terms of weight for Miniopterus scheibersii and Myotis myotis. The Myotis Bats are heavier than any of the Miniopterus Bats.

% The decimal was represented as “,” in the database that needed to be replaced by “.” for correct conversion of character type column into numeric type. Following code was used for this purpose: 


% Univariate abd Bivariate
% The Univariate Analysis focuses on a single variable, describing its distribution, central tendency, and dispersion (e.g., mean, median, standard deviation, histograms). 



% Correlation cannot be calculated in univariate analysis because it requires a second variable to evaluate a relationship.
% Bivariate Analysis: explores the relationship between two variables. Correlation, such as Pearson’s correlation coefficient or Spearman’s rank correlation, is a key part of bivariate analysis, quantifying how strongly two variables are linearly or monotonically related.
% For Univariate, mean median and standard deviation were checked:

% \begin{table}[htbp]
% \caption{Bats data description}
% \begin{center}
% \begin{tabular}{|c|c|c|}
% \hline
% \textbf{Operation}&{\textbf{Morphological assessment}}& {\textbf{Weight (g)}} \\
% \hline
% Mean	&51.69728&	 17.04515\\
% Median	&45.93&	12.36\\
% Standard Deviation&	 8.500533	& 7.037895\\
% \hline

% \end{tabular}
% \label{tab2}
% \end{center}
% \end{table}

% Correlation checks are inherently bivariate, as they measure the relationship or association between two variables. Below, the correlation check perfoermed between two variables in the database.


% Figure \ref{figcorr1} shows correaltion among all the variables however, this might show some information not true.

% \begin{figure}[htbp]
% \centerline{\includegraphics[width=0.5\textwidth]{CorrealtionMatrixRplot.png}}
% \caption{Correlation matrix showing pearson coefficient among numeric variables}
% \label{figcorr1}
% \end{figure}

% Then the Tukey test was performed in R-Studio: The Tukey test runs pairwise comparisons among each of the groups, and uses a conservative error estimate to find the groups which are statistically different from one another.

% Fisher’s Exact Test is computationally intensive for large tables but exact, unlike the Chi-Square test.
% P-value = 0.1663
% Since the p-value was high, the null hypothesis was not rejected. It suggested that the type of morphological characteristics chosen are not influenced by age group.

% Wilcoxon Test:

% W = 1526.5, p-value = 0.1145

% Since the p-value was high, the null hypothesis was not rejected. Hence, no significant difference in morphological characteristics values was found due to age (idade).



% To check correlation between weight of morphological assessment and weights of bats.

% Now performed Pearson correlation check:
% Pearson correlation coefficient: 0.9741522
% Hence, very strong correlation exists between morphological assessment and weight of the bats.

% This means that with more Morphological assessment value the weight increased for the bats which is natural.


% Now, divided the weight into three categories
% Heavy: weight greater than 25
% Medium: weight between 15 to 25
% Light: weight less than 15


% For Normality Check:
% Checking normality within the morphological assessment:
% Heavy weight:   W = 0.95748, p-value = 0.3665
% Medium weight:  W = 0.94314, p-value = 0.4601
% Light weight:   W = 0.62442, p-value = 1.591e-11


% I wanted to perform ANOVA with the three groups however, it is not possible because one of the group is doesnot hold on noramlity assumption, only two groups are normally distributted and hence the ANOVA cannot be performed.
% Hence performed the wilcoxon test:
% A wilcoxon test was used to test Total Expenditure across the morphological assessment and gender of the bats . Following were the results:
% W = 1957.5, p-value = 2.194e-05
% alternative hypothesis: true location shift is not equal to 0.
% Since, p-value is less than 0.05 hence, this time the null hypothesis is rejected. Significant difference between the two groups. 



\subsection{Visualisations and Plots}

Several graphical methods were used to explore the dataset and identify patterns.

\textbf{Scatterplots} were used to examine relationships between two continuous variables, helping to identify clusters, trends, and outliers. They are a preliminary step before calculating Pearson's correlation coefficient to ensure that the correlation reflects a true linear relationship.

\textbf{Histograms} were employed to visualise the distribution of numerical variables. They divide data into bins (intervals) and count the frequency of values within each bin. Unlike bar charts, which are used for categorical data, histograms are useful for detecting skewness (left, right, or symmetric) and assessing whether the data follows a normal distribution.

\textbf{Boxplots} (also called box-and-whisker plots) provide a summary of the data distribution. They display the minimum, first quartile (Q1), median (Q2), third quartile (Q3), and maximum values, along with potential outliers. Boxplots were particularly useful in detecting skewness, spread, and central tendency in the dataset.

\subsection{Hypothesis Testing}

To validate the assumptions of ANOVA, normality and homogeneity tests were applied using the Shapiro-Wilk and Levene’s tests.
\textbf{P-value Threshold:}  
A significance level of 0.05 was used to determine statistical significance, as defined in Equation \ref{eq1}:
\begin{equation}
p < 0.05
\label{eq1}
\end{equation}

\textbf{Shapiro-Wilk Test:}  
This test checks whether a given dataset follows a normal distribution. It is commonly used before applying parametric tests like ANOVA, t-tests, or regression analysis. A p-value greater than 0.05 indicates that the data follows a normal distribution (fail to reject the null hypothesis), whereas a p-value less than 0.05 suggests a violation of normality. While the Shapiro-Wilk test is effective for small datasets, for larger datasets, the Kolmogorov-Smirnov test is more appropriate.

\textbf{Levene’s Test:}  
Levene’s test assesses the homogeneity of variances across groups. Many statistical tests, including ANOVA and t-tests, assume that the variances of different groups being compared are roughly equal. If the p-value is greater than 0.05, homogeneity is maintained, and parametric tests can be applied. Otherwise, alternative methods such as Welch’s ANOVA or non-parametric tests are considered.



\textbf{Tukey's Honestly Significant Difference (HSD) Test:}  
A post-hoc test used after ANOVA to determine which specific group means are significantly different. It controls the Type I error rate in multiple comparisons. A p-value below 0.05 indicates a significant difference between group means.

\textbf{Fisher's Exact Test:}  
Used to assess associations between two categorical variables in a 2x2 contingency table. It is particularly useful for small sample sizes where the Chi-Square test may not be appropriate.

\textbf{ANOVA (Analysis of Variance):}  
ANOVA is used to compare the means of three or more independent groups to determine if there are significant differences between them \citep{ML_muller2024anova}. The test is applicable when data are normally distributed and homogeneity of variance is satisfied.

\textbf{Wilcoxon Test:}  
A non-parametric alternative to the paired t-test, used when data do not follow a normal distribution. The Wilcoxon-Mann-Whitney U Test is used for comparing two independent samples without assuming normality \citep{ML2_thakkar2025continuous}.





\FloatBarrier
\section{Results and Analysis}

\subsection{Tourism Data}

\subsubsection{Descriptive Statistics}

A box plot (Figure \ref{figbox1}) was used to observe the variables collectively, showing the median and outliers.

\begin{figure}[htbp]
\centerline{\includegraphics[width=0.5\textwidth]{BoxPlot_1.png}}
\caption{Box plot showing the spread of numeric variables}
\label{figbox1}
\end{figure}

A scatterplot was generated to analyse the relationship between Nights Spent and Total Expenditure. The data points cluster around specific ranges, with no clear upward or downward trend, indicating that expenditure is not directly proportional to the number of nights spent.

\begin{figure}[htbp]
\centerline{\includegraphics[width=0.5\textwidth]{Histogram1_RPlot.png}}
\caption{Histogram showing (a) distribution of Accommodation Type (b) distribution of Total Expenditure}
\label{figbox2}
\end{figure}

The histogram of Total Expenditure reveals a skewed distribution, with most spending concentrated in lower ranges, suggesting that the majority of trips had modest expenditures, with only a few high-spending outliers.

\textbf{Summary Statistics:}  
- Mean Total Expenditure: 104.86 units.  
- Median Total Expenditure: 100 units.  
- Standard Deviation: 16.8 units.  

The correlation matrix showed weak relationships among numerical variables. For example, the correlation between Nights Spent and Total Expenditure was negligible, suggesting that additional factors influence expenditure patterns.

\subsubsection{Relationships}

A correlation analysis was conducted to find relationships among the variables. Pearson’s correlation coefficient assumptions include:  
1. Both variables should be continuous.  
2. Each observation should include paired values for the two variables.  
3. Absence of extreme outliers.  
4. A linear relationship between the variables.  

If these assumptions are not met, alternative correlation measures such as Spearman’s correlation are used.

A scatter plot (Figure \ref{figscatter1}) shows a strong linear relationship between Organizer Expenditure and Private Expenditure with significant anomalies.

\begin{figure}[H]
\centerline{\includegraphics[width=0.5\textwidth]{ScatterPlot1_Rplot.png}}
\caption{Scatterplot showing association between Private and Organizer Expenditure}
\label{figscatter1}
\end{figure}

A Pearson correlation coefficient of 0.5789 indicates a moderately strong correlation between these variables. In contrast, a weak correlation was found between Nights Spent and Total Expenditure, implying that spending is not strongly linked to the duration of stay.

\textbf{Categorical Relationships:}  
A Fisher’s Exact Test was used to assess associations between Accommodation Type and Purpose of Trip, yielding a p-value of 1e-05, indicating a significant relationship. Business travelers were more likely to stay in business accommodations, while leisure travelers preferred private accommodations.

Before conducting a T-test for total expenditure with accommodation type, homogeneity test and normality test was conducted.
Using Levene test, it was known that the groups are homogeneous. However, Shapiro-wilk test revealed that groups do not have a normality distribution. Hence, further testing would be trivial. 


\begin{comment}
    
\subsection{Bats Morphometric Data}

The dataset comprised captured records of tagged bats along with their morphological measurements. The data was collected to investigate the relationship between body size and weight across different sexes and species. Additionally, this dataset allows for evaluating body condition, defined as the ratio of body weight to forearm length.

\textbf{Summary Statistics:}  
Table \ref{tab2} provides a summary of numeric morphological variables.

\begin{table}[htbp]
\caption{Bats Data Description}
\begin{center}
\begin{tabular}{|c|c|c|}
\hline
Operation & Forearm Length (mm) & Weight (g) \\
\hline
Mean & 51.70 & 17.04 \\
Median & 45.93 & 12.36 \\
Standard Deviation & 8.50 & 7.04 \\
\hline
\end{tabular}
\label{tab2}
\end{center}
\end{table}

The Pearson correlation coefficient of 0.9741 indicated a very strong relationship between forearm length and weight. This suggests that bats with larger forearm lengths tend to weigh more.

To verify normality assumptions, the Shapiro-Wilk test was performed. Bats data also failed the normality test.

- Heavy-weight bats: W = 0.9574, p-value = 0.3665 

- Medium-weight bats: W = 0.9431, p-value = 0.4601  

- Light-weight bats: W = 0.6244, p-value = 1.59e-11  

Since the light-weight group failed the normality test, ANOVA was not appropriate, and a Wilcoxon test was conducted instead:

- Wilcoxon test: W = 1957.5, p-value = 2.19e-05  

Hence, a significant difference exists between the groups.

\end{comment}


% \section{Conclusion}
% The following conclusions were drawn from the analysis: 
% For Tourism:

% %1.Spending Patterns:
% %There is no significant difference in total expenditure
% - There is no significant difference in total expenditure based on accommodation type or trip purpose. This suggests that travelers spend similarly regardless of their trip's purpose or the accommodation they choose.

% %2.Relationships:
% - A  strong correlation was found between Organizer Expenditure and Private Expenditure. Weak correlations were observed between Nights Spent and Total Expenditure, indicating that spending is not strongly influenced by the duration of stay.

% %3.Categorical Association:
% - No significant difference was found between the Total Expenditure across the two types of Accommodation. A significant association exists between Accommodation Type and Purpose of Trip, highlighting the influence of trip purpose on accommodation choice.



% For Bats:

%  -   Very strong correlation exists between morphological assessment and weight of the bats.
 
% -  Significant difference exists between morphological assessment of male and female bats.

% -  No significant difference in exist between morphological assessment of younger (juv) and elder (adulto) bats.


\section{Conclusion}

This study analysed tourism expenditure patterns %and bat morphometric data 
using statistical and visualisation techniques. While significant relationships were identified among certain variables, %ANOVA could not be applied in this case due to unmet assumptions.
ANOVA was not applicable in this case because its assumptions were not met. The key findings are summarized below:

% \subsection{Tourism Data}
- A significant association was found between Accommodation Type and Purpose of Trip, highlighting the influence of trip purpose on accommodation choice.
%

- A strong correlation was observed between Organizer Expenditure and Private Expenditure, while a weak correlation was found between Nights Spent and Total Expenditure. This indicates that spending is not strongly influenced by the duration of stay.  
% 

- No significant difference was found in total expenditure based on accommodation type and 
trip purpose, suggesting that travelers spend similarly regardless of their trip's purpose or lodging preference.  
%
%However, 
%Also, no significant difference was observed in Total Expenditure across different accommodation types.  
Any observed differences could be due to random variation rather than a true underlying effect.


%Additionally, 
This research provides an overview of the R programming environment and  %The study contributes to the understanding of %various
%statistical analysis methods in R. 
it contributes to the comprehension of statistical analysis methodologies and libraries in R. 
Further analysis of the dataset may reveal valuable patterns, trends, and relationships, providing deeper insights into the underlying phenomena. By applying advanced statistical and machine learning techniques, more meaningful conclusions can be derived to support data-driven decision-making.

\begin{comment}
    
\subsection{Bat Morphometric Analysis}
A very strong correlation was found between morphological assessment and bat weight, indicating that larger bats tend to weigh more.  
A significant difference was observed in morphological assessments between male and female bats.  
No significant difference was found in morphological assessments between juvenile and adult bats, suggesting that size differences may not be substantial between younger and older individuals.  

These findings contribute to a better understanding of tourism spending behavior and bat morphological variations, providing insights for both economic analysis and ecological research.


\end{comment}




%Bibliography

%1.RStudio Team (2020). RStudio: Integrated Development for R. RStudio, PBC, Boston, MA URL http://www.rstudio.com/.
%2.———. 2016. Bookdown: Authoring Books and Technical Documents with R Markdown.
%3.Boca Raton, Florida: Chapman; Hall/CRC. https://bookdown.org/yihui/bookdown. ———. 2024a. Bookdown: Authoring Books and Technical Documents with r Markdown. https://github.com/rstudio/bookdown.
%4.Verzani, J. (2011). Getting started with RStudio. " O'Reilly Media, Inc.".
%5.Bevans, R. (2024, May 09). One-way ANOVA | When and How to Use It (With Examples). Scribbr. Retrieved January 13, 2025, from https://www.scribbr.com/statistics/one-way-anova/




\section*{Acknowledgment}

The author acknowledge the teaching from Dr. Dulce Gomez, J. Tiago Marques, Pedro A. Salgueiro and Sara Santos during the course: Fundamentos de análise de dados em ambiente R in Fall 2024. % The author acknowledge the facilities provided by Universidade de Evora. 
The author was supported by EURAXESS Grant ID:186817, Mobilizing Agenda: Sines Nexus, with reference "C645112083-00000059" co-financed by the PRR - Recovery and Resilience Plan. 








%\section*{References}

\bibliography{refbib.bib}
\bibliographystyle{IEEEtranN}



%\section*{Appendix}









\vspace{12pt}
%\color{red}

\end{document}
