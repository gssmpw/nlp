\section{Literature Review}
Statistical analysis plays a crucial role in tourism and ecological research, providing data-driven insights into expenditure trends, behavioral patterns, and biological adaptations. Researchers commonly use R for statistical computing and data visualization due to its flexibility and robust package ecosystem ____. The ability to perform exploratory data analysis (EDA), hypothesis testing, and regression modeling makes R an essential tool for empirical research ____.

In tourism research, various statistical methods have been applied to analyze spending behaviors and travel patterns. ____ examined domestic tourism consumption in China and found that economic indicators such as GDP and disposable income significantly influenced tourism expenditure. Statistical models such as ANOVA and t-tests have been widely used to assess variations among different categories of travelers, such as business versus leisure tourists ____. Additionally, multivariate statistical techniques, including time-series models and clustering algorithms, have been employed to predict seasonal fluctuations in travel demand, assisting policymakers in optimizing tourism strategies.

Ecological research has extensively utilized statistical methods to analyze morphometric traits and their ecological significance. Studies indicate that biometric measurements, such as forearm length and body weight in bats, serve as key indicators of species adaptation and health ____. Non-parametric statistical methods, such as the Wilcoxon rank-sum test, are particularly useful when normality assumptions are violated. ____ conducted a comparative analysis of statistical methodologies in ecological research, emphasizing the importance of selecting appropriate methods based on data distribution. Furthermore, geometric morphometric techniques implemented in R have facilitated the quantification of shape variations and evolutionary patterns in ecological studies ____.

The existing literature highlights the effectiveness of statistical techniques in tourism and ecological research. The integration of R-based data visualization, hypothesis testing, and multivariate statistical methods has enhanced the reproducibility and accuracy of findings in these fields. This study builds upon previous research by applying inferential statistical methods to analyze tourism expenditure and ecological morphometrics, contributing to the growing field of data-driven decision-making.

\end{comment}























\begin{comment}
R is a high-level programming language widely used in statistical computing, data analysis, and geospatial applications. It was developed by ____ at the University of Auckland. The latest stable version, R ‘4.4.1’ was used, is available for free download at \url{https://cran.r-project.org/}.

The primary goal 
is to analyze the Tourism Dataset, which contains information on Polish household trips. The analysis aims to explore relationships between key variables, understand patterns in expenditure, and test hypotheses about spending behavior in relation to trip purposes, accommodation types, and nights spent.
This analysis uses statistical methods such as descriptive statistics, correlation analysis, hypothesis testing (e.g., t-tests, ANOVA), and graphical visualizations. These methods provide insights into how different factors contribute to total expenditure and whether differences exist among specific groups.
The  objectives were as follows:
\begin{itemize}

\item Characterizing the dataset and its variables descriptively.
\item Examining relationships between variables.
\item Conducting hypothesis tests and interpreting confidence intervals.
\item Presenting key findings and proposing areas for further research.
\end{itemize}
\end{comment}