\newpage
\appendix
\onecolumn

% \section{Multiple agents}
% The action space $\mathcal{A}^r$ is defined as $\mathcal{A}^r = \{$\texttt{[No Retrieval]},\texttt{[Retrieval]<query>}, \texttt{[Planning]}$\}$, where \texttt{<query>} represents the space of possible queries.


% The action space $\mathcal{A}^d$ is defined as $\mathcal{A}^d = \{\texttt{Retrieval} \times \mathcal{Q}\} \cup \{\texttt{LLM}\}$, where $\mathcal{Q}$ represents the space of possible subqueries.

% \section{Discussions}

% \subsection{Future Work}

% \subsection{Limitations}
% sft warm-up  from zero

\section{More Implementation Details}\label{app:imple_details}

In this section, we provide a comprehensive implementation details of our proposed method. For additional insights and more intricate details, we refer the reader to our supplementary materials.


\subsection{RL Training Process}\label{app:ppo_details}
Having obtained the credit rewards that reflect each agent's contribution, we develop an optimization framework to guide end-to-end training across all agents. % encourage collaborative behaviors among agents
The key idea is to use these credit signals for optimizing the collaborative behavior of the entire system.
The optimization objective for our multi-agent system can be formulated as maximizing the expected credit rewards:
\begin{equation}
    \mathcal{J}(\theta) = \mathbb{E}_{\tau \sim \pi_\theta}\left[\sum_{i\in \mathcal{N}}\sum_{t} r_{\text{credit}}(s^i_t, a^i_t)\right]
\end{equation}
% We optimize this objective using Proximal Policy Optimization (PPO)~\citep{SchulmanWDRK17}.
% Since each agent's action is a sequence of tokens, we optimize this objective using Proximal Policy Optimization (PPO)~\citep{SchulmanWDRK17} and decompose the optimization at the token level~\citep{Ouyang0JAWMZASR22}. Specifically, we define:
Since each agent's action is a sequence of tokens, we decompose this optimization using Proximal Policy Optimization (PPO)~\citep{SchulmanWDRK17,abs-2312-01058,ZhuDW24} as follows:
\begin{equation}
    \mathcal{L}_{\text{\modelname}} = \sum_{i\in\mathcal{N}} \mathcal{L}_{\text{PPO}}^i(\theta, \phi)
\end{equation}
Specifically, for each agent $i$, we define:
\begin{equation}
    \mathcal{L}_{\text{CLIP}}^i(\theta)
    = \mathbb{E}_{\tau \sim \pi_\theta}\Big[\sum_{t}\sum_{m} \min\big(r_{t,m}^i(\theta)\hat{A}_{t,m}^i, \text{clip}(r_{t,m}^i(\theta), 1-\epsilon, 1+\epsilon)\hat{A}_{t,m}^i\big)\Big]
\end{equation}
% \begin{equation}
%     \mathcal{L}_{\text{CLIP}}(\theta)
%     = \mathbb{E}_{\tau \sim \pi_\theta}\Big[\sum_{i\in \mathcal{N}}\sum_{t}\sum_{m} \min\big(r_{t,m}^i(\theta)\hat{A}_{t,m}^i, \text{clip}(r_{t,m}^i(\theta), 1-\epsilon, 1+\epsilon)\hat{A}_{t,m}^i\big)\Big]
% \end{equation}
where $r_{t,m}^i(\theta) = \frac{\pi_\theta(a_{t,m}^i|s_{t,m}^i)}{\pi_{\theta_{\text{old}}}(a_{t,m}^i|s_{t,m}^i)}$ is the probability ratio, $s_{t,m}^i$ represents the concatenation of current state and the first $m-1$ tokens in the action sequence for agent $i$ at time step $t$, and $a_{t,m}^i$ denotes its $m$-th token.
We compute the advantage estimate using GAE~\citep{SchulmanMLJA15}: $\hat{A}_{t,m}^i = \sum_{l=0}^{M-m-1}(\gamma\lambda)^l\delta_{t,m+l}^i$, where $M$ is the token length of the action sequence.
% We compute the advantage estimate using GAE~\citep{SchulmanMLJA15}: $\hat{A}_{t,m}^i = \sum_{l=0}^{M-m-1}(\gamma\lambda)^l\delta_{t,m+l}^i$, where $M$ is the token length of the action sequence, and $\delta_{t,m}^i = r_{\text{token}}(s_{t,m}^i, a_{t,m}^i) + \gamma V_\phi(s_{t,m+1}^i) - V_\phi(s_{t,m}^i)$ is the TD-error.
% The advantage estimate $\hat{A}_{t,m}^i$ is computed using Generalized Advantage Estimation (GAE):
% \begin{equation}
%     \hat{A}_{t,m}^i = \delta_{t,m}^i + (\gamma\lambda)\delta_{t,m+1}^i + ... + (\gamma\lambda)^{M-m+1}\delta_{t,M-1}^i
% \end{equation}
% where $M$ is the token length of the corresponding response sequence, and $\delta_{t,m}^i = r_{\text{token}}(s_{t,m}^i, a_{t,m}^i) + \gamma V_\phi(s_{t,m+1}^i) - V_\phi(s_{t,m}^i)$ is the TD-error.
% The token-level reward $r_{\text{token}}$ incorporates both the tree credit and a KL penalty term:
% \begin{equation}
%     r_{\text{token}}(s_{t,m}^i, a_{t,m}^i) = \begin{cases}
%         r_{\text{credit}}(s_t^i, a_t^i) -\beta \text{KL}(m) & m = M \\
%         -\beta \text{KL}(m) & \text{otherwise}
%     \end{cases}
% \end{equation}
% where $\text{KL}(m) = \log \frac{\pi_{\theta_{\text{old}}}(a_{t,m}^i|s_{t,m}^i)}{\pi_{\theta_{\text{ref}}}(a_{t,m}^i|s_{t,m}^i)}$, and $\beta$ is a hyperparameter that controls the strength of the KL penalty.

To estimate state values across the multi-agent system, we employ a centralized state-value function $V_\phi$ that takes each agent's state $s_{t,m}^i$ as input. The value function is optimized to minimize the mean squared error~\citep{LoweWTHAM17}:
\begin{equation}
    \mathcal{L}_V^i(\phi) = \mathbb{E}_{\tau \sim \pi_\theta}\left[\sum_{t}\sum_{m} (V_\phi(s_{t,m}^i) - \hat{G}_{t,m}^i)^2\right]
\end{equation}
where $\hat{G}_{t,m}^i = \hat{A}_{t,m}^i + V_\phi(s_{t,m}^i)$ is the empirical return.
The final optimization objective combines the policy and value losses:
\begin{equation}
    \mathcal{L}_{\text{PPO}}^{i}(\theta, \phi) = \mathcal{L}_{\text{CLIP}}^{i}(\theta) + c_v \mathcal{L}_V^i(\phi)
\end{equation}
where $c_v$ controls the weight of the value loss. This joint objective enables end-to-end training of both policy and value networks across all agents.
% Therefore, for multi-agent reinforcement learning~\citep{abs-2312-01058,ZhuDW24}, we have the final training objective:
% \begin{align}
%     \mathcal{L}_{\text{\modelname}} &= \sum_{i\in\mathcal{N}} \mathcal{L}_{\text{PPO}}^i(\theta, \phi) \nonumber \\
%     &= sum_{i\in\mathcal{N}} \Big(
%         \mathcal{L}_{\text{CLIP}}^{i}(\theta) + c_v \mathcal{L}_V^i(\phi)
%     \Big)
%     &= \sum_{i\in \mathcal{N}} \Big( \mathbb{E}_{\tau \sim \pi_\theta}\Big[\sum_{t}\sum_{m} \min\big(r_{t,m}^i(\theta)\hat{A}_{t,m}^i, \text{clip}(r_{t,m}^i(\theta), 1-\epsilon, 1+\epsilon)\hat{A}_{t,m}^i\big)\Big]
%     + \nonumber \\
%     &\mathbb{E}_{\tau \sim \pi_\theta}\left[\sum_{t}\sum_{m} (V_\phi(s_{t,m}^i) - \hat{G}_{t,m}^i)^2\right] \nonumber \\
%     &= \mathbb{E}_{\tau \sim \pi_\theta}\Big[\sum_{i\in \mathcal{N}}\sum_{t}\sum_{m} \min\big(r_{t,m}^i(\theta)\hat{A}_{t,m}^i, \text{clip}(r_{t,m}^i(\theta), 1-\epsilon, 1+\epsilon)\hat{A}_{t,m}^i\big)\Big]
% \end{align}


% In cooperative multi-agent system, we employ a centralized state-value function $V_\phi$ shared across all agents to estimate the value of each agent's state $s_{t,m}^i$. This value function is optimized to minimize the mean squared error between its predictions and the empirical returns across all agents:
% \begin{equation}
%     L_V(\phi) = \mathbb{E}_{\tau \sim \pi_\theta}\left[\sum_{i\in \mathcal{N}}\sum_{t}\sum_{m} (V_\phi(s_{t,m}^i) - \hat{G}_{t,m}^i)^2\right]
% \end{equation}
% where $\hat{G}_{t,m}^i = \hat{A}_{t,m}^i + V_\phi(s_{t,m}^i)$ is the empirical return.
% The final optimization objective combines the policy and value losses:
% \begin{equation}
%     L(\theta, \phi) = L^{\text{CLIP}}(\theta) - c_v L_V(\phi)
% \end{equation}
% where $c_v$ is a coefficient for the value loss.
% This overall optimization is performed in an end-to-end manner across all agents simultaneously.


\subsection{Implementation Details}
\textbf{Supervised Warm-up Phase:} 
We utilize \texttt{Llama-Factory}~\citep{zheng2024llamafactory} as our training framework for the initial supervised fine-tuning phase. The detailed hyper-parameters for this phase are presented in Table~\ref{tab:sft_params}.
\begin{table}[h]
\centering
\caption{Key hyperparameters in the supervised warm-up phase.}
\label{tab:sft_params} 
% \resizebox{0.99\linewidth}{!}{
% \begin{small}
\begin{tabular}{@{}lc@{}}
\toprule
\textbf{Hyperparameter} & \textbf{Value}                         \\ \midrule
Learning Rate           & 4e-5                                   \\
Batch size              & 512                                    \\
\#Epochs                & 3                                      \\
Optimizer type          & AdamW~\citep{LoshchilovH19}            \\
Chat template           & \texttt{Qwen}~\citep{qwen2}            \\
Base model              & Qwen2-1.5B or Qwen2-0.5B~\citep{qwen2} \\
Cutoff length           & 4096                                   \\
Warmup ratio            & 0.03                                   \\
LR scheduler type       & Cosine                                 \\ \bottomrule
\end{tabular}
% }
% \end{small}
\end{table}

\textbf{Reinforcement Learning Phase:} 
For the RL training phase, we adopt \texttt{OpenRLHF}~\citep{hu2024openrlhf} as our primary training framework, coupled with \texttt{VLLM}~\citep{kwon2023efficient} inference engine. The complete set of RL training hyper-parameters is detailed in Table~\ref{tab:rl_params}. To initialize both the policy and value models, we leverage the model obtained after one epoch of supervised fine-tuning, with the language model head replaced by a value head for the value model.
\begin{table}[h]
\centering
\caption{Key hyperparameters in the RL phase.}
\label{tab:rl_params} 
% \resizebox{0.99\linewidth}{!}{
% \begin{small}
\begin{tabular}{@{}lc@{}}
\toprule
\textbf{Hyperparameter}       & \textbf{Value} \\ \midrule
Learning Rate of Policy model & 5e-7           \\
Learning Rate of Value model  & 5e-6           \\
Batch size                    & 1024           \\
KL Coefficient                & 0.005          \\
Optimizer type                & Adam           \\
Prompt max len                & 4096           \\
Generate max len              & 2048           \\
Maximal depth                 & 13             \\
LR scheduler type             & Cosine         \\ \bottomrule
\end{tabular}
% }
% \end{small}
\end{table}

% \textbf{Inference Phase\footnote{We have released all the code in the Supplementary Material.}:}
% We construct our retrieval server using the 2018 Wikipedia dump~\citep{yang-etal-2018-hotpotqa} as the knowledge source and use contriever-msmarco~\citep{IzacardCHRBJG22} as our dense retriever.
% We also adopt the \href{https://docs.sglang.ai/}{\texttt{SGLang}}\footnote{\url{https://docs.sglang.ai/}} as the LLM server to support different model, such as Qwen2-72B-Instruct~\citep{qwen2}, Llama3.3-70B-Instruct~\citep{llama3}.
% Our inference code support two efficient inference engine: \texttt{SGLang} and \texttt{VLLM}.

\textbf{Inference Phase\footnote{The complete implementation code is available in the Supplementary Material.}:}
For the deployment of our system, we establish a comprehensive infrastructure that integrates multiple components:

\begin{itemize}[topsep=1pt, partopsep=1pt, leftmargin=12pt, itemsep=-1pt]
    \item \textbf{Retriever Server:} We construct our retrieval server using the 2018 Wikipedia dump~\citep{yang-etal-2018-hotpotqa} as the primary knowledge source. We employ contriever-msmarco~\citep{IzacardCHRBJG22} as our dense retriever for efficient and effective document retrieval. Our inference code also supports Google search engine~\citep{schmidt2014google} as the retriever server.
    
    \item \textbf{LLM Service:} We integrate \href{https://docs.sglang.ai/}{\texttt{SGLang}}\footnote{\url{https://docs.sglang.ai/}} as our LLM server, which provides compatibility with various state-of-the-art language models, including Qwen2-72B-Instruct~\citep{qwen2} and Llama3.3-70B-Instruct~\citep{llama3}. Moreover, we also support GPT series models\footnote{We use \texttt{gpt-4o-mini-2024-07-18} in our out-of-generalization experiments.}.
    
    \item \textbf{Inference Optimization:} Our implementation supports two high-performance inference engines: \texttt{SGLang} and \texttt{VLLM}, allowing users to optimize for different deployment scenarios and hardware configurations.
\end{itemize}

This modular architecture ensures both flexibility in model selection and efficiency in deployment, while maintaining robust performance across different configurations.

% \textbf{Other Details in our \modelname.}


\subsection{Dataset Details}
\begin{table*}[p!]
    \centering
    \setlength{\tabcolsep}{2.1pt}
    \begin{NiceTabular}{p{12em}ccccccc|cc|cc}
        \toprule
        \rowcolor[gray]{.9}
        \textbf{Datasets}                 & \Block[c]{1-7}{\textbf{Lemon}} &                 &              &                 &              &              &              & \Block[c]{1-2}{\textbf{CSCD-NS}} &               & \Block[c]{1-2}{\textbf{C2EC}} &               \\
        \textbf{Subsets}                  & \textit{Car}                   & \textit{Cot}    & \textit{Enc} & \textit{Gam}    & \textit{Mec} & \textit{New} & \textit{Nov} & \textit{dev}                     & \textit{Test} & \textit{dev}                  & \textit{Test} \\
        \midrule
        \textbf{Language}                 & \Block[c]{1-11}{Chinese}       &                 &              &                 &              &              &              &                                  &               &                               &               \\
        \midrule
        \textbf{All Sentences}            & 3,410                          & 1,026           & 3,434        & \phantom{0,}400 & 2,090        & 5,892        & 6,000        & 5,000                            & 5,000         & 1,995                         & 5,711         \\
        \textbf{Evaluation Sentences}     & 3,245                          & \phantom{0,}993 & 3,274        & \phantom{0,}393 & 1,942        & 5,887        & 6,000        & 5,000                            & 5,000         & 1,995                         & 5,711         \\
        \textbf{Erroneous Sentence Ratio} & 48.60                          & 44.41           & 48.59        & 37.66           & 46.60        & 49.96        & 50.23        & 46.28                            & 46.06         & 50.08                         & 49.92         \\
        \textbf{Average Length}           & 43.44                          & 40.11           & 39.95        & 32.81           & 39.18        & 25.15        & 36.24        & 57.45                            & 57.63         & 51.93                         & 41.88         \\
        \textbf{Average Error Character}  & \wz1.20                        & \wz1.10         & \wz1.12      & \wz1.10         & \wz1.13      & \wz1.11      & \wz1.13      & \wz1.10                          & \wz1.10       & \wz1.10                       & \wz1.16       \\
        \bottomrule
    \end{NiceTabular}
    \caption{
        The statistics of the datasets used in the experiments.
    }
    \label{tab:dataset_statistics}
\end{table*}

\textbf{In-domain Datasets.}
As shown in Table~\ref{tab:data_statistics}, we conduct extensive in-domain experiments on three single-hop and three multi-hop datasets.
For each dataset, we randomly sampled 6,000 instances as the training set, with sampling ratios detailed in Table~\ref{tab:data_statistics}. Overall, we utilize only 8\% of the original data as the training set.
For the in-domain test sets, we randomly sampled 1,000 instances as the test set.

\begin{table}[h]
\centering
\caption{Out-of-generalization Dataset Statistics.}
\label{tab:ood_statistics} 
% \resizebox{\linewidth}{!}{
% \begin{small}
\begin{tabular}{@{}lcc@{}}
\toprule
          & \textbf{FreshQA} & \textbf{Multihop-RAG} \\ \midrule
Data Size & 500              & 2556                  \\ \bottomrule
\end{tabular}
% }
% \end{small}
\end{table}

\textbf{Out-of-generalization Datasets.}
To comprehensively evaluate the plug-and-play capability of our \modelname in out-of-distribution generalization scenarios, we introduce two recent challenging datasets: FreshQA~\citep{VuI0CWWTSZLL24} and Multihop-RAG~\citep{multihop_rag}. 
The statistics of the OOD datasets are summarized in Table~\ref{tab:ood_statistics}.

% \subsection{Overall Algorithm}
% In this section, we present the inference process of \modelname for reference, as shown in the Algorithm~\ref{alg:infer_alg}.

% \begin{algorithm}[h]
    \caption{Inference Process of our \modelname}
    \label{alg:infer_alg}
    \begin{algorithmic}[1]
        \INPUT question $q$, the retrieval server (\textbf{Retriever}), the LLM server (\textbf{LLM}), the proxy model in our \modelname $\pi$, instruction for different agent (Reasoning Router, Information Filter, and Decision Maker).
        \OUTPUT The Answer.

        \STATE $a^1 \leftarrow \pi(q, \text{instruction}_1)$ \COMMENT{Reasoning Router agent}

        \IF{$a^1$==\texttt{[No Retrieval]}}
        \STATE $\text{Answer} \leftarrow \textbf{LLM}(q)$ \COMMENT{Direct Answering Strategy, \textcolor{red}{if $q$ does not require retrieval}}
        \ELSIF{$a^1$==\texttt{[Retrieval]<query content>}} 
        \STATE $\text{docs} \leftarrow \textbf{Retrieval}(\texttt{<query content>})$
        \STATE $\text{selected docs} (a^2) \leftarrow \pi(q, \text{docs}, \text{instruction}_2)$ \COMMENT{Information Filter agent}
        \STATE $\text{Answer} \leftarrow \textbf{LLM}(q, \text{selected docs})$ \COMMENT{Single-pass Strategy, \textcolor{red}{if $q$ requires retrieval and is simple question}}
        \ELSE
        \STATE Accumulated\_docs $\leftarrow \emptyset$ 
        \STATE $\text{Roadmap}\leftarrow \textbf{LLM}(q)$ \COMMENT{$a^1==\texttt{[Planning]}$}
        \STATE $a^3 \leftarrow \pi(q, \text{Roadmap}, \text{Accumulated\_docs}, \text{instruction}_3)$ \COMMENT{Decision Maker agent}
        \WHILE{$a^3 \neq \texttt{[LLM]}$ }
            \STATE $\text{docs} \leftarrow \textbf{Retrieval}(\texttt{<subquery content>}~\text{in}~a^3)$
            \STATE $\text{selected docs} (a^2) \leftarrow \pi(q, \text{docs}, \text{instruction}_2)$ \COMMENT{Information Filter agent}
            \STATE Accumulated\_docs $\leftarrow \{\text{Accumulated\_docs}\} \cup \{\text{selected docs}\}$ 
            \STATE $a^3 \leftarrow \pi(q, \text{Roadmap}, \text{Accumulated\_docs}, \text{instruction}_3)$ \COMMENT{Decision Maker agent}
        \ENDWHILE
        \STATE $\text{Answer} \leftarrow \textbf{LLM}(q, \text{Accumulated\_docs})$ \COMMENT{Multi-step Reasoning Strategy, \textcolor{red}{if $q$ requires retrieval and is complex}}
        \ENDIF
    \end{algorithmic}
\end{algorithm}


\section{Instructions and State Transition Function}\label{app:instructions}
\subsection{Instructions for Each Agent}
In this section, we details the state space and action space fo each agent in our \modelname.

\textbf{Reasoning Router.}
The Reasoning Router agent operates with state space $\mathcal{S}^1=\{q\}$, where $q$ represents the input question.
This agent is responsible for determining whether retrieval is necessary for the given question and assessing the question complexity when retrieval is needed.
For a question that does not require retrieval, this agent outputs \texttt{[No Retrieval]}.
If the retrieval is needed, the agent outputs one of the following actions based on the complexity of the question $q$: for simple questions requiring retrieval: \texttt{[Retrieval]<query content>}, initiating a single-pass retrieval-filter loop, where \texttt{<query content>} defines the space of possible queries;
for complex questions: \texttt{[Planning]}, triggering the multi-step reasoning strategy.
The specific examples are as follows:
\begin{tcolorbox}[title=Reasoning Router,width=\linewidth, breakable]
\begin{small}
\textcolor{red}{Instruction for Reasoning Router}\\
You are an intelligent assistant tasked with evaluating whether a given question requires further information through retrieval or needs planning to arrive at an accurate answer. You will have access to a large language model (LLM) for planning or answering the question and a retrieval system to provide relevant information about the query. \\

Instructions:\\
1. **Evaluate the Question**: Assess whether a precise answer can be provided based on the existing knowledge of LLM. Consider the specificity, complexity, and clarity of the question.\\
2. **Decision Categories:**\\
    - If the question is complex and requires a planning phase before retrieval, your response should be:\\
    \texttt{[Planning]}\\
    - If the question requests specific information that you believe the LLM does not possess or pertains to recent events or niche topics outside LLM's knowledge scope, format your response as follows: \\
    \texttt{[Retrieval] `YOUR QUERY HERE`}\\
    - If you think the LLM can answer the question without additional information, respond with:\\
    \texttt{[No Retrieval]}\\
3. **Focus on Assessment**: Avoid providing direct answers to the questions. Concentrate solely on determining the necessity for retrieval or planning.\\

\textcolor{red}{State of Reasoning Router}\\
Now, process the following question:\\
\\
Question: \{question\}\\

\textcolor{red}{Output (All possible Actions) of Reasoning Router}\\
\% For No Retrieval\\
\texttt{[No Retrieval]}\\
\% For Retrieval\\
\texttt{[Retrieval]<query content>} (for simple questions)\\
\texttt{[Planning]} (for complex questions)
\end{small}
\end{tcolorbox}


\textbf{Information Filter.}
The state space of Information Filter consists of the question $q$, the retrieved documents, and the current objective (if in \texttt{[Planning]} mode), i.e., $\mathcal{S}^2=\{q, \text{retrieved documents}\}$ for single-pass strategy (\texttt{Retrieval<query content}), or $\mathcal{S}^2=\{q, \text{retrieved documents}, \text{current objective}\}$ for multi-step reasoning strategy (\texttt{[Planning]}).
\begin{tcolorbox}[title=Information Filter,width=\linewidth, breakable]
\begin{small}
\textcolor{red}{Instruction for Information Filter}\\
You are an intelligent assistant tasked with analyzing the retrieved documents based on a given question and the current step's objectives. Your role is to determine the relevance of each document in relation to the question and the specified objectives.\\

Instructions:\\
1. **Analyze Relevance**: Evaluate each document whether it aligns with the objectives of the current retrieval step and contains a direct answer to the question.\\
2. **Thought Process**: Provide a brief analysis for each document, considering both the answer content and the retrieval objectives.\\
3. **Filter Documents**: After your thought process, generate a list of document indices indicating which documents to retain.\\

\textcolor{red}{State of Information Filter}\\
Now, process the following question:\\
\\
Current step's objectives: \{objective\} (only for \texttt{[Planning]} mode)\\

Question: \{question\}\\

Documents:
\{documents\}\\

\textcolor{red}{Output of Information Filter}\\
Thought: $<$Analysis of each documents$>$\\
Action: [$<$Selected document IDs$>$]
\end{small}
\end{tcolorbox}



\textbf{Decision Maker.}
The Decision Maker agent operates with state space $\mathcal{S}^3=\{q, \text{Accumulated Documents}, \text{Roadmap}\}$.
Based on the current state, this agent outputs one of two possible actions: \texttt{[Retrieval]<subquery content>} (requesting additional retrieval-filtering loop through the sub-query) or \texttt{[LLM]} (passing all accumulated documents to LLM for generating the final answer).
\begin{tcolorbox}[title=Decision Maker,width=\linewidth, breakable]
\begin{small}
\textcolor{red}{Instruction for Decision Maker}\\
You are an intelligent assistant tasked with determining the next appropriate action based on the provided existing documents, plan, and question. You have access to a large language model (LLM) for answering question and a retrieval system for gathering additional documents. Your objective is to decide whether to write a query for retrieving relevant documents or to generate a comprehensive answer using the LLM based on the existing documents and plan.\\

Instructions:\\
1. **Evaluate Existing Documents**: Assess the existing documents to determine if it is sufficient to answer the question.\\
2. **Follow the Plan**: Understand the next steps outlined in the plan.\\
3. **Decision Categories:**\\
    - If the existing documents is insufficient and requires additional retrieval, respond with:\\
        \texttt{[Retrieval] `YOUR QUERY HERE`}\\
    - If the existing documents is adequate to answer the question, respond with:\\
        \texttt{[LLM]}\\
4. **Focus on Action**: Do not answer the question directly; concentrate on identifying the next appropriate action based on the existing documents, plan, and question.\\

\textcolor{red}{State of Decision Maker}\\
Now, process the following question:\\
\\
Existing Documents: \{accumulated documents\}\\

Roadmap: \{roadmap\}\\

Question: \{question\}\\

\textcolor{red}{Output of Decision Maker}\\
Thought: {[}Your analysis for current situation (need retrieval for additional informations or use LLM to answer){]}\\
Action: {[}Your decision based on the analysis (\texttt{[Retrieval]<subquery content>} or \texttt{[LLM]}){]}
\end{small}
\end{tcolorbox}



\subsection{State Transition Function}\label{app:transition_details}
Given a state $s_t^i$ and an action $a_t^i$ in each agent $i\in \mathcal{N}$, the transition function $\mathcal{T}$ in our framework is deterministic.
Based on the three collaborative strategies introduced in Section~\ref{sec:strategy}, the state transitions are defined as follows:

\textbf{Direct Answering Strategy} (\texttt{[No Retrieval]}): In this strategy, the LLM directly generates the answer without retrieval, resulting in no state transitions between agents.

\textbf{Single-pass Strategy} (\texttt{[Retrieval]<query content>}): This strategy involves a state transition between the Reasoning Router and Information Filter agents:
\begin{equation}
    \mathcal{T}: \mathcal{S}^1=\{q\} \times \mathcal{A}=\{\texttt{[Retrieval]<query content>}\} \xrightarrow{\text{retrieval}} \mathcal{S}^2 = \{q, \text{retrieved documents}\}
\end{equation}
where $\mathcal{S}^1$ represents the initial state with the question $q$, and $\mathcal{S}^2$ represents the state for the Information Filter agent after retrieval. The Information Filter is responsible for filtering the helpful documents based on $\mathcal{S}^2$.


\textbf{Multi-Step Reasoning Strategy} (\texttt{[Planning]}): This strategy involves multiple state transitions in a cyclic manner:


\begin{itemize}[topsep=1pt, partopsep=1pt, leftmargin=12pt, itemsep=-1pt]
    \item Reasoning Router $\rightarrow$ Decision Maker: %Given the LLM-generated roadmap after \texttt{[Planning]}, we have the state $\mathcal{S}^3=\{q, \text{Accumulated Documents}, \text{Roadmap}\}$ in Decision Maker agent.
    \begin{equation}
        \mathcal{T}: \mathcal{S}^1=\{q\} \times \mathcal{A}=\{\texttt{[Planning]}\} \xrightarrow{\text{\texttt{[planning]}}} \mathcal{S}^3 = \{q, \text{Accumulated Documents}, \text{Roadmap}\},
    \end{equation}
    where the roadmap is generated by the LLM and the accumulated documents is empty for initial step.
    \item Decision Maker $\rightarrow$ Information Filter: %Given the current step's objective and retrieved documents based on the \texttt{[Retrieval]<subquery content>} outputted by the Decision Maker, we have the state $\mathcal{S}^2=\{q, \text{retrieved documents}, \text{current objective}\}$ in the Information Filter agent.
    \begin{align}
        \mathcal{T}: \mathcal{S}^3=\{q, \text{Accumulated Documents}, \text{Roadmap}\} \times \mathcal{A}=\{\texttt{[Retrieval]<subquery content>}, \nonumber \\ \text{current objective}\} \xrightarrow{\text{retrieval}} \mathcal{S}^2 = \{q, \text{retrieved documents}, \text{current objective}\},
    \end{align}
    where the current objective is generated by the Decision Maker agent in $\mathcal{S}^3$.
    \item Information Filter $\rightarrow$ Decision Maker: % Given the filtered documents, we need to update the exisiting documents (accumulated documents) in the state of Decision Maker agent: $\mathcal{S}^3=\{q, \text{New Accumulated Documents}, \text{Roadmap}\}$.
    \begin{align}
        \mathcal{T}: \mathcal{S}^2=\{q, \text{retrieved documents}, \text{current objective}\} \times \mathcal{A}=\{\text{Selected Documents}\} \nonumber \\ \xrightarrow{\text{filter}} \mathcal{S}^3_{\text{new}} = \{q, \text{Updated Accumulated Documents}, \text{Roadmap}\}.
    \end{align}
\end{itemize}

This retrieval-filter loop between the Decision Maker agent and the Information Filter agent continues until the Decision Maker outputting \texttt{[LLM]} or a termination condition is met. The state transitions in our \modelname are deterministic and well-defined, ensuring consistent behavior across the multi-agent system.


\section{Additional Experimental Results}\label{app:additional_experiments}

\subsection{Comparative Analysis of \modelname-ICL and \modelname-RL}
\begin{table}[h]
\centering
\caption{Comparative Analysis between \modelname-ICL and \modelname-RL.}
\label{tab:icl} 
\resizebox{\linewidth}{!}{
% \begin{small}
\begin{tabular}{@{}lc|cccccc|cc@{}}
\toprule
Method         & Proxy      & 2Wiki & HQA  & Musique & NQ   & PopQA & TQA  & Average & Efficiency \\ \midrule
\modelname-ICL & Qwen2-72B  & 54.1  & 62.5 & 45.5    & 63.4 & 45.7  & 82.9 & 59.01   & 10.7s      \\
\modelname-RL  & Qwen2-1.5B & 65.2  & 69.0 & 54.2    & 65.9 & 44.8  & 82.1 & 63.53   & 4.8s       \\ \bottomrule
\end{tabular}
}
% \end{small}
\end{table}
In this section, we further conduct a comprehensive comparison between \modelname-ICL (Qwen2-72B-Instruct) and \modelname-RL (Qwen2-1.5B), where \modelname-ICL (Qwen2-72B-Instruct) is used to generate seed data through rejection sampling in our supervised warm-up phase.
Our experimental results, as presented in Table~\ref{tab:icl}, reveal several important findings. First, \modelname-ICL demonstrates remarkable performance, surpassing all baseline methods across different datasets (as shown in Table~\ref{tab:main_result} and Table~\ref{tab:ood_results}). 
This result validates the effectiveness of our framework design, where collaboration among multiple agents enables effective alignment of the LLM and the retriever.
However, this approach faces practical limitations due to substantial inference overhead from multiple LLM queries, making it less suitable for efficient responses and edge deployment.

To address these limitations, we introduce a compact proxy that significantly reduces computational requirements while maintaining framework effectiveness.
Our analysis reveals that while \modelname-ICL performs well overall, it may not achieve optimal performance on more challenging tasks (e.g., 2WikiMultiHopQA, HotpotQA, and Musique).
Through reinforcement learning, we further optimize individual agent capabilities, leading to substantial improvements on the complex tasks.
In conclusion, our proxy-centric alignment framework demonstrates strong performance across both variants. While \modelname-ICL showcases the framework's effectiveness through few-shot learning, \modelname-RL offers practical advantages through reduced computational requirements and enhanced performance on challenging tasks. Our \modelname-RL successfully aligns the retriever and LLM without modifying either component, while facilitating edge deployment and robust performance across diverse scenarios.

\subsection{More Analysis in RL}
% strategy ratio
% depth distribution
\begin{figure}[h]
    \centering
    \includegraphics[width=0.5\linewidth]{images/strategy.pdf}
    \caption{Strategy Ratio in RL training process.}
    \label{fig:strategy_ratio}
\end{figure}

\textbf{Strategy Ratio during RL Training Process.}
As introduced in the Section~\ref{sec:strategy}, our \modelname incorporates three distinct strategies: Direct Answering Strategy (\texttt{[No Retrieval]}), Single-pass Strategy (\texttt{[Retrieval]<query content>}), and Multi-Step Reasoning Strategy (\texttt{[Planning]}), each designed for different question complexities. 
Figure~\ref{fig:strategy_ratio} reveals how \modelname dynamically adapts its strategy selection during the RL training process.

The evolution of strategy ratios shows a clear trend: the Multi-Step Reasoning Strategy gradually dominates the decision space, stabilizing at approximately 60-70\%, while the Single-pass Strategy decreases to around 30\%. The Direct Answering Strategy maintains a consistent but low ratio of about 5\%. This distribution pattern offers several insights into our framework's learning behavior:
\textbf{First}, the limited use of Direct Answering Strategy aligns with our experimental findings in Table~\ref{tab:main_result}, confirming that solely relying on the model's inherent knowledge is insufficient for complex question-answering tasks.
\textbf{Second}, the substantial proportion of Single-pass Strategy usage demonstrates our \modelname's ability to identify scenarios where simple external information retrieval suffices.
\textbf{Most notably}, the increasing preference for Multi-Step Reasoning Strategy indicates that our \modelname recognizes the importance of multi-step reasoning in handling complex queries effectively.
These learned ratios demonstrate that our framework effectively develops a balanced strategy selection mechanism.
By dynamically choosing appropriate strategies based on question complexity, our \modelname achieves a balance between computational efficiency and reasoning capability, making it well-suited for real-world applications.

% Figure~\ref{fig:strategy_ratio} illustrates the evolution of strategy selection ratios throughout the RL training process, revealing interesting patterns in how our \modelname learns to adaptively choose different strategies in RL.
% We observe that the Multi-Step Reasoning Strategy gradually becomes dominant, reaching and maintaining approximately 60-70\% of all decisions, while the Single-pass Strategy steadily decreases to around 30\%. The Direct Answering Strategy consistently maintains a low ratio of about 5\%.
% This distribution aligns with our intuition about efficient question-answering: 
% for many questions, Direct Answering Strategy (\texttt{[No Retrieval]}) is more efficient; however, relying solely on the model’s inherent knowledge remains challenging, as shown in Table~\ref{tab:main_result}.
% The significant proportion of Single-pass Strategy (\texttt{[Retrieval]<query content>}) indicates that our framework learns to identify cases where external information is necessary.
% The utilization of Multi-Step Reasoning Strategy (\texttt{[Planning]}) steadily increases, indicating that multi-step reasoning is essential for improving the performance of RAG systems on complex problems.
% These learned ratios demonstrate that our framework effectively develops a balanced strategy selection mechanism, avoiding the computational overhead of unnecessary retrievals while maintaining the capability to handle complex queries when needed. This adaptive behavior contributes to both efficiency and effectiveness in real-world applications.

\begin{figure}[h]
    \centering
    \includegraphics[width=\linewidth]{images/depth_distribution.pdf}
    \caption{Depth Distribution in Test set.}
    \label{fig:depth_dist}
\end{figure}

\textbf{Depth Distribution.}
Figure~\ref{fig:depth_dist} presents the depth distribution of reasoning processes across different datasets, revealing distinct patterns that align with the inherent complexity of each task. We observe three clear categories of reasoning depth requirements:
\textbf{(1) Simple Complexity (Depth 3-5):} Datasets like NaturalQuestions, PopQA, and TriviaQA show concentrated distributions around depths 3-5, indicating that most questions in these datasets can be effectively addressed with the Direct Answering Strategy (\texttt{[No Retrieval]}) and Single-pass Strategy (\texttt{[Retrieval]<query content>}). 
This aligns with the nature of these datasets, which primarily contain straightforward factual questions.
\textbf{(2) Mixed Complexity:} HotpotQA and 2WikiMultiHopQA exhibit multiple peaks in the depth distribution, with notable concentrations around depths 3-4 and depths 9-15, indicating a diverse range of question complexity. This bimodal distribution suggests that while some questions require simple reasoning steps, others need more complex reasoning chains.
\textbf{(3) Complex Complexity:} Musique displaies broader distributions with significant density at higher depths (9-13), particularly pronounced in their rightward skew. 
Musique's distribution is notably spread across higher depths, consistent with its design for multi-step reasoning questions.

These distributions validate our framework's adaptive capability in handling queries of varying complexity. The framework naturally adjusts its reasoning depth based on task requirements, demonstrating efficient resource utilization while maintaining the ability to perform deep reasoning when necessary.


\section{Additional Prompts}
In this section, we supplement additional prompts based on Appendix~\ref{app:instructions}.


\subsection{Roadmap}
In the multi-step reasoning strategy, we introduce an LLM-generated roadmap as high-level guidance for our proxy. 
The specific prompt and example are as follows:

\begin{tcolorbox}[title=An example of Roadmap,width=\linewidth, breakable]
\begin{small}
\textcolor{red}{Prompt for Roadmap}\\
You are an expert assistant tasked with analyzing the following question and formulating a detailed plan. You will utilize a retrieval system to gather relevant information in your planning. Your goal is to analysis the question and provide a structured sequence of actions to address it effectively.\\

Instructions:\\
1. **Question Analysis**: Identifying the core components of the question. Determine what key information we currently know and what additional information is needed through retrieval.\\
2. **Step By Step Planning**: Develop a detailed plan step by step. Focus on the planning process rather than providing direct answers.\\
3. **Focus on Planning**: Keep your response clear and structured, concentrating solely on the analysis and planning aspects.\\

Now, process the following question:\\
\\
Question: \{question\}\\

\textcolor{red}{Example of generated roadmap}\\
(Take \texttt{What nationality is the director of film The Caper Of The Golden Bulls?} as an example)\\
To answer the question, we need to find information about the director of the film "The Caper of the Golden Bulls." Then we should determine which nationality is the director born using the retrieval.\\
Step 1: Retrieve the relevant documents that mention the film `The Caper of the Golden Bulls.`\\
Step 2: Identify the director of the film from the retrieved documents.\\
Step 3: Retrieve the relevant information about `Which nationality is the director born?`.\\
Step 4: Provide the answer based on the retrieved information.
\end{small}
\end{tcolorbox}


\subsection{Evaluation}
In our experiments, we found that traditional evaluation metrics such as Exact Match (EM) are often inaccurate, as they strictly require identical generated answers.
To address this issue, following previous work~\citep{ZhengC00WZL0LXZ23,VuI0CWWTSZLL24}, we leverage an LLM to assess answer correctness by comparing the predicted answer with the ground truth.
The specific example is as follows:

\section{Evaluation}
We provide three sets of insights into this section, organised as \textit{findings (F*)}. We quantitatively study the effect of the adversarial and counterfactual perturbations on the performance of informal reasoners and autoformalisation methods. Then, we dive deeper into method variants. Finally, 
we analyse the nature of formalisation errors made by the models.

\subsection{Robustness Analysis}
\paragraph{\textbf{\emph{F1: Noise perturbations have a stronger effect on formalisation methods than informal \ac{LLM} reasoners.}}}
Table~\ref{tab:distraction_k4_formalisation} shows that, on average, the accuracy of both direct and \ac{CoT} informal reasoning remains between $73\%$ and $74\%$ in the face of added noise. While the autoformalisation method performs similarly to informal reasoners on the original dataset, its performance decreases between $4\%$ and $11\%$. The accuracy drops especially with logical (L) and tautological (T) distractions, whose logical language formats trick the \ac{LLM} into formalizing the noisy clauses. On the other hand, the linguistically complex and more natural sentences of encyclopedic distractions show a minor effect, suggesting that \acp{LLM} successfully avoids formalizing the more complicated sentences.

\paragraph{\textbf{\emph{F2: All \ac{LLM}-based reasoning methods suffer a drop for counterfactual perturbations.}}} % influence .}}}
Table~\ref{tab:distraction_k4_formalisation} shows that counterfactual statements cause a significant decrease in performance for both the informal reasoners and autoformalisation methods of between $12\%$ and $13\%$ on average. 
Moreover, this observation also holds for all tested models, i.e., none are robust towards counterfactual perturbations across every evaluated dimension. Even the strongest model, GPT 4o-mini, yields a performance of 63-68\%, which is relatively close to the random performance of 50\%. The high impact of counterfactual statements (the single ``not'' inserted) could be due to the inability of \acp{LLM} to overwrite prior knowledge with explicitly stated information or memorization of the answers. We study the error sources further in §\ref{subsec:errors}.  

\noindent \paragraph{\textbf{\emph{F3: Introducing multiple noise sentences has an effect only for logical distractions.}}}
We show the impact of introducing between one and four sentences for the two top-performing autoformalisation models in Figure~\ref{fig:length_distraction}. The figure shows similar trends with and without counterfactual perturbations.
As additional logical distractions are introduced, the model performance consistently decreases. Tautological (T) distractions lead to a decline in accuracy with a single disruptive sentence, yet adding more noise does not worsen the outcome. 
The tautological corpus introduces truth constants for all sentences as a persistent unseen logical construct. Given that this leads only to a decrease for a single occurrence, we can assume that a model can consistently handle the same unseen logical construct. In contrast, the logical corpus increases the chance of adding text, requiring new, previously unseen reasoning constructs for each added sentence. The impact of encyclopedic noise remains negligible, generalising F1 to $k$ sentences. Similarly, counterfactual perturbations remain much more effective for all settings, generalising F2.

\begin{table}[!t]
\small
\setlength{\modelspacing}{2pt}
\setlength{\tabcolsep}{1.7pt} % Default value: 6pt
\setlength{\belowrulesep}{4pt}
\begin{threeparttable}
    \centering
    \begin{tabular}{cc l r rrr @{\quad} rrrr}
\toprule
\multirow{2}{*}{} & \multirow{2}{*}{} & Reasoning & \multirow{2}{*}{O} & \multicolumn{3}{c}{Distraction} & \multicolumn{4}{c}{Counterfactual} \\
 & & Format & & E& L & T & $\text{O}_C$ & $\text{E}_C$& $\text{L}_C$ & $\text{T}_C$\\
\midrule
\multirow{6}{*}{\rotatebox{90}{Gemma-2}} & \multirow{3}{*}{\rotatebox{90}{9b}}
   & Informal (direct) & \textbf{0.78} & \textbf{0.80} & \textbf{0.79} & \textbf{0.77} & 0.58 & 0.52 & 0.50 & 0.59 \\
 & & Informal (CoT) & 0.72 & 0.78 & 0.73 & 0.76 & 0.61 & \textbf{0.57} & \textbf{0.60} & \textbf{0.66} \\
 & & Formal (FOL) & 0.62 & 0.58 & 0.52 & 0.53 & \textbf{0.63} & 0.52 & 0.46 & 0.46 \\[\modelspacing]
\cmidrule{2-11}
 & \multirow{3}{*}{\rotatebox{90}{27b}} 
   & Informal (direct) & 0.71 & 0.69 & \textbf{0.66} & \textbf{0.68} & 0.59 & 0.51 & 0.54 & 0.59 \\
 & & Informal (CoT) & 0.66 & 0.65 & 0.64 & 0.63 & 0.62 & 0.58 & \textbf{0.62} & \textbf{0.64} \\
 & & Formal (FOL) & \textbf{0.74} & \textbf{0.74} & 0.61 & 0.61 & \underline{\textbf{0.72}} & \underline{\textbf{0.67}} & 0.58 & 0.51 \\[\modelspacing]
\midrule
\multirow{6}{*}{\rotatebox{90}{Mistral}} & \multirow{3}{*}{\rotatebox{90}{7B}} 
   & Informal (direct) & 0.77 & \textbf{0.77} & 0.75 & \textbf{0.79} & \textbf{0.63} & \textbf{0.54} & \textbf{0.54} & \textbf{0.66} \\
 & & Informal (CoT) & \textbf{0.79} & 0.75 & \textbf{0.77} & 0.78 & 0.55 & 0.52 & \textbf{0.54} & 0.58 \\
 & & Formal (FOL) & 0.62 & 0.58 & 0.54 & 0.57 & 0.50 & \textbf{0.54} & 0.51 & 0.52 \\[\modelspacing]
\cmidrule{2-11}
 & \multirow{3}{*}{\rotatebox{90}{Small}} 
   & Informal (direct) & \textbf{0.77} & \textbf{0.76} & \textbf{0.76} & \textbf{0.75} & 0.61 & 0.51 & 0.56 & 0.59 \\
 & & Informal (CoT) & 0.72 & 0.72 & 0.72 & 0.71 & \textbf{0.62} & \textbf{0.59} & \textbf{0.62} & \textbf{0.68} \\
 & & Formal (FOL) & 0.68 & 0.59 & 0.53 & 0.64 & 0.54 & 0.55 & 0.49 & 0.51 \\[\modelspacing]
\midrule
\multirow{6}{*}{\rotatebox{90}{Llama-3.1}} & \multirow{3}{*}{\rotatebox{90}{8B}} 
   & Informal (direct) & 0.63 & 0.61 & 0.64 & 0.66 & 0.61 & \textbf{0.62} & 0.59 & 0.61 \\
 & & Informal (CoT) & 0.73 & \textbf{0.73} & \textbf{0.71} & \textbf{0.72} & \textbf{0.62} & 0.59 & \textbf{0.61} & \textbf{0.65} \\
 & & Formal (FOL) & \textbf{0.77} & 0.71 & 0.63 & 0.52 & 0.60 & 0.58 & 0.55 & 0.52 \\[\modelspacing]
\cmidrule{2-11}
 & \multirow{3}{*}{\rotatebox{90}{70B}} 
   & Informal (direct) & 0.77 & 0.74 & 0.74 & 0.73 & 0.62 & 0.53 & 0.56 & 0.64 \\
 & & Informal (CoT) & \textbf{0.78} & \textbf{0.75} & \textbf{0.76} & \textbf{0.76} & 0.64 & 0.61 & \textbf{0.66} & \underline{\textbf{0.73}} \\
 & & Formal (FOL) & 0.74 & 0.73 & 0.71 & 0.71 & \textbf{0.66} & \textbf{0.62} & 0.59 & 0.57 \\[\modelspacing]
 \midrule
\multirow{3}{*}{\rotatebox{90}{GPT}} & \multirow{3}{*}{\rotatebox{90}{4o-mini}} 
   & Informal (direct) & 0.78 & 0.77 & 0.79 & 0.79 & 0.64 & 0.61 & 0.61 & 0.63 \\
 & & Informal (CoT) & 0.80 & 0.80 & \underline{\textbf{0.81}} & \underline{\textbf{0.82}} & \textbf{0.68} & \textbf{0.63} & \underline{\textbf{0.68}} & \textbf{0.64} \\
 & & Formal (FOL) & \underline{\textbf{0.84}} & \underline{\textbf{0.82}} & 0.73 & 0.79 & 0.63 & 0.62 & 0.57 & 0.54 \\[\modelspacing]
 \midrule
\multicolumn{2}{c}{\multirow{3}{*}{\textbf{Avg}}} 
 & Informal (direct) & 0.74 & 0.73 & 0.73 & 0.73 & 0.61 & 0.55 & 0.56 & 0.62 \\
 & & Informal (CoT) & 0.74 & 0.74 & 0.73 & 0.74 & 0.62 & 0.58 & 0.62 & 0.65 \\
  & & Formal (FOL) & 0.72 & 0.68 &	0.61 & 0.62 & 0.61 & 0.59 & 0.54 & 0.52 \\
\bottomrule
\end{tabular}
\caption{Accuracies of informal and autoformalisation-based deductive reasoners. The best overall model per dataset is underlined; the best model version is marked in bold.}
\label{tab:distraction_k4_formalisation}
\end{threeparttable}
\end{table} 

\begin{figure}[!t]
    \centering
    \scriptsize
    \begin{tikzpicture}
        \begin{axis}[name=gpt,
            title={GPT-4o-mini},
            width=0.6\linewidth,
            height=0.6\linewidth,
            xlabel={\# Noise sentences},
            ylabel={Accuracy},
            xmin=-0.1, xmax=4.1,
            ymin=0.5, ymax=0.9,
            xtick={1,2,4},
            ytick={0.55, 0.6, 0.65, 0.75, 0.8, 0.85},
            title style={yshift=-0.6em},
            legend style={at={(1,-0.15)},
	           anchor=north,legend columns=-1},
            x label style={at={(axis description cs:1,-0.05)},anchor=north},
            y label style={at={(axis description cs:-0.15,0.5)},anchor=south},
            ymajorgrids=true,
            grid style=dashed,
        ]
            \addplot[color=blue, mark=square,]
                coordinates {
                (0,0.848076939582825)(1,0.823076903820038)(2,0.826923072338104)(4,0.821153819561005)
                };
            \addplot[color=red, mark=triangle,]
                coordinates {
                (0,0.848076939582825)(1,0.817307710647583)(2,0.801923096179962)(4,0.759615361690521)
                };
            \addplot[color=green, mark=diamond,] 
                coordinates {
                (0,0.848076939582825)(1,0.767307698726654)(2,0.769230782985687)(4,0.803846180438995)
                };
            \addplot[color=blue, mark=square*] 
                coordinates {
                (0,0.627777755260468)(1,0.622222244739533)(2,0.600000023841858)(4,0.633333325386047)
                };
            \addplot[color=red, mark=triangle*,] 
                coordinates {
                (0,0.627777755260468)(1,0.611111104488373)(2,0.611111104488373)(4,0.594444453716278)
                };
            \addplot[color=green, mark=diamond*,] 
                coordinates {
                (0,0.627777755260468)(1,0.572222232818604)(2,0.538888871669769)(4,0.555555582046509)
                };
                \legend{E,L,T,$\text{E}_C$, $\text{L}_C$ , $\text{T}_C$}
        \end{axis}

        \begin{axis}[name=llama, at={($(gpt.east)+(0.1cm,0)$)},anchor=west,
            title={Llama 3.1 70b},
            width=0.6\linewidth,
            height=0.6\linewidth,
            xmin=-0.1,, xmax=4.1,
            ymin=0.5, ymax=0.9,
            xtick={1,2,4},
            ytick={0.55, 0.6, 0.65, 0.75, 0.8, 0.85},
            title style={yshift=-0.6em},
            yticklabel=\empty,
            ymajorgrids=true,
            grid style=dashed,
        ]
            \addplot[color=blue, mark=square,]
                coordinates {
                (0,0.838461518287659)(1,0.817307710647583)(2,0.805769205093384)(4,0.817307710647583)
                };
            \addplot[color=red, mark=triangle,]
                coordinates {
                (0,0.838461518287659)(1,0.819230794906616)(2,0.803846180438995)(4,0.771153867244721)
                };
            \addplot[color=green, mark=diamond,]
                coordinates {
                (0,0.838461518287659)(1,0.803846180438995)(2,0.807692289352417)(4,0.805769205093384)
                };
            \addplot[color=blue, mark=square*]
                coordinates {
                (0,0.627777755260468)(1,0.622222244739533)(2,0.577777802944183)(4,0.594444453716278)
                };
            \addplot[color=red, mark=triangle*,]
                coordinates {
                (0,0.627777755260468)(1,0.583333313465118)(2,0.561111092567444)(4,0.577777802944183)
                };
            \addplot[color=green, mark=diamond*,]
                coordinates {
                (0,0.627777755260468)(1,0.627777755260468)(2,0.566666662693024)(4,0.577777802944183)
                };
        \end{axis}
    \end{tikzpicture}
    \caption{Influence of the number of noisy sentences for FOL.}
    \label{fig:length_distraction}
\end{figure}



\subsection{Impact of Method Design}
\paragraph{\textbf{\emph{F4: \ac{CoT} prompting is most impactful when both noise and counterfactual perturbations are applied.}}}
The accuracies for the individual \acp{LLM} in Table~\ref{tab:distraction_k4_formalisation} show that the impact of \ac{CoT} is negligible for noise-only datasets (first four columns). Meanwhile, the benefit from \ac{CoT} is most pronounced in the datasets that combine noise and counterfactual perturbations.
The better-performing informal prompting strategy for a model remains stable for all types of distractions. Still, the decline in performance due to counterfactuals leads to a less consistent preference for a specific prompting style.

\paragraph{\textbf{\emph{F5: The best-performing grammar differs per model and is unstable across data versions.}}}

The evaluation of different logical forms for formal \ac{LLM}-based reasoning in Table~\ref{tab:distraction_k4_logical_form} shows the preference of some models for specific syntactic formats.
Llama 3.1 70B has a considerable improvement of $12\%$ with TPTP syntax on the original set, while Llama 3.1 8B benefits from the R-FOL syntax. However, all grammars show a declining accuracy trend and increased syntax errors for noise perturbations, where the best grammar loses its advantage over the rest. 
When comparing the grammars on the counterfactual partitions, we observe that TPTP is consistently more robust than the standard first-order logic grammar. Here, GPT 4o-mini shows a reduction from $O$ to $O_C$ of $20\%$ for FOL and only $12\%$ for the TPTP grammar. Since this does not correlate with fewer syntax errors, the formalisation in TPTP prevents semantical errors for counterfactual premises. 
A positive reading of these results, especially the minor differences between FOL and R-FOL, is that autoformalisation \acp{LLM} can adapt to the grammar syntax prescribed in the prompt without further loss in performance.

\begin{table}[!t]
\small
\setlength{\modelspacing}{2pt}
\setlength{\tabcolsep}{1.7pt} % Default value: 6pt
\setlength{\belowrulesep}{4pt}
\begin{threeparttable}
    \centering
    \begin{tabular}{cc l r rrr @{\quad} rrrr}
\toprule
\multirow{2}{*}{} & \multirow{2}{*}{} & Grammar & \multirow{2}{*}{O} & \multicolumn{3}{c}{Distraction} & \multicolumn{4}{c}{Counterfactual} \\
 & & Syntax & & E& L & T & $\text{O}_C$ & $\text{E}_C$& $\text{L}_C$ & $\text{T}_C$\\
\midrule
\multirow{6}{*}{\rotatebox{90}{Llama-3.1}} & \multirow{3}{*}{\rotatebox{90}{8B}} 
   & FOL & 0.77 & \textbf{0.71} & 0.61 & \textbf{0.53} & 0.58 & \textbf{0.55} & 0.52 & \textbf{0.56} \\
 & & R-FOL & \textbf{0.78} & 0.69 & \textbf{0.62} & \textbf{0.53} & 0.58 & \textbf{0.55} & \textbf{0.54} & 0.52 \\
 & & TPTP & 0.73 & 0.67 & 0.55 & 0.51 & \textbf{0.68} & 0.54 & 0.46 & 0.51 \\[\modelspacing]
\cmidrule{2-11}
 & \multirow{3}{*}{\rotatebox{90}{70B}} 
   & FOL & 0.76 & 0.73 & 0.71 & \textbf{0.72} & 0.67 & 0.57 & 0.63 & 0.56 \\
 & & R-FOL & 0.76 & 0.73 & 0.67 & 0.71 & 0.64 & 0.57 & 0.53 & 0.64 \\
 & & TPTP & \underline{\textbf{0.88}} & \underline{\textbf{0.84}} & \underline{\textbf{0.81}} & \textbf{0.72} & \underline{\textbf{0.81}} & \underline{\textbf{0.68}} & \underline{\textbf{0.67}} & \underline{\textbf{0.68}} \\[\modelspacing]
\midrule
\multirow{3}{*}{\rotatebox{90}{GPT}} & \multirow{3}{*}{\rotatebox{90}{4o-mini}} 
   & FOL & \textbf{0.84} & \textbf{0.82} & \textbf{0.72} & \underline{\textbf{0.78}} & 0.64 & \textbf{0.63} & \textbf{0.61} & 0.51 \\
 & & R-FOL & \textbf{0.84} & 0.77 & 0.70 & \underline{\textbf{0.78}} & \textbf{0.72} & 0.56 & 0.54 & \textbf{0.63} \\
 & & TPTP & 0.83 & \textbf{0.82} & 0.71 & 0.71 & 0.69 & \textbf{0.63} & 0.57 & 0.57 \\
\bottomrule
\end{tabular}
\caption{Accuracies of different formalisation grammars for autoformalisation.}
\label{tab:distraction_k4_logical_form}
\end{threeparttable}
\end{table} 

\paragraph{\textbf{\emph{F6: Feedback does not help \acp{LLM} self-correct to mitigate robustness issues.}}}
\autoref{tab:distraction_k4_feedback} shows the results with different error recovery mechanisms. The results indicate that no feedback strategy emerges as a winner in the different datasets. 
All feedback variants reduce syntax errors for noise perturbations, but given the lack of a consistent increase in accuracy, the corrected formalisations are most likely to contain semantic errors still. 
The type of feedback message only has a minor influence on correcting syntax errors, whereas Llama 3.1 70b and GPT 4o-mini correct slightly more syntax errors with specific error messages. This finding aligns with \cite{huang2023large}, who also found that \acp{LLM} cannot consistently self-correct their reasoning after receiving relevant feedback.

\begin{table}[!ht]
\small
\setlength{\modelspacing}{2pt}
\setlength{\tabcolsep}{1.7pt} % Default value: 6pt
\setlength{\belowrulesep}{4pt}
\begin{threeparttable}
    \centering
    \begin{tabular}{cc l r rrr @{\quad} rrrr}
\toprule
\multirow{2}{*}{} & \multirow{2}{*}{} & \multirow{2}{*}{Feedback} & \multirow{2}{*}{O} & \multicolumn{3}{c}{Distraction} & \multicolumn{4}{c}{Counterfactual} \\
 & & & & E& L & T & $\text{O}_C$ & $\text{E}_C$& $\text{L}_C$ & $\text{T}_C$\\
\midrule
\multirow{8}{*}{\rotatebox{90}{Llama-3.1}} & \multirow{4}{*}{\rotatebox{90}{8B}} 
   & No recovery & 0.77 & \textbf{0.72} & 0.62 & 0.53 & 0.59 & 0.58 & 0.56 & \textbf{0.56} \\
 & & Error type & \textbf{0.79} & 0.71 & 0.63 & \textbf{0.56} & \textbf{0.66} & 0.54 & 0.52 & 0.51 \\
 & & Error message & 0.78 & 0.71 & \textbf{0.67} & 0.55 & 0.59 & 0.53 & \underline{\textbf{0.64}} & 0.49 \\
 & & Warning & 0.74 & 0.66 & 0.58 & 0.55 & 0.55 & \textbf{0.60} & 0.49 & 0.49 \\[\modelspacing]
\cmidrule{2-11}
 & \multirow{4}{*}{\rotatebox{90}{70B}} 
   & No recovery & \textbf{0.77} & \textbf{0.72} & \textbf{0.73} & 0.71 & \textbf{0.64} & 0.59 & \textbf{0.61} & 0.56 \\
 & & Error type & 0.72 & 0.70 & 0.72 & \textbf{0.73} & 0.62 & 0.56 & 0.60 & 0.58 \\
 & & Error message & 0.71 & 0.70 & \textbf{0.73} & 0.71 & \textbf{0.64} & 0.59 & 0.54 & \underline{\textbf{0.64}} \\
 & & Warning & 0.69 & \textbf{0.72} & 0.72 & 0.72 & 0.62 & \underline{\textbf{0.65}} & \textbf{0.61} & 0.63 \\[\modelspacing]
\midrule
\multirow{4}{*}{\rotatebox{90}{GPT}} & \multirow{4}{*}{\rotatebox{90}{4o-mini}} 
   & No recovery & \underline{\textbf{0.84}} & \underline{\textbf{0.82}} & 0.73 & 0.79 & 0.64 & \textbf{0.62} & 0.56 & \textbf{0.56} \\
 & & Error type & 0.83 & 0.79 & 0.74 & 0.76 & 0.67 & 0.57 & 0.56 & \textbf{0.56} \\
 & & Error message & \underline{\textbf{0.84}} & 0.78 & \underline{\textbf{0.77}} & \underline{\textbf{0.80}} & 0.62 & 0.59 & 0.56 & \textbf{0.56} \\
 & & Warning & \underline{\textbf{0.84}} & 0.75 & 0.73 & 0.76 & \underline{\textbf{0.70}} & 0.61 & \textbf{0.61} & 0.55 \\
 \bottomrule
\end{tabular}
\caption{Accuracies of error recovery strategies.}
\label{tab:distraction_k4_feedback}
\end{threeparttable}
\end{table} 

\subsection{Error Analysis}
\label{subsec:errors}
\paragraph{\textbf{\emph{F7: Autoformalisation increases syntax errors for noise perturbations.}}}
The low performance for noise perturbations correlates with more syntax errors for all models and distraction categories (cf. execution rates in Table~\ref{tab:appendix_k4_formalisation_exec}). The three worst-performing models (both Mistral models, Gemma-2 9b) generate, at best, for $37\%$  and, at worst, for only $4\%$ of the samples, a valid logical form.
Gemma-2 9b and Llama3.1 8b produce more syntax errors than the larger counterparts, suggesting that larger models are more robust towards noise perturbations. 
The accuracy of syntactically valid samples is higher than the informal reasoning methods for most distractions (Table~\ref{tab:appendix_k4_formalisation_vacc}), motivating informal reasoning as a backup strategy for formal reasoning. The error message feedback reveals two common syntax errors: 1) errors by models with an initial low execution rate exhibit issues with the template structure, including using incorrect keywords or adding conversational phrases;
2) perturbation-related errors, the most common of which is using undefined truth constants as part of tautological distractions. 

\paragraph{\textbf{\emph{F8: Autoformalisation increases semantic errors for counterfactuals.}}}
Unlike the introduced noise, counterfactual perturbations do not lead to more syntax errors. The execution rate in Table~\ref{tab:appendix_k4_formalisation_exec} is stable or improves for counterfactuals. However, we see a drop in accuracy for the counterfactual column $\text{O}_C$ in Table~\ref{tab:distraction_k4_formalisation} and can conclude that the number of logical forms with semantic errors has to increase. This suggests that the introduced negation is not correctly formalised. Looking at the warnings generated by the feedback mechanism, for GPT 4o-mini, $161$ warning messages are generated on the unperturbed data. $54$ of these were fixed with a single iteration. Not considering predicates and individuals as part of the context is the most frequent warning across all models. 