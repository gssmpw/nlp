\section{Upper bound for total out-degrees of nodes w.r.t. $\Ntwo \cup \Q$}
\label{sec:upper_bound_2}

In this section, we show an upper bound for the total out-degrees of nodes corresponding to strings that are elements of $\Ntwo \cup \Q \subseteq \M(T')$.

We first present properties of the strings in $\Ntwo \cup \Q$.
In particular, we focus on the strings in $\Ntwo \cup \Qn$,
as the strings in $\Qnotn$ are less important and can be handled in a trivial manner.

  \begin{lemma} \label{lem:exist2}
    Any $x \in \Ntwo \cup \Qn$ occurs in $T$.
  \end{lemma}

  \begin{proof}
    Since $x \in \Ntwo \cup \Qn$, $x \in \LeftM(T)$. Thus $x \in \Ntwo \cup \Qn$ occurs in $T$.
  \end{proof}

  \begin{lemma} \label{lem:sp124}
    For any $x \in \Ntwo \cup \Qn$ that is of Type $\rm{(i)}$, $\rm{(ii)}$ or $\rm{(iv)}$,
    there does not exist $y\in \Ntwo \cup \Qn$ such that $|y|>|x|$ and $P_{x_{G}}=P_{y_{F}}$, where $G,F \in \{L,R\}$.
  \end{lemma}

  \begin{proof}
    The case that $x \in \Ntwo$ follows from a symmetrical argument to Lemma~\ref{lem:sp123}, in which $y$ may belong to $\Ntwo$ or $\Qn$.
    \rhnote*{delete $\Qn = \M(T) \cap \M(T')$}{%
    Let us consider the case that $x \in \Qn$.
    }%
    Suppose that for $x \in \Qn$ which is of Type $\rm{(i)}$, $\rm{(ii)}$ or $\rm{(iv)}$,
    there is $y\in \Ntwo \cup \Qn$ such that $|y|>|x|$ and $P_{x_{G}}=P_{y_{F}}$, where $G,F \in \{L,R\}$.
    If $x_R$ is a suffix of $T'$, then there is no $y$ such that $|y|>|x|$ and $P_{x_{G}}=P_{y_{F}}$.
    From now on consider the case that $x_R$ is not a suffix of $T'$.
    %
    Let $b$ be the character immediately after $x_{G}$.
    Then, since $x$ is of Type $\rm{(i)}$, $\rm{(ii)}$ or $\rm{(iv)}$,
    character $b$ immediately follows every crossing occurrence of $x$ in $T'$.
    Note that $xb$ is a prefix of $y$.
    Due to Lemma~\ref{lem:exist2}, $y$ occurs in $T$,
    implying $xb$ also occurs in $T$.
    Thus the number of right-extensions of $x$ in $T'$
    is no more than the number of right-extensions of $x$ in $T$.
    However, this contradicts $x \in \Qn$.
  \end{proof}

  \begin{lemma} \label{lem:sp35}
    For any $x \in \Ntwo \cup \Qn$ that is of Type $\rm{(iii)}$ or $\rm{(v)}$, 
    there do not exist $y,z \in \Ntwo \cup \Qn$ with
    $|y|>|x|$ and $|z|>|x|$
    satisfying $P_{x_L}=P_{{y}_{G}}$ and $P_{x_R}=P_{{z}_{F}}$ simultaneously,
    where $G,F \in \{L,R\}$.
  \end{lemma}

  \begin{proof}
    If $x_R$ is a suffix of $T'$, then clearly there is no $z$ satisfying
    $|z|>|x|$ and $P_{x_R}=S_{{z}_{G}}$.
    In what follows, we consider the case that $x_R$ is not a suffix of $T'$.
    
    Suppose that for $x \in \Ntwo \cup \Qn$ which is of Type $\rm{(iii)}$ or $\rm{(v)}$, 
    there exist $y,z \in \Ntwo \cup \Qn$ with $|y|>|x|$, $|z|>|x|$
    that satisfy $P_{x_L}=P_{{y}_{G}}$ and $P_{x_R}=P_{{z}_{F}}$ at the same time,
    where $G,F \in \{L,R\}$.
    Let $b$ and $d$~($b \neq d$) be the character immediately after $x_L$ in $T'$ and the character immediately after $x_R$ in $T'$, respectively.
    By Lemma~\ref{lem:exist2}, $y$ and $z$ occur in $T$, and hence $xb$ that is a prefix of $y$ and $xd$ that is a prefix of $z$ also occur in $T$.
    Therefore, $x \in \RightM(T)$. However, if $x \in \Ntwo(T)$, this contradicts $x \notin \RightM(T)$.
    Also, if $x \in \Qn$, the number of right-extensions of $x$ in $T'$ do not increase from the number of right-extensions of $x$ in $T$. However, this contradicts $x \in \Qn$.
  \end{proof}

  \subsection{Correspondence between $\Ntwo \cup \Qn$ and $\M(T)$}

  For any $x \in \Ntwo \cup \Qn$ that is of Type $\rm{(i)}$, $\rm{(ii)}$ or $\rm{(iv)}$, then we associate $x$ with both $P_{x_L}$ and $P_{x_R}$.
  For any $x \in \Ntwo \cup \Qn$ that is of Type $\rm{(iii)}$ or $\rm{(v)}$, 
  \begin{itemize}
    \item if there exists $y\in \Ntwo \cup \Qn$ with $|y|>|x|$
  such that $P_{x_R}=P_{{y}_{G}}$ where $G \in \{L,R\}$, then we associate $x$ with $P_{x_L}$ (see Figure~\ref{fig:pxrex});
    \item if there exists $y\in \Ntwo \cup \Qn$ with $|y|>|x|$ such that $P_{x_L}=P_{{y}_{G}}$ where $G \in \{L,R\}$, then we associate $x$ with $P_{x_R}$ (see Figure~\ref{fig:pxlex});
    \item otherwise, we associate $x$ with both $P_{x_L}$ and $P_{x_R}$.
  \end{itemize}

  \begin{figure}[H]
    \centering
    \includegraphics[keepaspectratio,scale=0.33]{pxrex.pdf}
    \caption{When there exists $y\in \Ntwo \cup \Qn$ with $|y|>|x|$
      such that $P_{x_R}=P_{{y}_{G}}$, where $G \in \{L,R\}$.}
    \label{fig:pxrex}
  \end{figure}

  \begin{figure}[H]
    \centering
    \includegraphics[keepaspectratio,scale=0.33]{pxlex.pdf}
    \caption{When there exists $y\in \Ntwo \cup \Qn$ with $|y|>|x|$ such that $P_{x_L}=P_{{y}_{G}}$, where $G \in \{L,R\}$.}
    \label{fig:pxlex}
  \end{figure}

  
  By Lemmas~\ref{lem:sp124} and~\ref{lem:sp35}, each $x \in \Ntwo \cup \Qn$ corresponds to a distinct string $P_{x_G}$, where $G \in \{L,R\}$.
  Below, for each $x \in \Ntwo \cup \Qn$,
  we define $H(x)$ and $I(x)$ to which $x$ corresponds:

  \begin{definition} \label{def:H_x_I_x}
    For each $x \in \Ntwo \cup \Qn$
    associated to ${P_{x_L}}$, let $H(x)=\rrep_T({P_{x_L}})$.
    For each $x \in \Ntwo \cup \Qn$
    associated to ${P_{x_R}}$, let $I(x)=\rrep_T({P_{x_R}})$.
    See Figure~\ref{fig:H_xI_x}.
    \sinote*{added}{%
    When there is only one crossing occurrence of $x$ (i.e. $x_L = x_R$),
    only $H(x)$ is defined as above and $I(x)$ is undefined.
    }%
  \end{definition}
$H(x)$ (resp. $I(x)$) is undefined
for any $x \in \Ntwo \cup \Qn$ that is \emph{not} associated to ${P_{x_L}}$
(resp. ${P_{x_R}}$).

\sinote*{added}{%
By Lemma~\ref{lem:exist2} every $x \in \Ntwo \cup \Qn$ occurs in $T$,
and thus $H(x)$ and $I(x)$ are well defined
when $x$ is associated to $P_{x_L}$ and $P_{x_R}$, respectively.
}%
  
%  \begin{definition} \label{def:H_x_I_x}
%    For any $x \in \Ntwo \cup \Qn$,
%    let $H(x)=\rrep_T({P_{x_L}})$.
%    let $I(x)=\rrep_T({P_{x_R}})$.
%  \end{definition}
%
%  Note that when we associate $x$ with only $P_{x_L}$ or $P_{x_R}$ for $x$, we only define $H(x)$ or $I(x)$ for $x$ and 
%  even if we only define $H(x)$ or $I(x)$ for $x$, we state both in follow proofs, but the proof is not incomplete.


  \begin{figure}[H]
    \centering
    \includegraphics[keepaspectratio,scale=0.33]{H_xI_x.pdf}
    \caption{Illustration for $H(x)$ and $I(x)$ ($a\neq b, c\neq d$).}
    \label{fig:H_xI_x}
  \end{figure}

  \begin{lemma} \label{lem:H_x_I_x}
    For any $x \in \Ntwo \cup \Qn$, $H(x) \in \M(T)$ if $H(x)$ is defined,
    and $I(x) \in \M(T)$ if $I(x)$ is defined.
  \end{lemma}
  
%  \begin{lemma} \label{lem:H_x_I_x}
%    For any $x \in \Ntwo \cup \Qn$, $H(x), I(x) \in \M(T)$.
%  \end{lemma}

  \begin{proof}
    By Definition~\ref{def:H_x_I_x}, $H(x), I(x)\in \RightM(T)$.
    Therefore, it suffices for us to prove $H(x), I(x)\in \LeftM(T)$.
    For any $x \in \Ntwo \cup \Qn$, $x \in \LeftM(T)$.
    Since $P_{x_G} \: (G \in \{L,R\})$ is a prefix of $x$,
    we have $P_{x_G} \in \LeftM(T)$.
    Hence $H(x), I(x)\in \LeftM(T)$ holds.
  \end{proof}

  \begin{lemma} \label{lem:H_xH_y}
    For any $x,y \in \None \cup \Nnotv$ with $x \neq y$,
    let $\mathcal{L}$ be a list of $H(x)$, $I(x)$, $H(y)$, $I(y)$
    which are defined.
    Then the elements in $\mathcal{L}$ differ from each other.
  \end{lemma}

  \begin{proof}
    By a symmetrical argument to Lemma~\ref{lem:U_xU_y}.
  \end{proof}

  \subsection{Upper bound w.r.t. $\Ntwo \cup \Q$}

  \begin{lemma} 
    \label{lem:dt2}
    $\sum_{x \in \Ntwo \cup \Q}\D_{T'}(x) \le 3\size+2$.
  \end{lemma}

  \begin{proof}
    Below, we consider all the four possible cases depending on whether $x \in \Ntwo$ or $x \in \Qn$, and whether $H(x), I(x) \neq T$. 

    \noindent {\large \textbf{When $x\in \Ntwo$ and $H(x), I(x) \neq T$:}}
    \begin{itemize}
    \item
    First, we consider the case that $x$ is associated with both $H(x)$ and $I(x)$.
    Since $x \in \Ntwo$, then
      $x \notin \RightM(T)$.
    Therefore, the number of characters that are immediately after $x$ in $T$ is at most one.
    Moreover, there are at most two distinct characters immediately after the crossing occurrences of $x$.
    Hence, there are at most three distinct characters immediately after $x$ in $T'$, namely we have 
    \begin{equation}\label{equ:equ1}
      \D_{T'}(x) \le 3.
    \end{equation}
    In addition, since ${H(x),I(x) \ne T}$, it holds that $\D_{T}(H(x)),\D_{T}(I(x)) \ge 1$.
    By Inequality~\ref{equ:equ1}, we get $\D_{T'}(x) \le 3 \le \D_{T}(H(x))+\D_{T}(I(x))+1 \le 2\D_{T}(H(x))+2\D_{T}(I(x)).$

    \item Second, we consider the case that $x$ is associated with only one of $H(x)$ or $I(x)$.
    \begin{itemize}
    \item
    Assume that we associate $x$ with $H(x)$.
    Since $x \in \Ntwo$, then
      $x \notin \RightM(T)$.
    Therefore, the number of characters immediately after $x$ in $T$ is at most one.
    \rhnote*{added the case $x$ has only one crossing occurrence}{%
    In this case, we do not associate $x$ with $I(x)$, hence, $x$ has only one crossing occurrence or there exists $y\in \Ntwo \cup \Qn$ such that $|y|>|x|$ and $P_{x_R}=P_{{y}_{G}}$ where $G \in \{L,R\}$.
    When $x$ has only one crossing occurrence, there are at most one character immediately after the crossing occurrence of $x$.
    When there exists  $y\in \Ntwo \cup \Qn$ such that $|y|>|x|$ and $P_{x_R}=P_{{y}_{G}}$ where $G \in \{L,R\}$, then such $y$ occurs in $T$ due to Lemma~\ref{lem:exist2}.
    Therefore, although there are at most two distinct characters immediately after the crossing occurrences of $x$, one of them is the character immediately after $x$ in $T$ as shown in Figure~\ref{fig:xaex}.
    }%
    Hence, we have 
    \begin{equation}\label{equ:equ2}
      \D_{T'}(x) \le 2.
    \end{equation}
    In addition, since ${H(x) \ne T}$, then $\D_{T}(H(x)) \ge 1$ holds.
    By Inequality~\ref{equ:equ2}, we get $\D_{T'}(x) \le 2 \le \D_{T}(H(x))+1 \le 2\D_{T}(H(x)).$
    
    \item
    Let us assume that we associate $x$ with $I(x)$. In the same way as we associate $x$ with $H(x)$, we get $\D_{T'}(x) \le 2 \le \D_{T}(I(x))+1 \le 2\D_{T}(I(x))$.
    \end{itemize}
    \end{itemize}

    \begin{figure}[H]
      \centering
      \includegraphics[keepaspectratio,scale=0.33]{xaex.pdf}
      \caption{$xa$ occurs in $T$, where $a$ is the character immediately after the crossing occurrence $x_R$.}
      \label{fig:xaex}
    \end{figure}
    

    \noindent {\large \textbf{When $x\in \Ntwo$ and $H(x) = T$ or $I(x) = T$:}}
    \begin{itemize}
    \item
    Let $H(x) = T$. Now that $H(x)$ is defined, $x$ is associated with $H(x)$.
    \begin{itemize}
     \item First, we consider the case that we associate $x$ with both $H(x)$ and $I(x)$.
    In the same way as in Inequality~\ref{equ:equ1}, we get $\D_{T'}(x) \le 3$.
    In addition, since $H(x)=T$ and Lemma~\ref{lem:H_xH_y} holds, $I(x) \neq T$ and thus $\D_{T}(I(x)) \ge 1$ holds.
    Hence, we have $\D_{T'}(x) \le 3 \le \D_{T}(I(x))+2 \le 2\D_{T}(I(x))+2 \le 2\D_{T}(H(x))+2\D_{T}(I(x))+2$.
    \item Second, we consider the case that we only associate $x$ with $H(x)$.
    Since $H(x)=T$, $\D_{T}(H(x)) =0$ holds.
    In the same way as in Inequality~\ref{equ:equ2}, we get $\D_{T'}(x) \le 2$.
    Thus, we have $\D_{T'}(x) \le 2 \le 2\D_{T}(H(x))+2$.
    \end{itemize}

    \item
    Let $I(x) = T$. Now that $I(x)$ is defined, $x$ is associated with $I(x)$.
    In the same way as in the case for $H(x)=T$, 
    we get $\D_{T'}(x) \le 3 \le \D_{T}(H(x))+2 \le 2\D_{T}(H(x))+2 \le 2\D_{T}(H(x))+2\D_{T}(I(x))+2$ in the case that we associate $x$ with both $H(x)$ and $I(x)$,
    and we get $\D_{T'}(x) \le 2 \le 2\D_{T}(I(x))+2$ in the case that we only associate $x$ with $I(x)$.
   \end{itemize}

%   \medskip 
   \noindent {\large \textbf{When $x\in \Qn$ and $H(x), I(x) \neq T$:}}    
    Here, we analyze $\D_{T'}(x)-\D_T{(x)}$ since $x \in \Qn$.
    \begin{itemize}
    \item
    First, we consider the case that we associate $x$ with both $H(x)$ and $I(x)$.
    There are at most two distinct characters immediately after the crossing occurrences of $x$.
    Hence,
    \begin{equation}\label{equ:equ3}
      \D_{T'}(x)-\D_T{(x)} \le 2.
    \end{equation}
    In addition, since ${H(x),I(x) \ne T}$, then $\D_{T}(H(x)),\D_{T}(I(x)) \ge 1$ holds.
    By Inequality~\ref{equ:equ3}, we get $\D_{T'}(x)-\D_T{(x)} \le 2 \le \D_{T}(H(x))+\D_{T}(I(x)) \le \D_{T}(H(x))+\D_{T}(I(x))$.

    \item
    Second, we consider the case that we associate $x$ with only one of $H(x)$ or $I(x)$.
    Here, let us assume that we associate $x$ with $H(x)$.
    \rhnote*{added the case $x$ has only one crossing occurrence}{%
    In this case, we do not associate $x$ with $I(x)$, hence, $x$ has only one crossing occurrence or there exists $y\in \Ntwo \cup \Qn$ such that $|y|>|x|$ and $P_{x_R}=P_{{y}_{G}}$ where $G \in \{L,R\}$.
    When $x$ has only one crossing occurrence, there are at most one character immediately after the crossing occurrence of $x$.
    When there exists  $y\in \Ntwo \cup \Qn$ such that $|y|>|x|$ and $P_{x_R}=P_{{y}_{G}}$ where $G \in \{L,R\}$, then such $y$ occurs in $T$ due to Lemma~\ref{lem:exist2}.
    Therefore, although there are at most two distinct characters immediately after the crossing occurrences of $x$, one of them is the character immediately after $x$ in $T$ as shown in Figure~\ref{fig:xaex}.
    }%
    Hence, we have
    \begin{equation}\label{equ:equ4}
      \D_{T'}(x)-\D_T{(x)} \le 1.
    \end{equation}
    In addition, since ${H(x) \ne T}$, then $\D_{T}(H(x)) \ge 1$ holds.
    By Inequality~\ref{equ:equ4}, we get $\D_{T'}(x)-\D_T{(x)} \le 1 \le \D_{T}(H(x))$.
    In the case that we associate $x$ with $I(x)$, in the same way as we associate $x$ with $H(x)$, we get $\D_{T'}(x)-\D_T{(x)} \le 1 \le \D_{T}(I(x))$.
    \end{itemize}

%   \medskip 
   \noindent {\large \textbf{When $x\in \Qn$ and $H(x) = T$ or $I(x) = T$:}}
   Here, we analyze $\D_{T'}(x)-\D_T{(x)}$ since $x \in \Qn$.
   \begin{itemize}
    \item
    Let $H(x) = T$. Since $H(x)$ is defined, $x$ is associated with $H(x)$.
    \begin{itemize}  
     \item First, let us consider the case that we associate $x$ with both $H(x)$ and $I(x)$.
    In the same way as in Inequality~\ref{equ:equ3}, we get $\D_{T'}(x)-\D_T{(x)} \le 2$.
    In addition, since $H(x)=T$ and Lemma~\ref{lem:H_xH_y} holds, $I(x) \neq T$ and thus $\D_{T}(I(x)) \ge 1$ holds.
    Hence $\D_{T'}(x)-\D_T{(x)} \le 2 \le \D_{T}(I(x)) + 1 \le \D_{T}(H(x)) + \D_{T}(I(x)) + 1$.

     \item Second, let us consider the case that we only associate $x$ with $H(x)$.
    Since $H(x)=T$, $\D_{T}(H(x))=0$ holds.
    In the same way as in Inequality~\ref{equ:equ4}, we get $\D_{T'}(x)-\D_T{(x)} \le 1$.
    Thus, we have $\D_{T'}(x)-\D_T{(x)} \le 1 \le \D_{T}(H(x))+1$.
    \end{itemize}

   \item
    Let $I(x) = T$. Since $I(x)$ is defined, $x$ is associated with $I(x)$.
    In the same way as in the case for $H(x)=T$, 
    we get $\D_{T'}(x)-\D_T{(x)} \le 2 \le \D_{T}(H(x)) + 1 \le \D_{T}(H(x)) + \D_{T}(I(x)) +1$ in the case that we associate $x$ with both $H(x)$ and $I(x)$,
    and we get $\D_{T'}(x)-\D_T{(x)} \le 1 \le \D_{T}(I(x))+1$ in the case that we only associate $x$ with
    \rhnote*{deleted "only one of $H(x)$"}{%
    $I(x)$.
    }%
   \end{itemize}
    
    \begin{table}[h]
      \centering
      \caption{Upper bounds for each case of Lemma~\ref{lem:dt2}.} 
      \label{inequality}
      \fontsize{9pt}{10pt}\selectfont
      \begin{tabular}{|c|c|c|} \hline
        & When $H(x) \ne T \land I(x) \ne T$ & When $H(x) = T \lor I(x) = T$ \\ \hline
        $\D_{T'}(x)-\D_T{(x)} \: (x \in \Qn)$ & $\leq \D_T{(H(x))}+\D_T{(I(x))}$ & $\leq \D_T{(H(x))}+\D_T{(I(x))}+1$ \\\hline
        $\D_{T'}(x) \: (x \in \Ntwo)$ & $\leq 2(\D_T{(H(x))}+\D_T{(I(x))})$ & $\leq 2(\D_T{(H(x))}+\D_T{(I(x))})+2$ \\ \hline
      \end{tabular}
    \end{table}

%  For simplicity, we state both $H(x)$ and $I(x)$ even if we only associate $x$ with $H(x)$ or $I(x)$.
  %  For instance, if we only associate $x$ with $H(x)$, let $\D_T{(I(x))}$ be zero.

  \noindent {\large \textbf{Wrapping up:}}    
  Table~\ref{inequality} summarizes the bounds obtained above.
  For simplicity,
  let $\D_T{(I(x))} = 0$ when $I(x)$ is undefined,
  and let $\D_T{(H(x))} = 0$ when $H(x)$ is undefined.
  Note that this does not affect our upper bound analysis,
  since no maximal repeats in $T'$ are associated to the undefined $H(x)$'s and $I(x)$'s.
  %
  By Lemma~\ref{lem:H_xH_y}, there is at most one string $x$ such that $H(x) = T$ or $I(x) = T$.
  Thus, by using Lemma~\ref{lem:H_xH_y} and summing up the values in Table~\ref{inequality}, 
  we obtain $\sum_{x\in \Ntwo}\D_{T'}(x) + \sum_{x\in \Qn}(\D_{T'}(x)-\D_T(x)) \le 2\size+2$.
  %
  Also, since the number of out-edges of $x \in \Qnotn$ does not increase,
  we get
  $\sum_{x\in \Qnotn}\D_{T'}(x) + \sum_{x\in \Qn}\D_{T}(x) \le 
  \sum_{x\in \Qnotn}\D_{T}(x) + \sum_{x\in \Qn}\D_{T}(x) \le
  \sum_{x\in \Q}\D_{T}(x) \le \sum_{x\in \M(T)}\D_{T}(x) = \size$.
  %
  By adding $\sum_{x\in \Ntwo}\D_{T'}(x) + \sum_{x\in \Qn}(\D_{T'}(x)-\D_T(x)) \le 2\size+2$, 
  we get $\sum_{x \in \Ntwo \cup \Q}\D_{T'}(x) \le 3\size+2$.
  \end{proof}
