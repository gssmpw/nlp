\section{Preliminaries}

\subsection{Strings}

Let $\Sigma$ be an \emph{alphabet} of size $\sigma$.
An element of $\Sigma^*$ is called a \emph{string}.
%The set of characters that occur in a string $T$ is denoted by ${\Sigma_T}$.
For a string $T \in \Sigma^*$, the length of $T$ is denoted by $|T|$.
The \emph{empty string}, denoted by $\varepsilon$, is the string of length $0$.
%Let $\Sigma^+ = \Sigma^* \setminus \{\varepsilon\}$.
For any non-negative integer $n \geq 0$,
let $\Sigma^n$ denote the set of strings of length $n$.
%
For any two strings $S$ and $T$,
let $\ed(S,T)$ denote the edit distance between $S$ and $T$.
For any string $T$ and a non-negative integer $\ell \geq 0$,
let $\mathcal{K}(T,\ell) = \{S \mid \ed(S,T) = \ell\}$.

For string $T = uvw$, $u$, $v$, and $w$ are called a \emph{prefix}, \emph{substring},
and \emph{suffix} of $T$, respectively.
The sets of prefixes, substrings, and suffixes of string $T$ are denoted by
$\Prefix(T)$, $\Substr(T)$, and $\Suffix(T)$, respectively.
For a string $T$ of length $n$, $T[i]$ denotes the $i$th character of $T$
for $1 \leq i \leq n$,
and $T[i..j] = T[i] \cdots T[j]$ denotes the substring of $T$ that begins at position $i$ and ends at position $j$ on $T$ for $1 \leq i \leq j \leq n$.

\subsection{Maximal substrings and maximal repeats}

A substring $w \in \Substr(T)$ of $T$ is said to be
\emph{left-maximal} if (1) $w \in \Prefix(T)$
or (2) there exist two distinct characters $a,b \in \Sigma$ such that
$aw, bw \in \Substr(T)$,
and it is said to be \emph{right-maximal}
if (1) $w \in \Suffix(T)$ or (2) there exist two distinct characters $a,b \in \Sigma$ such that $wa, wb \in \Substr(T)$.
These substrings $w$ that occur at least twice in $T$
are also called \emph{left-maximal repeats}
and \emph{right-maximal repeats} in $T$, respectively.

Let $\LeftM(T)$ and $\RightM(T)$ denote the sets of left-maximal
and right-maximal substrings in $T$.
Let $\M(T) = \LeftM(T) \cap \RightM(T)$.
The elements in $\M(T)$ are called \emph{maximal substrings} in $T$,
and the elements in $\M(T) \setminus \{T\}$ are called
\emph{maximal repeats} in $T$.
A character $a \in \Sigma$ is said to be 
a \emph{right-extension} of a maximal repeat $w$ of $T$
if $wa \in \Substr(T)$.

%Let $\LeftM(T)$ and $\RightM(T)$ denote the sets of left-maximal
%and right-maximal repeats in $T$.
%Let $\M(T) = \LeftM(T) \cap \RightM(T)$.
%The elements in $\M(T)$ are called \emph{maximal-repeats} in $T$.
%A character $a \in \Sigma$ is said to be 
%a \emph{right-extension} of a maximal repeat $w$ of $T$
%if $wa \in \Substr(T)$.

For any substring $w$ of a string $T$,
we define its \emph{left-representation} and \emph{right-representation}
by $\lrep_T(w) = \alpha w$ and $\rrep_T(w) = w \beta$,
where $\alpha, \beta \in \Sigma^*$ are the shortest strings
such that $\alpha w$ is left-maximal in $T$
and $w\beta$ is right-maximal in $T$, respectively.

\subsection{CDAWGs}

The \emph{compact directed acyclic word graph} (\emph{CDAWG}) of a string $T$,
denoted $\CDAWG(T)$,
is the minimal DFA that recognizes all substrings of $T$,
in which each transition (edge) is labeled by a non-empty substring of $T$.
$\CDAWG(T)$ has a unique source that represents the empty string $\varepsilon$
and a unique sink that represents $T$.
All the other internal nodes represent the maximal repeats in $T$,
namely, the set of the longest strings represented by the nodes of $\CDAWG(T)$ are equal to $\M(T)$.
See Figure~\ref{fig:cdawg} for a concrete example of CDAWGs.

The \emph{size} of $\CDAWG(T)$ for a string $T$ of length $n$
is the number $\size(T)$ of edges in $\CDAWG(T)$,
which is equal to the number of right-extensions of maximal repeats in $T$.

In what follows, we will identify
maximal substrings with CDAWG nodes,
and right-extensions of maximal repeats with CDAWG edges,
respectively.

For each $x \in \M(T)$,
let $\D_T(x)$ denote the number of out-edges of of node $x$ in $\CDAWG(T)$.
It is clear that $\size(T) = \sum_{x \in \M(T)} \D_T(x)$.

\begin{figure}[tbh]
  \centering
  \includegraphics[keepaspectratio,scale=0.35]{cdawg.pdf}
  \caption{Illustration for $\CDAWG(T)$ of string $T=\mathrm{(ab)^2 c(ab)^2d}$. The longest strings represented by the nodes of $\CDAWG(T)$ are the maximal substrings in $\M(T) = \{\mathrm{\varepsilon, ab, (ab)^2, (ab)^2 c(ab)^2d}\}$.}
  \label{fig:cdawg}
\end{figure}

\subsection{Sensitivity of CDAWG size and our results}

Using the measure $\size$, we define
the worst-case multiplicative \emph{sensitivity} of the CDAWG
with edit operations (resp. insertion, deletion, and substitution) by:
\begin{eqnarray*}
  \MSIns(\mathsf{\size}, n) & = & \max_{T \in \Sigma^n, T' \in \mathcal{K}(T,1) \cap \Sigma^{n+1}} \{\size(T')/ \size(T)\}, \\
  \MSDel(\size, n) & = & \max_{T \in \Sigma^n, T' \in \mathcal{K}(T,1) \cap \Sigma^{n-1}} \{\size(T')/\size(T)\}, \\
  \MSSub(\size, n) & = & \max_{T \in \Sigma^n, T' \in \mathcal{K}(T, 1) \cap \Sigma^{n}} \{\size(T')/\size(T)\}.
\end{eqnarray*}

We will prove the following:
\begin{theorem} \label{theo:CDAWG_sensitivity}
  For any string $T$ of length $n$,
  $\MSIns(\mathsf{\size}, n) \leq (8\size+4)/\size$,
  $\MSDel(\mathsf{\size}, n) \leq (8\size+4)/\size$,
  $\MSSub(\mathsf{\size}, n) \leq (8\size+4)/\size$ hold,
  where $\size = \sum_{x \in \M(T)}\D_T(x)$.
\end{theorem}

Our proof for Theorem~\ref{theo:CDAWG_sensitivity} 
handles all cases of insertions, deletions, and substitutions.

