\documentclass[11pt]{article}

\usepackage{a4wide}
\usepackage{authblk}
\usepackage{amsmath,amssymb}
\usepackage{amsthm}
\usepackage[dvipdfmx]{graphicx}
\usepackage{here}
\usepackage{url}

\setlength{\topmargin}{-10mm}

\pdfinclusionerrorlevel=1
\pdfminorversion=7

\pdfoutput=1

\usepackage{thmtools}

\newtheorem{theorem}{Theorem}
\newtheorem{lemma}{Lemma}
\newtheorem{corollary}{Corollary}

\theoremstyle{definition}
\newtheorem{definition}{Definition}
\newtheorem{problem}{Problem}
\newtheorem{remark}{Remark}
\newtheorem{example}{Example}
\newtheorem*{claim}{Claim}

%\renewenvironment{proof}{\begin{trivlist} \item{\textit{Proof.}}}{\end{trivlist}}

\usepackage[final,mode=multiuser]{fixme}
%\usepackage[draft,mode=multiuser]{fixme}
\fxsetup{theme=color}
\definecolor{fxtarget}{rgb}{0.0000,0.0000,0.7000}
\fxuseenvlayout{colorsig}
\fxusetargetlayout{color}
\FXRegisterAuthor{rh}{ahr}{RH}
\FXRegisterAuthor{hf}{ahf}{HF}
\FXRegisterAuthor{si}{asi}{SI}
\fxsetface{env}{}

%%%%%%%%%%%%%%%%%%%%%%%%%%%%%%%%%%%%%%%%%%%%%%%%%%%

\newcommand{\Prefix}{\mathsf{Prefix}}
\newcommand{\Substr}{\mathsf{Substr}}
\newcommand{\Suffix}{\mathsf{Suffix}}
\newcommand{\occ}{\mathsf{occ}}
\newcommand{\Long}{\mathsf{long}}
\newcommand{\CDAWG}{\mathsf{CDAWG}}

\newcommand{\lrep}{\mathsf{lexp}}
\newcommand{\rrep}{\mathsf{rexp}}

%\newcommand{\lext}[2]{\overset{#2}{\overleftarrow{#1}}}
%\newcommand{\lext}[2]{\overset{#2}{\overrightarrow{#1}}}

\newcommand{\EqrL}{\equiv^{\mathrm{L}}}
\newcommand{\EqrR}{\equiv^{\mathrm{R}}}
\newcommand{\EqcL}[1]{[{#1}]^{\mathrm{L}}}
\newcommand{\EqcR}[1]{[{#1}]^{\mathrm{R}}}
\newcommand{\EqcB}[1]{[{#1}]}

\newcommand{\LeftM}{\mathsf{LeftM}}
\newcommand{\RightM}{\mathsf{RightM}}
\newcommand{\M}{\mathsf{M}}
\newcommand{\D}{\mathsf{d}}

\newcommand{\size}{\mathsf{e}}

\newcommand{\MSIns}{\mathsf{MS}_{\mathrm{Ins}}}
\newcommand{\MSDel}{\mathsf{MS}_{\mathrm{Del}}}
\newcommand{\MSSub}{\mathsf{MS}_{\mathrm{Sub}}}

\newcommand{\N}{\mathsf{N}}
\newcommand{\None}{\mathsf{N}_{\mathrm{1}}}
\newcommand{\Ntwo}{\mathsf{N}_{\mathrm{2}}}
\newcommand{\Nthree}{\mathsf{N}_{\mathrm{3}}}
\newcommand{\Nv}{\mathsf{N}_{\mathrm{3B}}}
\newcommand{\Q}{\mathsf{Q}}
\newcommand{\Qn}{\mathsf{Q}_{\mathrm{1}}}
\newcommand{\Qnotn}{\mathsf{Q}_{\mathrm{2}}}
\newcommand{\Nnotv}{\mathsf{N}_{\mathrm{3A}}}

\newcommand{\ed}{\mathsf{ed}}

\newcommand{\polylog}{\mathrm{polylog}}

%%%%%%%%%%%%%%%%%%%%%%%%%%%%%%%%%%%%%%%%%%%%%%%%%%%


\begin{document}

\title{Constant sensitivity on the CDAWGs}

\author[1]{Rikuya~Hamai}
\author[1]{Hiroto~Fujimaru}
\author[2]{Shunsuke~Inenaga}

\affil[1]{Department of Information Science and Technology, Kyushu University, Japan}
\affil[2]{Department of Informatics, Kyushu University, Japan}

\date{}
\maketitle

\begin{abstract}
\emph{Compact directed acyclic word graphs} (\emph{CDAWGs}) [Blumer et al. 1987] are a fundamental data structure on strings with applications in text pattern searching, data compression, and pattern discovery. Intuitively, the CDAWG of a string $T$ is obtained by merging isomorphic subtrees of the suffix tree [Weiner 1973] of the same string $T$, and thus CDAWGs are a compact indexing structure. In this paper, we investigate the sensitivity of CDAWGs when a single character edit operation is performed at an arbitrary position in $T$.
We show that the size of the CDAWG after an edit operation on $T$
is asymptotically at most 8 times larger than the original CDAWG before the edit.
\end{abstract}

\documentclass[../main.tex]{subfiles}
\graphicspath{{../images/}}
\makeatletter
\def\input@path{{../images/}}
\makeatother
\begin{document}
\section{Introduction}
\begin{figure}
\centering
\begin{tikzpicture}
\node[inner sep=0pt] (ws) at (0, 0) {
\includegraphics[height=.4\textwidth, trim={10cm 0 10cm 0},clip]{world_space.png}};
\node[inner sep=0pt] (cs) at (6,0) {\includegraphics[height=.4\textwidth, trim={10cm 1cm 10cm 4cm},clip]{conf_space.png}};
\end{tikzpicture}
\vspace{-5pt}
\label{fig:pbrm_intro}
\caption{\textbf{Left}: Shows world space obstacles as grey spheres. Robots start and goal configuration is colored red and green, respectively. Configurations along the computed path are colored transparent blue. \textbf{Right:} Mapped world space scenario to configuration space. Obstacle region is the grey mesh. Red spheres are collision-free regions computed by the neural SCDF. The optimized shortest path in the convex corridor is the blue curve.}
\vspace{-25pt}
\end{figure}
Motion planning is the problem of finding a collision-free trajectory that connects a given start and goal configuration. The planning takes place in the configuration space of the robot. For single body robots, like mobile robots or drones, the configuration space and the world space are usually the same. This simplifies the planning, since explicit obstacle representations are available which enables geometrical tools like separating hyperplanes, smallest distance to obstacles etc., to be used when designing motion planning algorithms. For multi-body robots like manipulators, the situation is completely different. The world space obstacles are usually mapped to non-convex regions, and to make the problem even harder, the mapping is usually not known. Forming explicit representations of the obstacle region in the configuration space is usually too expensive or intractable. Despite all of this, sampling based planners are used with great success, which mainly is due to their use of implicit representations of the obstacle region. The basic idea is to construct a graph in the configuration space that covers and connects the collision-free region. From this graph, a path can be extracted that connects a given start and goal configuration. The approach is computationally expensive, since the graph is constructed with the smallest geometrical building block available, points, which represents a collision-check. Furthermore, the extracted paths from the graph are non-smooth and jagged due to the stochastic nature of the approach. This adds an additional post-processing step to the process, where the paths are shortcutted and smoothened, before the path can be used for tracking. Clearly a lot of time is invested to form this graph and produce smooth paths. Thus, if the obstacles start to move, then all of this work is done in no use, since all points that make up this graph need to be re-verified, which is simply too time consuming to be done in real time.
\\\\
In this work, we want to address the existing drawbacks of the sampling based planners. Our main contribution is an improved motion planner where each vertex in the graph covers a collision-free region in the form of a sphere instead of a point and where the edges are formed with neighboring intersecting spheres. This representation has the advantage of instead of returning piecewise linear paths, returning a sequence of overlapping spheres, i.e. a convex corridor, that connects a given start and goal configuration, illustrated in Figure \ref{fig:pbrm_intro}. This convex corridor allows us to use convex optimization to produce smooth trajectories, instead of computationally expensive post-processing methods. The representation further allows us to estimate the coverage of the collision-free space, which gives us awareness and feedback in the offline roadmap construction phase. Finally, our representation is simple to adapt to moving obstacles, simply requery for the new radii and recheck for intersections. 
\\\\
The spherical collision-free regions are formed using a signed distance function (SDF), which is a function that returns the smallest distance from an arbitrary point to the boundary of an obstacle. As the name implies, the distance is signed, thus if the point is inside the obstacle it is negative otherwise positive. If the distance is positive, a sphere with radius equal to the distance is guaranteed to cover a collision-free region. Using an SDF in motion planning is not new, but what is novel about our approach is that we express the distance in the configuration space instead of the world space and by doing so allows us to form these convex collision-free regions. We refer to the resulting SDF as a signed configuration distance function (SCDF). Computing an SCDF analytically is non-trivial, our approach is therefore to parameterize the SCDF with a deep neural network and learn the mapping by supervised learning. Our resulting neural SCDF can compute distances for different parameter values of obstacle shapes and we also show how multiple distances can be combined, thus making our approach flexible.
\section{Related work}
Motion planning algorithms can roughly be divided into three families, grid-based, sampling based and optimization based methods. Grid-based methods (GBM) discretize the planning space from which a graph is then compiled. A standard search method is A$^\star$ \citep{a_star}, which is classified as an \textit{informed} search method, since it employs a heuristic function to speed up the search. A$^\star$ guarantees to return an optimal path at the level of discretization used. GBMs usually discretize the planning space by a regular lattice and this limits the GBMs to problems with low dimensionality due to the curse of dimensionality. Thus, GBMs are usually limited to single-body robots where the degrees of freedom (DOF) are low. To overcome the inherent scaling problem with the GBMs, stochastic methods are usually used for multi-body robots. These methods are termed as sampling-based methods (SBM) and core members within this family are the rapidly-exploring random trees (RRT) \citep{rrt} and the probabilistic roadmap (PRM) \citep{prm}. RRT grows a tree from the start configuration and explores the collision-free region in a rapid way until it is able to connect to the goal region. RRT is usually improved by bi-directional planning \citep{rrt_connect}, i.e. an additional tree is grown from the goal configuration and the trees are tested for connection after any tree has been expanded. RRT is a single-query method, thus it searches for a path from scratch each time it is queried. Contrary to this, PRM is a multi-query method, which solves for multiple queries without starting from scratch. PRM does this by creating a roadmap (graph) that covers the collision-free space as an offline step. The graph is then used to solve for multiple queries. PRMs are used in cases where the environment does not change since the extra offline step is too computationally costly and needs to be re-done if the environment is changed. In our work, we address this inherent issue by using a different roadmap representation. Our vertices in the graph cover a collision-free region in the form of spheres and we form the edges by checking for intersecting spheres. If something in the environment changes, we recompute the spheres radii and recheck the intersections, without relying on collision detection. We use a trained neural network to compute the sphere radius, therefore querying for the radius can be done fast, hence our representation enables the PRM for dynamic environments.
\\\\
In the recent decades, optimization based methods (OBM) \citep{chomp, schulman, itomp, stomp} have been introduced as an alternative to SBM for multi-body robots. Like the SBM, the OBMs scale well to higher dimensional problems and produce smoother motion. It is common to use a SDF in the optimization since it is a smooth function, thus enabling gradient-based methods. However, the standard way of expressing the SDF is in world space. The distance therefore needs to be mapped to the configuration space by the forward kinematics. This mapping makes the optimization problem a non-linear program (NLP), which is computationally expensive to solve. Recently, a different approach has been proposed. In \cite{mp_gcs} motion planning is formulated as a convex optimization problem by using the graph of convex sets framework \citep{gcs}. The underlying idea is to decompose the collision-free space into intersecting convex sets from which a convex optimization problem is formulated. In cases where an explicit representation of the obstacles in the configuration space exists, like for single-body robots, creating collision-free convex regions can be done fast \citep{iris}. For multi-body robots, this is non-trivial. Existing work does this successfully \citep{iris_nlp, iris_c} by an optimization based approach, but the methods are still too time consuming to be used in the presence of moving obstacles. Our approach is instead to use deep learning to learn an SDF expressed in the configuration space. With this, we can query for shortest distances to the collision boundary, which allows us to expand spherical regions which are collision-free. Our approach is fast and therefore enables our suggested roadmap planner to be used in dynamic environments.
\\\\
Recent research has focused on learning collision detection \citep{fk_kernel_distance, diffco, graphdistnet} by predicting the signed distance between the robot links and the surrounding obstacles in the world space. The learned SDF is used in trajectory optimization but since the distance is expressed in the world space, the problem becomes an NLP and therefore takes a long time to solve. We take a novel approach and suggest to instead express the signed distance in the configuration space. This allows us to improve the PRM at the same time as it enables convex optimization for trajectory optimization, which runs faster and is more reliable than NLP solvers. In \cite{cspf} a learned signed distance function in the configuration space is proposed similar to our approach. However, their approach is restricted to point cloud representations, while we propose to represent the obstacles as parameterized geometric shapes, e.g. spheres. Furthermore, we also show how to use our learned SCDF to improve an existing roadmap planner.
\section{Problem formulation}
A robot is located in the world space, $\W \subset \R^3 $. The unique location of the robot is given by its configuration $\q \in \C$, where $\C$ is the configuration space. The set of points covered by the robots bodies at a certain configuration is expressed as $\B(\q) \subset \W$. The robot is surrounded by $\NrObst$ obstacles $\O = \bigcup_{i=1}^{\NrObst} \O_i$, where  $\O_i \subset \W$. The representation of the obstacle in the configuration space is the set $\C\O_i = \{\q \in \C \: |\: \B(\q) \cap \O_i \neq \emptyset \}$. The obstacle space is formed as $\Co = \bigcup_{i=1}^{\NrObst} \C \O_i$. The complement is referred to as the free space, $\Cf = \C \setminus \Co$. The path planning problem is a tuple, ($\Cf$, $\qStart$, $\qGoal$), where we want to connect a query pair, consisting of a start, $\qStart$, and goal configuration, $\qGoal$, with a geometric path, $\q(s): [0, 1] \mapsto \Cf$, such that $\q(0)=\qStart$ and $\q(1)=\qGoal$, or report correctly when such a path does not exist.
\end{document}


\section{Preliminaries}\label{sec:preliminaries}



%We denote by $(\Ac(x_\Ac),\Bc(x_\Bc))(z)$ a random execution of $\pi$ with private inputs $(x_\Ac,y_\Ac)$, and common input $z$.

%\Jnote{Move to DP}
% At the end of such an execution, the protocol outputs a public transcript denoted by the random variable $\trans_\pi(x_\Ac,x_\Ac,z)$ we denotes the common as $\out(\trans_\pi(x_\Ac,x_\Ac,z)$, and each party $\Pc \in \set{\Ac,\Bc}$ obtains his view denoted $\view^\Pc_\pi(x_\Ac,x_\Bc,z)$, which may also contain a ``local output'' \Jnote{Local} $\out^\Pc(x_\Ac,x_\Bc,z)$ (if the protocol specifies such an output). \Jnote{Common output, and parties output}


\subsection{Distributions and Random Variables}\label{sec:prelim:dist}
The support of a distribution $P$ over a finite set $\cS$ is defined by $\Supp(P) \eqdef \set{x\in \cS: P(x)>0}$. For a distribution or a random variable $D$, let $d\from D$ denote that $d$ was sampled according to $D$. Similarly,  for a set $\cS$, let $x \from \cS$ denote that $x$ is drawn uniformly from $\cS$, and denote by $\cU_{\cS}$ the uniform distribution over $\cS$. For a finite set $\cX$ and a distribution $C_X$ over $\cX$, we use the capital letter $X$ to denote the random variable that takes values in $\cX$ and is sampled according to $C_X$. The {\sf statistical distance} (\aka {\sf~variation distance}) of two distributions $P$ and $Q$ over a discrete domain $\cX$ is defined by $\sdist{P}{Q} \eqdef \max_{\cS\subseteq \cX} \size{P(\cS)-Q(\cS)} = \frac{1}{2} \sum_{x \in \cS}\size{P(x)-Q(x)}$. 
For a vector $x = (x_1,\ldots,x_n)$ and index $i\in [n]$, we let $x_{-i} = (x_1,\ldots,x_{i-1},x_{i+1},\ldots,x_n)$ and $x^{(i)} = (x_1,\ldots,x_{i-1}, -x_i, x_{i+1},\ldots,x_n)$, for a set $\cS \subseteq [n]$ we let $x_{\cS} = (x_i)_{i \in \cS}$ and $x_{-\cS} = (x_i)_{i \in [n]\setminus \cS}$, and for a vector $r \in \zo^n$ we let $x_r = (x_i)_{\set{i \colon r_i = 1}}$ and $x_{-r} = (x_i)_{\set{i \colon r_i = 0}}$.

%For $n \in \N$ we let $U_n$ be the uniform distribution over $\oo^n$, and let $S_n$ be the distribution induces by the sum of $n$ i.i.d.\ random variables, each is distributed according to $U_1$. Let $\cN(0,1)$ be the standard normal distribution.
%For a distribution $\cD$ and a function $f$, we define by $f(\cD)$ the distribution that is induced by the output of $f(x)$ for $x \from \cD$. 





% \begin{theorem}[\cite{McGregorMPRTV10}]\label{thm:sv-extracotr}
% 	\Enote{Remove if not needed}
% 	There is a constant $c$ to make the following holds. Let $X$ be an $\alpha$-SV source on $\{0,1\}^n$, let $Y$ be a source on $\{0,1\}^n$ with min-entropy at least $\beta n$ (independent from $X$), and let $Z=\ip{X,Y}\mbox{mod m}$ for some $m\in\mathbb{N}$. Then for every $\delta\in[0,1]$, the random variable $(Y,Z)$ is $\delta$-close to $(Y,U)$ where $U$ is uniform on $\mathbb{Z}_m$ and independent of $Y$, provided that
% 	$$
% 	n\geq c\cdot\frac{m^2}{\alpha\beta}\cdot\log(\frac{m}{\beta})\cdot\log(\frac{m}{\delta}).
% 	$$
% \end{theorem}



\Enote{I removed the definition of DP since it already appears in the intro}
\remove{
\subsection{Differential Privacy}\label{sec:prelim:DP}
We use the following standard definition of (information theoretic) differential privacy, due to \citet{DMNS06}. For notational convenience, we focus on databases over $\oo$.
\begin{definition}[Differentially private mechanisms]\label{def:mech}
	A randomized function $f\colon\oo^n\mapsto \zs$ is an {\sf $n$-size, $(\eps,\delta)$-differentially private mechanism} (denoted $(\eps,\delta)$-\DP) if for every neighboring $w,w'\in \oo^n$ and every function $g\colon \zs\mapsto \zo$, it holds that 
	$$
	\pr{g(f(w))=1}\leq e^{\eps}\cdot \pr{g(f(w'))=1} +\delta.
	$$ 	
	If $\delta=0$, we omit it from the notation.
\end{definition}
}


\subsubsection{Computational Differential Privacy}
There are several ways for defining computational differential privacy (see \cref{sec:related-works}). We use the most relaxed version due to \cite{BNO08}. For notational convenience, we focus on databases over $\oo$.
\begin{definition}[Computational differentially private mechanisms]\label{def:ComMech}
	A randomized function ensemble $f=\set{f_\pk\colon\oo^{n(\pk)}\mapsto \zs}$ is an {\sf $n$-size, $(\eps,\delta)$-computationally differentially private} (denoted $(\eps,\delta)$-$\CDP$) if for every poly-size circuit family $\set{\Ac_\pk}_{\pk\in \N}$, the following holds for every large enough $\pk$ and every neighboring $w,w'\in\oo^{n(\pk)}$:
	$$
	\pr{\Ac_\pk(f_\pk(w))=1}\leq e^{\eps(\pk)}\cdot \pr{\Ac_\pk(f_\pk(w'))=1} +\delta(\pk).
	$$ 
	If $\delta(\pk) = \negl(\pk)$, we omit it from the notation. 
\end{definition}



\subsubsection{Two-Party Differential Privacy}\label{sec:DP}
In this section we formally define distributed differential privacy mechanism (\ie protocols). %For the ease of notation, we consider protocol with no common input.

\begin{definition}\label{def:DP}%\Nnote{fix security parameter}
	A two-party protocol $\Pi=(\Ac,\Bc)$ is {\sf $(\eps,\delta)$-differentially private}, denoted $(\eps,\delta)$-$\DP$, if the following holds for every algorithm $\Dc$: let $\V^\Pc(x,y)(\pk)$ be the view of party $\Pc$ in a random execution of $\Pi(x,y)(1^\pk)$. Then for every $\pk,n \in \N$, $x\in \oo^n$ and neighboring $y,y'\in\oo^n$:
	\begin{align*}
	\pr{\Dc(V^\Ac(x,y)(\pk))=1}\le e^{\eps(\pk)}\cdot \pr{\Dc(V^\Ac (x,y')(\pk))=1}+\delta(\pk),
	\end{align*} 
	and for every $y\in \oo^n$ and neighboring $x,x'\in\oo^{n}$:
	\begin{align*}
	\pr{\Dc(V^\Bc(x,y)(\pk))=1}\le e^{\eps(\pk)}\cdot \pr{\Dc(V^\Bc (x',y)(\pk))=1}+\delta(\pk).
	\end{align*} 	
	Protocol $\Pi$ is {\sf $(\eps,\delta)$-computational differentially private}, denoted $(\eps,\delta)$-$\CDP$, if the above inequalities only hold for a non-uniform \ppt $\Dc$ and large enough $\pk$. We omit $\delta = \negl(\pk)$ from the notation. \footnote{Note that define we give for two-party differentially private protocols is a semi-honest definition, in which we ask for the security to hold when the parties interact in an honest execution of the protocol. Since we are proving a lower bound, starting from this weaker guarantee (as opposed to security against malicious players), yields a stronger result.}
\end{definition}
%We omit $\delta$ from the notation if $\delta$ is a negligible function of $n$.

%\Enote{simulation-based}
\begin{remark}[The definition for computational differential privacy we use]\label{rem:comDPChannel} 
	An alternative, stronger definition of computational differential privacy, known as simulation-based computational differential privacy, requires that the distribution of each party’s view be computationally indistinguishable from a distribution that ensures privacy in an information-theoretic sense. \cref{def:DP} is a weaker notion in comparison. Consequently, establishing a lower bound for a protocol that satisfies this weaker guarantee (as we do in this work) yields a stronger result.%Actually, our lower bound only requires the privacy to hold against \emph{uniform} external observer.
	%\Nnote{Maybe add: When only interesting in \Dp against external observer, the two definitions can be achieve using key-agreement and (single-party) \Dp mechanism. }
\end{remark}




\subsection{Useful Claims}
\remove{
In this section, we state generic lemmas and propositions that we will use later in our proofs.

The following lemma which we prove in \cref{sec:missing-proofs:distance-I}, measures the distance between two uniform stings conditioned one a random index $i$ either being fixed to $0$ or to $1$.

\def\distanceILemma{
    Let $R \la \zo^n$. For any (randomized) function $f:\{0,1\}^n\rightarrow \{0,1\}$ and $\alpha > 0$, it holds that
    \begin{align}\label{eq:f-alpha}
        \ppr{i \la [n]}{\size{\:\ex{f(R) \mid R_i = 0}-\ex{f(R) \mid R_i = 1}\:}\geq \alpha} \leq \frac{2}{n \alpha^2},
    \end{align}
    where the expectations are taken over $R$ and the randomness of $f$.
}

\begin{lemma}\label{lem:distance-I}
    \distanceILemma
\end{lemma}
}

The following two propositions state that given the output of a differentially private function, it is not possible to predict well even a random index (even if all other indexes are leaked). The first proposition handles the information-theoretic case and the second handles the computation case. Both propositions are proven in \cref{sec:missing-proofs:hard-to-guess}. 

\def\propHardToGuessInf{
    Let $f\colon \oo^n \rightarrow \cY$ be an $(\eps,\delta)$-\DP function, let $g \colon [n] \times \oo^{n-1} \times \cY \rightarrow \set{-1,1,\bot}$ be a (randomized) function, and let $X = (X_1,\ldots,X_n) \la \oo^n$. Then the following holds for every $i \in [n]$ where $X_i^* = g(i,X_{-i},f(X_1,\ldots,X_n))$:
    \begin{align*}
        \pr{X_i^* = X_i} \leq e^{\eps}\cdot \pr{X_i^* = -X_i} + \delta.
    \end{align*}
}

\begin{proposition}\label{prop:hard-to-guess-inf}
    \propHardToGuessInf
\end{proposition}


\def\propHardToGuessComp{
    Let $f = \set{f_{\pk} \colon \oo^{n(\pk)} \rightarrow \zo^{m(\pk)}}_{\pk \in \bbN}$ be an $(\eps,\delta)$-\CDP function ensemble, and let $\set{g_{\pk}}_{\pk \in \bbN}$ be a poly-size circuit family. Then, for large enough $\pk$ and $X = (X_1,\ldots,X_{n(\pk)}) \la \oo^{n(\pk)}$, the following holds for every $i \in [n(\pk)]$ where $X_i^* = g_{\pk}(i,X_{-i},f_{\pk}(X_1,\ldots,X_n))$:
    \begin{align*}
        \pr{X_i^* = X_i} \leq e^{\eps(\pk)}\cdot \pr{X_i^* = -X_i} + \delta(\pk).
    \end{align*}
}

\begin{proposition}\label{prop:hard-to-guess-comp}
    \propHardToGuessComp
\end{proposition}





\remove{
\Enote{Chao's old statement:}
\begin{lemma}\label{lem:distance-I-old}
        Let $R \la \zo^n$. 
	For any function $f:\{0,1\}^n\rightarrow \{0,1\}$ and $\alpha<0.01$, it holds that
	$$
	\Pr_{i\la[n]}\left[\: \size{\:\mathbb{E}[f(R) \mid R_i = 0]-\mathbb{E}[f(R) \mid R_i = 1]\:}\geq \alpha\right]\leq \frac{2+2\log(\frac{1}{\alpha})}{n\alpha^2}.
	$$
\end{lemma}
\begin{proof}
	Define $S_1=\{r \in \zo^n \colon f(r)=1\}$. Then for any $i\in[n]$, we have
	$$
	\begin{array}{rl}
		\size{\mathbb{E}[f(R) \mid R_i = 0]-\mathbb{E}[f(R) \mid R_i = 1]}
		&=\size{\Pr[R\in S_1|R_i=0]-\Pr[R\in S_1|R_i=1]}\\
		&=\size{\frac{\Pr[R_i=0|R\in S_1]\cdot\Pr[R\in S_1]}{\Pr[R_i=0]}-\frac{\Pr[R_i=1|R\in S_1]\cdot\Pr[R\in S_1]}{\Pr[R_i=1]}}\\
		&=\frac{2\size{S_1}}{2^n}\size{\Pr[R_i=0|R\in S_1]-\Pr[R_i=1|R\in S_1]}
	\end{array}
	$$
	When $|S_1|\leq \alpha\cdot 2^{n-1}$, we have $\size{\mathbb{E}[f(R) \mid R_i = 0]-\mathbb{E}[f(R) \mid R_i = 1]}\leq\frac{2\size{S_1}}{2^n}\leq \alpha$ for any $i\in[n]$. Hence, in the following, we assume $|S_1|> \alpha\cdot 2^{n-1}$.

	%Define $I_{bad}=\{i|\size{\Pr[R_i=0|R\in S_1]-\Pr[R_i=1|R\in S_1]}>2\alpha\}$ and $k=\size{I_{bad}}$, then for any $i\notin I_{bad}$, we have 
    %$$
    %\begin{array}{rl}
    %    2\alpha&\geq \size{\Pr[R_i=0|R\in S_1]-\Pr[R_i=1|R\in S_1]}\\
    %    &=\size{\frac{\Pr[R\in S_1|R_i=0]\cdot\Pr[R_i=0]}{\Pr[R\in S_1]}-\frac{\Pr[R\in S_1|R_i=1]\cdot\Pr[R_i=1]}{\Pr[R\in S_1]}}\\
    %    &=\size{\Pr[R\in S_1|R_i=0]-\Pr[R\in S_1|R_i=1]}\cdot\frac{1}{2\Pr[R\in S_1]}\\
    %    &\geq \size{\mathbb{E}[f(R) \mid R_i = 0]-\mathbb{E}[f(R) \mid R_i = 1]}\cdot \frac{1}{2},
    %\end{array}
    %$$ 
    %where the last inequality is because $\Pr[R\in S_1]\leq 1$. So that $\size{\mathbb{E}}[f(R) \mid R_i = 0]-\mathbb{E}[f(R) \mid R_i = 1]\leq %4\alpha$.
    Define $I_{bad}=\{i \colon \size{\Pr[R_i=0|R\in S_1]-\Pr[R_i=1|R\in S_1]} \geq 2\alpha\}$ and $k=\size{I_{bad}}$, and denote $I_{bad}=\{i_1,\dots,i_k\}$. Define $(X_{i_1}, \ldots X_{i_k}) = (R_{i_1},\dots,R_{i_k})\mid_{R \in S_1}$. 
    Consider the min-entropy
	$$
	\begin{array}{rl}
		H_{min}(X_{i_1},\dots,X_{i_k})&\leq H(X_{i_1},\dots,X_{i_k})\\
		&\leq \sum_{j=1}^k H(X_{i_j})\\
		&\leq k\cdot \left(-(\frac{1}{2}+2\alpha)\cdot\log(\frac{1}{2}+2\alpha)-(\frac{1}{2}-2\alpha)\cdot\log(\frac{1}{2}-2\alpha)\right)\\
            &=k\cdot \left(-(\frac{1}{2}+2\alpha)\cdot(\log(1+4\alpha)-1)-(\frac{1}{2}-2\alpha)\cdot(\log(1-4\alpha)-1)\right)\\
            &=k\cdot \left(1-(\frac{1}{2}+2\alpha)\cdot\log(1+4\alpha)-(\frac{1}{2}-2\alpha)\cdot\log(1-4\alpha)\right),
		
	\end{array}
	$$
	where $H_{min}(Y)$ is the minimum entropy of $Y$ and $H(Y)$ is the Shannon entropy of $Y$.\Enote{add to preliminaries.}
        The third inequality holds since by the definition of $I_{bad}$, for every $j \in [k]$ it holds that $\size{\pr{X_{i_j} = 1}-\pr{X_{i_j} = 0}} > 2\alpha$, and therefore $H(X_{i_j}) \leq H(1/2 + 2\alpha)$\Enote{define}.
	
	Therefore, there exists $b_1,\dots,b_k\in\{0,1\}$, such that 
	
	\begin{align}\label{eq:min-entropy-result}
		\Pr\left[(R_{i_1},\ldots,R_{i_k}) = (b_1,\ldots,b_k) \mid R\in S_1\right]
		&= \pr{(X_{i_1},\ldots,X_{i_k}) = (b_1,\ldots,b_k)}\\
		&= 2^{-H_{min}(X_{i_1},\dots,X_{i_k})}\nonumber\\
		&\geq 2^{k\cdot \left(-1+(\frac{1}{2}+2\alpha)\cdot\log(1+4\alpha)+(\frac{1}{2}-2\alpha)\cdot\log(1-4\alpha)\right)}.\nonumber
	\end{align}
	
	Let $S_{bad}=\{r \in \zo^n  \colon \set{(r_{i_1},\ldots,r_{i_k}) = (b_1,\ldots,b_k)} \land \set{r\in S_1}\}$.
	It holds that
	\begin{align*}
		|S_{bad}|
		&= \size{S_1} \cdot \Pr\left[(R_{i_1},\ldots,R_{i_k}) = (b_1,\ldots,b_k) \mid R\in S_1\right]\\
		&\geq \alpha\cdot 2^{n-1}\cdot2^{k\cdot \left(-1+(\frac{1}{2}+2\alpha)\cdot\log(1+4\alpha)+(\frac{1}{2}-2\alpha)\cdot\log(1-4\alpha)\right)},
	\end{align*} 
	where the inequality holds by \cref{eq:min-entropy-result} and since $\size{S_1} \geq \alpha\cdot 2^{n-1}$.
	Notice that any string in $S_{bad}$ depends on at most $n-k$ bits. It implies that $|S_{bad}|\leq 2^{n-k}$. Therefore, we have
	$$
	\begin{array}{rl}
		&2^{n-k}\geq \alpha\cdot 2^{n-1}\cdot2^{k\cdot \left(-1+(\frac{1}{2}+2\alpha)\cdot\log(1+4\alpha)+(\frac{1}{2}-2\alpha)\cdot\log(1-4\alpha)\right)} \\
		\Rightarrow& n-k \geq \log \alpha+n-1+k\cdot \left(-1+(\frac{1}{2}+2\alpha)\cdot\log(1+4\alpha)+(\frac{1}{2}-2\alpha)\cdot\log(1-4\alpha)\right)\\
		\Rightarrow& 1-\log \alpha \geq k\cdot((\frac{1}{2}+2\alpha)\cdot\log(1+4\alpha)+(\frac{1}{2}-2\alpha)\cdot\log(1-4\alpha))\\
		\Rightarrow& 1-\log \alpha \geq k\cdot(4\alpha\cdot\log(1+4\alpha)+(\frac{1}{2}-2\alpha)\cdot\log(1-16\alpha^2))\\
        \Rightarrow& 1-\log\alpha \geq k\cdot(15.9\alpha^2-8\alpha^2+32\alpha^3)=k\cdot(7.9\alpha^2+32\alpha^3)>0.5k\alpha^2\\
		\Rightarrow& k\leq \frac{2-2\log \alpha}{\alpha^2} = \frac{2+2\log (1/\alpha)}{\alpha^2},
	\end{array}
	$$
	Where the third transition holds since 
	\begin{align*}
		\lefteqn{(\frac{1}{2}+2\alpha)\cdot\log(1+4\alpha)+(\frac{1}{2}-2\alpha)\cdot\log(1-4\alpha)}\\
		&= 4\alpha\cdot\log(1+4\alpha) + (\frac{1}{2}-2\alpha)\paren{\log(1+4\alpha)+\log(1-4\alpha)}\\
		&= 4\alpha\cdot\log(1+4\alpha)+(\frac{1}{2}-2\alpha)\cdot\log(1-16\alpha^2),
	\end{align*}
	and the forth transition holds since $4\alpha\cdot\log(1+4\alpha)+(\frac{1}{2}-2\alpha)\cdot\log(1-16\alpha^2) > 15.9\alpha^2-8\alpha^2+32\alpha^3$ for $\alpha < 0.01$.
	Thus, we conclude that 
	$$
	\Pr_{i\la[n]}\left[\size{\mathbb{E}[f(R) \mid R_i=0]-\mathbb{E}[f(R) \mid R_i = 1]}\geq \alpha\right]\leq \frac{k}{n}\leq \frac{2+2\log (1/\alpha)}{n\alpha^2}.
	$$
\end{proof}
}


\subsection{Channels and Two-Party Protocols}\label{sec:protocol}

\paragraph{Channels.}A channel is simply a distribution of a pair of tuples defined as follows. 
\begin{definition}[Channels]\label{def:channel} A {\sf channel} $C_{(X,U)(Y,V)}$ of size $\isize$ over alphabet $\Sigma$ is a probability distribution over $(\Sigma^\isize \times\zo^\ast) \times(\Sigma^\isize \times\zo^\ast)$. The ensemble $C_{(X,U)(Y,V)}= \set{C_{(X_\pk,U_\pk)(Y_\pk,V_\pk)}}_{\pk\in \N}$ is an $\isize$-size channel ensemble, if for every $\pk\in \N$, $C_{(X_\pk,U_\pk)(Y_\pk,V_\pk)}$ is an $\isize(\pk)$-size channel. %We denote a channel of size one by a \emph{single-bit} channel. 
We refer to $X$ and $Y$ as the {\sf local outputs}, and to $U$ and $V$ as the {\sf views}.	
\end{definition}

We view a  channel as the experiment in which there are two parties $\Ac$ and $\Bc$.  Party $\Ac$ receives ``output'' $X$ and ``view'' $U$, and party $\Bc$ receives ``output'' $Y$ and ``view'' $V$. Unless stated otherwise, the channels we consider are over the alphabet $\Sigma = \oo$. We naturally identify channels with the distribution that characterizes their output.








\subsubsection{Two-Party Protocols}

A two-party protocol $\Pi=(\Ac,\Bc)$ is \ppt if the running time of both parties is polynomial in their input length. We let $\Pi(x,y)(z)$ or $(\Ac(x),\Bc(y))(z)$ denote a random execution of $\Pi$ on a common input $z$, and private inputs $x,y$.%We assume \wlg that a protocol has a common output (part of its transcript).\Jnote{This is not really the case we consider in this paper..}

\begin{definition}[Oracle-aided protocols]\label{def:ChannelAidedProtocol}
	In a two-party protocol $\Pi$ with oracle access to a {\sf protocol} $\Psi$, denoted $\Pi^\Psi$, the parties make use of the \textit{next-message function} of $\Psi$.\footnote{The function that on a partial view of one of the parties, returns its next message.} In a two-party protocol $\Pi$ with oracle access to a {\sf channel} $C_{Z W}$, denoted $\Pi^C$, the parties can jointly invoke $C$ for several times. In each call, an independent pair $(z,w)$ is sampled according to $C_{Z W}$, one party gets $z$, the other gets $w$.
\end{definition}


\begin{definition}[The channel of a protocol]\label{def:ChannlOfProtocol}
	For a no-input two-party protocol $\Pi= (\Ac,\Bc)$, we associate the channel $C_\Pi$, defined by $\C_\Pi= C_{(X, U),(Y, V)}$, where $X$ and $Y$ are the local outputs of $\Ac$ and $\Bc$ (respectively) and
	$U$ and $V$ are the local views of $\Ac$ and $\Bc$ (respectively).
    
	For a two-party protocol $\Pi$ that gets a security parameter $1^\pk$ as its (only, common) input, we associate the channel ensemble $ \set{C_{\Pi(1^\pk)}}_{\pk\in \N}$. 
\end{definition}

\begin{definition}[$(\alpha,\gamma)$-Accurate channel]\label{def:accurate-func}
	A channel $C = C_{(X, U),(Y, V)}$ is {\sf $(\alpha,\gamma)$-accurate for the function $f$}, if $\ppr{C}{\size{\out(V)-f(X,Y)}\leq \alpha}\ge \gamma$, where $\out(V)$ is the designated output.
    A channel ensemble $C_{(X, U),(Y, V)}= \set{C_{(X_\pk, U_\pk),(Y_\pk, V_\pk)}}_{\pk\in \N}$ is  $(\alpha,\gamma)$-accurate for  $f$ if $C_{(X_\pk, U_\pk),(Y_\pk, V_\pk)}$ is $(\alpha(\pk),\gamma(\pk))$-accurate for $f$, for every $\pk \in \N$.
\end{definition}

\subsubsection{Differentially Private Channels}\label{sec:DPChannel}
Differentially private channels are naturally defined as follows:
\begin{definition}[Differentially private channels]\label{def:DPChannel}
	An $n$-size channel $C = C_{(X, U),(Y, V)}$ with $X, Y$ over $\oo^n$ 
	is {\sf$(\eps,\delta)$-differentially private} (denoted $(\eps,\delta)$-$\DP$) if for every $x \in \Supp(X)$ there exists an $n$-size $(\eps,\delta)$-$\DP$ mechanisms $\Mc_x$ such that $(X,Y,U) \equiv (X,Y,\Mc_X(Y))$, and for every $y \in \Supp(Y)$ there exists an $n$-size $(\eps,\delta)$-$\DP$ mechanisms $\Mc_y'$ such that $(X,Y,V) \equiv (X,Y,\Mc_Y'(X))$. In addition, we say that the channel is \emph{uniform} if $X$ and $Y$ are independent random variables uniformly distributed in $\oo^n$. 
\end{definition}

\begin{definition}[Computational differentially private channels]\label{def:CDPChannel}
	An $n$-size channel ensemble $C = \set{C_{(X_\pk, U_\pk),(Y_\pk, V_\pk)}}_{\pk\in\N}$ with $X_\pk, Y_\pk$ over $\oo^n$ 
	is {\sf$(\eps,\delta)$-computationally differentially private} (denoted $(\eps,\delta)$-$\CDP$) if for every ensemble $\set{x_\pk \in \Supp(X_\pk)}_{\pk\in\N}$ there exists an $n$-size $(\eps,\delta)$-\CDP mechanisms ensemble $\set{\Mc_{x_\pk}}_{\pk\in\N}$ such that $(X_\pk,Y_\pk,U_\pk) \equiv (X_\pk,Y_\pk,\Mc_{X_\pk}(Y_\pk))$, for every $\pk\in\N$, and for every ensemble $\set{y_\pk \in \Supp(Y_\pk)}_{\pk\in\N}$ there exists an $n$-size $(\eps,\delta)$-$\CDP$ mechanisms ensemble $\set{\Mc'_{y_\pk}}_{\pk\in\N}$ such that $(X_\pk,Y_\pk,V_\pk) \equiv (X_\pk,Y_\pk,\Mc_{Y_\pk}'(X_\pk))$ for every $\pk\in \N$. In addition, we say that the channel is \emph{uniform} if $X_\pk$ and $Y_\pk$ are independent random variables uniformly distributed in $\{\pm 1\}^n$ for all $\pk\in\N$.
\end{definition}




% \begin{lemma}~\label{lem:dp-sv-source}
% 	Let $P$ be an $\varepsilon$-DP randomized protocol. Let $X$ and $Y$ be independent random variables uniformly distributed in $\{\pm 1\}^n$ and let random variable $\Pi(X,Y)$ denote the transcript of running $P(X,y)$. Then for every $\pi\in Supp(\Pi)$, the random variables corresponding to the inputs conditioned on transcript $\pi$, $X_\pi$ and $Y_\pi$, are independent $e^{-\varepsilon}$-strong SV source.
% \end{lemma}





\subsubsection{Weak Erasure Channel (\WEC)}

\begin{definition}[\WEC]\label{def:WEC}
	A channel $((O_A,V_A), (O_B,V_B))$ with $O_A \in \set{0,1}$ and $O_B \in \set{0,1,\bot}$ is a {\sf weak erasure channel}, denoted $(\alpha,p,q)$-$\WEC$, if:
	\begin{itemize}
		%\item $O_A\in \set{-1,1}$ and $O_B\in \set{-1,1,\bot}$.
		\item Random erasure: $\pr{O_B = \perp} = 1/2$.
		
		\item Agreement: $\pr{O_A\ne O_B\mid O_B\ne \bot}\le \alpha$.
		
		\item Secrecy:
		
		\begin{enumerate}
			\item For every algorithm $\Dc$ it holds that\label{WEC:item:A}
			\begin{align*}
				%\size{\pr{\Ac(O_A,V_A) = 1 \mid O_B \neq \perp} - \pr{\Ac(O_A,V_A) = 1 \mid O_B = \perp}} \le p
				\size{\pr{\Dc(V_A) = 1 \mid O_B \neq \perp} - \pr{\Dc(V_A) = 1 \mid O_B = \perp}} \le p
			\end{align*}
			(Alice doesn't know if $O_B = \perp$.)
			
			\item For every algorithm $\Dc$ it holds that\label{WEC:item:B}
			\begin{align*}
				\pr{\Dc(V_B) = O_A \mid O_B=\bot} \leq \frac{1+q}{2}.
			\end{align*}
			(i.e., if $O_B=\bot$, Bob don't know what is the value of $O_A$).
			
			%\item $SD((O_A U|O_B=\bot),(O_A U|O_B\ne \bot))\le p$ (The sender don't know if $O_B=\bot$).
			
			%\item $SD(V O_A|O_B=\bot,V(-O_A)|O_B=\bot)\le q$ (If $O_B=\bot$, Bob don't know what the value of $O_A$).
		\end{enumerate}
	\end{itemize}
   We say that a channel ensemble $C=\set{C_\pk}_{\pk\in N}$ is a {\sf computational weak erasure channel}, denoted $(\alpha,p,q)$-\CompWEC, if for every \ppt algorithm $\Dc$ and every sufficiently large $\pk\in\N$, $C_\pk$ satisfies the properties stated in the items above, where the secrecy property holds with respect to a \ppt algorithm $\Dc$. A protocol $\Lambda$ is said to be $(\alpha,p,q)$-$\CompWEC$, if the ensemble induces by the protocol (that is, $C=\set{C_{\Lambda(\pk)}}_{\pk\in\N}$) is $(\alpha,p,q)$-$\CompWEC$.  
\end{definition}



\subsubsection{Approximate Weak Erasure Channel (\AWEC)}\label{sec:AWEC}

\begin{definition}[\AWEC]\label{def:AWEC}
	A channel $C = ((O_A,V_A), (O_B,V_B))$ over $([-n,n] \times \zo^*) \times (([-n,n] \cup \bot)  \times \zo^*)$ is an {\sf approximate weak erasure channel}, denoted $(\ell,\alpha,p,q)$-\AWEC if:
	\begin{itemize}
		
		\item Random erasure: $\pr{O_B = \perp} = 1/2$.
		
		\item Accuracy: $\pr{\size{O_A - O_B} > \ell \mid O_B \ne \bot}\le \alpha$.
		
		\item Secrecy:
		
		\begin{enumerate}
			\item For every algorithm $\Dc$ it holds that\label{AWEC:item:A}
			\begin{align*}
				%\size{\pr{\Ac(O_A,V_A) = 1 \mid O_B \neq \perp} - \pr{\Ac(O_A,V_A) = 1 \mid O_B = \perp}} \le p
				\size{\pr{\Dc(V_A) = 1 \mid O_B \neq \perp} - \pr{\Dc(V_A) = 1 \mid O_B = \perp}} \le p
			\end{align*}
			(Alice doesn't know if $O_B=\bot$).
			
			\item For every algorithm $\Dc$ it holds that\label{AWEC:item:B}
			\begin{align*}
				\pr{\size{\Dc(V_B) - O_A} \leq 1000 \ell \mid O_B=\bot} \leq q.
			\end{align*}
			(i.e., if $O_B=\bot$, Bob can't estimate the value of $O_A$ with error $\leq 1000 \ell$).
		\end{enumerate}
	\end{itemize}
     We say that a channel ensemble $C=\set{C_\pk}_{\pk\in N}$ is a {\sf computational approximate weak erasure channel}, denoted $(\ell,\alpha,p,q)$-\CompAWEC, if for every \ppt algorithm $\Dc$ and every sufficiently large $\pk\in\N$, $C_\pk$ satisfies the properties stated in the items above. A protocol $\Gamma$ is said to be $(\ell,\alpha,p,q)$-$\CompAWEC$, if the ensemble induced by the protocol (that is, $C=\set{C_{\Gamma(\pk)}}_{\pk\in\N}$) is $(\ell,\alpha,p,q)$-$\CompAWEC$.  
\end{definition}

We will make use of the following lemma, which shows that for some choices of the parameters, \AWEC implies \WEC. The lemma is proven in \cref{sec:AWEC-to-WEC}.

\begin{lemma}\label{lemma:AWEC-to-WEC}
	For every $\ell> 0$, there exists a \ppt protocol $\Lambda = (\Pc_1,\Pc_2)$ such that given an oracle access to an $(\ell,\alpha,p,q)$-\AWEC $C$, the channel $\tilde{C}$ induced by $\Lambda^C$ is $(\alpha'=\alpha+0.001,\: p' = p ,\:  q' = 1/2 + 2(q+0.01))$-\WEC.
	Furthermore, the proof is constructive in a black-box manner:
	\begin{enumerate}
		\item There exists an oracle-aided \ppt algorithm $\Ec_1$ such that for every channel $C = ((\OA,\VA), (\OB,\VB))$ and algorithm $\Dc$ violating the \WEC secrecy property~\ref{WEC:item:A} of $\tilde{C}$, algorithm $\Ec_1^{\Dc}$ violates the \AWEC secrecy property~\ref{AWEC:item:A} of $C$.
		
		\item There exists an oracle-aided \ppt algorithm $\Ec_2$ such that for every channel $C = ((\OA,\VA), (\OB,\VB))$ and algorithm $\Dc$ violating the \WEC secrecy property~\ref{WEC:item:B} of $\tilde{C}$, algorithm $\Ec_2^{\Dc}$ violates the \AWEC secrecy property~\ref{AWEC:item:B} of $C$.
	\end{enumerate}
\end{lemma}

Since \cref{lemma:AWEC-to-WEC} is constructive, the following is an immediate corollary.
\begin{corollary}\label{cor:CompAWEC to CompWEC}
There exists an oracle aided \ppt protocol $\Lambda$, such that given a protocol $\Gamma$ that induces $(\ell,\alpha,p,q)$-\CompAWEC, it holds that $\Lambda^\Gamma$ is $(\alpha'=\alpha+0.001,\: p' = p ,\:  q' = 1/2 + 2(q+0.01))$-\CompWEC.  
\end{corollary}
\begin{proof}[Proof of \ref{cor:CompAWEC to CompWEC}]
Let $\Lambda$ be the \ppt algorithm guaranteed  by Lemma \ref{lemma:AWEC-to-WEC}. Given an $(\ell,\alpha,p,q)$-\CompAWEC protocol $\Gamma$, we define $\Lambda(\pk)={\Lambda^{\Gamma(\pk)}(\pk)}$. Assume towards a contradiction that $\Lambda$ is not a $(\alpha',p',q')$-\CompWEC. It follows that there exists a \ppt $\Dc$ that for infinity many $\pk\in\N$ contradicts one of the \WEC secrecy properties of channel ensemble $\set{C_{\Lambda(\pk)}}_{\pk\in\N}$. Fix $\pk\in\N$ for which this holds. By Lemma \ref{lemma:AWEC-to-WEC}, there exists a \ppt $\Ec^\Dc$ that for every such $\pk$  contradicts one of the secrecy properties of the channel $C_{\Gamma(\pk)}$. This implies that for infinity many $\pk\in\N$, $\Ec^\Dc$  contradict the secrecy of the channel ensemble $\set{C_{\Gamma(\pk)}}_{\pk\in\N}$, which is a contradiction since this would means that $\Gamma$ is not a $(\ell,\alpha,p,q)$-\CompAWEC.       
\end{proof}



\subsection{Oblivious Transfer (\OT)}

\paragraph{Secure Computation.}
We use the standard notion of securely computing a functionality, \cf  \cite{Goldreich04}.
\begin{definition}[Secure computation]\label{def:SFE}
	A two-party protocol {\sf securely computes a functionality $f$}, if it does so according to the real/ideal paradigm.   We add the term perfectly/statistically/computationally/non-uniform computationally, if the simulator's output is  perfect/statistical/computationally indistinguishable/  non-uniformly indistinguishable from  the real distribution.  The protocol have the above notions of security {\sf against semi-honest  adversaries}, if its security only  guaranteed to holds against an adversary that follows the prescribed protocol.   Finally, for the case of perfectly secure computation, we naturally apply the above notion also to the non-asymptotic case: the protocol with no security parameter perfectly  compute a functionality $f$.
	
	A two-party protocol {\sf securely computes a functionality ensemble $f$ with oracle to a channel $C$}, if it does so according to the above definition when the parties have access to a trusted party computing $C$. All the above adjectives naturally extend to this setting.
\end{definition}

\paragraph{Oblivious Transfer.}
The (one-out-of-two) oblivious transfer functionality is defined as follows.
\begin{definition}[oblivious transfer functionality $f_{\OT}$]\label{def:OTfunc}
	The oblivious transfer functionality over $\zo \times (\zs)^2$ is defined by  $f_{\OT} (i,(\sigma_0,\sigma_1)) = (\perp,\sigma_i)$.
\end{definition}
A protocol is $\ast$ secure OT,   for \\$\ast\in \set{\text{semi-honest statistically/computationally/computationally non-uniform}}$, if it  compute the $f_{\OT}$  functionality with $\ast$ security.





% \begin{definition}[Computational oblivious transfer, semi-honest model]
% A protocol $\Pi=(\Ac,\Bc)$ is a semi-honest 1-out-of-2 computational oblivious transfer (comp-OT) protocol if the following holds. Given a common input $1^{\pk}$, the parties $\Ac$ and $\Bc$ run the protocol $\Pi(1^\pk)$ (in an honest manner) and    
% $\Ac$ outputs $X=(m_1,m_2)\in \zo\times\zo$ and has a view $U$ and $\Bc$ outputs $Y=(i,\hat{m})\in\zo\times\zo$ and has a view $V$, and the following properties are satisfied:
% \begin{enumerate}
%     \item \textbf{Correctness:} 
%     $\pr{\hat{m}\neq m_i}<\negl(\pk).$ 
    
%     \item \textbf{A's Privacy:} For every \ppt $\Dc$ and every sufficiently large $\pk$:
%     $\pr{\Dc(V)=m_{i-1}}<(1+\negl(\pk))/2$
    
%     \item \textbf{B's Privacy:} For every \ppt $\Dc$ and every sufficiently large $\pk$:
%     $\pr{\Dc(U)=i}<(1+\negl(\pk))/2$  
% \end{enumerate}
% \end{definition}

We make use of the following useful results by Wullschleger on oblivious transfer amplification from weak channels.
\begin{theorem}[\cite{Wullschleger09}, from \WEC to statistically secure \OT]\label{thm:WEC TO OT IT}
    There exists an oracle aided protocol $\Pi$ such that the following holds: Given a $(\alpha,p,q)$-\WEC $C$, if $44(\alpha+p)\le 1-q$ then $\Pi^{C}(1^\pk)$ is a semi-honest statistically secure \OT.
\end{theorem}

The following computational version of \cref{thm:WEC TO OT IT} is implicit in \cite{Wullschleger09} and is based on the computational proof explicitly stated in \cite{Wul07} (see Section 6 in \cite{Wullschleger09} for discussion).   

\begin{theorem}[\cite{Wullschleger09,   Wul07}, from \CompWEC to computinally secure \OT]\label{thm:WEC TO OT Comp}
    There exists an oracle aided protocol $\Pi$ such that the following holds: Given a $(\alpha,p,q)$-\CompWEC protocol $\Lambda$, if $44(\alpha+p)\le 1-q$ then $\Pi^{\Lambda}$ is a semi-honest computational secure \OT.
\end{theorem}



% \begin{definition}[Computational 1-out-of-2 Oblivious Transfer, semi-honest model]
% A protocol $\Pi=(\Ac,\Bc)$ is a semi-honest 1-out-of-2 $(\eps,\alpha,\beta)$-oblivious transfer (OT) protocol if the following holds. 

% The parties $\Ac$ and $\Bc$ run the protocol (in an honest manner) and    
% $\Ac$ outputs $X=(m_1,m_2)\in \zo\times\zo$ and has a view $U$ and $\Bc$ outputs $Y=(i,\hat{m})\in\zo\times\zo$ and has a view $V$, and following properties are satisfied:
% \begin{enumerate}
%     \item \textbf{Correctness:} 
%     $\pr{\hat{m}\neq m_i}<\eps.$ 
    
%     \item \textbf{A's Privacy:} For every adversary $\Dc$:
%     $\pr{\Dc(V)=m_{i-1}}<(1+\alpha)/2$
    
%     \item \textbf{B's Privacy:} For every adversary $\Dc$: $\pr{\Dc(U)=i}<(1+\beta)/2$  
% \end{enumerate}
% \end{definition}
\begin{tikzpicture}
    \begin{axis}[
        width=\linewidth,
        ylabel style={font=\scriptsize,yshift=-0.6em},
        y tick label style={font=\scriptsize},
        x tick label style={font=\scriptsize},
        ybar,
        %axis lines=left,  
        ymajorgrids,
        symbolic x coords={XGBoost, gMLP, PedCA-FT},
        %xtick={XGBoost, LightGBM, {ours}},
        ylabel={Sensitivity},
        ymin=0,
        ymax=55,
        bar shift=0pt,
        %bar width=0.5cm,
        nodes near coords, 
        nodes near coords style={font=\scriptsize}, 
        %enlargelimits=0.10,
    ]
        \addplot[
            fill=Set2-A,
            ybar,
            error bars/.cd,
            y dir=both,
            y explicit,
        ] coordinates {
            (XGBoost, 39.04) += (0, 8.1) -= (0, 7.53)
        };
        \addplot[
            fill=Set2-B,
            ybar,
            error bars/.cd,
            y dir=both,
            y explicit,
        ] coordinates {
            (gMLP, 8.22) += (0, 5.6) -= (0, 3.46)
        };
        \addplot[
            fill=Set2-C,
            ybar,
            error bars/.cd,
            y dir=both,
            y explicit,
        ] coordinates {
            (PedCA-FT, 42.47) += (0, 8.11) -= (0, 4.73)
        };
    \end{axis}
\end{tikzpicture}
\section{Upper bound for total out-degrees of nodes w.r.t. $\None \cup \Nnotv$}
\label{sec:upper_bound_1}

  In this section, we show an upper bound for the total out-degrees of the nodes corresponding to strings in $\None \cup \Nnotv  \subseteq \M(T')$.
  Recall that $x \in \None \cup \Nnotv$ implies $x \notin \LeftM(T)$.
  
We first describe useful properties of
strings $x \in \None \cup \Nnotv$.
  
  \begin{lemma} \label{lem:exist1}
    Any $x \in \None \cup \Nnotv$ occurs in $T$.
  \end{lemma}

  \begin{proof}
%  We prove Lemma~\ref{lem:exist1} by exhaustion.
    In the case $x \in \None$, since $x \in \RightM(T)$, 
    $x$ occurs in $T$. 

    Let us consider the case $x \in \Nnotv$ that is of Type $\rm{(i)}$, $\rm{(ii)}$, $\rm{(iii)}$ or $\rm{(iv)}$.
    Since $x$ is not of Type $\rm{(v)}$, 
    if all occurrences of $x$ in $T'$ are crossing occurrences of $x$ in $T'$, then $x \notin \M(T')$.
    Therefore, $x$ occurs in $T$.

    Let us consider the case $x \in \Nnotv$ that is of Type $\rm{(v)}$.
    Due to the definition of $\Nnotv$, there exists a distinct right-extension of $x$ in $T'$ other than the right-extension(s) of the crossing occurrence(s) of $x$.
    Therefore, there is a non-crossing occurrence of $x$ in $T'$
    as shown in Figure~\ref{fig:exsit1},
    implying that $x$ occurs in $T$.
  \end{proof}

  \begin{figure}[hbt]
    \centering
    \includegraphics[keepaspectratio,scale=0.33]{exist1.pdf}
    \caption{Illustration for Lemma~\ref{lem:exist1} where $i$ is the edited position and $a, b, c$ differ from each other.}
    \label{fig:exsit1}
  \end{figure}

%  \subsection{Case that $x\in \None \cup \Nnotv$ contains the edit position}

  \begin{lemma} \label{lem:sp123}
    For any $x \in \None \cup \Nnotv$ that is of Type $\rm{(i)}$, $\rm{(ii)}$ or $\rm{(iii)}$,
    there does not exist $y\in \None \cup \Nnotv$ such that
    $|y|>|x|$ and $S_{x_L}=S_{{y}_{G}}$,
    where $G \in \{L,R\}$.
  \end{lemma}
  
  \begin{proof}
    If $x_L$ is a prefix of $T'$, then clearly there is no $y$ satisfying
    $|y|>|x|$ and $S_{x_L}=S_{{y}_{G}}$.
    In what follows, we consider the case that $x_L$ is not a prefix of $T'$.
    
    For a contrary, suppose that for $x \in \None \cup \Nnotv$ that is of Type $\rm{(i)}$, $\rm{(ii)}$ or $\rm{(iii)}$, 
    there exists $y\in \None \cup \Nnotv$ such that
    $|y|>|x|$ and $S_{x_L}=S_{{y}_{G}}$, where $G \in \{L,R\}$.
    See also Figure~\ref{fig:sp123}.
    Let $a$ be the character immediately before $x_L$.
    Since $x$ is of Type $\rm{(i)}$, $\rm{(ii)}$ or $\rm{(iii)}$,
    every crossing occurrence of $x$ in $T'$ is immediately
    preceded by $a$.
    Because $x \in \M(T')$,
    it holds that $x \in \Prefix(T')$,
    or there is a distinct character $b \in \Sigma \setminus \{a\}$
    such that $bx$ occurs in $T'$.
    This implies that there is a non-crossing occurrence of $x$ in $T'$,
    which is as a prefix of $T$ or is immediately preceded by $b$ in $T$.
    By Lemma~\ref{lem:exist1}, $y$ occurs in $T$, and thus $ax$ that is a suffix of $y$ also occurs in $T$.
    Hence $x \in \LeftM(T)$, however, this contradicts that $x \notin \LeftM(T)$.
  \end{proof}

%  Lemma~\ref{lem:sp123} states that $x$ and $y$ cannot occur as in Figure~\ref{fig:sp123}

  \begin{figure}[bth]
    \centering
    \includegraphics[keepaspectratio,scale=0.33]{sp123.pdf}
    \caption{Illustration for Lemma~\ref{lem:sp123}: impossible occurrences of $x$ and $y$ with $S_{x_L} = S_{y_G}$.}
    \label{fig:sp123}
  \end{figure}

  \begin{lemma} \label{lem:sp45}
    For any $x \in \None \cup \Nnotv$ that is of Type $\rm{(iv)}$ or $\rm{(v)}$, 
    there do not exist $y,z \in \None \cup \Nnotv$ with
    $|y|>|x|$ and $|z|>|x|$
    satisfying
    $S_{x_L}=S_{{y}_{G}}$ and $S_{x_R}=S_{{z}_{F}}$ simultaneously,
    where $G,F \in \{L,R\}$.
  \end{lemma}

  
  \begin{proof}
    If $x_L$ is a prefix of $T'$, then clearly there is no $y$ satisfying
    $|y|>|x|$ and $S_{x_L}=S_{{y}_{G}}$.
    In what follows, we consider the case that $x_L$ is not a prefix of $T'$.
    
    For a contrary, 
    suppose that for $x \in \None \cup \Nnotv$ that is of Type $\rm{(iv)}$ or $\rm{(v)}$, 
    there exist $y,z \in \None \cup \Nnotv$ with $|y|>|x|$ and $|z|>|x|$ such that $S_{x_L}=S_{{y}_{G}}$ and $S_{x_R}=S_{{z}_{F}}$ at the same time, where $G,F \in \{L,R\}$.
    Let the character immediately before $x_L$ and the character immediately before $x_R$ be $a$ and $c$~($a \neq c$), respectively.
    \rhnote*{changed "b" to "c"}{%
    By Lemma~\ref{lem:exist1}, $y$ and $z$ occur in $T$, and thus $ax$ that is a suffix of $y$ and $cx$ that is a suffix of $z$ both occur in $T$.
    }%
    Therefore, $x \in \LeftM(T)$, however, this contradicts that $x \notin \LeftM(T)$.
  \end{proof}
  

  \subsection{Correspondence between $\None \cup \Nnotv$ and $\M(T)$}

  For any $x \in \None \cup \Nnotv$ that is of Type $\rm{(i)}$, $\rm{(ii)}$ or $\rm{(iii)}$, we associate $x$ with $S_{x_L}$.
  For any $x \in \None \cup \Nnotv$ that is of Type $\rm{(iv)}$ or $\rm{(v)}$, 
  if there does not exist $y\in \None \cup \Nnotv$
  such that $|y|>|x|$ and 
  $S_{x_L}=S_{{y}_{G}}$ with $G \in \{L,R\}$, we associate $x$ with $S_{x_L}$, 
  and otherwise we associate $x$ with $S_{x_R}$.

  By Lemma~\ref{lem:sp123} and Lemma~\ref{lem:sp45}, each $x \in \None \cup \Nnotv$ can be associated to a distinct string $S_{x_G}$ with $G \in \{L,R\}$.
  %
  Note however that $S_{x_G}$ may not be maximal in $T$.
  Thus we introduce a function $U$
  that bridges each $x \in \None \cup \Nnotv$ to a distinct maximal substring in $T$.
%  For any $x \in \None \cup \Nnotv$
%  we define $U(x)$ to which $x$ corresponds, by using $S_{x_G} \: (G \in \{L,R\})$ which $x$ corresponds,
%  as follows:

  \begin{definition} \label{def:U_x}
    For any $x \in \None \cup \Nnotv$,
    let $U(x)=\lrep_T({S_{x_G}})$ (see Figure~\ref{fig:U_x}).
  \end{definition}
    By Lemma~\ref{lem:exist1}, $x$ occurs in $T$ and thus
    its suffix $S_{x_G}$ also occurs in $T$.
    Hence $U(x)=\lrep_T({S_{x_G}})$ is well defined.

%  \sinote*{added}{%
%  Recall that $x$ is a maximal repeat in $T'$
%  and thus $x$ occurs at least twice in $T'$,
%  implying that $S_{x_G}$ occurs at least once in $T$.
%  Thus $U(x)=\lrep_T({S_{x_G}})$ is well defined.
%  }%
  

%  \sinote*{modified}{%
%  \begin{definition} \label{def:U_x}
%    For any $x \in \None \cup \Nnotv$,
%    let
%    \[
%    U(x) =
%    \begin{cases}
%      \lrep_T({S_{x_G}}) & \mbox{for deletion} \\
%      \lrep_T({T[i]S_{x_G}[2..|S_{x_G}|]}) & \mbox{for insertion and substitution}
%    \end{cases}
%    \]
%  \end{definition}
%  }%

%  See Figure~\ref{fig:U_x} for an illustration for $U(x)$.
  


  \begin{figure}[H]
    \centering
    \includegraphics[keepaspectratio,scale=0.33]{U_x.pdf}
    \caption{Illustration for $U(x)$ $(a\neq b)$.}
    \label{fig:U_x}
  \end{figure}

  \begin{lemma} \label{lem:U_x}
    For any $x \in \None \cup \Nnotv$, $U(x) \in \M(T)$.
  \end{lemma}

  \begin{proof}
    By Definition~\ref{def:U_x}, $U(x) = \lrep_T({S_{x_G}}) \in \LeftM(T)$.
    Therefore, it suffices for us to prove $U(x) \in \RightM(T)$.
    %
    From now on, we consider the four following cases:
%    \begin{enumerate}
%    \item[(a)] $x \in \None$. 
%    \item[(b)] $x \in \Nnotv$ and $x$ is of Type $\rm{(i)}$, $\rm{(ii)}$ or $\rm{(iv)}$.
%    \item[(c)] $x \in \Nnotv$ and $x$ is of Type $\rm{(iii)}$.
%    \item[(d)] $x \in \Nnotv$ and $x$ is of Type $\rm{(v)}$.
%    \end{enumerate}

    \noindent \textbf{Case (a) $x \in \None$:}
    In this case, $x \in \RightM(T)$, therefore $S_{x_{G}}$ that is a suffix of $x$ also satisfies $S_{x_{G}} \in \RightM(T)$.
    Hence, $U(x) = \lrep_T({S_{x_G}}) \in \RightM(T)$.

    \noindent \textbf{Case (b) $x \in \Nnotv$ and $x$ is of Type $\rm{(i)}$, $\rm{(ii)}$ or $\rm{(iv)}$:}
%    Let us consider the case that $x \in \Nnotv$ and $x$ is in $\rm{(i)}$, $\rm{(ii)}$ or $\rm{(iv)}$.
    Let the character immediately after all crossing occurrences of $x$ in $T'$ be $a$.
    There exists $xb \: (b \ne a)$ in $T'$ or $x \in \Suffix(T')$ because $x \in \M(T')$.
    Since the character immediately after all crossing occurrences of $x$ in $T'$ is $a$, 
    then there exists $xb \: (b \ne a)$ or $x \in \Suffix(T)$ in $T$.
    Hence, $S_{x_{G}} \in \RightM(T)$ since the character immediately after $S_{x_{G}}$ is $a$.
    Thus, $U(x) = \lrep_T({S_{x_G}}) \in \RightM(T)$.
  % $x \in \Suffix(T)$だと$x \in \RightM(T)$となるから考慮しなくていいけど論理的におかしなことは書いてない

   \noindent \textbf{Case (c) $x \in \Nnotv$ and $x$ is of Type $\rm{(iii)}$:}
%    Let us consider the case that $x \in \Nnotv$ and $x$ is in $\rm{(iii)}$.
    Since $x$ is of Type $\rm{(iii)}$, we associate $x$ with $S_{x_L}$.
    Because $x$ is of Type $\rm{(iii)}$ and $S_{x_L}$ is a suffix of a $S_{x_ R}$, $S_{x_{G}} \in \RightM(T)$ holds.
    Thus, $U(x) = \lrep_T({S_{x_G}}) \in \RightM(T)$.

   \noindent \textbf{Case (d) $x \in \Nnotv$ and $x$ is of Type $\rm{(v)}$:}
%    Let us consider the case that $x \in \Nnotv$ and $x$ is in $\rm{(v)}$.
    Let the character immediately after $S_{x_ G}$ be $a$.
    Since there exists a distinct right-extension of $x$ in $T'$ other than the right-extension(s) of the crossing occurrence(s) of $x$, there exists $xb$ $(b \neq a)$ in $T$.
    Therefore, $S_{x_ G} \in \RightM(T)$.
    Thus, $U(x) = \lrep_T({S_{x_G}}) \in \RightM(T)$.

    Consequently, we have $U(x) \in \M(T)$.
  \end{proof}

  The next lemma states the uniqueness of $U(x)$.
  \begin{lemma} \label{lem:U_xU_y}
    For any $x,y\in \None \cup \Nnotv$ with $x \neq y$, $U(x) \neq U(y)$.
  \end{lemma}

  \begin{proof}
    Suppose that there exist $x,y\in \None \cup \Nnotv$ such that
    $x \neq y$ and $U(x) = U(y)$.
    Let $x$ and $y$ correspond to $S_{x_{G}}$ and $S_{y_{F}}$, respectively,
    where $G,F \in \{L,R\}$.
    Let $U(x)=AS_{x_{G}},U(y)=BS_{y_{F}} \:(A,B\in \Substr(T))$,
    and assume without loss of generality that $|S_{x_{G}}|<|S_{y_{F}}|$.
    Then $|A|>|B|$ because $U(x) = U(y)$.
    \rhnote*{added "$U(y)=BS_{y_{F}} \in \M(T)$ by Lemma~\ref{lem:U_x}"}{%
    Since $U(y)=BS_{y_{F}} \in \M(T)$ by Lemma~\ref{lem:U_x}, and since $BS_{x_{G}}$ is a prefix of $BS_{y_{F}}$ (see Figure~\ref{fig:U_xU_y}), we have $BS_{x_{G}}\in \LeftM(T)$.
    }% 
    This contradicts $\lrep_T({S_{x_{G}}})=AS_{x_{G}}$.
  \end{proof}

  \begin{figure}[H]
    \centering
    \includegraphics[keepaspectratio,scale=0.33]{U_xU_y.pdf}
    \caption{Illustration for the proof of Lemma~\ref{lem:U_xU_y}, where $U(x) = U(y)$.}
    \label{fig:U_xU_y}
  \end{figure}

  \begin{comment}
  \subsection{Case that $x\in \None \cup \Nnotv$ does not contain the edit position}

  From now on, in this subsection, we consider the case that $x\in \None \cup \Nnotv$ does not contain the edit position.
  
  \begin{lemma}
    \label{lem:eps1}
    The number of $x\in \None \cup \Nnotv$ that does not contain the edit position is 1.
  \end{lemma}

  \begin{proof}
    It can be proved by making the same arguments as for Lemma~\ref{lem:sp123} and Lemma~\ref{lem:sp45}.
  \end{proof}%
  \end{comment}

  \subsection{Upper bound w.r.t. $\None \cup \Nnotv$}

  \begin{lemma} 
    \label{lem:dt1}
    $\sum_{x \in \None \cup \Nnotv}\D_{T'}(x) \le 3\size+2$.
  \end{lemma}

  \begin{proof}\rhnote*{changed}{%
    Let $U(x)=\lrep_T({S_{x_G}})$, where $G \in \{L,R\}$.
    Since $S_{x_G}$ is a suffix of $x$, $\D_{T}(x) \leq \D_{T}(U(x))$.
    Since there are at most two distinct characters immediately after the crossing occurrences of $x$, 
    $\D_{T'}(x) \leq \D_{T}(U(x))+2$.
    For $U(x) \neq T$, we have $\D_T(U(x)) \ge 1$. Thus $\D_{T'}(x) \le \D_T(U(x))+2 \le 3\D_T(U(x))$.
    For $U(x) = T$, we have $\D_T(U(x))=0$. Thus $\D_{T'}(x) \le 2$.
    %
    By using Lemma~\ref{lem:U_xU_y} and summing up these, we get 
    $\sum_{x \in \None \cup \Nnotv}\D_{T'}(x) \le {\sum_{x \in \None \cup \Nnotv} 3\D_T(U(x))}+2 \le 3\size+2$.
  \end{proof}
  }%


\section{Upper bound for total out-degrees of nodes w.r.t. $\Ntwo \cup \Q$}
\label{sec:upper_bound_2}

In this section, we show an upper bound for the total out-degrees of nodes corresponding to strings that are elements of $\Ntwo \cup \Q \subseteq \M(T')$.

We first present properties of the strings in $\Ntwo \cup \Q$.
In particular, we focus on the strings in $\Ntwo \cup \Qn$,
as the strings in $\Qnotn$ are less important and can be handled in a trivial manner.

  \begin{lemma} \label{lem:exist2}
    Any $x \in \Ntwo \cup \Qn$ occurs in $T$.
  \end{lemma}

  \begin{proof}
    Since $x \in \Ntwo \cup \Qn$, $x \in \LeftM(T)$. Thus $x \in \Ntwo \cup \Qn$ occurs in $T$.
  \end{proof}

  \begin{lemma} \label{lem:sp124}
    For any $x \in \Ntwo \cup \Qn$ that is of Type $\rm{(i)}$, $\rm{(ii)}$ or $\rm{(iv)}$,
    there does not exist $y\in \Ntwo \cup \Qn$ such that $|y|>|x|$ and $P_{x_{G}}=P_{y_{F}}$, where $G,F \in \{L,R\}$.
  \end{lemma}

  \begin{proof}
    The case that $x \in \Ntwo$ follows from a symmetrical argument to Lemma~\ref{lem:sp123}, in which $y$ may belong to $\Ntwo$ or $\Qn$.
    \rhnote*{delete $\Qn = \M(T) \cap \M(T')$}{%
    Let us consider the case that $x \in \Qn$.
    }%
    Suppose that for $x \in \Qn$ which is of Type $\rm{(i)}$, $\rm{(ii)}$ or $\rm{(iv)}$,
    there is $y\in \Ntwo \cup \Qn$ such that $|y|>|x|$ and $P_{x_{G}}=P_{y_{F}}$, where $G,F \in \{L,R\}$.
    If $x_R$ is a suffix of $T'$, then there is no $y$ such that $|y|>|x|$ and $P_{x_{G}}=P_{y_{F}}$.
    From now on consider the case that $x_R$ is not a suffix of $T'$.
    %
    Let $b$ be the character immediately after $x_{G}$.
    Then, since $x$ is of Type $\rm{(i)}$, $\rm{(ii)}$ or $\rm{(iv)}$,
    character $b$ immediately follows every crossing occurrence of $x$ in $T'$.
    Note that $xb$ is a prefix of $y$.
    Due to Lemma~\ref{lem:exist2}, $y$ occurs in $T$,
    implying $xb$ also occurs in $T$.
    Thus the number of right-extensions of $x$ in $T'$
    is no more than the number of right-extensions of $x$ in $T$.
    However, this contradicts $x \in \Qn$.
  \end{proof}

  \begin{lemma} \label{lem:sp35}
    For any $x \in \Ntwo \cup \Qn$ that is of Type $\rm{(iii)}$ or $\rm{(v)}$, 
    there do not exist $y,z \in \Ntwo \cup \Qn$ with
    $|y|>|x|$ and $|z|>|x|$
    satisfying $P_{x_L}=P_{{y}_{G}}$ and $P_{x_R}=P_{{z}_{F}}$ simultaneously,
    where $G,F \in \{L,R\}$.
  \end{lemma}

  \begin{proof}
    If $x_R$ is a suffix of $T'$, then clearly there is no $z$ satisfying
    $|z|>|x|$ and $P_{x_R}=S_{{z}_{G}}$.
    In what follows, we consider the case that $x_R$ is not a suffix of $T'$.
    
    Suppose that for $x \in \Ntwo \cup \Qn$ which is of Type $\rm{(iii)}$ or $\rm{(v)}$, 
    there exist $y,z \in \Ntwo \cup \Qn$ with $|y|>|x|$, $|z|>|x|$
    that satisfy $P_{x_L}=P_{{y}_{G}}$ and $P_{x_R}=P_{{z}_{F}}$ at the same time,
    where $G,F \in \{L,R\}$.
    Let $b$ and $d$~($b \neq d$) be the character immediately after $x_L$ in $T'$ and the character immediately after $x_R$ in $T'$, respectively.
    By Lemma~\ref{lem:exist2}, $y$ and $z$ occur in $T$, and hence $xb$ that is a prefix of $y$ and $xd$ that is a prefix of $z$ also occur in $T$.
    Therefore, $x \in \RightM(T)$. However, if $x \in \Ntwo(T)$, this contradicts $x \notin \RightM(T)$.
    Also, if $x \in \Qn$, the number of right-extensions of $x$ in $T'$ do not increase from the number of right-extensions of $x$ in $T$. However, this contradicts $x \in \Qn$.
  \end{proof}

  \subsection{Correspondence between $\Ntwo \cup \Qn$ and $\M(T)$}

  For any $x \in \Ntwo \cup \Qn$ that is of Type $\rm{(i)}$, $\rm{(ii)}$ or $\rm{(iv)}$, then we associate $x$ with both $P_{x_L}$ and $P_{x_R}$.
  For any $x \in \Ntwo \cup \Qn$ that is of Type $\rm{(iii)}$ or $\rm{(v)}$, 
  \begin{itemize}
    \item if there exists $y\in \Ntwo \cup \Qn$ with $|y|>|x|$
  such that $P_{x_R}=P_{{y}_{G}}$ where $G \in \{L,R\}$, then we associate $x$ with $P_{x_L}$ (see Figure~\ref{fig:pxrex});
    \item if there exists $y\in \Ntwo \cup \Qn$ with $|y|>|x|$ such that $P_{x_L}=P_{{y}_{G}}$ where $G \in \{L,R\}$, then we associate $x$ with $P_{x_R}$ (see Figure~\ref{fig:pxlex});
    \item otherwise, we associate $x$ with both $P_{x_L}$ and $P_{x_R}$.
  \end{itemize}

  \begin{figure}[H]
    \centering
    \includegraphics[keepaspectratio,scale=0.33]{pxrex.pdf}
    \caption{When there exists $y\in \Ntwo \cup \Qn$ with $|y|>|x|$
      such that $P_{x_R}=P_{{y}_{G}}$, where $G \in \{L,R\}$.}
    \label{fig:pxrex}
  \end{figure}

  \begin{figure}[H]
    \centering
    \includegraphics[keepaspectratio,scale=0.33]{pxlex.pdf}
    \caption{When there exists $y\in \Ntwo \cup \Qn$ with $|y|>|x|$ such that $P_{x_L}=P_{{y}_{G}}$, where $G \in \{L,R\}$.}
    \label{fig:pxlex}
  \end{figure}

  
  By Lemmas~\ref{lem:sp124} and~\ref{lem:sp35}, each $x \in \Ntwo \cup \Qn$ corresponds to a distinct string $P_{x_G}$, where $G \in \{L,R\}$.
  Below, for each $x \in \Ntwo \cup \Qn$,
  we define $H(x)$ and $I(x)$ to which $x$ corresponds:

  \begin{definition} \label{def:H_x_I_x}
    For each $x \in \Ntwo \cup \Qn$
    associated to ${P_{x_L}}$, let $H(x)=\rrep_T({P_{x_L}})$.
    For each $x \in \Ntwo \cup \Qn$
    associated to ${P_{x_R}}$, let $I(x)=\rrep_T({P_{x_R}})$.
    See Figure~\ref{fig:H_xI_x}.
    \sinote*{added}{%
    When there is only one crossing occurrence of $x$ (i.e. $x_L = x_R$),
    only $H(x)$ is defined as above and $I(x)$ is undefined.
    }%
  \end{definition}
$H(x)$ (resp. $I(x)$) is undefined
for any $x \in \Ntwo \cup \Qn$ that is \emph{not} associated to ${P_{x_L}}$
(resp. ${P_{x_R}}$).

\sinote*{added}{%
By Lemma~\ref{lem:exist2} every $x \in \Ntwo \cup \Qn$ occurs in $T$,
and thus $H(x)$ and $I(x)$ are well defined
when $x$ is associated to $P_{x_L}$ and $P_{x_R}$, respectively.
}%
  
%  \begin{definition} \label{def:H_x_I_x}
%    For any $x \in \Ntwo \cup \Qn$,
%    let $H(x)=\rrep_T({P_{x_L}})$.
%    let $I(x)=\rrep_T({P_{x_R}})$.
%  \end{definition}
%
%  Note that when we associate $x$ with only $P_{x_L}$ or $P_{x_R}$ for $x$, we only define $H(x)$ or $I(x)$ for $x$ and 
%  even if we only define $H(x)$ or $I(x)$ for $x$, we state both in follow proofs, but the proof is not incomplete.


  \begin{figure}[H]
    \centering
    \includegraphics[keepaspectratio,scale=0.33]{H_xI_x.pdf}
    \caption{Illustration for $H(x)$ and $I(x)$ ($a\neq b, c\neq d$).}
    \label{fig:H_xI_x}
  \end{figure}

  \begin{lemma} \label{lem:H_x_I_x}
    For any $x \in \Ntwo \cup \Qn$, $H(x) \in \M(T)$ if $H(x)$ is defined,
    and $I(x) \in \M(T)$ if $I(x)$ is defined.
  \end{lemma}
  
%  \begin{lemma} \label{lem:H_x_I_x}
%    For any $x \in \Ntwo \cup \Qn$, $H(x), I(x) \in \M(T)$.
%  \end{lemma}

  \begin{proof}
    By Definition~\ref{def:H_x_I_x}, $H(x), I(x)\in \RightM(T)$.
    Therefore, it suffices for us to prove $H(x), I(x)\in \LeftM(T)$.
    For any $x \in \Ntwo \cup \Qn$, $x \in \LeftM(T)$.
    Since $P_{x_G} \: (G \in \{L,R\})$ is a prefix of $x$,
    we have $P_{x_G} \in \LeftM(T)$.
    Hence $H(x), I(x)\in \LeftM(T)$ holds.
  \end{proof}

  \begin{lemma} \label{lem:H_xH_y}
    For any $x,y \in \None \cup \Nnotv$ with $x \neq y$,
    let $\mathcal{L}$ be a list of $H(x)$, $I(x)$, $H(y)$, $I(y)$
    which are defined.
    Then the elements in $\mathcal{L}$ differ from each other.
  \end{lemma}

  \begin{proof}
    By a symmetrical argument to Lemma~\ref{lem:U_xU_y}.
  \end{proof}

  \subsection{Upper bound w.r.t. $\Ntwo \cup \Q$}

  \begin{lemma} 
    \label{lem:dt2}
    $\sum_{x \in \Ntwo \cup \Q}\D_{T'}(x) \le 3\size+2$.
  \end{lemma}

  \begin{proof}
    Below, we consider all the four possible cases depending on whether $x \in \Ntwo$ or $x \in \Qn$, and whether $H(x), I(x) \neq T$. 

    \noindent {\large \textbf{When $x\in \Ntwo$ and $H(x), I(x) \neq T$:}}
    \begin{itemize}
    \item
    First, we consider the case that $x$ is associated with both $H(x)$ and $I(x)$.
    Since $x \in \Ntwo$, then
      $x \notin \RightM(T)$.
    Therefore, the number of characters that are immediately after $x$ in $T$ is at most one.
    Moreover, there are at most two distinct characters immediately after the crossing occurrences of $x$.
    Hence, there are at most three distinct characters immediately after $x$ in $T'$, namely we have 
    \begin{equation}\label{equ:equ1}
      \D_{T'}(x) \le 3.
    \end{equation}
    In addition, since ${H(x),I(x) \ne T}$, it holds that $\D_{T}(H(x)),\D_{T}(I(x)) \ge 1$.
    By Inequality~\ref{equ:equ1}, we get $\D_{T'}(x) \le 3 \le \D_{T}(H(x))+\D_{T}(I(x))+1 \le 2\D_{T}(H(x))+2\D_{T}(I(x)).$

    \item Second, we consider the case that $x$ is associated with only one of $H(x)$ or $I(x)$.
    \begin{itemize}
    \item
    Assume that we associate $x$ with $H(x)$.
    Since $x \in \Ntwo$, then
      $x \notin \RightM(T)$.
    Therefore, the number of characters immediately after $x$ in $T$ is at most one.
    \rhnote*{added the case $x$ has only one crossing occurrence}{%
    In this case, we do not associate $x$ with $I(x)$, hence, $x$ has only one crossing occurrence or there exists $y\in \Ntwo \cup \Qn$ such that $|y|>|x|$ and $P_{x_R}=P_{{y}_{G}}$ where $G \in \{L,R\}$.
    When $x$ has only one crossing occurrence, there are at most one character immediately after the crossing occurrence of $x$.
    When there exists  $y\in \Ntwo \cup \Qn$ such that $|y|>|x|$ and $P_{x_R}=P_{{y}_{G}}$ where $G \in \{L,R\}$, then such $y$ occurs in $T$ due to Lemma~\ref{lem:exist2}.
    Therefore, although there are at most two distinct characters immediately after the crossing occurrences of $x$, one of them is the character immediately after $x$ in $T$ as shown in Figure~\ref{fig:xaex}.
    }%
    Hence, we have 
    \begin{equation}\label{equ:equ2}
      \D_{T'}(x) \le 2.
    \end{equation}
    In addition, since ${H(x) \ne T}$, then $\D_{T}(H(x)) \ge 1$ holds.
    By Inequality~\ref{equ:equ2}, we get $\D_{T'}(x) \le 2 \le \D_{T}(H(x))+1 \le 2\D_{T}(H(x)).$
    
    \item
    Let us assume that we associate $x$ with $I(x)$. In the same way as we associate $x$ with $H(x)$, we get $\D_{T'}(x) \le 2 \le \D_{T}(I(x))+1 \le 2\D_{T}(I(x))$.
    \end{itemize}
    \end{itemize}

    \begin{figure}[H]
      \centering
      \includegraphics[keepaspectratio,scale=0.33]{xaex.pdf}
      \caption{$xa$ occurs in $T$, where $a$ is the character immediately after the crossing occurrence $x_R$.}
      \label{fig:xaex}
    \end{figure}
    

    \noindent {\large \textbf{When $x\in \Ntwo$ and $H(x) = T$ or $I(x) = T$:}}
    \begin{itemize}
    \item
    Let $H(x) = T$. Now that $H(x)$ is defined, $x$ is associated with $H(x)$.
    \begin{itemize}
     \item First, we consider the case that we associate $x$ with both $H(x)$ and $I(x)$.
    In the same way as in Inequality~\ref{equ:equ1}, we get $\D_{T'}(x) \le 3$.
    In addition, since $H(x)=T$ and Lemma~\ref{lem:H_xH_y} holds, $I(x) \neq T$ and thus $\D_{T}(I(x)) \ge 1$ holds.
    Hence, we have $\D_{T'}(x) \le 3 \le \D_{T}(I(x))+2 \le 2\D_{T}(I(x))+2 \le 2\D_{T}(H(x))+2\D_{T}(I(x))+2$.
    \item Second, we consider the case that we only associate $x$ with $H(x)$.
    Since $H(x)=T$, $\D_{T}(H(x)) =0$ holds.
    In the same way as in Inequality~\ref{equ:equ2}, we get $\D_{T'}(x) \le 2$.
    Thus, we have $\D_{T'}(x) \le 2 \le 2\D_{T}(H(x))+2$.
    \end{itemize}

    \item
    Let $I(x) = T$. Now that $I(x)$ is defined, $x$ is associated with $I(x)$.
    In the same way as in the case for $H(x)=T$, 
    we get $\D_{T'}(x) \le 3 \le \D_{T}(H(x))+2 \le 2\D_{T}(H(x))+2 \le 2\D_{T}(H(x))+2\D_{T}(I(x))+2$ in the case that we associate $x$ with both $H(x)$ and $I(x)$,
    and we get $\D_{T'}(x) \le 2 \le 2\D_{T}(I(x))+2$ in the case that we only associate $x$ with $I(x)$.
   \end{itemize}

%   \medskip 
   \noindent {\large \textbf{When $x\in \Qn$ and $H(x), I(x) \neq T$:}}    
    Here, we analyze $\D_{T'}(x)-\D_T{(x)}$ since $x \in \Qn$.
    \begin{itemize}
    \item
    First, we consider the case that we associate $x$ with both $H(x)$ and $I(x)$.
    There are at most two distinct characters immediately after the crossing occurrences of $x$.
    Hence,
    \begin{equation}\label{equ:equ3}
      \D_{T'}(x)-\D_T{(x)} \le 2.
    \end{equation}
    In addition, since ${H(x),I(x) \ne T}$, then $\D_{T}(H(x)),\D_{T}(I(x)) \ge 1$ holds.
    By Inequality~\ref{equ:equ3}, we get $\D_{T'}(x)-\D_T{(x)} \le 2 \le \D_{T}(H(x))+\D_{T}(I(x)) \le \D_{T}(H(x))+\D_{T}(I(x))$.

    \item
    Second, we consider the case that we associate $x$ with only one of $H(x)$ or $I(x)$.
    Here, let us assume that we associate $x$ with $H(x)$.
    \rhnote*{added the case $x$ has only one crossing occurrence}{%
    In this case, we do not associate $x$ with $I(x)$, hence, $x$ has only one crossing occurrence or there exists $y\in \Ntwo \cup \Qn$ such that $|y|>|x|$ and $P_{x_R}=P_{{y}_{G}}$ where $G \in \{L,R\}$.
    When $x$ has only one crossing occurrence, there are at most one character immediately after the crossing occurrence of $x$.
    When there exists  $y\in \Ntwo \cup \Qn$ such that $|y|>|x|$ and $P_{x_R}=P_{{y}_{G}}$ where $G \in \{L,R\}$, then such $y$ occurs in $T$ due to Lemma~\ref{lem:exist2}.
    Therefore, although there are at most two distinct characters immediately after the crossing occurrences of $x$, one of them is the character immediately after $x$ in $T$ as shown in Figure~\ref{fig:xaex}.
    }%
    Hence, we have
    \begin{equation}\label{equ:equ4}
      \D_{T'}(x)-\D_T{(x)} \le 1.
    \end{equation}
    In addition, since ${H(x) \ne T}$, then $\D_{T}(H(x)) \ge 1$ holds.
    By Inequality~\ref{equ:equ4}, we get $\D_{T'}(x)-\D_T{(x)} \le 1 \le \D_{T}(H(x))$.
    In the case that we associate $x$ with $I(x)$, in the same way as we associate $x$ with $H(x)$, we get $\D_{T'}(x)-\D_T{(x)} \le 1 \le \D_{T}(I(x))$.
    \end{itemize}

%   \medskip 
   \noindent {\large \textbf{When $x\in \Qn$ and $H(x) = T$ or $I(x) = T$:}}
   Here, we analyze $\D_{T'}(x)-\D_T{(x)}$ since $x \in \Qn$.
   \begin{itemize}
    \item
    Let $H(x) = T$. Since $H(x)$ is defined, $x$ is associated with $H(x)$.
    \begin{itemize}  
     \item First, let us consider the case that we associate $x$ with both $H(x)$ and $I(x)$.
    In the same way as in Inequality~\ref{equ:equ3}, we get $\D_{T'}(x)-\D_T{(x)} \le 2$.
    In addition, since $H(x)=T$ and Lemma~\ref{lem:H_xH_y} holds, $I(x) \neq T$ and thus $\D_{T}(I(x)) \ge 1$ holds.
    Hence $\D_{T'}(x)-\D_T{(x)} \le 2 \le \D_{T}(I(x)) + 1 \le \D_{T}(H(x)) + \D_{T}(I(x)) + 1$.

     \item Second, let us consider the case that we only associate $x$ with $H(x)$.
    Since $H(x)=T$, $\D_{T}(H(x))=0$ holds.
    In the same way as in Inequality~\ref{equ:equ4}, we get $\D_{T'}(x)-\D_T{(x)} \le 1$.
    Thus, we have $\D_{T'}(x)-\D_T{(x)} \le 1 \le \D_{T}(H(x))+1$.
    \end{itemize}

   \item
    Let $I(x) = T$. Since $I(x)$ is defined, $x$ is associated with $I(x)$.
    In the same way as in the case for $H(x)=T$, 
    we get $\D_{T'}(x)-\D_T{(x)} \le 2 \le \D_{T}(H(x)) + 1 \le \D_{T}(H(x)) + \D_{T}(I(x)) +1$ in the case that we associate $x$ with both $H(x)$ and $I(x)$,
    and we get $\D_{T'}(x)-\D_T{(x)} \le 1 \le \D_{T}(I(x))+1$ in the case that we only associate $x$ with
    \rhnote*{deleted "only one of $H(x)$"}{%
    $I(x)$.
    }%
   \end{itemize}
    
    \begin{table}[h]
      \centering
      \caption{Upper bounds for each case of Lemma~\ref{lem:dt2}.} 
      \label{inequality}
      \fontsize{9pt}{10pt}\selectfont
      \begin{tabular}{|c|c|c|} \hline
        & When $H(x) \ne T \land I(x) \ne T$ & When $H(x) = T \lor I(x) = T$ \\ \hline
        $\D_{T'}(x)-\D_T{(x)} \: (x \in \Qn)$ & $\leq \D_T{(H(x))}+\D_T{(I(x))}$ & $\leq \D_T{(H(x))}+\D_T{(I(x))}+1$ \\\hline
        $\D_{T'}(x) \: (x \in \Ntwo)$ & $\leq 2(\D_T{(H(x))}+\D_T{(I(x))})$ & $\leq 2(\D_T{(H(x))}+\D_T{(I(x))})+2$ \\ \hline
      \end{tabular}
    \end{table}

%  For simplicity, we state both $H(x)$ and $I(x)$ even if we only associate $x$ with $H(x)$ or $I(x)$.
  %  For instance, if we only associate $x$ with $H(x)$, let $\D_T{(I(x))}$ be zero.

  \noindent {\large \textbf{Wrapping up:}}    
  Table~\ref{inequality} summarizes the bounds obtained above.
  For simplicity,
  let $\D_T{(I(x))} = 0$ when $I(x)$ is undefined,
  and let $\D_T{(H(x))} = 0$ when $H(x)$ is undefined.
  Note that this does not affect our upper bound analysis,
  since no maximal repeats in $T'$ are associated to the undefined $H(x)$'s and $I(x)$'s.
  %
  By Lemma~\ref{lem:H_xH_y}, there is at most one string $x$ such that $H(x) = T$ or $I(x) = T$.
  Thus, by using Lemma~\ref{lem:H_xH_y} and summing up the values in Table~\ref{inequality}, 
  we obtain $\sum_{x\in \Ntwo}\D_{T'}(x) + \sum_{x\in \Qn}(\D_{T'}(x)-\D_T(x)) \le 2\size+2$.
  %
  Also, since the number of out-edges of $x \in \Qnotn$ does not increase,
  we get
  $\sum_{x\in \Qnotn}\D_{T'}(x) + \sum_{x\in \Qn}\D_{T}(x) \le 
  \sum_{x\in \Qnotn}\D_{T}(x) + \sum_{x\in \Qn}\D_{T}(x) \le
  \sum_{x\in \Q}\D_{T}(x) \le \sum_{x\in \M(T)}\D_{T}(x) = \size$.
  %
  By adding $\sum_{x\in \Ntwo}\D_{T'}(x) + \sum_{x\in \Qn}(\D_{T'}(x)-\D_T(x)) \le 2\size+2$, 
  we get $\sum_{x \in \Ntwo \cup \Q}\D_{T'}(x) \le 3\size+2$.
  \end{proof}

\section{Upper bound for total out-degrees of nodes w.r.t. $\Nv$}
\label{sec:upper_bound_3}

In this section, we show an upper bound for the total out-degrees of nodes corresponding to strings that are elements of $\Nv  \subseteq \M(T')$.

We first describe useful properties of
strings $x \in \Nv$.

\begin{definition}
    For any $x \in \Nv$, let $J_x$ be the string that is obtained by removing $P_{x_R}$ and $S_{x_L}$ from $x$, namely $x = P_{x_R} J_x S_{x_L}$.
  \end{definition}

Note that, by the definition of Type $\rm{(v)}$,
each $x \in \Nv$ has two or more crossing occurrences in $T'$.
Hence $J_x$ always exists (possibly the empty string).
See Figures~\ref{fig:J_x} and~\ref{fig:J_x2}.

  \begin{figure}[H]
    \centering
    \includegraphics[keepaspectratio,scale=0.35]{J_x.pdf}
    \caption{Illustration for $J_x$ in case of insertions and substitutions.
      %      In the case of deletion, the right-end of $J_x$ in $x_L$ touches the left-end of $j_x$ in $x_R$.
    }
    \label{fig:J_x}
  \end{figure}

  \begin{figure}[H]
    \centering
    \includegraphics[keepaspectratio,scale=0.35]{J_x2.pdf}
    \caption{Illustration for $J_x$ in case of deletions.}
    \label{fig:J_x2}
  \end{figure}

  \begin{lemma}\label{lem:J_xJ_y}
    For any $x, y \in \Nv$ with \sinote*{added}{$x \neq y$}, $J_x \neq J_y$.
  \end{lemma}

  \begin{proof}
    For a contrary, suppose that there exist $x, y \in \Nv$ such that $x \neq y$ and $J_x = J_y$.
%    and assume withoug loss of generality that $|S_{x_{L}}| \leq |S_{y_{L}}|$.
    Since $x$ is of Type $\rm{(v)}$, the characters immediately after $x_L$ and $x_R$ are different and let $a, b$~($a \neq b$) be these characters, respectively.
    If $|S_{x_{L}}|<|S_{y_{L}}|$, then both $S_{x_{L}}a$ and $S_{x_{L}}b$ must be prefixes of $S_{y_{L}}$ (see Figures~\ref{fig:J_xJ_y} and~\ref{fig:J_xJ_y2}), which contradicts that $a \neq b$.
    The other case where $|S_{x_{L}}| > |S_{y_{L}}|$ also leads to a contradiction.
    Hence $S_{x_{L}}=S_{y_{L}}$.
    Also, $P_{x_{R}}=P_{y_{R}}$ follows in a symmetric manner.
    These imply $x=y$, which is a contradiction.
  \end{proof}

  \begin{figure}[H]
    \centering
    \includegraphics[keepaspectratio,scale=0.35]{J_xJ_y.pdf}
    \caption{Illustration for Lemma~\ref{lem:J_xJ_y} in case of insertions and substitutions, where $J_x = J_y$.
      %      In the case of deletion, the right-end of $J_x$ in $x_L$ touches the left-end of $J_x$ in $x_R$, and the right-end of $J_y$ touches the left-end of $J_y$ in $y_R$.
    }
    \label{fig:J_xJ_y}
  \end{figure}

  \begin{figure}[H]
    \centering
    \includegraphics[keepaspectratio,scale=0.35]{J_xJ_y2.pdf}
    \caption{Illustration for Lemma~\ref{lem:J_xJ_y} in case of deletions, where $J_x = J_y$.}
    \label{fig:J_xJ_y2}
  \end{figure}

  \subsection{Correspondence between $\Nv$ and $\M(T)$}
  For any $x \in \Nv$, we associate $x$ with $J_x$.
  For any $x \in \Nv$
  we define $K(x)$ to which $x$ corresponds, by using $J_x$,
  as follows:

  \begin{definition}\label{def:K_x}
    For any $x \in \Nv$, let $K(x)=\lrep_T({\rrep_T({J_x})}) = \rrep_T({\lrep_T({J_x})})$ (see also Figure~\ref{fig:K_x} for illustration).
  \end{definition}
  We note that $K(x)$ is well defined since $J_x$ is a substring of $T$.

  \begin{figure}[H]
    \centering
    \includegraphics[keepaspectratio,scale=0.35]{K_x.pdf}
    \caption{Illustration for Definition~\ref{def:K_x}, where $a\neq b$ and $c\neq d$.}
    \label{fig:K_x}
  \end{figure}

  \begin{lemma}\label{lem:K_xK_y}
    For any $x, y \in \Nv$ \sinote*{added}{with $x \neq y$}, $K(x)\neq K(y)$.
  \end{lemma}

  \begin{proof}
    Suppose that there exist $x, y \in \Nv$ such that $x \neq y$ and $K(x)= K(y)$,
    and without loss of generality that $|J_x| \leq |J_y|$.
    Then, $J_x$ occurs at least twice in $J_y$.
    Therefore, $K(x) \neq K(y)$, however, this is a contradiction.
  \end{proof}

  \subsection{Upper bound w.r.t. $\Nv$}

  \begin{lemma} 
    \label{lem:dt3}
    $\sum_{x \in \Nv}\D_{T'}(x) \le 2\size$.
  \end{lemma}
  
  \begin{proof}
    By the definition of $\Nv$,
    there is no other right-extension of $x$ in $T'$ than the right-extension(s) of the crossing occurrence(s) of $x$.
    %    Now, since there are at most two distinct characters immediately after the crossing occurrences of $x$,
    Thus 
    there are at most two distinct characters immediately after $x$ in $T'$.
    Since $K(x)$ occurs at least twice in $T$, $\D_T(K(x)) \geq 1$.
    Hence, we get $\D_{T'}(x) \leq 2 \leq 2\D_T(K(x))$.
    Therefore, we have $\sum_{x \in \Nv}\D_{T'}(x) \le 2\size$ by Lemma~\ref{lem:K_xK_y}.
  \end{proof}


\section{Conclusion Remarks}
This work proposes a RBG graph model for disease spreading via hubs. We study the joint effect of the agent density, hub density, and connection function. The existence of a critical hub density depends only on the boundedness of the support of the connection function, which relates to curbing the traveling distance of individuals. When it comes to dispersion, both the degree distribution and the percolation threshold suggest that increasing dispersion helps spread the disease. The percolation properties of RBG graphs relate to unipartite graphs with modified connection functions. 
An interesting question in this direction is if and when the properties of the RBG graphs can be well represented by unipartite graphs with some modified connection functions. Our conjecture is that for independent connections between different pairs of agents, such representation is unlikely due to the oblivion of the local dependence (present in the RBG models). 
 Another direction is to consider hybrid models where agents may get infected either through common hubs or direct interactions between agents. The former infection mechanism is more centralized than the latter. 

\bibliographystyle{abbrv}
\bibliography{ref}

\end{document}
