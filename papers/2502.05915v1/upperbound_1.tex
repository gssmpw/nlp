\section{Upper bound for total out-degrees of nodes w.r.t. $\None \cup \Nnotv$}
\label{sec:upper_bound_1}

  In this section, we show an upper bound for the total out-degrees of the nodes corresponding to strings in $\None \cup \Nnotv  \subseteq \M(T')$.
  Recall that $x \in \None \cup \Nnotv$ implies $x \notin \LeftM(T)$.
  
We first describe useful properties of
strings $x \in \None \cup \Nnotv$.
  
  \begin{lemma} \label{lem:exist1}
    Any $x \in \None \cup \Nnotv$ occurs in $T$.
  \end{lemma}

  \begin{proof}
%  We prove Lemma~\ref{lem:exist1} by exhaustion.
    In the case $x \in \None$, since $x \in \RightM(T)$, 
    $x$ occurs in $T$. 

    Let us consider the case $x \in \Nnotv$ that is of Type $\rm{(i)}$, $\rm{(ii)}$, $\rm{(iii)}$ or $\rm{(iv)}$.
    Since $x$ is not of Type $\rm{(v)}$, 
    if all occurrences of $x$ in $T'$ are crossing occurrences of $x$ in $T'$, then $x \notin \M(T')$.
    Therefore, $x$ occurs in $T$.

    Let us consider the case $x \in \Nnotv$ that is of Type $\rm{(v)}$.
    Due to the definition of $\Nnotv$, there exists a distinct right-extension of $x$ in $T'$ other than the right-extension(s) of the crossing occurrence(s) of $x$.
    Therefore, there is a non-crossing occurrence of $x$ in $T'$
    as shown in Figure~\ref{fig:exsit1},
    implying that $x$ occurs in $T$.
  \end{proof}

  \begin{figure}[hbt]
    \centering
    \includegraphics[keepaspectratio,scale=0.33]{exist1.pdf}
    \caption{Illustration for Lemma~\ref{lem:exist1} where $i$ is the edited position and $a, b, c$ differ from each other.}
    \label{fig:exsit1}
  \end{figure}

%  \subsection{Case that $x\in \None \cup \Nnotv$ contains the edit position}

  \begin{lemma} \label{lem:sp123}
    For any $x \in \None \cup \Nnotv$ that is of Type $\rm{(i)}$, $\rm{(ii)}$ or $\rm{(iii)}$,
    there does not exist $y\in \None \cup \Nnotv$ such that
    $|y|>|x|$ and $S_{x_L}=S_{{y}_{G}}$,
    where $G \in \{L,R\}$.
  \end{lemma}
  
  \begin{proof}
    If $x_L$ is a prefix of $T'$, then clearly there is no $y$ satisfying
    $|y|>|x|$ and $S_{x_L}=S_{{y}_{G}}$.
    In what follows, we consider the case that $x_L$ is not a prefix of $T'$.
    
    For a contrary, suppose that for $x \in \None \cup \Nnotv$ that is of Type $\rm{(i)}$, $\rm{(ii)}$ or $\rm{(iii)}$, 
    there exists $y\in \None \cup \Nnotv$ such that
    $|y|>|x|$ and $S_{x_L}=S_{{y}_{G}}$, where $G \in \{L,R\}$.
    See also Figure~\ref{fig:sp123}.
    Let $a$ be the character immediately before $x_L$.
    Since $x$ is of Type $\rm{(i)}$, $\rm{(ii)}$ or $\rm{(iii)}$,
    every crossing occurrence of $x$ in $T'$ is immediately
    preceded by $a$.
    Because $x \in \M(T')$,
    it holds that $x \in \Prefix(T')$,
    or there is a distinct character $b \in \Sigma \setminus \{a\}$
    such that $bx$ occurs in $T'$.
    This implies that there is a non-crossing occurrence of $x$ in $T'$,
    which is as a prefix of $T$ or is immediately preceded by $b$ in $T$.
    By Lemma~\ref{lem:exist1}, $y$ occurs in $T$, and thus $ax$ that is a suffix of $y$ also occurs in $T$.
    Hence $x \in \LeftM(T)$, however, this contradicts that $x \notin \LeftM(T)$.
  \end{proof}

%  Lemma~\ref{lem:sp123} states that $x$ and $y$ cannot occur as in Figure~\ref{fig:sp123}

  \begin{figure}[bth]
    \centering
    \includegraphics[keepaspectratio,scale=0.33]{sp123.pdf}
    \caption{Illustration for Lemma~\ref{lem:sp123}: impossible occurrences of $x$ and $y$ with $S_{x_L} = S_{y_G}$.}
    \label{fig:sp123}
  \end{figure}

  \begin{lemma} \label{lem:sp45}
    For any $x \in \None \cup \Nnotv$ that is of Type $\rm{(iv)}$ or $\rm{(v)}$, 
    there do not exist $y,z \in \None \cup \Nnotv$ with
    $|y|>|x|$ and $|z|>|x|$
    satisfying
    $S_{x_L}=S_{{y}_{G}}$ and $S_{x_R}=S_{{z}_{F}}$ simultaneously,
    where $G,F \in \{L,R\}$.
  \end{lemma}

  
  \begin{proof}
    If $x_L$ is a prefix of $T'$, then clearly there is no $y$ satisfying
    $|y|>|x|$ and $S_{x_L}=S_{{y}_{G}}$.
    In what follows, we consider the case that $x_L$ is not a prefix of $T'$.
    
    For a contrary, 
    suppose that for $x \in \None \cup \Nnotv$ that is of Type $\rm{(iv)}$ or $\rm{(v)}$, 
    there exist $y,z \in \None \cup \Nnotv$ with $|y|>|x|$ and $|z|>|x|$ such that $S_{x_L}=S_{{y}_{G}}$ and $S_{x_R}=S_{{z}_{F}}$ at the same time, where $G,F \in \{L,R\}$.
    Let the character immediately before $x_L$ and the character immediately before $x_R$ be $a$ and $c$~($a \neq c$), respectively.
    \rhnote*{changed "b" to "c"}{%
    By Lemma~\ref{lem:exist1}, $y$ and $z$ occur in $T$, and thus $ax$ that is a suffix of $y$ and $cx$ that is a suffix of $z$ both occur in $T$.
    }%
    Therefore, $x \in \LeftM(T)$, however, this contradicts that $x \notin \LeftM(T)$.
  \end{proof}
  

  \subsection{Correspondence between $\None \cup \Nnotv$ and $\M(T)$}

  For any $x \in \None \cup \Nnotv$ that is of Type $\rm{(i)}$, $\rm{(ii)}$ or $\rm{(iii)}$, we associate $x$ with $S_{x_L}$.
  For any $x \in \None \cup \Nnotv$ that is of Type $\rm{(iv)}$ or $\rm{(v)}$, 
  if there does not exist $y\in \None \cup \Nnotv$
  such that $|y|>|x|$ and 
  $S_{x_L}=S_{{y}_{G}}$ with $G \in \{L,R\}$, we associate $x$ with $S_{x_L}$, 
  and otherwise we associate $x$ with $S_{x_R}$.

  By Lemma~\ref{lem:sp123} and Lemma~\ref{lem:sp45}, each $x \in \None \cup \Nnotv$ can be associated to a distinct string $S_{x_G}$ with $G \in \{L,R\}$.
  %
  Note however that $S_{x_G}$ may not be maximal in $T$.
  Thus we introduce a function $U$
  that bridges each $x \in \None \cup \Nnotv$ to a distinct maximal substring in $T$.
%  For any $x \in \None \cup \Nnotv$
%  we define $U(x)$ to which $x$ corresponds, by using $S_{x_G} \: (G \in \{L,R\})$ which $x$ corresponds,
%  as follows:

  \begin{definition} \label{def:U_x}
    For any $x \in \None \cup \Nnotv$,
    let $U(x)=\lrep_T({S_{x_G}})$ (see Figure~\ref{fig:U_x}).
  \end{definition}
    By Lemma~\ref{lem:exist1}, $x$ occurs in $T$ and thus
    its suffix $S_{x_G}$ also occurs in $T$.
    Hence $U(x)=\lrep_T({S_{x_G}})$ is well defined.

%  \sinote*{added}{%
%  Recall that $x$ is a maximal repeat in $T'$
%  and thus $x$ occurs at least twice in $T'$,
%  implying that $S_{x_G}$ occurs at least once in $T$.
%  Thus $U(x)=\lrep_T({S_{x_G}})$ is well defined.
%  }%
  

%  \sinote*{modified}{%
%  \begin{definition} \label{def:U_x}
%    For any $x \in \None \cup \Nnotv$,
%    let
%    \[
%    U(x) =
%    \begin{cases}
%      \lrep_T({S_{x_G}}) & \mbox{for deletion} \\
%      \lrep_T({T[i]S_{x_G}[2..|S_{x_G}|]}) & \mbox{for insertion and substitution}
%    \end{cases}
%    \]
%  \end{definition}
%  }%

%  See Figure~\ref{fig:U_x} for an illustration for $U(x)$.
  


  \begin{figure}[H]
    \centering
    \includegraphics[keepaspectratio,scale=0.33]{U_x.pdf}
    \caption{Illustration for $U(x)$ $(a\neq b)$.}
    \label{fig:U_x}
  \end{figure}

  \begin{lemma} \label{lem:U_x}
    For any $x \in \None \cup \Nnotv$, $U(x) \in \M(T)$.
  \end{lemma}

  \begin{proof}
    By Definition~\ref{def:U_x}, $U(x) = \lrep_T({S_{x_G}}) \in \LeftM(T)$.
    Therefore, it suffices for us to prove $U(x) \in \RightM(T)$.
    %
    From now on, we consider the four following cases:
%    \begin{enumerate}
%    \item[(a)] $x \in \None$. 
%    \item[(b)] $x \in \Nnotv$ and $x$ is of Type $\rm{(i)}$, $\rm{(ii)}$ or $\rm{(iv)}$.
%    \item[(c)] $x \in \Nnotv$ and $x$ is of Type $\rm{(iii)}$.
%    \item[(d)] $x \in \Nnotv$ and $x$ is of Type $\rm{(v)}$.
%    \end{enumerate}

    \noindent \textbf{Case (a) $x \in \None$:}
    In this case, $x \in \RightM(T)$, therefore $S_{x_{G}}$ that is a suffix of $x$ also satisfies $S_{x_{G}} \in \RightM(T)$.
    Hence, $U(x) = \lrep_T({S_{x_G}}) \in \RightM(T)$.

    \noindent \textbf{Case (b) $x \in \Nnotv$ and $x$ is of Type $\rm{(i)}$, $\rm{(ii)}$ or $\rm{(iv)}$:}
%    Let us consider the case that $x \in \Nnotv$ and $x$ is in $\rm{(i)}$, $\rm{(ii)}$ or $\rm{(iv)}$.
    Let the character immediately after all crossing occurrences of $x$ in $T'$ be $a$.
    There exists $xb \: (b \ne a)$ in $T'$ or $x \in \Suffix(T')$ because $x \in \M(T')$.
    Since the character immediately after all crossing occurrences of $x$ in $T'$ is $a$, 
    then there exists $xb \: (b \ne a)$ or $x \in \Suffix(T)$ in $T$.
    Hence, $S_{x_{G}} \in \RightM(T)$ since the character immediately after $S_{x_{G}}$ is $a$.
    Thus, $U(x) = \lrep_T({S_{x_G}}) \in \RightM(T)$.
  % $x \in \Suffix(T)$だと$x \in \RightM(T)$となるから考慮しなくていいけど論理的におかしなことは書いてない

   \noindent \textbf{Case (c) $x \in \Nnotv$ and $x$ is of Type $\rm{(iii)}$:}
%    Let us consider the case that $x \in \Nnotv$ and $x$ is in $\rm{(iii)}$.
    Since $x$ is of Type $\rm{(iii)}$, we associate $x$ with $S_{x_L}$.
    Because $x$ is of Type $\rm{(iii)}$ and $S_{x_L}$ is a suffix of a $S_{x_ R}$, $S_{x_{G}} \in \RightM(T)$ holds.
    Thus, $U(x) = \lrep_T({S_{x_G}}) \in \RightM(T)$.

   \noindent \textbf{Case (d) $x \in \Nnotv$ and $x$ is of Type $\rm{(v)}$:}
%    Let us consider the case that $x \in \Nnotv$ and $x$ is in $\rm{(v)}$.
    Let the character immediately after $S_{x_ G}$ be $a$.
    Since there exists a distinct right-extension of $x$ in $T'$ other than the right-extension(s) of the crossing occurrence(s) of $x$, there exists $xb$ $(b \neq a)$ in $T$.
    Therefore, $S_{x_ G} \in \RightM(T)$.
    Thus, $U(x) = \lrep_T({S_{x_G}}) \in \RightM(T)$.

    Consequently, we have $U(x) \in \M(T)$.
  \end{proof}

  The next lemma states the uniqueness of $U(x)$.
  \begin{lemma} \label{lem:U_xU_y}
    For any $x,y\in \None \cup \Nnotv$ with $x \neq y$, $U(x) \neq U(y)$.
  \end{lemma}

  \begin{proof}
    Suppose that there exist $x,y\in \None \cup \Nnotv$ such that
    $x \neq y$ and $U(x) = U(y)$.
    Let $x$ and $y$ correspond to $S_{x_{G}}$ and $S_{y_{F}}$, respectively,
    where $G,F \in \{L,R\}$.
    Let $U(x)=AS_{x_{G}},U(y)=BS_{y_{F}} \:(A,B\in \Substr(T))$,
    and assume without loss of generality that $|S_{x_{G}}|<|S_{y_{F}}|$.
    Then $|A|>|B|$ because $U(x) = U(y)$.
    \rhnote*{added "$U(y)=BS_{y_{F}} \in \M(T)$ by Lemma~\ref{lem:U_x}"}{%
    Since $U(y)=BS_{y_{F}} \in \M(T)$ by Lemma~\ref{lem:U_x}, and since $BS_{x_{G}}$ is a prefix of $BS_{y_{F}}$ (see Figure~\ref{fig:U_xU_y}), we have $BS_{x_{G}}\in \LeftM(T)$.
    }% 
    This contradicts $\lrep_T({S_{x_{G}}})=AS_{x_{G}}$.
  \end{proof}

  \begin{figure}[H]
    \centering
    \includegraphics[keepaspectratio,scale=0.33]{U_xU_y.pdf}
    \caption{Illustration for the proof of Lemma~\ref{lem:U_xU_y}, where $U(x) = U(y)$.}
    \label{fig:U_xU_y}
  \end{figure}

  \begin{comment}
  \subsection{Case that $x\in \None \cup \Nnotv$ does not contain the edit position}

  From now on, in this subsection, we consider the case that $x\in \None \cup \Nnotv$ does not contain the edit position.
  
  \begin{lemma}
    \label{lem:eps1}
    The number of $x\in \None \cup \Nnotv$ that does not contain the edit position is 1.
  \end{lemma}

  \begin{proof}
    It can be proved by making the same arguments as for Lemma~\ref{lem:sp123} and Lemma~\ref{lem:sp45}.
  \end{proof}%
  \end{comment}

  \subsection{Upper bound w.r.t. $\None \cup \Nnotv$}

  \begin{lemma} 
    \label{lem:dt1}
    $\sum_{x \in \None \cup \Nnotv}\D_{T'}(x) \le 3\size+2$.
  \end{lemma}

  \begin{proof}\rhnote*{changed}{%
    Let $U(x)=\lrep_T({S_{x_G}})$, where $G \in \{L,R\}$.
    Since $S_{x_G}$ is a suffix of $x$, $\D_{T}(x) \leq \D_{T}(U(x))$.
    Since there are at most two distinct characters immediately after the crossing occurrences of $x$, 
    $\D_{T'}(x) \leq \D_{T}(U(x))+2$.
    For $U(x) \neq T$, we have $\D_T(U(x)) \ge 1$. Thus $\D_{T'}(x) \le \D_T(U(x))+2 \le 3\D_T(U(x))$.
    For $U(x) = T$, we have $\D_T(U(x))=0$. Thus $\D_{T'}(x) \le 2$.
    %
    By using Lemma~\ref{lem:U_xU_y} and summing up these, we get 
    $\sum_{x \in \None \cup \Nnotv}\D_{T'}(x) \le {\sum_{x \in \None \cup \Nnotv} 3\D_T(U(x))}+2 \le 3\size+2$.
  \end{proof}
  }%

