\section{Upper bound for total out-degrees of nodes w.r.t. $\Nv$}
\label{sec:upper_bound_3}

In this section, we show an upper bound for the total out-degrees of nodes corresponding to strings that are elements of $\Nv  \subseteq \M(T')$.

We first describe useful properties of
strings $x \in \Nv$.

\begin{definition}
    For any $x \in \Nv$, let $J_x$ be the string that is obtained by removing $P_{x_R}$ and $S_{x_L}$ from $x$, namely $x = P_{x_R} J_x S_{x_L}$.
  \end{definition}

Note that, by the definition of Type $\rm{(v)}$,
each $x \in \Nv$ has two or more crossing occurrences in $T'$.
Hence $J_x$ always exists (possibly the empty string).
See Figures~\ref{fig:J_x} and~\ref{fig:J_x2}.

  \begin{figure}[H]
    \centering
    \includegraphics[keepaspectratio,scale=0.35]{J_x.pdf}
    \caption{Illustration for $J_x$ in case of insertions and substitutions.
      %      In the case of deletion, the right-end of $J_x$ in $x_L$ touches the left-end of $j_x$ in $x_R$.
    }
    \label{fig:J_x}
  \end{figure}

  \begin{figure}[H]
    \centering
    \includegraphics[keepaspectratio,scale=0.35]{J_x2.pdf}
    \caption{Illustration for $J_x$ in case of deletions.}
    \label{fig:J_x2}
  \end{figure}

  \begin{lemma}\label{lem:J_xJ_y}
    For any $x, y \in \Nv$ with \sinote*{added}{$x \neq y$}, $J_x \neq J_y$.
  \end{lemma}

  \begin{proof}
    For a contrary, suppose that there exist $x, y \in \Nv$ such that $x \neq y$ and $J_x = J_y$.
%    and assume withoug loss of generality that $|S_{x_{L}}| \leq |S_{y_{L}}|$.
    Since $x$ is of Type $\rm{(v)}$, the characters immediately after $x_L$ and $x_R$ are different and let $a, b$~($a \neq b$) be these characters, respectively.
    If $|S_{x_{L}}|<|S_{y_{L}}|$, then both $S_{x_{L}}a$ and $S_{x_{L}}b$ must be prefixes of $S_{y_{L}}$ (see Figures~\ref{fig:J_xJ_y} and~\ref{fig:J_xJ_y2}), which contradicts that $a \neq b$.
    The other case where $|S_{x_{L}}| > |S_{y_{L}}|$ also leads to a contradiction.
    Hence $S_{x_{L}}=S_{y_{L}}$.
    Also, $P_{x_{R}}=P_{y_{R}}$ follows in a symmetric manner.
    These imply $x=y$, which is a contradiction.
  \end{proof}

  \begin{figure}[H]
    \centering
    \includegraphics[keepaspectratio,scale=0.35]{J_xJ_y.pdf}
    \caption{Illustration for Lemma~\ref{lem:J_xJ_y} in case of insertions and substitutions, where $J_x = J_y$.
      %      In the case of deletion, the right-end of $J_x$ in $x_L$ touches the left-end of $J_x$ in $x_R$, and the right-end of $J_y$ touches the left-end of $J_y$ in $y_R$.
    }
    \label{fig:J_xJ_y}
  \end{figure}

  \begin{figure}[H]
    \centering
    \includegraphics[keepaspectratio,scale=0.35]{J_xJ_y2.pdf}
    \caption{Illustration for Lemma~\ref{lem:J_xJ_y} in case of deletions, where $J_x = J_y$.}
    \label{fig:J_xJ_y2}
  \end{figure}

  \subsection{Correspondence between $\Nv$ and $\M(T)$}
  For any $x \in \Nv$, we associate $x$ with $J_x$.
  For any $x \in \Nv$
  we define $K(x)$ to which $x$ corresponds, by using $J_x$,
  as follows:

  \begin{definition}\label{def:K_x}
    For any $x \in \Nv$, let $K(x)=\lrep_T({\rrep_T({J_x})}) = \rrep_T({\lrep_T({J_x})})$ (see also Figure~\ref{fig:K_x} for illustration).
  \end{definition}
  We note that $K(x)$ is well defined since $J_x$ is a substring of $T$.

  \begin{figure}[H]
    \centering
    \includegraphics[keepaspectratio,scale=0.35]{K_x.pdf}
    \caption{Illustration for Definition~\ref{def:K_x}, where $a\neq b$ and $c\neq d$.}
    \label{fig:K_x}
  \end{figure}

  \begin{lemma}\label{lem:K_xK_y}
    For any $x, y \in \Nv$ \sinote*{added}{with $x \neq y$}, $K(x)\neq K(y)$.
  \end{lemma}

  \begin{proof}
    Suppose that there exist $x, y \in \Nv$ such that $x \neq y$ and $K(x)= K(y)$,
    and without loss of generality that $|J_x| \leq |J_y|$.
    Then, $J_x$ occurs at least twice in $J_y$.
    Therefore, $K(x) \neq K(y)$, however, this is a contradiction.
  \end{proof}

  \subsection{Upper bound w.r.t. $\Nv$}

  \begin{lemma} 
    \label{lem:dt3}
    $\sum_{x \in \Nv}\D_{T'}(x) \le 2\size$.
  \end{lemma}
  
  \begin{proof}
    By the definition of $\Nv$,
    there is no other right-extension of $x$ in $T'$ than the right-extension(s) of the crossing occurrence(s) of $x$.
    %    Now, since there are at most two distinct characters immediately after the crossing occurrences of $x$,
    Thus 
    there are at most two distinct characters immediately after $x$ in $T'$.
    Since $K(x)$ occurs at least twice in $T$, $\D_T(K(x)) \geq 1$.
    Hence, we get $\D_{T'}(x) \leq 2 \leq 2\D_T(K(x))$.
    Therefore, we have $\sum_{x \in \Nv}\D_{T'}(x) \le 2\size$ by Lemma~\ref{lem:K_xK_y}.
  \end{proof}

