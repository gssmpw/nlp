\section{Occurrences of maximal repeats crossing the edited position}

% In what follows, we concentrate on the case of substitutions
% and show that $\size(T') \leq 7\size+3$
% for any strings $T$ and $T'$, where $T'$ is obtained by substituting
% a character in $T$ with a different character.

To present an upper bound for the sensitivity of CDAWG size,
it is essential to consider new occurrences of maximal repeats
that contain or touch the edited position $i$.
%
In this section, we introduce several new definitions regarding
those new occurrences of maximal repeats.
%and categorize them
%in a way that allows for charging new maximal repeats in $T'$ to existing maximal repeats in $T$ and allows for bounding the number of edges in $\CDAWG(T')$
%just a constant factor away from the number of edges in $\CDAWG(T)$.

For ease of discussions, we sometimes identify an interval where
a substring $w$ occurs in $T'$ with the substring $w$ itself,
when there is no risk of confusion.

% %\sinote*{changed}{%
% \begin{definition}
% Let $x = T'[j..k]$ be a non-empty substring of $T'$ that \emph{touches or contains}
% the edited position $i$, namely
% (1) $k = i-1$ (touching $i$ from left),
% (2) $j \leq i \leq k$ (containing $i$), or
% (3) $j = i+1$ (touching $i$ from right).
% These occurrence of a substring $x$ in $T'$
% are said to be \emph{crossing occurrences} for
% the edited position $i$.
% We will call these occurrences simply as crossing occurrences of $x$.
% See also Figure~\ref{fig:sub} for illustration.
% \end{definition}

% Let $T[i] = \alpha \in \Sigma$ and $T'[i] = \beta \in \Sigma$.
% The leftmost crossing occurrence $x_L$ of $x$ in $T'$ is denoted as
% \[
% x_L =
%     \begin{cases}
%      P_{x_L} & \mbox{if $x_L$ touches $i$ from left}  \\
%      P_{x_L} \beta S_{x_L} & \mbox{if $x_L$ contains $i$} \\
%      S_{x_L} & \mbox{if $x_L$ touches $i$ from right}
%     \end{cases}
% \]
% where $P_{x_L}, S_{x_L} \in \Sigma^*$.
% We denote the leftmost crossing occurrence $x_R$ of $x$ in $T'$ analogously.

\subsection{Crossing occurrences}

\begin{definition}\rhnote*{changed}{%
  Let $x = T'[j..k]$ be a non-empty substring of $T'$ that \emph{touches or contains}
  the edited position $i$.
  That is, if the edit operation is insertion and substitution,
  (1) $k = i-1$ (touching $i$ from left),
  (2) $j \leq i \leq k$ (containing $i$), or
  (3) $j = i+1$ (touching $i$ from right).
  If the edit operation is deletion, 
  (1) $k = i-1$ (touching $i$ from left),
  (2) $j \leq i-1 \land i \leq k$ (containing $i$), or
  (3) $j = i$ (touching $i$ from right).
  }%
  These occurrences of a substring $x$ in $T'$
  are said to be \emph{crossing occurrences} for
  the edited position $i$.
  We will call these occurrences simply as crossing occurrences of $x$.
%  See also Figure~\ref{fig:crossing} for illustration.
  \end{definition}

\rhnote*{added "of x" and changed "the original string T" to "T'"}{%
We denote the left most crossing occurrence $T'[j'..k']$ of $x$ as $x_L$.
For $x_L$, we consider the following substrings $P_{x_L}$ and $S_{x_L}$
of $T'$ (see Figure~\ref{fig:x_Lp_L} for illustration):
}%

In the case that the edit operation is insertion or substitution, let
\begin{eqnarray*}
P_{x_L} & = & 
    \begin{cases}
     T'[j'..i] & \mbox{if $x_L$ touches $i$ from left or contains $i$,}\\
     \varepsilon & \mbox{if $x_L$ touches $i$ from right,}
    \end{cases}\\
S_{x_L} & = &
    \begin{cases}
     \varepsilon & \mbox{if $x_L$ touches $i$ from left,}  \\
     T'[i..k'] & \mbox{if $x_L$ contains $i$ or touches $i$ from right.}
    \end{cases}
\end{eqnarray*}
In the case that the edit operation is deletion, let
\begin{eqnarray*}
P_{x_L} & = &
    \begin{cases}
     T'[j'..i-1] & \mbox{if $x_L$ touches $i$ from left or contains $i$,}\\
     \varepsilon & \mbox{if $x_L$ touches $i$ from right,}
    \end{cases}\\
S_{x_L} & = &
    \begin{cases}
     \varepsilon & \hspace*{6mm} \mbox{if $x_L$ touches $i$ from left,}  \\
     T'[i..k'] & \hspace*{6mm} \mbox{if $x_L$ contains $i$ or touches $i$ from right.}
    \end{cases}
\end{eqnarray*}
We define the rightmost crossing occurrence $x_R$, together with $P_{x_R}$ and $S_{x_R}$, analogously.

%}%
%\begin{definition}
%  Let $T'$ be the string obtained from $T$ by substituting the $i$th character $T[i] = \alpha$ with $\beta~(\beta \neq \alpha)$.
%  Assume that $x=P \beta S \: (P,S \in {\Sigma_x}^*, \beta \in \Sigma_x)$.
%  Then we define crossing occurrences of string as follows.
%  The occurrence of $x$ in $T'$ beginning at position $i-|P|$ is called a \emph{crossing occurrence} of $x$ in $T'$
%  (See Figure~\ref{fig:sub} for an illustration).
%  The leftmost crossing occurrence of $x$ in $T'$ is denoted by
%  $x_L = P_{x_L} \beta S_{x_L}$.
%  Similarly, the rightmost crossing occurrence of $x$ in $T'$ is denoted by $x_R = P_{x_R} \beta S_{x_R}$.
%\end{definition}

%\begin{figure}[H]
%  \centering
%  \includegraphics[keepaspectratio,scale=0.35]{crossing.pdf}
%  \caption{Illustration for crossing occurrences of $x$ in $T'$.}
%  \label{fig:crossing}
%\end{figure}

\begin{figure}[tbh]
  \centering
  \includegraphics[keepaspectratio,scale=0.35]{x_Lp_L.pdf}
  \caption{Illustration of $x_L$ in $T'$ for the case where $x_L$ contains $i$, with insertion and substitution.}
  \label{fig:x_Lp_L}
\end{figure}

\begin{definition}
  We categorize strings $x$ that have crossing occurrence(s) in the edited string in $T'$ into the five following types, depending on the properties of $x$:
  \begin{description}
    \item {Type (i)}: $x$ has only one crossing occurrence of $x$ in $T'$.
    \item {Type (ii)}: 
      \begin{enumerate}
        \item $x$ has two or more crossing occurrences of $x$ in $T'$.
        \item If all occurrences of $x$ in $T'$ are crossing occurrences of $x$ in $T'$, then $x \notin \LeftM(T')$ and $x \notin \RightM(T')$.
      \end{enumerate}
    \item {Type (iii)}:
      \begin{enumerate}
        \item $x$ has two or more crossing occurrences of $x$ in $T'$.
        \item If all occurrences of $x$ in $T'$ are crossing occurrences of $x$ in $T'$, then $x \notin \LeftM(T')$ and $x \in \RightM(T')$.
      \end{enumerate}
    \item {Type (iv)}:
      \begin{enumerate}
        \item $x$ has two or more crossing occurrences of $x$ in $T'$.
        \item If all occurrences of $x$ in $T'$ are crossing occurrences of $x$ in $T'$, then $x \in \LeftM(T')$ and $x \notin \RightM(T')$.
      \end{enumerate}
    \item {Type (v)}
      \begin{enumerate}
        \item $x$ has two or more crossing occurrences of $x$ in $T'$.
        \item If all occurrences of $x$ in $T'$ are crossing occurrences of $x$ in $T'$, then $x \in \M(T')$.
      \end{enumerate}
  \end{description}  
\end{definition}

Figure~\ref{fig:case} illustrates the aforementioned five types of a string $x$ when $x_L \notin \Prefix(T')$ and $x_R \notin \Suffix(T')$.

\begin{figure}[H]
  \centering
  \includegraphics[keepaspectratio,scale=0.35]{cases_of_string.pdf}
  \caption{Illustration for the five cases of a string $x$ when $x_L \notin \Prefix(T')$ and $x_R \notin \Suffix(T')$, where $a \ne c$ and $b \ne d$ for characters $a,b,c,d \in \Sigma$.}
  \label{fig:case}
\end{figure}

The reason why we only consider $x_L$ and $x_R$ is due to the periodicity for $x$.
We remark that when $x$ have multiple crossing occurrences which contain or touch the edited position $i$ in $T'$,
then the characters immediately before all crossing occurrences of $x$ in $T'$ except for $x_L$ are the same and the characters immediately after all crossing occurrences of $x$ in $T'$ except for $x_R$ are the same.
Therefore, we do not need to consider all crossing occurrences of $x$ in $T'$ except for $x_L$ and $x_R$ to examine whether $x$ is maximal in $\M(T')$ or not.

\subsection{New/Existing maximal repeats}

In Definitions~\ref{def:new_maximal_repeats} and~\ref{def:existing_maximal_repeats} below, we introduce $\N$ and $\Q$ which are respectively
the sets of new maximal repeats and existing maximal repeats in $T'$,
such that $\N \cup \Q = \M(T') \setminus \{T'\}$.
We then partition each of them into smaller subsets that are suitable for our needs.

\begin{definition} \label{def:new_maximal_repeats}
  Let
  \sinote*{changed}{%
    $\N= (\M(T')\setminus \M(T)) \setminus \{T'\}$
  }%
%  $\N=\M(T')\setminus \M(T)$
  denote the set of new maximal repeats in $T'$.
  We divide $\N$ into the three following disjoint subsets:
%  $\None = \N\cap \RightM(T)$,
%  $\Ntwo = \N\cap \LeftM(T)$, and
%  $\Nthree = \N \setminus (\None \cup \Ntwo)$.
%  Let the subsets $\None$, $\Ntwo$ and $\Nthree$ of $\N$ be defined as
    \begin{eqnarray*}
    \None & = & \N\cap \RightM(T), \\
    \Ntwo & = & \N\cap \LeftM(T), \\
    \Nthree & = & \N \setminus (\None \cup \Ntwo).
    \end{eqnarray*}
  Further, we divide $\Nthree$ into two disjoint subsets $\Nv$ and $\Nnotv$ as follows:
  $\Nv$ is the set of strings $x \in \Nthree$ such that
  (1) $x$ is of Type $\rm{(v)}$, and
  (2) there is no other right-extension of $x$ in $T'$ than the right-extension(s) of the crossing occurrence(s) of $x$,
  and $\Nnotv = \Nthree \setminus \Nv$.
%    let $\Nv$ denote a set of $x \in \Nthree$ satisfying the two following conditions:
%    (1) $x$ is of Type $\rm{(v)}$; 
%    (2) There is no distinct right-extension of $x$ in $T'$ other than the right-extension(s) of the crossing occurrence(s) of $x$.
%    Let $\Nnotv = \Nthree \setminus \Nv$.


%  Further, let $\Nv$ denote the subset of $\Nthree$
%  of which strings is not substring of $T$.
%  Let $\Nnotv = \Nthree \setminus \Nv$, which is the subset of strings contained within $T$.
%
  %  In addition, define a set of strings that are element of $\Nthree$ and case (v) as $\Nv$, 
%  also, define a set of strings that are element of $\Nthree$ and not case (v) as $\Nnotv$.
  %
\end{definition}

We note that the following properties hold for $\None,\Ntwo$ and $\Nthree$ by definition:
\begin{itemize}
\item $x\in \None\Rightarrow x\notin \LeftM(T)$ because $x \notin \M(T)$.
\item $x\in \Ntwo\Rightarrow x\notin \RightM(T)$ because $x \notin \M(T)$.
\item $x\in \Nthree\Rightarrow x\notin \RightM(T) \land x\notin \LeftM(T)$.
\end{itemize}

\begin{definition} \label{def:existing_maximal_repeats}
  Let
  \sinote*{changed}{%
    $\Q= (\M(T') \cap \M(T)) \setminus \{T'\}$
  }%
  %  $\Q=\M(T')\setminus \N = \M(T') \cap \M(T)$
  denote
  the set of existing maximal repeats in $T'$.
  We divide $\Q$ into the two following subsets:
  \begin{eqnarray*}
    \Qn & = & \{x \in \Q \mid \D_{T'}(x) > \D_T(x)\},\\
    \Qnotn & = & \{x \in \Q \mid \D_{T'}(x) \leq \D_T(x)\}.
  \end{eqnarray*}
%  Let $\Qn$ be the subset of $Q$ whose string's right-extensions increase.
%  Let $\Qnotn$ be the subset of $Q$ whose string's right-extensions do not increase.
\end{definition}
Namely, $\Qn$ (resp. $\Qnotn$) is the set of existing nodes of the CDAWG
for which the number of out-edges increase (resp. do not increase). 

\begin{example}
  \label{ex:mr}
  Consider the string $T = \mathrm{cabcabcdabca}\mathbf{d}\mathrm{bcabcdabcabdcabcabcabdabcab}$
  and \\
  the string $T' = \mathrm{cabcabcdabca|bcabcdabcabdcabcabcabdabcab}$ obtained by deleting the character $T[13] = \mathbf{d}$ from $T$ highlighted in bold.
  The edited position in $T'$ is designated by a~$|$.
  For instance, 
  $\mathrm{dabcab}\in\Qn, \mathrm{bcabc}\in\Qnotn, \mathrm{abcabc}\in\None, \mathrm{abcabcab}\in\Nnotv, \mathrm{cabcabcdabcab}\in\Nv$
  (also see Figure~\ref{fig:ex_mr}).
\end{example}

\begin{figure}[H]
  \centering
  \includegraphics[keepaspectratio,scale=0.3]{ex_mr.pdf}
  \caption{Illustration for $T'= \mathrm{cabcabcdabca|bcabcdabcabdcabcabcabdabcab}$ in Example~\ref{ex:mr} and the occurrences of $\mathrm{abcabc}\in\None$ and $\mathrm{cabcabcdabcab}\in\Nv$ in $T'$.
  The~$|$ symbol in $T'$ exhibits the edit position. The solid line boxes exhibit the crossing occurrences of $\mathrm{abcabc}$ and $\mathrm{cabcabcdabcab}$ in $T'$,
  and the dashed line boxes exhibit the non-crossing occurrences of them in $T'$.}
  \label{fig:ex_mr}
\end{figure}

Recall that $\size = \sum_{x \in \M(T)}\D_T(x)$ denotes the number of edges
in $\CDAWG(T)$ before the edit.
In the subsequent sections,
we work on the three disjoint subsets
$\None \cup \Nnotv$, 
$\Ntwo \cup \Q$, and
$\Nv$ of $\M(T')$,
and show that $\sum_{x \in \None \cup \Nnotv}\D_{T'}(x) \le 3\size+2$ (Section~\ref{sec:upper_bound_1}),
$\sum_{x \in \Ntwo \cup \Q}\D_{T'}(x) \le 3\size+2$ (Section~\ref{sec:upper_bound_2}),
and $\sum_{x \in \Nv}\D_{T'}(x) \le 2\size$ (Section~\ref{sec:upper_bound_3}).
All these immediately lead to Theorem~\ref{theo:CDAWG_sensitivity}
that upper bounds the number of edges in $\CDAWG(T')$ after the edit to $8\size + 4$.


%  By Lemma~\ref{lem:dt1}, Lemma~\ref{lem:dt2} and Lemma~\ref{lem:dt3},
%  we obtain Theorem~\ref{theo:CDAWG_sensitivity}.
