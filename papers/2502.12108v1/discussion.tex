\section{Discussion} \label{sec:discussion}

In this paper, we identified key limitations of path-based attribution methods such as Integrated Gradients (IG), particularly the artefacts that arise from ignoring the model’s curvature. To address these issues, we introduced a novel path-based method, Geodesic IG, that integrates gradients along geodesic paths on a manifold defined by the model, rather than straight lines.

By avoiding regions of high gradient in the input space, Geodesic IG effectively mitigates these artefacts while preserving all the axioms established by \citet{sundararajan2017axiomatic}. Additionally, we introduced a new axiom, Strong Completeness, which, when satisfied, prevents such misattributions. We proved that Geodesic IG is the only path-based method that satisfies this axiom. Through both theoretical analysis and empirical evaluation—using metrics such as Comprehensiveness and Log-Odds—we demonstrated the advantages of our approach.

To approximate geodesic paths, we proposed two methods: one based on $k$-Nearest Neighbour and another leveraging Stochastic Variational Inference. While these methods outperform existing alternatives, they also present challenges. One such challenge is computational cost, as discussed in Section \ref{sec:experiments}. Another is the inherent noise in sampling-based geodesic approximations. Even though in our experiments we demonstrated noise reduction relative to the original IG, we believe further improvements can be achieved. A promising future direction is to solve the geodesic equation directly, which could reduce noise and improve accuracy. Additionally, depending on the chosen solution method, this approach may offer greater computational efficiency compared to the current reliance on SVI.