\section{Related Work}
\label{sec:related_work}

Approximating geodesic paths is a widely studied area of research, and many methods to do so have been developed. For a comprehensive survey on this subject, please refer to \citet{crane2020survey}.

The idea of using a $k$NN algorithm to avoid computing gradients on out of distribution data points has also been used in Enhanced Integrated Gradients \citet{jha2020enhanced}. However, this method creates a path which is model agnostic, as it does not necessarily avoid high gradients regions. As a result, it can lead to significant artefacts which do not reflect the model's behaviour. To support this argument, we provide an example where this method fails on the half-moons datasets. In Fig. \ref{fig:enhanced_ig}, similar to the example in section \ref{sec:introduction}, we see the Enhanced IG attributes different importance to the horizontal and vertical features of the half-moon data points, in a model that is flat everywhere, other than the decision boundary. Furthermore, in Fig. \ref{fig:enhanced_ig}, we observer that the method violates Strong Completeness axiom, Axiom \ref{ax:strong}. This is expected, given Theorem \ref{theo:strong}.

\begin{figure}[t]
\vskip -0.2in
\begin{center}
\centerline{\includegraphics[width=0.6\columnwidth]{figures/enhanced_ig.pdf}}
\caption{\textbf{Enhanced IG attributions.} Enhanced IG applies a $k$-NN algorithm, then uses Dijkstra’s algorithm to find the shortest path between an input and a reference baseline, and computes gradients along this path. However, since this method is model-agnostic, it does not account for high-gradient regions. As a result, the attributions exhibit an unjustified shading that does not accurately reflect the model's true behaviour. Comparing the horizontal and vertical attributions (left and right plots, respectively), we confirm that this method does not satisfy Axiom \ref{ax:strong}, as expected. In this example, similar to those in Fig. \ref{fig:ig}, we sample 10,000 points from the half-moons dataset with noise drawn from  $\mathcal{N}(0, 0.15)$.}
\label{fig:enhanced_ig}
\end{center}
\vskip -0.2in
\end{figure}

The idea of adapting the path to avoid high gradient regions has be proposed by \citet{kapishnikov2021guided}, calling their method Guided Integrated Gradients. This method has a heuristic approach to finding such paths. As a result it does not guarantee to find paths of minimal accumulated gradients. In contrast, Geodesic Integrated Gradients offers a more principled approach by directly approximating the path of least resistance using a Riemannian manifold framework. As a result, as we see in Section \ref{sec:experiments}, our method significantly outperforms Guided IG in terms of attribution accuracy.