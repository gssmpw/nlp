\section{Related Work}
\label{sec:related-work}

\subsection{Research Studies}

Urban Search And Rescue (USAR) systems employ semi-autonomous control schemes, which incorporate hierarchical reinforcement learning and utilized sensors comparable to those in the proposed system. These systems also include manual and tele-operated fall-back mechanisms \cite{doroodgarLearningBasedSemiAutonomousController2014}.

The literature on the use of aerial AMRs focuses on an extensive examination of several topics, including platforms, sensors on board, Simultaneous Localization And Mapping (SLAM) methodologies, terrain coverage techniques, autonomous navigation strategies, and human-swarm interfaces \cite{recchiutoPostdisasterAssessmentUnmanned2018}. The detection of damaged infrastructure is also a subject of research, with studies investigating the use of UGV platforms to detect damage in simulated earthquake scenarios \cite{chenDetectionDamagedInfrastructure2019}.

In parallel, researchers investigate Multi-Robot Task Allocation (MRTA) problems in the context of flood response scenarios, assessing the performance of multiple UAV robots subject to range and payload constraints \cite{ghassemiMultirobotTaskAllocation2022}. The development of custom-made robots for earthquake scenarios is also an active area of research, with proposals for modular and snake-like robots that are lightweight and compact in size \cite{narayanSearchReconnaissanceRobot2022, jadejaSurvivorDetectionApproach2024}. Moreover, researchers have evaluated the deployment of Unmanned Surface Vehicles (USVs) alongside UAVs and UGVs, which require sophisticated frameworks to manage disaster scenarios \cite{pillaiHeterogeneousRobotsCollaboration2024}.

Indoor post-disaster human detection is crucial for FRs, and Micro-Aerial Vehicles (MAVs) possess the advantage of maneuvering in confined spaces. MAVs equipped with thermal cameras and real-time algorithms can precisely detect survivors, as well as map the scene for additional analysis \cite{tavasoliAutonomousPostdisasterIndoor2025}.

\subsection{International Projects}

\begin{table}
  \centering
  \caption{Key research and development efforts for deploying robotic solutions in post-disaster scenarios.}
  \begin{tblr}{colspec = {X[c, 1.5cm]|X[c]|X[c]|X[c]|X[c]|X[c]},
    row{1, 2} = {font=\bfseries},
        cell{1}{2} = {c=2}{c},
        cell{1}{4} = {c=3}{c},
        cell{1}{1} = {r=2}{c},
      }
    \toprule
    Work                                                      & Robots     &            & Setting    &            &              \\ \midrule
                                                              & UAV        & UGV        & Wildfire   & Flood      & Earthquake   \\ \midrule
    \cite{tavasoliAutonomousPostdisasterIndoor2025}           & \checkmark &            &            &            & \checkmark   \\
    \cite{pillaiHeterogeneousRobotsCollaboration2024}         & \checkmark & \checkmark &            & \checkmark &              \\
    \cite{jadejaSurvivorDetectionApproach2024}                &            & \checkmark &            &            & \checkmark   \\
    \cite{narayanSearchReconnaissanceRobot2022}               &            & \checkmark &            &            & \checkmark   \\
    \cite{chenDetectionDamagedInfrastructure2019}             &            & \checkmark &            &            & \checkmark   \\
    \cite{recchiutoPostdisasterAssessmentUnmanned2018}        & \checkmark &            &            &            & \checkmark   \\
    \cite{doroodgarLearningBasedSemiAutonomousController2014} &            & \checkmark &            &            & \checkmark   \\ \midrule
    CARMA \cite{CARMAProject2024}                             &            & \checkmark & \checkmark &            &              \\
    SILVANUS \cite{SILVANUSIntegratedTechnological2021}       & \checkmark & \checkmark & \checkmark &            &              \\
    CURSOR \cite{ristmaeCURSORSearchRescue2021}               & \checkmark & \checkmark &            &            & \checkmark   \\
    \midrule TRIFFID                                          & \checkmark & \checkmark & \checkmark & \checkmark & \checkmark & \\
    \bottomrule
  \end{tblr}
  \label{tab:system_comparison}
\end{table}

\begin{figure*}
  \centering
  \includegraphics[width=0.65\linewidth]{TRIFFID.png}
  \caption{TRIFFID system functional architecture.}
  \label{fig:triffid-functional-arch}
\end{figure*}

Apart from dedicated and task-oriented research works, more large-scale and integrated robotic systems for disaster management have recently been launched.

SILVANUS is an ongoing project focused on developing an integrated wildfire management platform, using UAV and UGV robots, public Internet data, and IoT devices \cite{SILVANUSIntegratedTechnological2021}.

CARMA is an active project that uses off-road tracked and legged UGVs to assist FRs in their operations during disaster scenarios. These scenarios include urban zones affected by earthquakes, underground parking garage fires, cargo incidents, and Chemical, Biological, Radiological, Nuclear, and Explosive (CBRN-E) incidents \cite{CARMAProject2024}.

The CURSOR project develops a search and rescue kit that includes drones, minified robotic equipment, and advanced sensors. This kit is designed to reduce the time required for detection and rescue of victims trapped under debris, while enhancing the personal safety of the search and rescue team \cite{ristmaeCURSORSearchRescue2021}.

While the above projects have substantially contributed to the successful integration of robotics to FR procedures, the TRIFFID system proposes a harmonized solution of both a UAV and UGV for autonomous reconnaissance in three demanding real-world scenarios, namely wildfire, urban flood, and earthquake. Table \ref{tab:system_comparison} summarizes key research and development efforts for deploying robotic solutions in post-disaster scenarios.