The rapid advancement of semiconductor technology has enabled the integration of multiple functionalities and features into a single \ac{SoC}. However, design verification remains a significant bottleneck in \ac{SoC} development, consuming a substantial portion of the overall project time and resources. According to a recent study by the Wilson Research Group, verification accounts for around \SI{60}{\percent} of the overall project time, highlighting the need for more efficient verification methodologies \cite{VerStudy_22}.

The simulation-based design verification is a well-established and powerful technique that leverages \ac{CRV} and coverage-driven methodologies, which typically use simulation regressions in the end. The goal is to improve verification objectives, including coverage, by repeating test simulations with random seeds. However, these classical simulation regressions often involve redundant test simulations, which does not guarantee an improvement in the total coverage. This process utilizes significant simulation resources and time to manually adjust constraints, leading to an increase in verification costs and time-to-market for \ac{SoC} products. Additionally, the generation of huge amounts of data in simulation regression makes design verification an ideal field for \ac{ML} application.

Numerous initiatives have been made to optimize functional verification through the use of \ac{ML} techniques, as reviewed in the surveys \cite{article} and \cite{siemens_whitepaper}. Most of these studies focused on functional coverage improvement, simulation speedup, and reducing the test simulations in the simulation regressions. This paper presents a novel approach to addressing the high number of test simulations and longer run-time in classical simulation regression, while meeting the target coverage goals. This study aims to promote the adoption of automation methodologies that utilize data science and ML, showcasing the performance and comparing various supervised models.

The contributions of this work are as follows:
\begin{itemize}
	\item Novel simulation regression optimization using supervised learning (Section \ref{sec:implementation})
	\item \ac{PyUVM}-based testbench creation for \ac{DUVs} (Section \ref{tbcreation})
	\item \ac{ML} environment integration into simulation environment (Section \ref{datacnp}, \ref{mlp})
	\item Automatic testbench updates with ML-predicted constraints, sequences, and tests (Section \ref{automatictb})
\end{itemize}

