\documentclass[10pt]{article} % For LaTeX2e
% \usepackage{tmlr}
% If accepted, instead use the following line for the camera-ready submission:
% \usepackage[accepted]{tmlr}
% To de-anonymize and remove mentions to TMLR (for example for posting to preprint servers), instead use the following:
\usepackage[preprint]{tmlr}

% Optional math commands from https://github.com/goodfeli/dlbook_notation.
%%%%% NEW MATH DEFINITIONS %%%%%

% \usepackage{amsmath,amsfonts,bm}
\usepackage{amsmath,amsfonts}

\usepackage{pifont}


\newcommand{\R}{\mathbb{R}}


\def\va{{\mathbf{a}}}
\def\vg{{\mathbf{g}}}

% Sets
\def\sR{\mathbb{R}}
\def\sC{\mathbb{C}}
\def\sZ{\mathbb{Z}}
\def\sN{\mathbb{N}}
\def\sQ{\mathbb{Q}}

\def\sS{\mathcal{S}}



% Vectors
\def\vzero{{\mathbf{0}}}
\def\vone{{\mathbf{1}}}
\def\vmu{{\mathbf{\mu}}}
\def\vtheta{{\mathbf{\theta}}}
\def\va{{\mathbf{a}}}
\def\vb{{\mathbf{b}}}
\def\vc{{\mathbf{c}}}
\def\vd{{\mathbf{d}}}
\def\ve{{\mathbf{e}}}
\def\vf{{\mathbf{f}}}
\def\vg{{\mathbf{g}}}
\def\vh{{\mathbf{h}}}
\def\vi{{\mathbf{i}}}
\def\vj{{\mathbf{j}}}
\def\vk{{\mathbf{k}}}
\def\vl{{\mathbf{l}}}
\def\vm{{\mathbf{m}}}
\def\vn{{\mathbf{n}}}
\def\vo{{\mathbf{o}}}
\def\vp{{\mathbf{p}}}
\def\vq{{\mathbf{q}}}
\def\vr{{\mathbf{r}}}
\def\vs{{\mathbf{s}}}
\def\vt{{\mathbf{t}}}
\def\vu{{\mathbf{u}}}
\def\vv{{\mathbf{v}}}
\def\vw{{\mathbf{w}}}
\def\vx{{\mathbf{x}}}
\def\vy{{\mathbf{y}}}
\def\vz{{\mathbf{z}}}
\def\vzeta{{\mathbf{\zeta}}}

% Matrix
\def\mA{{\mathbf{A}}}
\def\mB{{\mathbf{B}}}
\def\mC{{\mathbf{C}}}
\def\mD{{\mathbf{D}}}
\def\mE{{\mathbf{E}}}
\def\mF{{\mathbf{F}}}
\def\mG{{\mathbf{G}}}
\def\mH{{\mathbf{H}}}
\def\mI{{\mathbf{I}}}
\def\mJ{{\mathbf{J}}}
\def\mK{{\mathbf{K}}}
\def\mL{{\mathbf{L}}}
\def\mM{{\mathbf{M}}}
\def\mN{{\mathbf{N}}}
\def\mO{{\mathbf{O}}}
\def\mP{{\mathbf{P}}}
\def\mQ{{\mathbf{Q}}}
\def\mR{{\mathbf{R}}}
\def\mS{{\mathbf{S}}}
\def\mT{{\mathbf{T}}}
\def\mU{{\mathbf{U}}}
\def\mV{{\mathbf{V}}}
\def\mW{{\mathbf{W}}}
\def\mX{{\mathbf{X}}}
\def\mY{{\mathbf{Y}}}
\def\mZ{{\mathbf{Z}}}
\def\mBeta{{\mathbf{\beta}}}
\def\mPhi{{\mathbf{\Phi}}}
\def\mLambda{{\mathbf{\Lambda}}}
\def\mSigma{{\mathbf{\Sigma}}}


% Expectation
% \def\eE{\mathop{\mathbb{E}}\limits}
\def\eE{\mathbb{E}}

% Probability
\def\pP{\mathbb{P}}

% Tilde
\def\tf{\tilde{f}}
\def\tS{\tilde{S}}
\def\wtF{\widetilde{\mathcal{F}}}
\def\whR{\widehat{R}}
\def\tvx{\tilde{\mathbf{x}}}
\def\ty{\tilde{y}}


\def\defeq{\overset{\textup{def}}{=}}
% \def\defeq{\overset{.}{=}}
\def\defone{\overset{\text{\ding{172}}}{=}}
\def\deftwo{\overset{\text{\ding{173}}}{=}}
\def\leqone{\overset{\text{\ding{172}}}{\leq}}
\def\leqtwo{\overset{\text{\ding{173}}}{\leq}}
\def\leqthree{\overset{\text{\ding{174}}}{\leq}}
\def\leqfour{\overset{\text{\ding{175}}}{\leq}}
\def\eqone{\overset{\text{\ding{172}}}{=}}
\def\eqtwo{\overset{\text{\ding{173}}}{=}}
\def\eqthree{\overset{\text{\ding{174}}}{=}}
\def\eqfour{\overset{\text{\ding{175}}}{=}}
\def\geqfive{\overset{\text{\ding{176}}}{\geq}}

\usepackage{hyperref}
\usepackage{url}
\usepackage{booktabs} 
\usepackage{graphicx}
\usepackage{multirow}
\usepackage{multicol}
\usepackage{makecell}

\definecolor{ForestGreen}{HTML}{228B22}
\definecolor{BrickRed}{HTML}{8F1402}

\title{NeoBERT: A Next-Generation BERT}

% Authors must not appear in the submitted version. They should be hidden
% as long as the tmlr package is used without the [accepted] or [preprint] options.
% Non-anonymous submissions will be rejected without review.

\author{
Lola Le Breton$^{1,2,3}$ \quad Quentin Fournier$^{2}$ \quad Mariam El Mezouar$^{4}$ \quad Sarath Chandar$^{1,2,3,5}$ \\~\\
\textnormal{$^1$Chandar Research Lab \quad
$^2$Mila – Quebec AI Institute \quad $^3$Polytechnique Montréal \quad 
\\ $^4$Royal Military College of Canada \quad
$^5$Canada CIFAR AI Chair \\}
}





% The \author macro works with any number of authors. Use \AND 
% to separate the names and addresses of multiple authors.

\newcommand{\fix}{\marginpar{FIX}}
\newcommand{\new}{\marginpar{NEW}}

\def\month{MM}  % Insert correct month for camera-ready version
\def\year{YYYY} % Insert correct year for camera-ready version
\def\openreview{\url{https://openreview.net/forum?id=XXXX}} % Insert correct link to OpenReview for camera-ready version



\begin{document}


\maketitle

\begin{abstract}
	Recent innovations in architecture, pre-training, and fine-tuning have led to the remarkable in-context learning and reasoning abilities of large auto-regressive language models such as LLaMA and DeepSeek. In contrast, encoders like BERT and RoBERTa have not seen the same level of progress despite being foundational for many downstream NLP applications. To bridge this gap, we introduce NeoBERT, a next-generation encoder that redefines the capabilities of bidirectional models by integrating state-of-the-art advancements in architecture, modern data, and optimized pre-training methodologies. NeoBERT is designed for seamless adoption: it serves as a plug-and-play replacement for existing base models, relies on an optimal depth-to-width ratio, and leverages an extended context length of 4,096 tokens. Despite its compact 250M parameter footprint, it achieves state-of-the-art results on the massive MTEB benchmark, outperforming BERT$_{large}$, RoBERTa$_{large}$, NomicBERT, and ModernBERT under identical fine-tuning conditions. In addition, we rigorously evaluate the impact of each modification on GLUE and design a uniform fine-tuning and evaluation framework for MTEB. We release all code, data, checkpoints, and training scripts to accelerate research and real-world adoption\footnote{\url{https://huggingface.co/chandar-lab/NeoBERT}}\textsuperscript{,}\footnote{\url{https://github.com/chandar-lab/NeoBERT}}.
\end{abstract}

\section{Introduction}
\label{sec:introduction}

Auto-regressive language models have made tremendous progress since the introduction of GPT~\citep{radford2018improving}, and modern large language models (LLMs) such as LLaMA 3~\citep{dubey2024llama3herdmodels}, Mistral~\citep{jiang2023mistral7b}, OLMo~\citep{groeneveld2024olmoacceleratingsciencelanguage}, and DeepSeek-r1~\citep{deepseekai2025deepseekr1incentivizingreasoningcapability} now exhibit remarkable reasoning and in-context learning capabilities. These improvements result from scaling both the models and the web-scraped text datasets they are trained on, as well as from innovations in architecture and optimization techniques. However, while decoders have continuously evolved, encoders have not seen the same level of progress. Worse, their knowledge has become increasingly outdated despite remaining critical for a wide range of downstream NLP tasks that depend on their embeddings, notably for retrieval-augmented generation~\citep{ram2023incontextretrievalaugmentedlanguagemodels} and toxicity classification~\citep{hartvigsen-etal-2022-toxigen}. Despite being five years old, BERT~\citep{devlin2019bertpretrainingdeepbidirectional} and RoBERTa~\citep{liu2019robertarobustlyoptimizedbert} remain widely used, with more than 110 million combined downloads from Hugging Face as of the writing of this paper.

Similar to decoders, which undergo multi-stage processes of pre-training, instruction-tuning, and alignment, encoders also require successive training phases to perform well on downstream tasks. First, encoders go through self-supervised pre-training on large corpora of text with the masked language modeling objective. By predicting masked or replaced tokens, this stage enables models to learn the structural patterns of language and the semantics of words. However, the pre-training task is disconnected from downstream applications, and models require additional training to achieve strong performance in clustering or retrieval. Thus, a second fine-tuning phase is often achieved through multiple stages of semi-supervised contrastive learning, where models learn to differentiate between positive and negative sentence pairs, refining their embeddings in the latent space.

Recently, substantial progress has been made in improving the fine-tuning stage of pre-trained encoders, with models like GTE~\citep{li_towards_2023}, jina-embeddings~\citep{sturua2024jinaembeddingsv3multilingualembeddingstask}, SFR-embeddings~\citep{SFR-embedding-2}, and CDE~\citep{morris2024contextualdocumentembeddings} significantly outperforming previous encoders on the MTEB leaderboard, a recent and challenging benchmark spanning 7 tasks and 56 datasets. However, all these approaches focus on proposing complex fine-tuning methods and do not address the inherent limitations of their pre-trained backbone models.

As a result, there is a dire need for a new generation of BERT-like pre-trained models that incorporate up-to-date knowledge and leverage both architectural and training innovations, forming stronger backbones for these more advanced fine-tuning procedures. 

In response, we introduce NeoBERT, a next-generation BERT that integrates the latest advancements in architecture, data, and pre-training strategies. The improvements are rigorously validated on GLUE by fully pre-training 10 different models that successively incorporate the modifications. This validation ensures that the improvements benefit encoder architectures and highlights how some design choices drastically affect the model's abilities. Additionally, we design and experimentally validate a two-stage training procedure to increase NeoBERT's maximum context window from $1,024$ to $4,096$. To ensure a fair evaluation of NeoBERT against existing baselines and to isolate the impact of fine-tuning procedures, we propose a model-agnostic and systematic fine-tuning strategy with straightforward contrastive learning. All models are fine-tuned using this standardized approach and subsequently evaluated on the MTEB benchmark. 

On MTEB, NeoBERT consistently outperforms all competing pre-trained models while being 100M parameters smaller than the typical \textit{large}-sized encoders. With a context window of 4,096 tokens, it processes sequences 8× longer than RoBERTa \citep{liu2019robertarobustlyoptimizedbert} and two times longer than NomicBERT \citep{nussbaum2024nomicembedtrainingreproducible}. It is also the fastest encoder of its kind, significantly outperforming ModernBERT \textit{base} and \textit{large} in terms of inference speed. Despite its compact 250M parameter size, NeoBERT is trained for over 2T tokens, prioritizing training over scale to maximize accessibility for both academic researchers and industry users without requiring large-scale compute resources. This makes NeoBERT the most extensively trained model among modern encoders, ensuring robust generalization and superior downstream performance. Furthermore, NeoBERT maintains the same hidden size as \textit{base} models, allowing for seamless plug-and-play adoption without modifications to existing architectures. As the only fully open-source model of its kind, we release the code, data, training scripts, and model checkpoints, reinforcing our commitment to reproducible research.

\section{Related work}
\label{sec:related work}

In 2019, \citet{devlin2019bertpretrainingdeepbidirectional} introduced BERT, a novel approach to embedding text using bi-directional Transformers pre-trained without supervision on large corpora. Shortly after, \citet{liu2019robertarobustlyoptimizedbert} improved over BERT's pre-training by removing the next-sentence prediction objective and drastically increasing the amount of data, leading to RoBERTa. Since then, the primary focus of the community has shifted towards optimizing the fine-tuning phase of these models through contrastive learning, where the model is trained to maximize the similarity between positive text pairs while pushing them apart from negative samples.

Among the earliest contrastive learning approaches designed for encoders, SimCSE~\citep{gao2022simcsesimplecontrastivelearning} demonstrated that sentence pairs could be easily generated by feeding the same input to the model twice and applying dropout to introduce noise. However, this simple approach was soon outperformed by models like GTE~\citep{li_towards_2023}, which introduced more advanced contrastive learning techniques. GTE employed a weakly supervised stage that takes advantage of the vast number of successive sentence pairs available in traditional unlabeled datasets, followed by a semi-supervised stage incorporating labeled sentence pairs from high-quality datasets such as NLI~\citep{bowman2015largeannotatedcorpuslearning} and FEVER~\citep{thorne2018feverlargescaledatasetfact}. Recently, fine-grained strategies have emerged to better adapt models to both task and context. For instance, Jina-embeddings~\citep{sturua2024jinaembeddingsv3multilingualembeddingstask} introduced task-specific Low-Rank Adaptation (LoRA) adapters. As of January 2025, CDE~\citep{morris2024contextualdocumentembeddings} ranks at the top of the MTEB leaderboard for models under 250M parameters thanks to two key innovations: grouping samples with related contexts into the same batch and providing contextual embeddings for the entire corpus in response to individual queries.

However, pre-training has not seen the same level of effort, and thus progress, most likely due to its prohibitively high computational cost. RoBERTa, for instance, required a total of $1,024$ V100 days for its pre-training. As a result, GTE, Jina-embeddings, and CDE all rely on pre-trained BERT, XLM-RoBERTa~\citep{conneau2020unsupervisedcrosslingualrepresentationlearning}, and NomicBERT~\citep{nussbaum2024nomicembedtrainingreproducible} to initialize their respective models. The latter, NomicBERT, represents a recent effort to refine BERT's architecture and pre-training. NomicBERT incorporates architectural improvements such as SwiGLU and RoPE, utilizes FlashAttention, and extends the context length to $2,048$ tokens. Despite these innovations, NomicBERT still relied on sub-optimal choices, as discussed in \autoref{sec:neobert}. In parallel with the development of NeoBERT, \cite{warner2024smarterbetterfasterlonger} released ModernBERT with the goal of further refining the pre-training of NomicBERT. Although we share some of the modifications, we make distinct design choices and conduct thorough ablations that ultimately lead to greater performance on MTEB.

\section{NeoBERT}
\label{sec:neobert}

The following section describes NeoBERT's improvements over BERT and RoBERTa, as well as the recent NomicBERT and ModernBERT models. Since GTE and CDE use BERT and NomicBERT as their pre-trained backbone, they inherit their respective characteristics. \autoref{tab:hyperparam_data} summarizes the modifications.


\begin{table*}[!ht]
	\centering
	\small
	\setlength{\tabcolsep}{6pt}
	\renewcommand{\arraystretch}{1.2}
	\caption{Comparison of Model Architectures, Training Data, and Pre-Training Configurations.}
	\label{tab:hyperparam_data}
	\resizebox{\textwidth}{!}{
		\begin{tabular}{lcccccccc}
			\toprule
			& \multicolumn{2}{c}{\textbf{BERT}}& \multicolumn{2}{c}{\textbf{RoBERTa}}& \textbf{NomicBERT} & \multicolumn{2}{c}{\textbf{ModernBERT}}& \textbf{NeoBERT} \\ 
			                         & \textit{base} & \textit{large} & \textit{base} & \textit{large} & \textit{base} & \textit{base} & \textit{large} & \textit{medium} \\ 
			\midrule
			  
			\textbf{Layers}          & 12            & 24             & 12            & 24             & 12            & 22            & 28             & 28              \\
			\textbf{Hidden Size}     & 768           & $1,024$        & 768           & $1,024$        & 768           & 768           & $1,024$        & 768             \\
			\textbf{Attention Heads} & 12            & 16             & 12            & 16             & 12            & 12            & 16             & 12              \\
			\textbf{Parameters}      & 120M          & 350M           & 125M          & 355M           & 137M          & 149M          & 395M           & 250M            \\
			\textbf{Activation Function} & \multicolumn{4}{c}{GeLU}& SwiGLU & \multicolumn{2}{c}{GeGLU}& SwiGLU \\
			\textbf{Positional Encoding} & \multicolumn{4}{c}{Positional Embeddings}& RoPE & \multicolumn{2}{c}{RoPE}& RoPE \\
			\textbf{Normalization} & \multicolumn{4}{c}{Post-LayerNorm}& Post-LayerNorm & \multicolumn{2}{c}{Pre-LayerNorm}& Pre-RMSNorm \\
			
			\midrule
			 
			\textbf{Data Sources} & \multicolumn{2}{c}{\begin{tabular}{@{}c@{}}BooksCorpus \\ Wikipedia\end{tabular}} & \multicolumn{2}{c}{\begin{tabular}{@{}c@{}}BooksCorpus  \\ OpenWebText \\ Stories / CC-News\end{tabular}} & \begin{tabular}{@{}c@{}}BooksCorpus \\ Wikipedia\end{tabular} & \multicolumn{2}{c}{Undisclosed}& RefinedWeb \\
			\textbf{Dataset Size} & \multicolumn{2}{c}{13GB} & \multicolumn{2}{c}{160GB}  & 13GB & \multicolumn{2}{c}{-}& 2.8TB \\
			\textbf{Dataset Year} & \multicolumn{2}{c}{2019} & \multicolumn{2}{c}{2019}  & 2023 & \multicolumn{2}{c}{-}& 2023 \\
			\textbf{Tokenizer Level} & \multicolumn{2}{c}{Character} & \multicolumn{2}{c}{Byte}  & Character & \multicolumn{2}{c}{Character}& Character \\
			\textbf{Vocabulary Size} & \multicolumn{2}{c}{30K} & \multicolumn{2}{c}{50K}  & 30K & \multicolumn{2}{c}{50K}& 30K \\
			
			\midrule
			 
			\textbf{Sequence Length} & \multicolumn{2}{c}{512} & \multicolumn{2}{c}{512}  & $2,048$ & \multicolumn{2}{c}{$1,024$ $\rightarrow$ $8,192$}& $1,024$ $\rightarrow$ $4,096$ \\
			\textbf{Objective} & \multicolumn{2}{c}{MLM + NSP} & \multicolumn{2}{c}{MLM}  & MLM & \multicolumn{2}{c}{MLM}& MLM \\
			\textbf{Masking Rate} & \multicolumn{2}{c}{15\%} & \multicolumn{2}{c}{15\%}  & 30\% & \multicolumn{2}{c}{30\%}& 20\% \\
			\textbf{Masking Scheme} & \multicolumn{2}{c}{80/10/10} & \multicolumn{2}{c}{80/10/10}  & - & \multicolumn{2}{c}{-}& 100 \\
			\textbf{Optimizer} & \multicolumn{2}{c}{Adam} & \multicolumn{2}{c}{Adam}  & AdamW & \multicolumn{2}{c}{StableAdamW}& AdamW \\
			\textbf{Scheduler} & \multicolumn{2}{c}{-} & \multicolumn{2}{c}{-}  & - & \multicolumn{2}{c}{WSD}& CosineDecay \\
			\textbf{Batch Size} & \multicolumn{2}{c}{131k tokens} & \multicolumn{2}{c}{131k}  & 8M & \multicolumn{2}{c}{448k to 5M}& 2M \\
			\textbf{Tokens Seen} & \multicolumn{2}{c}{131B} & \multicolumn{2}{c}{131B}  & - & \multicolumn{2}{c}{$\sim$ 2T}& 2.1T \\
			\textbf{Training} & \multicolumn{2}{c}{DDP} & \multicolumn{2}{c}{DDP}  & \begin{tabular}{@{}c@{}}DeepSpeed \\ FlashAttention\end{tabular} & \multicolumn{2}{c}{\begin{tabular}{@{}c@{}}Alternate Attention \\ Unpadding \\ FlashAttention\end{tabular}}& \begin{tabular}{@{}c@{}}DeepSpeed \\ FlashAttention\end{tabular}\\
			\bottomrule
		\end{tabular}
	}
\end{table*}


\subsection{Architecture}
\label{sec:architecture}

The Transformer architecture has been refined over the years and has now largely stabilized, with models like LLaMA 3 being essentially the same as the original LLaMA. NeoBERT integrates the latest modifications that have, for the most part, become standard.

\paragraph{Depth-to-Width} The concept of depth efficiency has long been recognized in neural network architectures. In the case of Transformers, stacking self-attention layers is so effective that it can quickly saturate the network's capacity. Recognizing this, \citet{levine_limits_2020} provided theoretical and empirical evidence for an optimal depth-to-width ratio in Transformers. Their findings suggested that most language models were operating in a ``depth-inefficiency'' regime, where allocating more parameters to width rather than depth would have improved performance. In contrast, small language models like BERT, RoBERTa, and NomicBERT are instead in a width-inefficiency regime. To maximize NeoBERT's parameter efficiency while ensuring it remains a seamless plug-and-play replacement, we retain the original BERT$_{base}$ width of 768 and instead increase its depth to achieve this optimal ratio.

\paragraph{Positional Information} Transformers inherently lack the ability to distinguish token positions. Early models like BERT and RoBERTa addressed this by adding absolute positional embeddings to the token embeddings before the first Transformer block. However, this approach struggles to generalize to longer sequences and requires the positional information to be propagated across layers. To overcome these limitations, \citet{su2023roformerenhancedtransformerrotary} proposed Rotary Position Embeddings (RoPE), which integrate relative positional information directly into the self-attention mechanism. RoPE has quickly become the default in modern Transformers due to its significant improvements in performance and extrapolation capabilities. NeoBERT, like all newer encoders, integrates RoPE. Nevertheless, degradation still occurs with sequences significantly longer than those seen during training. As a solution, \citet{peng2023yarnefficientcontextwindow} introduced Yet Another RoPE Extension (YaRN), which allows to efficiently fine-tune models on longer contexts. NeoBERT is readily compatible with YaRN, making it well-suited for tasks requiring extended context.

\paragraph{Layer Normalization} Consistent with standard practices in modern Transformer architectures, we move the normalization layer inside the residual connections of each feed-forward and attention block, a technique known as Pre-Layer Normalization (Pre-LN). Pre-LN improves stability, allows for larger learning rates, and accelerates model convergence~\citep{xiong2020layernormalizationtransformerarchitecture}. While all newer encoder models adopt Pre-LN, they typically continue to use the classical LayerNorm rather than Root Mean Square Layer Normalization (RMSNorm). In NeoBERT, we substitute the classical LayerNorm with RMSNorm~\citep{zhang2019rootmeansquarelayer}, which achieves comparable training stability while being slightly less computationally intensive, as it requires one fewer statistic.

\paragraph{Activations} BERT and RoBERTa utilize the standard Gaussian Error Linear Unit (GELU) activation function. However, \citet{shazeer2020gluvariantsimprovetransformer} demonstrated the benefits of the Gated Linear Unit in Transformer architectures. These activation functions have since been adopted in several language models, including the LLaMA family. Following previous works, NeoBERT incorporates the SwiGLU activation function, and because it introduces a third weight matrix, we scale the number of hidden units by a factor of $\frac{2}{3}$ to keep the number of parameters constant.

\subsection{Data}
\label{sec:data}

Data has emerged as one of the most critical aspects of pre-training, and datasets with increasing quantity and diversity are frequently released. NeoBERT takes advantage of the latest datasets that have proven to be effective.

\paragraph{Dataset} BERT and NomicBERT were pre-trained on two carefully curated and high-quality datasets: Wikipedia and BookCorpus~\citep{zhu2015aligningbooksmoviesstorylike}. As \citet{baevski-etal-2019-cloze} demonstrated that increasing data size can improve downstream performance, \citet{liu2019robertarobustlyoptimizedbert} pre-trained RoBERTa on 10 times more data from BookCorpus, CC-News, OpenWebText, and Stories. However, RoBERTa's pre-training corpus has become small in comparison to modern web-scraped datasets built by filtering and deduplicating Common Crawl dumps. Following the same trend, we pre-trained NeoBERT on RefinedWeb~\citep{penedo2023refinedwebdatasetfalconllm}, a massive dataset containing 600B tokens, nearly 18 times larger than RoBERTa's. Although RefinedWeb does not have strict high-quality constraints, we believe that exposing the model to such a large and diverse dataset will improve its real-world utility.

\paragraph{Sequence Length} BERT and RoBERTa were pre-trained on sequences up to 512 tokens, which limits their downstream utility, especially without RoPE and YaRN. NomicBERT increased the maximum length to $2,048$ and employed Dynamic NTK interpolation at inference to scale to 8192. To further broaden NeoBERT's utility, we seek to increase the context length. However, due to the computational cost associated with pre-training, we adopt a two-stage pre-training procedure similar to LLMs like LLaMA 3. In the first stage, we train the model for 1M steps (2T tokens) using sequences truncated to a maximum length of $1,024$ tokens, referring to this version as NeoBERT$_{1024}$. In the second stage, we extend the training for an additional 50k steps (100B tokens), increasing the maximum sequence length to $4,096$ tokens. We refer to this final model as NeoBERT$_{4096}$. To ensure the model encounters longer sequences during this stage, we create two additional sub-datasets, Refinedweb$_{1024+}$ and Refinedweb$_{2048+}$, containing only sequence lengths greater than $1,024$ and $2,048$ tokens, respectively, alongside the original Refinedweb dataset. Each batch is sampled from Refinedweb, Refinedweb$_{1024+}$ and Refinedweb$_{2048+}$ with probabilities 20\%, 40\%, and 40\%. Since longer sequences tend to represent more complex or academic content, this strategy helps mitigate the distribution shift typically observed when filtering for longer sequences. We explore the benefits of this two-stage training strategy in \autoref{sec:seq_length}.


\subsection{Pre-Training}
\label{sec:training}

Encoder pre-training has received less attention than the data and architecture. However, many improvements made to decoders also apply to encoders. NeoBERT combines encoder-specific modifications with widely accepted decoder improvements.

\paragraph{Objective} In light of RoBERTa's findings that dropping the next-sentence prediction task does not harm performance, both NomicBERT and NeoBERT were only pre-trained on masked language modeling. Moreover, \citet{wettig_should_2023} challenged the assumption that the 15\% masking rate of  BERT and RoBERTa is universally optimal. Instead, their findings suggest that the optimal masking rate is actually 20\% for base models and 40\% for large models. Intuitively, a model learns best when the difficulty of its training tasks aligns with its capabilities. Based on their insight, we increase the masking rate to 20\%, while NomicBERT exceeds it by opting for 30\%.

\paragraph{Optimization} Following standard practice, we use the AdamW optimizer~\citep{loshchilov2019decoupledweightdecayregularization} with the same hyperparameters as LLaMA 2: $\beta_1 = 0.9$, $\beta_2 = 0.95$, and $\epsilon = 10^{-8}$. In preliminary experiments, we also considered SOAP \citep{vyas2025soapimprovingstabilizingshampoo}, a recent extension of the Shampoo optimizer, but it failed to outperform Adam and AdamW and has been omitted from the list of ablations. We employ a linear warmup for $2,000$ steps to reach a peak learning rate of $6 \times 10^{-4}$, followed by a cosine decay to 10\% of the peak learning rate over 90\% of the training steps. Once fully decayed, the learning rate remains constant for the last 100k steps at a sequence length of $1,024$ and 50k steps at a sequence length of $4,096$. We use a weight decay of 0.1 and apply gradient clipping with a maximum norm of 1.0.

\paragraph{Scale} Recent generative models like the LLaMA family~\citep{touvron2023llama2openfoundation, dubey2024llama3herdmodels} have demonstrated that language models benefit from being trained on significantly more tokens than was previously standard. Recently, LLaMA-3.2 1B was successfully trained on up to 9T tokens without showing signs of saturation. Moreover, encoders are less sample-efficient than decoders since they only make predictions for masked tokens. Therefore, it is reasonable to believe that encoders of similar sizes can be trained on an equal or even greater number of tokens without saturating. For NeoBERT's pre-training, we use batch sizes of 2M tokens over 1M steps in the first stage and 50k steps in the second, resulting in a theoretical total of 2.1T tokens. Note that because sequences are padded to the maximum length, this represents a theoretical number of tokens. In terms of tokens, this is comparable to RoBERTa and represents a 2x increase over NomicBERT. In terms of training steps, this amounts to a 2x increase over RoBERTa and a 10x increase over NomicBERT.

\paragraph{Efficiency} We improve efficiency by parallelizing the model across devices using DeepSpeed~\citep{aminabadi2022deepspeed-inference} with the ZeRO~\citep{rajbhandari2020zeromemoryoptimizationstraining} optimizer, reducing memory usage by eliminating data duplication across GPUs and increasing the batch size. We further optimize the GPU utilization by employing fused operators from the \texttt{xFormers} library to reduce overhead, selecting all dimensions to be multiples of 64 to align with GPU architectures, and removing biases to simplify computations without sacrificing performance. To address the quadratic demands of attention, we integrate FlashAttention~\citep{dao2023flashattention2fasterattentionbetter}, which computes exact attention without storing the full matrices.

\section{Effect of Design Choices}
\label{sec:ablations}

% add low batch size here
We conduct a series of ablations in controlled settings to evaluate our improvements to the original BERT architecture. We fully train each model for 1M steps, controlling for the seed and dataloader states to ensure successive models are trained with identical setups. These resource-intensive ablations were crucial to confirm our design choices, as they are based on the literature of pre-training decoder models. The baseline model, referred to as $M0$, is similar to BERT$_{base}$ but includes pre-layer normalization. Following RoBERTa, $M0$ also drops the next sentence prediction objective. We introduce modifications iteratively, resulting in a total of ten different models, as detailed in \autoref{tab:ablations}. To mitigate computational costs, the ablations are evaluated on the GLUE benchmark with a limited hyperparameter grid search of batch sizes $\in \{16, 32\}$ and learning rates $\in \{1e-5, 2e-5, 3e-5\}$. For the final model $M10$, we extend the grid search, as detailed in \autoref{app:glue}. Results are in \autoref{fig:glue-ablations}.

\begin{table}
	\caption{Modifications between successive ablations. The initial $M0$ baseline corresponds to a model similar to BERT, while $M9$ corresponds to NeoBERT.}
	\label{tab:ablations}
	\begin{center}
		\begin{tabular}{llll}
			\toprule
			\multicolumn{2}{c}{\textbf{Modification}} & \textbf{Before} & \textbf{After} \\\midrule
			\multirow{3}{*}{$M1$} & Embedding        & Positional          & RoPE       \\
			                      & Activation       & GELU                & SwiGLU     \\
			                      & Pre-LN           & LayerNorm           & RMSNorm    \\\midrule
			$M2$                  & Dataset          & Wiki + Book         & RefinedWeb \\\midrule
			$M3$                  & Tokenizer        & Google WordPiece    & LLaMA BPE  \\\midrule
			\multirow{2}{*}{$M4$} & Optimizer        & Adam                & AdamW      \\ 
			                      & Scheduler        & Linear              & Cosine     \\\midrule
			$M5$                  & Masking Scheme   & 15\% (80 / 10 / 10) & 20\% (100) \\\midrule
			$M6$                  & Sequence packing & False               & True       \\\midrule
			$M7$                  & Model Size       & 120M                & 250M       \\\midrule
			$M8$                  & Depth - Width    & 16 - 1056           & 28 - 768   \\\midrule
			\multirow{2}{*}{$M9$} & Batch size       & 131k                & 2M         \\
			                      & Context length   & 512                 & $4,096$    \\
			\bottomrule
		\end{tabular}
	\end{center}
\end{table}

\begin{figure}[!ht]
	\centering
	\caption{GLUE ablation scores on the development set. Modifications in grey are not included in the subsequent models. Increasing data size and diversity leads to the highest relative improvement ($M2$, \textcolor{ForestGreen}{$+3.6\%$}), followed by the model size ($M7$, \textcolor{ForestGreen}{$+2.9\%$}). Packing the sequences and using the LLaMA 2 tokenizer cause the largest relative drops ($M6$, \textcolor{BrickRed}{$-2.9\%$}, $M3$, \textcolor{BrickRed}{$-2.1\%$}).}
	\label{fig:glue-ablations}
	\includegraphics[width=0.9\linewidth]{figures/glue_ablations.pdf}
\end{figure}

\paragraph{Key Performance-Enhancing Modifications} As expected, the two modifications that had the greatest impact on the average GLUE score were related to scale. In $M2$, replacing Wikitext and BookCorpus with the significantly larger and more diverse RefinedWeb dataset improved the score by \textcolor{ForestGreen}{$+3.6\%$}, while increasing the model size from 120M to 250M in $M7$ led to a \textcolor{ForestGreen}{$+2.9\%$} relative improvement. Note that to assess the impact of the depth-to-width ratio, we first scale the number of parameters in $M7$ to 250M while maintaining a similar ratio to BERT$_{base}$, resulting in 16 layers of dimension 1056. In $M8$, the ratio is then adjusted to 28 layers of dimension 768.

\paragraph{Modifications That Were Discarded} In $M3$, replacing the Google WordPiece tokenizer with LLaMA BPE results in a \textcolor{BrickRed}{$-2.1\%$} performance decrease. We hypothesize that while the heterogeneous and multilingual nature of the LLaMA 2 vocabulary enhances broader applicability in decoders, it trades off performance for more compact encoder representations. In $M6$, we un-pad the sequences by concatenating samples of the same batch. While this removes unnecessary computation on padding tokens, packing sequences without accounting for cross-sequence attention results in a relative performance drop of \textcolor{BrickRed}{$-2.8\%$}. Although this modification was discarded from our subsequent ablations, we incorporate methods of un-padding with accurate cross-attention in our released codebase, following \cite{kundu2024enhancingtrainingefficiencyusing}.

\paragraph{Modifications Retained Despite Performance Trade-offs} Unexpectedly, using AdamW \citep{loshchilov2019decoupledweightdecayregularization} and cosine decay \citep{loshchilov2017sgdrstochasticgradientdescent} in $M4$ decreases performance by \textcolor{BrickRed}{$-0.5\%$}. As AdamW introduces additional regularization with weight decay, we expect that it will become beneficial when extending training by mitigating overfitting. Similarly, increasing the masking ratio from 15\% to 20\% in $M5$ decreases performance by \textcolor{BrickRed}{$-0.7\%$}. We hypothesize that increasing the task difficulty initially hinders downstream task performance but is likely to become advantageous when training larger models on more tokens. Consequently, we retain both modifications despite being unable to verify these hypotheses at scale due to the computational costs.


\section{Experiments}
\label{sec:experiments}

Selecting appropriate metrics and benchmarks is crucial for properly assessing the downstream performance and utility of language models. Following both early and recent studies, we include the GLUE and MTEB benchmarks in our evaluations.

\subsection{GLUE}
\label{sec:glue}

The GLUE benchmark \citep{wang2019gluemultitaskbenchmarkanalysis} is a cornerstone of language modeling evaluations and has played a significant role in the field. Although GLUE is now 6 years old and the community has long recognized its limitations, we report the GLUE score due to its widespread adoption and to facilitate the comparison of NeoBERT with existing encoders. Following standard practices, we fine-tune NeoBERT on the development set of GLUE with a classical hyperparameter search and introduce transfer learning between related tasks. Complete details of the fine-tuning and best hyperparameters are presented in \autoref{app:glue}. NeoBERT achieves a score of $89.0\%$ comparable to previous \textit{large} models while being $100M$ to $150M$ parameters smaller. We present the results in \autoref{tab:glue_dev}.

\begin{table*}[!ht]
	\centering
	\caption{GLUE scores on the development set. Baseline scores were retrieved as follows: BERT from Table 1 of \cite{devlin2019bertpretrainingdeepbidirectional}, RoBERTa from Table 8 of \cite{liu2019robertarobustlyoptimizedbert}, DeBERTa from Table 3 of \cite{he_debertav3_2023}, NomicBERT from Table 2 of \cite{nussbaum2024nomicembedtrainingreproducible}, GTE from Table 13 of \cite{zhang-etal-2024-mgte}, and ModernBERT from Table 5 of \cite{warner2024smarterbetterfasterlonger}.}
	\resizebox{\textwidth}{!}{
		\begin{tabular}{clcccccccc|c}
			\toprule
			\textbf{Size}                    & \textbf{Model}     & \textbf{MNLI}    & \textbf{QNLI}    & \textbf{QQP}     & \textbf{RTE}     & \textbf{SST}     & \textbf{MRPC}    & \textbf{CoLA}    & \textbf{STS}     & \textbf{Avg.}    \\
			\midrule
			\multirow{4}{*}{\makecell{Base \\ \small{($\leq 150M$)}}}&
			BERT     & 84.0& 90.5& 71.2& 66.4& 93.5& 88.9& 52.1& 85.8&79.6\\
			                                 & RoBERTa            & 87.6             & 92.8             & 91.9             & 78.7             & 94.8             & 90.2             & 63.6             & 91.2             & 86.4             \\
			                                 & GTE-en-8192        & 86.7             & 91.9             & 88.8             & 84.8             & 93.3             & 92.1             & 57.0             & 90.2             & 85.6             \\
			                                 & NomicBERT$_{2048}$ & 86.0             & 92.0             & 92.0             & 82.0             & 93.0             & 88.0             & 50.0             & 90.0             & 84.0             \\
			                                 & ModernBERT         & \underline{89.1} & \underline{93.9} & \underline{92.1} & \underline{87.4} & \underline{96.0} & \underline{92.2} & \underline{65.1} & \underline{91.8} & \underline{88.5} \\
			\midrule
			\multirow{2}{*}{\makecell{Medium \\ \small{$250M$}}} & NeoBERT$_{1024}$          & 88.9& \underline{93.9} & 90.7& 91.0& \underline{95.8}& 93.4& 64.8& \underline{92.1}&88.8\\
			                                 & NeoBERT$_{4096}$   & \underline{89.0} & 93.7             & \underline{90.7} & \underline{91.3} & 95.6             & \underline{93.4} & \underline{66.2} & 91.8             & \underline{89.0} \\
			\midrule
			\multirow{5}{*}{\makecell{Large \\ \small{($\geq 340M$)}}} &   BERT    & 86.3& 92.7& 72.1& 70.1& 94.9& 89.3& 60.5& 86.5&82.1\\
			                                 & RoBERTa            & 90.2             & 94.7             & 92.2             & 86.6             & 96.4             & 90.9             & 68.0             & 92.4             & 88.9             \\
			                                 & DeBERTaV3          & \textbf{91.9}    & \textbf{96.0}    & \textbf{93.0}    & \textbf{92.7}    & 96.9             & 91.9             & \textbf{75.3}    & \textbf{93.0}    & \textbf{91.4}    \\
			                                 & GTE-en-8192        & 89.2             & 93.9             & 89.2             & 88.1             & 95.1             & \textbf{93.5}    & 60.4             & 91.4             & 87.6             \\
			                                 & ModernBERT         & 90.8             & 95.2             & 92.7             & 92.1             & \textbf{97.1}    & 91.7             & 71.4             & 92.8             & 90.5             \\
			\bottomrule
		\end{tabular}
	}
	\label{tab:glue_dev}
\end{table*}

\subsection{MTEB}
\label{sec:mteb}

Beyond the GLUE benchmark, we consider the more recent and challenging MTEB benchmark~\citep{muennighoff_mteb_2023}, which has quickly become a standard for evaluating embedding models, with a wide coverage of 7 tasks and 56 datasets in its English subset.

MTEB tasks rely on the cosine similarity of embeddings pooled across tokens in a sentence. The most common and straightforward pooling strategy is computing the average of each token's final hidden representation. However, because out-of-the-box encoders are trained with the masked language modeling objective, they provide no guarantee that mean embeddings will produce meaningful representations without further fine-tuning. As a result, models require expensive fine-tuning strategies crafted for MTEB to achieve strong scores. For instance, GTE~\citep{li_towards_2023} with multi-stage contrastive learning, Jina-embeddings~\citep{sturua2024jinaembeddingsv3multilingualembeddingstask} with task-specific Low-Rank Adaptation (LoRA) adapters, and CDE~\citep{morris2024contextualdocumentembeddings}, with batch clustering and contextual corpus embeddings all pushed the limits of the leaderboard in their respective categories.

These coupled stages make it challenging to disentangle the respective impacts of pre-training and fine-tuning on the final model’s performance. To isolate and fairly evaluate the improvements introduced during pre-training, we implemented an affordable, model-agnostic fine-tuning strategy based on classical contrastive learning. This fine-tuning approach was designed in accordance with established methods in the literature. Its controlled settings ensured that all models were fine-tuned and evaluated under identical conditions.


\paragraph{Method} Given a dataset of positive pairs $\mathbb{D}=\{q_i, d_i^+\}_{i = 1}^{n}$, a similarity metric $s$, a temperature parameter $\tau$, and a set of negative documents $N_q$ for each query $q$, the contrastive loss is defined as:
\begin{equation*}
	\mathcal{L}=-\log \frac{e^{s(q, d^+) / \tau}}{e^{s(q, d^+) / \tau} + \sum_{d^- \in N_q} e^{s(q, d^-) / \tau}}
\end{equation*}

Negative documents can be either generic samples of the same format or tailored hard negatives, which exhibit a high degree of similarity to the contrasted sample in their original representation. We constructed a dataset of positive query-document pairs with optional hard negatives based on open-source datasets previously employed in the literature \citep{li2023generaltextembeddingsmultistage} for a total of nine million documents. In addition to the optional hard negatives, we also leverage in-batch, task-homogeneous negatives. In line with prior research ~\citep{li2023generaltextembeddingsmultistage}, we employ task-specific instructions and temperature-scaled sampling of the datasets. Complete details about the data, training, and evaluation can be found in \autoref{app:contrastive}.

\paragraph{Results} We found that training for more than 2,000 steps resulted in minimal performance gains. \autoref{tab:mteb_full} presents the complete MTEB-English evaluation of all fine-tuned models. Although NeoBERT is $100M$ parameters smaller than all \textit{large} baselines, it is the best model overall with a \textcolor{ForestGreen}{$+4.5\%$} relative increase over the second best model, demonstrating the benefits of its architecture, data, and pre-training improvements.

\begin{table*}[!ht]
	\centering
	\caption{MTEB scores on the English subset after 2,000 steps of fine-tuning with contrastive learning.}
	\resizebox{\textwidth}{!}{
		\begin{tabular}{clccccccc|c}
			\toprule
			\textbf{Size}                     & \textbf{Model}     & \textbf{Class.} & \textbf{Clust.} & \textbf{PairClass.} & \textbf{Rerank.} & \textbf{Retriev.} & \textbf{STS}  & \textbf{Summ.} & \textbf{Avg.} \\
			\midrule
			\multirow{4}{*}{\makecell{Base}}&
			BERT     & 60.6& 37.0& 71.5& 48.9& 28.3& 69.9& 31.1 &48.1\\
			                                  & RoBERTa            & 58.7            & 36.7            & 71.2                & 49.8             & 26.9              & 71.8          & 29.1           & 47.7          \\
			                                  & DeBERTaV3          & 45.9            & 15.2            & 44.3                & 39.0             & 3.5               & 42.2          & 25.0           & 26.9          \\
			                                  & NomicBERT$_{2048}$ & 55.0            & 35.3            & 69.0                & 48.8             & 30.5              & 70.1          & 30.1           & 47.1          \\
			                                  & ModernBERT         & 58.9            & 38.1            & 63.8                & 48.5             & 21.0              & 66.2          & 30.1           & 45.0          \\
			\midrule
			Medium                            & NeoBERT$_{4096}$   & 61.6            & \textbf{40.8}   & \textbf{76.2}       & 51.2             & \textbf{31.6}     & \textbf{74.8} & 30.7           & \textbf{51.3} \\
			\midrule
			\multirow{5}{*}{\makecell{Large}} & BERT               & 59.8            & 39.3            & 70.9                & 49.7             & 29.6              & 71.4          & \textbf{31.2}  & 49.1          \\
			                                  & RoBERTa            & 57.1            & 39.3            & 72.5                & \textbf{51.3}    & 30.0              & 71.7          & 31.1           & 48.9          \\
			                                  & DeBERTaV3          & 45.9            & 16.4            & 45.0                & 40.8             & 4.0               & 40.1          & 29.9           & 27.1          \\
			                                  & ModernBERT         & \textbf{62.4}   & 38.7            & 65.5                & 50.1             & 23.1              & 68.3          & 27.8           & 46.9          \\
			\bottomrule
		\end{tabular}
	}
	\label{tab:mteb_full}
\end{table*}

\subsection{Sequence Length}
\label{sec:seq_length}

Following previous literature, NeoBERT underwent an additional 50k pre-training steps, during which it was exposed to extended sequences of up to 4,096 tokens. To assess the impact of this additional training, we randomly sampled 2,467 long sequences from the English subset of Wikipedia. For each sequence, we independently masked each token at position $i$ and computed its cross-entropy loss, $l_i$. The pseudo-perplexity of the entire sentence is then defined as \( \mathcal{P} = \exp \left( \frac{1}{n} \sum_{i=1}^{n} l_i \right) \). We present the results in \autoref{fig:ppl}.

\begin{figure}[!htb]
	\centering
	\caption{Pseudo-Perplexity in function of the sequence length for NeoBERT$_{1024}$ \textit{(left)} and NeoBERT$_{4096}$ \textit{(right)}. This validates the effectiveness of the final pre-training stage on NeoBERT's ability to model long sequences.}
	\begin{minipage}{0.49\textwidth}
		\centering
		\includegraphics[width=\textwidth]{figures/ppl_NeoBERT_1024.pdf}
		\label{fig:plot1}
	\end{minipage}
	\hfill
	\begin{minipage}{0.49\textwidth}
		\centering
		\includegraphics[width=\textwidth]{figures/ppl_NeoBERT_4096.pdf}
		\label{fig:plot2}
	\end{minipage}
	\label{fig:ppl}
\end{figure}

Although NeoBERT$_{1024}$ was trained exclusively on sequences of up to $1,024$ tokens, it generalizes effectively to context lengths approaching 3,000 tokens. This demonstrates the robustness of RoPE embeddings to out-of-distribution inputs. Moreover, after an additional 50k training steps with sequences up to $4,096$ tokens, NeoBERT$_{4096}$ successfully models longer sequences. This approach provides a compute-efficient strategy for extending the model’s maximum context window beyond its original length.


\subsection{Efficiency}
\label{sec:efficiency}

To assess model efficiency, we construct a synthetic dataset consisting of maximum-length sequences of sizes $\{512, 1024, 2048, 4096, 8192\}$. For each sequence length, we scale the batch size from 1 to 512 samples or until encountering out-of-memory errors. Inference is performed for $100$ steps on a single A100 GPU, and we report the highest throughput achieved for each model and sequence length. \autoref{fig:efficiency} presents the results. 

Due to their low parameter count and relatively simple architecture, BERT and RoBERTa are the most efficient for sequences up to 512 tokens. However, their use of positional embeddings prevents them from further scaling the context window. For extended sequences, NeoBERT significantly outperforms ModernBERT$_{base}$, despite having $100M$ more parameters, achieving a $46.7\%$ speedup on sequences of $4,096$ tokens.

\begin{figure}[!ht]
	\centering
	\caption{Model throughput (tokens per second) as a function of sequence length ($\uparrow$ is better). Above $1,024$ in sequence length, NeoBERT surpasses ModernBERT$_{base}$ despite having $100M$ more parameters.}
	\label{fig:efficiency}
	\includegraphics[width=0.9\linewidth]{figures/efficiency.pdf}
\end{figure}


\section{Discussion}
\label{sec:discussion}

Encoders are compact yet powerful tools for language understanding and representation tasks. They require fewer parameters and significantly lower training costs compared to their decoder counterparts, making them strong alternatives for applications that do not involve text generation. Traditionally, the representational capacity of these models has been assessed through downstream tasks such as classification, in particular through the GLUE benchmark. 

While GLUE has played a pivotal role in guiding model adoption, it includes only nine sequence classification datasets, four of which are entailment tasks. Moreover, its small dataset sizes and occasionally ambiguous labeling make it prone to overfitting, with models long surpassing human performance on the benchmark. Although DeBERTa-v3 achieves state-of-the-art performance on GLUE by a significant margin, our fine-tuning experiments reveal its comparatively poor performance on the more recent MTEB benchmark. MTEB encompasses a broader range of datasets and tasks, but attaining high performance on its leaderboard necessitates carefully crafted fine-tuning strategies with costly training requirements. As more complex fine-tuning strategies emerge, it becomes unclear what the source of score improvements is. Moreover, these strategies are not easily reproducible or accessible, limiting the possibility of fair comparison between pre-trained backbones.

This underscores the limitations of current evaluation paradigms and highlights the need for more affordable and standardized frameworks. We advocate for future research to focus on the development of standardized fine-tuning protocols and the exploration of new zero-shot evaluation methodologies to ensure a more comprehensive and unbiased assessment of encoder-only models.


\section{Conclusion}
\label{sec:conclusion}

We introduced NeoBERT, a state-of-the-art encoder pre-trained from scratch with the latest advancements in language modeling, architecture, and data selection. To ensure rigorous validation, we systematically evaluated every design choice by fully training and benchmarking ten distinct models in controlled settings. On GLUE, NeoBERT outperforms BERT$_{large}$ and NomicBERT and is comparable with RoBERTa$_{large}$ despite being 100M parameters smaller and supporting sequences eight times longer. To further validate its effectiveness, we conducted a comprehensive evaluation on MTEB, carefully isolating the effects of pre-training and fine-tuning to provide a fair comparison against the best open-source embedding models. Under identical fine-tuning conditions, NeoBERT consistently outperforms all baselines. With its unparalleled efficiency, optimal depth-to-width, and plug-and-play compatibility, NeoBERT represents the next generation of encoder models. To foster transparency and collaboration, we release all code, data, model checkpoints, and training scripts, making NeoBERT the only fully open-source model of its kind.

\subsubsection*{Broader Impact Statement}

Despite its improvements, NeoBERT inherits the biases and limitations of its pre-training data. Moreover, the greatest jump in performance stems from the pre-training dataset, and as newer, larger, and more diverse datasets become available, retraining will likely be needed to further improve its performance. Nonetheless, NeoBERT stands today as an affordable state-of-the-art pre-trained encoder with great potential for downstream applications.

\section*{Acknowledgements}
\label{sec:acknowledgements}

Sarath Chandar is supported by the Canada CIFAR AI Chairs program, the Canada Research Chair in Lifelong Machine Learning, and the NSERC Discovery Grant. Quentin Fournier is supported by the Lambda research grant program. The authors acknowledge the computational resources provided by Mila and the Royal Military College of Canada.

% \bibliographystyle{tmlr}
\documentclass{MITstyle}

%\usepackage[table]{xcolor}
\usepackage{chngcntr}
\usepackage{hyperref}
\usepackage{microtype}

\title{A Lightweight and Extensible Cell Segmentation and Classification Model for Whole Slide Images}

\author{Nikita Shvetsov~$^{1, }$\footnote{Correspondence e-mail: nikita.shvetsov@uit.no}, Thomas K. Kilvaer~$^{2, 3}$, Masoud Tafavvoghi~$^{4}$, Anders Sildnes~$^{1}$, \\ Kajsa Møllersen~$^{4}$, Lill-Tove Rasmussen Busund~$^{5, 6}$, Lars Ailo Bongo~$^{1}$ \\
%
\vspace{1em} % Space between authors and afilliations
%
\normalfont{\small $^{1}$Department of Computer Science, UiT The Arctic University of Norway}\\
\normalfont{\small $^{2}$Department of Oncology, University Hospital of North Norway}\\
\normalfont{\small $^{3}$Department of Clinical Medicine, UiT The Arctic University of Norway}\\
\normalfont{\small $^{4}$Department of Community Medicine, UiT The Arctic University of Norway}\\
\normalfont{\small $^{5}$Department of Medical Biology, UiT The Arctic University of Norway} \\
\normalfont{\small $^{6}$Department of Clinical Pathology, University Hospital of North Norway} %\vspace{2em}
}

\begin{document}
\maketitle

\section*{Abstract}

% \begin{abstract}
% Developing clinically useful cell-level analysis tools in digital pathology remains challenging due to limitations in dataset granularity, inconsistent annotations, computational demands of advanced models, and difficulties in integrating new technologies into clinical workflows. To address these challenges, we propose a multi-faceted solution that enhances data quality, model performance, and usability to create a lightweight and extensible cell segmentation and classification model.

% First, we update data labels by employing a cross-relabeling process that refines the labels of two existing datasets, PanNuke and MoNuSAC, to create a new unified dataset with enhanced granularity, encompassing seven distinct cell types. Second, we leverage the H-Optimus foundation model as a fixed encoder to improve feature representation for simultaneous cell segmentation and classification tasks. Third, to address the computational demands of foundation models, we employ knowledge distillation to reduce model size and complexity while maintaining comparable performance. Finally, to facilitate integration into clinical workflows, we integrate the distilled model into the QuPath software, a widely used open-source platform in digital pathology.

% Our results demonstrate improvements in cell segmentation and classification performance using the H‑Optimus-based model compared to a CNN-based model. Specifically, the average $R^2$ improved from 0.575 to 0.871, and the average $PQ$ score improved from 0.450 to 0.492, indicating better alignment with actual cell counts and enhanced segmentation and classification quality. Furthermore, the distilled student model maintains performance comparable to the larger foundation model while reducing the parameter count by a factor of 48.
% Overall, by reducing computational complexity and integrating it into existing workflows, the proposed approach may significantly impact diagnostic processes, reduce the workload of pathologists, and contribute to improved patient outcomes. Though our approach shows potential enhancements in efficiency and usability of cell segmentation and classification models in digital pathology, extensive validation is needed to deploy these models in clinical practice.
% \end{abstract}

%%% shortened abstract
\begin{abstract}
Developing clinically useful cell-level analysis tools in digital pathology remains challenging due to limitations in dataset granularity, inconsistent annotations, high computational demands, and difficulties integrating new technologies into workflows. To address these issues, we propose a solution that enhances data quality, model performance, and usability by creating a lightweight, extensible cell segmentation and classification model. 

First, we update data labels through cross-relabeling to refine annotations of PanNuke and MoNuSAC, producing a unified dataset with seven distinct cell types. Second, we leverage the H-Optimus foundation model as a fixed encoder to improve feature representation for simultaneous segmentation and classification tasks. Third, to address foundation models' computational demands, we distill knowledge to reduce model size and complexity while maintaining comparable performance. Finally, we integrate the distilled model into QuPath, a widely used open-source digital pathology platform. 

Results demonstrate improved segmentation and classification performance using the H-Optimus-based model compared to a CNN-based model. Specifically, average $R^2$ improved from 0.575 to 0.871, and average $PQ$ score improved from 0.450 to 0.492, indicating better alignment with actual cell counts and enhanced segmentation quality. The distilled model maintains comparable performance while reducing parameter count by a factor of 48. By reducing computational complexity and integrating into workflows, this approach may significantly impact diagnostics, reduce pathologist workload, and improve outcomes. Although the method shows promise, extensive validation is necessary prior to clinical deployment.
\end{abstract}
\clearpage

\section{Introduction}
In digital pathology, accurate segmentation and classification of cells are crucial for many diagnostic, prognostic, and predictive analyses \cite{Jaber_Beziaeva_etal._2019,Lin_Pan_etal._2022,Park_Ock_etal._2022,Shen_Choi_etal._2024}. Nowadays, developments in computational pathology offer multiple solutions \cite{H._Qu_P._Wu_etal._2020,Javed_Mahmood_etal._2020} to utilize cell-level datasets to train machine learning models that solve these problems. The quality and specificity of training datasets are critical for robust and accurate models. Adhering to the principle of "garbage in, garbage out", it is essential to ensure that these datasets are extensively and accurately labeled with distinct classes that reflect the diverse biological characteristics of different cell types. Unfortunately, the number of open-source datasets comprising such high-quality annotations is limited. Existing cell segmentation datasets \cite{Gamper_Koohbanani_etal._2019,Graham_Vu_etal._2019,Verma_Kumar_etal._2021} may offer extensive annotations for certain cell types while providing more general labels for others. For example, in PanNuke, which is one of the largest open-source datasets comprising labeled cells, various types of morphologically and functionally different inflammatory cells like macrophages and lymphocytes are clustered in a broad "inflammatory" class. Consequently, these classes are frequently omitted from analyses or aggregated into broader meta-classes \cite{Gamper_Koohbanani_etal._2020} and likely interfere with other cell classes included in the dataset. This and similar inconsistencies in annotation granularity limit the ability of machine learning models to learn the comprehensive and nuanced features necessary for accurate cell segmentation and classification. To address these challenges, methods for refining and standardizing dataset annotations are essential to enhance the quality of training data.

A complementary approach to mitigate the absence of high-quality training data is the use of foundation models. Foundation models as encoders are defined as large-scale, versatile networks pre-trained on vast, diverse datasets using self-supervised learning, contrasting with convolutional neural network (CNN) pre-trained encoders that rely on supervised learning with labeled data. In practice, foundation models leverage enormous amounts of weakly or unlabeled data from millions of whole slide images (WSIs) and employ self-attention mechanisms to capture long-range dependencies and global context \cite{Chen_Ding_etal._2024,Saillard_Jenatton_etal._2024,Vorontsov_Bozkurt_etal._2024,Xu_Usuyama_etal._2024}. As a consequence, foundation models are able to produce transferable feature representations across different cell types and tissue environments. The feature representations can be leveraged by decoder networks to produce segmentation masks and pixel-level classifications. Because foundation models have comprehensive feature representations, they can be effectively fine-tuned using much smaller amounts of cell-level data compared to the large datasets needed to train models from scratch. Furthermore, foundation models incorporate adversarial training elements or contrastive learning \cite{Chen_Ding_etal._2024,Xu_Usuyama_etal._2024}, enhancing their resilience and adaptability by exposing them to challenging and varied scenarios during training. This may result in more generalizable models, often making them well-suited for diverse and complex tasks in digital pathology.

Despite the inherent advantages of foundation models, their deployment for practical use faces its own obstacles. In particular, they require substantial computational power, financial investments and rigorous testing to ensure reliability and efficacy for a given task \cite{Akkus_Dangott_etal._2022,Dragomir_Cocuz_etal._2022,Go_2022,Jafri_Farooqui_etal._2024}. Moreover, while foundation models enhance feature representation and performance, they depend on the quality of available annotations for decoder fine-tuning and, like any other model, cannot resolve existing inconsistencies or ambiguities in data labels. Therefore, there remains a critical need for solutions that address both data quality and practical deployment considerations.
Further, integrating new technologies into existing clinical workflows often encounters resistance, as it necessitates adjustments to established diagnostic processes. So, there is a need to develop solutions that could be integrated into current practices, minimizing the burden on medical professionals to adopt new tools \cite{King_Williams_etal._2023}.

Existing solutions \cite{Goldsborough_Philps_etal._2024,Hörst_Rempe_etal._2024}, while addressing some aspects of these challenges, fall short in providing a comprehensive approach. To address the data quality and clinical deployment issues, we propose a multi-faceted solution that encompasses data refinement, model optimization, and integration with existing pathology tools (\hyperref[fig:fig1]{Figure 1}). The outcome is a lightweight cell segmentation and classification model that can be integrated into digital pathology workflows for practical clinical use.

\begin{figure}[h!]
    \centering
    \includegraphics[width=\textwidth, height=0.82\textheight, keepaspectratio]{images/Figure_1.pdf}
    \caption{Overview of the proposed solution, including 1) Data refinement using cross-relabeling, 2) Teacher model development and fine tuning, 3) Student model optimization with knowledge distillation and 4) Student model and QuPath integration}
    \label{fig:fig1}
\end{figure}
\clearpage

Our approach begins with preparing the data for the fine-tuning and training of the machine learning models. We create a refined dataset, acquired via cross-relabeling two cell-level datasets, enhancing annotation specificity and consistency of the labeled data. Subsequently, we create a cell segmentation and classification model based on the foundation model. We leverage the foundation model as a fixed encoder and fine-tune a decoder using the refined dataset to improve generalization across diverse tissue- and cell types.
To ensure that the model remains lightweight and deployable in a possibly resource-constrained environment, we employ knowledge distillation to approximate the functionality of the foundation model. Finally, to facilitate the practical application of our model in digital pathology workflows, we integrate it with the QuPath \cite{Bankhead_Loughrey_etal._2017} application. Each methodological component contributes to the overarching goal of enhancing model performance, generalizability, and usability in clinical settings.

The primary contributions of this paper are:
\begin{enumerate}
    \item \textit{Data labels refinement through cross-relabeling:}
    
    We propose a new method for refining labels of cell-level datasets through cross-relabeling. This method employs classification models to re-label broad and ambiguous instances, resulting in a more diverse dataset. Our evaluation demonstrates that these classification models achieve high accuracy on test subsets, indicating the reliability of the method for label refinement.

    \item \textit{Enhanced model performance via foundation models:}
    
    We employ a foundation model as a feature extractor for the cell segmentation and classification task. In comparison with training a CNN model from scratch, the foundation model backbone only needs fine-tuning, which significantly reduces training time, computational resources and data requirements. We show that using a foundation model encoder leads to better performance in cell segmentation and classification networks than using a CNN-based encoder. This improvement may enable the model to generalize more effectively across various tissue types and imaging methods.
    
    \item \textit{Model optimization through knowledge distillation:}
    
    We show that a smaller student model trained using knowledge distillation on the refined dataset obtained via our cross-relabeling approach from a foundation model achieves comparable performance in cell segmentation and quantification tasks. As a result, this model is more suitable for deployment in environments without high-performance computing resources.
    
    \item \textit{Integration with QuPath:}
    
    We integrate the distilled cell segmentation and classification model into QuPath, a widely used open-source digital pathology platform, to accelerate clinical adaptation by enabling pathologists to more easily incorporate advanced computational tools into their existing workflows.
\end{enumerate}

Through these methodological steps, we aim to bridge the gap between advanced machine learning techniques and practical clinical applications, making accurate and efficient digital pathology accessible in a broader range of healthcare settings.

\section{Refining Existing Datasets Using Cross-Relabeling}
To address the limitations of sparse and ambiguous labeling of cell-level datasets, we propose a generalizable cross-relabeling strategy that can be applied to any dataset containing broadly categorized or imprecisely labeled cell types. This approach involves training and subsequently leveraging classification models to refine broad categories into more specific or biologically relevant classes.
When applied to cell-level data, the methodology includes extracting individual cell images from the dataset patches, preprocessing these images to standardize the size and accommodate partial cells, and then training deep learning classifiers capable of distinguishing between the finer cell subtypes within the coarser categories. 
To illustrate our approach, we focus on the PanNuke \cite{Gamper_Koohbanani_etal._2020, Gamper_Koohbanani_etal._2019} and MoNuSAC \cite{Verma_Kumar_etal._2021} datasets that we have used to train models for cell quantification in our previous works \cite{Shvetsov_Grønnesby_etal._2022,Shvetsov_Sildnes_etal._2024}. We find that for better cell differentiation we have to introduce more granular labels. PanNuke includes a broad classification of "inflammatory" cells, encompassing lymphocytes, macrophages, and neutrophils. Each cell type differs significantly in structure, function, and clinical relevance. Conversely, MoNuSAC uses the label "epithelial" for a class that comprises both benign epithelial cells and malignant neoplastic cells. This practice makes it challenging to differentiate between benign and malignant epithelial cells in the dataset, which is a critical distinction when identifying tumor areas within tissue samples. To address these issues, we implement a cross-relabeling strategy as shown in \hyperref[fig:fig2]{Figure 2}. The key components are two classification models: one is trained on singular cell images from PanNuke data to classify the epithelial meta-class into epithelial and neoplastic classes. The other is trained on MoNuSAC to refine the inflammatory class into lymphocytes, neutrophils, and macrophages.

\begin{figure}[h!]
    \centering
    \includegraphics[width=\textwidth]{images/Figure_2.pdf}
    \caption{Refined dataset generation via cross relabeling}
    \label{fig:fig2}
\end{figure}

The refining approach consists of three consecutive steps. The first is the preprocessing step, in which we extract individual cells from both datasets (\hyperref[fig:fig3]{Figure 3}). The specifics of PanNuke and MoNuSAC patch preparation before cell preprocessing are provided in \hyperref[chap:S1]{Appendix S1}.

\begin{figure}[h!]
    \centering
    \includegraphics[width=\textwidth]{images/Figure_3.pdf}
    \caption{Cell instances preprocessing including (1) cell map extraction, (2) bounding box delineation, (3) adjusting cell boxes and (4) cropping and resizing of cell images}
    \label{fig:fig3}
\end{figure}

During preprocessing, we extract cell type maps from the ground truth label mask and calculate bounding boxes around each cell instance. To accommodate partial cells at patch borders, a common issue in cropped patch images, we employ mirror padding and extend the field of view of the cell label by 15 pixels to capture adjacent cells. We then crop and resize the identified regions to $64 \times 64$ pixels using bicubic interpolation.

The preprocessed PanNuke dataset comprises 68,031 neoplastic and 23,207 epithelial cell images, while MoNuSAC comprises  33,104 lymphocytes, 1,252 neutrophils, and 1,695 macrophages, which we subsequently use in training cell classification models and classifying the cell image data \hyperref[fig:S2]{Appendix Figure S2 (1)}. 

The next step is to train two distinct ResNet50-based classifiers tailored to address the specific labeling challenges inherent in each dataset. We use ResNet50 for classification models due to its proven effectiveness for image classification tasks in histopathology \cite{pan2022reviewmachinelearningapproaches}, and its compatibility with small images. For the PanNuke dataset, we design the classifier, trained on MoNuSAC data, to disaggregate the heterogeneous "inflammatory" cell category into distinct subtypes: lymphocytes, macrophages, and neutrophils. Similarly, for the MoNuSAC dataset, the classifier is trained on PanNuke data and distinguishes between benign and malignant epithelial cells within the overarching "epithelial" label. By applying these targeted classifiers to their respective datasets, we assign more specific labels to individual cell instances, thus enabling us to create a unified dataset.
To ensure a balanced representation of classes, we train both models on datasets that had been equalized to match the size of the least represented class. Thus, we obtain datasets comprising 23,207 samples per class for PanNuke and 1,252 samples per class for MoNuSAC data. Next, we partition both of them into training (70\%), validation (20\%), and testing (10\%) subsets. To mitigate the risk of overfitting, we use a single dropout layer with a rate of p=0.5 in both models and data augmentation using randomized color perturbations, rotation, and horizontal and vertical flipping. We employ AdamW optimizer and the cross-entropy loss function for the training criterion.

To evaluate the two trained models, we measure the classification accuracy on the respective test subsets. The accuracies on the test subset for both classifiers are presented in \hyperref[tab:1]{Table 1}. The PanNuke model achieves an average accuracy of 93.57\%, with higher accuracy for neoplastic cells (96.06\%) compared to epithelial cells (86.26\%). The confusion matrix in Figure A3.1 shows that the model predominantly distinguishes accurately between epithelial and neoplastic tissues, with a substantial number of correct classifications and relatively few misclassifications. The MoNuSAC model demonstrates an average accuracy of 98.92\%, excelling in classifying lymphocytes (99.67\%) and macrophages (94.12\%), with lower performance for neutrophils (85.71\%). The confusion matrix in Figure A3.2 shows that the model identifies lymphocytes and performs reasonably well with macrophages and neutrophils.

\begin{table}[h!]
\renewcommand{\arraystretch}{1.5}
  \centering
  \caption{Cell classification results for PanNuke and MoNuSAC trained models (CI 95\%).}
  \label{tab:1}
  \begin{tabular}{|l|c|c|}
   \hline
   %\rowcolor{gray!30}
    Accuracy               & PanNuke model              & MoNuSAC model              \\
    \hline
    Average      & 0.936 (0.931--0.941)         & 0.989 (0.986--0.993)        \\
    \hline
    Neoplastic   & 0.961 (0.956--0.965)         & -                          \\
    \hline
    Epithelial   & 0.863 (0.849--0.877)         & -                          \\
    \hline
    Lymphocytes  & -                          & 0.997 (0.995--0.999)        \\
    \hline
    Neutrophils  & -                          & 0.857 (0.796--0.918)        \\
    \hline
    Macrophages  & -                          & 0.941 (0.906--0.976)        \\
    \hline
  \end{tabular}
\end{table}

Finally, during the last step, we use the model trained on PanNuke data for epithelial cells in MoNuSAC and the model trained on MoNuSAC for the inflammatory cells class in PanNuke. Specifically, we use classifier models to relabel epithelial cells in MoNuSAC and inflammatory cells in PanNuke data. Then we combine cells with refined labels and the rest of the cells in both datasets to create a refined dataset (\hyperref[fig:S2]{Appendix Figure S2 (2)}). The process of relabeling cells and visualizing them on a patch is shown in \hyperref[fig:fig4]{Figure 4}. The cell counts in the refined dataset are provided in \hyperref[tab:S4]{Appendix Table S4}.

\begin{figure}[h!]
    \centering
    \includegraphics[width=\textwidth, height=0.42\textheight, keepaspectratio]{images/Figure_4.pdf}
    \caption{Cell relabeling procedure for epithelial and inflammatory cell classes}
    \label{fig:fig4}
\end{figure}

%\hfill

Relabeling and combining datasets have been explored in a prior study \cite{Parulekar_Kanwat_etal._2023}, where consecutive fine-tuning on multiple datasets was employed to account for hierarchical class label structures. While the method presented in \cite{Parulekar_Kanwat_etal._2023} is intuitive, it often lacks consistency and requires multiple fine-tuning runs, which can be cumbersome and time-consuming. 
In contrast, cross-relabeling simplifies this process by using specialized classification models tailored to each dataset's specific labeling challenges. This approach provides better transparency and produces a unified dataset encompassing seven distinct cell types across multiple tissue samples, enhancing data diversity for further model training or fine-tuning.

Despite these improvements, cross-relabeling does not entirely resolve issues related to poor labeling quality or the amount of labeled data. Specifically, our results show lower accuracies persist for underrepresented classes, such as macrophages, which may stem from a limited sample availability and intrinsic challenges in distinguishing these cells based solely on H\&E staining. Furthermore, while our method enhances label specificity, it relies on the initial quality of the broad labels; thus, any fundamental inaccuracies in the original annotations can propagate through the relabeling process. Addressing the overall problem of limited data labels may require integrating additional data sources or utilizing complementary immunohistochemical staining methods.
Although the reported performance metrics are obtained from evaluations on the native test sets of each dataset, it is important to note that the primary application of these classifiers is to perform cross-relabeling, where a model trained on one dataset (e.g., PanNuke) is applied to another (e.g., MoNuSAC) and vice versa. We acknowledge that a more systematic evaluation of cross-dataset generalization is needed and could be performed in future work.

Overall, the refined dataset produced by our approach can enhance the supervised training or fine-tuning of cell segmentation and classification models, especially those that utilize pre-trained foundation models to improve feature extraction robustness. In addition, these models can detect nuanced classes that enable researchers to conduct more detailed analyses of biological processes in computational pathology.

\section{Foundation models for robust cell segmentation and classification}

Accurate cell segmentation and classification in digital pathology are hindered by limited labeled data and the fact that conventional CNNs are unable to capture global contextual information due to their local receptive field constraints \cite{Gheflati_Rivaz_2022,Yang_Marcus_etal.}. Traditional approaches in cell quantification have predominantly relied on CNN encoders, such as ResNet50, given their proven effectiveness in semantic segmentation tasks \cite{Deshmane_2023,Graham_Vu_etal._2019,Mukasheva_Koishiyeva_etal._2024,Stringer_Wang_etal._2021}. However, approaches that include fine-tuning of pretrained CNNs, data augmentation, and stain normalization to partially increase data variability and address staining differences often fail to achieve the necessary generalization and robustness across diverse tissue types and staining conditions \cite{G._Wang_W._Li_etal._2018,Gao_Bagci_etal._2018,Karim_El_Khoury_Martin_Fockedey_etal._2021}.

To overcome these challenges, we leverage an encoder-decoder network that uses a foundation model as the encoder and a CNN upsampling decoder (\hyperref[fig:fig5]{Figure 5}) for simultaneous cell segmentation and classification in 2D patches extracted from WSIs. Foundation models with transformer-based architectures are viable alternatives to CNN-based encoders \cite{Shamshad_Khan_etal._2023,Sourget_2023}. They enable the creation of more advanced architectures that can decode or transform learned features more effectively \cite{Chen_Duan_etal._2023,Cheng_Misra_etal._2022,Xie_Wang_etal._2021}.

\begin{figure}[h!]
    \centering
    \includegraphics[width=\textwidth]{images/Figure_5.pdf}
    \caption{UNETR-like model with foundational model as backbone}
    \label{fig:fig5}
\end{figure}

By utilizing a transformer-based encoder, we incorporate global contextual information into the feature extraction process, which is a key advantage of such architectures \cite{Chen_Lu_etal._2021}. This foundation model integration facilitates accurate pixel-wise segmentation and classification without the need for extensive encoder training, thereby potentially improving generalization across varied cellular structures and tissue types.
In our implementation, we employ a modified UNETR \cite{Hatamizadeh_Tang_etal._2021} architecture that combines a vision transformer (ViT) \cite{Dosovitskiy_Beyer_etal._2021} encoder with a CNN-based decoder. The encoder utilizes the pretrained H-Optimus foundation model, which contains 1.1 billion parameters and is trained on over 500,000 H\&E stained WSIs \cite{Saillard_Jenatton_etal._2024}. We extract outputs from four evenly spaced transformer blocks $Z_i$, where $i \in [1, 14, 26, 38]$, to serve as residual connections for the CNN decoder. We select these blocks based on our observation that features from non-adjacent levels of the encoder lead to better overall performance on the test subset.

The CNN decoder upsamples the feature representations, acquired from the transformer blocks, to generate an intermediate vector that is handled by two task-specific layers that generate cell segmentation and classification masks. The first task-specific layer is the ‘Cellpose head’,  which is used to delineate cell instances. The layer generates horizontal and vertical gradient maps to form vector fields that are refined through gradient tracking in a post-processing step using the Cellpose algorithm \cite{Stringer_Wang_etal._2021}, known for its efficacy in cell segmentation tasks and generalizability across multiple domains \cite{Pachitariu_Stringer_2022,Stringer_Pachitariu_2024}. The second task-specific layer is the "Cell type head", which assigns labels to individual pixels. In the post-processing step, we determine the output classification label of each segmented cell instance by majority voting over the labeled pixels that comprise the cell in the segmentation map.

To evaluate model performance and measure the impact of adding a foundation model as backbone, we compare it to a ResNet50-based model. ResNet50 is a widely used solution for encoders in segmentation architectures in the medical domain \cite{Deshmane_2023,Graham_Vu_etal._2019,Mukasheva_Koishiyeva_etal._2024,Stringer_Wang_etal._2021}. For the H-Optimus-based model, we utilize frozen weights for the encoder and only fine-tune the decoder to take advantage of the extensive pre-training of the foundation model. For the ResNet50-based model we start with ImageNet \cite{Deng_Dong_etal.} weights and train both encoder and decoder parts. Hyperparameters for the training step are set to be identical, where possible, for comparable evaluation. 
For this evaluation, we deliberately use the PanNuke dataset to provide a standardized and controlled comparison between the H‑Optimus and ResNet50-based models (\hyperref[fig:S2]{Appendix Figure S2 (3)}). Specifically, we use two of the default PanNuke dataset splits (66\%) for training and validation, and reserve the third split (33\%) for testing.

To address the challenge of cell class imbalance in the PanNuke dataset, which is a common characteristic in most cell-level H\&E patch datasets, both models’ training processes employ a weighted loss function comprising cross-entropy and focal loss \cite{Lin_Goyal_etal._2018}. The focal loss component is adjusted with coefficients derived from each cell class' instance frequency, emphasizing learning from underrepresented classes and enhancing the model's sensitivity to rare but significant cellular patterns. The cross-entropy loss is augmented with spectral decoupling regularization \cite{Pezeshki_Kaba_etal._2021,Pohjonen_Stürenberg_etal._2022} and spatially varying label smoothing \cite{Islam_Glocker_2021}, which potentially stabilizes training and improves generalization in case of complex tissue morphologies. For optimization, we employ the \textit{AdamW} \cite{Loshchilov_Hutter_2019} to counter unbalanced class scenarios, with cosine annealing learning rate scheduler.

We utilize the scikit-learn library \cite{Van_der_Walt_Schönberger_etal._2014} and HoVer-Net \cite{Graham_Vu_etal._2019} implementations of $R^2$ (the coefficient of determination) and $PQ$ (panoptic quality) to evaluate our experiments. Complete mathematical formulations and detailed explanations of these metrics are provided in \hyperref[chap:S5]{Appendix S5}. To compute confidence intervals, we use nonparametric bootstrapping, where after calculating the metric on the full sample, we generated 1000 bootstrap replicates by resampling with replacement and then determined the 95\% confidence intervals as the 2.5th and 97.5th percentiles of the resulting empirical distribution.

%\hfill

The model comparisons are summarized in \hyperref[tab:2]{Table 2}. The H‑Optimus-based model achieves higher $R^2$ across all cell classes compared to the ResNet50-based model, which means that its predictions are more closely aligned with the PanNuke cell counts, indicating a stronger correlation with the observed data. Notably, the improvement of $R^2_{dead}$ may be an indicator of better global contextual representations provided by the foundation model backbone. In terms of segmentation and classification quality combined, measured by the PQ score, the H‑Optimus-based model demonstrates notable improvements across most cell classes. Overall, the average $R^2$ improved from 0.575 to 0.871, while the average $PQ$ score improved from 0.450 to 0.492, demonstrating better performance of the H-Optimus-based model.

\begin{table}[h!]
\renewcommand{\arraystretch}{1.5}
  \centering
  \caption{Cell quantification metrics for baseline and proposed models (CI 95\%).}
  \label{tab:2}
  \begin{tabular}{|l|c|c|}
    \hline
    %\rowcolor{gray!30}
    Metric             & Resnet50-based            & H-optimus-based              \\
    \hline
    $R^2_{neoplastic}$    & 0.681 (0.576--0.769)       & \textbf{0.941 (0.917--0.960)} \\
    \hline
    $R^2_{inflammatory}$  & 0.863 (0.778--0.903)       & \textbf{0.949 (0.918--0.966)} \\
    \hline
    $R^2_{connective}$    & 0.600 (0.488--0.698)       & 0.609 (0.436--0.772)          \\
    \hline
    $R^2_{dead}$          & 0.097 (-11.389--0.669)     & 0.925 (0.404--0.982)          \\
    \hline
    $R^2_{epithelial}$    & 0.635 (0.490--0.747)       & \textbf{0.930 (0.886--0.964)} \\
    \hline
    $PQ_{neoplastic}$       & 0.517 (0.499--0.535)       & \textbf{0.589 (0.575--0.604)} \\
    \hline
    $PQ_{inflammatory}$     & 0.455 (0.429--0.482)       & \textbf{0.528 (0.507--0.549)} \\
    \hline
    $PQ_{connective}$       & 0.416 (0.400--0.431)       & \textbf{0.451 (0.436--0.465)} \\
    \hline
    $PQ_{dead}$             & 0.374 (0.342--0.408)       & 0.292 (0.209--0.365)          \\
    \hline
    $PQ_{epithelial}$       & 0.488 (0.460--0.519)       & \textbf{0.599 (0.579--0.618)} \\
    \hline
  \end{tabular}
\end{table}

Our results  show that integrating the H‑Optimus foundation model within the UNETR architecture enhances the model's ability to segment and classify cells across diverse tissues from PanNuke data. The pretrained transformer encoder provides robust feature representations, resulting in higher average $R^2$ and $PQ$ scores compared to the CNN-based model. This leads to more reliable cell quantification and more accurate downstream analysis. Additionally, the streamlined fine-tuning process reduces computational overhead and training time, making the model more adaptable for new data.

Despite these advancements, the foundation model-based approach does not fully resolve all challenges related to cell segmentation and classification. We observe lower metric scores for underrepresented classes in the training data. Furthermore, foundation models typically encompass billions of parameters, resulting in substantial computational and memory requirements. It therefore poses challenges for deployment in resource-constrained environments, limiting their practical applicability in certain clinical settings.

\section{Model optimization via Knowledge Distillation}

To address the limitations posed by the extensive size of foundation models, we implement knowledge distillation — a model compression technique that leverages the teacher-student paradigm \cite{Hinton_Vinyals_etal._2015}. By training a smaller, more efficient student model to replicate the output of a larger, pre-trained teacher model, we retain performance while significantly reducing the model's complexity and resource requirements (\hyperref[fig:fig6]{Figure 6}).

\begin{figure}[h!]
    \centering
    \includegraphics[width=\textwidth, height=0.45\textheight, keepaspectratio]{images/Figure_6.pdf}
    \caption{Knowledge distillation framework for training a student model using a pre-trained teacher}
    \label{fig:fig6}
\end{figure}

We employ knowledge distillation to compress the H‑Optimus-based teacher model into a more efficient student model. The teacher model is the modified UNETR architecture with the H‑Optimus foundation model described in the previous chapter. The student model is based on a UNet architecture augmented with residual connections and incorporates a smaller ViT encoder with 9 million parameters \cite{Steiner_Kolesnikov_etal._2022,Wightman_2019}. 

First, we fine-tune the teacher model using the refined dataset from the cross-relabeling procedure (Section 2). Initially we train the decoder of the teacher model while keeping the encoder weights frozen. We split the refined dataset into train (70\%), validation (20\%) and test (10\%) subsets (\hyperref[fig:S2]{Appendix Figure S2 (4)}). During fine-tuning, we use the train and validation subsets, while leaving the test subset for model evaluation. We set the training procedure and model hyperparameters to be identical to those that were used to demonstrate the utility of foundation models for the simultaneous cell segmentation and classification task.

Next, we perform knowledge distillation from teacher to student using the refined dataset used to fine-tune the teacher model. The student model is trained to replicate the teacher model's outputs. We utilize a specialized loss function that aligns the student's predicted probability distribution with the teacher's, incorporating the teacher's class probability distribution derived from the output. Following the methodology of Hinton et al. \cite{Hinton_Vinyals_etal._2015}, we experiment with various hyperparameter settings for the temperature ($T$) and the balancing coefficients ($\alpha$ and $\beta$) in the loss function. We vary $T$ from 1 to 20 and adjust $\alpha$ and $\beta$ to balance the distillation and student losses. Through iterative tuning and evaluation, we identify that setting $T=14$, $\alpha=0.3$, and $\beta=0.7$ yields a configuration that converges and closely approximates the teacher model's performance during training.

Finally, we assess the performance of both models using the $R^2$ and $PQ$ (defined in \hyperref[chap:S5]{Appendix S5}) on the test set of the refined dataset (\hyperref[tab:3]{Table 3}). We observe that the 95\% confidence intervals overlap for most cell types, so we cannot claim statistically significant performance differences between the teacher and student models. One exception appears in the neoplastic class. The teacher model produces an $R^2$ of 0.919, while the student model shows an $R^2$ of 0.852. In addition, the student model achieves higher $PQ$ values for the neoplastic and connective classes, though the confidence intervals show overlap.

\begin{table}[h!]
\renewcommand{\arraystretch}{1.5}
  \centering
  \caption{Cell quantification metrics for teacher and distilled student models (CI 95\%).}
  \label{tab:3}
  \begin{tabular}{|l|c|c|}
    \hline
    %\rowcolor{gray!30}
    Metric & Teacher & Student \\
    \hline
    $R^2_{neoplastic}$    & \textbf{0.919} (0.898--0.939) & 0.852 (0.800--0.891) \\
    \hline
    $R^2_{lymphocyte}$    & 0.969 (0.956--0.977)         & 0.969 (0.956--0.978) \\
    \hline
    $R^2_{connective}$    & 0.694 (0.548--0.809)         & 0.618 (0.469--0.741) \\
    \hline
    $R^2_{dead}$          & 0.755 (0.400--0.908)         & 0.424 (0.100--0.731) \\
    \hline
    $R^2_{epithelial}$    & 0.922 (0.870--0.958)         & 0.843 (0.738--0.917) \\
    \hline
    $R^2_{macrophage}$    & 0.384 (-0.369--0.724)        & 0.704 (0.352--0.859) \\
    \hline
    $R^2_{neutrofil}$     & 0.854 (0.578--0.929)         & 0.833 (0.502--0.925) \\
    \hline
    $PQ_{neoplastic}$       & 0.581 (0.569--0.593)         & 0.601 (0.588--0.613) \\
    \hline
    $PQ_{lymphocyte}$       & 0.536 (0.520--0.553)         & 0.563 (0.544--0.579) \\
    \hline
    $PQ_{connective}$       & 0.436 (0.421--0.451)         & 0.457 (0.441--0.474) \\
    \hline
    $PQ_{dead}$             & 0.272 (0.235--0.315)         & 0.279 (0.201--0.369) \\
    \hline
    $PQ_{epithelial}$       & 0.522 (0.500--0.545)         & 0.530 (0.506--0.555) \\
    \hline
    $PQ_{macrophage}$       & 0.524 (0.459--0.588)         & 0.474 (0.405--0.543) \\
    \hline
    $PQ_{neutrofil}$        & 0.541 (0.490--0.592)         & 0.565 (0.522--0.607) \\
    \hline
  \end{tabular}
\end{table}


We further decompose the $PQ$ metric into its $SQ$ and $DQ$ components (\hyperref[tab:S6]{Appendix Table S6}). Both models produce nearly identical $SQ$ values, which indicates that they predict instance boundaries with similar precision. Although the student model shows some improvement in $DQ$ scores for certain classes, the confidence intervals overlap and do not confirm a statistically significant difference.

We observe that the student and teacher models yield comparable detection performance despite the student model using a much smaller and simpler architecture. A model with fewer parameters reduces the risk of overfitting when training data are scarce relative to the model’s complexity \cite{Farias_Ludermir_etal._2022}. The knowledge distillation process also encourages the student model to focus on the most generalizable detection features learned from the teacher. These factors enable the student model to achieve similar detection performance across different cell types.

Additionally, considering the model sizes reported in \hyperref[tab:4]{Table 4}, the distilled model achieves a significant reduction compared to the teacher model, with a 48-fold decrease in parameter count and a 5.5-fold reduction in on-disk size. In inference mode, the teacher model requires 16 GB of VRAM for a batch size of 32, while the distilled model only needs 3 GB of VRAM for the same batch size. These reductions make the distilled model significantly more practical for fine-tuning and deployment in resource-constrained environments.

\begin{table}[h!]
\renewcommand{\arraystretch}{1.5}
  \centering
  \caption{Parameter counts and size of teacher and distilled model}
  \label{tab:4}
  \adjustbox{max width=\textwidth}{%
  \begin{tabular}{|l|c|c|c|}
    \hline
    %\rowcolor{gray!30}
    Metric & H-optimus-based (Teacher) & mobileViT-based (Student) & Magnitude of difference \\
    \hline
    Parameters count       & 1,158,917,906   & \textbf{24,093,393}   & \textbf{48x}  \\
    \hline
    Estimated Total Size (MB) & 87,912       & \textbf{15,935}    & \textbf{5.5x} \\
    \hline
  \end{tabular}%
}
\end{table}

%\hfill

With recent advancements in complex network architectures and the use of pretrained encoders to achieve state-of-the-art performance \cite{Baumann_Dislich_etal._2024,Hörst_Rempe_etal._2024} in cell segmentation and classification tasks, model size, computational complexity, and processing times have increased. This limits the scalability and accessibility of these models. As we demonstrate, this may be mitigated using knowledge distillation. Studies in the field of natural language processing have demonstrated the efficacy of knowledge distillation in retaining the capabilities of the teacher model while achieving significant reductions in size and complexity \cite{Huangpu_Gao_2024,Sun_Yu_etal.}. 

We demonstrate the feasibility of knowledge distillation in digital pathology, specifically for cell segmentation and classification tasks. Moreover, we achieve this performance while also significantly reducing the parameter count. In addressing the challenge of knowledge transfer, we found that distillation from a transformer-based model to a smaller transformer is more straightforward than attempting to map transformer features to CNN blocks. In our experiments, using a CNN-based network as a student results in worse cell quantification performance due to the structural constraints of CNN feature space dimensions. 

Although our primary approach relies on a transformer-based student model that performs well, it can be further optimized to incorporate advantages from CNN architectures. For example, employing alternative techniques such as using ViT adapters \cite{Chen_Duan_etal._2023} or $1 \times 1$ convolutions to adjust feature map sizes may be beneficial for harnessing CNN advantages like enhanced local feature extraction. Moreover, if additional performance improvements are desired, the process can be further enhanced by applying supplementary knowledge distillation techniques, such as self-distillation \cite{Zhang_Song_etal._2019} or online distillation \cite{Houyon_Cioppa_etal._2023}.

Despite these promising results, further validation on independent datasets is necessary to fully understand the model's limitations. Underrepresented classes may pose challenges when addressing complex cases. Pathologists need to validate these models to adopt them in clinical settings. While the distilled models are smaller and more deployable, a technological gap persists because pathologists traditionally rely on established methods for inspecting WSIs and diagnosing diseases. Addressing the complexities involved in deploying models for inference and supporting pathologists in adopting new tools is essential for integrating these models into clinical workflows.

\section{Model integration with QuPath}
Digital pathology tools with graphical user interfaces are essential for visualizing and analyzing WSIs. To make our student model useful in clinical pathology workflows, it needs to be integrated into a tool that enables inspecting regions, creating annotations, and providing quantitative analyses of biomarkers. Therefore, we integrate the trained student model from the previous chapter into the QuPath open‑source platform \cite{Bankhead_Loughrey_etal._2017}. QuPath provides the required annotation, visualization, and analysis tools to interpret complex histological data, including workflows for cell segmentation, classification, and quantification (\hyperref[fig:fig7]{Figure 7}). 

\begin{figure}[h!]
    \centering
    \includegraphics[width=\textwidth]{images/Figure_7.pdf}
    \caption{Visualization of model-generated cell quantification annotations (left) and the corresponding unannotated slide (right) in QuPath}
    \label{fig:fig7}
\end{figure}

To identify the regions in a WSI critical for prognosticating tumor development, such as specific tumor areas or border regions without overlapping healthy tissue, the pathologist uses QuPath to outline these regions. Then, the pathologist initiates a cell segmentation and classification script through the QuPath interface for the selected regions. The resulting annotations and quantified cell information are then directly overlaid onto the WSI in the QuPath interface. Additional design and implementation details are in \hyperref[chap:S7]{Appendix S7}. 

Two common approaches for integrating deep learning models into QuPath are Java‑based native QuPath extensions \cite{Goldsborough_Philps_etal._2024} and the execution of RESTful API requests to a model server coupled with handling the response via an extension, as demonstrated in the application of cell segmentation models applied to immunofluorescence images \cite{Sugawara_2023}. While the community is actively working on these integration strategies, there is currently no universal solution that fully addresses all integration and performance requirements.

Extensions may offer better integration with QuPath, allowing slightly improved performance and more widespread usage of the built-in QuPath models, but they lack the flexibility to customize models and modify their behavior. For example, the newest version of QuPath includes models such as StarDist \cite{Weigert_Schmidt} and InstanSeg \cite{Goldsborough_Philps_etal._2024} that can perform cell segmentation. Both models pose limitations when applied to simultaneous cell segmentation and classification. StarDist performs well only on convex, round shapes by design, whereas some neoplastic, inflammatory, and connective cells exhibit complex and non-convex shapes. InstanSeg provides only semantic segmentation without assigning classes to the segmented cells.

%\hfill

In contrast, our approach offers an alternative integration strategy. It utilizes the paquo library to directly interact with QuPath’s internal application programming interface from within Python. This enables data exchange and processing without the need for intermediate conversion steps and provides greater control over model customization, retraining, and the incorporation of custom processing steps.

The integration of our custom model with QuPath underscores its potential to significantly enhance the diagnostic process by reducing the time burden on pathologists and enabling them to focus on more complex interpretative tasks using familiar software. Leveraging a tool that is already well-established among pathologists increases the likelihood of its adoption into daily clinical workflows. The quantitative data generated through the automated workflow is critical for both clinical decision-making and research, facilitating more accurate biomarker analysis, enabling robust statistical evaluations, and supporting hypothesis generation and testing. Additionally, by streamlining cell segmentation and classification, the tool enhances the scalability and reproducibility of pathological assessments, ultimately contributing to improved diagnostic accuracy and patient outcomes.

\section{Conclusion and future work}

In this study, we address critical challenges in digital pathology and tackle the usability and deployment issues of the developed models in standard computing environments without the need for high-performance computing systems. Our multi-faceted approach encompasses data refinement through cross-relabeling, leveraging foundation models for robust cell segmentation and classification, optimizing model performance via knowledge distillation, and integrating the optimized model into the QuPath software for practical application. This approach is used to construct a capable, versatile, and adjustable model for cell segmentation and classification, with enhanced performance and usability.

\begin{sloppypar}
While our approach shows potential in the field of computational pathology, certain limitations persist. 
For example, our implementation currently exhibits lower performance in detecting macrophages. 
This serves as an instance of the broader challenge of accurately identifying complex cell types. In order to address this issue, extending our approach to incorporate additional data sources, exploring alternative modeling approaches, and integrating other imaging modalities such as immunohistochemical staining may help improve detection accuracy. Moreover, although the distilled model reduces computational demands, integrating advanced deep learning models into clinical practice requires addressing technological gaps and potential resistance to adopting new tools within established diagnostic processes.
\end{sloppypar}

Future work could focus on several key areas to refine the proposed approach and facilitate its adoption in clinical environments. Enhancing the cell-relabeling process with additional datasets \cite{Graham_Jahanifar_etal._2021} could improve the representation of underrepresented cell types and enhance overall model performance. Also, incorporating additional data sources, such as multi-modal imaging or complementary staining methods, may address limitations related to cell type differentiation and class imbalance. Exploring other foundation models \cite{Vorontsov_Bozkurt_etal._2024,Zimmermann_Vorontsov_etal._2024} or introducing additional modalities \cite{Ding_Wagner_etal._2024,Vaidya_Zhang_etal._2025} may provide alternative architectures better suited to specific tasks or offer improved efficiency. Implementing more complex knowledge distillation techniques \cite{Houyon_Cioppa_etal._2023,Zhang_Song_etal._2019} could further optimize the model's performance and adaptability. Additionally, deeper integration with QuPath or other digital pathology software could provide pathologists more control over cell quantification analysis directly within the QuPath interface, thereby increasing accessibility and usability. Such enhancements would not only refine model performance but also ensure greater adaptability and scalability within various clinical environments. Finally, extensive validation of the model by pathologists and benchmarking against independent datasets are essential steps toward establishing the model's reliability and fostering confidence in its clinical utility.

\section*{Acknowledgments} 
This work was funded in part by the Research Council of Norway grant no. 309439 SFI Visual Intelligence, and the North Norwegian Health Authority grant no. HNF1521-20.

\bibliographystyle{IEEEtran}
\begin{sloppypar}
\begin{thebibliography}{99}

\bibitem{chaplot2020neural} Chaplot, Devendra Singh, et al. "Neural topological slam for visual navigation." Proceedings of the IEEE/CVF conference on computer vision and pattern recognition. 2020.

\bibitem{maksymets2021thda} Maksymets, Oleksandr, et al. "Thda: Treasure hunt data augmentation for semantic navigation." Proceedings of the IEEE/CVF International Conference on Computer Vision. 2021.

\bibitem{mezghan2022memory} Mezghan, Lina, et al. "Memory-augmented reinforcement learning for image-goal navigation." 2022 IEEE/RSJ International Conference on Intelligent Robots and Systems (IROS). IEEE, 2022.

\bibitem{al2022zero} Al-Halah, Ziad, Santhosh Kumar Ramakrishnan, and Kristen Grauman. "Zero experience required: Plug \& play modular transfer learning for semantic visual navigation." Proceedings of the IEEE/CVF Conference on Computer Vision and Pattern Recognition. 2022.

\bibitem{ye2021auxiliary} Ye, Joel, et al. "Auxiliary tasks and exploration enable objectgoal navigation." Proceedings of the IEEE/CVF international conference on computer vision. 2021.

\bibitem{chaplot2020object} Chaplot, Devendra Singh, et al. "Object goal navigation using goal-oriented semantic exploration." Advances in Neural Information Processing Systems 33 (2020)

\bibitem{ramakrishnan2022poni} Ramakrishnan, Santhosh Kumar, et al. "Poni: Potential functions for objectgoal navigation with interaction-free learning." Proceedings of the IEEE/CVF Conference on Computer Vision and Pattern Recognition. 2022.

\bibitem{ramrakhya2022habitat} Ramrakhya, Ram, et al. "Habitat-web: Learning embodied object-search strategies from human demonstrations at scale." Proceedings of the IEEE/CVF Conference on Computer Vision and Pattern Recognition. 2022.

\bibitem{mousavian2019visual} Mousavian, Arsalan, et al. "Visual representations for semantic target driven navigation." 2019 International Conference on Robotics and Automation (ICRA). IEEE, 2019.

\bibitem{dhariwal2021diffusion} Dhariwal, Prafulla, and Alexander Nichol. "Diffusion models beat gans on image synthesis." Advances in neural information processing systems 34 (2021)

\bibitem{ho2022classifier} Ho, Jonathan, and Tim Salimans. "Classifier-free diffusion guidance." arXiv preprint arXiv:2207.12598 (2022).

\bibitem{nichol2021glide} Nichol, Alex, et al. "Glide: Towards photorealistic image generation and editing with text-guided diffusion models." arXiv preprint arXiv:2112.10741 (2021)

\bibitem{brooks2023instructpix2pix} Brooks, Tim, Aleksander Holynski, and Alexei A. Efros. "Instructpix2pix: Learning to follow image editing instructions." Proceedings of the IEEE/CVF Conference on Computer Vision and Pattern Recognition. 2023.

\bibitem{fu2023guiding} Fu, Tsu-Jui, et al. "Guiding instruction-based image editing via multimodal large language models." arXiv preprint arXiv:2309.17102 (2023).

\bibitem{geng2024instructdiffusion} Geng, Zigang, et al. "Instructdiffusion: A generalist modeling interface for vision tasks." Proceedings of the IEEE/CVF Conference on Computer Vision and Pattern Recognition. 2024.

\bibitem{zhou2024minedreamer} Zhou, Enshen, et al. "Minedreamer: Learning to follow instructions via chain-of-imagination for simulated-world control." arXiv preprint arXiv:2403.12037 (2024).

\bibitem{zhou2023esc} Zhou, Kaiwen, et al. "Esc: Exploration with soft commonsense constraints for zero-shot object navigation." International Conference on Machine Learning. PMLR, 2023.

\bibitem{yu2023l3mvn} Yu, Bangguo, Hamidreza Kasaei, and Ming Cao. "L3mvn: Leveraging large language models for visual target navigation." 2023 IEEE/RSJ International Conference on Intelligent Robots and Systems (IROS). IEEE, 2023.

\bibitem{gadre2023cows} Gadre, Samir Yitzhak, et al. "Cows on pasture: Baselines and benchmarks for language-driven zero-shot object navigation." Proceedings of the IEEE/CVF Conference on Computer Vision and Pattern Recognition. 2023.

\bibitem{shah2023navigation} Shah, Dhruv, et al. "Navigation with large language models: Semantic guesswork as a heuristic for planning." Conference on Robot Learning. PMLR, 2023.

\bibitem{cai2024bridging} Cai, Wenzhe, et al. "Bridging zero-shot object navigation and foundation models through pixel-guided navigation skill." 2024 IEEE International Conference on Robotics and Automation (ICRA). IEEE, 2024.

\bibitem{yu2023co} Yu, Bangguo, Hamidreza Kasaei, and Ming Cao. "Co-NavGPT: Multi-robot cooperative visual semantic navigation using large language models." arXiv preprint arXiv:2310.07937 (2023).

\bibitem{wu2024voronav} Wu, Pengying, et al. "Voronav: Voronoi-based zero-shot object navigation with large language model." arXiv preprint arXiv:2401.02695 (2024).

\bibitem{qin2023mp5} Qin, Yiran, et al. "Mp5: A multi-modal open-ended embodied system in minecraft via active perception." arXiv preprint arXiv:2312.07472 (2023).

\bibitem{du2024learning} Du, Yilun, et al. "Learning universal policies via text-guided video generation." Advances in Neural Information Processing Systems 36 (2024).

\bibitem{ajay2024compositional} Ajay, Anurag, et al. "Compositional foundation models for hierarchical planning." Advances in Neural Information Processing Systems 36 (2024).

\bibitem{liang2024skilldiffuser} Liang, Zhixuan, et al. "Skilldiffuser: Interpretable hierarchical planning via skill abstractions in diffusion-based task execution." Proceedings of the IEEE/CVF Conference on Computer Vision and Pattern Recognition. 2024.

\bibitem{heusel2017gans} Heusel, Martin, et al. "Gans trained by a two time-scale update rule converge to a local nash equilibrium." Advances in neural information processing systems 30 (2017).

\bibitem{zhang2018unreasonable} Zhang, Richard, et al. "The unreasonable effectiveness of deep features as a perceptual metric." Proceedings of the IEEE conference on computer vision and pattern recognition. 2018.

\bibitem{brown2020language} Brown, Tom B. "Language models are few-shot learners." arXiv preprint arXiv:2005.14165 (2020).

\bibitem{podell2023sdxl} Podell, Dustin, et al. "Sdxl: Improving latent diffusion models for high-resolution image synthesis." arXiv preprint arXiv:2307.01952 (2023).

\bibitem{brohan2022rt} Brohan, Anthony, et al. "Rt-1: Robotics transformer for real-world control at scale." arXiv preprint arXiv:2212.06817 (2022).

\bibitem{brohan2023rt} Brohan, Anthony, et al. "Rt-2: Vision-language-action models transfer web knowledge to robotic control." arXiv preprint arXiv:2307.15818 (2023).

\bibitem{li2024manipllm} Li, Xiaoqi, et al. "Manipllm: Embodied multimodal large language model for object-centric robotic manipulation." Proceedings of the IEEE/CVF Conference on Computer Vision and Pattern Recognition. 2024.

\bibitem{shah2023vint} Shah, Dhruv, et al. "ViNT: A foundation model for visual navigation." arXiv preprint arXiv:2306.14846 (2023).

\bibitem{liu2024visual} Liu, Haotian, et al. "Visual instruction tuning." Advances in neural information processing systems 36 (2024).

\bibitem{hu2021lora} Hu, Edward J., et al. "Lora: Low-rank adaptation of large language models." arXiv preprint arXiv:2106.09685 (2021).

\bibitem{qin2023supfusion} Qin, Yiran, et al. "SupFusion: Supervised LiDAR-camera fusion for 3D object detection." Proceedings of the IEEE/CVF International Conference on Computer Vision. 2023.

\bibitem{qin2024worldsimbench} Qin, Yiran, et al. "Worldsimbench: Towards video generation models as world simulators." arXiv preprint arXiv:2410.18072 (2024).

\bibitem{yu2025gamefactory} Yu, Jiwen, et al. "GameFactory: Creating New Games with Generative Interactive Videos." arXiv preprint arXiv:2501.08325 (2025).

\bibitem{zhou2024code} Zhou, Enshen, et al. "Code-as-Monitor: Constraint-aware Visual Programming for Reactive and Proactive Robotic Failure Detection." arXiv preprint arXiv:2412.04455 (2024).

\bibitem{zhang2024ad} Zhang, Zaibin, et al. "AD-H: Autonomous Driving with Hierarchical Agents." arXiv preprint arXiv:2406.03474 (2024).

\bibitem{wang2024toward} Wang, Chaoqun, et al. "Toward Accurate Camera-based 3D Object Detection via Cascade Depth Estimation and Calibration." arXiv preprint arXiv:2402.04883 (2024).

\bibitem{huang2024story3d} Huang, Yuzhou, et al. "Story3d-agent: Exploring 3d storytelling visualization with large language models." arXiv preprint arXiv:2408.11801 (2024).

\bibitem{savinov2018semi} Savinov, Nikolay, Alexey Dosovitskiy, and Vladlen Koltun. "Semi-parametric topological memory for navigation." arXiv preprint arXiv:1803.00653 (2018).

\bibitem{majumdar2022zson} Majumdar, Arjun, et al. "Zson: Zero-shot object-goal navigation using multimodal goal embeddings." Advances in Neural Information Processing Systems 35 (2022): 32340-32352.

\bibitem{yadav2023offline} Yadav, Karmesh, et al. "Offline visual representation learning for embodied navigation." Workshop on Reincarnating Reinforcement Learning at ICLR 2023. 2023.

\bibitem{yadav2023ovrl} Yadav, Karmesh, et al. "Ovrl-v2: A simple state-of-art baseline for imagenav and objectnav." arXiv preprint arXiv:2303.07798 (2023).

\bibitem{sun2024fgprompt} Sun, Xinyu, et al. "FGPrompt: fine-grained goal prompting for image-goal navigation." Advances in Neural Information Processing Systems 36 (2024).

\bibitem{zhu2017target} Zhu, Yuke, et al. "Target-driven visual navigation in indoor scenes using deep reinforcement learning." 2017 IEEE international conference on robotics and automation (ICRA). IEEE, 2017.

\bibitem{koh2024generating} Koh, Jing Yu, Daniel Fried, and Russ R. Salakhutdinov. "Generating images with multimodal language models." Advances in Neural Information Processing Systems 36 (2024).

\bibitem{krantz2022instance} Krantz, Jacob, et al. "Instance-specific image goal navigation: Training embodied agents to find object instances." arXiv preprint arXiv:2211.15876 (2022).

\bibitem{schulman2017proximal} Schulman, John, et al. "Proximal policy optimization algorithms." arXiv preprint arXiv:1707.06347 (2017).

\bibitem{anderson2018evaluation} Anderson, Peter, et al. "On evaluation of embodied navigation agents." arXiv preprint arXiv:1807.06757 (2018).

\bibitem{lin2024navcot} Lin, Bingqian, et al. "NavCoT: Boosting LLM-Based Vision-and-Language Navigation via Learning Disentangled Reasoning." arXiv preprint arXiv:2403.07376 (2024).

\bibitem{NavGPT} Zhou, Gengze, Yicong Hong, and Qi Wu. "Navgpt: Explicit reasoning in vision-and-language navigation with large language models." Proceedings of the AAAI Conference on Artificial Intelligence.

\bibitem{hahn2021no} Hahn, Meera, et al. "No rl, no simulation: Learning to navigate without navigating." Advances in Neural Information Processing Systems 34 (2021): 26661-26673.

\bibitem{li2025t2isafety} Li, Lijun, et al. "T2ISafety: Benchmark for Assessing Fairness, Toxicity, and Privacy in Image Generation." arXiv preprint arXiv:2501.12612 (2025).

\bibitem{an2024agfsync} An, Jingkun, et al. "AGFSync: Leveraging AI-Generated Feedback for Preference Optimization in Text-to-Image Generation." arXiv preprint arXiv:2403.13352 (2024).


\end{thebibliography}
\end{sloppypar}

\clearpage
\beginsupplement
\section*{Appendix}
\renewcommand{\thesubsection}{S\arabic{subsection}}

\subsection{\label{chap:S1}PanNuke and MoNuSAC preprocessing}
The PanNuke dataset comprises a set of 7,901 RGB patches, each with dimensions of $256 \times 256$ pixels, which we set as the standard patch size for our analysis. In contrast, the MoNuSAC dataset encompasses 294 images of heterogeneous dimensions. To standardize the MoNuSAC images with our experiments, we implement a standardization protocol. Specifically, for images exceeding the dimensions of $256 \times 256$ pixels, we segment them into equal-sized patches and apply mirror padding to the remaining portions to avoid information loss at the peripherals. Patches with dimensions less than $128 \times 128$ pixels are excluded from the dataset due to the insufficient resolution to capture relevant cellular details. For patches where either dimension falls between 128 and 256 pixels, we employ upsampling to achieve the standard patch size. As a result, we obtain a total of 2,823 RGB patches derived from the MoNuSAC dataset for subsequent analysis. For additional details on the MoNuSAC data preparation process, refer to the source code \cite{Shvetsov_2025a}.
\clearpage

\subsection{\label{chap:S2}Data usage for the methodology}

\counterwithin{figure}{subsection}
\renewcommand{\thefigure}{S\arabic{subsection}}

\begin{figure}[h!]
    \centering
    \includegraphics[width=\textwidth, height=0.85\textheight, keepaspectratio]{images/A2.pdf}
    \caption{Overview of the methodology for cross-labeling, dataset refinement, and model comparison. (1) Cross-relabeling - training and testing cell classification models, (2) Cross-relabeling - using cell classification models to create refined dataset, (3) Fine-tuning and training models for comparison, (4) Student knowledge distillation with refined dataset}
    \label{fig:S2}
\end{figure}
\clearpage

\subsection{\label{chap:S3}Confusion matrices for classification models}
\counterwithin{figure}{subsection}
\renewcommand{\thefigure}{S\arabic{subsection}.\arabic{figure}}

\begin{figure}[h!]
    \centering
    \includegraphics[width=\textwidth, height=0.4\textheight, keepaspectratio]{images/A3_1.pdf}
    \caption{Confusion matrix for PanNuke trained model}
    \label{fig:S3.1}
\end{figure}

\begin{figure}[h!]
    \centering
    \includegraphics[width=\textwidth, height=0.4\textheight, keepaspectratio]{images/A3_2.pdf}
    \caption{Confusion matrix for MoNuSAC trained model}
    \label{fig:S3.2}
\end{figure}

\clearpage

\subsection{\label{chap:S4}Datasets cell counts}

\counterwithin{table}{subsection}
\renewcommand{\thetable}{S\arabic{subsection}}

\begin{table}[h!]
\renewcommand{\arraystretch}{2.0}
\centering
\caption{\label{tab:S4}Cell counts for PanNuke, MoNuSAC and refined datasets. Numbers in parentheses indicate preprocessed cell counts for cell classifier models training and testing.}
%\adjustbox{max width=\textwidth}{%
\begin{tabular}{|l|c|c|c|}
\hline
%\rowcolor{gray!30}
Cell type & PanNuke & MoNuSAC & Refined \\
\hline
Neoplastic & 77,403 (68,031) & - & 105,451 \\
\hline
Epithelial & 26,572 (23,207) & - & 29,926 \\
\hline
Epithelial (benign and malignant) & - & 31,402 & - \\
\hline
Inflammatory & 32,276 & - & - \\
\hline
Lymphocytes & - & 37,045 (33,104) & 65,275 \\
\hline
Neutrophils & - & 1,355 (1,252) & 3,833 \\
\hline
Macrophage & - & 1,842 (1,695) & 3,410 \\
\hline
Dead & 2,908 & - & 2,908 \\
\hline
Connective & 50,585 & - & 50,585 \\
\hline
\end{tabular}
%
%}
\end{table}



\clearpage

\subsection{\label{chap:S5}Definition of validation metrics}
\counterwithin{equation}{subsection}
\renewcommand{\theequation}{\arabic{equation}}

\subsubsection{\label{chap:S5.1}R\textsuperscript{2}}
The coefficient of determination, denoted as $R^2$, is a statistical measure that represents the proportion of variance in the dependent variable that is predictable from the independent variables. In the context of cell quantification in pathology, $R^2$ is used to assess how well the predicted quantities of different cell types in a patch align with the actual quantities observed in the ground truth data, with higher values representing more accurate quantification. $R^2$ is defined as
\begin{equation*}
R^2 = 1 - \frac{\sum_{i=1}^n (y_i - \hat{y}_i)^2}{\sum_{i=1}^n (y_i - \bar{y})^2},
\end{equation*}
where $y_i$ represents the actual number of cells of a specific type in the $i$-th image, $\hat{y}_i$ represents the predicted number of cells of that type in the $i$-th image, $\bar{y}$ is the mean of the actual numbers across all images, and $n$ is the total number of images in the dataset.

The $R^2$ metric has a range of $(-\infty, 1]$. An $R^2$ of 1 indicates perfect prediction, where all predicted values exactly match the actual values. An $R^2$ of 0 suggests that the model explains none of the variability of the response data around its mean. If $R^2$ is negative, it indicates that the model performs worse than a model that simply predicts the mean of the actual values for all observations.

\subsubsection{\label{chap:S5.2}PQ}
Panoptic Quality ($PQ$) is a comprehensive metric used to evaluate the performance of segmentation models in tasks that require both instance segmentation and classification. $PQ$ provides a single score that encapsulates both the detection accuracy (i.e., how many objects were correctly identified) and the segmentation quality (i.e., how accurately the objects' boundaries were delineated). This metric is particularly useful in multiclass scenarios where each pixel is classified into distinct categories, such as different cell types in pathology images.

$PQ$ is calculated as the product of two terms: Detection Quality ($DQ$) and Segmentation Quality ($SQ$). It can be expressed as
\begin{equation*}
PQ = DQ \cdot SQ,
\end{equation*}
where
\begin{equation*}
DQ = \frac{TP}{TP + 0.5\, FP + 0.5\, FN},
\end{equation*}
\begin{equation*}
SQ = \frac{\sum_{(p, g) \in \mathcal{M}} IoU(p, g)}{TP}.
\end{equation*}
In these formulas, $TP$ denotes the number of correctly matched instances between ground truth and prediction, $FP$ denotes the predicted instances that have no corresponding ground truth, $FN$ denotes the ground truth instances that were not detected, $IoU(p, g)$ is the Intersection over Union for a pair of matched instances $p$ (prediction) and $g$ (ground truth), and $\mathcal{M}$ is the set of matched pairs.

The $PQ$ metric is calculated for each class and is averaged across classes to provide a global performance measure.

The $PQ$ score has a range of $[0, 1.0]$, where a higher score indicates better performance in both detecting and segmenting the instances correctly. A $PQ$ of 1 signifies perfect identification and segmentation of all instances, whereas a $PQ$ of 0 indicates that no instances were correctly identified and segmented.

\clearpage

\subsection{\label{chap:S6}Segmentation and Detection quality metrics for teacher and student models}

\begin{table}[h!]
\renewcommand{\arraystretch}{2.0}
\centering
\caption{Segmentation and detection quality for student and teacher models (CI 95\%)}
\label{tab:S6}
%\adjustbox{max width=\textwidth}{%
\begin{tabular}{|l|c|c|}
\hline
%\rowcolor{gray!30}
Metric & Teacher & Student \\
\hline
$SQ_{neoplastic}$ & 0.819 (0.815--0.823) & 0.824 (0.819--0.828) \\
\hline
$SQ_{lymphocyte}$ & 0.795 (0.788--0.802) & 0.790 (0.783--0.796) \\
\hline
$SQ_{connective}$ & 0.770 (0.762--0.776) & 0.780 (0.772--0.786) \\
\hline
$SQ_{dead}$ & 0.659 (0.623--0.688) & 0.657 (0.624--0.695) \\
\hline
$SQ_{epithelial}$ & 0.780 (0.770--0.790) & 0.788 (0.779--0.797) \\
\hline
$SQ_{macrophage}$ & 0.788 (0.760--0.810) & 0.757 (0.730--0.783) \\
\hline
$SQ_{neutrofil}$ & 0.782 (0.761--0.801) & 0.775 (0.759--0.792) \\
\hline
$DQ_{neoplastic}$ & 0.706 (0.692--0.719) & 0.727 (0.712--0.741) \\
\hline
$DQ_{lymphocyte}$ & 0.675 (0.656--0.698) & 0.713 (0.691--0.734) \\
\hline
$DQ_{connective}$ & 0.566 (0.546--0.584) & 0.583 (0.565--0.602) \\
\hline
$DQ_{dead}$ & 0.410 (0.361--0.465) & 0.435 (0.306--0.561) \\
\hline
$DQ_{epithelial}$ & 0.668 (0.639--0.694) & 0.673 (0.644--0.702) \\
\hline
$DQ_{macrophage}$ & 0.657 (0.583--0.727) & 0.615 (0.531--0.703) \\
\hline
$DQ_{neutrofil}$ & 0.691 (0.625--0.753) & 0.729 (0.679--0.778) \\
\hline
\end{tabular}
%
%}
\end{table}

\clearpage

\subsection{\label{chap:S7}QuPath integration method}
We adopt an integration strategy leveraging the paquo \cite{Bayer_AG} library, a Python package that enables direct interaction with QuPath’s internal API, thereby facilitating seamless data exchange without intermediate conversion steps. The data processing pipeline (\hyperref[fig:S7]{Appendix Figure S7}) begins with the acquisition of WSIs and their associated annotations from QuPath, which are represented as Shapely \cite{Gillies_Wel_etal._2024} polygons. Utilizing paquo, we directly read, create, and modify these annotations and detections within a QuPath project in the Python environment. Images are then cropped using these polygons and processed by cell segmentation and classification models employing standard vision processing toolkits such as OpenCV, pyvips, and PyTorch. Additionally, QuPath employs Groovy scripts to initiate a Python process that starts the entire pipeline from QuPath graphical interface: fetching polygons, extracting images from them, and running deep learning model inference on the cropped images. 
The results are returned to QuPath, leveraging paquo's Python bindings to manipulate QuPath data while minimizing the computational overhead typically associated with cross-environment communication.

\counterwithin{figure}{subsection}
\renewcommand{\thefigure}{S\arabic{subsection}}

\begin{figure}[h!]
    \centering
    \includegraphics[width=\textwidth]{images/A7.pdf}
    \caption{QuPath integration workflow using Python environment}
    \label{fig:S7}
\end{figure}

Compared to traditional workflows that involve exporting annotations as GeoJSON, classifying them in Python, and reimporting them into QuPath, our approach offers several advantages. We eliminate the need to switch between programming languages, providing a cohesive and streamlined development process entirely within QuPath software and removing the necessity to use other tools. Meanwhile, we avoid storing annotations as intermediate JSON files unless required for external use or archiving. By conducting the entire inference and post-processing workflow within the Python environment, we leverage the power and flexibility of Python libraries for image processing and machine learning. This approach also enables adjustments to any set of labels and models, thereby improving its applicability.

%\hfill

The distilled model and QuPath integration code are packaged into a Docker container, enabling streamlined execution with the Docker engine. Detailed integration code and deployment instructions can be found in the GitHub repository \cite{Shvetsov_2025b}.

Despite these benefits, we acknowledge that the paquo library is a proof‑of‑concept project in its early development stage and has not been tested across all versions of QuPath.

\clearpage

\subsection{\label{chap:S8}Data and code availability statement}
All datasets, models, and code used in this study are publicly available and can be obtained from the repositories listed below. 
The PanNuke \cite{Gamper_Koohbanani_etal._2019} and MoNuSAC \cite{Verma_Kumar_etal._2021} datasets are publicly accessible, and download information along with detailed descriptions can be found in their respective articles. Preprocessing scripts for PanNuke and MoNuSAC data, as well as individual cell extraction scripts, are available on GitHub \cite{Shvetsov_2025a}. The H-Optimus foundation model used in our experiments can be downloaded from the HuggingFace repository \cite{hoptimus2024}, and model information is available on GitHub \cite{Saillard_Jenatton_etal._2024}. In addition, the integration code for QuPath and the distilled model packaged in a Docker container are provided in the repository \cite{Shvetsov_2025b}, and paquo Python library is available from the authors GitHub repository \cite{Bayer_AG}.
\clearpage

\end{document}


\clearpage
\appendix


\section{Training details}
\label{app:training}

NeoBERT was trained on 8 H100 for 1,050,000 steps, for a total of 6,000 GPU hours. In the first stage of training, we used a local batch size of 32, 8 gradient accumulation steps, and a maximum sequence length of $1,024$, for a total batch size of 2M tokens. In the second stage of training, we keep the theoretical batch size constant and increase the maximum sequence length to $4,096$.

\section{Ablations}
\label{app:ablations}

Our first model, $M0$ is modeled after BERT$_{base}$ in terms of architecture. The only two differences are the absence of the next-sentence-prediction objective, as well as Pre-Layer Normalization. Each successive model, up until $M8$ is identical to the previous one on every point except for the change reported in \autoref{tab:ablations}.

\section{GLUE}
\label{app:glue}

We perform a classical parameter search with learning rates in $\{5e-6, 6e-6, 1e-5, 2e-5, 3e-5\}$, batch sizes in $\{4, 8, 16, 32\}$ and weight decay in $\{1e-2, 1e-5\}$. In addition, we start training from the best MNLI checkpoint for RTE, STS, MRPC, and QNLI.

We fine-tune on the training splits of every glue dataset for 10 epochs, with evaluation on the validation splits every $n$ steps, $n$ being defined as $\min(500, \text{len(dataloader) // } 10)$ with early stopping after 15 evaluation cycles if scores have not improved.

Following BERT, we exclude WNLI from our evaluation\footnote{See 12 in https://gluebenchmark.com/faq}. For tasks with two scores and for MNLI matched and mismatched, we report the average between the two metrics.

\begin{table*}[!ht]
	\centering
	\setlength{\tabcolsep}{8pt} % Adjust column spacing
	\renewcommand{\arraystretch}{1.2} % Adjust row spacing
	\begin{tabular}{l|c|c|c|c}
		\toprule
		\textbf{Model} & \textbf{Task} & \textbf{Batch Size} & \textbf{Learning Rate} & \textbf{Weight Decay} \\
		\midrule
		\multirow{8}{*}{NeoBERT$_{1024}$} 
		               & CoLA          & 4                   & 6e-6                   & 1e-5                  \\
		               & MNLI          & 16                  & 6e-6                   & 1e-2                  \\
		               & MRPC          & 8                   & 2e-5                   & 1e-5                  \\
		               & QNLI          & 8                   & 5e-6                   & 1e-5                  \\
		               & QQP           & 32                  & 1e-5                   & 1e-2                  \\
		               & RTE           & 8                   & 6e-6                   & 1e-5                  \\
		               & SST-2         & 16                  & 1e-5                   & 1e-5                  \\
		               & STS-B         & 8                   & 1e-5                   & 1e-2                  \\
		\midrule
		\multirow{8}{*}{NeoBERT$_{4096}$} 
		               & CoLA          & 8                   & 8e-6                   & 1e-5                  \\
		               & MNLI          & 16                  & 5e-6                   & 1e-5                  \\
		               & MRPC          & 2                   & 1e-5                   & 1e-5                  \\
		               & QNLI          & 8                   & 5e-6                   & 1e-5                  \\
		               & QQP           & 32                  & 8e-6                   & 1e-5                  \\
		               & RTE           & 32                  & 5e-6                   & 1e-5                  \\
		               & SST-2         & 32                  & 8e-6                   & 1e-2                  \\
		               & STS-B         & 32                  & 2e-5                   & 1e-5                  \\
		\bottomrule
	\end{tabular}
	\caption{Optimal hyperparameters for GLUE tasks. The grid search was conducted over batch sizes $\{2, 4, 8, 16, 32\}$, learning rates $\{5e-6, 6e-6, 8e-6, 1e-5, 2e-5, 3e-5\}$, and weight decay values $\{1e-2, 1e-5\}$.}
	\label{tab:glue_hp}
\end{table*}

\section{MTEB}
\label{app:contrastive}

\subsection{Evaluation of pre-trained models}

As demonstrated in \autoref{fig:mteb-pre-trained}, evaluating out-of-the-box pre-trained models on MTEB is inconclusive. In that setting, BERT$_{base}$ outperforms both BERT$_{large}$ and RoBERTa$_{large}$, highlighting the importance of fine-tuning to ensure representative evaluation on the MTEB benchmark.

\begin{figure}[!htb]
	\caption{Zero-shot evaluation of BERT and RoBERTa on the English subset of MTEB.}
	\centering
	\includegraphics[width=0.6\linewidth]{figures/mteb_pretrained.png}
	\label{fig:mteb-pre-trained}
\end{figure}

\subsection{Contrastive learning}

Following the existing literature, we designed a simple fine-tuning strategy entirely agnostic to the models evaluated. We used cosine similarity and $\tau = 0.07$ as a temperature parameter in the contrastive learning loss. Additionally, we sampled datasets with a multinomial distribution based on their sizes $(n_j)_{j=1}^m$ with $\alpha=0.5$:

\[
	\pi = \frac{n_i^{\alpha}}{\sum_{j=1}^{m} n^{\alpha}_j}
\]

We trained on the following fully-open datasets: AG-News~\citep{zhang2016characterlevelconvolutionalnetworkstext}, All-NLI~\citep{bowman2015largeannotatedcorpuslearning, williams2018broadcoveragechallengecorpussentence}, AmazonQA~\citep{gupta2019amazonqareviewbasedquestionanswering}, ConcurrentQA~\citep{arora2022reasoningpublicprivatedata}, GitHub Issues
\citep{li2023angle}, GooAQ~\citep{khashabi2021gooaqopenquestionanswering}, MedMCQA~\citep{pal2022medmcqalargescalemultisubject}, NPR\footnote{\url{https://huggingface.co/datasets/sentence-transformers/npr}}, 
PudMedQA~\citep{jin2019pubmedqadatasetbiomedicalresearch}, SentenceCompression~\citep{filippova-altun-2013-overcoming} StackExchange\footnote{\url{https://huggingface.co/datasets/sentence-transformers/stackexchange-duplicates}}, TriviaQA~\citep{han2019episodicmemoryreaderlearning}, Wikihow~\citep{koupaee2018wikihowlargescaletext}, Yahoo! Answers~\citep{zhang2016characterlevelconvolutionalnetworkstext} as well as the available training splits of MTEB datasets (StackOverFlowDupQuestion, Fever~\citep{thorne2018feverlargescaledatasetfact}, MS MARCO~\citep{bajaj2018msmarcohumangenerated}, STS12, and STSBenchmark~\citep{Cer_2017}).

We fine-tune every model for 2,000 steps and evaluate on MTEB in float16. The complete results are presented in \autoref{tab:mteb_full}.


\begin{figure*}[!htb]
	\caption{Average MTEB scores of fine-tuned encoders grouped by task type. The average score is computed across the 56 tasks of MTEB-English. NeoBERT is the best model on five out of seven task types and the best model overall. See \autoref{tab:mteb_full} for complete scores.}
	\label{fig:mteb-results}
	\centering
	\includegraphics[width=1\linewidth]{figures/mteb.pdf}
\end{figure*}

\subsection{Task instructions}

We provide the set of instructions used for fine-tuning in \autoref{tab:finetuning_instructions} and evaluation in \autoref{tab:mteb_instruct2} and \autoref{tab:mteb_instruct1}.

\begin{table*}
	\setlength{\tabcolsep}{1.5pt}
	\centering
	\caption{Instructions for fine-tuning on the different contrastive learning datasets.}
	\label{tab:finetuning_instructions}
	\begin{tabular}{ll}
		\toprule
		Dataset           & Instruction                                                                        \\ \midrule
		AGNEWS            & Given a news title, retrieve relevant articles.                                    \\
		ALLNLI            & Given a premise, retrieve a hypothesis that is entailed by the premise.            \\
		AMAZONQA          & Given a question, retrieve Amazon posts that answer the question.                  \\
		CONCURRENTQA      & Given a multi-hop question, retrieve documents that can help answer the            \\
		                  & question.                                                                          \\
		FEVER             & Given a claim, retrieve documents that support or refute the claim.                \\
		GITHUBISSUE       & Given a question, retrieve questions from Github that are duplicates to the given  \\
		                  & question.                                                                          \\
		GOOAQ             & Given a question, retrieve relevant documents that best answer the question.       \\
		MEDMCQA           & Given a medical question, retrieve relevant passages that answer the question.     \\
		MEDMCQA$_{CLUST}$ & Identify the main category of medical exams based on their questions               \\
		                  & and answers.                                                                       \\
		MSMARCO           & Given a web search query, retrieve relevant passages that answer the query.        \\
		NPR               & Given a news title, retrieve relevant articles.                                    \\
		PAQ               & Given a question, retrieve Wikipedia passages that answer the question.            \\
		PUBMEDQA          & Given a medical question, retrieve documents that best answer the question.        \\
		QQP               & Given a question, retrieve questions from Quora forum that are semantically        \\
		                  & equivalent to the given question.                                                  \\
		SENTENCECOMP      & Given a sentence, retrieve semantically equivalent summaries.                      \\
		STACKEXCHANGE     & Given a Stack Exchange post, retrieve posts that are duplicates to the given post. \\
		STACKOVERFLOW     & Retrieve duplicate questions from StackOverflow forum.                             \\
		STS12             & Retrieve semantically similar text.                                                \\
		STSBENCHMARK      & Retrieve semantically similar text.                                                \\
		TRIVIAQA          & Given a question, retrieve documents that answer the question.                     \\
		WIKIHOW           & Given a Wikihow post, retrieve titles that best summarize the post.                \\
		YAHOO$_{CLUST}$   & Identify the main topic of Yahoo posts based on their titles and answers.          \\ \bottomrule
	\end{tabular}
\end{table*}

\begin{table*}
	\setlength{\tabcolsep}{2pt}
	\centering
	\caption{Instructions for evaluation on the different MTEB tasks.}
	\label{tab:mteb_instruct2}
	\begin{tabular}{ll}
		\toprule
		Task name      & Instruction                                                                       \\ \midrule
		DBPedia        & Given a query, retrieve relevant entity descriptions from DBPedia.                \\
		FEVER          & Given a claim, retrieve documents that support or refute the claim.               \\
		FiQA2018       & Given a financial question, retrieve user replies that best answer the question.  \\
		HotpotQA       & Given a multi-hop question, retrieve documents that can help answer the question. \\
		MSMARCO        & Given a web search query, retrieve relevant passages that answer the query.       \\
		NFCorpus       & Given a question, retrieve relevant documents that best answer the question.      \\
		NQ             & Given a question, retrieve Wikipedia passages that answer the question.           \\
		QuoraRetrieval & Given a question, retrieve questions that are semantically equivalent to the      \\
		               & given question.                                                                   \\
		SCIDOCS        & Given a scientific paper title, retrieve paper abstracts that are cited by the    \\
		               & given paper.                                                                      \\
		SciFact        & Given a scientific claim, retrieve documents that support or refute the claim .   \\
		Touche2020     & Given a question, retrieve detailed and persuasive arguments that answer          \\
		               & the question.                                                                     \\
		TRECCOVID      & Given a query on COVID-19, retrieve documents that answer the query.              \\
		SICK-R         & Retrieve semantically similar text.                                               \\
		STS            & Retrieve semantically similar text.                                               \\
		BIOSSES        & Retrieve semantically similar text from the biomedical field.                     \\
		SummEval       & Given a news summary, retrieve other semantically similar summaries.              \\
		\bottomrule
	\end{tabular}
\end{table*}

\begin{table*}
	\setlength{\tabcolsep}{1.5pt}
	\centering
	\caption{Instructions for evaluation on the different MTEB tasks.}
	\label{tab:mteb_instruct1}
	\begin{tabular}{ll}
		\toprule
		Task name                      & Instruction                                                            \\ \midrule
		AmazonCounterfactualClass.     & Given an Amazon customer review, classify it as either counterfactual  \\
		                               & or not-counterfactual.                                                 \\
		AmazonPolarityClass.           & Given an Amazon review, classify its main sentiment into positive      \\
		                               & or negative.                                                           \\
		AmazonReviewsClass.            & Given an Amazon review, classify it into its appropriate rating        \\
		                               & category.                                                              \\
		Banking77Class.                & Given a online banking query, find the corresponding intents.          \\
		EmotionClass.                  & Given a Twitter message, classify the emotion expressed into one of    \\
		                               & the six emotions: anger, fear, joy, love, sadness, and surprise.       \\
		ImdbClass.                     & Given an IMDB movie review, classify its sentiment into positive or    \\
		                               & negative.                                                              \\
		MassiveIntentClass.            & Given a user utterance, find the user intents.                         \\
		MassiveScenarioClass.          & Given a user utterance, find the user scenarios.                       \\
		MTOPDomainClass.               & Given a user utterance, classify the domain in task-oriented           \\
		                               & conversation.                                                          \\
		MTOPIntentClass.               & Given a user utterance, classify the intent in task-oriented           \\
		                               & conversation.                                                          \\
		ToxicConversationsClass.       & Given comments, classify them as either toxic or not toxic.            \\
		TweetSentimentExtractionClass. & Given a tweet, classify its sentiment as either positive, negative, or \\
		                               & neutral.                                                               \\
		<dataset>ClusteringP2P         & Identify the main and secondary category of <dataset> papers based on  \\
		                               & their titles and abstracts.                                            \\
		<dataset>ClusteringS2S         & Identify the main and secondary category of <dataset> papers based on  \\
		                               & their titles.                                                          \\
		<dataset>Clustering            & Identify the topic or theme of <dataset> posts based on their titles.  \\
		TwentyNewsgroupsClustering     & Identify the topic or theme of the given news articles.                \\
		SprintDuplicateQuestions       & Retrieve duplicate questions from Sprint forum.                        \\
		TwitterSemEval2015             & Given a tweet, retrieve tweets that are semantically similar.          \\
		TwitterURLCorpus               & Given a tweet, retrieve tweets that are semantically similar.          \\
		AskUbuntuDupQuestions          & Retrieve duplicate questions from AskUbuntu forum.                     \\
		MindSmallReranking             & Given a user browsing history, retrieve relevant news articles.        \\
		SciDocsRR                      & Given a title of a scientific paper, retrieve the                      \\
		                               & relevant papers.                                                       \\
		StackOverflowDupQuestions      & Retrieve duplicate questions from StackOverflow forum.                 \\
		ArguAna                        & Given a claim, find documents that refute the claim. Document          \\
		ClimateFEVER                   & Given a claim about climate change, retrieve documents that support    \\
		                               & or refute the claim.                                                   \\
		CQADupstackRetrieval           & Given a question, retrieve detailed question descriptions from         \\
		                               & Stackexchange that are duplicates to the given question.               \\
		\bottomrule
	\end{tabular}
\end{table*}

\section{Efficiency}
\label{app:efficiency}

\autoref{tab:efficiency} presents the complete results of model efficiency evaluations.

\begin{table*}[!ht]
	\centering
	\caption{Throughput ($10^3$ tokens / second) in function of the sequence length, with optimal batch size.}
	\begin{tabular}{clrrrrr}
		\toprule
		\textbf{Size}                     & \textbf{Model}       & \textbf{512}            & \textbf{1024}           & \textbf{2048}           & \textbf{4096}           & \textbf{8192}           \\
		\midrule
		\multirow{3}{*}{\makecell{Base}}  & BERT$_{base}$        & $\textbf{27.6} \pm 3.6$ & -                       & -                       & -                       & -                       \\
		                                  & RoBERTa$_{base}$     & $24.9 \pm 3.0$          & -                       & -                       & -                       & -                       \\
		                                  & ModernBERT$_{base}$  & $25.4 \pm 2.3$          & $\textbf{22.6} \pm 2.7$ & $17.2 \pm 1.7$          & $11.7 \pm 0.8$          & $6.8 \pm 0.2$           \\
		\midrule
		Medium                            & NeoBERT              & $24.5 \pm 1.4$          & $\textbf{22.2} \pm 1.7$ & $\textbf{20.5} \pm 1.6$ & $\textbf{17.2} \pm 1.2$ & $\textbf{13.0} \pm 0.2$ \\
		\midrule
		\multirow{3}{*}{\makecell{Large}} & BERT$_{large}$       & $19.5 \pm 0.6$          & -                       & -                       & -                       & -                       \\
		                                  & RoBERTa$_{large}$    & $15.9 \pm 0.3$          & -                       & -                       & -                       & -                       \\
		                                  & ModernBERT$_{large}$ & $13.4 \pm 0.2$          & $11.4 \pm 1.1$          & $9.2 \pm 0.7$           & $6.5 \pm 0.3$           & $3.8 \pm 0.1$           \\
		\bottomrule
	\end{tabular}
	\label{tab:efficiency}
\end{table*}

\end{document}