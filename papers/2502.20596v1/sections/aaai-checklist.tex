\begin{enumerate}

\item \textbf{This paper}:
    \item[] \textbf{Question:} Includes a conceptual outline and/or pseudocode description of AI methods introduced
    \item[] Answer: [YES]
    
    \item[] \textbf{Question:} Clearly delineates statements that are opinions, hypothesis, and speculation from objective facts and results (yes/no)
    Provides well marked pedagogical references for less-familiare readers to gain background necessary to replicate the paper (yes/no)
    \item[] Answer: [YES]
    
\item \textbf{Does this paper make theoretical contributions?}
[NO]

\item \textbf{Does this paper rely on one or more datasets?}
[YES]

    \item[] \textbf{Question:} A motivation is given for why the experiments are conducted on the selected datasets (yes/partial/no/NA)
    \item[] Answer: [NO]. We would like to include the reason here: we follow the previous baselines in terms of benchmarking datasets, in order to conduct fair comparisons.
    
    \item[] \textbf{Question:} All novel datasets introduced in this paper are included in a data appendix. (yes/partial/no/NA)
    \item[] Answer: [N/A]
    
    \item[] \textbf{Question:} All novel datasets introduced in this paper will be made publicly available upon publication of the paper with a license that allows free usage for research purposes. (yes/partial/no/NA)
    \item[] Answer: [N/A]
    
    \item[] \textbf{Question:} All datasets drawn from the existing literature (potentially including authors' own previously published work) are accompanied by appropriate citations. (yes/no/NA)
    \item[] Answer: [YES]
    
    \item[] \textbf{Question:} All datasets drawn from the existing literature (potentially including authors' own previously published work) are publicly available. (yes/partial/no/NA)
    \item[] Answer: [YES]
    
    \item[] \textbf{Question:} All datasets that are not publicly available are described in detail, with explanation why publicly available alternatives are not scientifically satisficing. (yes/partial/no/NA)
    \item[] Answer: [N/A]

\item \textbf{Does this paper include computational experiments?}
[YES]

    \item[] \textbf{Question:} Any code required for pre-processing data is included in the appendix. (yes/partial/no).
    \item[] Answer: [YES]. The source code for our experiments, along with detailed specifications for all external libraries utilized, has been submitted via the AAAI submission system. This provision ensures that researchers attempting to replicate our findings have comprehensive access to the necessary code base and dependencies.

    \item[] \textbf{Question:} All source code required for conducting and analyzing the experiments is included in a code appendix. (yes/partial/no)
    \item[] Answer: [YES]
    
    \item[] \textbf{Question:} All source code required for conducting and analyzing the experiments will be made publicly available upon publication of the paper with a license that allows free usage for research purposes. (yes/partial/no)
    \item[] Answer: [YES]
     
    \item[] \textbf{Question:} All source code implementing new methods have comments detailing the implementation, with references to the paper where each step comes from (yes/partial/no)
    \item[] Answer: [YES]
    
    \item[] \textbf{Question:} If an algorithm depends on randomness, then the method used for setting seeds is described in a way sufficient to allow replication of results. (yes/partial/no/NA)
    \item[] Answer: [YES]
    
    \item[] \textbf{Question:} This paper specifies the computing infrastructure used for running experiments (hardware and software), including GPU/CPU models; amount of memory; operating system; names and versions of relevant software libraries and frameworks. (yes/partial/no)
    \item[] Answer: [YES]. This information is included in the Supplementary Materials accompanying the main PDF. We would like to specify here once more: Our BERT-based experiments were conducted on an NVIDIA RTX 3090 GPU with 24GB of memory. For experiments with the LLM2Vec backbone, we utilized an NVIDIA A100 GPU with 80GB of VRAM. The operating system used across all experiments was Ubuntu Server 18.04.3 LTS.
    
    \item[] \textbf{Question:} This paper formally describes evaluation metrics used and explains the motivation for choosing these metrics. (yes/partial/no)
    \item[] Answer: [YES]. Evaluation metrics are described in the Supplementary Materials accompanying the main PDF.
    
    \item[] \textbf{Question:} This paper states the number of algorithm runs used to compute each reported result. (yes/no)
    \item[] Answer: [YES]. We run the algorithms 6 times. We include this information in the Supplementary Materials accompanying the main PDF.
    
    \item[] \textbf{Question:} Analysis of experiments goes beyond single-dimensional summaries of performance (e.g., average; median) to include measures of variation, confidence, or other distributional information. (yes/no)
    \item[] Answer: [YES]
    
    \item[] \textbf{Question:} The significance of any improvement or decrease in performance is judged using appropriate statistical tests (e.g., Wilcoxon signed-rank). (yes/partial/no)
    \item[] Answer: [NO]
    
    \item[] \textbf{Question:} This paper lists all final (hyper-)parameters used for each model/algorithm in the paper's experiments. (yes/partial/no/NA)
    \item[] Answer: [YES]
    
    \item[] \textbf{Question:} This paper states the number and range of values tried per (hyper-) parameter during development of the paper, along with the criterion used for selecting the final parameter setting. (yes/partial/no/NA)
    \item[] Answer: [YES]. We include this information in the Supplementary Materials accompanying the main PDF.
    
\end{enumerate}

% This paper:
% \begin{itemize}
%     \item Includes a conceptual outline and/or pseudocode description of AI methods introduced (\textbf{\underline{yes}}/partial/no/NA)
%     \item Clearly delineates statements that are opinions, hypothesis, and speculation from objective facts and results (\textbf{\underline{yes}}/no)
%     \item Provides well marked pedagogical references for less-familiare readers to gain background necessary to replicate the paper (\textbf{\underline{yes}}/no)
% \end{itemize}

% \noindent
% Does this paper make theoretical contributions? (yes/\textbf{\underline{no}})

% \noindent
% Does this paper rely on one or more datasets? (\textbf{\underline{yes}}/no)
% \begin{itemize}
%     \item A motivation is given for why the experiments are conducted on the selected datasets (\textbf{\underline{yes}}/partial/no/NA)
%     \item All novel datasets introduced in this paper are included in a data appendix. (yes/partial/no/\textbf{\underline{NA}}).
%     We do not introduce novel datasets in this study.
%     \item All novel datasets introduced in this paper will be made publicly available upon publication of the paper with a license that allows free usage for research purposes. (yes/partial/no/\textbf{\underline{NA}}).
%     We do not introduce novel datasets in this study.
%     \item All datasets drawn from the existing literature (potentially including authors’ own previously published work) are accompanied by appropriate citations. (\textbf{\underline{yes}}/no/NA)
%     \item All datasets drawn from the existing literature (potentially including authors’ own previously published work) are publicly available. (\textbf{\underline{yes}}/partial/no/NA)
%     \item All datasets that are not publicly available are described in detail, with explanation why publicly available alternatives are not scientifically satisficing. (yes/partial/no/\textbf{\underline{NA}}).
%     We do not use datasets that are not publicly available.
% \end{itemize}

% \noindent
% Does this paper include computational experiments? (\textbf{\underline{yes}}/no)
% \begin{itemize}
%     \item Any code required for pre-processing data is included in the appendix. (\textbf{\underline{yes}}/partial/no).
%     \item All source code required for conducting and analyzing the experiments is included in a code appendix. (\textbf{\underline{yes}}/partial/no)
%     \item All source code required for conducting and analyzing the experiments will be made publicly available upon publication of the paper with a license that allows free usage for research purposes. (\textbf{\underline{yes}}/partial/no)
%     \item All source code implementing new methods have comments detailing the implementation, with references to the paper where each step comes from (\textbf{\underline{yes}}/partial/no)
%     \item If an algorithm depends on randomness, then the method used for setting seeds is described in a way sufficient to allow replication of results. (\textbf{\underline{yes}}/partial/no/NA)
%     \item This paper specifies the computing infrastructure used for running experiments (hardware and software), including GPU/CPU models; amount of memory; operating system; names and versions of relevant software libraries and frameworks. (\textbf{\underline{yes}}/partial/no)
%     \item This paper formally describes evaluation metrics used and explains the motivation for choosing these metrics. (\textbf{\underline{yes}}/partial/no)
%     \item This paper states the number of algorithm runs used to compute each reported result. (\textbf{\underline{yes}}/no)
%     \item Analysis of experiments goes beyond single-dimensional summaries of performance (e.g., average; median) to include measures of variation, confidence, or other distributional information. (\textbf{\underline{yes}}/no)
%     \item The significance of any improvement or decrease in performance is judged using appropriate statistical tests (e.g., Wilcoxon signed-rank). (yes/partial/\textbf{\underline{no}})
%     \item This paper lists all final (hyper-)parameters used for each model or algorithm in the paper’s experiments. 
%     (\textbf{\underline{yes}}/partial/no/NA)
%     \item This paper states the number and range of values tried per (hyper-) parameter during development of the paper, along with the criterion used for selecting the final parameter setting. (\textbf{\underline{yes}}/partial/no/NA)
%     \item \textbf{Source Code and Dependencies}: The source code for our experiments, along with detailed specifications for all external libraries utilized, has been submitted via the AAAI Rolling Review submission system. This provision ensures that researchers attempting to replicate our findings have comprehensive access to the necessary codebase and dependencies.
%     \item \textbf{Computing Infrastructure}: Our experiments involving a BERT-based model were conducted on an NVIDIA RTX 3090 GPU with 24GB of memory. For training the LLM2vec backbone, we utilized an NVIDIA A100 GPU with 80GB of VRAM. The operating system used across all experiments was Ubuntu Server 18.04.3 LTS.

% \end{itemize}

% \begin{itemize}
% \item \textbf{Source Code and Dependencies}: The source code for our experiments, along with detailed specifications for all external libraries utilized, has been submitted via the AAAI Rolling Review submission system. This provision ensures that researchers attempting to replicate our findings have comprehensive access to the necessary codebase and dependencies.
% \item \textbf{Computing Infrastructure}: Our experiments involving a BERT-based model were conducted on an NVIDIA RTX 3090 GPU with 24GB of memory. For training the LLM2vec backbone, we utilized an NVIDIA A100 GPU with 80GB of VRAM. The operating system used across all experiments was Ubuntu Server 18.04.3 LTS.
% \end{itemize}