%%%%%%%% ICML 2024 EXAMPLE LATEX SUBMISSION FILE %%%%%%%%%%%%%%%%%

\documentclass{article}

%%%%% NEW MATH DEFINITIONS %%%%%

\usepackage{amsmath,amsfonts,bm}
\usepackage{derivative}
% Mark sections of captions for referring to divisions of figures
\newcommand{\figleft}{{\em (Left)}}
\newcommand{\figcenter}{{\em (Center)}}
\newcommand{\figright}{{\em (Right)}}
\newcommand{\figtop}{{\em (Top)}}
\newcommand{\figbottom}{{\em (Bottom)}}
\newcommand{\captiona}{{\em (a)}}
\newcommand{\captionb}{{\em (b)}}
\newcommand{\captionc}{{\em (c)}}
\newcommand{\captiond}{{\em (d)}}

% Highlight a newly defined term
\newcommand{\newterm}[1]{{\bf #1}}

% Derivative d 
\newcommand{\deriv}{{\mathrm{d}}}

% Figure reference, lower-case.
\def\figref#1{figure~\ref{#1}}
% Figure reference, capital. For start of sentence
\def\Figref#1{Figure~\ref{#1}}
\def\twofigref#1#2{figures \ref{#1} and \ref{#2}}
\def\quadfigref#1#2#3#4{figures \ref{#1}, \ref{#2}, \ref{#3} and \ref{#4}}
% Section reference, lower-case.
\def\secref#1{section~\ref{#1}}
% Section reference, capital.
\def\Secref#1{Section~\ref{#1}}
% Reference to two sections.
\def\twosecrefs#1#2{sections \ref{#1} and \ref{#2}}
% Reference to three sections.
\def\secrefs#1#2#3{sections \ref{#1}, \ref{#2} and \ref{#3}}
% Reference to an equation, lower-case.
\def\eqref#1{equation~\ref{#1}}
% Reference to an equation, upper case
\def\Eqref#1{Equation~\ref{#1}}
% A raw reference to an equation---avoid using if possible
\def\plaineqref#1{\ref{#1}}
% Reference to a chapter, lower-case.
\def\chapref#1{chapter~\ref{#1}}
% Reference to an equation, upper case.
\def\Chapref#1{Chapter~\ref{#1}}
% Reference to a range of chapters
\def\rangechapref#1#2{chapters\ref{#1}--\ref{#2}}
% Reference to an algorithm, lower-case.
\def\algref#1{algorithm~\ref{#1}}
% Reference to an algorithm, upper case.
\def\Algref#1{Algorithm~\ref{#1}}
\def\twoalgref#1#2{algorithms \ref{#1} and \ref{#2}}
\def\Twoalgref#1#2{Algorithms \ref{#1} and \ref{#2}}
% Reference to a part, lower case
\def\partref#1{part~\ref{#1}}
% Reference to a part, upper case
\def\Partref#1{Part~\ref{#1}}
\def\twopartref#1#2{parts \ref{#1} and \ref{#2}}

\def\ceil#1{\lceil #1 \rceil}
\def\floor#1{\lfloor #1 \rfloor}
\def\1{\bm{1}}
\newcommand{\train}{\mathcal{D}}
\newcommand{\valid}{\mathcal{D_{\mathrm{valid}}}}
\newcommand{\test}{\mathcal{D_{\mathrm{test}}}}

\def\eps{{\epsilon}}


% Random variables
\def\reta{{\textnormal{$\eta$}}}
\def\ra{{\textnormal{a}}}
\def\rb{{\textnormal{b}}}
\def\rc{{\textnormal{c}}}
\def\rd{{\textnormal{d}}}
\def\re{{\textnormal{e}}}
\def\rf{{\textnormal{f}}}
\def\rg{{\textnormal{g}}}
\def\rh{{\textnormal{h}}}
\def\ri{{\textnormal{i}}}
\def\rj{{\textnormal{j}}}
\def\rk{{\textnormal{k}}}
\def\rl{{\textnormal{l}}}
% rm is already a command, just don't name any random variables m
\def\rn{{\textnormal{n}}}
\def\ro{{\textnormal{o}}}
\def\rp{{\textnormal{p}}}
\def\rq{{\textnormal{q}}}
\def\rr{{\textnormal{r}}}
\def\rs{{\textnormal{s}}}
\def\rt{{\textnormal{t}}}
\def\ru{{\textnormal{u}}}
\def\rv{{\textnormal{v}}}
\def\rw{{\textnormal{w}}}
\def\rx{{\textnormal{x}}}
\def\ry{{\textnormal{y}}}
\def\rz{{\textnormal{z}}}

% Random vectors
\def\rvepsilon{{\mathbf{\epsilon}}}
\def\rvphi{{\mathbf{\phi}}}
\def\rvtheta{{\mathbf{\theta}}}
\def\rva{{\mathbf{a}}}
\def\rvb{{\mathbf{b}}}
\def\rvc{{\mathbf{c}}}
\def\rvd{{\mathbf{d}}}
\def\rve{{\mathbf{e}}}
\def\rvf{{\mathbf{f}}}
\def\rvg{{\mathbf{g}}}
\def\rvh{{\mathbf{h}}}
\def\rvu{{\mathbf{i}}}
\def\rvj{{\mathbf{j}}}
\def\rvk{{\mathbf{k}}}
\def\rvl{{\mathbf{l}}}
\def\rvm{{\mathbf{m}}}
\def\rvn{{\mathbf{n}}}
\def\rvo{{\mathbf{o}}}
\def\rvp{{\mathbf{p}}}
\def\rvq{{\mathbf{q}}}
\def\rvr{{\mathbf{r}}}
\def\rvs{{\mathbf{s}}}
\def\rvt{{\mathbf{t}}}
\def\rvu{{\mathbf{u}}}
\def\rvv{{\mathbf{v}}}
\def\rvw{{\mathbf{w}}}
\def\rvx{{\mathbf{x}}}
\def\rvy{{\mathbf{y}}}
\def\rvz{{\mathbf{z}}}

% Elements of random vectors
\def\erva{{\textnormal{a}}}
\def\ervb{{\textnormal{b}}}
\def\ervc{{\textnormal{c}}}
\def\ervd{{\textnormal{d}}}
\def\erve{{\textnormal{e}}}
\def\ervf{{\textnormal{f}}}
\def\ervg{{\textnormal{g}}}
\def\ervh{{\textnormal{h}}}
\def\ervi{{\textnormal{i}}}
\def\ervj{{\textnormal{j}}}
\def\ervk{{\textnormal{k}}}
\def\ervl{{\textnormal{l}}}
\def\ervm{{\textnormal{m}}}
\def\ervn{{\textnormal{n}}}
\def\ervo{{\textnormal{o}}}
\def\ervp{{\textnormal{p}}}
\def\ervq{{\textnormal{q}}}
\def\ervr{{\textnormal{r}}}
\def\ervs{{\textnormal{s}}}
\def\ervt{{\textnormal{t}}}
\def\ervu{{\textnormal{u}}}
\def\ervv{{\textnormal{v}}}
\def\ervw{{\textnormal{w}}}
\def\ervx{{\textnormal{x}}}
\def\ervy{{\textnormal{y}}}
\def\ervz{{\textnormal{z}}}

% Random matrices
\def\rmA{{\mathbf{A}}}
\def\rmB{{\mathbf{B}}}
\def\rmC{{\mathbf{C}}}
\def\rmD{{\mathbf{D}}}
\def\rmE{{\mathbf{E}}}
\def\rmF{{\mathbf{F}}}
\def\rmG{{\mathbf{G}}}
\def\rmH{{\mathbf{H}}}
\def\rmI{{\mathbf{I}}}
\def\rmJ{{\mathbf{J}}}
\def\rmK{{\mathbf{K}}}
\def\rmL{{\mathbf{L}}}
\def\rmM{{\mathbf{M}}}
\def\rmN{{\mathbf{N}}}
\def\rmO{{\mathbf{O}}}
\def\rmP{{\mathbf{P}}}
\def\rmQ{{\mathbf{Q}}}
\def\rmR{{\mathbf{R}}}
\def\rmS{{\mathbf{S}}}
\def\rmT{{\mathbf{T}}}
\def\rmU{{\mathbf{U}}}
\def\rmV{{\mathbf{V}}}
\def\rmW{{\mathbf{W}}}
\def\rmX{{\mathbf{X}}}
\def\rmY{{\mathbf{Y}}}
\def\rmZ{{\mathbf{Z}}}

% Elements of random matrices
\def\ermA{{\textnormal{A}}}
\def\ermB{{\textnormal{B}}}
\def\ermC{{\textnormal{C}}}
\def\ermD{{\textnormal{D}}}
\def\ermE{{\textnormal{E}}}
\def\ermF{{\textnormal{F}}}
\def\ermG{{\textnormal{G}}}
\def\ermH{{\textnormal{H}}}
\def\ermI{{\textnormal{I}}}
\def\ermJ{{\textnormal{J}}}
\def\ermK{{\textnormal{K}}}
\def\ermL{{\textnormal{L}}}
\def\ermM{{\textnormal{M}}}
\def\ermN{{\textnormal{N}}}
\def\ermO{{\textnormal{O}}}
\def\ermP{{\textnormal{P}}}
\def\ermQ{{\textnormal{Q}}}
\def\ermR{{\textnormal{R}}}
\def\ermS{{\textnormal{S}}}
\def\ermT{{\textnormal{T}}}
\def\ermU{{\textnormal{U}}}
\def\ermV{{\textnormal{V}}}
\def\ermW{{\textnormal{W}}}
\def\ermX{{\textnormal{X}}}
\def\ermY{{\textnormal{Y}}}
\def\ermZ{{\textnormal{Z}}}

% Vectors
\def\vzero{{\bm{0}}}
\def\vone{{\bm{1}}}
\def\vmu{{\bm{\mu}}}
\def\vtheta{{\bm{\theta}}}
\def\vphi{{\bm{\phi}}}
\def\va{{\bm{a}}}
\def\vb{{\bm{b}}}
\def\vc{{\bm{c}}}
\def\vd{{\bm{d}}}
\def\ve{{\bm{e}}}
\def\vf{{\bm{f}}}
\def\vg{{\bm{g}}}
\def\vh{{\bm{h}}}
\def\vi{{\bm{i}}}
\def\vj{{\bm{j}}}
\def\vk{{\bm{k}}}
\def\vl{{\bm{l}}}
\def\vm{{\bm{m}}}
\def\vn{{\bm{n}}}
\def\vo{{\bm{o}}}
\def\vp{{\bm{p}}}
\def\vq{{\bm{q}}}
\def\vr{{\bm{r}}}
\def\vs{{\bm{s}}}
\def\vt{{\bm{t}}}
\def\vu{{\bm{u}}}
\def\vv{{\bm{v}}}
\def\vw{{\bm{w}}}
\def\vx{{\bm{x}}}
\def\vy{{\bm{y}}}
\def\vz{{\bm{z}}}

% Elements of vectors
\def\evalpha{{\alpha}}
\def\evbeta{{\beta}}
\def\evepsilon{{\epsilon}}
\def\evlambda{{\lambda}}
\def\evomega{{\omega}}
\def\evmu{{\mu}}
\def\evpsi{{\psi}}
\def\evsigma{{\sigma}}
\def\evtheta{{\theta}}
\def\eva{{a}}
\def\evb{{b}}
\def\evc{{c}}
\def\evd{{d}}
\def\eve{{e}}
\def\evf{{f}}
\def\evg{{g}}
\def\evh{{h}}
\def\evi{{i}}
\def\evj{{j}}
\def\evk{{k}}
\def\evl{{l}}
\def\evm{{m}}
\def\evn{{n}}
\def\evo{{o}}
\def\evp{{p}}
\def\evq{{q}}
\def\evr{{r}}
\def\evs{{s}}
\def\evt{{t}}
\def\evu{{u}}
\def\evv{{v}}
\def\evw{{w}}
\def\evx{{x}}
\def\evy{{y}}
\def\evz{{z}}

% Matrix
\def\mA{{\bm{A}}}
\def\mB{{\bm{B}}}
\def\mC{{\bm{C}}}
\def\mD{{\bm{D}}}
\def\mE{{\bm{E}}}
\def\mF{{\bm{F}}}
\def\mG{{\bm{G}}}
\def\mH{{\bm{H}}}
\def\mI{{\bm{I}}}
\def\mJ{{\bm{J}}}
\def\mK{{\bm{K}}}
\def\mL{{\bm{L}}}
\def\mM{{\bm{M}}}
\def\mN{{\bm{N}}}
\def\mO{{\bm{O}}}
\def\mP{{\bm{P}}}
\def\mQ{{\bm{Q}}}
\def\mR{{\bm{R}}}
\def\mS{{\bm{S}}}
\def\mT{{\bm{T}}}
\def\mU{{\bm{U}}}
\def\mV{{\bm{V}}}
\def\mW{{\bm{W}}}
\def\mX{{\bm{X}}}
\def\mY{{\bm{Y}}}
\def\mZ{{\bm{Z}}}
\def\mBeta{{\bm{\beta}}}
\def\mPhi{{\bm{\Phi}}}
\def\mLambda{{\bm{\Lambda}}}
\def\mSigma{{\bm{\Sigma}}}

% Tensor
\DeclareMathAlphabet{\mathsfit}{\encodingdefault}{\sfdefault}{m}{sl}
\SetMathAlphabet{\mathsfit}{bold}{\encodingdefault}{\sfdefault}{bx}{n}
\newcommand{\tens}[1]{\bm{\mathsfit{#1}}}
\def\tA{{\tens{A}}}
\def\tB{{\tens{B}}}
\def\tC{{\tens{C}}}
\def\tD{{\tens{D}}}
\def\tE{{\tens{E}}}
\def\tF{{\tens{F}}}
\def\tG{{\tens{G}}}
\def\tH{{\tens{H}}}
\def\tI{{\tens{I}}}
\def\tJ{{\tens{J}}}
\def\tK{{\tens{K}}}
\def\tL{{\tens{L}}}
\def\tM{{\tens{M}}}
\def\tN{{\tens{N}}}
\def\tO{{\tens{O}}}
\def\tP{{\tens{P}}}
\def\tQ{{\tens{Q}}}
\def\tR{{\tens{R}}}
\def\tS{{\tens{S}}}
\def\tT{{\tens{T}}}
\def\tU{{\tens{U}}}
\def\tV{{\tens{V}}}
\def\tW{{\tens{W}}}
\def\tX{{\tens{X}}}
\def\tY{{\tens{Y}}}
\def\tZ{{\tens{Z}}}


% Graph
\def\gA{{\mathcal{A}}}
\def\gB{{\mathcal{B}}}
\def\gC{{\mathcal{C}}}
\def\gD{{\mathcal{D}}}
\def\gE{{\mathcal{E}}}
\def\gF{{\mathcal{F}}}
\def\gG{{\mathcal{G}}}
\def\gH{{\mathcal{H}}}
\def\gI{{\mathcal{I}}}
\def\gJ{{\mathcal{J}}}
\def\gK{{\mathcal{K}}}
\def\gL{{\mathcal{L}}}
\def\gM{{\mathcal{M}}}
\def\gN{{\mathcal{N}}}
\def\gO{{\mathcal{O}}}
\def\gP{{\mathcal{P}}}
\def\gQ{{\mathcal{Q}}}
\def\gR{{\mathcal{R}}}
\def\gS{{\mathcal{S}}}
\def\gT{{\mathcal{T}}}
\def\gU{{\mathcal{U}}}
\def\gV{{\mathcal{V}}}
\def\gW{{\mathcal{W}}}
\def\gX{{\mathcal{X}}}
\def\gY{{\mathcal{Y}}}
\def\gZ{{\mathcal{Z}}}

% Sets
\def\sA{{\mathbb{A}}}
\def\sB{{\mathbb{B}}}
\def\sC{{\mathbb{C}}}
\def\sD{{\mathbb{D}}}
% Don't use a set called E, because this would be the same as our symbol
% for expectation.
\def\sF{{\mathbb{F}}}
\def\sG{{\mathbb{G}}}
\def\sH{{\mathbb{H}}}
\def\sI{{\mathbb{I}}}
\def\sJ{{\mathbb{J}}}
\def\sK{{\mathbb{K}}}
\def\sL{{\mathbb{L}}}
\def\sM{{\mathbb{M}}}
\def\sN{{\mathbb{N}}}
\def\sO{{\mathbb{O}}}
\def\sP{{\mathbb{P}}}
\def\sQ{{\mathbb{Q}}}
\def\sR{{\mathbb{R}}}
\def\sS{{\mathbb{S}}}
\def\sT{{\mathbb{T}}}
\def\sU{{\mathbb{U}}}
\def\sV{{\mathbb{V}}}
\def\sW{{\mathbb{W}}}
\def\sX{{\mathbb{X}}}
\def\sY{{\mathbb{Y}}}
\def\sZ{{\mathbb{Z}}}

% Entries of a matrix
\def\emLambda{{\Lambda}}
\def\emA{{A}}
\def\emB{{B}}
\def\emC{{C}}
\def\emD{{D}}
\def\emE{{E}}
\def\emF{{F}}
\def\emG{{G}}
\def\emH{{H}}
\def\emI{{I}}
\def\emJ{{J}}
\def\emK{{K}}
\def\emL{{L}}
\def\emM{{M}}
\def\emN{{N}}
\def\emO{{O}}
\def\emP{{P}}
\def\emQ{{Q}}
\def\emR{{R}}
\def\emS{{S}}
\def\emT{{T}}
\def\emU{{U}}
\def\emV{{V}}
\def\emW{{W}}
\def\emX{{X}}
\def\emY{{Y}}
\def\emZ{{Z}}
\def\emSigma{{\Sigma}}

% entries of a tensor
% Same font as tensor, without \bm wrapper
\newcommand{\etens}[1]{\mathsfit{#1}}
\def\etLambda{{\etens{\Lambda}}}
\def\etA{{\etens{A}}}
\def\etB{{\etens{B}}}
\def\etC{{\etens{C}}}
\def\etD{{\etens{D}}}
\def\etE{{\etens{E}}}
\def\etF{{\etens{F}}}
\def\etG{{\etens{G}}}
\def\etH{{\etens{H}}}
\def\etI{{\etens{I}}}
\def\etJ{{\etens{J}}}
\def\etK{{\etens{K}}}
\def\etL{{\etens{L}}}
\def\etM{{\etens{M}}}
\def\etN{{\etens{N}}}
\def\etO{{\etens{O}}}
\def\etP{{\etens{P}}}
\def\etQ{{\etens{Q}}}
\def\etR{{\etens{R}}}
\def\etS{{\etens{S}}}
\def\etT{{\etens{T}}}
\def\etU{{\etens{U}}}
\def\etV{{\etens{V}}}
\def\etW{{\etens{W}}}
\def\etX{{\etens{X}}}
\def\etY{{\etens{Y}}}
\def\etZ{{\etens{Z}}}

% The true underlying data generating distribution
\newcommand{\pdata}{p_{\rm{data}}}
\newcommand{\ptarget}{p_{\rm{target}}}
\newcommand{\pprior}{p_{\rm{prior}}}
\newcommand{\pbase}{p_{\rm{base}}}
\newcommand{\pref}{p_{\rm{ref}}}

% The empirical distribution defined by the training set
\newcommand{\ptrain}{\hat{p}_{\rm{data}}}
\newcommand{\Ptrain}{\hat{P}_{\rm{data}}}
% The model distribution
\newcommand{\pmodel}{p_{\rm{model}}}
\newcommand{\Pmodel}{P_{\rm{model}}}
\newcommand{\ptildemodel}{\tilde{p}_{\rm{model}}}
% Stochastic autoencoder distributions
\newcommand{\pencode}{p_{\rm{encoder}}}
\newcommand{\pdecode}{p_{\rm{decoder}}}
\newcommand{\precons}{p_{\rm{reconstruct}}}

\newcommand{\laplace}{\mathrm{Laplace}} % Laplace distribution

\newcommand{\E}{\mathbb{E}}
\newcommand{\Ls}{\mathcal{L}}
\newcommand{\R}{\mathbb{R}}
\newcommand{\emp}{\tilde{p}}
\newcommand{\lr}{\alpha}
\newcommand{\reg}{\lambda}
\newcommand{\rect}{\mathrm{rectifier}}
\newcommand{\softmax}{\mathrm{softmax}}
\newcommand{\sigmoid}{\sigma}
\newcommand{\softplus}{\zeta}
\newcommand{\KL}{D_{\mathrm{KL}}}
\newcommand{\Var}{\mathrm{Var}}
\newcommand{\standarderror}{\mathrm{SE}}
\newcommand{\Cov}{\mathrm{Cov}}
% Wolfram Mathworld says $L^2$ is for function spaces and $\ell^2$ is for vectors
% But then they seem to use $L^2$ for vectors throughout the site, and so does
% wikipedia.
\newcommand{\normlzero}{L^0}
\newcommand{\normlone}{L^1}
\newcommand{\normltwo}{L^2}
\newcommand{\normlp}{L^p}
\newcommand{\normmax}{L^\infty}

\newcommand{\parents}{Pa} % See usage in notation.tex. Chosen to match Daphne's book.

\DeclareMathOperator*{\argmax}{arg\,max}
\DeclareMathOperator*{\argmin}{arg\,min}

\DeclareMathOperator{\sign}{sign}
\DeclareMathOperator{\Tr}{Tr}
\let\ab\allowbreak


% Recommended, but optional, packages for figures and better typesetting:
\usepackage{microtype}
\usepackage{graphicx}
% \usepackage{subfigure}
\usepackage{booktabs} % for professional tables

% hyperref makes hyperlinks in the resulting PDF.
% If your build breaks (sometimes temporarily if a hyperlink spans a page)
% please comment out the following usepackage line and replace
% \usepackage{icml2025} with \usepackage[nohyperref]{icml2025} above.
\usepackage{hyperref}
\usepackage{xurl}
\usepackage{wrapfig}
\usepackage{multirow}
\usepackage{CJKutf8}

\def\<{\left\langle} % Angle brackets
\def\>{\right\rangle}
\providecommand{\argmax}{\mathop{\rm argmax}} 
\providecommand{\argmin}{\mathop{\rm argmin}}

\newcommand{\vthe}{\bm{\theta}}
\newcommand{\x}{\bm{x}}


% Attempt to make hyperref and algorithmic work together better:
\newcommand{\theHalgorithm}{\arabic{algorithm}}
\newcommand{\myparagraph}[1]{\noindent\textbf{#1}}

% Use the following line for the initial blind version submitted for review:
% \usepackage{icml2025}

% If accepted, instead use the following line for the camera-ready submission:
\usepackage[accepted]{icml2025}

% For theorems and such
\usepackage{amsmath}
\usepackage{amssymb}
\usepackage{mathtools}
\usepackage{amsthm}
\usepackage{subcaption}

% if you use cleveref..
\usepackage[capitalize,noabbrev]{cleveref}

%%%%%%%%%%%%%%%%%%%%%%%%%%%%%%%%
% THEOREMS
%%%%%%%%%%%%%%%%%%%%%%%%%%%%%%%%
\theoremstyle{plain}
\newtheorem{theorem}{Theorem}[section]
\newtheorem{proposition}[theorem]{Proposition}
\newtheorem{lemma}[theorem]{Lemma}
\newtheorem{corollary}[theorem]{Corollary}
\theoremstyle{definition}
\newtheorem{definition}[theorem]{Definition}
\newtheorem{assumption}[theorem]{Assumption}
\theoremstyle{remark}
\newtheorem{remark}[theorem]{Remark}

\newcommand\xie[1]{[{\color{red}Xie: #1}]}
\newcommand\cgy[1]{{\color{blue}{cgy: #1}}}

% Todonotes is useful during development; simply uncomment the next line
%    and comment out the line below the next line to turn off comments
%\usepackage[disable,textsize=tiny]{todonotes}
\usepackage[textsize=tiny]{todonotes}


% The \icmltitle you define below is probably too long as a header.
% Therefore, a short form for the running title is supplied here:
\icmltitlerunning{Principled Data Selection for Alignment}

\begin{document}

\twocolumn[
\icmltitle{Principled Data Selection for Alignment: The Hidden Risks of Difficult Examples}

% It is OKAY to include author information, even for blind
% submissions: the style file will automatically remove it for you
% unless you've provided the [accepted] option to the icml2025
% package.

% List of affiliations: The first argument should be a (short)
% identifier you will use later to specify author affiliations
% Academic affiliations should list Department, University, City, Region, Country
% Industry affiliations should list Company, City, Region, Country

% You can specify symbols, otherwise they are numbered in order.
% Ideally, you should not use this facility. Affiliations will be numbered
% in order of appearance and this is the preferred way.
\icmlsetsymbol{equal}{*}

\begin{icmlauthorlist}
\icmlauthor{Chengqian Gao}{mbzuai}
\icmlauthor{Haonan Li}{mbzuai}
\icmlauthor{Liu Liu}{tencent}
\icmlauthor{Zeke Xie}{hkust-gz}
\icmlauthor{Peilin Zhao}{tencent}
\icmlauthor{Zhiqiang Xu}{tencent}
\end{icmlauthorlist}

\icmlaffiliation{mbzuai}{MBZUAI}
\icmlaffiliation{tencent}{Tencent Inc}
\icmlaffiliation{hkust-gz}{HKUST (Guangzhou)}

\icmlcorrespondingauthor{Liu Liu}{first1.last1@xxx.edu}
\icmlcorrespondingauthor{Peilin Zhao}{first2.last2@www.uk}
\icmlcorrespondingauthor{Zhiqiang Xu}{first2.last2@www.uk}

% You may provide any keywords that you
% find helpful for describing your paper; these are used to populate
% the "keywords" metadata in the PDF but will not be shown in the document
\icmlkeywords{Large Language Model, Alignment}

\vskip 0.3in
]

% this must go after the closing bracket ] following \twocolumn[ ...

% This command actually creates the footnote in the first column
% listing the affiliations and the copyright notice.
% The command takes one argument, which is text to display at the start of the footnote.
% The \icmlEqualContribution command is standard text for equal contribution.
% Remove it (just {}) if you do not need this facility.

% \printAffiliationsAndNotice{}  % leave blank if no need to mention equal contribution

\begin{abstract}
The alignment of large language models (LLMs) often assumes that using more clean data yields better outcomes, overlooking the match between model capacity and example difficulty. Challenging this, we propose a new principle: \textit{``Preference data vary in difficulty, and overly difficult examples hinder alignment, by exceeding the model's capacity."} Through systematic experimentation, we validate this principle with three key findings: (1) preference examples vary in difficulty, as evidenced by consistent learning orders across alignment runs; (2) overly difficult examples significantly degrade performance across four LLMs and two datasets; and (3) the capacity of a model dictates its threshold for handling difficult examples, underscoring a critical relationship between data selection and model capacity. Building on this principle, we introduce \textit{Selective DPO}, which filters out overly difficult examples. This simple adjustment improves alignment performance by 9-16\% in win rates on the AlpacaEval 2 benchmark compared to the DPO baseline, suppressing a series of DPO variants with different algorithmic adjustments. 
Together, these results illuminate the importance of aligning data difficulty with model capacity, offering a transformative perspective for improving alignment strategies in LLMs.\footnote{Code is available at \url{https://github.com/glorgao/SelectiveDPO}.} 
\end{abstract}

%!TEX root = gcn.tex
\section{Introduction}
Graphs, representing structural data and topology, are widely used across various domains, such as social networks and merchandising transactions.
Graph convolutional networks (GCN)~\cite{iclr/KipfW17} have significantly enhanced model training on these interconnected nodes.
However, these graphs often contain sensitive information that should not be leaked to untrusted parties.
For example, companies may analyze sensitive demographic and behavioral data about users for applications ranging from targeted advertising to personalized medicine.
Given the data-centric nature and analytical power of GCN training, addressing these privacy concerns is imperative.

Secure multi-party computation (MPC)~\cite{crypto/ChaumDG87,crypto/ChenC06,eurocrypt/CiampiRSW22} is a critical tool for privacy-preserving machine learning, enabling mutually distrustful parties to collaboratively train models with privacy protection over inputs and (intermediate) computations.
While research advances (\eg,~\cite{ccs/RatheeRKCGRS20,uss/NgC21,sp21/TanKTW,uss/WatsonWP22,icml/Keller022,ccs/ABY318,folkerts2023redsec}) support secure training on convolutional neural networks (CNNs) efficiently, private GCN training with MPC over graphs remains challenging.

Graph convolutional layers in GCNs involve multiplications with a (normalized) adjacency matrix containing $\numedge$ non-zero values in a $\numnode \times \numnode$ matrix for a graph with $\numnode$ nodes and $\numedge$ edges.
The graphs are typically sparse but large.
One could use the standard Beaver-triple-based protocol to securely perform these sparse matrix multiplications by treating graph convolution as ordinary dense matrix multiplication.
However, this approach incurs $O(\numnode^2)$ communication and memory costs due to computations on irrelevant nodes.
%
Integrating existing cryptographic advances, the initial effort of SecGNN~\cite{tsc/WangZJ23,nips/RanXLWQW23} requires heavy communication or computational overhead.
Recently, CoGNN~\cite{ccs/ZouLSLXX24} optimizes the overhead in terms of  horizontal data partitioning, proposing a semi-honest secure framework.
Research for secure GCN over vertical data  remains nascent.

Current MPC studies, for GCN or not, have primarily targeted settings where participants own different data samples, \ie, horizontally partitioned data~\cite{ccs/ZouLSLXX24}.
MPC specialized for scenarios where parties hold different types of features~\cite{tkde/LiuKZPHYOZY24,icml/CastigliaZ0KBP23,nips/Wang0ZLWL23} is rare.
This paper studies $2$-party secure GCN training for these vertical partition cases, where one party holds private graph topology (\eg, edges) while the other owns private node features.
For instance, LinkedIn holds private social relationships between users, while banks own users' private bank statements.
Such real-world graph structures underpin the relevance of our focus.
To our knowledge, no prior work tackles secure GCN training in this context, which is crucial for cross-silo collaboration.


To realize secure GCN over vertically split data, we tailor MPC protocols for sparse graph convolution, which fundamentally involves sparse (adjacency) matrix multiplication.
Recent studies have begun exploring MPC protocols for sparse matrix multiplication (SMM).
ROOM~\cite{ccs/SchoppmannG0P19}, a seminal work on SMM, requires foreknowledge of sparsity types: whether the input matrices are row-sparse or column-sparse.
Unfortunately, GCN typically trains on graphs with arbitrary sparsity, where nodes have varying degrees and no specific sparsity constraints.
Moreover, the adjacency matrix in GCN often contains a self-loop operation represented by adding the identity matrix, which is neither row- nor column-sparse.
Araki~\etal~\cite{ccs/Araki0OPRT21} avoid this limitation in their scalable, secure graph analysis work, yet it does not cover vertical partition.

% and related primitives
To bridge this gap, we propose a secure sparse matrix multiplication protocol, \osmm, achieving \emph{accurate, efficient, and secure GCN training over vertical data} for the first time.

\subsection{New Techniques for Sparse Matrices}
The cost of evaluating a GCN layer is dominated by SMM in the form of $\adjmat\feamat$, where $\adjmat$ is a sparse adjacency matrix of a (directed) graph $\graph$ and $\feamat$ is a dense matrix of node features.
For unrelated nodes, which often constitute a substantial portion, the element-wise products $0\cdot x$ are always zero.
Our efficient MPC design 
avoids unnecessary secure computation over unrelated nodes by focusing on computing non-zero results while concealing the sparse topology.
We achieve this~by:
1) decomposing the sparse matrix $\adjmat$ into a product of matrices (\S\ref{sec::sgc}), including permutation and binary diagonal matrices, that can \emph{faithfully} represent the original graph topology;
2) devising specialized protocols (\S\ref{sec::smm_protocol}) for efficiently multiplying the structured matrices while hiding sparsity topology.


 
\subsubsection{Sparse Matrix Decomposition}
We decompose adjacency matrix $\adjmat$ of $\graph$ into two bipartite graphs: one represented by sparse matrix $\adjout$, linking the out-degree nodes to edges, the other 
by sparse matrix $\adjin$,
linking edges to in-degree nodes.

%\ie, we decompose $\adjmat$ into $\adjout \adjin$, where $\adjout$ and $\adjin$ are sparse matrices representing these connections.
%linking out-degree nodes to edges and edges to in-degree nodes of $\graph$, respectively.

We then permute the columns of $\adjout$ and the rows of $\adjin$ so that the permuted matrices $\adjout'$ and $\adjin'$ have non-zero positions with \emph{monotonically non-decreasing} row and column indices.
A permutation $\sigma$ is used to preserve the edge topology, leading to an initial decomposition of $\adjmat = \adjout'\sigma \adjin'$.
This is further refined into a sequence of \emph{linear transformations}, 
which can be efficiently computed by our MPC protocols for 
\emph{oblivious permutation}
%($\Pi_{\ssp}$) 
and \emph{oblivious selection-multiplication}.
% ($\Pi_\SM$)
\iffalse
Our approach leverages bipartite graph representation and the monotonicity of non-zero positions to decompose a general sparse matrix into linear transformations, enhancing the efficiency of our MPC protocols.
\fi
Our decomposition approach is not limited to GCNs but also general~SMM 
by 
%simply 
treating them 
as adjacency matrices.
%of a graph.
%Since any sparse matrix can be viewed 

%allowing the same technique to be applied.

 
\subsubsection{New Protocols for Linear Transformations}
\emph{Oblivious permutation} (OP) is a two-party protocol taking a private permutation $\sigma$ and a private vector $\xvec$ from the two parties, respectively, and generating a secret share $\l\sigma \xvec\r$ between them.
Our OP protocol employs correlated randomnesses generated in an input-independent offline phase to mask $\sigma$ and $\xvec$ for secure computations on intermediate results, requiring only $1$ round in the online phase (\cf, $\ge 2$ in previous works~\cite{ccs/AsharovHIKNPTT22, ccs/Araki0OPRT21}).

Another crucial two-party protocol in our work is \emph{oblivious selection-multiplication} (OSM).
It takes a private bit~$s$ from a party and secret share $\l x\r$ of an arithmetic number~$x$ owned by the two parties as input and generates secret share $\l sx\r$.
%between them.
%Like our OP protocol, o
Our $1$-round OSM protocol also uses pre-computed randomnesses to mask $s$ and $x$.
%for secure computations.
Compared to the Beaver-triple-based~\cite{crypto/Beaver91a} and oblivious-transfer (OT)-based approaches~\cite{pkc/Tzeng02}, our protocol saves ${\sim}50\%$ of online communication while having the same offline communication and round complexities.

By decomposing the sparse matrix into linear transformations and applying our specialized protocols, our \osmm protocol
%($\prosmm$) 
reduces the complexity of evaluating $\numnode \times \numnode$ sparse matrices with $\numedge$ non-zero values from $O(\numnode^2)$ to $O(\numedge)$.

%(\S\ref{sec::secgcn})
\subsection{\cgnn: Secure GCN made Efficient}
Supported by our new sparsity techniques, we build \cgnn, 
a two-party computation (2PC) framework for GCN inference and training over vertical
%ly split
data.
Our contributions include:

1) We are the first to explore sparsity over vertically split, secret-shared data in MPC, enabling decompositions of sparse matrices with arbitrary sparsity and isolating computations that can be performed in plaintext without sacrificing privacy.

2) We propose two efficient $2$PC primitives for OP and OSM, both optimally single-round.
Combined with our sparse matrix decomposition approach, our \osmm protocol ($\prosmm$) achieves constant-round communication costs of $O(\numedge)$, reducing memory requirements and avoiding out-of-memory errors for large matrices.
In practice, it saves $99\%+$ communication
%(Table~\ref{table:comm_smm}) 
and reduces ${\sim}72\%$ memory usage over large $(5000\times5000)$ matrices compared with using Beaver triples.
%(Table~\ref{table:mem_smm_sparse}) ${\sim}16\%$-

3) We build an end-to-end secure GCN framework for inference and training over vertically split data, maintaining accuracy on par with plaintext computations.
We will open-source our evaluation code for research and deployment.

To evaluate the performance of $\cgnn$, we conducted extensive experiments over three standard graph datasets (Cora~\cite{aim/SenNBGGE08}, Citeseer~\cite{dl/GilesBL98}, and Pubmed~\cite{ijcnlp/DernoncourtL17}),
reporting communication, memory usage, accuracy, and running time under varying network conditions, along with an ablation study with or without \osmm.
Below, we highlight our key achievements.

\textit{Communication (\S\ref{sec::comm_compare_gcn}).}
$\cgnn$ saves communication by $50$-$80\%$.
(\cf,~CoGNN~\cite{ccs/KotiKPG24}, OblivGNN~\cite{uss/XuL0AYY24}).

\textit{Memory usage (\S\ref{sec::smmmemory}).}
\cgnn alleviates out-of-memory problems of using %the standard 
Beaver-triples~\cite{crypto/Beaver91a} for large datasets.

\textit{Accuracy (\S\ref{sec::acc_compare_gcn}).}
$\cgnn$ achieves inference and training accuracy comparable to plaintext counterparts.
%training accuracy $\{76\%$, $65.1\%$, $75.2\%\}$ comparable to $\{75.7\%$, $65.4\%$, $74.5\%\}$ in plaintext.

{\textit{Computational efficiency (\S\ref{sec::time_net}).}} 
%If the network is worse in bandwidth and better in latency, $\cgnn$ shows more benefits.
$\cgnn$ is faster by $6$-$45\%$ in inference and $28$-$95\%$ in training across various networks and excels in narrow-bandwidth and low-latency~ones.

{\textit{Impact of \osmm (\S\ref{sec:ablation}).}}
Our \osmm protocol shows a $10$-$42\times$ speed-up for $5000\times 5000$ matrices and saves $10$-2$1\%$ memory for ``small'' datasets and up to $90\%$+ for larger ones.

\section{Preliminaries}
\label{sec:preliminaries}

Let $S$ be a set of vertices of a graph $G=(V, E)$.
We denote the total degree of vertices in $S$ by $d(S) = \sum_{u \in S} d_u$, where $d_u$ represents the degree of vertex $u$.
A simple observation is that the size of the edge cut $\partial(S)$ is determined by the degree of the vertices in $S$ and the edges between vertices of $S$.
If there are $k$ edges between vertices of $S$, then $\size{\partial(S)}=d(S)- 2k$.
%
Since the number of edges in $S$ may vary from $0$ to $\binom{\size{S}}{2}$, a necessary condition for the realizability of an \GRC{} instance is as follows.

\begin{remark}
\label{thm:feasible_cut_sizes}
    A \GRC{} instance $(\texttt{d}, \call)$ is realizable only if, for each cut $(S, \ell) \in \call$, we have $\ell \in \set{ d(S) - 2k : 0 \leq k \leq \binom{\size{S}}{2} }$.
\end{remark}

Since this condition is easily verifiable, we assume henceforth that it holds for any \GRC{} instance. In particular, for cuts of size two, this observation implies that only two feasible values are possible, determining whether an edge must exist between the corresponding vertices, as detailed below.

\begin{remark}
\label{thm:fixed_forbidden_edges}
    Given an instance $I = (\texttt{d}, \call)$ of \GRC{}, in any realization $G$ of $I$, if $(\set{u, v}, d_u + d_v - 2) \in \call$, then $uv \in E(G)$, and if $(\set{u, v}, d_u + d_v) \in \call$, then $uv \notin E(G)$.
\end{remark}

Based on this, we say that an edge $uv$ is \textit{fixed} if $(\set{u, v}, d_u + d_v - 2) \in \call$ and is \textit{forbidden} if $(\set{u, v}, d_u + d_v) \in \call$.
We apply similar terminology when constructing an instance of \GRC{}.
Given an instance $(\texttt{d}, \call)$ of \GRC{}, to \textit{fix} or \textit{forbid} an edge $uv$ means adding the cut $(\set{u, v}, d_u + d_v - 2)$ or $(\set{u, v}, d_u + d_v)$ to $\call$, respectively.

\cref{thm:fixed_forbidden_edges} implies that the \GRC{} problem, when limited to cuts of size two, is equivalent to the \GR{} problem with added constraints: a subset of edges is fixed, and another disjoint one is forbidden. Moreover, we can simplify the problem by focusing only on forbidden edges by reducing the degree of vertices incident to fixed edges and then marking those edges as forbidden.
%
Formally, given an instance $(\texttt{d}, \call)$ and a cut $(\set{u, v}, d_u + d_v - 2) \in \call$, in which case the edge $uv$ is fixed, we can produce an equivalent instance $(\texttt{d}', \call')$ as follows.  For all $i \notin \{u, v\}$  set $d'_i = d_i$. Reduce $d'_u = d_u - 1$ and   $d'_v = d_v - 1$;  
%\begin{align*}
    % \begin{cases}
    %     d'_i = d_i, &\text{ if  $i \notin \{u, v\}$ }\\
    %     d'_u = d_u - 1 \\
    %     d'_v = d_v - 1 
    % \end{cases}
    %
    % d'_i &= d_i, &\mbox{ if  }i \notin \{u, v\};\\
    %d'_i &= d_i, &\forall\ i \notin \{u, v\};\\
    %d'_u &= d_u - 1 ;\\
    %d'_v &= d_v - 1 ;
%\end{align*}
and $\call'$ is obtained from $\call$ by replacing $(\set{u, v}, d_u + d_v - 2)$ with $(\set{u, v}, d_u + d_v)$.


The resulting instance $(\texttt{d}', \call')$ has a realization if and only if $(\texttt{d}, \call)$ has a realization. If $G = (V, E)$ is a realization of $(\texttt{d}, \call)$, then, as discussed above, we must have $uv \in E$, and $G - uv$ is a realization of $\call'$. Conversely, if $G' = (V, E')$ is a realization of $(\texttt{d}', \call')$, then necessarily $uv \notin E'$ due to the cut $(\set{u, v}, d_u + d_v)$, and $G' + uv$ is a realization of $(\texttt{d}, \call)$.

Thus, cut restrictions involving sets of size two can be simply reinterpreted as forbidding edges.
%
Let $F$ be the set of all forbidden edges that cannot appear in any realization of instance $(\texttt{d}, \call)$. Then $\calg = K_n - F$ is what we call the \emph{possibility graph}, which must be a supergraph of any valid realization of~$(\texttt{d}, \call)$.

\section{Preference Examples Vary in Difficulty}
\label{sec:curricula}

One surprising finding here is that the order in which examples are learned is remarkably consistent across runs. Such robustness reveals the underlying presence of example difficulty. We then validate the effectiveness of validation loss as a measure of example difficulty for alignment tasks.

\subsection{The Underlying Example Difficulty}\label{sec:consistent_learning_order}

While various metrics such as length~\cite{spitkovsky2010baby,tay2019simple,nagatsuka2023length} and perplexity~\cite{wu2024curriculum} have been proposed to measure difficulty of text samples,
their ability to reliably capture example difficulty remains controversial~\cite{campos2021curriculum}. 
We address this concern by demonstrating: (1) examples have distinct learned steps (see Eq.\ref{eq:learned-step}), indicating different difficulty levels, and (2) these learned steps are consistent across runs with different training data and random seeds.

In Figure~\ref{fig:implicit_curricula} (left), we visualize the learned steps of 300 test examples from \textit{Ultrafeedback-binarized}\footnote{\url{https://huggingface.co/datasets/HuggingFaceH4/ultrachat\_200k}},% using color coding, where a larger learned step indicates more training steps required for the LLM to understand the corresponding example. 
where darker colors indicate more training steps needed for model comprehension. 
Results from 10 runs show consistent learning order across different models~\citep{jiang2023mistral,llama3modelcard,team2024gemma} varying in size (2B--9B), training stage, and data sampling. This consistency confirms that examples vary in difficulty, allowing us to discuss difficult examples without debating various definitions of difficulty.



\subsection{Validation Loss as a Proxy for Learned Step}

The robust learning order suggests the existence of difficult examples---some examples are consistently harder for LLMs to understand.  
However, identifying these examples at scale is computationally expensive, as the computing of learned step requires evaluating the model after each gradient update. To address this, we adopt validation loss from the curriculum learning literature~\cite{wucurricula,rampp2024does} (see Eq(\ref{eq:validation-loss})). Specifically, we train six reference models using the DPO objective on the randomly sampled half training set and evaluate the validation loss for examples on the other half.
We refer the difficult examples to examples with large validation loss.

\begin{definition}[Difficult example]
    \label{def:difficult-example}
    A preference example $(x, y_w, y_l)$ is considered a \textit{difficult example} if its \textit{validation loss} exceeds or equals a specified value: $$\text{VL}(x, y_w, y_l) \geq Q(\tau).$$ 
\end{definition}

\begin{remark}
    We introduce a flexible threshold $Q(\tau)$, \textit{i.e.,} the $\tau$-quantile of the validation loss distribution. This variability arises mainly from: (1) There is no formal definition of the easy and difficult samples~\cite{zhu2024exploring}, and (2) Different pre-train models have different training dynamics and thus different scales and distributions of validation loss. 
\end{remark}

\begin{figure*}
    \centering
    \includegraphics[width=\linewidth]{fig/base-model-argilla-armo.pdf}\vspace{-0.1cm}
    \includegraphics[width=\linewidth]{fig/base-model-uf-armo.pdf}
    \vspace{-0.4cm}
    \caption{
    \textbf{Direct Preference Optimization (DPO) struggles with difficult examples, broadly and significantly.} We present the defined WR$'$ evolution for four models trained on the \textit{argilla-mix-dpo-7k} and \textit{ultrafeedback-binarized} datasets. The results are based on checkpoints from three 1-eopch runs with different seeds. 
    \textit{\underline{Random Ordering (DPO)}}: Training data are presented in a randomized sequence. 
    \textit{\underline{Sorted by VL (From Easy to Difficult)}}: Training examples are ranked by their \textit{validation loss} (VL) and presented from easy to difficult, following a curriculum learning approach. 
    \textit{\underline{Selected by VL (Shuffled)}}: The easiest 60\% (for Argilla-7K) or 50\% (for UF-binarized) of the data is selected based on VL, and examples are sampled in a random order for training. The VL measurements are displayed as bar plots. We include evaluation results (dashed lines) from the two corresponding DPO models released by~\citet{meng2024simpo} for reference.
    }
    \label{fig:base-model-struggles}
\end{figure*}



% \subsection{Validation Loss as an Effective Alternative}
To assess whether validation loss effectively approximates the learned step, we examine the correlation between difficulty rankings produced by these two measures. Using \textit{Spearman’s} rank correlation, we compared rankings across different runs and models. As shown in the middle panel of Figure~\ref{fig:implicit_curricula}, validation loss exhibits patterns remarkably similar to the learned step. Furthermore, the high correlation coefficients between average learned step and average validation loss across the four models (0.9258, 0.9227, 0.9336, and 0.9283) validate the effectiveness of validation loss as a computationally efficient proxy for learned step. 
Additionally, the \textit{Jaccard similarity} between difficult example sets (defined as top 50\% by either metric) remains consistently high for each model (Figure~\ref{fig:implicit_curricula}, right), confirming that both measures identify similar sets of difficult examples. 



\section{Difficult Examples Hinder the Alignment}
\label{sec:hinders}
In this section, we first demonstrate that difficult examples significantly degrade alignment performance across various datasets and model scales. We then investigate the factors that contribute to their difficulty through a series of systematically designed empirical studies.

\subsection{Investigation Setup}
\textbf{Models:} We evaluate SFT models trained on the \textit{UltraChat-200k} dataset: \textbf{Mistral-7B-SFT}~\citep{jiang2023mistral}, \textbf{Qwen-2.5-7B-SFT}~\cite{qwen2.5}, \textbf{Llama3-8B-SFT}~\citep{llama3modelcard} and \textbf{Gemma-2-9B-SFT}~\citep{team2024gemma}. We focus on SFT models as they better demonstrate the effects of alignment procedures~\cite{meng2024simpo}. 

\textbf{Datasets:} We use \textit{UltraFeedback-binarized}, a widely adopted alignment dataset~\cite{tunstall2023zephyr,meng2024simpo,zhou2024wpo,pattnaik2024curry}, and \textit{Argilla-dpo-mix-7k}\footnote{\url{https://huggingface.co/datasets/argilla/dpo-mix-7k}}, a small but high-quality dataset.

\textbf{Hyper-parameters:} Following prior work, we set $\beta=0.01$~\cite{zhou2024wpo}. The learning rate is sweeped for DPO with random ordering and directly applied to DPO with other settings. We conduct the alignment with one epoch following~\citet{meng2024simpo}. 

\textbf{Evaluation:} We employ \textbf{WR$'$}, the win rate against \textit{gpt-4-turbo} on 805 testing examples from \textit{AlpacaEval 2}~\citep{dubois2024length} with \textit{ArmoRM}~\citep{wang2024interpretable}, a reward model with impressive performance on the RewardBench~\cite{lambert2024rewardbench}, as the evaluator.


\subsection{Difficult Examples Hinder Preference Alignment} 
As shown in Figure~\ref{fig:base-model-struggles}, training on difficult examples leads to significant performance declines. We compare three example-ordering strategies: (1) random ordering (standard DPO), (2) easy-to-difficult sorting by validation loss, and (3) random ordering with only easy examples. Despite using the same training recipes, models consistently perform better when trained on easier examples across all four architectures and both datasets. Notably, the benefits are mainly unlocked by excluding difficult examples rather than the ordering itself, as shown by the similar performance of sorted and shuffled easy examples (setting 2 and 3).


The performance drop from difficult examples is more pronounced in \textit{Ultrafeedback-binarized}. This aligns the observation that \textit{Ultrafeedback-binarized} contains mislabeled examples~\cite{ultrafeedback_preferences, notus2023} and \textit{Argilla-dpo-mix-7k} is characterized by high-quality data. 

\begin{figure*}[!ht]
    \centering
    \begin{subfigure}[t]{0.32\textwidth}
        \centering
        \includegraphics[width=\textwidth]{fig/mislabeled.pdf}
        \vspace{-0.7cm}
        \caption{Label flipping.}
    \end{subfigure}
    \hfill
    \begin{subfigure}[t]{0.32\textwidth}
        \centering
        \includegraphics[width=\textwidth]{fig/distribution-shift.pdf}
        \vspace{-0.7cm}
        \caption{Distribution shift.}
    \end{subfigure}
    \hfill
    \begin{subfigure}[t]{0.32\textwidth}
    \centering
    \includegraphics[width=\textwidth]{fig/improper-learning-rate.pdf}
    \vspace{-0.7cm}
    \caption{Improper learning rate.}
    \end{subfigure}
    \vspace{-0.2cm}
    \caption{
    \textbf{Difficulty examples are not necessarily data errors.} \underline{\textit{(a)}}: flipping the last 40\% examples with higher \textit{validation loss}. \underline{\textit{(b)}}: sorting the examples with the \textit{$\epsilon$-greedy sorting} algorithm. In this case, each mini-batch data contains (1-$\epsilon$) part of easy-to-difficult examples and ($\epsilon$) part of randomly sampled examples. \underline{\textit{(c)}}: increasing and decreasing the learning rate. 
    All experiments are conducted on the Mistral-7B-SFT model with \textit{Argilla-dpo-mix-7k} dataset.
    }
    \label{fig:insights}
\end{figure*}


\begin{figure*}[!ht]
    \centering
    \includegraphics[width=0.97\linewidth]{fig/scaling_law_qwen_argilla_armo_detailed_figs.pdf}\vspace{-0.2cm}
    \caption{
    \textbf{Difficult examples benefit larger models with greater capacities.} Examples are sorted by their \textit{validation loss}, ranging from easy to difficult. We fit the measured \textbf{WR$'$} (scatter points) using a second-degree polynomial (dashed line), identifying the peak of each parabola as the \textit{sweet spot} (marker).
    Notably, larger models reach sweet spots at higher data percentages, indicating that model with greater capacity can manage more challenging examples. The results are from ten runs per model type, evaluated using \textit{ArmoRM}~\cite{wang2024interpretable}. 
    }
    \vspace{-0.4cm}
    \label{fig:scaling_law}
\end{figure*}


\subsection{Difficult Examples Are Not Necessarily Data Errors}
\label{sec:insights}
Before proposing our solution to filter out difficult and harmful examples, we here investigate their characters, ensuring their removal is justified. For statistics and case study on difficult examples, please refer to Appendix~\ref{app:feature-analysis} and~\ref{app:case_study_difficult_example}. 

\textbf{Mislabeled data (Figure~\ref{fig:insights} (a)).} Prior work suggests that difficult examples might be mislabeled~\cite{ultrafeedback_preferences,notus2023}. To test this hypothesis, we sort the examples by their validation loss and flip the labels of last $40\%$ (the most difficult) examples. However, this modification does not alleviate the performance drop, suggesting that label noise is not the primary cause.

\textbf{Distribution shift (Figure~\ref{fig:insights} (b)).} 
%It is possible that the easy examples and difficult examples follow two distinct distributions. In such, the learned LLM forgets preference presented in the easy examples when switch to the difficult examples. We disprove this hypothesis by conducting alignment with a designated $\epsilon$-greedy sorting function. By which, we ensure that every mini-batch data are with  $\epsilon$ part from random ordering while $(1-\epsilon)$ part are sorted by the validation loss globally. However, we do not observe significant benefits compared to the greedy sorting setting. 
Another possibility is that difficult examples represent a distinct distribution, causing catastrophic forgetting when models transition from easy to difficult examples. We test this using $\epsilon$-greedy sorting: each mini-batch contains $\epsilon$ portion of randomly sampled examples and $(1-\epsilon)$ portion of examples sorted by validation loss. This ensures continuous exposure to both distributions, yet shows no improvement over the greedy sorting.

\textbf{Learning rate sensitivity (Figure~\ref{fig:insights} (c)).}
We argue that the performance drop is not caused by simply the improper learning rate. We investigate this by varying the learning rate.  However, adjusting the learning rate neither alleviates performance drops nor delays the decline, demonstrating that the issue is unrelated to improper optimization settings.

\begin{figure*}
    \centering
    \includegraphics[width=\linewidth]{fig/illustrative-figure.pdf}
    \vspace{-0.9cm}
    \caption{
    \textbf{The pipeline of \textit{Selective DPO}.} It extends DPO~\cite{rafailov2024direct} with an principled data selection process: selecting preference examples within the model's capacity. Specifically, Selective DPO comprises three steps: \underline{\textit{(1)}} Train a set of reference models using the DPO loss on different subsets of the training data. \underline{\textit{(2)}} Evaluate the reference models to compute the validation loss, which serves as a proxy for example difficulty. \underline{\textit{(3)}} Selectively align LLMs on examples with low validation loss from easy to difficult examples.
    }
    \vspace{-0.4cm}
    \label{fig:illustrative-figure}
\end{figure*}


\subsection{Difficult Example Exceeds Model's Capacity}
We hypothesize that difficult examples represent tasks beyond the model's current capabilities, requiring larger models to properly understand the nuanced preference differences.
To validate this hypothesis, we conduct experiments using Qwen-2.5 models~\cite{qwen2.5} of three sizes: 3B, 7B, and 14B. The dataset is \textit{Argilla-dpo-mix-7k}. 
Figure~\ref{fig:scaling_law} shows a clear relationship between model size and manageable example difficulty: the optimal percentage of training data (the \textit{sweet spot}) increases from 64\% for the 3B model to 81\% for the 14B model. This scaling pattern demonstrates that larger models can effectively learn from more difficult examples, confirming the direct relationship between model capacity and example difficulty threshold.


\section{Selective DPO}
\label{sec:method}

Having verified the three key claims underpinning our data selection principle, we are now well-positioned to propose an instantiated algorithm, \textit{Selective DPO}. It extends the standard DPO~\cite{rafailov2024direct} by selectively training on examples within the model’s capacity. The algorithm consists of three main steps, as illustrated in Figure~\ref{fig:illustrative-figure}:


\begin{itemize}
    \item \textbf{Train reference models.} 
    The training dataset is randomly split into two partitions. Using the standard DPO loss (Eq.~\ref{eq:dpo_loss}), SFT models are trained separately on each partition, resulting in two reference models per split. This process is repeated three times, yielding six reference models. Unlike the reference SFT model used in the DPO objective to control KL divergence, these reference models are specifically employed for computing validation loss.

    \item \textbf{Rank examples by their validation loss.} 
    The trained reference models evaluate held-out examples from their respective complementary partitions ($\mathcal{D} \backslash \hat{\mathcal{D}}$). Each example is assessed three times using different reference models, and the mean validation loss is computed to rank the examples in ascending order. 

    \item \textbf{Align with the selected data.} 
    The easiest examples, comprising the lowest $\tau$ percent of validation loss rankings, are selected for alignment training. The alignment algorithm, such as DPO, is applied exclusively to these examples. To fully utilize the difficulty ranking, examples are processed sequentially from easy to difficult.
\end{itemize}


\begin{remark}[Flexible hyper-parameter $\tau$]
    The hyper-parameter $\tau$ determines the percentage of data selected for training. Optimal $\tau$ values depend on the data difficulty distribution and the model’s capacity. In practice, $\tau$ can be tuned using a third-party evaluator such as AlpacaEval 2~\cite{dubois2024length}. 
\end{remark}

For the evaluation in the next section, we set $\tau=50$ for the \textit{UltraFeedback-binarized} dataset, based on insights from Figure~\ref{fig:base-model-struggles}. For clarity and reproducibility, pseudo-code for Selective DPO is provided in Appendix~\ref{app:selective-dpo-code}.


\setlength{\tabcolsep}{2pt}
\begin{table*}[!t]
\centering
\small 
\caption{Benchmarking results from AlpacaEval 2~\cite{dubois2024length}, Arena-Hard~\cite{li2024crowdsourced}, and MT-Bench~\cite{zheng2023judging}. In AlpacaEval 2, \textbf{WR} and \textbf{LC} indicate the win rate and length-controlled win rate against GPT-4-Turbo. In Arena-Hard, \textbf{WR} represents the win rate against GPT-4-0314, with GPT-4-Turbo serving as the evaluator. MT-Bench scores the quality of generated responses on a scale from 1 to 10, using either GPT-4 or GPT-4-Turbo as the evaluator. All results are based on full parameter fine-tuning (FPFT), except for the row labeled with LoRA~\cite{hulora}. We run this comparison on the \textit{UltraFeedback-binarized} dataset. 
}
\resizebox{\textwidth}{!}{
\begin{tabular}{lcccccccccc}
\toprule
\multirow{3}{*}{\textbf{Method}} & \multicolumn{5}{c}{\textbf{Mistral-7B-SFT}} & \multicolumn{5}{c}{\textbf{Llama-3-8B-SFT}} \\ 
\cmidrule(lr){2-6}\cmidrule(lr){7-11}
& \multicolumn{2}{c}{\textbf{AlpacaEval 2}} & \multicolumn{1}{c}{\textbf{Arena-Hard}} & \multicolumn{2}{c}{\textbf{MT-Bench}} & \multicolumn{2}{c}{\textbf{AlpacaEval 2}} & \multicolumn{1}{c}{\textbf{Arena-Hard}} & \multicolumn{2}{c}{\textbf{MT-Bench}} \\
\cmidrule(lr){2-3}\cmidrule(lr){4-4} \cmidrule(lr){5-6} \cmidrule(lr){7-8}\cmidrule(lr){9-9}\cmidrule(lr){10-11} 
& {\scriptsize \bf LC (\%)} & {\scriptsize \bf WR (\%)} & {\scriptsize \bf WR (\%)} & {\scriptsize \bf GPT-4 Turbo} & {\scriptsize \bf GPT-4} & {\scriptsize \bf LC (\%)}  & {\scriptsize \bf WR (\%)} & {\scriptsize \bf WR (\%)} & {\scriptsize \bf GPT-4 Turbo} & {\scriptsize \bf GPT-4} \\
\midrule
SFT &  8.4 & 6.2 & 1.3 & 4.8 & 6.3 & 6.2 & 4.6 & 3.3 & 5.2 & 6.6 \\
DPO~\cite{rafailov2024direct} & 15.1 & 12.5 & 10.4 & 5.9 & 7.3 & 18.2 & 15.5 & 15.9 & 6.5 & 7.7 \\
\ + Label Flipping~\cite{wang2024secrets} & 15.4 & 13.1 & 10.9 & -  & 7.3 & 19.1 & 15.9 & 16.2 & - & 7.7 \\
\ + Label Smoothing~\cite{mitchell2023note} & 15.2 & 12.7 & 10.2 & -  & 7.3 & 17.7 & 14.8 & 15.7 & - & 7.6 \\
\midrule 
RRHF~\cite{yuan2023rrhf}   & 11.6 & 10.2 &  5.8 & 5.4 & 6.7 & 12.1 & 10.1 &  6.3 & 5.8 & 7.0 \\
SLiC-HF~\cite{Zhao2023SLiCHFSL} & 10.9 &  8.9 &  7.3 & 5.8 & \textbf{7.4} & 12.3 & 13.7 &  6.0 & 6.3 & 7.6 \\
IPO~\cite{Azar2023AGT} & 11.8 & 9.4 & 7.5 & 5.5 & 7.2 & 14.4 & 14.2 & 17.8 & 6.5 & 7.4 \\
CPO~\cite{xu2024contrastive} &  9.8 &  8.9 &  6.9 & 5.4 & 6.8 & 10.8 &  8.1 &  5.8 & 6.0 & 7.4 \\
KTO~\cite{Ethayarajh2024KTOMA} & 13.1 & 9.1 & 5.6 & 5.4 & 7.0 & 14.2 & 12.4 & 12.5 & 6.3 & \textbf{7.8}  \\
ORPO~\cite{Hong2024ORPOMP} & 14.7 & 12.2 & 7.0 & 5.8 & 7.3 & 12.2 & 10.6 & 10.8 & 6.1 & 7.6 \\
R-DPO~\cite{Park2024DisentanglingLF} & 17.4 & 12.8 & 8.0 & 5.9 & \textbf{7.4} & 17.6 & 14.4 & 17.2 & 6.6 & 7.5 \\
SimPO~\cite{meng2024simpo} & 21.5 & 20.8 & 16.6 & 6.0 & 7.3 & 22.0 & 20.3 & \textbf{23.4} & 6.6 & 7.7 \\
WPO~\cite{zhou2024wpo} & 24.4 & 23.7 & \textbf{16.7} & - & \textbf{7.4} & \textbf{23.1} & \textbf{22.2} & 23.1 & - & 7.7 \\ 
\midrule
\textbf{Selective DPO} (Ours w/ LoRA) & \textbf{25.4} & \textbf{27.4} &  16.2 & \textbf{-} & 7.3 & 21.1 & {18.3} & 22.7 & \textbf{-} & \textbf{7.8} \\
\textbf{Selective DPO} (Ours) & \textbf{27.1} & \textbf{28.9} & \textbf{17.0} & \textbf{-} & \textbf{7.4} & \textbf{24.9} & \textbf{25.3} & \textbf{24.1} & \textbf{-} & \textbf{8.0} \\
\bottomrule
\end{tabular}
}
\label{tab:benchmarking-results}
%\vspace{-.5em}
\end{table*}
\begin{figure*}[!ht]
    \centering
    \includegraphics[width=\linewidth]{fig/additional_benchmarking_results.pdf}
    \vspace{-0.8cm}
    \caption{Comparison results against SimPO and WPO, with all methods tuned for their learning rates. Selective DPO (S$^{+}$DPO) demonstrates superior performance in win rate (WR) and comparable results in length-controlled win rate (LC).}
    \vspace{-0.4cm}
    \label{fig:additional-benchmarking-results}
\end{figure*}

\section{Experiments}
\label{sec:benchmarking}
We evaluate the benefits of the proposed preference data selection principle by benchmarking the Selective DPO algorithm, against state-of-the-art alignment algorithms using the formal benchmarks: \textit{AlpacaEval 2}~\cite{dubois2024length}, \textit{Arena-Hard v0.1}~\cite{li2024crowdsourced}, and \textit{MT-Bench}~\cite{zheng2023judging}. We report scores following each benchmark’s evaluation protocol. 


\subsection{Performance Comparison}

\myparagraph{Baselines.} We compare Selective DPO with DPO~\cite{rafailov2024direct} and its variants, including IPO~\cite{Azar2023AGT}, KTO~\cite{Ethayarajh2024KTOMA}, and ORPO~\cite{Hong2024ORPOMP}, borrowing their results from the SimPO paper~\cite{meng2024simpo} to ensure consistency. For SimPO and WPO~\cite{zhou2024wpo}, we rerun the released code on Llama, Gemma, and Qwen models. Hyper-parameter tuning is performed on the learning rate for all runs, see Appendix~\ref{app:experiment_details}. 
Additional baselines include techniques designed to address noisy labels, such as \textit{label flipping} and \textit{label smoothing}. Label flipping corrects mislabeled data identified by the \textit{ArmoRM} reward model, while label smoothing assumes the dataset label is correct with probability 0.6. 


\myparagraph{Results (Table~\ref{tab:benchmarking-results} and Figure~\ref{fig:additional-benchmarking-results}).} Table~\ref{tab:benchmarking-results} compares results on the Mistral-7B~\cite{jiang2023mistral} and Llama-3-8B~\cite{llama3modelcard} models. Label flipping yields only marginal gains, supporting our insight that difficult examples are not necessarily data errors. In contrast, Selective DPO, which carefully selects 50\% of the training data, significantly outperforms all baselines across all three benchmarks, demonstrating the strength of our data selection principle for alignment tasks. 
Figure~\ref{fig:additional-benchmarking-results} extends the comparison to Gemma-2-9B~\cite{team2024gemma} and Qwen-2.5-7B~\cite{qwen2.5}, showing exceptional performance in win rate (WR) on AlpacaEval 2 and comparable performance on length-controlled win rate (LC). The slightly lower performance on LC is consistent with results in Table~\ref{tab:benchmarking-results}, where Selective DPO demonstrates better performance under WR. 

We emphasize that our goal is not to propose the best ever alignment algorithm, but to verify the proposed data selection principle for alignment: selecting examples that align with the model's capacity. The length exploitation issue, while beyond the scope of this paper, could potentially be addressed using techniques from SimPO~\cite{meng2024simpo} or WPO~\cite{zhou2024wpo}, which we leave as future work.

\begin{figure*}
    \centering
    % \begin{subfigure}[t]{0.24\textwidth}
    %     \centering
    %     \includegraphics[width=\textwidth]{fig/ablation-study-reference-models.pdf}
    %     \vspace{-0.6cm}
    %     \caption{}
    % \end{subfigure}
    %     \centering
    % \begin{subfigure}[t]{0.24\textwidth}
    %     \centering
    %     \includegraphics[width=\textwidth]{fig/ablation-study-tau.pdf}
    %     \vspace{-0.6cm}
    %     \caption{}
    % \end{subfigure}
    %     \centering
    % \begin{subfigure}[t]{0.24\textwidth}
    %     \centering
    %     \includegraphics[width=\textwidth]{fig/indepth-analysis-nlls.pdf}
    %     \vspace{-0.6cm}
    %     \caption{}
    % \end{subfigure}
    % \begin{subfigure}[t]{0.24\textwidth}
    %     \centering
    %     \includegraphics[width=\textwidth]{fig/indepth-analysis-reward-margin.pdf}
    %     \vspace{-0.6cm}
    %     \caption{}
    % \end{subfigure}
    \includegraphics[width=\textwidth]{fig/ablation-study-and-indepth-analysis.pdf}
    \vspace{-0.8cm}
    \caption{\textbf{Hyper-parameter study and in-depth analysis of Selective DPO.}  \underline{\textit{(a)}}: Relationship between the number of reference models and performance. \underline{\textit{(b)}}: Performance with different percentages of selected easy examples. \underline{\textit{(c)}}: Negative log-likelihoods distributions on the generated responses. \underline{\textit{(d)}}: Reward margin distributions of the implicit reward models.}
    \label{fig:ablation-and-analysis}
    \vspace{-0.4cm}
\end{figure*}
\begin{figure}
    \centering
    \includegraphics[width=\linewidth]{fig/weak-to-strong-curricula.pdf}
    \vspace{-0.8cm}
    \caption{\textbf{Weak-to-strong curriculum under-performs.}  Aligning a 7B model with examples ordered by 3B reference models yields compromised results.}
    \label{fig:weak-to-strong-curriculum}
    \vspace{-0.4cm}
\end{figure}


\subsection{Hyper-Parameter Study}
Selective DPO introduces two implicit hyper-parameters. \textbf{Number of reference models (Figure~\ref{fig:ablation-and-analysis} (a))}: Increasing the number of reference models used to compute the validation loss improves performance on \textit{AlpacaEval 2} (LC). However, considering computational costs, training six reference models strikes a balance between performance and efficiency. 
\textbf{Percentage of selected easy examples (Figure~\ref{fig:ablation-and-analysis} (b))}: 
Increasing $\tau$ incorporates examples exceeding the model’s capacity, leading to performance degradation, while excessively low values limit training to the simplest examples, also resulting in suboptimal performance.

\subsection{In-Depth Analysis of DPO vs. Selective DPO}
Selective DPO outperforms DPO in terms of likelihood distribution and reward margin distribution. As shown in Figure~\ref{fig:ablation-and-analysis}(c), Selective DPO achieves a distribution of negative log-likelihoods (NLLs) closer to zero on test prompts, indicating higher confidence in generated responses. Additionally, the implicit reward model learned by Selective DPO exhibits better accuracy and larger reward margins on testing examples (Figure~\ref{fig:ablation-and-analysis}(d)). Together, these results suggest that by filtering out overly difficult examples, Selective DPO produces more robust reward models and reduces undesired hallucinations.

\subsection{Weak-to-Strong Curriculum}
To investigate whether difficult examples can be identified using smaller reference models, we compare alignment experiments where a 7B SFT model is trained with its own curriculum versus a curriculum derived from a smaller 3B model. 
Results in Figure~\ref{fig:weak-to-strong-curriculum} show moderate benefits from the smaller model’s curriculum, though slightly inferior to the model’s own curriculum. This suggests that while smaller models can provide insights, data selection remains more effective when tailored to the target model’s capacity.

% \section{Related Work}
\label{sec:RelatedWorks}

\paragraph{Code Generation Benchmarks}
In code generation tasks,
ambiguous user instructions hinder the evaluation of code
suggestions generated by the model. Since the cause of
ambiguity is missing details, clarifying questions become
neessary~\cite{mu2023clarifygptempoweringllmbasedcode}. Interactive, test-driven workflows mitigate this ambiguity by first generating test cases aligned with user expectations, which users validate before code generation~\cite{lahiri2023interactivecodegenerationtestdriven}. Extensions of this approach employ runtime techniques to generate, mutate, and rank candidate code suggestions and test cases based on user feedback~\cite{Fakhoury_2024_LLM_Code_Gen}. Although effective, these workflows can burden users, highlighting the need to minimize intervention to essential cases. 

\paragraph{Interactive ML Systems}
In task-oriented settings, ambiguity between generated outputs and user expectations remains a challenge. AmbigNLG addresses this by introducing a taxonomy of instruction ambiguities and applying targeted disambiguation based on the identified ambiguity type~\cite{ambignlp}. These ambiguities include unclear output lengths, mandatory keywords, and contextual nuances in instructions. NoisyToolBench~\cite{NoisyToolBench} offers a dataset for evaluating LLM tool use with ambiguous instructions, though it focuses on simpler tasks. Reinforcement learning frameworks like ReHAC balance user interaction by modeling optimal intervention points~\cite{feng2024largelanguagemodelbasedhumanagent}, but more effective strategies are needed for complex, multi-step workflows.

\paragraph{LLMs and Ambiguity}
The current state-of-the-art LLMs
are not inherently trained to handle ambiguity through
user interaction~\cite{zhang2024clamberbenchmarkidentifyingclarifying}, but, their instruction
tuning enables improved performance with prompt engineering~\cite{white2023prompt}. Ambiguity detection has been
tackled with uncertainty estimation to measure the utility
of seeking clarification~\cite{zhang2023clarifynecessaryresolvingambiguity, park2024claraclassifyingdisambiguatinguser}. Meanwhile, the quality of clarifying questions and
the resulting performance remain critical to overall success~\cite{learning-good-questions, clarifydelphi, kuhn2023clamselectiveclarificationambiguous}. Despite advances, state-of-the-art techniques such as few-shot prompting and Chain-of-Thought
reasoning offer limited relief in ambiguous scenarios~\cite{zhang2024clamberbenchmarkidentifyingclarifying}. Self-disambiguation uses the internal knowledge of a
model to reduce query ambiguity~\cite{keluskar2024llmsunderstandambiguitytext,sterner2022explaining,sumanathilaka2024llmsassistambiguityquantitative}. For example,
Alignment with Perceived Ambiguity (APA) employs self-
disambiguation to quantify perceived ambiguity using information gain, improving the model’s processing of such
inputs ~\cite{kim2024aligninglanguagemodelsexplicitly}. Although inference-only
methods are cost-effective, they are less robust than training-based approaches for handling ambiguity.

\section{Conclusion}
We reveal a tradeoff in robust watermarks: Improved redundancy of watermark information enhances robustness, but increased redundancy raises the risk of watermark leakage. We propose DAPAO attack, a framework that requires only one image for watermark extraction, effectively achieving both watermark removal and spoofing attacks against cutting-edge robust watermarking methods. Our attack reaches an average success rate of 87\% in detection evasion (about 60\% higher than existing evasion attacks) and an average success rate of 85\% in forgery (approximately 51\% higher than current forgery studies). 
\bibliography{reference}

\newpage
\appendix
\section{Appendix}

\subsection{Conversational agent prompts for generating stable diffusion prompts in art-making phase}

\textbf{Role:} You will be able to capture the essence of the sessions and drawings in the recordings based on the art therapy session recordings I have given you and summarize them into a short sentence that will be used to guide the PROMPT for the Stable Diffusion model.

\vspace{0.5em} % 添加一些垂直间距

\textbf{Example input:}

\begin{itemize}[leftmargin=*]
    \item \textbf{USER:} [user-drawn] I drew the ocean. [canvas content] There is nothing on the canvas right now.
    \item \textbf{ASSISTANT:} What kind of ocean is this?
    \item \textbf{USER:} [user-drawn] I drew grass. [canvas content] Now there is an ocean on the canvas.
    \item \textbf{ASSISTANT:} What kind of grass is this?
    \item \textbf{USER:} [user-drawn] I drew the sky. [canvas content] Now there is grass and ocean on the canvas.
    \item \textbf{ASSISTANT:} What kind of sky is this?
    \item \textbf{USER:} [user-drawn] I drew mountains. [canvas content] Now there is sky, grass, and ocean on the canvas.
    \item \textbf{ASSISTANT:} What kind of mountain is this?
    \item \textbf{USER:} [user-drawn] I drew clouds. [canvas content] Now there is sky, mountain, grass, and ocean on the canvas.
    \item \textbf{ASSISTANT:} What kind of cloud is this?
    \item \textbf{USER:} [user dialogue] Colorful clouds, emerald green mountains and grass, choppy ocean
\end{itemize}

\vspace{0.5em} % 添加一些垂直间距

\textbf{Task:}

\begin{enumerate}[label=\textbf{Step \arabic*:}]
    \item \textbf{[Step 0]} Read the given transcript of the art therapy session, focusing on the content of \texttt{user: [user drawing]} and \texttt{user: [user dialog]}: Go to \textbf{[Step 1]}.
    \item \textbf{[Step 1]} Based on the input, find the last entry of user's input with \texttt{[canvas content]}, find the keywords of the screen elements that the canvas now contains (in the example input above, it is: sky, grass, sea), separate the keywords of each element with a comma, and add them to the generated result. Examples: [keyword1], [keyword2], [keyword3], \dots, [keyword n].
    \item \textbf{[Step 2]} Find whether there are more specific descriptions of the keywords of the painting elements in \texttt{[Step 1]} in \texttt{[User Dialog]} according to the input. If not, this step ends into \textbf{[Step 3]}; if there are, combine these descriptions and the keywords corresponding to the descriptions into a new descriptive phrase, and replace the previous keywords with the new phrases. Examples: [description of keyword 1] [keyword 1], [keyword 2 description of keyword 2], [description of keyword 3], \dots. Based on the above example input, the output is: rough sea, lush grass, blue sky.
    \item \textbf{[Step 3]} Based on the input, find out if there is a description of the painting style in the \texttt{[User Dialog]} in the dialog record, and if there is, add the style of the picture as a separate phrase after the corresponding phrase generated in \texttt{[Step 2]}, separated by commas. For example: [description of keyword 1] [keyword 1], [description of keyword 2] [keyword 2], \dots, [screen style phrase 1], [screen style phrase 2], [screen style phrase 3], \dots, [Picture Style Phrase n].
\end{enumerate}

\vspace{0.5em} % 添加一些垂直间距

\textbf{Output:} 

Only need to output the generated result of \textbf{[Step 3]}.

\vspace{0.5em} % 添加一些垂直间距

\textbf{Example output:} 

\emph{Rough sea, lush grass}

\subsection{Conversational agent prompts for discussion phase}

\textbf{Role:} <therapist\_name>, Professional Art Therapist

\textbf{Characteristics:} Flexible, empathetic, honest, respectful, trustworthy, non-judgmental.

\vspace{0.5em} % 添加垂直间距

\textbf{Task:} Based on the user's dialogic input, start sequentially from step [A], then step [B], to step [C], step [D], step [E] \dots Step [N] will be asked in a dialogical order, and after step [N], you can go to \textbf{Concluding Remarks}. You can select only one question to be asked at a time from the sample output display of step [N]! You have the flexibility to ask up to one round of extended dialog questions at step [N] based on the user's answers. Lead the user to deeper self-exploration and emotional expression, rather than simply asking questions.

\vspace{0.5em} % 添加垂直间距

\textbf{Operational Guidelines:}

\begin{enumerate}
    \item You must start with the first question and proceed sequentially through the steps in the conversational process (step [A], step [B], step [C], step [D], step [E], \dots, step [N]).
    \item Do not include references like step '[A]', step '[B]' directly in your reply text.
    \item You may include one round of extended dialog questions at any step [N] depending on the user's responses and situation. After that, move on to the next step.
    \item Always ensure empathy and respect are present in your responses, e.g., re-telling or summarizing the user's previous answer to show empathy and attention.
\end{enumerate}

\vspace{0.5em} % 添加垂直间距

\textbf{Therapist’s Configuration:}

\textbf{Principle 1:}  
\textit{Sample question:} How are you feeling about what you are creating in this moment?

\vspace{0.5em}

\textbf{Principle 2:}  
\textit{Sample question:} Can you share with me what this artwork represents to you personally? 

\vspace{0.5em}

\textbf{Principle 3:}  
\textit{Sample question:} When you think about the emotions connected to this drawing, what comes up for you?

\vspace{0.5em}

\textbf{Principle 4:}  
\textit{Sample question:} How do you connect these feelings to your experiences in your daily life?

\vspace{0.5em} % 添加垂直间距

\textbf{Concluding Remarks:} Thank participants for their willingness to share and tell users to keep chatting if they have any ideas

\vspace{1em} % 添加额外的间距

\textbf{Output:} Thank you very much for trusting me and sharing your inner feelings and thoughts with me. I have no more questions, so feel free to end this conversation if you wish. Or, if you wish, we can continue to talk.

\subsection{AI summary prompts}

\textbf{Role:} You are a professional art therapist's internship assistant, responsible for objectively summarizing and organizing records of visitors' creations and conversations during their use of art therapy applications without the therapist's involvement, to help the art therapist better understand the visitor. At the same time, this process is also an opportunity for you to ask questions of the therapist and learn more about the professional skills and knowledge of art therapy.

\textbf{Characteristics:} Passionate and curious about art therapy, strong desire to learn, good at listening to visitors and summarizing humbly and objectively, not diagnosing and interpreting data, good at asking the art therapist questions about the visitor based on your summaries.

\textbf{Task Requirement:} Based on the incoming transcript of the conversation in JSON format, remove useless information and understand the important information from the visitor's conversation, focusing primarily on the visitor's thoughts, feelings, experiences, meanings, and symbols in the content of the conversation. Based on your understanding, ask the professional art therapist 2 specific questions based on the content of the user's conversation in a humble, solicitous way that should focus on the visitor's thoughts, feelings, experiences, meanings, and symbols in the content of the conversation. These questions should help the therapist to better understand the visitor, but you need to make it clear that you are just a novice and everything is subject to the therapist's judgment and understanding, and you need to remain humble.

\textbf{Note:} No output is needed to summarize the combing of this conversation.



\bibliographystyle{icml2025}


%%%%%%%%%%%%%%%%%%%%%%%%%%%%%%%%%%%%%%%%%%%%%%%%%%%%%%%%%%%%%%%%%%%%%%%%%%%%%%%
%%%%%%%%%%%%%%%%%%%%%%%%%%%%%%%%%%%%%%%%%%%%%%%%%%%%%%%%%%%%%%%%%%%%%%%%%%%%%%%
% APPENDIX
%%%%%%%%%%%%%%%%%%%%%%%%%%%%%%%%%%%%%%%%%%%%%%%%%%%%%%%%%%%%%%%%%%%%%%%%%%%%%%%
%%%%%%%%%%%%%%%%%%%%%%%%%%%%%%%%%%%%%%%%%%%%%%%%%%%%%%%%%%%%%%%%%%%%%%%%%%%%%%%
% \newpage
\appendix
\section{Appendix}

\subsection{Conversational agent prompts for generating stable diffusion prompts in art-making phase}

\textbf{Role:} You will be able to capture the essence of the sessions and drawings in the recordings based on the art therapy session recordings I have given you and summarize them into a short sentence that will be used to guide the PROMPT for the Stable Diffusion model.

\vspace{0.5em} % 添加一些垂直间距

\textbf{Example input:}

\begin{itemize}[leftmargin=*]
    \item \textbf{USER:} [user-drawn] I drew the ocean. [canvas content] There is nothing on the canvas right now.
    \item \textbf{ASSISTANT:} What kind of ocean is this?
    \item \textbf{USER:} [user-drawn] I drew grass. [canvas content] Now there is an ocean on the canvas.
    \item \textbf{ASSISTANT:} What kind of grass is this?
    \item \textbf{USER:} [user-drawn] I drew the sky. [canvas content] Now there is grass and ocean on the canvas.
    \item \textbf{ASSISTANT:} What kind of sky is this?
    \item \textbf{USER:} [user-drawn] I drew mountains. [canvas content] Now there is sky, grass, and ocean on the canvas.
    \item \textbf{ASSISTANT:} What kind of mountain is this?
    \item \textbf{USER:} [user-drawn] I drew clouds. [canvas content] Now there is sky, mountain, grass, and ocean on the canvas.
    \item \textbf{ASSISTANT:} What kind of cloud is this?
    \item \textbf{USER:} [user dialogue] Colorful clouds, emerald green mountains and grass, choppy ocean
\end{itemize}

\vspace{0.5em} % 添加一些垂直间距

\textbf{Task:}

\begin{enumerate}[label=\textbf{Step \arabic*:}]
    \item \textbf{[Step 0]} Read the given transcript of the art therapy session, focusing on the content of \texttt{user: [user drawing]} and \texttt{user: [user dialog]}: Go to \textbf{[Step 1]}.
    \item \textbf{[Step 1]} Based on the input, find the last entry of user's input with \texttt{[canvas content]}, find the keywords of the screen elements that the canvas now contains (in the example input above, it is: sky, grass, sea), separate the keywords of each element with a comma, and add them to the generated result. Examples: [keyword1], [keyword2], [keyword3], \dots, [keyword n].
    \item \textbf{[Step 2]} Find whether there are more specific descriptions of the keywords of the painting elements in \texttt{[Step 1]} in \texttt{[User Dialog]} according to the input. If not, this step ends into \textbf{[Step 3]}; if there are, combine these descriptions and the keywords corresponding to the descriptions into a new descriptive phrase, and replace the previous keywords with the new phrases. Examples: [description of keyword 1] [keyword 1], [keyword 2 description of keyword 2], [description of keyword 3], \dots. Based on the above example input, the output is: rough sea, lush grass, blue sky.
    \item \textbf{[Step 3]} Based on the input, find out if there is a description of the painting style in the \texttt{[User Dialog]} in the dialog record, and if there is, add the style of the picture as a separate phrase after the corresponding phrase generated in \texttt{[Step 2]}, separated by commas. For example: [description of keyword 1] [keyword 1], [description of keyword 2] [keyword 2], \dots, [screen style phrase 1], [screen style phrase 2], [screen style phrase 3], \dots, [Picture Style Phrase n].
\end{enumerate}

\vspace{0.5em} % 添加一些垂直间距

\textbf{Output:} 

Only need to output the generated result of \textbf{[Step 3]}.

\vspace{0.5em} % 添加一些垂直间距

\textbf{Example output:} 

\emph{Rough sea, lush grass}

\subsection{Conversational agent prompts for discussion phase}

\textbf{Role:} <therapist\_name>, Professional Art Therapist

\textbf{Characteristics:} Flexible, empathetic, honest, respectful, trustworthy, non-judgmental.

\vspace{0.5em} % 添加垂直间距

\textbf{Task:} Based on the user's dialogic input, start sequentially from step [A], then step [B], to step [C], step [D], step [E] \dots Step [N] will be asked in a dialogical order, and after step [N], you can go to \textbf{Concluding Remarks}. You can select only one question to be asked at a time from the sample output display of step [N]! You have the flexibility to ask up to one round of extended dialog questions at step [N] based on the user's answers. Lead the user to deeper self-exploration and emotional expression, rather than simply asking questions.

\vspace{0.5em} % 添加垂直间距

\textbf{Operational Guidelines:}

\begin{enumerate}
    \item You must start with the first question and proceed sequentially through the steps in the conversational process (step [A], step [B], step [C], step [D], step [E], \dots, step [N]).
    \item Do not include references like step '[A]', step '[B]' directly in your reply text.
    \item You may include one round of extended dialog questions at any step [N] depending on the user's responses and situation. After that, move on to the next step.
    \item Always ensure empathy and respect are present in your responses, e.g., re-telling or summarizing the user's previous answer to show empathy and attention.
\end{enumerate}

\vspace{0.5em} % 添加垂直间距

\textbf{Therapist’s Configuration:}

\textbf{Principle 1:}  
\textit{Sample question:} How are you feeling about what you are creating in this moment?

\vspace{0.5em}

\textbf{Principle 2:}  
\textit{Sample question:} Can you share with me what this artwork represents to you personally? 

\vspace{0.5em}

\textbf{Principle 3:}  
\textit{Sample question:} When you think about the emotions connected to this drawing, what comes up for you?

\vspace{0.5em}

\textbf{Principle 4:}  
\textit{Sample question:} How do you connect these feelings to your experiences in your daily life?

\vspace{0.5em} % 添加垂直间距

\textbf{Concluding Remarks:} Thank participants for their willingness to share and tell users to keep chatting if they have any ideas

\vspace{1em} % 添加额外的间距

\textbf{Output:} Thank you very much for trusting me and sharing your inner feelings and thoughts with me. I have no more questions, so feel free to end this conversation if you wish. Or, if you wish, we can continue to talk.

\subsection{AI summary prompts}

\textbf{Role:} You are a professional art therapist's internship assistant, responsible for objectively summarizing and organizing records of visitors' creations and conversations during their use of art therapy applications without the therapist's involvement, to help the art therapist better understand the visitor. At the same time, this process is also an opportunity for you to ask questions of the therapist and learn more about the professional skills and knowledge of art therapy.

\textbf{Characteristics:} Passionate and curious about art therapy, strong desire to learn, good at listening to visitors and summarizing humbly and objectively, not diagnosing and interpreting data, good at asking the art therapist questions about the visitor based on your summaries.

\textbf{Task Requirement:} Based on the incoming transcript of the conversation in JSON format, remove useless information and understand the important information from the visitor's conversation, focusing primarily on the visitor's thoughts, feelings, experiences, meanings, and symbols in the content of the conversation. Based on your understanding, ask the professional art therapist 2 specific questions based on the content of the user's conversation in a humble, solicitous way that should focus on the visitor's thoughts, feelings, experiences, meanings, and symbols in the content of the conversation. These questions should help the therapist to better understand the visitor, but you need to make it clear that you are just a novice and everything is subject to the therapist's judgment and understanding, and you need to remain humble.

\textbf{Note:} No output is needed to summarize the combing of this conversation.





%%%%%%%%%%%%%%%%%%%%%%%%%%%%%%%%%%%%%%%%%%%%%%%%%%%%%%%%%%%%%%%%%%%%%%%%%%%%%%%
%%%%%%%%%%%%%%%%%%%%%%%%%%%%%%%%%%%%%%%%%%%%%%%%%%%%%%%%%%%%%%%%%%%%%%%%%%%%%%%


\end{document}
