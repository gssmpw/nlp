

% 往 inference time 上靠

% \section{Analysis}
\subsection{Efficiency Analysis}
% To show the efficiency of our methods, we compare the average retrieve counts over the test samples on 2WikiMultihopQA and HotpotQA. 
% As shown in Table~\ref{tab:efficiency}, DRAGIN的检索次数受到阈值和不同数据集的影响很大,特别的对于不同的数据需要不同的阈值,as we know 2WikiMultihopQA的数据需要的evidence比 hotpotqa包含的数据需要的evidence要多一些,由于它有更多复杂的多跳问题,然后,对比相同的threshold,hotpotqa调用检索的次数却比2wiki的多,which展现了它的不鲁棒性。
% 相反的,我们的方法普遍使用了较少次数的检索,同时检索的次数和query的难度相关度较大

% To demonstrate the efficiency of our methods, we compare the average retrieval counts across test samples on 2WikiMultihopQA and HotpotQA. 

% As shown in Table~\ref{tab:efficiency}, we statistics the retrieval counts and answer accuracy for both the DRAGIN baseline and our method.
% Our method achieves a great balance between effectiveness and efficiency.
% In contrast, the retrieval counts of DRAGIN are significantly influenced by the threshold and the specific dataset.
% Moreover, different datasets require different thresholds due to the complexity of their queries.
% For instance, 2WikiMultihopQA requires more evidence because it contains more complex multi-hop questions compared to HotpotQA.
% However, when DRAGIN uses the same threshold, HotpotQA requires more retrievals than 2WikiMultihopQA, highlighting DRAGIN's lack of robustness.
% Conversely, our method consistently uses fewer retrievals, and the number of retrievals is more closely related to the difficulty of the queries.

\begin{table}[t!]
\centering
    \scriptsize
    \setlength{\tabcolsep}{0.0035\linewidth}
    \caption{\textbf{Computational efficiency of EvSSC across different datasets.} Memory denotes training memory usage.}
    %\vskip-1ex
\setlength{\tabcolsep}{4pt} %5pt 设定列之间的宽度
\resizebox{\columnwidth}{!}{%
\begin{tabular}{l|>{\columncolor{gray!10}}l>{\columncolor{blue!8}}l|>{\columncolor{gray!10}}l>{\columncolor{blue!8}}l}
\toprule
\textbf{ } & \textbf{VoxFormer-S} & \textbf{EvSSC (VoxFormer)} & \textbf{SGN-S} & \textbf{EvSSC (SGN)}\\
\midrule\midrule
\multicolumn{5}{c}{\textit{DSEC-SSC}} \\ \midrule 
\textbf{mIoU} &25.62 & 26.34 & 29.06 & 29.55\\ 
\textbf{IoU}  & 47.25 & 47.29 & 43.70 & 43.99\\ 
\textbf{Memory}  & 9.74G & 10.52G & 10.19G & 10.70G\\ 
\textbf{Latency} & 0.732s & 0.836s & 0.941s & 1.193s \\\midrule 
\multicolumn{5}{c}{\textit{SemanticKITTI-E}} \\ \midrule 
\textbf{mIoU} & 12.86 & 13.61 & 14.55 & 15.15\\ 
\textbf{IoU}  & 44.42 & 45.01 & 43.60 & 43.17\\ 
\textbf{Memory} & 14.87G & 15.78G & 15.29G & 17.79G \\ 
\textbf{Latency}  & 0.996s & 1.005s & 0.855s &1.005s\\ \midrule 
\multicolumn{5}{c}{\textit{SemanticKITTI-C Shot Noise}} \\ \midrule 
\textbf{mIoU} & 8.29 & 12.64 & 13.62 & 14.32\\ 
\textbf{IoU}  & 44.26 & 45.04 & 42.05 & 42.54\\ 
\textbf{Memory} & 14.87G & 15.78G & 15.29G & 17.79G \\ 
\textbf{Latency}  & 0.996s & 1.005s & 0.855s &1.005s\\
\bottomrule
\end{tabular}
}
\label{table:efficiency}
%\vskip-3ex
\end{table}



To demonstrate the efficiency of our method, we compare the average number of retrievals across test samples on 2WikiMultihopQA and HotpotQA. As shown in Table~\ref{tab:efficiency}, our method has two main advantages:

\begin{itemize}
    \item \textbf{Fewer Retrievals Required} Our method requires significantly fewer retrievals than the DRAGIN baseline while maintaining high answer accuracy. This efficiency highlights our method's ability to achieve effectiveness with less computational effort.
    \item \textbf{Better Correlation with Query Difficulty} The number of retrievals in our method is more closely related to the difficulty of the queries. Complex queries naturally demand more retrievals, and our method adjusts accordingly. In contrast, DRAGIN's retrieval counts are significantly influenced by threshold settings and vary inconsistently across different datasets. For instance, even though 2WikiMultihopQA contains more complex multi-hop questions that typically require more evidence, DRAGIN retrieves more times on HotpotQA when using the same threshold, underscoring its lack of robustness.
\end{itemize}

In summary, our method not only reduces the number of retrievals needed but also aligns the retrieval effort with the actual difficulty of the queries, demonstrating both efficiency and adaptability.


\subsection{Self-Evolving Trajectory Analysis}
% 三个阶段检索次数分布线箱图
% In this section, we visualize how the model enhances its RAG capability across different stages. We first 展示了模型的检索次数分布在cot、stage I和stage II. Then 我们展示了在不同stage模型的检索次数的变化趋势

we conduct a comprehensive analysis to understand how our approach progressively enhances the model's RAG capabilities, with a particular focus on retrieval efficiency. Our analysis consists of two key aspects: 1) We examine the distribution of retrieval counts across three stages (CoT baseline, Stage I, and Stage II) to understand the evolution of the model's retrieval behavior. 2) We track the temporal dynamics of retrieval counts through different training stages to reveal the optimization trajectory of our approach.

\paragraph{Retrieval Counts Distribution} As shown in Figure~\ref{fig:retrieve-counts}, we analyze the distribution of retrieval counts across three stages: CoT baseline, Stage I, and Stage II. Compared to the CoT baseline, Stage I demonstrates a significant reduction in retrieval frequency, indicating the model's learned ability to make more efficient retrieval decisions. Stage II further optimizes the retrieval strategy, showing a more pronounced shift towards minimal retrieval attempts while maximizing the utilization of internal knowledge.

\begin{figure}
    \centering
    \includegraphics[width=0.9\linewidth]{figure/retrieve_counts_distribution.pdf}
    \caption{Retrieval counts distribution across different stages on 2WikiMultihopQA dataset.}
    \label{fig:retrieve-counts}
\end{figure}


\paragraph{Retrieval Counts Trajectory}


% As shown in Figure~\ref{}, we illustrate the trajectory of Retrieve counts for the differenct stages. Speciffically, we seperate them by whether they answer correctly in the Stage II. Based on it, we can find that 1. 对于模型会的问题,随着不同stage的evolve,模型逐渐变得倾向于使用内部知识回答,figure(a)中,大多数数据逐渐呈现从检索次数多到少的流动趋势,而极少数数据的检索次数增多,which may due to the intrinsic uncerntainty of model. 
% 2. 对于模型回答错误的问题,随着不同stage的演进,模型倾向于探索多的次数来尝试回答,例如figure(b)中流向dpo3,4,5的数据比figre(a)中多。

\begin{figure}[htbp]
    \centering
    \begin{minipage}[b]{0.48\linewidth}
        \centering
        \includegraphics[width=\linewidth]{figure/correct.pdf}
        % \caption{Retrieval counts trajectory during training}
        % \label{fig:trajectory-counts}
        (a) Correct
    \end{minipage}
    \hfill
    \begin{minipage}[b]{0.48\linewidth}
        \centering
        \includegraphics[width=\linewidth]{figure/incorrect.pdf}
        % \caption{Answer accuracy trajectory during training}
        % \label{fig:trajectory-acc}
        (b) Incorrect
    \end{minipage}
    \caption{Training trajectories showing the evolution of retrieval counts and answer accuracy across different stages.}
    \label{fig:trajectory}
\end{figure}


As shown in Figure~\ref{fig:trajectory}, we depict the trajectory of retrieval counts across different stages of our model. We specifically separate the data based on whether the model answers correctly in Stage II. Based on it, we observe the following:

For questions that the model answers correctly, the model progressively relies more on its internal knowledge as it evolves through the stages. In Figure~\ref{fig:trajectory} (a), most data points exhibit a trend of decreasing retrieval counts over time, indicating a shift from external retrieval to internal reasoning. Only a small fraction of data points show increased retrieval counts, which may be due to the model's intrinsic uncertainty.

For questions that the model answers incorrectly, the model tends to increase its number of retrieval attempts in subsequent stages to find the correct answer. In Figure~\ref{fig:trajectory} (b), there is a higher proportion of data flowing into stages with more retrievals (DPO3, DPO4, DPO5) compared to Figure~\ref{fig:trajectory} (a). This suggests that when the model struggles to provide the correct answer, it compensates by exploring more retrievals.

These findings highlight that our model not only reduces retrieval counts when confident but also adjusts its retrieval efforts based on the difficulty of the queries, demonstrating an adaptive retrieval strategy.


% 召回率分析?

% impact of 两类偏序数据的比例
\subsection{Adaptability to Different Retrievers}

% Table generated by Excel2LaTeX from sheet 'dpr'
\begin{table}[htbp]
  \centering
  \resizebox{\linewidth}{!}{
    \begin{tabular}{cccc}
    \toprule
    Dataset      & Method & EM    & F1 \\
    \midrule
    \multirow{3}[1]{*}{Hotpot} & SRAG-BM25 & 42.60  & 55.10  \\
          & FLARE-DPR & 15.50 &  24.63  \\
          & SRAG-DPR & 30.80  & 41.17  \\ \midrule
    \multirow{3}[1]{*}{2WikiMultihopQA} & SRAG-BM25 & 48.60  & 54.39  \\
          & FLARE-DPR & 22.50 & 29.52    \\
          & SRAG-DPR &  33.20 & 37.80   \\
    \bottomrule
    \end{tabular}}%
    \caption{Add caption}
  \label{tab:retriever}%
\end{table}%


\section{Case Study on SafeSPLE}
We now demonstrate via a case study one way to implement a SafeSPL and parameterized safety cases.  The first part of our process is a hazard analysis. We then build a feature model. The features are then used to parameterize our safety case. Lastly, we can generate safety-case instances as requested for any of the concrete combinations of features.  

\subsection{Hazard Analysis}
To begin the SafeSPLE process for a UAS flight, we analyze the hazards of that flight, which is an important first step before creating a safety case \cite{Knig12}. A hazard is a state or event that can potentially result in an accident \cite{ericson2015hazard}. In this work, we do not describe this part of the process in depth but rather list a few of the key hazards we identified. We utilized several sources to create our list of hazards. First, we referenced several papers describing hazard analysis or safety cases for sUAS flights \cite{denpai2016, clodenpai2017, sora}. Next, we discussed sUAS hazards with colleagues and experts who have studied sUAS and flown them. This investigation gave us an extensive list of hazards, which was too long and broad to include in this paper. We narrowed down this extensive list to focus on the following hazards. 

\begin{itemize}
    \item Too much precipitation
    \item Insufficient visibility
    \item Temperatures outside the operating specifications of the sUAS
    \item Wind gusts outside the operating specifications of the sUAS
    \item Insufficient battery for the mission
\end{itemize}

The hazards above are not intended to be fully described or defined, and we do not include prevention or recovery controls or escalation factors for any of these hazards (see \cite{denpai2016} for a more in-depth discussion of hazard analysis). The ultimate consequences of each of the above hazards are generally either loss of separation from the ground or loss of separation from other air traffic. Either of these consequences could lead to the destruction of property, injury, or death. A complete risk analysis of these consequences is likewise beyond the scope of this paper. We illustrate our family-based approach below using a subset of the identified hazards in order to show how the parameterized safety case addresses the hazards for different sUAS.       

\subsection{Feature Model}

\begin{figure}[ht]
    \centering
    \includegraphics[width=.8\textwidth]{Figures/safety-case-blow-up.pdf}
    \caption{Two parts of the feature model that we focus on for this case study.}
    \label{fig:feature_model_focus}
\end{figure}

The next step in the SafeSPLE process is to create a partial feature model that could apply to a wide variety of sUAS models and missions in controlled airspace as described in Section \ref{sec:SafeSPLE} and Figure \ref{fig:featuremodel}. Since this feature model includes information about the pilot, airspace, mission, vehicle, and weather (among other things), it allows for a wide variety of different types of parameters to be used in our parameterized safety case. In figure \ref{fig:feature_model_focus} we show the two parts of the feature model that are the focus of our safety cases here - the pilot and the weather. These parameterized safety cases are described in the next section. 


\subsection{Parameterized Safety Case}

\begin{figure}[ht]
    \centering
    \includegraphics[width=.7\textwidth]{Figures/pilot_only.pdf}
    \caption{Pilot Safety Case: A safety case based only on whether the pilot is certified and has sufficient experience.}
    \label{fig:pilot_only}
\end{figure}

The next step in our case study (based on SafeSPLE) is to create two illustrative parameterized safety cases for our controlled airspace. The first safety case, seen in Figure \ref{fig:pilot_only}, is based solely on the pilot. It checks whether the pilot is certified and has sufficient flight hours. We assume that in non-commercial airspaces, flight regulations would trust a certified pilot with sufficient reputation (i.e., no significant history of problems) to perform safety checks consistent with the lower-level details of our safety cases. In other words, the pilot is in charge of ensuring a safe flight in whatever airspace they are in. Regulators often do not exclude pilots legally allowed to be in the airspace unless there is some serious prior issue \cite{FAA_TRUST, FAA_part107}. So it is our belief that any UAS Traffic Management system will likely allow certified pilots to enter the airspace unless it has some reason not to.

As shown in Figure \ref{fig:pilot_only}, our safety case checks to see if the pilot is certified to fly their sUAS, here represented using the FAA's Part 107 certification \cite{FAA_part107}. We also check to see if the pilot has sufficient flight hours to be competent to complete this flight, which is something that our managed airspace should know.  In the future, this flight-hours check might be replaced or augmented with different checks, such as the pilot's score on a competency-reputation metric, future certifications, or temporary notices to pilots that the FAA might put out. If evidence of these checks confirms that the pilot is certified and has sufficient experience to enter the controlled airspace, the associated strategy node (S1) in the safety case is satisfied.

\begin{figure}[ht]
    \centering
    \includegraphics[width=\columnwidth]{Figures/wind_only.pdf}
    \caption{Wind Safety Case: A parameterized safety case based only on the weather and the drone's capabilities. This safety case creates the instances seen in Figures \ref{fig:instance_1} and \ref{fig:instance_2}.}
    \label{fig:wind_only}
\end{figure}

Our second safety case is relevant when the pilot lacks the evidence required to satisfy our initial safety case above. There needs to be an opportunity for newer pilots to learn and fly if such flights can be done safely. Thus, our second safety case focuses on giving such pilots the information that they will need in order to complete a safe flight. This second safety case (Figure \ref{fig:wind_only}) focuses on the weather because poor weather is a common reason for a pilot to decide that a flight will not be safe or for in-flight failures \cite{weather_hazards_for_UAV}. The weather portion of the feature model also has several parameters that can map to portions of our safety case. This sort of weather-focused safety case would normally involve far more attributes than we show in Figure \ref{fig:wind_only}, but we focus only on the weather and a small amount of information (evidence) about the battery here.

In the wind safety case from Figure \ref{fig:wind_only}, we constructed a general safety case that involves a number of parameters that are found in our feature model. These parameters are indicated using square brackets, such as [Precipitation] and [UASAllowedPrecipitation]. The data types of each parameter are left intentionally vague, as there are a number of ways for these parameters to be stored. We assume that information about each [Vehicle] is publicly available and that published sUAS specifications can be converted into the same data format and type that the feature model and safety case parameters have. If a [Vehicle] does not contain information in its specifications for certain parameters, then there is an option to assume some default values that could apply to almost all drones. 

For instance, most sUAS specifications will include information about the maximum allowed wind speed within which the manufacturer states the sUAS can operate. Likewise, most sUAS specifications include both maximum and minimum allowed temperatures in which to operate (often from -10 \textcelsius \;or 0 \textcelsius \;up to 40 \textcelsius) \cite{DEERCD20, DJI_MiniPro_4_Specs}. Fewer sUAS specifications contain specific information about visibility requirements since those depend on the type of mission being flown, especially whether it needs to be flown in a visual line of sight (VLOS) or beyond a visual line of sight (BVLOS). If the pilot does not provide visibility requirement information, we thus assume that the flight must take place VLOS and proceed accordingly. Similarly, if no information is provided about an sUAS's ability to fly in various forms of precipitation, we assume that the sUAS can only operate with no precipitation. 

Note that in the wind safety case (Figure \ref{fig:wind_only}), many of the goals share a similar structure. For instance, "The forecast precipitation is within acceptable level..." and "The forecast visibility is within acceptable levels...". The repetition of these elements is intentional and allows for greater ease of human understanding of the safety case, as well as for simpler extension of the safety case when we add additional hazards we need to mitigate. 

Some of the values of the parameters in the safety case may not be available at the time of a flight request. For instance, if a pilot is applying to complete a flight several weeks or months in the future, the forecast weather conditions will be unreliable. In such a case, the safety case might not contain concrete values until closer to the flight. The pilot could still access the parameterized safety case in order to study the safety requirements for the flight. As the time of the flight approaches, a more fully instantiated safety case could be sent to the pilot. 

The information for instantiating these parameterized safety cases will need to be pulled from a variety of sources, such as publicly available weather data and manufacturers' specifications for commercially available sUAS. However, some of the parameters' information will need to come from the pilot, including their certification status, the sUAS model they will fly, their flight plan, and any additional sUAS capabilities they have added (such as detect-and-avoid systems). 
In the event that the sUAS being flown was completely home-built, there may be no public documentation of its abilities, and all of its specifications will need to be provided (or inferred) by the pilot. 
Therefore, some of the individual safety cases will necessarily contain a fair amount of uncertainty while still serving as a guideline for the pilot. 


\subsection{Instances}
As a final step in our SafeSPLE process, we demonstrate how to create instances of our parameterized safety case. This process involves obtaining the information required for all parameters and checking if all the solution nodes of the safety case remain true. In all of the safety case diagrams in Figures \ref{fig:pilot_only}, \ref{fig:wind_only}, \ref{fig:instance_1}, \ref{fig:instance_2}, these solution nodes are the bottom nodes labeled E1-E6, and have propagated from the context. If any solution node becomes false, then we can say that the pilot should either reconsider the flight, or should implement further mitigations to reduce the risk from the relevant hazard. For instance, if the safety case shows that the current wind gusts are too high, the pilot might delay the flight until the wind calms, or the pilot might decide to make the flight with a larger and more capable UAS (if available).

\begin{figure}[ht]
    \centering
    \includegraphics[width=.95\textwidth]{Figures/Instance_1.pdf}
    \caption{Safety Case Instance 1: An instance of the wind safety case (Figure \ref{fig:wind_only}) based on a mission with a DJI Mini 4 Pro drone.}
    \label{fig:instance_1}
\end{figure}

The first instance of our parameterized safety case is shown in Figure \ref{fig:instance_1}. This mission will be performed by a DJI Mini 4 Pro, a widely available drone that currently sells for just over \$1000, depending on accessories. The Mini 4 Pro is fully charged, and the mission, as planned, should take 16 minutes, flown entirely within VLOS of the pilot. This information about battery charge and the mission plan is provided by the pilot. The wind is gusting up to 6 meters/sec, with temperatures in the mid 20s \textcelsius, unlimited visibility, and no precipitation. This weather information is provided to the safety case by a commercial or governmental weather service. 

Once we obtain the information about the make and model of the drone, we can look up the DJI's published specifications. According to DJI \cite{DJI_MiniPro_4_Specs}, the Mini 4 Pro is able to fly in wind speeds up to 10 m/s, and with a fully charged battery can fly up to 34 minutes. The Mini 4 Pro can operate in temperatures between -10 \textcelsius \;and 40 \textcelsius. Using all this information, we can instantiate the safety case seen in Figure \ref{fig:instance_1}. Note that every solution node 
(labeled E1-E6) is satisfied by the above information. There is no precipitation; visibility is unlimited; the temperatures are not too hot or cold; the wind gusts are below the max allowed for the drone; and the battery reserves are more than twice as much as needed. So in this instance of the safety case the the top-level goal is satisfied. 

\begin{figure}[ht]
    \centering
    \includegraphics[width=.95\textwidth]{Figures/Instance_2.pdf}
    \caption{Safety Case Instance 2: An instance of the wind safety case (Figure \ref{fig:wind_only}) based on a mission with a DEERC D20 drone. Note that this instance fails to fulfill our safety requirements at node E4 (marked in darker red).}
    \label{fig:instance_2}
\end{figure}
%Nice examples!

In Figure \ref{fig:instance_2} we can see a second instance of our safety case. This mission will be performed by a DEERC D20 drone, another widely available drone that currently sells for around \$50. The D20 is also fully charged, and the planned mission will only take 5 minutes of flying, entirely within VLOS. The wind is gusting up to 8 m/s, with temperatures in the mid-30s \textcelsius, 3 km visibility, and no precipitation.

According to the DEERC documentation \cite{DEERCD20}, the D20 drone is capable of about 10 minutes of flight time in temperatures between 0 \textcelsius \ and 40 \textcelsius. However, the D20 documentation does not specify the maximum speed of the winds that the drone is capable of flying in. Instead, the documentation reads, "DO NOT use this drone in adverse weather conditions such as rain, snow, fog, and wind." Therefore the safety case takes a conservative approach and assigns a default value of 3 m/s to the variable [MaxAllowedWindSpd] (3 m/s is slightly less than 7 mph). This default value could, of course, be set to 0 m/s, although this seems unrealistic for most outdoor flying. Other default values might be justified.

Plugging in all of these values, we see that while most solution nodes are satisfied, the current wind conditions (gusts up to 8 m/s) do not allow for a safe flight with the D20 (default max wind speed of 3 m/s). In Figure \ref{fig:instance_2}, this is shown at solution node E4, which is colored a darker red than the other solution nodes. The safety case is designed to serve as input to the UTM on-entry decision. At this point there are two main options for how the UAS Traffic Manager could behave. The UTM could refuse entry to this pilot until the wind speed is lower, or the UTM could send the safety case to the pilot with the recommendation that the pilot make modifications to the flight plan while leaving the ultimate flight decision up to the pilot.

Creating instances of safety cases with SafeSPL should be quick and relatively straightforward, if the information it needs is available. If information on the drone's capabilities is lacking, default values can still allow the safety case to create a reasonable instance. If information about the weather is unknown, then those portions of the safety case can be left uninstantiated until more detailed information becomes available. At the very least, we can generate a partially instantiated safety case so the pilot can see the areas where information is lacking or is based on default values. This information could allow the pilot to focus on mitigation measures in those areas if needed. 

\subsection{Connecting to Safe Entry}
\label{sec:safe_entry}

The parameterized safety cases created by SafeSPLE and described above could play an important role in a to-be-developed UTM system. When a pilot requests permission to fly in the airspace controlled by the UTM, the information needed to instantiate the safety case is either submitted by the pilot or looked up by the UTM system. Once a safety case has been created for that flight, there are at least two options for what the UTM system might do with it. 

\begin{enumerate}
    \item Closed Access: The UTM system accepts or denies requests based on whether each generated safety case "passes" or "fails". In other words, if the safety case goals are not satisfied, the UTM system  denies the flight. 
    \item Open Access: The UTM system accepts or denies the flight based solely on whether the pilot is certified or trusted. The safety case then becomes a guideline that can be provided to the pilot as something of a checklist to encourage a safer flight.
\end{enumerate}

Which action the UTM should take is an ongoing discussion with no immediate correct answer.
Currently the regulations in the US appear to generally favor approach (2), the open-access model. Regardless of which approach is taken for a specific controlled airspace, we believe the use of SafeSPLE will generate valuable on-the-fly information.  This information may offer an effective and useful checklist for decision-making. 


% In this section, we study whether our method can 适应其他的检索器,并进行有效的检索。具体来说,我们主要探究SRAG在DPR上的表现和BM25的有什么区别,因此我们和一个基于DPR的baseline相比,并且和基于BM25的SRAG相比。

In this section, we investigate SRAG's adaptability to different retrieval mechanisms and its effectiveness across various retrieval architectures. 
Specifically, we examine SRAG's performance when integrated with dense passage retrieval (DPR) compared to its BM25-based implementation. 
We conduct comparative experiments against both a DPR-based baseline and our BM25-based SRAG to evaluate the method's generalizability across different retrieval paradigms.


For DPR implementation, we utilize the RAGRetriever pipeline from HuggingFace\footnote{\url{https://huggingface.co/facebook/rag-sequence-nq}}. The question encoder is \texttt{facebook/rag-sequence-nq}, and we use the compressed DPR embeddings from \texttt{facebook/wiki\_dpr}\footnote{\url{https://huggingface.co/datasets/facebook/wiki_dpr}} for efficient retrieval.


As shown in Table~\ref{tab:retriever}, we derive two key findings:
% 1. 和dragin论文中提到的相同,dpr的检索器有时候不如bm25
% 2. 比基于dpr的baseline效果好
1) Our SRAG method performs exceptionally well when integrated with the DPR retriever. We implemented both the FLARE baseline and xxx baseline using the DPR retriever, and our SRAG method outperforms these DPR-based baselines. This demonstrates SRAG's effectiveness across different retrieval mechanisms, indicating its adaptability and ability to enhance retrieval performance regardless of the underlying retrieval architecture.
2) Consistent with observations in the DRAGIN paper, we find that the DPR retriever sometimes underperforms compared to BM25. We speculate this is mainly due to two reasons: a)  Existing literature has shown that BM25 retrieval offers greater robustness, especially in some scenarios. b) Current QA datasets are usually constructed with queries based on Wikipedia. The entities in question often use the same expressions as in the documents in such datasets. Therefore, keyword-matching methods like BM25 can perform better because they leverage exact term overlaps between queries and documents.


These findings highlight SRAG's flexibility and robustness, showing that it not only adapts to various retrieval systems but also consistently improves performance over baseline methods.

\subsection{Case study}
% In Table~\ref{tab:casestudy}, we show a 典型的 case that 展示了SRAG的整个学习过程。In 对于问题“Where is the company that Sachin Warrier worked for as a software engineer headquartered”。 在stage1,SRAG 会首先obtain knowledge that Sachin Warrier worked for Tata Consultancy Services. Then, it 搜索 the headquarters of Tata Consultancy Services 从而获得答案是mumbai。而对于stage2, SRAG在检索到Sachin Warrier worked for Tata Consultancy Services之后,它利用内部知识回答了Tata Consultancy Services的首都在Mumbai。

Table~\ref{tab:casestudy} provides a representative case study demonstrating the complete learning process of SRAG. For the query, ``Where is the company that Sachin Warrier worked for as a software engineer headquartered?'', SRAG follows distinct approaches in each stage. In Stage 1, the model initially acquires the knowledge that Sachin Warrier was employed at Tata Consultancy Services. Subsequently, it conducts a search to locate the headquarters of Tata Consultancy Services, retrieving the answer Mumbai. In Stage 2, however, after identifying that Sachin Warrier worked for Tata Consultancy Services, SRAG leverages its internal knowledge base to directly determine that the company's headquarters is in Mumbai, bypassing the need for an external search.