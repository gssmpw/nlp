\section{Related work}
\paragraph{Data-Driven Path Planning}
Path planning (in particular point-to-point shortest path planning) refers to the task of finding a low-cost, collision-free path from start to goal points in a map, which has a long history in robotics and AI____. Sampling-based path planning, such as Probabilistic Roadmaps____, Rapidly-exploring Random Trees (RRT)____, or their asymptotically optimal versions____, is a popular approach to the path planning tasks especially when the robotic state space is either continuous or high-dimensional. This approach approximates the state space by a fixed number of samples (\ie, nodes of roadmap graphs or random trees), where biased sampling around potential solution paths can improve planning efficiency. Machine learning has been a promising tool for learning how to bias the sampling from expert demonstration data____. Nevertheless, no prior work on such data-driven path planning can incorporate natural language instructions into their planning results.

\paragraph{LLMs for Robotic Applications}
Large language models (LLMs) have been used for various applications including robotics____. Existing work utilizes LLMs for task planning and action selection____, replanning and self recovery____, or human-mediated navigation____. Another interesting work is ROS-GPT____ which integrates OpenAI's ChatGPT with the Robot Operating System (ROS) 2, facilitating the generation of command prompts from natural language inputs. In contrast, our work is the first to utilize LLMs for modulating cost maps in the context of data-driven path planning.