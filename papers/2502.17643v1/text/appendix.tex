
\section{Snapshots of Human Experiments.}
Figs. \ref{fig. larger UI for task}-\ref{fig. larger aicoach intervention} are larger versions of Figs. \ref{fig. ui}-\ref{fig. aicoach ui} mentioned in the main text.

\newcommand\gap{0.92}
\begin{figure}[h]
  \centering
  \includegraphics[width=\gap\linewidth]{images/movers_task_cropped.png}
  \caption{Task scene and UI to control Alice (Fig. \ref{fig. ui for task}).}
  \label{fig. larger UI for task}
\end{figure}
  
\begin{figure}[h]
  \centering
  \includegraphics[width=\gap\linewidth]{images/movers_annotation_cropped.png}
  \caption{UI for intent annotation (Fig. \ref{fig. ui for intent selection}).}
  \label{fig. larger UI for intent selection}
\end{figure}

\begin{figure}[h]
  \centering
  \includegraphics[width=\gap\linewidth, frame]{images/movers_review_cropped.png}
  \caption{UI for after-action review (Fig. \ref{fig. ui for post-session review}).}
  \label{fig. larger UI for post-session review}
\end{figure}


\begin{figure}[h]
  \centering
  \includegraphics[width=\gap\linewidth]{images/exp2_movers_group_b_0.png}
  \caption{AI coach: no intervention (Fig. \ref{fig. aicoach no intervention}).}
  \label{fig. larger aicoach no intervention}
\end{figure}

\begin{figure}[h]
  \centering
  \includegraphics[width=\gap\linewidth]{images/exp2_movers_group_b_2.png}
      \caption{AI coach: \coach-generated intervention (Fig. \ref{fig. aicoach intervention}).}
  \label{fig. larger aicoach intervention}
\end{figure}

\section{Additional Discussion on the results in Section \ref{sec: model learning results}}
As demonstrated in Table \ref{table: btil performance}, the success rates differ between the intents. This difference is due to the varying characteristics of the environments and the degree of heterogeneity across demonstrations. For instance, when a human is carrying a box, they are almost always heading toward the truck. This enables the learned model to clearly associate the intent "truck" with the behavior of moving toward the truck. In contrast, when a human is not carrying a box, they could be heading toward any of Box 1, Box 2, or Box 3. This ambiguity makes intent inference for Box 1, Box 2, and Box 3 more challenging than intent inference for "truck."


\section{Statistical Analysis of survey results in Section \ref{sec: survey results}}

Statements \#1-3 pertained to the robot teammate and were presented to both groups. Statements \#4-9 focus on the AI Coach and were asked only of the experimental group.
Using a scale of 1-5 to represent "Strongly Disagree" to "Strongly Agree," we performed statistical analysis for statements \#1-3. As shown in Table \ref{table: survey stats}, participant's responses were not statistically significant between the control and experimental groups. This suggests that participants from both groups had similar impressions of their robot teammate, indicating that no confounds were introduced by the robot teammate.

\begin{table}[h]
    \caption{Statistical Analysis for Survey Statements \#1-3.}
    \label{table: survey stats}
    \centering
    \begin{tabular}{c c c c c}
        \toprule
        \textbf{Domain} & \# & \textbf{Control} & \textbf{Experimental} & \textbf{p-value} \\
        \midrule
        \multirow{3}{*}{Movers} 
         & 1 & 3.23 $\pm$ 1.01 & 3.43 $\pm$ 0.94 & 0.43 \\
         & 2 & 2.77 $\pm$ 1.01 & 3.20 $\pm$ 1.06 & 0.11 \\
         & 3 & 3.57 $\pm$ 1.25 & 3.63 $\pm$ 1.19 & 0.83 \\
        \midrule
        \multirow{3}{*}{Flood} 
         & 1 & 3.30 $\pm$ 0.99 & 3.27 $\pm$ 1.11 & 0.90 \\
         & 2 & 3.43 $\pm$ 1.17 & 3.43 $\pm$ 1.14 & 1.00 \\
         & 3 & 3.30 $\pm$ 1.09 & 3.43 $\pm$ 1.25 & 0.66 \\
        \bottomrule
    \end{tabular}
\end{table}

For responses regarding the AI Coach usefulness (statements \#4-9), using combined scores rather than individual scores is recommended \cite{schrum2020four}. Overall, the \% of positive responses was 77.8\% for Movers and 57.8\% for Flood. Fig. \ref{fig: survey plot} illustrates this distribution, with the right side of the vertical line being more prominent. 

