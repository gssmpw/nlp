\subsection{Domains}
\label{sec. domains}
We first describe the collaborative tasks used in our evaluations: \movers and \rescue. Introduced in \cite{seo2023automated}, these dyadic tasks require teams to maintain a shared plan for effective execution. However, due to partial observability and lack of communication, achieving coordination and high task performance is challenging.

\subsubsection{\movers}
As shown in \cref{fig: movers}, Alice and Rob are tasked with moving three boxes to the truck as quickly as possible. The boxes are heavy and require both teammates to lift them together. Teamwork is effective as long as the teammates agree on which box to move and act accordingly, regardless of the order. However, as depicted in \cref{fig: movers human sight}, each team member has a limited view of the environment and cannot communicate with the other during task execution, making coordination challenging. The task ends after $150$ time steps or when all boxes are moved to the truck, whichever comes first. The cumulative team reward is defined as $150$ minus the time step at which the task terminates. 

\subsubsection{\rescue}
The second task is inspired by time-critical disaster response scenarios. As shown in \cref{fig: rescue}, the environment includes victims at three sites: one at City Hall, two at the Campsite, and four at the Mall. A rescue team, consisting of a police car and a fire truck, must save all victims within a time limit of $30$ time steps. While victims at City Hall and the Campsite can be rescued by a single vehicle, rescuing those at the Mall requires both vehicles to collaborate in repairing one of two bridges. Teamwork in this task is more complex: sometimes the team must work together (e.g., at the Mall), while in other cases, dividing sub-tasks is more efficient (e.g., at City Hall and the Campsite). As depicted in \cref{fig: rescue human sight}, team members can only observe each other when at the same location or a landmark, complicating coordination. The total team reward is defined as the number of victims rescued within the time limit. 
