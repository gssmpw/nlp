\begin{table}
\caption{Survey Statements}
\label{table: survey}
\begin{center}
\small
% \setlength\tabcolsep{4.5pt} % Column spacing adjustment (default: 6)
\setlength\dashlinegap{3pt}
\begin{tabular}{cl} \toprule
\# & Statement (rated on a 5-point Scale) \\ \midrule
1 & \makecell[tl]{The team worked fluently together.} \\
2 & \makecell[tl]{The robot contributed to the fluency of the interaction.} \\
3 & \makecell[tl]{The team improved over time.}  \\ \midrule %\hdashline \noalign{\vskip 0.5ex}
4 & \makecell[tl]{During the task, I followed the AI Coach suggestions in general.} \\
5 & \makecell[tl]{The AI Coach was intelligent.} \\
6 & \makecell[tl]{The AI Coach was trustworthy.} \\
7 & \makecell[tl]{The AI Coach's suggestions were effective.} \\
8 & \makecell[tl]{The AI Coach's suggestions were timely.} \\
9 & \makecell[tl]{The AI Coach contributed to the fluency of the interaction.} \\
\bottomrule  
\end{tabular} 
\end{center}
\end{table}
\subsection{Study 2: Validation}
\label{sec. validation}
After collecting the training data, we conducted a second study to evaluate \coach's performance (\textbf{Q2}) and perceived usefulness (\textbf{Q3}). The study was a randomized control trial, where only the experimental group received coaching from \coach. %The control group completed tasks without AI-enabled coaching.

\subsubsection{Participants}
We recruited participants via Prolific~\cite{palan2018prolific}. Of the 73 users who accessed the experiment, 61 completed it. To ensure balanced group sizes, we used the first 30 participants from each group. The control group consisted of 13 females and 17 males (age: 28.7$\pm$8.4 years), while the experimental group included 11 females, 17 males, and 2 non-binary participants (27.8$\pm$9.1 years).

\subsubsection{Materials and Setup}
Similar to the first study, we developed a website featuring the two tasks with an interactive user interface. However, instead of intent annotation mechanisms, this version incorporated \coach on the backend and its user interface on the frontend for interacting with the team during task execution. As shown in \cref{fig. aicoach ui}, the interface features an AI coach icon with a speech balloon above the task screen.
During task execution, the speech balloon nominally displays: \textit{``Keep up the good work.''}
However, if \coach detects misaligned intents and decides to intervene, it pauses the task, prompts the team to reflect on their plans, and recommends an optimal course of action corresponding to $x^*$. As illustrated in \cref{fig. aicoach intervention}, the speech balloon displays:
\begin{quote}
\centering
\textit{``I've spotted a potential opportunity to enhance our teamwork: Please $\langle recommendation\rangle$''}. 
\end{quote}
The suggestion is highlighted with a red circle, and participants must click the ``Confirm'' button to resume the task.
While \coach offers recommendations, participants ultimately decide whether to accept them.
To account for the fact that not all recommendations will be followed, we set $p_a = 0.9$ for this study. 
\coach utilizes two additional hyperparameters: the cost of intervention $c$ and the threshold $\delta$. For \movers, the intervention cost is set to 1, representing the loss of one time step to pause and reflect on the recommendation. In contrast, for the life-critical \rescue task, the cost is considered negligible ($c = 0$) as any small delays caused by interventions are justified if they assist in rescue efforts. The threshold $\delta$ was determined through a grid search over the hyperparameter space, with values set to $5$ for \movers and $0.1$ for \rescue based on simulated experiments with the learned teamwork model.

\begin{table}
\centering
\caption{Success Rate of the Learned Model}%
\label{table: btil performance}
% \setlength\tabcolsep{4.5pt} % Column spacing adjustment (default: 6)
\setlength\dashlinegap{4pt}
\small
\begin{tabular}{@{}c@{}c@{\hskip 6pt}c@{\hskip 6pt}c@{\hskip 6pt}c@{}} \toprule
~~Domain~~ & ~~Intent~~ & \makecell[c]{Success (\%)} & \makecell[c]{Wrong (\%)} & \makecell[c]{Nowhere (\%)} \\ \midrule
\multirow{5}{*}{\small \movers} & Box 1 & 75.4 & 5.9 & 18.7  \\
                                & Box 2 & 72.5 & 12.5 & 15.0  \\
                                & Box 3 & 69.3 & 13.8 & 16.9  \\
                                & Truck & 99.6 & 0.0 & 0.4  \\ \cmidrule{2-5}
                                & Mean & 79.2 & 8.05 & 12.75  \\ \midrule
\multirow{5}{*}{\small \rescue} & City Hall & 88.3 & 0.3 & 11.4  \\
                                & Campsite & 38.2 & 9.2 & 52.6  \\
                                & Bridge 1 & 59.4 & 2.2 & 38.4  \\
                                & Bridge 2 & 45.4 & 3.1 & 51.5  \\ \cmidrule{2-5}
                                & Mean & 57.8 & 3.7 & 38.5  \\ \bottomrule
\end{tabular} 
\end{table}
\subsubsection{Procedure}
The overall structure of this experiment closely mirrors that of the first study. It is web-based and includes a study overview, a demographic survey, \movers and \rescue domains, and a post-experiment survey. For each domain, participants completed an interactive tutorial followed by four task trials. While the tutorials and trials were similar to the first study, intent annotation features were removed. Only for the experimental group, \coach's features were integrated into the tutorial and task trials. Each domain involved one practice trial to help participants familiarize themselves with the task and the robot teammate, followed by three test trials. Neither group received assistance from \coach during the practice trial. In the test trials, the control group performed the task without coaching, while the experimental group received task-time interventions from \coach. After the trials, participants completed the survey described next.

\subsubsection{Measures}
We assess \textbf{Q1} by quantifying the intent-condition-ed success rate of the learned model. For \textbf{Q2}, team performance is evaluated using task scores. Beyond improving teamwork, the perceived usefulness of \coach is essential for its adoption by human users. Hence, to address \textbf{Q3}, we use subjective statements adapted from a widely used scale~\cite{hoffman2019evaluating}. The first three questions solicited participants perception regarding the robot teammate, while the rest regarding \coach the AI coach. Control group rated the first three statements listed in \cref{table: survey}, while the experimental group rated all statements. Responses were recorded on a 5-point scale, ranging from \textit{strongly disagree} (1) to \textit{strongly agree} (5).
