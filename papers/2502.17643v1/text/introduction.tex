\section{Introduction}
\label{sec. intro}
Consider your favorite sports team --- whether it is soccer, cricket, basketball, or another team sport --- working together to achieve a common goal. Even though all the team members are trained professionals, some teams consistently outperform others. Indeed, \textit{a team of individual experts does not necessarily make for an expert team}~\cite{bisbey2021transforming}; building a successful team requires the confluence of multiple factors~\cite{mathieu2000influence}. Human factors research has identified key drivers of team effectiveness, including capability, coordination, communication, and coaching~\cite{tannenbaum2020teams}. Through targeted training and interventions, human teams can significantly improve coordination and enhance their performance in collaborative tasks.
\ifarxiv
\blfootnote{This article is an extended version of an identically-titled paper accepted at the \textit{International Conference on Autonomous Agents and Multiagent Systems (AAMAS 2025)}.}
\else
\blfootnote{An extended version of this paper, which includes supplementary material mentioned in the text, is available at \url{http://tiny.cc/socratic-appendix}}
\fi

Coaches play a crucial role in both team training and interventions. Rather than performing tasks themselves, they enhance collaboration by offering expert insights. These insights are provided both during task execution, such as in games, and during training sessions, such as in practice. While coaches are common in professional sports, integrating them into life-critical fields presents significant challenges~\cite{seo2021towards, orlov2024rw4t}. Resource constraints and a shortage of experts make it difficult to employ coaches during task execution.
For example, in surgical teamwork, a coach could be invaluable in reducing preventable medical errors~\cite{wahr2013patient, makary2016medical, seo2021towards}. Reducing these errors would significantly improve patient health outcomes. However, due to the shortage of medical professionals, it is not feasible for a specialist to continuously serve in this coaching role. Similarly, in aviation, coaches assist with simulation-based training, but they cannot accompany a flight crew on every flight~\cite{kolander2019flight}. 

\begin{figure*}[t]
  \centering
  \includegraphics[width=0.9\linewidth]{images/socratic_schematic.pdf}
  \caption{Schematic of \coach: an AI coach for enhancing teamwork during task execution. Blue arrows represent the workflow during the training phase, whereas black arrows indicate the workflow during the execution phase.} 
  \label{fig. schematic} 
  \Description{A schematic diagram that provides an overview of \coach.}
\end{figure*}

Recognizing the need for coaching assistance in life- and safety-critical applications, we propose an innovative use of artificial intelligence (AI): task-time team coaching. Specifically, we envision an AI agent that complements a human coach by monitoring a team during task execution and providing real-time guidance to improve teamwork, particularly in situations where the human coach may be busy or unavailable. While coaches offer a variety of feedback before, during, and after tasks, in this work, we limit our scope to delivering task-time feedback in time-critical tasks. For this setting, we present a novel proof-of-concept AI agent, \coach, designed to complement a human coach and thereby enhance teamwork.

Illustrated in \cref{fig. schematic}, the overall design of \coach is grounded in the extensive literature on human team training. Specifically, \coach operates by observing task execution and identifying points where the team's mental models regarding shared plans may become misaligned. When such a misalignment is detected, \coach prompts the team to pause, reflect on their plans, and offers suggestions for improvement. By encouraging the team to reconsider future actions that could lead to inefficiencies or errors, \coach aims to enhance collaborative decision-making. 

From an AI perspective, \coach leverages recent advances in imitation learning and multi-agent systems. First, it employs multi-agent imitation learning to model team behavior based on demonstrations from previously executed tasks. Using this model and data from an ongoing task, it builds on \tic -- a recent algorithm for agent-based teamwork -- to algorithmically detect points of misalignment and generate recommendations. Finally, through an interactive user interface, \coach delivers the automatically generated interventions aimed at aligning the team's understanding and improving overall performance.

To evaluate \coach, we conducted two human subject experiments: one focused on training and the other on validation. Both experiments involved two collaborative tasks and dyadic teams. In the training experiment, we curated a novel dataset of human demonstrations annotated with intents and used it to train \coach. In the validation experiment, we conducted a randomized controlled trial to evaluate both \coach's objective performance and the users' subjective perceptions of the system. The experimental results show that Socratic significantly improves team performance with minimal interventions. Equally important for its adoption, participants perceive \coach as helpful to improving teamwork. The evaluations also suggest promising directions for both AI research and the proposed applications, highlighting the potential of AI agents to support human teamwork.
