\subsection{System Overview}

\subsubsection{Scope}
We limit our scope to collaborative tasks modeled as Dec-POMDPs (\cref{sec. task model}) and teams that include at least one human member.
Importantly, we do not make assumptions about team members' rationality or expertise levels.
As reviewed in \cref{sec: team training}, the science of teamwork identifies several key drivers of effective teamwork.
In this proof-of-concept work, we focus on team alignment\footnote{Investigating other drivers of effective teamwork, intervention mechanisms, and teamwork settings is an important avenue for future research.} 
 --- ensuring that the team is ``on the same page.''
Misalignment is particularly common in time-critical scenarios, where teams may lack sufficient time to communicate and coordinate shared plans.
Additionally, real-world factors such as partial observability, fatigue, and uncertainty can further degrade team member's understanding of each other's beliefs, desires, and intentions.

\subsubsection{Design Requirements}
\label{sec: problem statement}
\label{sec. requirements}
With this scope defined, we design \coach: an AI-enabled coaching agent to improve teamwork during task execution. The design process began by identifying system requirements through brainstorming sessions with an interdisciplinary team of researchers in human factors, team training, AI, and usability. We determined that an AI agent capable of \textit{detecting misalignments in team members' intents} and \textit{alerting the team to pause, reflect, and adjust their plans} is both feasible to develop and can significantly enhance collaboration.
For the successful realization and adoption of such an agent, we distilled key requirements (\textbf{Rx}); namely, \coach must be:
\begin{itemize}
    \item[\textbf{R1.}] able to sense and monitor teamwork;
    \item[\textbf{R2.}] able to accurately infer intents of the team members;
    \item[\textbf{R3.}] able to accurately anticipate future actions of the team;
    \item[\textbf{R4.}] able to generate effective task-time interventions;
    \item[\textbf{R5.}] able to effectively deliver the interventions; and
    \item[\textbf{R6.}] perceived as useful by the team members.
\end{itemize}
