\subsubsection{System Architecture}
To meet the design requirements, \coach leverages recent advancements in imitation learning and multi-agent systems, incorporating an interactive user interface to monitor the team and deliver interventions.
For \textbf{R1} (sensing and monitoring teamwork), we assume \coach is equipped with sensors to observe both the team and task environment. Similar to sport scenarios, where team members may have partial observability, the coach has full visibility of the environment.
To meet \textbf{R2} (inferring intents) and \textbf{R3} (anticipating future actions), \coach employs a recent multi-agent imitation learning algorithm \btil that explicitly models team members' intents and learns a generative model of team behavior~\cite{seo2022semi}.
Building on this model, \coach utilizes a specialized instance of the \tic framework to generate task-time interventions to meet \textbf{R4}~\cite{seo2023automated}.
Lastly, \coach includes a user interface to deliver these interventions to the team, addressing \textbf{R5}. 

\subsubsection{System Operation}
\coach operates in two phases: training and execution. During the training phase, \coach observes the team performing tasks in practice sessions, collecting teamwork data and learning generative models of team behavior. During task execution, \coach uses the learnt generative model to infer team intents, detect misalignments, and compute and deliver effective interventions. We now detail each system component.
