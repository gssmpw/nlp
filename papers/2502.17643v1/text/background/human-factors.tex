\subsection{The Science of Human Teamwork}
\label{sec: team training}

% \subsubsection{Drivers of Effective Teamwork}
The science of human teamwork focuses on the question: \textit{What makes teams work?}~\cite{salas2018science}.
Over the past four decades, psychologists and human factors researchers have systematically identified the factors that make teamwork challenging and developed methods to improve it~\cite{salas2008teams,cooke2013interactive,cooke2021effective,salas2024science}.
We briefly review key insights that inform our work, while directing readers to recent survey by ~\citet{tannenbaum2020teams} for more details.
There is broad consensus that teamwork is especially challenging in time-critical scenarios, where success depends on the convergence of multiple factors.
A major challenge is that humans often make suboptimal decisions due to bounded rationality~\cite{simon1997models, kahneman2003maps, kahneman2013prospect} and limited situational awareness~\cite{endsley2000situation, parush2011communication, stanton2017state, endsley2021situation}, especially under time constraints.
Hence, \coach does not assume perfect rationality or situational awareness from team members.
Even teams composed of experts may not function optimally due to a lack of shared mental models, leading to poor coordination and even fatal errors~\cite{cannon1993shared, mathieu2000influence, jonker2010shared, van2011team, mccomb2014concept, harari2024misalignment}.
Thus, \coach explicitly considers team members' intent and allows for potential misalignment, which can lead to suboptimal teamwork.

% \subsubsection{Team Training}
To enhance teamwork, the science of teamwork recommends several methods and best practices, including effective communication, simulation-based training, and coaching -- the latter being the focus of this paper. Coaches play a crucial role by assessing teamwork and providing feedback to improve it. While human coaches rely on their expertise and experience for these activities, the science of teamwork has developed principled methods and formalized best practices for coaching. Researchers have established robust methods for assessing  teams~\cite{salas2018science,granaasen2019towards,costar2020improving,grimm2023dynamical,kennedy2024novel} and generating targeted insights to enhance teamwork~\cite{hackman2005theory, salas2008does, peters2013team, weaver2014team, britton2015expanding}. However, these assessments are typically post-hoc, lack automation, and are limited to contexts where a human coach is available. Thus, we explore the design of an AI coach capable of operationalizing these insights, detecting misalignments in team members' shared intents, and providing real-time feedback during task execution. 
