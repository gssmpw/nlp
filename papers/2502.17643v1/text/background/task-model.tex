\begin{figure*}[h]
  \centering
  \begin{subfigure}[b]{0.245\linewidth}
      \centering
      \includegraphics[width=0.95\textwidth, frame]{images/movers_map.png}
      \caption{Map of \movers}
      \label{fig: movers}
  \end{subfigure}
  \hfill  
  \begin{subfigure}[b]{0.245\linewidth}
      \centering
      \includegraphics[width=0.95\textwidth, frame]{images/movers_human_sight.png}
      \caption{\movers: Alice's perspective}
      \label{fig: movers human sight}
  \end{subfigure}
  \hfill  
  \begin{subfigure}[b]{0.245\linewidth}
      \centering
      \includegraphics[width=0.95\textwidth, frame]{images/rescue_map.png}
      \caption{Map of \rescue}
      \label{fig: rescue}
  \end{subfigure}
  \hfill  
  \begin{subfigure}[b]{0.245\linewidth}
      \centering
      \includegraphics[width=0.95\textwidth, frame]{images/rescue_police_sight.png}
      \caption{\rescue: Police's perspective}
      \label{fig: rescue human sight}
  \end{subfigure}

  \captionsetup{subrefformat=parens}
  \caption{\movers and \rescue domains, detailed in \cref{sec. domains}. Team members can observe only the unshaded region of the environment.}
  \label{fig: domains}
\end{figure*}
\subsection{Collaborative Tasks}
\label{sec. task model}
Teamwork is fundamental to many human endeavors, spanning scenarios such as sports, healthcare, aviation, and more. Our focus is on \textit{time-critical scenarios}, such as healthcare and disaster response, where effective teamwork is crucial for mission success. Teamwork occurs at various levels, ranging from large organizations to small ad-hoc teams. We focus on \textit{mission-oriented, sequential tasks}, where an established team works toward a clearly defined mission (e.g., \cref{fig: domains}). Although the mission is well-defined, there are often multiple ways to achieve the task. Real-world challenges, such as uncertainty, information asymmetry, and partial observability, can create barriers to efficient teamwork and task completion. Finally, we consider teams composed of human members, either in human-only teams or hybrid human-AI teams.

To develop an AI agent capable of supporting such teamwork, the first step is to mathematically model the task and team dynamics. Fortunately, research in multi-agent systems offers several established formalisms for modeling collaborative tasks, including belief-desire-intention frameworks, Markov models, and game theory~\cite{tambe2005conflicts, bernstein2002complexity, nair2003taming, nair2005hybrid, chao2016timed, stone2010ad, grosz1999planning, grosz1996collaborative, modi2005adopt, harbers2012measuring, semsar2009multi, fridovich2020efficient, tian2022safety}. In our work, we leverage decentralized multi-agent partially observable Markov decision processes (Dec-POMDPs)~\cite{oliehoek2016concise}. This choice is motivated by their ability to model tasks with well-defined missions, structured teams, time constraints, action uncertainties, and partial observability, as well as their prior use in modeling time-critical collaboration scenarios like disaster response~\cite{chen2015decentralized, unhelkar2016contact, liu2017learning, lee2021multi,  dong2023optimizing}. 

We define a task as the tuple $\mathcal{M} = (n, S, A, \Omega, T, O, R, \gamma, h)$, where $n$ is the number of agents, $S$, $A\doteq\times_i A_i$ and $\Omega\doteq\times_i\Omega_i$ denote a state space, an action space and an observation space, respectively, $T(s'|s, a)$ denotes a probability of a state $s$ transitioning to another state $s'$ given a joint action $a\doteq(a_1, \cdots, a_n)$, $O(o|s', a)$ is a probability of a joint observation $o\doteq(o_1, \cdots, o_n)$ given a state $s'$ and a joint action $a$, $R$ is the task reward (objective), $\gamma$ is the discount factor, and $h$ is the task horizon. In theory, a team could act optimally by computing a decentralized policy using Dec-POMDP solvers based on the task model. However, it is unrealistic to expect human team members to compute and execute such a policy flawlessly and without errors. Therefore, we draw on human factors research to model team behaviors and identify strategies for improving teamwork.
