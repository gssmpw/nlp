\subsection{AI-Assisted Teamwork}
AI-assisted human teamwork is an emerging area of research with applications being explored across various domains~\cite{kamar2009incorporating, rajvsp2020systematic, seo2021towards, pynadath2023effectiveness, pynadath2023improving, van2023markov, seo2024ai, harari2024deep}. For instance, DeepMind and Liverpool FC are investigating data-driven approaches to analyze and enhance team strategies in football~\cite{tuyls2021game}. For applications in healthcare and disaster response, researchers have applied AI to analyze team conversations and improve extended-duration teamwork~\cite{kim2016improving, amir2016mutual}. Closer to our focus on time-critical scenarios, domain-specific methods for automated teamwork assessment have been developed~\cite{granaasen2019towards, kotlyar2023assessing}. However, these methods, to our knowledge, provide only post-hoc support, and AI has not yet been used for task-time coaching.

Approaches for assessing and improving teamwork in human-robot or robot-only teams are also relevant to our work~\cite{riley2004advice, riley2002empirical, wu2022evaluating, thomaz2016computational, unhelkar2016contact}. Research in human-robot collaboration introduces metrics for evaluating teamwork~\cite{hoffman2019evaluating, ma2022metrics, norton2022metrics} and algorithms for improving it~\cite{buisan2021human, semeraro2023human, mukherjee2022survey, nikolaidis2017human, unhelkar2020decision}. However, these methods focus on training robots to work with humans. In contrast, our work centers on an AI agent that provides coaching and decision support, without directly performing the task.
Closest to our work are the recent frameworks TIC~\cite{seo2023automated} and TARS~\cite{zhang2024risk}, which generate task-time interventions to enhance multi-agent teamwork. TARS uses Dynamic Epistemic Logic-POMDP to generate interventions through planning algorithms~\cite{zhang2024risk}. TIC employs Dec-POMDPs and multi-agent imitation learning to generate interventions through a learned model~\cite{seo2023automated,seo2022semi}. However, these methods have not been applied or evaluated in settings with human team members.

Our work builds on these methods but differs in key ways. First, we adopt a systems perspective to develop \coach that includes both an intervention algorithm and a user interface, enabling interaction with and coaching for human users. Second, our methodology incorporates mechanisms to collect training data on human teamwork, including their cognitive states. Finally, we validate the effectiveness of the solution through human subject experiments.
