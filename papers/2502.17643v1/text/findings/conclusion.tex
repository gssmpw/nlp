\section{Conclusion}
\label{sec: conclusion}
We introduce \coach: a system that provides AI-enabled coaching to teams with human members during task execution. Through human subject experiments on challenging dyadic tasks, we demonstrated that \coach not only enhances team performance but is also perceived as useful by participants. Since \coach does not perform the tasks itself, it has the potential to assist in various domains, including those where AI agents may lack the capability to act but can still analyze and enhance human task execution.

Along with its strengths, we also highlight the limitations of this proof-of-concept work, which suggest exciting future research directions. First, while the experimental tasks captured challenging elements of real-world collaboration, they were conducted in a web-based environment. Future work should investigate AI-enabled coaching in more complex scenarios that include dynamic environments, larger teams, multiple objectives, ad-hoc collaboration, or members with diverse expertise~\cite{stone2010ad, seo2024idil, seo2025hierarchical}. Additionally, expanding AI coaching to physical teamwork settings using multimodal perception is an important next step~\cite{wang2019ai, xia2025sportu}.

Second, from a human-centered perspective, our study opens up new opportunities to examine how AI can support team training and complement human coaches. Informed by the science of teamwork, expanding \coach's interventions to include more varied recommendations would further enhance its utility~\cite{peters2013team, weaver2014team, britton2015expanding}. To enhance usability, interventions could be delivered through user-friendly interfaces such as screens, audio systems, or augmented/virtual reality.
Lastly, given AI coaches can make errors, a critical area for further investigation is ensuring the safe and responsible deployment of AI coaches~\cite{hoffman2013trust,  yang2017evaluating, qian2022evaluating, quintero2023robotic, qian2024pps, meneses2024perceptions}. This includes examining how trust in AI coaches can be effectively built, calibrated, and maintained to foster successful human-AI collaboration.
