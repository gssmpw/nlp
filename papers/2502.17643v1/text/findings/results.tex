\subsection{Experimental Results}
\label{sec:findings}

\subsubsection{\coach learns intent-driven models of team behavior.}
\label{sec: model learning results}
To address \textbf{Q1}, \coach first learns models of team behavior using the training data. We then evaluate if the learned model captures intent-driven behaviors by simulating the policy $1000$ times for each intent $x$ and measuring its success rate in completing the intended sub-task within 20 time steps. For instance, if the specified intent is to rescue victims at City Hall, we check how often the model succeeds.
\cref{table: btil performance} presents the success rates of the learned model for each intent. Failures are categorized as either \textit{Wrong} (where the model accomplishes a sub-task associated with a different intent, such as rescuing victims at the Camp Site when the specified intent was City Hall) or \textit{Nowhere} (where the model fails to complete any sub-task within the time limit). 
On the challenging task of modeling team behaviors from human data, the model achieved an average success rate of 79\% for \movers and 58\% for \rescue. Most failures belong to the \textit{Nowhere} category, suggesting that model learns intent-driven models of team behavior. Through the second study, we find that this model learning performance is sufficient for \coach to deliver effective task-time interventions to improve teamwork.

\begin{figure}
\centering
\includegraphics[width=\linewidth]{images/task_results_score.png} \vspace{-1em}
\caption{Team Scores: with and without \coach.} \vspace{-1em}
\label{fig. task results} 
\end{figure}
\subsubsection{\coach improves teamwork via targeted interventions.}
To answer \textbf{Q2}, we compared the performance of the two groups, using a cost-adjusted score that accounts for the time spent on processing and responding to interventions. Specifically, \score is defined as $R - C$, where $R$ is the cumulative team reward and $C$ is the total cost of interventions. For the control group, \score is equal to the task score, as no interventions took place.
As shown in \cref{fig. task results}, the experimental group outperformed the control group. In \movers, teams coached by \coach scored on average $90.8 (\pm 8.4)$ compared to $78.6 (\pm 21.7)$ for the control group, with a statistical significance of $p < 0.001$. Similarly, for \rescue, the average score of the experimental group was $6.1 (\pm 1.5)$ versus $5.3 (\pm 1.7)$ for the control group, with $p < 0.01$. The average number of interventions was 3.9 ($\pm$1.7) for \movers and 2.3 ($\pm$2.1) for \rescue. These results highlight that \coach effectively enhanced teamwork with minimal interventions. This targeted approach to delivering interventions is crucial for improving team performance and building human users' trust in AI-enabled coaching.

\subsubsection{\coach is perceived as useful by human users.}
\label{sec: survey results}
To answer \textbf{Q3}, we analyzed participants' survey responses. \cref{fig: survey plot} displays the percentage of positive, neutral, and negative assessments for each statement. 
Responses to statements \#1-3, which evaluated the robot teammate, were largely similar across both groups, indicating that participants had comparable perceptions of the robot teammate's capabilities. This consistency ensures a fair comparison between the groups, allowing us to accurately evaluate the AI coach's utility.

Statements \#4-9, which evaluated \coach and were rated only by the experimental group, indicate that participants perceived \coach as useful, effective, intelligent, and trustworthy.
Based on these statements, the average rating of \coach was $3.81 (\pm 1.03)$ for \movers and $3.28 (\pm 1.32)$ for \rescue on a $1-5$ scale.
Except for statement \#8 for \rescue task, the positive responses outweighed the negative ones for all statements.

Open-ended feedback suggested that \coach was seen as more helpful in the \movers task, while participants found \rescue more challenging. 
Regarding statement \#8, which asked about the timeliness of \coach’s recommendations, one participant commented:
\begin{quote}
\textit{“There was one occasion when the AI’s suggestion came a bit late, causing me to waste a few moves.”}
\end{quote}
While \coach is already designed to provide proactive guidance using a predictive model of teamwork, participants' responses suggest that they value this proactivity and may expect even more planning support from an AI Coach.  Informed by these findings, we conclude by summarizing our contributions and discussing their implications for both team training and AI research.
\begin{figure}
  \centering
  \includegraphics[width=\linewidth]{images/survey_divstacked_common.png}
  \newline
  \includegraphics[width=\linewidth]{images/survey_divstacked_coach.png} \vspace{-1em}
  \caption{Participant Responses to Survey Statements}
  \label{fig: survey plot} 
\end{figure}
