\section{Related Works}
\label{sec:related}
Advancements have influenced the evolution of crop yield monitoring in sensor technology and data analysis methodologies. While early studies highlighted rainfall as the primary determinant of crop yield, a key shift occurred in 1968 with the recognition of soil moisture as a more reliable predictor by Baier and Robertson ____. Their work leverages spectral data to estimate crop yield based on vegetation health indicators. Over time, numerous vegetation indices (VIs) have been developed to assess vegetative conditions and physiological characteristics of crops. These VIs, including the Normalized Difference Vegetation Index (NDVI), Leaf Area Index (LAI) ____, and Transformed Soil Adjusted Vegetation Index (TSAVI) ____, play a crucial role in crop prediction models. Advancements in hyperspectral imaging have enabled the capture of fine-grained spectral data, facilitating the development of biochemical indices for quantifying plant constituents.

Rice crop yield estimation relies on understanding the crop's growth stages and environmental factors. Water level in paddy fields, rather than direct precipitation, is crucial for irrigated rice fields. Accumulating temperature is more important than temperature at certain times, as it affects the crop's development stages. These factors are integrated into crop models to predict yield accurately.

In the early 2000s, research surged leveraging imaging and machine learning technologies for crop yield prediction. Studies introduced novel methodologies, such as artificial neural networks (ANN) and SVR, to analyse remote sensing data and historical yield records. These approaches demonstrated more precise results compared to traditional models and prepared for more precise and scalable methods for estimating crop yields. Uno et al. ____ analyse hyperspectral images of corn plots in Canada using statistical and ANN approaches, demonstrating the potential of ANNs in predicting yield with higher accuracy compared to conventional models. Li et al. ____ introduce a methodology employing ANN models to predict corn and soybean yields in the United States "corn belt" region, achieving high prediction accuracy through historical yield data and NDVI time series. Bala and Islam ____ estimate potato yields in Bangladesh using TERRA MODIS reflectance data and Vegetation Indices (VIs), demonstrating the effectiveness of VIs derived from remote sensing for early yield estimation. Li et al. ____ employ SVR and multi-temporal Landsat TM NDVIs to predict winter wheat yield in China, showcasing the precision and effectiveness of SVR models in yield estimation. Stojanova et al. ____ integrate LiDAR and Landsat satellite data using machine learning techniques to model vegetation characteristics in Slovenia. Their approach combines the precision of LiDAR data with the broad coverage of satellite data, facilitating effective forest management and monitoring processes.

Furthermore, Mosleh et al. ____ evaluated the efficacy of remote sensing imagery in mapping rice areas and forecasting production, highlighting challenges such as spatial resolution limitations and issues with radar imagery. Johnson et al. ____ developed crop yield forecasting models for the Canadian Prairies, revealing the effectiveness of satellite-derived vegetation indices, particularly NDVI, in predicting yield potential. Pantazi et al. ____ proposed a model for winter wheat yield prediction, integrating soil spectroscopy and remote sensing data to visually depict yield-influencing factors. Ramos et al. ____ introduced an optimised Random Forest algorithm for maize-crop yield prediction, emphasising the importance of vegetation indices like NDVI, NDRE, and GNDVI. Li et al. ____ utilised extreme gradient boosting machine learning to accurately predict vegetation growth in China, achieving high predictive accuracy and demonstrating effectiveness under diverse conditions. Zhang et al. ____ employed field-surveyed data to predict smallholder maize yield, with novel insights into the performance of various vegetation indices and machine learning techniques.

Recent studies have demonstrated the efficacy of utilising Sentinel-2 satellite imagery and machine learning techniques for predicting crop yields and mapping crop types in various agricultural settings. Son et al. ____ employed Sentinel-2 image composites and machine learning algorithms to forecast rice crop yields in Taiwan, finding that Support Vector Machines (SVM) outperformed RF and ANN at the field level, indicating their potential for accurate yield predictions approximately one month before harvesting. Perich et al. ____ utilised Sentinel-2 imagery to map crop yields at the pixel level in small-scale agriculture, with machine learning models utilising spectral index and raw reflectance data proving effective, even in the presence of cloudy satellite time series. Khan et al. ____ combined ground-based surveys with Sentinel-2 satellite images and deep learning techniques to map crop types, achieving high accuracy in identifying staple crops like rice, wheat, and sugarcane within the first four weeks of sowing. 

Along with Sentinel 2, UAV and other sensor spectral information have also been used in the literature in the last couple of years. Shafi et al. ____ propose XGBoost, LASSO and RF regression models to be utilised via Drone-based multispectral imagery whilst Islam et al. ____ combine remote sensing and meteorological data in stacking multiple regression models for rice crop yield prediction. Zhou et al. ____ compare CNN and LSTM-based models for predicting annual rice yield in Hubei Province, China, utilising ERA5 temperature data, and MODIS vegetation indices, demonstrating that the CNN-LSTM model with spatial heterogeneity outperforms models using only remote sensing data. Arshad et al. ____ evaluate the performance of RF and SVR, in predicting wheat yield in southern Pakistan using a combination of remote sensing indices and climatic variables where RF outperforms other methods. Asadollah et al. ____ assess the effectiveness of using a novel Randomized Search cross-validation (RScv) optimization algorithm with four machine learning models to predict annual yields of four crops (Barley, Oats, Rye, and Wheat) across 20 European countries, demonstrating improved prediction accuracy through satellite-based climate and soil data. 

Furthermore, Lu et al. ____ present a state-of-the-art CNN-BiGRU model enhanced by GOA and a novel attention mechanism (GCBA) for accurate county-level soybean yield estimation in the U.S., leveraging multi-source remote sensing data and outperforming existing models in yield prediction accuracy. Killeen et al. ____ investigate UAV-based corn yield prediction using RF and linear regression models and find that spatial cross-validation reduces over-optimism in yield prediction compared to standard 10-fold cross-validation, with LR showing better spatial generalizability than RF. Dhaliwal and Williams II ____ use a 26-year dataset on US sweet corn production to evaluate machine learning models for yield prediction, finding that RF performs best, with year, location, and seed source identified as the most influential variables.

Recently, Gadupudi et al. ____ demonstrated integrating ML strategies like RF and Decision Trees alongside DL models such as LSTM and RNN to enhance crop prediction accuracy, incorporating soil attributes, climate data, and cost analyses to optimize outcomes. Similarly, Rao et al. ____ employed attention-based CNNs and bidirectional LSTMs with hyperparameter tuning to predict crop yields, showcasing significant improvements in detection performance through methods like the shuffling shepherd optimization algorithm. Sharma et al. ____ explored the fusion of AI algorithms, including logistic regression and IoT-enabled analytics, to tailor recommendations based on regional agricultural parameters, advancing productivity and diversification. 

The diverse literature outcomes examined previously, along with numerous others, have highlighted the multidimensional potential of AI in transforming traditional agricultural practices into more data-driven, adaptive systems. They have employed a variety of data types from different sources and machine learning models. This diversity presents challenges in generalising techniques across different datasets, yet it also enhances performance for specific datasets. The adoption of multi-modal data usage, multi-modal AI techniques, and Ensemble methods has emerged as the current practice in this research field. Shahhosseini et al. ____ explore the predictive performance of two novel CNN-DNN machine learning ensemble models for forecasting county-level corn yields across the US Corn Belt. By combining management, environmental, and historical yield data from 1980 to 2019, the study compares the effectiveness of homogeneous and heterogeneous ensemble creation methods, finding that homogeneous ensembles provide the most accurate yield predictions, offering the potential for the development of a reliable tool to aid agronomic decision-making. Gavahi et al. ____ introduce DeepYield, a novel approach for crop yield forecasting that combines Convolutional Long Short-Term Memory (ConvLSTM) and 3-Dimensional CNN (3DCNN). By integrating spatiotemporal features extracted from remote sensing data, including MODIS Land Surface Temperature (LST), Surface Reflectance (SR), and Land Cover (LC), DeepYield outperforms traditional methods and demonstrates more precise forecasting accuracy for soybean yields across the Contiguous United States (CONUS). Zare et al. ____ investigate the impact of data assimilation techniques on improving crop yield predictions by assimilating LAI data into three single crop models and their multimode ensemble using a particle filtering algorithm. Results from a case study in southwestern Germany reveal that data assimilation significantly enhances LAI simulation accuracy and grain yield prediction, particularly for certain crop models, highlighting the potential for further improvements in data assimilation applications through regional model calibration and input uncertainty analysis.

Moreover, Gopi and Karthikeyan ____ introduce the Red Fox Optimization with Ensemble Recurrent Neural Network for Crop Recommendation and Yield Prediction (RFOERNN-CRYP) model, which leverages deep learning methods to provide automated crop recommendations and yield predictions. By employing ensemble learning with three different deep learning models (LSTM, bidirectional LSTM (BiLSTM), and gated recurrent unit (GRU)) and optimising hyperparameters using the RFO algorithm, the proposed model demonstrates improved performance compared to individual classifiers, offering valuable support for farmers in decision-making processes related to crop cultivation. From a similar perspective, Boppudi et al. ____ propose a deep ensemble model for accurately predicting crop yields in India, addressing the challenge posed by variations in weather and environmental factors. The model (Deep Max Out, Bi-GRU and CNN) incorporates improved preprocessing techniques, feature selection using the IBS-BOA algorithm, and prediction through a combination of Deep Ensemble Model and Ensemble classifiers, resulting in significantly reduced error rates compared to existing methods.

Recently, Umamaheswari and Madhumathi ____ applied a stacking ensemble approach using regressors like SVR, KNN, and RF as base learners, with LASSO regression as the meta-learner, achieving enhanced prediction precision. Osibo et al. ____ demonstrated the integration of weighted ensemble methods with remote sensing data, achieving better results compared to state-of-the-art models while simplifying data integration. Zhang et al. ____ constructed the StackReg framework, combining UAV-acquired multispectral data with ridge regression, SVM, Cubist, and XGBoost, which consistently outperformed base models, particularly in multi-stage settings. These studies highlight the critical role of ensemble learning in improving prediction reliability, offering versatile solutions adaptable to diverse agricultural contexts.

In the sequel, we introduce a novel framework for predicting crop yields, named ``\textit{RicEns-Net}." This framework incorporates advanced data engineering processes involving five unique data sources, namely Sentinel 1/2/3, NASA's Goddard Earth Sciences (GES) Data and Information Services Centre (DISC), and field measurements. The novelty of RicEns-Net lies in its integration of these diverse and rich sources of multi-modal data, comprising 15 features selected from a pool of over 100, within an advanced Deep Ensemble model. This model encompasses widely-used CNN and MLP architectures, as well as less explored DenseNet and Autoencoder architectures, which have been infrequently utilised or entirely unexplored in existing literature.

\subsection{Objectives \& Contributions}
The key objectives and related contributions of this research are as follows:
\begin{itemize}
    \item Integrate diverse remote sensing and meteorological data to enhance the accuracy and reliability of crop yield forecasting.
    
    \textit{Contribution:} We successfully unified radar, optical imagery, and meteorological data into a coherent multimodal dataset. This integration demonstrated the practical benefits of combining diverse data sources, providing a foundation for robust and accurate crop yield forecasting models.
    
    \item Address feature complexity and dimensionality challenges through advanced feature engineering and selection.
    
    \textit{Contribution:} We developed and implemented novel feature engineering and selection techniques, effectively reducing feature dimensionality while preserving essential predictive attributes. This contribution advanced the methodology for handling multimodal datasets in agricultural forecasting.
    
    \item Develop a novel deep ensemble learning model to improve the precision of crop yield prediction using multimodal data sources.
    
    \textit{Contribution:} We designed a deep ensemble learning framework that leverages data from SAR, MSI, and meteorological sources. The model achieved state-of-the-art performance, significantly enhancing precision and reliability in crop yield prediction tasks.
    
    \item Demonstrate the effectiveness of multimodal data fusion through performance comparisons with state-of-the-art machine learning techniques.
    
    \textit{Contribution:} We performed extensive benchmarking of the proposed model against leading machine learning techniques, validating the effectiveness of multimodal data fusion. The results highlighted the higher performance of the model and provided insights into the role of data integration in improving predictive accuracy.
\end{itemize}