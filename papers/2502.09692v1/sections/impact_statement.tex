\section{Impact Statement}
Neural network-based simulation surrogates will play an important and potentially transformative role across many industries. 
Once trained, a surrogate model enables faster design-cycle times in iterative verification and optimization applications. 
The reduction of cycle times in turn allows to explore a broader design space, potentially yielding better and more efficient designs. 
In this paper, we focus on automotive aerodynamics, where surrogate-optimized designs can contribute to increasing the range of electrical vehicles, thereby having a direct, positive impact on carbon emission reduction. 
Simulation surrogate models can imitate numerical \ac{CFD} simulations within a matter of a few seconds compared to several hours or days required by traditional methods. 
This implies that the surrogate models will also help to save compute (orders of magnitude) currently still invested into high-fidelity numerical simulations. 
We also want to emphasize that aerodynamics surrogate models, in particular, are a classic example of a dual-use technology that can be used for both, civilian as well as military applications.
We want to explicitly state, that the latter is not our intention.
