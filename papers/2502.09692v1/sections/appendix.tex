\section{Results}





\subsection{Evaluation of generated images} \label{subsection: Evaluation of generated images}

%The generative capabilities of the model are evaluated with quantitative metrics and with a visual Turing test. 
%To fairly evaluate the generated images directly to the reference CAMUS images, the evaluation experiments in this subsection do not use the sector width augmentations explained in subsection \ref{subsection: Training of the DDPM}. These additional augmentations do not affect the metrics by a lot, but would \newline


The ImageNet Fréchet inception distance (FID) \cite{heusel2017gans} and inception score (IS) \cite{salimans2016improved}
of the diffusion model are 23.87 and 1.47 respectively. However, these metrics can give misleading results for generative models that are not trained on ImageNet \cite{deng2009imagenet, barratt2018note, rosca2017variational}. To qualitatively assess the performance of the model, Fig.~\ref{fig: similar_samples} shows random samples generated together with the most similar cases from the CAMUS dataset identified automatically using the structural similarity index measure (SSIM) \cite{wang2004image}. This shows the model does not simply memorize cases from the training set, and produces realistic and varied samples. \newline




\begin{figure*}[h]
\centering
  \centering
  \includegraphics[trim={0.75cm 0.25cm 0.75cm 0.25cm}, clip,width = 1\linewidth]{figures/similar_samples_v2.drawio.pdf}
  \caption{Generated samples, together with most similar cases in the train and validation set and the test set of the CAMUS dataset, based on SSIM \cite{wang2004image}.}
  \label{fig: similar_samples}
\end{figure*}

\subsection{Survey results}

On the 45 pairs with one real and one synthetic image, participants correctly identified the synthetic image 56.4\% of the time. When broken down by group, cardiologists achieved an accuracy of 63.7\%, while clinical researchers and engineers both identified the correct frame 53.3\% of the time. Fig.~\ref{fig: survey} shows the explanations given when the participants correctly identified the synthetic frame, when they were wrong, and when both frames were real in the 5 cases mentioned above.
\newline

Using a binomial test with a significance level of 5\%, the accuracy of the cardiologists was found to be statistically significantly higher than random guessing ($P=0.09\%$). However, the engineers and clinical researchers in the survey did not show statistically significant higher accuracy compared to random guessing ($P=24.6\%$).

\begin{figure*}[h]
\centering
  \centering
  \includegraphics[trim={0cm 0cm 0cm 0cm}, clip,width = 1\linewidth]{figures/combined_reasons_grouped_barplot.pdf}
  \caption{Explanations given during the survey}
  \label{fig: survey}
\end{figure*}





%In the visual Turing test experiment, two ultrasound engineers and one clinician were shown an image and asked to determine whether it was real or synthetic. The images consisted of 50 frames sampled randomly from the CAMUS dataset and 50 frames sampled from the DDPM trained on the CAMUS dataset. These 100 images were presented to the respondents in a random order. If the respondents identified a frame as synthetic, they had to select a reason from the pre-defined options "Anatomically incorrect," "Speckle patterns," or "Image artifacts." If none of these options matched their reasoning, they could select "Other" and provide an explanation in a text field. Figure \ref{fig: survey} shows the results of the survey.

%\begin{figure}
%     \centering
%     \begin{subfigure}[b]{0.8\linewidth}
%         \centering \includegraphics[trim={0.2cm 0.2cm 0.2cm 0.2cm}, clip, width=1\linewidth]{figures/survey_results.pdf}
%         \caption{Results of the visual Turing test.}
%     \end{subfigure}
%     \begin{subfigure}[b]{0.8\linewidth}
%         \centering \includegraphics[trim={0.2cm 0.2cm 0.2cm 0.2cm}, clip, width=1\linewidth]{figures/survey_reasons.pdf}\caption{Explanations given for selecting synthetic}
%         \label{fig: survey_reasons}
%     \end{subfigure}
%   \caption{Results of the visual Turing test survey. For each image labeled as synthetic, the respondents where asked to indicate a reason for selecting synthetic.}
%    \label{fig: survey}
%\end{figure}

\subsection{Segmentation ablation study results}

Table \ref{table: ablation_study_1} shows the results of the ablation study on the CAMUS dataset, using Dice score and Hausdorff distance as metrics. The bottom part of Fig.~\ref{fig: heatmaps} shows the heatmaps of pixels belonging to the LV after applying the combination of all generative augmentations. Comparing these to the original illustrates that the generative augmentations increase the variety of LV location in the image. \newline

The increase in segmentation accuracy of the HUNT4 model on CAMUS originate mostly from an improvement in segmentation accuracy for samples outside the HUNT4 image distribution. Table \ref{table: camus_subsets_results} lists the segmentation results for the HUNT4 models on different subsets of CAMUS. The subsets are based on depth and sector angle cutoff values visualized in Figs.~\ref{fig: depths_hist} and \ref{fig: sector_angles_hist}.


%Subsection \ref{subsection: results of segmentation ablation study} contains the result of the ablation study. 



\begin{table*}[h]
\scriptsize
  \centering
  \renewcommand{\arraystretch}{1} % Increase vertical spacing
  \caption{Segmentation results of the ablation study using different datasets (HUNT4 and CAMUS) for training and testing. For all experiments, regular augmentations are applied in addition to the generative augmentations (see Table \ref{table: characteristics nnunet}).The Dice score and Hausdorff distance are only for the LV lumen label. We elaborate on this choice in the Discussion. Since the two datasets have been annotated by different experts with different annotation conventions, there is a considerably lower segmentation accuracy when the training and test sets are different. }
  \begin{tabular}{m{60pt}m{40pt}m{120pt}m{60pt}m{100pt}}
    \toprule
      Training set  & Test set & Generative Augmentations & Dice score & Hausdorff distance (mm)\\
    \midrule
    \multirow{7}{1.4cm}{HUNT4} & \multirow{7}{1.4cm}{CAMUS} & None & 0.802 $\pm$ 0.15 & 29.03 $\pm$ 26.01\\
    && Depth increase & 0.887 $\pm$ 0.05 & \textbf{7.49 $\pm$ 3.25} \\
    && Tilt variation  & 0.829 $\pm$ 0.14 & 17.31 $\pm$ 20.98 \\
    && Sector width & 0.847 $\pm$ 0.11 & 21.36 $\pm$ 23.84\\
    && Translation & 0.840 $\pm$ 0.12 & 16.55 $\pm$ 19.71 \\
    && Combination & \textbf{0.887 $\pm$ 0.05} & 8.17 $\pm$ 5.32 \\
    && Combination without repaint & 0.810 $\pm$ 0.15  & 26.90 $\pm$ 25.07\\
    \midrule
    \multirow{7}{1.4cm}{CAMUS} &  \multirow{7}{1.4cm}{CAMUS} & None & 0.943 $\pm$ 0.03 & 4.46 $\pm$ 2.52 \\
    && Depth increase & 0.945 $\pm$ 0.03 & \textbf{4.27 $\pm$ 2.34} \\
    && Tilt variation  &  0.945 $\pm$ 0.03 &  4.30 $\pm$ 2.43 \\
    && Sector width variation & \textbf{0.946 $\pm$ 0.03} & 4.34 $\pm$ 2.41 \\
    && Translation & 0.944 $\pm$ 0.03 & 4.44 $\pm$ 2.43\\
    && Combination & 0.944 $\pm$ 0.03 & 4.37 $\pm$ 2.43 \\
    && Combination without repaint & 0.934 $\pm$ 0.03 & 5.39 $\pm$ 2.85 \\

      \midrule
      \midrule
        \multirow{7}{1.4cm}{HUNT4} & \multirow{7}{1.4cm}{HUNT4} & None &  0.952 $\pm$ 0.02 &  3.34 $\pm$ 1.21 \\
    && Depth increase & 0.954 $\pm$ 0.02 & 3.24 $\pm$ 0.99 \\
    && Tilt variation & 0.954 $\pm$ 0.02 & 3.38 $\pm$ 1.06 \\
    && Sector width variation & 0.953 $\pm$ 0.02  & \textbf{3.23 $\pm$ 1.00} \\
    && Translation & 0.954 $\pm$  0.02  & 3.32 $\pm$  0.97 \\
    && Combination & \textbf{0.954 $\pm$ 0.02} & 3.31 $\pm$ 0.99 \\
    && Combination without repaint &  0.947 $\pm$ 0.02 & 4.14 $\pm$ 1.85 \\
    \midrule
    \multirow{7}{1.4cm}{CAMUS} &\multirow{7}{1.4cm}{HUNT4} & None & 0.886 $\pm$ 0.04 & 6.70 $\pm$ 1.81 \\
    && Depth increase & 0.891 $\pm$ 0.04 & 6.55 $\pm$ 1.84 \\
    && Tilt variation  & 0.887 $\pm$ 0.04 & 6.69 $\pm$ 1.91 \\
    && Sector width variation &  0.892 $\pm$ 0.04 & \textbf{6.54 $\pm$ 1.78} \\
    && Translation &  0.890 $\pm$ 0.04  & 6.55 $\pm$ 1.83   \\
    && Combination & \textbf{0.892 $\pm$ 0.04} & 6.59 $\pm$ 1.82 \\
    && Combination without repaint & 0.875 $\pm$ 0.04 & 7.71 $\pm$ 2.11 \\
    \bottomrule
  \end{tabular}
      \label{table: ablation_study_1}
\end{table*}



\begin{table*}
\scriptsize
  \centering
  \renewcommand{\arraystretch}{1} % Increase vertical spacing
  \caption{Segmentation results on different CAMUS subsets for a segmentation model trained on HUNT4 without generative augmentations and with the combination of all
  generative augmentations. }
  \begin{tabular}{m{100pt}m{120pt}m{100pt}m{100pt}}
    \toprule
      Training dataset   & CAMUS Test subset & Dice score & Hausdorff distance (mm)\\
        \midrule
      \multirow{4}{4cm}{HUNT4 without generative augmentations} & Depth $< 150$ mm ($n=1088$) &   0.855 $\pm$ 0.11  & 14.48 $\pm$ 16.61 \\
          &  Depth $\geq 150$ mm ($n=912$) & 0.729 $\pm$ 0.18 & 45.83 $\pm$ 30.19 \\
      &  Sector angle $< 70^\circ$ ($n=146$)& 0.869 $\pm$ 0.10 & 12.47 $\pm$ 16.47 \\
      &   Sector angle $\geq 70^\circ$ ($n=1854$) & 0.792 $\pm$ 0.16  & 30.06 $\pm$ 28.80 \\
      \midrule
      \multirow{4}{4cm}{HUNT4 with generative augmentations} & Depth $< 150$ mm ($n=1088$) &  \textbf{0.893 $\pm$ 0.05} & \textbf{7.45 $\pm$ 3.80}   \\
      &  Depth $\geq 150$ mm ($n=912$) & \textbf{0.886 $\pm$ 0.07} & \textbf{9.34 $\pm$ 8.37} \\
      &  Sector angle $< 70^\circ$ ($n=146$)& \textbf{0.893 $\pm$ 0.05}  & \textbf{7.11 $\pm$ 3.10} \\
      &   Sector angle $\geq 70^\circ$ ($n=1854$) &  \textbf{0.890 $\pm$ 0.07} & \textbf{8.40 $\pm$ 6.56} \\
    \bottomrule
  \end{tabular}
      \label{table: camus_subsets_results}
\end{table*}


% describe train on hunt4 and camus with augmentations. Also describe baseline of 'black' augmentations











\subsection{Clinical evaluation on HUNT4 results}


Similar to the segmentation results, the performance gains of the HUNT4 model originate mostly from an improvement in segmentation accuracy for frames outside the normal range. Fig.~\ref{fig: ef_main_text} shows the Bland-Altman plots comparing the manual reference EF with the automatic EF for segmentation models trained with and without generative augmentations for data both inside and outside of the HUNT4 acquisition normal range of depth $> 150$mm and sector angle $> 70^\circ$. Appendix \ref{appendix: exensive EF evaluation} contains additional analysis of automatic EF and also evaluates automatic on CAMUS.



%Figure \ref{fig: ef} shows the Bland-Altman plots comparing the automatic EF measurements with the manual reference for segmentation models trained without generative augmentations and with the combination of all generative augmentations. 









\section{Supplementary material for 3D predictions}
In Figure~\ref{fig:ahmedml_visualizations}, we visually compare the 3D GP-UPT predictions
of the experiments in Section~\ref{sec:experiments-volume}
with the corresponding ground truth \ac{CFD} simulation results on a test set sample.

Table~\ref{table:gino_3d_hyper_parameters} and \ref{table:gpupt_3d_hyper_parameters} we report the hyper-parameters for GINO and GP-UPT for the 3D volume prediction task (Section~\ref{sec:experiments-volume}).

\begin{figure*}[h!]
     \centering
     \begin{subfigure}[b]{0.43\textwidth}
         \centering
         \includegraphics[width=\textwidth,trim={0 1.2cm 1.4cm 1.7cm},clip]{figures/ahmed_3d/pred_cut.png}
         \caption{GP-UPT horizontal and vertical cut plane}
         \label{fig:ahmedml_visualizations_cut_plane_upt}
     \end{subfigure}
     \hfill
     \begin{subfigure}[b]{0.43\textwidth}
         \centering
         \includegraphics[width=\textwidth,trim={0 1.2cm 1.4cm 1.7cm},clip]{figures/ahmed_3d/gt_cut.png}
         \caption{CFD horizontal and vertical cut plane}
         \label{fig:ahmedml_visualizations_cut_plane_cfd}
     \end{subfigure}
     \hfill
     \begin{subfigure}[b]{0.43\textwidth}
         \centering
         \includegraphics[width=\textwidth,trim={0 2.5cm 0 1.0cm},clip]{figures/ahmed_3d/pred_iso.png}
         \caption{GP-UPT $C_{pt}=0.9$ iso-surface}
         \label{fig:ahmedml_iso_surface_upt}
     \end{subfigure}
        \hfill
     \begin{subfigure}[b]{0.43\textwidth}
         \centering
         \includegraphics[width=\textwidth,trim={0 2.5cm 0 1.0cm},clip]{figures/ahmed_3d/gt_iso.png}
         \caption{CFD $C_{tb}=0.9$ iso-surface}
         \label{fig:ahmedml_iso_surface_cfd}
     \end{subfigure}
        \caption{Visualization of GP-UPT total pressure coefficient $C_{pt}$ predictions (left) and ground truth HRLES CFD simulations (right).}
        \label{fig:ahmedml_visualizations}
\end{figure*}

\begin{table*}[h!]
\centering
\caption{Hyper-parameter configuration for \ac{GP-UPT} for 3D prediction.}
\begin{tabular}{cc}
\hline
Hyper-parameter           & AhmedML                    \\ \hline
Input normalization       & rescaling \\
Number of supernodes $S$     & 8000                             \\
Radius $r_{sn}$                        & 2                                     \\
Max degree supernode & 32                                    \\
Hidden dimensionality     & 768                               \\
Encoder blocks $K$          & 3                                   \\
Decoder blocks $C$           & 2                                   \\
Number of attention heads $h$ & 8                                   \\
Number of epochs & 4000                                   \\
Optimizer & LION \\
Learning rate & \num{5e-5}                                   \\
\hline
\end{tabular}
\label{table:gpupt_3d_hyper_parameters}
\end{table*}

\begin{table*}[h!]
\centering
\caption{Hyper-parameter configuration fo GINO for 3D prediction.}
\begin{tabular}{cc}
\hline
Hyper-parameter           & AhmedML                    \\ \hline
Input normalization       & rescaling \\
Grid resolution& 64x64x64\\
Hidden dimension & 384\\
Latent dimension & $64^3$ \\
Number of blocks & 2\\
Fourier modes & 24\\
Number of epochs & 4000                                   \\
Optimizer & LION \\
Learning rate & \num{5e-5}                                   \\
\hline
\end{tabular}
\label{table:gino_3d_hyper_parameters}
\end{table*}

\newpage
\section{Model inference characteristics}
\label{appendix:model_inference_characteristics}
To complement the quantitative model benchmark in terms of surface quantity prediction accuracy of Section~\ref{sec:experiments-benchmark}, we additionally provide profiling results, analyzing inference characteristics of the respective models.
Table \ref{table:inference-characteristics-drivaerml} summarizes model latency as well as peak memory consumption for inference forward passes of the DrivAerML experiments of Table~\ref{table:baseline-models}. 
Point-based models have the same number of input and output points, while for field-based methods we vary the number of output points only. 

\begin{table}[H]
\centering
\setlength{\tabcolsep}{1.5pt}
\caption{GPU peak memory usage and inference times measured on 40k points.}
\begin{tabular}{lcccc}
\toprule
 & & & \multicolumn{2}{c}{DrivAerML} \\ 
                       \cmidrule(lr){4-5}
Model - (\#params) & Field & Point & \multicolumn{1}{c}{Peak Memory [\si{\giga\byte}]} & \multicolumn{1}{c}{Model Latency [\si{\milli\second}]} \\ \hline
PointNet - (\num{3.5}M)      &  \xmark & \cmark    & 0.41 &2.47  \\   %
RegDGCNN %
- (\num{1.4}M) & \xmark & \cmark & 19.30 &171.90 \\ %
GINO - (15.7M) %
& \cmark & \xmark & 6.04 & 72.68 \\%& 2427.63  &  0.1378  & 24.05  \\                       
Transolver - (\num{3.9}M)    & \xmark & \cmark & 0.44& 24.26 \\ %
UPT - (\num{4.0}M)    & \xmark & \cmark & 0.86& 52.30 \\
\hline
\begin{tabular}[c]{@{}l@{}}GP-UPT  (\num{3.6}M) \end{tabular}& \xmark & \cmark & 0.86 & 34.04 \\ %
GP-UPT (\num{4.7}M) & \cmark  & \xmark & 0.86  & 69.00 \\% & 617.99 & 0.0694  & 12.38 \\
\bottomrule
\end{tabular}
\label{table:inference-characteristics-drivaerml}
\end{table}


\section{CAD mesh model fine tuning}
\begin{figure}[h!]
     \centering
     \begin{subfigure}[b]{0.85\textwidth}
         \centering
         \includegraphics[width=\textwidth]{figures/stl_fine_tuning_part1_page_1.pdf}
         \caption{Stage 1: Pre-trained GP-UPT model on CFD simulation mesh cells optimized with field- and supernode level multi-task learning.}
         \label{fig:cad_mesh_fine_tuning_stage1}
     \end{subfigure}
     \begin{subfigure}[b]{0.63\textwidth}
         \centering
         \includegraphics[width=\textwidth]{figures/stl_fine_tuning_part2_page_1.pdf}
         \caption{Stage 2: Fine-tuned model working with point clouds sampled directly for the CAD surface mesh.}
         \label{fig:cad_mesh_fine_tuning_stage2}
     \end{subfigure}
        \caption{Overview of the two-stage process for CAD mesh fine tuning.}
        \label{fig:cad_mesh_fine_tuning}
\end{figure}
\label{appendix:cad_mesh_fine_tuning}
We emphasize the practical implication (benefits) of models with a decoupled geometry encoder and field decoder in Section~\ref{sec:methodology}, as well as experiments (Section~\ref{sec:experiments}) of the main paper.
In this additional experiment, we demonstrate how to utilize this property to produce a model that is capable of directly inferring aerodynamic properties from a provided \acf{CAD} geometry (e.g., an STL file).
Note that this setup produces a model that does not depend on expensive \ac{CFD} simulation meshing when deployed in down-stream applications such as drag coefficient estimation for design verification and optimization. 

\textbf{Motivation}: When estimating the latency of a \ac{CFD} surrogate model for a new design,
we need to take two main computations into account:
\begin{enumerate*}[label=(\arabic*)]
    \item  input pre-processing including surface meshing, and
    \item the model inference forward pass:~$t_{CFD\,surrogate} = t_{meshing} + t_{inference}$.
\end{enumerate*}
Hence, circumventing the expensive \ac{CFD} meshing stage drastically reduces the time for initial feedback on the aerodynamics of a design (e.g., a few seconds vs. up to 1 hour~\cite{ashton2024drivaerml}).
Next, we explain the two-stage training procedure yielding such a model.
Figure \ref{fig:cad_mesh_fine_tuning} provides a visual summary of both stages.

\paragraph{Stage 1:} In this stage, we train a model as described in the methods section (Section~\ref{sec:methodology}) and as outlined in Figure~\ref{fig:gp-upt-architecture}.
For a direct comparison with Stage 2, we show a simplified version of this model in Figure~\ref{fig:cad_mesh_fine_tuning_stage1}.
Recall, that for utilizing field- and supernode-level multi-task learning, we need to feed the CFD simulation cell centers as an input to both
the geometry encoder and the field decoder for computing losses and optimization gradients.
Once trained, the query points for the field decoder can be arbitrary points
sampled from the surface of the geometry.
However, the geometry encoder was trained on, and hence still expects,
the point cloud distributions originating from the CFD simulation cell centers
(i.e., denser sampling of regions around mirrors or wheels).
In the next stage, we describe how to overcome this practical limitation of the model.

\paragraph{Stage 2:} Given the trained model of the previous stage, we now fine-tune it to accept, instead of the \ac{CFD} simulation cell centers, uniformly sampled points from the CAD surface as an input.
For training and evaluation purposes we still keep the simulation cell centers as query locations although this is not required when deploying the model in practice.
Note that in this fine-tuning stage, we need to drop the supernode-level component loss ($\cL_{point}$) of the multi-task objective ($\cL_{multi}$) as there is no relation between geometry input points and simulated surface quantities.
Figure~\ref{fig:cad_mesh_fine_tuning_stage2} provides an overview of the inference process of such a model.
In Table~\ref{tab:cad_mesh_fine_tuning} we compare the performance of the original model with a fine-tuned Stage 2 \ac{CAD} inference model.
We observe that, in addition to the beneficial properties with respect to practical applicability,
the \ac{CAD} inference model also maintains the same prediction accuracy on the respective surface quantities.
\begin{table}[h!]
\centering
\caption{Field decoder results for GP-UPT base model and CAD mesh fine-tuning on DrivAerML.}
\begin{tabular}{lccc}
\toprule
Model (\#params)& MSE & L2 &MAE \\
\hline
GP-UPT (4.7M) & 314.25 & 0.0497 &10.38 \\
GP-UPT-CAD (4.7M) & 302.39 & 0.0486 & 10.16 \\
\bottomrule
\end{tabular}
\label{tab:cad_mesh_fine_tuning}
\end{table}

\clearpage

\section{Surface mesh resolutions for drag and lift estimation}
\label{appendix:mesh_resolutions}
In Section~\ref{sec:experiments-surface}, we evaluate the impact of different surface mesh resolutions on drag coefficient estimation.
For a better intuition on the granularity of the respective surface meshes we show the first four re-meshing stages in Figure~\ref{fig:surface_remeshing}.
\begin{figure}[h!]
     \centering
     \begin{subfigure}[b]{0.47\textwidth}
         \centering
         \includegraphics[width=\textwidth]{figures/meshes/N3k.png}
         \caption{$3\cdot10^3$ mesh cells}
         \label{fig:surface_remeshing_1}
     \end{subfigure}
     \hfill
     \begin{subfigure}[b]{0.47\textwidth}
         \centering
         \includegraphics[width=\textwidth]{figures/meshes/N10k.png}
         \caption{$10^4$ mesh cells}
         \label{fig:surface_remeshing_1}
     \end{subfigure}
     \hfill
     \begin{subfigure}[b]{0.47\textwidth}
         \centering
         \includegraphics[width=\textwidth]{figures/meshes/N30k.png}
         \caption{$3\cdot10^4$ mesh cells}
         \label{fig:surface_remeshing_2}
     \end{subfigure}
     \begin{subfigure}[b]{0.47\textwidth}
         \centering
         \includegraphics[width=\textwidth]{figures/meshes/N100k.png}
         \caption{$10^5$ mesh cells}
         \label{fig:surface_remeshing_3}
     \end{subfigure}
     \hfill
     \begin{subfigure}[b]{0.47\textwidth}
         \centering
         \includegraphics[width=\textwidth]{figures/meshes/N300k.png}
         \caption{$3 \cdot 10^5$ mesh cells}
         \label{fig:surface_remeshing_4}
     \end{subfigure}
     \hfill
     \begin{subfigure}[b]{0.47\textwidth}
         \centering
         \includegraphics[width=\textwidth]{figures/meshes/N1000k.png}
         \caption{$10^6$ mesh cells}
         \label{fig:surface_remeshing_4}
     \end{subfigure}
    \caption{Visualization of different surface mesh resolutions for the drag coefficient experiments in Section~\ref{sec:experiments-surface}.}
    \label{fig:surface_remeshing}
\end{figure}
