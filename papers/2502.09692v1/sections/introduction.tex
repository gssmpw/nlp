\begin{figure}[ht!]
    \centering
    \includegraphics[width=0.99\linewidth]{figures/fig1_final.pdf}
    \caption{The NeuralCFD paradigm. We introduce \acf{GP-UPT} for industry scale \ac{CFD} prediction, where geometry encoding and field decoding are disentangled and handled by distinct parts of the model. This allows us to: (i) infer the model’s latent space from a limited number of sample points, which are representative of the raw geometry input, i.e., \ac{GP-UPT} entirely circumvents the need for simulation re-meshing; (ii) predict (decode) values on arbitrary (number of) points (sampling patterns) on the geometry or in the respective 3D volume. Point-wise predictions can be inferred in parallel from a cached latent space representation.}
    \label{fig:fig1}
\end{figure}
\section{Introduction}
\label{sec:introduction}

\Acf{CFD} is central to automotive aerodynamics, offering in-depth analysis of entire flow fields, and complementing wind tunnels by simulating open-road conditions. 
The fundamental basis of almost all \ac{CFD} simulations is the \ac{NS} equations, describing the motion of viscous fluid substances around objects.
However, the computational cost of solving the \ac{NS} equations necessitates modeling approximations, most notably regarding the onset and effects of turbulence. 
Therefore, \ac{CFD} employs different turbulence modeling strategies, balancing accuracy and cost. 
In this context, two seminal datasets, DrivAerNet \cite{elrefaie2024drivaernet, elrefaie2024drivaernet++} and DrivAerML \cite{ashton2024drivaerml}, have been released, allowing for in-depth study of deep learning surrogates for automotive aerodynamics. 
DrivAerNet runs \ac{CFD} simulations on 8 to 16 million volumetric mesh cells with low-fidelity \ac{RANS} methods \cite{reynolds1895iv, alfonsi2009reynolds, ashton2015comparison}, whereas DrivAerML runs \ac{CFD} simulations on 160 million volumetric cells with \ac{HRLES}~\cite{spalart2006new, chaouat2017state, heinz2020review, ashton2022hlpw}, which is the highest-fidelity \ac{CFD} approach routinely deployed by the automotive industry \cite{hupertz2022towards,ashton2024drivaerml}.

In recent years, deep neural network-based surrogates have emerged as a computationally efficient alternative in science and engineering~\cite{Thuerey:21,Zhang:23,brunton2020machine}, impacting e.g., weather forecasting~\cite{pathak2022fourcastnet,bi2023accurate,lam2023learning,Nguyen:23,bodnar2024aurora}, protein folding~\cite{jumper2021highly,abramson2024accurate}, or material design~\cite{merchant2023scaling, zeni2023mattergen, yang2024mattersim}.
In automotive aerodynamics, however, key challenges must be overcome before deep neural network-based surrogates can be implemented at an industry scale:
\begin{enumerate}[label={(\Roman*)}, noitemsep,topsep=0pt]
\item \label{challenge1} Surrogates must be able to deliver accurate predictions and be scalable to large surface and volume meshes, ideally taking raw geometries as inputs,
i.e., without relying on the \ac{CFD} simulation meshing procedure.
\item \label{challenge2} Surrogates must be capable of achieving the required performance levels while being trainable with a limited number of samples, as ground-truth numerical simulation datasets are both scarce and costly to generate.
\end{enumerate}
In order to address these challenges, we introduce \acf{GP-UPT}, the first neural operator designed to provide scalable solutions for high-fidelity aerodynamics simulations. 
\ac{GP-UPT} separates geometry encoding and physics predictions, ensuring flexibility with respect to the geometry representations and surface sampling strategies.
It builds on the \acf{UPT}~\cite{alkin2024universal} framework, which operates without grid- or particle-based latent structures, enabling flexibility and scalability across meshes and particles. \ac{GP-UPT} extends this framework by:
\begin{enumerate*}[label={(\roman*)}]
    \item preserving geometry information when encoding, and
    \item guiding output predictions to conform to the input geometry.
\end{enumerate*}
\ac{GP-UPT} enables independent scaling of the respective model parts (e.g., encoder or decoder) according to the practical requirements.
Qualitatively, \ac{GP-UPT} demonstrates favorable performance and scaling compared to state-of-the-art neural operators, offering a clear solution to open challenges of automotive aerodynamics. 
Key highlights include: 
\begin{enumerate*}[label={(\roman*)}]
    \item converging model outputs for different input sampling patterns;
    \item achieving the first near-perfect accuracy in drag and lift coefficient predictions relative to numerical \ac{CFD} simulations, where predictions across the entire surface meshes of DrivAerML (8.8 million surface \ac{CFD} mesh cells) are obtained within seconds on a single GPU;
    \item attaining accurate 3D velocity field predictions at 20 million mesh cells, even when only the geometry representation is input to the model;
    \item establishing a low-fidelity to high-fidelity simulation transfer learning approach, requiring only half of the high-fidelity data to match the performance of models trained from scratch.
\end{enumerate*}
