\section{Conclusion and outlook}
\label{sec:conclusion}
This paper is motivated by two key challenges that need to be overcome for industry scale applicability of neural network-based \ac{CFD} surrogate models.
Following these challenges, we define architectural requirements for such models and introduce \ac{GP-UPT}.
By decoupling geometry encoding and physics predictions, \ac{GP-UPT} ensures flexibility with respect to geometry representations and surface sampling strategies.
Given the proposed architecture, we show that:
\begin{enumerate*}[label=(\arabic*)]
    \item both, surface-level, as well as volume predictions, work out of the box without requiring any changes to the architecture.
    \item Encoder - decoder - decoupling allows us to infer on arbitrary meshes circumventing the need for expensive \ac{CFD} meshing while still producing accurate estimates for integral quantities such as drag and lift.
    \item Transfer learning substantially reduces data requirements and additionally is beneficial for overall performance.
\end{enumerate*}
In future work, we aim to transfer the general \ac{GP-UPT} framework to different application domains and different underlying simulations beyond \ac{CFD} and aerodynamics. 
