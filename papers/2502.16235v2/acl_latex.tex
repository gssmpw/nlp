% This must be in the first 5 lines to tell arXiv to use pdfLaTeX, which is strongly recommended.
\pdfoutput=1
% In particular, the hyperref package requires pdfLaTeX in order to break URLs across lines.

\documentclass[11pt]{article}

% Change "review" to "final" to generate the final (sometimes called camera-ready) version.
% Change to "preprint" to generate a non-anonymous version with page numbers.

\usepackage[dvipsnames,svgnames]{xcolor}

\usepackage[preprint]{acl}
% \usepackage[review]{acl}

% Standard package includes
\usepackage{times}
\usepackage{latexsym}
\usepackage{booktabs}
\usepackage{marvosym}

% For proper rendering and hyphenation of words containing Latin characters (including in bib files)
\usepackage[T1]{fontenc}
% For Vietnamese characters
% \usepackage[T5]{fontenc}
% See https://www.latex-project.org/help/documentation/encguide.pdf for other character sets

% This assumes your files are encoded as UTF8
\usepackage[utf8]{inputenc}

% This is not strictly necessary, and may be commented out,
% but it will improve the layout of the manuscript,
% and will typically save some space.
\usepackage{microtype}

% This is also not strictly necessary, and may be commented out.
% However, it will improve the aesthetics of text in
% the typewriter font.
\usepackage{inconsolata}

%Including images in your LaTeX document requires adding
%additional package(s)
\usepackage{graphicx}

\usepackage{amsmath}
\usepackage{algorithm}
\usepackage{algorithmic}

\usepackage{subcaption} % for 'subtable' env
\usepackage{multirow} % for cmd 'multirow', 'multicolumn'
\usepackage{multicol}
\usepackage{colortbl}

\usepackage{enumerate}
\usepackage{enumitem}
\usepackage{amssymb}
\usepackage{listings}
\usepackage{minitoc} 


\lstset{
  language=Python,
  backgroundcolor=\color{lightgray},
  basicstyle=\ttfamily\small,
  breaklines=true,
  numbers=left,
  numberstyle=\tiny,
  captionpos=b,
  frame=single,
  rulecolor=\color{black},
  showstringspaces=false
}

\newcommand{\dyf}[1]{\textcolor{black}{#1}}

\definecolor{jwtcolor}{RGB}{0, 123, 255}
\newcommand{\jwt}[1]{{\color{jwtcolor}#1}}

\renewcommand{\algorithmicrequire}{\textbf{Input:}}
\renewcommand{\algorithmicensure}{\textbf{Output:}}

% ======================
\newcommand{\yc}[1]{\textcolor{red}{#1}}
\newcommand{\ycj}[1]{\textcolor{red}{\bf [Comments: #1] }} 
% ======================

% If the title and author information does not fit in the area allocated, uncomment the following
%
%\setlength\titlebox{<dim>}
%
% and set <dim> to something 5cm or larger.

\makeatletter
\def\thanks#1{\protected@xdef\@thanks{\@thanks\protect\footnotetext{#1}}}
\makeatother
% \title{Hybrid Prior-Posterior Tree Search: Pushing the Boundaries of Inference Scaling Laws for LLM Reasoning}
% \title{Explore \& Exploit Tree Search: Eliminating the Redundant Branch for Parallel LLM Reasoning }
\title{Dynamic Parallel Tree Search for Efficient LLM Reasoning}

% Dynamic Parallel Reasoning


% Author information can be set in various styles:
% For several authors from the same institution:
% \author{Author 1 \and ... \and Author n \\
%         Address line \\ ... \\ Address line}
% if the names do not fit well on one line use
%         Author 1 \\ {\bf Author 2} \\ ... \\ {\bf Author n} \\
% For authors from different institutions:
% \author{Author 1 \\ Address line \\  ... \\ Address line
%         \And  ... \And
%         Author n \\ Address line \\ ... \\ Address line}
% To start a separate ``row'' of authors use \AND, as in
% \author{Author 1 \\ Address line \\  ... \\ Address line
%         \AND
%         Author 2 \\ Address line \\ ... \\ Address line \And
%         Author 3 \\ Address line \\ ... \\ Address line}

% Yifu Ding, Wentao Jiang, Shunyu Liu, Yongcheng Jing, Jinyang Guo, Yingjie Wang, Jing Zhang, Zengmao Wang, Ziwei Liu, Bo Du, Xianglong Liu, Dacheng Tao

\author{Yifu Ding\textsuperscript{1,2}, Wentao Jiang\textsuperscript{3}, Shunyu Liu\textsuperscript{2}, Yongcheng Jing\textsuperscript{2}, Jinyang Guo\textsuperscript{1}, Yingjie Wang\textsuperscript{2}, \\
{\bf Jing Zhang\textsuperscript{3}, Zengmao Wang\textsuperscript{3,*}, Ziwei Liu\textsuperscript{2}, Bo Du\textsuperscript{3}, Xianglong Liu\textsuperscript{1,*}, Dacheng Tao\textsuperscript{2,*}}
  % \texttt{} \\
  % Affiliation / Address line 1 \\
  % Affiliation / Address line 2 \\
  % Affiliation / Address line 3 \\
  % \texttt{email@domain} \\
  \thanks{
  \textsuperscript{*}Corresponding author. \textsuperscript{1}Beihang University, China. 
  \textsuperscript{2}Nanyang Technological University, Singapore. 
  \textsuperscript{3}Wuhan University, China. }
  }

%\author{
%  \textbf{First Author\textsuperscript{1}},
%  \textbf{Second Author\textsuperscript{1,2}},
%  \textbf{Third T. Author\textsuperscript{1}},
%  \textbf{Fourth Author\textsuperscript{1}},
%\\
%  \textbf{Fifth Author\textsuperscript{1,2}},
%  \textbf{Sixth Author\textsuperscript{1}},
%  \textbf{Seventh Author\textsuperscript{1}},
%  \textbf{Eighth Author \textsuperscript{1,2,3,4}},
%\\
%  \textbf{Ninth Author\textsuperscript{1}},
%  \textbf{Tenth Author\textsuperscript{1}},
%  \textbf{Eleventh E. Author\textsuperscript{1,2,3,4,5}},
%  \textbf{Twelfth Author\textsuperscript{1}},
%\\
%  \textbf{Thirteenth Author\textsuperscript{3}},
%  \textbf{Fourteenth F. Author\textsuperscript{2,4}},
%  \textbf{Fifteenth Author\textsuperscript{1}},
%  \textbf{Sixteenth Author\textsuperscript{1}},
%\\
%  \textbf{Seventeenth S. Author\textsuperscript{4,5}},
%  \textbf{Eighteenth Author\textsuperscript{3,4}},
%  \textbf{Nineteenth N. Author\textsuperscript{2,5}},
%  \textbf{Twentieth Author\textsuperscript{1}}
%\\
%\\
%  \textsuperscript{1}Affiliation 1,
%  \textsuperscript{2}Affiliation 2,
%  \textsuperscript{3}Affiliation 3,
%  \textsuperscript{4}Affiliation 4,
%  \textsuperscript{5}Affiliation 5
%\\
%  \small{
%    \textbf{Correspondence:} \href{mailto:email@domain}{email@domain}
%  }
%}

\begin{document}
\maketitle

\begin{abstract}


Despite the significant advances in neural machine translation, performance remains subpar for low-resource language pairs.
Ensembling multiple systems is a widely adopted technique to enhance performance, often accomplished by combining probability distributions.
However, the previous approaches face the challenge of high computational costs for training multiple models.
Furthermore, for black-box models, averaging token-level probabilities at each decoding step is not feasible.
To address the problems of multi-model ensemble methods, we present a pivot-based single model ensemble.
The proposed strategy consists of two steps: pivot-based candidate generation and post-hoc aggregation.
In the first step, we generate candidates through pivot translation.
This can be achieved with only a single model and facilitates knowledge transfer from high-resource pivot languages, resulting in candidates that are not only diverse but also more accurate.
Next, in the aggregation step, we select $\textit{k}$ high-quality candidates from the generated candidates and merge them to generate a final translation that outperforms the existing candidates.
Our experimental results show that our method produces translations of superior quality by leveraging candidates from pivot translation to capture the subtle nuances of the source sentence.


\end{abstract}
\section{Introduction}

In recent years, with advancements in generative models and the expansion of training datasets, text-to-speech (TTS) models \cite{valle, voicebox, ns3} have made breakthrough progress in naturalness and quality, gradually approaching the level of real recordings. However, low-latency and efficient dual-stream TTS, which involves processing streaming text inputs while simultaneously generating speech in real time, remains a challenging problem \cite{livespeech2}. These models are ideal for integration with upstream tasks, such as large language models (LLMs) \cite{gpt4} and streaming translation models \cite{seamless}, which can generate text in a streaming manner. Addressing these challenges can improve live human-computer interaction, paving the way for various applications, such as speech-to-speech translation and personal voice assistants.

Recently, inspired by advances in image generation, denoising diffusion \cite{diffusion, score}, flow matching \cite{fm}, and masked generative models \cite{maskgit} have been introduced into non-autoregressive (NAR) TTS \cite{seedtts, F5tts, pflow, maskgct}, demonstrating impressive performance in offline inference.  During this process, these offline TTS models first add noise or apply masking guided by the predicted duration. Subsequently, context from the entire sentence is leveraged to perform temporally-unordered denoising or mask prediction for speech generation. However, this temporally-unordered process hinders their application to streaming speech generation\footnote{
Here, “temporally” refers to the physical time of audio samples, not the iteration step $t \in [0, 1]$ of the above NAR TTS models.}.


When it comes to streaming speech generation, autoregressive (AR) TTS models \cite{valle, ellav} hold a distinct advantage because of their ability to deliver outputs in a temporally-ordered manner. However, compared to recently proposed NAR TTS models,  AR TTS models have a distinct disadvantage in terms of generation efficiency \cite{MEDUSA}. Specifically, the autoregressive steps are tied to the frame rate of speech tokens, resulting in slower inference speeds.  
While advancements like VALL-E 2 \cite{valle2} have boosted generation efficiency through group code modeling, the challenge remains that the manually set group size is typically small, suggesting room for further improvements. In addition,  most current AR TTS models \cite{dualsteam1} cannot handle stream text input and they only begin streaming speech generation after receiving the complete text,  ignoring the latency caused by the streaming text input. The most closely related works to SyncSpeech are CosyVoice2 \cite{cosyvoice2.0} and IST-LM \cite{yang2024interleaved}, both of which employ interleaved speech-text modeling to accommodate dual-stream scenarios. However, their autoregressive process generates only one speech token per step, leading to low efficiency.



To seamlessly integrate with  upstream LLMs and facilitate dual-stream speech synthesis, this paper introduces \textbf{SyncSpeech}, designed to keep the generation of streaming speech in synchronization with the incoming streaming text. SyncSpeech has the following advantages: 1) \textbf{low latency}, which means it begins generating speech in a streaming manner as soon as the second text token is received,
and
2) \textbf{high efficiency}, 
which means for each arriving text token, only one decoding step is required to generate all the corresponding speech tokens.

SyncSpeech is based on the proposed \textbf{T}emporal \textbf{M}asked generative \textbf{T}ransformer (TMT).
During inference, SyncSpeech adopts the Byte Pair Encoding (BPE) token-level duration prediction, which can access the previously generated speech tokens and performs top-k sampling. 
Subsequently, mask padding and greedy sampling are carried out based on  the duration prediction from the previous step. 

Moreover, sequence input is meticulously constructed to incorporate duration prediction and mask prediction into a single decoding step.
During the training process, we adopt a two-stage training strategy to improve training efficiency and model performance. First, high-efficiency masked pretraining is employed to establish a rough alignment between text and speech tokens within the sequence, followed by fine-tuning the pre-trained model to align with the inference process.

Our experimental results demonstrate that, in terms of generation efficiency, SyncSpeech operates at 6.4 times the speed of the current dual-stream TTS model for English and at 8.5 times the speed for Mandarin. When integrated with LLMs, SyncSpeech achieves latency reductions of 3.2 and 3.8 times, respectively, compared to the current dual-stream TTS model for both languages.
Moreover, with the same scale of training data, SyncSpeech performs comparably to traditional AR models in terms of the quality of generated English speech. For Mandarin, SyncSpeech demonstrates superior quality and robustness compared to current dual-stream TTS models. This showcases the potential of  SyncSpeech as a foundational model to integrate with upstream LLMs.


\section{Related Work}
% \label{sec:related_work}

\paragraph{Reasoning with LLMs. }
LLMs have evolved from System 1 tasks (e.g., translation)~\cite{Brown_2020_Language} to System 2 reasoning (e.g., math, logic)~\cite{Kojima_2022_Large}. CoT~\cite{Wei_2022_Chain} enhances multi-step reasoning, with variants like Self-Consistent CoT~\cite{wang2022self}, but its exploration scope remains constrained, limiting its effectiveness.~~\cite{chu2023survey}.
% \vspace{-4pt}
% \noindent\textbf{Tree Search for Reasoning. }
Furthermore, ToT~\cite{Yao_2023_Tree} enables multi-path exploration, leveraging MCTS~\cite{chaslot2008monte} for backtracking and heuristic rollouts~\cite{wan2024alphazero,wang2024q}. However, MCTS remains computationally expensive, with limited work on acceleration methods.

% \vspace{-pt}
\paragraph{LLM Inference Acceleration. }
While LLM inference has been optimized for linear decoding~\cite{lin2024awq}, tree-structured reasoning remains underexplored~\cite{li2024large}. Approaches like Deft~\cite{yao2024deft} optimize prefix sharing, while others use self-consistency for early stopping~\cite{li2024escape}. Efficient tree search for LLM reasoning remains an open challenge.

Due to page limit, we have included a more detailed discussion of related work 
% (1) Reasoning with LLMs, (2) Tree Search for Reasoning, and (3) LLM Inference Acceleration 
in Appendix~\ref{app:sec:related_work}. 
% \section{The Rationale of DPTS}
\section{Method Rationale}

\label{sec:motivation}

% \ycj{Rationale?}
% \ycj{Pre-analysis/Motivation and Problem Definition/Motivation and Challenge Analysis/Pilot Study?}
 
% \ycj{Add a short overview statement here.}


{
In this section, we present empirical findings that highlight the key challenges of tree search in LLM and provide the rationale behind our proposed DPTS. 
% First, the inherent sequential nature of tree search complicates parallel execution, leading to irregular node expansions and varying path lengths. Second, excessive exploitation of low-confidence paths wastes computational resources, suggesting that a confidence-based pruning strategy could mitigate this inefficiency. Finally, tree search methods that prioritize breadth often suffer from frequent switching between paths, hindering deep exploitation and resulting in token and expansion redundancy. Addressing these challenges can significantly enhance the efficiency and effectiveness of tree search methods.
}
First, the frequent switch between paths complicates parallel execution and causes shallow thinking, disrupting the model’s ability to engage in efficient deep reasoning (Sec.~\ref{sec:3.1}). 
% sequential nature of tree search complicates parallel execution, leading to irregular node expansions and varying path lengths. Existing tree search algorithms frequently switch between paths, causing shallow thinking and disrupting the model’s ability to engage in deep reasoning, ultimately degrading generation quality. 
Second, excessive exploitation of low-confidence paths results in redundant rollouts and wastes effort on fewer possible candidates (Sec.~\ref{sec:3.2}). 
% consuming unnecessary computational resources. Without an effective pruning strategy,..., reducing overall efficiency and slowing down search convergence.

% \subsection{Unpredictable Growth Behavior}
\subsection{Frequent Switching}
\label{sec:3.1}

\begin{figure}[t]
    \centering
    \includegraphics[width=0.93\linewidth]
    {figs/draw_switch_times.pdf}
    \vspace{-0.1in}
    \caption{Statistics for switch from the best path to the suboptimal (blue), and total switch (green).} 
    \vspace{-0.1in}
    \label{fig:motivation_switch_path}
\end{figure}
% \ycj{Analysis of why vanilla solution doesn't work well?}

% Tree search has become one of the key paradigms for enabling deep reasoning in large language models. However, the inherent sequential nature of tree structures presents significant challenges for GPU parallelism. Regardless of the search algorithm employed, the hierarchical and distributed nature of tree-structured data introduces intrinsic difficulties in memory management—whether using a contiguous memory pool or fragmented memory blocks, the irregularity of tree search remains unavoidable.  % 这一段说的 tree-structured data 难以 parallel

Tree search inherently exhibits retrospective and recursive behaviors, making efficient parallel execution difficult. Even if each node is constrained to generate the same number of tokens, the focus switching between different reasoning trajectories and the diverse path lengths makes it incompatible with the end-to-end parallelism on GPUs. The detailed illustrations for this phenomenon can be found in Appendix~\ref{sec:app:trees}. 

% This behavior demonstrates the fundamental characteristics of tree search that make the efficient parallelization a challenging problem: \textit{diverse path lengths, varying child depths, irregular node jumps, and recursive retrospective exploitation}. 

The focus switching between paths also makes the tree search fail in focused reasoning trajectory~\cite{wang2025thoughts}, which prevents deep thinking and leads to a tendency of shallow exploitation. We quantify the switch times of the reasoning focus on each sample in the Math500 dataset. Figure~\ref{fig:motivation_switch_path} counts the total switch, which is about 35 on average. As well as the switch from the best path to a suboptimal or incorrect one, which is up to 3 times for a single sample. It demonstrates the instability of the tree search algorithm in maintaining a focused reasoning trajectory. 


% 下面这段我放 appendix 了
\jwt{
% Tree search is key for enabling deep reasoning in large language models, but its sequential nature presents challenges for efficient GPU parallelism. Tree-structured data introduces difficulties in memory management, with irregular patterns in node expansions and varying path lengths. This leads to difficulties in maintaining parallelism, as tree search is inherently recursive and retrospective, causing inefficient execution when different paths vary in depth and termination points.

% Figure~\ref{fig:motivation_dfs_trees} visualizes depth-first search (DFS) trees, highlighting the irregular expansion process. Darker nodes represent high-confidence paths, while lighter nodes indicate lower-confidence ones. The figure demonstrates that tree search does not follow a predictable spatial or hierarchical pattern, resulting in diverse path lengths, varying child depths, and irregular node jumps. These behaviors complicate parallel execution and efficient resource utilization.

% To address these challenges, it is crucial to focus on improving the handling of irregular growth in tree structures. By introducing strategies that can better manage the dynamic nature of tree search, we can optimize memory usage and reduce the inherent complexity, enabling more effective parallelization.

}


% These findings highlight a critical shortcoming of BFS: frequent switching between different reasoning paths prevents deep exploitation, leading to significant computational redundancy. This behavior results in excessive token generation, unnecessary expansions on suboptimal paths, and instability in maintaining a coherent line of reasoning—factors that ultimately hinder search efficiency and inference speed.

% 下面这段我放 appendix 了
% \jwt{
%  Tree search focusing on width expansion explores a wide range of paths but suffers from frequent switching between them, preventing deep reasoning and leading to shallow exploitation. This behavior causes two inefficiencies: incomplete reasoning and excessive expansions. These algorithms often generates more tokens and expansions than necessary, exploring many suboptimal paths before finding the best one, which results in significant computational redundancy.

%  Figure~\ref{fig:motivation_bfs_trees} shows how BFS expands in a flat, top-down manner, leading to shallow exploitation. Figure~\ref{fig:motivation_waste_tokens}(left) compares the total tokens generated (blue line) to those required for the best path (yellow line), revealing excessive token redundancy. Figure~\ref{fig:motivation_waste_tokens}(right) highlights unnecessary node expansions, where many explored nodes do not contribute to the final solution. Finally, Figure~\ref{fig:motivation_switch_path} analyzes node-switching frequency, showing that BFS frequently shifts between optimal and suboptimal paths, leading to instability in reasoning.

%  The high frequency of path-switching can be mitigated by introducing mechanisms that help the model maintain focus on the most promising paths.
% }



\subsection{Redundant Exploration}
\label{sec:3.2}


The lack of early termination in existing tree search algorithms leads to excessive exploitation and redundant searching. Observations in Figure~\ref{fig:motivation_low_confidence} show that low-confidence nodes rarely contribute to the best solutions, either terminated with suboptimal results (yellow) or failing to be the first to reach the best path (orange). The average probability of the suboptimal results brought by low confidence is 91.3\%, while the probability of those nodes being the earliest best path is only 6.2\%. It suggests that low-confidence nodes have little potential to reach the best solution, it is even hard to be the first one. It means that most low-confidence nodes have less contribution to the final results but waste computational resources. 
% We reorder the nodes based on the former probability for clearer illustration (the plot with original order is showcased in Appendix~\ref{app:sec:low_reward_original}). 

% One of the fundamental inefficiencies in traditional tree search lies in its inability to terminate early when exploring suboptimal paths. In most cases, a search path is only abandoned when it reaches the termination condition, regardless of whether it is already apparent that the path is unlikely to yield an optimal solution. This behavior can lead to substantial computational redundancy, as a large number of unnecessary expansions and token generations are performed on low-confidence nodes. 


% To quantify the extent of this redundancy, we conducted an observational experiment. In Figure~\ref{fig:motivation_low_confidence}, we analyze the probability that continuing the rollout from a low-confidence node leads to less contribution. Here, we define a node as low confidence if its score is lower than the average confidence of previously visited nodes (refer to Eq. (\ref{eq:theta}))—a deliberately aggressive threshold since such occurrences are relatively frequent (highlighted in yellow). And we also reorder the nodes based this probability for clearer illustration (the plot with original order is showcased in Appendix~\ref{app:sec:low_reward_original}). However, as our observations indicate, the probability that these low-confidence nodes ultimately contribute to the optimal path is extremely low (highlighted in orange). Additionally, in most cases where the optimal path is eventually reached, it is not the first time an optimal solution is discovered (highlighted in blue), meaning that a higher-confidence node had already identified the correct path earlier. 

% Between the yellow and orange regions lies an additional scenario: the probability that continuing a rollout from a low-confidence node either fails to generate an answer at termination or produces an incorrect answer. This category accounts for the majority of cases, further reinforcing the inefficiency of expanding low-confidence paths.

% Based on these observations, it suggest that a node’s prior confidence may serve as a reasonable predictor of whether the search path will ultimately lead to a valid solution. While this does not guarantee a perfect pruning strategy, it indicates that integrating confidence-based heuristics could significantly reduce unnecessary rollouts, improving the overall efficiency of tree search methods.

% 合并到上面的黑色文本里了
\jwt{
% A major inefficiency in traditional tree search is the lack of early termination when exploring low-confidence paths. This results in wasted computational resources, as the algorithm continues expanding these paths even when their confidence scores remain low. Observations show that low-confidence nodes rarely contribute to optimal solutions, suggesting that confidence-based pruning can reduce unnecessary exploitation and improve efficiency.

% Figure~\ref{fig:motivation_dfs_trees} (Tree 2) illustrates the issue, showing that low-confidence nodes (rightmost branches) are expanded despite their low likelihood of contributing to the final answer. Figure~\ref{fig:motivation_low_confidence} further quantifies this inefficiency: yellow regions indicate frequent occurrences of low-confidence nodes, while orange regions show that they rarely lead to optimal solutions. The blue regions reveal that even when the best path is found, a higher-confidence node had typically identified it earlier, confirming the redundancy of expanding low-confidence nodes.

% These findings emphasize the importance of selectively pruning low-confidence paths early in the search process. By incorporating mechanisms that assess the likelihood of a node contributing to the optimal solution, unnecessary expansions can be avoided, leading to a significant reduction in computational overhead.
}

\begin{figure}[t]
    % \centering
    \includegraphics[width=0.85\linewidth]{figs/low_reward_original.pdf}
    \vspace{-0.1in}
    \caption{Probabilities with reordered samples of those have prior confidence below $\theta_{es}(\lambda=1)$ in Eq.~\ref{eq:theta} and do not terminate with the highest reward score (yellow), and paths that are not the earliest best path (orange), which means there is already at least one path that has terminated with the same reward score. }
    \vspace{-0.2in}
    \label{fig:motivation_low_confidence}
\end{figure}

These findings emphasize the importance of maintaining the focus on deep reasoning and pruning low-confidence paths for efficient inference. 


%%%%%%%%%%%%%%%%%%%%%%%%%%%%%%%%%%%%%%%%%%%%%%%%%%%
% Model
%%%%%%%%%%%%%%%%%%%%%%%%%%%%%%%%%%%%%%%%%%%%%%%%%%%
\section{\Method Design}
\label{sec:model}


%The collected QA dataset enables a concrete benchmark for the QA problem in daily-life activities monitoring. 
%Analyzing the dataset reveals unique challenges, such as handling heterogeneous question types and variable time spans. 
%In this section, we provide the details of \Method. 
%\Method is the first end-to-end QA system that enables natural QA interactions with users. 
%We envision \Method as a next-generation mobile system that assists in person's everyday tasks. \Method uses the most intuitive communication of natural language to ``chat'' with users, thus seamlessly embedding into users' lives.
%

%\Method advances SOTA systems from the following perspectives. \textbf{First,} \Method handles a wide range of natural language questions from existence to time queries, and provides clear, understandable answers through LLMs. This addresses the challenges in prior systems~\cite{xing2021deepsqa,nie2022conversational,englhardt2024classification} which restricted users to predefined question or answer sets.
%\textbf{Second,}  \Method can answer questions based on long durations of fine-grained timeseries sensor data, such as counting the total exercising time in a week, by integrating a sensor query stage. This differs from existing systems that use low-dimensional sensor inputs (e.g., daily step counts) or fixed windows~\cite{englhardt2024classification,kim2024health,yang2024drhouse,moon2023anymal,han2024onellm,moon-etal-2023-imu2clip}. %To do so, \Method designs an intermediate stage for accurate and efficient encoding and query of sensor data.
%\textbf{Third}, \Method generates accurate answers to quantitative questions where SOTA systems struggle with~\cite{xing2021deepsqa,zhang2023llama,moon-etal-2023-imu2clip,moon2023anymal}. This is achieved via a carefully designed pipeline that integrates LLMs and the sensor data query stage.
%\textbf{Last but not least,} \Method is optimized for edge systems, delivering real-time responses on the edge desktop. 
%\Method is the first real system that enables natural and real-time interactions with multimodal sensors on edge devices.






%%%%% Overall steps, why do you design in this way
%\vspace{-2mm}
\subsection{Overview of \Method} 

%This is critical for \Method to handle quantitative questions.
%, which process natural language, with a sensor data query stage, which processes raw sensor signals. 
%The integration must leverage the strengths of each module while mitigating their weaknesses. Specifically, LLMs excel at understanding diverse questions but can generate hallucinations that distort quantitative results. On the other hand, the sensor query stage provides accurate quantitative data but may struggle with unexpected queries. Additionally, the complexity of the query search must be carefully managed to ensure real-time responses.
%The challenge of creating a pipeline where these modules operate accurately and robustly in real-world scenarios remains unresolved.
%
To address the challenge of accurate QA over long-duration sensor data, we design a novel three-stage pipeline in \Method, consisting of \textit{question decomposition, sensor data query}, and \textit{answer assembly}.
Specifically, we use LLMs in question decomposition and answer assembly, recognizing their essential roles in correctly interpreting user queries and generating natural language answers.
The intermediate sensor query stage is the \textit{key component} of \Method.
First, the query stage uses a pretrained sensor encoder to effectively encode high-dimensional and long-duration multimodal sensor timeseries into meaningful embeddings. Second, the query stage performs a similarity search in the embedding space to retrieve all sensor information relevant to the original question. This ensures precise extraction of sensor context, making a significant contribution to the accuracy of the final answer.
To the best of the authors' knowledge, \Method is the first system to incorporate an explicit sensor data query stage for accurately handling sensor-based language tasks in long-term monitoring.
%We discuss the potential to further improve \Method with vector databases~\cite{zhou2024llm} or retrival augmented generation~\cite{zhao2024retrieval} in Sec.~\ref{sec:future-work}.
%We recognize the potential to further improve \Method with vector databases~\cite{zhou2024llm} or retrival augmented generation~\cite{zhao2024retrieval}, which we leave for future investigation and discuss 

%Each stage and its interfaces are carefully designed to leverage the strengths of each module while mitigating their weaknesses. 
%The \textbf{key technical challenge} for \Method lies in creating a pipeline where these modules operate \textit{accurately, robustly and efficiently} in real-world scenarios, a problem that remains unsolved before \Method.
%, such as the hallucinations in LLMs.
%The key innovation is to use the LLM as both the front- and back-end for processing questions and answers, with the sensor data query stage in the middle to ensure accurate sensor information extraction. E
%The key technical challenge for \Method lies in \textit{effectively integrating LLM with accurate sensory queries}, so that the whole pipeline works accurately and robustly in real-world scenarios.


The complete pipeline of \Method is illustrated in Fig.~\ref{fig:overview}. For example, given a question like, “\textit{How long did I exercise last week in the morning?}” (\textcircled{1} in Fig.~\ref{fig:overview}), \Method first decomposes the question and generates specific sensor data queries using LLMs. These queries include details such as the context of interest, date and time ranges, and the summarization function (\textcircled{3}). In this case, the query might specify a context of “\textit{do exercise},” a time span of “\textit{last week},” a time of day of “\textit{morning},” and a summarization function of \texttt{CalculateDuration}. To enhance the precision of question decomposition, \Method uses solution templates (\textcircled{2}) with carefully designed prompts. Next, the sensor query stage encodes the activity text “\textit{exercise}” into the embedding space (\textcircled{4}) and retrieves sensor embeddings that are sufficiently similar to the text embedding (\textcircled{5}). These sensor embeddings are encoded offline from the full-history raw sensor data. Both the sensor and text encoders are pretrained offline to align their outputs in the same embedding space, ensuring accurate sensor data retrieval.
Therefore, only text encoding and similarity-based query searches are performed online.
Additional properties, such as the date (“\textit{last week}”) and time of day (“\textit{morning}”), are used to constrain the sensor query range. The final step of the sensor data query stage involves summarizing the relevant sensor embeddings into a textual context with a summarization function (\textcircled{6}). For instance, with the \texttt{CalculateDuration} function, the summarization step calculates the total duration of the retrieved sensor embeddings. This results in a sensor context such as: “\textit{Among all days last week, you exercised for 35 minutes on Monday morning and 55 minutes on Thursday morning.}” The summarization function is identified during question decomposition. Finally, the original question and the sensor context are fed into the answer assembly stage, where a fine-tuned LLM generates the final answer (\textcircled{7}). With the precise sensor context, the model produces an accurate response, such as: “\textit{You exercised a total of 1 hour and 30 minutes last week.}” (\textcircled{8}).

\begin{figure*}[tb]
  \centering
  \includegraphics[width=0.95\textwidth]{figs/overview3.png}
  \vspace{-3mm}
  \caption{The system diagram of \Method including three stages.}
  \vspace{-4mm}
  \label{fig:overview}
\end{figure*}
%\textbf{First,} with the goal of handling heterogeneous questions and answers in the real world, \Method integrates LLMs in the first stage of question decomposition and the third stage of answer assembly. This design leverages the extensive knowledge of pretrained/finetuned LLMs to ensure accurate and natural interactions with users.
%\textbf{Second,} to achieve an exhaustive search across the entire sensor lifespan, \Method develops a sensor data query mechanism to improve answer accuracy.
%The sensor information (in text) and natural language questions are fused in the third stage.
%Without introducing a new modality, \Method avoids the intensive training required by multimodal LLMs, making it a practical solution for real-world applications.

%Inspired by the limitations of prior designs as discussed in Sec.~\ref{sec:analysis}, \Method focuses on handling the diverse interactions in the real world, especially the diverse scenarios and long sensor time spans.
%In \Method, we design a novel framework that fuses LLMs with sensor database query to leverage the strengths of both.
%\Method addresses the limited question and answer types in prior QA systems~\cite{xing2021deepsqa,nie2022ai} as well as the fixed sensor window in existing multimodal LLM works~\cite{moon2023anymal,han2024onellm}.


%leverages the exceptional reasoning capability of pretrained LLMs to handle arbitrary questions, while integrating with a carefully designed a sensor query database to effectively extract relevant multimodal sensor data in the full history.

%In this section, we first give an overview of \Method and then provide more details about its key designs.


%In \Method, we propose a novel framework to address two key limitations in previous designs: (i) heterogeneous question and answer types and (ii) variable time scales of the queries.
%These capabilities are crucial for real-world sensor applications but have not been sufficiently addressed in existing studies~\cite{xing2021deepsqa,han2024onellm,moon2023anymal}.


%The framework also needs to integrate with accurate sensor data queries to ensure correct answer generation, rather than relying on the LLM's potentially inaccurate predictions.
%\Method is the first design to effectively fuse sensor data with natural language. Although recent works in multimodal LLM~\cite{han2024onellm,moon2023anymal} provide a path for integrating data from other modalities into LLMs, they require vast amounts of multimodal data (e.g., over 1M samples) for finetuning. 
%\Dataset is insufficient to obtain satisfactory finetuning results, as shown in the previous section.

 
%Unfortunately, \Dataset is the only dataset currently available for our natural QA task with sensors, making multimodal LLM infeasible in our scenario.


%To address the limitations of (1) heterogeneous question and answer types and (2) fixed sensor scales, our approach leverages the extensive knowledge of pretrained LLMs to handle diverse questions while designing a sensor data query mechanism to adaptively search the entire database. 



%We use \Dataset to finetune the LLM in the last stage thus the model adapts to our sensor-specific task and is able to generate answers in the desired format.

%In \Method, all three stages work together to produce accurate, high-quality answers. The sensor query stage in the middle stage effectively integrates sensor data with natural language: its query search is guided by the first stage's decomposition results, while its extracted information (in text) is fused with the original question in the last stage, both powered by pretrained LLMs. This approach avoids the data-intensive training required by multimodal LLMs, making \Method a practical solution for real-world applications. 


In the following lines, we describe the key design elements of \Method: the offline contrastive sensor-text encoder pretraining (Sec.~\ref{sec:pretraining}) and the online three-stage pipeline (Sec.~\ref{sec:three-stage}).

%that enable accurate answers from long-duration raw sensor timeseries.
%We first detail the offline contrastive sensor-text pretraining in \Method (Sec.~\ref{sec:pretraining}), which is critical for training well-aligned sensor and text encoders and ensuring precise sensor data queries. To achieve this, we introduce a novel contrastive pretraining loss tailored for partial contexts. 
%Next, we detail the online three-stage pipeline of \Method in Sec.~\ref{sec:three-stage}, including how LLMs are utilized for question decomposition and answer assembly, and how the sensor data query stage enables effective fusion of sensor and text information through similarity-based searches.




\subsection{Contrastive Sensor-Text Pretraining for Partial Contexts}
\label{sec:pretraining}

The sensor and text encoders are crucial for effective sensor-text fusion, serving as a "bridge" between high-dimensional sensor timeseries and semantic text. A high-quality sensor encoder ensures that \Method can accurately and comprehensively retrieve relevant sensor embeddings during the sensor data query stage. Similarly, a robust text encoder is essential for handling the arbitrary contextual text generated during the question decomposition stage.

However, achieving this level of alignment poses significant challenges due to the lack of suitable techniques. Existing pretraining methods, such as CLIP and its variants~\cite{radford2021learning,moon-etal-2023-imu2clip}, are effective for pretraining encoders with paired inputs, i.e., one piece of sensor data and one sentence. Unfortunately, encoders trained in this manner struggle to accurately identify similar sensor embeddings when provided with partial or arbitrary context instead of complete sentences.
For instance, using CLIP, a piece of sensor data might align well with the full sentence "The person is sitting and working on computers at school." However, when a user is specifically interested in a partial context like "working on computers," the encoded text embedding may not closely match the original sensor embeddings, leading to reduced accuracy in sensor data query as we show in Sec.~\ref{sec:ablation}. To address this challenge, we introduce a novel contrastive sensor-text pretraining loss for partial contexts.



%tailored for multi-label contexts. 
We pretrain our model on a large-scale multimodal sensor dataset with annotations, where each sample $\{\mathbf{x}_t, w_t\}$ consists of raw time series sensor data $\mathbf{x}_t$ collected at time $t$ and an associated set of partial context labels $w_t$. $w_t$ can be extracted from label annotations. For instance, in a single-label human activity classification dataset,  $w_t$ may contain a single phrase (e.g., $\{\textrm{"standing"}\}$), whereas in a multi-label dataset, it may include multiple phrases (e.g., $\{ \textrm{"at school", "working on computers"}\}$).
We employ separate encoders for sensor and text inputs. 
Formally, let $\theta$ denote the complete sensor encoder model. The encoded sensor embedding is given by $\mathbf{z}^s_t = \theta(\mathbf{x}_t)$.
The text encoder, denoted by $\phi$, maps arbitrary phrases to a text embedding $\mathbf{z}^w_t = \phi(w_t)$.
Key notations used throughout our work are summarized in Table~\ref{tbl:notation}.
%
The details of the sensor encoder are shown in Fig.~\ref{fig:sensor_encoder}. We use distinct sensor encoder for each sensor modality (e.g., IMUs, audio and phone status) to accommodate the varying complexity of different sensor data types. For instance, a Transformer-based encoder is used for high-dimensional time series data while a simple linear layer is designed for encoding phone status. The outputs from these modality-specific encoders are concatenated and passed through a fusion layer to generate the final sensor embedding. Our framework allows for missing modalities by padding with mean values and is flexible for future expansion to additional modalities.

\begin{figure}[t]
\begin{center}
\includegraphics[width=0.75\textwidth]{figs/sensor_encoder.png} 
\vspace{-4mm}
\caption{Visualization of the contrastive sensor-text pretraining in \Method.}
\label{fig:sensor_encoder}
\end{center}
\vspace{-6mm}
\end{figure}


%which effectively aligns sensor embeddings with the semantic information in labels~\cite{zhang2023navigating}. 
%Important notations are listed in Table~\ref{tbl:notation}.
%We integrate the timeseries collected from $d$ multimodal sensors into a single sample $\mathbf{x}_t \in \mathbb{R}^{d \times \tau}$, where $t$ denotes the timestamp and $\tau$ denotes the time window length. 
%In the preprocessing stage, we normalize data based on complete readings from the same sensor.
%Missing modalities or readings are padded with zeros.
%The sensor encoder $\theta$ encodes $\mathbf{x}_t$, while the label encoder $\phi$ encodes text labels $w_c$, such as "at school". We then compute the logits by taking the dot product between the sensor and label embeddings:
%$g(\mathbf{x}_t; \theta, \phi) = \sigma \left ( \left [ f(\mathbf{x}_t;\theta) \circ f(w_c;\phi) \right ]_{w_c \in \mathcal{W}} \right )$,
%where $\sigma$ is a sigmoid function to obtain the predicted probabilities.
%
We introduce a new pretraining loss to align sensor embeddings $\mathbf{z}^s_t$ and partial text embeddings $\mathbf{z}^w_t$. Different from CLIP~\cite{radford2021learning}, our loss function enables alignment between sensor data and all partial phrases, defined as follows:
\begin{equation}
 \mathcal{L} = \sum_t {\frac{-1}{|w_t|} \sum_{w \in w_t} \log \frac{\exp(\mathbf{z}^s_t \cdot \mathbf{z}^w_t / \tau)}{\sum_{a \in A(w)} \exp (\mathbf{z}^s_t \cdot \mathbf{z}^a / \tau)}}, \label{eq:loss} 
\end{equation}
$|w_t|$ is the cardinality of set $w_t$, $\tau$ is a scalar temperature parameter. The term $A(w) \equiv A \setminus \{w\}$ represents the collection of negative contexts, defined as all possible phrases except the positive phrase $w$.
The intuition behind our loss function is illustrated in Fig.~\ref{fig:sensor_encoder}. Consider a sensor embedding $\mathbf{x}_t$ and a set of partial text labels $w_t = \{ \textrm{sitting, working on computers}\}$.
Our loss function encourages high similarity between the sensor embedding $\mathbf{z}^s_t$ and all positive text embeddings corresponding to $w_t$, namely "sitting" and "working on computers", while distinguishing them from other negative contexts such as "walking" or "at school". We draw inspiration from the supervised contrast learning loss~\cite{khosla2020supervised}. However, our loss function differs in that it explicitly models the similarity between sensor embeddings and multiple text phrases rather than between samples of the same modality.
%During encoder training, we minimize binary cross-entropy loss using logits $g(\mathbf{x}_t; \theta, \phi)$ and ground truth multi-class labels $y_t$. 
%In the final deployment of \Method, the pretrained encoders are fixed, and new sensor data are encoded into logits.
%The logits is a vector with the same size as $y_t$, indicating the probability of each activity.
After pretraining, we store all sensor embeddings in a database for online queries.
%We further optimize the efficiency in Sec.~\ref{sec:edge}.

\iffalse
\begin{wraptable}{r}{0.44\textwidth}
\footnotesize
\vspace{-5mm}
\caption{List of important notations.}
\label{tbl:notation}
\vspace{-5mm}
\begin{center}
\hspace{-2mm}\begin{tabular}{p{3em} p{22em}} 
 \toprule
 \hspace{-2mm}Symbol & Meaning \\
 \midrule
 %$(q^i, a^i)$ & The $i$th question-answer pair \\
 %$\mathbf{x}^i$ & The full-history sensor time series data w.r.t. $i$th QA pair \\
 $d$ & Number of different sensors \\
 %$T^i$ & Total duration of sensor data $\mathbf{x}^i$ \\
 $\tau$ & Window length of each sensor sample \\
 $\{\mathbf{x}_t, y_t\}$ & The sensor time series data and context labels at time $t$ \\
 %$c^i$ & The extracted sensor context w.r.t. $i$th QA pair \\
 %$T$ & Time window length of the raw timeseries sensor data \\
 $\theta, \phi$ & Pretrained sensor and label encoders \\
 $h$ & Threshold in query search \\
 %$\mathbf{t}$ & Global timestamps of sensor data \\
 %$N$ & Total number of QA pairs in our \Method dataset \\
 %$l, m$ & Maximum length of tokenized questions and answers \\
 %$\theta$ & Autoregressive Large Language Model to finetuning \\
 %$q$ & Number of provided templates during question decomposition \\
 %$k$ & Number of generated variants during dataset augmentation \\
 %$lr$ & Learning rate of during finetuning \\
 %$\tau$ & Temperature for LLM generation \\
 %$\phi$ & Pretrained transformer encoder for encoding raw sensor data \\
 %$b$ & The dimension of sensor embeddings \\
 \bottomrule
\end{tabular}
\end{center}
\vspace{-4mm}
\end{wraptable}
\fi

\begin{table}[t]
\small
\caption{List of important notations.}
\label{tbl:notation}
\vspace{-4mm}
\begin{center}
\begin{tabular}{p{3em} p{21em} | p{3em} p{21em} } 
\toprule
Symbol & Meaning & Symbol & Meaning \\
\midrule
 $\mathbf{x}_t$ & Multimodal sensor data sampled at $t$ &
 $w_t$ & Partial context labels at $t$ \\
 $d$ & Number of different sensors &
 $T$ & Total duration of sensor data \\
 $\theta$ & Sensor encoder &
 $\phi$ & Text encoder \\
 $\mathbf{z}^s_t$ & Sensor embedding of $\mathbf{x}_t$ &
 $\mathbf{z}^w_t$ & Text embedding of $w_t$ \\
 $A(w)$ & Set of negative contexts of $w$ &
 $\tau$ & Temperature scalar \\
 $f$ & Trained similarity function between embeddings & 
 $h$ & Threshold in query search \\
\bottomrule
\end{tabular}
\end{center}
\vspace{-4mm}
\end{table}


\subsection{Three Stages in \Method}
\label{sec:three-stage}

In this section, we explain the detailed designs in each stage of \Method, including question decomposition, sensor data query and answer assembly.
All stages need to work accurately, robustly and collaboratively to achieve natural and precise answer generation in the end.

%\Method uses LLMs in the question decomposition and answer assembly stages to interact freely with users using natural language.
%Our major goal is to utilize the extensive knowledge in LLMs for comprehension and summarizing with minimal hallucinations.
%The first and third stages of \Method use LLMs to to interact freely with users using natural language.
%correctly understand diverse types of questions and construct accurate answers. These stages allow \Method 
%More specifically, as shown in Fig.~\ref{fig:overview}, the question decomposition stage uses GPT~\cite{gpt-4} or LLaMA~\cite{touvron2023llama}, treating the decomposition as a zero-shot reasoning task due to the lack of finetuning data.
%The answer assembly stage uses a custom LLaMA~\cite{touvron2023llama} from fine-tuning on \Dataset. The finetuning process helps \Method produce answers in a desired format. 
%We introduce more details in the following.

%Due to the lack of data to this question decomposition, we formulate the task as zero-shot generation, enhanced by in-context learning and chain-of-thought reasoning to improve solution quality.
%As detailed in the left portion of Fig.~\ref{fig:overview}, we utilize both closed-source LLMs, such as GPT~\cite{gpt-4}, and open-source LLMs, like Llama~\cite{touvron2023llama}. 
\iffalse
\begin{figure}[t]
\begin{center}
\begin{tabular}{c}
    \includegraphics[width=0.95\textwidth]{figs/solution_template.png} \\
    {\small (a) Example solution template for ICL. \vspace{1mm}} \\
    
    \includegraphics[width=0.95\textwidth]{figs/question_decompose.png} \\
    {\small (b) The prompt design for the question decomposition stage includes ICL solution templates (in blue) and bolded text for CoT. \vspace{1mm}}  \\

    \includegraphics[width=0.5\textwidth]{figs/finetuning.png} \\
    {\small (c) Prompt design in the answer assembly stage for LLM finetuning.} \\
\end{tabular}
\vspace{-4mm}
\caption{Prompts design in \Method to better leverage the power of LLMs.}
\label{fig:prompts}
\end{center}
\vspace{-6mm}
\end{figure}
\fi


\subsubsection{Question Decomposition}
\label{sec:decomposition}

%Explicit question decomposition has not been utilized in existing sensor-based QA systems, which either used a limited question set~\cite{xing2021deepsqa,nie2022conversational,englhardt2024classification} or relied on a single LLM for end-to-end problem solving~\cite{englhardt2024classification,kim2024health,yang2024drhouse,zhang2023llama,moon-etal-2023-imu2clip,moon2023anymal}.
%In contrast, \Method integrates explicit question decomposition to guide accurate sensor data queries, thus empowering \Method to handle quantitative questions.

%Question decomposition stage is crucial for accurately interpreting various question types and determining the corresponding sensor query requests. 
The question decomposition stage processes the user question and identifies specific query triggers for the sensor data query stage, such as the context of interest, date and time ranges, and the summarization function, as illustrated in the leftmost box in Fig.~\ref{fig:overview}.
The primary challenge in designing this stage lies in generating \textit{accurate} decompositions for \textit{arbitrary} user inputs, including various questions types as discussed in Sec.~\ref{sec:motivation}.
We choose to rely on LLMs for this stage due to their outstanding capability in handling natural language inputs.
However, LLMs are not without limitations as they can produce erroneous outputs or hallucinations~\cite{huang2023survey}, especially in our case where a dedicated dataset for similar decomposition tasks is unavailable.


%%%Difference from SOTA, what makes this challenging


%However, a significant challenge is the lack of a dedicated dataset for this decomposition. Without such a dataset and finetuning, it is difficult to ensure that LLMs generate desired decompositions with minimal errors due to hallucinations.
%
%
%The \textit{primary challenge} in question decomposition is handling arbitrary user questions without a dedicated dataset. Existing datasets like DeepSQA~\cite{xing2021deepsqa} and SensorQA~\cite{} focus on end-to-end answers, not decomposition. While GPT models excel in general reasoning~\cite{gpt-4}, guiding them to generate precise query arguments in our specific format without pre-existing data is difficult. 
To address this challenge, we propose a \textit{few-shot learning} approach to prompt pre-trained LLMs, enhanced with in-context learning~\cite{alayrac2022flamingo,shao2023prompting} and chain of thought techniques~\cite{chuCoTReasoningSurvey2024,lu2022learn}.
An example prompt is illustrated in Fig.~\ref{fig:question-decompose}, where LLM is instructed to mark different arguments with distinct symbols, such as "<<" and ">>" for function names. This helps in accurately extracting various arguments from the LLM output. 
In \Method, we accommodate a variety of real-life scenarios by extracting contexts (such as activities), dates, times of day, and summarization functions during the query process, as shown in Fig.~\ref{fig:question-decompose}.
\Method supports multiple extracted phrases to handle complex questions such as "How long did I work at school on Monday and Tuesday?"
\Method is designed to be flexible, allowing for future expansion with additional decomposition terms.

We next explain more details on the in-context learning and chain-of-thought techniques used in \Method, both for ensuring high-quality question decompositions.



%We explain more details about each technique next.
%In in-context learning, we adaptively select solution templates and feed them to GPTs along with the question to decompose.
%The CoT design requires LLMs to perform detailed reasoning 
%We also introduce in-context learning and chain-of-thought prompting to improve zero-shot reasoning and avoid too ``creative'' generations.
%In this stage, we introduce in-context learning and chain-of-thought prompting to enhance the effectiveness of zero-shot reasoning, as detailed below.
%Due to the lack of data for this specific task, we opt for prompting GPTs in a zero-shot manner. 




\textbf{In-Context Learning (ICL)} integrates a few examples directly into the prompt during inference, enabling LLMs to adapt effectively to specific tasks without requiring fine-tuning~\cite{alayrac2022flamingo,shao2023prompting}. 
Building on this insight, we design a library of general solution templates covering diverse QA scenarios and incorporate them as in-context examples to improve decomposition accuracy (Fig.~\ref{fig:question-decompose}).
We observe that decomposition solutions for the same question type often share similarities. For instance, "how long" questions typically map to the \texttt{CalculateDuration} function. To utilize this question-specific property, in \Method, we first classify the question type using a BERT model~\cite{devlin2018bert}. Depending on the question type, we dynamically select templates that best match the question category, a strategy shown to enhance ICL performance~\cite{fu2023gpt4aigchip}.
Fig.~\ref{fig:solution-template} illustrates an example of a solution template for a time query. Including an explanation in the solution template is particularly helpful for complex reasoning tasks. %, as detailed with chain of thought.
%For example, a solution to ``\textit{How long did I work yesterday?}'' could be ``\textit{Query the database with the function <<CalculateDuration>> using the activity ((in the main workplace)), and the date [[yesterday]]}''.
%During decomposition, we adaptively select the two templates that solve the most similar questions, which has been shown to improve ICL performance~\cite{fu2023gpt4aigchip}. Using just two templates helps balance prompt length and performance.
%To ensure accurate decomposition across various question types, we create distinct templates tailored to each category.
%For example, the template in Fig.\ref{fig} (b) is used for time queries, while another template addresses action queries like ``What did I do on Tuesday?''
%This adaptive template selection offers specialized guidance based on the question category, thereby enhancing decomposition accuracy. 
%In \Method, we provide 2-3 templates for each question category listed in Table~\ref{tab:question_profile}.



%\textbf{In-context learning}
%includes task examples directly into the prompt during inference, which has shown superb performances in adapting LLM outputs to specific tasks without finetuning~\cite{alayrac2022flamingo,shao2023prompting}. Based on this insight, we design solution templates as in-context examples, which are embedded into prompts as illustrated in Fig.~\ref{fig:prompts} (a).
%An example of such a solution template is shown in Fig.\ref{fig:prompts} (b).
%For each question, we provide a corresponding solution and a detailed explanation. To ensure accurate decomposition across various question types, we create distinct templates tailored to different question categories.
%For instance, the template example in Fig.~\ref{fig:prompts} (b) is used for time queries, while a separate template with ``What did I do on Tuesday?'' is used for action queries. 
%This adaptive template selection offers specialized guidance based on the question category, thereby enhancing decomposition accuracy. In \Method, we provide 2-3 templates for each question category outlined in Table~\ref{tab:question_profile}.

%These templates serve two main purposes: first, they guide LLMs to highlight different properties with various symbols, enabling accurate extraction of query arguments from the generated text. Secondly, 
%We create multiple templates for each question category and use different templates based on the question to be decomposed.
%A concrete template example for the category of time query questions is shown in Fig.~\ref{fig:prompts} (b). We select a typical question within this category and provide a solution as well as an explanation. 

\textbf{Chain-of-Thought (CoT) Prompting} explicitly requests LLMs to generate their reasoning process, which improves accuracy in zero- or few-shot logical reasoning~\cite{chuCoTReasoningSurvey2024,lu2022learn}. In \Method, we include ``please generate step-by-step explanations'' in prompts to encourage CoT, as shown in Fig.~\ref{fig:question-decompose}.
CoT and detailed explanations in solution templates improve the quality of logical reasoning, especially for questions requiring multi-step thoughts.
For example, applying CoT to the question ``What did I do right after waking up on Wednesday?'' helps produce a step-by-step thinking process, such as: ``To find the activity right after waking up on Wednesday, we can query the start and end times of all activities on Wednesday, therefore we can use the function <<DetectFirstTime>> and <<DetectLastTime>> with ((all activities)) on [[Wednesday]]''.

\begin{figure}[t]
    \centering    
    \begin{subfigure}[b]{0.95\textwidth}
        \centering
        \includegraphics[width=\textwidth]{figs/question_decompose.png}
        \vspace{-6mm}
        \caption{The prompt design for the question decomposition stage includes ICL and CoT.}
        \label{fig:question-decompose}
    \end{subfigure}

    \begin{subfigure}[b]{0.95\textwidth}
        \centering
        \includegraphics[width=\textwidth]{figs/solution_template.png}
        \vspace{-6mm}
        \caption{Example solution template for the time query question category.}
        \label{fig:solution-template}
    \end{subfigure} %\hspace{0.05\textwidth} % Add horizontal space between subfigures if needed

    \begin{subfigure}[b]{0.6\textwidth}
        \centering
        \includegraphics[width=0.8\textwidth]{figs/finetuning.png}
        \vspace{-2mm}
        \caption{Prompt design in the answer assembly stage for LLM finetuning.}
        \label{fig:finetuning}
    \end{subfigure}
    \vspace{-4mm}
    \caption{Prompts design in \Method to better leverage the power of LLMs.}
    \label{fig:prompts}
    \vspace{-4mm}
\end{figure}


\subsubsection{Sensor Data Query}
\label{sec:sensor-data-query}

The sensor query stage is the core module of \Method, responsible for searching the sensor database and extracting relevant information, which directly impacts the accuracy of \Method's answers.
Existing sensor-based QA systems did not employ an explicit search module, restricting to low-dimensional sensor features~\cite{englhardt2024classification,kim2024health,yang2024drhouse} or fixed windows of sensor signals~\cite{xing2021deepsqa,moon-etal-2023-imu2clip,moon2023anymal,chen2024sensor2text,arakawa2024prism}.
In contrast, \Method introduces this explicit query stage to handle long-duration, fine-grained timeseries sensor signals and provide accurate information to queries, which is not possible with prior QA designs.

%In this stage, \Method invokes functions with arguments provided by the question decomposition stage and returns the extracted information as text
%For example, a query calling \texttt{CountingDays} using activity ``\textit{at home}'' and dates ``\textit{last week}'' may return one line of text ``\textit{You spent 4 days at home last week}'', which is then passed to the answer assembly stage.


%%%Difference from SOTA - what is missing from SOTA? what makes this challenging

%Sensor data query is the core component of \Method, which extracts relevant sensor data from raw timeseries in a long time span and composes the factual sensor information in text, , as shown in the middle part of Fig.~\ref{fig:overview}.

The sensor and label encoders are pretrained offline as explained in Sec.~\ref{sec:pretraining}. 
With the pretrained sensor encoder, the full-history raw sensor signals are converted into an embedding database beforehand.
During online user interactions, as shown in the middle part of Fig.~\ref{fig:overview}, the sensor query stage only needs to encode the context (e.g., activity) generated in the question decomposition stage using the text encoder (\textcircled{4} in Fig.~\ref{fig:overview}) , performs a similarity search among the sensor database given the text (\textcircled{5}), and summarizes the retrieved sensor embeddings into textual information for the final answer assembly (\textcircled{6}).
The design of the sensor query stage faces two primary challenges: (1) ensuring accurate and efficient searches within the sensor embedding database, and (2) constructing adequate output context based on users' queries. 
\Method addresses these challenges through an efficient similarity-based embedding search mechanism and a set of carefully designed summarization functions, as detailed below.
%(1) handling diverse queries from practical user questions, and (2) performing accurate and efficient searches on timeseries data.
%The sensor data encoder converts raw timeseries data to lightweight embeddings, enabling accurate and efficient searches within the embedding database.
%We next explain the detailed designs.



%We include the list of important notations in Table~\ref{tbl:notation}.



%While designing many functions can enable finer-grained sensor queries, it may also confuse the LLM during the decomposition stage. To balance these factors, we designed six functions in \Method, covering common question and answer types listed in Table~\ref{tab:sensorqa_profile}.
%\textcolor{red}{Note: we don't need to cover exhaustive functions, as stage 1 and stage 3 will project the scenarios to these questions.}
%%% What are the main challenge of this stage?
%The \textit{key challenge} lies in accurately extracting sensor data from an extensive duration. 
%This involves two major components: first, training a sensor encoder that accurately encodes raw data into embeddings for database storage, ensuring query \textit{precision}. Second, developing a simple yet comprehensive set of query functions to extract the most \textit{relevant} information. These components jointly determine \Method's ability to provide correct answers in the final stage.
%first, training a sensor encoder that accurately encodes raw sensor data into embeddings for database storage. A well-designed encoder ensures the \textit{precision} of sensor queries. Secondly, it is crucial to develop a simple yet comprehensive set of query functions to effectively extract the most \textit{relevant} sensor information from the long duration. Together, these two components determine whether \Method can provide correct answers in the final stage.

%Both sensor data encoder and query functions are crucial in ensuring the relevance and precision of the extracted sensor information, which 



\textbf{Similarity-based Embedding Search} 
%We design efficient query search in \Method based on the lightweight logits.
Our goal is to accurately identify all sensor samples relevant to the query context.
For example, if the query context is ``exercise'', \Method aims to retrieve all sensor samples associated with activities similar to exercise.
If multiple text are generated during question decomposition, \Method performs an embedding search for each query text individually.
The pretraining in Sec.~\ref{sec:pretraining} ensures that the sensor embedding space is well aligned with the text embedding space even using partial context. 
Therefore, the similarity comparison given any pair of sensor and text embeddings can be achieved by training a similarity function $f$ in the embedding space: 
\begin{equation}
f(\mathbf{z}^s_t, \mathbf{z}^w) = f \big(\theta(\mathbf{x}_t), \phi(w) \big) \in [0, 1]
\end{equation}
The output of $f$ is a scalar value between 0 and 1, representing the similarity between the sensor and text samples.
With $f$, the similarity search in the embedding space is significantly more efficient than searching directly through raw, high-dimensional time-series data. The efficiency can be further improved by narrowing the search scope based on the date and time-of-day arguments identified during question decomposition.

\textbf{Summarization Function Design}
We design a set of summarization functions to "summarize" the queried sensor samples and generate contextual information for the answer assembly stage. The specific summarization function to be used is determined by \Method during question decomposition.
For example, a question of "\textit{how long}" should be directed to the \texttt{CalculateDuration} function, while a question of "\textit{what did I do}" should be handled by the \texttt{DetectActivity} function.
Each summarization function uses a unique template to return text information.
The numerical value in the returned text is determined by the embedding search results.
For instance, when querying \texttt{CalculateDuration} with the activity ``\textit{cooking}'', date ``\textit{Sunday}'', and time ``\textit{morning}'', the text output can be: "\textit{You spent {$\gamma$} minutes cooking on Sunday morning}", where $\gamma$ is calculated as follows:
\begin{equation}
\gamma = \sum_{t \in T_{SundayMorning}} \Bigg[ f \big( \theta(\mathbf{x}_t), \phi( \textrm{"cooking"}) \big) > h \Bigg].
\end{equation}
%\begin{equation}
%    \gamma = \sum_{t \in T_{SundayMorning}} [ g(\mathbf{x}_t;\theta, \phi) > h ].
%\end{equation}
Here $T_{SundayMorning}$ represents all timestamps within Sunday morning. 
The notation $[Cond]$ gives $1$, when the inner condition {\it Cond} is met; otherwise $0$.
$h$ is a predetermined threshold.

In \Method, we carefully design a set of summarization functions to account for diverse scenarios in real life, including time quries, activity quries, counting, etc. The details of those functions are explained in Appendix~\ref{sec:query-function}.

\subsubsection{Answer Assembly}
\label{sec:answer-assembly}
As shown in the right box of Fig.~\ref{fig:overview}, the final stage of answer assembly integrates question and sensor information to generate the final answer.
State-of-the-art methods~\cite{xing2021deepsqa,zhang2023llama,moon-etal-2023-imu2clip} rely on ``black-box'' fusion of natural language and sensory data, often leading to ineffective fusion and inaccurate answers (see Sec.~\ref{sec:motivation}).
In contrast, \Method summarizes query results to text and directly fuses them with the question in the prompt, as illustrated in Fig.~\ref{fig:finetuning}.
Our intuition is that, in contrast to processing and fusing with other modalities, LLMs are the most professional in dealing with text.
\Method is capable of answering both qualitative and quantitative questions by combining the original question and the extracted fine-grained activity information from long-duration, high-dimensional sensors.
At this answer assembly stage, we finetune a LLM such as LLaMA~\cite{zhang2023llama} to adapt the model to the desired answer style. Fine-tuning is chosen over few-shot learning as it delivers better performance with the presence of high-quality datasets like \Dataset~\citesensorqa (see Sec.\ref{sec:ablation}). We use Low Rank Adaptation (LoRA)~\cite{hu2021lora} due to its parameter efficiency and comparable performance to full fine-tuning.




%the \texttt{CalculateDuration} function is widely used in time compare, time query and existence questions.
%\texttt{CalculateFrequency} and \texttt{ActivityDetection} are designed for the types of counting and activity query questions, respectively. The \texttt{CalculateDays} function is added to address day counting as a subset of the counting category.
%The \texttt{DetectFirstTime} and \texttt{DetectLastTime} are for counter the needs of concrete timestamp queries.
%Finally, the arguments \texttt{Activity, Date} and \texttt{Time\_of\_Day} are given as additional constraints to limit the search range within the sensor's duration.
%The above four functions are able to cover the majority of question and answer types.
%The sensor database comprises a full history of sensor data with timestamps and a pretrained label classifier, which we explain further in Sec.~\ref{sec:reality}.

%, since days counting differs from regular activity frequency counting.
%is the most commonly used, calculating the duration of a specified activity on a specified date for a certain period. \texttt{CalculateDuration} is widely used in handling time compare, time query and existence questions. Functions 

%We recognize a few potential directions to further improve the sensor query such as using vector databases~\cite{zhou2024llm} or retrival augmented generation~\cite{zhao2024retrieval}, which we leave for future investigation.
%We recognize the potential of enhancing \Method with vector database~\cite{zhou2024llm} or retrieval augmented generation techniques~\cite{zhao2024retrieval}. However, we emphasize that   We discuss the possible extensions to \Method in Sec.~\ref{sec:discussion}.
% 

%\subsection{Optimizing \Method for the Edge}
%\label{sec:edge}
%Deploying \Method on edge devices is critical in preserving user's privacy.
%As shown from Fig.~\ref{fig:overview}, the computational complexity of \Method mainly comes from two parts: (1) the LLMs in question decomposition and answer assembly, and (2) the sensor data encoder and and logits search.
%We use two strategies to optimize them.

%\textbf{Optimizing LLM deployments}
%We optimize the LLMs in \Method using quantization, which has achieved real-time token generation on desktop-level systems in recent works~\cite{lin2023awq,kim2023squeezellm}.
%Thanks to \Method's design of only using text inputs for both question decomposition and answer assembly, the LLMs in \Method can be integrated with any state-of-the-art quantization techniques, such as AWQ~\cite{lin2023awq}.
%In contrast, multimodal LLMs cannot be fully quantized by these techniques due to the additional adapter modules.
%The only thing needed is a calibration dataset, which can be created using our solution templates and finetuning data.
%We construct a calibration dataset from our data 

%\textbf{Optimizing the Encoding of Sensor Data}
%Although inferences on the sensor and label encoders can be inefficient on edge devices in real time, these encoding operations can be performed offline when the QA part is idle.
%We further optimize the encoding stage by performing the encoding inference offline, when the QA part is idle.
%During user interactions, \Method only needs to search the precomputed logits and ensure real-time responses without delay.
%Moreover, the sensor encoder ``compresses'' raw time series data into logits, thus significantly reducing storage requirements as shown in Sec.~\ref{sec:memory}.

\section{Experiments}
\label{sec:exp}
We conduct extensive experiments to evaluate the efficiency of the DPTS framework. We benchmark its performance against various search algorithms across multiple models and datasets to ensure a comprehensive analysis.


\subsection{Settings}
\label{sec:exp_setting}

\paragraph{Models.} We include Qwen-2.5-1.5B-Instruct, Qwen-2.5-7B-Instruct~\cite{yang2024qwen2}, LLaMA-3.1-8B-Instruct, and LLaMA-3.2-3B-Instruct~\cite{touvron2023llama} to cover various model sizes.  

\paragraph{Datasets.} The evaluation datasets include Math500~\cite{hendrycks2021measuring} and GSM8K~\cite{cobbe2021training}, both are widely used for reasoning and mathematical problem-solving tasks. We implement the evaluation by referring the approaches used in Qwen-2.5-Math~\cite{yang2024qwen2}, supplemented in our submission. 

\paragraph{Comparison Methods.} We compare DPTS against three widely used search algorithms: Monte Carlo Tree Search~(MCTS)~\cite{sprueill2023monte}
% balances exploration and exploitation when sampling, while the selected paths rollout till termination
, Best-of-N~\cite{cobbe2021training}
% performs multiple independent rollouts and selects the highest-scoring output
, Beam Search~\cite{Yao_2023_Tree}.
% expands multiple hypotheses in parallel, pruning low-scoring candidates at each step to maintain a fixed-width search~\cite{snell2024scaling}. 
Since efficient tree search algorithms have recently regained attention after the emergence of LLM reasoning, the strong baselines are limited. As a result, we primarily compare DPTS against these typical and well-established search algorithms to demonstrate the effectiveness of our method. An introduction of the comparison methods and other details about experimental settings are provided in Appendix~\ref{app:sec:exp}.

% 实验参数: tree_width:4 tree_depth:16 mini_step:128 % 其实 103 确实实验结果更好不知道为什么 otz
% max token:2048 mcts_max_time:120 

\subsection{Comparisons on Search Algorithms}


% \begin{table}
%     \centering
%     \caption{Comparisons across existing search algorithms on LLM reasoning tasks. }
%     \label{tab:comparisons}
%     \resizebox{\linewidth}{!}{
%     \begin{tabular}{cllrlr}
%     \toprule
%     \multirow{2}{*}{\textbf{Model}}  & \multirow{2}{*}{\textbf{Algo.}}  & \multicolumn{2}{c}{\textbf{Math500}} & \multicolumn{2}{c}{\textbf{GSM8K}}  \\ 
%     & &  \textbf{Acc.} & \textbf{Time (s)} & \textbf{Acc.} & \textbf{Time (s)}  \\ \midrule
%     \multirow{4}{*}{\begin{tabular}{c} Qwen2.5\\1.5B\end{tabular}} 
%     & MCTS & 56.6 & 117.37 & 75.1 & 73.28 \\ 
%     ~ & Best-of-N & 52.6 & 89.87 & 70.1 & 33.37  \\ 
%     ~ & Beam & 52.4 & 104.58 & 71.5 & 41.27 \\ 
%     ~ & \cellcolor[gray]{0.9}{DPTS} & \cellcolor[gray]{0.9}{59.2} & \cellcolor[gray]{0.9}{\textbf{45.10}} & \cellcolor[gray]{0.9}{75.2} & \cellcolor[gray]{0.9}{\textbf{18.32}}  \\ \midrule
    
%     \multirow{4}{*}{\begin{tabular}{c} Qwen2.5\\7B\end{tabular}}  
%     & MCTS & 75.2 & 121.46 & 89.6 & 79.68 \\ 
%     ~ & Best-of-N & 71.6 & 91.29 & 88.2 & 34.89 \\ 
%     ~ & Beam & 72.4 & 106.89 & 86.7 & 36.49 \\ 
%     ~ & \cellcolor[gray]{0.9}{DPTS} & \cellcolor[gray]{0.9}{76.2} & \cellcolor[gray]{0.9}{\textbf{53.50}} & \cellcolor[gray]{0.9}{89.4} & \cellcolor[gray]{0.9}{\textbf{19.95}} \\ \midrule
    
%     \multirow{4}{*}{\begin{tabular}{c} Llama3\\3B\end{tabular}}  
%     & MCTS & 48.6 & 111.80 & 64.0 & 57.19  \\ 
%     ~ & Best-of-N & 46.4 & 91.34 & 57.1 & 27.27 \\ 
%     ~ & Beam & 45.2 & 104.36 & 58.4 & 28.27 \\ 
%     ~ & \cellcolor[gray]{0.9}{DPTS} & \cellcolor[gray]{0.9}{50.8} & \cellcolor[gray]{0.9}{\textbf{47.75}} & \cellcolor[gray]{0.9}{67.8} & \cellcolor[gray]{0.9}{\textbf{27.74}} \\ \midrule
    
%     \multirow{4}{*}{\begin{tabular}{c} Llama3\\8B\end{tabular}}  
%     & MCTS & 54.2 & 143.36 & 69.5 & 69.74 \\ 
%     ~ & Best-of-N & 49.8 & 122.63 & 67.6 & 33.48  \\ 
%     ~ & Beam & 49.6 & 142.21 & 68.3 & 34.51 \\ 
%     ~ & \cellcolor[gray]{0.9}{DPTS} & \cellcolor[gray]{0.9}{55.4} & \cellcolor[gray]{0.9}{\textbf{37.98}} & \cellcolor[gray]{0.9}{68.2} & \cellcolor[gray]{0.9}{\textbf{17.82}} \\ \bottomrule
%     \end{tabular}
% }
% \end{table}

\begin{table}[ht]
    \centering
    \footnotesize
    \caption{Comparisons across existing search algorithms on LLM reasoning tasks.}
    \vspace{-0.1in}
    \label{tab:comparisons}
    \begin{tabular}{@{}c@{\hskip 4pt} l@{\hskip 6pt} c@{\hskip 6pt} r c@{\hskip 6pt} r}
    % \begin{tabular}{@{}c@{\hskip 4pt} l@{\hskip 6pt} c@{\hskip 6pt} r r@{\hskip 6pt} r@{}}
        \toprule
        \multirow{2}{*}{\textbf{Model}} & \multirow{2}{*}{\textbf{Algo.}} & \multicolumn{2}{c}{\textbf{Math500}} & \multicolumn{2}{c}{\textbf{GSM8K}} \\
        & & \textbf{Acc.} & \textbf{Time (s)} & \textbf{Acc.} & \textbf{Time (s)} \\
        \midrule
        \multirow{4}{*}{\begin{tabular}{c} Qwen-2.5 \\ 1.5B \end{tabular}} & MCTS & 56.6 & 117.37 & 75.1 & 73.28 \\
        & Best-of-N & 52.6 & 89.87 & 70.1 & 33.37 \\
        & Beam & 52.4 & 104.58 & 71.5 & 41.27 \\
        & \cellcolor[HTML]{D3D3D3} \textbf{DPTS} & \cellcolor[HTML]{D3D3D3} 59.2 & \cellcolor[HTML]{D3D3D3} \textbf{45.10} & \cellcolor[HTML]{D3D3D3} 75.2 & \cellcolor[HTML]{D3D3D3} \textbf{18.32} \\
        \midrule
        \multirow{4}{*}{\begin{tabular}{c} Qwen-2.5 \\ 7B \end{tabular}} & MCTS & 75.2 & 121.46 & 89.6 & 79.68 \\
        & Best-of-N & 71.6 & 91.29 & 88.2 & 34.89 \\
        & Beam & 72.4 & 106.89 & 86.7 & 36.49 \\
        & \cellcolor[HTML]{D3D3D3} \textbf{DPTS} & \cellcolor[HTML]{D3D3D3} 76.2 & \cellcolor[HTML]{D3D3D3} \textbf{53.50} & \cellcolor[HTML]{D3D3D3} 89.4 & \cellcolor[HTML]{D3D3D3} \textbf{19.95} \\
        \midrule
        \multirow{4}{*}{\begin{tabular}{c} Llama-3 \\ 3B \end{tabular}} & MCTS & 48.6 & 111.80 & 64.0 & 57.19 \\
        & Best-of-N & 46.4 & 91.34 & 57.1 & 27.27 \\
        & Beam & 45.2 & 104.36 & 58.4 & 28.27 \\
        & \cellcolor[HTML]{D3D3D3} \textbf{DPTS} & \cellcolor[HTML]{D3D3D3} 50.8 & \cellcolor[HTML]{D3D3D3} \textbf{47.75} & \cellcolor[HTML]{D3D3D3} 67.8 & \cellcolor[HTML]{D3D3D3} \textbf{27.74} \\
        \midrule
        \multirow{4}{*}{\begin{tabular}{c} Llama-3 \\ 8B \end{tabular}} & MCTS & 54.2 & 143.36 & 69.5 & 69.74 \\
        & Best-of-N & 49.8 & 122.63 & 67.6 & 33.48 \\
        & Beam & 49.6 & 142.21 & 68.3 & 34.51 \\
        & \cellcolor[HTML]{D3D3D3} \textbf{DPTS} & \cellcolor[HTML]{D3D3D3} 55.4 & \cellcolor[HTML]{D3D3D3} \textbf{37.98} & \cellcolor[HTML]{D3D3D3} 68.2 & \cellcolor[HTML]{D3D3D3} \textbf{17.82} \\
        \bottomrule
    \end{tabular}
    \vspace{-0.2in}
\end{table}

% \vspace{-20pt}

We conduct a comprehensive comparison across different search algorithms on various models and sizes. We emphasize the search efficiency of our method while maintaining accuracy. 

For efficiency, results in Table~\ref{tab:comparisons} show that DPTS significantly reduces inference time compared to other search methods across various models and datasets, demonstrating superior efficiency. 
On Math500, DPTS achieves the lowest inference time across all models. Particularly, in Qwen-2.5, DPTS reduces the search time from 117.37s (MCTS) to 45.10s in the 1.5B model, achieving nearly a $2.6\times$ speedup, and reduces from 121.46s (MCTS) to 53.50s in 7B model, accelerating nearly $2.2\times$. 
The impact is even more pronounced on the GSM8K, where DPTS achieves a $3.9\times$ speedup from 79.68s to 19.95s in Qwen-2.5-7B. And DPTS even only requires 17.82s for each sample using Llama-3-8B, also $3.9\times$. It forcefully suggests that DPTS effectively mitigates redundant rollouts and optimizes search efficiency. 
% Moreover, while Best-of-N and Beam search show competitive efficiency, they often sacrifice accuracy, whereas DPTS balances both accuracy and computational effectiveness. 
We highlight that, especially on the more challenging tasks, the Early Stop plays a crucial role. Without it, trees often run till timeout on Math500, significantly increasing inference time. In contrast, our approach allows the search tree to terminate earlier within a limited number of expansions, effectively reducing computation time. 
% On simpler datasets, termination primarily occurs when all paths output an end token, making early stop triggers less frequent. However, our method still maintains a clear efficiency advantage overall, demonstrating superior computational efficiency across different levels of task complexity.

For accuracy, DPTS maintains the searching quality and even outperforms the existing algorithms with half or even less of the reasoning time. 
On the Math500 dataset, DPTS achieves the highest accuracy in all experiment cases, surpassing MCTS, Best-of-N, and Beam search. Notably, for Qwen-2.5-1.5B, DPTS improves accuracy from 56.6\% (MCTS) to 59.2\%. 
A similar trend is observed on GSM8K, where DPTS either matches or slightly improves accuracy over MCTS, and surpasses Best-of-N and Beam Search by a wide margin. On Llama-3-3B, DPTS has 67.8\% accuracy, outperforming the previous best MCTS by 3.8\% with only 48.5\% time consumption. 
These results highlight that DPTS maintains or even enhances solution quality while significantly improving inference speed, making it a more robust and efficient search algorithm for complex reasoning tasks. 




\begin{table}
    \centering
    \caption{Ablation study of each component in DPTS framework. ``AP'': adaptive parallelism. ``S'': Searching. ``T'': Transition. ``Best Index'': The average index of terminated path leads to the best solution.  }
    \vspace{-0.1in}
    \label{tab:ablation}
    \resizebox{0.95\linewidth}{!}{
    \begin{tabular}{lccrc}
    \toprule
        \textbf{Algo.} & \textbf{${|P|}$} & \textbf{Acc.}  & \textbf{Time (s)} & \textbf{Best Index} \\ \midrule
         Baseline & 1 & 56.6 & 117.37 & 10.45 \\ \midrule
         % Baseline & 4 & 59.8 & 105.61 & 7.79 \\ %  数据不是很好解释,acc 太高了,虽然推理速度慢
         Baseline & AP & 58.8 & 108.06 & 8.27 \\ 
         + S & AP & 58.2 & 76.81 & 4.66 \\ 
         + T & AP & 57.0 & 32.22 & 2.51 \\ 
         + S + T & AP & 59.2 & 45.10 & 4.17 \\ 
     \bottomrule
    \end{tabular}
    }
    \vspace{-0.2in}
\end{table}


\subsection{Ablation Study}

To analyze the contribution of each component within the DPTS framework, we conduct an ablation study on Qwen-2.5-1.5B with Math500 in Table~\ref{tab:ablation}. In this study, we use the classical MCTS as the baseline and incrementally integrate our proposed techniques to evaluate their impact. 

We begin with the original MCTS (non-parallel, $|P|=1$) as the baseline. It spends the most time per sample and has the largest best index 10.45. 
% , which means that after 10.45 terminated paths, it finds the best solution. 
% P=4 这行数据不是很好,这段先不加了
% Simply increasing $|P|$ does not reduce inference time, due to the severer redundant exploitation when conducting parallelism. This occurs because of two reasons. One is the waiting for all paths to terminate. Without supplementary nodes (such as our Deep Seek: explore → exploit), rollout latency is determined by the longest path, which diminishes efficiency. The other is the larger possibility of being terminated by memory overflow. The excessive computation and KV cache storage on suboptimal rollouts may aggravate the memory occupation. As a result, finalize the search process before it reaches the optimal path due to memory overflow. 
% When the finalization condition is set to a timeout or memory overflow, accuracy may degrade due to excessive computation and KV cache storage on suboptimal rollouts, restricting the model’s ability to store and exploit promising paths. 
We then apply Parallelism Streamline with Adaptive Parallel Generation (AP), and accuracy improves. It shows that the trees are able to grow faster and larger to include a better solution with parallelism. 
% the model allocates available GPU resources more effectively. 

Next, we assign the node status as exploit or explore nodes for each expansion with Search Mechanism, denoted as ``+S | AP'' in Table~\ref{tab:ablation}. 
% By incorporating a depth and breadth balanced strategy, 
The search process becomes significantly more structured and targeted, leading to a boost in efficiency. The time of each sample saves by 31.24s (28.9\%$\downarrow$). It finds the best path within an average of 4.66 terminated paths, much fewer than Baseline AP. 
% demonstrating the impact of the balanced exploitation and exploration. 

Moreover, when only applying the Early Stop strategy in the Transition Mechanism (denoted as ``+T | AP), we obtain fast inference with much less time and paths. However, since we only use the exploitation nodes without exploring the possible branches, the accuracy is relatively low. Therefore, we claim that the Search and Transition Mechanism should be used as a whole: the Search mechanism provides different node statuses for exploitation and exploration, while the Transition mechanism makes them flexibly change and update. 

Finally, we combine our Search and Transition Mechanism (denoted as ``+S+T | AP''), enabling Early Stop and Deep Seek. It shows the best search results in accuracy and efficiency. 
% with only 54.32s per sample, and 4.17 paths to reach the optimal solution on average. 
It demonstrates that DPTS is efficient in quickly identifying optimal solutions and conducting deep reasoning. 

Results show that each component of DPTS contributes significantly to improving inference speed and reasoning accuracy, making it a robust and scalable framework for parallel tree search. 




\subsection{Hyperparameter Analysis}

\begin{table}
    \centering
    \caption{Hyperparameter $\lambda_\mathrm{es}$ and $\lambda_\mathrm{ds}$ in transition thresholds $\theta_\mathrm{es}$ and $\theta_\mathrm{ds}$. ``ES (Early Stop)~\%'' and ``DS (Deep Seek)~\%'' are the ratios of the node type transition between the exploitation and exploration. More results can be found in Appendix~\ref{sec:app:hyperparameter}. }
    \vspace{-0.05in}
    \label{tab:hyperparameter}
    \resizebox{\linewidth}{!}{
    \begin{tabular}{cccccc}
    \toprule
        \textbf{$\lambda_\mathrm{es}$} & \textbf{$\lambda_\mathrm{ds}$} & \textbf{Acc.} & \textbf{Time (s)} & \textbf{ES~(\%)}  & \textbf{DS~(\%)} \\ \midrule
        1.0 & 1.0 & 53.0 & 47.59 & 41.4 & 10.5 \\ 
        % 0.95 & 0.95 & 57.8 & 43.99 & 23.4 & 18.2 \\ 
        0.9 & 0.9  & 58.6 & 43.30 & 15.6 & 20.9 \\
        % 0.85 & 0.85 & 58.4 & 42.60 & 9.67 & 23.7 \\
        0.8 & 0.8 & 58.0 & 46.33 & 8.1  & 23.9 \\
        % 0.7 & 0.8 & & \\
        % 0.7 & 0.7 & 58.4 & 41.58 & 5.3 & 27.5 \\
        0.6 & 0.6 & 57.4 & 44.39 & 6.1 & 32.3   \\
        0.4 & 0.4 & 56.6 & 38.41 & 0 & 0.1 \\
        % 0.2 & 0.2 & 57.0 & 40.71 & 0 & 0.1  \\
        % 0   & 0   & 54.4 & 42.78 & 0 & 0.1  \\
        \bottomrule
    \end{tabular}
    }
    \vspace{-0.12in}
    % \vspace{5pt}
\end{table}

We conduct a hyperparameter study in Table \ref{tab:hyperparameter} on the thresholds $\theta_\mathrm{es}$ and $\theta_\mathrm{ds}$ in the Transition mechanism. 
When $t < t^*$, the threshold $\theta$ follows the mean-based strategy determined by $\lambda$. When $t \geq t^*$, it turns to a max-based one. 
Empirically, we set  $t^* = 5$  to balance the flexibility and efficiency. 

Experimental results demonstrate that our method is robust to $\lambda$. We try different $\lambda$ and report the average ES/DS ratios per sample. 
We highlight that $\lambda_\mathrm{es}$ and $\lambda_\mathrm{ds}$ can be set differently based on the specific task. But DPTS consistently works well when $\lambda\in[0.6, 0.8]$. It demonstrates that the Transition mechanism is effective in mitigating the redundancy issue during search progress. 
% regardless of the transition ratio setting. 
However, it should be noticed that, if $\lambda$ is large, the ES transition may be aggressive, which leads to unsatisfactory results (e.g. $\lambda=1.0$). Meanwhile, if $\lambda$ is too small, it degrades to all exploitation nodes, resulting in low efficiency as well. 

\begin{figure}
    \centering
    \includegraphics[width=0.95\linewidth]{figs/singlecol_ee_proportion.pdf}
    \vspace{-0.1in}
    \caption{Proportions of exploit and explore nodes throughout the search process. }
    \label{fig:ee_proportion}
    \vspace{-0.12in}
\end{figure}

\subsection{Visualizations}

To provide an intuitive understanding of the effectiveness of our proposed method, we present visualizations of searching trajectories. % sample tree growth during the search process. 

First, we analyze the dynamic changes in the number of exploitation and exploration nodes throughout the search process in Figure~\ref{fig:ee_proportion}. The Deep Seek transition temporarily increases the proportion of exploit paths, allowing promising nodes to receive deeper reasoning. However, as the threshold $\theta_\mathrm{es}$ increases, exploit nodes are more likely to reach the threshold and stop. As a result, the number of exploit nodes naturally decreases, reinforcing a balance between exploitation and exploration. This dynamic adaptation ensures that DPTS stretches on the most promising branches under the constraint of computational resources. 

\begin{figure}
    \centering
    \includegraphics[width=\linewidth]{figs/visualization_dpts.pdf}
    \vspace{-0.2in}
    \caption{Visualization of DPTS Tree. The green boxes are early stopped nodes based on their prior confidence using our \textit{Early Stop} mechanism, and the purple boxes are the terminated nodes with posterior reward scores. }
    \vspace{-0.12in}
    \label{fig:dpts_tree}
\end{figure}

Second, we show the trees generated by DPTS and analyze the search behavior in Figure~\ref{fig:dpts_tree}. It does not continue exploitation on low-confidence nodes, effectively pruning unpromising branches after shallow exploration. 
Additionally, the trees are capable of stable reasoning focus, with deep exploitation on promising paths. 
Therefore, the generated trees exhibit a relatively narrow width, as DPTS primarily expands nodes that are more relevant to the optimal path and spend less time on unnecessary regions. 
It demonstrates that DPTS 
% efficiently 
% focuses the thinking trajectory, 
ensures high-potential paths receive deeper thinking within a limited time and memory budget. 


\section{Limitations and Conclusion}
This work has a few limitations. To start, we focused our search on GenAI-enabled work practices performed in the HCI community. For this purspose, we limited ourselves to the ACM digital library. As more work emerges around how GenAI is being used, looking at broader research communities will help to tell a more comprehensive story. Further, the papers that we found relevant to our research objective were mostly qualitative. While this was appropriate to the nature of our question, quantitative survey studies can complement our narratives that we identified.

Finally, although GenAI tools are becoming accessible in fields beyond technology, the reviewed studies predominantly focused on technology-related occupations, highlighting a critical need for HCI studies to examine GenAI's impact across a broader range of professions.

In summary, this paper analyzed 23 papers to understand how GenAI is being used by practitioners to craft their jobs. We found that practitioners used GenAI to transform targeted aspects of the tasks they were performing, as well as to shape their roles and relationships. Based on our findings, we discussed how bottom-up usage of these tools was changing roles in unconventional ways, shifting task demand from high-level abstract thinking to more routine tasks, and facilitating the decomposition of roles into piecework. 
%We also suggest a need to expand the job crafting framework to consider ways in which practitioners craft the technology they use to transform their work experiences.



\newpage 
\section*{Limitations}

% Since December 2023, a "Limitations" section has been required for all papers submitted to ACL Rolling Review (ARR). This section should be placed at the end of the paper, before the references. The "Limitations" section (along with, optionally, a section for ethical considerations) may be up to one page and will not count toward the final page limit. Note that these files may be used by venues that do not rely on ARR so it is recommended to verify the requirement of a "Limitations" section and other criteria with the venue in question.

% 和 ethical considerations 一块儿合计最多一页,且不算在最终页数里


% 并行路径的共享前缀可以通过 DEFT 等算法进一步减少数据搬运操作来减少 latency % 可以和其他reasoning方法结合

% 局限于 reasoning 任务,并没有在其他任务上做验证  % t* 的设置比较经验性,可能扩展到其他任务的时候会不适用。

% 我们认为这种方法可以用于online training, 通过提升生成质量来提升policy improvement的效果,但由于设备受限而没有实验证明。

Our DPTS framework focuses on selecting and refining the search paths, but does not involve hardware design. Therefore, it is orthogonal to low-level methods. For example, we can integrate DEFT~\cite{yao2024deft} to reduce the data transportation of the shared prefixes, leading to further acceleration. 
Also, our method is validated only on math reasoning tasks, and has not been tested on other domains, such as coding or scientific problems. However, we believe its generalizable capabilities make it applicable across a wide range of fields. 
Additionally, we envision that this method can also be applied to online training by improving generation quality. We leave these attempts to our future work. 





% \section*{Acknowledgments}

% This document has been adapted
% by Steven Bethard, Ryan Cotterell and Rui Yan
% from the instructions for earlier ACL and NAACL proceedings, including those for
% ACL 2019 by Douwe Kiela and Ivan Vuli\'{c},
% NAACL 2019 by Stephanie Lukin and Alla Roskovskaya,
% ACL 2018 by Shay Cohen, Kevin Gimpel, and Wei Lu,
% NAACL 2018 by Margaret Mitchell and Stephanie Lukin,
% Bib\TeX{} suggestions for (NA)ACL 2017/2018 from Jason Eisner,
% ACL 2017 by Dan Gildea and Min-Yen Kan,
% NAACL 2017 by Margaret Mitchell,
% ACL 2012 by Maggie Li and Michael White,
% ACL 2010 by Jing-Shin Chang and Philipp Koehn,
% ACL 2008 by Johanna D. Moore, Simone Teufel, James Allan, and Sadaoki Furui,
% ACL 2005 by Hwee Tou Ng and Kemal Oflazer,
% ACL 2002 by Eugene Charniak and Dekang Lin,
% and earlier ACL and EACL formats written by several people, including
% John Chen, Henry S. Thompson and Donald Walker.
% Additional elements were taken from the formatting instructions of the \emph{International Joint Conference on Artificial Intelligence} and the \emph{Conference on Computer Vision and Pattern Recognition}.


% Bibliography entries for the entire Anthology, followed by custom entries
%\bibliography{anthology,custom}
% Custom bibliography entries only
\bibliography{acl_latex}

\appendix
\newpage

% \appendix  
\section*{Appendix}
\label{sec:appendix}
% \tableofcontents  

\subsection*{Contents}
\begin{description}
    \item [\textbf{A}] \textbf{More Observations of Motivation} .............  \pageref{sec:app:motivation}
        \begin{description}
            \item [A.1] Wasted Tokens and Expansions ......... \pageref{sec:app:wasted}
            \item [A.2] Examples of DFS and BFS Trees ...... \pageref{sec:app:trees}
        \end{description}
    \item [\textbf{B}] \textbf{Formulas and Algorithms} ........................... \pageref{sec:app:formulas_and_algorithms}
        \begin{description}
            \item[B.1] Formulas in Parallelism Streamline ... \pageref{app:sec:parallel_reasoning}
            \item[B.2] Algorithms for Searching and Transition Mechanism ....................................... \pageref{sec:app:algorithms}
        \end{description}
    \item [\textbf{C}] \textbf{Additional Details about Experiment} ...... \pageref{app:sec:exp}
        \begin{description}
            \item [C.1] Comparison Methods ......................... \pageref{app:sec:comparison_methods}
            \item [C.2] Experimental Settings ....................... \pageref{app:sec:exp_setting}
            \item [C.3] Distribution of Best Path Index ......... \pageref{app:sec:best_path_index}
            \item [C.4] Additional Results of $\lambda$ ...................... \pageref{sec:app:hyperparameter}
        \end{description}
    \item [\textbf{D}] \textbf{Related Work} ............................................. \pageref{app:sec:related_work}
\end{description}

\hspace{0pt}

\section{More Observations of Motivation}
\label{sec:app:motivation}
\subsection{Wasted Tokens and Expansions}
\label{sec:app:wasted}

\begin{figure}[ht]
    \centering
    \includegraphics[width=\linewidth]{figs/draw_wasted_tokens.pdf}
    \caption{The proportion of tokens required for the best path relative to the total tokens generated (left), and the proportion of expansions on suboptimal paths relative to the total number of expansions (right). }
    \label{fig:motivation_waste_tokens}
\end{figure}

To better understand the inefficiencies caused by frequent node switching, we conduct a statistical analysis on Qwen-2.5-1.5B with the Math500 dataset and evaluate the redundancy in token generation and node expansion. 

Token redundancy analysis: In Figure~\ref{fig:motivation_waste_tokens}(left), we compare the total number of tokens generated for each sample (blue line) against the number of tokens required for the best path (yellow line). The samples are sorted in descending order primarily by total token count and secondarily by best-path token count. Our analysis shows that the total token count does not exhibit a strict multiplicative relationship with the best-path token count, but in general, the number of tokens required for the best path is significantly lower than the total token count.
% , with an average difference of XX times. 
This suggests that traditional tree search algorithms generate a large number of unnecessary tokens during exploration.

Expansion redundancy analysis: We also examined the number of node expansions during tree growth (Figure~\ref{fig:motivation_waste_tokens}(right)). The blue line represents the total number of expansions for each sample, while the green line represents the number of expansions on suboptimal paths (i.e., nodes that do not contain any part of the optimal solution). While there is no strict multiplicative correlation between these two metrics, the green line closely follows the blue line, indicating that a significant proportion of expansions occur on suboptimal paths. This further supports the observation that traditional tree search algorithms frequently explore unnecessary areas before finding the best solution.



\subsection{Examples of DFS and BFS Trees}
\label{sec:app:trees}

\begin{figure}[ht]
    \centering
    \includegraphics[width=\linewidth]{figs/motivation_dfs_trees.pdf}
    \caption{Growth of two trees with DFS algorithm. }
    \label{fig:motivation_dfs_trees}
\end{figure}

As illustrated in Figure~\ref{fig:motivation_dfs_trees}, which shows two typical depth-first search (DFS) trees, we visualize the node expansion process in layers based on tree depth. 
The darker reddish-brown nodes represent high-confidence nodes, while the lighter nodes indicate lower-confidence ones. The arrows denote parent-child relationships, where dark blue arrows indicate later-generated nodes and light blue arrows represent earlier-generated nodes. 

From the figure, we can clearly observe the reasoning trajectory of tree search: starting from the root node, the search prioritizes the child node with the highest confidence, then recursively expands deeper by selecting the most promising child node at each level. This continues until a termination condition is met, at which point the search backtracks and explores alternative paths from the root node. Due to the nature of this process, different paths vary significantly in their depth and termination points. Moreover, the next explored path does not follow a strict spatial or hierarchical pattern within the tree.

We also observe redundant exploration issues in the right two branches. At tree depths 4/5, the confidence scores of the expanded nodes are noticeably lower compared to previously explored nodes. However, due to the inherent mechanics of depth-first search (DFS), the algorithm continues expanding these nodes until the termination condition is met, even if the intermediate confidence scores remain consistently low. As a result, considerable computation is wasted on redundant expansions and token generations, with little contribution to improving the final output quality.


\begin{figure}[ht]
    \centering
    \includegraphics[width=\linewidth]{figs/motivation_bfs_trees.pdf}
    \caption{Growth of two trees with BFS algorithm. }
    \label{fig:motivation_bfs_trees}
\end{figure}

% While breadth-first search (BFS) provides a more structured and evenly distributed exploration pattern, it suffers from frequent node switching, which prevents deep reasoning and leads to a tendency of shallow exploration. Unlike depth-first search (DFS), which aggressively expands a single path before backtracking, BFS systematically explores a wider range of possibilities but often fails to fully develop any single thought process. 

As illustrated in Figure~\ref{fig:motivation_bfs_trees}, BFS results in a flatter, more uniform, top-down expansion structure compared to the trees observed in Figure~\ref{fig:motivation_dfs_trees}. This behavior creates two key inefficiencies: 
(1) Incomplete reasoning before termination: In our experiments on the Math500 dataset, a pure BFS approach resulted in 178 (about 35.6\%)  of reasoning paths terminating without generating an answer (e.g., Tree 1 in Figure~\ref{fig:motivation_bfs_trees}). The algorithm explores many different areas of the tree but often fails to pursue any one path deeply enough to reach a valid conclusion. 
(2) Excessive expansions and token redundancy: Even when BFS eventually finds a correct answer, it tends to consume significantly more expansions and tokens than necessary (e.g., Tree 2 in Figure~\ref{fig:motivation_bfs_trees}). The best path (highlighted in yellow arrows) has a depth of only 4, yet before discovering this optimal solution, BFS explores a large number of additional nodes (light blue arrows), many of which do not contain any part of the optimal path.

\section{Formulas and Algorithms}
\label{sec:app:formulas_and_algorithms}

\subsection{Formulas in Parallelism Streamline}
\label{app:sec:parallel_reasoning}

% \textbf{Adaptive Parallelism Queue}: Parallelism queue $P$ has dynamically adjustable length, which adapts to the available GPU memory during inference. Specifically, the queue size $|P|$ is defined as: $|P| = \frac{O_{{max}} - O_{{init}}}{O_{{peak}} - O_{{init}}}$, where  $O_{{peak}}$  represents the peak memory usage during the last generation, and  $O_{{init}}$  is the memory consumption at model initialization phase. This ensures a reasonable parallel number, allowing an efficient utilization of computational resources. 

% \textbf{Node Data Structure}: In our framework, each node maintains the following key attributes at the top level: a node identifier, a pointer to the parent node, the prior confidence, complete sequence from the root node to the current node ($\mathrm{Seq}^{1 \sim n}$), past key-value cache specific to this node ($\mathrm {KV}^n$). Therefore, the node structure can be written as:
% \begin{equation}
%     Node = [\mathrm{id}, \mathrm {parent}, \mathrm {conf.}, \mathrm {KV}^n, \mathrm {Seq}^{1 \sim n}].
% \end{equation}
% A major challenge in memory optimization arises from the KV cache, which consumes significant GPU memory. To avoid redundant storage, each node only retains its own past KV cache, rather than storing the entire sequence of past KV.  


\paragraph{Data Collection and Preparation.} 
Before executing parallel inference, we should collect the input data that is stored in separate memory locations. Based on the Node Data Structure (refer to Appendix~\ref{app:sec:parallel_reasoning}), which is $[\mathrm{id}, \mathrm {parent}, \mathrm {conf.}, \mathrm {KV}^n, \mathrm {Seq}^{1 \sim n}]$, we need to concatenate the past KV caches and input sequences of different nodes into single large batch matrices. However, as discussed in Sec.~\ref{sec:motivation}, tree search paths exhibit varying path lengths, meaning that both past KV caches and context sequences have different sizes. 
To handle the length disparity and support arbitrary node parallelism, we apply padding for shorter past KV and context sequence: 
% we apply left padding for nodes with shorter past KV caches, and right padding for nodes with shorter input sequences: 
% {\footnotesize
\begin{equation}
\label{eq:kv}
\begin{split}
    & \mathrm{KV}^{1\sim n} = \mathtt{concat}\left(\mathrm{KV}^{r^n}, \cdots, \mathrm{KV}^{a^n_1}, \mathrm{KV}^n\right), \\
    & \mathrm{padding}^n = \mathbf{0}_{\max\left(\forall_{m\in P}|\mathrm{KV}^{1\sim m}| \right) - |\mathrm{KV}^{1\sim n}|},  \\
    & \mathrm{KV}^{1\sim n}_{\mathrm{pad}} =  \mathtt{concat} \left(\mathrm{padding}^{n}, \mathrm{KV}^{1\sim n} \right), \\
    & \mathrm{KV}^{all} =  \mathtt{stack}\left ( [\forall_{n\in P}\ \mathrm{KV}^{1\sim n}_{\mathrm{pad}}]\right), 
\end{split}
\end{equation}
% }
where $r$ and  $a_1, a_2, \dots$ are the root node and $1^{st}, 2^{nd}, \dots$ ancestors of $n$, respectively. $\mathbf{0}_{len}$ is a matrix of zeros with length $len$. Similar to above, the input sequences are also padded and stacked: 
% {\footnotesize
\begin{equation}
\label{eq:seq}
\begin{split}
& \mathrm{padding}^n = padding\_token_{\max(\forall_{n\in P} |\mathrm{Seq}^{1\sim n}|)},  \\
& \mathrm{Seq^{1\sim n}_{pad}} =  \mathtt{concat} \left(  \mathrm{Seq^{1\sim n}}, \mathrm{padding}^n\right), \\ 
& \mathrm{Seq}^{all}=\mathtt{stack}\left( \forall_{n\in P}\mathrm{Seq^{1\sim n}_{pad}} \right), 
\end{split}
\end{equation}
% }
where $padding\_token_{len}$ is a vector of the predefined padding token id with length $len$. 
In this way, the data is fed into the LLM for parallel generation. Additionally, techniques like DEFT~\cite{yao2024deft} are orthogonal to our DPTS and can be integrated to identify and merge shared prefixes, further optimizing inference efficiency.

\paragraph{Generation and Updating.} After the generation phase, we obtain new output sequences and past KV caches. These are then partitioned based on the tree width. Specifically, we segment past KV caches and only store those corresponding to new tokens. Output sequences are completely stored rather than fragmented, due to the negligible memory overhead. For clearer demonstration, the output of $i^\mathrm{th}$ node in $P$ can be written as 
% {\footnotesize 
% \begin{equation}
% \label{eq:n_new}
% \begin{split}
%     &  \mathrm{KV}^{n^{ij}}  = \mathbf{n}.\mathtt{past\_kv}_{[ij, \dots, |\mathrm{KV}^{all}|:]}, \forall j\in[1, \dots, \mathtt{tw}] \\
%     & \mathrm{Seq}^{1\sim n^{ij}} = \mathbf{n}.\mathtt{output}_{[ij]} \\ 
%     & n^{ij}  = [\mathrm{id}, n^i, \mathrm{KV}^{n^{ij}}, \mathrm{Seq}^{1\sim n^{ij}}] \\ 
%     & \mathbf{n^{new}} = \{n^{i1}, \dots, n^{i\mathtt{tw}}\}, 
% \end{split}
% \end{equation}
% }
\begin{center}
\begin{equation}
\label{eq:n_new}
\begin{split}
    & \mathbf{KV}^{n^{ij}} = \mathbf{n}.\mathtt{past\_kv}_{[ij, \dots, |\mathbf{KV}^{\text{all}}|:]}, \forall j \in \left[1, \dots, w\right] \\
    & \mathbf{Seq}^{1 \sim n^{ij}} = \mathbf{n}.\mathtt{output}_{[ij]} \\
    & n^{ij} = \left[\mathrm{id}, n^i, \mathbf{KV}^{n^{ij}}, \mathbf{Seq}^{1 \sim n^{ij}}\right] \\
    & \mathbf{n^{new}} = \left\{ n^{i1}, \dots, n^{iw} \right\}
\end{split}
\end{equation}
\end{center}
where $w$ is the tree width, $\mathbf{n}$ is the generation output in parallel manner. 
If sequences were stored in a fragmented manner, every inference step would require additional collection and concatenation, introducing unnecessary latency.  This approach is a trade-off between inference speed and memory consumption. 
Newly generated nodes are then updated into the candidate node pool $N$, making them available for subsequent selection processes.



\subsection{Algorithms for Searching and Transition Mechanism}
\label{sec:app:algorithms}

In the main text, due to paper length constraints, we only present the overall process in Algorithm~\ref{alg:algorithm_parallel}, which connects the entire framework’s algorithms and formulations, including Parallelism Streamline and the Search and Transition Mechanism.

In the above section, we provided the mathematical formulation of the Parallelism Streamline. In the following, we further supplement the algorithmic details of Search, Transition, and Reward, offering readers a clear and intuitive representation of the algorithmic process. 



\begin{algorithm}[ht]
    \caption{Searching}
\label{alg:searching}
\begin{algorithmic}[1]
\REQUIRE Parallel queue $P$, current parallel queue size $\tau_P$, candidate node pool $N$. 
\ENSURE Updated $P$. 
\STATE $e_1\gets 0$
\FORALL{$n\in P$}
    \IF{$n$.mode$ = \mathtt{EXPLOIT}$}
    \STATE $e_1 \gets e_1  + 1$
    % \ELSE
    % \STATE $e_2 \gets e_2 + 1$
    \ENDIF
\ENDFOR
\IF{$|P|<\tau_P$}
    \STATE $N'\gets$ Descending $N$ based on conf. \\
    % , $\forall n \in N$
    \STATE $\mathbf{u} \gets N'[:\tau_P-|P|]$ 
    \FORALL{$u \in \mathbf{u}$}
        \IF {$ e_1<p|P| $}
        \STATE $u$.mode $\gets \mathtt{EXPLOIT}$
        \STATE $e_1 \gets e_1  + 1$
        \ELSE
        \STATE $u$.mode $\gets \mathtt{EXPLORE}$
        \ENDIF 
    \ENDFOR
    \STATE $P \gets P \cup \mathbf{u}$
\ENDIF
\RETURN $P$
\end{algorithmic}
\end{algorithm}

Firstly, Algorithm~\ref{alg:searching} demonstrates the searching mechanism. Specifically, during initialization or when the number of nodes in the parallel queue $P$ is less than the maximum parallelism $\tau_P$, the algorithm selects the $\tau_P- |P|$ highest-confidence nodes from the candidate node pool $N$ as supplementary nodes (Lines 8-9).

Then, the highest-confidence nodes are designated as exploit nodes (Line 12) until the proportion of exploit nodes reaches the ratio $p$. The remaining selected nodes are assigned as explore nodes. Finally, these newly selected nodes are merged into $P$ to prepare for the next cycle of parallel expansion. 

\begin{algorithm}[ht]
    \caption{Transition}
\label{alg:transition}
\begin{algorithmic}[1]
\REQUIRE Parallel queue $P$, transition thresholds $\theta_\mathrm{es}$ and $\theta_\mathrm{ds}$. 
\ENSURE Updated $P$. 
\FORALL{$n \in P$}
    \STATE $n^*=\max_\mathrm{conf.}$($n$.children)
    \IF{$n$.mode $=$ $\mathtt{EXPLOIT}$ \AND $c_{n^*} >\theta_\mathrm{es}$ \OR $n$.mode $=$ $\mathtt{EXPLORE}$ \AND $c_{n^*} >\theta_\mathrm{ds}$}
        \STATE $P \gets P\cup \{n^*\}$
        \STATE $n^*$.mode $\gets \mathtt{EXPLOIT}$ 
    % \ELSE
    \ENDIF
    \STATE $P \gets P \setminus n$
\ENDFOR
\RETURN $P$
\end{algorithmic}
\end{algorithm}

Next is the Transition Algorithm~\ref{alg:transition}. After each expansion, we iterate through all nodes in the parallel queue $P$ and identify the best child node  $n^*$ for each node.

Then, based on the category of node $n$, we compare $n^*$ with the corresponding threshold  $\theta$. If the early stop condition is not met or the deep seek condition is met, we add  $n^*$ as a new exploit node into $P$. Otherwise, the node will no longer be expanded and will be evicted from $P$.

\begin{algorithm}[ht]
    \caption{Reward}
\label{app:algo:reward}
\begin{algorithmic}[1]
\REQUIRE Parallel queue $P$, candidate node pool $N$. 
\ENSURE Updated $N$. 
\FORALL{$n \in P$} 
    % \PARALLEL
        \STATE $n$.children $\gets \text{Eq. (\ref{eq:n_new})}$ 
        \FORALL{$m \in n$.children}
            \IF{$\mathtt{is\_terminate}(m)$}
                \STATE $m$.reward $\gets \mathtt{reward}(m)$
            \ELSE
                \STATE $N \gets N \cup \{m\}$
            \ENDIF
        \ENDFOR
    \ENDFOR
\RETURN $N$
\end{algorithmic}
\end{algorithm}

After completing an expansion, we check whether each node’s path meets the termination conditions, such as reaching the maximum token limit or generating an end token. 
If a path satisfies the termination condition, exploration of that path stops, and a reward is computed as the final path score. We demonstrate this process in Algorithm~\ref{app:algo:reward}, which is identical to previous tree search algorithms and is not discussed in detail. However, for the sake of algorithmic completeness, we explicitly include it here.

\section{Additional Details about Experiment}
\label{app:sec:exp}

\subsection{Comparison Methods}
\label{app:sec:comparison_methods}

We compare DPTS against three widely used search algorithms: (1) Monte Carlo Tree Search~(MCTS)~\cite{sprueill2023monte} balances exploration and exploitation when sampling, while the selected paths rollout till termination, (2) Best-of-N~\cite{cobbe2021training} performs multiple independent rollouts and selects the highest-scoring output, (3) Beam Search~\cite{Yao_2023_Tree} expands multiple hypotheses in parallel, pruning low-scoring candidates at each step to maintain a fixed-width search~\cite{snell2024scaling}. 
Since efficient tree search algorithms have recently regained attention after the emergence of LLM reasoning, the strong baselines are limited. As a result, we primarily compare DPTS against these typical and well-established search algorithms to demonstrate the effectiveness of our method. 

 
\subsection{Experimental Settings}
\label{app:sec:exp_setting}

% 实验参数: tree_width:4 tree_depth:16 mini_step:128 % 其实 103 确实实验结果更好不知道为什么 otz
% max token:2048 mcts_max_time:120

The experimental settings are as follows: we set the tree width to 4, tree depth to 16, mini step to 100, and the maximum token limit to 2048. The MCTS time limit is 120 seconds, and the threshold parameter is empirically set to $t^*=5$. All models were downloaded from Hugging Face.

For evaluation, we implemented a custom codebase, which is included in the supplementary materials. Inference automatically terminates if it exceeds the timeout limit or encounters a memory overflow. Within these constraints, the search tree can expand and roll out indefinitely, ensuring comprehensive exploration during inference. 

\subsection{Distribution of Best Path Index}
\label{app:sec:best_path_index}

We use a histogram to visualize the earliest (blue bar) and shortest (green bar) best path index. Through an ablation study, we examine how the best solution in the search path evolves as our method is progressively introduced.

In the baseline method, the best path typically appears around the 8th terminated path. Incorporating the parallelism streamline does not directly affect the accuracy of the search path. However, after adding the searching and transition mechanism, DPTS finds the best solution region more quickly and reaches the best path earlier.

\begin{figure}
    \centering
    \includegraphics[width=\linewidth]{figs/best_path_distribution.pdf}
    \caption{The distribution of the earliest (blue bar) and shortest (green bar) best path index. }
    \label{fig:best_path_index}
\end{figure}


\subsection{Additional Results of $\lambda$}
\label{sec:app:hyperparameter}

We conducted a more detailed experiment on $\lambda$, with results presented in Table~\ref{app:tab:hyperparameter} and Figure~\ref{app:fig:lambda_curve}. It is evident that when $\lambda$ is set within a reasonable range (e.g., $[0.7,0.9]$), both accuracy and inference time exhibit optimal performance. In this range, the proportion of DS\% (deep seek transitions) is higher than ES\% (early stop transitions), indicating that more high-confidence nodes are being timely converted into exploit nodes. At the same time, a small number of paths are reassigned as low-score paths during inference and subsequently terminated.

However, when $\lambda$ is too large (e.g., close to $1$), the proportion of es increases aggressively. This suggests that many exploit nodes being expanded have scores within the range of $[0.9,1.0]$, and setting the threshold in this range may cause some correct paths to be prematurely stopped. Conversely, when $\lambda$ is too small (e.g., $< 0.4$), both ES and DS proportions drop to nearly zero. This occurs because most node scores exceed the threshold, causing nearly all paths to expand under exploitation mode. Moreover, an overly small early stop threshold causes no paths to terminate, effectively degrading the search into an exploitation-only strategy. 

Therefore, selecting a suitable $\lambda$ is important. A larger  $\lambda$ imposes stricter exploitation conditions, leading to more paths being stopped and fewer paths being converted to deep seeking. Conversely, a smaller $\lambda$ results in looser conditions, allowing more exploiting paths to continue rolling out until they reach a termination condition, while more high-confidence paths transition into deep seeking.


\begin{figure}
    \centering
    \includegraphics[width=\linewidth]{figs/draw_lambda_curve.pdf}
    \caption{Illustrations of hyperparameter analysis. }
    \label{app:fig:lambda_curve}
\end{figure}

\begin{table}
    \centering
    \caption{Hyperparameter $\lambda_\mathrm{es}$ and $\lambda_\mathrm{ds}$ in transition thresholds $\theta_\mathrm{es}$ and $\theta_\mathrm{ds}$. ``ES (Early Stop)~\%'' and ``DS (Deep Seek)~\%'' are the ratios of the node type transition between the exploitation and exploration.  }
    \label{app:tab:hyperparameter}
    \resizebox{\linewidth}{!}{
    \begin{tabular}{cccccc}
    \toprule
        \textbf{$\lambda_\mathrm{es}$} & \textbf{$\lambda_\mathrm{ds}$} & \textbf{Acc.} & \textbf{Time (s)} & \textbf{ES~(\%)}  & \textbf{DS~(\%)} \\ \midrule
        1.0 & 1.0 & 53.0 & 47.59 & 41.4 & 10.5 \\ 
        0.95 & 0.95 & 57.8 & 43.99 & 23.4 & 18.2 \\ 
        0.9 & 0.9  & 58.6 & 43.30 & 15.6 & 20.9 \\
        0.85 & 0.85 & 58.4 & 42.60 & 9.67 & 23.7 \\
        0.8 & 0.8 & 58.0 & 46.33 & 8.1  & 23.9 \\
        % 0.7 & 0.8 & & \\
        0.7 & 0.7 & 58.4 & 41.58 & 5.3 & 27.5 \\
        0.6 & 0.6 & 57.4 & 44.39 & 6.1 & 32.3   \\
        0.4 & 0.4 & 56.6 & 38.41 & 0 & 0.1 \\
        0.2 & 0.2 & 57.0 & 40.71 & 0 & 0.1  \\
        0   & 0   & 54.4 & 42.78 & 0 & 0.1  \\
        \bottomrule
    \end{tabular}
    }
    \vspace{5pt}
\end{table}

\section{Related Work}


\minisection{Pivot-based approaches}
Pivot translation is an approach that decomposes the translation task into two sequential steps~\cite{wu-wang-2007-pivot, utiyama-isahara-2007-comparison}.
By transferring knowledge from high-resource pivot languages, pivoting is especially effective in translation between low-resource languages \cite{zoph-etal-2016-transfer, aji-etal-2020-neural, he-etal-2022-tencent}.
In this study, pivot translation enables us to obtain high-quality candidates for the ensemble.
\citet{kim-etal-2019-pivot} discusses a pivot-based transfer learning technique where source$\rightarrow$pivot and pivot$\rightarrow$target models are first trained separately, then use pre-trained models to initialize the source$\rightarrow$target model, allowing effective training of a single, direct NMT model.
\citet{zhang-etal-2022-triangular} further investigate the transfer learning approach by utilizing auxiliary monolingual data.


Pivot translation typically employs English as the bridge language.
Nonetheless, previous studies have explored the use of diverse pivot languages, taking into account factors such as data size and the relationships between languages~\cite{paul2009importance, dabre-etal-2015-leveraging}.
By leveraging the ability of pivot translation to produce diverse outputs, several studies have focused on generating paraphrases~\cite{mallinson-etal-2017-paraphrasing, guo2019zeroshot}.
More recently, \citet{mohammadshahi-etal-2024-investigating} uses pivot translation for ensemble, but it requires computing token-level probabilities and fails to improve translation.
Our work shares the motivation with these studies, generating translations depending on the pivot path to obtain a variety of candidates.


\minisection{Ensemble in NLG tasks}
Ensemble learning is a widely adopted strategy to obtain more accurate predictions by employing multiple systems~\cite{sagi2018ensemble}.
In NMT, the traditional approach involves averaging the probability distributions of the next target token, which is predicted at each decoding step by multiple models ~\cite{bojar-etal-2014-findings} or by different snapshots~\cite{huang2017snapshot}.
When multiple sources are available, an ensemble can be conducted with predictions obtained by different sources~\cite {firat-etal-2016-zero}.
Also, a token-level ensemble through vocabulary alignment across LLMs has also been proposed~\cite{eva}.
However, these methods are not applicable to recent black-box models as they cannot compute token-level probabilities at decoding time.


Selection-based ensemble has also been explored, which chooses the final output among the existing candidates.
This can be achieved through majority voting by selecting the most frequent one~\cite{wang2022rationaleaugmented} or selecting the best candidate with QE~\cite{fernandes-etal-2022-quality, howgood}.
Recently, MBR decoding~\cite{GOEL2000115, mbr}, which aims to find the hypothesis with the highest expected utility, has gained attention.
However, this approach limits the final output space to the existing candidate pool.


\begin{figure*}[t]
  \centering
  \includegraphics[width=0.95\textwidth]{Figures/overview.pdf} 
  \caption{Overview of \ours framework.}
  \label{fig:overall}
\end{figure*}


On the other hand, the generation-based ensemble method involves generating a new final prediction.
Fusion-in-Decoder~\cite{fid} proposes an architecture that aggregates additional information with a given input.
More recently, within the context of LLMs, \citet{llm-blender} and \citet{exchangeofthought} investigate a method of using LLMs to generate multiple outputs and aggregate them.
Generating new output through LLMs offers the benefit of explicitly harnessing their pre-trained knowledge within the ensemble process.

% \newpage

% \section{Critical Code}
% \begin{figure*}[t]
%     \centering
%     \begin{lstlisting}[caption=Code]

% print("hello world")
% print("hello world")

%     \end{lstlisting}
%     \label{fig:lst_code}
% \end{figure*}

% \minitoc  % 仅显示附录的目录




\end{document}
