% This must be in the first 5 lines to tell arXiv to use pdfLaTeX, which is strongly recommended.
\pdfoutput=1
% In particular, the hyperref package requires pdfLaTeX in order to break URLs across lines.

\documentclass[11pt]{article}

% Change "review" to "final" to generate the final (sometimes called camera-ready) version.
% Change to "preprint" to generate a non-anonymous version with page numbers.

\usepackage[dvipsnames,svgnames]{xcolor}

\usepackage[preprint]{acl}
% \usepackage[review]{acl}

% Standard package includes
\usepackage{times}
\usepackage{latexsym}
\usepackage{booktabs}
\usepackage{marvosym}

% For proper rendering and hyphenation of words containing Latin characters (including in bib files)
\usepackage[T1]{fontenc}
% For Vietnamese characters
% \usepackage[T5]{fontenc}
% See https://www.latex-project.org/help/documentation/encguide.pdf for other character sets

% This assumes your files are encoded as UTF8
\usepackage[utf8]{inputenc}

% This is not strictly necessary, and may be commented out,
% but it will improve the layout of the manuscript,
% and will typically save some space.
\usepackage{microtype}

% This is also not strictly necessary, and may be commented out.
% However, it will improve the aesthetics of text in
% the typewriter font.
\usepackage{inconsolata}

%Including images in your LaTeX document requires adding
%additional package(s)
\usepackage{graphicx}

\usepackage{amsmath}
\usepackage{algorithm}
\usepackage{algorithmic}

\usepackage{subcaption} % for 'subtable' env
\usepackage{multirow} % for cmd 'multirow', 'multicolumn'
\usepackage{multicol}
\usepackage{colortbl}

\usepackage{enumerate}
\usepackage{enumitem}
\usepackage{amssymb}
\usepackage{listings}
\usepackage{minitoc} 


\lstset{
  language=Python,
  backgroundcolor=\color{lightgray},
  basicstyle=\ttfamily\small,
  breaklines=true,
  numbers=left,
  numberstyle=\tiny,
  captionpos=b,
  frame=single,
  rulecolor=\color{black},
  showstringspaces=false
}

\newcommand{\dyf}[1]{\textcolor{black}{#1}}

\definecolor{jwtcolor}{RGB}{0, 123, 255}
\newcommand{\jwt}[1]{{\color{jwtcolor}#1}}

\renewcommand{\algorithmicrequire}{\textbf{Input:}}
\renewcommand{\algorithmicensure}{\textbf{Output:}}

% ======================
\newcommand{\yc}[1]{\textcolor{red}{#1}}
\newcommand{\ycj}[1]{\textcolor{red}{\bf [Comments: #1] }} 
% ======================

% If the title and author information does not fit in the area allocated, uncomment the following
%
%\setlength\titlebox{<dim>}
%
% and set <dim> to something 5cm or larger.

\makeatletter
\def\thanks#1{\protected@xdef\@thanks{\@thanks\protect\footnotetext{#1}}}
\makeatother
% \title{Hybrid Prior-Posterior Tree Search: Pushing the Boundaries of Inference Scaling Laws for LLM Reasoning}
% \title{Explore \& Exploit Tree Search: Eliminating the Redundant Branch for Parallel LLM Reasoning }
\title{Dynamic Parallel Tree Search for Efficient LLM Reasoning}

% Dynamic Parallel Reasoning


% Author information can be set in various styles:
% For several authors from the same institution:
% \author{Author 1 \and ... \and Author n \\
%         Address line \\ ... \\ Address line}
% if the names do not fit well on one line use
%         Author 1 \\ {\bf Author 2} \\ ... \\ {\bf Author n} \\
% For authors from different institutions:
% \author{Author 1 \\ Address line \\  ... \\ Address line
%         \And  ... \And
%         Author n \\ Address line \\ ... \\ Address line}
% To start a separate ``row'' of authors use \AND, as in
% \author{Author 1 \\ Address line \\  ... \\ Address line
%         \AND
%         Author 2 \\ Address line \\ ... \\ Address line \And
%         Author 3 \\ Address line \\ ... \\ Address line}

% Yifu Ding, Wentao Jiang, Shunyu Liu, Yongcheng Jing, Jinyang Guo, Yingjie Wang, Jing Zhang, Zengmao Wang, Ziwei Liu, Bo Du, Xianglong Liu, Dacheng Tao

\author{Yifu Ding\textsuperscript{1,2}, Wentao Jiang\textsuperscript{3}, Shunyu Liu\textsuperscript{2}, Yongcheng Jing\textsuperscript{2}, Jinyang Guo\textsuperscript{1}, Yingjie Wang\textsuperscript{2}, \\
{\bf Jing Zhang\textsuperscript{3}, Zengmao Wang\textsuperscript{3,*}, Ziwei Liu\textsuperscript{2}, Bo Du\textsuperscript{3}, Xianglong Liu\textsuperscript{1,*}, Dacheng Tao\textsuperscript{2,*}}
  % \texttt{} \\
  % Affiliation / Address line 1 \\
  % Affiliation / Address line 2 \\
  % Affiliation / Address line 3 \\
  % \texttt{email@domain} \\
  \thanks{
  \textsuperscript{*}Corresponding author. \textsuperscript{1}Beihang University, China. 
  \textsuperscript{2}Nanyang Technological University, Singapore. 
  \textsuperscript{3}Wuhan University, China. }
  }

%\author{
%  \textbf{First Author\textsuperscript{1}},
%  \textbf{Second Author\textsuperscript{1,2}},
%  \textbf{Third T. Author\textsuperscript{1}},
%  \textbf{Fourth Author\textsuperscript{1}},
%\\
%  \textbf{Fifth Author\textsuperscript{1,2}},
%  \textbf{Sixth Author\textsuperscript{1}},
%  \textbf{Seventh Author\textsuperscript{1}},
%  \textbf{Eighth Author \textsuperscript{1,2,3,4}},
%\\
%  \textbf{Ninth Author\textsuperscript{1}},
%  \textbf{Tenth Author\textsuperscript{1}},
%  \textbf{Eleventh E. Author\textsuperscript{1,2,3,4,5}},
%  \textbf{Twelfth Author\textsuperscript{1}},
%\\
%  \textbf{Thirteenth Author\textsuperscript{3}},
%  \textbf{Fourteenth F. Author\textsuperscript{2,4}},
%  \textbf{Fifteenth Author\textsuperscript{1}},
%  \textbf{Sixteenth Author\textsuperscript{1}},
%\\
%  \textbf{Seventeenth S. Author\textsuperscript{4,5}},
%  \textbf{Eighteenth Author\textsuperscript{3,4}},
%  \textbf{Nineteenth N. Author\textsuperscript{2,5}},
%  \textbf{Twentieth Author\textsuperscript{1}}
%\\
%\\
%  \textsuperscript{1}Affiliation 1,
%  \textsuperscript{2}Affiliation 2,
%  \textsuperscript{3}Affiliation 3,
%  \textsuperscript{4}Affiliation 4,
%  \textsuperscript{5}Affiliation 5
%\\
%  \small{
%    \textbf{Correspondence:} \href{mailto:email@domain}{email@domain}
%  }
%}

\begin{document}
\maketitle

\begin{abstract}
Tree of Thoughts (ToT) enhances Large Language Model (LLM) reasoning by structuring problem-solving as a spanning tree. However, recent methods focus on search accuracy while overlooking computational efficiency. The challenges of accelerating the ToT lie in the frequent switching of reasoning focus, and the redundant exploration of suboptimal solutions. To alleviate this dilemma, we propose Dynamic Parallel Tree Search (DPTS), a novel parallelism framework that aims to dynamically optimize the reasoning path in inference. 
It includes the Parallelism Streamline in the generation phase to build up a flexible and adaptive parallelism with arbitrary paths by fine-grained cache management and alignment. 
Meanwhile, the Search and Transition Mechanism filters potential candidates to dynamically maintain the reasoning focus on more possible solutions and have less redundancy. Experiments on Qwen-2.5 and Llama-3 with Math500 and GSM8K datasets show that DPTS significantly improves efficiency by 2-4$\times$ on average while maintaining or even surpassing existing reasoning algorithms in accuracy, making ToT-based reasoning more scalable and computationally efficient. 
% Codes are provided in the submission.
% Tree of Thoughts~(ToT) enhances Large Language Model~(LLM) reasoning by structuring problem-solving as a spanning tree. However, recent methods focus on search accuracy while overlooked the computational efficiency. 
% The challenges of accelerating the ToT lies in the frequent switching of reasoning focus, and the redundant exploration on suboptimal solutions. 
% % its sequential data structure limits the utilization of GPU parallelism. In particular, existing algorithms typically cause redundant exploration and frequent switching issues, which becomes even severer when paralleled. 
% To alleviate this dilemma, we propose Dynamic Parallel Tree Search~(DPTS), a novel parallelism framework that aims to dynamically optimize reasoning path in inference. 
% % DPTS balances deep exploration and broad expansion through the search algorithm, and the bidirectional transition mechanism allows the tree to focus on high-confidence solutions. 
% % To this end, we design a flexible parallel framework to support arbitrary nodes with various path lengths for  simultaneous expanding. Meanwhile, the searching algorithm and bidirectional transition mechanism allows the trees to focus on high-confidence solutions with less redundant generation tokens. 
% It includes the Parallelism Streamline in generation phase to build up a finer-grained and flexible 
% % path arrangement, allowing a flexible 
% parallelism with arbitrary paths. Meanwhile, the Search and Transition Mechanism filters potential candidates 
% % with balanced exploitation and exploration during selection phase, which maintains 
% to maintain the reasoning focus on more possible solutions and have less redundancy. 
% Experiments on Qwen2.5 and Llama3 with Math500 and GSM8K datasets 
% % \ycj{xx} different models and datasets 
% show that DPTS significantly improves efficiency by $2\times$ on average while maintaining or even surpassing the existing reasoning algorithms in accuracy, making ToT-based reasoning more scalable and computationally efficient. 
% Our code can be found in the supplementary material.
% \ycj{Showcase your speed-up ratio}  % updated 
\end{abstract}
\section{Introduction}

In recent years, with advancements in generative models and the expansion of training datasets, text-to-speech (TTS) models \cite{valle, voicebox, ns3} have made breakthrough progress in naturalness and quality, gradually approaching the level of real recordings. However, low-latency and efficient dual-stream TTS, which involves processing streaming text inputs while simultaneously generating speech in real time, remains a challenging problem \cite{livespeech2}. These models are ideal for integration with upstream tasks, such as large language models (LLMs) \cite{gpt4} and streaming translation models \cite{seamless}, which can generate text in a streaming manner. Addressing these challenges can improve live human-computer interaction, paving the way for various applications, such as speech-to-speech translation and personal voice assistants.

Recently, inspired by advances in image generation, denoising diffusion \cite{diffusion, score}, flow matching \cite{fm}, and masked generative models \cite{maskgit} have been introduced into non-autoregressive (NAR) TTS \cite{seedtts, F5tts, pflow, maskgct}, demonstrating impressive performance in offline inference.  During this process, these offline TTS models first add noise or apply masking guided by the predicted duration. Subsequently, context from the entire sentence is leveraged to perform temporally-unordered denoising or mask prediction for speech generation. However, this temporally-unordered process hinders their application to streaming speech generation\footnote{
Here, “temporally” refers to the physical time of audio samples, not the iteration step $t \in [0, 1]$ of the above NAR TTS models.}.


When it comes to streaming speech generation, autoregressive (AR) TTS models \cite{valle, ellav} hold a distinct advantage because of their ability to deliver outputs in a temporally-ordered manner. However, compared to recently proposed NAR TTS models,  AR TTS models have a distinct disadvantage in terms of generation efficiency \cite{MEDUSA}. Specifically, the autoregressive steps are tied to the frame rate of speech tokens, resulting in slower inference speeds.  
While advancements like VALL-E 2 \cite{valle2} have boosted generation efficiency through group code modeling, the challenge remains that the manually set group size is typically small, suggesting room for further improvements. In addition,  most current AR TTS models \cite{dualsteam1} cannot handle stream text input and they only begin streaming speech generation after receiving the complete text,  ignoring the latency caused by the streaming text input. The most closely related works to SyncSpeech are CosyVoice2 \cite{cosyvoice2.0} and IST-LM \cite{yang2024interleaved}, both of which employ interleaved speech-text modeling to accommodate dual-stream scenarios. However, their autoregressive process generates only one speech token per step, leading to low efficiency.



To seamlessly integrate with  upstream LLMs and facilitate dual-stream speech synthesis, this paper introduces \textbf{SyncSpeech}, designed to keep the generation of streaming speech in synchronization with the incoming streaming text. SyncSpeech has the following advantages: 1) \textbf{low latency}, which means it begins generating speech in a streaming manner as soon as the second text token is received,
and
2) \textbf{high efficiency}, 
which means for each arriving text token, only one decoding step is required to generate all the corresponding speech tokens.

SyncSpeech is based on the proposed \textbf{T}emporal \textbf{M}asked generative \textbf{T}ransformer (TMT).
During inference, SyncSpeech adopts the Byte Pair Encoding (BPE) token-level duration prediction, which can access the previously generated speech tokens and performs top-k sampling. 
Subsequently, mask padding and greedy sampling are carried out based on  the duration prediction from the previous step. 

Moreover, sequence input is meticulously constructed to incorporate duration prediction and mask prediction into a single decoding step.
During the training process, we adopt a two-stage training strategy to improve training efficiency and model performance. First, high-efficiency masked pretraining is employed to establish a rough alignment between text and speech tokens within the sequence, followed by fine-tuning the pre-trained model to align with the inference process.

Our experimental results demonstrate that, in terms of generation efficiency, SyncSpeech operates at 6.4 times the speed of the current dual-stream TTS model for English and at 8.5 times the speed for Mandarin. When integrated with LLMs, SyncSpeech achieves latency reductions of 3.2 and 3.8 times, respectively, compared to the current dual-stream TTS model for both languages.
Moreover, with the same scale of training data, SyncSpeech performs comparably to traditional AR models in terms of the quality of generated English speech. For Mandarin, SyncSpeech demonstrates superior quality and robustness compared to current dual-stream TTS models. This showcases the potential of  SyncSpeech as a foundational model to integrate with upstream LLMs.


\section{Related Work}
% \label{sec:related_work}

\paragraph{Reasoning with LLMs. }
LLMs have evolved from System 1 tasks (e.g., translation)~\cite{Brown_2020_Language} to System 2 reasoning (e.g., math, logic)~\cite{Kojima_2022_Large}. CoT~\cite{Wei_2022_Chain} enhances multi-step reasoning, with variants like Self-Consistent CoT~\cite{wang2022self}, but its exploration scope remains constrained, limiting its effectiveness.~~\cite{chu2023survey}.
% \vspace{-4pt}
% \noindent\textbf{Tree Search for Reasoning. }
Furthermore, ToT~\cite{Yao_2023_Tree} enables multi-path exploration, leveraging MCTS~\cite{chaslot2008monte} for backtracking and heuristic rollouts~\cite{wan2024alphazero,wang2024q}. However, MCTS remains computationally expensive, with limited work on acceleration methods.

% \vspace{-pt}
\paragraph{LLM Inference Acceleration. }
While LLM inference has been optimized for linear decoding~\cite{lin2024awq}, tree-structured reasoning remains underexplored~\cite{li2024large}. Approaches like Deft~\cite{yao2024deft} optimize prefix sharing, while others use self-consistency for early stopping~\cite{li2024escape}. Efficient tree search for LLM reasoning remains an open challenge.

Due to page limit, we have included a more detailed discussion of related work 
% (1) Reasoning with LLMs, (2) Tree Search for Reasoning, and (3) LLM Inference Acceleration 
in Appendix~\ref{app:sec:related_work}. 
% \section{The Rationale of DPTS}
\section{Method Rationale}

\label{sec:motivation}

% \ycj{Rationale?}
% \ycj{Pre-analysis/Motivation and Problem Definition/Motivation and Challenge Analysis/Pilot Study?}
 
% \ycj{Add a short overview statement here.}


{
In this section, we present empirical findings that highlight the key challenges of tree search in LLM and provide the rationale behind our proposed DPTS. 
% First, the inherent sequential nature of tree search complicates parallel execution, leading to irregular node expansions and varying path lengths. Second, excessive exploitation of low-confidence paths wastes computational resources, suggesting that a confidence-based pruning strategy could mitigate this inefficiency. Finally, tree search methods that prioritize breadth often suffer from frequent switching between paths, hindering deep exploitation and resulting in token and expansion redundancy. Addressing these challenges can significantly enhance the efficiency and effectiveness of tree search methods.
}
First, the frequent switch between paths complicates parallel execution and causes shallow thinking, disrupting the model’s ability to engage in efficient deep reasoning (Sec.~\ref{sec:3.1}). 
% sequential nature of tree search complicates parallel execution, leading to irregular node expansions and varying path lengths. Existing tree search algorithms frequently switch between paths, causing shallow thinking and disrupting the model’s ability to engage in deep reasoning, ultimately degrading generation quality. 
Second, excessive exploitation of low-confidence paths results in redundant rollouts and wastes effort on fewer possible candidates (Sec.~\ref{sec:3.2}). 
% consuming unnecessary computational resources. Without an effective pruning strategy,..., reducing overall efficiency and slowing down search convergence.

% \subsection{Unpredictable Growth Behavior}
\subsection{Frequent Switching}
\label{sec:3.1}

\begin{figure}[t]
    \centering
    \includegraphics[width=0.93\linewidth]
    {figs/draw_switch_times.pdf}
    \vspace{-0.1in}
    \caption{Statistics for switch from the best path to the suboptimal (blue), and total switch (green).} 
    \vspace{-0.1in}
    \label{fig:motivation_switch_path}
\end{figure}
% \ycj{Analysis of why vanilla solution doesn't work well?}

% Tree search has become one of the key paradigms for enabling deep reasoning in large language models. However, the inherent sequential nature of tree structures presents significant challenges for GPU parallelism. Regardless of the search algorithm employed, the hierarchical and distributed nature of tree-structured data introduces intrinsic difficulties in memory management—whether using a contiguous memory pool or fragmented memory blocks, the irregularity of tree search remains unavoidable.  % 这一段说的 tree-structured data 难以 parallel

Tree search inherently exhibits retrospective and recursive behaviors, making efficient parallel execution difficult. Even if each node is constrained to generate the same number of tokens, the focus switching between different reasoning trajectories and the diverse path lengths makes it incompatible with the end-to-end parallelism on GPUs. The detailed illustrations for this phenomenon can be found in Appendix~\ref{sec:app:trees}. 

% This behavior demonstrates the fundamental characteristics of tree search that make the efficient parallelization a challenging problem: \textit{diverse path lengths, varying child depths, irregular node jumps, and recursive retrospective exploitation}. 

The focus switching between paths also makes the tree search fail in focused reasoning trajectory~\cite{wang2025thoughts}, which prevents deep thinking and leads to a tendency of shallow exploitation. We quantify the switch times of the reasoning focus on each sample in the Math500 dataset. Figure~\ref{fig:motivation_switch_path} counts the total switch, which is about 35 on average. As well as the switch from the best path to a suboptimal or incorrect one, which is up to 3 times for a single sample. It demonstrates the instability of the tree search algorithm in maintaining a focused reasoning trajectory. 


% 下面这段我放 appendix 了
\jwt{
% Tree search is key for enabling deep reasoning in large language models, but its sequential nature presents challenges for efficient GPU parallelism. Tree-structured data introduces difficulties in memory management, with irregular patterns in node expansions and varying path lengths. This leads to difficulties in maintaining parallelism, as tree search is inherently recursive and retrospective, causing inefficient execution when different paths vary in depth and termination points.

% Figure~\ref{fig:motivation_dfs_trees} visualizes depth-first search (DFS) trees, highlighting the irregular expansion process. Darker nodes represent high-confidence paths, while lighter nodes indicate lower-confidence ones. The figure demonstrates that tree search does not follow a predictable spatial or hierarchical pattern, resulting in diverse path lengths, varying child depths, and irregular node jumps. These behaviors complicate parallel execution and efficient resource utilization.

% To address these challenges, it is crucial to focus on improving the handling of irregular growth in tree structures. By introducing strategies that can better manage the dynamic nature of tree search, we can optimize memory usage and reduce the inherent complexity, enabling more effective parallelization.

}


% These findings highlight a critical shortcoming of BFS: frequent switching between different reasoning paths prevents deep exploitation, leading to significant computational redundancy. This behavior results in excessive token generation, unnecessary expansions on suboptimal paths, and instability in maintaining a coherent line of reasoning—factors that ultimately hinder search efficiency and inference speed.

% 下面这段我放 appendix 了
% \jwt{
%  Tree search focusing on width expansion explores a wide range of paths but suffers from frequent switching between them, preventing deep reasoning and leading to shallow exploitation. This behavior causes two inefficiencies: incomplete reasoning and excessive expansions. These algorithms often generates more tokens and expansions than necessary, exploring many suboptimal paths before finding the best one, which results in significant computational redundancy.

%  Figure~\ref{fig:motivation_bfs_trees} shows how BFS expands in a flat, top-down manner, leading to shallow exploitation. Figure~\ref{fig:motivation_waste_tokens}(left) compares the total tokens generated (blue line) to those required for the best path (yellow line), revealing excessive token redundancy. Figure~\ref{fig:motivation_waste_tokens}(right) highlights unnecessary node expansions, where many explored nodes do not contribute to the final solution. Finally, Figure~\ref{fig:motivation_switch_path} analyzes node-switching frequency, showing that BFS frequently shifts between optimal and suboptimal paths, leading to instability in reasoning.

%  The high frequency of path-switching can be mitigated by introducing mechanisms that help the model maintain focus on the most promising paths.
% }



\subsection{Redundant Exploration}
\label{sec:3.2}


The lack of early termination in existing tree search algorithms leads to excessive exploitation and redundant searching. Observations in Figure~\ref{fig:motivation_low_confidence} show that low-confidence nodes rarely contribute to the best solutions, either terminated with suboptimal results (yellow) or failing to be the first to reach the best path (orange). The average probability of the suboptimal results brought by low confidence is 91.3\%, while the probability of those nodes being the earliest best path is only 6.2\%. It suggests that low-confidence nodes have little potential to reach the best solution, it is even hard to be the first one. It means that most low-confidence nodes have less contribution to the final results but waste computational resources. 
% We reorder the nodes based on the former probability for clearer illustration (the plot with original order is showcased in Appendix~\ref{app:sec:low_reward_original}). 

% One of the fundamental inefficiencies in traditional tree search lies in its inability to terminate early when exploring suboptimal paths. In most cases, a search path is only abandoned when it reaches the termination condition, regardless of whether it is already apparent that the path is unlikely to yield an optimal solution. This behavior can lead to substantial computational redundancy, as a large number of unnecessary expansions and token generations are performed on low-confidence nodes. 


% To quantify the extent of this redundancy, we conducted an observational experiment. In Figure~\ref{fig:motivation_low_confidence}, we analyze the probability that continuing the rollout from a low-confidence node leads to less contribution. Here, we define a node as low confidence if its score is lower than the average confidence of previously visited nodes (refer to Eq. (\ref{eq:theta}))—a deliberately aggressive threshold since such occurrences are relatively frequent (highlighted in yellow). And we also reorder the nodes based this probability for clearer illustration (the plot with original order is showcased in Appendix~\ref{app:sec:low_reward_original}). However, as our observations indicate, the probability that these low-confidence nodes ultimately contribute to the optimal path is extremely low (highlighted in orange). Additionally, in most cases where the optimal path is eventually reached, it is not the first time an optimal solution is discovered (highlighted in blue), meaning that a higher-confidence node had already identified the correct path earlier. 

% Between the yellow and orange regions lies an additional scenario: the probability that continuing a rollout from a low-confidence node either fails to generate an answer at termination or produces an incorrect answer. This category accounts for the majority of cases, further reinforcing the inefficiency of expanding low-confidence paths.

% Based on these observations, it suggest that a node’s prior confidence may serve as a reasonable predictor of whether the search path will ultimately lead to a valid solution. While this does not guarantee a perfect pruning strategy, it indicates that integrating confidence-based heuristics could significantly reduce unnecessary rollouts, improving the overall efficiency of tree search methods.

% 合并到上面的黑色文本里了
\jwt{
% A major inefficiency in traditional tree search is the lack of early termination when exploring low-confidence paths. This results in wasted computational resources, as the algorithm continues expanding these paths even when their confidence scores remain low. Observations show that low-confidence nodes rarely contribute to optimal solutions, suggesting that confidence-based pruning can reduce unnecessary exploitation and improve efficiency.

% Figure~\ref{fig:motivation_dfs_trees} (Tree 2) illustrates the issue, showing that low-confidence nodes (rightmost branches) are expanded despite their low likelihood of contributing to the final answer. Figure~\ref{fig:motivation_low_confidence} further quantifies this inefficiency: yellow regions indicate frequent occurrences of low-confidence nodes, while orange regions show that they rarely lead to optimal solutions. The blue regions reveal that even when the best path is found, a higher-confidence node had typically identified it earlier, confirming the redundancy of expanding low-confidence nodes.

% These findings emphasize the importance of selectively pruning low-confidence paths early in the search process. By incorporating mechanisms that assess the likelihood of a node contributing to the optimal solution, unnecessary expansions can be avoided, leading to a significant reduction in computational overhead.
}

\begin{figure}[t]
    % \centering
    \includegraphics[width=0.85\linewidth]{figs/low_reward_original.pdf}
    \vspace{-0.1in}
    \caption{Probabilities with reordered samples of those have prior confidence below $\theta_{es}(\lambda=1)$ in Eq.~\ref{eq:theta} and do not terminate with the highest reward score (yellow), and paths that are not the earliest best path (orange), which means there is already at least one path that has terminated with the same reward score. }
    \vspace{-0.2in}
    \label{fig:motivation_low_confidence}
\end{figure}

These findings emphasize the importance of maintaining the focus on deep reasoning and pruning low-confidence paths for efficient inference. 


\section{Proposed Method}
\label{sec:method}
To address the aforementioned challenges, we propose an innovative framework that allows for efficient reasoning, termed Dynamic Parallel Tree Search (DPTS). 
% The proposed parallel framework serves as the foundation of DPTS. Furthermore, the search and the transition mechanism enables efficient and adaptive exploitation. 
In the generation phase, the Parallelism Streamline in Sec.~\ref{sec:method_parallel} supports fine-grained and flexible paralleled expansion for arbitrary paths. 
In the selection phase, the Search and Transition Mechanism in Sec.~\ref{sec:search_and_transition} enables less redundant exploration by identifying the highly potential solutions to focus reasoning. 

% \begin{algorithm}[ht]
% \caption{Algorithmic process DPTS}
% \label{alg:algorithm_parallel}
% \begin{algorithmic}[1]
% \REQUIRE LLM generation function $llm(x)$, PRM reward function $prm(x)$, Query $q$, Candidate Node Pool $N = \varnothing$, Parallel Queue $P = \varnothing$, Exploit Node Proportion $p$, Tree Width $w$. \\
% \ENSURE End node with best path reward $n^*$.

% {\color{ForestGreen}{// Step 1: Initialize the root node}} \\
% \STATE $r \gets \text{generate\_node}(q, \mathrm{None})$
% \STATE $N \gets N \cup \{r\}$

% \WHILE{\text{within computational budget}} 

%     % {\color{ForestGreen}{// Step 2: Adjust Parallel Queue}} \\
%     \STATE $ P_\text{size} \gets \text{Eq. (\ref{eq:queue_size})}$  \\
    
%     {\color{ForestGreen}{// Step 3: Perform searching}} \\
%     \STATE $P \gets \text{Search}(P, P_\text{size}, N)$ (Algorithm~\ref{alg:searching})
%     \STATE $\theta_{\mathrm{es}}, \theta_{\mathrm{ds}} \gets \text{Eq. (\ref{eq:theta})}$ \\
    
%     {\color{ForestGreen}{// Step 4: Parallelism by Eq. (\ref{eq:kv}) and (\ref{eq:seq})}} \\
%     \STATE $\mathbf{n} \gets \text{generate\_node}(
%     \mathrm{Seq}^{all}, \mathrm{KV}^{all})$ \\ 
%     {\color{ForestGreen}{// Step 5: Update new nodes by Eq.~(\ref{eq:n_new})}} \\
%     % \STATE $\mathrm{KV}^{\text{all}}, \mathrm{Seq}^{\text{all}} \gets \text{Eq. (\ref{eq:kv}), Eq. (\ref{eq:seq})}$
%     % \STATE $\mathbf{n} \gets \mathtt{generate\_node}(\mathrm{Seq}^{\text{all}}, \mathrm{KV}^{\text{all}})$
%     {\color{ForestGreen}{// Step 6: Terminate and reward}} \\
%     \STATE $N\gets \text{Reward}(P, N)$ (\text{Algorithm~\ref{app:algo:reward}}) \\
%     {\color{ForestGreen}{// Step 7: Perform transition}} \\
    
%     \STATE $P \gets \text{Transition}(P, \theta_{\mathrm{es}}, \theta_{\mathrm{ds}})$ (\text{Algorithm~\ref{alg:transition}})
% \ENDWHILE

% \RETURN $\max_{\text{reward}}(\forall n \in N)$
% \end{algorithmic}
% \end{algorithm}

\begin{figure*}
    \centering
    \includegraphics[width=\linewidth]{figs/overview.pdf}
    \vspace{-0.35in}
    \caption{Overview of the proposed DPTS framework. The right part demonstrates the Parallelism Streamline, while the left and middle illustrate the proposed Search and Transition Mechanism. }
    \label{fig:overview}
    \vspace{-0.1in}
\end{figure*}

\subsection{Parallelism Streamline}
\label{sec:method_parallel}

% As the foundation of our framework, we fully parallelize the generation process to maximize GPU utilization. Thus, we first introduce the parallel execution architecture, as illustrated in the right side of Figure~\ref{fig:overview}. It consists of three phases: Adaptive Parallelism Queue, which has dynamically adjustable length determined by the available GPU memory; Data Collection and Preparation for parallel inference; Generation and Updating for efficient storage. Additional details can be found in Appendix~\ref{app:sec:parallel_reasoning} due to the limited length.  
% The core component of this architecture are the parallel queue $P$, which holds the currently selected optimal nodes; candidate node pool $N$, which collects all the unexpanded nodes; the data structure of each node. 

% 

% The selection process will be detailed in later sections on searching and transition mechanisms.


% \begin{algorithm}[ht]
% \caption{Process of EETS}
% \label{alg:algorithm_parallel}
% % \resizebox{\linewidth}{!}{
% \begin{algorithmic}[1]
% \REQUIRE Query $q$, candidate node pool $N=\emptyset$, parallel queue $P=\emptyset$, exploit node proportion $p$, tree width $\mathtt{tw}$. \\ 
% \ENSURE  Node with best path $n^*$. \\ 
% % \FOR {$s \gets S$}
%     \STATE $t_0 = \mathtt{time()}$
    
%     {\color{ForestGreen}{// Initializing root node $r$}} 
%     \STATE $r \gets \mathtt{generate\_node}(q, \mathrm{None})$
%     \STATE $N \gets N \cup \{r\}$
    
%     \WHILE{$\mathtt{time()}-t_0<t_\mathrm{{max}}$} 
%     \STATE $\mathtt{queue\_size} \gets \frac{1-O_{init}}{O_{peak} - O_{init}}$  
%     % {\color{ForestGreen}{// Adaptive parallelism number}} 
    
%     % \IF{$|P|<\mathtt{queue\_size}$}
%     % \STATE $N'\gets$ Descending $N$ based on conf. \\
%     % % , $\forall n \in N$
%     % {\color{ForestGreen}{// EE Searching}} \\
%     % \STATE $P \gets P \cup N'[:\mathtt{queue\_size}-|P|]$
%     % \STATE $P'\gets$ Descending $P$ based on conf.
%     % % , $\forall n \in P$
%     % \STATE $n$.mode $\gets \mathtt{EXPLORE}$, $\forall n \in P'[:p|P|]$
%     % \STATE $n$.mode $\gets \mathtt{EXPLOIT}$, $\forall n \in P'[p|P|:]$
%     % \ENDIF
%     {\color{ForestGreen}{// E\&E Searching}} \\
%     \STATE $P\gets$ Algorithm\ref{alg:searching}$(P$, $\mathtt{queue\_size}, N)$ \\
%     % 更新 theta
%     \STATE $\theta_{early\_stop}, \theta_{deep\_seek} \gets $ Eq. (\ref{eq:theta})
    
%     {\color{ForestGreen}{// E\&E Parallelism}} \\
%     % data preparation & generation
%     \STATE $\mathrm{KV}^{all}, \mathrm{Seq}^{all} \gets$ Eq. (\ref{eq:kv}), Eq. (\ref{eq:seq})
%     \STATE $\mathbf{n} \gets \mathtt{generate\_node}(\mathrm{Seq}^{all}, \mathrm{KV}^{all})$
    
%     % updating new node
%     \FORALL{$n \in P$}
%         \STATE $n$.children $\gets$ Eq. (\ref{eq:n_new}) 
%         \FORALL{$m\in n$.children}
%             \IF{$\mathtt{is\_terminate}(m)$}
%                 \STATE $m$.reward$\gets\mathtt{reward}(m)$
%             \ELSE
%                 \STATE $N\gets N\cup \{m\}$ \\ 
%             \ENDIF
%         \ENDFOR
%         % \STATE $a^i \gets P'[i]$
%         % \FOR{$j \gets 1$ to $\mathtt{tw}$}
%         % \STATE $\mathrm{KV}^{n^{ij}}\gets \mathbf{n}.\mathtt{past\_kv}_{[ij, \dots, |\mathrm{KV}^{all}|:]}$
%         % \STATE $\mathrm{Seq}^{1\sim n^{ij}}\gets \mathbf{n}.\mathtt{output}_{[ij]}$
%         % \STATE $n^{ij}\gets[id, a, \mathrm{KV}^{n^{ij}}, \mathrm{Seq}^{1\sim n^{ij}}]$
%         % \ENDFOR
%         % \STATE $a^i$.children$\gets[n^{i1}, \dots, n^{i\mathtt{tw}}]$
%     \ENDFOR
%     % updating N and transition
    
%     {\color{ForestGreen}{// E\&E Transition}} \\
%     \STATE $P \gets$Algorithm\ref{alg:transition}$(P, \theta_{early\_stop}, \theta_{deep\_seek})$
%     % \FORALL{$n \in P$}
%     % \STATE $n^*=\max_\mathrm{conf.}$($n$.children)
%     % \IF{$n$.mode $=$ $\mathtt{EXPLORE}$ \AND $n^*$.conf. $<\theta_{early\_stop}$ \OR $n$.mode $=$ $\mathtt{EXPLOIT}$ \AND $n^*$.conf. $<\theta_{deep\_seek}$}
%     %     % \IF{}
%     %     \STATE $P \gets P \setminus n$
%     %     \ELSE
%     %     \STATE $P \gets P\cup \{n^*\}$
%     %     \STATE $n^*$.mode $\gets \mathtt{EXPLORE}$ 
%     % \ENDIF
%     % \ENDFOR
%     % \FORALL{$n \in P$}
%     % \IF{$\mathtt{is\_terminate}(n)$}
%     % \STATE $P\gets P \setminus n$
%     % \STATE $n$.reward$\gets\mathtt{reward}(n)$
%     % \ENDIF
%     % \ENDFOR
% \ENDWHILE
% \RETURN $\max_\mathrm{reward}(\forall n \in N)$
% \end{algorithmic}
% \end{algorithm}

% \jwt{
As illustrated in Figure~\ref{fig:overview}, We fully parallelize the tree search process in our framework with three main components: \textbf{Tree Structure Building}, \textbf{KV Cache Handling}, and \textbf{Adaptive Parallel Generation}. Each component is designed to optimize memory usage and parallel execution during the reasoning process. 



\subsubsection{Tree Structure Building}
The tree search framework relies on a tree structure where each node represents a reasoning state.  Specifically, the node data structure includes the following elements:
\begin{itemize}[itemsep=0pt,parsep=0pt,left=0pt,topsep=0pt]
    \item \textbf{Node ID}: A unique identifier for each node.
    \item \textbf{Parent Node}: A reference to the parent node, establishing the hierarchical structure of the tree.
    \item \textbf{Prior Confidence}: The confidence of the node, based on prior knowledge and model predictions.
    % \item \textbf{Posterior Reward}:
    \item \textbf{Key-Value Cache} (\(\text{KV}^n\)): The key-value cache specific to this node, storing intermediate results during the reasoning process.
    \item \textbf{Token Sequence} (\(\text{Seq}^{1 \sim n}\)): The complete token sequence from the root node to the current node, representing the reasoning path taken so far.
\end{itemize}
The key challenge lies in managing the KV cache~\cite{floridi2020gpt}. Instead of storing the entire sequences, each node only retains its own KV cache. This significantly reduces memory usage, particularly when dealing with a large number of nodes in the tree. By keeping each node's cache isolated, we avoid redundant memory usage while ensuring that each node has necessary information to continue reasoning process.

\subsubsection{KV Cache Handling}

The KV cache for each node is stored separately, and during parallel execution, these caches need to be collected and concatenated for efficient parallelism. The key challenge is that tree search paths have varying lengths, which means that both the KV caches and the input sequences for different nodes will vary in size and be hard to parallel. 

To address this, we use a simple but straightforward padding technique to ensure that all sequences have consistent lengths before being processed. Specifically, for nodes with shorter KV caches, we apply left padding with zeros. Similarly, input sequences are padded with a predefined padding token to match the longest sequence in the batch. 
This padding ensures that all nodes are processed in parallel with consistent sequence lengths and corresponding KV cache, allowing for efficient batch processing across the tree search. \dyf{The details of padding and concatenating are given in Appendix Eq. (\ref{eq:kv}) and (\ref{eq:seq}).}

Besides data collecting and preparation, we also clean up the useless KV cache either the leaf node is terminated, or all the children's branches are exploited and finished. In this way, we release the memory, making room for new reasoning paths. 

\subsubsection{Adaptive Parallel Generation}
To further utilize the computational resources, we introduce an adaptive parallelism queue, which dynamically adjusts the number of parallel paths based on the available GPU memory. The parallelism queue size, denoted \( |P| \), is used to restrict the number of exploitation and exploration paths in Sec.~\ref{sec:method_searching}. It is calculated by the available and the peak memory usage during previous generations: 
% $|P| = \frac{O_{\text{max}} - O_{\text{init}}}{O_{\text{peak}} - O_{\text{init}}},$ \label{eq:queue_size}
\begin{equation}
\label{eq:queue_size}
|P| = \frac{O_{\text{max}} - O_{\text{init}}}{O_{\text{peak}} - O_{\text{init}}},
\end{equation}
where $O_{\text{max}}$ is the total memory budget, \( O_{\text{peak}} \) represents the peak memory usage from the previous generation, and \( O_{\text{init}} \) is the memory consumption during model initialization. 
\dyf{As the tree grows, the memory occupation of intermediate results continues to increase even with KV cache cleaning. Since memory overflow is one of the termination conditions, it is important to adaptively adjust the parallel number, preventing excessive memory allocation and early termination.} 
% mechanism ensures that the number of parallel paths is adjusted by the available memory resources, preventing excessive memory allocation and early termination of searching process}. 
 

After the generation phase, the newly generated sequences and KV caches are stored based on the tree width. The sequences for each node are completely stored, while the KV caches are stored partially with only the new tokens generated at this step (details are provided in Appendix~\ref{eq:n_new}). 
The new nodes are then added to the candidate node pool \( N \), where they will be available for subsequent selection processes in the tree search.

% \hspace{0pt}
\vspace{0.05in}

In summary, our Parallelism Streamline is a well-structured streamline to optimize both memory usage and parallel execution. 
% By carefully engineering the tree structure, KV cache, and parallel generation processes, we ensure that the reasoning capabilities of LLM are significantly enhanced while maintaining computational efficiency. 
The overall process is showcased in Algorithm~\ref{alg:algorithm_parallel}. More details can be found in Appendix~\ref{app:sec:parallel_reasoning} due to the limited length. 


\begin{algorithm}[ht]
\caption{Algorithmic process DPTS}
\label{alg:algorithm_parallel}
\begin{algorithmic}[1]
\REQUIRE LLM generation function $llm(x)$, PRM reward function $prm(x)$, Query $q$, Candidate Node Pool $N = \varnothing$, Parallel Queue $P = \varnothing$, Exploit Node Proportion $p$, Tree Width $w$. \\
\ENSURE End node with best path reward $n^*$.

{\color{ForestGreen}{// Step 1: Initialize the root node}} \\
\STATE $r \gets \text{generate\_node}(q, \mathrm{None})$
\STATE $N \gets N \cup \{r\}$

\WHILE{\text{within computational budget}} 

    % {\color{ForestGreen}{// Step 2: Adjust Parallel Queue}} \\
    \STATE $ P_\text{size} \gets \text{Eq. (\ref{eq:queue_size})}$  \\
    
    {\color{ForestGreen}{// Step 2: Perform searching}} \\
    \STATE $P \gets \text{Search}(P, P_\text{size}, N)$ (Algorithm~\ref{alg:searching})
    \STATE $\theta_{\mathrm{es}}, \theta_{\mathrm{ds}} \gets \text{Eq. (\ref{eq:theta})}$ \\
    
    {\color{ForestGreen}{// Step 3: Parallelism by Eq. (\ref{eq:kv}) and (\ref{eq:seq})}} \\
    \STATE $\mathbf{n} \gets \text{generate\_node}(
    \mathrm{Seq}^{all}, \mathrm{KV}^{all})$ \\ 
    {\color{ForestGreen}{// Step 4: Update new nodes by Eq.~(\ref{eq:n_new})}} \\
    % \STATE $\mathrm{KV}^{\text{all}}, \mathrm{Seq}^{\text{all}} \gets \text{Eq. (\ref{eq:kv}), Eq. (\ref{eq:seq})}$
    % \STATE $\mathbf{n} \gets \mathtt{generate\_node}(\mathrm{Seq}^{\text{all}}, \mathrm{KV}^{\text{all}})$
    {\color{ForestGreen}{// Step 5: Terminate and reward}} \\
    \STATE $N\gets \text{Reward}(P, N)$ (\text{Algorithm~\ref{app:algo:reward}}) \\
    {\color{ForestGreen}{// Step 6: Perform transition}} \\
    
    \STATE $P \gets \text{Transition}(P, \theta_{\mathrm{es}}, \theta_{\mathrm{ds}})$ (\text{Algorithm~\ref{alg:transition}})
\ENDWHILE

\RETURN $\max_{\text{reward}}(\forall n \in N)$
\end{algorithmic}
\end{algorithm}

% The dynamic adjustment of parallelism, coupled with effective padding strategies and efficient node management, allows for scalable and high-performance inference in complex tree search scenarios.  



% \textbf{Data Structure.} For preliminary, we first organize every thought steps as nodes of ToT. Each node in the parallel reasoning process is represented by a data structure that maintains essential information for efficient tree search. This structure includes the node's unique identifier, a reference to its parent node, prior confidence scores, the complete sequence of tokens from the root to the current node, and the node-specific key-value (KV) cache. To optimize memory usage, we avoid redundant storage of the entire sequence of past KV caches, retaining only the KV cache specific to each node. This ensures that the memory consumption is minimized without sacrificing the ability to track relevant state information.

% \The core of our parallel reasoning approach lies in the dynamic management of the parallelism queue, which plays a central role in balancing memory usage and computational efficiency. The size of this queue is adaptively adjusted based on the available GPU memory, ensuring that the framework can handle varying memory demands throughout the inference process. Specifically, 
% % the queue size is determined by the peak memory usage observed during prior generations, allowing us to dynamically allocate computational resources while avoiding memory overload.

% For data preparation phrase, to handle the variable path lengths in the tree search, input data is prepared by padding both the KV caches and input sequences to achieve uniform dimensions. Since the path lengths can differ significantly between nodes, we perform padding on both the KV caches and input sequences to ensure consistent input dimensions. The padding process involves left-padding the KV caches for shorter paths and right-padding the input sequences, resulting in a unified matrix that can be processed in parallel by the large language model (LLM). This approach allows the model to handle varying sequence lengths efficiently without introducing unnecessary complexity.

% Once the data is prepared, the framework proceeds to the generation and updating stage. In this phase, new output sequences and their corresponding KV caches are generated for each node in parallel. The results are then partitioned based on the tree width, with each partition containing only the newly generated information. This segmentation minimizes the memory overhead by retaining only the relevant KV cache segments associated with the new tokens. The output sequences, on the other hand, are stored in their entirety to avoid fragmentation, which would otherwise introduce unnecessary latency. Newly generated nodes are subsequently updated into the candidate node pool, where they become available for further exploitation in subsequent cycles.

% Together, these components of the parallel reasoning process allow the framework to efficiently leverage computational resources, maintain flexibility in handling diverse paths, and ensure high-speed inference while keeping memory consumption in check.
% }

\subsection{Search and Transition Mechanism}
\label{sec:search_and_transition}
In this section, we introduce the \textbf{Search} and \textbf{Transition} Mechanism in DPTS, which is a hybrid search algorithm that balances exploitation and exploration through separate management and dynamic conversion. 

\subsubsection{Search}
\label{sec:method_searching}

% To achieve a balance between exploitation and exploration, the parallel queue  $P$  is dynamically partitioned: the top  $p|P|$  highest-scoring nodes are designated as exploit nodes, while the remaining  $(1 - p)|P|$  nodes are explore nodes. These nodes are selected from the candidate node pool $N$ based on their confidence, as illustrated in Figure~\ref{fig:overview} (left). 

% \textbf{Exploit nodes updating} follows a dynamic node inheritance mechanism. if the new node confidence above the threshold, the best child of the exploit node can proceed to the next generation. However, if the confidence of new node falls below the threshold, or the number of exploit node less than $p|P|$, additional nodes are selected from the highest-scoring candidates in the pool  $N$  to ensure the total number of exploitation paths. This dynamic adjustment ensures that the exploitation process remains continuous and adaptive to the evolving search space. 

% \textbf{Explore nodes selection} focuses on extensive searching high-confidence solutions and refining reasoning direction. Unlike exploit nodes, explore nodes are not inherited from the previous generation cycle. Instead, before each expansion step, the highest-scoring nodes from pool $N$, excepting the exploit nodes that are already selected, are collected as explore nodes. This mechanism ensures that explore nodes are always updated based on the latest search progress, allowing for efficient and extensive exploration while avoid unnecessary detours into suboptimal regions. 

% \jwt{
The Search Mechanism aims to balance exploration and exploitation by dynamically partitioning the nodes in parallel queue \( P \) into two categories: \textit{explore nodes} and \textit{exploit nodes}. 
% By allocating resources efficiently to both types of nodes, the search process is able to explore new areas of the search space while exploiting high-confidence paths for refinement. 

As illustrated in Figure~\ref{fig:overview} (left), these nodes are selected from the candidate node pool \( N \).
% 设置了一个超参数p,这个超参数根据什么设置的 % 目前是 half-half
The partition ratio $p$ can be manually adjusted according to the task and memory budget. 
%And we simply set to $p=0.5$ for balance. 这句应该放实验里面 % okk
At initialization, the top \( p|P| \) highest-scoring nodes are assigned as \textit{exploitation nodes}, while the remaining \( (1 - p)|P| \) nodes are assigned as \textit{exploration nodes}. While during searching progress, the proportion of the two types of nodes dynamically fluctuates based on the transition mechanism in Sec.~\ref{sec:method_transition}. 
The primary distinction between these two nodes lies in their origins and roles during the search process.

\paragraph{Exploitation Nodes} 
 The exploitation nodes are primarily inherited from parent exploitation nodes, focusing on refining the most promising paths in the search space. When a child node’s confidence exceeds a predefined threshold, it inherits the status of its parent exploitation node and continues that path. This inheritance ensures that the most promising paths are deepened and further refined. Additionally, when the number of exploitation nodes falls below a predefined threshold, new high-confidence candidate nodes from the pool \( N \) are selected to fill the gap, ensuring that the number of exploitation nodes remains adequate for the search process. This strategy enables the exploitation of high-potential paths while maintaining the focus on areas with high confidence.
 % These nodes are selected before each expansion step from the candidate node pool \( N \), excluding those already designated as exploitation nodes. Exploitation nodes are not inherited from the previous generation; instead, they are always chosen based on the latest progress in the search space. This approach ensures that exploitation targets the most confident solutions while avoiding redundant exploration of paths that are unlikely to lead to better results.

\paragraph{Exploration Nodes} 
In contrast to exploitation nodes, the exploration nodes are not inherited from previous nodes but are dynamically selected from the candidate nodes pool. These nodes are responsible for discovering new paths that may have high potential but low current confidence in the search space. At each reasoning step, the exploration nodes are reselected from the candidate pool \( N \), choosing the highest-confidence nodes that are not already assigned as exploitation nodes. The dynamic re-selection of exploration nodes allows the search process to adapt to changing circumstances and uncover new regions of the search space that may lead to better solutions.

% These nodes are dynamically updated through an inheritance mechanism, where a new child node is inherited by an exploration node only if its confidence exceeds a certain threshold. If the confidence of the new child node falls below the threshold, or if the number of explore nodes drops below \( p|P| \), additional nodes are selected from the pool \( N \) to maintain the desired number of explore nodes. This dynamic selection ensures that exploitation  remains adaptive to the evolving search space, allowing the algorithm to discover new, potentially fruitful paths as it progresses.


% }


% \begin{algorithm}[ht]
%     \caption{Searching}
% \label{alg:searching}
% \begin{algorithmic}[1]
% \REQUIRE Parallel queue $P$, parallel number $\mathtt{queue\_size}$, candidate node pool $N$. 
% \ENSURE Updated $P$. 
% \IF{$|P|<\mathtt{queue\_size}$}
%     \STATE $N'\gets$ Descending $N$ based on conf. \\
%     % , $\forall n \in N$
%     \STATE $P \gets P \cup N'[:\mathtt{queue\_size}-|P|]$
%     \STATE $P'\gets$ Descending $P$ based on conf.
%     % , $\forall n \in P$
%     \STATE $n$.mode $\gets \mathtt{EXPLOIT}$, $\forall n \in P'[:p|P|]$
%     \STATE $n$.mode $\gets \mathtt{EXPLORE}$, $\forall n \in P'[p|P|:]$
% \ENDIF
% \RETURN $P$
% \end{algorithmic}
% \end{algorithm}

\subsubsection{Transition}
\label{sec:method_transition}

% While partitioning parallel nodes into exploit and explore helps mitigate redundant exploitation and shallow expansion issues, it does not entirely eliminate inefficiencies. One key limitation is that the initially selected exploit nodes are not guaranteed to be on the optimal path. Even when an exploit node is later found to be suboptimal, it continues expanding until reach the termination condition (timeout or out of memory), leading to unnecessary resource consumption. Additionally, before an exploit node terminates, newly identified high-confidence nodes cannot be deeply exploit since they are still classified as explore nodes. These weaken the flexibility and limit the efficiency of parallel reasoning. 

% To alleviate this issue, we introduce a bidirectional transition mechanism in the DPTS framework. As illustrated in Figure~\ref{fig:overview} (middle), it enables dynamic transition between exploit and explore nodes, enhancing flexibility and adaptability. It is a two-way transition: \textit{Early Stop (Exploit → Explore)} and \textit{Deep Seek (Explore → Exploit)}. 
% % These two strategies allow the search process to adaptively reallocate computational resources, ensuring efficient reasoning while maintaining a balance between exploitation and exploration.

% In the \textbf{Early Stop transition}, a threshold $\theta_{{early\_stop}}$ is introduced to dynamically evict low-potential exploit nodes. Specifically, we evaluate the best child node after each expansion. If the confidence is lower than $\theta_{{early\_stop}}$, none of the child would be added in the queue $P$ for the next cycle, preventing further redundant exploitation. Conversely, if the best child’s confidence exceeds $\theta_{{es}}$, it inherits the parent’s status and continues to exploit in the next cycle. As discussed in Sec.~\ref{sec:motivation}, the prior confidence is a reasonable indicator for forecasting the potential of the path. The threshold $\theta_{{es}}$ is empirically set as
% \begin{equation}
% \label{eq:theta}
% \theta_{{es}}  = \begin{cases}
% \lambda_1\frac{\sum^{N_\mathrm{Exp.}}_{n}n.\mathrm{conf.}}{|N_\mathrm{Exp.}|}, \text{if $t\le t^*$} \\ 
% \max(\forall_{n\in N_\mathrm{Exp.}} n.\mathrm{conf.}), \text{otherwise}
% \end{cases}
% \end{equation}
% where $N_\mathrm{Exp.}$ means all previously selected and expanded nodes, $\lambda_1$ is a coefficient, $t$ is the number of current terminated paths and $t^*$ is a predefined threshold to change the $\theta_{{early\_stop}}$. 
% We provide experimental evidence in Appendix~\ref{app:best_path_index} to show that early-terminated paths are more likely to be the optimal solution. 

% In contrast, the \textbf{Deep Seek transition} allows high-confidence explore nodes to convert into exploit nodes, enabling promising nodes to dig deeper. Specifically, explore node with a confidence exceeding the threshold $\theta_{{deep\_seek}}$ is promoted to an exploit node, where $\theta_{{deep\_seek}}$ has the similar formula as $\theta_{early\_stop}$ with $\lambda_{2}$. 
% % which is empirically set as $\theta_{{deep\_seek}}=\theta_{{early\_stop}}$ and we consistently obtain good results. 
% Using this mechanism, the number of exploit nodes may temporarily exceed $p|P|$. Fortunately, the increasing number of high-confidence nodes naturally raises $\theta_{{early\_stop}}$, which in turn a larger ratio of exploit nodes meet the Early Stop condition. This creates an adaptive system where the two types of nodes are dynamically balanced throughout the searching process. 

% \jwt{
While the Search Mechanism ensures a balance between exploration and exploitation, the redundant issue is not entirely mitigated. 
% there are cases where inefficiencies arise. 
One example is that the initial exploitation nodes are not guaranteed to be the optimal solution. However, they only stop exploiting till reach the termination condition. 
% may fail to lead to optimal paths.
Another issue occurs when high-confidence nodes are assigned as exploration nodes, but they will wait for the computation resources and do not roll out till the previous paths terminate. 
% which may waste resources because of over-exploring through already existing promising paths. 

To address these issues, we introduce the Transition Mechanism, which consists of two main strategies: \textit{Early Stop} (Exploitation → Exploration) and \textit{Deep Seek} (Exploration → Exploitation). As illustrated in Figure~\ref{fig:overview} (middle), these strategies allow an evolving search space with node transits between the two statuses. 
It helps the tree maintain focused reasoning, ensuring the efficient allocation and utilization of limited computational resources throughout the whole search process. 
% This flexibility enhances the search process by reallocating computational resources as needed to maintain an optimal balance between exploration and exploitation.

\paragraph{Early Stop (Exploitation → Exploration)} 
The Early Stop~\cite{yao2007early} strategy allows relatively low-confidence exploitation nodes to transition into explore nodes, eliminating redundant exploitation on suboptimal paths. 
During the expansion process, if the best child node of an explore node has a confidence lower than a certain threshold \( \theta_{\mathrm{es}} \), the child node will be excluded from the queue \( P \) in the next cycle. This prevents further exploration of paths that are unlikely to lead to optimal solutions, saving computational resources. Conversely, if the child node’s confidence exceeds \( \theta_{\mathrm{es}} \), it inherits the parent’s status and continues to explore in the next cycle. This mechanism ensures that only the most promising explore nodes continue to expand, optimizing both exploration and resource usage.
The threshold \( \theta_{\mathrm{es}} \) is defined as follows:
\begin{equation}
\label{eq:theta}
\mathsf{\theta}_{\mathrm{es}} = \begin{cases}
\lambda_\mathrm{es} \, \frac{1}{|\mathcal{N}|} \sum\limits_{i \in \mathcal{N}} c_i, & \text{if } t \leq t^* \\
\underset{i \in \mathcal{N}}{\max} \, c_i, & \text{otherwise}
\end{cases}
\end{equation}
where \( \mathcal{N} \) is the set of previously expanded nodes, \( c_i \) represents the confidence of node \( i \), \( \lambda_\mathrm{es} \) is a coefficient that adjusts the threshold, \( t \) is the number of currently terminated paths, and \( t^* \) is a predefined threshold after which \( \theta_{\mathrm{es}} \) is adjusted.


\paragraph{Deep Seek (Exploration → Exploitation)} 
The Deep Seek strategy addresses the issue of inefficient over-exploration and shallow thinking, ensuring promising exploration nodes can be dug deeper. Specifically, exploration nodes with confidence exceeding a threshold \( \theta_{\mathrm{ds}} \) with $\lambda_\mathrm{ds}$ are promoted to exploitation nodes. 
% This threshold is typically set to \( \theta_{\mathrm{ds}} = \theta_{\mathrm{es}} \), according to empirical results. % 这个实验里也打算加一下不等的,因为我想了想感觉等号不一定理论上合理。。
As a result, the number of exploitation nodes may temporarily exceed \( p|P| \). But as more high-confidence nodes are promoted, \( \theta_{\mathrm{es}} \) increases, and thus more exploration nodes are stopped under the Early Stop strategy. This creates a dynamic balance between exploration and exploitation throughout the search process. 

% \hspace{0pt}
\vspace{0.07in}

In a word, the proposed Search and Transition Mechanism in DPTS effectively manages the trade-off between exploitation and exploration with dynamic and bidirectional transition. 
% By dynamically partitioning the parallel queue and  the transition strategies, our approach adapts to the evolving search space and optimizes the utilization of computational resources. It enables the model to focus on the most promising branches while avoiding redundant or suboptimal expansions. 
Detailed algorithms in this part can be found in Appendix~\ref{app:sec:parallel_reasoning}.  

% \begin{algorithm}[ht]
%     \caption{Transition}
% \label{alg:transition}
% \begin{algorithmic}[1]
% \REQUIRE Parallel queue $P$, transition thresholds $\theta_{early\_stop}$ and $\theta_{deep\_seek}$. 
% \ENSURE Updated $P$. 
% \FORALL{$n \in P$}
%     \STATE $n^*=\max_\mathrm{conf.}$($n$.children)
%     \IF{$n$.mode $=$ $\mathtt{EXPLOIT}$ \AND $n^*$.conf. $<\theta_{early\_stop}$ \OR $n$.mode $=$ $\mathtt{EXPLORE}$ \AND $n^*$.conf. $<\theta_{deep\_seek}$}
%         \STATE $P \gets P \setminus n$
%     \ELSE
%         \STATE $P \gets P\cup \{n^*\}$
%         \STATE $n^*$.mode $\gets \mathtt{EXPLOIT}$ 
%     \ENDIF
% \ENDFOR
% \RETURN $P$
% \end{algorithmic}
% \end{algorithm}



\section{Experiments}
\label{sec:exp}
We conduct extensive experiments to evaluate the efficiency of the DPTS framework. We benchmark its performance against various search algorithms across multiple models and datasets to ensure a comprehensive analysis.


\subsection{Settings}
\label{sec:exp_setting}

\paragraph{Models.} We include Qwen-2.5-1.5B-Instruct, Qwen-2.5-7B-Instruct~\cite{yang2024qwen2}, LLaMA-3.1-8B-Instruct, and LLaMA-3.2-3B-Instruct~\cite{touvron2023llama} to cover various model sizes.  

\paragraph{Datasets.} The evaluation datasets include Math500~\cite{hendrycks2021measuring} and GSM8K~\cite{cobbe2021training}, both are widely used for reasoning and mathematical problem-solving tasks. We implement the evaluation by referring the approaches used in Qwen-2.5-Math~\cite{yang2024qwen2}, supplemented in our submission. 

\paragraph{Comparison Methods.} We compare DPTS against three widely used search algorithms: Monte Carlo Tree Search~(MCTS)~\cite{sprueill2023monte}
% balances exploration and exploitation when sampling, while the selected paths rollout till termination
, Best-of-N~\cite{cobbe2021training}
% performs multiple independent rollouts and selects the highest-scoring output
, Beam Search~\cite{Yao_2023_Tree}.
% expands multiple hypotheses in parallel, pruning low-scoring candidates at each step to maintain a fixed-width search~\cite{snell2024scaling}. 
Since efficient tree search algorithms have recently regained attention after the emergence of LLM reasoning, the strong baselines are limited. As a result, we primarily compare DPTS against these typical and well-established search algorithms to demonstrate the effectiveness of our method. An introduction of the comparison methods and other details about experimental settings are provided in Appendix~\ref{app:sec:exp}.

% 实验参数: tree_width:4 tree_depth:16 mini_step:128 % 其实 103 确实实验结果更好不知道为什么 otz
% max token:2048 mcts_max_time:120 

\subsection{Comparisons on Search Algorithms}


% \begin{table}
%     \centering
%     \caption{Comparisons across existing search algorithms on LLM reasoning tasks. }
%     \label{tab:comparisons}
%     \resizebox{\linewidth}{!}{
%     \begin{tabular}{cllrlr}
%     \toprule
%     \multirow{2}{*}{\textbf{Model}}  & \multirow{2}{*}{\textbf{Algo.}}  & \multicolumn{2}{c}{\textbf{Math500}} & \multicolumn{2}{c}{\textbf{GSM8K}}  \\ 
%     & &  \textbf{Acc.} & \textbf{Time (s)} & \textbf{Acc.} & \textbf{Time (s)}  \\ \midrule
%     \multirow{4}{*}{\begin{tabular}{c} Qwen2.5\\1.5B\end{tabular}} 
%     & MCTS & 56.6 & 117.37 & 75.1 & 73.28 \\ 
%     ~ & Best-of-N & 52.6 & 89.87 & 70.1 & 33.37  \\ 
%     ~ & Beam & 52.4 & 104.58 & 71.5 & 41.27 \\ 
%     ~ & \cellcolor[gray]{0.9}{DPTS} & \cellcolor[gray]{0.9}{59.2} & \cellcolor[gray]{0.9}{\textbf{45.10}} & \cellcolor[gray]{0.9}{75.2} & \cellcolor[gray]{0.9}{\textbf{18.32}}  \\ \midrule
    
%     \multirow{4}{*}{\begin{tabular}{c} Qwen2.5\\7B\end{tabular}}  
%     & MCTS & 75.2 & 121.46 & 89.6 & 79.68 \\ 
%     ~ & Best-of-N & 71.6 & 91.29 & 88.2 & 34.89 \\ 
%     ~ & Beam & 72.4 & 106.89 & 86.7 & 36.49 \\ 
%     ~ & \cellcolor[gray]{0.9}{DPTS} & \cellcolor[gray]{0.9}{76.2} & \cellcolor[gray]{0.9}{\textbf{53.50}} & \cellcolor[gray]{0.9}{89.4} & \cellcolor[gray]{0.9}{\textbf{19.95}} \\ \midrule
    
%     \multirow{4}{*}{\begin{tabular}{c} Llama3\\3B\end{tabular}}  
%     & MCTS & 48.6 & 111.80 & 64.0 & 57.19  \\ 
%     ~ & Best-of-N & 46.4 & 91.34 & 57.1 & 27.27 \\ 
%     ~ & Beam & 45.2 & 104.36 & 58.4 & 28.27 \\ 
%     ~ & \cellcolor[gray]{0.9}{DPTS} & \cellcolor[gray]{0.9}{50.8} & \cellcolor[gray]{0.9}{\textbf{47.75}} & \cellcolor[gray]{0.9}{67.8} & \cellcolor[gray]{0.9}{\textbf{27.74}} \\ \midrule
    
%     \multirow{4}{*}{\begin{tabular}{c} Llama3\\8B\end{tabular}}  
%     & MCTS & 54.2 & 143.36 & 69.5 & 69.74 \\ 
%     ~ & Best-of-N & 49.8 & 122.63 & 67.6 & 33.48  \\ 
%     ~ & Beam & 49.6 & 142.21 & 68.3 & 34.51 \\ 
%     ~ & \cellcolor[gray]{0.9}{DPTS} & \cellcolor[gray]{0.9}{55.4} & \cellcolor[gray]{0.9}{\textbf{37.98}} & \cellcolor[gray]{0.9}{68.2} & \cellcolor[gray]{0.9}{\textbf{17.82}} \\ \bottomrule
%     \end{tabular}
% }
% \end{table}

\begin{table}[ht]
    \centering
    \footnotesize
    \caption{Comparisons across existing search algorithms on LLM reasoning tasks.}
    \vspace{-0.1in}
    \label{tab:comparisons}
    \begin{tabular}{@{}c@{\hskip 4pt} l@{\hskip 6pt} c@{\hskip 6pt} r c@{\hskip 6pt} r}
    % \begin{tabular}{@{}c@{\hskip 4pt} l@{\hskip 6pt} c@{\hskip 6pt} r r@{\hskip 6pt} r@{}}
        \toprule
        \multirow{2}{*}{\textbf{Model}} & \multirow{2}{*}{\textbf{Algo.}} & \multicolumn{2}{c}{\textbf{Math500}} & \multicolumn{2}{c}{\textbf{GSM8K}} \\
        & & \textbf{Acc.} & \textbf{Time (s)} & \textbf{Acc.} & \textbf{Time (s)} \\
        \midrule
        \multirow{4}{*}{\begin{tabular}{c} Qwen-2.5 \\ 1.5B \end{tabular}} & MCTS & 56.6 & 117.37 & 75.1 & 73.28 \\
        & Best-of-N & 52.6 & 89.87 & 70.1 & 33.37 \\
        & Beam & 52.4 & 104.58 & 71.5 & 41.27 \\
        & \cellcolor[HTML]{D3D3D3} \textbf{DPTS} & \cellcolor[HTML]{D3D3D3} 59.2 & \cellcolor[HTML]{D3D3D3} \textbf{45.10} & \cellcolor[HTML]{D3D3D3} 75.2 & \cellcolor[HTML]{D3D3D3} \textbf{18.32} \\
        \midrule
        \multirow{4}{*}{\begin{tabular}{c} Qwen-2.5 \\ 7B \end{tabular}} & MCTS & 75.2 & 121.46 & 89.6 & 79.68 \\
        & Best-of-N & 71.6 & 91.29 & 88.2 & 34.89 \\
        & Beam & 72.4 & 106.89 & 86.7 & 36.49 \\
        & \cellcolor[HTML]{D3D3D3} \textbf{DPTS} & \cellcolor[HTML]{D3D3D3} 76.2 & \cellcolor[HTML]{D3D3D3} \textbf{53.50} & \cellcolor[HTML]{D3D3D3} 89.4 & \cellcolor[HTML]{D3D3D3} \textbf{19.95} \\
        \midrule
        \multirow{4}{*}{\begin{tabular}{c} Llama-3 \\ 3B \end{tabular}} & MCTS & 48.6 & 111.80 & 64.0 & 57.19 \\
        & Best-of-N & 46.4 & 91.34 & 57.1 & 27.27 \\
        & Beam & 45.2 & 104.36 & 58.4 & 28.27 \\
        & \cellcolor[HTML]{D3D3D3} \textbf{DPTS} & \cellcolor[HTML]{D3D3D3} 50.8 & \cellcolor[HTML]{D3D3D3} \textbf{47.75} & \cellcolor[HTML]{D3D3D3} 67.8 & \cellcolor[HTML]{D3D3D3} \textbf{27.74} \\
        \midrule
        \multirow{4}{*}{\begin{tabular}{c} Llama-3 \\ 8B \end{tabular}} & MCTS & 54.2 & 143.36 & 69.5 & 69.74 \\
        & Best-of-N & 49.8 & 122.63 & 67.6 & 33.48 \\
        & Beam & 49.6 & 142.21 & 68.3 & 34.51 \\
        & \cellcolor[HTML]{D3D3D3} \textbf{DPTS} & \cellcolor[HTML]{D3D3D3} 55.4 & \cellcolor[HTML]{D3D3D3} \textbf{37.98} & \cellcolor[HTML]{D3D3D3} 68.2 & \cellcolor[HTML]{D3D3D3} \textbf{17.82} \\
        \bottomrule
    \end{tabular}
    \vspace{-0.2in}
\end{table}

% \vspace{-20pt}

We conduct a comprehensive comparison across different search algorithms on various models and sizes. We emphasize the search efficiency of our method while maintaining accuracy. 

For efficiency, results in Table~\ref{tab:comparisons} show that DPTS significantly reduces inference time compared to other search methods across various models and datasets, demonstrating superior efficiency. 
On Math500, DPTS achieves the lowest inference time across all models. Particularly, in Qwen-2.5, DPTS reduces the search time from 117.37s (MCTS) to 45.10s in the 1.5B model, achieving nearly a $2.6\times$ speedup, and reduces from 121.46s (MCTS) to 53.50s in 7B model, accelerating nearly $2.2\times$. 
The impact is even more pronounced on the GSM8K, where DPTS achieves a $3.9\times$ speedup from 79.68s to 19.95s in Qwen-2.5-7B. And DPTS even only requires 17.82s for each sample using Llama-3-8B, also $3.9\times$. It forcefully suggests that DPTS effectively mitigates redundant rollouts and optimizes search efficiency. 
% Moreover, while Best-of-N and Beam search show competitive efficiency, they often sacrifice accuracy, whereas DPTS balances both accuracy and computational effectiveness. 
We highlight that, especially on the more challenging tasks, the Early Stop plays a crucial role. Without it, trees often run till timeout on Math500, significantly increasing inference time. In contrast, our approach allows the search tree to terminate earlier within a limited number of expansions, effectively reducing computation time. 
% On simpler datasets, termination primarily occurs when all paths output an end token, making early stop triggers less frequent. However, our method still maintains a clear efficiency advantage overall, demonstrating superior computational efficiency across different levels of task complexity.

For accuracy, DPTS maintains the searching quality and even outperforms the existing algorithms with half or even less of the reasoning time. 
On the Math500 dataset, DPTS achieves the highest accuracy in all experiment cases, surpassing MCTS, Best-of-N, and Beam search. Notably, for Qwen-2.5-1.5B, DPTS improves accuracy from 56.6\% (MCTS) to 59.2\%. 
A similar trend is observed on GSM8K, where DPTS either matches or slightly improves accuracy over MCTS, and surpasses Best-of-N and Beam Search by a wide margin. On Llama-3-3B, DPTS has 67.8\% accuracy, outperforming the previous best MCTS by 3.8\% with only 48.5\% time consumption. 
These results highlight that DPTS maintains or even enhances solution quality while significantly improving inference speed, making it a more robust and efficient search algorithm for complex reasoning tasks. 




\begin{table}
    \centering
    \caption{Ablation study of each component in DPTS framework. ``AP'': adaptive parallelism. ``S'': Searching. ``T'': Transition. ``Best Index'': The average index of terminated path leads to the best solution.  }
    \vspace{-0.1in}
    \label{tab:ablation}
    \resizebox{0.95\linewidth}{!}{
    \begin{tabular}{lccrc}
    \toprule
        \textbf{Algo.} & \textbf{${|P|}$} & \textbf{Acc.}  & \textbf{Time (s)} & \textbf{Best Index} \\ \midrule
         Baseline & 1 & 56.6 & 117.37 & 10.45 \\ \midrule
         % Baseline & 4 & 59.8 & 105.61 & 7.79 \\ %  数据不是很好解释,acc 太高了,虽然推理速度慢
         Baseline & AP & 58.8 & 108.06 & 8.27 \\ 
         + S & AP & 58.2 & 76.81 & 4.66 \\ 
         + T & AP & 57.0 & 32.22 & 2.51 \\ 
         + S + T & AP & 59.2 & 45.10 & 4.17 \\ 
     \bottomrule
    \end{tabular}
    }
    \vspace{-0.2in}
\end{table}


\subsection{Ablation Study}

To analyze the contribution of each component within the DPTS framework, we conduct an ablation study on Qwen-2.5-1.5B with Math500 in Table~\ref{tab:ablation}. In this study, we use the classical MCTS as the baseline and incrementally integrate our proposed techniques to evaluate their impact. 

We begin with the original MCTS (non-parallel, $|P|=1$) as the baseline. It spends the most time per sample and has the largest best index 10.45. 
% , which means that after 10.45 terminated paths, it finds the best solution. 
% P=4 这行数据不是很好,这段先不加了
% Simply increasing $|P|$ does not reduce inference time, due to the severer redundant exploitation when conducting parallelism. This occurs because of two reasons. One is the waiting for all paths to terminate. Without supplementary nodes (such as our Deep Seek: explore → exploit), rollout latency is determined by the longest path, which diminishes efficiency. The other is the larger possibility of being terminated by memory overflow. The excessive computation and KV cache storage on suboptimal rollouts may aggravate the memory occupation. As a result, finalize the search process before it reaches the optimal path due to memory overflow. 
% When the finalization condition is set to a timeout or memory overflow, accuracy may degrade due to excessive computation and KV cache storage on suboptimal rollouts, restricting the model’s ability to store and exploit promising paths. 
We then apply Parallelism Streamline with Adaptive Parallel Generation (AP), and accuracy improves. It shows that the trees are able to grow faster and larger to include a better solution with parallelism. 
% the model allocates available GPU resources more effectively. 

Next, we assign the node status as exploit or explore nodes for each expansion with Search Mechanism, denoted as ``+S | AP'' in Table~\ref{tab:ablation}. 
% By incorporating a depth and breadth balanced strategy, 
The search process becomes significantly more structured and targeted, leading to a boost in efficiency. The time of each sample saves by 31.24s (28.9\%$\downarrow$). It finds the best path within an average of 4.66 terminated paths, much fewer than Baseline AP. 
% demonstrating the impact of the balanced exploitation and exploration. 

Moreover, when only applying the Early Stop strategy in the Transition Mechanism (denoted as ``+T | AP), we obtain fast inference with much less time and paths. However, since we only use the exploitation nodes without exploring the possible branches, the accuracy is relatively low. Therefore, we claim that the Search and Transition Mechanism should be used as a whole: the Search mechanism provides different node statuses for exploitation and exploration, while the Transition mechanism makes them flexibly change and update. 

Finally, we combine our Search and Transition Mechanism (denoted as ``+S+T | AP''), enabling Early Stop and Deep Seek. It shows the best search results in accuracy and efficiency. 
% with only 54.32s per sample, and 4.17 paths to reach the optimal solution on average. 
It demonstrates that DPTS is efficient in quickly identifying optimal solutions and conducting deep reasoning. 

Results show that each component of DPTS contributes significantly to improving inference speed and reasoning accuracy, making it a robust and scalable framework for parallel tree search. 




\subsection{Hyperparameter Analysis}

\begin{table}
    \centering
    \caption{Hyperparameter $\lambda_\mathrm{es}$ and $\lambda_\mathrm{ds}$ in transition thresholds $\theta_\mathrm{es}$ and $\theta_\mathrm{ds}$. ``ES (Early Stop)~\%'' and ``DS (Deep Seek)~\%'' are the ratios of the node type transition between the exploitation and exploration. More results can be found in Appendix~\ref{sec:app:hyperparameter}. }
    \vspace{-0.05in}
    \label{tab:hyperparameter}
    \resizebox{\linewidth}{!}{
    \begin{tabular}{cccccc}
    \toprule
        \textbf{$\lambda_\mathrm{es}$} & \textbf{$\lambda_\mathrm{ds}$} & \textbf{Acc.} & \textbf{Time (s)} & \textbf{ES~(\%)}  & \textbf{DS~(\%)} \\ \midrule
        1.0 & 1.0 & 53.0 & 47.59 & 41.4 & 10.5 \\ 
        % 0.95 & 0.95 & 57.8 & 43.99 & 23.4 & 18.2 \\ 
        0.9 & 0.9  & 58.6 & 43.30 & 15.6 & 20.9 \\
        % 0.85 & 0.85 & 58.4 & 42.60 & 9.67 & 23.7 \\
        0.8 & 0.8 & 58.0 & 46.33 & 8.1  & 23.9 \\
        % 0.7 & 0.8 & & \\
        % 0.7 & 0.7 & 58.4 & 41.58 & 5.3 & 27.5 \\
        0.6 & 0.6 & 57.4 & 44.39 & 6.1 & 32.3   \\
        0.4 & 0.4 & 56.6 & 38.41 & 0 & 0.1 \\
        % 0.2 & 0.2 & 57.0 & 40.71 & 0 & 0.1  \\
        % 0   & 0   & 54.4 & 42.78 & 0 & 0.1  \\
        \bottomrule
    \end{tabular}
    }
    \vspace{-0.12in}
    % \vspace{5pt}
\end{table}

We conduct a hyperparameter study in Table \ref{tab:hyperparameter} on the thresholds $\theta_\mathrm{es}$ and $\theta_\mathrm{ds}$ in the Transition mechanism. 
When $t < t^*$, the threshold $\theta$ follows the mean-based strategy determined by $\lambda$. When $t \geq t^*$, it turns to a max-based one. 
Empirically, we set  $t^* = 5$  to balance the flexibility and efficiency. 

Experimental results demonstrate that our method is robust to $\lambda$. We try different $\lambda$ and report the average ES/DS ratios per sample. 
We highlight that $\lambda_\mathrm{es}$ and $\lambda_\mathrm{ds}$ can be set differently based on the specific task. But DPTS consistently works well when $\lambda\in[0.6, 0.8]$. It demonstrates that the Transition mechanism is effective in mitigating the redundancy issue during search progress. 
% regardless of the transition ratio setting. 
However, it should be noticed that, if $\lambda$ is large, the ES transition may be aggressive, which leads to unsatisfactory results (e.g. $\lambda=1.0$). Meanwhile, if $\lambda$ is too small, it degrades to all exploitation nodes, resulting in low efficiency as well. 

\begin{figure}
    \centering
    \includegraphics[width=0.95\linewidth]{figs/singlecol_ee_proportion.pdf}
    \vspace{-0.1in}
    \caption{Proportions of exploit and explore nodes throughout the search process. }
    \label{fig:ee_proportion}
    \vspace{-0.12in}
\end{figure}

\subsection{Visualizations}

To provide an intuitive understanding of the effectiveness of our proposed method, we present visualizations of searching trajectories. % sample tree growth during the search process. 

First, we analyze the dynamic changes in the number of exploitation and exploration nodes throughout the search process in Figure~\ref{fig:ee_proportion}. The Deep Seek transition temporarily increases the proportion of exploit paths, allowing promising nodes to receive deeper reasoning. However, as the threshold $\theta_\mathrm{es}$ increases, exploit nodes are more likely to reach the threshold and stop. As a result, the number of exploit nodes naturally decreases, reinforcing a balance between exploitation and exploration. This dynamic adaptation ensures that DPTS stretches on the most promising branches under the constraint of computational resources. 

\begin{figure}
    \centering
    \includegraphics[width=\linewidth]{figs/visualization_dpts.pdf}
    \vspace{-0.2in}
    \caption{Visualization of DPTS Tree. The green boxes are early stopped nodes based on their prior confidence using our \textit{Early Stop} mechanism, and the purple boxes are the terminated nodes with posterior reward scores. }
    \vspace{-0.12in}
    \label{fig:dpts_tree}
\end{figure}

Second, we show the trees generated by DPTS and analyze the search behavior in Figure~\ref{fig:dpts_tree}. It does not continue exploitation on low-confidence nodes, effectively pruning unpromising branches after shallow exploration. 
Additionally, the trees are capable of stable reasoning focus, with deep exploitation on promising paths. 
Therefore, the generated trees exhibit a relatively narrow width, as DPTS primarily expands nodes that are more relevant to the optimal path and spend less time on unnecessary regions. 
It demonstrates that DPTS 
% efficiently 
% focuses the thinking trajectory, 
ensures high-potential paths receive deeper thinking within a limited time and memory budget. 


\section{Limitations and Conclusion}
This work has a few limitations. To start, we focused our search on GenAI-enabled work practices performed in the HCI community. For this purspose, we limited ourselves to the ACM digital library. As more work emerges around how GenAI is being used, looking at broader research communities will help to tell a more comprehensive story. Further, the papers that we found relevant to our research objective were mostly qualitative. While this was appropriate to the nature of our question, quantitative survey studies can complement our narratives that we identified.

Finally, although GenAI tools are becoming accessible in fields beyond technology, the reviewed studies predominantly focused on technology-related occupations, highlighting a critical need for HCI studies to examine GenAI's impact across a broader range of professions.

In summary, this paper analyzed 23 papers to understand how GenAI is being used by practitioners to craft their jobs. We found that practitioners used GenAI to transform targeted aspects of the tasks they were performing, as well as to shape their roles and relationships. Based on our findings, we discussed how bottom-up usage of these tools was changing roles in unconventional ways, shifting task demand from high-level abstract thinking to more routine tasks, and facilitating the decomposition of roles into piecework. 
%We also suggest a need to expand the job crafting framework to consider ways in which practitioners craft the technology they use to transform their work experiences.



\newpage 
\section*{Limitations}

% Since December 2023, a "Limitations" section has been required for all papers submitted to ACL Rolling Review (ARR). This section should be placed at the end of the paper, before the references. The "Limitations" section (along with, optionally, a section for ethical considerations) may be up to one page and will not count toward the final page limit. Note that these files may be used by venues that do not rely on ARR so it is recommended to verify the requirement of a "Limitations" section and other criteria with the venue in question.

% 和 ethical considerations 一块儿合计最多一页,且不算在最终页数里


% 并行路径的共享前缀可以通过 DEFT 等算法进一步减少数据搬运操作来减少 latency % 可以和其他reasoning方法结合

% 局限于 reasoning 任务,并没有在其他任务上做验证  % t* 的设置比较经验性,可能扩展到其他任务的时候会不适用。

% 我们认为这种方法可以用于online training, 通过提升生成质量来提升policy improvement的效果,但由于设备受限而没有实验证明。

Our DPTS framework focuses on selecting and refining the search paths, but does not involve hardware design. Therefore, it is orthogonal to low-level methods. For example, we can integrate DEFT~\cite{yao2024deft} to reduce the data transportation of the shared prefixes, leading to further acceleration. 
Also, our method is validated only on math reasoning tasks, and has not been tested on other domains, such as coding or scientific problems. However, we believe its generalizable capabilities make it applicable across a wide range of fields. 
Additionally, we envision that this method can also be applied to online training by improving generation quality. We leave these attempts to our future work. 





\input{pages/b.Acknowledgements}

% Bibliography entries for the entire Anthology, followed by custom entries
%\bibliography{anthology,custom}
% Custom bibliography entries only
\bibliography{acl_latex}

\appendix
\newpage

% \appendix  
\section*{Appendix}
\label{sec:appendix}
% \tableofcontents  

\subsection*{Contents}
\begin{description}
    \item [\textbf{A}] \textbf{More Observations of Motivation} .............  \pageref{sec:app:motivation}
        \begin{description}
            \item [A.1] Wasted Tokens and Expansions ......... \pageref{sec:app:wasted}
            \item [A.2] Examples of DFS and BFS Trees ...... \pageref{sec:app:trees}
        \end{description}
    \item [\textbf{B}] \textbf{Formulas and Algorithms} ........................... \pageref{sec:app:formulas_and_algorithms}
        \begin{description}
            \item[B.1] Formulas in Parallelism Streamline ... \pageref{app:sec:parallel_reasoning}
            \item[B.2] Algorithms for Searching and Transition Mechanism ....................................... \pageref{sec:app:algorithms}
        \end{description}
    \item [\textbf{C}] \textbf{Additional Details about Experiment} ...... \pageref{app:sec:exp}
        \begin{description}
            \item [C.1] Comparison Methods ......................... \pageref{app:sec:comparison_methods}
            \item [C.2] Experimental Settings ....................... \pageref{app:sec:exp_setting}
            \item [C.3] Distribution of Best Path Index ......... \pageref{app:sec:best_path_index}
            \item [C.4] Additional Results of $\lambda$ ...................... \pageref{sec:app:hyperparameter}
        \end{description}
    \item [\textbf{D}] \textbf{Related Work} ............................................. \pageref{app:sec:related_work}
\end{description}

\hspace{0pt}

\section{More Observations of Motivation}
\label{sec:app:motivation}
\subsection{Wasted Tokens and Expansions}
\label{sec:app:wasted}

\begin{figure}[ht]
    \centering
    \includegraphics[width=\linewidth]{figs/draw_wasted_tokens.pdf}
    \caption{The proportion of tokens required for the best path relative to the total tokens generated (left), and the proportion of expansions on suboptimal paths relative to the total number of expansions (right). }
    \label{fig:motivation_waste_tokens}
\end{figure}

To better understand the inefficiencies caused by frequent node switching, we conduct a statistical analysis on Qwen-2.5-1.5B with the Math500 dataset and evaluate the redundancy in token generation and node expansion. 

Token redundancy analysis: In Figure~\ref{fig:motivation_waste_tokens}(left), we compare the total number of tokens generated for each sample (blue line) against the number of tokens required for the best path (yellow line). The samples are sorted in descending order primarily by total token count and secondarily by best-path token count. Our analysis shows that the total token count does not exhibit a strict multiplicative relationship with the best-path token count, but in general, the number of tokens required for the best path is significantly lower than the total token count.
% , with an average difference of XX times. 
This suggests that traditional tree search algorithms generate a large number of unnecessary tokens during exploration.

Expansion redundancy analysis: We also examined the number of node expansions during tree growth (Figure~\ref{fig:motivation_waste_tokens}(right)). The blue line represents the total number of expansions for each sample, while the green line represents the number of expansions on suboptimal paths (i.e., nodes that do not contain any part of the optimal solution). While there is no strict multiplicative correlation between these two metrics, the green line closely follows the blue line, indicating that a significant proportion of expansions occur on suboptimal paths. This further supports the observation that traditional tree search algorithms frequently explore unnecessary areas before finding the best solution.



\subsection{Examples of DFS and BFS Trees}
\label{sec:app:trees}

\begin{figure}[ht]
    \centering
    \includegraphics[width=\linewidth]{figs/motivation_dfs_trees.pdf}
    \caption{Growth of two trees with DFS algorithm. }
    \label{fig:motivation_dfs_trees}
\end{figure}

As illustrated in Figure~\ref{fig:motivation_dfs_trees}, which shows two typical depth-first search (DFS) trees, we visualize the node expansion process in layers based on tree depth. 
The darker reddish-brown nodes represent high-confidence nodes, while the lighter nodes indicate lower-confidence ones. The arrows denote parent-child relationships, where dark blue arrows indicate later-generated nodes and light blue arrows represent earlier-generated nodes. 

From the figure, we can clearly observe the reasoning trajectory of tree search: starting from the root node, the search prioritizes the child node with the highest confidence, then recursively expands deeper by selecting the most promising child node at each level. This continues until a termination condition is met, at which point the search backtracks and explores alternative paths from the root node. Due to the nature of this process, different paths vary significantly in their depth and termination points. Moreover, the next explored path does not follow a strict spatial or hierarchical pattern within the tree.

We also observe redundant exploration issues in the right two branches. At tree depths 4/5, the confidence scores of the expanded nodes are noticeably lower compared to previously explored nodes. However, due to the inherent mechanics of depth-first search (DFS), the algorithm continues expanding these nodes until the termination condition is met, even if the intermediate confidence scores remain consistently low. As a result, considerable computation is wasted on redundant expansions and token generations, with little contribution to improving the final output quality.


\begin{figure}[ht]
    \centering
    \includegraphics[width=\linewidth]{figs/motivation_bfs_trees.pdf}
    \caption{Growth of two trees with BFS algorithm. }
    \label{fig:motivation_bfs_trees}
\end{figure}

% While breadth-first search (BFS) provides a more structured and evenly distributed exploration pattern, it suffers from frequent node switching, which prevents deep reasoning and leads to a tendency of shallow exploration. Unlike depth-first search (DFS), which aggressively expands a single path before backtracking, BFS systematically explores a wider range of possibilities but often fails to fully develop any single thought process. 

As illustrated in Figure~\ref{fig:motivation_bfs_trees}, BFS results in a flatter, more uniform, top-down expansion structure compared to the trees observed in Figure~\ref{fig:motivation_dfs_trees}. This behavior creates two key inefficiencies: 
(1) Incomplete reasoning before termination: In our experiments on the Math500 dataset, a pure BFS approach resulted in 178 (about 35.6\%)  of reasoning paths terminating without generating an answer (e.g., Tree 1 in Figure~\ref{fig:motivation_bfs_trees}). The algorithm explores many different areas of the tree but often fails to pursue any one path deeply enough to reach a valid conclusion. 
(2) Excessive expansions and token redundancy: Even when BFS eventually finds a correct answer, it tends to consume significantly more expansions and tokens than necessary (e.g., Tree 2 in Figure~\ref{fig:motivation_bfs_trees}). The best path (highlighted in yellow arrows) has a depth of only 4, yet before discovering this optimal solution, BFS explores a large number of additional nodes (light blue arrows), many of which do not contain any part of the optimal path.

\section{Formulas and Algorithms}
\label{sec:app:formulas_and_algorithms}

\subsection{Formulas in Parallelism Streamline}
\label{app:sec:parallel_reasoning}

% \textbf{Adaptive Parallelism Queue}: Parallelism queue $P$ has dynamically adjustable length, which adapts to the available GPU memory during inference. Specifically, the queue size $|P|$ is defined as: $|P| = \frac{O_{{max}} - O_{{init}}}{O_{{peak}} - O_{{init}}}$, where  $O_{{peak}}$  represents the peak memory usage during the last generation, and  $O_{{init}}$  is the memory consumption at model initialization phase. This ensures a reasonable parallel number, allowing an efficient utilization of computational resources. 

% \textbf{Node Data Structure}: In our framework, each node maintains the following key attributes at the top level: a node identifier, a pointer to the parent node, the prior confidence, complete sequence from the root node to the current node ($\mathrm{Seq}^{1 \sim n}$), past key-value cache specific to this node ($\mathrm {KV}^n$). Therefore, the node structure can be written as:
% \begin{equation}
%     Node = [\mathrm{id}, \mathrm {parent}, \mathrm {conf.}, \mathrm {KV}^n, \mathrm {Seq}^{1 \sim n}].
% \end{equation}
% A major challenge in memory optimization arises from the KV cache, which consumes significant GPU memory. To avoid redundant storage, each node only retains its own past KV cache, rather than storing the entire sequence of past KV.  


\paragraph{Data Collection and Preparation.} 
Before executing parallel inference, we should collect the input data that is stored in separate memory locations. Based on the Node Data Structure (refer to Appendix~\ref{app:sec:parallel_reasoning}), which is $[\mathrm{id}, \mathrm {parent}, \mathrm {conf.}, \mathrm {KV}^n, \mathrm {Seq}^{1 \sim n}]$, we need to concatenate the past KV caches and input sequences of different nodes into single large batch matrices. However, as discussed in Sec.~\ref{sec:motivation}, tree search paths exhibit varying path lengths, meaning that both past KV caches and context sequences have different sizes. 
To handle the length disparity and support arbitrary node parallelism, we apply padding for shorter past KV and context sequence: 
% we apply left padding for nodes with shorter past KV caches, and right padding for nodes with shorter input sequences: 
% {\footnotesize
\begin{equation}
\label{eq:kv}
\begin{split}
    & \mathrm{KV}^{1\sim n} = \mathtt{concat}\left(\mathrm{KV}^{r^n}, \cdots, \mathrm{KV}^{a^n_1}, \mathrm{KV}^n\right), \\
    & \mathrm{padding}^n = \mathbf{0}_{\max\left(\forall_{m\in P}|\mathrm{KV}^{1\sim m}| \right) - |\mathrm{KV}^{1\sim n}|},  \\
    & \mathrm{KV}^{1\sim n}_{\mathrm{pad}} =  \mathtt{concat} \left(\mathrm{padding}^{n}, \mathrm{KV}^{1\sim n} \right), \\
    & \mathrm{KV}^{all} =  \mathtt{stack}\left ( [\forall_{n\in P}\ \mathrm{KV}^{1\sim n}_{\mathrm{pad}}]\right), 
\end{split}
\end{equation}
% }
where $r$ and  $a_1, a_2, \dots$ are the root node and $1^{st}, 2^{nd}, \dots$ ancestors of $n$, respectively. $\mathbf{0}_{len}$ is a matrix of zeros with length $len$. Similar to above, the input sequences are also padded and stacked: 
% {\footnotesize
\begin{equation}
\label{eq:seq}
\begin{split}
& \mathrm{padding}^n = padding\_token_{\max(\forall_{n\in P} |\mathrm{Seq}^{1\sim n}|)},  \\
& \mathrm{Seq^{1\sim n}_{pad}} =  \mathtt{concat} \left(  \mathrm{Seq^{1\sim n}}, \mathrm{padding}^n\right), \\ 
& \mathrm{Seq}^{all}=\mathtt{stack}\left( \forall_{n\in P}\mathrm{Seq^{1\sim n}_{pad}} \right), 
\end{split}
\end{equation}
% }
where $padding\_token_{len}$ is a vector of the predefined padding token id with length $len$. 
In this way, the data is fed into the LLM for parallel generation. Additionally, techniques like DEFT~\cite{yao2024deft} are orthogonal to our DPTS and can be integrated to identify and merge shared prefixes, further optimizing inference efficiency.

\paragraph{Generation and Updating.} After the generation phase, we obtain new output sequences and past KV caches. These are then partitioned based on the tree width. Specifically, we segment past KV caches and only store those corresponding to new tokens. Output sequences are completely stored rather than fragmented, due to the negligible memory overhead. For clearer demonstration, the output of $i^\mathrm{th}$ node in $P$ can be written as 
% {\footnotesize 
% \begin{equation}
% \label{eq:n_new}
% \begin{split}
%     &  \mathrm{KV}^{n^{ij}}  = \mathbf{n}.\mathtt{past\_kv}_{[ij, \dots, |\mathrm{KV}^{all}|:]}, \forall j\in[1, \dots, \mathtt{tw}] \\
%     & \mathrm{Seq}^{1\sim n^{ij}} = \mathbf{n}.\mathtt{output}_{[ij]} \\ 
%     & n^{ij}  = [\mathrm{id}, n^i, \mathrm{KV}^{n^{ij}}, \mathrm{Seq}^{1\sim n^{ij}}] \\ 
%     & \mathbf{n^{new}} = \{n^{i1}, \dots, n^{i\mathtt{tw}}\}, 
% \end{split}
% \end{equation}
% }
\begin{center}
\begin{equation}
\label{eq:n_new}
\begin{split}
    & \mathbf{KV}^{n^{ij}} = \mathbf{n}.\mathtt{past\_kv}_{[ij, \dots, |\mathbf{KV}^{\text{all}}|:]}, \forall j \in \left[1, \dots, w\right] \\
    & \mathbf{Seq}^{1 \sim n^{ij}} = \mathbf{n}.\mathtt{output}_{[ij]} \\
    & n^{ij} = \left[\mathrm{id}, n^i, \mathbf{KV}^{n^{ij}}, \mathbf{Seq}^{1 \sim n^{ij}}\right] \\
    & \mathbf{n^{new}} = \left\{ n^{i1}, \dots, n^{iw} \right\}
\end{split}
\end{equation}
\end{center}
where $w$ is the tree width, $\mathbf{n}$ is the generation output in parallel manner. 
If sequences were stored in a fragmented manner, every inference step would require additional collection and concatenation, introducing unnecessary latency.  This approach is a trade-off between inference speed and memory consumption. 
Newly generated nodes are then updated into the candidate node pool $N$, making them available for subsequent selection processes.



\subsection{Algorithms for Searching and Transition Mechanism}
\label{sec:app:algorithms}

In the main text, due to paper length constraints, we only present the overall process in Algorithm~\ref{alg:algorithm_parallel}, which connects the entire framework’s algorithms and formulations, including Parallelism Streamline and the Search and Transition Mechanism.

In the above section, we provided the mathematical formulation of the Parallelism Streamline. In the following, we further supplement the algorithmic details of Search, Transition, and Reward, offering readers a clear and intuitive representation of the algorithmic process. 



\begin{algorithm}[ht]
    \caption{Searching}
\label{alg:searching}
\begin{algorithmic}[1]
\REQUIRE Parallel queue $P$, current parallel queue size $\tau_P$, candidate node pool $N$. 
\ENSURE Updated $P$. 
\STATE $e_1\gets 0$
\FORALL{$n\in P$}
    \IF{$n$.mode$ = \mathtt{EXPLOIT}$}
    \STATE $e_1 \gets e_1  + 1$
    % \ELSE
    % \STATE $e_2 \gets e_2 + 1$
    \ENDIF
\ENDFOR
\IF{$|P|<\tau_P$}
    \STATE $N'\gets$ Descending $N$ based on conf. \\
    % , $\forall n \in N$
    \STATE $\mathbf{u} \gets N'[:\tau_P-|P|]$ 
    \FORALL{$u \in \mathbf{u}$}
        \IF {$ e_1<p|P| $}
        \STATE $u$.mode $\gets \mathtt{EXPLOIT}$
        \STATE $e_1 \gets e_1  + 1$
        \ELSE
        \STATE $u$.mode $\gets \mathtt{EXPLORE}$
        \ENDIF 
    \ENDFOR
    \STATE $P \gets P \cup \mathbf{u}$
\ENDIF
\RETURN $P$
\end{algorithmic}
\end{algorithm}

Firstly, Algorithm~\ref{alg:searching} demonstrates the searching mechanism. Specifically, during initialization or when the number of nodes in the parallel queue $P$ is less than the maximum parallelism $\tau_P$, the algorithm selects the $\tau_P- |P|$ highest-confidence nodes from the candidate node pool $N$ as supplementary nodes (Lines 8-9).

Then, the highest-confidence nodes are designated as exploit nodes (Line 12) until the proportion of exploit nodes reaches the ratio $p$. The remaining selected nodes are assigned as explore nodes. Finally, these newly selected nodes are merged into $P$ to prepare for the next cycle of parallel expansion. 

\begin{algorithm}[ht]
    \caption{Transition}
\label{alg:transition}
\begin{algorithmic}[1]
\REQUIRE Parallel queue $P$, transition thresholds $\theta_\mathrm{es}$ and $\theta_\mathrm{ds}$. 
\ENSURE Updated $P$. 
\FORALL{$n \in P$}
    \STATE $n^*=\max_\mathrm{conf.}$($n$.children)
    \IF{$n$.mode $=$ $\mathtt{EXPLOIT}$ \AND $c_{n^*} >\theta_\mathrm{es}$ \OR $n$.mode $=$ $\mathtt{EXPLORE}$ \AND $c_{n^*} >\theta_\mathrm{ds}$}
        \STATE $P \gets P\cup \{n^*\}$
        \STATE $n^*$.mode $\gets \mathtt{EXPLOIT}$ 
    % \ELSE
    \ENDIF
    \STATE $P \gets P \setminus n$
\ENDFOR
\RETURN $P$
\end{algorithmic}
\end{algorithm}

Next is the Transition Algorithm~\ref{alg:transition}. After each expansion, we iterate through all nodes in the parallel queue $P$ and identify the best child node  $n^*$ for each node.

Then, based on the category of node $n$, we compare $n^*$ with the corresponding threshold  $\theta$. If the early stop condition is not met or the deep seek condition is met, we add  $n^*$ as a new exploit node into $P$. Otherwise, the node will no longer be expanded and will be evicted from $P$.

\begin{algorithm}[ht]
    \caption{Reward}
\label{app:algo:reward}
\begin{algorithmic}[1]
\REQUIRE Parallel queue $P$, candidate node pool $N$. 
\ENSURE Updated $N$. 
\FORALL{$n \in P$} 
    % \PARALLEL
        \STATE $n$.children $\gets \text{Eq. (\ref{eq:n_new})}$ 
        \FORALL{$m \in n$.children}
            \IF{$\mathtt{is\_terminate}(m)$}
                \STATE $m$.reward $\gets \mathtt{reward}(m)$
            \ELSE
                \STATE $N \gets N \cup \{m\}$
            \ENDIF
        \ENDFOR
    \ENDFOR
\RETURN $N$
\end{algorithmic}
\end{algorithm}

After completing an expansion, we check whether each node’s path meets the termination conditions, such as reaching the maximum token limit or generating an end token. 
If a path satisfies the termination condition, exploration of that path stops, and a reward is computed as the final path score. We demonstrate this process in Algorithm~\ref{app:algo:reward}, which is identical to previous tree search algorithms and is not discussed in detail. However, for the sake of algorithmic completeness, we explicitly include it here.

\section{Additional Details about Experiment}
\label{app:sec:exp}

\subsection{Comparison Methods}
\label{app:sec:comparison_methods}

We compare DPTS against three widely used search algorithms: (1) Monte Carlo Tree Search~(MCTS)~\cite{sprueill2023monte} balances exploration and exploitation when sampling, while the selected paths rollout till termination, (2) Best-of-N~\cite{cobbe2021training} performs multiple independent rollouts and selects the highest-scoring output, (3) Beam Search~\cite{Yao_2023_Tree} expands multiple hypotheses in parallel, pruning low-scoring candidates at each step to maintain a fixed-width search~\cite{snell2024scaling}. 
Since efficient tree search algorithms have recently regained attention after the emergence of LLM reasoning, the strong baselines are limited. As a result, we primarily compare DPTS against these typical and well-established search algorithms to demonstrate the effectiveness of our method. 

 
\subsection{Experimental Settings}
\label{app:sec:exp_setting}

% 实验参数: tree_width:4 tree_depth:16 mini_step:128 % 其实 103 确实实验结果更好不知道为什么 otz
% max token:2048 mcts_max_time:120

The experimental settings are as follows: we set the tree width to 4, tree depth to 16, mini step to 100, and the maximum token limit to 2048. The MCTS time limit is 120 seconds, and the threshold parameter is empirically set to $t^*=5$. All models were downloaded from Hugging Face.

For evaluation, we implemented a custom codebase, which is included in the supplementary materials. Inference automatically terminates if it exceeds the timeout limit or encounters a memory overflow. Within these constraints, the search tree can expand and roll out indefinitely, ensuring comprehensive exploration during inference. 

\subsection{Distribution of Best Path Index}
\label{app:sec:best_path_index}

We use a histogram to visualize the earliest (blue bar) and shortest (green bar) best path index. Through an ablation study, we examine how the best solution in the search path evolves as our method is progressively introduced.

In the baseline method, the best path typically appears around the 8th terminated path. Incorporating the parallelism streamline does not directly affect the accuracy of the search path. However, after adding the searching and transition mechanism, DPTS finds the best solution region more quickly and reaches the best path earlier.

\begin{figure}
    \centering
    \includegraphics[width=\linewidth]{figs/best_path_distribution.pdf}
    \caption{The distribution of the earliest (blue bar) and shortest (green bar) best path index. }
    \label{fig:best_path_index}
\end{figure}


\subsection{Additional Results of $\lambda$}
\label{sec:app:hyperparameter}

We conducted a more detailed experiment on $\lambda$, with results presented in Table~\ref{app:tab:hyperparameter} and Figure~\ref{app:fig:lambda_curve}. It is evident that when $\lambda$ is set within a reasonable range (e.g., $[0.7,0.9]$), both accuracy and inference time exhibit optimal performance. In this range, the proportion of DS\% (deep seek transitions) is higher than ES\% (early stop transitions), indicating that more high-confidence nodes are being timely converted into exploit nodes. At the same time, a small number of paths are reassigned as low-score paths during inference and subsequently terminated.

However, when $\lambda$ is too large (e.g., close to $1$), the proportion of es increases aggressively. This suggests that many exploit nodes being expanded have scores within the range of $[0.9,1.0]$, and setting the threshold in this range may cause some correct paths to be prematurely stopped. Conversely, when $\lambda$ is too small (e.g., $< 0.4$), both ES and DS proportions drop to nearly zero. This occurs because most node scores exceed the threshold, causing nearly all paths to expand under exploitation mode. Moreover, an overly small early stop threshold causes no paths to terminate, effectively degrading the search into an exploitation-only strategy. 

Therefore, selecting a suitable $\lambda$ is important. A larger  $\lambda$ imposes stricter exploitation conditions, leading to more paths being stopped and fewer paths being converted to deep seeking. Conversely, a smaller $\lambda$ results in looser conditions, allowing more exploiting paths to continue rolling out until they reach a termination condition, while more high-confidence paths transition into deep seeking.


\begin{figure}
    \centering
    \includegraphics[width=\linewidth]{figs/draw_lambda_curve.pdf}
    \caption{Illustrations of hyperparameter analysis. }
    \label{app:fig:lambda_curve}
\end{figure}

\begin{table}
    \centering
    \caption{Hyperparameter $\lambda_\mathrm{es}$ and $\lambda_\mathrm{ds}$ in transition thresholds $\theta_\mathrm{es}$ and $\theta_\mathrm{ds}$. ``ES (Early Stop)~\%'' and ``DS (Deep Seek)~\%'' are the ratios of the node type transition between the exploitation and exploration.  }
    \label{app:tab:hyperparameter}
    \resizebox{\linewidth}{!}{
    \begin{tabular}{cccccc}
    \toprule
        \textbf{$\lambda_\mathrm{es}$} & \textbf{$\lambda_\mathrm{ds}$} & \textbf{Acc.} & \textbf{Time (s)} & \textbf{ES~(\%)}  & \textbf{DS~(\%)} \\ \midrule
        1.0 & 1.0 & 53.0 & 47.59 & 41.4 & 10.5 \\ 
        0.95 & 0.95 & 57.8 & 43.99 & 23.4 & 18.2 \\ 
        0.9 & 0.9  & 58.6 & 43.30 & 15.6 & 20.9 \\
        0.85 & 0.85 & 58.4 & 42.60 & 9.67 & 23.7 \\
        0.8 & 0.8 & 58.0 & 46.33 & 8.1  & 23.9 \\
        % 0.7 & 0.8 & & \\
        0.7 & 0.7 & 58.4 & 41.58 & 5.3 & 27.5 \\
        0.6 & 0.6 & 57.4 & 44.39 & 6.1 & 32.3   \\
        0.4 & 0.4 & 56.6 & 38.41 & 0 & 0.1 \\
        0.2 & 0.2 & 57.0 & 40.71 & 0 & 0.1  \\
        0   & 0   & 54.4 & 42.78 & 0 & 0.1  \\
        \bottomrule
    \end{tabular}
    }
    \vspace{5pt}
\end{table}

\section{Related Work}


\minisection{Pivot-based approaches}
Pivot translation is an approach that decomposes the translation task into two sequential steps~\cite{wu-wang-2007-pivot, utiyama-isahara-2007-comparison}.
By transferring knowledge from high-resource pivot languages, pivoting is especially effective in translation between low-resource languages \cite{zoph-etal-2016-transfer, aji-etal-2020-neural, he-etal-2022-tencent}.
In this study, pivot translation enables us to obtain high-quality candidates for the ensemble.
\citet{kim-etal-2019-pivot} discusses a pivot-based transfer learning technique where source$\rightarrow$pivot and pivot$\rightarrow$target models are first trained separately, then use pre-trained models to initialize the source$\rightarrow$target model, allowing effective training of a single, direct NMT model.
\citet{zhang-etal-2022-triangular} further investigate the transfer learning approach by utilizing auxiliary monolingual data.


Pivot translation typically employs English as the bridge language.
Nonetheless, previous studies have explored the use of diverse pivot languages, taking into account factors such as data size and the relationships between languages~\cite{paul2009importance, dabre-etal-2015-leveraging}.
By leveraging the ability of pivot translation to produce diverse outputs, several studies have focused on generating paraphrases~\cite{mallinson-etal-2017-paraphrasing, guo2019zeroshot}.
More recently, \citet{mohammadshahi-etal-2024-investigating} uses pivot translation for ensemble, but it requires computing token-level probabilities and fails to improve translation.
Our work shares the motivation with these studies, generating translations depending on the pivot path to obtain a variety of candidates.


\minisection{Ensemble in NLG tasks}
Ensemble learning is a widely adopted strategy to obtain more accurate predictions by employing multiple systems~\cite{sagi2018ensemble}.
In NMT, the traditional approach involves averaging the probability distributions of the next target token, which is predicted at each decoding step by multiple models ~\cite{bojar-etal-2014-findings} or by different snapshots~\cite{huang2017snapshot}.
When multiple sources are available, an ensemble can be conducted with predictions obtained by different sources~\cite {firat-etal-2016-zero}.
Also, a token-level ensemble through vocabulary alignment across LLMs has also been proposed~\cite{eva}.
However, these methods are not applicable to recent black-box models as they cannot compute token-level probabilities at decoding time.


Selection-based ensemble has also been explored, which chooses the final output among the existing candidates.
This can be achieved through majority voting by selecting the most frequent one~\cite{wang2022rationaleaugmented} or selecting the best candidate with QE~\cite{fernandes-etal-2022-quality, howgood}.
Recently, MBR decoding~\cite{GOEL2000115, mbr}, which aims to find the hypothesis with the highest expected utility, has gained attention.
However, this approach limits the final output space to the existing candidate pool.


\begin{figure*}[t]
  \centering
  \includegraphics[width=0.95\textwidth]{Figures/overview.pdf} 
  \caption{Overview of \ours framework.}
  \label{fig:overall}
\end{figure*}


On the other hand, the generation-based ensemble method involves generating a new final prediction.
Fusion-in-Decoder~\cite{fid} proposes an architecture that aggregates additional information with a given input.
More recently, within the context of LLMs, \citet{llm-blender} and \citet{exchangeofthought} investigate a method of using LLMs to generate multiple outputs and aggregate them.
Generating new output through LLMs offers the benefit of explicitly harnessing their pre-trained knowledge within the ensemble process.

% \newpage

% \section{Critical Code}
% \begin{figure*}[t]
%     \centering
%     \begin{lstlisting}[caption=Code]

% print("hello world")
% print("hello world")

%     \end{lstlisting}
%     \label{fig:lst_code}
% \end{figure*}

% \minitoc  % 仅显示附录的目录




\end{document}
