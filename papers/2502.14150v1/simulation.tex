\section{Simulation Results} % (fold)
\label{sec:simulation_results}

\begin{figure*}[!ht]
    \centering
    % \subfloat[\label{fig:24bus.network}]{\includegraphics[height=1.5in]{images/IEEE_24_network.png}} \qquad
    \subfloat[\label{fig:24bus.cost.load}]{\includegraphics[height=1.5in]{images/cost_load_alpha_24.pdf}}
    \subfloat[\label{fig:24bus.ms.uplift}]{\includegraphics[height=1.5in]{images/ms_uplift_alpha24.pdf}}
    \subfloat[\label{fig:24bus.lmp.alpha06}]{\includegraphics[height=1.6in]{images/pglib_opf_case24_ieee_rts_06_-10.png}}
    \subfloat[\label{fig:24bus.lmp.alpha09}]{\includegraphics[height=1.6in]{images/pglib_opf_case24_ieee_rts_09_-10.png}}
    \includegraphics[height=1.6in]{images/pglib_opf_case24_ieee_rts_cm.png}
    \caption{Simulation results on the IEEE 24-bus RTS test network with \protect\subref{fig:24bus.cost.load} load shed (average and total) and dispatch cost (nominal and total including reserve), \protect\subref{fig:24bus.ms.uplift} merchandising surplus and uplift payments under \LMPnom{} and \LMPmar{} schemes, and \LMPmar{} for \protect\subref{fig:24bus.lmp.alpha06} $\alpha = 0.6$, and \protect\subref{fig:24bus.lmp.alpha09} $\alpha = 0.9$.}
    \label{fig:24bus.sim}
    \label{fig:24bus.network}
\end{figure*}

% \begin{figure}
%     \centering
%     % \subfloat[\label{fig:24bus.lmp.alpha0}]{\includegraphics[height=1.5in]{images/pglib_opf_case24_ieee_rts_00.png}}
%     % \subfloat[\label{fig:24bus.lmp.alpha02}]{\includegraphics[height=1.5in]{images/pglib_opf_case24_ieee_rts_02.png}}
%     \subfloat[\label{fig:24bus.lmp.alpha06} $\alpha = 0.6$]{\includegraphics[height=1.75in]{images/pglib_opf_case24_ieee_rts_06.png}}
%     \subfloat[\label{fig:24bus.lmp.alpha09} $\alpha = 0.9$]{\includegraphics[height=1.75in]{images/pglib_opf_case24_ieee_rts_09.png}}
%     \includegraphics[height=1.75in]{images/pglib_opf_case24_ieee_rts_cm.png}
%     \caption{LMPs of the 24-bus network RTS test case for various $\alpha$.}
%     \label{fig:24bus.network}
% \end{figure}

% \begin{enumerate}
%     \item Present the 30-bus network.
%     \item Show results on the 30-bus network and discuss \CSCED{} vs \RSCED{} and effect of risk-sensitivity.
%     \item Demonstrate increased computational burden associated with risk aversion.
%     \item Need to discuss pricing properties.
%     \item Sweep pglib-opf cases for comparison of runtime and demonstrate (cubic?) growth.
% \end{enumerate}

We now present results from the numerical experiments of our \RSCED{} formulation on the IEEE 24-bus RTS test network shown in Figure \ref{fig:24bus.network} from the PGLib v21 \cite{pglib2021}. The network is augmented with a uniform \VoLL{} of $\$9100/$MW, and relaxed line ratings of $\v{f}^{\DA} = 1.7 \v{f}$ and $\v{f}^{\SE} = 1.2 \v{f}$. \rev{Our scenario set consists of every single-line failure, and we assume a uniform distribution over these scenarios.} As seen in Figure \ref{fig:24bus.cost.load}, increasing risk-aversion results in a decrease in the expected and total load shed. In this example, there does not exist a solution without load shed and thus, the canonical \CSCED{} formulation in \eqref{eq:csced} that does not model load shed, would be infeasible. However, the \RSCED{} problem is able to find a solution. As Figure \ref{fig:24bus.cost.load} illustrates, while increasing risk-aversion, i.e., larger $\alpha$ values, corresponds to decreased load shed, it also results in greater dispatch cost, including nominal and reserve capacities. 
% \nate{Scenarios uniformly distributed.}

Under \LMPnom{}, as illustrated in Figure \ref{fig:24bus.ms.uplift}, the merchandising surplus is not positive for $\alpha>0.7$, an instance of the potential lack of revenue adequacy described in Theorem \ref{thm:ms} (i). As discussed in Section \ref{sec:IRandLOC}, \LMPnom{} LOC payments grow rapidly for $\alpha>0.8$ as the SO becomes more averse to worst-case scenario congestion. For $\alpha>0.4$, merchandising surplus is less than the required LOC payments, meaning that the SO runs cash negative after settling the payments of the market participants. On the other hand, the corresponding \LMPmar{} merchandising surplus is always positive and greater than the LOC payments, affirming the result in Theorem \ref{thm:revenue}. 
% Following the discussion in Section \ref{sec:IRandLOC}, this merchandising surplus 
In fact, the LOC payments under \LMPmar{} are identically zero for this example across all values of $\alpha$. While we cannot guarantee that this is true in general, the payments are substantially lower using \LMPmar{} than \LMPnom{}, and remain similar across all levels of risk aversion.

\revision{In Figures \ref{fig:24bus.lmp.alpha06} and \ref{fig:24bus.lmp.alpha09}, we provide two heat-maps of \LMPmar{}s across the 24-bus system for two different values of the risk parameter $\alpha$. 
% Notice that with higher value of $\alpha$, prices at certain buses (e.g., at bus 6) vary more than in others (e.g., at bus 19) with $\alpha$. 
Notice that the prices at certain buses (e.g., at bus 6 in the lower righthand corner of the network) are more sensitive to increases in the tunable parameter $\alpha$ than others (e.g., bus 19 in the upper center part of the network). 
% Variation in prices with location even with classical LMPs (defined as $\v{\pi}^{\ED}$ in \eqref{eq:ED_LMP}) result from congestion across the network at an optimal dispatch by making it cheaper to deliver incremental demand at certain locations than others. 
Under the classical LMP pricing mechanism (defined as $\v{\pi}^{\ED}$ in \eqref{eq:ED_LMP}) variation in prices across network buses follows from line congestion at the optimal dispatch, which makes it cheaper to service incremental demand at certain locations than others. 
When that dispatch is calculated using the \RSCED{} formulation, \LMPmar{}-based prices are further affected by \emph{possible} congestion that might arise under optimal recourse decisions in contingency scenarios. Since \RSCED{} balances nominal dispatch, recourse costs, and their probabilities in a unified framework, network price variations, in a sense, reflect the combined effect of network constraints and responses to uncertain contingencies. When SO's risk tolerance is low (with higher $\alpha$), meeting incremental load at one location can be much more costly than at another, as both the nominal dispatch may vary substantially to meet the stringent risk requirements, as well as the congestion patterns in those scenarios. Just as congestion components of LMPs in $\v{\pi}^{\ED}$ can be used as signals for identifying candidate network sections for transmission expansion, maintenance, and upgrades, the congestion components of \RSCED{} \emph{across} scenarios can serve the same purpose. In addition, \LMPmar{}-based prices may also highlight the line failure scenarios that contribute more heavily. }
% These lines are possibly prime candidates for maintenance and upgrades.}

\begin{table}[h]
    \centering
    \caption{The effect of adjusting the line failure probability of the line between buses 3 and 24 with $\alpha = 0.6$.}
    \label{tab:24bus.prob.compare}
    % \begin{tabular}{p{0.12\columnwidth}>{\centering}p{0.24\columnwidth}>{\centering}p{0.127\columnwidth}>{\centering}p{0.127\columnwidth}{\centering}p{0.155\columnwidth}}
    \begin{tabular}{l c c}
         \toprule
         \textbf{Failure} & \textbf{Nominal}
         % & \textbf{Reserve}
         & \textbf{Max load} \\
         \textbf{Probability} & \textbf{cost (\$/h)}
         % & \textbf{cost (\$/h)}
         & \textbf{shed (p.u.)}  \\
         \midrule
         $0.$ & 1042.854 & 1.7026 \\
         $0.001$ & 1042.941 & 0.6994 \\
         $0.01$ & 1113.926 & 0.5104 \\
         % $0.$ & 1042.854 & 31.615 & 1.7026 \\
         % $0.001$ & 1042.941 & 35.050 & 0.6994 \\
         % $0.01$ & 1113.926 & 28.800 & 0.5104 \\
         % $0.1$ & 1113.926 & 28.800 & 0.5104 \\
         % $0.2$ & 1113.926 & 28.800 & 0.5104 \\
         % $0.3$ & 1113.926 & 28.800 & 0.5104 \\
         \bottomrule \\
    \end{tabular}
\end{table}

\revision{In Table \ref{tab:24bus.prob.compare}, we study the impact of changing the line failure probability of a single line (joining buses 3 and 24), while holding the probabilities of other line failure scenarios to 0.0014. The results display the same trends as observed for the 3-bus system in Section \ref{sec:rsced}.}



\subsection{Evaluating Benders' decomposition for arbitrary networks}

To demonstrate the viability of the Benders' decomposition approach to solving the \RSCED{} problem for practical networks, we evaluate the solution time for a number of PGLib v21 networks. \rev{As in the previous simulation, we take as our scenario set every possible single-line failure, and assume a uniform distribution over these scenarios}. The run-times are shown in Table \ref{tab:runtimes}, where the column labeled ``Gurobi" lists runtimes for solving the large linear program in \eqref{eq:RSCED.lp} using Gurobi. As the table indicates, run-times for both the large LP solution and Benders' decomposition generally increase with network size. Benders' decomposition becomes competitive with solving the large LP for systems larger than 14 buses and significantly 
% outperforms it 
faster for networks larger than 73 buses.
%
\begin{table}[ht]
    \centering
    \caption{Runtime comparison of Benders' Decomposition and direct solution of the large LP formulation of the \RSCED{} problem.}
    \csvreader[column names={casename=\cname,g0=\gzero,b0=\bzero,g9=\gnine,b9=\bnine},tabular=lrrrr,head,respect underscore=true,table head=\toprule & \multicolumn{2}{c}{\uline{\hspace{0.6cm}${\alpha = 0}$\hspace{0.6cm}}} & \multicolumn{2}{c}{\uline{\hspace{0.5cm}$\alpha = 0.9$\hspace{0.5cm}}} \\ Case Name & Gurobi & Benders' & Gurobi & Benders' \\ \midrule, table foot=\bottomrule \\]{data/rsced_runtimes.csv}{}{\cname & \gzero & \bzero & \gnine & \bnine }
    \label{tab:runtimes}
    \setlength\belowcaptionskip{-1in}
\end{table}
% \vspace*{-\baselineskip}
%

% \rev{
% \raggedbottom
% \vspace{-0.5in}
\rev{The results presented in Table \ref{tab:runtimes} constitute a proof of concept, demonstrating that our formulation is amenable to decomposition schemes such as Benders'. 
Algorithmic enhancements to the base Benders' approach (e.g., in \cite{liu2015computational}) will likely improve run-times over the results in Table \ref{tab:runtimes}, and facilitate solution of \RSCED{} over larger power networks.} \revision{We emphasize that \RSCED{}, with its flexibility in specifying SO's risk preference, falls in the same optimization class as \CSCED{}. In addition, \RSCED{} admits the decomposition in \eqref{eq:decomp.sup.epi}-\eqref{eq:decomp.sub} that shares parallels with that for \CSCED{} in \cite{liu2015computational}, permitting algorithmic innovations (including and beyond Benders') to apply seamlessly across formulations.}
% }  
% several algorithmic enhancements to the standard Benders' approach  to the  
% The extension of such a method to orders of magnitude larger-scale networks introduces significant computational challenges. However, we anticipate that the application of algorithmic enhancements such as those presented in \cite{liu2015computational} will help to mitigate such difficulties, allowing future work to include expanded results addressing the scalability of our formulation.
% }
% We see that the Benders' Decomposition algorithm is competitive with solving the large LP for cases with more than 14 buses, and demonstrates significantly reduced computation times for networks larger than 73-buses.

% section simulation_results (end)