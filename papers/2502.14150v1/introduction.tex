%!TEX root = ./mainRSCED.tex

\section{Introduction}

Power procurement is typically performed by solving an \emph{economic dispatch} (ED) problem matching aggregate supply to demand and ensuring an economically efficient dispatch satisfying network constraints. These problems are often solved in forward planning operations prior to the time of power delivery. Such forward planning faces uncertainty in many aspects including available supply, demand fluctuations, and component availability. In this paper, we focus on uncertainty in transmission line availability. In order to ensure that system operators (SOs) can respond adequately to any line failures, they consider ED problems augmented with additional security constraints known as \emph{security-constrained economic dispatch} (\SCED{}) problems. Various \SCED{} formulations exist, many of which fall under two primary categories: preventive security (\PSCED{}), which ensures dispatches satisfy line flow capacities across considered contingencies \cite{alsac1974optimal}, and corrective security (\CSCED{}), which allows for an SO to take corrective action in the form of reserve capacity deployment \cite{monticelli1987security,capitanescu2007improving,li2016adaptive,capitanescu2011state} and in some cases load shed \cite{bouffard2008stochastic}.

Reserve capacity is itself procured in forward planning operations. Historically, this procurement has been guided by heuristics, e.g., securing reserves up to a fixed percentage of the total anticipated load as in the case of CAISO \cite{CAISO}. However, such methods 
% fail to guarantee that sufficient reserve capacity is procured, and 
can result in either larger or smaller than necessary quantities of procured capacity. The former results in additional overhead costs and inefficient dispatches, while the latter can result in involuntary blackouts. Here we consider a \SCED{} formulation explicitly incorporating and thus automating the reserve procurement process, 
 % an automated approach to nodal reserve procurement by incorporating it explicitly in the \SCED{} formulation, 
similar to the approaches in \cite{capitanescu2011state,shi2022scenario,galiana2005scheduling}. Specifically, we simultaneously optimize over-dispatch and corrective actions, yielding explicit quantities of reserve capacity required across failure scenarios as a byproduct. Therefore, sufficient reserve capacity is procured while minimizing the cost associated with this capacity.

In the context of power system operation, security can be defined in either a deterministic or probabilistic sense \cite{galiana2005scheduling}. Deterministic definitions require that demand is met without load shed across all contingencies, while probabilistic definitions allow that scenario-dependent portions of load may be shed at an additional cost to be minimized. Traditionally this additional cost has been modeled as the expected \emph{value of lost load} (\VoLL{}) with respect to a reference probability distribution over failure scenarios, e.g., as in \cite{bouffard2008stochastic}. However, SOs usually treat load shed as a measure of last resort, and optimization of expected \VoLL{} can in practice expose power systems to undesirably high levels of risk for load shed. On the other hand, robust approaches based on deterministic notions of security tend to result in overly conservative dispatches suffering similar overhead costs as in some of the aforementioned heuristic approaches. Thus, a fundamental tradeoff exists between dispatch reliability in the face of uncertainty and economic efficiency, and existing \SCED{} formulations tend to implicitly impose a choice between one or the other. 

In this work, we propose a \emph{risk-sensitive} \SCED{} (\RSCED{}) formulation which features simultaneous primary and reserve generation, as well as a simple parametrization capable of capturing a continuum of risk preferences towards load shed. 
% procurement and accounts for potential load shed, and further allows for a . The quantity of this load shed is dependent on the realized scenario, implying a distribution of costs drawing values from each possible scenario. Meaningfully evaluating the cost associated with load shed thus requires adoption of a probability measure. In \cite{bouffard2008stochastic}, authors consider the expected cost with respect to a reference distribution. However, this can lead to higher levels of load shed than necessary. SOs are, in general, averse to potential load shed. On the other hand, robust approaches that seek to avoid load shed entirely can result in overly conservative dispatches with large associated costs. Thus, a fundamental tradeoff exists between dispatch reliability in the face of uncertainty and economic efficiency. 
In particular, we model the risk associated with load shed costs via the \emph{conditional value at risk} (\CVaR{}) measure. This risk measure, parameterized by a single scalar $\alpha$, measures the average losses over the $1 - \alpha$ fraction of worst-case scenarios. Adjusting $\alpha$ allows SOs to express their tolerance towards potential load shed. For example, setting $\alpha = 0$ recovers consideration of expected cost over all scenarios, while setting $\alpha$ close to 1 focuses optimization on worst-case, i.e., highest-cost scenarios, approaching deterministically robust solutions.
% For general distributions, it is given by \nate{should just explain this without equation at this point}
% \begin{equation}
%     \cvar[\alpha]{\mathpzc{x}} := \min_{\mathpzc{z}} z + \frac{1}{1 - \alpha} \mathbb{E}[ \mathpzc{x} - z ]^+,
% \end{equation}
% where $[\cdot]^+$ denotes the positive part of its argument.

The \CVaR{} measure has garnered recent attention in power system applications \cite{jabr2005robust,asensio2015stochastic,madavan2019risk,li2018flexible,wang2015distributionally} due to several properties that make it highly amenable to optimization. Most notably, 
% This includes the property that 
\CVaR{} is a coherent risk measure, which implies that it preserves convexity \cite{shapiro2021lectures} of input cost functions, and as a special case retains the linearity of input linear cost structures. 
% Thus, \CVaR{}-based objectives maintain existing convexity, and often allow for equivalent linear programming formulations. 
As such, optimization of \CVaR{} objectives allows for the use of off-the-shelf convex optimization solvers and adoption of decomposition methods such as Benders' decomposition that have been applied to similar problems in \cite{liu2015computational}.

The convexity of \CVaR{} further aids in establishing meaningful prices for the \RSCED{}-based dispatches via Lagrangian analysis. SO revenue adequacy and individual rationality for participating generators are crucial prerequisites to a functioning market. In the context of the proposed \RSCED{} formulation, we examine whether these properties can be guaranteed under two deterministic market clearing mechanisms suggested for use in concert with \SCED{} formulations. The first,
% is an ex-post pricing scheme, 
which we refer to as the \emph{locational marginal price of the nominal} (\LMPnom{}), establishes prices based solely on the nominal dispatch, i.e., the base case dispatch with respect to which potential corrective actions are taken. Proponents of these prices observe that explicitly pricing against security constraints results in increased cost due to unrealized scenarios \cite{boucher1998security}. In contrast, authors in \cite{hogan2014electricity} suggest that these prices are inadequate to capture the cost borne by SOs or support grid maintenance and development, and instead advocate for the use of an ex-ante pricing scheme that we refer to as the \emph{locational marginal price under security} (\LMPmar{}). These prices reflect the total marginal cost of generation in \CSCED{} problems, including our \RSCED{} formulation.
% (or \CSCED{}) \nate{\CSCED{} not discussed yet to this point}. 

Prior work on pricing design for \SCED{} problems includes a variety of results on both revenue adequacy and cost recovery. Note that cost recovery is implied by individual rationality. Considering energy-only formulations and two-stage, stochastic pricing mechanisms \cite{pritchard2010single} and \cite{morales2012pricing} establish both properties in expectation, while \cite{morales2014electricity} establishes cost recovery in expectation and revenue adequacy in all scenarios. In \cite{wong2007pricing}, an energy-reserve co-optimization formulation paired with \LMPmar{} pricing is likewise shown to yield both properties in expectation. 
% In a market simultaneously clearing energy and reserves for primary and tertiary regulation, the authors in \cite{galiana2005scheduling} advocate for the use of \LMPmar{} pricing via numerical examples but do not establish guarantees regarding revenue adequacy or individual rationality. 
Incorporating separate costs associated with reserve deployment in an energy-reserve formulation, authors in \cite{shi2022scenario} demonstrated revenue-adequacy in expectation and cost recovery per scenario. Our risk-sensitive \CVaR{}-based formulation generalizes these former ones, which optimize expected costs. \rev{While risk-sensitive dispatch formulations, as well as \LMPnom{} and \LMPmar{} pricing mechanisms, have been proposed separately in the literature, the market-relevant properties such as revenue adequacy and individual rationality of these pricing mechanisms remain poorly understood, including in risk-sensitive contexts.}

The main contributions of this paper are as follows:
% \begin{enumerate}
    % \item 
    
    \noindent $\bullet$ A scenario-based \emph{risk-aware} energy-reserve co-optimization model is proposed. Risk is modeled via the \CVaR{} risk measure, resulting in a convex optimization problem incorporating network constraints in all contingency scenarios. 
    
    % \item 
    \noindent $\bullet$ \rev{Individual rationality is established for \LMPnom{} and \LMPmar{} variants of locational marginal prices derived in our risk-aware setting, when paired with associated \emph{lost opportunity cost} (\LOC{}) payments made by the SO to generators 
    % in compensation 
    for deviation from individual profit-maximizing dispatches.}
    
    % \item 
    \noindent $\bullet$ Revenue adequacy is established for  \LMPmar{} pricing of \RSCED{}, while we show that \LMPnom{} may result in negative revenue. Including \LOC{} payments, we further show that the total revenue under \LMPmar{} pricing is nonnegative. 
    % \item 

    \noindent $\bullet$ We provide a Benders' decomposition algorithm to efficiently solve the \RSCED{} problem in a manner analogous to \rev{risk-neutral or robust SCED approaches}. 
% \end{enumerate}
% provide related results for a pricing scheme similar to \LMPmar{} for a setting in the risk-neutral case, i.e., minimizing costs in expectation, which our risk-sensitive \CVaR{}-based formulation generalizes. 

% We provide rigorous proofs demonstrating that \LMPmar{} ensures the SO operates with net non-negative revenue, while \LMPnom{} may not. 
% in work performed in parallel. 
% \anm{They claim that they have cost recovery, but our numerical results show that this is not true.} \nate{Need to detail/highlight contributions.}

% Stochastic market clearing methods have gained attention recently to address the concern of greater renewable penetration, as well as potential failure scenarios. These market mechanisms take a two-stage approach, expanding on the ex-post pricing method above by adjusting prices and payments at the time of power delivery. They have been shown to satisfy revenue adequacy and cost recovery in expectation \cite{morales2012pricing,pritchard2010single}, or guarantee either on a per-scenario basis while the other is met in expectation \cite{zakeri2019pricing,morales2014electricity}, or satisfy both on a per-scenario basis, but result in inefficient outcomes \cite{kazempour2018stochastic}.

% % Our proposed \RSCED{} formulation combines this automated reserve procurement with existing \SCED{} formulations that model corrective action. In this, we now have load shed dependent on the realized contingency. This means that the cost associated with load shed has some distribution. One can use average, however this can lead to higher levels of load shed than is necessary and SOs are generally averse to shedding load. On the other hand, robust approaches that avoid load shed can be expensive and there is a tradeoff between cost and reliably providing electricity. For this reason, we model the cost of load shed via the conditional value at risk measure (\CVaR{}). It provides a tunable parameter by which the SO can express their tolerance towards potential load shed.

% % One of the primary advantages of \CVaR{} lies in the fact that under linear costs, it retains linearity. It can thus be expressed as an LP. This allows us to apply off-the-shelf solvers and adopt existing techniques, such as Benders' decomposition, for solving these problems efficiently.

% Another advantage of \CVaR{} is that it is a convex measure. This means that we define meaningful prices. We study two such pricing schemes that have been studied. Authors suggest that one price is better than the other. In this paper, we provide a rigorous proofs on the profitability of these pricing schemes for a SO in clearing the market.

% \anm{Up to here.}

% Decision-making in power systems is performed in advance of the real time. In solving these models, operators face two potential risks: availability of supply in the form of renewable energy and component availability with regard to potential failures. Discrepancies in either of these that manifest in the real time can lead to an inability to supply adequate energy, and if not properly accounted for in forward markets can lead to load shedding or, in extreme cases blackouts. These events can be very costly. To mitigate the impact of these potentially adverse scenarios, system operators (SOs) must explicitly model these risks in forward decision-making.

% They do so by modeling these adverse scenarios explicitly in market clearing formulations, potentially considering topology reconfigurations, in order to ensure reliable delivery of electricity, even in the event of component failures. These techniques of ensuring reliability introduce greater complexity in the model and in the response of SOs in the event of failure, requiring greater levels of automation.

% In the event of failures, SOs may draw from reserve capacity or, in extreme cases, shed load to compensate for changes in the grid topology. These reserve capacities have historically been procured using heuristics, which typically are acquired at far higher capacities than required in order to ensure sufficient resources, while still not being able to provide guarantees that sufficient reserve is available. Instead, in modeling recourse behavior, SOs know how much reserve capacity is required. By incorporating this forward reserve procurement into the model explicitly, SOs can not only guarantee sufficient resources are procured, but also reduce the quantity required at less pivotal nodes, thus increasing efficiency.

% In extreme cases, SOs face the problem of potentially shedding load. While they are averse to doing so, small levels of temporary load shed may prevent the event of large-scale blackouts and should be modeled accordingly. Existing formulations consider the expected cost associated with load shed, however this can lead to higher than necessary levels of load shed, failing to capture the natural aversion SOs have towards any potential load shed.

% \RSCED{} addresses both of these concerns, by providing automated reserve procurement and explicitly modeling potential load shed. Naturally, SOs are averse to shedding any load, which needs to be captured appropriately in the formulation. We do so by modeling it using the conditional value at risk (\CVaR{}) measure. This, like other risk measures, allows a SO to express a tolerance towards extreme values. In this case, high levels of load shed. Simply by adjusting a parameter, they can specify how heavily to penalize high levels of load shed.

% The main reason for using \CVaR{} over other risk measures is that it can be efficiently optimized over. It is a convex risk measure and admits a variational form that makes it easy to optimize over, because minimization over \CVaR{} can be accomplished without explicitly evaluating the measure, and the expectation over a discrete set with a fixed probability distribution can be expressed as a simple sum. Thus, \CVaR{} allows us to express \RSCED{} as a large-scale LP similar to existing \SCED{} formulations. This allows the use of the Benders' decomposition algorithm that is popular for such applications, and can be used to efficiently solve these problems for large-scale problems.

% \anm{The following is the sentiment I want to convey but hasn't been incorporated.}

% After failure, SOs must return to N-1 reliability within 30 minutes. The additional complexity of dynamic topology reconfiguration requires solution of a large-scale MIP. Efficiently solving the underlying (fixed-integer) problem is integral to ensuring this security. Approaches that are LPs can be solved efficiently. Hence the need for \RSCED{}.

% \anm{Ignore below.}

% The design of a robust and resilient grid is a challenge that has been studied extensively. \anm{Needs a better intro.} The effect of potential failures can have dramatic consequences and historically, single line failures leading to cascading failures has been the predominant cause of large-scale blackouts in the United States. Such challenges will only continue to grow as the climate crisis increases the frequency of extreme weather events, further stressing grid components and resulting in potential outages. It is imperative that system operators (SOs) explicitly model failure scenarios to mitigate their impact on grid operations and ensure that such large-scale blackouts do not occur.

% Significant interest has been devoted to such problems, specifically to enforce the \emph{$N - 1$ security criterion}, that requires SOs to ensure that the grid is resilient to any single component failure. They do so by solving a \emph{security-constrained economic dispatch} (\SCED{}) problem. These seek to minimize some dispatch cost, while ensuring that it satisfies grid constraints such as Kirchoff's laws, line flow limits, and generation capacity limits both under nominal operation and in the event of potential failures.

% Various formulations of \SCED{} exist within the literature. Perhaps the simplest is preventive-\SCED{} (\PSCED{}), that simply restricts line flows in the post-failure scenario, as well as maintaining nominal network constraints. Corrective-\SCED{} (\CSCED{}) expands on this by allowing for temporary line flow relaxation and allows SOs to respond to the failure through recourse actions such as modifying the dispatch of fast-ramping generators or drawing from spinning reserve capacity. While traditionally such formulations do not consider the cost associated with such recourse action, authors consider the expected cost of these actions and have even allowed for potential load shed. Such minimization of the expected cost of shedding load does not encode the SO's aversion to potential load shed and can lead to higher levels than are necessary, and must be accounted for accordingly.

% One of the primary difficulties of allowing for generator re-dispatching is ensuring adequate reserve supply from which to draw. We adopt the convention of ISO-NE, who considers reserves from three different categories: spinning reserve, non-spinning reserve, and operating reserve. The former two are expected to respond within 10 minutes, while the latter is expected to respond within 30. These capacities are typically procured based on  heuristic rules. In order to ensure that adequate resources are procured, this requires large overestimates, while still not providing provable guarantees of sufficiency.

% We take the view of authors in \addcite who advocate for explicitly incorporating reserve procurement, specifically spinning and non-spinning reserve, in the solution of \CSCED{} and similar formulations. While the primary motivation is to ensure sufficient resources are procured, an added benefit of such an approach is the total cost of procuring reserve is minimized, resulting in more efficient grid operations for SOs.

% Our proposed, risk-sensitive \SCED{} (\RSCED{}), approach expands on the general \CSCED{} formulation, by explicitly including reserve procurement, as well as modeling the cost associated with load shed via the popular conditional value at risk (\CVaR{}) measure. This measure allows SOs to express a tolerance towards adverse scenarios. To illustrate, consider a random variable with a smooth cumulative distribution function. Over such a random variable, \CVaR{} measures the average over the $(1 - \alpha)$ fraction of worst-case scenarios, for some parameter $\alpha \in [0, 1)$. By adjusting $\alpha$, a SO is able to express their tolerance, with $\alpha = 0$ corresponding to expectation and $\alpha \uparrow 1$ to the essential supremum. This risk measure has found popularity in financial literature, and more recently in power systems applications \addcite due to its ability to capture this risk-aversion and the ability to design efficient algorithms around it.

% We additionally study the problem of market design for the \RSCED{} and \CSCED{} formulations. Two pricing schemes have been considered for these problems. The first, which we refer to as the \emph{locational marginal price of the nominal} (\LMPnom{}), establishes prices based solely the nominal dispatch, ignoring the impact of the security constraints. Advocates take the view that prices should not reflect unrealized instances. Authors in \addcite suggest that such prices are inadequate to capture the costs borne by SOs, proposing instead to use, what we refer to as the \emph{locational marginal price under security} (\LMPmar{}). Such prices derive their values from the marginal cost of \RSCED{} (or \CSCED{}).

% In the following, we will study the market design under both \LMPnom{} and \LMPmar{}. Specifically extending results from the \ED{} problem on revenue adequacy and the quantity of payments required to make sure generators are dis-incentivized from deviating from their prescribed dispatch. We will demonstrate that \LMPmar{} ensures that SOs are able to operate with non-negative net revenue. Such a result is integral to ensuring sustained operation of SOs without additional support. Furthermore, the profits can be used to support grid maintenance and ancillary services. To our knowledge, we are the first to provide concrete results on revenue adequacy of \LMPmar{} as well as establishing this property, including out-of-market settlements.

% % understanding the profitability of the SO in clearing the market. For example, we seek to understand whether SOs are revenue adequate, that is are the profits from clearing the market non-negative. Such a property is impot

% % Cascading failures has historically been one of the most significant causes of large-scale blackouts in the United States. This will only be further exacerbated by the climate crisis and the increasing frequency of extreme weather events that it forebodes. In order to mitigate the impact of these events, system operators (SOs) must explicitly account for these potential failure scenarios, or contingencies, in making dispatch decisions. 

% % SOs do so by augmenting the economic dispatch (\ED{}) problem, which seeks to minimize dispatch cost subject to engineering constraints of the grid, with additional security-constraints. Such formulations are called \emph{security-constrained \ED{}} (\SCED{}) problems. One method of ensuring security is through \emph{preventive}-\SCED{} (\PSCED{}), that ensures that the line flows remain within their ratings after the flows rebalance following the failure \cite{alsac1974optimal}. The solutions are often conservative, and are relaxed by allowing generators to take corrective action, in \emph{corrective}-\SCED{} (\CSCED{}) formulations. Such problems typically allow for generator redispatching within procured reserve capacity and temporarily relax line flow capacity limits during this adjustment period. This temporary relaxation takes advantage of the thermal capacitance of the lines to produce less conservative solutions, while guaranteeing security. Typically, \CSCED{} formulations do not consider the cost of reserve or re-dispatching nor do they allow for any potential load shed \cite{monticelli1987security,capitanescu2007improving,li2016adaptive,capitanescu2011state}. This can lead to high nominal costs of generation and in some cases, infeasibility. On the other hand, \cite{bouffard2008stochastic} allows for potential load shed, whose cost is modeled via the \emph{value of lost load}, and minimizes the total expected cost, including re-dispatching. The resulting dispatch can lead to higher levels of load shed than necessary and does not capture the SOs' inherent aversion to potential load shed.

% % Our contribution is a risk-sensitive security-constrained economic dispatch (\RSCED{}) formulation which makes use of the conditional value at risk (\CVaR{}) measure. For a random variable with a smooth cumulative distribution function, it measures the average over the $(1 - \alpha)$ fraction of worst case scenarios. As the cost associated with load shed is typically much higher than the cost of generation, minimizing the \CVaR{} of the cost with higher risk parameters $\alpha$ would result in lower levels of load shed. We note that the \CVaR{} measure has been widely used in financial literature, and more recently has garnered attention for applications in power systems due to its amenability to optimization and useful mathematical properties.

% % We additionally, consider the problem of reserve procurement explicitly in the \SCED{} formulation. Reserve procurement is often performed using heuristic rules, which provide no guarantees of sufficiency and can lead to significantly higher costs. In \cite{galiana2005scheduling}, the authors note that such rules will prove poorer as greater uncertainty is introduced in the form of renewable generation. Instead, they highlight the need to procure reserves jointly with the nominal dispatch. We adopt this approach of \emph{automated reserve procurement} to ensure that adequate reserve capacity is procured at minimal cost.

% % Establishing meaningful prices has proved for \SCED{} formulations has proved to be a difficult task. \addcite notes that prices based solely on the nominal cost of generation provide insufficient funds for under security, suggesting without proof that certain conditions such as revenue adequacy are difficult to establish. Authors in \cite{galiana2005scheduling} consider a pricing scheme that evaluates to the locational \emph{marginal cost of energy}, including security constraints, sum the \emph{marginal cost of security}, measured as the cost associated with the security constraints. This is able to price both the nominal dispatch, as well as reserve services.

% % Furthermore, \SCED{} formulations consider two time-periods: the nominal dispatch and the reserve or redispatching. Such problems do not admit a competitive equilibrium \addcite, specifically the condition of individual rationality. That is, generators under \RSCED{}, or \CSCED{}, may have incentive to deviate from their prescribed dispatch. This is due to cost incentives of participating solely in a single stage, producing inefficient solutions. In order to ensure that the dispatch is maintained, generators must be compensated for their \emph{lost opportunity cost}, or the difference between the revenue of their desired dispatch and the prescribed. This is similar to uplift payments in the unit commitment problem. Ideally, such payments should be kept to a minimum.

% % We study two different pricing schemes, the first considers prices based solely on the nominal dispatch, which we refer to as the \emph{location marginal price of the nominal} (\LMPnom{}), and the marginal cost of demand, or the \emph{locational marignal price under security} (\LMPmar{}). We demonstrate that \LMPmar{} has several useful properties, including revenue adequacy and low lost opportunity cost payments, as compared to \LMPnom{}.

% % These decisions must satisfy the \emph{$N-1$ security criterion}, which ensures that the system remains stable in the event of any single component failure. In order to satisfy this criterion, SOs augment the economic dispatch problem, which canonically tries to minimize the procurement cost of energy subject to physical constraints of the grid, such as line capacity limits, generation capacity limits, and Kirchoff's laws, with additional \emph{security-constraints} to ensure reliability of the in the event of a component failure. \ED{} problems augmented with these constraints are referred to as security-constrained economic dispatch (\SCED{}) problems. The definition of these problems highlights two competing goals of the SO: minimize cost of procurement and maintain reliability against potential failures.

% % A variety of \SCED{} formulations exist in the literature, the simplest of which is preventive SCED (\PSCED{}). This formulation ensures that the system satisfies the nominal system constraints in the event of any potential failure \cite{alsac1974optimal}. These formulations tend to be overly conservative resulting in high procurement costs. Corrective \SCED{} (\CSCED{}) formulations have been proposed to address this concern, by allowing for the temporary relaxation of line flow capacity limits and allowing SOs to take corrective actions, such as re-dispatching fast ramping generators or shedding load \cite{monticelli1987security,capitanescu2007improving,li2016adaptive,capitanescu2011state}. \CSCED{} formulations typically do not allow for any potential load shed and neglect the costs associated with re-dispatching. These considerations have been included in a recent work \cite{bouffard2008stochastic}, in which the authors associate probabilities of failures to each failure, assign a cost to load shed through the value of lost load, and minimize the expected cost of generation. In minimizing the expected cost of generation, this formulation can lead to higher levels of potential load shed than are necessary. SOs, however, are generally averse to any potential load shed. This is captured in the risk-sensitive variant of the \SCED{} problem in \addcite, which models the uncertain load shed with the conditional value at risk (\CVaR{}) measure.

% % The conditional value at risk is a popular risk measure in financial literature \addcite, and has recently gained attention for power systems applications \addcite. \CVaR{}, parameterized by $\alpha$, measures the average over the $(1 - \alpha)$ fraction of worst case scenarios. For a random variable $\mathpzc{x}$ with a continuous cumulative distribution function, $F$, this is given by 
% % \begin{equation}
% %     \CVaR{}_{\alpha}[\mathpzc{x}] := \E\left[ \mathpzc{x} \mid \mathpzc{x} \ge F^{-1}\left( \alpha\right) \right].
% % \end{equation}
% % The risk aversion parameter $\alpha$ provides a tunable parameter to express the level of aversion towards adverse scenarios (ex. high costs or constraint violation).  For example, $\CVaR{}_0[\mathpzc{x}] = \E[\mathpzc{x}]$, and $\lim_{\alpha \uparrow 1} \CVaR{}_{\alpha}[\mathpzc{x}] = {\ess\sup} \; {\mathpzc{x}}$. \CVaR{} also shares a close relationship with chance-constraints, which have found widespread appeal in power systems applications \addcite. Such constraints seek to ensure that the random variable is bounded by some value with a certain probability. Without loss of generality, assume an upper bound of $0$, for which the chance-constraint is given by
% % \begin{equation}
% %     \mathbb{P}\left[\mathpzc{x} \le 0\right] \ge 1 - \alpha.
% % \end{equation}
% % Notice that $\CVaR{}_{\alpha}[\mathpzc{x}] \le 0 \implies \mathbb{P}\left[\mathpzc{x} \le 0\right] \ge 1 - \alpha$. In other words, \CVaR{} is a convex inner approximation of chance-constraints. Additionally, \CVaR{} also limits the extent of violation by taking the expectation over all adverse scenarios. We augment the \CSCED{} problem with potential load shed and minimize the random cost of generation under \CVaR{} in our proposed risk-sensitive \SCED{} (\RSCED{}) formulation.

% The \RSCED{} formulation, like other \CSCED{} formulations, must include additional variables and constraints due to each realizable contingency. This results in large problem descriptions for practical power networks, that are typically too large to solve within the time periods required for power system operation. In order to deal with this challenge, contingency pre-filtering has been suggested to reduce the number of contingencies included in the problem and thereby reduce its dimension; see \cite{capitanescu2007contingency} for a survey of such methods. In this work, we apply the popular Benders' decomposition method, that has been applied to \CSCED{} formulations in \cite{li2016adaptive} \addcite. Such decomposition-based approaches are easily amenable to parallelization.

\revision{Our proposed \RSCED{} formulation can be used within existing electricity market operations in a variety of contexts. For example, one can use \RSCED{} to clear the 5-15 minute ahead real-time market, for which our analyzed pricing schemes define the  settlements for the market participants. \RSCED{} can also be used within a residual unit commitment module that SOs often run an hour to several hours in advance of power delivery. An enhanced multi-period version of the same with inter-temporal constraints can be used within a security-constrained unit commitment and economic dispatch module to clear the day-ahead market within organized wholesale market structures. One can theoretically use our framework to address long-term planning or conceive a capacity market design with additional work. However, as it will become clear, the size of \RSCED{} grows with the number of scenarios considered, which can be prohibitively large to capture operating points across planning horizons.}

\revision{In this paper, we concretely focus on single line failures as possible contingencies throughout. Our design principle and the formulation technically does not require that we only model single line failures. A power system in practice has a large number of components; one can create a contingency scenario by considering the failure of any possible subset among these components, and construct a finite collection of such scenarios. Our design philosophy will apply to this variant, where the optimization problem will seek to compute a nominal dispatch and reserve procurement in a way that penalizes the risk-based cost of said procurement and possible recourse actions within these scenarios, given operational constraints.}

\revision{In our analysis of the settlements, we narrowly focus on market-relevant properties such as individual rationality and revenue-adequacy. Organized wholesale market settlements are complex; their payments can depend on considerations that we do not explicitly model. However, our analysis is premised on the belief that, if a market settlement scheme, sans those complications, is theoretically favored to another based on the market-relevant properties we analyze, then we expect that design to fare better in practice with additional considerations. As will become evident, \LMPmar{} outperforms \LMPnom{} in the metrics we consider, and hence, defines a more sound pricing scheme from a theoretical standpoint. Said succinctly, \LMPmar{} explicitly accounts for the marginal costs of recourse actions in contingencies, while \LMPnom{} ignores them. As the power system grows in complexity, and the range of uncertainty increases, ignoring components due to responses to these uncertain scenarios in prices offered to the generators might not reflect the true value of these resources within the system. 
% While our focus in this work lies solely on the mathematical analysis of these candidate prices, the insights reveal distinct advantages of utilizing \LMPmar{} over \LMPnom{}.
}

The paper is organized as follows. We begin by presenting established results on the deterministic economic dispatch problem, LMP-based market design, and properties of such markets in Section \ref{sec:ED}. We then formulate the \RSCED{} problem in Section \ref{sec:rsced}, contextualizing it within existing work. In Section \ref{sec:pricing}, we study the market design under \LMPnom{} and \LMPmar{}, providing rigorous proofs of the revenue achieved in clearing the market. We present, in Section \ref{sec:algorithm}, a Benders' decomposition algorithm to solve \RSCED{} efficiently at scale. Finally, Section \ref{sec:simulation_results} describes simulation results of the \RSCED{} formulation on the IEEE 24-bus RTS network, with a scalability demonstration of the Benders' decomposition approach for a large array of networks. \revision{We introduce notation where used, but key symbols are included in Table \ref{tab:symbols}.}

% \bose{Major symbols to introduce}
{\small
\begin{table}[htbp]
\caption{Major symbols}\label{tab:symbols}
\begin{center}
\vspace{-1em}
\begin{tabular}{ll}
\hline
$\v{g}$& Nominal nodal generation dispatch\\
$\ul{\v{r}},\,\ol{\v{r}}$& Negative and positive nodal reserve capacity\\
$\v{\delta g}_k$&Nodal generator re-dispatch in contingency $k$\\
$\v{\delta d}_k$&Nodal load shed in contingency $k$\\
$\alpha$&Tunable risk parameter\\
$\MS{}$& Merchandising surplus\\
$\LOC{}$& Lost opportunity cost\\
$\v{v}$&Value of lost load (VoLL)\\
$\v{\pi}^{\N}$&Locational marginal prices of the nominal\\
$\v{\pi}^{\S}$&Locational marginal prices under security\\
\hline
\end{tabular}
\end{center}
% \label{tab:symbols}
\end{table}
}