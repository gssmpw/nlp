% \section{The deterministic economic dispatch and its market design}
\section{Preliminaries}
\label{sec:ED}

% SOs are tasked with solving forward dispatch problems at various intervals before the real-time in order to make dispatch decisions. 
SOs are tasked with making a series of forward dispatch decisions prior to real-time demand fulfillment. 
They do so by solving \emph{economic dispatch} (ED) problems, which seek to minimize the cost of energy procurement, or dispatch cost, subject to engineering constraints of the grid \cite{varaiya2010smart}. We begin by studying the deterministic economic dispatch problem, without modeling potential component failures or recourse action, in order to identify desirable properties characterizing its outcomes to be extended to its risk-sensitive variants. We refer to the works in \cite{varaiya2010smart,ma2009security,rajagopal2013risk} for further details on existing economic dispatch formulations and practices. 

Consider an $n$-bus power network, connected by $\ell$ lines. Throughout the paper, for a positive integer $m$, we use $[m]:=\{1,\dots,m\}$. 
% to denote the set of positive integers from 1 to $m$ inclusive. 
Located at each bus are both demand and generation, the vector values of which are denoted by $\v{d} \in \mathbb{R}^n$ and $\v{g} \in \mathbb{R}^n$. Assume that generator $i$ has a linear cost $c_i$, collected by the vector $\v{c}$,
% \footnote{Such cost structures are able to represent piecewise-affine costs of generation that are frequently adopted in such problems.}, 
and that power production is limited to the set $\Gcal := [\underline{\v{g}}, \overline{\v{g}}]$. The total cost of generation is then given by $\v{c}^\top \v{g}$. We note that piecewise affine costs may be adopted as well without introducing additional conceptual difficulties. Let $\v{f} \in \mathbb{R}^{2\ell}$ denote the vector of (directed) line flow limits. We adopt the DC power flow approximation, which assumes lossless lines, negligible reactive power, and small voltage phase angle differences. Such assumptions allow us to represent the nonlinear constraints dictated by  Kirchhoff's laws as linear constraints. Specifically, let $\v{H} \in \mathbb{R}^{2 \ell \times n}$ be the injection-shift factor (ISF) matrix mapping the injections to (directed) line flows, and $\bone$ denote the vector of all ones of appropriate size. Then, the ED problem takes the form
% we may formally state the ED problem as
\begin{subequations}
\begin{align}
    \text{(ED)}\,\,\,&\underset{\v{g} \in \Gcal}{\text{minimize}} && \v{c}^\top \v{g}, \label{eq:ED.obj} \\
    &\text{subject to} && \bone^\top \left(\v{g} - \v{d}\right) = 0, \quad \v{H} \left( \v{g} - \v{d} \right) \le \v{f}. \label{eq:ED.cons}
\end{align}
\label{eq:ED}
\end{subequations}
% In order to design a market for this problem, we 
In order to design a compensation structure,
assign dual multipliers $\lambda \in \mathbb{R}$ and $\v{\mu} \in \mathbb{R}^{2\ell}$ to each of the constraints in \eqref{eq:ED.cons}. Let $(\v{g}^\star, \lambda^\star, \v{\mu}^\star)$ be an optimal primal-dual solution to the ED problem \eqref{eq:ED}. Then the \emph{nodal prices} or \emph{locational marginal prices} (LMPs) are defined as,
\begin{equation}\label{eq:ED_LMP}
    \v{\pi}^{\ED} := \lambda^\star \bone - \v{H}^\top \v{\mu}^\star.
\end{equation}
These prices are known to satisfy a number of useful properties. Notably, this LMP is the marginal sensitivity of the optimal cost to demand. That is, a unit increase in demand at node $i$ would result in a corresponding increase in the total procurement cost of magnitude $\pi^{\ED}_i$.

In clearing the market, SOs accept payments $\pi_i d_i$ from each consumer $i\in[n]$ and pay $\pi_ig^{\star}_i$ to each supplier $i\in[n]$ for the quantities procured and provided, respectively. 
% \sout{The sum of payments need not be identical,} 
The summed consumer payments need not equal the summed payments made to suppliers, i.e., SOs may operate at a profit or loss. This difference is captured by the \emph{merchandising surplus} (\MS{}). 
% \sout{defined as the difference between the total payments by consumers and the total payments to suppliers. With prices given by $\v{\pi}$, this is given by} 
Given nodal prices $\v{\pi}^{\ED}$, the merchandising surplus is 
\begin{equation}\label{eq:MS}
    \MS{}[\v{\pi}^{\ED}] := (\v{\pi}^{\ED})^\top (\v{d} - \v{g}^\star),
\end{equation}
It can be shown that $\MS{}[\v{\pi}^{\ED}] \ge 0$. 
% \sout{It can be shown that for a solution $(\v{g}^\star, \lambda^\star, \v{\mu}^\star)$ of \eqref{eq:ED}, the merchandising surplus is \emph{revenue adequate}, that is} 
% It can be shown that the merchandising surplus associated with a solution $(\v{g}^\star, \lambda^\star, \v{\mu}^\star)$ of \eqref{eq:ED} is \emph{revenue adequate}, i.e., 
% \begin{equation}\nonumber
%     \MS{}[\v{\pi}^{\ED}] \ge 0.
% \end{equation}
% \nate{Make \MS{} a function of prices, demand, supply?}
Such a market is called \emph{revenue adequate}, which guarantees that SOs do not run cash-negative from operating a market. Moreover, in the case of strictly positive merchandising surplus, additional revenue is disseminated through financial derivatives such as financial transmission rights. 
% Under revenue adequacy, SOs are guaranteed to avoid operating at a loss, ensuring that adequate resources are available to support the market. Moreover, in the case of strictly positive merchandising surplus, additional revenue can be appropriated towards grid maintenance or facilitating transactions in ancillary markets.

Assuming price-taking behavior amongst market participants, LMP-based market clearing satisfies the condition of \emph{individual rationality}. That is, for any $i\in[n]$ and corresponding LMP $\pi^{\ED}_i$, the SO prescribed dispatch $g^\star_i$ maximizes the profit of generator $i$, or
\begin{equation}\nonumber
    g_i^\star \in \underset{\hat{g}_i \in \mathcal{G}_i}{\arg\max} \; (\pi^{\ED}_i - c_i) \hat{g}_i,
\end{equation}
where $\mathcal{G}_i$ denotes the interval $[\ul{g}_i,\ol{g}_i]$. Thus, under individually rational prices, generators have no incentive to deviate from the prescribed dispatch. Failing to enforce this property obligates SOs to compensate generators for any differences between the prescribed and profit-maximizing dispatch. Such a concern arises elsewhere in power system operations, e.g., in the unit commitment problem, where generators are paid \emph{uplift payments} in compensation for the costs associated with startup and shutdown \cite{yang2019unified}. Payments of this sort are problematic in that they are out-of-market settlements, whose costs are not reflected in objective functions representing cost minimization or social welfare maximization.