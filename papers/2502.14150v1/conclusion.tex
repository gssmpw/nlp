%!TEX root = ./mainRSCED.tex

% \begin{figure*}[!t]
\framebox[\linewidth][c]{
    \begin{minipage}{0.98\linewidth}
    \begin{itemize}
        \small
        \item Primal feasibility conditions: 
        % \eqref{eq:RSCED.nominal} --- \eqref{eq:RSCED.se.bounds}, 
        \eqref{eq:RSCED.lp.nominal} --- 
        \eqref{eq::RSCED.lp.epigraph}.
        \item Dual feasibility conditions: $\v{\mu}, \underline{\v{\gamma}}, \overline{\v{\gamma}}, \v{\mu}_k^\DA{}, \v{\mu}_k^\SE{}, \underline{\v{\gamma}}_k, \overline{\v{\gamma}}_k, \underline{\v{\rho}}_k, \overline{\v{\rho}}_k, \underline{\v{\eta}}, \overline{\v{\eta}}, \underline{\v{\sigma}}_k, \overline{\v{\sigma}}_k, \underline{\nu}_k, \overline{\nu}_k \ge 0$ for all $k \in[K]$.
        \item 
        % \nate{32b should not have sum over $k$, negative sign issue with $\ul{\v{r}}$} 
        Stationarity conditions: For $k \in[K]$,
        \begin{subequations}
        \begin{gather}
          \v{c} - \lambda \bone + \v{H}^\top \v{\mu} - \underline{\v{\gamma}} + \overline{\v{\gamma}} + \sum_{k = 1}^K \left(\v{H}_k^\top (\v{\mu}_k^\DA{} + \v{\mu}_k^\SE{}) - \underline{\v{\gamma}}_k + \overline{\v{\gamma}}_k\right) = 0, \;
          \underline{\v{c}}_r - \underline{\v{\eta}} - \sum_{k = 1}^K \underline{\v{\rho}}_k  = 0, \;
          \overline{\v{c}}_r - \overline{\v{\eta}} - \sum_{k = 1}^K \overline{\v{\rho}}_k = 0, \label{eq:rsced.kkt.stat.gr} \\
          -\lambda_k \bone + \v{H}_k^\top \v{\mu}_k^\SE{} - \underline{\v{\gamma}}_k + \overline{\v{\gamma}}_k + \overline{\v{\rho}}_k - \underline{\v{\rho}}_k = 0, \;
          -\lambda_k \bone + \v{H}_k^\top \v{\mu}_k^\SE{} - \underline{\v{\sigma}}_k + \overline{\v{\sigma}}_k + \overline{\nu}_k \voll = 0, \label{eq:rsced.kkt.stat.d} \\
          1 - \sum_{k = 1}^K \overline{\nu}_k = 0, \;
          \frac{1}{1 - \alpha} p_k - \underline{\nu}_k - \overline{\nu}_k = 0. \label{eq:rsced.stat.cvar}
        \end{gather}
        \end{subequations}
        \item Complementary slackness conditions: 
        \begin{subequations}
        \begin{gather}
            \v{\mu}^\top [\v{H}(\v{g} - \v{d}) - \v{f}] = 0, \; \underline{\v{\gamma}}^\top [ \underline{\v{g}} - \v{g}] = 0, \;
            \overline{\v{\gamma}}^\top [\v{g} - \overline{\v{g}} ] = 0, \label{eq:rsced.kkt.cs.g} \\
            [\v{\mu}_k^\DA{}]^\top [\v{H}_k (\v{g} - \v{d}) - \v{f}_k^\DA{}] = 0, \;
            [\v{\mu}_k^\SE{}]^\top [\v{H}_k(\v{g} + \v{\delta g}_k - \v{d} + \v{\delta d}_k) - \v{f}_k^\SE{}] = 0, \label{eq:rsced.kkt.cs.flow} \\
            \underline{\v{\gamma}}_k^\top [ \underline{\v{g}} - \v{g} - \v{\delta g}_k] = 0, \;
            \overline{\v{\gamma}}_k^\top [ \v{g} + \v{\delta g}_k - \overline{\v{g}} ] = 0, \;
            \underline{\v{\rho}}_k^\top [ -\underline{\v{r}} - \v{\delta g}_k ] = 0, \;
            \overline{\v{\rho}}_k^\top [\v{\delta g}_k - \overline{\v{r}}] = 0, \label{eq:rsced.kkt.cs.dg} \\
            \underline{\v{\eta}}^\top \underline{\v{r}} = 0, \;
            \overline{\v{\eta}}^\top \overline{\v{r}} = 0, \; 
            \underline{\v{\sigma}}_k^\top \v{\delta d}_k = 0, \;
            \overline{\v{\sigma}}_k^\top [ \v{\delta d}_k - \v{\Delta}_d ] = 0, \; \underline{\nu}_k y_k = 0, \;
            \overline{\nu}_k [ y_k + z - \v{\voll}^\top \v{\delta d}_k ] = 0. \label{eq:rsced.kkt.cs.rdcvar}
        \end{gather}
        \end{subequations}
    \end{itemize}
    \end{minipage}
}
\caption{The KKT conditions for \eqref{eq:RSCED.lp}.}
\label{fig:rsced.kkt}
\end{figure*}

\section{Conclusion} % (fold)
\label{sec:conclusion}

% e\rev{pull up with vspace} 
% \vspace{-0.1in}
In this paper, we studied a risk-sensitive security-constrained economic dispatch formulation that allows an SO to trade off between the nominal dispatch cost and reliability of meeting demand. \rev{In this risk-sensitive context}, we studied two pricing mechanisms, one based solely on the nominal dispatch and another based on the total marginal cost of demand. We established results on revenue adequacy, and evaluated the lost opportunity cost payments under either scheme, \rev{closing a gap in understanding of market properties of these mechanisms under risk aversion. Our results on pricing policies apply to existing \CSCED{} formulations as well. Despite offering a generalization over prior approaches in the inclusion of an SO's risk attitudes, our formulation remains amenable to solution via decomposition techniques such as Benders'.} We studied the effect of risk aversion on the \RSCED{} dispatch and the pricing schemes in an IEEE 24-bus RTS network, and demonstrated the utility of Benders' decomposition in reducing solution time of \RSCED{} over large power networks. 


\revision{While we concretely presented a framework to optimize energy and reserve procurement under single line failure scenarios, our risk-sensitive design framework extends to any forward market context that must factor in possible contingencies, their probabilities, the available recourse actions in said contingencies, and their associated costs. Such a framework will allow the system to successfully respond to scenarios that might arise during real-time operations.
Our analysis revealed that a pricing mechanism that included components due to costs associated with recourse decisions in such contingencies fared better than another that ignored them. These components help the market settlements better reflect the value of the response to the uncertain scenarios. As the power system becomes increasingly complex, the number of possible scenarios will undoubtedly increase, growing the computational burden on the market clearing mechanism. A judicious scenario screening procedure will then play a central role in the efficacy of the overall market design. Studying the result of explicit incorporation of unit commitment decisions and scenario selection procedures in \RSCED{} remains an interesting direction for future endeavors.}

% forward market design, even though we has selected single-line failures, multiple lines 
% general design principle and settlement mechanism
% one pricing is better than other can be rigorously shown and we expect this result to extend 
% as systems become more complex, identifying key scenarios via scenario selection procedure 
% section conclusion (end)