%!TEX root = ./mainRSCED.tex

\section{Pricing \RSCED{}-based Dispatch}
\label{sec:pricing}

% In the following we establish various results for the \RSCED{} problem for two different pricing mechanisms. We highlight the fact that the \RSCED{} problem is a generalization of the \CSCED{} problem minimizing the total expected cost of generation including recourse actions and the results defined hold for that problem as well. 

Pricing \SCED{} problems is a well-known challenge \cite{shi2022scenario}. Desirable properties of LMP-based pricing for the ED problem, such as revenue adequacy for SOs and individual rationality for generators cannot be easily extended to security-constrained counterparts. One way to extend such prices to SCED problems is to establish LMP-style prices based solely upon nominal constraints, an approach known as \LMPnom{}. 
% In recent work, \cite{hogan2013electricity}, authors have challenged this method,
As argued in, e.g., \cite{hogan2013electricity}, 
% , which establishes prices based only on the nominal constraints in the absence of outages, 
% positing that 
\LMPnom{} based markets often result in insufficient revenue for appropriate grid maintenance and require large out-of-market settlements to ensure that generators follow the SO's dispatch signals. 
% to ensure that dispatches are maintained. 

Alternatively, pricing based on the 
% \emph{nodal marginal price of electricity} 
\emph{locational marginal price under security} (\LMPmar{})
has been proposed in \cite{galiana2005scheduling}. This model incorporates anticipated congestion across all credible contingencies into nodal prices. In this section, we will study \LMPnom{} and \LMPmar{}, establishing results on revenue adequacy and evaluating the out-of-market settlements required to ensure individual rationality amongst generators. \rev{To the best of our knowledge, such results thus far have been limited to approaches that optimize expected costs of recourse actions, or exclude recourse altogether. We emphasize that the results we present apply to the existing \CSCED{} formulation described in \eqref{eq:csced} as well our \RSCED{} formulation.} 

We use the superscript $\star$ in this section to denote an element of a primal-dual optimal solution of the \RSCED{} problem \eqref{eq:RSCED.lp}. See Appendix \ref{app:proof.thm.ms} for the complete Karush-Kuhn-Tucker (KKT) optimality conditions for problem \eqref{eq:RSCED.lp}. 

\begin{definition}
  The locational marginal prices of the nominal (\LMPnom{}) are defined as
  \begin{equation}
      \v{\pi}^{\N} := \lambda^\star \bone - \v{H}^\top \v{\mu}^\star.
      \label{eq:price.nominal}
  \end{equation}
\end{definition}
% \sout{The network-wide cost of generation is captured by $\lambda^\star$, while $\v{\mu}^\star$ captures the congestion rent in the nominal case.} 
In \eqref{eq:price.nominal}, $\lambda^\star$ captures the network energy price, while $\v{H}^{\top}\v{\mu}^{\star}$ captures the congestion prices under nominal conditions. Note that while the form of the LMPs \eqref{eq:price.nominal} is the same as in the standard ED case given in \eqref{eq:ED_LMP}, the Lagrange multipliers arise from different optimization problems, i.e., \eqref{eq:RSCED.lp} and \eqref{eq:ED}, respectively. 
% \sout{It, however, ignores the congestion rent associated with any potential contingencies. As noted in \cite{hogan2013electricity}, this can lead to dispatches from generators that deviate significantly from their profit-maximization strategies and, as we will see, fail to guarantee revenue adequacy for the SO.} 
As noted in \cite{hogan2013electricity}, in ignoring congestion costs due to potential contingencies, the \LMPnom{} approach can lead to realized generator dispatches that deviate significantly from their profit-maximization strategies and, as we will see, fail to guarantee revenue adequacy for the SO. 

\begin{definition}
  The locational marginal prices under security (\LMPmar{}) are defined as
  \begin{equation}
    \v{\pi}^{\S} := \lambda^{\star} \bone - \v{H}^\top \v{\mu}^\star - \sum_{k = 1}^K \v{H}_k^\top (\v{\mu}_k^{\DA{}\star} + \v{\mu}_k^{\SE\star}).
    \label{eq:price.marginal}
  \end{equation}
\end{definition}
Unlike the \LMPnom{} pricing, this model explicitly incorporates an additional cost associated with potential recourse actions under the contingency scenarios. \revision{In a sense, \LMPmar{} adds contingency-specific price components to \LMPnom{}.}
The payment in \eqref{eq:price.marginal} is further justified by the fact that akin to the deterministic case, entries of $\v{\pi}^{\S}$
 measure the marginal sensitivity of the optimal cost in \eqref{eq:RSCED.lp} to demand at each node. 

Let $\v{\pi}\in\{\v{\pi}^{\N},\v{\pi}^{\S}\}$. Under either of these pricing schemes, we define the payment made by the demander at bus $i$ as 
$$\Pi^d_i[\v{\pi}] := \pi_i d_i,$$ where $d_i$ is the quantity demanded and $\pi_i$ is the LMP at node $i\in[n]$. Supplier $i\in[n]$ is compensated 
\begin{equation}\label{eq:supp_comp}
    \Pi^g_i[\v{\pi}] = \pi_i g^{\star}_i + \sum_{k=1}^K(\overline{\rho}^{\star}_{k,i}\overline{r}^{\star}_i + \underline{\rho}^{\star}_{k,i} \underline{r}^{\star}_i),
\end{equation} where $g^{\star}_i$ is the quantity produced, $\overline{\rho}^{\star}_{k,i}$ and $\underline{\rho}^{\star}_{k,i}$ are the dual multipliers associated with the reserve capacity requirements in \eqref{eq:RSCED.lp}, and $\ul{r}^{\star}_i$ and $\ol{r}^{\star}_i$ are the reserve capacities procured at node $i\in[n]$. Note that $\overline{\rho}^{\star}_{k,i}$ and $\underline{\rho}^{\star}_{k,i}$ are the marginal costs of reserve capacity at node $i$ in outage scenario $k$. Further, both pricing schemes consist entirely of \emph{ex-ante} payments, i.e., payments are made prior to the realization of a particular scenario and thus do not depend upon the scenario realized. Therefore, the revenue-related results we derive in subsequent sections always apply, rather than in a statistical sense, e.g., in expectation.

\subsection{Revenue Adequacy} % (fold)
\label{sub:revenue_adequacy}

We first address the issue of revenue adequacy under the \LMPnom{} and \LMPmar{} pricing schemes. Specifically, we aim to answer whether the merchandising surplus of the SO is nonnegative with these candidate pricing schemes to accompany the dispatch signals computed from the R-SCED problem \eqref{eq:RSCED.lp}. 
% That is, we study whether the  we wish to establish that the SO, in running the market, is able to ensure that they do not operate at a loss. 
% This is done by evaluating the merchandising surplus $(\MS{})$, or the difference between the total payments from demanders and the total cost of procurement from suppliers, which for pricing scheme $\v{\pi}$ given by
Augmenting the expression in \eqref{eq:MS} with additional compensation for generator reserve procurement, the merchandising surplus for pricing scheme $\v{\pi}$ under \RSCED{} is given by
\begin{equation}\label{eq:MS_RSCED}
\begin{split}
    \MS{}[\v{\pi}] &= \sum_{i=1}^n\Pi^d_i[\v{\pi}] - \sum_{i=1}^n\Pi^g_i[\v{\pi}]\\
    &= \v{\pi}^\top \v{d} - \v{\pi}^\top \v{g}^{\star} - \sum_{k=1}^K(\overline{\v{\rho}}_k^{\star\top} \overline{\v{r}}^{\star} + \underline{\v{\rho}}^{\star\top}_k \underline{\v{r}}^{\star}).
\end{split}
    % \MS{}[\v{\pi}] = \v{\pi}^\top \v{d} - \v{\pi}^\top \v{g} - \sum_{k=1}^K(\overline{\v{\rho}}_k^\top \overline{\v{r}} + \underline{\v{\rho}}^\top_k \underline{\v{r}}).
\end{equation}

Applying the price definitions \eqref{eq:price.nominal} and \eqref{eq:price.marginal} in \eqref{eq:MS_RSCED}, we have the following result.

\begin{theorem}
  \label{thm:ms}~
  % \begin{itemize}
          % \item[(i)] 
          Revenue adequacy is not guaranteed under the \LMPnom{} scheme \eqref{eq:price.nominal}. 
      % \item[(ii)] 
      The \LMPmar{} pricing scheme \eqref{eq:price.marginal}, however, is revenue adequate, i.e., $\MS{}[\v{\pi}^{\S}] \ge 0$.
  % \begin{equation}
  %     \MS{}[\v{\pi}^{\S}] \ge 0.
  % \end{equation}
  % \end{itemize}
\end{theorem}
See Section \ref{sec:simulation_results} for a numerical example demonstrating a case wherein $\MS[\pi^{\N}]<0$. In particular, in Figure \ref{fig:24bus.ms.uplift}, $\MS[\pi^{\N}]<0$ for values of $\alpha$ larger than 0.7.
The proof of the claim regarding S-LMP in Theorem \ref{thm:ms} can be found in Appendix \ref{app:proof.thm.ms}. 
% The implication of this theorem is that pricing under $\v{\pi}^{\S}$ ensures that using \LMPmar{} pricing SOs do not operate at a loss. 
Note that the second result in Theorem \ref{thm:ms} holds for all values of the \CVaR{} parameter $\alpha$, i.e., regardless of risk preference, \rev{including the special case where $\alpha = 0$, i.e., when expected load shed is minimized.}
% \sout{The importance of a positive merchandising surplus extends not only to ensuring continued operation by the SO, but includes resources for grid maintenance and ancillary services.} 
% Aside from allowing for continued operation by the SO, positive merchandising surplus can be used to support grid maintenance and ancillary services. 
% The introduction of security constraints increases the complexity of dispatch decision-making, as the grid must be maintained not only in the nominal case, but must also under line outage scenarios as well. 
As we will see empirically in Section \ref{sec:simulation_results}, while $\MS{}[\v{\pi}^{\S}]$ is nonnegative, it also generally increases with risk-aversion. 
% There is an additional cost associated with being averse to load shed, and the increased \MS{} associated with this can be used to perform grid upgrades to improve reliability accordingly.
In the long run, increased merchandising surplus can be used to perform grid upgrades to improve reliability accordingly. \revision{In effect, in ignoring price components from the various contingencies, \LMPnom{} can fail to collect enough rents from demanders to pay the generators--a scenario that can never arise under \LMPmar{}. Reflecting costs from unrealized scenarios makes \LMPmar{} a better pricing scheme than \LMPnom{} based on revenue adequacy properties. However, such a pricing scheme requires a careful selection of modeled contingencies and their probabilities from the SO that market participants must agree upon.}

\subsection{Individual Rationality and Lost Opportunity Cost Payments}\label{sec:IRandLOC}
We now address the question of individual rationality of generators under the \LMPnom{} and \LMPmar{} pricing schemes. That is, are generators being compensated at the same level as their profit-maximizing dispatch, given the price? While implemented via a single stage of ex-ante transactions, the payment schemes that we have presented here have two parts: payment for nominal generation, and compensation for ancillary, scenario-dependent generation by way of payments for reserve procurement. 
% \nate{Clarify - mechanism is still single-stage/static} 
As a result, defining what it means to maximize profit is not straightforward. In particular, one must determine whether to account for profit associated with reserve capacity.

The \RSCED{} formulation \eqref{eq:RSCED.lp} and generator compensation structure \eqref{eq:supp_comp}, entails simultaneous settlement of two markets, one for nominal case generation, and another for reserve capacity. Setting aside the concern that the co-existence of these markets may create opportunities for arbitrage or the exercise of market power, as described in e.g., \cite{pritchard2010single} and \cite{borenstein2002measuring}, assessment of the profitability of multi-product firms such as the generators we consider is complicated \cite{hirst2000maximizing,buchsbaum2020spillovers}. 
% % Authors in \addcite highlight a 
% concern of considering ancillary resources in profit maximization. 
% Without careful consideration, the inclusion of ancillary resources may undermine market operation in the presence of profit-maximizing generators \cite{pritchard2010single,morales2012pricing}.
% \cite{borenstein2002measuring,buchsbaum2020spillovers}. 
% Specifically, this results in a two time-scale market; 
% The difficulty lies in handling settlements of two markets, one for nominal-case generation, and another for reserve capacity. 
% As pointed out in \cite{pritchard2010single},  
% Due to the typically higher costs associated with reserve capacity deployment, generators are incentivized to supply solely in the reserve market, producing inefficient outcomes. 
Depending upon energy and reserve capacity prices, as well as equipment characteristics and operating costs, generators may be incentivized to supply more heavily in one market over another, potentially producing inefficient outcomes.

Instead, we assume that generators act as \emph{energy-only} profit-maximizers solely in the nominal case, viewing reserve procurement as an additional ``bonus'' sum. In this way, we can restrict attention to settlements for nominal generation. 
% a single time period. 
For the nominal case, if generators are asked to act contrary to their energy-only profit-maximizing dispatch, given the price, they must be paid a \emph{lost opportunity cost} (\LOC{}) payment corresponding to lost revenue due to deviation from their individually optimal dispatch, described below.

The \LOC{} payment for a generator at bus $i$ is calculated as the difference in the energy-only profit from the prescribed dispatch, $(\pi_i - c_i) g_i^\star$, and the energy-only profit-maximizing dispatch with prices derived from the solution of \eqref{eq:RSCED.lp}, given by
\begin{equation}
  \label{eq:ind.rat}
\begin{aligned}
    & \underset{\hat{g}_i}{\text{maximize}} && (\pi_i - c_i) \hat{g}_i, & \text{subject to} && \underline{g}_i \le \hat{g}_i \le \overline{g}_i.
\end{aligned}
\end{equation}
The optimal solution to \eqref{eq:ind.rat}, given price $\pi_i$ is 
  \begin{equation}\label{eq:opt_gen_sol}
      \hat{g}_i^\star := \begin{cases}
        \overline{g}_i & \pi_i \geq c_i, \\
        \underline{g}_i & \pi_i < c_i. 
      \end{cases}
  \end{equation}
Thus, defining the difference between dispatched generation and \eqref{eq:opt_gen_sol} as
\begin{equation}
      \Gamma_i[\v{\pi}] := \begin{cases}
        \overline{g}_i - g_i^\star & \pi_i \ge c_i, \\
        g_i^\star - \underline{g}_i & \pi_i < c_i, 
      \end{cases}
    %   \max\{ & \sign(\pi_i - c_i) (\overline{g}_i - g_i^\star), \\
    %     & \sign(\pi_i - c_i) (\underline{g}_i - g_i^\star) \}.
  \end{equation}
we have that $\LOC{}_i[\v{\pi}]$, the \LOC{} payment to generator $i$ under pricing scheme $\pi$ is given by 
\begin{equation}\label{eq:LOC_i}
      \LOC{}_i[\v{\pi}] = | \pi_i - c_i | \Gamma_i[\v{\pi}]. 
  \end{equation}

% Evaluating this difference for each generator, we arrive at the following proposition. The function $\sign(a)$ returns $1$ if $a > 0$, $-1$ if $a < 0$ and $0$ otherwise. The following proposition describes these \LOC{} payments under both pricing schemes. \nate{Can just give optimal solution to generator problem and take difference - show directly by calculation}

% \begin{proposition}
%   \label{prop:loc}
%   The \LOC{} payments for generator $i$ under a pricing scheme $\v{\pi}$ is given by,
%   \begin{equation}
%       \LOC{}_i[\v{\pi}] = | \pi_i - c_i | \Gamma_i[\v{\pi}],
%   \end{equation}
%   where $\Gamma_i[\v{\pi}]$ is defined as
%   \begin{equation}
%       \Gamma_i[\v{\pi}] := \begin{cases}
%         \overline{g}_i - g_i^\star & \pi_i \ge c_i, \\
%         g_i^\star - \underline{g}_i & \pi_i < c_i.
%       \end{cases}
%     %   \max\{ & \sign(\pi_i - c_i) (\overline{g}_i - g_i^\star), \\
%     %     & \sign(\pi_i - c_i) (\underline{g}_i - g_i^\star) \}.
%   \end{equation}
% \end{proposition}
% The proof of Proposition \ref{prop:loc} is included in Appendix \ref{app:proof.prop.loc}.
Summing \eqref{eq:LOC_i} over $i$ then gives the total \LOC{} payments SOs must pay under a given pricing scheme. Using the KKT condition in \eqref{eq:rsced.kkt.stat.gr}, and price definitions \eqref{eq:price.nominal} and \eqref{eq:price.marginal}, we have 
{\small
\begin{equation}\label{eq:loc_pi_N}
    \begin{split}
        &\LOC{}[\v{\pi}^{\N}]\\
        &= \left| \overline{\v{\gamma}}^{\star} - \underline{\v{\gamma}}^{\star} + \sum_{k = 1}^K \v{H}_k^\top (\v{\mu}_k^{\DA{}\star} + \v{\mu}_k^{\SE{}\star}) + \overline{\v{\gamma}}^{\star}_k - \underline{\v{\gamma}}^{\star}_k \right|^\top \v{\Gamma}[\v{\pi}^{\N}],
    \end{split}
\end{equation}
% and 
% while $\LOC{}[\v{\pi}^{\S}]$ is given by
\begin{equation}\label{eq:loc_pi_S}
    \LOC{}[\v{\pi}^{\S}]=\left| \overline{\v{\gamma}}^{\star} - \underline{\v{\gamma}}^{\star} + \sum_{k = 1}^K \overline{\v{\gamma}}^{\star}_k - \underline{\v{\gamma}}^{\star}_k \right|^\top \v{\Gamma}[\v{\pi}^{\S}].
\end{equation}
}
% It is easy to see to $\v{\Gamma}[\v{\pi}]$ is bounded by $\overline{\v{g}} - \underline{\v{g}}$ for any pricing scheme $\v{\pi}$. The magnitude of the \LOC{} payments is thus governed by the preceding term. 
Notice that the $\LOC[\v{\pi}^{\N}]$ expression in \eqref{eq:loc_pi_N} includes additional scenario-dependent congestion components $\v{H}_k^\top (\v{\mu}_k^{\DA{}\star} + \v{\mu}_k^{\SE{}\star})$ for each $k$. Intuitively speaking, as an SO becomes increasingly risk-averse, optimal values of the \RSCED{} objective in \eqref{eq:RSCED.lp} become more sensitive to worst-case scenarios. This sensitivity is reflected in the magnitude of these scenario-dependent congestion components. 
% it is expected that the effect of congestion will become more pronounced as more scenarios are considered risky. 
As a result, $\v{\mu}_k^{\DA{}\star}$ and $\v{\mu}_k^{\SE{}\star}$ are expected to grow significantly, resulting in increased $\LOC{}[\v{\pi}^{\N}]$. 

The value of these \LOC{} payments is highly relevant to the previous study of the \MS{}. SOs seek pricing schemes for which the \emph{total revenue}, i.e., \MS{} less aggregate \LOC{} payments, is non-negative. To that end, we  provide the following result for \LMPmar{}, the proof of which is included in Appendix \ref{app:proof.thm.revenue}.
\begin{theorem}
  \label{thm:revenue}
  The total revenue of a SO under the pricing scheme \LMPmar{} is positive, or
  \begin{equation}
      \MS{}[\v{\pi}^{\S}] - \bone^\top \LOC{}[\v{\pi}^{\S}] \ge 0.
  \end{equation}
\end{theorem}
Thus, even while ensuring individual rationality via \LOC{} payments, SOs are able to operate with the pricing scheme \LMPmar{} without fear of operating at a deficit. Intuitively speaking, under the \LMPmar{} scheme, payments from demanders and to suppliers account for contingencies, i.e., the market clearing process itself sufficiently accounts for reserve procurement and recourse actions. By Theorem \ref{thm:ms}, this process yields a nonnegative merchandising surplus, which the proof of Theorem \ref{thm:revenue} shows is ample to cover the remaining \LOC{} payments. The \LMPnom{} scheme, on the other hand, does not necessarily yield nonnegative \MS{}. Furthermore, \LMPnom{} \LOC{} payments have a congestion component that is expected to increase with risk-aversion, but no corresponding increase in revenue. This is problematic as out-of-market settlements may exceed the merchandising surplus. In fact, we will observe this phenomenon in our empirical analysis of the IEEE 24-bus RTS network in Section \ref{sec:simulation_results}. \revision{Again, similar to our observation for merchandising surplus, \LMPmar{} fares better than \LMPnom{} in ensuring that rents collected from demanders can sufficiently incentivize the generators to follow the SO-intended dispatch. Allowing prices to reflect what might occur in contingency scenarios helps to pay generators to follow a nominal dispatch whose calculation through R-SCED already captures the effect of said contingencies. In a way, \LMPmar{} ties the pricing mechanism more closely to the process through which the dispatch is computed than does \LMPnom{}.}

% \begin{theorem}
%   \label{thm:loc}
%   The \LOC{} payments under pricing schemes $\v{\pi}_s$ and $\v{\pi}_n$ are given by,
%   \begin{subequations}
%   \begin{align}
%       \LOC[\v{\pi}_s] &= \left| \overline{\v{\gamma}} - \underline{\v{\gamma}} + \sum_{k = 1}^K \overline{\v{\gamma}}_k - \underline{\v{\gamma}}_k \right|^\top \v{\Gamma}[\v{\pi}_s], \\
%       \LOC[\v{\pi}_n] &= \left| \overline{\v{\gamma}} - \underline{\v{\gamma}} + \sum_{k = 1}^K \v{H}_k^\top (\v{\mu}_k^\DA{} + \v{\mu}_k^\SE{}) + \overline{\v{\gamma}}_k - \underline{\v{\gamma}}_k \right|^\top \v{\Gamma}[\v{\pi}_n],
%   \end{align}
%   \end{subequations}
%   where $\v{\Gamma}[\v{\pi}]$ is a vector whose indices are given by
%     \begin{equation}
%     \begin{aligned}
%         \Gamma_i[\v{\pi}] := \max\{ & \sign(\pi_i - c_i) (\overline{g}_i - g_i^\star), \\
%         & \sign(\pi_i - c_i) (\underline{g}_i - g_i^\star) \}.
%     \end{aligned}
%     \end{equation}
% \end{theorem}

% The \LOC{} payments suggest that generators under the nodal marginal pricing scheme are reimbursed for (incorrect?) commitment decisions. The nominal pricing scheme includes a congestion-dependent component. As a SO becomes more risk-averse, the dual multipliers $\v{\mu}_k^\DA{}$ and $\v{\mu}_k^\SE{}$ would be expected to grow significantly, resulting in larger \LOC{} payments.

% It is evident that there is an additional cost associated with maintaining network reliability. In the nominal pricing method, producing larger \LOC{} payments, while the nodal marginal pricing produced higher LMPs and higher cost to consumers. The higher revenue allows for guarantees of revenue adequacy.

% \begin{theorem}
%   \label{thm:loc}
%   The \LOC{} payments under $\v{\pi}_s$ and $\v{\pi}_n$ pricing satisfies,
%   \begin{equation}
%       \LOC{}[\v{\pi_s}] \le \LOC{}[\v{\pi}_n].
%   \end{equation}
% \end{theorem}


% Under the nominal pricing scheme we are able to establish the following result.
% %
% \begin{lemma} %
%     \label{lem:ms.nominal}
%     Under the nominal prices in \eqref{eq:price.nominal}, the merchandising surplus satisfies,
%     \begin{equation}
%         \MS{} = \v{\mu}^\top \v{f} - \v{c}_r^\top \overline{\v{r}} - \v{c}_r^\top \underline{\v{r}}.
%         \label{eq:ms.nominal}
%     \end{equation}
% \end{lemma}
% %
% It can be seen that both $\v{\mu}^\top \v{f} \ge 0$ and $[ \overline{\v{\rho}}_k - \underline{\v{\rho}}_k ]^\top \v{\delta g}_k \ge 0$ for all $k$. As such, the merchandising surplus can be negative, which will be demonstrated in the empirical analysis.

% On the other hand, we are able to show the following with the marginal pricing scheme.
% %
% \begin{lemma} %
%     \label{lem:ms.marginal}
%     Under the marginal prices in \eqref{eq:price.marginal}, the merchandising surplus satisfies,
%     \begin{equation}
%     \begin{aligned}
%         \MS{} &= \v{\mu}^\top \v{f} + \sum_{k = 1}^K [\v{\mu}_k^\DA{}]^\top \v{f}_k^\DA{} + [\v{\mu}_k^\SE{}]\v{f}_k^\SE{} \\
%         &\qquad + \sum_{k = 1}^K (\overline{\v{\gamma}}_k - \underline{\v{\gamma}}_k)^\top \v{\delta g}_k + \overline{\v{\sigma}}_k^\top \v{\delta}_d + \overline{\nu}_k \voll^\top \v{\delta d}_k \ge 0.
%     \end{aligned}
%     \label{eq:ms.marginal}
%     \end{equation}
% \end{lemma}
% %
% Unlike the nominal mechanism, marginal pricing ensures that the SO sustains a positive merchandising surplus.

% subsection revenue_adequacy (end)

% \subsection{Competitive equilbiria and lost opportunity cost payments} % (fold)
% \label{sub:competitive_equilbiria_and_uplift_payments}

% A notable property of the \ED{} problem is that its solution supports a competitive equilibrium. That is, generators produce maximum profit at the prescribed quantities and therefore have no incentive to deviate their dispatch. Uncertainty in forward market procurement for multi-period markets poses a significant challenge to this notion, as hilighted in \addcite. \SCED{} formulations can be considered as instances of multi-period markets as they encode behavior in a secondary stage, i.e. corrective actions, raising the question of whether a competitive equilibrium can be established. In the event that generators are asked to deviate from their competitive behavior, they must be compensated accordingly via lost opportunity cost (\LOC{}) payments, similar to the case of generators with startup and shutdown costs in the unit-commitment problem \addcite.

% In order to establish such results, we consider the problem of profit maximization for a generator. We assume that generators are acting competitively, and therefore they express the true marginal cost of generation. As such the profit maximization problem of a generator at bus $i$ is given by
% %
% \begin{equation}
% \begin{aligned}
%     & \max_{g_i} && (\pi_i - c_i) g_i,
%     & \text{subject to} && \underline{g}_i \le g_i \le \overline{g_i}.
% \end{aligned}
% \end{equation}
% %
% Fortunately, this problem admits a trivial solution given by
% %
% \begin{equation}
%     g_i^\star = \begin{cases}
%         \overline{g}_i & \pi_i > c_i, \\
%         g_i \in [\underline{g}_i, \overline{g}_i] & \pi = c_i, \\
%         \underline{g}_i & \pi < c_i,
%     \end{cases}
% \end{equation}
% and the corresponding profit is given by
% \begin{equation}
%     J_i^\star = \begin{cases}
%         (\pi_i - c_i) \overline{g}_i & \pi_i \ge c_i, \\
%         (\pi_i - c_i) \underline{g}_i & \pi_i < c_i.
%     \end{cases}
% \end{equation}
% %
% The LOC payments, $\LOC{}$, or the difference between the competitive profit, $J_i^\star$, and the actual profit $(\pi_i - c_i) g_i^\star$ is then
% %
% \begin{equation}
%     \LOC{} = (\pi_i - c_i) \delta g_i,
% \end{equation}
% where $\delta g_i = \overline{g}_i - g_i^*$ if $\pi_i \ge c_i$ and $\delta g_i = \underline{g}_i - g_i^*$ otherwise.

% With these payments established, we can study their values under both of our pricing schemes. We have the following result for the nominal mechanism.
% %
% \begin{lemma} %
%     \label{lem:up.nominal}
%     Adopting the nominal pricing scheme in \eqref{eq:price.nominal}, the LOC payments satisfy,
%     \begin{equation}
%         \LOC{} = \left[ \overline{\v{\gamma}} - \underline{\v{\gamma}} + \sum_{k = 1}^K \v{H}_k^\top (\v{\mu}_k^\DA{} + \v{\mu}_k^\SE{}) + \overline{\v{\gamma}}_k - \underline{\v{\gamma}}_k \right]^\top \v{\delta g}^\star.
%         \label{eq:up.nominal}
%     \end{equation}
% \end{lemma}
% %
% For the marginal scheme we are able to provide the following result.
% %
% \begin{lemma} %
%     \label{lem:up.marginal}
%     Under the marginal pricing model in \eqref{eq:price.marginal}, the LOC payments satisfy,
%     \begin{equation*}
%         \LOC{} = \left[ \overline{\v{\gamma}} - \underline{\v{\gamma}} + \sum_{k = 1}^K \overline{\v{\gamma}}_k - \underline{\v{\gamma}}_k \right]^\top \v{\delta g}^\star.
%         \label{eq:up.marginal}
%     \end{equation*}
% \end{lemma}
% %
% Observe that the additional terms in \eqref{eq:up.nominal} as compared to \eqref{eq:up.marginal} are the congestion terms in \eqref{eq:ms.marginal}. As a result, increasing congestion across the network will result in greater costs...
