%!TEX root = ./mainRSCED.tex

\section{Formulating the security-constrained economic dispatch problem}
\label{sec:formulation}

System operators are tasked with solving forward dispatch problems at various intervals before the real-time in order to make dispatch decisions. Such problems seek to minimize the cost of procuring energy, or dispatch cost, subject to certain engineering constraints of the grid. We begin by presenting the deterministic \ED{} problem, which we will use to derive the risk-sensitive counterpart.

\subsection{The deterministic economic dispatch problem}
Consider an $n$-bus power network, connected by $\ell$ lines. Each bus is equipped with some demand and generation capacity, the vector values of which are denoted $\v{d}$ and $\v{g}$. Assume that each generator has a linear cost, collected by the vector $\v{c}$ \footnote{Such cost structures are able to represent piecewise-affine costs of generation that are frequently adopted in such problems.}, and can only produce power within $\Gcal := [\underline{\v{g}}, \overline{\v{g}}]$. The total cost of generation is then $\v{c}^\top \v{g}$. We adopt the DC approximations, that assume lossless lines, negligible reactive power, and small voltage phase angle differences. Such assumptions allow us to represent the nonlinear equations of Kirchoff's laws as linear constraints. 

Letting $\bone$ denote the vector of all ones, and $\v{H}$ be the injection-shift factor (ISF) matrix, the \ED{} problem is stated formally as
\begin{subequations}
\begin{align}
    &\underset{\v{g} \in \Gcal}{\text{minimize}} && \v{c}^\top \v{g}, \label{eq:ED.obj} \\
    &\text{subject to} && \bone^\top \left(\v{g} - \v{d}\right) = 0, \quad \v{H} \left( \v{g} - \v{d} \right) \le \v{f}. \label{eq:ED.cons}
\end{align}
\label{eq:ED}
\end{subequations}
This problem assumes full component availability, and cannot provide any meaningful guarantees in the event of a component failure. In order to ensure satisfaction of the $N-1$ security criterion, this formulation must be augmented with security-constraints, that ensure the network remains resilient to any potential component failure. For simplicity, we restrict attention to potential line outages.
% Generator outages can be accommodated, but will
Denote the set of all potential outages by $\Kset$, which is indexed by $k = 1, \dots, K$.

\subsection{The risk-sensitive economic dispatch problem}
We now consider the risk-sensitive counterpart of the economic dispatch problem in \eqref{eq:ED}, in which we consider potential line failures. Denote the set of all potential line outages by $\Kset$, and let it be indexed by $k = 1, \dots, K$. In the event of a failure of line $k$, we must ensure satisfaction of similar constraints to \eqref{eq:ED.cons}. To that end, let $\v{H}_k$ be the ISF matrix with the removal of line $k$. Note that line capacity limits arise primarily from thermal considerations and can be relaxed temporarily. We adopt the convention in \cite{national2010transmission} summarized in Table \ref{tab:relaxed.line.limits}. We denote these post-contingency limits as $\v{f}_k^{\DA{}}$ and $\v{f}_k^{\SE{}}$ for the drastic-action and short-term emergency limits, respectively.

\begin{table}[ht]
    \centering
    \begin{tabular}{l c c}
        \toprule
         \textbf{Name} & \textbf{Maximum allowable time} \\
         \midrule
         Nominal & Indefinite \\
         Long-Term Emergency (LE) & 4 hours \\
         Short-Term Emergency (SE) & 15 minutes \\
         Drastic Action (DA) & 5 minutes \\
         \bottomrule \\
    \end{tabular}
    \caption{Relaxed line flow capacity limits for emergency scenarios.}
    \label{tab:relaxed.line.limits}
\end{table}

%
\begin{subequations} %
\label{eq:RSCED} %
\begin{align} %
    &\underset{\v{g}, \underline{\v{r}}, \overline{\v{r}}, \v{\delta g}, \v{\delta d}}{\text{minimize}} && \CVaR{}_{\alpha} \left[ \v{c}^\top \v{g} + \underline{\v{c}}_r^\top \underline{\v{r}} + \overline{\v{c}}_r^\top \overline{\v{r}} + \mathcal{C}(\v{\delta d} ) \right] \label{eq:RSCED.obj}, \\
    &\text{subject to} && \bone^\top \left(\v{g} - \v{d}\right) = 0, \; \v{H} \left( \v{g} - \v{d} \right) \le \v{f}, \; \v{g} \in \Gcal \label{eq:RSCED.nominal} \\
    &&& \v{H}_k \left( \v{g} - \v{d} \right) \le \v{f}_k^{\DA}, \label{eq:RSCED.da} \\
    &&& \bone^\top \left( \v{\delta g}_k + \v{\delta d}_k \right) = 0, \label{eq:RSCED.se.balance} \\
    &&& \v{H}_k \left( \v{g} + \v{\delta g}_k - \v{d} + \v{\delta d}_k \right) \le \v{f}_k^{\SE}, \label{eq:RSCED.se.flow} \\
    &&& \v{g} + \v{\delta g}_k \in \Gcal, \; \v{\delta g}_k \in [-\underline{\v{r}}, \overline{\v{r}}], \\
    &&& 0 \le \underline{\v{r}}, \; \overline{\v{r}} \le \v{R}, \; \v{\delta d}_k \in \v{\Delta}_d, \label{eq:RSCED.se.bounds} \\
    &&& \forall k = 1, \dots, K. %
\end{align} %
\end{subequations} %
%

\subsection{Relation to existing formulations}


\subsection{Incorporating preventive constraints}
The simplest method of satisfying the $N-1$ criterion is to ensure that the dispatch satisfies the network constraints in the event of any potential failure. This is referred to as preventive-\SCED{} (\PSCED{}). Under this formulation, suppose there is a failure of line $k$. Let $\v{H}_k$ be the injection-shift factor matrix in the event of this failure, and $\v{f}_k$ be the line capacity limits sans the failed line. Then, the line capacity constraint in the event of a failure is given by $\v{H}_k \left( \v{g} - \v{d} \right) \le \v{f}_k$. Together with \eqref{eq:ED}, the \PSCED{} problem is then given by
\begin{subequations}
\begin{align}
    &\underset{\v{g} \in \Gcal}{\text{minimize}} && \v{c}^\top \v{g}, \label{eq:PSCED.obj} \\
    &\text{subject to} && \bone^\top \left(\v{g} - \v{d}\right) = 0, \quad \v{H} \left( \v{g} - \v{d} \right) \le \v{f}, \label{eq:PSCED.nom} \\
    &&& \v{H}_k \left( \v{g} - \v{d} \right) \le \v{f}_k, \quad \forall k = 1, \dots, l. \label{eq:PSCED.cont}
\end{align}
\label{eq:PSCED}
\end{subequations}
It is clear that the dispatch derived from this model ensures satisfaction of the $N-1$ security criterion. However, the formulation fails to account for the inherent slack in the system and often leads to conservative solutions with higher than necessary dispatch costs. Specifically, we can leverage the thermal capacitance of power lines to relax their limits temporarily.

\subsection{Modeling corrective action}
Line flow capacity limits arise, primarily, from thermal considerations and can be relaxed temporarily. Here, we adopt the convention of \cite{national2010transmission}, which allows for the relaxed limits summarized in Table \ref{tab:relaxed.line.limits}. We see that, in the event of a failure, we can temporarily relax the lines to their drastic action level. However, typically, market clearing problems are solved 15 minute intervals \addcite. As a result, we must consider an additional time period, 5 minutes post-failure, and ensure that the limits are satisfied at their short-term emergency level. During this time, we allow SOs to make recourse actions by modifying the dispatch of fast-ramping generators. Let $\v{\delta g}_k$ denote this change in dispatch and $\v{g}$ be the collection of these deviations. Assume the reserve capacity limit of the nodal generation is given by $\v{\Delta}_g := [\underline{\v{\delta}}_g, \overline{\v{\delta}}_g]$.
%
% The Corrective-SCED formulation (\CSCED{}) seeks to expand on the \PSCED{} formulation by permitting temporary line flow capacity limit relaxation and allowing SOs to respond to failures in order to reduce the cost of generation. This is done by taking advantage of the thermal capacitance of transmission lines, which are temporarily allowed to exceed their rated capacity during failure scenarios. We adopt a similar notation to that in existing ISO operation manuals \addcite, which is summarized in Table \addcite.
%
%
This leads to the corrective-\SCED{} (\CSCED{}) formulation, which is given by
\begin{subequations}
\begin{align}
    &\underset{\v{g}, \v{\delta g}}{\text{minimize}} && \v{c}^\top \v{g}, \label{eq:CSCED.obj} \\
    &\text{subject to} && \bone^\top \left(\v{g} - \v{d}\right) = 0, \; \v{H} \left( \v{g} - \v{d} \right) \le \v{f}, \; \v{g} \in \Gcal \label{eq:CSCED.nominal} \\
    &&& \v{H}_k \left( \v{g} - \v{d} \right) \le \v{f}_k^{\DA}, \label{eq:CSCED.da} \\
    &&& \bone^\top \v{\delta g}_k = 0, \; \v{H}_k \left( \v{g} + \v{\delta g}_k - \v{d} \right) \le \v{f}_k^{\SE}, \label{eq:CSCED.se} \\
    &&& \v{g} + \v{\delta g}_k \in \Gcal, \; \v{\delta g}_k \in \v{\Delta}_g, \label{eq:CSCED.se.gen} \\
    &&& \forall k = 1, \dots, K.
\end{align}
\label{eq:CSCED}
\end{subequations}
Notice that we do not ensure a return to nominal levels. Instead, we assume that dispatches are cleared at 15 minutes intervals, after which more significant decisions can be made to return the system to its nominal state. The presented formulation is able to guarantee satisfaction of the $N-1$ security criterion, while being less conservative than \eqref{eq:PSCED}.

The \CSCED{} formulation assumes a solution with no potential load shed. In heavily-stressed conditions, such as those under extreme weather events and/or previously failed components, load shed may be required in order to mitigate large-scale blackouts. Additionally, this formulation neglects the cost associated with the cost of recourse generation, which should be encoded in the problem formulation itself.

% \subsection{Handling potential load shed and incorporating reserve procurement}

% This formulation assumes a fixed demand profile, however under heavily loaded conditions load shed may be a necessary corrective action in order to ensure network stability. Additionally, we have assumed a fixed reserve capacity, the cost of which is ignored in the formulation. To address these concerns, authors in \cite{bouffard2008stochastic} associate probabilities, $p_k$, to each potential failure, and allow for potential load shed in each scenario $k$, denoted $\v{\delta}_k$, which can take values in $\v{\Delta}_d := [0, \overline{\v{\delta}}_d$. The cost of load shed is modeled through the value of lost load (\VoLL{}), which we denote by $\voll$. Let $\v{\delta g}$ and $\v{\delta d}$ collect $\v{\delta g}_k$  and $\v{\delta d}_k$ respectively. The cost of recourse load shed is random, depending on the realized contingency, which we will denote by $\mathcal{C}(\v{\delta d})$ and takes values $\voll^\top \v{\delta d}_k$ with probability $p_k$ and $0$ otherwise. For simplicity, we will assume that the up and down reserve capacities are the same and have a nodal cost of $\v{c}_r$. Extending it to the case of different directional reserve capacities with different costs does not add any conceptual difficulty to the subsequent analysis. Under these conditions, the authors in \cite{bouffard2008stochastic} seek to minimize the expected cost, formulating the complete problem as
% %
% \begin{equation}
% \begin{aligned}
%     &\underset{\v{g}, \v{r}, \v{\delta g}, \v{\delta d}}{\text{minimize}} && \E\left[ \v{c}^\top \v{g} + \v{c}_r^\top \v{r} + \mathcal{C}(\v{\delta d} )\right], \\
%     &\text{subject to} && \eqref{eq:CSCED.nominal} \text{ --- } \eqref{eq:CSCED.da}, \\
%     % \eqref{eq:CSCED.se.gen}, \\
%     &&& \bone^\top \left( \v{\delta g}_k + \v{\delta d}_k \right) = 0, \\
%     &&& \v{H}_k \left( \v{g} + \v{\delta g}_k - \v{d} + \v{\delta d}_k \right) \le \v{f}_k^{\SE}, \\
%     &&& \v{g} + \v{\delta g}_k \in \Gcal, \; \v{\delta g}_k \in [-\v{r}, \v{r}], \; \v{\delta d}_k \in \v{\Delta}_d, \\
%     &&& \forall k = 1, \dots, K.
% \end{aligned}
% \label{eq:CSCEDE}
% \end{equation}
% %
% The objective in this formulation can be thought of as maximizing the expected utility, where the value of lost load measures the utility lost from being unable to meet demand. In practice, however, SOs typically assume a fixed demand profile and are highly averse to any potential load shed. This can be captured in this formulation by augmenting the \VoLL{} to capture this natural aversion. We adopt a different approach, selecting a measure that capture this aversion to high potential load shed.

% \subsection{Capturing the SOs aversion to load shed}
% In order to capture the SOs natural aversion to load shed, we consider the risk-sensitive \SCED{} formulation (\RSCED{}), previously proposed in \addcite. This formulation minimizes the conditional value at risk (\CVaR{}) of the random cost, rather than the expectation, and is defined as the following for some fixed parameter $\alpha$.
% %
% \begin{subequations}
% \begin{align}
%     &\underset{\v{g}, \v{r}, \v{\delta g}, \v{\delta d}}{\text{minimize}} && \CVaR{}_{\alpha} \left[ \v{c}^\top \v{g} + \v{c}_r^\top \v{r} + \mathcal{C}(\v{\delta d} ) \right], \\
%     &\text{subject to} && \bone^\top \left(\v{g} - \v{d}\right) = 0, \; \v{H} \left( \v{g} - \v{d} \right) \le \v{f}, \; \v{g} \in \Gcal \label{eq:RSCED.nominal} \\
%     &&& \v{H}_k \left( \v{g} - \v{d} \right) \le \v{f}_k^{\DA}, \label{eq:RSCED.da} \\
%     &&& \bone^\top \left( \v{\delta g}_k + \v{\delta d}_k \right) = 0, \label{eq:RSCED.se.balance} \\
%     &&& \v{H}_k \left( \v{g} + \v{\delta g}_k - \v{d} + \v{\delta d}_k \right) \le \v{f}_k^{\SE}, \label{eq:RSCED.se.flow} \\
%     &&& \v{g} + \v{\delta g}_k \in \Gcal, \; \v{\delta g}_k \in [-\v{r}, \v{r}], \; \v{\delta d}_k \in \v{\Delta}_d, \label{eq:RSCED.se.bounds} \\
%     &&& \forall k = 1, \dots, K,
% \end{align}
% \label{eq:RSCED}
% \end{subequations}
% Here, we make use of \CVaR{}, that, parameterized by $\alpha$, measures the expected tail loss over the $(1-\alpha)$-fraction of worst case scenarios. The choice of $\alpha$ allows a SO to express their level of tolerance to high cost scenarios; setting $\alpha = 0$ renders \CVaR{} equivalent to expected value, recovering the formulation in \eqref{eq:CSCEDE}, and in taking $\alpha \uparrow 1$, \CVaR{} becomes the essential supremum, akin to robust optimization. Figure \ref{fig:cvar} provides an illustration of this risk-measure for some fixed distribution.
% %
% \begin{figure}
%     \centering
%     \includegraphics[width=0.7\linewidth]{images/cvar_beta3.pdf}
%     \caption{A description of \CVaR{} over the cost distribution induced by a set of fixed decision variables}
%     \label{fig:cvar}
% \end{figure}
% %
% The definition of $\CVaR{}$ assumes some fixed decision variables. When optimizing over $\CVaR$, we seek to determine decision variables whose induced distribution has the lowest expected tail loss.