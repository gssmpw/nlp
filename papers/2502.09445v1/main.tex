
\documentclass{article} % For LaTeX2e
\usepackage{iclr2025_conference,times}

% Optional math commands from https://github.com/goodfeli/dlbook_notation.
%%%%% NEW MATH DEFINITIONS %%%%%

\usepackage{amsmath,amsfonts,bm}
\usepackage{derivative}
% Mark sections of captions for referring to divisions of figures
\newcommand{\figleft}{{\em (Left)}}
\newcommand{\figcenter}{{\em (Center)}}
\newcommand{\figright}{{\em (Right)}}
\newcommand{\figtop}{{\em (Top)}}
\newcommand{\figbottom}{{\em (Bottom)}}
\newcommand{\captiona}{{\em (a)}}
\newcommand{\captionb}{{\em (b)}}
\newcommand{\captionc}{{\em (c)}}
\newcommand{\captiond}{{\em (d)}}

% Highlight a newly defined term
\newcommand{\newterm}[1]{{\bf #1}}

% Derivative d 
\newcommand{\deriv}{{\mathrm{d}}}

% Figure reference, lower-case.
\def\figref#1{figure~\ref{#1}}
% Figure reference, capital. For start of sentence
\def\Figref#1{Figure~\ref{#1}}
\def\twofigref#1#2{figures \ref{#1} and \ref{#2}}
\def\quadfigref#1#2#3#4{figures \ref{#1}, \ref{#2}, \ref{#3} and \ref{#4}}
% Section reference, lower-case.
\def\secref#1{section~\ref{#1}}
% Section reference, capital.
\def\Secref#1{Section~\ref{#1}}
% Reference to two sections.
\def\twosecrefs#1#2{sections \ref{#1} and \ref{#2}}
% Reference to three sections.
\def\secrefs#1#2#3{sections \ref{#1}, \ref{#2} and \ref{#3}}
% Reference to an equation, lower-case.
\def\eqref#1{equation~\ref{#1}}
% Reference to an equation, upper case
\def\Eqref#1{Equation~\ref{#1}}
% A raw reference to an equation---avoid using if possible
\def\plaineqref#1{\ref{#1}}
% Reference to a chapter, lower-case.
\def\chapref#1{chapter~\ref{#1}}
% Reference to an equation, upper case.
\def\Chapref#1{Chapter~\ref{#1}}
% Reference to a range of chapters
\def\rangechapref#1#2{chapters\ref{#1}--\ref{#2}}
% Reference to an algorithm, lower-case.
\def\algref#1{algorithm~\ref{#1}}
% Reference to an algorithm, upper case.
\def\Algref#1{Algorithm~\ref{#1}}
\def\twoalgref#1#2{algorithms \ref{#1} and \ref{#2}}
\def\Twoalgref#1#2{Algorithms \ref{#1} and \ref{#2}}
% Reference to a part, lower case
\def\partref#1{part~\ref{#1}}
% Reference to a part, upper case
\def\Partref#1{Part~\ref{#1}}
\def\twopartref#1#2{parts \ref{#1} and \ref{#2}}

\def\ceil#1{\lceil #1 \rceil}
\def\floor#1{\lfloor #1 \rfloor}
\def\1{\bm{1}}
\newcommand{\train}{\mathcal{D}}
\newcommand{\valid}{\mathcal{D_{\mathrm{valid}}}}
\newcommand{\test}{\mathcal{D_{\mathrm{test}}}}

\def\eps{{\epsilon}}


% Random variables
\def\reta{{\textnormal{$\eta$}}}
\def\ra{{\textnormal{a}}}
\def\rb{{\textnormal{b}}}
\def\rc{{\textnormal{c}}}
\def\rd{{\textnormal{d}}}
\def\re{{\textnormal{e}}}
\def\rf{{\textnormal{f}}}
\def\rg{{\textnormal{g}}}
\def\rh{{\textnormal{h}}}
\def\ri{{\textnormal{i}}}
\def\rj{{\textnormal{j}}}
\def\rk{{\textnormal{k}}}
\def\rl{{\textnormal{l}}}
% rm is already a command, just don't name any random variables m
\def\rn{{\textnormal{n}}}
\def\ro{{\textnormal{o}}}
\def\rp{{\textnormal{p}}}
\def\rq{{\textnormal{q}}}
\def\rr{{\textnormal{r}}}
\def\rs{{\textnormal{s}}}
\def\rt{{\textnormal{t}}}
\def\ru{{\textnormal{u}}}
\def\rv{{\textnormal{v}}}
\def\rw{{\textnormal{w}}}
\def\rx{{\textnormal{x}}}
\def\ry{{\textnormal{y}}}
\def\rz{{\textnormal{z}}}

% Random vectors
\def\rvepsilon{{\mathbf{\epsilon}}}
\def\rvphi{{\mathbf{\phi}}}
\def\rvtheta{{\mathbf{\theta}}}
\def\rva{{\mathbf{a}}}
\def\rvb{{\mathbf{b}}}
\def\rvc{{\mathbf{c}}}
\def\rvd{{\mathbf{d}}}
\def\rve{{\mathbf{e}}}
\def\rvf{{\mathbf{f}}}
\def\rvg{{\mathbf{g}}}
\def\rvh{{\mathbf{h}}}
\def\rvu{{\mathbf{i}}}
\def\rvj{{\mathbf{j}}}
\def\rvk{{\mathbf{k}}}
\def\rvl{{\mathbf{l}}}
\def\rvm{{\mathbf{m}}}
\def\rvn{{\mathbf{n}}}
\def\rvo{{\mathbf{o}}}
\def\rvp{{\mathbf{p}}}
\def\rvq{{\mathbf{q}}}
\def\rvr{{\mathbf{r}}}
\def\rvs{{\mathbf{s}}}
\def\rvt{{\mathbf{t}}}
\def\rvu{{\mathbf{u}}}
\def\rvv{{\mathbf{v}}}
\def\rvw{{\mathbf{w}}}
\def\rvx{{\mathbf{x}}}
\def\rvy{{\mathbf{y}}}
\def\rvz{{\mathbf{z}}}

% Elements of random vectors
\def\erva{{\textnormal{a}}}
\def\ervb{{\textnormal{b}}}
\def\ervc{{\textnormal{c}}}
\def\ervd{{\textnormal{d}}}
\def\erve{{\textnormal{e}}}
\def\ervf{{\textnormal{f}}}
\def\ervg{{\textnormal{g}}}
\def\ervh{{\textnormal{h}}}
\def\ervi{{\textnormal{i}}}
\def\ervj{{\textnormal{j}}}
\def\ervk{{\textnormal{k}}}
\def\ervl{{\textnormal{l}}}
\def\ervm{{\textnormal{m}}}
\def\ervn{{\textnormal{n}}}
\def\ervo{{\textnormal{o}}}
\def\ervp{{\textnormal{p}}}
\def\ervq{{\textnormal{q}}}
\def\ervr{{\textnormal{r}}}
\def\ervs{{\textnormal{s}}}
\def\ervt{{\textnormal{t}}}
\def\ervu{{\textnormal{u}}}
\def\ervv{{\textnormal{v}}}
\def\ervw{{\textnormal{w}}}
\def\ervx{{\textnormal{x}}}
\def\ervy{{\textnormal{y}}}
\def\ervz{{\textnormal{z}}}

% Random matrices
\def\rmA{{\mathbf{A}}}
\def\rmB{{\mathbf{B}}}
\def\rmC{{\mathbf{C}}}
\def\rmD{{\mathbf{D}}}
\def\rmE{{\mathbf{E}}}
\def\rmF{{\mathbf{F}}}
\def\rmG{{\mathbf{G}}}
\def\rmH{{\mathbf{H}}}
\def\rmI{{\mathbf{I}}}
\def\rmJ{{\mathbf{J}}}
\def\rmK{{\mathbf{K}}}
\def\rmL{{\mathbf{L}}}
\def\rmM{{\mathbf{M}}}
\def\rmN{{\mathbf{N}}}
\def\rmO{{\mathbf{O}}}
\def\rmP{{\mathbf{P}}}
\def\rmQ{{\mathbf{Q}}}
\def\rmR{{\mathbf{R}}}
\def\rmS{{\mathbf{S}}}
\def\rmT{{\mathbf{T}}}
\def\rmU{{\mathbf{U}}}
\def\rmV{{\mathbf{V}}}
\def\rmW{{\mathbf{W}}}
\def\rmX{{\mathbf{X}}}
\def\rmY{{\mathbf{Y}}}
\def\rmZ{{\mathbf{Z}}}

% Elements of random matrices
\def\ermA{{\textnormal{A}}}
\def\ermB{{\textnormal{B}}}
\def\ermC{{\textnormal{C}}}
\def\ermD{{\textnormal{D}}}
\def\ermE{{\textnormal{E}}}
\def\ermF{{\textnormal{F}}}
\def\ermG{{\textnormal{G}}}
\def\ermH{{\textnormal{H}}}
\def\ermI{{\textnormal{I}}}
\def\ermJ{{\textnormal{J}}}
\def\ermK{{\textnormal{K}}}
\def\ermL{{\textnormal{L}}}
\def\ermM{{\textnormal{M}}}
\def\ermN{{\textnormal{N}}}
\def\ermO{{\textnormal{O}}}
\def\ermP{{\textnormal{P}}}
\def\ermQ{{\textnormal{Q}}}
\def\ermR{{\textnormal{R}}}
\def\ermS{{\textnormal{S}}}
\def\ermT{{\textnormal{T}}}
\def\ermU{{\textnormal{U}}}
\def\ermV{{\textnormal{V}}}
\def\ermW{{\textnormal{W}}}
\def\ermX{{\textnormal{X}}}
\def\ermY{{\textnormal{Y}}}
\def\ermZ{{\textnormal{Z}}}

% Vectors
\def\vzero{{\bm{0}}}
\def\vone{{\bm{1}}}
\def\vmu{{\bm{\mu}}}
\def\vtheta{{\bm{\theta}}}
\def\vphi{{\bm{\phi}}}
\def\va{{\bm{a}}}
\def\vb{{\bm{b}}}
\def\vc{{\bm{c}}}
\def\vd{{\bm{d}}}
\def\ve{{\bm{e}}}
\def\vf{{\bm{f}}}
\def\vg{{\bm{g}}}
\def\vh{{\bm{h}}}
\def\vi{{\bm{i}}}
\def\vj{{\bm{j}}}
\def\vk{{\bm{k}}}
\def\vl{{\bm{l}}}
\def\vm{{\bm{m}}}
\def\vn{{\bm{n}}}
\def\vo{{\bm{o}}}
\def\vp{{\bm{p}}}
\def\vq{{\bm{q}}}
\def\vr{{\bm{r}}}
\def\vs{{\bm{s}}}
\def\vt{{\bm{t}}}
\def\vu{{\bm{u}}}
\def\vv{{\bm{v}}}
\def\vw{{\bm{w}}}
\def\vx{{\bm{x}}}
\def\vy{{\bm{y}}}
\def\vz{{\bm{z}}}

% Elements of vectors
\def\evalpha{{\alpha}}
\def\evbeta{{\beta}}
\def\evepsilon{{\epsilon}}
\def\evlambda{{\lambda}}
\def\evomega{{\omega}}
\def\evmu{{\mu}}
\def\evpsi{{\psi}}
\def\evsigma{{\sigma}}
\def\evtheta{{\theta}}
\def\eva{{a}}
\def\evb{{b}}
\def\evc{{c}}
\def\evd{{d}}
\def\eve{{e}}
\def\evf{{f}}
\def\evg{{g}}
\def\evh{{h}}
\def\evi{{i}}
\def\evj{{j}}
\def\evk{{k}}
\def\evl{{l}}
\def\evm{{m}}
\def\evn{{n}}
\def\evo{{o}}
\def\evp{{p}}
\def\evq{{q}}
\def\evr{{r}}
\def\evs{{s}}
\def\evt{{t}}
\def\evu{{u}}
\def\evv{{v}}
\def\evw{{w}}
\def\evx{{x}}
\def\evy{{y}}
\def\evz{{z}}

% Matrix
\def\mA{{\bm{A}}}
\def\mB{{\bm{B}}}
\def\mC{{\bm{C}}}
\def\mD{{\bm{D}}}
\def\mE{{\bm{E}}}
\def\mF{{\bm{F}}}
\def\mG{{\bm{G}}}
\def\mH{{\bm{H}}}
\def\mI{{\bm{I}}}
\def\mJ{{\bm{J}}}
\def\mK{{\bm{K}}}
\def\mL{{\bm{L}}}
\def\mM{{\bm{M}}}
\def\mN{{\bm{N}}}
\def\mO{{\bm{O}}}
\def\mP{{\bm{P}}}
\def\mQ{{\bm{Q}}}
\def\mR{{\bm{R}}}
\def\mS{{\bm{S}}}
\def\mT{{\bm{T}}}
\def\mU{{\bm{U}}}
\def\mV{{\bm{V}}}
\def\mW{{\bm{W}}}
\def\mX{{\bm{X}}}
\def\mY{{\bm{Y}}}
\def\mZ{{\bm{Z}}}
\def\mBeta{{\bm{\beta}}}
\def\mPhi{{\bm{\Phi}}}
\def\mLambda{{\bm{\Lambda}}}
\def\mSigma{{\bm{\Sigma}}}

% Tensor
\DeclareMathAlphabet{\mathsfit}{\encodingdefault}{\sfdefault}{m}{sl}
\SetMathAlphabet{\mathsfit}{bold}{\encodingdefault}{\sfdefault}{bx}{n}
\newcommand{\tens}[1]{\bm{\mathsfit{#1}}}
\def\tA{{\tens{A}}}
\def\tB{{\tens{B}}}
\def\tC{{\tens{C}}}
\def\tD{{\tens{D}}}
\def\tE{{\tens{E}}}
\def\tF{{\tens{F}}}
\def\tG{{\tens{G}}}
\def\tH{{\tens{H}}}
\def\tI{{\tens{I}}}
\def\tJ{{\tens{J}}}
\def\tK{{\tens{K}}}
\def\tL{{\tens{L}}}
\def\tM{{\tens{M}}}
\def\tN{{\tens{N}}}
\def\tO{{\tens{O}}}
\def\tP{{\tens{P}}}
\def\tQ{{\tens{Q}}}
\def\tR{{\tens{R}}}
\def\tS{{\tens{S}}}
\def\tT{{\tens{T}}}
\def\tU{{\tens{U}}}
\def\tV{{\tens{V}}}
\def\tW{{\tens{W}}}
\def\tX{{\tens{X}}}
\def\tY{{\tens{Y}}}
\def\tZ{{\tens{Z}}}


% Graph
\def\gA{{\mathcal{A}}}
\def\gB{{\mathcal{B}}}
\def\gC{{\mathcal{C}}}
\def\gD{{\mathcal{D}}}
\def\gE{{\mathcal{E}}}
\def\gF{{\mathcal{F}}}
\def\gG{{\mathcal{G}}}
\def\gH{{\mathcal{H}}}
\def\gI{{\mathcal{I}}}
\def\gJ{{\mathcal{J}}}
\def\gK{{\mathcal{K}}}
\def\gL{{\mathcal{L}}}
\def\gM{{\mathcal{M}}}
\def\gN{{\mathcal{N}}}
\def\gO{{\mathcal{O}}}
\def\gP{{\mathcal{P}}}
\def\gQ{{\mathcal{Q}}}
\def\gR{{\mathcal{R}}}
\def\gS{{\mathcal{S}}}
\def\gT{{\mathcal{T}}}
\def\gU{{\mathcal{U}}}
\def\gV{{\mathcal{V}}}
\def\gW{{\mathcal{W}}}
\def\gX{{\mathcal{X}}}
\def\gY{{\mathcal{Y}}}
\def\gZ{{\mathcal{Z}}}

% Sets
\def\sA{{\mathbb{A}}}
\def\sB{{\mathbb{B}}}
\def\sC{{\mathbb{C}}}
\def\sD{{\mathbb{D}}}
% Don't use a set called E, because this would be the same as our symbol
% for expectation.
\def\sF{{\mathbb{F}}}
\def\sG{{\mathbb{G}}}
\def\sH{{\mathbb{H}}}
\def\sI{{\mathbb{I}}}
\def\sJ{{\mathbb{J}}}
\def\sK{{\mathbb{K}}}
\def\sL{{\mathbb{L}}}
\def\sM{{\mathbb{M}}}
\def\sN{{\mathbb{N}}}
\def\sO{{\mathbb{O}}}
\def\sP{{\mathbb{P}}}
\def\sQ{{\mathbb{Q}}}
\def\sR{{\mathbb{R}}}
\def\sS{{\mathbb{S}}}
\def\sT{{\mathbb{T}}}
\def\sU{{\mathbb{U}}}
\def\sV{{\mathbb{V}}}
\def\sW{{\mathbb{W}}}
\def\sX{{\mathbb{X}}}
\def\sY{{\mathbb{Y}}}
\def\sZ{{\mathbb{Z}}}

% Entries of a matrix
\def\emLambda{{\Lambda}}
\def\emA{{A}}
\def\emB{{B}}
\def\emC{{C}}
\def\emD{{D}}
\def\emE{{E}}
\def\emF{{F}}
\def\emG{{G}}
\def\emH{{H}}
\def\emI{{I}}
\def\emJ{{J}}
\def\emK{{K}}
\def\emL{{L}}
\def\emM{{M}}
\def\emN{{N}}
\def\emO{{O}}
\def\emP{{P}}
\def\emQ{{Q}}
\def\emR{{R}}
\def\emS{{S}}
\def\emT{{T}}
\def\emU{{U}}
\def\emV{{V}}
\def\emW{{W}}
\def\emX{{X}}
\def\emY{{Y}}
\def\emZ{{Z}}
\def\emSigma{{\Sigma}}

% entries of a tensor
% Same font as tensor, without \bm wrapper
\newcommand{\etens}[1]{\mathsfit{#1}}
\def\etLambda{{\etens{\Lambda}}}
\def\etA{{\etens{A}}}
\def\etB{{\etens{B}}}
\def\etC{{\etens{C}}}
\def\etD{{\etens{D}}}
\def\etE{{\etens{E}}}
\def\etF{{\etens{F}}}
\def\etG{{\etens{G}}}
\def\etH{{\etens{H}}}
\def\etI{{\etens{I}}}
\def\etJ{{\etens{J}}}
\def\etK{{\etens{K}}}
\def\etL{{\etens{L}}}
\def\etM{{\etens{M}}}
\def\etN{{\etens{N}}}
\def\etO{{\etens{O}}}
\def\etP{{\etens{P}}}
\def\etQ{{\etens{Q}}}
\def\etR{{\etens{R}}}
\def\etS{{\etens{S}}}
\def\etT{{\etens{T}}}
\def\etU{{\etens{U}}}
\def\etV{{\etens{V}}}
\def\etW{{\etens{W}}}
\def\etX{{\etens{X}}}
\def\etY{{\etens{Y}}}
\def\etZ{{\etens{Z}}}

% The true underlying data generating distribution
\newcommand{\pdata}{p_{\rm{data}}}
\newcommand{\ptarget}{p_{\rm{target}}}
\newcommand{\pprior}{p_{\rm{prior}}}
\newcommand{\pbase}{p_{\rm{base}}}
\newcommand{\pref}{p_{\rm{ref}}}

% The empirical distribution defined by the training set
\newcommand{\ptrain}{\hat{p}_{\rm{data}}}
\newcommand{\Ptrain}{\hat{P}_{\rm{data}}}
% The model distribution
\newcommand{\pmodel}{p_{\rm{model}}}
\newcommand{\Pmodel}{P_{\rm{model}}}
\newcommand{\ptildemodel}{\tilde{p}_{\rm{model}}}
% Stochastic autoencoder distributions
\newcommand{\pencode}{p_{\rm{encoder}}}
\newcommand{\pdecode}{p_{\rm{decoder}}}
\newcommand{\precons}{p_{\rm{reconstruct}}}

\newcommand{\laplace}{\mathrm{Laplace}} % Laplace distribution

\newcommand{\E}{\mathbb{E}}
\newcommand{\Ls}{\mathcal{L}}
\newcommand{\R}{\mathbb{R}}
\newcommand{\emp}{\tilde{p}}
\newcommand{\lr}{\alpha}
\newcommand{\reg}{\lambda}
\newcommand{\rect}{\mathrm{rectifier}}
\newcommand{\softmax}{\mathrm{softmax}}
\newcommand{\sigmoid}{\sigma}
\newcommand{\softplus}{\zeta}
\newcommand{\KL}{D_{\mathrm{KL}}}
\newcommand{\Var}{\mathrm{Var}}
\newcommand{\standarderror}{\mathrm{SE}}
\newcommand{\Cov}{\mathrm{Cov}}
% Wolfram Mathworld says $L^2$ is for function spaces and $\ell^2$ is for vectors
% But then they seem to use $L^2$ for vectors throughout the site, and so does
% wikipedia.
\newcommand{\normlzero}{L^0}
\newcommand{\normlone}{L^1}
\newcommand{\normltwo}{L^2}
\newcommand{\normlp}{L^p}
\newcommand{\normmax}{L^\infty}

\newcommand{\parents}{Pa} % See usage in notation.tex. Chosen to match Daphne's book.

\DeclareMathOperator*{\argmax}{arg\,max}
\DeclareMathOperator*{\argmin}{arg\,min}

\DeclareMathOperator{\sign}{sign}
\DeclareMathOperator{\Tr}{Tr}
\let\ab\allowbreak


\usepackage{hyperref}
\usepackage{url}

%%%%%%%%%%%%% added %%%%%%%%%%%%%
\usepackage{url}
\usepackage{longtable}
\usepackage{array}
\usepackage{soul}

\usepackage{nicematrix, booktabs}       % professional-quality tables
\usepackage{amsfonts}       % blackboard math symbols
\usepackage{nicefrac}       % compact symbols for 1/2, etc.
\usepackage{microtype}      % microtypography
\usepackage{xcolor}         % colors
\usepackage{amsmath}
\usepackage{graphicx} 
\usepackage{xargs}
\usepackage{mathrsfs}
\usepackage{bbm}
\usepackage{upgreek}

\usepackage{arydshln}

\usepackage{multirow}
\usepackage{bbm}
\usepackage{xcolor}
\usepackage{amssymb}
\usepackage{amsthm}
\usepackage{adjustbox}
\usepackage{url}
\usepackage{enumitem}
\usepackage{soul}
\usepackage{wrapfig}
\usepackage{outlines}
\usepackage{subcaption}  %
\usepackage[noend]{algorithmic}
\usepackage{algorithm}
\usepackage{thmtools}
\usepackage{thm-restate}
\usepackage{cleveref}


\usepackage{siunitx}
\sisetup{output-exponent-marker=\ensuremath{\mathrm{e}}}




\declaretheorem[name=Theorem,numberwithin=section]{thm}


\newtheorem{asu}{Assumption}
\newtheorem{asuH}{Assumption}
\renewcommand\theasu{A\arabic{asu}}
\renewcommand\theasuH{H\arabic{asuH}}
\definecolor{apricot}{rgb}{0.98, 0.81, 0.69}
\definecolor{celadon}{rgb}{0.67, 0.88, 0.69}

\newcommand{\ucmd}[1]{{\color{red}#1}}

\newcommand\numberthis{\addtocounter{equation}{1}\tag{\theequation}}
\newcounter{relctr} %% <- counter for relations
\everydisplay\expandafter{\the\everydisplay\setcounter{relctr}{0}} %% <- reset every eq
\renewcommand*\therelctr{\alph{relctr}} %% <- label format

\newcommand*\diff{\mathop{}\!\mathrm{d}}

\newcommand\labelrel[2]{%
  \begingroup
    \refstepcounter{relctr}%
    \stackrel{\textnormal{(\alph{relctr})}}{\mathstrut{#1}}%
    \originallabel{#2}%
  \endgroup
}
\AtBeginDocument{\let\originallabel\label}

\newtheorem{theorem}{Theorem}
\newtheorem{proposition}{Proposition}
\newtheorem{lemma}{Lemma}
\newtheorem{corollary}{Corollary}
\newtheorem{definition}{Definition}
\newtheorem{assumption}{Assumption}
\newcommand{\LeventIssue}[1]{\textcolor{orange}{Levent: {#1}}}
\newcommand{\Kruno}[1]{\textcolor{blue}{Kruno: {#1}}}


% Lists
\usepackage[inline]{enumitem}
\newlist{enuminline}{enumerate*}{1}
\setlist[enuminline]{label=(\roman*)}

% Math
\newcommand{\vol}{\mathrm{vol}}
\newcommand{\E}{\mathbb E}
\newcommand{\var}{\mathrm{var}}
\newcommand{\cov}{\mathrm{cov}}
\newcommand{\Normal}{\mathcal N}
\newcommand{\slowvar}{\mathcal L}
\newcommand{\bbR}{\mathbb R}
\newcommand{\bbE}{\mathbb E}
\newcommand{\bbP}{\mathbb P}
\newcommand{\calC}{\mathcal C}
\newcommand{\calR}{\mathcal R}
\newcommand{\calT}{\mathcal T}
\newcommand{\loA}{\underline{A}}
\newcommand{\upA}{\overline{A}} 
\newcommand{\cv}{\mathrm{cv}} 
\newcommand{\pp}{\mathrm{pp}}
\newcommand{\HS}{\mathrm{HS}}
\newcommand{\erfi}{\mathrm{erfi}}
\newcommand{\tX}{\widetilde X}
\newcommand{\OneFOne}{{}_1F_1}
\newcommand{\PP}{\mathrm{\texttt{PP}}}
\newcommand{\PPpp}{\mathrm{\texttt{PP+}}}
\newcommand{\BPP}{\mathrm{\texttt{BPP}}}
\newcommand{\BPPpp}{\mathrm{\texttt{BPP+}}}
\newcommand{\FABPPI}{\mathrm{\texttt{FABPP}}}
\newcommand{\asto}{\overset{\text{a.s.}}{\to}}

%%%%%%%%%%%%% added end %%%%%%%%%%%%%

\title{A differentiable rank-based objective for better feature learning}

% A differentiable conditional dependence based regularizer for better feature learning
% A DIFFERENTIABLE RANK-BASED REGULARIZER FOR BETTER FEATURE LEARNING
% Authors must not appear in the submitted version. They should be hidden
% as long as the \iclrfinalcopy macro remains commented out below.
% Non-anonymous submissions will be rejected without review.

\author{Krunoslav Lehman Pavasovic\thanks{Correspondence to krunolp@meta.com. \textsuperscript{\textdagger} Joint last author.}\\
Meta FAIR, Paris\\
 \\
\And
David Lopez-Paz\\
Meta FAIR, Paris\\
\\
\And
Giulio Biroli\textsuperscript{\textdagger}\\
ENS Paris\\
\\
\And
Levent Sagun\textsuperscript{\textdagger} \\
Meta FAIR, Paris \\
\\
}

% The \author macro works with any number of authors. There are two commands
% used to separate the names and addresses of multiple authors: \And and \AND.
%
% Using \And between authors leaves it to \LaTeX{} to determine where to break
% the lines. Using \AND forces a linebreak at that point. So, if \LaTeX{}
% puts 3 of 4 authors names on the first line, and the last on the second
% line, try using \AND instead of \And before the third author name.

\newcommand{\fix}{\marginpar{FIX}}
\newcommand{\new}{\marginpar{NEW}}

\iclrfinalcopy % Uncomment for camera-ready version, but NOT for submission.
\begin{document}


\maketitle

% \begin{abstract}


The choice of representation for geographic location significantly impacts the accuracy of models for a broad range of geospatial tasks, including fine-grained species classification, population density estimation, and biome classification. Recent works like SatCLIP and GeoCLIP learn such representations by contrastively aligning geolocation with co-located images. While these methods work exceptionally well, in this paper, we posit that the current training strategies fail to fully capture the important visual features. We provide an information theoretic perspective on why the resulting embeddings from these methods discard crucial visual information that is important for many downstream tasks. To solve this problem, we propose a novel retrieval-augmented strategy called RANGE. We build our method on the intuition that the visual features of a location can be estimated by combining the visual features from multiple similar-looking locations. We evaluate our method across a wide variety of tasks. Our results show that RANGE outperforms the existing state-of-the-art models with significant margins in most tasks. We show gains of up to 13.1\% on classification tasks and 0.145 $R^2$ on regression tasks. All our code and models will be made available at: \href{https://github.com/mvrl/RANGE}{https://github.com/mvrl/RANGE}.

\end{abstract}


% \section{Introduction}

Video generation has garnered significant attention owing to its transformative potential across a wide range of applications, such media content creation~\citep{polyak2024movie}, advertising~\citep{zhang2024virbo,bacher2021advert}, video games~\citep{yang2024playable,valevski2024diffusion, oasis2024}, and world model simulators~\citep{ha2018world, videoworldsimulators2024, agarwal2025cosmos}. Benefiting from advanced generative algorithms~\citep{goodfellow2014generative, ho2020denoising, liu2023flow, lipman2023flow}, scalable model architectures~\citep{vaswani2017attention, peebles2023scalable}, vast amounts of internet-sourced data~\citep{chen2024panda, nan2024openvid, ju2024miradata}, and ongoing expansion of computing capabilities~\citep{nvidia2022h100, nvidia2023dgxgh200, nvidia2024h200nvl}, remarkable advancements have been achieved in the field of video generation~\citep{ho2022video, ho2022imagen, singer2023makeavideo, blattmann2023align, videoworldsimulators2024, kuaishou2024klingai, yang2024cogvideox, jin2024pyramidal, polyak2024movie, kong2024hunyuanvideo, ji2024prompt}.


In this work, we present \textbf{\ours}, a family of rectified flow~\citep{lipman2023flow, liu2023flow} transformer models designed for joint image and video generation, establishing a pathway toward industry-grade performance. This report centers on four key components: data curation, model architecture design, flow formulation, and training infrastructure optimization—each rigorously refined to meet the demands of high-quality, large-scale video generation.


\begin{figure}[ht]
    \centering
    \begin{subfigure}[b]{0.82\linewidth}
        \centering
        \includegraphics[width=\linewidth]{figures/t2i_1024.pdf}
        \caption{Text-to-Image Samples}\label{fig:main-demo-t2i}
    \end{subfigure}
    \vfill
    \begin{subfigure}[b]{0.82\linewidth}
        \centering
        \includegraphics[width=\linewidth]{figures/t2v_samples.pdf}
        \caption{Text-to-Video Samples}\label{fig:main-demo-t2v}
    \end{subfigure}
\caption{\textbf{Generated samples from \ours.} Key components are highlighted in \textcolor{red}{\textbf{RED}}.}\label{fig:main-demo}
\end{figure}


First, we present a comprehensive data processing pipeline designed to construct large-scale, high-quality image and video-text datasets. The pipeline integrates multiple advanced techniques, including video and image filtering based on aesthetic scores, OCR-driven content analysis, and subjective evaluations, to ensure exceptional visual and contextual quality. Furthermore, we employ multimodal large language models~(MLLMs)~\citep{yuan2025tarsier2} to generate dense and contextually aligned captions, which are subsequently refined using an additional large language model~(LLM)~\citep{yang2024qwen2} to enhance their accuracy, fluency, and descriptive richness. As a result, we have curated a robust training dataset comprising approximately 36M video-text pairs and 160M image-text pairs, which are proven sufficient for training industry-level generative models.

Secondly, we take a pioneering step by applying rectified flow formulation~\citep{lipman2023flow} for joint image and video generation, implemented through the \ours model family, which comprises Transformer architectures with 2B and 8B parameters. At its core, the \ours framework employs a 3D joint image-video variational autoencoder (VAE) to compress image and video inputs into a shared latent space, facilitating unified representation. This shared latent space is coupled with a full-attention~\citep{vaswani2017attention} mechanism, enabling seamless joint training of image and video. This architecture delivers high-quality, coherent outputs across both images and videos, establishing a unified framework for visual generation tasks.


Furthermore, to support the training of \ours at scale, we have developed a robust infrastructure tailored for large-scale model training. Our approach incorporates advanced parallelism strategies~\citep{jacobs2023deepspeed, pytorch_fsdp} to manage memory efficiently during long-context training. Additionally, we employ ByteCheckpoint~\citep{wan2024bytecheckpoint} for high-performance checkpointing and integrate fault-tolerant mechanisms from MegaScale~\citep{jiang2024megascale} to ensure stability and scalability across large GPU clusters. These optimizations enable \ours to handle the computational and data challenges of generative modeling with exceptional efficiency and reliability.


We evaluate \ours on both text-to-image and text-to-video benchmarks to highlight its competitive advantages. For text-to-image generation, \ours-T2I demonstrates strong performance across multiple benchmarks, including T2I-CompBench~\citep{huang2023t2i-compbench}, GenEval~\citep{ghosh2024geneval}, and DPG-Bench~\citep{hu2024ella_dbgbench}, excelling in both visual quality and text-image alignment. In text-to-video benchmarks, \ours-T2V achieves state-of-the-art performance on the UCF-101~\citep{ucf101} zero-shot generation task. Additionally, \ours-T2V attains an impressive score of \textbf{84.85} on VBench~\citep{huang2024vbench}, securing the top position on the leaderboard (as of 2025-01-25) and surpassing several leading commercial text-to-video models. Qualitative results, illustrated in \Cref{fig:main-demo}, further demonstrate the superior quality of the generated media samples. These findings underscore \ours's effectiveness in multi-modal generation and its potential as a high-performing solution for both research and commercial applications.
% \input{content/2_preliminaries_and_technical_background}
% \input{content/3_main_results}
% \section{Experiments}

\subsection{Setups}
\subsubsection{Implementation Details}
We apply our FDS method to two types of 3DGS: 
the original 3DGS, and 2DGS~\citep{huang20242d}. 
%
The number of iterations in our optimization 
process is 35,000.
We follow the default training configuration 
and apply our FDS method after 15,000 iterations,
then we add normal consistency loss for both
3DGS and 2DGS after 25000 iterations.
%
The weight for FDS, $\lambda_{fds}$, is set to 0.015,
the $\sigma$ is set to 23,
and the weight for normal consistency is set to 0.15
for all experiments. 
We removed the depth distortion loss in 2DGS 
because we found that it degrades its results in indoor scenes.
%
The Gaussian point cloud is initialized using Colmap
for all datasets.
%
%
We tested the impact of 
using Sea Raft~\citep{wang2025sea} and 
Raft\citep{teed2020raft} on FDS performance.
%
Due to the blurriness of the ScanNet dataset, 
additional prior constraints are required.
Thus, we incorporate normal prior supervision 
on the rendered normals 
in ScanNet (V2) dataset by default.
The normal prior is predicted by the Stable Normal 
model~\citep{ye2024stablenormal}
across all types of 3DGS.
%
The entire framework is implemented in 
PyTorch~\citep{paszke2019pytorch}, 
and all experiments are conducted on 
a single NVIDIA 4090D GPU.

\begin{figure}[t] \centering
    \makebox[0.16\textwidth]{\scriptsize Input}
    \makebox[0.16\textwidth]{\scriptsize 3DGS}
    \makebox[0.16\textwidth]{\scriptsize 2DGS}
    \makebox[0.16\textwidth]{\scriptsize 3DGS + FDS}
    \makebox[0.16\textwidth]{\scriptsize 2DGS + FDS}
    \makebox[0.16\textwidth]{\scriptsize GT (Depth)}

    \includegraphics[width=0.16\textwidth]{figure/fig3_img/compare3/gt_rgb/frame_00522.jpg}
    \includegraphics[width=0.16\textwidth]{figure/fig3_img/compare3/3DGS/frame_00522.jpg}
    \includegraphics[width=0.16\textwidth]{figure/fig3_img/compare3/2DGS/frame_00522.jpg}
    \includegraphics[width=0.16\textwidth]{figure/fig3_img/compare3/3DGS+FDS/frame_00522.jpg}
    \includegraphics[width=0.16\textwidth]{figure/fig3_img/compare3/2DGS+FDS/frame_00522.jpg}
    \includegraphics[width=0.16\textwidth]{figure/fig3_img/compare3/gt_depth/frame_00522.jpg} \\

    % \includegraphics[width=0.16\textwidth]{figure/fig3_img/compare1/gt_rgb/frame_00137.jpg}
    % \includegraphics[width=0.16\textwidth]{figure/fig3_img/compare1/3DGS/frame_00137.jpg}
    % \includegraphics[width=0.16\textwidth]{figure/fig3_img/compare1/2DGS/frame_00137.jpg}
    % \includegraphics[width=0.16\textwidth]{figure/fig3_img/compare1/3DGS+FDS/frame_00137.jpg}
    % \includegraphics[width=0.16\textwidth]{figure/fig3_img/compare1/2DGS+FDS/frame_00137.jpg}
    % \includegraphics[width=0.16\textwidth]{figure/fig3_img/compare1/gt_depth/frame_00137.jpg} \\

     \includegraphics[width=0.16\textwidth]{figure/fig3_img/compare2/gt_rgb/frame_00262.jpg}
    \includegraphics[width=0.16\textwidth]{figure/fig3_img/compare2/3DGS/frame_00262.jpg}
    \includegraphics[width=0.16\textwidth]{figure/fig3_img/compare2/2DGS/frame_00262.jpg}
    \includegraphics[width=0.16\textwidth]{figure/fig3_img/compare2/3DGS+FDS/frame_00262.jpg}
    \includegraphics[width=0.16\textwidth]{figure/fig3_img/compare2/2DGS+FDS/frame_00262.jpg}
    \includegraphics[width=0.16\textwidth]{figure/fig3_img/compare2/gt_depth/frame_00262.jpg} \\

    \includegraphics[width=0.16\textwidth]{figure/fig3_img/compare4/gt_rgb/frame00000.png}
    \includegraphics[width=0.16\textwidth]{figure/fig3_img/compare4/3DGS/frame00000.png}
    \includegraphics[width=0.16\textwidth]{figure/fig3_img/compare4/2DGS/frame00000.png}
    \includegraphics[width=0.16\textwidth]{figure/fig3_img/compare4/3DGS+FDS/frame00000.png}
    \includegraphics[width=0.16\textwidth]{figure/fig3_img/compare4/2DGS+FDS/frame00000.png}
    \includegraphics[width=0.16\textwidth]{figure/fig3_img/compare4/gt_depth/frame00000.png} \\

    \includegraphics[width=0.16\textwidth]{figure/fig3_img/compare5/gt_rgb/frame00080.png}
    \includegraphics[width=0.16\textwidth]{figure/fig3_img/compare5/3DGS/frame00080.png}
    \includegraphics[width=0.16\textwidth]{figure/fig3_img/compare5/2DGS/frame00080.png}
    \includegraphics[width=0.16\textwidth]{figure/fig3_img/compare5/3DGS+FDS/frame00080.png}
    \includegraphics[width=0.16\textwidth]{figure/fig3_img/compare5/2DGS+FDS/frame00080.png}
    \includegraphics[width=0.16\textwidth]{figure/fig3_img/compare5/gt_depth/frame00080.png} \\



    \caption{\textbf{Comparison of depth reconstruction on Mushroom and ScanNet datasets.} The original
    3DGS or 2DGS model equipped with FDS can remove unwanted floaters and reconstruct
    geometry more preciously.}
    \label{fig:compare}
\end{figure}


\subsubsection{Datasets and Metrics}

We evaluate our method for 3D reconstruction 
and novel view synthesis tasks on
\textbf{Mushroom}~\citep{ren2024mushroom},
\textbf{ScanNet (v2)}~\citep{dai2017scannet}, and 
\textbf{Replica}~\citep{replica19arxiv}
datasets,
which feature challenging indoor scenes with both 
sparse and dense image sampling.
%
The Mushroom dataset is an indoor dataset 
with sparse image sampling and two distinct 
camera trajectories. 
%
We train our model on the training split of 
the long capture sequence and evaluate 
novel view synthesis on the test split 
of the long capture sequences.
%
Five scenes are selected to evaluate our FDS, 
including "coffee room", "honka", "kokko", 
"sauna", and "vr room". 
%
ScanNet(V2)~\citep{dai2017scannet}  consists of 1,613 indoor scenes
with annotated camera poses and depth maps. 
%
We select 5 scenes from the ScanNet (V2) dataset, 
uniformly sampling one-tenth of the views,
following the approach in ~\citep{guo2022manhattan}.
To further improve the geometry rendering quality of 3DGS, 
%
Replica~\citep{replica19arxiv} contains small-scale 
real-world indoor scans. 
We evaluate our FDS on five scenes from 
Replica: office0, office1, office2, office3 and office4,
selecting one-tenth of the views for training.
%
The results for Replica are provided in the 
supplementary materials.
To evaluate the rendering quality and geometry 
of 3DGS, we report PSNR, SSIM, and LPIPS for 
rendering quality, along with Absolute Relative Distance 
(Abs Rel) as a depth quality metrics.
%
Additionally, for mesh evaluation, 
we use metrics including Accuracy, Completion, 
Chamfer-L1 distance, Normal Consistency, 
and F-scores.




\subsection{Results}
\subsubsection{Depth rendering and novel view synthesis}
The comparison results on Mushroom and 
ScanNet are presented in \tabref{tab:mushroom} 
and \tabref{tab:scannet}, respectively. 
%
Due to the sparsity of sampling 
in the Mushroom dataset,
challenges are posed for both GOF~\citep{yu2024gaussian} 
and PGSR~\citep{chen2024pgsr}, 
leading to their relative poor performance 
on the Mushroom dataset.
%
Our approach achieves the best performance 
with the FDS method applied during the training process.
The FDS significantly enhances the 
geometric quality of 3DGS on the Mushroom dataset, 
improving the "abs rel" metric by more than 50\%.
%
We found that Sea Raft~\citep{wang2025sea}
outperforms Raft~\citep{teed2020raft} on FDS, 
indicating that a better optical flow model 
can lead to more significant improvements.
%
Additionally, the render quality of RGB 
images shows a slight improvement, 
by 0.58 in 3DGS and 0.50 in 2DGS, 
benefiting from the incorporation of cross-view consistency in FDS. 
%
On the Mushroom
dataset, adding the FDS loss increases 
the training time by half an hour, which maintains the same
level as baseline.
%
Similarly, our method shows a notable improvement on the ScanNet dataset as well using Sea Raft~\citep{wang2025sea} Model. The "abs rel" metric in 2DGS is improved nearly 50\%. This demonstrates the robustness and effectiveness of the FDS method across different datasets.
%


% \begin{wraptable}{r}{0.6\linewidth} \centering
% \caption{\textbf{Ablation study on geometry priors.}} 
%         \label{tab:analysis_prior}
%         \resizebox{\textwidth}{!}{
\begin{tabular}{c| c c c c c | c c c c}

    \hline
     Method &  Acc$\downarrow$ & Comp $\downarrow$ & C-L1 $\downarrow$ & NC $\uparrow$ & F-Score $\uparrow$ &  Abs Rel $\downarrow$ &  PSNR $\uparrow$  & SSIM  $\uparrow$ & LPIPS $\downarrow$ \\ \hline
    2DGS&   0.1078&  0.0850&  0.0964&  0.7835&  0.5170&  0.1002&  23.56&  0.8166& 0.2730\\
    2DGS+Depth&   0.0862&  0.0702&  0.0782&  0.8153&  0.5965&  0.0672&  23.92&  0.8227& 0.2619 \\
    2DGS+MVDepth&   0.2065&  0.0917&  0.1491&  0.7832&  0.3178&  0.0792&  23.74&  0.8193& 0.2692 \\
    2DGS+Normal&   0.0939&  0.0637&  0.0788&  \textbf{0.8359}&  0.5782&  0.0768&  23.78&  0.8197& 0.2676 \\
    2DGS+FDS &  \textbf{0.0615} & \textbf{ 0.0534}& \textbf{0.0574}& 0.8151& \textbf{0.6974}&  \textbf{0.0561}&  \textbf{24.06}&  \textbf{0.8271}&\textbf{0.2610} \\ \hline
    2DGS+Depth+FDS &  0.0561 &  0.0519& 0.0540& 0.8295& 0.7282&  0.0454&  \textbf{24.22}& \textbf{0.8291}&\textbf{0.2570} \\
    2DGS+Normal+FDS &  \textbf{0.0529} & \textbf{ 0.0450}& \textbf{0.0490}& \textbf{0.8477}& \textbf{0.7430}&  \textbf{0.0443}&  24.10&  0.8283& 0.2590 \\
    2DGS+Depth+Normal &  0.0695 & 0.0513& 0.0604& 0.8540&0.6723&  0.0523&  24.09&  0.8264&0.2575\\ \hline
    2DGS+Depth+Normal+FDS &  \textbf{0.0506} & \textbf{0.0423}& \textbf{0.0464}& \textbf{0.8598}&\textbf{0.7613}&  \textbf{0.0403}&  \textbf{24.22}& 
    \textbf{0.8300}&\textbf{0.0403}\\
    
\bottomrule
\end{tabular}
}
% \end{wraptable}



The qualitative comparisons on the Mushroom and ScanNet dataset 
are illustrated in \figref{fig:compare}. 
%
%
As seen in the first row of \figref{fig:compare}, 
both the original 3DGS and 2DGS suffer from overfitting, 
leading to corrupted geometry generation. 
%
Our FDS effectively mitigates this issue by 
supervising the matching relationship between 
the input and sampled views, 
helping to recover the geometry.
%
FDS also improves the refinement of geometric details, 
as shown in other rows. 
By incorporating the matching prior through FDS, 
the quality of the rendered depth is significantly improved.
%

\begin{table}[t] \centering
\begin{minipage}[t]{0.96\linewidth}
        \captionof{table}{\textbf{3D Reconstruction 
        and novel view synthesis results on Mushroom dataset. * 
        Represents that FDS uses the Raft model.
        }}
        \label{tab:mushroom}
        \resizebox{\textwidth}{!}{
\begin{tabular}{c| c c c c c | c c c c c}
    \hline
     Method &  Acc$\downarrow$ & Comp $\downarrow$ & C-L1 $\downarrow$ & NC $\uparrow$ & F-Score $\uparrow$ &  Abs Rel $\downarrow$ &  PSNR $\uparrow$  & SSIM  $\uparrow$ & LPIPS $\downarrow$ & Time  $\downarrow$ \\ \hline

    % DN-splatter &   &  &  &  &  &  &  &  & \\
    GOF &  0.1812 & 0.1093 & 0.1453 & 0.6292 & 0.3665 & 0.2380  & 21.37  &  0.7762  & 0.3132  & $\approx$1.4h\\ 
    PGSR &  0.0971 & 0.1420 & 0.1196 & 0.7193 & 0.5105 & 0.1723  & 22.13  & 0.7773  & 0.2918  & $\approx$1.2h \\ \hline
    3DGS &   0.1167 &  0.1033&  0.1100&  0.7954&  0.3739&  0.1214&  24.18&  0.8392& 0.2511 &$\approx$0.8h \\
    3DGS + FDS$^*$ & 0.0569  & 0.0676 & 0.0623 & 0.8105 & 0.6573 & 0.0603 & 24.72  & 0.8489 & 0.2379 &$\approx$1.3h \\
    3DGS + FDS & \textbf{0.0527}  & \textbf{0.0565} & \textbf{0.0546} & \textbf{0.8178} & \textbf{0.6958} & \textbf{0.0568} & \textbf{24.76}  & \textbf{0.8486} & \textbf{0.2381} &$\approx$1.3h \\ \hline
    2DGS&   0.1078&  0.0850&  0.0964&  0.7835&  0.5170&  0.1002&  23.56&  0.8166& 0.2730 &$\approx$0.8h\\
    2DGS + FDS$^*$ &  0.0689 &  0.0646& 0.0667& 0.8042& 0.6582& 0.0589& 23.98&  0.8255&0.2621 &$\approx$1.3h\\
    2DGS + FDS &  \textbf{0.0615} & \textbf{ 0.0534}& \textbf{0.0574}& \textbf{0.8151}& \textbf{0.6974}&  \textbf{0.0561}&  \textbf{24.06}&  \textbf{0.8271}&\textbf{0.2610} &$\approx$1.3h \\ \hline
\end{tabular}
}
\end{minipage}\hfill
\end{table}

\begin{table}[t] \centering
\begin{minipage}[t]{0.96\linewidth}
        \captionof{table}{\textbf{3D Reconstruction 
        and novel view synthesis results on ScanNet dataset.}}
        \label{tab:scannet}
        \resizebox{\textwidth}{!}{
\begin{tabular}{c| c c c c c | c c c c }
    \hline
     Method &  Acc $\downarrow$ & Comp $\downarrow$ & C-L1 $\downarrow$ & NC $\uparrow$ & F-Score $\uparrow$ &  Abs Rel $\downarrow$ &  PSNR $\uparrow$  & SSIM  $\uparrow$ & LPIPS $\downarrow$ \\ \hline
    GOF & 1.8671  & 0.0805 & 0.9738 & 0.5622 & 0.2526 & 0.1597  & 21.55  & 0.7575  & 0.3881 \\
    PGSR &  0.2928 & 0.5103 & 0.4015 & 0.5567 & 0.1926 & 0.1661  & 21.71 & 0.7699  & 0.3899 \\ \hline

    3DGS &  0.4867 & 0.1211 & 0.3039 & 0.7342& 0.3059 & 0.1227 & 22.19& 0.7837 & 0.3907\\
    3DGS + FDS &  \textbf{0.2458} & \textbf{0.0787} & \textbf{0.1622} & \textbf{0.7831} & 
    \textbf{0.4482} & \textbf{0.0573} & \textbf{22.83} & \textbf{0.7911} & \textbf{0.3826} \\ \hline
    2DGS &  0.2658 & 0.0845 & 0.1752 & 0.7504& 0.4464 & 0.0831 & 22.59& 0.7881 & 0.3854\\
    2DGS + FDS &  \textbf{0.1457} & \textbf{0.0679} & \textbf{0.1068} & \textbf{0.7883} & 
    \textbf{0.5459} & \textbf{0.0432} & \textbf{22.91} & \textbf{0.7928} & \textbf{0.3800} \\ \hline
\end{tabular}
}
\end{minipage}\hfill
\end{table}


\begin{table}[t] \centering
\begin{minipage}[t]{0.96\linewidth}
        \captionof{table}{\textbf{Ablation study on geometry priors.}}
        \label{tab:analysis_prior}
        \resizebox{\textwidth}{!}{
\begin{tabular}{c| c c c c c | c c c c}

    \hline
     Method &  Acc$\downarrow$ & Comp $\downarrow$ & C-L1 $\downarrow$ & NC $\uparrow$ & F-Score $\uparrow$ &  Abs Rel $\downarrow$ &  PSNR $\uparrow$  & SSIM  $\uparrow$ & LPIPS $\downarrow$ \\ \hline
    2DGS&   0.1078&  0.0850&  0.0964&  0.7835&  0.5170&  0.1002&  23.56&  0.8166& 0.2730\\
    2DGS+Depth&   0.0862&  0.0702&  0.0782&  0.8153&  0.5965&  0.0672&  23.92&  0.8227& 0.2619 \\
    2DGS+MVDepth&   0.2065&  0.0917&  0.1491&  0.7832&  0.3178&  0.0792&  23.74&  0.8193& 0.2692 \\
    2DGS+Normal&   0.0939&  0.0637&  0.0788&  \textbf{0.8359}&  0.5782&  0.0768&  23.78&  0.8197& 0.2676 \\
    2DGS+FDS &  \textbf{0.0615} & \textbf{ 0.0534}& \textbf{0.0574}& 0.8151& \textbf{0.6974}&  \textbf{0.0561}&  \textbf{24.06}&  \textbf{0.8271}&\textbf{0.2610} \\ \hline
    2DGS+Depth+FDS &  0.0561 &  0.0519& 0.0540& 0.8295& 0.7282&  0.0454&  \textbf{24.22}& \textbf{0.8291}&\textbf{0.2570} \\
    2DGS+Normal+FDS &  \textbf{0.0529} & \textbf{ 0.0450}& \textbf{0.0490}& \textbf{0.8477}& \textbf{0.7430}&  \textbf{0.0443}&  24.10&  0.8283& 0.2590 \\
    2DGS+Depth+Normal &  0.0695 & 0.0513& 0.0604& 0.8540&0.6723&  0.0523&  24.09&  0.8264&0.2575\\ \hline
    2DGS+Depth+Normal+FDS &  \textbf{0.0506} & \textbf{0.0423}& \textbf{0.0464}& \textbf{0.8598}&\textbf{0.7613}&  \textbf{0.0403}&  \textbf{24.22}& 
    \textbf{0.8300}&\textbf{0.0403}\\
    
\bottomrule
\end{tabular}
}
\end{minipage}\hfill
\end{table}




\subsubsection{Mesh extraction}
To further demonstrate the improvement in geometry quality, 
we applied methods used in ~\citep{turkulainen2024dnsplatter} 
to extract meshes from the input views of optimized 3DGS. 
The comparison results are presented  
in \tabref{tab:mushroom}. 
With the integration of FDS, the mesh quality is significantly enhanced compared to the baseline, featuring fewer floaters and more well-defined shapes.
 %
% Following the incorporation of FDS, the reconstruction 
% results exhibit fewer floaters and more well-defined 
% shapes in the meshes. 
% Visualized comparisons
% are provided in the supplementary material.

% \begin{figure}[t] \centering
%     \makebox[0.19\textwidth]{\scriptsize GT}
%     \makebox[0.19\textwidth]{\scriptsize 3DGS}
%     \makebox[0.19\textwidth]{\scriptsize 3DGS+FDS}
%     \makebox[0.19\textwidth]{\scriptsize 2DGS}
%     \makebox[0.19\textwidth]{\scriptsize 2DGS+FDS} \\

%     \includegraphics[width=0.19\textwidth]{figure/fig4_img/compare1/gt02.png}
%     \includegraphics[width=0.19\textwidth]{figure/fig4_img/compare1/baseline06.png}
%     \includegraphics[width=0.19\textwidth]{figure/fig4_img/compare1/baseline_fds05.png}
%     \includegraphics[width=0.19\textwidth]{figure/fig4_img/compare1/2dgs04.png}
%     \includegraphics[width=0.19\textwidth]{figure/fig4_img/compare1/2dgs_fds03.png} \\

%     \includegraphics[width=0.19\textwidth]{figure/fig4_img/compare2/gt00.png}
%     \includegraphics[width=0.19\textwidth]{figure/fig4_img/compare2/baseline02.png}
%     \includegraphics[width=0.19\textwidth]{figure/fig4_img/compare2/baseline_fds01.png}
%     \includegraphics[width=0.19\textwidth]{figure/fig4_img/compare2/2dgs04.png}
%     \includegraphics[width=0.19\textwidth]{figure/fig4_img/compare2/2dgs_fds03.png} \\
      
%     \includegraphics[width=0.19\textwidth]{figure/fig4_img/compare3/gt05.png}
%     \includegraphics[width=0.19\textwidth]{figure/fig4_img/compare3/3dgs03.png}
%     \includegraphics[width=0.19\textwidth]{figure/fig4_img/compare3/3dgs_fds04.png}
%     \includegraphics[width=0.19\textwidth]{figure/fig4_img/compare3/2dgs02.png}
%     \includegraphics[width=0.19\textwidth]{figure/fig4_img/compare3/2dgs_fds01.png} \\

%     \caption{\textbf{Qualitative comparison of extracted mesh 
%     on Mushroom and ScanNet datasets.}}
%     \label{fig:mesh}
% \end{figure}












\subsection{Ablation study}


\textbf{Ablation study on geometry priors:} 
To highlight the advantage of incorporating matching priors, 
we incorporated various types of priors generated by different 
models into 2DGS. These include a monocular depth estimation
model (Depth Anything v2)~\citep{yang2024depth}, a two-view depth estimation 
model (Unimatch)~\citep{xu2023unifying}, 
and a monocular normal estimation model (DSINE)~\citep{bae2024rethinking}.
We adapt the scale and shift-invariant loss in Midas~\citep{birkl2023midas} for
monocular depth supervision and L1 loss for two-view depth supervison.
%
We use Sea Raft~\citep{wang2025sea} as our default optical flow model.
%
The comparison results on Mushroom dataset 
are shown in ~\tabref{tab:analysis_prior}.
We observe that the normal prior provides accurate shape information, 
enhancing the geometric quality of the radiance field. 
%
% In contrast, the monocular depth prior slightly increases 
% the 'Abs Rel' due to its ambiguous scale and inaccurate depth ordering.
% Moreover, the performance of monocular depth estimation 
% in the sauna scene is particularly poor, 
% primarily due to the presence of numerous reflective 
% surfaces and textureless walls, which limits the accuracy of monocular depth estimation.
%
The multi-view depth prior, hindered by the limited feature overlap 
between input views, fails to offer reliable geometric 
information. We test average "Abs Rel" of multi-view depth prior
, and the result is 0.19, which performs worse than the "Abs Rel" results 
rendered by original 2DGS.
From the results, it can be seen that depth order information provided by monocular depth improves
reconstruction accuracy. Meanwhile, our FDS achieves the best performance among all the priors, 
and by integrating all
three components, we obtained the optimal results.
%
%
\begin{figure}[t] \centering
    \makebox[0.16\textwidth]{\scriptsize RF (16000 iters)}
    \makebox[0.16\textwidth]{\scriptsize RF* (20000 iters)}
    \makebox[0.16\textwidth]{\scriptsize RF (20000 iters)  }
    \makebox[0.16\textwidth]{\scriptsize PF (16000 iters)}
    \makebox[0.16\textwidth]{\scriptsize PF (20000 iters)}


    % \includegraphics[width=0.16\textwidth]{figure/fig5_img/compare1/16000.png}
    % \includegraphics[width=0.16\textwidth]{figure/fig5_img/compare1/20000_wo_flow_loss.png}
    % \includegraphics[width=0.16\textwidth]{figure/fig5_img/compare1/20000.png}
    % \includegraphics[width=0.16\textwidth]{figure/fig5_img/compare1/16000_prior.png}
    % \includegraphics[width=0.16\textwidth]{figure/fig5_img/compare1/20000_prior.png}\\

    % \includegraphics[width=0.16\textwidth]{figure/fig5_img/compare2/16000.png}
    % \includegraphics[width=0.16\textwidth]{figure/fig5_img/compare2/20000_wo_flow_loss.png}
    % \includegraphics[width=0.16\textwidth]{figure/fig5_img/compare2/20000.png}
    % \includegraphics[width=0.16\textwidth]{figure/fig5_img/compare2/16000_prior.png}
    % \includegraphics[width=0.16\textwidth]{figure/fig5_img/compare2/20000_prior.png}\\

    \includegraphics[width=0.16\textwidth]{figure/fig5_img/compare3/16000.png}
    \includegraphics[width=0.16\textwidth]{figure/fig5_img/compare3/20000_wo_flow_loss.png}
    \includegraphics[width=0.16\textwidth]{figure/fig5_img/compare3/20000.png}
    \includegraphics[width=0.16\textwidth]{figure/fig5_img/compare3/16000_prior.png}
    \includegraphics[width=0.16\textwidth]{figure/fig5_img/compare3/20000_prior.png}\\
    
    \includegraphics[width=0.16\textwidth]{figure/fig5_img/compare4/16000.png}
    \includegraphics[width=0.16\textwidth]{figure/fig5_img/compare4/20000_wo_flow_loss.png}
    \includegraphics[width=0.16\textwidth]{figure/fig5_img/compare4/20000.png}
    \includegraphics[width=0.16\textwidth]{figure/fig5_img/compare4/16000_prior.png}
    \includegraphics[width=0.16\textwidth]{figure/fig5_img/compare4/20000_prior.png}\\

    \includegraphics[width=0.30\textwidth]{figure/fig5_img/bar.png}

    \caption{\textbf{The error map of Radiance Flow and Prior Flow.} RF: Radiance Flow, PF: Prior Flow, * means that there is no FDS loss supervision during optimization.}
    \label{fig:error_map}
\end{figure}




\textbf{Ablation study on FDS: }
In this section, we present the design of our FDS 
method through an ablation study on the 
Mushroom dataset to validate its effectiveness.
%
The optional configurations of FDS are outlined in ~\tabref{tab:ablation_fds}.
Our base model is the 2DGS equipped with FDS,
and its results are shown 
in the first row. The goal of this analysis 
is to evaluate the impact 
of various strategies on FDS sampling and loss design.
%
We observe that when we 
replace $I_i$ in \eqref{equ:mflow} with $C_i$, 
as shown in the second row, the geometric quality 
of 2DGS deteriorates. Using $I_i$ instead of $C_i$ 
help us to remove the floaters in $\bm{C^s}$, which are also 
remained in $\bm{C^i}$.
We also experiment with modifying the FDS loss. For example, 
in the third row, we use the neighbor 
input view as the sampling view, and replace the 
render result of neighbor view with ground truth image of its input view.
%
Due to the significant movement between images, the Prior Flow fails to accurately 
match the pixel between them, leading to a further degradation in geometric quality.
%
Finally, we attempt to fix the sampling view 
and found that this severely damaged the geometric quality, 
indicating that random sampling is essential for the stability 
of the mean error in the Prior flow.



\begin{table}[t] \centering

\begin{minipage}[t]{1.0\linewidth}
        \captionof{table}{\textbf{Ablation study on FDS strategies.}}
        \label{tab:ablation_fds}
        \resizebox{\textwidth}{!}{
\begin{tabular}{c|c|c|c|c|c|c|c}
    \hline
    \multicolumn{2}{c|}{$\mathcal{M}_{\theta}(X, \bm{C^s})$} & \multicolumn{3}{c|}{Loss} & \multicolumn{3}{c}{Metric}  \\
    \hline
    $X=C^i$ & $X=I^i$  & Input view & Sampled view     & Fixed Sampled view        & Abs Rel $\downarrow$ & F-score $\uparrow$ & NC $\uparrow$ \\
    \hline
    & \ding{51} &     &\ding{51}    &    &    \textbf{0.0561}        &  \textbf{0.6974}         & \textbf{0.8151}\\
    \hline
     \ding{51} &           &     &\ding{51}    &    &    0.0839        &  0.6242         &0.8030\\
     &  \ding{51} &   \ding{51}  &    &    &    0.0877       & 0.6091        & 0.7614 \\
      &  \ding{51} &    &    & \ding{51}    &    0.0724           & 0.6312          & 0.8015 \\
\bottomrule
\end{tabular}
}
\end{minipage}
\end{table}




\begin{figure}[htbp] \centering
    \makebox[0.22\textwidth]{}
    \makebox[0.22\textwidth]{}
    \makebox[0.22\textwidth]{}
    \makebox[0.22\textwidth]{}
    \\

    \includegraphics[width=0.22\textwidth]{figure/fig6_img/l1/rgb/frame00096.png}
    \includegraphics[width=0.22\textwidth]{figure/fig6_img/l1/render_rgb/frame00096.png}
    \includegraphics[width=0.22\textwidth]{figure/fig6_img/l1/render_depth/frame00096.png}
    \includegraphics[width=0.22\textwidth]{figure/fig6_img/l1/depth/frame00096.png}

    % \includegraphics[width=0.22\textwidth]{figure/fig6_img/l2/rgb/frame00112.png}
    % \includegraphics[width=0.22\textwidth]{figure/fig6_img/l2/render_rgb/frame00112.png}
    % \includegraphics[width=0.22\textwidth]{figure/fig6_img/l2/render_depth/frame00112.png}
    % \includegraphics[width=0.22\textwidth]{figure/fig6_img/l2/depth/frame00112.png}

    \caption{\textbf{Limitation of FDS.} }
    \label{fig:limitation}
\end{figure}


% \begin{figure}[t] \centering
%     \makebox[0.48\textwidth]{}
%     \makebox[0.48\textwidth]{}
%     \\
%     \includegraphics[width=0.48\textwidth]{figure/loss_Ignatius.pdf}
%     \includegraphics[width=0.48\textwidth]{figure/loss_family.pdf}
%     \caption{\textbf{Comparison the photometric error of Radiance Flow and Prior Flow:} 
%     We add FDS method after 2k iteration during training.
%     The results show
%     that:  1) The Prior Flow is more precise and 
%     robust than Radiance Flow during the radiance 
%     optimization; 2) After adding the FDS loss 
%     which utilize Prior 
%     flow to supervise the Radiance Flow at 2k iterations, 
%     both flow are more accurate, which lead to
%     a mutually reinforcing effects.(TODO fix it)} 
%     \label{fig:flowcompare}
% \end{figure}






\textbf{Interpretive Experiments: }
To demonstrate the mutual refinement of two flows in our FDS, 
For each view, we sample the unobserved 
views multiple times to compute the mean error 
of both Radiance Flow and Prior Flow. 
We use Raft~\citep{teed2020raft} as our default optical flow model
for visualization.
The ground truth flow is calculated based on 
~\eref{equ:flow_pose} and ~\eref{equ:flow} 
utilizing ground truth depth in dataset.
We introduce our FDS loss after 16000 iterations during 
optimization of 2DGS.
The error maps are shown in ~\figref{fig:error_map}.
Our analysis reveals that Radiance Flow tends to 
exhibit significant geometric errors, 
whereas Prior Flow can more accurately estimate the geometry,
effectively disregarding errors introduced by floating Gaussian points. 

%





\subsection{Limitation and further work}

Firstly, our FDS faces challenges in scenes with 
significant lighting variations between different 
views, as shown in the lamp of first row in ~\figref{fig:limitation}. 
%
Incorporating exposure compensation into FDS could help address this issue. 
%
 Additionally, our method struggles with 
 reflective surfaces and motion blur,
 leading to incorrect matching. 
 %
 In the future, we plan to explore the potential 
 of FDS in monocular video reconstruction tasks, 
 using only a single input image at each time step.
 


\section{Conclusions}
In this paper, we propose Flow Distillation Sampling (FDS), which
leverages the matching prior between input views and 
sampled unobserved views from the pretrained optical flow model, to improve the geometry quality
of Gaussian radiance field. 
Our method can be applied to different approaches (3DGS and 2DGS) to enhance the geometric rendering quality of the corresponding neural radiance fields.
We apply our method to the 3DGS-based framework, 
and the geometry is enhanced on the Mushroom, ScanNet, and Replica datasets.

\section*{Acknowledgements} This work was supported by 
National Key R\&D Program of China (2023YFB3209702), 
the National Natural Science Foundation of 
China (62441204, 62472213), and Gusu 
Innovation \& Entrepreneurship Leading Talents Program (ZXL2024361)
% \input{content/5_real_experiments}
% \paragraph{Summary}
Our findings provide significant insights into the influence of correctness, explanations, and refinement on evaluation accuracy and user trust in AI-based planners. 
In particular, the findings are three-fold: 
(1) The \textbf{correctness} of the generated plans is the most significant factor that impacts the evaluation accuracy and user trust in the planners. As the PDDL solver is more capable of generating correct plans, it achieves the highest evaluation accuracy and trust. 
(2) The \textbf{explanation} component of the LLM planner improves evaluation accuracy, as LLM+Expl achieves higher accuracy than LLM alone. Despite this improvement, LLM+Expl minimally impacts user trust. However, alternative explanation methods may influence user trust differently from the manually generated explanations used in our approach.
% On the other hand, explanations may help refine the trust of the planner to a more appropriate level by indicating planner shortcomings.
(3) The \textbf{refinement} procedure in the LLM planner does not lead to a significant improvement in evaluation accuracy; however, it exhibits a positive influence on user trust that may indicate an overtrust in some situations.
% This finding is aligned with prior works showing that iterative refinements based on user feedback would increase user trust~\cite{kunkel2019let, sebo2019don}.
Finally, the propensity-to-trust analysis identifies correctness as the primary determinant of user trust, whereas explanations provided limited improvement in scenarios where the planner's accuracy is diminished.

% In conclusion, our results indicate that the planner's correctness is the dominant factor for both evaluation accuracy and user trust. Therefore, selecting high-quality training data and optimizing the training procedure of AI-based planners to improve planning correctness is the top priority. Once the AI planner achieves a similar correctness level to traditional graph-search planners, strengthening its capability to explain and refine plans will further improve user trust compared to traditional planners.

\paragraph{Future Research} Future steps in this research include expanding user studies with larger sample sizes to improve generalizability and including additional planning problems per session for a more comprehensive evaluation. Next, we will explore alternative methods for generating plan explanations beyond manual creation to identify approaches that more effectively enhance user trust. 
Additionally, we will examine user trust by employing multiple LLM-based planners with varying levels of planning accuracy to better understand the interplay between planning correctness and user trust. 
Furthermore, we aim to enable real-time user-planner interaction, allowing users to provide feedback and refine plans collaboratively, thereby fostering a more dynamic and user-centric planning process.



\begin{abstract}
In this paper, we leverage existing statistical methods to better understand feature learning from data.
We tackle this by modifying the model-free variable selection method, Feature Ordering by Conditional Independence (FOCI), which is introduced in \cite{azadkia2021simple}. 
While FOCI is based on a non-parametric coefficient of conditional dependence, we introduce its parametric, differentiable approximation. With this approximate coefficient of correlation, we present a new algorithm called difFOCI, which is applicable to a wider range of machine learning problems thanks to its differentiable nature and learnable parameters.
We present difFOCI in three contexts: (1) as a variable selection method with baseline comparisons to FOCI, 
(2) as a trainable model parametrized with a neural network, 
and (3) as a generic, widely applicable neural network regularizer, one that improves feature learning with better management of spurious correlations. We evaluate difFOCI on increasingly complex problems ranging from basic variable selection in toy examples to saliency map comparisons in convolutional networks. We then show how difFOCI can be incorporated in the context of fairness to facilitate classifications without relying on sensitive data.
\end{abstract}

\section{Introduction}

\looseness=-1Feature learning is crucial in machine learning (ML), enabling models to learn meaningful representations of the data. It can improve performance, reduce dimensionality, increase interpretability, and provide flexibility for adapting to new data distributions and tasks \citep{bengio2012unsupervised, bengio2013representation}. However, increasing model transparency \citep{arrieta2020explainable,rauker2023toward}, improving disentanglement and understanding architectural biases \citep{bouchacourt2021grounding, roth2022disentanglement}, as well as learning invariances to improve robustness \citep{arjovsky2019invariant} have proven to be challenging. 



\looseness=-1In this paper, we propose a new feature-learning approach that relies on ranks, notion seldom explored in the literature but thoroughly studied in statistics. The importance of ranks is evident, from independence tests \citep{bergsma2014consistent, blum1961distribution,  csorgHo1985testing, deb2023multivariate,drton2020high} and sensitivity analysis \citep{gamboa2018sensitivity}, to multivariate analysis \citep{sen1971nonparametric} and measuring deviation \citep{rosenblatt1975quadratic}. However, most of these methods are nonparametric and, therefore, not easily extendable to feature learning with neural networks (NNs). While there are a handful of feature learning works that rely on rank notions \citep{kuo2017feature, wojtas2020feature, fan2023few,li2023deep}, these works do so indirectly and through reliance on two NNs; one that optimizes for a non-rank-based feature-learning objective and another that learns how to rank those learned features according to some similarity measure. 


To fill this gap, we propose difFOCI, a parametric relaxation of the nonparametric, rank-based measures of correlation \citep{chatterjee2020original, azadkia2021simple}, which generalizes the measure proposed by \citet{dette2013copula}, roots of the idea can be related to the Rényi correlation \citep{bickel1993efficient, renyi1959measures}. To the best of our knowledge, difFOCI is the first parametric framework that directly optimizes a rank-based objective, making it directly applicable to numerous applications in machine learning, including end-to-end trainable neural networks. We demonstrate that difFOCI yields strong results in various areas, including \emph{(i)} feature selection, \emph{(ii)} domain shift and spurious correlation, and \emph{(iii)} fairness experiments.


\paragraph{Organization of the paper.} In Section 2, we introduce the notation and technical background. In Section 3, we outline the main results of this paper, explaining the proposed metric, and establishing its theoretical properties. We analyze it in toy examples that demonstrate solid performance. In Section 4, we extend difFOCI, showcasing its strengths in three examples. In Section 5, we highlight its wide applicability to real-world data, showing it achieves state-of-the-art performance on feature selection and dimensionality reduction, and competitive performance in domain shift and fairness literature. Finally, in Section 6, we conclude with a few remarks on the potential future applications.


\section{Preliminaries and technical background}
\label{sec:preliminaries_and_technical}

\subsection{Notation and preliminary definitions}

We let $\Idd$ denote the $d \times d$ identity matrix and $[n] = \{1, \dots, n \}$. We let $S(A)=\pi_1(A),...,\pi_{n!}(A)$ be the set of all permutations of a set $A$, with $|A|=n$. For a matrix $\bfX$, we denote the set of all permutations its columns by $S(\bfX)$ and by $\pi_i^j(\bfX)$, we represent the $i$-th element of the $j$-th permutation. We denote its $p$-th through $q$-th column as ${\bf{X}}_{p:q}$, with $p>q$, $p,q\in\mathbb{N}$.  We define the Hadamard product between a vector $\mathbf{\alpha}\in \mathbb{R}^p$ and a matrix $\mathbf{X}\in\mathbb{R}^{n,p}$ as $(\alpha\odot\mathbf{X})_{i,j}:=\alpha_i\mathbf{X}_{i,j}$. We represent the scaled Softmax function with $\sigma_\beta(x)$, where $\sigma_\beta(x)_{i}=e^{\beta x_i}/\sum_{j=1}^d e^{\beta x_j}$, for $x\in\mathbb{R}^d$, $\beta\in\mathbb{R}^+$. Finally, we use $c(x,p)$ to denote zeroing out any $x_i$ with $|x_i|\leq p$, for $x\in\mathbb{R}^d$ and $p\in\mathbb{R}$.

\subsection{Chaterjee's coefficient}
\label{sec:chaterjee}

We present the novel rank-based estimator developed by \citet{chatterjee2020original}, which is the first of two foundational works necessary for our approach. Consider a random vector $(X, Y)$ on a probability space $(\Omega ,\mathcal {F},\mathbb{P})$, with $Y$ being non-constant and governed by the law $\mu$. The estimator approximates the following rank-based measure \citep{dette2013copula}:


\begin{align}
        \xi(X, Y):=\frac{\int \operatorname{Var}\left(\mathbb{E}\left(\mathbbm{1}_{\{Y \geq t\}} \mid X\right)\right) d \mu(t)}{\int \operatorname{Var}\left(\mathbbm1_{\{Y \geq t\}}\right) d \mu(t)}.
        \label{eqn:chaterjee_measure}
\end{align}

\looseness=-1\citet{chatterjee2020original} establishes a straightforward estimator for (\ref{eqn:chaterjee_measure}) that has simple asymptotic theory, enjoys several consistency results and exhibits several natural properties; \emph{(i)} normalization: $\xi(X, Y)\in[0,1]$, \emph{(ii)} independence: $\xi(X, Y)=0 \iff Y \independent X$, \emph{(iii)} complete dependence: $\xi(X, Y)=1 \iff Y \text{ a measurable function of } X$ a.s., and \emph{(iv)} scale invariance: $\xi(aX, Y)=\xi(X, Y), a\in\mathbb{R}^*$.  To estimate $\xi$, consider \iid~pairs $\left(X_i, Y_i\right)_{i=1}^n\sim(X, Y)$, with $n \geq 2$. Rearrange the data as $\left(X_{(1)}, Y_{(1)}\right), \ldots,\left(X_{(n)}, Y_{(n)}\right)$, such that $X_{(1)} \leq \cdots \leq X_{(n)}$, breaking ties uniformly at random. Define $r_i$ as the rank of $Y_{(i)}$, i.e., the number of $j$ for which $Y_{(j)} \leq Y_{(i)}$, and $l_i$ as the number of $j$ such that $Y_{(j)} \geq Y_{(i)}$. The estimator is then defined as:

\begin{align}
    \xi_n(X, Y):=1-\frac{n \sum_{i=1}^{n-1}\left|r_{i+1}-r_i\right|}{2 \sum_{i=1}^n l_i\left(n-l_i\right)}.
    \label{eqn:chaterjee_estim}
\end{align}


Furthermore, \citet{chatterjee2020original} establishes the following consistency result for $\xi_n$:

\begin{theorem}\citep{chatterjee2020original}
    If $Y$ is not almost surely a constant, then as $n \rightarrow \infty$, $\xi_n(X, Y)$ converges almost surely to the deterministic limit $\xi(X, Y)$.
    \label{chaterjee_thm1}
\end{theorem}


\looseness=-1In simulations by \citet{chatterjee2020original}, this estimator demonstrates greater efficacy than most signal-detection tests. Its applications span diverse areas: approximate unlearning \citep{mehta2022deep}, topology \citep{deb2020measuring}, black carbon concentration estimation \citep{tang2023black}, sensitivity analysis \citep{gamboa2022global}, and causal discovery \citep{li2023nonlinear}. Extensive further research has been conducted: its limiting variance under independence \citep{han2022azadkiachatterjees}, permutation testing \citep{kim2022local}, bootstrapping \citep{lin2024failure}, rate efficiency \citep{lin2023boosting}, minimax optimality \citep{auddy2023exact} and kernel extension \citep{huang2022kernel}; \citet{bickel2022measures} analyzed it for independence testing, showing it might have no power or prove misleading\footnote{This does not impact us, however, as we do not utilize it for independence testing.}.

\subsection{Extending the coefficient for estimating conditional dependence}


In a subsequent study, \citet{azadkia2021simple} extend the coefficient (\ref{eqn:chaterjee_measure}) $\xi$ to a measure $T(Y, \bfZ \mid \bfX)$, capturing the strength of the conditional dependence between $Y$ and $\bfZ$, given $\bfX$. $T$ can be interpreted as a non-linear extension of the partial $R^2$ statistic \citep{draper1998applied}, and reads as follows: 

\begin{align*}
    T=T(Y, \mathbf{Z} \mid \mathbf{X}):=\frac{\int \mathbb{E}(\operatorname{Var}(\mathbb{P}(Y \geq t \mid \mathbf{Z}, \mathbf{X}) \mid \mathbf{X})) d \mu(t)}{\int \mathbb{E}\left(\operatorname{Var}\left(\mathbbm{1}_{\{Y \geq t\}} \mid \mathbf{X}\right)\right) d \mu(t)},
\end{align*}


where $Y$ denotes a random variable governed by $\mu$, and $\mathbf{X} = (X_1, \ldots, X_p)$ and $\mathbf{Z} = (Z_1, \ldots, Z_q)$ are random vectors, defined within the same probability space, with \iid~copies $\left(\bfX_i, \bfZ_i, Y_i\right)_{i=1}^n\sim(\bfX, \bfZ, Y), n\geq2$. Here, $q \geq 1$ and $p \geq 0$, with $p = 0$ indicating $\mathbf{X}$ has no components.



The statistic $T$ generalizes the univariate measure in (\ref{eqn:chaterjee_measure}). To construct its estimator, for each index $i$, define $N(i)$ as the index $j$ where $\mathbf{X}_j$ is the closest to $\mathbf{X}_i$, and $M(i)$ as the index $j$ where the pair $(\mathbf{X}_j, \mathbf{Z}_j)$ is closest to $(\mathbf{X}_i, \mathbf{Z}_i)$ in $\mathbb{R}^{p+q}$ w.r.t. the Euclidean metric and resolving ties randomly. The estimate of $T$ is given by:

\begin{align}
    T_n=T_n(Y, \mathbf{Z} \mid \mathbf{X}):=\frac{\sum_{i=1}^n\left(\min \left\{r_i, r_{M(i)}\right\}-\min \left\{r_i, r_{N(i)}\right\}\right)}{\sum_{i=1}^n\left(r_i-\min \left\{r_i, r_{N(i)}\right\}\right)}.
    \label{eqn:mona_estim_with_x}
\end{align}


with $M(i)$ denoting the index $j$ such that $\mathbf{Z}_j$ is the nearest neighbor of $\mathbf{Z}_i$, $p\geq1$ and $r_i$, $l_i$ as defined in Sec. \ref{sec:chaterjee}\footnote{The expression for $p=0$ is given in Appendix \ref{appx:aux_results}.}. The authors establish the same four natural properties for $T$ as for the estimator in (\ref{eqn:chaterjee_measure}) - normalization, independence, complete dependence, and scale invariance:


\begin{theorem}\citep{azadkia2021simple}
    Suppose that $Y$ is not almost surely equal to a measurable function of $\mathbf{X}$. Then $T$ is well-defined and $0 \leq T \leq 1$. Moreover, $T=0$ iff $Y$ and $\mathbf{Z}$ are conditionally independent given $\mathbf{X}$, and $T=1$ iff $Y$ is almost surely equal to a measurable function of $\mathbf{Z}$ given $\mathbf{X}$.
    \label{thm:21_mona}
\end{theorem} 

The authors further demonstrate that $T_n$ is indeed a consistent estimator of $T$:

\begin{theorem}\citep{azadkia2021simple}
    Suppose that $Y$ is not almost surely equal to a measurable function of $\mathbf{X}$. Then as $n \rightarrow \infty, T_n \rightarrow T$ almost surely.
    \label{thm:22_mona}
\end{theorem} 



\subsection{FOCI: A new paradigm for feature selection}

 \looseness=-1\citet{azadkia2021simple} utilize the estimator $T_n$ to propose a novel, model-independent, step-wise feature selection method. The method, termed FOCI: Feature Ordering by Conditional Independence, is free from tuning parameters and demonstrates provable consistency. FOCI is outlined in Alg. \ref{algo:FOCI}, where we observe its iterative nature: variables are chosen one by one until the estimator's value drops below zero.

\begin{algorithm}[htbp]
\caption{FOCI}
\begin{algorithmic}
\STATE \textbf{Input:} $n$ \iid~copies of $(Y, \bfX)$, with the set of predictors $\mathbf{X}=$ $\left(X_j\right)_{j\in[p]}$ and response $Y$
\STATE $j_1\gets\argmax_{j\in[p]} T_n(Y, X_j)$
\IF{$T_n(Y,X_{j_1})\leq0$}
    \STATE $\hat{S}=\emptyset$
\ELSE
    \WHILE{$T_n\left(Y, X_j \mid X_{j_1}, \ldots, X_{j_k}\right)>0$}
        \STATE $j_{k+1}\gets\argmax_{[p]\setminus\{j_1,...,j_k\}}$ $T_n\left(Y, X_j \mid X_{j_1}, \ldots, X_{j_k}\right)$ %\hfill // Choose index that maximizes $T_n$
    \ENDWHILE
    \STATE $\hat{S}=\{j_1,...,j_{k'}\}$
    
    \ENDIF
\STATE \textbf{Output:} Set $\hat{S}$ of chosen predictors' indices
\end{algorithmic}
\label{algo:FOCI}
\end{algorithm} 


FOCI performs well on both simulated and real-world datasets. In a toy example with $Y = X_1 X_2 + \sin(X_1 X_3)$, where $X_i\sim\mathrm{N}(0, \sigma^2 \Idp)$, $\sigma^2=1$, and $i\in[2000], p=100$, FOCI selects the correct subset 70 percent of the time. In contrast, popular scikit-learn feature selection algorithms \citep{pedregosa2011scikit}, explained in Sec. \ref{sec:experiments}, almost never identify the correct subset (difFOCI, proposed in the next section, consistently selects the correct subset while preserving the same relative feature importance as FOCI during its correct runs). When applied to real-world datasets, FOCI matches the performance of established methods while requiring up to four times fewer features. 

\subsection{Extending $T$ to machine and deep learning}

From a statistical point of view, both $\xi_n$ and $T_n$ exhibit several strengths: well-established theoretical properties, are non-parametric, have no tunable parameters nor any distributional assumptions. Furthermore, a simple application of $T_n$ results in a strong feature-selection baseline. However, the non-smooth nature of the objectives in (\ref{eqn:chaterjee_estim}) and (\ref{eqn:mona_estim_with_x}) renders them non-differentiable, and therefore not applicable to most ML applications\footnote{Even if applicable, FOCI is often not well-suited for deep learning applications, as shown in Sec. \ref{sec:domain_shift}.}.


In the following section, we make these objectives differentiable using straightforward, well-known tricks in the ML community. This allows us to extend them to various ML and deep learning applications (as showcased in Sec. \ref{sec:experiments}). Moreover, it also allows to account for interactions between all features simultaneously (rather than in a step-wise fashion as in FOCI). Although FOCI could account for this in principle, as can be seen from Alg. \ref{algo:FOCI}, this would increase FOCI's complexity from $O(p^2)$ to potentially $O(2^p)$ thus preventing its practical use.

\section{Main results}
\label{sec:main_results}
We now propose an alternative formulation to the estimator $T_n$ in (\ref{eqn:mona_estim_with_x}), the objective of FOCI. As we will show later, this variation allows for the retention of FOCI's strengths as well as the improvement of its shortcomings. 

\subsection{difFOCI: towards a differentiable version of FOCI}
\label{sec:making_foci_dffble}

The initial step involves making the objective $T_n(Y, \bfZ | \bfX)$ differentiable w.r.t inputs $\bfZ$. Implementing this can be accomplished using straightforward techniques. We employ the following approach:

\begin{enumerate}
\item Compute the pairwise distance matrix $\mathbf{M}\in\mathbb{R}^{n,n}$ where $M_{i,j}=\|\bfX_i-\bfX_j\|$.
\item Calculate ${\bf{S}}_{\beta} \in \mathbb{R}^{n,n}$ such that ${\bf{S}}_{\beta}=\sigma_\beta(-({\bf{M}}+\lambda\Idn))$\footnote{Throughout the experiments, we use $\lambda=\max(1e^{10}, \max_{i,j}{\bf{M}}_{i,j}+\epsilon)$.}.
\item Instead of indexing $r_{N(i)}=r[N(i)]$, utilize $r^\top\mathbf{S}_{\beta_{i,\cdot}}$.
\end{enumerate}


Similarly, for ${\bf{U}}_{\beta}:=\sigma_\beta(-(\bf{\hat{M}}+\lambda\Idn))$, and $\hat{M}_{i,j}=\|(\bfX_i, \bfZ_i)-(\bfX_j, \bfZ_j)\|$. This allows us to present difFOCI, a differentiable version of the estimator in (\ref{eqn:mona_estim_with_x}):

\begin{align}
    T_{n,\beta}=T_{n,\beta}(Y, \bfZ|\bfX):=\frac{\sum_{i=1}^n (\min\{r_i, r^\top{\bf{U}}_{{\beta}_{i, \cdot}}\}\} - \min\{r_i, r^\top{\bf{S}}_{{\beta}_{i, \cdot}}\}\})}{\sum_{i=1}^n (r_i -  \min\{r_i, r^\top{\bf{S}}_{{\beta}_{i, \cdot}}\}\})}.
    \label{eqn:new_obj}
\end{align}
 Using the following theorem, we establish that our new estimator (\ref{eqn:new_obj}) enjoys the same limiting theoretical properties as the estimator in (\ref{eqn:mona_estim_with_x}):

\begin{theorem}%{thm}{ours}
     Let $\beta\in\mathbb{R}^+$. Suppose that $Y$ is not almost surely equal to a measurable function of $\bfX$. Then, $\lim_{n\rightarrow\infty} \lim_{\beta\rightarrow\infty} T_{n,\beta}= T$ almost surely. 
    \label{thm:ours}
\end{theorem}

The proof's core argument (given in Appendix \ref{appx:sec_B}) is based on demonstrating that the quantities $r^\top{\bf{U}}_{{\beta}_{i, \cdot}}$ and $r^\top{\bf{S}}_{{\beta}_{i, \cdot}}$ converge to $r_{M(i)}$ and $r_{N(i)}$ respectively as the inverse temperature parameter $\beta$ approaches infinity. Once this convergence is established, the remainder of the proof follows easily from Theorems \ref{thm:91_mona} and \ref{thm:92_mona} in \citet{azadkia2021simple}, outlined in Appendix \ref{appx:sec_A}.


Making the estimator differentiable allows us to use $T_{n,\beta}$ in various ways. Considering the predictors $\bfX$, response variable $Y$ and potentially available sensitive attributes $\bfX_S$ or group affiliations $\bfX_G$, parameterization $f_{\mathbf{\theta}}$, we highlight three ways to use $T_{n, \beta}$:


\begin{enumerate}[label=\textbf{(dF\arabic*)}]
    \item $T_{n, \beta}(Y, f_{\mathbf{\theta}}(\bfX))$: as a maximization objective, learning features that preserve ranks in the same fashion as the response \label{dF1}
    \item $\ell(Y, \hat{Y}) + \lambda T_{n, \beta}(\mathbf{X_G}, f_{\mathbf{\theta}}(\bfX))$: as a regularizer, penalizing the outputs (or learned features) $f_{\mathbf{\theta}}(\bfX)$ for being dependent on the protected groups $\bfX_G$, where $\ell$ denotes the standard loss used in machine learning \label{dF2}
    \item $T_{n, \beta}(Y, f_{\mathbf{\theta}}(\bfX) | \bfX_S)$: as a conditioning objective, allowing to learn features that contain information about the response only after conditioning out the sensitive information $\bfX_S$ \label{dF3}
\end{enumerate}

For instance, \ref{dF1} can be utilized for feature selection or dimensionality reduction techniques. \ref{dF2} can be employed to prevent the network from relying on spurious correlations when group attributes are available. \ref{dF3} can be applied in fairness scenarios where we aim to avoid predictions based on certain personal information.

The remaining task is to select the parameterization $f_{\mathbf{\theta}}(\cdot)$. In the following sections, we will focus on two options: \emph{(i)} \textit{vec} - a dot product parameterization $f_{\mathbf{\theta}}(\bfX)=\theta \odot \bfX$, or \emph{(ii)} \textit{NN} - a neural network parameterization, $f_{\mathbf{\theta}}(\cdot)$\footnote{For example, with \textit{vec}-\ref{dF1} we denote using \ref{dF1} with vector parameterization.}$^{,}$\footnote{We also tried \textit{vec-NN} parameterization $f_{\mathbf{\theta}}(\bfX)=\theta_2 \odot f_{\mathbf{\theta}_1}(\bfX)$, with $\mathbf{\theta}=\{\mathbf\theta_1, \mathbf\theta_2\}$ but it did not show any improvements over the \textit{NN} parameterization.}. Algorithm \ref{algo:general} provides a general outline for using the $T_{n,\beta}$ with a chosen parameterization, and specific instances of the algorithm are given in Appendix \ref{appx:sec_G}.


\begin{algorithm}[ht]
\caption{Differentiable FOCI (difFOCI)}
\begin{algorithmic}
\STATE \textbf{Input:} predictor $\mathbf{Z}\in\mathbb{R}^{n,p}$, response $Y\in\mathbb{R}^n$, and optional $\bfX\in\{\emptyset, S, G\}$, for sensitive $S\in\mathbb{R}^{n, d}$ or group info. $G\in\mathbb{R}^{n,d}$, $d\geq1$
\STATE \textbf{Input:} parameterization $f_{\mathbf{\theta}}\in\{\textit{vec}, \textit{NN}\}$, objective choice $T_{n,\beta}\in \{\ref{dF1}, \ref{dF2}, \ref{dF3}\}$
\STATE Initialize ${\mathbf{\theta}}$ 
 \FOR{$t=1,...,n_{\text{iter}}$}
    \STATE $\mathcal{L} \gets T_{n, \beta}(Y, f_{{\mathbf{\theta}}_t}(\bfZ) | \bfX)$ \hfill // Applying difFOCI
    \STATE Update ${\mathbf{\theta}}_{t+1} \gets \text{Optim}(\mathcal{L}, {\mathbf{\theta}}_t)$    \hfill // Parameter update
\ENDFOR
\STATE \textbf{Output:} parameterization $f_{\mathbf{\theta}}$
\end{algorithmic}
\label{algo:general}
\end{algorithm} 



\begin{figure}[t]
    \centering
    \begin{subfigure}[b]{\textwidth}
        \centering
        \includegraphics[width=\textwidth]{images/2a_functions.pdf} % Adjust the path and filename
        \caption{Generating functions of functional process}
        \label{plt:2a_functions}
    \end{subfigure}
    \hfill % Optional: add some horizontal separation
    \begin{subfigure}[b]{\textwidth}
        \centering
        \includegraphics[width=\textwidth]{images/2b_selected.pdf} % Adjust the path and filename
        \caption{First plot: norms of $\theta$. Remaining plots: features with 5 largest param. norms (only first 3 selected).}
        \label{plt:2b_selected}
    \end{subfigure}
    \caption{Synthetic dataset experiment, detailed in Sec. \ref{sec:preliminary_synthetic_study}. Out of 240 total features, our \textit{vec}-\ref{dF1} selects three informative, yet diverse features (corresponding to norms $0.27$, $0.23$, and $0.18$).}
    \label{fig:wholefigure}
\end{figure}



\begin{table}[b]
  \caption{Feature selection benchmark results in terms of test MSE. Our algorithms consistently yield the most accurate predictions while selecting one of the smallest feature subsets (as seen in (\ref{sim_data})). With $\hat\mu_y$, we denote predicting the overall mean and with \textit{Full}, regressing to the whole dataset.}
  \centering
  \begin{subtable}{\textwidth}
    \centering
    \resizebox{\textwidth}{!}{
    \begin{tabular}{l|ccccccccccc}
      \toprule
          & GUS   & S.Per. & FPR & FDR & FWE & K.B. 2 & K.B. 50 & K.B. 75 & FOCI     & \textit{vec}-\ref{dF1}  & \textit{NN}-\ref{dF1}     \\ 
      \midrule
  \# Feat. Select. & 1     & 24    & 112   & 95    & 53    & 2*    & 50*   & 100*  & 6     & 2   & N/A   \\
  Test MSE              & 0.086 & 0.028 & 0.027 & 0.028 & 0.030 & 0.084 & 0.030 & 0.028 & 0.030 & 0.016 $\pm$ 0.02 & \textbf{0.012} $\pm$ \textbf{0.01}
       \\ \bottomrule
      \end{tabular}}
      \caption{Results from simulated data study, detailed in Sec. \ref{sec:preliminary_synthetic_study}. Both \ref{dF1} versions successfully inherit FOCI's strengths: they select a small number of features while exhibiting solid performance.}
      \label{sim_data}
  \end{subtable}
  
  \vspace{0.25cm}
  
  \begin{subtable}{\textwidth}
    \centering
    \resizebox{\textwidth}{!}{
    \begin{tabular}{l|ccccccccccccccc}
      \toprule
          & $\hat\mu_y$ & Full  & GUS   & S.Per. & FPR & FDR & FWE & K.B. & UMAP  & PCA & FOCI    & \textit{vec}-\ref{dF1}    & \textit{NN}-\ref{dF1}\\ 
      \midrule
  Exp 1. & 1.38 & 0.22 & 0.93 & 0.94 & 0.53 & 0.54 & 0.68 & 0.54  & 1.14  & 1.02 & 0.21  & \textbf{0.02 $\pm$ 0.00} & 0.08 $\pm$ 0.01 \\
  Exp 2. & 0.49 & 0.58 & 0.53 & 0.58 & 0.58 & 0.59 & 0.58 & 0.58 & 0.55 & 0.52 & 0.53 & 0.24 $\pm$ 0.00 & \textbf{0.02 $\pm$ 0.01} \\
  Exp 3. & 0.35 & 0.31 & 0.32 & 0.32 & 0.34 & 0.34 & 0.33 & 0.33 & 0.33 & 0.34 & 0.30  & 0.23 $\pm$ 0.00 & \textbf{0.18 $\pm$ 0.01}  
       \\ \bottomrule
      \end{tabular}}
      \caption{Results from three toy experiments, described in Sec. \ref{sec:toy_exps}, show that both versions of \ref{dF1} enhance FOCI's strengths. In Experiments 2 and 3, they are the only methods that outperform regressing to the mean \(\hat{\mu}_y\).}
      \label{tab:sinusoidal_experiments}
  \end{subtable}
\end{table}


We proceed by testing whether difFOCI performs well at FOCI's main application - feature selection. We begin with a simulated dataset, followed by three experiments with increasing complexity.


\subsection{Preliminary synthetic study}
\label{sec:preliminary_synthetic_study}

To evaluate the feature selection performance of difFOCI, we utilize \textit{vec}-\ref{dF1} to obtain the objective $T_{n, \beta}(Y, \theta \odot \bfX)$. Unlike FOCI, which returns a binary vector indicating whether a feature is selected or rejected, difFOCI's version \textit{vec}-\ref{dF1} yields a real-valued vector with components $(\theta_i)_{i\in[p]}$ representing the predictive informativeness of each corresponding feature (which allows taking into account feature variability). To perform feature selection, we need to choose a cutoff parameter $\upsilon$ and select the features with $|\theta_i|\geq\upsilon$.



Alg. \ref{algo:general} therefore requires the following hyperparameters: softmax temperature $\beta$, cutoff value $\upsilon$, and optimization parameters (e.g., learning rate $\gamma$, weight decay $\lambda$, minibatch size $b$, etc.). Our experimental analyses show that $\beta=5$ and $\upsilon=0.1$ yield consistently good performance, so we set these as fixed\footnote{A further discussion on this can be found in Appendix \ref{sec:parameter_beta_choice}}. As a result, our algorithm simplifies solely to the hyperparameters used in conventional optimization methods, which are in Appendix \ref{appx:sec_G} for all experiments.


\paragraph{Environment.} As an initial example, we consider a data-generation process ideal for FOCI: from a large pool of features, a handful is sufficient for strong performance with $n\sim p$. The functional process is illustrated in Fig. \ref{plt:2a_functions}, crafted to generate a diverse set of features: \emph{informative ones}, such as straight lines, sinusoids, or parabolas, and functions \emph{individually uninformative, yet informative in multidimensional contexts}, e.g., ellipses, rotated parabolas, and more involved curves. This process includes 60 functions, each noised four times, resulting in $p=240$ features with $n=100$ points. Ideally, a feature selection method should pinpoint a small but diverse set of features\footnote{The exact data-generating process is given in Appendix \ref{sec:synthetic_env}}.

\paragraph{Baselines.} For comparative analysis, we employ various feature selection techniques from the scikit-learn library \citep{pedregosa2011scikit}. These include: \textit{GenericUnivariateSelect} (GUS) for univariate feature selection, \textit{SelectPercentile} (S.Per.), retaining only the top user-specified percentile of features, and statistical test-based methods: \textit{SelectFpr} (FPR), \textit{SelectFdr} (FDR), and \textit{SelectFwe} (FWE) addressing false positive rate, false discovery rate, and family-wise error, respectively. Additionally, we employ \textit{SelectKBest} (K.B) to select the best 25\%, 50\%, or 75\% of features based on the ANOVA F-value test \citep{girden1992anova}. We also benchmark against dimensionality reduction techniques including Linear Discriminant Analysis (LDA, \citet{fisher1936lda}), Principal Component Analysis (PCA, \citet{wold1987principal}), and Uniform Manifold Approximation and Projection (UMAP, \citet{mcinnes2018umap}), retaining 25\%, 50\%, and 75\% of the features/principal components.

Throughout this and Sec. \ref{sec:toy_exps}, we measure the performance by looking at the test error using Support Vector Regression (SVR, $C=1.0$, $\epsilon=0.2$) \citep{svr2, svr3, svr1}. For SelectKBest, PCA, and UMAP, instead of reporting for $25, 50$, and $75\%$ of features/components separately, we only provide the results yielding the lowest mean-squared test error.

\looseness=-1\paragraph{Results.} Our approach selects a small, diverse, and informative set of features, resulting in good performance and showcasing successful inheritance of FOCI's main strengths (see Table \ref{sim_data}). The norms of the selection parameter $\theta$ are shown in Fig. \ref{plt:2b_selected}, demonstrating the evident relationship between the predictive informativeness of the features and the corresponding parameter norms. 


We have discussed the recent advances and methodologies necessary to introduce difFOCI, as well as provided experimental analysis on a synthetic examples. We now proceed to more challenging examples, and ultimately to real-world datasets.


\section{From feature selection to feature learning}
\label{sec:toy_exps}


\begin{table}
    \caption{Feat. selection and dim. reduction benchmarks in terms of logistic test loss. Reported are the mean and std. across five random seeds. Our algorithms yield competitive predictions.  \vspace{-10pt}}
  \centering
  \resizebox{\textwidth}{!}{
  \begin{tabular}{l|cccccccccccc}
    \toprule
        & GUS   & S.Per. & FPR & FDR & FWE & K.B. & UMAP & LDA & PCA & FOCI     & \textit{vec}-\ref{dF1}    & \textit{NN}-\ref{dF1}   \\ 
    \midrule
Spambase             & 10.70 & 6.05  & 2.92 & 2.92 & 2.92  & 3.39 & 2.97 & 3.20 & 3.12 & 3.04  &   \textbf{2.56 $\pm$ 0.13}  & \textbf{2.57 $\pm$ 0.19}  \\
Toxicity             & 14.41  & 12.98 & 17.30 & 12.98  & 18.02    & \textbf{10.09} &  12.98  & 15.86 & \textbf{10.00} & 16.30 & 11.61 $\pm$ 0.80  & \textbf{9.61 $\pm$ 1.50} \\
QSAR                 & 2.88  & 3.16 & 3.16 & 3.76 & 2.92 & 3.52 & 2.32 &  \textbf{2.16} & \textbf{2.16} & 3.44  & 2.54 $\pm$ 0.07  & \textbf{2.11} $\pm$ \textbf{0.11}  \\
Breast Cancer        & 4.66  & 1.69 & 0.42  & 0.42  & 0.42  & \textbf{0.00}  & 2.48 & 1.42 & 1.24 & 0.62 & \textbf{0.00 $\pm$ 0.00} &  \textbf{0.00  $\pm$ 0.00}    \\
Religious             & 0.84  & 0.56 & 0.65   & 0.57 & 0.56  & 0.48 & 6.63 & 1.61 & 0.60 & 0.53  & \textbf{0.48 $\pm$ 0.03}  & 0.56 $\pm$ 0.04 \\ \bottomrule
    \end{tabular}}
    \vspace{-4pt}
  \label{tab:real_data}
\end{table}


With a high-level understanding of difFOCI in place, we continue to assess its performance. We begin by highlighting two key observations we encountered during our preliminary experiments. We consider the following toy example:  $Y = \sin(X_1)+2\sin(X_2)+3\sin(X_3)+\epsilon$, where $\epsilon_i\sim \mathrm{N}(0,\sigma_\epsilon^2)$, $i\in[n]$, and $\bfX\sim N(0,\sigma^2_x\Idp)$, with $n=2000$, $p=10$, $\sigma_x=\sigma_\epsilon=0.1$.


\looseness=-1\textbf{Observation 1.}\hypertarget{obs_1}{($\dagger$)} The objectives (\ref{eqn:chaterjee_estim}) and (\ref{eqn:mona_estim_with_x}) consistently capture the correct feature functional forms. Specifically, the values \emph{(i)} $T_n\left(Y,\left[\sum_{i=1}^2\sin(\pi^j_i(\bfX_{1:3}), \sin(\pi^j_3(\bfX_{1:3}))\right]\right)$, $j\in[3]$, \emph{(ii)} $T_n(Y ,\sin(\bfX_{1:3}))$,  and \emph{(iii)} $T_n(Y ,\sum_{i=1}^3\sin(\bfX_i))$ are all significantly greater than \emph{(i)} $T_n\left(Y , \left[\sum_{i=1}^2\pi^j_i(\bfX_{1:3}),\pi^j_3(\bfX_{1:3})\right]\right)$, \emph{(ii)} $T_n(Y ,\bfX_{1:3})$, and \emph{(iii)} $T_n(Y ,\sum_{i=1}^3\bfX_i)$ (as illustrated in Figure \ref{plt:0_error_bar} in the Appendix). 
Therefore, a more complex parameterization (than $f_{\mathbf{\theta}}(\bfX)=\theta \odot \bfX$) might learn a nonlinear transformation of the features, maintaining ranks in a manner more consistent with the true functional forms.


\textbf{Observation 2.}\hypertarget{obs_2}{($\ddagger$)} The objectives (\ref{eqn:chaterjee_estim}) and (\ref{eqn:mona_estim_with_x}) consistently prefer correct, lower-dimensional bases of the features. Specifically, $T_n(Y , \sum_{i=1}^3\sin(\bfX_i))$ remains consistently greater than $T_n(Y , \sin(\bfX_{1:3}))$. Therefore, a more elaborate parameterization could learn an appropriate, possibly lower dimensional, basis transformation. 


\looseness=-1Motivated by these observations, we propose \textit{NN} parameterizations to further explore the capabilities of difFOCI. We begin with simple one-hidden-layer Multi-layer Perceptrons (MLPs) as $f_{\mathbf{\theta}}$ parameterizations. We set the output dimension to match the input, as this performed well across all experiments, though treating it as a hyperparameter might further enhance performance.

\subsection{Initial assessments of \ref{dF1}}

We now evaluate both \textit{vec}-\ref{dF1} and \textit{NN}-\ref{dF1} across three progressively challenging examples. We note that across all examples, FOCI selects the correct subset of the features more than 95 percent of the time. We set $p=10$ throughout the experiments, and both $\sigma_\epsilon=\sigma_x=0.1$. Full experimental details are given in Appendix \ref{appx:sec_F}.


\looseness=-1 \textbf{Toy example 1: difFOCI successfully accounts for feature variability.} Here, we test whether \textit{vec}-\ref{dF1} and \textit{NN}-\ref{dF1} on the following example, previously introduced in Sec. \ref{sec:main_results}: $Y = \sin(X_1)+2\sin(X_2)+3\sin(X_3)+\epsilon$, where $\epsilon_i\sim \mathrm{N}(0,\sigma_\epsilon^2)$, $i\in[n]$, and $\bfX\sim N(0,\sigma^2_x\Idp)$, with $n=2000$. In Table \ref{tab:sinusoidal_experiments}, we observe that \textit{vec}-\ref{dF1} and \textit{NN}-\ref{dF1} successfully pinpoint the correct feature subset and account for feature variability, resulting in improved performance to that of FOCI. We expand on this in Appendix, Fig. \ref{plt:1_param_evol} for \textit{vec}-\ref{dF1}, where we can observe the correct proportionality of the coefficients in the regression equation and the learned parameters $\theta_1, \theta_2$ and $\theta_3$.\footnote{Note that this is already an improvement to FOCI, as it cannot take into account feature variability.}. 


\textbf{Toy example 2: difFOCI can learn appropriate basis transformations.} The goal of this toy example is to examine whether \textit{NN}-\ref{dF1} effectively learns basis transformations. Data are generated as follows: $Y = \sin(X_1+2X_2+3X_3)+\epsilon$, where $\epsilon_i\sim \mathrm{N}(0,\sigma_\epsilon^2)$, $i\in[n]$, and $\bfX\sim N(0,\sigma^2_x\Idp)$, with $n=2000$. We affirmatively demonstrate its efficacy by examining the test loss after fitting the SVR - the substantially lower test error can be observed in Table \ref{tab:sinusoidal_experiments}.


\textbf{Toy example 3: difFOCI simultaneously addresses mutual interactions, basis, and nonlinear transformations.} Our final example seeks to explore the full capabilities of \ref{dF1} with NN parameterization, examining whether it can simultaneously discern complex, interrelated relationships as well as multiple transformations, encompassing both nonlinear and basis transformations. The data generation process is as follows: $Y = \sin((X_1X_2)^2+(X_2X_3)^2+(X_1X_3)^2)+\epsilon$, where $\epsilon_i\sim \mathrm{N}(0,\sigma_\epsilon^2)$, $i\in[n]$, and $\bfX\sim N(0,\sigma^2_x\Idp)$, with $n=5000$. As evidenced in Table  \ref{tab:sinusoidal_experiments} (using a two-hidden-layer MLP\footnote{For this example, we found one-hidden-layer MLP not to be expressive enough.
}), we successfully learn effective transformations that result in strong performance. 

\paragraph{Summary.} Throughout the experiments, both \textit{vec}-\ref{dF1} and \textit{NN}-\ref{dF1} yield strong performance, as seen in Table \ref{tab:sinusoidal_experiments}. The two penultimate examples emphasize the potential capabilities of difFOCI; not only can it correctly identify the relevant subsets, but it also learns useful transformation, yielding the only method that outperforms random guessing (see $\hat{\mu}_y$ column in Table \ref{tab:sinusoidal_experiments}).


\section{Experiments}
\label{sec:experiments}


Having examined \ref{dF1} on synthetic problems and toy datasets, we now proceed to real-world datasets. We attempt to demonstrate the flexibility of difFOCI and highlight the promising potential of all three objectives: \ref{dF1}-\ref{dF3}. Our aim in this section is not solely to outperform existing benchmarks, but rather to showcase difFOCI's broad applicability, inspire further investigation into these applications, and explorations of new areas where the method can be applied.


\subsection{Real-world data}
\label{sec:real_world_data}
In this section, we compare \textit{vec}-\ref{dF1} and \textit{NN}-\ref{dF1} to feature selection and dimensionality reduction methods using real-world datasets.

\looseness=-1\textbf{Environments.} We evaluate our methods on five UCI datasets \citep{uci2019}: Breast Cancer Wisconsin \citep{street1993breastcancer}, involving benign/malignant cancer prediction; Toxicity \citep{gul2021toxicity}, aimed at determining the toxicity of molecules affecting circadian rhythms; Spambase \citep{hopkins1999spambase}, classifying emails as spam or not; QSAR \citep{qsar}, a set containing molecular fingerprints used for chemical toxicity classification, and Religious \citep{sah2019biblical}, aimed at identifying the source of religious books texts.  We perform Logistic Regression \citep{cox1958regression} with default scikit-learn \citep{pedregosa2011scikit} parameters ($\text{tol}=$\num{1e-4}, $C=1.0$). Dataset information is provided in Appendix \ref{appx:sec_C}. 


\paragraph{difFOCI is competitive in feature selection and dimensionality reduction.} As seen in Table \ref{tab:real_data} difFOCI achieves solid performance in the experiments. For \textit{NN}-\ref{dF1}, we use two-hidden-layer MLPs. The findings, which employ logistic loss, demonstrate that taking into account feature variability and using parameterization are crucial for improved performance compared to FOCI.

\subsection{Domain shift/spurious correlations}
\label{sec:domain_shift}
Here, we investigate an application of difFOCI to deep learning in the form of \textit{NN}-\ref{dF2}. The data consists of triplets $(Y, \bfX, \bfX_G)$, denoting the predictor, response variables, and group attributes, respectively. In this context, difFOCI can be employed as a regularizer to enforce the learning of uncorrelated features with respect to spurious attributes, thereby mitigating relying on spurious correlations and shortcuts in the model \citep{kenney1982beware}.

\begin{wraptable}{r}{0.5\textwidth}
\vspace{-.5cm}
\caption{Average and worst group accuracies for the Waterbirds dataset. We compare to the ERM and DRO, where e.s. stands for early-stopping and $l2$ for Ridge regularization. We can see that difFOCI performs comparably to state-of-the-art spurious correlation methods.} 
\centering
  \resizebox{0.5\textwidth}{!}{
  \begin{tabular}{l|cc|cc|}
    & \multicolumn{2}{c|}{Average acc.} & \multicolumn{2}{c|}{Worst group acc.} \\ 
    \toprule
    & Train & Test & Train & Test \\
    \midrule 
    ERM   &  100 & \textbf{97.3} &  100 &  60.0        \\
    ERM (e.s. $+$ strong $l2$)   &  97.6 & 95.7 &  35.7 &  21.3                  \\
    ERM + FOCI  &  99.9  & 77.8 & 1.1 & 0.0  \\
    ERM + \textit{NN}-\ref{dF2}    &  99.9 & 93.7 &  92.0 &  \textbf{85.7} \\ \midrule
    DRO  &  100.0 & \textbf{97.4} &  100.0 &  76.9                \\
    DRO (e.s. $+$ strong $l2$)  &  99.1 & 96.6 &  74.2 &  86.0                \\
    DRO + FOCI & 99.5  & 74.5 & 6.1 &  3.9                \\
    DRO + \textit{NN}-\ref{dF2} &  80.1 & 93.5 &  99.2 &  \textbf{87.2}                \\ \bottomrule
\end{tabular}}
  \label{tab:waterb_table}
  \vspace{-.5cm}
\end{wraptable}

\looseness=-1\paragraph{Environment.} We use Waterbirds dataset \citep{sagawa2019distributionally}, which combines bird photographs from the Caltech-UCSD Birds-200-2011 dataset \citep{wah2011caltech} with image backgrounds from the Places dataset \citep{zhou2017places}. The labels $Y = \{\text{waterbirds}, \text{landbirds}\}$ are placed against $G = \{\text{water}, \text{land}\}$ backgrounds, with waterbirds (landbirds) more frequently appearing against a water (land) background (exact details given in Table \ref{tab:waterbirds_counts}, Appx. \ref{appx:sec_E}). Due to this spurious correlation, \citep{sagawa2019distributionally, idrissi2022simple, bell2024reassessing} observed that NNs (i.e., ResNet-50 \citep{he2016deep}, pre-trained on ImageNet \citep{imagenet}) tend to rely on the background to infer the label, rather than solely focusing on birds. 

\paragraph{Preventing reliance on spurious correlations.} We investigate the potential benefits of employing \textit{NN}-\ref{dF2} as a regularization technique, which penalizes the reliance of extracted features $f_{FE_{\theta}}$ on the spurious attribute $\bfX_G$ (i.e., the background) via $T_{n, \beta}(\bfX_G, f_{FE_{\theta}}(\bfX)\mid \bfX_G)$. From Tables \ref{tab:waterb_table}-\ref{tab:worst_group_acc}, we can see that \textit{NN}-\ref{dF2} (applied to both ERM and DRO) compares competitively to state-of-the-art methods. The exact algorithm is given in \ref{algo:appx_ex_2}. Experimental details, reported average accuracy and further examples are in Appendix \ref{appx:sec_G}. 


\begin{figure}[t]
  \centering
  \includegraphics[width=\linewidth]{images/3_saliency_plot.pdf}
    \caption{ResNet-50 \citep{he2016deep} saliency maps using the ERM \citep{vapnik2006estimation} loss, DRO \citep{sagawa2019distributionally} with standard regularization (early stopping and $\ell2$) or difFOCI. Without difFOCI, the models heavily rely on background (spurious features). difFOCI effectively resolves the problem (main focus is on relevant features: the bird). Further samples are shown in the Appendix \ref{appx:sec_E}.}
    \label{fig:birds}
\end{figure}


\looseness=-1\paragraph{difFOCI increases worst group accuracy while maintaining solid performance.} We can observe in Table \ref{tab:waterb_table} and Fig. \ref{fig:birds} that \textit{NN}-\ref{dF2} successfully prevents the network from relying on the spuriously correlated background while improving worst group accuracy for both ERM and DRO. Apart from Waterbirds dataset, we also tested difFOCI on 5 additional datasets: two text datasets: MultiNLI \citep{williams2017broad}, CivilComments \citep{borkan2019nuanced}, and four image datasets: NICO++ \citep{zhang2023nico}, CelebA \citep{liang2022metashift}, MetaShift \citep{liang2022metashift} and CheXpert \citep{irvin2019chexpert}. Full experimental details (including average accuracy performance) can be found in Appendix \ref{appx:sec_G}. We experimented with various architectures: in addition to the ResNet-50, we used BERT and ViT-B with pretraining strategies like DINO and CLIP. Furthermore, we compared to Just Train Twice \citep{liu2021just}, Mixup \citep{zhang2017mixup}, and Invariant Risk Minimization \citep{arjovsky2019invariant} as baselines. As shown in Table \ref{tab:fairness_experiments}, difFOCI demonstrates competitive performance in terms of both average and worst-group accuracy.




\begin{table}[t]
    \caption{Worst group accuracy across several datasets. difFOCI obtains competitive performance.}
    \centering
    \setlength{\tabcolsep}{4pt} 
    \resizebox{\textwidth}{!}{
        \begin{tabular}{lccccccc}
            \toprule
            Dataset & difFOCI+ERM & difFOCI+DRO & ERM & DRO & JTT & Mixup & IRM \\
            \midrule
            MultiNLI & $\mathbf{77.6 \pm 0.1}$ & $\mathbf{77.5 \pm 0.2}$ & $66.9 \pm 0.5$ & $77.0 \pm 0.1$ & $69.6 \pm 0.1$ & $69.5 \pm 0.4$ & $66.5 \pm 1.0$ \\
            CivilComments & $66.32 \pm 0.2$ & $\mathbf{70.3 \pm 0.2}$ & $64.1 \pm 1.1$ & $\mathbf{70.2 \pm 0.8}$ & $64.0 \pm 1.1$ & $65.1 \pm 0.9$ & $63.2 \pm 0.5$ \\
            CelebA & $\mathbf{89.32 \pm 0.4}$ & $\mathbf{89.8 \pm 0.9}$ & $65.0 \pm 2.5$ & $\mathbf{88.8 \pm 0.6}$ & $70.3 \pm 0.5$ & $57.6 \pm 0.5$ & $63.1 \pm 1.7$ \\
            NICO++ & $\mathbf{47.10 \pm 0.7}$ & $46.3 \pm 0.2$ & $39.3 \pm 2.0$ & $38.3 \pm 1.2$ & $40.0 \pm 0.0$ & $43.1 \pm 0.7$ & $40.0 \pm 0.0$ \\
            MetaShift & $83.10 \pm 0.5$ & $\mathbf{91.7 \pm 0.2}$ & $80.9 \pm 0.3$ & $86.2 \pm 0.6$ & $82.6 \pm 0.6$ & $80.9 \pm 0.8$ & $84.0 \pm 0.4$ \\
            CheXpert & $54.42 \pm 3.2$ & $\mathbf{75.3 \pm 0.3}$ & $50.1 \pm 3.5$ & $73.9 \pm 0.4$ & $61.5 \pm 4.3$ & $40.2 \pm 4.1$ & $35.1 \pm 1.2$ \\
            \bottomrule
        \end{tabular}%
    }
    \label{tab:worst_group_acc}
\end{table}



\subsection{Fairness study}
\label{sec:fairness}


\begin{table}[t]
    \caption{\textit{NN}-\ref{dF3} allows preserving predictivity of $y$ while significantly reducing predictivity of $X_s$.}
    \centering
    \setlength{\tabcolsep}{4pt}  % reduce column separation
    \resizebox{\textwidth}{!}{%
        \begin{tabular}{lccccccc}
            \toprule
            Dataset & Features & Train acc: $y$ & Val. Acc: $y$ & Test acc: $y$ & Train acc: $X_s$ & Val. Acc: $X_s$ & Test acc: $X_s$ \\
            \midrule
            \multirow{2}{*}{Bank marketing} & Stand. data & $91.32 \pm 2.3$ & $93.27 \pm 1.2$ & $90.05 \pm 2.0$ & $89.09 \pm 1.2$ & $72.26 \pm 1.5$ & $70.93 \pm 0.9$ \\
            & \ref{dF3} features & $90.81 \pm 1.8$ & $92.13 \pm 2.6$ & $89.35 \pm 1.1$ & $63.12 \pm 2.8$ & $62.24 \pm 0.7$ & $\mathbf{63.81 \pm 2.1}$ \\
                        \midrule

            \multirow{2}{*}{Student data} & Stand. data & $88.35 \pm 1.7$ & $79.63 \pm 0.9$ & $75.67 \pm 1.3$ & $95.68 \pm 2.1$ & $72.16 \pm 2.4$ & $71.21 \pm 1.5$ \\
            & \ref{dF3} features & $80.18 \pm 2.9$ & $72.16 \pm 1.6$ & $72.73 \pm 1.7$ & $59.47 \pm 1.1$ & $58.95 \pm 1.0$ & $\mathbf{48.89 \pm 1.1}$ \\
                        \midrule
            \multirow{2}{*}{ASCI Income} & Stand. data & $83.49 \pm 2.4$ & $85.10 \pm 2.1$ & $81.30 \pm 2.7$ & $68.97 \pm 1.6$ & $67.67 \pm 2.6$ & $66.00 \pm 0.7$ \\
            & \ref{dF3} features & $82.80 \pm 0.8$ & $81.99 \pm 1.5$ & $82.95 \pm 0.9$ & $56.58 \pm 1.2$ & $55.01 \pm 2.0$ & $\mathbf{52.73 \pm 2.0}$ \\
            \bottomrule
        \end{tabular}%
    }
    \label{tab:fairness_experiments}
\end{table}



Finally, we explore \textit{NN}-\ref{dF3}. This section, while not the primary focus of our contribution, offers a complementary illustration of the difFOCI objective's versatility through a heuristic example. We found that this form \ref{dF3} preserves the performance of the chosen parameterization while significantly reducing its predictivity of the sensitive attribute.


\looseness=-1\paragraph{Environments.} We utilize classification datasets with interpretable features and sensitive attributes: \emph{(i)} Student dataset \citep{cortez2008student}, aimed at predicting if a student's performance surpasses a specific threshold (sex as the sensitive); \emph{(ii)} Bank Marketing dataset \citep{moro2014bankmarketing} with predicting if a customer subscribes to a bank product (marital status as the sensitive); and two ACS datasets \citep{ding2021retiringASCIfolktables}, \emph{(iii)} Employment and \emph{(iv)} Income, for predicting individual's employment status and whether their income exceeds a threshold, with sex and race as sensitive attributes in both datasets. Exact experimental details are provided in Appendix \ref{sec:fairne_apx}.

\paragraph{Findings.} Leveraging the conditional dependence expression in (\ref{eqn:mona_estim_with_x}), our method flexibly incorporates sensitive features to facilitate fairer classification without exploiting sensitive data. Using NN-\ref{dF3}, we optimize $T_{n,\beta}(Y,  f_\theta(\bfX) \mid \bfX_s)$ to learn features that are informative about $Y$, offering an optimization that heuristically seems to favor solutions less predictive of $\bfX_s$. Specifically, we train two NNs to predict $y$: the first NN was trained on $X$ (without $X_s$), while the second NN was trained on features $f_\theta(X)$ obtained using \ref{dF3}. We then used the final layers of both NNs to predict the sensitive $X_s$. As can be observed from Table \ref{tab:fairness_experiments}, difFOCI \ref{dF3} significantly reduces the predictability of $X_s$ (sometimes to chance level) without significantly impacting accuracy on $y$ - in some cases even slightly improves it. 



\paragraph{Despite conditioning out sensitive information, difFOCI delivers solid performance.} From Table  \ref{tab:fairness_experiments}, we see that \textit{vec}-\ref{dF3} demonstrates strong performance by effectively debiasing the network (as it cannot predict the sensitive $X_s$ well), while keeping informativeness regarding $y$. In Appendix \ref{sec:fairne_apx}, we conduct another experiment with similar findings showcasing the promising potential of \ref{dF3}.


\section{Conclusion}
\label{sec:conclusion}


In this paper, we discussed two recent advancements in rank-based measures of correlation, critically examining the proposed estimators, including the FOCI algorithm and its barriers to adoption in machine learning. Leveraging these advancements, we introduced three enhanced and more adaptable versions of FOCI. We conducted several studies to showcase the retention of FOCI's strengths and the improvement of its weaknesses. We evaluated difFOCI's capabilities from toy examples, where our method was the sole one exceeding random guessing, to comprehensive real-world datasets involving feature selection and spurious correlations, where it demonstrated state-of-the-art performance. Finally, we proposed a direct application of our algorithm in fairness research, showcasing that difFOCI successfully debiases neural networks on several datasets.

\section*{Acknowledgements and funding}
This work has received funding from the French government, managed by the National Research Agency (ANR), under the France 2030 program with the reference ANR-23-IACL-0008. We extend our thanks to Samuel Bell, João Maria Janeiro, Badr Youbi Idrissi, Theo Moutakanni, Stéphane d'Ascoli and Timothée Darcet for feedback and support. Finally, we also thank Carolyn Krol for extensive consultation and support throughout this project.


\bibliography{iclr2025_conference}
\bibliographystyle{iclr2025_conference}

\newpage
\appendix
\subsection{Lloyd-Max Algorithm}
\label{subsec:Lloyd-Max}
For a given quantization bitwidth $B$ and an operand $\bm{X}$, the Lloyd-Max algorithm finds $2^B$ quantization levels $\{\hat{x}_i\}_{i=1}^{2^B}$ such that quantizing $\bm{X}$ by rounding each scalar in $\bm{X}$ to the nearest quantization level minimizes the quantization MSE. 

The algorithm starts with an initial guess of quantization levels and then iteratively computes quantization thresholds $\{\tau_i\}_{i=1}^{2^B-1}$ and updates quantization levels $\{\hat{x}_i\}_{i=1}^{2^B}$. Specifically, at iteration $n$, thresholds are set to the midpoints of the previous iteration's levels:
\begin{align*}
    \tau_i^{(n)}=\frac{\hat{x}_i^{(n-1)}+\hat{x}_{i+1}^{(n-1)}}2 \text{ for } i=1\ldots 2^B-1
\end{align*}
Subsequently, the quantization levels are re-computed as conditional means of the data regions defined by the new thresholds:
\begin{align*}
    \hat{x}_i^{(n)}=\mathbb{E}\left[ \bm{X} \big| \bm{X}\in [\tau_{i-1}^{(n)},\tau_i^{(n)}] \right] \text{ for } i=1\ldots 2^B
\end{align*}
where to satisfy boundary conditions we have $\tau_0=-\infty$ and $\tau_{2^B}=\infty$. The algorithm iterates the above steps until convergence.

Figure \ref{fig:lm_quant} compares the quantization levels of a $7$-bit floating point (E3M3) quantizer (left) to a $7$-bit Lloyd-Max quantizer (right) when quantizing a layer of weights from the GPT3-126M model at a per-tensor granularity. As shown, the Lloyd-Max quantizer achieves substantially lower quantization MSE. Further, Table \ref{tab:FP7_vs_LM7} shows the superior perplexity achieved by Lloyd-Max quantizers for bitwidths of $7$, $6$ and $5$. The difference between the quantizers is clear at 5 bits, where per-tensor FP quantization incurs a drastic and unacceptable increase in perplexity, while Lloyd-Max quantization incurs a much smaller increase. Nevertheless, we note that even the optimal Lloyd-Max quantizer incurs a notable ($\sim 1.5$) increase in perplexity due to the coarse granularity of quantization. 

\begin{figure}[h]
  \centering
  \includegraphics[width=0.7\linewidth]{sections/figures/LM7_FP7.pdf}
  \caption{\small Quantization levels and the corresponding quantization MSE of Floating Point (left) vs Lloyd-Max (right) Quantizers for a layer of weights in the GPT3-126M model.}
  \label{fig:lm_quant}
\end{figure}

\begin{table}[h]\scriptsize
\begin{center}
\caption{\label{tab:FP7_vs_LM7} \small Comparing perplexity (lower is better) achieved by floating point quantizers and Lloyd-Max quantizers on a GPT3-126M model for the Wikitext-103 dataset.}
\begin{tabular}{c|cc|c}
\hline
 \multirow{2}{*}{\textbf{Bitwidth}} & \multicolumn{2}{|c|}{\textbf{Floating-Point Quantizer}} & \textbf{Lloyd-Max Quantizer} \\
 & Best Format & Wikitext-103 Perplexity & Wikitext-103 Perplexity \\
\hline
7 & E3M3 & 18.32 & 18.27 \\
6 & E3M2 & 19.07 & 18.51 \\
5 & E4M0 & 43.89 & 19.71 \\
\hline
\end{tabular}
\end{center}
\end{table}

\subsection{Proof of Local Optimality of LO-BCQ}
\label{subsec:lobcq_opt_proof}
For a given block $\bm{b}_j$, the quantization MSE during LO-BCQ can be empirically evaluated as $\frac{1}{L_b}\lVert \bm{b}_j- \bm{\hat{b}}_j\rVert^2_2$ where $\bm{\hat{b}}_j$ is computed from equation (\ref{eq:clustered_quantization_definition}) as $C_{f(\bm{b}_j)}(\bm{b}_j)$. Further, for a given block cluster $\mathcal{B}_i$, we compute the quantization MSE as $\frac{1}{|\mathcal{B}_{i}|}\sum_{\bm{b} \in \mathcal{B}_{i}} \frac{1}{L_b}\lVert \bm{b}- C_i^{(n)}(\bm{b})\rVert^2_2$. Therefore, at the end of iteration $n$, we evaluate the overall quantization MSE $J^{(n)}$ for a given operand $\bm{X}$ composed of $N_c$ block clusters as:
\begin{align*}
    \label{eq:mse_iter_n}
    J^{(n)} = \frac{1}{N_c} \sum_{i=1}^{N_c} \frac{1}{|\mathcal{B}_{i}^{(n)}|}\sum_{\bm{v} \in \mathcal{B}_{i}^{(n)}} \frac{1}{L_b}\lVert \bm{b}- B_i^{(n)}(\bm{b})\rVert^2_2
\end{align*}

At the end of iteration $n$, the codebooks are updated from $\mathcal{C}^{(n-1)}$ to $\mathcal{C}^{(n)}$. However, the mapping of a given vector $\bm{b}_j$ to quantizers $\mathcal{C}^{(n)}$ remains as  $f^{(n)}(\bm{b}_j)$. At the next iteration, during the vector clustering step, $f^{(n+1)}(\bm{b}_j)$ finds new mapping of $\bm{b}_j$ to updated codebooks $\mathcal{C}^{(n)}$ such that the quantization MSE over the candidate codebooks is minimized. Therefore, we obtain the following result for $\bm{b}_j$:
\begin{align*}
\frac{1}{L_b}\lVert \bm{b}_j - C_{f^{(n+1)}(\bm{b}_j)}^{(n)}(\bm{b}_j)\rVert^2_2 \le \frac{1}{L_b}\lVert \bm{b}_j - C_{f^{(n)}(\bm{b}_j)}^{(n)}(\bm{b}_j)\rVert^2_2
\end{align*}

That is, quantizing $\bm{b}_j$ at the end of the block clustering step of iteration $n+1$ results in lower quantization MSE compared to quantizing at the end of iteration $n$. Since this is true for all $\bm{b} \in \bm{X}$, we assert the following:
\begin{equation}
\begin{split}
\label{eq:mse_ineq_1}
    \tilde{J}^{(n+1)} &= \frac{1}{N_c} \sum_{i=1}^{N_c} \frac{1}{|\mathcal{B}_{i}^{(n+1)}|}\sum_{\bm{b} \in \mathcal{B}_{i}^{(n+1)}} \frac{1}{L_b}\lVert \bm{b} - C_i^{(n)}(b)\rVert^2_2 \le J^{(n)}
\end{split}
\end{equation}
where $\tilde{J}^{(n+1)}$ is the the quantization MSE after the vector clustering step at iteration $n+1$.

Next, during the codebook update step (\ref{eq:quantizers_update}) at iteration $n+1$, the per-cluster codebooks $\mathcal{C}^{(n)}$ are updated to $\mathcal{C}^{(n+1)}$ by invoking the Lloyd-Max algorithm \citep{Lloyd}. We know that for any given value distribution, the Lloyd-Max algorithm minimizes the quantization MSE. Therefore, for a given vector cluster $\mathcal{B}_i$ we obtain the following result:

\begin{equation}
    \frac{1}{|\mathcal{B}_{i}^{(n+1)}|}\sum_{\bm{b} \in \mathcal{B}_{i}^{(n+1)}} \frac{1}{L_b}\lVert \bm{b}- C_i^{(n+1)}(\bm{b})\rVert^2_2 \le \frac{1}{|\mathcal{B}_{i}^{(n+1)}|}\sum_{\bm{b} \in \mathcal{B}_{i}^{(n+1)}} \frac{1}{L_b}\lVert \bm{b}- C_i^{(n)}(\bm{b})\rVert^2_2
\end{equation}

The above equation states that quantizing the given block cluster $\mathcal{B}_i$ after updating the associated codebook from $C_i^{(n)}$ to $C_i^{(n+1)}$ results in lower quantization MSE. Since this is true for all the block clusters, we derive the following result: 
\begin{equation}
\begin{split}
\label{eq:mse_ineq_2}
     J^{(n+1)} &= \frac{1}{N_c} \sum_{i=1}^{N_c} \frac{1}{|\mathcal{B}_{i}^{(n+1)}|}\sum_{\bm{b} \in \mathcal{B}_{i}^{(n+1)}} \frac{1}{L_b}\lVert \bm{b}- C_i^{(n+1)}(\bm{b})\rVert^2_2  \le \tilde{J}^{(n+1)}   
\end{split}
\end{equation}

Following (\ref{eq:mse_ineq_1}) and (\ref{eq:mse_ineq_2}), we find that the quantization MSE is non-increasing for each iteration, that is, $J^{(1)} \ge J^{(2)} \ge J^{(3)} \ge \ldots \ge J^{(M)}$ where $M$ is the maximum number of iterations. 
%Therefore, we can say that if the algorithm converges, then it must be that it has converged to a local minimum. 
\hfill $\blacksquare$


\begin{figure}
    \begin{center}
    \includegraphics[width=0.5\textwidth]{sections//figures/mse_vs_iter.pdf}
    \end{center}
    \caption{\small NMSE vs iterations during LO-BCQ compared to other block quantization proposals}
    \label{fig:nmse_vs_iter}
\end{figure}

Figure \ref{fig:nmse_vs_iter} shows the empirical convergence of LO-BCQ across several block lengths and number of codebooks. Also, the MSE achieved by LO-BCQ is compared to baselines such as MXFP and VSQ. As shown, LO-BCQ converges to a lower MSE than the baselines. Further, we achieve better convergence for larger number of codebooks ($N_c$) and for a smaller block length ($L_b$), both of which increase the bitwidth of BCQ (see Eq \ref{eq:bitwidth_bcq}).


\subsection{Additional Accuracy Results}
%Table \ref{tab:lobcq_config} lists the various LOBCQ configurations and their corresponding bitwidths.
\begin{table}
\setlength{\tabcolsep}{4.75pt}
\begin{center}
\caption{\label{tab:lobcq_config} Various LO-BCQ configurations and their bitwidths.}
\begin{tabular}{|c||c|c|c|c||c|c||c|} 
\hline
 & \multicolumn{4}{|c||}{$L_b=8$} & \multicolumn{2}{|c||}{$L_b=4$} & $L_b=2$ \\
 \hline
 \backslashbox{$L_A$\kern-1em}{\kern-1em$N_c$} & 2 & 4 & 8 & 16 & 2 & 4 & 2 \\
 \hline
 64 & 4.25 & 4.375 & 4.5 & 4.625 & 4.375 & 4.625 & 4.625\\
 \hline
 32 & 4.375 & 4.5 & 4.625& 4.75 & 4.5 & 4.75 & 4.75 \\
 \hline
 16 & 4.625 & 4.75& 4.875 & 5 & 4.75 & 5 & 5 \\
 \hline
\end{tabular}
\end{center}
\end{table}

%\subsection{Perplexity achieved by various LO-BCQ configurations on Wikitext-103 dataset}

\begin{table} \centering
\begin{tabular}{|c||c|c|c|c||c|c||c|} 
\hline
 $L_b \rightarrow$& \multicolumn{4}{c||}{8} & \multicolumn{2}{c||}{4} & 2\\
 \hline
 \backslashbox{$L_A$\kern-1em}{\kern-1em$N_c$} & 2 & 4 & 8 & 16 & 2 & 4 & 2  \\
 %$N_c \rightarrow$ & 2 & 4 & 8 & 16 & 2 & 4 & 2 \\
 \hline
 \hline
 \multicolumn{8}{c}{GPT3-1.3B (FP32 PPL = 9.98)} \\ 
 \hline
 \hline
 64 & 10.40 & 10.23 & 10.17 & 10.15 &  10.28 & 10.18 & 10.19 \\
 \hline
 32 & 10.25 & 10.20 & 10.15 & 10.12 &  10.23 & 10.17 & 10.17 \\
 \hline
 16 & 10.22 & 10.16 & 10.10 & 10.09 &  10.21 & 10.14 & 10.16 \\
 \hline
  \hline
 \multicolumn{8}{c}{GPT3-8B (FP32 PPL = 7.38)} \\ 
 \hline
 \hline
 64 & 7.61 & 7.52 & 7.48 &  7.47 &  7.55 &  7.49 & 7.50 \\
 \hline
 32 & 7.52 & 7.50 & 7.46 &  7.45 &  7.52 &  7.48 & 7.48  \\
 \hline
 16 & 7.51 & 7.48 & 7.44 &  7.44 &  7.51 &  7.49 & 7.47  \\
 \hline
\end{tabular}
\caption{\label{tab:ppl_gpt3_abalation} Wikitext-103 perplexity across GPT3-1.3B and 8B models.}
\end{table}

\begin{table} \centering
\begin{tabular}{|c||c|c|c|c||} 
\hline
 $L_b \rightarrow$& \multicolumn{4}{c||}{8}\\
 \hline
 \backslashbox{$L_A$\kern-1em}{\kern-1em$N_c$} & 2 & 4 & 8 & 16 \\
 %$N_c \rightarrow$ & 2 & 4 & 8 & 16 & 2 & 4 & 2 \\
 \hline
 \hline
 \multicolumn{5}{|c|}{Llama2-7B (FP32 PPL = 5.06)} \\ 
 \hline
 \hline
 64 & 5.31 & 5.26 & 5.19 & 5.18  \\
 \hline
 32 & 5.23 & 5.25 & 5.18 & 5.15  \\
 \hline
 16 & 5.23 & 5.19 & 5.16 & 5.14  \\
 \hline
 \multicolumn{5}{|c|}{Nemotron4-15B (FP32 PPL = 5.87)} \\ 
 \hline
 \hline
 64  & 6.3 & 6.20 & 6.13 & 6.08  \\
 \hline
 32  & 6.24 & 6.12 & 6.07 & 6.03  \\
 \hline
 16  & 6.12 & 6.14 & 6.04 & 6.02  \\
 \hline
 \multicolumn{5}{|c|}{Nemotron4-340B (FP32 PPL = 3.48)} \\ 
 \hline
 \hline
 64 & 3.67 & 3.62 & 3.60 & 3.59 \\
 \hline
 32 & 3.63 & 3.61 & 3.59 & 3.56 \\
 \hline
 16 & 3.61 & 3.58 & 3.57 & 3.55 \\
 \hline
\end{tabular}
\caption{\label{tab:ppl_llama7B_nemo15B} Wikitext-103 perplexity compared to FP32 baseline in Llama2-7B and Nemotron4-15B, 340B models}
\end{table}

%\subsection{Perplexity achieved by various LO-BCQ configurations on MMLU dataset}


\begin{table} \centering
\begin{tabular}{|c||c|c|c|c||c|c|c|c|} 
\hline
 $L_b \rightarrow$& \multicolumn{4}{c||}{8} & \multicolumn{4}{c||}{8}\\
 \hline
 \backslashbox{$L_A$\kern-1em}{\kern-1em$N_c$} & 2 & 4 & 8 & 16 & 2 & 4 & 8 & 16  \\
 %$N_c \rightarrow$ & 2 & 4 & 8 & 16 & 2 & 4 & 2 \\
 \hline
 \hline
 \multicolumn{5}{|c|}{Llama2-7B (FP32 Accuracy = 45.8\%)} & \multicolumn{4}{|c|}{Llama2-70B (FP32 Accuracy = 69.12\%)} \\ 
 \hline
 \hline
 64 & 43.9 & 43.4 & 43.9 & 44.9 & 68.07 & 68.27 & 68.17 & 68.75 \\
 \hline
 32 & 44.5 & 43.8 & 44.9 & 44.5 & 68.37 & 68.51 & 68.35 & 68.27  \\
 \hline
 16 & 43.9 & 42.7 & 44.9 & 45 & 68.12 & 68.77 & 68.31 & 68.59  \\
 \hline
 \hline
 \multicolumn{5}{|c|}{GPT3-22B (FP32 Accuracy = 38.75\%)} & \multicolumn{4}{|c|}{Nemotron4-15B (FP32 Accuracy = 64.3\%)} \\ 
 \hline
 \hline
 64 & 36.71 & 38.85 & 38.13 & 38.92 & 63.17 & 62.36 & 63.72 & 64.09 \\
 \hline
 32 & 37.95 & 38.69 & 39.45 & 38.34 & 64.05 & 62.30 & 63.8 & 64.33  \\
 \hline
 16 & 38.88 & 38.80 & 38.31 & 38.92 & 63.22 & 63.51 & 63.93 & 64.43  \\
 \hline
\end{tabular}
\caption{\label{tab:mmlu_abalation} Accuracy on MMLU dataset across GPT3-22B, Llama2-7B, 70B and Nemotron4-15B models.}
\end{table}


%\subsection{Perplexity achieved by various LO-BCQ configurations on LM evaluation harness}

\begin{table} \centering
\begin{tabular}{|c||c|c|c|c||c|c|c|c|} 
\hline
 $L_b \rightarrow$& \multicolumn{4}{c||}{8} & \multicolumn{4}{c||}{8}\\
 \hline
 \backslashbox{$L_A$\kern-1em}{\kern-1em$N_c$} & 2 & 4 & 8 & 16 & 2 & 4 & 8 & 16  \\
 %$N_c \rightarrow$ & 2 & 4 & 8 & 16 & 2 & 4 & 2 \\
 \hline
 \hline
 \multicolumn{5}{|c|}{Race (FP32 Accuracy = 37.51\%)} & \multicolumn{4}{|c|}{Boolq (FP32 Accuracy = 64.62\%)} \\ 
 \hline
 \hline
 64 & 36.94 & 37.13 & 36.27 & 37.13 & 63.73 & 62.26 & 63.49 & 63.36 \\
 \hline
 32 & 37.03 & 36.36 & 36.08 & 37.03 & 62.54 & 63.51 & 63.49 & 63.55  \\
 \hline
 16 & 37.03 & 37.03 & 36.46 & 37.03 & 61.1 & 63.79 & 63.58 & 63.33  \\
 \hline
 \hline
 \multicolumn{5}{|c|}{Winogrande (FP32 Accuracy = 58.01\%)} & \multicolumn{4}{|c|}{Piqa (FP32 Accuracy = 74.21\%)} \\ 
 \hline
 \hline
 64 & 58.17 & 57.22 & 57.85 & 58.33 & 73.01 & 73.07 & 73.07 & 72.80 \\
 \hline
 32 & 59.12 & 58.09 & 57.85 & 58.41 & 73.01 & 73.94 & 72.74 & 73.18  \\
 \hline
 16 & 57.93 & 58.88 & 57.93 & 58.56 & 73.94 & 72.80 & 73.01 & 73.94  \\
 \hline
\end{tabular}
\caption{\label{tab:mmlu_abalation} Accuracy on LM evaluation harness tasks on GPT3-1.3B model.}
\end{table}

\begin{table} \centering
\begin{tabular}{|c||c|c|c|c||c|c|c|c|} 
\hline
 $L_b \rightarrow$& \multicolumn{4}{c||}{8} & \multicolumn{4}{c||}{8}\\
 \hline
 \backslashbox{$L_A$\kern-1em}{\kern-1em$N_c$} & 2 & 4 & 8 & 16 & 2 & 4 & 8 & 16  \\
 %$N_c \rightarrow$ & 2 & 4 & 8 & 16 & 2 & 4 & 2 \\
 \hline
 \hline
 \multicolumn{5}{|c|}{Race (FP32 Accuracy = 41.34\%)} & \multicolumn{4}{|c|}{Boolq (FP32 Accuracy = 68.32\%)} \\ 
 \hline
 \hline
 64 & 40.48 & 40.10 & 39.43 & 39.90 & 69.20 & 68.41 & 69.45 & 68.56 \\
 \hline
 32 & 39.52 & 39.52 & 40.77 & 39.62 & 68.32 & 67.43 & 68.17 & 69.30  \\
 \hline
 16 & 39.81 & 39.71 & 39.90 & 40.38 & 68.10 & 66.33 & 69.51 & 69.42  \\
 \hline
 \hline
 \multicolumn{5}{|c|}{Winogrande (FP32 Accuracy = 67.88\%)} & \multicolumn{4}{|c|}{Piqa (FP32 Accuracy = 78.78\%)} \\ 
 \hline
 \hline
 64 & 66.85 & 66.61 & 67.72 & 67.88 & 77.31 & 77.42 & 77.75 & 77.64 \\
 \hline
 32 & 67.25 & 67.72 & 67.72 & 67.00 & 77.31 & 77.04 & 77.80 & 77.37  \\
 \hline
 16 & 68.11 & 68.90 & 67.88 & 67.48 & 77.37 & 78.13 & 78.13 & 77.69  \\
 \hline
\end{tabular}
\caption{\label{tab:mmlu_abalation} Accuracy on LM evaluation harness tasks on GPT3-8B model.}
\end{table}

\begin{table} \centering
\begin{tabular}{|c||c|c|c|c||c|c|c|c|} 
\hline
 $L_b \rightarrow$& \multicolumn{4}{c||}{8} & \multicolumn{4}{c||}{8}\\
 \hline
 \backslashbox{$L_A$\kern-1em}{\kern-1em$N_c$} & 2 & 4 & 8 & 16 & 2 & 4 & 8 & 16  \\
 %$N_c \rightarrow$ & 2 & 4 & 8 & 16 & 2 & 4 & 2 \\
 \hline
 \hline
 \multicolumn{5}{|c|}{Race (FP32 Accuracy = 40.67\%)} & \multicolumn{4}{|c|}{Boolq (FP32 Accuracy = 76.54\%)} \\ 
 \hline
 \hline
 64 & 40.48 & 40.10 & 39.43 & 39.90 & 75.41 & 75.11 & 77.09 & 75.66 \\
 \hline
 32 & 39.52 & 39.52 & 40.77 & 39.62 & 76.02 & 76.02 & 75.96 & 75.35  \\
 \hline
 16 & 39.81 & 39.71 & 39.90 & 40.38 & 75.05 & 73.82 & 75.72 & 76.09  \\
 \hline
 \hline
 \multicolumn{5}{|c|}{Winogrande (FP32 Accuracy = 70.64\%)} & \multicolumn{4}{|c|}{Piqa (FP32 Accuracy = 79.16\%)} \\ 
 \hline
 \hline
 64 & 69.14 & 70.17 & 70.17 & 70.56 & 78.24 & 79.00 & 78.62 & 78.73 \\
 \hline
 32 & 70.96 & 69.69 & 71.27 & 69.30 & 78.56 & 79.49 & 79.16 & 78.89  \\
 \hline
 16 & 71.03 & 69.53 & 69.69 & 70.40 & 78.13 & 79.16 & 79.00 & 79.00  \\
 \hline
\end{tabular}
\caption{\label{tab:mmlu_abalation} Accuracy on LM evaluation harness tasks on GPT3-22B model.}
\end{table}

\begin{table} \centering
\begin{tabular}{|c||c|c|c|c||c|c|c|c|} 
\hline
 $L_b \rightarrow$& \multicolumn{4}{c||}{8} & \multicolumn{4}{c||}{8}\\
 \hline
 \backslashbox{$L_A$\kern-1em}{\kern-1em$N_c$} & 2 & 4 & 8 & 16 & 2 & 4 & 8 & 16  \\
 %$N_c \rightarrow$ & 2 & 4 & 8 & 16 & 2 & 4 & 2 \\
 \hline
 \hline
 \multicolumn{5}{|c|}{Race (FP32 Accuracy = 44.4\%)} & \multicolumn{4}{|c|}{Boolq (FP32 Accuracy = 79.29\%)} \\ 
 \hline
 \hline
 64 & 42.49 & 42.51 & 42.58 & 43.45 & 77.58 & 77.37 & 77.43 & 78.1 \\
 \hline
 32 & 43.35 & 42.49 & 43.64 & 43.73 & 77.86 & 75.32 & 77.28 & 77.86  \\
 \hline
 16 & 44.21 & 44.21 & 43.64 & 42.97 & 78.65 & 77 & 76.94 & 77.98  \\
 \hline
 \hline
 \multicolumn{5}{|c|}{Winogrande (FP32 Accuracy = 69.38\%)} & \multicolumn{4}{|c|}{Piqa (FP32 Accuracy = 78.07\%)} \\ 
 \hline
 \hline
 64 & 68.9 & 68.43 & 69.77 & 68.19 & 77.09 & 76.82 & 77.09 & 77.86 \\
 \hline
 32 & 69.38 & 68.51 & 68.82 & 68.90 & 78.07 & 76.71 & 78.07 & 77.86  \\
 \hline
 16 & 69.53 & 67.09 & 69.38 & 68.90 & 77.37 & 77.8 & 77.91 & 77.69  \\
 \hline
\end{tabular}
\caption{\label{tab:mmlu_abalation} Accuracy on LM evaluation harness tasks on Llama2-7B model.}
\end{table}

\begin{table} \centering
\begin{tabular}{|c||c|c|c|c||c|c|c|c|} 
\hline
 $L_b \rightarrow$& \multicolumn{4}{c||}{8} & \multicolumn{4}{c||}{8}\\
 \hline
 \backslashbox{$L_A$\kern-1em}{\kern-1em$N_c$} & 2 & 4 & 8 & 16 & 2 & 4 & 8 & 16  \\
 %$N_c \rightarrow$ & 2 & 4 & 8 & 16 & 2 & 4 & 2 \\
 \hline
 \hline
 \multicolumn{5}{|c|}{Race (FP32 Accuracy = 48.8\%)} & \multicolumn{4}{|c|}{Boolq (FP32 Accuracy = 85.23\%)} \\ 
 \hline
 \hline
 64 & 49.00 & 49.00 & 49.28 & 48.71 & 82.82 & 84.28 & 84.03 & 84.25 \\
 \hline
 32 & 49.57 & 48.52 & 48.33 & 49.28 & 83.85 & 84.46 & 84.31 & 84.93  \\
 \hline
 16 & 49.85 & 49.09 & 49.28 & 48.99 & 85.11 & 84.46 & 84.61 & 83.94  \\
 \hline
 \hline
 \multicolumn{5}{|c|}{Winogrande (FP32 Accuracy = 79.95\%)} & \multicolumn{4}{|c|}{Piqa (FP32 Accuracy = 81.56\%)} \\ 
 \hline
 \hline
 64 & 78.77 & 78.45 & 78.37 & 79.16 & 81.45 & 80.69 & 81.45 & 81.5 \\
 \hline
 32 & 78.45 & 79.01 & 78.69 & 80.66 & 81.56 & 80.58 & 81.18 & 81.34  \\
 \hline
 16 & 79.95 & 79.56 & 79.79 & 79.72 & 81.28 & 81.66 & 81.28 & 80.96  \\
 \hline
\end{tabular}
\caption{\label{tab:mmlu_abalation} Accuracy on LM evaluation harness tasks on Llama2-70B model.}
\end{table}

%\section{MSE Studies}
%\textcolor{red}{TODO}


\subsection{Number Formats and Quantization Method}
\label{subsec:numFormats_quantMethod}
\subsubsection{Integer Format}
An $n$-bit signed integer (INT) is typically represented with a 2s-complement format \citep{yao2022zeroquant,xiao2023smoothquant,dai2021vsq}, where the most significant bit denotes the sign.

\subsubsection{Floating Point Format}
An $n$-bit signed floating point (FP) number $x$ comprises of a 1-bit sign ($x_{\mathrm{sign}}$), $B_m$-bit mantissa ($x_{\mathrm{mant}}$) and $B_e$-bit exponent ($x_{\mathrm{exp}}$) such that $B_m+B_e=n-1$. The associated constant exponent bias ($E_{\mathrm{bias}}$) is computed as $(2^{{B_e}-1}-1)$. We denote this format as $E_{B_e}M_{B_m}$.  

\subsubsection{Quantization Scheme}
\label{subsec:quant_method}
A quantization scheme dictates how a given unquantized tensor is converted to its quantized representation. We consider FP formats for the purpose of illustration. Given an unquantized tensor $\bm{X}$ and an FP format $E_{B_e}M_{B_m}$, we first, we compute the quantization scale factor $s_X$ that maps the maximum absolute value of $\bm{X}$ to the maximum quantization level of the $E_{B_e}M_{B_m}$ format as follows:
\begin{align}
\label{eq:sf}
    s_X = \frac{\mathrm{max}(|\bm{X}|)}{\mathrm{max}(E_{B_e}M_{B_m})}
\end{align}
In the above equation, $|\cdot|$ denotes the absolute value function.

Next, we scale $\bm{X}$ by $s_X$ and quantize it to $\hat{\bm{X}}$ by rounding it to the nearest quantization level of $E_{B_e}M_{B_m}$ as:

\begin{align}
\label{eq:tensor_quant}
    \hat{\bm{X}} = \text{round-to-nearest}\left(\frac{\bm{X}}{s_X}, E_{B_e}M_{B_m}\right)
\end{align}

We perform dynamic max-scaled quantization \citep{wu2020integer}, where the scale factor $s$ for activations is dynamically computed during runtime.

\subsection{Vector Scaled Quantization}
\begin{wrapfigure}{r}{0.35\linewidth}
  \centering
  \includegraphics[width=\linewidth]{sections/figures/vsquant.jpg}
  \caption{\small Vectorwise decomposition for per-vector scaled quantization (VSQ \citep{dai2021vsq}).}
  \label{fig:vsquant}
\end{wrapfigure}
During VSQ \citep{dai2021vsq}, the operand tensors are decomposed into 1D vectors in a hardware friendly manner as shown in Figure \ref{fig:vsquant}. Since the decomposed tensors are used as operands in matrix multiplications during inference, it is beneficial to perform this decomposition along the reduction dimension of the multiplication. The vectorwise quantization is performed similar to tensorwise quantization described in Equations \ref{eq:sf} and \ref{eq:tensor_quant}, where a scale factor $s_v$ is required for each vector $\bm{v}$ that maps the maximum absolute value of that vector to the maximum quantization level. While smaller vector lengths can lead to larger accuracy gains, the associated memory and computational overheads due to the per-vector scale factors increases. To alleviate these overheads, VSQ \citep{dai2021vsq} proposed a second level quantization of the per-vector scale factors to unsigned integers, while MX \citep{rouhani2023shared} quantizes them to integer powers of 2 (denoted as $2^{INT}$).

\subsubsection{MX Format}
The MX format proposed in \citep{rouhani2023microscaling} introduces the concept of sub-block shifting. For every two scalar elements of $b$-bits each, there is a shared exponent bit. The value of this exponent bit is determined through an empirical analysis that targets minimizing quantization MSE. We note that the FP format $E_{1}M_{b}$ is strictly better than MX from an accuracy perspective since it allocates a dedicated exponent bit to each scalar as opposed to sharing it across two scalars. Therefore, we conservatively bound the accuracy of a $b+2$-bit signed MX format with that of a $E_{1}M_{b}$ format in our comparisons. For instance, we use E1M2 format as a proxy for MX4.

\begin{figure}
    \centering
    \includegraphics[width=1\linewidth]{sections//figures/BlockFormats.pdf}
    \caption{\small Comparing LO-BCQ to MX format.}
    \label{fig:block_formats}
\end{figure}

Figure \ref{fig:block_formats} compares our $4$-bit LO-BCQ block format to MX \citep{rouhani2023microscaling}. As shown, both LO-BCQ and MX decompose a given operand tensor into block arrays and each block array into blocks. Similar to MX, we find that per-block quantization ($L_b < L_A$) leads to better accuracy due to increased flexibility. While MX achieves this through per-block $1$-bit micro-scales, we associate a dedicated codebook to each block through a per-block codebook selector. Further, MX quantizes the per-block array scale-factor to E8M0 format without per-tensor scaling. In contrast during LO-BCQ, we find that per-tensor scaling combined with quantization of per-block array scale-factor to E4M3 format results in superior inference accuracy across models. 


\end{document}
