\section{Introduction}\label{sec:intro}


In the \textbf{I}nternet \textbf{o}f \textbf{T}hings (IoT) paradigm, physical objects are equipped with sensing, computing, and networking capabilities. This allows them to monitor or react to their environment and exchange messages with other objects and with remote servers \cite{iot_survey}.
A common use case of this paradigm is the Smart Home, composed of household objects, ranging from small power outlets to large appliances. The popularity of Smart Home devices has increased sharply in the past years, with a market size estimated at 101.07 billion USD in 2024 \cite{smart-home-market-share-fortune}.
Their main appeal is to provide home automation, and therefore convenience, to the user.

% Smart Home devices pose a critical security and safety risk, as they can act directly on the user's personal space, digitally and physically. Moreover, as such devices are basically always connected to the internet, they form a large attack surface.
% Smart Home devices are therefore interesting targets for cyber attacks. In fact, a recent report by NETGEAR \cite{netgear} concluded that, in average, a home network protected by their solution undergoes ten network attacks per day.
% Once a device has been hacked, the attacker can manipulate it or use it to launch attacks against other hosts, as demonstrated by the Mirai botnet \cite{mirai}. An attacker can also use the sensors of IoT devices to obtain sensitive data about their owners. In all cases, there is a risk that the functioning of the Smart Home will be negatively affected or that privacy-sensitive data will be leaked \cite{smart_home_privacy_survey,peek-a-boo}.

% Smart Home networks are an environment that is notoriously difficult to secure \cite{touqeer2021smart}. One reason is the limited awareness of users, who are not IT experts and install Smart Home devices for convenience, leaving the duty to correctly configure and maintain them to the device manufacturers, e.g., in the form of firmware updates. However, many manufacturers focus on expanding the functionalities of the modest hardware of IoT devices as much as possible, and neglect security aspects in the process \cite{toys,in-the-room}. It also cannot be ruled out that the manufacturer itself has equipped the device with undesirable behavior \cite{amazon-alexa}. 
% Furthermore, IoT devices offer only limited computing resources, which makes it difficult to run state-of-the-art security software on them \cite{nbaiot}.

As IoT devices have very low computing resources,
it is a technical challenge to embark them with all the technology
necessary for their correct operation.
Notoriously, to stay economically competitive,
manufacturers tend to expand the user-visible functionalities
on the devices' limited hardware,
neglecting transversal aspects such as robustness or security \cite{toys,in-the-room}.
Ironically, this compromise might impede the initial incentive behind the design of such systems,
i.e. user convenience.
Indeed, if devices do not provide robustness in the face of network communication
instability, the automation system may be unresponsive,
penalizing the user.
The motivation behind this work is to assess the robustness of Smart Home devices
against network instabilities.
More precisely, we envision to discover and describe network communication patterns issued by such devices
in the case where their default traffic fails.

% For these reasons, firewalls and network-based \textbf{I}ntrusion \textbf{D}etection \textbf{S}ystems (IDS) for Smart Homes have been extensively researched in the past decade \cite{passban_ids, argus, duan_iota_2023, smart-home-firewall}. The aim of such security solutions is to detect and block unwanted network communications from and to the devices. Typically, the firewall or IDS is given a description of the traffic allowed and/or disallowed for a particular device. This description, called a \emph{device profile} in the following, must be as complete as possible, as an incomplete profile would cause the firewall or IDS to block legitimate or allow unwanted communication attempts. Unfortunately, only very few manufacturers provide insight into the communication behavior of their IoT devices and services, although the IETF has even standardized a format dubbed the Manufacturer Usage Description (MUD) \cite{mud} to write down profiles in a compact, machine-readable form, albeit admittedly relatively limited \cite{matheu_extending_mud_2019, singh_clearer_than_mud_2019, smart-home-firewall}.

Due to their goal-oriented nature,
Smart Home devices generally exhibit simple, predictable
network communication patterns \cite{sivanathan_classifying_2019}.
Consequently, researchers have designed solutions to express their network behavior
in a reduced and exhaustive form, a so-called \emph{profile}.
The IETF has even standardized a format for such profiles,
the \textbf{M}anufacturer \textbf{U}sage \textbf{D}escription (MUD) \cite{mud},
albeit admittedly relatively limited \cite{matheu_extending_mud_2019, singh_clearer_than_mud_2019, smart-home-firewall}.
Several methods have been presented in the past to automatically create the profile of a device \cite{hamza_clear_2018,ping-pong,homesnitch}. For this purpose, its network communications are observed over a certain period of time in the wild or under laboratory conditions. Data mining and analysis techniques are then used to extract compact representations of the network traffic and associate them with specific device events. In order to gain a comprehensive view on the communication profile of a device, relatively complex experimental setups are required \cite{saidi_haystack_2020, behaviot},
due to the variety of possible interactions with the device.
%, ranging from simply using the manufacturer-provided smartphone application,
%to leveraging a third-party \emph{Smart Home automation platform}
%where the device is part of a user-defined routine.

%
% RAMIN: I don't think we should talk about the quality of signature extraction algorithms,
% since this is not a topic we address in the paper.
%
%% Signature extraction algorithms
%However, existing algorithms to extract signatures for device events from network traffic
%are quite complex.
%Numerous leverage demanding \textbf{M}achine \textbf{L}earning (ML) models,
%requiring a heavy quantity of labeled data,
%which can hinder their applicability in a wide range of systems.
%Moreover, whereas such algorithms indeed usually showcase very accurate results
%when applied to the dataset and system they were trained on,
%their usefulness tends to decline when applied to different systems \CP{Add ref}.
%We argue that the task at hand, i.e. extracting event signatures from packet captures,
%can be achieved with a simpler algorithm,
%i.e. merely extracting the network flows which have occurred
%in all network captures corresponding to successful event executions.

We argue that existing approaches to profiling are insufficient for our goal of gaining comprehensive insight into the network behavior of a Smart Home device, as they do not aim at triggering communication patterns that only become visible under certain network conditions, in particular when the default communication mechanism does not succeed.
For example, an IoT device might have a list of alternative domain names for its cloud hosted services that it will only contact if the default domain name fails to resolve or the server behind that name does not respond. Ignoring such behaviors leads to a profile, in which some of the device's communication patterns are missing or incompletely described.

%\CP{We'll need a section where we show how to use the information we discover to strengthen security}.

In this paper, we present a framework to comprehensively uncover the communication patterns of a Smart Home device. For each type of interaction with the device, we first observe the traffic flows that occur in an unconstrained network and build a set of flow descriptors from them.
Such descriptors contain Internet and transport layer information as well as application layer data, such as domain names. We then iteratively block flows corresponding to certain descriptors and repeat the interaction with the intent of making new flows appear. 
Deciding which flows to block and the sequence of blocking experiments is not straightforward. 
Our aim is to discover as many hidden flows as possible but keep the number of experiments under control. 
To do this efficiently and not end up in an infinite loop where the same patterns appear over and over again, we store the found descriptors in a tree that we traverse using a breadth-first strategy,
and prune using an \textit{ad hoc} heuristic.
The result is a set of \emph{multi-level}, tree-shaped profiles (one for each type of interaction) for the device that describe the device's default and alternative communication patterns that we then explore to assess the robustness of the devices' network communications.
Our profiles are conceptually similar to the aforementioned ones, e.g. MUD,
and can therefore also be used as input configuration for allow-list firewalls.
Current state-of-the-art approaches,
which do not cover alternative communication paths,
might consider the latter as malicious,
whereas they are actually part of the device's intended behavior,
only rarer. 
Profiles obtained from our framework can thus provide more accurate security configurations that do not block the legitimate communications triggered by a robust device.

%In the context of this work, we then leverage the resulting signatures to assess the robustness
%of the devices' network communications.
%However, they might also be repurposed and be used as configuration
%for allow-list firewalls.


% Workflow overview
%In this work, we position ourselves from the perspective of Smart Home events;
%our objective is then to apply our aforementioned strategy to profile the covered events.
%For each event, we will produce a \emph{multi-level profile},
%comprising the first-level traffic patterns,
%as well as the newly emerged patterns.
%In practice, we instrument a corpus of Smart Home devices,
%issue their typical events (e.g. for a smart plug, toggle it),
%and extract the event's network signature.
%We repeat this,
%while iteratively blocking the previously-seen signatures,
%with the intent of making new signatures appear.
%As we want to trigger all the event's potential signatures,
%we stop iterating when no new signature occurs.

% Contributions
The contributions of our work can be summarized as follows:
\begin{itemize}
%\item \textbf{We design a new, simple algorithm to profile Smart Home devices at the network flow level.}
%For a device interaction, the algorithm builds an event signature, i.e., the description of flows appearing for that interaction.
% by extracting the network flows present at each successful event iteration.
% Our algorithm does not rely on Machine Learning nor heavy statistical computation,
% and therefore does not require a big amount of (labeled) data,
% making it portable and easy to apply in various small networks.

\item \textbf{We develop and implement a framework},
modular and easily extensible,
which instruments Smart Home devices and \textbf{automatically extracts multi-level profiles} for them. The core of the framework is a new algorithm that, for a given device interaction, builds an event signature, i.e., the description of flows appearing for that interaction, and then iteratively blocks previously observed flows, until the interaction fails or no new flow is discovered.

\item \textbf{We apply our framework} to a testbed network comprising ten off-the-shelf devices,
and generate multi-level profiles for 36 unique events,
which sum up to a total of 254 unique network flows discovered,
of which 70 were ``hidden'', i.e., not part of the default communication patterns of the tested devices, accounting for over 27\% of the total number of unique flows.

\item \textbf{We assess the robustness} of the instrumented devices,
by computing robustness-related metrics.
We conclude that most devices provide at least one backup communication strategy,
if a default communication pattern fails.

\item \textbf{We publish the source code}
of our signature extraction algorithm and our experimental framework.
Those will be made available after the anonymous review process.
\end{itemize}

% Paper structure
The rest of the paper is structured as follows. We describe the problem and the scope of our work in Section~\ref{sec:problem}.
In Section~\ref{sec:framework},
our methodology is explained in detail.
Its evaluation is presented in Section~\ref{sec:results}.
Section~\ref{sec:related-work} presents research works related to ours,
and positions them compared to ours.
Ultimately, the paper concludes in Section~\ref{sec:conclusion}.
