\section{Related Work}
\label{sec:related-work}

Research on IoT device profiling has been fruitful in the last years.
Such works lie at differing layers of abstraction,
which can be classified along two orthogonal axes:
\begin{itemize}
    \item \emph{Object abstraction}: which semantic unit is profiled, from individual events,
    to single devices, to the IoT network as a whole.
    \item \emph{Behavior abstraction}: whether the work intends to profile the low-level network traffic,
    which can be further dissected into individual packets or aggregate flows,
    or the higher-level abstract device events retrieved from home automation systems.
\end{itemize}

% Following this classification,
% our work intends to profile \emph{individual events},
% at the \emph{network flow} level enhanced with application-layer data.

In the following,
we will summarize related works,
and position them along the aforementioned abstraction axes;
Table~\ref{tab:abstraction} shows the resulting classification.

\begin{table}
    \centering
    \begin{tabular}{|c|c|c|c|}
        % \cline{2-4}
        % \multicolumn{1}{c|}{} & \multicolumn{3}{c|}{\textbf{Object}} \\
        \cline{2-4}
        \multicolumn{1}{c|}{} & \textbf{Event} & \textbf{Device} & \textbf{Network} \\
        \hline
        \textbf{Packets} & \makecell{\emph{PingPong} \cite{ping-pong},\\\emph{D-interact} \cite{sun_inferring_2022}} & \emph{IoTa} \cite{duan_iota_2023} & De Keersmaeker \cite{smart-home-firewall} \\
        \hline
        \textbf{Flows}  & \textbf{This work} & \makecell{MUD \cite{mud},\\Hamza \cite{hamza_clear_2018},\\Mandalari \cite{blocking-without-breaking}} & \makecell{De Keersmaeker \cite{smart-home-firewall},\\Girish \cite{in-the-room}} \\
        \hline
        \textbf{Events} & / & \multicolumn{2}{c|}{\emph{BehavIoT} \cite{behaviot}} \\
        \hline
    \end{tabular}
    \caption{Abstraction levels of device behavior models}
    \label{tab:abstraction}
\end{table}

Girish \textit{et al.} \cite{in-the-room} conducted the first comprehensive analysis
of LAN Smart Home traffic.
They exhaustively characterized the protocols used,
highlighting the security and privacy vulnerabilities,
of intra-network device communication.

Among the numerous proposals for IoT device behavior models,
the most widespread is probably the IETF's MUD standard \cite{mud},
as it has been standardized and backed up by multiple big players,
including Cisco \cite{mud-cisco}.
Subsequently, research leveraging this standard was fueled,
including Hamza \textit{et al.}'s solution to
generate MUD profiles from network traces of IoT devices' traffic \cite{hamza_clear_2018}.
The generated profiles, however, suffer from the shortcomings inherent to the MUD standard,
namely, among others, the lack of support for protocols other than
IP (v4 \& v6) on layer 3,
and TCP, UDP and ICMP on layer 4.
De Keersmaeker \textit{et al.} \cite{smart-home-firewall} proposed a syntax
to express the intended network behavior of devices,
inspired by MUD,
while overcoming some of its shortcomings,
and including support for traffic related to cross-device interactions.

Regarding research efforts to extract event signatures from network traffic,
three works follow a workflow similar to ours:
\emph{PingPong} \cite{ping-pong},
\emph{D-interact} \cite{sun_inferring_2022},
and \emph{IoTa} \cite{duan_iota_2023}.
Nevertheless, all three lie at a finer granularity level than our work,
as their signatures are composed of individual packets,
and leverage packet-level metadata,
such as the packet size.
We argue that such features can be subject to network churn
and therefore vary from one event execution to the other,
making them imprecise to accurately characterize network behavior.

% In an effort similar to ours,
% Sun \textit{et al.} \cite{sun_inferring_2022},
% for their system \emph{D-interact},
% designed a mathematical algorithm to
% extract event signatures,
% comprising their constitutive network packets,
% from Smart Home traffic.
% To extract the signature for an event,
% their workflow starts by collecting $m$
% potentially different packet sequences,
% each corresponding to one event activation.
% For each sequence, they compute the \emph{Levenshtein distance} with all the other sequences.
% The extracted event signature is the sequence for which the sum of distances is minimal.
% The main difference with our work is that they use packet-level signatures,
% whereas we chose to use flow-level signatures,
% as packet-specific metadata (e.g. packet size)
% can be subject to network churn
% and therefore vary from one event execution to the other.
% % Moreover, we include event success verification in our algorithm,
% % which allows us to use a less statistics-heavy mathematical model.
% %\CP{Why are statistics mentioned here ? Should be presented before.}


% \emph{PingPong}, by Trimanada \textit{et al.} \cite{ping-pong},
% is a tool similar to ours:
% it automatically generates device event signatures by instrumenting the devices.
% Their abstraction level is however one layer below ours,
% as their signatures are composed of individual packets.
% Furthermore, they consider a set of identifying packet features disjoint to ours:
% they only leverage the packet length and direction,
% while skipping the flow-representative features we use,
% i.e. hosts, protocol, and port numbers.
% They evaluated their strategy to be accurate,
% but we argue the packet count and length are only precise
% when the network stays perfectly stable.
% On the contrary, our research aims to
% extract signatures taking into account potential network instabilities.

Hu \textit{et al.} proposed \emph{BehavIoT} \cite{behaviot},
a system which models the behavior of a whole Smart Home network,
with the intent of using it to detect when the network produces unintended behavior,
potentially due to unwanted communication, or even a security breach.
Their model is twofold:
on the one hand, they model individual devices events;
on the other hand, cross-device interactions.
Their complete model is generated based on the network traffic produced by the devices,
and takes the form of a probabilistic Finite State Machine (FSM),
representing all the possible event transitions in the home network.

%By covering all individual device events as well as cross-device interactions,
%their work exhibit a broad scope;
%yet, the model is \emph{ad hoc},
%as it can only represent the network in which it was generated.

All of the above-mentioned works have in common that, unlike our work, they do not actively attempt to discover hidden communication patterns.


% \subsection{Model-based Smart Home security systems}

% Various research works have leveraged Smart Home network models
% similar to the aforementioned ones
% as a component of a security system,
% intended to protect the home network from unintended traffic.

% Duan \textit{et al.}, for their \emph{IoTa} system \cite{duan_iota_2023},
% apply a strategy similar to \emph{PingPong}:
% they extract packet-level signatures of devices,
% only considering their length and direction.
% Subsequently, they model the complete device's network pattern with an FSM,
% each node representing one packet signature.
% Additionally, they leveraged the models
% as a traffic monitoring system:
% packets which do not comply with the FSM
% are flagged as unwanted.

% De Keersmaeker \textit{et al.} \cite{smart-home-firewall}
% defined a syntax to express to intended behavior of devices,
% inspired by MUD,
% while overcoming some of MUD's shortcomings,
% and including support for traffic related to cross-device interactions
% They then built a deny-list firewall system,
% taking their profiles as configuration files,
% and leveraging the NFTables Linux firewall \cite{nftables};
% we used their firewall as a building block of our framework.
