\section{Profiling methodology}
\label{sec:framework}

In this section, we describe our profiling methodology. We start with an outline of our approach, followed by a detailed presentation of its components.

\subsection{General workflow}
\label{sec:overview}

Fig.~\ref{fig:workflow} gives an overview on the workflow of our methodology to obtain a profile for a Smart Home device under a certain type of interaction. An interaction can require the usage of a triggering source, such as a companion app installed on a smartphone. In practice, a Smart Home device may support different interactions (e.g., for a smart lamp: switching on/off, changing its color, etc.) and triggering sources, and the workflow described here must be repeated for each of them. The workflow consists of multiple steps summarized below.

%For all devices, the considered interactions have been chosen to closely mimic a real user interacting with the device in a Smart Home setting, by using its companion app; e.g., for a smart lamp, we consider the following interactions: switching on/off, changing the brightness, changing the color.

\section{Problem definition}
\label{sec:problem-def}

The first step in building and using a \CSE{} or SciML model is defining the problem scope: the model's intended purpose, application domain and operating environment, required quantities of interest (QoI) and their scales, and how prior knowledge informs model conceptualization.

\subsection{Model purpose}

\begin{essrec}[Specify prior knowledge and model purpose]
Define the model's intended use and document the essential model properties that must be satisfied. Document any known limitations and constraints of the chosen approach. This ensures appropriate data selection and physics-informed objectives while preventing model misuse outside its intended scope.
\end{essrec}

A SciML model's purpose, as discussed in Section~\ref{sec:scope}, dictates all subsequent modeling choices.
This purpose determines required outer-loop processes and essential properties.

To highlight the importance of specifying the target outer-loop process, consider a model used for explanatory modeling. An explanatory model must simulate all system processes, like ice-sheet thickness and velocity evolution. In contrast, a risk assessment model needs only decision-relevant quantities, like ice-sheet mass loss under varying emissions scenarios. Design and control models, meanwhile, have different requirements than those for risk assessment.
The model purpose dictates the data types and formulations needed to train a SciML model. The impact of this purpose on data requirements and physics-informed objectives varies by application domain. Thus, the exact model formulation should be chosen in light of these problem-specific considerations.


\subsection{Verification, calibration, validation and application domain}

\begin{essrec}[Specify verification, calibration, validation, and application domains]
Define the specific conditions under which the model will operate across the verification, calibration, validation, and prediction phases. These domains are specified by relevant boundary conditions, forcing functions, geometry, and timescales. Account for potential differences between these domains and address any data distribution shifts that could affect model performance. This ensures the selected model architecture and training data align with the intended use while preventing unreliable predictions when operating outside validated conditions.
\end{essrec}

The trustworthy development and deployment of a model (see Figure~\ref{fig:model-development}) requires using the model in regards to verification, calibration, validation, and application domains. These domains are defined by the conditions under which the model operates during these respective phases and must be defined before model construction because they determine viable model classes. Key features include boundary conditions, forcing functions, geometry, and timescales. For ice sheets, examples include surface mass balance, land mass topography, and ocean temperatures.

Each domain will often require the prediction of different quantities of interest under different conditions. Moreover the complexity of the processes being modeled typically increase when transitioning from verification to calibration, to validation to prediction. Additionally the amount of data to complement or inform the model decreases as we move through these domains. For example, the verification domain for our ice-sheet examplar predicts the entire state of the ice-sheet for simple manufactured or analytical solutions. The calibration domain predicts Humboldt Glacier surface velocity under steady-state preindustrial conditions. The validation domain predicts grounding-line change rates from the first decade of this century. The application domain predicts glacier mass change in 2100. Figure~\ref{fig:computational-domains} illustrates these distinct domains. When transitioning between domains, data shifts across domains must be considered. For example, a model trained only on calibration data from recent ice-sheet forcings may fail to predict ice-sheet properties under different future conditions.

\begin{figure}[htb]
    \centering
    \includegraphics[width=0.65\linewidth]{application-domain.pdf}
    \caption{Verification, validation, calibration and application domains.}
    \label{fig:computational-domains}
\end{figure}


\subsection{Quantities of interest}

\begin{essrec}[Carefully select and specify the quantities of interest]
Select and specify the model outputs (quantities of interest, QoI) required for the intended use, considering their form and scale. For risk assessment and design applications, identify the minimal set of QoIs needed for decision-making or optimization. For explanatory modeling, specify the broader range of QoIs needed to capture system behavior. This choice fundamentally determines the required model complexity, training data requirements, and computational approach needed to achieve reliable predictions.
\end{essrec}

Quantities of interest (QoI) are the model outputs required by users. Their form and scale depend on modeling purpose and application domain. We now discuss key considerations for QoI selection.

Risk assessment requires reproducing only decision-critical QoI. For ice-sheets, these include sea-level rise from mass loss and infrastructure damage costs. Design applications similarly need few QoI to evaluate objectives and constraints, like thermal and structural stresses in aerospace vehicles. Design models need accurate QoI predictions only along optimizer trajectories\footnote{For each design iteration the model may still need to be accurate across all uncertain model inputs}, while risk assessment models must predict across many conditions. Explanatory modeling demands more extensive QoI sets, such as complete ice-sheet depth and velocity fields for studying calving. Therefore, simple surrogates often suffice for risk assessment and design, but explanatory modeling may require operators or reduced order models.


\subsection{Model conceptualization}


\begin{essrec}[Select and document model structure]
Select a model structure that fits the model's purpose, domain, and quantities of interest based on relevant prior knowledge such as conservation laws or system properties. Document the alternative model structures considered and the reasoning behind the final selection, including how available resources and computational constraints influenced the choice. This systematic approach ensures the model balances usability, reliability, and feasibility while maintaining transparency about structural assumptions and limitations.
\end{essrec}

Model conceptualization, which follows problem definition, involves selecting model structure based on prior knowledge. While essential to \CSE{} model development~\cite{Jakeman_LN_EMS_2006}, this step requires clear identification of the application domain and relevant QoI.

Model structure selection draws on key prior knowledge: conservation laws, system invariances like rotational and translational symmetries. These guide method selection---for example, symplectic time integrators~\cite{ruth1983canonical} preserve system dynamics properties. Moreover, this knowledge informs and justifies the selection of candidate model structures.
A \CSE{} modeler chooses between model types like lumped versus distributed PDE models, and linear versus nonlinear PDEs. The optimal choice depends on application domain, QoI, and available resources. For example, linear PDEs may introduce more error but their lower computational cost enables better error and uncertainty characterization for tasks like optimal design.
Similar considerations guide SciML model selection. For example, Gaussian processes excel at predicting scalar QoI with few inputs and limited data, but become intractable for larger datasets without variational inference approximations~\cite{Liu_CO_KBS_2018}. In contrast, deep neural networks handle high-dimensional data but require large datasets. The intended use also shapes model structure and training, e.g., optimization applications require controlling derivative errors~\cite{bouhlel2020scalable,o2024derivative} to ensure convergence~\cite{cao2024lazy,luo2023efficient}. These approximation errors must be understood and quantified where possible.

\CSE{} has a strong history of using prior knowledge to formulate governing equations for complex phenomena and deriving numerical methods that respect important physical properties. However, all models are approximate and the best model must balance usability, reliability, and feasibility~\cite{Hamilton_PSFJEMS_2022}. While SciML methods can be usable and feasible, more attention is needed to establish their trustworthiness. In the following two sections, we discuss how \CSE{} V\&V can improve the trustworthiness of SciML research.

\section{Verification}
\label{sec:verification}

Verification increases the trustworthiness of numerical models by demonstrating that the numerical method can adequately solve the equations of the desired mathematical model and the code correctly implements the algorithm. Verification consists of code verification and solution verification, which enhance credibility and trust in the model's predictions. Code and solution verification are well-established in \CSE{} to reduce algorithmic errors. However, verification for SciML models has received less attention due to the field's young age and unique challenges. Moreoever, because SciML models heavily rely on data, unlike \CSE{} models, existing \CSE{} verification notions need to be adapted for SciML.

\subsection{Code verification}
\label{sec:code-verification}

\begin{essrec}[Verify code implementation with test problems]
Evaluate the SciML model's accuracy on simple manufactured test problems using verification data that is independent from training data. Assess how the model error responds to variations in training data samples and optimization parameters while increasing both model complexity and training data size. This systematic testing approach reveals implementation issues, quantifies the impact of sampling and optimization choices, and establishes confidence in the model's numerical implementation.
\end{essrec}

Code verification ensures that a computer code correctly implements the intended mathematical model. For \CSE{} models, this involves confirming that numerical methods and algorithms are free from programming errors (``bugs"). PDE-based \CSE{} models commonly use the method of manufactured solutions (MMS) to verify code on arbitrarily complex solutions. MMS substitutes a user-provided solution into the governing equations, then differentiates it to obtain the exact forcing function and boundary conditions. These solutions check if the code produces the known theoretical convergence rate as the numerical discretization is refined. If the observed order of convergence is less than theoretical, causes such as software bugs, insufficient mesh refinement, or singularities and discontinuities affecting convergence must be identified.

Code verification for SciML models is important but challenging due to the large role of data and nonconvex numerical optimization. Three main challenges limit code verification for many SciML models.
First, while theoretical analysis of SciML models is increasing~\cite{schwab2019deep,schwab2023deep,opschoor2022exponential,leshno1993multilayer,lanthaler2023curse,kovachki2021universal,kovachki2023neural}, many models like neural networks do not generally admit known convergence rates outside specific map classes~\cite{schwab2019deep,schwab2023deep,opschoor2022exponential,herrmann2024neural}, despite their universal approximation properties~\cite{hornik1989multilayer,cybenko1989approximation,leshno1993multilayer}.
Second, generalizable procedures to refine models, such as consistently refining neural-network width and depth as data increases, do not exist.
Finally, regardless of data amount and model unknowns, modeling error often plateaus at a much higher level than machine precision due to nonconvex optimization issues like local minima and saddle points~\cite{Dauphin_PGCGB_NIPS_2014,Bottouleon_CN_SIAMR_2018}.

Developing theoretical and algorithmic advances to address the three main challenges limiting code verification can substantially improve the trustworthiness of SciML models. Convergence-based code verification is currently possible only for certain SciML models with theory that bounds approximation errors in terms of model complexity and training data amount, such as operator methods~\cite{Turnage_et_al_arxiv_2024}, polynomial chaos expansions~\cite{Cohen_M_SMAIJCM_2017,xiu2002wiener}, and Gaussian processes~\cite{Burt_RV_PMLR_2019}.

For SciML models without supporting theory, convergence tests should still be conducted and reported. Studies providing evidence of model convergence engender greater trustworthiness than those that do not, even when the empirically estimated convergence rate cannot be compared to theoretical rates. For example, observing Monte Carlo-type sampling rates in a regime of interest for a fixed overparametrized model can provide intuition into whether the model should be enhanced.

To account for the heavy reliance of SciML models on training data optimization, code verification should be adapted in two ways.
First, report errors in the ML model for a given complexity and data amount for different realizations of the training data to quantify the impact of sampling error, which is not present in \CSE{} models.
Second, because most SciML algorithms introduce optimization error, conduct verification studies that artificially generate data from a random realization of an ML model, then compare the recovered parameter values with the true parameter values or compare the predictions of the learned and true approximations, or at the very least compare the predictions of the two models. Additionally, quantify the sensitivity of the SciML model error to randomness in the optimization by varying the random seed and initial guess passed to the optimizer (see Section~\ref{sec:loss-and-opt}).
All verification tests must employ test data or \emph{verification data}, independent of the training data, to measure the accuracy of the SciML model.


\subsection{Solution verification}

\begin{essrec}[Verify solution accuracy with realistic benchmarks]
Test the model's performance on well-designed, realistic benchmark problems that reflect the intended application domain. Quantify how the model error varies with different training data samples and optimization parameters. When feasible, examine error patterns across different model complexities and data amounts; otherwise, focus on verifying the specific configuration intended for deployment. This ensures the model meets accuracy requirements under realistic conditions while accounting for uncertainties in training and optimization.
\end{essrec}

Code verification establishes a code's ability to reproduce known idealized solutions, while solution verification, performed after code verification, assesses the code's accuracy on more complex yet tractable problems defined by more realistic boundary conditions, forcing, and data. For example, code verification of ice sheets may use manufactured solutions, whereas solution verification may use more realistic MISMIP benchmarks~\cite{Cornford_et_al_TC_2020}. In solution verification, the numerical solution cannot be compared to a known exact solution, and the convergence rate to a known solution cannot be established. Instead, solution verification must use other procedures to estimate the error introduced by the numerical discretization.

Solution verification establishes whether the exact conditions of a model result in the expected theoretical convergence rate or if unexpected features like shocks or singularities prevent it. The most common approach for \CSE{} models compares the difference between consecutive solutions as the numerical discretization is refined and uses Richardson extrapolation to estimate errors. A posteriori error estimation techniques that require solving an adjoint equation can also be used.

While thorough solution verification of CSE models is challenging, these difficulties are further amplified for SciML models. Currently, solution verification of SciML models simply consists of evaluating a trained model's performance using test data separate from the training data. However, this is insufficient as solution verification requires quantifying the impact of increasing data and model complexity on model error. Yet, unfortunately, performing a posteriori error estimation for many SciML models using techniques like Richardson extrapolation is difficult due to the confounding of model expressivity, statistical sampling errors, and variability introduced by converging to local solutions or saddle points of nonconvex optimization, making it challenging to monotonically decrease the error of SciML models such as neural networks. 

Until convergence theory for SciML models improves and automated procedures are developed to change SciML model hyperparameters as data increases, solution verification of SciML models should repeat the sensitivity tests proposed for code verification (Section~\ref{sec:code-verification}) with two key differences:
First, verification experiments used to generate verification data must be specifically designed for solution verification, as not all verification data equally informs solution verification efforts, similar to observations made when creating validation datasets for \CSE{} models~\cite{Oberkampf_T_PAS_2002}. See Section~\ref{sec:data-sources} for more information on important properties of verification benchmarks.
Second, while ideally the convergence of SciML errors on realistic benchmarks should be investigated, it may be computationally impractical. Thus, solution verification should prioritize quantifying errors using the model complexity and data amount that will be used when deploying the SciML model to its application domain.

\section{Validation}
\label{sec:validation}

Verification establishes if a model can accurately produce the behavior of a system described by governing equations. In contrast, validation assesses whether a \CSE{} model's governing equations---or data for SciML models---and the model's implementation can reproduce the physical system's important properties, as determined by the model's purpose.

Validation requires three main steps: (1) solve an inverse problem to calibrate the model to observational data; (2) compare the model's output with observational data collected explicitly for validation; and (3) quantify the uncertainty in model predictions when interpolating or extrapolating from the validation domain to the application domain. We will expand on these steps below.
But first note that the issues affecting the verification of SciML models also affect calibration and validation. Consequently, we will not revisit them here but rather will highlight the unique challenges in validating SciML models.

\subsection{Calibration}

\begin{essrec}[Perform probabilistic calibration]
Calibrate the trained SciML model using observational data to optimize its predictive accuracy for the application domain. Implement Bayesian inference when possible to generate probabilistic parameter estimates and quantify model uncertainty. Choose calibration metrics that account for both model and experimental uncertainties, and select calibration data strategically to maximize information content within experimental constraints. This approach enables reliable uncertainty estimation and optimal use of available observational data.
\end{essrec}

Once a \CSE{} model has been verified, it must be calibrated to match experimental data that contains observational noise. This calibration requires solving an inverse problem~\cite{Stuart_AN_2010}, which can be either deterministic or statistical (e.g., Bayesian). The deterministic approach formulates the inverse problem as a (nonlinear) optimization problem that minimizes the mismatch between model and experimental data. This formulation requires regularization to ensure well-posedness, typically chosen using the L-curve~\cite{hansen1999curve} or the Morozov discrepancy principle~\cite{anzengruber2009morozov}. The Bayesian approach replaces the misfit with a likelihood function based on the noise model, while using a prior distribution for regularization. This prior distribution ensures well-posedness while encoding typical parameter ranges and correlation lengths. We recommend Bayesian methods for calibration because they provide insight into the uncertainty of the reconstructed model parameters. 

The calibration of SciML operator, reduced-order, and hybrid CSE-SciML models is distinct from SciML training and follows similar principles to \CSE{} model calibration. These models are first trained using simulation data for solution verification. Next, observational data (called \emph{calibration data}) determines the optimal model input values that match experimental outputs. For instance, calibrating a SciML ice-sheet model such as that of Ref.~\cite{He_PKS_JCP_2023} requires finding optimal friction field parameters of a trained SciML model, which best predict observed glacier surface velocities, given the noise in the observational data.

Calibration typically improves a model's predictive accuracy on its application domain, but the informative value of calibration data varies significantly. Therefore, researchers should select calibration data strategically to maximize information content within their experimental budget. See Section~\ref{sec:data-sources} for further discussion on collecting informative data.

\subsection{Model validation}

\begin{essrec}[Validate model against purpose-specific requirements]
Define validation metrics that align with the model's intended purpose. Then validate the model using independent data that was not used for training or calibration, ensuring it captures essential physics and boundary conditions of interest. If validation reveals inadequate performance, iterate by collecting additional training data, refining the model structure, or gathering more calibration data until the model achieves satisfactory accuracy for its intended application. This systematic approach will help ensure the model meets stakeholder requirements while maintaining scientific rigor.
\end{essrec}

Model validation is the ``substantiation that a model within its domain of applicability possesses a satisfactory range of accuracy consistent with the intended application of the model''~\cite{Refsgaard_H_AWR_2004}. Validation involves comparing computational results with observational data, then determining if the agreement meets the model's intended purpose~\cite{Lee_et_all_AIAA_2016}. For \CSE{} models with unacceptable validation agreement, modelers must either collect additional calibration data or refine the model structure until reaching acceptable accuracy. SciML models follow a similar iterative process but offer an additional option: to collect more training data.

Model validation must occur after calibration and requires independent data not used for calibration or training. For our conceptual ice-sheet model, calibration matches surface velocities assumed to represent pre-industrial conditions, while validation assesses the calibrated model's ability to predict grounding line change rates at the start of this decade. Performance metrics must target the specific modeling purpose. For optimization tasks, metrics should measure the distance from true optima obtained via the SciML model or bound the associated error~\cite{cao2024lazy}. For uncertainty estimation, metrics should quantify errors in uncertainty statistics through moment discrepancies or density-based measures like (shifted) reverse and forward Kullback--Leibler divergences.
For explanatory SciML modeling, validation metrics must also assess physical fidelity: adherence to physical laws, conservation properties (such as mass and energy), and other constraints. As with verification, the validation concept should encompass \emph{data validation}, particularly whether training data adequately represents the application space.

Validation determines whether a model is acceptable for its specific purpose rather than universally correct. The definition of acceptable is subjective, depending on validation metrics and accuracy requirements established by model stakeholders in alignment with the problem definition and model purpose (see Section~\ref{sec:problem-def}). Moreoever, validation itself does not constitute final model acceptance, which must be based on model accuracy in the application domain, as discussed in Section~\ref{sec:prediction}.

Two additional considerations complete our discussion of model validation. First, this validation differs from the concept of \emph{cross validation}, which estimates ML model accuracy on data representative of the training domain during development. The validation described here assesses accuracy in a separate validation domain. Second, validation data varies in informative value. Validation experiments should ``capture the essential physics of interest, including all relevant physical modeling data and initial and boundary conditions required by the code''~\cite{Oberkampf_T_NED_2008}. Most critically, validation data must remain independent from training and calibration data. 

\subsection{Prediction}
\label{sec:prediction}

\begin{essrec}[Quantify prediction uncertainties]
Assess and quantify all sources of uncertainty affecting model predictions in the application domain, including numerical approximation errors, input and parameter uncertainties, sampling errors from finite training data, and optimization errors. Propagate these uncertainties through the model using appropriate techniques to estimate relevant statistics that match validation criteria. Define acceptance thresholds for prediction uncertainty to ensure the model's reliability for its intended use while acknowledging inherent limitations in uncertainty quantification.
\end{essrec}

Although extensive data may be available for model calibration, validation data is typically scarcer and may not represent the model's intended application domain. According to Schwer~\cite{Schwer_EWC_2007}, ``The original reason for developing a model was to make predictions for applications of the model where no experimental data could, or would, be obtained.'' Therefore, minimizing validation metrics at nominal conditions cannot sufficiently validate a model. Modelers must also quantify accuracy and uncertainty when predictions are extrapolated to the application domain.

SciML models, like \CSE{} models, are subject to numerous sources of uncertainty. The impact of these uncertainties on model predictions must be quantified. Several sources of uncertainty affect \CSE{} models. These include: numerical errors, from approximating the solution to governing equations; input uncertainty arises, which is caused by inexact knowledge of model inputs; parameter uncertainty, which stems from inexact knowledge of model coefficients; and model structure error representing the difference between the model and reality. SciML models contain all these uncertainties. They also incorporate additional uncertainties from sampling and optimization errors, as discussed previously.

Sampling error arises from training a model with a finite amount of possibly noisy data. For a fixed ML model structure and zero optimization error, this error decreases as the amount of data increases. Optimization error represents the difference between the optimized solution, which is often a local optimum, and the global solution for fixed training data. Optimization error can enter \CSE{} models during calibration. Optimization error affects SciML models more significantly because it occurs both during calibration and training. Linear approximations, for example, based on polynomials, achieve zero optimization error during training to machine precision. However, nonlinear approximations such as neural networks often produce non-trivial optimization errors. Stochastic gradient descent demonstrates this by producing different parameter estimates due to stochastic optimization randomness and initial guesses.

The identified sources of modeling uncertainty require parameterization for sampling. Expert knowledge typically guides the construction of prior distributions that represent parametric uncertainty. This parameterization should occur during problem definition. Bayesian calibration updates these priors into posterior distributions using calibration data. The model must then propagate all uncertainties onto predictions in the application domain. Monte Carlo quadrature accomplishes this propagation by drawing random samples from the uncertainty distributions. The method collects model predictions at these samples and computes empirical estimates of important statistics defined by validation criteria, such as mean and variance.

We emphasize the impact of all sources of error and uncertainty must be quantified. Simply estimating the impact of error caused by using finite sample sets, for example estimated by generative models such as variational autoencoders of Gaussian processes is insufficient. Moreover, complete elimination of uncertainty is impossible. Consequently, model acceptance, like validation, must rely on subjective accuracy criteria established through stakeholder communication. For instance, acceptance criteria for predicted sea-level change from melting ice-sheets at year 2100 may specify that prediction precision reaches 1\% of the mean value. Yet, engineering applications, such as those focused on aerospace design, may have much higher accuracy requirements.

The aforementioned Monte-Carlo based UQ procedure effectively quantifies the impact of parameterized uncertainties on model predictions. However, model structure error remains difficult to parameterize in both SciML and \CSE{} modeling. Validation can partially assess model structure error. However, experiments rarely cover all conditions of use. Specifically, validation tests only the model's interpolation ability within the convex hull of available data and assumptions. This limitation creates challenges when applying the model outside its validation domain. Some progress exists in quantifying extrapolation error for ``models based upon highly-reliable theory that is augmented with less-reliable embedded models''~\cite{Oliver_TSM_CMAME_2015}. However, such hybrid CSE-SciML models rely on well-established physics-based governing equations to support extrapolation confidence. Pure SciML models still require substantial research to develop reliable methods for estimating model structure uncertainty.



\hyperref[sec:traffic-capture]{Step 1} consists in performing the interaction with the device under investigation and recording the  network traffic. To this end, we connect the Smart Home device (or, for non-IP devices, their gateway or hub) and the device running the triggering source to a (wireless or wired) LAN and capture the traffic on the Access Point (AP), respectively switch. We call an interaction and the resulting device state an \emph{event}. 
%% FDK: commented out because already explained in detail in the next subsection
%We believe, guided by previous work on device profiling \cite{ping-pong}, that the traffic of the Smart Home device and of the controlling device triggering the event are of equal importance, therefore we collect all their traffic. Step 1 is repeated multiple times.

In \hyperref[sec:sig-ex]{step 2}, the traffic traces obtained are filtered and analyzed, and an \emph{event signature} is extracted. The latter consists of a set of traffic flow descriptors, called \emph{Flow IDs} in the following, that describe the relevant features of the bidirectional flows observed in the traces, and therefore associated with the event. A Flow ID contains network and transport layer information, such as host addresses and port numbers, as well as application layer information (when available). The extracted Flow IDs are stored in a tree-shaped data structure that allows tracking which Flow IDs have already been observed.

In \hyperref[sec:event-tree]{step 3}, we select a Flow ID from the tree, configure a firewall running on the switch or AP to block packets matching that ID, and repeat steps 1 and 2. The idea is that by deliberately blocking some of the traffic associated with the interaction and repeating the steps 1 to 2, the device and/or app will try alternative communication patterns.

% until the event fails, i.e., the desired change of the device is not achieved because required communication has been blocked, or 

The steps above are repeated until no new patterns are discovered.
The final result is the profile of the device for the investigated event.
It takes the form of a tree,
where each node is a Flow ID linked to the event,
and the path from the root towards a given node gives
the set of parent Flow IDs which were blocked to let the current Flow ID appear.
%i.e. all the event signatures linked to that event together with the information which flows were blocked to let them appear
We define the Flow IDs initially occurring without any traffic blocking required as \emph{first-level} Flow IDs,
whereas the ones appearing as a result of traffic blocking are \emph{hidden} Flow IDs.
The latter are the communication patterns which can be discovered
only through our multi-level approach,
and therefore missed by state-of-the-art techniques.

%In the following sections, we describe the different steps of the workflow in more detail.
%An algorithmic representation of the workflow is shown in Algorithm~\ref{algo:exp} and will be referenced in the corresponding sections.

% \begin{algorithm}
%     \caption{Device profiling}
%     \label{algo:exp}
%     \begin{algorithmic}[1]
%         \Procedure{ProfileDevice}{$\mathit{device},\mathit{interaction}$} $\rightarrow$ $\mathit{signature\_tree}$
%             \State tree $\gets$ root node\label{algoempty1}
% 			\State firewall\_rules = \{ \}
% 			\State current\_node $\gets$ root node\label{algoempty2}
%             \Repeat						
% 				\State traces $\gets$ []\label{capturestart}
%                 \For {1 \textbf{to} $m$}
% 					\State trace $\gets$ start traffic capture
% 					\State trigger\_event(device, interaction)
% 					\State stop traffic capture after duration $d$
% 					\If{event is successful}
% 			    		\State add trace to traces
% 					\EndIf
% 					\State sleep random period
%                 \EndFor\label{captureend}

%                 \If {$\|$ traces $\|$ $\leq m/2$}\label{algotest1}
%                     \State continue with next iteration
%                 \EndIf\label{algotest2}
                
%                 \State $s$ $\gets$ extract\_event\_signature(traces)\label{algoextract}
%                 \For{flow\_id \textbf{in} $s$}\label{addloop1}
%                     \If{flow\_id \textbf{not} \textbf{in} tree}\label{algoprune}
%                         \State current\_node.add\_child(flow\_id)
%                     \EndIf
%                 \EndFor\label{addloop2}
                
%                 \State mark current\_node as visited
                
%                 \If{tree has unvisited nodes}\label{algoselect1}
%                     \State current\_node $\gets$ get\_unvisited\_node(tree)\label{algoselect}
%                     \State firewall\_rules $\gets$ reject ( $\{$ current\_node $\}\ \cup$ ancestors(current\_node) )
%                     \State apply firewall\_rules
%                 \EndIf\label{algoselect2}
%             \Until{no unvisited node in tree}

%             \State \Return tree
%         \EndProcedure
%     \end{algorithmic}
% \end{algorithm}

\subsection{Step 1: Traffic capturing}
\label{sec:traffic-capture}

Similarly to \emph{PingPong} \cite{ping-pong} and \emph{IoTAthena} \cite{wan_iotathena_2022}, we start capturing the network traffic, perform the interaction, then stop the traffic capture after a predefined duration $d$. We repeat each interaction and the traffic capturing $m$ times.
Our framework automates interactions triggered by a companion app by leveraging the Android Debug Bridge (ADB) tool \cite{adb} to generate touch events on the smartphone running the companion app. 
We also consider power-cycling the Smart home device as an interaction (called \emph{boot interaction} in the following). To automate the boot interaction, we plug the device in a smart power outlet controlled by our framework.

We collect all incoming and outgoing traffic of the Smart Home device and of the smartphone, since we consider the activities of both of them potentially relevant for the characterization of the event. Between captures, we introduce random waiting intervals to avoid the systematic capturing of periodic background traffic unrelated to the interaction.
%(lines~\ref{capturestart}--\ref{captureend} in Algorithm~\ref{algo:exp}).
Control plane packets are filtered out, in particular all ARP, ICMP, DHCP packets, TCP Handshake packets, and TLS Handshake packets, except \texttt{Client Hello} packets containing the Server Name Indication (SNI) extension.

As our approach is based on blocking traffic, it can happen that the interaction does not cause the desired state change in the Smart Home device. For the event signature extraction described in step 2, it is important to filter out such unsuccessful event executions. We do this by verifying the device state, e.g. whether the lamp has turned on, after each capture. To obtain the device state, we leverage the device's API. We believe this method is more precise and reliable than the approach followed by Mandalari \textit{et al.} \cite{blocking-without-breaking} which compares screenshots of the companion app to check whether the displayed status of the Smart Home device has changed. If there is no API, we fall back to screenshots, too, but we compute the Structural Similarity Measure (SSIM) \cite{ssim} and consider two screenshots identical if the SSIM is over an empirically defined threshold, instead of directly comparing image pixels like Mandalari \textit{et al.} The result are $m^+ \leq m$ traffic captures of successful event executions.

% If the device's state was correctly updated,
% we consider the related network traffic to build the event signature.
% Otherwise, we dismiss the captured traffic,
% and proceed to the next event iteration.

\subsection{Step 2: Event signature extraction}
\label{sec:sig-ex}

For each of the $m^+$ packet traces, the recorded packets are first aggregated to Flow IDs, and then the event signature is extracted. This happens in multiple steps that we now describe.
%(line~\ref{algoextract} in Algorithm~\ref{algo:exp}).

\subsubsection{Replacing IP addresses by domain names}

During an event, the Smart Home device or the companion app may communicate with cloud services provided by the device vendor.
Where possible, we replace non-local source and destination IP addresses by domain names. This makes the signature extraction more robust against changes in the IP addresses caused by cloud migrations or load balancing. To do so, we keep a table matching encountered domain names with their corresponding IP address(es). We first populate the table with entries from the LAN gateway's DNS cache, and then update it whenever a packet bearing domain data is observed, i.e. a DNS query/response or a TLS \texttt{Client Hello} message with the SNI extension.\footnote{We also tried reverse DNS lookups, but since most of the servers contacted during our experiments were hosted at big cloud providers such as Amazon, no meaningful data was obtained, and we abandoned this approach.}
%Therefore, we did not further consider reverse lookups.}


%\begin{itemize}
    %\item For DNS responses, we extract the domain name from the question resource record,
    %along with the related IP address(es) from the answer and additional resource records,
    %and save them in the table.
    %\item For TLS Client Hello packets with the SNI extension,
    %we extract the domain name from said extension,
    %and the IP address for the packet's IP layer.
%\end{itemize}

\subsubsection{Aggregating packets to Flow IDs}

Packets that share all attributes in the set of properties below are grouped to a bidirectional flow, and the attributes form the \emph{Flow ID} of that flow. %We use the following attributes:
\begin{itemize}
\item The hostname (domain name or IP address) of the \emph{source} and \emph{destination}. The Flow ID's \emph{source} is the source of the first packet in the bidirectional flow.
\item The \emph{protocol} (e.g. TCP or UDP).
\item The source and destination \emph{ports}. We only consider a port if it belongs to a well known service (e.g. port 80) or if it appears, for the given combination of the other attribute values, in all $m^+$ packet traces. This means, for example, that the packets of all TCP connections from a client $A$ to the port 80 of server $B$ are aggregated to a single flow and the random client ports used by $A$ are ignored and not further considered. The same is also done for vendor-specific ports, thanks to the above rule.
\item \emph{Application}-specific data for known application layer protocols, amongst others: query name and query type for DNS; method and URI for HTTP; message type, code and URI for CoAP. Consequently, for example, a DNS response is grouped with its query (based on the query type and name).
\end{itemize}

%Grouping along the application-layer data is not as strict as the other attributes,
%as, intuitively, varying application-layer data can be related to the same network flow.
%Therefore, we implemented custom, protocol-specific equality for application-layer data.
%For instance, as a pair of a DNS request and response must be grouped to the same Flow ID,
%we match them on the query type (e.g. A or AAAA) and queried name,
%but not on the QR flag (which indicates if the message is a query or a response).
%Other supported protocols include HTTP and CoAP, for which we respectively match on the method and URI fields,
%and the type (Confirmable or Non-confirmable), code (analogous to HTTP's method), and URI path.

\subsubsection{Building the event signature}

The aggregation of packets to Flow IDs produces a set $f_i$ of Flow IDs, with $1\leq i \leq m^+$, for each of the $m^+$ packet traces.
We define the \emph{event signature} $s$ for the investigated interaction as the set of Flow IDs that appear in all packet traces:
\[
    s := \bigcap_{i=1}^{m^+} f_i
\]
By only keeping the intersection, we filter out Flow IDs belonging to network communications that are not deterministically associated with the event, such as periodic or sporadic messages, as discussed in Section~\ref{sec:problem} and subsection~\ref{sec:traffic-capture}.


%% Moved this to subsubsection "Grouping packets per Flow ID"
% \subsubsection{Retrieve Flow ID's fixed port(s)}

% We consider, in turn, each flow comprised in the \emph{reference pattern},
% which will be dubbed \emph{base flow} in the following.
% Its hosts (regardless of the direction) and transport protocol
% will be used as is to provide data for one of the flows comprising the ultimate signature. \CP{What does it mean to provide data?}
% The ports, however, must not be taken as is;
% indeed, a lot of network flows showcase at least one port number which is random,
% usually on the client side,
% sometimes even on both sides.
% Therefore, we must first examine flows in the other patterns
% which correspond to the same data exchange as the given base flow,
% and consider their ports data.
% Once all related flows have been covered,
% we can potentially extract the port which was identical in all exchanges,
% as part of the signature's flow.
% It must be noted that,
% if at least one of the patterns does not contain any flow matching the base one,
% the base flow must be removed from the \emph{signature},
% as it is not present in every traffic capture.
% \CP{You can remove this last sentence if you talk of the intersection earlier.}


% \subsubsection{Construction of the event signature}

% By joining the hosts and transport protocol from the base flow,
% and the ports data derived from all matching flows,
% we obtain a flow which is part of the event signature.
% By repeating the process over each \emph{base flow} contained in the \emph{reference pattern},
% we construct the complete \emph{event signature} for the event under test.



% \begin{algorithm}
%     \caption{Signature extraction algorithm}
%     \label{algo:sig-extract}
%     \begin{algorithmic}
%         \Function{extract\_signature}{$\mathit{patterns}$}
%             \State sort(patterns)
%             \State ref\_pattern $\gets$ patterns[0]
%             \For{ flow \textbf{in} ref\_pattern }
%                 \State flow\_id $\gets$ extract\_id(flow)
%                 \For{ pattern \textbf{in} patterns \textbackslash\ \{ref\_pattern\} }
%                     \State matching\_flow $\gets$ find\_matching\_flow(pattern, ref\_flow)
%                     \If{ \textbf{no} matching\_flow }
%                         \State \textbf{skip} flow
%                     \EndIf
%                     \State flow\_id $\gets$ flow\_id + matching\_flow
%                 \EndFor
%             \State signature $\gets$ signature + flow\_id
%             \EndFor
%         \State \Return signature
%         \EndFunction
%     \end{algorithmic}
% \end{algorithm}


\subsection{Step 3: Blocking flows with the event signature tree}
\label{sec:event-tree}

The event signature tree is a \emph{connected}, \emph{acyclic}, \emph{rooted tree} where the nodes of the tree are Flow IDs, with the exception of the root node. The tree is used to keep track of the Flow IDs already seen and blocked.

%The intuition behind the tree is the following: After interacting with the device, we obtain an event signature consisting of Flow IDs. We then pick a Flow ID from the signature and configure the firewall to block packets matching it and repeat the interaction. To keep track which Flow IDs have been already blocked and which still have to be investigated, we use the tree.

\subsubsection{Tree creation}

At the beginning, when the profiling starts for a specific interaction of the device, the tree only consists of the root node. The firewall running on the switch or AP to which the Smart Home device and the companion app are connected does not block any traffic.
%(lines~\ref{algoempty1}--\ref{algoempty2} in Algorithm~\ref{algo:exp}).

After obtaining the event signature $s$ from step 2, we add the Flow IDs in the signature as a child node to the root node.
%(lines~\ref{addloop1}--\ref{addloop2} in Algorithm~\ref{algo:exp}).
For the next iteration of steps 1 through 3, we select a node, i.e. a Flow ID, that has not been visited yet from the tree and instruct the firewall to block all packets matching that node and all its ancestors in the tree. The node will be marked as visited and the newly found Flow IDs will be added as children to it, and the procedure is repeated.
%(lines~\ref{algoselect1}--\ref{algoselect2}).

The algorithm ends if no unvisited nodes are left. Theoretically, the algorithm could continue to run indefinitely if new Flow IDs are discovered in each iteration. However, in our experiments, the algorithm always terminated after a finite number of iterations because the more Flow IDs are blocked, the more constrained the communication of the device becomes and the fewer successful event traces are captured, until no more new Flow IDs are added to the tree.

\subsubsection{Tree traversal and pruning}

To select the next node,
%(line~\ref{algoselect})
we use a Breadth-First Search (BFS) \cite{algorithms-book} traversal,
i.e., we process all nodes at a given depth before proceeding to the next level.
%We chose BFS over DFS (\textbf{D}epth-\textbf{F}irst \textbf{S}earch) because it more closely fits to our experimental strategy.
Our objective is to trigger all possible network flows corresponding to a Smart Home device event, i.e., express all possible nodes in the event signature tree.
To manage resource usage and execution time, two strategies are used to limit the size of the tree.

Firstly, we already remove event signatures for which less than half of the trace captures were successful, i.e., $m^+ < m/2$, in step 2.
%(lines~\ref{algotest1}--\ref{algotest2} in Algorithm~\ref{algo:exp}).
In this way, we filter out event signatures for which there is not enough data to reliably determine the Flow IDs that are with a high probability linked to the event. In practice, we observe a polarization in terms of successful event execution:
either all event executions succeed ($m^+=m$) or none ($m^+=0$).
The latter occurs when the firewall rules prevent the event's success.
Setting the threshold at $m/2$ is therefore a conservative choice.

\begin{figure}
  \centering
  \begin{tikzpicture}

    % Root
    \node (root) {root};
  
    %% Depth 1
    % Nodes
    \node [anchor=west, right=0.3cm of root, yshift=1.2cm] (1-plug-phone-tcp) {\circled{A} plug$\leftrightarrow$phone:9999 {[TCP]}};
    \node [anchor=west, right=0.3cm of root, yshift=-0.3cm] (1-plug-server-https) {\circled{B} plug$\leftrightarrow$\texttt{use1-api.tplinkra.com}:443 {[HTTPS]}};
    % Edges
    \draw (root.east) -- (1-plug-phone-tcp.west);
    \draw (root.east) -- (1-plug-server-https.west);
  
    %% Children of 1-plug-phone-tcp
    \node [below=0.5cm of 1-plug-phone-tcp.west, anchor=west, xshift=0.5cm] (A1) (2-plug-server-https) {\circled{C} plug$\leftrightarrow$\texttt{use1-api.tplinkra.com}:443 {[HTTPS]} \xmark};
    \node [below=0.5cm of 2-plug-server-https.west, anchor=west] (2-plug-phone-udp_A) {\circled{D} plug$\leftrightarrow$phone:9999 {[UDP]}};
    \draw ([xshift=0.3cm,yshift=-2.5mm]1-plug-phone-tcp.west) |- (2-plug-server-https.west);
    \draw ([xshift=0.3cm,yshift=-5mm]1-plug-phone-tcp.west) |- (2-plug-phone-udp_A.west);

    %% Children of 1-plug-server-https
    \node [below=0.5cm of 1-plug-server-https.west, anchor=west, xshift=0.5cm] (2-plug-phone-tcp) {\circled{E} plug$\leftrightarrow$phone:9999 {[TCP]} \xmark};
    \node [below=0.5cm of 2-plug-phone-tcp.west, anchor=west] (2-plug-phone-udp_B) {\circled{F} plug$\leftrightarrow$phone:9999 {[UDP]} \xmark};
    \draw ([xshift=0.3cm,yshift=-2.5mm]1-plug-server-https.west) |- (2-plug-phone-tcp.west);
    \draw ([xshift=0.3cm,yshift=-5mm]1-plug-server-https.west) |- (2-plug-phone-udp_B.west);
  
  \end{tikzpicture}
  \caption{\emph{Node pruning}. The nodes \circled{C}, \circled{E}, and \circled{F} are not further explored in the BFS traversal because nodes with the same Flow ID have been already seen (nodes \circled{B}, \circled{A}, and \circled{D}, respectively).}
  \label{fig:tree-pruning}
\end{figure}


Secondly, we prune branches of the tree
which will likely not provide new information,
i.e. no new Flow IDs,
by applying an \textit{ad hoc} pruning heuristic.
To determine which heuristic we can apply without losing any information,
we ran preliminary experiments with one of our testbed's devices,
the TP-Link HS110 smart plug \cite{hs110}.
We observed that, for any two nodes with the same Flow ID,
their children were always identical.
We conclude it is likely that no new information will be found
by processing a node identical to one which has already been processed,
regardless of its position in the tree.
Therefore, we adopt the \emph{node pruning} heuristic,
illustrated in Fig.~\ref{fig:tree-pruning}:
a node is pruned if an equivalent node, i.e., a node with the same Flow ID, has already been processed.


%% FDK: shortened this section, as I feel it does not need such extensive explanation

% A first insight is that,
% since the firewall rules are all blocking,
% the rule order is not important.
% As such, two distinct tree paths from the root to a leaf are equivalent
% if they comprise the same set of nodes,
% regardless of their order.
% We considered two pruning strategies:
% \emph{path pruning} and \emph{node pruning}.

% The \emph{path pruning} strategy prunes a tree node
% if the set of nodes comprising the path from the root to it
% has already been processed, regardless of their order.
% Considering the example tree shown on Fig.~\ref{fig:tree-pruning}
% (an excerpt of a real event signature tree
% resulting from our experiments with the TP-Link HS110 smart plug \cite{hs110}),
% this strategy will prune the node depicted with $\square$,
% as a path composed of node \circled{A} and \circled{B} has already been processed.

% Secondly, we must decide when to prune the exploration of the tree.
% A first insight is that,
% since the firewall rules are all blocking,
% the rule order is not important.
% As such, two distinct tree paths from the root to a leaf are equivalent
% if they comprise the same set of nodes,
% regardless of their order.
% % Moreover, a path $P_b$ which contains a subset of the nodes in another path $P_a$
% % is completely covered by it. 
% % A first pruning strategy would then be to prune a path
% % if it is a subset of, or equivalent to,
% % an already processed path.
% This \emph{path pruning} strategy is illustrated in Figure~\ref{fig:tree-pruning},
% which showcases an excerpt of a real event signature tree
% resulting from our experiments with the TP-Link HS110 smart plug \cite{hs110}.
% The path comprising the nodes \circled{B} and \circled{A} is pruned,
% as an equivalent path, comprising \circled{A} and \circled{B},
% had been processed before.

% To potentially prune the tree even further,
% we explore \emph{node pruning},
% i.e. pruning a tree branch if a node equivalent to the one being processed has already been visited (line~\ref{algoprune} in the algorithm) in a previous branch,
% as illustrated in Figure~\ref{fig:tree-pruning}, where \circled{A}, \circled{B} and \circled{C} are not explored the second time they are encountered.
% The complete resulting event signature tree
% when applying \emph{path pruning}
% is shown in Appendix~\ref{app:preliminary}
% in textual form.

% The \emph{node pruning} strategy prunes a tree node
% if an equivalent singular node has already been processed.
% Considering the example tree shown on Fig.~\ref{fig:tree-pruning},
% this strategy will prune the nodes depicted with $\triangle$.
% This strategy is more aggressive,
% so we performed some preliminary experiments
% to ensure it will still allow to discover all possible flows.
% In practice, we instrumented one of our testbed's devices,
% the TP-Link HS110 smart plug \cite{hs110}.
% We performed a preliminary batch of experiments with it,
% while applying the \emph{path pruning} strategy.
% We observed that
% for any occurrence of a given node,
% its children were always identical.
% The takeaway is, therefore,
% that \emph{path pruning} might be a too conservative strategy in our case,
% and that we can safely apply \emph{node pruning}
% without losing information.


\subsubsection{How to block packets matching a Flow ID}\label{sec:firewall}

As explained, a Flow ID comprises traffic features from multiple networking layers, such as hostnames, port numbers, and application layer information. 
The presence of application layer information in the Flow ID requires a firewall that is able to match and reject traffic based on information found at that layer. A simple firewall, such as the default Linux kernel firewall NFTables \cite{nftables} (successor of the well-known IPTables) is not sufficient.
Instead, we leverage the open-source Smart Home firewall by De Keersmaeker \textit{et al.} \cite{smart-home-firewall}. It is based on NFTables, while enhancing its capabilities to match additional application-layer protocols.
However, this firewall can only work with \emph{allow} rules, while our profiling algorithm is based on blocking rules.
Therefore, we modified the firewall, turning it into a \emph{deny-list} firewall. 
In our prototype implementation of the profiling algorithm, we translate the list of Flow IDs into firewall blocking rules and let the firewall enforce them for the next batch of experiments.
The code of the modified firewall will be made publicly available.

%In our prototype implementation of the profiling algorithm, we translate the list of Flow IDs to a list of blocking rules and produce a configuration file that we give to the firewall before starting the next batch of event generations.

%These rules, applied through a configuration file before the next batch of experiments, force the device or the companion app to reveal alternative communication paths by preventing them from using previously discovered network flows.


%% FDK: Removed the following subsection, as it is less relevant now that the paper's focus is on event robustness.
% \subsection{Profiling outcome}

% The outcome of our methodology takes the form of an \emph{event signature tree}.
% We define the Flow IDs occurring at the first tree depth as \emph{first-level} Flow IDs,
% whereas the ones appearing at depths of two and deeper are \emph{hidden} Flow IDs.
% The latter are the communication patterns which can be discovered
% only through our multi-level approach,
% and therefore missed by state-of-the-art techniques.

% \FDK{Now that the paper's focus is on robustness, do we still need the following paragraph ?}
% One of the incentives behind our work is to strengthen the security of Smart Home networks, by providing more robust device profiles, which could be used, amongst others, as configuration input for allow-list firewalls or IDS. Conveniently, the outcome of the profiling algorithm is a fully explored multi-level event signature tree containing all Flow IDs encountered for a specific interaction, and the firewall configuration files produced during the profiling can be directly used with the allow-list firewall by De Keersmaeker \textit{et al.} (see Section~\ref{sec:firewall}). They are also translatable to MUD profiles \cite{mud}, albeit at the expense of dropping the extracted application layer data, as MUD does not support it.
