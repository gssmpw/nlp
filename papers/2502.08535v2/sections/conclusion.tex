\section{Conclusion}
\label{sec:conclusion}

In this article,
we presented a novel methodology and framework
to extract multi-level signatures
from Smart Home device events,
whereas existing works are limited to first-level signatures.
Our signatures take the form of trees, one per interaction type with the object,
with nodes being the constitutive flows.
We show that more than half of the devices
exhibit \emph{hidden} traffic patterns,
only occurring when tampering with the network they are connected to. 
In average, two hidden flows are discovered for each device, across all 
studied interactions with the device. 
Our work highlights the robustness capabilities of Smart Home devices,
enabling users to take more advised buying decisions. 
One can envisage that, in the future, devices 
are labeled with such a robustness score.

% The implications for Smart Home network security solutions are primordial.
% Indeed, whereas they usually only consider first-level signatures,
% they should encompass the whole tree depth to provide an exhaustive protection,
% the latter being paramount in networks so close to the user's personal space.
% We hope to fuel further research in this direction,
% either by covering the shortcomings of our solution,
% or by applying a similar methodology to other scenarios,
% with the global incentive of making our networks more secure.

As future work, we would like to apply our methodology to scenarios where multiple Smart Home devices are involved in a single interaction.  This is conceptually supported by our methodology, but will require an extension of the mechanisms we used for event triggering. In this case, approaches to further automate the latter should also be investigated.
