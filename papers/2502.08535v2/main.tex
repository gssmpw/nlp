% Package hacks
\PassOptionsToPackage{dvipsnames}{xcolor}
\PassOptionsToPackage{hyphens}{url}

% TODO: for submission, add option `letterpaper`
\documentclass[10pt, conference, letterpaper, anonymous]{IEEEtran}
%\IEEEoverridecommandlockouts

%% Packages
\usepackage{cite}
\usepackage{todonotes}
\usepackage{amsfonts}  % Mathematical sets
\usepackage{gensymb}   % Symbols
\usepackage{algpseudocode}
\usepackage{algorithm}
\usepackage{forest}  % Draw trees in TikZ
\usepackage{xcolor}
\usepackage{pifont}  % Checkmarks and crosses
\usepackage{subcaption}  % Subfigures
\usepackage{multirow}  % Tables: cells spanning multiple rows
\usepackage{makecell}
\usepackage{listings}  % Code listings
\usepackage{hyperref}
%\usepackage{url}  % Clickable URLs
\usepackage{stackengine}  % Overlapping images
\def\BibTeX{{\rm B\kern-.05em{\sc i\kern-.025em b}\kern-.08em
    T\kern-.1667em\lower.7ex\hbox{E}\kern-.125emX}}

%% Commands
\newcommand*\circled[1]{\raisebox{.5pt}{\textcircled{\raisebox{-.9pt} {#1}}}}
\newcommand{\cmark}{\textcolor{Green}{\ding{51}}}
\newcommand{\xmark}{\textcolor{Red}{\ding{55}}}


%% TikZ config
% Libraries
\usetikzlibrary{automata,positioning,backgrounds,patterns}
% Line styles
\tikzstyle{lan}    = [draw=blue, semithick]
\tikzstyle{wlan}   = [draw=blue, dotted, thick]
\tikzstyle{wan}    = [draw=Green, semithick]
\tikzstyle{zigbee} = [draw=red, dotted, thick]
% Styles
\tikzset{
  every node/.style={font=\footnotesize},
  connector/.style={
      -stealth,
      font=\scriptsize
  },
  rectangle connector/.style={
      connector,
      to path={(\tikztostart) -- ++(#1,0pt) \tikztonodes |- (\tikztotarget) },
      pos=0.5
  }
}


\newcommand{\FDK}[1]{\textcolor{orange}{FDK: #1}}
\newcommand{\RS}[1]{\textcolor{blue}{RS: #1}}
\newcommand{\CP}[1]{\textcolor{olive}{CP: #1}}
\newcommand{\RVB}[1]{\textcolor{purple}{RVB: #1}}
\newcommand{\texturl}[1]{{\small \texttt{#1}}}


%%
%% \BibTeX command to typeset BibTeX logo in the docs
\AtBeginDocument{%
  \providecommand\BibTeX{{%
    Bib\TeX}}}


%%
%% end of the preamble, start of the body of the document source.
\begin{document}

%%
%% The "title" command has an optional parameter,
%% allowing the author to define a "short title" to be used in page headers.
\title{The Forest Behind the Tree: Revealing Hidden Smart Home Communication Patterns}


%% AUTHORS %%
%% Uncomment for anonymous submission
%\author{\IEEEauthorblockN{Anonymous Author(s)}}

%% Uncomment for camera-ready
\author{
\IEEEauthorblockN{François De Keersmaeker}
\IEEEauthorblockA{\textit{ICTEAM/INGI} \\
\textit{UCLouvain}\\
Louvain-la-Neuve, Belgium \\
francois.dekeersmaeker@uclouvain.be}
\and
\IEEEauthorblockN{Rémi Van Boxem}
\IEEEauthorblockA{
\textit{JUNIA}\\
Lille, France \\
remi.van-boxem@student.junia.com}
\and
\IEEEauthorblockN{Cristel Pelsser}
\IEEEauthorblockA{\textit{ICTEAM/INGI} \\
\textit{UCLouvain}\\
Louvain-la-Neuve, Belgium \\
cristel.pelsser@uclouvain.be}
\and
\IEEEauthorblockN{Ramin Sadre}
\IEEEauthorblockA{\centerline{\textit{ICTEAM/INGI}} \\
\textit{UCLouvain} \\
Louvain-la-Neuve, Belgium \\
ramin.sadre@uclouvain.be}
}

% \received{20 February 2007}
% \received[revised]{12 March 2009}
% \received[accepted]{5 June 2009}

%%
%% This command processes the author and affiliation and title
%% information and builds the first part of the formatted document.
\maketitle


%%
%% The abstract is a short summary of the work to be presented in the
%% article.
\begin{abstract}

% %% CoNEXT 2025 winter submission
% % Network-connected Smart Home devices are becoming increasingly common, creating potential security and privacy risks. Previous research has shown these devices follow predictable network communication patterns, allowing researchers to model their normal network behavior and detect potential security breaches. However, existing approaches only observe traffic passively rather than actively trying to disturb it.
% % We present a framework that generates comprehensive network signatures for Smart Home devices by systematically blocking previously observed traffic patterns to reveal new, hidden patterns that other methods miss. These signatures are structured as behavior trees, where each child node represents network flows that occur when the parent node's traffic is blocked.
% % We applied this framework on ten real-world devices under 26 usage scenarios, discovering 138 unique flows, of which 27 (20\%) are information gained through our multi-level tree approach, compared to state-of-the-art single-level traffic analysis.

% Because of their widespread use, Smart Home devices have long attracted the interest of researchers who want to better understand how such devices behave in home networks. Compared to general purpose computers, their network activity follows relatively simple and predictable patterns. Existing work has shown that this property allows to describe the network behavior of a device in a compact profile, which can be used, for example, to configure a firewall.

% However, previous approaches have mostly been limited to studying device behavior under normal network conditions and have not attempted to actively disturb communication, thus missing potential hidden patterns that do not appear by default. We argue that uncovering these hidden patterns is important because they provide insight into how robust a device is to challenging network conditions, and because they lead to more accurate and complete profiles.
 
% In this paper, we present a framework for systematically and automatically revealing hidden communication patterns in Smart Home devices. It generates comprehensive profiles for devices by examining and blocking previously observed traffic patterns to reveal new patterns that other, more shallow methods miss. These profiles are structured as behavior trees, where a node represents a network flow which occurs when the parent nodes' flows are blocked. We apply our framework on ten real-world devices with 36 unique user interactions. We discover 254 unique flows, of which 70 ($>$ 27\%) appear only thanks to our approach and which are not found by state-of-the-art methods.

The widespread use of Smart Home devices has attracted significant research interest in understanding their behavior within home networks. Unlike general-purpose computers, these devices exhibit relatively simple and predictable network activity patterns. However, previous studies have primarily focused on normal network conditions, overlooking potential hidden patterns that emerge under challenging conditions. Discovering these hidden flows is crucial for assessing device robustness.

This paper addresses this gap by presenting a framework that systematically and automatically reveals these hidden communication patterns. By actively disturbing communication and blocking observed traffic, the framework generates comprehensive profiles structured as behavior trees, uncovering flows that are missed by more shallow methods. This approach was applied to ten real-world devices, identifying 254 unique flows, with over 27\% only discovered through this new method. These insights enhance our understanding of device robustness and can be leveraged to improve the accuracy of network security measures.
\end{abstract}


%%
%% Keywords. The author(s) should pick words that accurately describe
%% the work being presented. Separate the keywords with commas.
\begin{IEEEkeywords}
  IoT, Smart Home, networks, robustness, security, traffic profiling
\end{IEEEkeywords}


%% BODY TEXT SECTION %%
\section{Introduction}
\label{sec:introduction}
The business processes of organizations are experiencing ever-increasing complexity due to the large amount of data, high number of users, and high-tech devices involved \cite{martin2021pmopportunitieschallenges, beerepoot2023biggestbpmproblems}. This complexity may cause business processes to deviate from normal control flow due to unforeseen and disruptive anomalies \cite{adams2023proceddsriftdetection}. These control-flow anomalies manifest as unknown, skipped, and wrongly-ordered activities in the traces of event logs monitored from the execution of business processes \cite{ko2023adsystematicreview}. For the sake of clarity, let us consider an illustrative example of such anomalies. Figure \ref{FP_ANOMALIES} shows a so-called event log footprint, which captures the control flow relations of four activities of a hypothetical event log. In particular, this footprint captures the control-flow relations between activities \texttt{a}, \texttt{b}, \texttt{c} and \texttt{d}. These are the causal ($\rightarrow$) relation, concurrent ($\parallel$) relation, and other ($\#$) relations such as exclusivity or non-local dependency \cite{aalst2022pmhandbook}. In addition, on the right are six traces, of which five exhibit skipped, wrongly-ordered and unknown control-flow anomalies. For example, $\langle$\texttt{a b d}$\rangle$ has a skipped activity, which is \texttt{c}. Because of this skipped activity, the control-flow relation \texttt{b}$\,\#\,$\texttt{d} is violated, since \texttt{d} directly follows \texttt{b} in the anomalous trace.
\begin{figure}[!t]
\centering
\includegraphics[width=0.9\columnwidth]{images/FP_ANOMALIES.png}
\caption{An example event log footprint with six traces, of which five exhibit control-flow anomalies.}
\label{FP_ANOMALIES}
\end{figure}

\subsection{Control-flow anomaly detection}
Control-flow anomaly detection techniques aim to characterize the normal control flow from event logs and verify whether these deviations occur in new event logs \cite{ko2023adsystematicreview}. To develop control-flow anomaly detection techniques, \revision{process mining} has seen widespread adoption owing to process discovery and \revision{conformance checking}. On the one hand, process discovery is a set of algorithms that encode control-flow relations as a set of model elements and constraints according to a given modeling formalism \cite{aalst2022pmhandbook}; hereafter, we refer to the Petri net, a widespread modeling formalism. On the other hand, \revision{conformance checking} is an explainable set of algorithms that allows linking any deviations with the reference Petri net and providing the fitness measure, namely a measure of how much the Petri net fits the new event log \cite{aalst2022pmhandbook}. Many control-flow anomaly detection techniques based on \revision{conformance checking} (hereafter, \revision{conformance checking}-based techniques) use the fitness measure to determine whether an event log is anomalous \cite{bezerra2009pmad, bezerra2013adlogspais, myers2018icsadpm, pecchia2020applicationfailuresanalysispm}. 

The scientific literature also includes many \revision{conformance checking}-independent techniques for control-flow anomaly detection that combine specific types of trace encodings with machine/deep learning \cite{ko2023adsystematicreview, tavares2023pmtraceencoding}. Whereas these techniques are very effective, their explainability is challenging due to both the type of trace encoding employed and the machine/deep learning model used \cite{rawal2022trustworthyaiadvances,li2023explainablead}. Hence, in the following, we focus on the shortcomings of \revision{conformance checking}-based techniques to investigate whether it is possible to support the development of competitive control-flow anomaly detection techniques while maintaining the explainable nature of \revision{conformance checking}.
\begin{figure}[!t]
\centering
\includegraphics[width=\columnwidth]{images/HIGH_LEVEL_VIEW.png}
\caption{A high-level view of the proposed framework for combining \revision{process mining}-based feature extraction with dimensionality reduction for control-flow anomaly detection.}
\label{HIGH_LEVEL_VIEW}
\end{figure}

\subsection{Shortcomings of \revision{conformance checking}-based techniques}
Unfortunately, the detection effectiveness of \revision{conformance checking}-based techniques is affected by noisy data and low-quality Petri nets, which may be due to human errors in the modeling process or representational bias of process discovery algorithms \cite{bezerra2013adlogspais, pecchia2020applicationfailuresanalysispm, aalst2016pm}. Specifically, on the one hand, noisy data may introduce infrequent and deceptive control-flow relations that may result in inconsistent fitness measures, whereas, on the other hand, checking event logs against a low-quality Petri net could lead to an unreliable distribution of fitness measures. Nonetheless, such Petri nets can still be used as references to obtain insightful information for \revision{process mining}-based feature extraction, supporting the development of competitive and explainable \revision{conformance checking}-based techniques for control-flow anomaly detection despite the problems above. For example, a few works outline that token-based \revision{conformance checking} can be used for \revision{process mining}-based feature extraction to build tabular data and develop effective \revision{conformance checking}-based techniques for control-flow anomaly detection \cite{singh2022lapmsh, debenedictis2023dtadiiot}. However, to the best of our knowledge, the scientific literature lacks a structured proposal for \revision{process mining}-based feature extraction using the state-of-the-art \revision{conformance checking} variant, namely alignment-based \revision{conformance checking}.

\subsection{Contributions}
We propose a novel \revision{process mining}-based feature extraction approach with alignment-based \revision{conformance checking}. This variant aligns the deviating control flow with a reference Petri net; the resulting alignment can be inspected to extract additional statistics such as the number of times a given activity caused mismatches \cite{aalst2022pmhandbook}. We integrate this approach into a flexible and explainable framework for developing techniques for control-flow anomaly detection. The framework combines \revision{process mining}-based feature extraction and dimensionality reduction to handle high-dimensional feature sets, achieve detection effectiveness, and support explainability. Notably, in addition to our proposed \revision{process mining}-based feature extraction approach, the framework allows employing other approaches, enabling a fair comparison of multiple \revision{conformance checking}-based and \revision{conformance checking}-independent techniques for control-flow anomaly detection. Figure \ref{HIGH_LEVEL_VIEW} shows a high-level view of the framework. Business processes are monitored, and event logs obtained from the database of information systems. Subsequently, \revision{process mining}-based feature extraction is applied to these event logs and tabular data input to dimensionality reduction to identify control-flow anomalies. We apply several \revision{conformance checking}-based and \revision{conformance checking}-independent framework techniques to publicly available datasets, simulated data of a case study from railways, and real-world data of a case study from healthcare. We show that the framework techniques implementing our approach outperform the baseline \revision{conformance checking}-based techniques while maintaining the explainable nature of \revision{conformance checking}.

In summary, the contributions of this paper are as follows.
\begin{itemize}
    \item{
        A novel \revision{process mining}-based feature extraction approach to support the development of competitive and explainable \revision{conformance checking}-based techniques for control-flow anomaly detection.
    }
    \item{
        A flexible and explainable framework for developing techniques for control-flow anomaly detection using \revision{process mining}-based feature extraction and dimensionality reduction.
    }
    \item{
        Application to synthetic and real-world datasets of several \revision{conformance checking}-based and \revision{conformance checking}-independent framework techniques, evaluating their detection effectiveness and explainability.
    }
\end{itemize}

The rest of the paper is organized as follows.
\begin{itemize}
    \item Section \ref{sec:related_work} reviews the existing techniques for control-flow anomaly detection, categorizing them into \revision{conformance checking}-based and \revision{conformance checking}-independent techniques.
    \item Section \ref{sec:abccfe} provides the preliminaries of \revision{process mining} to establish the notation used throughout the paper, and delves into the details of the proposed \revision{process mining}-based feature extraction approach with alignment-based \revision{conformance checking}.
    \item Section \ref{sec:framework} describes the framework for developing \revision{conformance checking}-based and \revision{conformance checking}-independent techniques for control-flow anomaly detection that combine \revision{process mining}-based feature extraction and dimensionality reduction.
    \item Section \ref{sec:evaluation} presents the experiments conducted with multiple framework and baseline techniques using data from publicly available datasets and case studies.
    \item Section \ref{sec:conclusions} draws the conclusions and presents future work.
\end{itemize}
%\section{Background}\label{sec:backgrnd}

\subsection{Cold Start Latency and Mitigation Techniques}

Traditional FaaS platforms mitigate cold starts through snapshotting, lightweight virtualization, and warm-state management. Snapshot-based methods like \textbf{REAP} and \textbf{Catalyzer} reduce initialization time by preloading or restoring container states but require significant memory and I/O resources, limiting scalability~\cite{dong_catalyzer_2020, ustiugov_benchmarking_2021}. Lightweight virtualization solutions, such as \textbf{Firecracker} microVMs, achieve fast startup times with strong isolation but depend on robust infrastructure, making them less adaptable to fluctuating workloads~\cite{agache_firecracker_2020}. Warm-state management techniques like \textbf{Faa\$T}~\cite{romero_faa_2021} and \textbf{Kraken}~\cite{vivek_kraken_2021} keep frequently invoked containers ready, balancing readiness and cost efficiency under predictable workloads but incurring overhead when demand is erratic~\cite{romero_faa_2021, vivek_kraken_2021}. While these methods perform well in resource-rich cloud environments, their resource intensity challenges applicability in edge settings.

\subsubsection{Edge FaaS Perspective}

In edge environments, cold start mitigation emphasizes lightweight designs, resource sharing, and hybrid task distribution. Lightweight execution environments like unikernels~\cite{edward_sock_2018} and \textbf{Firecracker}~\cite{agache_firecracker_2020}, as used by \textbf{TinyFaaS}~\cite{pfandzelter_tinyfaas_2020}, minimize resource usage and initialization delays but require careful orchestration to avoid resource contention. Function co-location, demonstrated by \textbf{Photons}~\cite{v_dukic_photons_2020}, reduces redundant initializations by sharing runtime resources among related functions, though this complicates isolation in multi-tenant setups~\cite{v_dukic_photons_2020}. Hybrid offloading frameworks like \textbf{GeoFaaS}~\cite{malekabbasi_geofaas_2024} balance edge-cloud workloads by offloading latency-tolerant tasks to the cloud and reserving edge resources for real-time operations, requiring reliable connectivity and efficient task management. These edge-specific strategies address cold starts effectively but introduce challenges in scalability and orchestration.

\subsection{Predictive Scaling and Caching Techniques}

Efficient resource allocation is vital for maintaining low latency and high availability in serverless platforms. Predictive scaling and caching techniques dynamically provision resources and reduce cold start latency by leveraging workload prediction and state retention.
Traditional FaaS platforms use predictive scaling and caching to optimize resources, employing techniques (OFC, FaasCache) to reduce cold starts. However, these methods rely on centralized orchestration and workload predictability, limiting their effectiveness in dynamic, resource-constrained edge environments.



\subsubsection{Edge FaaS Perspective}

Edge FaaS platforms adapt predictive scaling and caching techniques to constrain resources and heterogeneous environments. \textbf{EDGE-Cache}~\cite{kim_delay-aware_2022} uses traffic profiling to selectively retain high-priority functions, reducing memory overhead while maintaining readiness for frequent requests. Hybrid frameworks like \textbf{GeoFaaS}~\cite{malekabbasi_geofaas_2024} implement distributed caching to balance resources between edge and cloud nodes, enabling low-latency processing for critical tasks while offloading less critical workloads. Machine learning methods, such as clustering-based workload predictors~\cite{gao_machine_2020} and GRU-based models~\cite{guo_applying_2018}, enhance resource provisioning in edge systems by efficiently forecasting workload spikes. These innovations effectively address cold start challenges in edge environments, though their dependency on accurate predictions and robust orchestration poses scalability challenges.

\subsection{Decentralized Orchestration, Function Placement, and Scheduling}

Efficient orchestration in serverless platforms involves workload distribution, resource optimization, and performance assurance. While traditional FaaS platforms rely on centralized control, edge environments require decentralized and adaptive strategies to address unique challenges such as resource constraints and heterogeneous hardware.



\subsubsection{Edge FaaS Perspective}

Edge FaaS platforms adopt decentralized and adaptive orchestration frameworks to meet the demands of resource-constrained environments. Systems like \textbf{Wukong} distribute scheduling across edge nodes, enhancing data locality and scalability while reducing network latency. Lightweight frameworks such as \textbf{OpenWhisk Lite}~\cite{kravchenko_kpavelopenwhisk-light_2024} optimize resource allocation by decentralizing scheduling policies, minimizing cold starts and latency in edge setups~\cite{benjamin_wukong_2020}. Hybrid solutions like \textbf{OpenFaaS}~\cite{noauthor_openfaasfaas_2024} and \textbf{EdgeMatrix}~\cite{shen_edgematrix_2023} combine edge-cloud orchestration to balance resource utilization, retaining latency-sensitive functions at the edge while offloading non-critical workloads to the cloud. While these approaches improve flexibility, they face challenges in maintaining coordination and ensuring consistent performance across distributed nodes.


\section{Problem Formulation} \label{sec:probdef}

This section formally defines the problem of restoring a given pruned network with only using its original pretrained CNN in a way free of data and fine-tuning.



% Unlike many existing works utilize data for identifying unimportant filters as well as fine-tuning to this end, we cannot evaluate the filter importance by data-dependent values like activation maps (\textit{a.k.a.} channels) as our focus in this paper is not to use any training data. Thus, in our problem setting, we can only exploit the values of filters in the original network, and thereby have to make some changes in the remaining filters of the pruned network so that the network can return the output not too much different from the original one.

% No matter how much we carefully select unimportant filters to be pruned, some kinds of retraining process appears inevitable as done by the most existing works to this end. However, since our focus in this paper is not to use any training data, we cannot evaluate the importance of filters by data-dependent values like activation maps (\textit{a.k.a.} channels). 

% To this end, they not only use a careful criterion (\textit{e.g.}, L1-norm), but also fine-tune the network using the original data.
% Most of filter pruning methods try to select filters to be pruned prudently so that pruned network's output be similar to the original network's. To this end, they prune the unimportant filters and then fine-tune the pruned network with using the train data. 

% How can we restore the the pruned networks without any data? In other words, it implies that we cannot use any data-driven values(i.e., activation maps) and we can only exploit the values of original filters. In that case, the only thing we can do maybe changing the weights of remained filters appropriately not to amplify the difference between pruned and unpruned network's outputs through the information of original filters.

\begin{figure*}[t]
	\centering
    \subfigure[\label{fig:matrix:a}Pruning matrix]{\hspace{6mm}\includegraphics[width=0.35\columnwidth]{./figure/LBYL_figure_2_1.pdf}\hspace{6mm}} 
    \subfigure[\label{fig:matrix:b}Delivery matrix for LBYL]{\hspace{6mm}\includegraphics[width=0.35\columnwidth]{./figure/LBYL_figure_2_2.pdf}\hspace{6mm}}
    \subfigure[\label{fig:matrix:c}Delivery matrix for one-to-one]{\hspace{9mm}\includegraphics[width=0.35\columnwidth]{./figure/LBYL_figure_2_3.pdf}\hspace{9mm}} 
    \caption{Comparison between pruning matrix and delivery matrix, where the $4$-th and $6$-th filters are being pruned among $6$ original filters}
	\label{fig:matrix}
	\vspace{-2mm}
\end{figure*}



\subsection{Filter Pruning in a CNN}
Consider a given CNN to be pruned with $L$ layers, where each $\ell$-th layer starts with a convolution operation on its input channels, which are the output of the previous $(\ell-1)$-th layer $\mathbf{A}^{(\ell-1)}$, with the group of convolution filters $\mathbf{W}^{{(\ell)}}$ and thereby obtain the set of \textit{feature maps} $\mathbf{Z}^{(\ell)}$ as follows:
\begin{equation}
\boldsymbol{\mathbf{Z}}^{(\ell)} = {\mathbf{A}^{(\ell-1)} \circledast {\mathbf{W}}^{(\ell)}},
\nonumber
\end{equation}
where $\circledast$ represents the convolution operation. Then, this convolution process is normally followed by a batch normalization (BN) process and an activation function such as ReLU, and the $\ell$-th layer finally outputs an \textit{activation map} $\mathbf{A}^{(\ell)}$ to be sent to the $(\ell+1)$-th layer through this sequence of procedures as:
\begin{equation}
\mathbf{A}^{(\ell)} = \F(\N(\mathbf{Z}^{(\ell)})),
\nonumber
\end{equation}
where $\F(\cdot)$ is an activation function and $\N(\cdot)$ is a BN procedure.

Note that all of $\mathbf{W}^{(\ell)}$, $\mathbf{Z}^{(\ell)}$, and $\mathbf{A}^{(\ell)}$ are tensors such that: $\mathbf{W}^{(\ell)} \in \mathbb{R}^{m \times n \times k \times k}$ and $\mathbf{Z}^{(\ell)},\mathbf{A}^{(\ell)} \in \mathbb{R}^{m \times w \times h}$, where (1) $m$ is the number of filters, which also equals the number of output activation maps, (2) $n$ is the number of input activation maps resulting from the $(\ell-1)$-th layer, (3) $k \times k$ is the size of each filter, and (4) $w \times h$ is the size of each output channel for the $\ell$-th layer.

\smalltitle{Filter pruning as n-mode product}
When filter pruning is performed at the $\ell$-th layer, all three tensors above are consequently modified to their \textit{damaged} versions, namely $\mathbf{\Tilde{W}}^{(\ell)}$, $\mathbf{\Tilde{Z}}^{(\ell)}$, and $\mathbf{\Tilde{A}}^{(\ell)}$, respectively, in a way that: $\mathbf{\Tilde{W}}^{(\ell)} \in \mathbb{R}^{t \times n \times k \times k}$ and $\mathbf{\Tilde{Z}}^{(\ell)},\mathbf{\Tilde{A}}^{(\ell)} \in \mathbb{R}^{t \times w \times h}$, where $t$ is the number of remaining filters after pruning and therefore $t < m$. Mathematically, the tensor of remaining filters, \textit{i.e.}, $\mathbf{\Tilde{W}}^{(\ell)}$, is obtained by the \textit{$1$-mode product} \cite{DBLP:journals/siamrev/KoldaB09} of the tensor of the original filters $\mathbf{W}^{(\ell)}$ with a \textit{pruning matrix} $\boldsymbol{\S} \in \mathbb{R}^{m \times t}$ (see Figure \ref{fig:matrix:a})
as follows:
\begin{eqnarray}\begin{split}\label{eq:pruning}
\mathbf{\Tilde{W}}^{(\ell)} = {\mathbf{W}}^{(\ell)} \times_{1} {\boldsymbol{\S}}^{T},\text{where }\boldsymbol{\S}_{i,k} = 
  \begin{cases} 
   1~ \text{if } i = i'_k \\
   0~ \text{otherwise}
  \end{cases} \\
  \text{s.t. } i, i'_k \in [1, m] 
  \text{ and } k \in [1, t].
  \end{split}
\end{eqnarray}
  
By Eq. (\ref{eq:pruning}), each $i'_k$-th filter is not pruned and the other $(m-t)$ filters are completely removed from $\mathbf{W}^{(\ell)}$ to be $\mathbf{\Tilde{W}}^{(\ell)}$.

This reduction at the $\ell$-th layer causes another reduction for each filter of the $(\ell+1)$-th layer so that $\mathbf{W}^{(\ell+1)}$ is now modified to $\mathbf{\Tilde{W}}^{(\ell+1)} \in \mathbb{R}^{m' \times t \times k' \times k'}$, where $m'$ is the number of filters of size $k' \times k'$ in the $(\ell+1)$-th layer. Due to this series of information losses, the resulting feature map (\textit{i.e.}, $\mathbf{Z}^{(\ell+1)}$) would severely be damaged to be $\mathbf{\Tilde{Z}}^{(\ell+1)}$ as shown below:
\begin{equation}
{\mathbf{\Tilde{Z}}}^{{(\ell+1)}} = \mathbf{\Tilde{A}}^{(\ell)} \circledast {\mathbf{\Tilde{W}}}^{(\ell+1)}~~~\not\approx~~~\mathbf{Z}^{(\ell+1)}
\label{eq:eq}\nonumber
\end{equation}
The shape of $\mathbf{\Tilde{Z}}^{(\ell+1)}$ remains the same unless we also prune filters for the $(\ell+1)$-th layer. If we do so as well, the loss of information will be accumulated and further propagated to the next layers. Note that $\mathbf{\Tilde{W}}^{(\ell+1)}$ can also be represented by the \textit{$2$-mode product} \cite{DBLP:journals/siamrev/KoldaB09} of $\mathbf{W}^{(\ell+1)}$ with the transpose of the same matrix $\boldsymbol{\S}$ as:
\begin{equation} \label{eq:pruning2}
\mathbf{\Tilde{W}}^{(\ell+1)} = {\mathbf{W}}^{(\ell+1)} \times_{2} {\boldsymbol{\S}^T}
\end{equation}




\subsection{Problem of Restoring a Pruned Network without Data and Fine-Tuning}
As mentioned earlier, our goal is to restore a pruned and thus damaged CNN without using any data and re-training process, which implies the following two facts. First, we have to use a pruning criterion exploiting only the values of filters themselves such as L1-norm. In this sense, this paper does not focus on proposing a sophisticated pruning criterion but intends to recover a network somehow pruned by such a simple criterion. Secondly, since we cannot make appropriate changes in the remaining filters by fine-tuning, we should make the best use of the original network and identify how the information carried by a pruned filter can be delivered to the remaining filters.

% For brevity, we formulate our problem here with respect to a specific layer, say $\ell$, and then it can trivially be generalized for the entire network. 
\smalltitle{Delivery matrix}
In order to represent the information to be delivered to the preserved filters, let us first think of what the pruning matrix $\boldsymbol{\S}$ means. As defined in Eq. (\ref{eq:pruning}) and shown in Figure \ref{fig:matrix:a}, each row is either a zero vector (for filters being pruned) or a one-hot vector (for remaining filters), which is intended only to remove filters without delivering any information. Intuitively, we can transform this pruning matrix into a \textit{delivery matrix} that carries information for filters being pruned by replacing some meaningful values with some of the zero values therein. Once we find such an \textit{ideal} $\boldsymbol{\S^*}$, we can plug it into $\boldsymbol{\S}$ of Eq. (\ref{eq:pruning2}) to deliver missing information propagated from the $\ell$-th layer to the filters at the $(\ell+1)$-th layer, which will hopefully generate an approximation $\mathbf{\hat{Z}}^{(\ell+1)}$ close to the original feature map as follows:
\begin{equation} \label{eq:fmap_approx}
{\mathbf{\hat{Z}}}^{{(\ell+1)}} = {\mathbf{\Tilde{A}}^{(\ell)} \circledast ({\mathbf{W}}^{(\ell+1)} \times_{2} {\boldsymbol{\S^*}^T})}
~~~\approx~~~\mathbf{Z}^{(\ell+1)}
\end{equation}
Thus, using the delivery matrix $\boldsymbol{\mathcal{S^*}}$, the information loss caused by pruning at each layer is recovered at the feature map of the next layer.

\smalltitle{Problem statement}
Given a pretrained CNN, our problem aims to find the best delivery matrix $\boldsymbol{\mathcal{S^*}}$ for each layer without any data and training process such that the following \textit{reconstruction error} is minimized:
\begin{equation}
\sum\limits_{i = 1}^{m'}\|{{\mathbf{Z}}_{i}^{{(\ell+1)}}-{\hat{\mathbf{Z}}}_{i}^{{(\ell+1)}}}\|_1,
\label{eq:goal}
\end{equation}
where ${\mathbf{Z}}_i^{{(\ell+1)}}$ and ${\hat{\mathbf{Z}}}_i^{{(\ell+1)}}$ indicate the $i$-th original feature map and its corresponding approximation, respectively, out of $m'$ filters in the $(\ell+1)$-th layer. Note that what is challenging here is that we cannot obtain the activation maps in $\mathbf{A}^{(\ell)}$ and $\mathbf{\Tilde{A}}^{(\ell)}$ without data as they are data-dependent values.

% = \sum\limits_{i = 1}^{m'}\|{{\mathbf{Z}}_{i}^{{(\ell+1)}}-{\mathbf{\Tilde{A}}^{(\ell)} \circledast ({\mathbf{W}}^{(\ell+1)} \times_{2} {\boldsymbol{\mathcal{S^*}^T}})}}\|_{1}


% Our goal is finding the approximation matrix $\boldsymbol{\mathcal{S}}$ to minimize the reconstruction error between the pruned model and the original model without any data, and effectively deliver missing information for pruned filters using this approximation matrix


% $\testit{s}$,which can be represented as below.

% \begin{equation}
% \boldsymbol{\mathcal{S}} =  \underset{{\boldsymbol{\mathcal{S}}}}{\mathrm{argmin}} \sum\limits_{{i} = 1}^{m_{\ell+1}} \|{{\mathbf{Z}}_{i,:,:}^{{(\ell+1)}}-{\hat{\mathbf{Z}}}_{i,:,:}^{{(\ell+1)}}}\|_{1} 
% \label{eq:eq1}
% \end{equation}



% Let us first recall that the ultimate goal of network pruning is to make the output of a pruned network as close as possible to that of its original network. Unlike many existing pruning methods, our focus is not to use any training data at all for the entire pruning and recovery process, and this implies the following two facts. First, we cannot evaluate the filter importance by data-dependent values like activation values or gradients, but have to use a pruning criterion exploiting only the values of filters themselves such as L1-norm. Furthermore, instead of fine-tuning with data, the only thing we can do for the pruned network is to make appropriate changes in the remaining filters by identifying some relationships between pruned filters and the other preserved ones without any support from data. Based on this intuition, this section mathematically and generally defines the problem of restoring a pruned neural network in a manner free of data and fine-tuning.


% Thus, we make approximation matrix $\testit{s}$ $\in$ $\mathbb{R}^{m_{\ell} \times t_{\ell}}$ with relationship between the pruned filter and preserved filters in $\ell$-th layer and then apply it to the original filters in $(\ell+1)$-th layer to compensate for pruned feature maps $\boldsymbol{\hat{\mathbf{Z}}}^{{(\ell+1)}}$ as shown below.
% (\textit{i.e.}, Let $\hat{\mathbf{W}}^{(\ell+1)}$ be ${\mathbf{W}}^{(\ell+1)}$ $\times_2$ ${{\textit{s}}} $, where $\times_2$ is 2-mode matrix product) 

% \begin{equation}
% \mathbf{Z}^{(\ell+1)} = {\mathbf{A}}^{(\ell)} \circledast {\mathbf{W}}^{(\ell+1)}
% \approx {\hat{\mathbf{A}}^{(\ell)} \circledast ({\mathbf{W}}^{(\ell+1)} \times_{2} {{s}}) = {\hat{\mathbf{Z}}}^{{(\ell+1)}}}
% \label{eq:eq}\nonumber
% \end{equation}




% For a Convolutional Neural Network (CNN) with $L$ layers, we denote $\mathcal{A}{^{(\ell-1)}}$ $\in$ $\mathbb{R}^{n_{\ell -1 } \times h_{\ell -1} \times w_{\ell -1}}$ is activation maps at $\ell-1$-th layer, where $n_{\ell -1}$, $h_{\ell -1}$ and $w_{\ell -1}$ are the number of channels, height and width in activation maps, respectively. and we denote $\mathbf{W}^{{(\ell )}}$ $\in$  $\mathbb{R}^{m_{\ell} \times n_{\ell -1}\times k \times k}$ is covolution filters in $\ell$-th layer,where $m_{\ell}$, $n_{\ell-1}$ and $k$ are the number of filters, number of channels and kernel size, respectively. Trough the convolution operation using activation map $\mathcal{A}{^{(\ell-1)}}$ and convolution filter $\mathbf{W}^{{(\ell)}}$ in $\ell$-th layer, the feature maps $\boldsymbol{\mathbf{Z}}^{{(\ell)}}$ $\in$ $\mathbb{R}^{m_{\ell} \times h_{\ell+1} \times w_{\ell+1}}$ is computed as shown as below.


% \begin{equation}
% \boldsymbol{\mathbf{Z}}^{(\ell)} = {\mathcal{A}^{(\ell-1)} \circledast {\mathbf{W}}^{(\ell)}}
% \label{eq:eq1}\nonumber
% \end{equation}
% where $\circledast$ is convolution operation.

% and the feature maps passed through the BN and ReLU layer are activation maps $\mathcal{A}{^{(\ell)}}$ $\in$ $\mathbb{R}^{m_{{\ell}} \times h_{\ell+1} \times w_{\ell+1}} $ in $\ell$-th layer as shown as below.

% \begin{equation}
% \mathcal{A}^{(\ell)} = \mathcal{F}(\mathbf{Z}^{(\ell)} \circledast {\mathbf{W}}^{(\ell)})
% \label{eq:eq2}\nonumber
% \end{equation}
% where $\mathcal{F}$ is the function that implement batch normalization and non-linear activation(\textit{e.g.}, ReLU).

% \smalltitle{Filter Pruning}
% If the filter pruning is performed in $\ell$-th layer, the shape of original filters $\mathbf{W}^{{(\ell)}}$ $\in$ $\mathbb{R}^{m_{\ell} \times n_{\ell-1}\times k \times k}$ is modified to ${\hat {\mathbf{W}}^{(\ell)}}$ $\in$ $\mathbb{R}^{t_{\ell} \times n_{\ell-1}\times k \times k}$, where $t_{\ell}$ $<$ $m_{\ell}$ by pruning criterion. Therefore, the pruned activation maps ${\hat {\mathcal{A}}}{^{({\ell+1})}}$ $\in$ $\mathbb{R}^{t_{{\ell}} \times h_{{\ell+2}} \times w_{{\ell+2}}}$ in (${\ell+1}$)-th layer is computed as below.

% \begin{equation}
% \mathbf{\hat{A}}^{(l+1)} = \mathcal{F}({\mathbf{A}^{(\ell)} \circledast {\mathbf{\hat{W}}}^{(\ell+1)}})
% \label{eq:eq3}\nonumber
% \end{equation}

% Moreover, corresponding channels of each filters in ($\ell +1$)-th layer are sequentially removed. As a result, shape of original filters $\mathbf{W}^{{(\ell+1)}}$ $\in$ $\mathbb{R}^{m_{\ell+1} \times m_{\ell}\times k \times k}$ in ($\ell+1$)-th layer is changed to  ${\hat {\mathbf{W}}^{(\ell+1)}}$ $\in$ $\mathbb{R}^{m_{\ell+1} \times t_{\ell}\times k \times k}$. Although feature maps ${\hat{\mathbf{Z}}}^{{(\ell+1)}}$ $\in$ $\mathbb{R}^{m_{\ell+1} \times h_{\ell+2} \times w_{\ell+2}}$ in ($\ell+1$)-th layer after pruning have same shape with original feature maps ${\mathbf{Z}}^{{(\ell+1)}}$ $\in$ $\mathbb{R}^{m_{\ell+1} \times h_{\ell+2} \times w_{\ell+2}}$, the pruned feature maps $\boldsymbol{\hat{\mathbf{Z}}}^{{(\ell+1)}}$ are damaged.
\begin{figure*}[ht]
    \centering
    \includegraphics[width=\textwidth, trim=79 280 93 123, clip]{figures/framework_img.pdf}
    \caption{The pipeline of the \ENDow{} framework 
    %where each component is specified in a given configuration. 
    which yields a downstream task score and a WER score of the transcript set input to the task. The pipeline is executed for several severeties of noising and types of cleaning techniques. %Acoustic noising is applied at $k$ intensities, providing $k+1$ audio versions (including the non-noised version), eventually producing $k+2$ transcript versions (including the source transcript). Applying transcript cleaning reveals the effect of \textit{types} of noise. 
    Resulting scores are plotted on a graph for the analyses, as in, e.g., \autoref{fig_cleaning_graphs}.}
    %The pipeline is executed on $k+1$ intensities of acoustic noising (including the non-noised version), producing $k+2$ scores for the downstream task (including execution on the source transcripts). This process eventually describes the effect of the \textit{intensity} of transcript noise on the downstream task. The process is repeated for $m$ cleaning techniques ($m+1$ when including no cleaning), to analyze the benefit of a cleaning approach and the effect of the \textit{types} of transcript noise.}
    \label{fig_framework}
\end{figure*}

\begin{table*}[t]
\centering
\fontsize{11pt}{11pt}\selectfont
\begin{tabular}{lllllllllllll}
\toprule
\multicolumn{1}{c}{\textbf{task}} & \multicolumn{2}{c}{\textbf{Mir}} & \multicolumn{2}{c}{\textbf{Lai}} & \multicolumn{2}{c}{\textbf{Ziegen.}} & \multicolumn{2}{c}{\textbf{Cao}} & \multicolumn{2}{c}{\textbf{Alva-Man.}} & \multicolumn{1}{c}{\textbf{avg.}} & \textbf{\begin{tabular}[c]{@{}l@{}}avg.\\ rank\end{tabular}} \\
\multicolumn{1}{c}{\textbf{metrics}} & \multicolumn{1}{c}{\textbf{cor.}} & \multicolumn{1}{c}{\textbf{p-v.}} & \multicolumn{1}{c}{\textbf{cor.}} & \multicolumn{1}{c}{\textbf{p-v.}} & \multicolumn{1}{c}{\textbf{cor.}} & \multicolumn{1}{c}{\textbf{p-v.}} & \multicolumn{1}{c}{\textbf{cor.}} & \multicolumn{1}{c}{\textbf{p-v.}} & \multicolumn{1}{c}{\textbf{cor.}} & \multicolumn{1}{c}{\textbf{p-v.}} &  &  \\ \midrule
\textbf{S-Bleu} & 0.50 & 0.0 & 0.47 & 0.0 & 0.59 & 0.0 & 0.58 & 0.0 & 0.68 & 0.0 & 0.57 & 5.8 \\
\textbf{R-Bleu} & -- & -- & 0.27 & 0.0 & 0.30 & 0.0 & -- & -- & -- & -- & - &  \\
\textbf{S-Meteor} & 0.49 & 0.0 & 0.48 & 0.0 & 0.61 & 0.0 & 0.57 & 0.0 & 0.64 & 0.0 & 0.56 & 6.1 \\
\textbf{R-Meteor} & -- & -- & 0.34 & 0.0 & 0.26 & 0.0 & -- & -- & -- & -- & - &  \\
\textbf{S-Bertscore} & \textbf{0.53} & 0.0 & {\ul 0.80} & 0.0 & \textbf{0.70} & 0.0 & {\ul 0.66} & 0.0 & {\ul0.78} & 0.0 & \textbf{0.69} & \textbf{1.7} \\
\textbf{R-Bertscore} & -- & -- & 0.51 & 0.0 & 0.38 & 0.0 & -- & -- & -- & -- & - &  \\
\textbf{S-Bleurt} & {\ul 0.52} & 0.0 & {\ul 0.80} & 0.0 & 0.60 & 0.0 & \textbf{0.70} & 0.0 & \textbf{0.80} & 0.0 & {\ul 0.68} & {\ul 2.3} \\
\textbf{R-Bleurt} & -- & -- & 0.59 & 0.0 & -0.05 & 0.13 & -- & -- & -- & -- & - &  \\
\textbf{S-Cosine} & 0.51 & 0.0 & 0.69 & 0.0 & {\ul 0.62} & 0.0 & 0.61 & 0.0 & 0.65 & 0.0 & 0.62 & 4.4 \\
\textbf{R-Cosine} & -- & -- & 0.40 & 0.0 & 0.29 & 0.0 & -- & -- & -- & -- & - & \\ \midrule
\textbf{QuestEval} & 0.23 & 0.0 & 0.25 & 0.0 & 0.49 & 0.0 & 0.47 & 0.0 & 0.62 & 0.0 & 0.41 & 9.0 \\
\textbf{LLaMa3} & 0.36 & 0.0 & \textbf{0.84} & 0.0 & {\ul{0.62}} & 0.0 & 0.61 & 0.0 &  0.76 & 0.0 & 0.64 & 3.6 \\
\textbf{our (3b)} & 0.49 & 0.0 & 0.73 & 0.0 & 0.54 & 0.0 & 0.53 & 0.0 & 0.7 & 0.0 & 0.60 & 5.8 \\
\textbf{our (8b)} & 0.48 & 0.0 & 0.73 & 0.0 & 0.52 & 0.0 & 0.53 & 0.0 & 0.7 & 0.0 & 0.59 & 6.3 \\  \bottomrule
\end{tabular}
\caption{Pearson correlation on human evaluation on system output. `R-': reference-based. `S-': source-based.}
\label{tab:sys}
\end{table*}



\begin{table}%[]
\centering
\fontsize{11pt}{11pt}\selectfont
\begin{tabular}{llllll}
\toprule
\multicolumn{1}{c}{\textbf{task}} & \multicolumn{1}{c}{\textbf{Lai}} & \multicolumn{1}{c}{\textbf{Zei.}} & \multicolumn{1}{c}{\textbf{Scia.}} & \textbf{} & \textbf{} \\ 
\multicolumn{1}{c}{\textbf{metrics}} & \multicolumn{1}{c}{\textbf{cor.}} & \multicolumn{1}{c}{\textbf{cor.}} & \multicolumn{1}{c}{\textbf{cor.}} & \textbf{avg.} & \textbf{\begin{tabular}[c]{@{}l@{}}avg.\\ rank\end{tabular}} \\ \midrule
\textbf{S-Bleu} & 0.40 & 0.40 & 0.19* & 0.33 & 7.67 \\
\textbf{S-Meteor} & 0.41 & 0.42 & 0.16* & 0.33 & 7.33 \\
\textbf{S-BertS.} & {\ul0.58} & 0.47 & 0.31 & 0.45 & 3.67 \\
\textbf{S-Bleurt} & 0.45 & {\ul 0.54} & {\ul 0.37} & 0.45 & {\ul 3.33} \\
\textbf{S-Cosine} & 0.56 & 0.52 & 0.3 & {\ul 0.46} & {\ul 3.33} \\ \midrule
\textbf{QuestE.} & 0.27 & 0.35 & 0.06* & 0.23 & 9.00 \\
\textbf{LlaMA3} & \textbf{0.6} & \textbf{0.67} & \textbf{0.51} & \textbf{0.59} & \textbf{1.0} \\
\textbf{Our (3b)} & 0.51 & 0.49 & 0.23* & 0.39 & 4.83 \\
\textbf{Our (8b)} & 0.52 & 0.49 & 0.22* & 0.43 & 4.83 \\ \bottomrule
\end{tabular}
\caption{Pearson correlation on human ratings on reference output. *not significant; we cannot reject the null hypothesis of zero correlation}
\label{tab:ref}
\end{table}


\begin{table*}%[]
\centering
\fontsize{11pt}{11pt}\selectfont
\begin{tabular}{lllllllll}
\toprule
\textbf{task} & \multicolumn{1}{c}{\textbf{ALL}} & \multicolumn{1}{c}{\textbf{sentiment}} & \multicolumn{1}{c}{\textbf{detoxify}} & \multicolumn{1}{c}{\textbf{catchy}} & \multicolumn{1}{c}{\textbf{polite}} & \multicolumn{1}{c}{\textbf{persuasive}} & \multicolumn{1}{c}{\textbf{formal}} & \textbf{\begin{tabular}[c]{@{}l@{}}avg. \\ rank\end{tabular}} \\
\textbf{metrics} & \multicolumn{1}{c}{\textbf{cor.}} & \multicolumn{1}{c}{\textbf{cor.}} & \multicolumn{1}{c}{\textbf{cor.}} & \multicolumn{1}{c}{\textbf{cor.}} & \multicolumn{1}{c}{\textbf{cor.}} & \multicolumn{1}{c}{\textbf{cor.}} & \multicolumn{1}{c}{\textbf{cor.}} &  \\ \midrule
\textbf{S-Bleu} & -0.17 & -0.82 & -0.45 & -0.12* & -0.1* & -0.05 & -0.21 & 8.42 \\
\textbf{R-Bleu} & - & -0.5 & -0.45 &  &  &  &  &  \\
\textbf{S-Meteor} & -0.07* & -0.55 & -0.4 & -0.01* & 0.1* & -0.16 & -0.04* & 7.67 \\
\textbf{R-Meteor} & - & -0.17* & -0.39 & - & - & - & - & - \\
\textbf{S-BertScore} & 0.11 & -0.38 & -0.07* & -0.17* & 0.28 & 0.12 & 0.25 & 6.0 \\
\textbf{R-BertScore} & - & -0.02* & -0.21* & - & - & - & - & - \\
\textbf{S-Bleurt} & 0.29 & 0.05* & 0.45 & 0.06* & 0.29 & 0.23 & 0.46 & 4.2 \\
\textbf{R-Bleurt} & - &  0.21 & 0.38 & - & - & - & - & - \\
\textbf{S-Cosine} & 0.01* & -0.5 & -0.13* & -0.19* & 0.05* & -0.05* & 0.15* & 7.42 \\
\textbf{R-Cosine} & - & -0.11* & -0.16* & - & - & - & - & - \\ \midrule
\textbf{QuestEval} & 0.21 & {\ul{0.29}} & 0.23 & 0.37 & 0.19* & 0.35 & 0.14* & 4.67 \\
\textbf{LlaMA3} & \textbf{0.82} & \textbf{0.80} & \textbf{0.72} & \textbf{0.84} & \textbf{0.84} & \textbf{0.90} & \textbf{0.88} & \textbf{1.00} \\
\textbf{Our (3b)} & 0.47 & -0.11* & 0.37 & 0.61 & 0.53 & 0.54 & 0.66 & 3.5 \\
\textbf{Our (8b)} & {\ul{0.57}} & 0.09* & {\ul 0.49} & {\ul 0.72} & {\ul 0.64} & {\ul 0.62} & {\ul 0.67} & {\ul 2.17} \\ \bottomrule
\end{tabular}
\caption{Pearson correlation on human ratings on our constructed test set. 'R-': reference-based. 'S-': source-based. *not significant; we cannot reject the null hypothesis of zero correlation}
\label{tab:con}
\end{table*}

\section{Results}
We benchmark the different metrics on the different datasets using correlation to human judgement. For content preservation, we show results split on data with system output, reference output and our constructed test set: we show that the data source for evaluation leads to different conclusions on the metrics. In addition, we examine whether the metrics can rank style transfer systems similar to humans. On style strength, we likewise show correlations between human judgment and zero-shot evaluation approaches. When applicable, we summarize results by reporting the average correlation. And the average ranking of the metric per dataset (by ranking which metric obtains the highest correlation to human judgement per dataset). 

\subsection{Content preservation}
\paragraph{How do data sources affect the conclusion on best metric?}
The conclusions about the metrics' performance change radically depending on whether we use system output data, reference output, or our constructed test set. Ideally, a good metric correlates highly with humans on any data source. Ideally, for meta-evaluation, a metric should correlate consistently across all data sources, but the following shows that the correlations indicate different things, and the conclusion on the best metric should be drawn carefully.

Looking at the metrics correlations with humans on the data source with system output (Table~\ref{tab:sys}), we see a relatively high correlation for many of the metrics on many tasks. The overall best metrics are S-BertScore and S-BLEURT (avg+avg rank). We see no notable difference in our method of using the 3B or 8B model as the backbone.

Examining the average correlations based on data with reference output (Table~\ref{tab:ref}), now the zero-shoot prompting with LlaMA3 70B is the best-performing approach ($0.59$ avg). Tied for second place are source-based cosine embedding ($0.46$ avg), BLEURT ($0.45$ avg) and BertScore ($0.45$ avg). Our method follows on a 5. place: here, the 8b version (($0.43$ avg)) shows a bit stronger results than 3b ($0.39$ avg). The fact that the conclusions change, whether looking at reference or system output, confirms the observations made by \citet{scialom-etal-2021-questeval} on simplicity transfer.   

Now consider the results on our test set (Table~\ref{tab:con}): Several metrics show low or no correlation; we even see a significantly negative correlation for some metrics on ALL (BLEU) and for specific subparts of our test set for BLEU, Meteor, BertScore, Cosine. On the other end, LlaMA3 70B is again performing best, showing strong results ($0.82$ in ALL). The runner-up is now our 8B method, with a gap to the 3B version ($0.57$ vs $0.47$ in ALL). Note our method still shows zero correlation for the sentiment task. After, ranks BLEURT ($0.29$), QuestEval ($0.21$), BertScore ($0.11$), Cosine ($0.01$).  

On our test set, we find that some metrics that correlate relatively well on the other datasets, now exhibit low correlation. Hence, with our test set, we can now support the logical reasoning with data evidence: Evaluation of content preservation for style transfer needs to take the style shift into account. This conclusion could not be drawn using the existing data sources: We hypothesise that for the data with system-based output, successful output happens to be very similar to the source sentence and vice versa, and reference-based output might not contain server mistakes as they are gold references. Thus, none of the existing data sources tests the limits of the metrics.  


\paragraph{How do reference-based metrics compare to source-based ones?} Reference-based metrics show a lower correlation than the source-based counterpart for all metrics on both datasets with ratings on references (Table~\ref{tab:sys}). As discussed previously, reference-based metrics for style transfer have the drawback that many different good solutions on a rewrite might exist and not only one similar to a reference.


\paragraph{How well can the metrics rank the performance of style transfer methods?}
We compare the metrics' ability to judge the best style transfer methods w.r.t. the human annotations: Several of the data sources contain samples from different style transfer systems. In order to use metrics to assess the quality of the style transfer system, metrics should correctly find the best-performing system. Hence, we evaluate whether the metrics for content preservation provide the same system ranking as human evaluators. We take the mean of the score for every output on each system and the mean of the human annotations; we compare the systems using the Kendall's Tau correlation. 

We find only the evaluation using the dataset Mir, Lai, and Ziegen to result in significant correlations, probably because of sparsity in a number of system tests (App.~\ref{app:dataset}). Our method (8b) is the only metric providing a perfect ranking of the style transfer system on the Lai data, and Llama3 70B the only one on the Ziegen data. Results in App.~\ref{app:results}. 


\subsection{Style strength results}
%Evaluating style strengths is a challenging task. 
Llama3 70B shows better overall results than our method. However, our method scores higher than Llama3 70B on 2 out of 6 datasets, but it also exhibits zero correlation on one task (Table~\ref{tab:styleresults}).%More work i s needed on evaluating style strengths. 
 
\begin{table}%[]
\fontsize{11pt}{11pt}\selectfont
\begin{tabular}{lccc}
\toprule
\multicolumn{1}{c}{\textbf{}} & \textbf{LlaMA3} & \textbf{Our (3b)} & \textbf{Our (8b)} \\ \midrule
\textbf{Mir} & 0.46 & 0.54 & \textbf{0.57} \\
\textbf{Lai} & \textbf{0.57} & 0.18 & 0.19 \\
\textbf{Ziegen.} & 0.25 & 0.27 & \textbf{0.32} \\
\textbf{Alva-M.} & \textbf{0.59} & 0.03* & 0.02* \\
\textbf{Scialom} & \textbf{0.62} & 0.45 & 0.44 \\
\textbf{\begin{tabular}[c]{@{}l@{}}Our Test\end{tabular}} & \textbf{0.63} & 0.46 & 0.48 \\ \bottomrule
\end{tabular}
\caption{Style strength: Pearson correlation to human ratings. *not significant; we cannot reject the null hypothesis of zero corelation}
\label{tab:styleresults}
\end{table}

\subsection{Ablation}
We conduct several runs of the methods using LLMs with variations in instructions/prompts (App.~\ref{app:method}). We observe that the lower the correlation on a task, the higher the variation between the different runs. For our method, we only observe low variance between the runs.
None of the variations leads to different conclusions of the meta-evaluation. Results in App.~\ref{app:results}.
%\section{Discussion of Assumptions}\label{sec:discussion}
In this paper, we have made several assumptions for the sake of clarity and simplicity. In this section, we discuss the rationale behind these assumptions, the extent to which these assumptions hold in practice, and the consequences for our protocol when these assumptions hold.

\subsection{Assumptions on the Demand}

There are two simplifying assumptions we make about the demand. First, we assume the demand at any time is relatively small compared to the channel capacities. Second, we take the demand to be constant over time. We elaborate upon both these points below.

\paragraph{Small demands} The assumption that demands are small relative to channel capacities is made precise in \eqref{eq:large_capacity_assumption}. This assumption simplifies two major aspects of our protocol. First, it largely removes congestion from consideration. In \eqref{eq:primal_problem}, there is no constraint ensuring that total flow in both directions stays below capacity--this is always met. Consequently, there is no Lagrange multiplier for congestion and no congestion pricing; only imbalance penalties apply. In contrast, protocols in \cite{sivaraman2020high, varma2021throughput, wang2024fence} include congestion fees due to explicit congestion constraints. Second, the bound \eqref{eq:large_capacity_assumption} ensures that as long as channels remain balanced, the network can always meet demand, no matter how the demand is routed. Since channels can rebalance when necessary, they never drop transactions. This allows prices and flows to adjust as per the equations in \eqref{eq:algorithm}, which makes it easier to prove the protocol's convergence guarantees. This also preserves the key property that a channel's price remains proportional to net money flow through it.

In practice, payment channel networks are used most often for micro-payments, for which on-chain transactions are prohibitively expensive; large transactions typically take place directly on the blockchain. For example, according to \cite{river2023lightning}, the average channel capacity is roughly $0.1$ BTC ($5,000$ BTC distributed over $50,000$ channels), while the average transaction amount is less than $0.0004$ BTC ($44.7k$ satoshis). Thus, the small demand assumption is not too unrealistic. Additionally, the occasional large transaction can be treated as a sequence of smaller transactions by breaking it into packets and executing each packet serially (as done by \cite{sivaraman2020high}).
Lastly, a good path discovery process that favors large capacity channels over small capacity ones can help ensure that the bound in \eqref{eq:large_capacity_assumption} holds.

\paragraph{Constant demands} 
In this work, we assume that any transacting pair of nodes have a steady transaction demand between them (see Section \ref{sec:transaction_requests}). Making this assumption is necessary to obtain the kind of guarantees that we have presented in this paper. Unless the demand is steady, it is unreasonable to expect that the flows converge to a steady value. Weaker assumptions on the demand lead to weaker guarantees. For example, with the more general setting of stochastic, but i.i.d. demand between any two nodes, \cite{varma2021throughput} shows that the channel queue lengths are bounded in expectation. If the demand can be arbitrary, then it is very hard to get any meaningful performance guarantees; \cite{wang2024fence} shows that even for a single bidirectional channel, the competitive ratio is infinite. Indeed, because a PCN is a decentralized system and decisions must be made based on local information alone, it is difficult for the network to find the optimal detailed balance flow at every time step with a time-varying demand.  With a steady demand, the network can discover the optimal flows in a reasonably short time, as our work shows.

We view the constant demand assumption as an approximation for a more general demand process that could be piece-wise constant, stochastic, or both (see simulations in Figure \ref{fig:five_nodes_variable_demand}).
We believe it should be possible to merge ideas from our work and \cite{varma2021throughput} to provide guarantees in a setting with random demands with arbitrary means. We leave this for future work. In addition, our work suggests that a reasonable method of handling stochastic demands is to queue the transaction requests \textit{at the source node} itself. This queuing action should be viewed in conjunction with flow-control. Indeed, a temporarily high unidirectional demand would raise prices for the sender, incentivizing the sender to stop sending the transactions. If the sender queues the transactions, they can send them later when prices drop. This form of queuing does not require any overhaul of the basic PCN infrastructure and is therefore simpler to implement than per-channel queues as suggested by \cite{sivaraman2020high} and \cite{varma2021throughput}.

\subsection{The Incentive of Channels}
The actions of the channels as prescribed by the DEBT control protocol can be summarized as follows. Channels adjust their prices in proportion to the net flow through them. They rebalance themselves whenever necessary and execute any transaction request that has been made of them. We discuss both these aspects below.

\paragraph{On Prices}
In this work, the exclusive role of channel prices is to ensure that the flows through each channel remains balanced. In practice, it would be important to include other components in a channel's price/fee as well: a congestion price  and an incentive price. The congestion price, as suggested by \cite{varma2021throughput}, would depend on the total flow of transactions through the channel, and would incentivize nodes to balance the load over different paths. The incentive price, which is commonly used in practice \cite{river2023lightning}, is necessary to provide channels with an incentive to serve as an intermediary for different channels. In practice, we expect both these components to be smaller than the imbalance price. Consequently, we expect the behavior of our protocol to be similar to our theoretical results even with these additional prices.

A key aspect of our protocol is that channel fees are allowed to be negative. Although the original Lightning network whitepaper \cite{poon2016bitcoin} suggests that negative channel prices may be a good solution to promote rebalancing, the idea of negative prices in not very popular in the literature. To our knowledge, the only prior work with this feature is \cite{varma2021throughput}. Indeed, in papers such as \cite{van2021merchant} and \cite{wang2024fence}, the price function is explicitly modified such that the channel price is never negative. The results of our paper show the benefits of negative prices. For one, in steady state, equal flows in both directions ensure that a channel doesn't loose any money (the other price components mentioned above ensure that the channel will only gain money). More importantly, negative prices are important to ensure that the protocol selectively stifles acyclic flows while allowing circulations to flow. Indeed, in the example of Section \ref{sec:flow_control_example}, the flows between nodes $A$ and $C$ are left on only because the large positive price over one channel is canceled by the corresponding negative price over the other channel, leading to a net zero price.

Lastly, observe that in the DEBT control protocol, the price charged by a channel does not depend on its capacity. This is a natural consequence of the price being the Lagrange multiplier for the net-zero flow constraint, which also does not depend on the channel capacity. In contrast, in many other works, the imbalance price is normalized by the channel capacity \cite{ren2018optimal, lin2020funds, wang2024fence}; this is shown to work well in practice. The rationale for such a price structure is explained well in \cite{wang2024fence}, where this fee is derived with the aim of always maintaining some balance (liquidity) at each end of every channel. This is a reasonable aim if a channel is to never rebalance itself; the experiments of the aforementioned papers are conducted in such a regime. In this work, however, we allow the channels to rebalance themselves a few times in order to settle on a detailed balance flow. This is because our focus is on the long-term steady state performance of the protocol. This difference in perspective also shows up in how the price depends on the channel imbalance. \cite{lin2020funds} and \cite{wang2024fence} advocate for strictly convex prices whereas this work and \cite{varma2021throughput} propose linear prices.

\paragraph{On Rebalancing} 
Recall that the DEBT control protocol ensures that the flows in the network converge to a detailed balance flow, which can be sustained perpetually without any rebalancing. However, during the transient phase (before convergence), channels may have to perform on-chain rebalancing a few times. Since rebalancing is an expensive operation, it is worthwhile discussing methods by which channels can reduce the extent of rebalancing. One option for the channels to reduce the extent of rebalancing is to increase their capacity; however, this comes at the cost of locking in more capital. Each channel can decide for itself the optimum amount of capital to lock in. Another option, which we discuss in Section \ref{sec:five_node}, is for channels to increase the rate $\gamma$ at which they adjust prices. 

Ultimately, whether or not it is beneficial for a channel to rebalance depends on the time-horizon under consideration. Our protocol is based on the assumption that the demand remains steady for a long period of time. If this is indeed the case, it would be worthwhile for a channel to rebalance itself as it can make up this cost through the incentive fees gained from the flow of transactions through it in steady state. If a channel chooses not to rebalance itself, however, there is a risk of being trapped in a deadlock, which is suboptimal for not only the nodes but also the channel.

\section{Conclusion}
This work presents DEBT control: a protocol for payment channel networks that uses source routing and flow control based on channel prices. The protocol is derived by posing a network utility maximization problem and analyzing its dual minimization. It is shown that under steady demands, the protocol guides the network to an optimal, sustainable point. Simulations show its robustness to demand variations. The work demonstrates that simple protocols with strong theoretical guarantees are possible for PCNs and we hope it inspires further theoretical research in this direction.

%\section{Related Work}
%\label{sec:related-work}

%\subsection{Background}

%Defect detection is critical to ensure the yield of integrated circuit manufacturing lines and reduce faults. Previous research has primarily focused on wafer map data, which engineers produce by marking faulty chips with different colors based on test results. The specific spatial distribution of defects on a wafer can provide insights into the causes, thereby helping to determine which stage of the manufacturing process is responsible for the issues. Although such research is relatively mature, the continual miniaturization of integrated circuits and the increasing complexity and density of chip components have made chip-level detection more challenging, leading to potential risks\cite{ma2023review}. Consequently, there is a need to combine this approach with magnified imaging of the wafer surface using scanning electron microscopes (SEMs) to detect, classify, and analyze specific microscopic defects, thus helping to identify the particular process steps where defects originate.

%Previously, wafer surface defect classification and detection were primarily conducted by experienced engineers. However, this method relies heavily on the engineers' expertise and involves significant time expenditure and subjectivity, lacking uniform standards. With the ongoing development of artificial intelligence, deep learning methods using multi-layer neural networks to extract and learn target features have proven highly effective for this task\cite{gao2022review}.

%In the task of defect classification, it is typical to use a model structure that initially extracts features through convolutional and pooling layers, followed by classification via fully connected layers. Researchers have recently developed numerous classification model structures tailored to specific problems. These models primarily focus on how to extract defect features effectively. For instance, Chen et al. presented a defect recognition and classification algorithm rooted in PCA and classification SVM\cite{chen2008defect}. Chang et al. utilized SVM, drawing on features like smoothness and texture intricacy, for classifying high-intensity defect images\cite{chang2013hybrid}. The classification of defect images requires the formulation of numerous classifiers tailored for myriad inspection steps and an Abundance of accurately labeled data, making data acquisition challenging. Cheon et al. proposed a single CNN model adept at feature extraction\cite{cheon2019convolutional}. They achieved a granular classification of wafer surface defects by recognizing misclassified images and employing a k-nearest neighbors (k-NN) classifier algorithm to gauge the aggregate squared distance between each image feature vector and its k-neighbors within the same category. However, when applied to new or unseen defects, such models necessitate retraining, incurring computational overheads. Moreover, with escalating CNN complexity, the computational demands surge.

%Segmentation of defects is necessary to locate defect positions and gather information such as the size of defects. Unlike classification networks, segmentation networks often use classic encoder-decoder structures such as UNet\cite{ronneberger2015u} and SegNet\cite{badrinarayanan2017segnet}, which focus on effectively leveraging both local and global feature information. Han Hui et al. proposed integrating a Region Proposal Network (RPN) with a UNet architecture to suggest defect areas before conducting defect segmentation \cite{han2020polycrystalline}. This approach enables the segmentation of various defects in wafers with only a limited set of roughly labeled images, enhancing the efficiency of training and application in environments where detailed annotations are scarce. Subhrajit Nag et al. introduced a new network structure, WaferSegClassNet, which extracts multi-scale local features in the encoder and performs classification and segmentation tasks in the decoder \cite{nag2022wafersegclassnet}. This model represents the first detection system capable of simultaneously classifying and segmenting surface defects on wafers. However, it relies on extensive data training and annotation for high accuracy and reliability. 

%Recently, Vic De Ridder et al. introduced a novel approach for defect segmentation using diffusion models\cite{de2023semi}. This approach treats the instance segmentation task as a denoising process from noise to a filter, utilizing diffusion models to predict and reconstruct instance masks for semiconductor defects. This method achieves high precision and improved defect classification and segmentation detection performance. However, the complex network structure and the computational process of the diffusion model require substantial computational resources. Moreover, the performance of this model heavily relies on high-quality and large amounts of training data. These issues make it less suitable for industrial applications. Additionally, the model has only been applied to detecting and segmenting a single type of defect(bridges) following a specific manufacturing process step, limiting its practical utility in diverse industrial scenarios.

%\subsection{Few-shot Anomaly Detection}
%Traditional anomaly detection techniques typically rely on extensive training data to train models for identifying and locating anomalies. However, these methods often face limitations in rapidly changing production environments and diverse anomaly types. Recent research has started exploring effective anomaly detection using few or zero samples to address these challenges.

%Huang et al. developed the anomaly detection method RegAD, based on image registration technology. This method pre-trains an object-agnostic registration network with various images to establish the normality of unseen objects. It identifies anomalies by aligning image features and has achieved promising results. Despite these advancements, implementing few-shot settings in anomaly detection remains an area ripe for further exploration. Recent studies show that pre-trained vision-language models such as CLIP and MiniGPT can significantly enhance performance in anomaly detection tasks.

%Dong et al. introduced the MaskCLIP framework, which employs masked self-distillation to enhance contrastive language-image pretraining\cite{zhou2022maskclip}. This approach strengthens the visual encoder's learning of local image patches and uses indirect language supervision to enhance semantic understanding. It significantly improves transferability and pretraining outcomes across various visual tasks, although it requires substantial computational resources.
%Jeong et al. crafted the WinCLIP framework by integrating state words and prompt templates to characterize normal and anomalous states more accurately\cite{Jeong_2023_CVPR}. This framework introduces a novel window-based technique for extracting and aggregating multi-scale spatial features, significantly boosting the anomaly detection performance of the pre-trained CLIP model.
%Subsequently, Li et al. have further contributed to the field by creating a new expansive multimodal model named Myriad\cite{li2023myriad}. This model, which incorporates a pre-trained Industrial Anomaly Detection (IAD) model to act as a vision expert, embeds anomaly images as tokens interpretable by the language model, thus providing both detailed descriptions and accurate anomaly detection capabilities.
%Recently, Chen et al. introduced CLIP-AD\cite{chen2023clip}, and Li et al. proposed PromptAD\cite{li2024promptad}, both employing language-guided, tiered dual-path model structures and feature manipulation strategies. These approaches effectively address issues encountered when directly calculating anomaly maps using the CLIP model, such as reversed predictions and highlighting irrelevant areas. Specifically, CLIP-AD optimizes the utilization of multi-layer features, corrects feature misalignment, and enhances model performance through additional linear layer fine-tuning. PromptAD connects normal prompts with anomaly suffixes to form anomaly prompts, enabling contrastive learning in a single-class setting.

%These studies extend the boundaries of traditional anomaly detection techniques and demonstrate how to effectively address rapidly changing and sample-scarce production environments through the synergy of few-shot learning and deep learning models. Building on this foundation, our research further explores wafer surface defect detection based on the CLIP model, especially focusing on achieving efficient and accurate anomaly detection in the highly specialized and variable semiconductor manufacturing process using a minimal amount of labeled data.

\section{Conclusion}
In this work, we propose a simple yet effective approach, called SMILE, for graph few-shot learning with fewer tasks. Specifically, we introduce a novel dual-level mixup strategy, including within-task and across-task mixup, for enriching the diversity of nodes within each task and the diversity of tasks. Also, we incorporate the degree-based prior information to learn expressive node embeddings. Theoretically, we prove that SMILE effectively enhances the model's generalization performance. Empirically, we conduct extensive experiments on multiple benchmarks and the results suggest that SMILE significantly outperforms other baselines, including both in-domain and cross-domain few-shot settings.


%%
%% The next two lines define the bibliography style to be used, and
%% the bibliography file.
\bibliographystyle{IEEEtran}
\bibliography{biblio_short}


%%
%% If your work has an appendix, this is the place to put it.
\appendix
\subsection{Lloyd-Max Algorithm}
\label{subsec:Lloyd-Max}
For a given quantization bitwidth $B$ and an operand $\bm{X}$, the Lloyd-Max algorithm finds $2^B$ quantization levels $\{\hat{x}_i\}_{i=1}^{2^B}$ such that quantizing $\bm{X}$ by rounding each scalar in $\bm{X}$ to the nearest quantization level minimizes the quantization MSE. 

The algorithm starts with an initial guess of quantization levels and then iteratively computes quantization thresholds $\{\tau_i\}_{i=1}^{2^B-1}$ and updates quantization levels $\{\hat{x}_i\}_{i=1}^{2^B}$. Specifically, at iteration $n$, thresholds are set to the midpoints of the previous iteration's levels:
\begin{align*}
    \tau_i^{(n)}=\frac{\hat{x}_i^{(n-1)}+\hat{x}_{i+1}^{(n-1)}}2 \text{ for } i=1\ldots 2^B-1
\end{align*}
Subsequently, the quantization levels are re-computed as conditional means of the data regions defined by the new thresholds:
\begin{align*}
    \hat{x}_i^{(n)}=\mathbb{E}\left[ \bm{X} \big| \bm{X}\in [\tau_{i-1}^{(n)},\tau_i^{(n)}] \right] \text{ for } i=1\ldots 2^B
\end{align*}
where to satisfy boundary conditions we have $\tau_0=-\infty$ and $\tau_{2^B}=\infty$. The algorithm iterates the above steps until convergence.

Figure \ref{fig:lm_quant} compares the quantization levels of a $7$-bit floating point (E3M3) quantizer (left) to a $7$-bit Lloyd-Max quantizer (right) when quantizing a layer of weights from the GPT3-126M model at a per-tensor granularity. As shown, the Lloyd-Max quantizer achieves substantially lower quantization MSE. Further, Table \ref{tab:FP7_vs_LM7} shows the superior perplexity achieved by Lloyd-Max quantizers for bitwidths of $7$, $6$ and $5$. The difference between the quantizers is clear at 5 bits, where per-tensor FP quantization incurs a drastic and unacceptable increase in perplexity, while Lloyd-Max quantization incurs a much smaller increase. Nevertheless, we note that even the optimal Lloyd-Max quantizer incurs a notable ($\sim 1.5$) increase in perplexity due to the coarse granularity of quantization. 

\begin{figure}[h]
  \centering
  \includegraphics[width=0.7\linewidth]{sections/figures/LM7_FP7.pdf}
  \caption{\small Quantization levels and the corresponding quantization MSE of Floating Point (left) vs Lloyd-Max (right) Quantizers for a layer of weights in the GPT3-126M model.}
  \label{fig:lm_quant}
\end{figure}

\begin{table}[h]\scriptsize
\begin{center}
\caption{\label{tab:FP7_vs_LM7} \small Comparing perplexity (lower is better) achieved by floating point quantizers and Lloyd-Max quantizers on a GPT3-126M model for the Wikitext-103 dataset.}
\begin{tabular}{c|cc|c}
\hline
 \multirow{2}{*}{\textbf{Bitwidth}} & \multicolumn{2}{|c|}{\textbf{Floating-Point Quantizer}} & \textbf{Lloyd-Max Quantizer} \\
 & Best Format & Wikitext-103 Perplexity & Wikitext-103 Perplexity \\
\hline
7 & E3M3 & 18.32 & 18.27 \\
6 & E3M2 & 19.07 & 18.51 \\
5 & E4M0 & 43.89 & 19.71 \\
\hline
\end{tabular}
\end{center}
\end{table}

\subsection{Proof of Local Optimality of LO-BCQ}
\label{subsec:lobcq_opt_proof}
For a given block $\bm{b}_j$, the quantization MSE during LO-BCQ can be empirically evaluated as $\frac{1}{L_b}\lVert \bm{b}_j- \bm{\hat{b}}_j\rVert^2_2$ where $\bm{\hat{b}}_j$ is computed from equation (\ref{eq:clustered_quantization_definition}) as $C_{f(\bm{b}_j)}(\bm{b}_j)$. Further, for a given block cluster $\mathcal{B}_i$, we compute the quantization MSE as $\frac{1}{|\mathcal{B}_{i}|}\sum_{\bm{b} \in \mathcal{B}_{i}} \frac{1}{L_b}\lVert \bm{b}- C_i^{(n)}(\bm{b})\rVert^2_2$. Therefore, at the end of iteration $n$, we evaluate the overall quantization MSE $J^{(n)}$ for a given operand $\bm{X}$ composed of $N_c$ block clusters as:
\begin{align*}
    \label{eq:mse_iter_n}
    J^{(n)} = \frac{1}{N_c} \sum_{i=1}^{N_c} \frac{1}{|\mathcal{B}_{i}^{(n)}|}\sum_{\bm{v} \in \mathcal{B}_{i}^{(n)}} \frac{1}{L_b}\lVert \bm{b}- B_i^{(n)}(\bm{b})\rVert^2_2
\end{align*}

At the end of iteration $n$, the codebooks are updated from $\mathcal{C}^{(n-1)}$ to $\mathcal{C}^{(n)}$. However, the mapping of a given vector $\bm{b}_j$ to quantizers $\mathcal{C}^{(n)}$ remains as  $f^{(n)}(\bm{b}_j)$. At the next iteration, during the vector clustering step, $f^{(n+1)}(\bm{b}_j)$ finds new mapping of $\bm{b}_j$ to updated codebooks $\mathcal{C}^{(n)}$ such that the quantization MSE over the candidate codebooks is minimized. Therefore, we obtain the following result for $\bm{b}_j$:
\begin{align*}
\frac{1}{L_b}\lVert \bm{b}_j - C_{f^{(n+1)}(\bm{b}_j)}^{(n)}(\bm{b}_j)\rVert^2_2 \le \frac{1}{L_b}\lVert \bm{b}_j - C_{f^{(n)}(\bm{b}_j)}^{(n)}(\bm{b}_j)\rVert^2_2
\end{align*}

That is, quantizing $\bm{b}_j$ at the end of the block clustering step of iteration $n+1$ results in lower quantization MSE compared to quantizing at the end of iteration $n$. Since this is true for all $\bm{b} \in \bm{X}$, we assert the following:
\begin{equation}
\begin{split}
\label{eq:mse_ineq_1}
    \tilde{J}^{(n+1)} &= \frac{1}{N_c} \sum_{i=1}^{N_c} \frac{1}{|\mathcal{B}_{i}^{(n+1)}|}\sum_{\bm{b} \in \mathcal{B}_{i}^{(n+1)}} \frac{1}{L_b}\lVert \bm{b} - C_i^{(n)}(b)\rVert^2_2 \le J^{(n)}
\end{split}
\end{equation}
where $\tilde{J}^{(n+1)}$ is the the quantization MSE after the vector clustering step at iteration $n+1$.

Next, during the codebook update step (\ref{eq:quantizers_update}) at iteration $n+1$, the per-cluster codebooks $\mathcal{C}^{(n)}$ are updated to $\mathcal{C}^{(n+1)}$ by invoking the Lloyd-Max algorithm \citep{Lloyd}. We know that for any given value distribution, the Lloyd-Max algorithm minimizes the quantization MSE. Therefore, for a given vector cluster $\mathcal{B}_i$ we obtain the following result:

\begin{equation}
    \frac{1}{|\mathcal{B}_{i}^{(n+1)}|}\sum_{\bm{b} \in \mathcal{B}_{i}^{(n+1)}} \frac{1}{L_b}\lVert \bm{b}- C_i^{(n+1)}(\bm{b})\rVert^2_2 \le \frac{1}{|\mathcal{B}_{i}^{(n+1)}|}\sum_{\bm{b} \in \mathcal{B}_{i}^{(n+1)}} \frac{1}{L_b}\lVert \bm{b}- C_i^{(n)}(\bm{b})\rVert^2_2
\end{equation}

The above equation states that quantizing the given block cluster $\mathcal{B}_i$ after updating the associated codebook from $C_i^{(n)}$ to $C_i^{(n+1)}$ results in lower quantization MSE. Since this is true for all the block clusters, we derive the following result: 
\begin{equation}
\begin{split}
\label{eq:mse_ineq_2}
     J^{(n+1)} &= \frac{1}{N_c} \sum_{i=1}^{N_c} \frac{1}{|\mathcal{B}_{i}^{(n+1)}|}\sum_{\bm{b} \in \mathcal{B}_{i}^{(n+1)}} \frac{1}{L_b}\lVert \bm{b}- C_i^{(n+1)}(\bm{b})\rVert^2_2  \le \tilde{J}^{(n+1)}   
\end{split}
\end{equation}

Following (\ref{eq:mse_ineq_1}) and (\ref{eq:mse_ineq_2}), we find that the quantization MSE is non-increasing for each iteration, that is, $J^{(1)} \ge J^{(2)} \ge J^{(3)} \ge \ldots \ge J^{(M)}$ where $M$ is the maximum number of iterations. 
%Therefore, we can say that if the algorithm converges, then it must be that it has converged to a local minimum. 
\hfill $\blacksquare$


\begin{figure}
    \begin{center}
    \includegraphics[width=0.5\textwidth]{sections//figures/mse_vs_iter.pdf}
    \end{center}
    \caption{\small NMSE vs iterations during LO-BCQ compared to other block quantization proposals}
    \label{fig:nmse_vs_iter}
\end{figure}

Figure \ref{fig:nmse_vs_iter} shows the empirical convergence of LO-BCQ across several block lengths and number of codebooks. Also, the MSE achieved by LO-BCQ is compared to baselines such as MXFP and VSQ. As shown, LO-BCQ converges to a lower MSE than the baselines. Further, we achieve better convergence for larger number of codebooks ($N_c$) and for a smaller block length ($L_b$), both of which increase the bitwidth of BCQ (see Eq \ref{eq:bitwidth_bcq}).


\subsection{Additional Accuracy Results}
%Table \ref{tab:lobcq_config} lists the various LOBCQ configurations and their corresponding bitwidths.
\begin{table}
\setlength{\tabcolsep}{4.75pt}
\begin{center}
\caption{\label{tab:lobcq_config} Various LO-BCQ configurations and their bitwidths.}
\begin{tabular}{|c||c|c|c|c||c|c||c|} 
\hline
 & \multicolumn{4}{|c||}{$L_b=8$} & \multicolumn{2}{|c||}{$L_b=4$} & $L_b=2$ \\
 \hline
 \backslashbox{$L_A$\kern-1em}{\kern-1em$N_c$} & 2 & 4 & 8 & 16 & 2 & 4 & 2 \\
 \hline
 64 & 4.25 & 4.375 & 4.5 & 4.625 & 4.375 & 4.625 & 4.625\\
 \hline
 32 & 4.375 & 4.5 & 4.625& 4.75 & 4.5 & 4.75 & 4.75 \\
 \hline
 16 & 4.625 & 4.75& 4.875 & 5 & 4.75 & 5 & 5 \\
 \hline
\end{tabular}
\end{center}
\end{table}

%\subsection{Perplexity achieved by various LO-BCQ configurations on Wikitext-103 dataset}

\begin{table} \centering
\begin{tabular}{|c||c|c|c|c||c|c||c|} 
\hline
 $L_b \rightarrow$& \multicolumn{4}{c||}{8} & \multicolumn{2}{c||}{4} & 2\\
 \hline
 \backslashbox{$L_A$\kern-1em}{\kern-1em$N_c$} & 2 & 4 & 8 & 16 & 2 & 4 & 2  \\
 %$N_c \rightarrow$ & 2 & 4 & 8 & 16 & 2 & 4 & 2 \\
 \hline
 \hline
 \multicolumn{8}{c}{GPT3-1.3B (FP32 PPL = 9.98)} \\ 
 \hline
 \hline
 64 & 10.40 & 10.23 & 10.17 & 10.15 &  10.28 & 10.18 & 10.19 \\
 \hline
 32 & 10.25 & 10.20 & 10.15 & 10.12 &  10.23 & 10.17 & 10.17 \\
 \hline
 16 & 10.22 & 10.16 & 10.10 & 10.09 &  10.21 & 10.14 & 10.16 \\
 \hline
  \hline
 \multicolumn{8}{c}{GPT3-8B (FP32 PPL = 7.38)} \\ 
 \hline
 \hline
 64 & 7.61 & 7.52 & 7.48 &  7.47 &  7.55 &  7.49 & 7.50 \\
 \hline
 32 & 7.52 & 7.50 & 7.46 &  7.45 &  7.52 &  7.48 & 7.48  \\
 \hline
 16 & 7.51 & 7.48 & 7.44 &  7.44 &  7.51 &  7.49 & 7.47  \\
 \hline
\end{tabular}
\caption{\label{tab:ppl_gpt3_abalation} Wikitext-103 perplexity across GPT3-1.3B and 8B models.}
\end{table}

\begin{table} \centering
\begin{tabular}{|c||c|c|c|c||} 
\hline
 $L_b \rightarrow$& \multicolumn{4}{c||}{8}\\
 \hline
 \backslashbox{$L_A$\kern-1em}{\kern-1em$N_c$} & 2 & 4 & 8 & 16 \\
 %$N_c \rightarrow$ & 2 & 4 & 8 & 16 & 2 & 4 & 2 \\
 \hline
 \hline
 \multicolumn{5}{|c|}{Llama2-7B (FP32 PPL = 5.06)} \\ 
 \hline
 \hline
 64 & 5.31 & 5.26 & 5.19 & 5.18  \\
 \hline
 32 & 5.23 & 5.25 & 5.18 & 5.15  \\
 \hline
 16 & 5.23 & 5.19 & 5.16 & 5.14  \\
 \hline
 \multicolumn{5}{|c|}{Nemotron4-15B (FP32 PPL = 5.87)} \\ 
 \hline
 \hline
 64  & 6.3 & 6.20 & 6.13 & 6.08  \\
 \hline
 32  & 6.24 & 6.12 & 6.07 & 6.03  \\
 \hline
 16  & 6.12 & 6.14 & 6.04 & 6.02  \\
 \hline
 \multicolumn{5}{|c|}{Nemotron4-340B (FP32 PPL = 3.48)} \\ 
 \hline
 \hline
 64 & 3.67 & 3.62 & 3.60 & 3.59 \\
 \hline
 32 & 3.63 & 3.61 & 3.59 & 3.56 \\
 \hline
 16 & 3.61 & 3.58 & 3.57 & 3.55 \\
 \hline
\end{tabular}
\caption{\label{tab:ppl_llama7B_nemo15B} Wikitext-103 perplexity compared to FP32 baseline in Llama2-7B and Nemotron4-15B, 340B models}
\end{table}

%\subsection{Perplexity achieved by various LO-BCQ configurations on MMLU dataset}


\begin{table} \centering
\begin{tabular}{|c||c|c|c|c||c|c|c|c|} 
\hline
 $L_b \rightarrow$& \multicolumn{4}{c||}{8} & \multicolumn{4}{c||}{8}\\
 \hline
 \backslashbox{$L_A$\kern-1em}{\kern-1em$N_c$} & 2 & 4 & 8 & 16 & 2 & 4 & 8 & 16  \\
 %$N_c \rightarrow$ & 2 & 4 & 8 & 16 & 2 & 4 & 2 \\
 \hline
 \hline
 \multicolumn{5}{|c|}{Llama2-7B (FP32 Accuracy = 45.8\%)} & \multicolumn{4}{|c|}{Llama2-70B (FP32 Accuracy = 69.12\%)} \\ 
 \hline
 \hline
 64 & 43.9 & 43.4 & 43.9 & 44.9 & 68.07 & 68.27 & 68.17 & 68.75 \\
 \hline
 32 & 44.5 & 43.8 & 44.9 & 44.5 & 68.37 & 68.51 & 68.35 & 68.27  \\
 \hline
 16 & 43.9 & 42.7 & 44.9 & 45 & 68.12 & 68.77 & 68.31 & 68.59  \\
 \hline
 \hline
 \multicolumn{5}{|c|}{GPT3-22B (FP32 Accuracy = 38.75\%)} & \multicolumn{4}{|c|}{Nemotron4-15B (FP32 Accuracy = 64.3\%)} \\ 
 \hline
 \hline
 64 & 36.71 & 38.85 & 38.13 & 38.92 & 63.17 & 62.36 & 63.72 & 64.09 \\
 \hline
 32 & 37.95 & 38.69 & 39.45 & 38.34 & 64.05 & 62.30 & 63.8 & 64.33  \\
 \hline
 16 & 38.88 & 38.80 & 38.31 & 38.92 & 63.22 & 63.51 & 63.93 & 64.43  \\
 \hline
\end{tabular}
\caption{\label{tab:mmlu_abalation} Accuracy on MMLU dataset across GPT3-22B, Llama2-7B, 70B and Nemotron4-15B models.}
\end{table}


%\subsection{Perplexity achieved by various LO-BCQ configurations on LM evaluation harness}

\begin{table} \centering
\begin{tabular}{|c||c|c|c|c||c|c|c|c|} 
\hline
 $L_b \rightarrow$& \multicolumn{4}{c||}{8} & \multicolumn{4}{c||}{8}\\
 \hline
 \backslashbox{$L_A$\kern-1em}{\kern-1em$N_c$} & 2 & 4 & 8 & 16 & 2 & 4 & 8 & 16  \\
 %$N_c \rightarrow$ & 2 & 4 & 8 & 16 & 2 & 4 & 2 \\
 \hline
 \hline
 \multicolumn{5}{|c|}{Race (FP32 Accuracy = 37.51\%)} & \multicolumn{4}{|c|}{Boolq (FP32 Accuracy = 64.62\%)} \\ 
 \hline
 \hline
 64 & 36.94 & 37.13 & 36.27 & 37.13 & 63.73 & 62.26 & 63.49 & 63.36 \\
 \hline
 32 & 37.03 & 36.36 & 36.08 & 37.03 & 62.54 & 63.51 & 63.49 & 63.55  \\
 \hline
 16 & 37.03 & 37.03 & 36.46 & 37.03 & 61.1 & 63.79 & 63.58 & 63.33  \\
 \hline
 \hline
 \multicolumn{5}{|c|}{Winogrande (FP32 Accuracy = 58.01\%)} & \multicolumn{4}{|c|}{Piqa (FP32 Accuracy = 74.21\%)} \\ 
 \hline
 \hline
 64 & 58.17 & 57.22 & 57.85 & 58.33 & 73.01 & 73.07 & 73.07 & 72.80 \\
 \hline
 32 & 59.12 & 58.09 & 57.85 & 58.41 & 73.01 & 73.94 & 72.74 & 73.18  \\
 \hline
 16 & 57.93 & 58.88 & 57.93 & 58.56 & 73.94 & 72.80 & 73.01 & 73.94  \\
 \hline
\end{tabular}
\caption{\label{tab:mmlu_abalation} Accuracy on LM evaluation harness tasks on GPT3-1.3B model.}
\end{table}

\begin{table} \centering
\begin{tabular}{|c||c|c|c|c||c|c|c|c|} 
\hline
 $L_b \rightarrow$& \multicolumn{4}{c||}{8} & \multicolumn{4}{c||}{8}\\
 \hline
 \backslashbox{$L_A$\kern-1em}{\kern-1em$N_c$} & 2 & 4 & 8 & 16 & 2 & 4 & 8 & 16  \\
 %$N_c \rightarrow$ & 2 & 4 & 8 & 16 & 2 & 4 & 2 \\
 \hline
 \hline
 \multicolumn{5}{|c|}{Race (FP32 Accuracy = 41.34\%)} & \multicolumn{4}{|c|}{Boolq (FP32 Accuracy = 68.32\%)} \\ 
 \hline
 \hline
 64 & 40.48 & 40.10 & 39.43 & 39.90 & 69.20 & 68.41 & 69.45 & 68.56 \\
 \hline
 32 & 39.52 & 39.52 & 40.77 & 39.62 & 68.32 & 67.43 & 68.17 & 69.30  \\
 \hline
 16 & 39.81 & 39.71 & 39.90 & 40.38 & 68.10 & 66.33 & 69.51 & 69.42  \\
 \hline
 \hline
 \multicolumn{5}{|c|}{Winogrande (FP32 Accuracy = 67.88\%)} & \multicolumn{4}{|c|}{Piqa (FP32 Accuracy = 78.78\%)} \\ 
 \hline
 \hline
 64 & 66.85 & 66.61 & 67.72 & 67.88 & 77.31 & 77.42 & 77.75 & 77.64 \\
 \hline
 32 & 67.25 & 67.72 & 67.72 & 67.00 & 77.31 & 77.04 & 77.80 & 77.37  \\
 \hline
 16 & 68.11 & 68.90 & 67.88 & 67.48 & 77.37 & 78.13 & 78.13 & 77.69  \\
 \hline
\end{tabular}
\caption{\label{tab:mmlu_abalation} Accuracy on LM evaluation harness tasks on GPT3-8B model.}
\end{table}

\begin{table} \centering
\begin{tabular}{|c||c|c|c|c||c|c|c|c|} 
\hline
 $L_b \rightarrow$& \multicolumn{4}{c||}{8} & \multicolumn{4}{c||}{8}\\
 \hline
 \backslashbox{$L_A$\kern-1em}{\kern-1em$N_c$} & 2 & 4 & 8 & 16 & 2 & 4 & 8 & 16  \\
 %$N_c \rightarrow$ & 2 & 4 & 8 & 16 & 2 & 4 & 2 \\
 \hline
 \hline
 \multicolumn{5}{|c|}{Race (FP32 Accuracy = 40.67\%)} & \multicolumn{4}{|c|}{Boolq (FP32 Accuracy = 76.54\%)} \\ 
 \hline
 \hline
 64 & 40.48 & 40.10 & 39.43 & 39.90 & 75.41 & 75.11 & 77.09 & 75.66 \\
 \hline
 32 & 39.52 & 39.52 & 40.77 & 39.62 & 76.02 & 76.02 & 75.96 & 75.35  \\
 \hline
 16 & 39.81 & 39.71 & 39.90 & 40.38 & 75.05 & 73.82 & 75.72 & 76.09  \\
 \hline
 \hline
 \multicolumn{5}{|c|}{Winogrande (FP32 Accuracy = 70.64\%)} & \multicolumn{4}{|c|}{Piqa (FP32 Accuracy = 79.16\%)} \\ 
 \hline
 \hline
 64 & 69.14 & 70.17 & 70.17 & 70.56 & 78.24 & 79.00 & 78.62 & 78.73 \\
 \hline
 32 & 70.96 & 69.69 & 71.27 & 69.30 & 78.56 & 79.49 & 79.16 & 78.89  \\
 \hline
 16 & 71.03 & 69.53 & 69.69 & 70.40 & 78.13 & 79.16 & 79.00 & 79.00  \\
 \hline
\end{tabular}
\caption{\label{tab:mmlu_abalation} Accuracy on LM evaluation harness tasks on GPT3-22B model.}
\end{table}

\begin{table} \centering
\begin{tabular}{|c||c|c|c|c||c|c|c|c|} 
\hline
 $L_b \rightarrow$& \multicolumn{4}{c||}{8} & \multicolumn{4}{c||}{8}\\
 \hline
 \backslashbox{$L_A$\kern-1em}{\kern-1em$N_c$} & 2 & 4 & 8 & 16 & 2 & 4 & 8 & 16  \\
 %$N_c \rightarrow$ & 2 & 4 & 8 & 16 & 2 & 4 & 2 \\
 \hline
 \hline
 \multicolumn{5}{|c|}{Race (FP32 Accuracy = 44.4\%)} & \multicolumn{4}{|c|}{Boolq (FP32 Accuracy = 79.29\%)} \\ 
 \hline
 \hline
 64 & 42.49 & 42.51 & 42.58 & 43.45 & 77.58 & 77.37 & 77.43 & 78.1 \\
 \hline
 32 & 43.35 & 42.49 & 43.64 & 43.73 & 77.86 & 75.32 & 77.28 & 77.86  \\
 \hline
 16 & 44.21 & 44.21 & 43.64 & 42.97 & 78.65 & 77 & 76.94 & 77.98  \\
 \hline
 \hline
 \multicolumn{5}{|c|}{Winogrande (FP32 Accuracy = 69.38\%)} & \multicolumn{4}{|c|}{Piqa (FP32 Accuracy = 78.07\%)} \\ 
 \hline
 \hline
 64 & 68.9 & 68.43 & 69.77 & 68.19 & 77.09 & 76.82 & 77.09 & 77.86 \\
 \hline
 32 & 69.38 & 68.51 & 68.82 & 68.90 & 78.07 & 76.71 & 78.07 & 77.86  \\
 \hline
 16 & 69.53 & 67.09 & 69.38 & 68.90 & 77.37 & 77.8 & 77.91 & 77.69  \\
 \hline
\end{tabular}
\caption{\label{tab:mmlu_abalation} Accuracy on LM evaluation harness tasks on Llama2-7B model.}
\end{table}

\begin{table} \centering
\begin{tabular}{|c||c|c|c|c||c|c|c|c|} 
\hline
 $L_b \rightarrow$& \multicolumn{4}{c||}{8} & \multicolumn{4}{c||}{8}\\
 \hline
 \backslashbox{$L_A$\kern-1em}{\kern-1em$N_c$} & 2 & 4 & 8 & 16 & 2 & 4 & 8 & 16  \\
 %$N_c \rightarrow$ & 2 & 4 & 8 & 16 & 2 & 4 & 2 \\
 \hline
 \hline
 \multicolumn{5}{|c|}{Race (FP32 Accuracy = 48.8\%)} & \multicolumn{4}{|c|}{Boolq (FP32 Accuracy = 85.23\%)} \\ 
 \hline
 \hline
 64 & 49.00 & 49.00 & 49.28 & 48.71 & 82.82 & 84.28 & 84.03 & 84.25 \\
 \hline
 32 & 49.57 & 48.52 & 48.33 & 49.28 & 83.85 & 84.46 & 84.31 & 84.93  \\
 \hline
 16 & 49.85 & 49.09 & 49.28 & 48.99 & 85.11 & 84.46 & 84.61 & 83.94  \\
 \hline
 \hline
 \multicolumn{5}{|c|}{Winogrande (FP32 Accuracy = 79.95\%)} & \multicolumn{4}{|c|}{Piqa (FP32 Accuracy = 81.56\%)} \\ 
 \hline
 \hline
 64 & 78.77 & 78.45 & 78.37 & 79.16 & 81.45 & 80.69 & 81.45 & 81.5 \\
 \hline
 32 & 78.45 & 79.01 & 78.69 & 80.66 & 81.56 & 80.58 & 81.18 & 81.34  \\
 \hline
 16 & 79.95 & 79.56 & 79.79 & 79.72 & 81.28 & 81.66 & 81.28 & 80.96  \\
 \hline
\end{tabular}
\caption{\label{tab:mmlu_abalation} Accuracy on LM evaluation harness tasks on Llama2-70B model.}
\end{table}

%\section{MSE Studies}
%\textcolor{red}{TODO}


\subsection{Number Formats and Quantization Method}
\label{subsec:numFormats_quantMethod}
\subsubsection{Integer Format}
An $n$-bit signed integer (INT) is typically represented with a 2s-complement format \citep{yao2022zeroquant,xiao2023smoothquant,dai2021vsq}, where the most significant bit denotes the sign.

\subsubsection{Floating Point Format}
An $n$-bit signed floating point (FP) number $x$ comprises of a 1-bit sign ($x_{\mathrm{sign}}$), $B_m$-bit mantissa ($x_{\mathrm{mant}}$) and $B_e$-bit exponent ($x_{\mathrm{exp}}$) such that $B_m+B_e=n-1$. The associated constant exponent bias ($E_{\mathrm{bias}}$) is computed as $(2^{{B_e}-1}-1)$. We denote this format as $E_{B_e}M_{B_m}$.  

\subsubsection{Quantization Scheme}
\label{subsec:quant_method}
A quantization scheme dictates how a given unquantized tensor is converted to its quantized representation. We consider FP formats for the purpose of illustration. Given an unquantized tensor $\bm{X}$ and an FP format $E_{B_e}M_{B_m}$, we first, we compute the quantization scale factor $s_X$ that maps the maximum absolute value of $\bm{X}$ to the maximum quantization level of the $E_{B_e}M_{B_m}$ format as follows:
\begin{align}
\label{eq:sf}
    s_X = \frac{\mathrm{max}(|\bm{X}|)}{\mathrm{max}(E_{B_e}M_{B_m})}
\end{align}
In the above equation, $|\cdot|$ denotes the absolute value function.

Next, we scale $\bm{X}$ by $s_X$ and quantize it to $\hat{\bm{X}}$ by rounding it to the nearest quantization level of $E_{B_e}M_{B_m}$ as:

\begin{align}
\label{eq:tensor_quant}
    \hat{\bm{X}} = \text{round-to-nearest}\left(\frac{\bm{X}}{s_X}, E_{B_e}M_{B_m}\right)
\end{align}

We perform dynamic max-scaled quantization \citep{wu2020integer}, where the scale factor $s$ for activations is dynamically computed during runtime.

\subsection{Vector Scaled Quantization}
\begin{wrapfigure}{r}{0.35\linewidth}
  \centering
  \includegraphics[width=\linewidth]{sections/figures/vsquant.jpg}
  \caption{\small Vectorwise decomposition for per-vector scaled quantization (VSQ \citep{dai2021vsq}).}
  \label{fig:vsquant}
\end{wrapfigure}
During VSQ \citep{dai2021vsq}, the operand tensors are decomposed into 1D vectors in a hardware friendly manner as shown in Figure \ref{fig:vsquant}. Since the decomposed tensors are used as operands in matrix multiplications during inference, it is beneficial to perform this decomposition along the reduction dimension of the multiplication. The vectorwise quantization is performed similar to tensorwise quantization described in Equations \ref{eq:sf} and \ref{eq:tensor_quant}, where a scale factor $s_v$ is required for each vector $\bm{v}$ that maps the maximum absolute value of that vector to the maximum quantization level. While smaller vector lengths can lead to larger accuracy gains, the associated memory and computational overheads due to the per-vector scale factors increases. To alleviate these overheads, VSQ \citep{dai2021vsq} proposed a second level quantization of the per-vector scale factors to unsigned integers, while MX \citep{rouhani2023shared} quantizes them to integer powers of 2 (denoted as $2^{INT}$).

\subsubsection{MX Format}
The MX format proposed in \citep{rouhani2023microscaling} introduces the concept of sub-block shifting. For every two scalar elements of $b$-bits each, there is a shared exponent bit. The value of this exponent bit is determined through an empirical analysis that targets minimizing quantization MSE. We note that the FP format $E_{1}M_{b}$ is strictly better than MX from an accuracy perspective since it allocates a dedicated exponent bit to each scalar as opposed to sharing it across two scalars. Therefore, we conservatively bound the accuracy of a $b+2$-bit signed MX format with that of a $E_{1}M_{b}$ format in our comparisons. For instance, we use E1M2 format as a proxy for MX4.

\begin{figure}
    \centering
    \includegraphics[width=1\linewidth]{sections//figures/BlockFormats.pdf}
    \caption{\small Comparing LO-BCQ to MX format.}
    \label{fig:block_formats}
\end{figure}

Figure \ref{fig:block_formats} compares our $4$-bit LO-BCQ block format to MX \citep{rouhani2023microscaling}. As shown, both LO-BCQ and MX decompose a given operand tensor into block arrays and each block array into blocks. Similar to MX, we find that per-block quantization ($L_b < L_A$) leads to better accuracy due to increased flexibility. While MX achieves this through per-block $1$-bit micro-scales, we associate a dedicated codebook to each block through a per-block codebook selector. Further, MX quantizes the per-block array scale-factor to E8M0 format without per-tensor scaling. In contrast during LO-BCQ, we find that per-tensor scaling combined with quantization of per-block array scale-factor to E4M3 format results in superior inference accuracy across models. 



\end{document}
\endinput
%%
