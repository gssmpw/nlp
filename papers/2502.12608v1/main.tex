%%
%% This is file `sample-sigconf-authordraft.tex',
%% generated with the docstrip utility.
%%
%% The original source files were:
%%
%% samples.dtx  (with options: `all,proceedings,bibtex,authordraft')
%% 
%% IMPORTANT NOTICE:
%% 
%% For the copyright see the source file.
%% 
%% Any modified versions of this file must be renamed
%% with new filenames distinct from sample-sigconf-authordraft.tex.
%% 
%% For distribution of the original source see the terms
%% for copying and modification in the file samples.dtx.
%% 
%% This generated file may be distributed as long as the
%% original source files, as listed above, are part of the
%% same distribution. (The sources need not necessarily be
%% in the same archive or directory.)
%%
%%
%% Commands for TeXCount
%TC:macro \cite [option:text,text]
%TC:macro \citep [option:text,text]
%TC:macro \citet [option:text,text]
%TC:envir table 0 1
%TC:envir table* 0 1
%TC:envir tabular [ignore] word
%TC:envir displaymath 0 word
%TC:envir math 0 word
%TC:envir comment 0 0
%%
%% The first command in your LaTeX source must be the \documentclass
%% command.
%%
%% For submission and review of your manuscript please change the
%% command to \documentclass[manuscript, screen, review]{acmart}.
%%
%% When submitting camera ready or to TAPS, please change the command
%% to \documentclass[sigconf]{acmart} or whichever template is required
%% for your publication.
%%
%%
\documentclass[sigconf]{acmart}
\usepackage{subcaption}
\usepackage{multirow}
\usepackage{pifont}
\usepackage{enumitem}
\usepackage{wrapfig}
% \usepackage{subfig}
\usepackage{graphicx}
\usepackage{natbib}
\usepackage{subcaption}
\theoremstyle{plain}
\newtheorem{theorem}{Theorem}[section]
\newtheorem{proposition}[theorem]{Proposition}
\newtheorem{lemma}[theorem]{Lemma}
\newtheorem{corollary}[theorem]{Corollary}
\theoremstyle{definition}
\newtheorem{definition}[theorem]{Definition}
\newtheorem{assumption}[theorem]{Assumption}
\theoremstyle{remark}
\newtheorem{remark}[theorem]{Remark}

\newcommand{\jt}[1]{{\color{red}[JT:#1]}}
\newcommand{\czk}[1]{{\color{blue}[czk:#1]}}
\newcommand{\lbh}[1]{{\color{green}[lbh:#1]}}
%%
%% \BibTeX command to typeset BibTeX logo in the docs
\AtBeginDocument{%
  \providecommand\BibTeX{{%
    Bib\TeX}}}

%% Rights management information.  This information is sent to you
%% when you complete the rights form.  These commands have SAMPLE
%% values in them; it is your responsibility as an author to replace
%% the commands and values with those provided to you when you
%% complete the rights form.
\setcopyright{acmlicensed}
\copyrightyear{2018}
\acmYear{2018}
\acmDOI{XXXXXXX.XXXXXXX}
%% These commands are for a PROCEEDINGS abstract or paper.
\acmConference[Conference acronym 'XX]{Make sure to enter the correct
  conference title from your rights confirmation email}{June 03--05,
  2018}{Woodstock, NY}
%%
%%  Uncomment \acmBooktitle if the title of the proceedings is different
%%  from ``Proceedings of ...''!
%%
%%\acmBooktitle{Woodstock '18: ACM Symposium on Neural Gaze Detection,
%%  June 03--05, 2018, Woodstock, NY}
\acmISBN{978-1-4503-XXXX-X/2018/06}


%%
%% Submission ID.
%% Use this when submitting an article to a sponsored event. You'll
%% receive a unique submission ID from the organizers
%% of the event, and this ID should be used as the parameter to this command.
%%\acmSubmissionID{123-A56-BU3}

%%
%% For managing citations, it is recommended to use bibliography
%% files in BibTeX format.
%%
%% You can then either use BibTeX with the ACM-Reference-Format style,
%% or BibLaTeX with the acmnumeric or acmauthoryear sytles, that include
%% support for advanced citation of software artefact from the
%% biblatex-software package, also separately available on CTAN.
%%
%% Look at the sample-*-biblatex.tex files for templates showcasing
%% the biblatex styles.
%%

%%
%% The majority of ACM publications use numbered citations and
%% references.  The command \citestyle{authoryear} switches to the
%% "author year" style.
%%
%% If you are preparing content for an event
%% sponsored by ACM SIGGRAPH, you must use the "author year" style of
%% citations and references.
%% Uncommenting
%% the next command will enable that style.
%%\citestyle{acmauthoryear}


%%
%% end of the preamble, start of the body of the document source.
\begin{document}

%%
%% The "title" command has an optional parameter,
%% allowing the author to define a "short title" to be used in page headers.
\title{Unveiling Mode Connectivity in Graph Neural Networks}

%%
%% The "author" command and its associated commands are used to define
%% the authors and their affiliations.
%% Of note is the shared affiliation of the first two authors, and the
%% "authornote" and "authornotemark" commands
%% used to denote shared contribution to the research.
\author{Bingheng Li}
% \author{Ben Trovato}
% \authornote{Both authors contributed equally to this research.}
% \email{trovato@corporation.com}
% \orcid{1234-5678-9012}
% \author{G.K.M. Tobin}
% \authornotemark[1]
% \email{webmaster@marysville-ohio.com}
\affiliation{%
    \institution{Department of Computer Science and Engineering,}
  \institution{Michigan State University}
  % \streetaddress{P.O. Box 1212}
  % \city{Dublin}
  % \state{Ohio}
  \country{}
  % \postcode{43017-6221}
}
% \institution{Jilin University}
\email{libinghe@msu.edu}

\author{Zhikai Chen}
\affiliation{%
    \institution{Department of Computer Science and Engineering,}
  \institution{Michigan State University}
  \country{}
}
\email{chenzh85@msu.edu}

\author{Haoyu Han}
\affiliation{%
    \institution{Department of Computer Science and Engineering,}
  \institution{Michigan State University}
  \country{}
  % \city{Rocquencourt}
  % \country{France}
}
\email{hanhaoy1@msu.edu}

\author{Shenglai Zeng}
\affiliation{%
    \institution{Department of Computer Science and Engineering,}
  \institution{Michigan State University}
  \country{}
}
\email{zengshe1@msu.edu}

\author{Jingzhe Liu}
\affiliation{%
    \institution{Department of Computer Science and Engineering,}
  \institution{Michigan State University}
  \country{}
}
\email{liujin33@msu.edu}

\author{Jiliang Tang}
\affiliation{
\institution{Department of Computer Science and Engineering,}
  \institution{Michigan State University}
  \country{}
}
\email{tangjili@msu.edu}


%%
%% By default, the full list of authors will be used in the page
%% headers. Often, this list is too long, and will overlap
%% other information printed in the page headers. This command allows
%% the author to define a more concise list
%% of authors' names for this purpose.
\renewcommand{\shortauthors}{Trovato et al.}
\newcommand\liu[1]{\textcolor{blue}{liu: #1}}
%%
%% The abstract is a short summary of the work to be presented in the
%% article.
\begin{abstract}
A fundamental challenge in understanding graph neural networks (GNNs) lies in characterizing their optimization dynamics and loss landscape geometry, critical for improving interpretability and robustness. While mode connectivity—a lens for analyzing geometric properties of loss landscapes—has proven insightful for other deep learning architectures, its implications for GNNs remain unexplored. This work presents the first investigation of mode connectivity in GNNs. We uncover that GNNs exhibit distinct non-linear mode connectivity, diverging from patterns observed in fully-connected networks or CNNs. Crucially, we demonstrate that graph structure, rather than model architecture, dominates this behavior, with graph properties like homophily correlating with mode connectivity patterns. We further establish a link between mode connectivity and generalization, proposing a generalization bound based on loss barriers and revealing its utility as a diagnostic tool. Our findings further bridge theoretical insights with practical implications: they rationalize domain alignment strategies in graph learning and provide a foundation for refining GNN training paradigms. 
\end{abstract}

%%
%% The code below is generated by the tool at http://dl.acm.org/ccs.cfm.
%% Please copy and paste the code instead of the example below.
%%
\begin{CCSXML}
<ccs2012>
<concept>
<concept_id>10003752.10003809.10003635</concept_id>
<concept_desc>Theory of computation~Graph algorithms analysis</concept_desc>
<concept_significance>500</concept_significance>
</concept>
<concept>
<concept_id>10010147.10010257.10010293.10010294</concept_id>
<concept_desc>Computing methodologies~Neural networks</concept_desc>
<concept_significance>500</concept_significance>
</concept>
</ccs2012>
\end{CCSXML}

\ccsdesc[500]{Theory of computation~Graph algorithms analysis}
\ccsdesc[500]{Computing methodologies~Neural networks}

% \ccsdesc[500]{Do Not Use This Code~Generate the Correct Terms for Your Paper}
% \ccsdesc[300]{Do Not Use This Code~Generate the Correct Terms for Your Paper}
% \ccsdesc{Do Not Use This Code~Generate the Correct Terms for Your Paper}
% \ccsdesc[100]{Do Not Use This Code~Generate the Correct Terms for Your Paper}

%%
%% Keywords. The author(s) should pick words that accurately describe
%% the work being presented. Separate the keywords with commas.
%\keywords{Mode Connectivity, Graph Neural Networks, Machine Learning theory}
%% A "teaser" image appears between the author and affiliation
%% information and the body of the document, and typically spans the
%% page.




%%
%% This command processes the author and affiliation and title
%% information and builds the first part of the formatted document.
\maketitle

\section{Introduction}
\label{sec:intro}
% Image editing methods in diffusion models depend on user-defined control directions - users can unlock their creativity using these methods by specifying the desired manipulation through prompts~\cite{gandikota2023concept}, reference images~\cite{ruiz2022dreambooth, kumari2022customdiffusion, gal2022image, chen2024trainingfreeregionalpromptingdiffusion}, or attribute vectors~\cite{parmar2023zero,hertz2022prompt}. In this work, we ask a fundamentally different question: \emph{Can we automatically discover the underlying visual structure of a concept within diffusion model's knowledge?} %Rather than requiring user-specified controls, we aim to decompose the model's internal knowledge into meaningful directions.

% This question touches on a fundamental limitation in how we interact with diffusion models. Current control methods ~\cite{zhang2023addingconditionalcontroltexttoimage, gandikota2023concept, ye2023ipadaptertextcompatibleimage,ye2023ipadaptertextcompatibleimage, hertz2024stylealignedimagegeneration, li2023photomaker, shi2024instantbooth, chen2024trainingfreeregionalpromptingdiffusion} require users to specify their desired manipulations in advance, limiting interactive creativity. This contrasts with natural human artistic workflows, where creators dynamically explore creative ideas while jointly refining them toward meaningful artistic outcomes~\cite{hoffmann2016modeling}. This synergy between specification and exploration is not new to generative models. Early GAN architectures naturally developed disentangled latent spaces that enabled continuous\cite{harkonen2020ganspace,radford2015unsupervised, wu2021stylespace, shen2020interfacegan}, compositional control over generated images. Users could explore these spaces to discover interesting variations that would be difficult to describe in words~\cite{wu2021stylespace}, then combine them to achieve their creative goals~\cite{grabe2022towards}. 


% While diffusion models have largely superseded GANs in conditional image synthesis~\cite{dhariwal2021diffusion},  their underlying structure remains less understood. Diffusion models achieve remarkable diversity through high-dimensional latents, unlike GANs' compact latent spaces.  With a single prompt, diffusion models can generate radically different variations through different random initializations of input noise. We ask - Is it possible to discover interpretable structure within this vast space of variations?

Text-to-image diffusion models are capable of generating remarkable visual variations from a single prompt through different random initializations. However, this vast creative potential remains largely opaque to users---while we can generate diverse images, we lack understanding of the underlying structure of these variations. This presents a fundamental challenge: how can we discover and expose the latent visual capabilities encoded within these models?

\let\thefootnote\relax \footnote{$^{*}$Correspondence to \texttt{gandikota.ro@northeastern.edu}}

The challenge touches on a key limitation in how we interact with diffusion models today. Current control methods require users to explicitly specify their desired edits in advance through prompts~\cite{gandikota2023concept}, reference images~\cite{zhang2023addingconditionalcontroltexttoimage, chen2024trainingfreeregionalpromptingdiffusion, ruiz2022dreambooth,kumari2022customdiffusion, Ryu_lora, hu2021lora}, or attribute vectors~\cite{ye2023ipadaptertextcompatibleimage, hertz2024stylealignedimagegeneration, li2023photomaker, shi2024instantbooth,parmar2023zero,hertz2022prompt}. That contrasts sharply with natural human creative workflows, where artists dynamically explore creative ideas and jointly refine them toward meaningful artistic outcomes~\cite{hoffmann2016modeling}. The need for pre-specified controls creates a barrier between users and the full creative potential of these models.

Interestingly, earlier generative models like GANs~\cite{gans,karras2019style,brock2018large} naturally developed more interpretable internal structures. Their compact latent spaces often exhibited emergent disentanglement~\cite{harkonen2020ganspace,radford2015unsupervised, wu2021stylespace, shen2020interfacegan}, enabling continuous and compositional control over generated images. Users could explore these spaces to discover interesting variations that would be difficult to describe in words~\cite{wu2021stylespace}, then combine them to achieve their creative goals~\cite{grabe2022towards}.

Diffusion models have largely superseded GANs in conditional image synthesis~\cite{dhariwal2021diffusion}, achieving greater diversity through much higher-dimensional latents. And yet an understanding of the underlying structure of these larger latent spaces has remained elusive. In this work, we ask a fundamental question: \emph{Can we automatically discover the visual structure within a diffusion model's knowledge of a concept?} Rather than requiring user-specified controls, we aim to decompose the model's internal representations into expressive directions that users can explore and combine.

To address these needs, we present \textbf{SliderSpace}, a framework that brings systematic explorability to diffusion models. Given just a text prompt, SliderSpace discovers a canonical set of meaningful, diverse, and controllable directions within the model's knowledge of that concept. Each direction is implemented as a low-rank adapter~\cite{hu2021lora} that can be scaled and composed with others, allowing users to explore and smoothly combine different aspects of variation, as shown in Figure~\ref{fig:intro}.

We ground SliderSpace discovery in three key requirements for meaningful decomposition of a diffusion model's visual manifold: 
\begin{enumerate}
    \item \textbf{Unsupervised Discovery:} The decomposition process should emerge from the intrinsic structure of the model's learned representation, rather than being guided by predefined attributes. This ensures we capture the true topology of the model's knowledge space rather than projecting our assumptions onto it.
    
    \item \textbf{Semantic Orthogonality:} Each discovered control must represent a distinct semantic direction. This is enforced in a semantic feature space, like CLIP, where every slider has an orthogonal effect in embeddings. This prevents discovering multiple controls that create similar semantic effects, making the system more efficient and easier.
    
    \item \textbf{Distribution Consistency:} Directions must induce consistent transformations across both random seeds and prompt variations. 
\end{enumerate}

These requirements naturally lead to our proposed framework, which we formalize in Section~\ref{sec:method}. As we show in our experiments, SliderSpace is architecture-agnostic, working with both conventional U-Net based models like Stable Diffusion~\cite{rombach2022high, rombach2022sd20, podell2023sdxl, turbo, dmd} and recent transformer-based architectures like Flux~\cite{flux}.

We demonstrate the expressiveness of SliderSpace through three applications: First, we show how SliderSpace can decompose high-level concepts into diverse and expressive components, revealing the natural axes of variation in the model's understanding. Second, we explore artistic style variation, where SliderSpace discovers directions that match or exceed the diversity of manually curated artist lists while being judged more useful by human evaluators. Finally, we show how SliderSpace can help reverse the mode collapse commonly observed in distilled diffusion models, restoring diversity while maintaining generation speed.

Beyond providing practical creative control, SliderSpace opens new avenues for understanding and utilizing the latent capabilities of diffusion models. By mapping these models' visual potential into intuitive, composable directions, we take a step toward making their creative possibilities more accessible and interpretable to users.

% Image editing methods in diffusion models unlock the creativity of users. In this work we ask an alternate question: \emph{Can we organize and expose what of the diffusion model is already capable of?}.
% Existing methods for controlling image generation typically require users to manually specify edit directions for desired changes. This process is time-consuming, requires technical expertise, and limits the spontaneity of the creative process. For instance, if a user wants to adjust the smile of a generated person, they must explicitly request this edit, often through imprecise prompt engineering or model fine-tuning. This approach of predefined controls or manual specifications restricts users from fully exploring the latent capabilities of the model. There may be interesting stylistic variations or attributes that the model can generate, but users have no easy way to discover or utilize these.

% Natural visual disentanglement was an emergent property in the latent space of Generative Adversarial Models (GANs) \cite{harkonen2020ganspace,radford2015unsupervised, wu2021stylespace, shen2020interfacegan}. In particular, it has been observed that StyleGAN~\cite{karras2019style} stylespace neurons offer detailed control over many meaningful aspects of images that would be difficult to describe in words~\cite{wu2021stylespace}. However, diffusion models do not share such a compact latent space~\cite{park2023unsupervised}; and efforts to uncover such a space in the semantic embeddings of the text conditioning have met with limited success \nik{Nick - is there a specific citation you were thinking about?}.

% In this work we introduce \textbf{SliderSpace}, which takes a step towards uncovering an analogous low dimensional representation of diffusion models' visual breadth; in essence treating the diffusion model as many generators sharing parameters, where a particular generator is defined by a specific prompt. For a given prompt we sample many random seeds (and optionally prompt expansions using an LLM), generate the corresponding images, and apply an off the shelf feature extractor (in this work CLIP, but our method can be applied to any differentiable feature extractor). We use PCA to analyze these features, and for each of the leading $k$ principal components we train a LoRA \cite{} which causes the diffusion model to produces images which increase the feature magnitude along that component when passed back through the same feature extractor. This leads to a 'Slider' for each principal component, because each LoRA can be scaled and applied to the original diffusion model, continuously varying those visual features in the generated results (as measured, in our case, by CLIP).

% There are many other works that enhance the controllability of diffusion models. One common approach is enabling users to add spatial constraints to a generation either manually, or via a reference image \cite{zhang2023addingconditionalcontroltexttoimage, chen2024trainingfreeregionalpromptingdiffusion}, a second is leveraging more abstract embeddings (e.g. identity, style) extracted from a reference image \cite{ye2023ipadaptertextcompatibleimage, hertz2024stylealignedimagegeneration, li2023photomaker, shi2024instantbooth}, a third is finetuning a foundation model to better generate a concept important to the user \cite{ruiz2022dreambooth, kumari2022customdiffusion, Ryu_lora, hu2021lora}, and a fourth (most relevant to this work) is finding low-rank adaptors of the model based on a prompt or small training set which can be scaled to provide continous control over one aspect of generated image (e.g. night vs day, basic vs luxury, etc.) \cite{gandikota2023concept}. SliderSpace is complementary to all of these methods and offers something distinct. All of the other methods we are aware require the user (and / or model designer) to know in advance what type of control they want. In contrast SliderSpace assists users in discovering and controlling hidden capabilities present in the diffusion model's distribution of possible generations.

%We propose that truly intuitive creative control in a text-to-image model should meet three key criteria: \emph{discoverability}, \emph{intuitiveness}, and \emph{specificity}. The model should reveal controllable attributes that may not be immediately obvious, offer controls that are easy to understand and manipulate, and ensure each control affects a distinct attribute of the generated image.

% We demonstrate the utility and power of SliderSpace using three applications built on top of SDXL-DMD \cite{dmd}, because its fast generation speed lends itself well to the continuous control offered by SliderSpace.

% First, we study concept decomposition (Section \ref{sec:concept_exp}), where we learn sliders for a specific concept (e.g. 'monster', 'waterfall', 'car'). Through quantitative metrics of diversity and text alignment we demonstrate that the learned sliders dramatically boost the diversity of generations when randomly applied without harming text alignment; we also ask humans to qualitatively judge these results in a user study where they find the SliderSpace results to be more 'Diverse', 'Useful', and 'Creative' than our baselines.

% Second, we attempt to compare the automatic discoveries of SliderSpace to a large scale manual study of artistic styles (Section \ref{sec:art_exp}), open-sourced by ParrotZone \cite{parrotzone}. In this study SDXL was prompted with over 4300 artist names,  and based on visual inspection the cases of successful stylistic mimicry recorded. Quantitatively SliderSpace more closely matches the distribution of artistic variation discovered by ParrotZone than other baselines, and in our user studies was judged to be significantly more 'Diverse' and 'Useful' than the baselines. To our surprise humans even judged SliderSpace results to be slightly more 'Diverse' than the results generated by the manually discovered artist names of \cite{parrotzone}.

% Third, we attempt to use SliderSpace to reverse the mode collapse commonly observed in distilled few-step diffusion models relative to the original teacher model (Section \ref{sec:diverse_exp}). We quantitatively demonstrate that applying SliderSpace to SDXL-DMD leads to more closely matching the distribution of images by the original teacher, SDXL.

%Through extensive experiments on various state-of-the-art text-to-image models, we demonstrate that SliderSpace significantly enhances user control and creative expression in AI-assisted image generation tasks. Our method enables a range of applications, including concept decomposition and control, diversity improvement in generated images, customization dissection and edits, and the exploration of artistic styles inherent in the model.

% SliderSpace goes beyond providing a practical tool for enhanced creative control. By mapping the visual potential of diffusion models it can open new avenues for generative creativity and deepens our understanding of each model's hidden potential.
We based our analyses on the labeled data created in previous work~\cite{sanei2023characterizing}. The dataset distinguished 305 usability issues from five popular OSS projects (Jupyter Lab,
Google Colab, CoCalc, VSCode, and Atom) and identified their posters. In this paper, we focus on individuals who have ever posted a usability issue in that dataset. 

\subsection{Discovering the Role of Issue Posters}\label{sec: Discovering_role}

To detect the background of the usability issue posters in the dataset, we checked each user's \textit{Profile page} on GitHub, examining their bios, shared personal websites, LinkedIn pages, and/or shared resumes. If they have not shared these information, we searched for their LinkedIn profiles using their full names to extract details on their backgrounds and expertise. We considered their job titles posted in the information acquired this way and categorized them into (1) UX professionals, (2) managers, (3) data scientists, and (4) developers. UX professionals were defined as those indicating positions such as \textit{UX designer} and \textit{user interface and user experience designer}.

Among the 224 usability issue posters in the dataset, we were able to identify the role of 180 users. Within those 180 users, 121 (67.2\%) were developers, 34 (18.9\%) identified as data scientists, 21 (11.7\%) held managerial positions, and only four (2.2\%) were UX professionals. The UX professionals included one male contributed to \textit{VSCode}, another male contributed to \textit{Atom}, and two involved in \textit{Jupyter Lab} project, one male and one female. Notably, there were no UX professionals involved in \textit{CoCalc} and \textit{Google Colab} projects in our data sample. For easier referencing, in the following we call the UX professionals of VSCode as \textit{VSCode\_pro}, Atom \textit{Atom\_pro}, male of Jupyter Lab as \textit{Jupyter\_pro\_M} and female as \textit{Jupyter\_pro\_F}.

\subsection{Characteristics of Issues Posted by UX Professionals (RQ1)}

Once we identified the roles of the usability issue posters, we extracted all the issues posted by the four UX professionals across the five OSS projects. Next, we analyzed the extracted issues by adopting the following steps. First, following the approach outlined in \cite{sanei2023characterizing}, we labeled each issue with either usability or non-usability; and for each usability issue, we identified the main \textit{usability dimension} touched by the issue using the ten Nielsen heuristics~\cite{nielsen2005ten}. Then, similar to \cite{sanei2021impacts}, we identified the specific \textit{sentiment} and \textit{tone} expressed by the UX professionals when posting the usability issues. In our study, the sentiment captures the valence of the emotion that includes three categories (positive, negative, and neutral), while the tone describes emotion with seven affective factors (excited, frustrated, impolite, polite, sad, satisfied, and sympathetic). Subsequently, we analyzed the \textit{argument structure} of the usability issues to better understand the discursive device that the issue posters adopted to convince other discussion participants. We particularly identified whether a \textit{claim} and a \textit{premise} appeared in a usability issue post, using criteria proposed in prior work~\cite{skitalinskaya_learning_2021, wachsmuth_argumentation_2017, dowden1993logical}. Statements were considered as claims if they explicitly indicate the position or stance of the issue posters to the discussed usability issues; and premise means that a statement contains reasoning, evidence, or examples that support a stance. We compared how the above characteristics (i.e., usability dimensions, sentiments, tones, and argument structures) differed in issues posted by UX professionals and those without UX expertise.

\subsection{UX Professionals' Purpose Following Up on Issues (RQ2)}

% After investigating how UX professionals posted the usability issues, we recognized the importance of understanding their participation afterwards, particularly in following up on the discussion threads of the issues they posted. 
Thus, we first isolated comments made by the UX professionals posted to the usability issues they created within the datasets. Then, we employed an inductive content analysis~\cite{wamboldt1992content, Hsieh2005} and categorized the various purposes behind their contributions in posting each comment. For our analysis, the \textit{purpose} specifies the distinct goal that a particular comment serves within the context of the discussion thread. The purpose of a comment may vary based on its content and the immediate objective of the issue posters to write in the discussion to address one specific comment posted by another contributor. We grouped the identified purposes into themes through an iterative approach conducted by the two authors.

\section{Experiments}
\label{sec:experiments}

\begin{figure*}[t]
\vspace{-6mm}
    \centering
    \includegraphics[width=0.8\linewidth]{figs/compare.pdf}
    \vspace{-4mm}
    \caption{\textbf{Qualitative comparison} with the baseline for generating a sequence of novel view images.  
    The results demonstrate that our method synthesizes more consistent multi-view images compared to our baseline model (Zero123). In addition, compared to SyncDreamer, our method visually maintains better similarity to the conditioned image and appears more natural.}
    \label{fig:sota_compare}
\vspace{-5mm}
\end{figure*}

\subsection{Experimental Setups}
\textbf{Dataset.}
Following previous work~\cite{zero123, SyncDreamer}, we evaluate our work on the Google Scanned Object (GSO)~\cite{GSO} dataset to verify the zero-shot novel view image synthesis capability. 
We also provide results for additional datasets in the Supplementary Material.
Specifically, we randomly select 30 objects from the GSO dataset with various object categories. 
Unlike recent approaches~\cite{mvdream, SyncDreamer} that aim to enhance the consistency of novel view synthesis models by generating multiple fixed-view images, our method can generate images from any camera pose and any number of views. Therefore, we conduct experiments under different camera pose settings to validate our approach:
specifically, 
1) \textit{16-views with free camera pose}: for each object, we circularly render 16 views with the elevation angles ranging in $[-10\degree, 40\degree]$ and the azimuth angles are evenly distributed in $[0\degree, 360\degree]$. 
2) \textit{16-views with fixed camera pose}: We maintain a constant elevation angle of $30\degree$ and uniformly sample azimuth angles (same as SyncDreamer~\cite{SyncDreamer}).
3) \textit{32-views with free camera pose}: Similar to the first setting, but we sample 32 views.
It's important to note that our method does not require additional training or fine-tuning on any datasets.

\noindent\textbf{Metrics.}
To validate the effectiveness of our method, we mainly evaluate it based on three criteria:
1) \textit{Quality Score}. We evaluate the image quality of synthesized multi-view images by measuring their similarity with ground truth images. Following prior research~\cite{zero123, sparsefusion}, we report the similarity between the synthesized images and the ground truth images with standard metrics: PSNR, SSIM~\cite{ssim}, and LPIPS~\cite{lpips}.
2) \textit{Multi-view Consistency Score}. As the primary goal of our work is to improve the consistency of generated images, we also employ the 3D consistency score~\cite{3dim} to verify the consistency among the synthesized images. Specifically, we train an Instant-NGP~\cite{instant_ngp} with the input image and part of the synthesized novel view images of our model and evaluate the similarity between the remaining synthesized images and the rendered images of Instant-NGP. For the synthesized multi-view images of each object, we allocate $3/4$ for training and reserve the remaining $1/4$ for validation.
Intuitively, if the consistency of synthesized images is improved, the NeRF-like model will train a better object representation, and the re-rendered images will agree more with the validation images.
3) \textit{Input Consistency Score}. To assess the faithfulness of synthesized images in preserving the identity of the input condition image, we introduce the input consistency score. This score calculates the similarity of each synthesized image with the input condition image, utilizing the LPIPS metric.

In addition, we use synthesized multi-view images to train a neural 3D reconstruction model (NeuS~\cite{neus}) and report commonly used Chamfer Distances (CD) and Volume IoUs between the trained 3D model and the ground truth.

\noindent\textbf{Baselines.}
Given that our main goal is to improve the consistency of the trained baseline model without further fine-tuning, we mainly compare our approach with the used baseline model Zero123~\cite{zero123}. Additionally, we compare our method to the SOTA approaches such as PGD~\cite{tseng2023consistent} and SyncDreamer~\cite{SyncDreamer} using the same Zero123 base model.

\noindent\textbf{Implementation Details.}
We use the official checkpoint provided by Zero123~\cite{zero123}, which is trained on objaverse~\cite{objaverse} for 165,000 steps. We inject our epipolar attention layer after step $T=4$ and layer $L=10$ by default. We find that feature fusion weight $\alpha=0.5$, and the number of context views $M=2$ work better.

\begin{table}[t]
\centering
\caption{Comparison of multi-view consistency, image quality, and input consistency of synthesized multi-view images at the 16-view setting with free camera pose.}
\label{tab:view16_free_compare}
\vspace{-2mm}
\scalebox{0.6}{
\begin{tabular}{c ccc ccc c}
\toprule
              & \multicolumn{3}{c}{Multi-view Consistency} & \multicolumn{3}{c}{Quality Score} & \multicolumn{1}{c}{Input Consis.} \\
              \cmidrule(lr){2-4} \cmidrule(lr){5-7} \cmidrule(lr){8-8}
              & PSNR$\uparrow$  & SSIM$\uparrow$ & LPIPS$\downarrow$ 
              & PSNR$\uparrow$  & SSIM$\uparrow$ & LPIPS$\downarrow$ 
              & LPIPS$\downarrow$ 
              \\ \midrule

Zero123
& 15.225        & 0.645       & 0.408
& 14.255        & 0.747       &	0.208
& 0.303         
\\
SyncDreamer
& 14.830        & 0.626       & 0.434
& 12.650        & 0.713       &	0.254
& 0.317         
\\
Ours 
& \best{18.300}	& \best{0.734}	& \best{0.355}
& \best{14.947}	& \best{0.763}	& \best{0.191}
& \best{0.282}
\\

\bottomrule
\end{tabular}
}
\end{table}

\begin{table}[t]
\vspace{-1mm}
\centering
\caption{Comparison of multi-view consistency, image quality, and input consistency at the 16-view setting with fixed camera pose as SyncDreamer~\cite{SyncDreamer}.}
\label{tab:view16_fxied_compare}
\vspace{-3mm}
\scalebox{0.6}{
\begin{tabular}{c ccc ccc c}
\toprule
              & \multicolumn{3}{c}{Multi-view Consistency} & \multicolumn{3}{c}{Quality Score} & \multicolumn{1}{c}{Input Consis.} \\
              \cmidrule(lr){2-4} \cmidrule(lr){5-7} \cmidrule(lr){8-8}
              & PSNR$\uparrow$  & SSIM$\uparrow$ & LPIPS$\downarrow$ 
              & PSNR$\uparrow$  & SSIM$\uparrow$ & LPIPS$\downarrow$ 
              & LPIPS$\downarrow$ 
              \\ \midrule

Zero123
& 16.556        & 0.682       & 0.378
& 14.592        & 0.750       &	0.207
& 0.305         
\\
SyncDreamer
& \best{22.424}        & \best{0.812}       & \best{0.268}
& 15.269        & 0.749       &	0.196
& 0.300         
\\
Ours 
& 21.151	& 0.780	& 0.302
& \best{15.293}	& \best{0.764}	& \best{0.184}
& \best{0.287}
\\

\bottomrule
\end{tabular}
}
\vspace{-4mm}
\end{table}


\subsection{Comparison With Baseline Models}
The quantitative comparison on three settings are shown in Tab.~\ref{tab:view16_free_compare}, Tab.~\ref{tab:view16_fxied_compare}, and Tab.~\ref{tab:view32_free_compare}. The qualitative comparison is shown in Fig.~\ref{fig:sota_compare}.

\begin{table}[t]
\centering
\caption{Comparison of multi-view consistency and image quality scores of synthesized multi-view images at the 32-view setting with free camera pose.}
\vspace{-3mm}
\label{tab:view32_free_compare}
\scalebox{0.7}{
\begin{tabular}{c ccc ccc}
\toprule
              & \multicolumn{3}{c}{Multi-view Consistency} & \multicolumn{3}{c}{Quality Score} \\
              \cmidrule(lr){2-4} \cmidrule(lr){5-7}
              & PSNR$\uparrow$  & SSIM$\uparrow$ & LPIPS$\downarrow$ 
              & PSNR$\uparrow$  & SSIM$\uparrow$ & LPIPS$\downarrow$ 
              \\ \midrule

Zero123
& 16.515        & 0.694       & 0.378
& 15.142        & 0.733       &	0.211
\\
PGD~\cite{tseng2023consistent}
& 18.481        & 0.720       & 0.343
& 15.281        & 0.739       &	0.205
\\
Ours 
& \best{20.655}	& \best{0.792}	& \best{0.305}
& \best{15.268}	& \best{0.742}	& \best{0.203}
\\

\bottomrule
\end{tabular}
}
\vspace{-3mm}
\end{table}

\begin{table*}
  [t]
  \centering
  \resizebox{\textwidth}{!}{%
  \begin{tabular}{cccccccccccc}
    \toprule \multicolumn{2}{c}{Components}                                                             & \multicolumn{5}{c}{Re-executability Rate (\%)} & \multicolumn{5}{c}{Readability (\#)} \\
    \cmidrule(lr){1-2} \cmidrule(lr){3-7} \cmidrule(lr){8-12}        \hspace{8pt}\labelemoji\hspace{8pt}                                                                & \hspace{8pt}\toolemoji\hspace{8pt}                                      & O0                                 & O1             & O2             & O3             & AVG            & O0             & O1             & O2             & O3             & AVG            \\
    \hline
    \rowcolor[rgb]{0.93,0.93,0.93}\multicolumn{12}{c}{\textbf{Initialize with LLM4Decompile-End-6.7B~\citep{llm4decompile}}}   \\
    \xmark                                                                                              & \xmark                                    & 69.51                              & 46.95          & 50.61          & 46.34          & 53.35          & 3.98 & 3.41 & 3.44 & 3.38 & 3.55 \\
    \cmark                                                                                              & \xmark                                    & 75.61                              & 50.61          & 50.00          & 50.00          & 56.55          & 4.01 & 3.44 & 3.39 & \textbf{3.49} & 3.58 \\
    \xmark                                                                                              & \cmark                                    & 83.54                     & \textbf{56.10}          & 51.22          & 50.61 & 60.37 & 4.05 & 3.51 & 3.51 & 3.42 & 3.62 \\
    \cmark                                                                                              & \cmark                                    & \textbf{85.37}                            & \textbf{56.10}                     & \textbf{51.83} & \textbf{52.43}          & \textbf{61.43} & \textbf{4.13} & \textbf{3.60} & \textbf{3.54} & \textbf{3.49} & \textbf{3.69} \\

    \rowcolor[rgb]{0.93,0.93,0.93}\multicolumn{12}{c}{\textbf{Initialize with Deepseek-Coder-6.7B-base~\citep{deepseekcoder}}} \\
    \xmark                                                                                              & \xmark                                    & 59.15                              & 35.98          & 39.02          & 37.80          & 42.99          & 3.71 & 3.05 & 3.16 & 3.05 & 3.24 \\
    \cmark                                                                                              & \xmark                                    & 66.46                              & 41.46          & 38.41          & 36.59          & 45.73          & 3.76 & 3.17 & \textbf{3.21} & 3.08 & 3.31 \\
    \xmark                                                                                              & \cmark                                    & 70.73                              & 39.63          & 39.02          & 40.24          & 47.41          & 3.90 & 3.17 & 3.08 & 3.11 & 3.31 \\
    \cmark                                                                                              & \cmark                                    & \textbf{79.88}                     & \textbf{45.73} & \textbf{43.90} & \textbf{42.68} & \textbf{53.05} & \textbf{3.96} & \textbf{3.21} & 3.18 & \textbf{3.19} & \textbf{3.38} \\
    \bottomrule
  \end{tabular}%
  }
  \caption{The ablation study of different methods across four optimization levels
  (O0, O1, O2, O3), as well as their average scores (AVG). The results in bold represent the optimal performance. The ~\labelemoji~ and ~\toolemoji~ means Relabedling and Function Call. \textbf{Bold} denotes the best performance.}
  \label{tab:ablation}
\end{table*}



\begin{figure*}[ht]
    \centering
    \begin{minipage}{0.65\textwidth}
        \centering
        \includegraphics[width=0.95\linewidth]{figs/ablation.pdf}
        \vspace{-2mm}
        \captionof{figure}{Qualitative Comparison for different design choices. Our method, employing multi-view epipolar attention, demonstrates the best consistency.}
        \label{fig:ablation}
    \end{minipage}\hfill
    \begin{minipage}{0.33\textwidth}
        \centering
        \includegraphics[width=0.8\linewidth]{figs/neus_ver.pdf}
        \vspace{-3mm}
        \caption{Our method shows better direct 3D reconstruction~\cite{neus}.}
        \label{fig:neus}
    \end{minipage}
    \vspace{-5mm}
\end{figure*}

\noindent\textbf{Multi-view Consistency.}
Tab.~\ref{tab:view16_fxied_compare} presents the 3D consistency scores compared to our baseline model (Zero123) and SyncDreamer. The results indicate a significant improvement across all three metrics achieved by our method when compared with Zero123.
While our method exhibits a marginally lower numerical consistency score compared to SyncDreamer, it enables the synthesis of images with arbitrary camera poses.	
This capability is illustrated in Tab.~\ref{tab:view16_free_compare}, where our method consistently enhances consistency with changes in camera pose settings, whereas SyncDreamer fails to do so and exhibits inferior results compared to Zero123.
Furthermore, our method facilitates the synthesis of multi-view images with any number of camera views. This versatility is demonstrated in Tab.~\ref{tab:view32_free_compare}, where our method continues to achieve significant improvements in consistency scores, while SyncDreamer is unable to operate under such conditions.	

Meanwhile, Fig.~\ref{fig:sota_compare} provides a qualitative comparison with the baseline. While both our method and SyncDreamer enhance consistency, our method visually preserves better similarity to the input image, including color and texture details. The input consistency score further corroborates this.

\noindent\textbf{Image Quality.}
While our primary goal centers around enhancing the consistency of synthesized multi-view images, we also evaluate the image quality by comparing the similarity with the ground truth images. The results shown in Tab.~\ref{tab:view16_free_compare}, Tab.~\ref{tab:view16_fxied_compare}, and Tab.~\ref{tab:view32_free_compare} indicate that our method also enhances the image quality under different settings besides improving the consistency.
Moreover, our method shows better image quality compared with SyncDreamer even in the 16-view setting with fixed camera pose.

\noindent\textbf{Input Consistency.}
Input consistency terms whether the results align with the input image.
Fig.~\ref{fig:sota_compare} illustrates that both our method and SyncDreamer enhance multi-view consistency. However, the color and texture details of SyncDreamer's results diverge from the input image and appear visually unnatural.
This discrepancy is evident in the input consistency score presented in Tab.~\ref{tab:view16_fxied_compare}, indicating lower similarity with the condition image in the SyncDreamer results.	

\subsection{Ablation Study}
The overall quantitative results are shown in Tab.~\ref{tab:ablation}, and the qualitative comparisons are shown in Fig.~\ref{fig:ablation}.

\noindent \textbf{Full Attention \vs Epipolar Attention.}
The results presented in Tab.\ref{tab:ablation} and Fig.\ref{fig:ablation} demonstrate that our epipolar attention mechanism can synthesize more consistent multi-view images compared with full attention. Furthermore, our epipolar attention achieves a greater performance improvement compared to full attention when using multiple reference images. This could be attributed to the fact that our epipolar attention more effectively localizes target information, as depicted in Fig.~\ref{fig:full_attn_compare}, thereby reducing noise from the reference images. In the multi-view setting, where multiple reference images are utilized, this noise reduction becomes particularly crucial.
Moreover, it is noteworthy that the epipolar attention mechanism consumes less GPU memory compared to our baseline, as discussed in Sec.~\ref{sec:attn_analysis}.

\noindent \textbf{Attending Single-View \vs Multi-View.}
Applying the epipolar attention significantly improves the consistency between the input and target views. However, the consistency between different views in the unobserved regions of the input view is not well preserved.
After implementing our epipolar attention in the multi-view setting, the consistency across the generated multi-view images is further improved. The last row in Tab.~\ref{tab:ablation} shows that after applying our multi-view epipolar attention, the consistency score is further improved compared with the single-view setting. Besides, the qualitative result in Fig.~\ref{fig:ablation} also shows better consistency among different target views.



\begin{table}[t]
\centering
\vspace{-1mm}
\caption{Comparison of 3D reconstruction results. Our method significantly improves the reconstruction quality.}
\vspace{-3mm}
\label{tab:neus}
\scalebox{0.7}{
\begin{tabular}{c cc}
\toprule
              &  Chamfer Dist.$\downarrow$  & Volume IoU$\uparrow$
\\ \midrule

            Zero123         & 0.017         & 0.819    \\
            SyncDreamer     & \best{0.013}         & \best{0.847}    \\
            Ours            & 0.014	& 0.842 \\

\bottomrule
\end{tabular}
}
\vspace{-5mm}
\end{table}


\vspace{-2mm}
\subsection{Downstream Application}
\vspace{-2mm}
To demonstrate the effectiveness of our method, we also applied it to the downstream 3D reconstruction task. Specifically, we trained the NeuS model~\cite{neus} directly using images synthesized by our method, Zero123, and SyncDreamer, respectively.
The quantitative results in Tab.~\ref{tab:neus} show that the consistent multi-view images synthesized by our method can significantly improve the 3D reconstruction quality.
Additionally, our method exhibits similar performance to SyncDreamer which requires time-consuming re-training.
The qualitative results in Fig.~\ref{fig:neus} show that it is challenging to train the NeuS model directly due to the lack of consistency in the images generated by Zero123. In contrast, our method generates more consistent multi-view images and, therefore, better reconstructs the geometry and texture details.
We show improvements on other downstream applications such as image-to-3D in the Supplementary Material.


\section*{Conclusion}
This paper aims to enhance our understanding of the computational complexity of computing various Shapley value variants. We found that for various ML models --- including decision trees, regression tree ensembles, weighted automata, and linear regression --- both local and global interventional and baseline SHAP can be computed in polynomial time under HMM modeled distributions. This extends popular algorithms, such as TreeSHAP, beyond their empirical distributional scope. We also establish strict complexity gaps between the various SHAP variants (baseline, interventional, and conditional) and prove the intractability of computing SHAP for tree ensembles and neural networks in simplified scenarios. Overall, we present SHAP as a versatile framework whose complexity depends on four key factors: \begin{inparaenum}[(i)] \item model type, \item SHAP variant, \item distribution modeling approach, \item and local vs. global explanations\end{inparaenum}. We believe this perspective provides deeper insight into the computational complexity of SHAP, paving the way for future work.




%We believe that our framework provides a more intricate understanding of SHAP computation complexity across different models, distributions, and variants, paving the way for further research.

Our work opens promising directions for future research. First, expanding our computational analysis to other SHAP-related metrics, such as asymmetric SHAP~\citep{frye20} and SAGE~\citep{covert2020understanding}, would be valuable. Additionally, we aim to explore more expressive distribution classes and relaxed assumptions beyond those in Section \ref{sec:tractable} while maintaining tractable SHAP computation. Finally, when exact computation is intractable (Section \ref{sec:intractable}), investigating the approximability of SHAP metrics through approximation and parameterized complexity theory~\citep{downey2012parameterized} is an important direction.

%Our work opens several promising avenues for future research on the computational properties of explainable AI methods, with a particular focus on SHAP. First, it would be interesting to broaden the computational analysis conducted in this work to include other popular SHAP-related metrics in the literature, such as asymmetric SHAP \cite{frye20} and SAGE \cite{covert2020understanding}. Also, in the future, we aim to explore more expressive distribution classes and relaxed distributional assumptions—extending beyond those examined in Section \ref{sec:tractable} —that still yield tractable SHAP computation. Finally, when exact computation proves intractable (Section \ref{sec:intractable}), it is worthwhile to theoretically investigate the question of the approximability of computing the SHAP metrics across various configurations, through the lens of approximation and parametrized complexity theory \cite{arora2009computational}.

%This paper aims to deepen our understanding of the computational complexity involved in obtaining different Shapley value variants. We found that for a variety of ML models, including decision trees, tree ensembles for regression, weighted automata, and linear regression models — computing both local and global interventional and baseline SHAP can be done in polynomial time when distributions are modeled by HMMs. This extends the distributional scope of popular algorithms like TreeSHAP, which is limited to empirical distributions. Additionally, we demonstrate a strict complexity gap between SHAP variants, showing that interventional and baseline SHAP can be strictly easier to compute than conditional SHAP. Despite these positive results, we uncovered intractability for various SHAP variants in neural networks and tree ensembles. Finally, we provided generalized complexity relations across SHAP variants. We believe that our framework offers a deeper understanding of the complexity involved in computing SHAP across various variants, models, distributions, as well as in both local and global computations, laying the groundwork for future research.


%%
%% The next two lines define the bibliography style to be used, and
%% the bibliography file.
\bibliographystyle{ACM-Reference-Format}
\bibliography{main}


%%
%% If your work has an appendix, this is the place to put it.
\appendix
\onecolumn
\section{More results about Mode connectivity in GNN}
\label{app: A}

\subsection{The performance of linear interpolations 
between two minima.}

\begin{figure*}[!ht]
    \centering
    \begin{subfigure}[b]{0.24\textwidth}
        \includegraphics[width=1.0\textwidth]{fig/init/init/cora_gcn_plot.png}
        \caption{Cora}
    \end{subfigure}
    \begin{subfigure}[b]{0.24\textwidth}
        \includegraphics[width=1.0\textwidth]{fig/init/init/citeseer_gcn_plot.png}
        \caption{CiteSeer}
    \end{subfigure}
    \begin{subfigure}[b]{0.24\textwidth}
        \includegraphics[width=1.0\textwidth]{fig/init/init/pubmed_gcn_plot.png}
        \caption{PubMed}
    \end{subfigure}
    \begin{subfigure}[b]{0.24\textwidth}
        \includegraphics[width=1.0\textwidth]{fig/init/init/amazon-computer_gcn_plot.png}
        \caption{Amazon-Computer}
    \end{subfigure}
    \begin{subfigure}[b]{0.24\textwidth}
        \includegraphics[width=1.0\textwidth]{fig/init/init/amazon-photo_gcn_plot.png}
        \caption{Amazon-Photo}
    \end{subfigure}
    \begin{subfigure}[b]{0.24\textwidth}
        \includegraphics[width=1.0\textwidth]{fig/init/init/coauthor-cs_gcn_plot.png}
        \caption{Coauthor-CS}
    \end{subfigure}
    \begin{subfigure}[b]{0.24\textwidth}
        \includegraphics[width=1.0\textwidth]{fig/init/init/coauthor-physics_gcn_plot.png}
        \caption{Coauthor-Physics}
    \end{subfigure}
    \begin{subfigure}[b]{0.24\textwidth}
        \includegraphics[width=1.0\textwidth]{fig/init/init/wikics_gcn_plot.png}
        \caption{WikiCS}
    \end{subfigure}
    \begin{subfigure}[b]{0.24\textwidth}
        \includegraphics[width=1.0\textwidth]{fig/init/init/squirrel_gcn_plot.png}
        \caption{Squirrel}
    \end{subfigure}
    \begin{subfigure}[b]{0.24\textwidth}
        \includegraphics[width=1.0\textwidth]{fig/init/init/chameleon_gcn_plot.png}
        \caption{Chameleon}
    \end{subfigure}
    \begin{subfigure}[b]{0.24\textwidth}
        \includegraphics[width=1.0\textwidth]{fig/init/init/roman-empire_gcn_plot.png}
        \caption{Roman-Empire}
    \end{subfigure}
    \begin{subfigure}[b]{0.24\textwidth}
        \includegraphics[width=1.0\textwidth]{fig/init/init/amazon-ratings_gcn_plot.png}
        \caption{Amazon-Ratings }
    \end{subfigure}
        \begin{subfigure}[b]{0.24\textwidth}
        \includegraphics[width=1.0\textwidth]{fig/init/init/minesweeper_gcn_plot.png}
        \caption{Minesweeper}
    \end{subfigure}
    \caption{The performance of interpolations along a non-linear path connecting two minima.}
    \label{fig:A1}
\end{figure*}

\clearpage



\subsection{ The performance of interpolations along quadratic Bézier curve connecting two minima}
\begin{figure*}[!ht]
    \centering
    \begin{subfigure}[b]{0.24\textwidth}
        \includegraphics[width=1.0\textwidth]{fig/Bezier/Bezier/cora_bezier_fitted_plot.png}
        \caption{Cora}
    \end{subfigure}
    \begin{subfigure}[b]{0.24\textwidth}
        \includegraphics[width=1.0\textwidth]{fig/Bezier/Bezier/citeseer_bezier_fitted_plot.png}
        \caption{CiteSeer}
    \end{subfigure}
    \begin{subfigure}[b]{0.24\textwidth}
        \includegraphics[width=1.0\textwidth]{fig/Bezier/Bezier/pubmed_bezier_fitted_plot.png}
        \caption{PubMed}
    \end{subfigure}
    \begin{subfigure}[b]{0.24\textwidth}
        \includegraphics[width=1.0\textwidth]{fig/Bezier/Bezier/amazon-computer_bezier_fitted_plot.png}
        \caption{Amazon-Computer}
    \end{subfigure}
    \begin{subfigure}[b]{0.24\textwidth}
        \includegraphics[width=1.0\textwidth]{fig/Bezier/Bezier/amazon-photo_bezier_fitted_plot.png}
        \caption{Amazon-Photo}
    \end{subfigure}
    \begin{subfigure}[b]{0.24\textwidth}
        \includegraphics[width=1.0\textwidth]{fig/Bezier/Bezier/coauthor-cs_bezier_fitted_plot.png}
        \caption{Coauthor-CS}
    \end{subfigure}
    \begin{subfigure}[b]{0.24\textwidth}
        \includegraphics[width=1.0\textwidth]{fig/Bezier/Bezier/coauthor-physics_bezier_fitted_plot.png}
        \caption{Coauthor-Physics}
    \end{subfigure}
    \begin{subfigure}[b]{0.24\textwidth}
        \includegraphics[width=1.0\textwidth]{fig/Bezier/Bezier/wikics_bezier_fitted_plot.png}
        \caption{WikiCS}
    \end{subfigure}
    \begin{subfigure}[b]{0.24\textwidth}
        \includegraphics[width=1.0\textwidth]{fig/Bezier/Bezier/squirrel_bezier_fitted_plot.png}
        \caption{Squirrel}
    \end{subfigure}
    \begin{subfigure}[b]{0.24\textwidth}
        \includegraphics[width=1.0\textwidth]{fig/Bezier/Bezier/chameleon_bezier_fitted_plot.png}
        \caption{Chameleon}
    \end{subfigure}
    \begin{subfigure}[b]{0.24\textwidth}
        \includegraphics[width=1.0\textwidth]{fig/Bezier/Bezier/roman-empire_bezier_fitted_plot.png}
        \caption{Roman-Empire}
    \end{subfigure}
    \begin{subfigure}[b]{0.24\textwidth}
        \includegraphics[width=1.0\textwidth]{fig/Bezier/Bezier/amazon-ratings_bezier_fitted_plot.png}
        \caption{Amazon-Ratings }
    \end{subfigure}
        \begin{subfigure}[b]{0.24\textwidth}
        \includegraphics[width=1.0\textwidth]{fig/Bezier/Bezier/minesweeper_bezier_fitted_plot.png}
        \caption{Minesweeper}
    \end{subfigure}
    \caption{The performance of interpolations along a non-linear path connecting two minima.}
    \label{fig:A2}
\end{figure*}
\clearpage



\subsection{Effect of convolution mechanism on mode connectivity}
\label{app: conv}
\begin{figure*}[!ht]
    \centering
    \begin{subfigure}[b]{0.24\textwidth}
        \includegraphics[width=1.0\textwidth]{fig/init_model/cora_comparison.png}
        \caption{Cora}
    \end{subfigure}
    \begin{subfigure}[b]{0.24\textwidth}
        \includegraphics[width=1.0\textwidth]{fig/init_model/citeseer_comparison.png}
        \caption{CiteSeer}
    \end{subfigure}
    \begin{subfigure}[b]{0.24\textwidth}
        \includegraphics[width=1.0\textwidth]{fig/init_model/pubmed_comparison.png}
        \caption{PubMed}
    \end{subfigure}
    \begin{subfigure}[b]{0.24\textwidth}
        \includegraphics[width=1.0\textwidth]{fig/init_model/amazon-computer_comparison.png}
        \caption{Amazon-Computer}
    \end{subfigure}
    \begin{subfigure}[b]{0.24\textwidth}
        \includegraphics[width=1.0\textwidth]{fig/init_model/amazon-photo_comparison.png}
        \caption{Amazon-Photo}
    \end{subfigure}
    \begin{subfigure}[b]{0.24\textwidth}
        \includegraphics[width=1.0\textwidth]{fig/init_model/coauthor-cs_comparison.png}
        \caption{Coauthor-CS}
    \end{subfigure}
    \begin{subfigure}[b]{0.24\textwidth}
        \includegraphics[width=1.0\textwidth]{fig/init_model/coauthor-physics_comparison.png}
        \caption{Coauthor-Physics}
    \end{subfigure}
    \begin{subfigure}[b]{0.24\textwidth}
        \includegraphics[width=1.0\textwidth]{fig/init_model/wikics_comparison.png}
        \caption{WikiCS}
    \end{subfigure}
    \begin{subfigure}[b]{0.24\textwidth}
        \includegraphics[width=1.0\textwidth]{fig/init_model/squirrel_comparison.png}
        \caption{Squirrel}
    \end{subfigure}
    \begin{subfigure}[b]{0.24\textwidth}
        \includegraphics[width=1.0\textwidth]{fig/init_model/chameleon_comparison.png}
        \caption{Chameleon}
    \end{subfigure}
    \begin{subfigure}[b]{0.24\textwidth}
        \includegraphics[width=1.0\textwidth]{fig/init_model/roman-empire_comparison.png}
        \caption{Roman-Empire}
    \end{subfigure}
    \begin{subfigure}[b]{0.24\textwidth}
        \includegraphics[width=1.0\textwidth]{fig/init_model/amazon-ratings_comparison.png}
        \caption{Amazon-Ratings }
    \end{subfigure}
        \begin{subfigure}[b]{0.24\textwidth}
        \includegraphics[width=1.0\textwidth]{fig/init_model/minesweeper_comparison.png}
        \caption{Minesweeper}
    \end{subfigure}
            \begin{subfigure}[b]{0.24\textwidth}
        \includegraphics[width=1.0\textwidth]{fig/init_model/legend.png}
        \caption{Legend}
    \end{subfigure}
    \caption{Performance of mode connectivity on different convolution mechanisms.}
    \label{fig:A3}
\end{figure*}

\clearpage
\subsection{Visualization of Loss basin and minimas }
\begin{figure*}[!ht]
    \centering
    \begin{subfigure}[b]{0.24\textwidth}
        \includegraphics[width=1.0\textwidth]{fig/basin/cora_basin.png}
        \caption{Cora}
    \end{subfigure}
    \begin{subfigure}[b]{0.24\textwidth}
        \includegraphics[width=1.0\textwidth]{fig/basin/Basin/loss_landscape_fixed_citeseer.png}
        \caption{CiteSeer}
    \end{subfigure}
    \begin{subfigure}[b]{0.24\textwidth}
        \includegraphics[width=1.0\textwidth]{fig/basin/Basin/loss_landscape_fixed_pubmed.png}
        \caption{PubMed}
    \end{subfigure}
    \begin{subfigure}[b]{0.24\textwidth}
        \includegraphics[width=1.0\textwidth]{fig/basin/Basin/loss_landscape_fixed_amazon-computer.png}
        \caption{Amazon-Computer}
    \end{subfigure}
    \begin{subfigure}[b]{0.24\textwidth}
        \includegraphics[width=1.0\textwidth]{fig/basin/Basin/loss_landscape_fixed_amazon-photo.png}
        \caption{Amazon-Photo}
    \end{subfigure}
    \begin{subfigure}[b]{0.24\textwidth}
        \includegraphics[width=1.0\textwidth]{fig/basin/Basin/loss_landscape_fixed_coauthor-cs.png}
        \caption{Coauthor-CS}
    \end{subfigure}
    \begin{subfigure}[b]{0.24\textwidth}
        \includegraphics[width=1.0\textwidth]{fig/basin/Basin/loss_landscape_fixed_coauthor-physics.png}
        \caption{Coauthor-Physics}
    \end{subfigure}
    \begin{subfigure}[b]{0.24\textwidth}
        \includegraphics[width=1.0\textwidth]{fig/basin/Basin/loss_landscape_fixed_wikics.png}
        \caption{WikiCS}
    \end{subfigure}
    \begin{subfigure}[b]{0.24\textwidth}
        \includegraphics[width=1.0\textwidth]{fig/basin/Basin/loss_landscape_fixed_squirrel.png}
        \caption{Squirrel}
    \end{subfigure}
    \begin{subfigure}[b]{0.24\textwidth}
        \includegraphics[width=1.0\textwidth]{fig/basin/Basin/loss_landscape_fixed_chameleon.png}
        \caption{Chameleon}
    \end{subfigure}
    \begin{subfigure}[b]{0.24\textwidth}
        \includegraphics[width=1.0\textwidth]{fig/basin/Basin/loss_landscape_fixed_roman-empire.png}
        \caption{Roman-Empire}
    \end{subfigure}
    \begin{subfigure}[b]{0.24\textwidth}
        \includegraphics[width=1.0\textwidth]{fig/basin/Basin/loss_landscape_fixed_amazon-ratings.png}
        \caption{Amazon-Ratings }
    \end{subfigure}
        \begin{subfigure}[b]{0.24\textwidth}
        \includegraphics[width=1.0\textwidth]{fig/basin/Basin/loss_landscape_fixed_minesweeper.png}
        \caption{Minesweeper}
    \end{subfigure}
    \caption{Performance of mode connectivity on different convolution mechanisms.}
    \label{fig:A4}
\end{figure*}


% 
\subsection{Proof for Satisfaction of Marginal Constraints.}
% In this section, we will first show that our procedure satisfying the marginal conditions for our coupling $q(\rvx_0, \rvx_1)$:
% \begin{equation}
%     \int q(\rvx_0, \rvx_1) d\rvx_1 = q_0(\rvx_0), \int q(\rvx_0, \rvx_1) d\rvx_0 = q_1(\rvx_1).
% \end{equation}

% \begin{itemize}
%     \item For independent couple $q(x_0) = \int q(\mathcal{S}) \int q(x_1 | \mathcal{S}) q(x_0) dx_0 d_\mathcal{S}$ and $q(x_1) = \int q(\mathcal{S}) \int q(x_1 | \mathcal{S}) q(x_0) dx_1 d_\mathcal{S}$, we just need to show $\int q(x_0, x_1 | \mathcal{S}) dx_0 = q(x_1 | \mathcal{S})$ and $\int q(x_0, x_1 | \mathcal{S}) dx_1 = q(x_0)$.
%     \item $q(x_0, x_1 | \mathcal{S})$ is independent, so we can decompose it into $\prod q(x_0^i, x_1^i | \mathcal{S})$.
%     \item we can show $\int q(x_0^i, x_1^i | \mathcal{S}) dx_0 = q(x_1^i | \mathcal{S})$ and $\int q(x_0^i, x_1^i | \mathcal{S}) dx_0 = q(x_1^i)$
%     \item $q(x_0| \mathcal{S})$ and $q(x_1| \mathcal{S})$ are independent, so we can decompose it into $\prod q(x_1^i | \mathcal{S})$ and $\prod q(x_0^i)$.
%     \item The first part is done.
%     \item The second part is to show adding noise will not affect $q(x_0^i)$

% \end{itemize}

% In particular, the proof will be divided into four parts.
% %
% First, we will introduce the main theorem to apply to obtain the results, and show the random subsampling of a Dense Gaussian noise will converge to Gaussian distribution if the sample superset is large enough.
% %
% Second, by a proper construction, we can show that subsampling of a dense point superset can converge to direct subsampling of the surfaces when the size of the superset is also large enough.
% %
% Third, by considering our random subsampling procedure, we can show that our sampling is still random subsampling for Gaussian noise superset and point superset.
% %
% Lastly, we show that even introduction of the coupling interpolation, the results mariginal remain the same due to careful considerations.

\newtheorem{proposition}{Proposition}
\newtheorem{lemma}{Lemma}
\subsubsection{Law of Large Numbers}


\begin{proposition}\label{prop:large_samples}
Given $(X_1, \cdots, X_n)$, which are independently and identically distributed (IID) real $d$-diemsnion random variables, following a probability distribution $p(X)$,~\ie, $X_i \sim p(X), X \in \mathbb{R}^d$.
%
We have an additional random variable $Y$ that is random uniform sample of these variables,~\ie, $P(Y = X_i) = \frac{1}{n}$.
%
The cumulative distribution function (CDF) $\bar{F}(t)$ of random variable $Y$ will converge to the $F(X)$,~\ie, CDF of $X$.
\end{proposition}



% Assume $(X_1, \cdots, X_n)$ are independently and identically distributed (IID) real $d$-diemsnion random variables following a probability distribution $p(X)$, \ie, $X_i  \sim p(X), X \in \mathbb{R}^d$.
% %
% We also denote the cumulative distribution function of $p(X)$ to be $F(x)$.
%
Proof:
We first define an empirical cumulative distribution function $\hat{F}_n(X)$ over the random variables $(X_1, \cdots, X_n)$:
\begin{equation}
    \hat{F}_n (t) = \frac{1}{n} \sum_{i=1}^{n} \mathbf{1}_{X_i \leq t},
\end{equation}
where $\mathbf{1}_{X_i \leq t}$ is an indicator for $X_i^d \leq t^d$ for all dimensions $\{1, \cdots, d\}$.

The Glivenko–Cantelli theorem states that this empirical distribution function $\hat{F}_n(X)$ will converge to the cumulative distribution $F(X)$ if $n$ is sufficiently large:
\begin{equation}
    \textbf{sup}_{t \in \mathbb{R}^d} | \hat{F}_n(t) - F(t) | \rightarrow 0.
\end{equation}

If we have an additional random variable $Y$ that its value is a random subsample of the variables $(X_1, \cdots, X_n)$:
\begin{equation}
    P(Y = X_i) = \frac{1}{n}, \forall i = 1, 2, \cdots, n.
\end{equation}

The CDF of this variable $\bar{F}(t)$ is:
\begin{equation}
    \bar{F}(t) = P(Y \leq t) = \sum_{i=1}^{n} P(Y = X_i) \cdot \mathbf{1}_{X_i \leq t} = \frac{1}{n} \sum_{i=1}^{n} \mathbf{1}_{X_i \leq t} = \hat{F}_n(t).
\end{equation}
Therefore, the CDF of $Y$ also converges to the original underlying CDF $F(t)$ if $n$ is sufficiently large.

\begin{proposition}\label{prop:ot}
Assume we have $n$ random samples $(X_1, \cdots, X_n) \sim p_1$, and another $n$ random samples $(Y_1, \cdots, Y_n) \sim p_2$, and we are also given an arbitrary bijective map between random variables, \ie, $\Pi: \{1, \cdots, n\} \leftrightarrow \{1, \cdots, n\}$.
%
If we construct a new random variable $Z : (X, Y)$ follows the following couplings:

\[
    P(X = X_i, Y = Y_j) =
    \begin{cases}
    \frac{1}{n}, & \text{if } j = \Pi(i) \\ 
        0, & \text{else } j \neq \Pi(i);
    \end{cases}
\]

The CDF of the marginal $P(X)$ will converge the CDF of $p_1$, while the CDF of the marginal $P(Y)$ will converge to the CDF of $p_2$.
\end{proposition}

Proof:
Since $\Pi$ is bijective, we can compute the marginal $P(X = X_i)$ directly:
\begin{equation}
    \begin{split}
            P(X = X_i) = \sum_{j=1}^{n} P(X = X_i, Y = Y_j) \\
            = P(X = X_i, Y = Y_{\Pi(i)}) + \sum_{j \neq \Pi(i)} P(X = X_i, Y = Y_j) \\
            = \frac{1}{n} + 0 = \frac{1}{n}
    \end{split}
\end{equation}

Similarly, we can show the marginal of P(Y) is also $\frac{1}{n}$.
%
By leveraging Proposition~\ref{prop:large_samples}, we show that $P(X)$ will converge the CDF of $p_1$, and the CDF of $P(Y)$ will converge to the CDF of $p_2$.

% \begin{lemma}\label{lemma:independent}
% The Gaussian noises $x_0$ are independently and identically distributed (IID), \ie, $q_0(x_0) = \prod_{i}^N q_0(x_0^i)$, where $x^i_0$ is the $i$-th noises and $x^i_0 \sim q_0$ .
% %
% Also, the point cloud $x_1$ given a 3D shape $S$ is also independently and identically distributed (IID), \ie, $q_1(x_1|S) = \prod_{i}^N q_1(x_1^i | S)$, where $x^i_1$ is the $i$-th point and $x^i_0 \sim q_{1|S}$.
% %
% Lastly, the training pair $(x_0, x_1)$ from our coupling  given a shape $S$ is also independently and identically distributed (IID), \ie, $q(x_0, x_1 | S) = \prod_{i}^N q(x_0^i, x_1^i | S)$, where $(x_0^i, x_1^i$) is the $i$-th pair in the training pair.
% \end{lemma}

% \begin{lemma}\label{lemma:joint}
%     The sample distribution of a point $x_1^i$ involves modeling of underlying shape $S$ and the modeling of the point distribution given $S$, \ie, $q_1(x_1^i) = \int q_1(x_1^i | S) q(S) dS$.
%     %
%     However, the distribution of noises $q_0(x^i_0)$ is unrelated to a given shape $S$, \ie, $q_0(x^i_0 | S) = q_0(x^i_0)$.
% \end{lemma}

% By considering the $p(X)$ be a Gaussian distribution $N(0, I)$ or a sampling distribution of 3D points given a Shape $\mathcal{S}$, \ie, $q(x|\mathcal{S})$, we can show the random sample $Y$ still follows the original distribution.

% If we consider $M$ random variables, where each of them is an 3D Gaussian noise, denoted as $\epsilon_i \sim N(0, I), \epsilon_i \in \mathbb{R}^3$.
% %
% We also define another variable $\epsilon$ is a random sample of these random variables, \ie, $P(\epsilon = \epsilon_i) = \frac{1}{M}$.
% %
% Since each dimension in $\epsilon$ is independent, CDF of $\epsilon^j$ will follows the by leverage the above results, where $j$ is the j-th dimension of the noise.
% We consider a dense 3D Gaussian noises with $M \times 3$ random variables, $\{x_1, y_1, z_1, \cdots, x_M, y_M, z_M\}$, where we denote $x_i$, $y_i$, and $z_i$ to be the coordinates of in x, y, and z dimensions, respectively and $x_i, y_i, z_i \sim N(0, I)$.
% %
% If we can consider a random variable $\hat{x}$, which is random sample of this dense gausian in x dimension, \ie,  P$(\hat{x} = x_i) = \frac{1}{M}$.
% %
% By the above results, the CDF follows the original distribution, which is the Gaussian distribution $N(0, I)$.
% %
% By considering also y and z dimension, we can show that a random sampling of Gaussian point converge to Gaussian distribution.
\newtheorem{theorem}{Theorem}
\subsubsection{Proof of Our OT Approximation}
\label{subsec:our_ot_proof}

We first give a definition of coupling $q(x_0, x_1)$ in our case before showing its marginal fullfils the marginal requirements.
%
In particular, we denote $x_0 \in R^{N \times 3}$ and $x_1 \in R^{N \times 3}$ as two random variables following the distributions, $q_0(x_0)$ and $q_1(x_1)$, respectively.
%
It is noted that $q_0 := N(0, I)$, which is the standard Gaussian for each dimension in $x_0$, and $q_1(x_1)$ is the distribution all possible point clouds, which involves the joint modeling of point cloud distribution given a shape $S$ ($q_{1}(x_1|S)$) and the distribution of shape ($q(S)$), \ie, $q_1(x_1) = \int q_{1}(x_1|S) q(S) dS$.
%

We can notice that each row in $x_0$ is independently and identically distributed (IID), \ie, $q_0(x_0) = \prod_{i}^N \hat{q_0}(x_0^i)$, where we denote the $i$-th row of $x_0$ as $x_0^i$ and distribution of $x_0^i$ as $\hat{q_0}(x_0^i)$, which is 3-dimensional unit Gaussian.
%
We can also assume each point in $x_1$ is IID given a shape, \ie, $q_{1}(x_1 | S) = \prod_{i}^N \hat{q_{1}}(x_1^i|S)$,  where we denote the $i$-th row of $x_1$ as $x^i_1$ and the distribution of $x^i_1$ as $\hat{q_{1}}(x_1^i|S)$. 

In our superset OT precomputation for a given shape $S$, we pre-sample a set of random variables $(x^1_0 \cdots, x^j_0, \cdots, x^M_0) \sim \hat{q}_0$, and a set of random variables  $(x^1_1, \cdots, x^k_1,\cdots, x^M_1) \sim \hat{q}_1$, and have a precomputed bijective mapping $\Pi : \{1, \cdots, M\} \leftrightarrow \{1, \cdots, M\}$.
%
With these defined, our coupling $\hat{q}(x^i_0, x^i_1 |S)$ for one row in the training pair $(x^i_0, x^i_1)$ given $S$ can be formulated as:
\[
    \hat{q}(x^i_0 = x^j_0, x^i_1 = x^k_1 | S) =
    \begin{cases}
    \frac{1}{n}, & \text{if } j = \Pi(k) \\ 
        0, & \text{else } j \neq \Pi(k);
    \end{cases}
\]
%
Since the each row in the training pairs are independently subsampled, the coupling of the training pair $(x_0, x_1)$ given a shape is defined as $q(x_0, x_1 |S) = \prod_{i}^N \hat{q}(x_0^i, x_1^i | S)$.
%
In the end, the coupling over all training pairs can be obtained by marginalize over all possible shapes, \ie, $\int q(x_0, x_1 | S) q(S) dS$.

\begin{theorem}

% Our coupling $q(x_0, x_1)$ for a given Gaussian noise $x_0 \in R^{N \times 3}$ and a given point cloud $x_1 \in R^{N \times 3}$

Our coupling without blending converge the following marginal if the superset size $M$ is sufficiently large:
\begin{equation}\label{eq:mariginals}
    \int q(\rvx_0, \rvx_1) d\rvx_1 = q_0(\rvx_0), \int q(\rvx_0, \rvx_1) d\rvx_0 = q_1(\rvx_1).
\end{equation}
\end{theorem}

Proof:
We first show the left constraint:
% \begin{equation}
\begin{align}
LHS & = \int q(x_0, x_1) dx_1 = \int \int q(x_0, x_1 | S) q(S) dS dx_1  \\
& = \int q(S) \int q(x_0, x_1 | S) dx_1 dS && \text{change the order of integration} \\
& = \int q(S) \int \prod_i^N \hat{q}(x_0^i, x_1^i|S) d(x_1^1, \cdots, x_1^N) dS  && \text{independent assumption of each row in training pair}\\
& = \int q(S) \prod_i^N \int \hat{q}(x_0^i, x_1^i|S) dx_1^j dS && \text{integrals of independent products}\\
& = \int q(S) \prod_i \sum_k^M \hat{q}(x_0^i, x_1^k|S) dS && \text{restricting to discrete values in supersets}\\
& = \int q(S) \prod_i \hat{q}_0(x^i_0) dS && \text{Proposition~\ref{prop:ot}}\\
& = \int q(S) q_0(x_0) dS = q_0(x_0) && \text{independent assumption of each row in Gaussian noises} \\
\end{align}
% \end{equation}

Similarly, we perform the same computation on the right constraint:
% \begin{equation}
\begin{align}
LHS & = \int q(x_0, x_1) dx_0 = \int \int q(x_0, x_1 | S) q(S) dS dx_0   \\
 & = \int q(S) \int q(x_0, x_1 | S) dx_0 dS && \text{change the order of integration} \\
& = \int q(S) \int \prod_i^N \hat{q}(x_0^i, x_1^i|S) d(x_0^1, \cdots, x_0^N) dS && \text{independent assumption of each row in training pair} \\
& = \int q(S) \prod_i^N \int \hat{q}(x_0^i, x_1^i|S) dx_0^i dS  && \text{integrals of independent products} \\
& = \int q(S) \prod_i \sum_j^M \hat{q}(x_0^j, x_1^i|S) dS 
 && \text{restricting to discrete values in supersets} \\
& = \int q(S) \prod_i \hat{q}_1(x^i_1 | S) dS  && \text{Proposition~\ref{prop:ot}} \\
& = \int q(S) q_1(x_1 | S) dS = q_1(x_1) && \text{independent assumption of each row in point cloud} \\
\end{align}
% \end{equation}


% We first consider the RHS of Left Constraints (Equation~\ref{eq:mariginals}), we can reformulate it as follows:
% \begin{equation}
%     \begin{split}
%             RHS = q_0(x_0) = \int q(S) q_0(x_0 | S) dS = \int q(S) q_0(x_0) dS \\
%             % = \int q_0(x_0) (\int q_1(x_1 |S) q(S) dS) dx_1 \text{, by Lemma~\ref{lemma:joint}} \\
%             % = \int q(S) \int q_0(x_0) q_1(x_1|S) dx_1 d_S \text{, by rearranging the integrals} \\
%     \end{split}
% \end{equation}
% Considering LHS:
% \begin{equation}
%     \begin{split}
%         LHS = \int q(x_0, x_1) dx_1 = \int \int q(x_0, x_1 | S) q(S) dS dx_1 \\
%         = \int q(S) \int q(x_0, x_1 | S) dx_1 dS
%     \end{split}
% \end{equation}

% By comparing LHS and RHS, it is sufficient to show that $\int q(x_0, x_1 |S) dx_1 = q_0(x_0)$ for the first constraint.
% Similarly, for the second constraint RHS:
% \begin{equation}
%     \begin{split}
%             RHS = q_1(x_1) = \int q(S) q_1(x_1|S) dS \\
%             % = \int q_0(x_0) (\int q_1(x_1 |S) q(S) dS) dx_1 \text{, by Lemma~\ref{lemma:joint}} \\
%             % = \int q(S) \int q_0(x_0) q_1(x_1|S) dx_1 d_S \text{, by rearranging the integrals} \\
%     \end{split}
% \end{equation}
% Considering LHS:
% \begin{equation}
%     \begin{split}
%         LHS = \int q(x_0, x_1) dx_0 = \int \int q(x_0, x_1 | S) q(S) dS dx_0 \\
%         = \int q(S) \int q(x_0, x_1 | S) dx_0 dS
%     \end{split}
% \end{equation}
% Then it is sufficient to show $\int q(x_0, x_1 | S) dx_0 = q_1(x_1|S) $.

% To show first equation, we can apply Lemma~\ref{lemma:independent}:
% \begin{equation}
% \label{eq:left_LHS}
%     \begin{split}
%         LHS = \int q(x_0, x_1 | S) dx_1 = \int \prod_i q(x_0^i, x_1^i | S) d(x_1^i, \cdots, x_1^N) \\
%         = \prod_i \int q(x_0^i, x_1^i|S) dx_1^i 
%     \end{split}
% \end{equation}

% \begin{equation}
% \label{eq:left_RHS}
%     RHS = q_0(x_0) = \prod_i q_0(x^i_0)
% \end{equation}
% By this computation, we are also sufficient to show $\int q(x_0^i, x_1^i | S) dx_1^i = q_0(x_0^i)$ and by similar computation:
% \begin{equation}
% \label{eq:right_LHS}
%     \begin{split}
%         LHS = \int q(x_0, x_1 | S) dx_0 = \int \prod_i q(x_0^i, x_1^i | S) d(x_0^i, \cdots, x_0^N) \\
%         = \prod_i \int q(x_0^i, x_1^i|S) dx_0^i 
%     \end{split}
% \end{equation}

% \begin{equation}
% \label{eq:right_RHS}
%     RHS = q_1(x_0|S) = \prod_i q_1(x^i_1|S)
% \end{equation}
% Therefore, we are sufficient to show $\int q(x^i_0, x^i_1) dx_0^i = q_1(x_1^i |S)$.

% By considering the fact that, we pre-sample a set of random variables $(x^1_0 \cdots, x^j_0, \cdots, x^M_0) \sim q_0$, and a set of random variables  $(x^1_1, \cdots, x^k_1,\cdots, x^M_1) \sim q_{1|S}$, and have a precomputed bijective mapping $\Pi : \{1, \cdots, M\} \leftrightarrow \{1, \cdots, M\}$.
% %
% With these defined, our coupling $q(x^i_0, x^i_1 |S)$ given $S$ can formulated as:
% \[
%     P(x^i_0 = x^j_0, x^i_1 = x^k_1) =
%     \begin{cases}
%     \frac{1}{n}, & \text{if } k = \Pi(j) \\ 
%         0, & \text{else } k \neq \Pi(j);
%     \end{cases}
% \]
% By Proposition~\ref{prop:ot}, if the superset size $M$ is large enough, we can show that the CDF of Equation~\ref{eq:left_LHS} converge to Equation~\ref{eq:left_RHS}, also the CDF of Equation~\ref{eq:left_LHS} converges to Equation~\ref{eq:left_RHS}.

% To show our coupling maintain the correct marginal, we assume we have $M$ random variables $(X_1, \cdots, X_M) \sim p_1$, and another $M$ random random variables $(Y_1, \cdots, Y_M) \sim p_2$.
% %
% We can additionally take an arbitrary bijective map $\Pi$ between random variables, \ie, $\Pi : \{1, \cdots, M\} \leftrightarrow \{1, \cdots, M\}$.
% %
% If we only sample the a pair of variables based on the bijective map, we can then construct a new random Variable $Z: \{X, Y\}$:
% \[
%     P(X = X_i, Y = Y_j) =
%     \begin{cases}
%     \frac{1}{M}, & \text{if } j = \Pi(i) \\ 
%         0, & \text{else } j \neq \Pi(i);
%     \end{cases}
% \]

% Since $\Pi$ is a bijective mapping, the mariginal distribution of $P(X = X_i)$ and $P(Y = Y_j)$ is also $\frac{1}{M}$.
% %
% Following the result in the previous section, we can show the random variable $X$ ($Y$) still follows $p_1$ ($p_2$).
% %
% In our case, we consider $p_1$ to be a 3D Gaussian distribution, and $p_2$ to be point sample distribution given a Shape $\mathcal{S}$.

% The last part we need to show is that $q_0(x_0)$ and $q_1(x_1|\mathcal{S})$ is independently sampled for each of the point, \ie, $q_0(x_0) = \prod_{i} q_0(x_0^i)$ and \ie, $q_1(x_1) = \prod_{i} q_1(x_1^i | \mathcal{S})$, where $x_0^i$ and $x_1^i$ is the $i$-th point in $x_0$ and $x_1$, respectively.
% %
% For Gaussian distribution $q_0(x_0)$, this is true because it is an unit Gaussian distribution.
% %
% For surface point distribution $q_1(x_1|S)$, it is also correct since the points are indepdently sampled.



% Additionally, for a Gaussian noise sets arranged in the matrix format, $x_0 \in \mathbb{R}^{N \times 3}, x_0 \sim$
\subsubsection{Proof of Hybrid Coupling}

In the last, we would like to show even with our hybrid coupling, the marginal still fulfills the requirements.
%
In particular, we define a new noises $x_0'$ after perturbation:
\begin{equation}
    x_0' = \sqrt{1 - \beta} x_0 + \sqrt{\beta} \epsilon, \epsilon \sim N(\epsilon; 0, I),
\end{equation}
where $\beta \in [0, 1]$ is the blending coefficient. We denoted this as a conditional distribution $q(x_0'| x_0)$, which has a form of $N(x_0'; \sqrt{1 - \beta}x_0, \beta)$.
%
It is noted that since $\epsilon \sim N(\epsilon, 0, I)$, each row of $x'_0$ is also IID given $x_0$, \ie, $q_0(x_0' | x_0) = \prod_i^N \hat{q}_0(x_0^{'i} | x_0^i)$.
%
Due to the independent properties, it is sufficient to show that:
\begin{equation}
    \int q(x_0^{i'}, x_1^i | S) dx_0^{i'} = q_1(x^i_1|S), 
    \int q(x_0^{i'}, x_1^i | S) dx_1^{i} = q_0(x_0^i).
\end{equation}

For the sake of simplicity, we remove all the index $i$ and shape $S$ in the folloings.
We first show the left constraint:
\iffalse
\begin{align}
    q(x_1) & = \int q(x_0', x_1) dx_0' = \int \int q_0(x_0) q(x_0'| x_0) q(x_1|x_0, x_0') dx_0 dx_0' \\
    & = \int \int q_0(x_0) q(x_0'| x_0) q(x_1|x_0) dx_0 dx_0' \\
    & =  \int \int  q_0(x_0) q(x_0'| x_0) q(x_1|x_0)  dx_0' dx_0 \\
    & = \int q_0(x_0) q(x_1|x_0) \int  q(x_0'| x_0)  dx_0' dx_0 \\
    & = \int q_0(x_0) q(x_1|x_0) (1) dx_0 \\
    & = \int q(x_0, x_1) dx_0  = \frac{1}{M} \\
\end{align}
\fi
\begin{align}
    \int q(x_0', x_1) dx_0' & = \int \int q_0(x_0', x_0, x_1) dx_0 dx_0' \\
    & = \int \int q_0(x_0'|x_0) q(x_0, x_1) dx_0 dx_0' \\
    & =  \int q(x_0, x_1) \int  q_0(x_0'|x_0)  dx_0' dx_0 \\
    & = \int q(x_0, x_1) (1) dx_0 \\
    & = q(x_1)
\end{align}
By Proposition~\ref{prop:large_samples}, we can show $q(x_1)$ still converge to the right CDF if $M$ is sufficient large.

On the other hand, we show that:
\iffalse
\begin{align}
    \int q(x_0', x_1) dx_1 &= \int \int q_1(x_0' | x_0, x_1) q_0(x_0|x_1) q(x_1) dx_0 dx_1 \\
    &= \int \int q(x_0' | x_0, x_1) q(x_0|x_1) q(x_1) dx_1 dx_0 \\
    &= \int \int q(x_0' | x_0) q(x_0|x_1) q(x_1) dx_1 dx_0 \\
    & = \int q(x_0'|x_0) \int q(x_0|x_1) q(x_1) dx_1 dx_0 \\
    & = \int q(x_0'|x_0) \sum_{x_1} q(x_0, x_1) dx_0 \\
    & = \int q(x_0'|x_0) \frac{1}{M} dx_0 \\
    & = \frac{1}{M} \sum_{x_0} q(x_0' | x_0) \\
    & = \frac{1}{M} \sum_{x_0} N(x_0'; \sqrt{1 - \beta} x_0, \beta)
\end{align}
\fi
\begin{align}
    \int q(x_0', x_1) dx_1 &= \int \int q_0(x_0', x_0, x_1) dx_0 dx_1 \\
    &= \int \int q_0(x_0', x_0) dx_0\\
    &= \int \int  q_0(x_0'|x_0) q(x_0) dx_0 \\
    & = N(0, I)
\end{align}
where the last equality is obtained by inserting $q(x_0) = N(0, I)$ and $q_0(x_0'|x_0) = N(x_0'; \sqrt{1 - \beta}x_0, \beta I)$.

\iffalse
When $M \rightarrow \infty$, it becomes a convolution of two Gaussian $N(0, (1 - \beta) I)$ and $N(0, \beta I)$.
%
By convolution of Gaussian, we can observe that:
\begin{align}
    \int q(x_0', x_1) dx_1 & = N(0, (1 - \beta)I + \beta I) \\
    & = N(0, I)\\
\end{align}
\fi
\section{Experimental Settings \label{app:implementation}}

We follow the model architecture design and hyperparameter settings of \citet{luo2024classic}. In order to accommodate the computational requirements for our extensive experiments, we harness a variety of high-capacity GPU resources. This includes: Tesla V100 32Gb, NVIDIA RTX A6000
48Gb, NVIDIA RTX A5000 24Gb, and Quadro RTX 8000 48Gb.

\section{Datasets}

In this paper, we adopt $12$ datasets from different domains.

\noindent{\small$\bullet$} \textbf{[Citation network]}. \emph{Cora}, \emph{Citeseer}, and \emph{Pubmed}~\citep{Yang2016RevisitingSL} are citation graphs where each node corresponds to a scientific paper. In these graphs, nodes are characterized by bag-of-words feature vectors, and each one is assigned a label that indicates its research field. It is important to note that all three datasets are examples of homophilous graphs.

\noindent{\small$\bullet$} \textbf{[Amazon network]}. \citet{Shchur2018PitfallsOG} In this network, products are nodes, and an edge signifies that two products are often bought together.  Each product has associated reviews, which are treated as a bag of words.  The task is to determine the product category for each item in the network.

\noindent{\small$\bullet$} \textbf{[Coauthor network]}. \citet{Shchur2018PitfallsOG} The network represents authors connected by co-authorship.  Using the keywords from their published papers, we aim to classify each author according to their research field.

\noindent{\small$\bullet$} \textbf{[Wikics network]}. WikiCS~\citep{mernyei2020wiki} is a hyperlink network in the field of computer science on Wikipedia. The categories correspond to different research directions in computer science, such as artificial intelligence, computer vision, network security, etc.


\noindent{\small$\bullet$} \textbf{[Wikipedia network]}.  Squirrel and Chameleon~\citep{platonov2023a} represent two distinct portions of the Wikipedia web.  The objective is to categorize each individual webpage (node) within these portions into one of five traffic-based classifications, determined by their respective average monthly page views.

\noindent{\small$\bullet$} \textbf{[Heterophilous network]}. These networks are from \citet{platonov2023a} Amazon Ratings, A co-purchasing network of products with reviews used to predict product category. Minesweeper, a synthetic grid-based graph where nodes represent cells, and the task is to identify mines using information about neighboring mines. Roman Empire, a word dependency graph from a Wikipedia article, where nodes are words, and edges represent sequential or syntactic relationships, with the task of classifying words by their syntactic roles.



\end{document}
\endinput
%%
%% End of file `sample-sigconf-authordraft.tex'.
