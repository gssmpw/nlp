%%
%% This is file `sample-sigconf-authordraft.tex',
%% generated with the docstrip utility.
%%
%% The original source files were:
%%
%% samples.dtx  (with options: `all,proceedings,bibtex,authordraft')
%% 
%% IMPORTANT NOTICE:
%% 
%% For the copyright see the source file.
%% 
%% Any modified versions of this file must be renamed
%% with new filenames distinct from sample-sigconf-authordraft.tex.
%% 
%% For distribution of the original source see the terms
%% for copying and modification in the file samples.dtx.
%% 
%% This generated file may be distributed as long as the
%% original source files, as listed above, are part of the
%% same distribution. (The sources need not necessarily be
%% in the same archive or directory.)
%%
%%
%% Commands for TeXCount
%TC:macro \cite [option:text,text]
%TC:macro \citep [option:text,text]
%TC:macro \citet [option:text,text]
%TC:envir table 0 1
%TC:envir table* 0 1
%TC:envir tabular [ignore] word
%TC:envir displaymath 0 word
%TC:envir math 0 word
%TC:envir comment 0 0
%%
%% The first command in your LaTeX source must be the \documentclass
%% command.
%%
%% For submission and review of your manuscript please change the
%% command to \documentclass[manuscript, screen, review]{acmart}.
%%
%% When submitting camera ready or to TAPS, please change the command
%% to \documentclass[sigconf]{acmart} or whichever template is required
%% for your publication.
%%
%%
\documentclass[sigconf]{acmart}
\usepackage{subcaption}
\usepackage{multirow}
\usepackage{pifont}
\usepackage{enumitem}
\usepackage{wrapfig}
% \usepackage{subfig}
\usepackage{graphicx}
\usepackage{natbib}
\usepackage{subcaption}
\theoremstyle{plain}
\newtheorem{theorem}{Theorem}[section]
\newtheorem{proposition}[theorem]{Proposition}
\newtheorem{lemma}[theorem]{Lemma}
\newtheorem{corollary}[theorem]{Corollary}
\theoremstyle{definition}
\newtheorem{definition}[theorem]{Definition}
\newtheorem{assumption}[theorem]{Assumption}
\theoremstyle{remark}
\newtheorem{remark}[theorem]{Remark}

\newcommand{\jt}[1]{{\color{red}[JT:#1]}}
\newcommand{\czk}[1]{{\color{blue}[czk:#1]}}
\newcommand{\lbh}[1]{{\color{green}[lbh:#1]}}
%%
%% \BibTeX command to typeset BibTeX logo in the docs
\AtBeginDocument{%
  \providecommand\BibTeX{{%
    Bib\TeX}}}

%% Rights management information.  This information is sent to you
%% when you complete the rights form.  These commands have SAMPLE
%% values in them; it is your responsibility as an author to replace
%% the commands and values with those provided to you when you
%% complete the rights form.
\setcopyright{acmlicensed}
\copyrightyear{2018}
\acmYear{2018}
\acmDOI{XXXXXXX.XXXXXXX}
%% These commands are for a PROCEEDINGS abstract or paper.
\acmConference[Conference acronym 'XX]{Make sure to enter the correct
  conference title from your rights confirmation email}{June 03--05,
  2018}{Woodstock, NY}
%%
%%  Uncomment \acmBooktitle if the title of the proceedings is different
%%  from ``Proceedings of ...''!
%%
%%\acmBooktitle{Woodstock '18: ACM Symposium on Neural Gaze Detection,
%%  June 03--05, 2018, Woodstock, NY}
\acmISBN{978-1-4503-XXXX-X/2018/06}


%%
%% Submission ID.
%% Use this when submitting an article to a sponsored event. You'll
%% receive a unique submission ID from the organizers
%% of the event, and this ID should be used as the parameter to this command.
%%\acmSubmissionID{123-A56-BU3}

%%
%% For managing citations, it is recommended to use bibliography
%% files in BibTeX format.
%%
%% You can then either use BibTeX with the ACM-Reference-Format style,
%% or BibLaTeX with the acmnumeric or acmauthoryear sytles, that include
%% support for advanced citation of software artefact from the
%% biblatex-software package, also separately available on CTAN.
%%
%% Look at the sample-*-biblatex.tex files for templates showcasing
%% the biblatex styles.
%%

%%
%% The majority of ACM publications use numbered citations and
%% references.  The command \citestyle{authoryear} switches to the
%% "author year" style.
%%
%% If you are preparing content for an event
%% sponsored by ACM SIGGRAPH, you must use the "author year" style of
%% citations and references.
%% Uncommenting
%% the next command will enable that style.
%%\citestyle{acmauthoryear}


%%
%% end of the preamble, start of the body of the document source.
\begin{document}

%%
%% The "title" command has an optional parameter,
%% allowing the author to define a "short title" to be used in page headers.
\title{Unveiling Mode Connectivity in Graph Neural Networks}

%%
%% The "author" command and its associated commands are used to define
%% the authors and their affiliations.
%% Of note is the shared affiliation of the first two authors, and the
%% "authornote" and "authornotemark" commands
%% used to denote shared contribution to the research.
\author{Bingheng Li}
% \author{Ben Trovato}
% \authornote{Both authors contributed equally to this research.}
% \email{trovato@corporation.com}
% \orcid{1234-5678-9012}
% \author{G.K.M. Tobin}
% \authornotemark[1]
% \email{webmaster@marysville-ohio.com}
\affiliation{%
    \institution{Department of Computer Science and Engineering,}
  \institution{Michigan State University}
  % \streetaddress{P.O. Box 1212}
  % \city{Dublin}
  % \state{Ohio}
  \country{}
  % \postcode{43017-6221}
}
% \institution{Jilin University}
\email{libinghe@msu.edu}

\author{Zhikai Chen}
\affiliation{%
    \institution{Department of Computer Science and Engineering,}
  \institution{Michigan State University}
  \country{}
}
\email{chenzh85@msu.edu}

\author{Haoyu Han}
\affiliation{%
    \institution{Department of Computer Science and Engineering,}
  \institution{Michigan State University}
  \country{}
  % \city{Rocquencourt}
  % \country{France}
}
\email{hanhaoy1@msu.edu}

\author{Shenglai Zeng}
\affiliation{%
    \institution{Department of Computer Science and Engineering,}
  \institution{Michigan State University}
  \country{}
}
\email{zengshe1@msu.edu}

\author{Jingzhe Liu}
\affiliation{%
    \institution{Department of Computer Science and Engineering,}
  \institution{Michigan State University}
  \country{}
}
\email{liujin33@msu.edu}

\author{Jiliang Tang}
\affiliation{
\institution{Department of Computer Science and Engineering,}
  \institution{Michigan State University}
  \country{}
}
\email{tangjili@msu.edu}


%%
%% By default, the full list of authors will be used in the page
%% headers. Often, this list is too long, and will overlap
%% other information printed in the page headers. This command allows
%% the author to define a more concise list
%% of authors' names for this purpose.
\renewcommand{\shortauthors}{Trovato et al.}
\newcommand\liu[1]{\textcolor{blue}{liu: #1}}
%%
%% The abstract is a short summary of the work to be presented in the
%% article.
\begin{abstract}
A fundamental challenge in understanding graph neural networks (GNNs) lies in characterizing their optimization dynamics and loss landscape geometry, critical for improving interpretability and robustness. While mode connectivity—a lens for analyzing geometric properties of loss landscapes—has proven insightful for other deep learning architectures, its implications for GNNs remain unexplored. This work presents the first investigation of mode connectivity in GNNs. We uncover that GNNs exhibit distinct non-linear mode connectivity, diverging from patterns observed in fully-connected networks or CNNs. Crucially, we demonstrate that graph structure, rather than model architecture, dominates this behavior, with graph properties like homophily correlating with mode connectivity patterns. We further establish a link between mode connectivity and generalization, proposing a generalization bound based on loss barriers and revealing its utility as a diagnostic tool. Our findings further bridge theoretical insights with practical implications: they rationalize domain alignment strategies in graph learning and provide a foundation for refining GNN training paradigms. 
\end{abstract}

%%
%% The code below is generated by the tool at http://dl.acm.org/ccs.cfm.
%% Please copy and paste the code instead of the example below.
%%
\begin{CCSXML}
<ccs2012>
<concept>
<concept_id>10003752.10003809.10003635</concept_id>
<concept_desc>Theory of computation~Graph algorithms analysis</concept_desc>
<concept_significance>500</concept_significance>
</concept>
<concept>
<concept_id>10010147.10010257.10010293.10010294</concept_id>
<concept_desc>Computing methodologies~Neural networks</concept_desc>
<concept_significance>500</concept_significance>
</concept>
</ccs2012>
\end{CCSXML}

\ccsdesc[500]{Theory of computation~Graph algorithms analysis}
\ccsdesc[500]{Computing methodologies~Neural networks}

% \ccsdesc[500]{Do Not Use This Code~Generate the Correct Terms for Your Paper}
% \ccsdesc[300]{Do Not Use This Code~Generate the Correct Terms for Your Paper}
% \ccsdesc{Do Not Use This Code~Generate the Correct Terms for Your Paper}
% \ccsdesc[100]{Do Not Use This Code~Generate the Correct Terms for Your Paper}

%%
%% Keywords. The author(s) should pick words that accurately describe
%% the work being presented. Separate the keywords with commas.
%\keywords{Mode Connectivity, Graph Neural Networks, Machine Learning theory}
%% A "teaser" image appears between the author and affiliation
%% information and the body of the document, and typically spans the
%% page.




%%
%% This command processes the author and affiliation and title
%% information and builds the first part of the formatted document.
\maketitle

\section{Introduction}
Backdoor attacks pose a concealed yet profound security risk to machine learning (ML) models, for which the adversaries can inject a stealth backdoor into the model during training, enabling them to illicitly control the model's output upon encountering predefined inputs. These attacks can even occur without the knowledge of developers or end-users, thereby undermining the trust in ML systems. As ML becomes more deeply embedded in critical sectors like finance, healthcare, and autonomous driving \citep{he2016deep, liu2020computing, tournier2019mrtrix3, adjabi2020past}, the potential damage from backdoor attacks grows, underscoring the emergency for developing robust defense mechanisms against backdoor attacks.

To address the threat of backdoor attacks, researchers have developed a variety of strategies \cite{liu2018fine,wu2021adversarial,wang2019neural,zeng2022adversarial,zhu2023neural,Zhu_2023_ICCV, wei2024shared,wei2024d3}, aimed at purifying backdoors within victim models. These methods are designed to integrate with current deployment workflows seamlessly and have demonstrated significant success in mitigating the effects of backdoor triggers \cite{wubackdoorbench, wu2023defenses, wu2024backdoorbench,dunnett2024countering}.  However, most state-of-the-art (SOTA) backdoor purification methods operate under the assumption that a small clean dataset, often referred to as \textbf{auxiliary dataset}, is available for purification. Such an assumption poses practical challenges, especially in scenarios where data is scarce. To tackle this challenge, efforts have been made to reduce the size of the required auxiliary dataset~\cite{chai2022oneshot,li2023reconstructive, Zhu_2023_ICCV} and even explore dataset-free purification techniques~\cite{zheng2022data,hong2023revisiting,lin2024fusing}. Although these approaches offer some improvements, recent evaluations \cite{dunnett2024countering, wu2024backdoorbench} continue to highlight the importance of sufficient auxiliary data for achieving robust defenses against backdoor attacks.

While significant progress has been made in reducing the size of auxiliary datasets, an equally critical yet underexplored question remains: \emph{how does the nature of the auxiliary dataset affect purification effectiveness?} In  real-world  applications, auxiliary datasets can vary widely, encompassing in-distribution data, synthetic data, or external data from different sources. Understanding how each type of auxiliary dataset influences the purification effectiveness is vital for selecting or constructing the most suitable auxiliary dataset and the corresponding technique. For instance, when multiple datasets are available, understanding how different datasets contribute to purification can guide defenders in selecting or crafting the most appropriate dataset. Conversely, when only limited auxiliary data is accessible, knowing which purification technique works best under those constraints is critical. Therefore, there is an urgent need for a thorough investigation into the impact of auxiliary datasets on purification effectiveness to guide defenders in  enhancing the security of ML systems. 

In this paper, we systematically investigate the critical role of auxiliary datasets in backdoor purification, aiming to bridge the gap between idealized and practical purification scenarios.  Specifically, we first construct a diverse set of auxiliary datasets to emulate real-world conditions, as summarized in Table~\ref{overall}. These datasets include in-distribution data, synthetic data, and external data from other sources. Through an evaluation of SOTA backdoor purification methods across these datasets, we uncover several critical insights: \textbf{1)} In-distribution datasets, particularly those carefully filtered from the original training data of the victim model, effectively preserve the model’s utility for its intended tasks but may fall short in eliminating backdoors. \textbf{2)} Incorporating OOD datasets can help the model forget backdoors but also bring the risk of forgetting critical learned knowledge, significantly degrading its overall performance. Building on these findings, we propose Guided Input Calibration (GIC), a novel technique that enhances backdoor purification by adaptively transforming auxiliary data to better align with the victim model’s learned representations. By leveraging the victim model itself to guide this transformation, GIC optimizes the purification process, striking a balance between preserving model utility and mitigating backdoor threats. Extensive experiments demonstrate that GIC significantly improves the effectiveness of backdoor purification across diverse auxiliary datasets, providing a practical and robust defense solution.

Our main contributions are threefold:
\textbf{1) Impact analysis of auxiliary datasets:} We take the \textbf{first step}  in systematically investigating how different types of auxiliary datasets influence backdoor purification effectiveness. Our findings provide novel insights and serve as a foundation for future research on optimizing dataset selection and construction for enhanced backdoor defense.
%
\textbf{2) Compilation and evaluation of diverse auxiliary datasets:}  We have compiled and rigorously evaluated a diverse set of auxiliary datasets using SOTA purification methods, making our datasets and code publicly available to facilitate and support future research on practical backdoor defense strategies.
%
\textbf{3) Introduction of GIC:} We introduce GIC, the \textbf{first} dedicated solution designed to align auxiliary datasets with the model’s learned representations, significantly enhancing backdoor mitigation across various dataset types. Our approach sets a new benchmark for practical and effective backdoor defense.



\section{Methodology}\label{sec:method}

This section describes our approach for estimating \gls{bp} from \gls{ecg} and \gls{ppg} waveforms using a large \gls{eeg}-based foundation model. We first detail how we adapt and fine-tune the CEReBrO architecture for \gls{bp} prediction, then describe our post-training quantization steps.

\subsection{Architecture and Fine-Tuning}\label{subsec:model}

\begin{figure}[htp]
    \centering
    \includegraphics[width=7.5cm]{images/architecture}
    \caption{The modified CEReBrO Architecture~\cite{CEReBrO}, supplemented with an additional MLP-head, which is utilized for the \gls{bp} estimation task.} 
    \vspace{-0.6cm}
    \label{fig:cerebro}
\end{figure}  

Our method builds on the \textbf{CEReBrO} transformer-encoder~\cite{CEReBrO}, originally pre-trained on a large \gls{eeg} dataset (TUEG~\cite{TUEG}). CEReBrO employs a tokenization scheme that splits time-series signals into non-overlapping patches and projects them into a latent space. Alternating self-attention blocks then process these tokens by focusing on intra-channel (temporal) correlations and inter-channel (spatial) relationships. This design efficiently captures both local and long-range dependencies in multi-channel biosignals. Although CEReBrO was trained on \gls{eeg} data, its attention-based encoder can generalize to other biosignals sharing similar temporal structures. To adapt CEReBrO for \gls{bp} estimation from \gls{ecg} and \gls{ppg}, we make the following modifications:

\begin{itemize}
    \item We feed \gls{ecg} and \gls{ppg} signals as two input channels, each sampled at 125,Hz and shaped into 10-second segments ($2 \times 1250$).
    \item  We replace the original classification head with a single fully connected layer (MLP) that outputs two values: \gls{sbp} and \gls{dbp}.
\end{itemize}

CEReBrO is then also available in three sizes—\emph{small} (3.58M parameters), \emph{medium} (39.95M parameters), and \emph{large} (85.15M parameters).

We then explore two fine-tuning strategies:
\begin{itemize}
    \item \textbf{Frozen Backbone}: All transformer layers except for the first input-embedding layer and the final MLP head are frozen. This preserves most \gls{eeg}-based representations while allowing partial adaptation to \gls{ecg}/\gls{ppg}.
    \item \textbf{Unfrozen Backbone}: All transformer layers are unfrozen to allow deeper domain alignment, albeit with a risk of forgetting learned \gls{eeg} features.
\end{itemize}
To measure the benefit of leveraging a pre-trained \gls{eeg} encoder, we compare fine-tuning against training from scratch (i.e., random initialization). In each setting, we train models of three different sizes (small, medium, large) for 100 epochs and extend unfrozen-backbone runs to 200 epochs to assess longer-term convergence. We use Xavier Initialization~\cite{xavier_init} when training from scratch. Performance is evaluated on the MIMIC-III and VitalDB datasets in terms of MAE, SD, and coefficient of determination ($R^2$), as well as clinical standards (\gls{bhs} and \gls{aami}).

\subsection{Quantization}\label{subsec:quantization}
We apply post-training quantization to our fine-tuned models to enable real-time deployment on resource-constrained devices. This step reduces the memory footprint and inference latency while preserving clinically relevant accuracy.

We use PyTorch’s FX Graph Mode Quantization pipeline~\cite{pytorch2} to insert quantization and dequantization operations systematically. Quantization is widely employed to map floating-point (32-bit) values to lower numerical precision, typically to 8-bit integers. The range of a floating-point value, denoted by \(x_{\mathrm{fp}}\), is defined as follows: \([x_{\min}, x_{\max}]\). Based on this, two characteristic values can be defined, which are essential for the quantization process: \textbf{Scale} \(\Delta\), which determines the step size (real-valued), while \textbf{Zero-point} \(z\), which is an integer offset whose primary function is to ensure that the zero is mapped onto an integer. In this work, we specifically adopt two types of quantization:

\begin{itemize}
    \item \emph{Symmetric Quantization} for weights (common when weight distributions are roughly zero-mean).
    \item \emph{Asymmetric Quantization} for activations (typical when ReLU shifts values positively).
\end{itemize}
For symmetric quantization, we have the following characteristic values:
\begin{equation*}
    \Delta = \frac{\max\bigl(|x_{\min}|, |x_{\max}|\bigr)}{2^{b-1}}, 
    \quad 
    z = 0
\end{equation*}

Based on these values, forward quantization is done using this equation:
\begin{equation*}
x_{\mathrm{int}} = \mathrm{clip}\!\Bigl(
        \mathrm{round}\bigl(\tfrac{x}{\Delta}\bigr), 
        \, -2^{b-1}, \, 2^{b-1} - 1\Bigr).
\end{equation*}

And for \emph{asymmetric} quantization the characteristic values can be calculated in the following way, where $b$ is the number of bits :
\begin{equation*}
    \Delta = \frac{x_{\max} - x_{\min}}{2^b - 1}, 
    \quad 
    z = \left\lfloor - \frac{x_{\min}}{\Delta} + 0.5 \right\rfloor.
\end{equation*}
When $\Delta$ and $z$ are determined, the forward quantization step is the following:
\begin{equation*}
    x_{\mathrm{int}} = \mathrm{clip}\!\Bigl(
        \mathrm{round}\bigl(\tfrac{x_{\mathrm{fp}}}{\Delta}\bigr) + z , 
        \, 0, \, 2^b - 1\Bigr),
\end{equation*}
where $\mathrm{clip}(\cdot,0,2^b-1)$ ensures $x_{\mathrm{int}}$ to stay in the range $[0, 2^b - 1]$~\cite{quant_data}.

Quantization typically involves three stages: (1) \emph{calibration}, where representative data is passed through the model to collect scaling statistics; (2) \emph{conversion}, transforming the floating-point model into a quantized version; and (3) \emph{execution}, running inference with reduced-precision operations.

We explore both \emph{static} quantization~\cite{FU20092937}, which precomputes scaling and zero points via a calibration dataset, and \emph{dynamic} quantization~\cite{vu2008stabilizing}, which calculates them on-the-fly, eliminating the calibration phase. While static quantization can offer speed benefits if the input distribution is stable, dynamic quantization is often more flexible for variable-length or varying data distributions and avoids the need for extra calibration data.

Our target precision is INT8, balancing memory savings and model fidelity. We evaluate symmetric quantization for weights and asymmetric for activations (shifted by ReLU). Different observers—\emph{MinMaxObserver}, \emph{MovingAverageMinMaxObserver}, and \emph{HistogramObserver}—estimate the range, each trading off complexity against robustness. We also employ per-channel quantization for \gls{ecg}/\gls{ppg} inputs, giving each signal channel a separate scale and zero point.

Our experiments reveal that dynamic per-channel quantization with symmetric weights yields an optimal model size, computational speed, and accuracy trade-off. Detailed results of these experiments are presented in Section~\ref{sec:results}. This approach is critical for enabling continuous, on-device \gls{bp} estimation, where both memory and energy constraints are strict.
\section{Experiments}

\subsection{Setups}
\subsubsection{Implementation Details}
We apply our FDS method to two types of 3DGS: 
the original 3DGS, and 2DGS~\citep{huang20242d}. 
%
The number of iterations in our optimization 
process is 35,000.
We follow the default training configuration 
and apply our FDS method after 15,000 iterations,
then we add normal consistency loss for both
3DGS and 2DGS after 25000 iterations.
%
The weight for FDS, $\lambda_{fds}$, is set to 0.015,
the $\sigma$ is set to 23,
and the weight for normal consistency is set to 0.15
for all experiments. 
We removed the depth distortion loss in 2DGS 
because we found that it degrades its results in indoor scenes.
%
The Gaussian point cloud is initialized using Colmap
for all datasets.
%
%
We tested the impact of 
using Sea Raft~\citep{wang2025sea} and 
Raft\citep{teed2020raft} on FDS performance.
%
Due to the blurriness of the ScanNet dataset, 
additional prior constraints are required.
Thus, we incorporate normal prior supervision 
on the rendered normals 
in ScanNet (V2) dataset by default.
The normal prior is predicted by the Stable Normal 
model~\citep{ye2024stablenormal}
across all types of 3DGS.
%
The entire framework is implemented in 
PyTorch~\citep{paszke2019pytorch}, 
and all experiments are conducted on 
a single NVIDIA 4090D GPU.

\begin{figure}[t] \centering
    \makebox[0.16\textwidth]{\scriptsize Input}
    \makebox[0.16\textwidth]{\scriptsize 3DGS}
    \makebox[0.16\textwidth]{\scriptsize 2DGS}
    \makebox[0.16\textwidth]{\scriptsize 3DGS + FDS}
    \makebox[0.16\textwidth]{\scriptsize 2DGS + FDS}
    \makebox[0.16\textwidth]{\scriptsize GT (Depth)}

    \includegraphics[width=0.16\textwidth]{figure/fig3_img/compare3/gt_rgb/frame_00522.jpg}
    \includegraphics[width=0.16\textwidth]{figure/fig3_img/compare3/3DGS/frame_00522.jpg}
    \includegraphics[width=0.16\textwidth]{figure/fig3_img/compare3/2DGS/frame_00522.jpg}
    \includegraphics[width=0.16\textwidth]{figure/fig3_img/compare3/3DGS+FDS/frame_00522.jpg}
    \includegraphics[width=0.16\textwidth]{figure/fig3_img/compare3/2DGS+FDS/frame_00522.jpg}
    \includegraphics[width=0.16\textwidth]{figure/fig3_img/compare3/gt_depth/frame_00522.jpg} \\

    % \includegraphics[width=0.16\textwidth]{figure/fig3_img/compare1/gt_rgb/frame_00137.jpg}
    % \includegraphics[width=0.16\textwidth]{figure/fig3_img/compare1/3DGS/frame_00137.jpg}
    % \includegraphics[width=0.16\textwidth]{figure/fig3_img/compare1/2DGS/frame_00137.jpg}
    % \includegraphics[width=0.16\textwidth]{figure/fig3_img/compare1/3DGS+FDS/frame_00137.jpg}
    % \includegraphics[width=0.16\textwidth]{figure/fig3_img/compare1/2DGS+FDS/frame_00137.jpg}
    % \includegraphics[width=0.16\textwidth]{figure/fig3_img/compare1/gt_depth/frame_00137.jpg} \\

     \includegraphics[width=0.16\textwidth]{figure/fig3_img/compare2/gt_rgb/frame_00262.jpg}
    \includegraphics[width=0.16\textwidth]{figure/fig3_img/compare2/3DGS/frame_00262.jpg}
    \includegraphics[width=0.16\textwidth]{figure/fig3_img/compare2/2DGS/frame_00262.jpg}
    \includegraphics[width=0.16\textwidth]{figure/fig3_img/compare2/3DGS+FDS/frame_00262.jpg}
    \includegraphics[width=0.16\textwidth]{figure/fig3_img/compare2/2DGS+FDS/frame_00262.jpg}
    \includegraphics[width=0.16\textwidth]{figure/fig3_img/compare2/gt_depth/frame_00262.jpg} \\

    \includegraphics[width=0.16\textwidth]{figure/fig3_img/compare4/gt_rgb/frame00000.png}
    \includegraphics[width=0.16\textwidth]{figure/fig3_img/compare4/3DGS/frame00000.png}
    \includegraphics[width=0.16\textwidth]{figure/fig3_img/compare4/2DGS/frame00000.png}
    \includegraphics[width=0.16\textwidth]{figure/fig3_img/compare4/3DGS+FDS/frame00000.png}
    \includegraphics[width=0.16\textwidth]{figure/fig3_img/compare4/2DGS+FDS/frame00000.png}
    \includegraphics[width=0.16\textwidth]{figure/fig3_img/compare4/gt_depth/frame00000.png} \\

    \includegraphics[width=0.16\textwidth]{figure/fig3_img/compare5/gt_rgb/frame00080.png}
    \includegraphics[width=0.16\textwidth]{figure/fig3_img/compare5/3DGS/frame00080.png}
    \includegraphics[width=0.16\textwidth]{figure/fig3_img/compare5/2DGS/frame00080.png}
    \includegraphics[width=0.16\textwidth]{figure/fig3_img/compare5/3DGS+FDS/frame00080.png}
    \includegraphics[width=0.16\textwidth]{figure/fig3_img/compare5/2DGS+FDS/frame00080.png}
    \includegraphics[width=0.16\textwidth]{figure/fig3_img/compare5/gt_depth/frame00080.png} \\



    \caption{\textbf{Comparison of depth reconstruction on Mushroom and ScanNet datasets.} The original
    3DGS or 2DGS model equipped with FDS can remove unwanted floaters and reconstruct
    geometry more preciously.}
    \label{fig:compare}
\end{figure}


\subsubsection{Datasets and Metrics}

We evaluate our method for 3D reconstruction 
and novel view synthesis tasks on
\textbf{Mushroom}~\citep{ren2024mushroom},
\textbf{ScanNet (v2)}~\citep{dai2017scannet}, and 
\textbf{Replica}~\citep{replica19arxiv}
datasets,
which feature challenging indoor scenes with both 
sparse and dense image sampling.
%
The Mushroom dataset is an indoor dataset 
with sparse image sampling and two distinct 
camera trajectories. 
%
We train our model on the training split of 
the long capture sequence and evaluate 
novel view synthesis on the test split 
of the long capture sequences.
%
Five scenes are selected to evaluate our FDS, 
including "coffee room", "honka", "kokko", 
"sauna", and "vr room". 
%
ScanNet(V2)~\citep{dai2017scannet}  consists of 1,613 indoor scenes
with annotated camera poses and depth maps. 
%
We select 5 scenes from the ScanNet (V2) dataset, 
uniformly sampling one-tenth of the views,
following the approach in ~\citep{guo2022manhattan}.
To further improve the geometry rendering quality of 3DGS, 
%
Replica~\citep{replica19arxiv} contains small-scale 
real-world indoor scans. 
We evaluate our FDS on five scenes from 
Replica: office0, office1, office2, office3 and office4,
selecting one-tenth of the views for training.
%
The results for Replica are provided in the 
supplementary materials.
To evaluate the rendering quality and geometry 
of 3DGS, we report PSNR, SSIM, and LPIPS for 
rendering quality, along with Absolute Relative Distance 
(Abs Rel) as a depth quality metrics.
%
Additionally, for mesh evaluation, 
we use metrics including Accuracy, Completion, 
Chamfer-L1 distance, Normal Consistency, 
and F-scores.




\subsection{Results}
\subsubsection{Depth rendering and novel view synthesis}
The comparison results on Mushroom and 
ScanNet are presented in \tabref{tab:mushroom} 
and \tabref{tab:scannet}, respectively. 
%
Due to the sparsity of sampling 
in the Mushroom dataset,
challenges are posed for both GOF~\citep{yu2024gaussian} 
and PGSR~\citep{chen2024pgsr}, 
leading to their relative poor performance 
on the Mushroom dataset.
%
Our approach achieves the best performance 
with the FDS method applied during the training process.
The FDS significantly enhances the 
geometric quality of 3DGS on the Mushroom dataset, 
improving the "abs rel" metric by more than 50\%.
%
We found that Sea Raft~\citep{wang2025sea}
outperforms Raft~\citep{teed2020raft} on FDS, 
indicating that a better optical flow model 
can lead to more significant improvements.
%
Additionally, the render quality of RGB 
images shows a slight improvement, 
by 0.58 in 3DGS and 0.50 in 2DGS, 
benefiting from the incorporation of cross-view consistency in FDS. 
%
On the Mushroom
dataset, adding the FDS loss increases 
the training time by half an hour, which maintains the same
level as baseline.
%
Similarly, our method shows a notable improvement on the ScanNet dataset as well using Sea Raft~\citep{wang2025sea} Model. The "abs rel" metric in 2DGS is improved nearly 50\%. This demonstrates the robustness and effectiveness of the FDS method across different datasets.
%


% \begin{wraptable}{r}{0.6\linewidth} \centering
% \caption{\textbf{Ablation study on geometry priors.}} 
%         \label{tab:analysis_prior}
%         \resizebox{\textwidth}{!}{
\begin{tabular}{c| c c c c c | c c c c}

    \hline
     Method &  Acc$\downarrow$ & Comp $\downarrow$ & C-L1 $\downarrow$ & NC $\uparrow$ & F-Score $\uparrow$ &  Abs Rel $\downarrow$ &  PSNR $\uparrow$  & SSIM  $\uparrow$ & LPIPS $\downarrow$ \\ \hline
    2DGS&   0.1078&  0.0850&  0.0964&  0.7835&  0.5170&  0.1002&  23.56&  0.8166& 0.2730\\
    2DGS+Depth&   0.0862&  0.0702&  0.0782&  0.8153&  0.5965&  0.0672&  23.92&  0.8227& 0.2619 \\
    2DGS+MVDepth&   0.2065&  0.0917&  0.1491&  0.7832&  0.3178&  0.0792&  23.74&  0.8193& 0.2692 \\
    2DGS+Normal&   0.0939&  0.0637&  0.0788&  \textbf{0.8359}&  0.5782&  0.0768&  23.78&  0.8197& 0.2676 \\
    2DGS+FDS &  \textbf{0.0615} & \textbf{ 0.0534}& \textbf{0.0574}& 0.8151& \textbf{0.6974}&  \textbf{0.0561}&  \textbf{24.06}&  \textbf{0.8271}&\textbf{0.2610} \\ \hline
    2DGS+Depth+FDS &  0.0561 &  0.0519& 0.0540& 0.8295& 0.7282&  0.0454&  \textbf{24.22}& \textbf{0.8291}&\textbf{0.2570} \\
    2DGS+Normal+FDS &  \textbf{0.0529} & \textbf{ 0.0450}& \textbf{0.0490}& \textbf{0.8477}& \textbf{0.7430}&  \textbf{0.0443}&  24.10&  0.8283& 0.2590 \\
    2DGS+Depth+Normal &  0.0695 & 0.0513& 0.0604& 0.8540&0.6723&  0.0523&  24.09&  0.8264&0.2575\\ \hline
    2DGS+Depth+Normal+FDS &  \textbf{0.0506} & \textbf{0.0423}& \textbf{0.0464}& \textbf{0.8598}&\textbf{0.7613}&  \textbf{0.0403}&  \textbf{24.22}& 
    \textbf{0.8300}&\textbf{0.0403}\\
    
\bottomrule
\end{tabular}
}
% \end{wraptable}



The qualitative comparisons on the Mushroom and ScanNet dataset 
are illustrated in \figref{fig:compare}. 
%
%
As seen in the first row of \figref{fig:compare}, 
both the original 3DGS and 2DGS suffer from overfitting, 
leading to corrupted geometry generation. 
%
Our FDS effectively mitigates this issue by 
supervising the matching relationship between 
the input and sampled views, 
helping to recover the geometry.
%
FDS also improves the refinement of geometric details, 
as shown in other rows. 
By incorporating the matching prior through FDS, 
the quality of the rendered depth is significantly improved.
%

\begin{table}[t] \centering
\begin{minipage}[t]{0.96\linewidth}
        \captionof{table}{\textbf{3D Reconstruction 
        and novel view synthesis results on Mushroom dataset. * 
        Represents that FDS uses the Raft model.
        }}
        \label{tab:mushroom}
        \resizebox{\textwidth}{!}{
\begin{tabular}{c| c c c c c | c c c c c}
    \hline
     Method &  Acc$\downarrow$ & Comp $\downarrow$ & C-L1 $\downarrow$ & NC $\uparrow$ & F-Score $\uparrow$ &  Abs Rel $\downarrow$ &  PSNR $\uparrow$  & SSIM  $\uparrow$ & LPIPS $\downarrow$ & Time  $\downarrow$ \\ \hline

    % DN-splatter &   &  &  &  &  &  &  &  & \\
    GOF &  0.1812 & 0.1093 & 0.1453 & 0.6292 & 0.3665 & 0.2380  & 21.37  &  0.7762  & 0.3132  & $\approx$1.4h\\ 
    PGSR &  0.0971 & 0.1420 & 0.1196 & 0.7193 & 0.5105 & 0.1723  & 22.13  & 0.7773  & 0.2918  & $\approx$1.2h \\ \hline
    3DGS &   0.1167 &  0.1033&  0.1100&  0.7954&  0.3739&  0.1214&  24.18&  0.8392& 0.2511 &$\approx$0.8h \\
    3DGS + FDS$^*$ & 0.0569  & 0.0676 & 0.0623 & 0.8105 & 0.6573 & 0.0603 & 24.72  & 0.8489 & 0.2379 &$\approx$1.3h \\
    3DGS + FDS & \textbf{0.0527}  & \textbf{0.0565} & \textbf{0.0546} & \textbf{0.8178} & \textbf{0.6958} & \textbf{0.0568} & \textbf{24.76}  & \textbf{0.8486} & \textbf{0.2381} &$\approx$1.3h \\ \hline
    2DGS&   0.1078&  0.0850&  0.0964&  0.7835&  0.5170&  0.1002&  23.56&  0.8166& 0.2730 &$\approx$0.8h\\
    2DGS + FDS$^*$ &  0.0689 &  0.0646& 0.0667& 0.8042& 0.6582& 0.0589& 23.98&  0.8255&0.2621 &$\approx$1.3h\\
    2DGS + FDS &  \textbf{0.0615} & \textbf{ 0.0534}& \textbf{0.0574}& \textbf{0.8151}& \textbf{0.6974}&  \textbf{0.0561}&  \textbf{24.06}&  \textbf{0.8271}&\textbf{0.2610} &$\approx$1.3h \\ \hline
\end{tabular}
}
\end{minipage}\hfill
\end{table}

\begin{table}[t] \centering
\begin{minipage}[t]{0.96\linewidth}
        \captionof{table}{\textbf{3D Reconstruction 
        and novel view synthesis results on ScanNet dataset.}}
        \label{tab:scannet}
        \resizebox{\textwidth}{!}{
\begin{tabular}{c| c c c c c | c c c c }
    \hline
     Method &  Acc $\downarrow$ & Comp $\downarrow$ & C-L1 $\downarrow$ & NC $\uparrow$ & F-Score $\uparrow$ &  Abs Rel $\downarrow$ &  PSNR $\uparrow$  & SSIM  $\uparrow$ & LPIPS $\downarrow$ \\ \hline
    GOF & 1.8671  & 0.0805 & 0.9738 & 0.5622 & 0.2526 & 0.1597  & 21.55  & 0.7575  & 0.3881 \\
    PGSR &  0.2928 & 0.5103 & 0.4015 & 0.5567 & 0.1926 & 0.1661  & 21.71 & 0.7699  & 0.3899 \\ \hline

    3DGS &  0.4867 & 0.1211 & 0.3039 & 0.7342& 0.3059 & 0.1227 & 22.19& 0.7837 & 0.3907\\
    3DGS + FDS &  \textbf{0.2458} & \textbf{0.0787} & \textbf{0.1622} & \textbf{0.7831} & 
    \textbf{0.4482} & \textbf{0.0573} & \textbf{22.83} & \textbf{0.7911} & \textbf{0.3826} \\ \hline
    2DGS &  0.2658 & 0.0845 & 0.1752 & 0.7504& 0.4464 & 0.0831 & 22.59& 0.7881 & 0.3854\\
    2DGS + FDS &  \textbf{0.1457} & \textbf{0.0679} & \textbf{0.1068} & \textbf{0.7883} & 
    \textbf{0.5459} & \textbf{0.0432} & \textbf{22.91} & \textbf{0.7928} & \textbf{0.3800} \\ \hline
\end{tabular}
}
\end{minipage}\hfill
\end{table}


\begin{table}[t] \centering
\begin{minipage}[t]{0.96\linewidth}
        \captionof{table}{\textbf{Ablation study on geometry priors.}}
        \label{tab:analysis_prior}
        \resizebox{\textwidth}{!}{
\begin{tabular}{c| c c c c c | c c c c}

    \hline
     Method &  Acc$\downarrow$ & Comp $\downarrow$ & C-L1 $\downarrow$ & NC $\uparrow$ & F-Score $\uparrow$ &  Abs Rel $\downarrow$ &  PSNR $\uparrow$  & SSIM  $\uparrow$ & LPIPS $\downarrow$ \\ \hline
    2DGS&   0.1078&  0.0850&  0.0964&  0.7835&  0.5170&  0.1002&  23.56&  0.8166& 0.2730\\
    2DGS+Depth&   0.0862&  0.0702&  0.0782&  0.8153&  0.5965&  0.0672&  23.92&  0.8227& 0.2619 \\
    2DGS+MVDepth&   0.2065&  0.0917&  0.1491&  0.7832&  0.3178&  0.0792&  23.74&  0.8193& 0.2692 \\
    2DGS+Normal&   0.0939&  0.0637&  0.0788&  \textbf{0.8359}&  0.5782&  0.0768&  23.78&  0.8197& 0.2676 \\
    2DGS+FDS &  \textbf{0.0615} & \textbf{ 0.0534}& \textbf{0.0574}& 0.8151& \textbf{0.6974}&  \textbf{0.0561}&  \textbf{24.06}&  \textbf{0.8271}&\textbf{0.2610} \\ \hline
    2DGS+Depth+FDS &  0.0561 &  0.0519& 0.0540& 0.8295& 0.7282&  0.0454&  \textbf{24.22}& \textbf{0.8291}&\textbf{0.2570} \\
    2DGS+Normal+FDS &  \textbf{0.0529} & \textbf{ 0.0450}& \textbf{0.0490}& \textbf{0.8477}& \textbf{0.7430}&  \textbf{0.0443}&  24.10&  0.8283& 0.2590 \\
    2DGS+Depth+Normal &  0.0695 & 0.0513& 0.0604& 0.8540&0.6723&  0.0523&  24.09&  0.8264&0.2575\\ \hline
    2DGS+Depth+Normal+FDS &  \textbf{0.0506} & \textbf{0.0423}& \textbf{0.0464}& \textbf{0.8598}&\textbf{0.7613}&  \textbf{0.0403}&  \textbf{24.22}& 
    \textbf{0.8300}&\textbf{0.0403}\\
    
\bottomrule
\end{tabular}
}
\end{minipage}\hfill
\end{table}




\subsubsection{Mesh extraction}
To further demonstrate the improvement in geometry quality, 
we applied methods used in ~\citep{turkulainen2024dnsplatter} 
to extract meshes from the input views of optimized 3DGS. 
The comparison results are presented  
in \tabref{tab:mushroom}. 
With the integration of FDS, the mesh quality is significantly enhanced compared to the baseline, featuring fewer floaters and more well-defined shapes.
 %
% Following the incorporation of FDS, the reconstruction 
% results exhibit fewer floaters and more well-defined 
% shapes in the meshes. 
% Visualized comparisons
% are provided in the supplementary material.

% \begin{figure}[t] \centering
%     \makebox[0.19\textwidth]{\scriptsize GT}
%     \makebox[0.19\textwidth]{\scriptsize 3DGS}
%     \makebox[0.19\textwidth]{\scriptsize 3DGS+FDS}
%     \makebox[0.19\textwidth]{\scriptsize 2DGS}
%     \makebox[0.19\textwidth]{\scriptsize 2DGS+FDS} \\

%     \includegraphics[width=0.19\textwidth]{figure/fig4_img/compare1/gt02.png}
%     \includegraphics[width=0.19\textwidth]{figure/fig4_img/compare1/baseline06.png}
%     \includegraphics[width=0.19\textwidth]{figure/fig4_img/compare1/baseline_fds05.png}
%     \includegraphics[width=0.19\textwidth]{figure/fig4_img/compare1/2dgs04.png}
%     \includegraphics[width=0.19\textwidth]{figure/fig4_img/compare1/2dgs_fds03.png} \\

%     \includegraphics[width=0.19\textwidth]{figure/fig4_img/compare2/gt00.png}
%     \includegraphics[width=0.19\textwidth]{figure/fig4_img/compare2/baseline02.png}
%     \includegraphics[width=0.19\textwidth]{figure/fig4_img/compare2/baseline_fds01.png}
%     \includegraphics[width=0.19\textwidth]{figure/fig4_img/compare2/2dgs04.png}
%     \includegraphics[width=0.19\textwidth]{figure/fig4_img/compare2/2dgs_fds03.png} \\
      
%     \includegraphics[width=0.19\textwidth]{figure/fig4_img/compare3/gt05.png}
%     \includegraphics[width=0.19\textwidth]{figure/fig4_img/compare3/3dgs03.png}
%     \includegraphics[width=0.19\textwidth]{figure/fig4_img/compare3/3dgs_fds04.png}
%     \includegraphics[width=0.19\textwidth]{figure/fig4_img/compare3/2dgs02.png}
%     \includegraphics[width=0.19\textwidth]{figure/fig4_img/compare3/2dgs_fds01.png} \\

%     \caption{\textbf{Qualitative comparison of extracted mesh 
%     on Mushroom and ScanNet datasets.}}
%     \label{fig:mesh}
% \end{figure}












\subsection{Ablation study}


\textbf{Ablation study on geometry priors:} 
To highlight the advantage of incorporating matching priors, 
we incorporated various types of priors generated by different 
models into 2DGS. These include a monocular depth estimation
model (Depth Anything v2)~\citep{yang2024depth}, a two-view depth estimation 
model (Unimatch)~\citep{xu2023unifying}, 
and a monocular normal estimation model (DSINE)~\citep{bae2024rethinking}.
We adapt the scale and shift-invariant loss in Midas~\citep{birkl2023midas} for
monocular depth supervision and L1 loss for two-view depth supervison.
%
We use Sea Raft~\citep{wang2025sea} as our default optical flow model.
%
The comparison results on Mushroom dataset 
are shown in ~\tabref{tab:analysis_prior}.
We observe that the normal prior provides accurate shape information, 
enhancing the geometric quality of the radiance field. 
%
% In contrast, the monocular depth prior slightly increases 
% the 'Abs Rel' due to its ambiguous scale and inaccurate depth ordering.
% Moreover, the performance of monocular depth estimation 
% in the sauna scene is particularly poor, 
% primarily due to the presence of numerous reflective 
% surfaces and textureless walls, which limits the accuracy of monocular depth estimation.
%
The multi-view depth prior, hindered by the limited feature overlap 
between input views, fails to offer reliable geometric 
information. We test average "Abs Rel" of multi-view depth prior
, and the result is 0.19, which performs worse than the "Abs Rel" results 
rendered by original 2DGS.
From the results, it can be seen that depth order information provided by monocular depth improves
reconstruction accuracy. Meanwhile, our FDS achieves the best performance among all the priors, 
and by integrating all
three components, we obtained the optimal results.
%
%
\begin{figure}[t] \centering
    \makebox[0.16\textwidth]{\scriptsize RF (16000 iters)}
    \makebox[0.16\textwidth]{\scriptsize RF* (20000 iters)}
    \makebox[0.16\textwidth]{\scriptsize RF (20000 iters)  }
    \makebox[0.16\textwidth]{\scriptsize PF (16000 iters)}
    \makebox[0.16\textwidth]{\scriptsize PF (20000 iters)}


    % \includegraphics[width=0.16\textwidth]{figure/fig5_img/compare1/16000.png}
    % \includegraphics[width=0.16\textwidth]{figure/fig5_img/compare1/20000_wo_flow_loss.png}
    % \includegraphics[width=0.16\textwidth]{figure/fig5_img/compare1/20000.png}
    % \includegraphics[width=0.16\textwidth]{figure/fig5_img/compare1/16000_prior.png}
    % \includegraphics[width=0.16\textwidth]{figure/fig5_img/compare1/20000_prior.png}\\

    % \includegraphics[width=0.16\textwidth]{figure/fig5_img/compare2/16000.png}
    % \includegraphics[width=0.16\textwidth]{figure/fig5_img/compare2/20000_wo_flow_loss.png}
    % \includegraphics[width=0.16\textwidth]{figure/fig5_img/compare2/20000.png}
    % \includegraphics[width=0.16\textwidth]{figure/fig5_img/compare2/16000_prior.png}
    % \includegraphics[width=0.16\textwidth]{figure/fig5_img/compare2/20000_prior.png}\\

    \includegraphics[width=0.16\textwidth]{figure/fig5_img/compare3/16000.png}
    \includegraphics[width=0.16\textwidth]{figure/fig5_img/compare3/20000_wo_flow_loss.png}
    \includegraphics[width=0.16\textwidth]{figure/fig5_img/compare3/20000.png}
    \includegraphics[width=0.16\textwidth]{figure/fig5_img/compare3/16000_prior.png}
    \includegraphics[width=0.16\textwidth]{figure/fig5_img/compare3/20000_prior.png}\\
    
    \includegraphics[width=0.16\textwidth]{figure/fig5_img/compare4/16000.png}
    \includegraphics[width=0.16\textwidth]{figure/fig5_img/compare4/20000_wo_flow_loss.png}
    \includegraphics[width=0.16\textwidth]{figure/fig5_img/compare4/20000.png}
    \includegraphics[width=0.16\textwidth]{figure/fig5_img/compare4/16000_prior.png}
    \includegraphics[width=0.16\textwidth]{figure/fig5_img/compare4/20000_prior.png}\\

    \includegraphics[width=0.30\textwidth]{figure/fig5_img/bar.png}

    \caption{\textbf{The error map of Radiance Flow and Prior Flow.} RF: Radiance Flow, PF: Prior Flow, * means that there is no FDS loss supervision during optimization.}
    \label{fig:error_map}
\end{figure}




\textbf{Ablation study on FDS: }
In this section, we present the design of our FDS 
method through an ablation study on the 
Mushroom dataset to validate its effectiveness.
%
The optional configurations of FDS are outlined in ~\tabref{tab:ablation_fds}.
Our base model is the 2DGS equipped with FDS,
and its results are shown 
in the first row. The goal of this analysis 
is to evaluate the impact 
of various strategies on FDS sampling and loss design.
%
We observe that when we 
replace $I_i$ in \eqref{equ:mflow} with $C_i$, 
as shown in the second row, the geometric quality 
of 2DGS deteriorates. Using $I_i$ instead of $C_i$ 
help us to remove the floaters in $\bm{C^s}$, which are also 
remained in $\bm{C^i}$.
We also experiment with modifying the FDS loss. For example, 
in the third row, we use the neighbor 
input view as the sampling view, and replace the 
render result of neighbor view with ground truth image of its input view.
%
Due to the significant movement between images, the Prior Flow fails to accurately 
match the pixel between them, leading to a further degradation in geometric quality.
%
Finally, we attempt to fix the sampling view 
and found that this severely damaged the geometric quality, 
indicating that random sampling is essential for the stability 
of the mean error in the Prior flow.



\begin{table}[t] \centering

\begin{minipage}[t]{1.0\linewidth}
        \captionof{table}{\textbf{Ablation study on FDS strategies.}}
        \label{tab:ablation_fds}
        \resizebox{\textwidth}{!}{
\begin{tabular}{c|c|c|c|c|c|c|c}
    \hline
    \multicolumn{2}{c|}{$\mathcal{M}_{\theta}(X, \bm{C^s})$} & \multicolumn{3}{c|}{Loss} & \multicolumn{3}{c}{Metric}  \\
    \hline
    $X=C^i$ & $X=I^i$  & Input view & Sampled view     & Fixed Sampled view        & Abs Rel $\downarrow$ & F-score $\uparrow$ & NC $\uparrow$ \\
    \hline
    & \ding{51} &     &\ding{51}    &    &    \textbf{0.0561}        &  \textbf{0.6974}         & \textbf{0.8151}\\
    \hline
     \ding{51} &           &     &\ding{51}    &    &    0.0839        &  0.6242         &0.8030\\
     &  \ding{51} &   \ding{51}  &    &    &    0.0877       & 0.6091        & 0.7614 \\
      &  \ding{51} &    &    & \ding{51}    &    0.0724           & 0.6312          & 0.8015 \\
\bottomrule
\end{tabular}
}
\end{minipage}
\end{table}




\begin{figure}[htbp] \centering
    \makebox[0.22\textwidth]{}
    \makebox[0.22\textwidth]{}
    \makebox[0.22\textwidth]{}
    \makebox[0.22\textwidth]{}
    \\

    \includegraphics[width=0.22\textwidth]{figure/fig6_img/l1/rgb/frame00096.png}
    \includegraphics[width=0.22\textwidth]{figure/fig6_img/l1/render_rgb/frame00096.png}
    \includegraphics[width=0.22\textwidth]{figure/fig6_img/l1/render_depth/frame00096.png}
    \includegraphics[width=0.22\textwidth]{figure/fig6_img/l1/depth/frame00096.png}

    % \includegraphics[width=0.22\textwidth]{figure/fig6_img/l2/rgb/frame00112.png}
    % \includegraphics[width=0.22\textwidth]{figure/fig6_img/l2/render_rgb/frame00112.png}
    % \includegraphics[width=0.22\textwidth]{figure/fig6_img/l2/render_depth/frame00112.png}
    % \includegraphics[width=0.22\textwidth]{figure/fig6_img/l2/depth/frame00112.png}

    \caption{\textbf{Limitation of FDS.} }
    \label{fig:limitation}
\end{figure}


% \begin{figure}[t] \centering
%     \makebox[0.48\textwidth]{}
%     \makebox[0.48\textwidth]{}
%     \\
%     \includegraphics[width=0.48\textwidth]{figure/loss_Ignatius.pdf}
%     \includegraphics[width=0.48\textwidth]{figure/loss_family.pdf}
%     \caption{\textbf{Comparison the photometric error of Radiance Flow and Prior Flow:} 
%     We add FDS method after 2k iteration during training.
%     The results show
%     that:  1) The Prior Flow is more precise and 
%     robust than Radiance Flow during the radiance 
%     optimization; 2) After adding the FDS loss 
%     which utilize Prior 
%     flow to supervise the Radiance Flow at 2k iterations, 
%     both flow are more accurate, which lead to
%     a mutually reinforcing effects.(TODO fix it)} 
%     \label{fig:flowcompare}
% \end{figure}






\textbf{Interpretive Experiments: }
To demonstrate the mutual refinement of two flows in our FDS, 
For each view, we sample the unobserved 
views multiple times to compute the mean error 
of both Radiance Flow and Prior Flow. 
We use Raft~\citep{teed2020raft} as our default optical flow model
for visualization.
The ground truth flow is calculated based on 
~\eref{equ:flow_pose} and ~\eref{equ:flow} 
utilizing ground truth depth in dataset.
We introduce our FDS loss after 16000 iterations during 
optimization of 2DGS.
The error maps are shown in ~\figref{fig:error_map}.
Our analysis reveals that Radiance Flow tends to 
exhibit significant geometric errors, 
whereas Prior Flow can more accurately estimate the geometry,
effectively disregarding errors introduced by floating Gaussian points. 

%





\subsection{Limitation and further work}

Firstly, our FDS faces challenges in scenes with 
significant lighting variations between different 
views, as shown in the lamp of first row in ~\figref{fig:limitation}. 
%
Incorporating exposure compensation into FDS could help address this issue. 
%
 Additionally, our method struggles with 
 reflective surfaces and motion blur,
 leading to incorrect matching. 
 %
 In the future, we plan to explore the potential 
 of FDS in monocular video reconstruction tasks, 
 using only a single input image at each time step.
 


\section{Conclusions}
In this paper, we propose Flow Distillation Sampling (FDS), which
leverages the matching prior between input views and 
sampled unobserved views from the pretrained optical flow model, to improve the geometry quality
of Gaussian radiance field. 
Our method can be applied to different approaches (3DGS and 2DGS) to enhance the geometric rendering quality of the corresponding neural radiance fields.
We apply our method to the 3DGS-based framework, 
and the geometry is enhanced on the Mushroom, ScanNet, and Replica datasets.

\section*{Acknowledgements} This work was supported by 
National Key R\&D Program of China (2023YFB3209702), 
the National Natural Science Foundation of 
China (62441204, 62472213), and Gusu 
Innovation \& Entrepreneurship Leading Talents Program (ZXL2024361)
\section{Conclusion}
In this work, we propose a simple yet effective approach, called SMILE, for graph few-shot learning with fewer tasks. Specifically, we introduce a novel dual-level mixup strategy, including within-task and across-task mixup, for enriching the diversity of nodes within each task and the diversity of tasks. Also, we incorporate the degree-based prior information to learn expressive node embeddings. Theoretically, we prove that SMILE effectively enhances the model's generalization performance. Empirically, we conduct extensive experiments on multiple benchmarks and the results suggest that SMILE significantly outperforms other baselines, including both in-domain and cross-domain few-shot settings.


%%
%% The next two lines define the bibliography style to be used, and
%% the bibliography file.
\bibliographystyle{ACM-Reference-Format}
\bibliography{main}


%%
%% If your work has an appendix, this is the place to put it.
\appendix
\onecolumn
\section{More results about Mode connectivity in GNN}
\label{app: A}

\subsection{The performance of linear interpolations 
between two minima.}

\begin{figure*}[!ht]
    \centering
    \begin{subfigure}[b]{0.24\textwidth}
        \includegraphics[width=1.0\textwidth]{fig/init/init/cora_gcn_plot.png}
        \caption{Cora}
    \end{subfigure}
    \begin{subfigure}[b]{0.24\textwidth}
        \includegraphics[width=1.0\textwidth]{fig/init/init/citeseer_gcn_plot.png}
        \caption{CiteSeer}
    \end{subfigure}
    \begin{subfigure}[b]{0.24\textwidth}
        \includegraphics[width=1.0\textwidth]{fig/init/init/pubmed_gcn_plot.png}
        \caption{PubMed}
    \end{subfigure}
    \begin{subfigure}[b]{0.24\textwidth}
        \includegraphics[width=1.0\textwidth]{fig/init/init/amazon-computer_gcn_plot.png}
        \caption{Amazon-Computer}
    \end{subfigure}
    \begin{subfigure}[b]{0.24\textwidth}
        \includegraphics[width=1.0\textwidth]{fig/init/init/amazon-photo_gcn_plot.png}
        \caption{Amazon-Photo}
    \end{subfigure}
    \begin{subfigure}[b]{0.24\textwidth}
        \includegraphics[width=1.0\textwidth]{fig/init/init/coauthor-cs_gcn_plot.png}
        \caption{Coauthor-CS}
    \end{subfigure}
    \begin{subfigure}[b]{0.24\textwidth}
        \includegraphics[width=1.0\textwidth]{fig/init/init/coauthor-physics_gcn_plot.png}
        \caption{Coauthor-Physics}
    \end{subfigure}
    \begin{subfigure}[b]{0.24\textwidth}
        \includegraphics[width=1.0\textwidth]{fig/init/init/wikics_gcn_plot.png}
        \caption{WikiCS}
    \end{subfigure}
    \begin{subfigure}[b]{0.24\textwidth}
        \includegraphics[width=1.0\textwidth]{fig/init/init/squirrel_gcn_plot.png}
        \caption{Squirrel}
    \end{subfigure}
    \begin{subfigure}[b]{0.24\textwidth}
        \includegraphics[width=1.0\textwidth]{fig/init/init/chameleon_gcn_plot.png}
        \caption{Chameleon}
    \end{subfigure}
    \begin{subfigure}[b]{0.24\textwidth}
        \includegraphics[width=1.0\textwidth]{fig/init/init/roman-empire_gcn_plot.png}
        \caption{Roman-Empire}
    \end{subfigure}
    \begin{subfigure}[b]{0.24\textwidth}
        \includegraphics[width=1.0\textwidth]{fig/init/init/amazon-ratings_gcn_plot.png}
        \caption{Amazon-Ratings }
    \end{subfigure}
        \begin{subfigure}[b]{0.24\textwidth}
        \includegraphics[width=1.0\textwidth]{fig/init/init/minesweeper_gcn_plot.png}
        \caption{Minesweeper}
    \end{subfigure}
    \caption{The performance of interpolations along a non-linear path connecting two minima.}
    \label{fig:A1}
\end{figure*}

\clearpage



\subsection{ The performance of interpolations along quadratic Bézier curve connecting two minima}
\begin{figure*}[!ht]
    \centering
    \begin{subfigure}[b]{0.24\textwidth}
        \includegraphics[width=1.0\textwidth]{fig/Bezier/Bezier/cora_bezier_fitted_plot.png}
        \caption{Cora}
    \end{subfigure}
    \begin{subfigure}[b]{0.24\textwidth}
        \includegraphics[width=1.0\textwidth]{fig/Bezier/Bezier/citeseer_bezier_fitted_plot.png}
        \caption{CiteSeer}
    \end{subfigure}
    \begin{subfigure}[b]{0.24\textwidth}
        \includegraphics[width=1.0\textwidth]{fig/Bezier/Bezier/pubmed_bezier_fitted_plot.png}
        \caption{PubMed}
    \end{subfigure}
    \begin{subfigure}[b]{0.24\textwidth}
        \includegraphics[width=1.0\textwidth]{fig/Bezier/Bezier/amazon-computer_bezier_fitted_plot.png}
        \caption{Amazon-Computer}
    \end{subfigure}
    \begin{subfigure}[b]{0.24\textwidth}
        \includegraphics[width=1.0\textwidth]{fig/Bezier/Bezier/amazon-photo_bezier_fitted_plot.png}
        \caption{Amazon-Photo}
    \end{subfigure}
    \begin{subfigure}[b]{0.24\textwidth}
        \includegraphics[width=1.0\textwidth]{fig/Bezier/Bezier/coauthor-cs_bezier_fitted_plot.png}
        \caption{Coauthor-CS}
    \end{subfigure}
    \begin{subfigure}[b]{0.24\textwidth}
        \includegraphics[width=1.0\textwidth]{fig/Bezier/Bezier/coauthor-physics_bezier_fitted_plot.png}
        \caption{Coauthor-Physics}
    \end{subfigure}
    \begin{subfigure}[b]{0.24\textwidth}
        \includegraphics[width=1.0\textwidth]{fig/Bezier/Bezier/wikics_bezier_fitted_plot.png}
        \caption{WikiCS}
    \end{subfigure}
    \begin{subfigure}[b]{0.24\textwidth}
        \includegraphics[width=1.0\textwidth]{fig/Bezier/Bezier/squirrel_bezier_fitted_plot.png}
        \caption{Squirrel}
    \end{subfigure}
    \begin{subfigure}[b]{0.24\textwidth}
        \includegraphics[width=1.0\textwidth]{fig/Bezier/Bezier/chameleon_bezier_fitted_plot.png}
        \caption{Chameleon}
    \end{subfigure}
    \begin{subfigure}[b]{0.24\textwidth}
        \includegraphics[width=1.0\textwidth]{fig/Bezier/Bezier/roman-empire_bezier_fitted_plot.png}
        \caption{Roman-Empire}
    \end{subfigure}
    \begin{subfigure}[b]{0.24\textwidth}
        \includegraphics[width=1.0\textwidth]{fig/Bezier/Bezier/amazon-ratings_bezier_fitted_plot.png}
        \caption{Amazon-Ratings }
    \end{subfigure}
        \begin{subfigure}[b]{0.24\textwidth}
        \includegraphics[width=1.0\textwidth]{fig/Bezier/Bezier/minesweeper_bezier_fitted_plot.png}
        \caption{Minesweeper}
    \end{subfigure}
    \caption{The performance of interpolations along a non-linear path connecting two minima.}
    \label{fig:A2}
\end{figure*}
\clearpage



\subsection{Effect of convolution mechanism on mode connectivity}
\label{app: conv}
\begin{figure*}[!ht]
    \centering
    \begin{subfigure}[b]{0.24\textwidth}
        \includegraphics[width=1.0\textwidth]{fig/init_model/cora_comparison.png}
        \caption{Cora}
    \end{subfigure}
    \begin{subfigure}[b]{0.24\textwidth}
        \includegraphics[width=1.0\textwidth]{fig/init_model/citeseer_comparison.png}
        \caption{CiteSeer}
    \end{subfigure}
    \begin{subfigure}[b]{0.24\textwidth}
        \includegraphics[width=1.0\textwidth]{fig/init_model/pubmed_comparison.png}
        \caption{PubMed}
    \end{subfigure}
    \begin{subfigure}[b]{0.24\textwidth}
        \includegraphics[width=1.0\textwidth]{fig/init_model/amazon-computer_comparison.png}
        \caption{Amazon-Computer}
    \end{subfigure}
    \begin{subfigure}[b]{0.24\textwidth}
        \includegraphics[width=1.0\textwidth]{fig/init_model/amazon-photo_comparison.png}
        \caption{Amazon-Photo}
    \end{subfigure}
    \begin{subfigure}[b]{0.24\textwidth}
        \includegraphics[width=1.0\textwidth]{fig/init_model/coauthor-cs_comparison.png}
        \caption{Coauthor-CS}
    \end{subfigure}
    \begin{subfigure}[b]{0.24\textwidth}
        \includegraphics[width=1.0\textwidth]{fig/init_model/coauthor-physics_comparison.png}
        \caption{Coauthor-Physics}
    \end{subfigure}
    \begin{subfigure}[b]{0.24\textwidth}
        \includegraphics[width=1.0\textwidth]{fig/init_model/wikics_comparison.png}
        \caption{WikiCS}
    \end{subfigure}
    \begin{subfigure}[b]{0.24\textwidth}
        \includegraphics[width=1.0\textwidth]{fig/init_model/squirrel_comparison.png}
        \caption{Squirrel}
    \end{subfigure}
    \begin{subfigure}[b]{0.24\textwidth}
        \includegraphics[width=1.0\textwidth]{fig/init_model/chameleon_comparison.png}
        \caption{Chameleon}
    \end{subfigure}
    \begin{subfigure}[b]{0.24\textwidth}
        \includegraphics[width=1.0\textwidth]{fig/init_model/roman-empire_comparison.png}
        \caption{Roman-Empire}
    \end{subfigure}
    \begin{subfigure}[b]{0.24\textwidth}
        \includegraphics[width=1.0\textwidth]{fig/init_model/amazon-ratings_comparison.png}
        \caption{Amazon-Ratings }
    \end{subfigure}
        \begin{subfigure}[b]{0.24\textwidth}
        \includegraphics[width=1.0\textwidth]{fig/init_model/minesweeper_comparison.png}
        \caption{Minesweeper}
    \end{subfigure}
            \begin{subfigure}[b]{0.24\textwidth}
        \includegraphics[width=1.0\textwidth]{fig/init_model/legend.png}
        \caption{Legend}
    \end{subfigure}
    \caption{Performance of mode connectivity on different convolution mechanisms.}
    \label{fig:A3}
\end{figure*}

\clearpage
\subsection{Visualization of Loss basin and minimas }
\begin{figure*}[!ht]
    \centering
    \begin{subfigure}[b]{0.24\textwidth}
        \includegraphics[width=1.0\textwidth]{fig/basin/cora_basin.png}
        \caption{Cora}
    \end{subfigure}
    \begin{subfigure}[b]{0.24\textwidth}
        \includegraphics[width=1.0\textwidth]{fig/basin/Basin/loss_landscape_fixed_citeseer.png}
        \caption{CiteSeer}
    \end{subfigure}
    \begin{subfigure}[b]{0.24\textwidth}
        \includegraphics[width=1.0\textwidth]{fig/basin/Basin/loss_landscape_fixed_pubmed.png}
        \caption{PubMed}
    \end{subfigure}
    \begin{subfigure}[b]{0.24\textwidth}
        \includegraphics[width=1.0\textwidth]{fig/basin/Basin/loss_landscape_fixed_amazon-computer.png}
        \caption{Amazon-Computer}
    \end{subfigure}
    \begin{subfigure}[b]{0.24\textwidth}
        \includegraphics[width=1.0\textwidth]{fig/basin/Basin/loss_landscape_fixed_amazon-photo.png}
        \caption{Amazon-Photo}
    \end{subfigure}
    \begin{subfigure}[b]{0.24\textwidth}
        \includegraphics[width=1.0\textwidth]{fig/basin/Basin/loss_landscape_fixed_coauthor-cs.png}
        \caption{Coauthor-CS}
    \end{subfigure}
    \begin{subfigure}[b]{0.24\textwidth}
        \includegraphics[width=1.0\textwidth]{fig/basin/Basin/loss_landscape_fixed_coauthor-physics.png}
        \caption{Coauthor-Physics}
    \end{subfigure}
    \begin{subfigure}[b]{0.24\textwidth}
        \includegraphics[width=1.0\textwidth]{fig/basin/Basin/loss_landscape_fixed_wikics.png}
        \caption{WikiCS}
    \end{subfigure}
    \begin{subfigure}[b]{0.24\textwidth}
        \includegraphics[width=1.0\textwidth]{fig/basin/Basin/loss_landscape_fixed_squirrel.png}
        \caption{Squirrel}
    \end{subfigure}
    \begin{subfigure}[b]{0.24\textwidth}
        \includegraphics[width=1.0\textwidth]{fig/basin/Basin/loss_landscape_fixed_chameleon.png}
        \caption{Chameleon}
    \end{subfigure}
    \begin{subfigure}[b]{0.24\textwidth}
        \includegraphics[width=1.0\textwidth]{fig/basin/Basin/loss_landscape_fixed_roman-empire.png}
        \caption{Roman-Empire}
    \end{subfigure}
    \begin{subfigure}[b]{0.24\textwidth}
        \includegraphics[width=1.0\textwidth]{fig/basin/Basin/loss_landscape_fixed_amazon-ratings.png}
        \caption{Amazon-Ratings }
    \end{subfigure}
        \begin{subfigure}[b]{0.24\textwidth}
        \includegraphics[width=1.0\textwidth]{fig/basin/Basin/loss_landscape_fixed_minesweeper.png}
        \caption{Minesweeper}
    \end{subfigure}
    \caption{Performance of mode connectivity on different convolution mechanisms.}
    \label{fig:A4}
\end{figure*}


% \subsection{Error Gap Between Aligned and Misaligned Data}\label{subsec:proof-align-misalign}







\thmalignment*

\begin{proof}

For the aligned case, we can derive the mean squared error (MSE) as follows:
\begin{equation}\label{eq:mse_aligned}
    \mathrm{MSE}_\mathrm{aligned} = \inf_{\boldsymbol{\alpha} \in R^{m^P}, \boldsymbol{\beta} \in R^{m^S}} \|\mathbf{y} - \mathbf{X}^P \boldsymbol{\alpha} - \mathbf{X}^S \boldsymbol{\beta}\|
\end{equation}
The ordinary least squares (OLS) estimator of $\boldsymbol{\alpha}$ is given by:
\begin{equation}
    \hat{\boldsymbol{\alpha}} := (\mathbf{X}^{P \top} \mathbf{X}^P)^{-1} \mathbf{X}^P (\mathbf{y} - \mathbb{E}[\mathbf{R}] \mathbf{X}^S \boldsymbol{\beta}) 
\end{equation}
For a permutation matrix $\mathbf{R}$ under uniform distribution, we have $\mathbb{E}[\mathbf{R}] = \frac{1}{n}\mathds{1}^\top \mathds{1}$. Therefore:
\begin{equation}\label{eq:alpha_hat}
    \hat{\boldsymbol{\alpha}} = (\mathbf{X}^{P \top} \mathbf{X}^P)^{-1} \mathbf{X}^P (\mathbf{y} - \frac{1}{n} \mathds{1}^\top \mathds{1} \mathbf{X}^S \boldsymbol{\beta}) 
\end{equation}
The MSE for the misaligned case can be expressed as:
\begin{align}
    \mathrm{MSE}_{\mathrm{misaligned}} 
    & = \inf_{\boldsymbol{\beta}} \inf_{\boldsymbol{\alpha}} \mathbb{E}_\mathbf{R} \|\mathbf{y} - \mathbf{X}^P \boldsymbol{\alpha} - \mathbf{R} \mathbf{X}^S \boldsymbol{\beta}\|_2^2 \\
    & = \inf_{\boldsymbol{\beta}} \mathbb{E}_\mathbf{R} \|\mathbf{y} - \mathbf{X}^P \hat{\boldsymbol{\alpha}} - \mathbf{R} \mathbf{X}^S \boldsymbol{\beta}\|_2^2 \\
\end{align}
Substituting $\hat{\boldsymbol{\alpha}}$ from equation~\ref{eq:alpha_hat}, we obtain:
\begin{align}
    \mathrm{MSE}_{\mathrm{misaligned}} 
    & = \inf_{\boldsymbol{\beta}} \mathbb{E}_\mathbf{R} \left\|\mathbf{y} - \mathbf{X}^P (\mathbf{X}^{P \top} \mathbf{X}^P)^{-1} (\mathbf{X}^P \mathbf{y} - \mathbf{X}^P \frac{1}{n} 1^\top 1 \mathbf{X}^S \boldsymbol{\beta}) - \mathbf{R} \mathbf{X}^S \boldsymbol{\beta}\right\|_2^2 \\
    & = \inf_{\boldsymbol{\beta}} \mathbb{E}_\mathbf{R} \left\| (\mathbf{I} - \mathbf{X}^P (\mathbf{X}^{P \top} \mathbf{X}^P)^{-1} \mathbf{X}^P)\mathbf{y} + (\mathbf{X}^P (\mathbf{X}^{P \top} \mathbf{X}^P)^{-1} \mathbf{X}^P \frac{1}{n} \mathds{1}^\top \mathds{1} \mathbf{X}^S \boldsymbol{\beta}) - \mathbf{R} \mathbf{X}^S \boldsymbol{\beta}\right\|_2^2 
\end{align}
Since $\mathbf{X}^P (\mathbf{X}^{P \top} \mathbf{X}^P)^{-1} \mathbf{X}^P$ is a projection matrix that projects any vector onto the column space of $\mathbf{X}^P$, and $\mathbf{X}^S \boldsymbol{\beta}$ is orthogonal to the column space of $\mathbf{X}^P$, the term $\mathbf{X}^P (\mathbf{X}^{P \top} \mathbf{X}^P)^{-1} \mathbf{X}^P \frac{1}{n} \mathds{1}^\top \mathds{1} \mathbf{X}^S \boldsymbol{\beta} = 0$. Thus:
\begin{align}
    \mathrm{MSE}_{\mathrm{misaligned}}
    & = \inf_{\boldsymbol{\beta}} \mathbb{E}_\mathbf{R} \left\| (\mathbf{I} - \mathbf{X}^P (\mathbf{X}^{P \top} \mathbf{X}^P)^{-1} \mathbf{X}^P)\mathbf{y} - \mathbf{R} \mathbf{X}^S \boldsymbol{\beta}\right\|_2^2 \\
    & = \inf_{\boldsymbol{\beta}} \mathbb{E}_\mathbf{R} \left[\left\|\mathbf{R} \mathbf{X}^S \boldsymbol{\beta}\right\|_2^2 - 2\left[(\mathbf{I} - \mathbf{X}^P (\mathbf{X}^{P \top} \mathbf{X}^P)^{-1} \mathbf{X}^P)\mathbf{y}\right]^\top \mathbf{R} \mathbf{X}^S \boldsymbol{\beta} + \left\|(\mathbf{I} - \mathbf{X}^P (\mathbf{X}^{P \top} \mathbf{X}^P)^{-1} \mathbf{X}^P)\mathbf{y}\right\|_2^2\right]
\end{align}
By properties of permutation matrices:
\begin{equation}
    \mathbb{E}_\mathbf{R}\| \mathbf{R} \mathbf{X}^S \boldsymbol{\beta}\|_2^2 = \|\mathbf{X}^S \boldsymbol{\beta}\|_2^2; \; \mathbb{E}_\mathbf{R} [\mathbf{R}]= \frac{1}{n}\mathds{1}^\top \mathds{1}
\end{equation}
Therefore:
\begin{align}
    \mathrm{MSE}_{\mathrm{misaligned}}
    & = \inf_{\boldsymbol{\beta}} \left[\left\|\mathbf{X}^S \boldsymbol{\beta}\right\|_2^2 - 2\left[(\mathbf{I} - \mathbf{X}^P (\mathbf{X}^{P \top} \mathbf{X}^P)^{-1} \mathbf{X}^P)\mathbf{y}\right]^\top \frac{1}{n}\mathds{1}^\top \mathds{1} \mathbf{X}^S \boldsymbol{\beta} + \left\|(\mathbf{I} - \mathbf{X}^P (\mathbf{X}^{P \top} \mathbf{X}^P)^{-1} \mathbf{X}^P)\mathbf{y}\right\|_2^2\right]
\end{align}
Since $\mathbf{I} - \mathbf{X}^P (\mathbf{X}^{P \top} \mathbf{X}^P)^{-1} \mathbf{X}^P$ projects any vector onto the orthogonal complement of the column space of $\mathbf{X}^P$, the term $\left[(\mathbf{I} - \mathbf{X}^P (\mathbf{X}^{P \top} \mathbf{X}^P)^{-1} \mathbf{X}^P)\mathbf{y}\right]^\top \frac{1}{n}\mathds{1}^\top \mathds{1} \mathbf{X}^S \boldsymbol{\beta} = 0$. Hence:
\begin{align}
    \mathrm{MSE}_{\mathrm{misaligned}}
    & = \inf_{\boldsymbol{\beta}} \left[\left\|\mathbf{X}^S \boldsymbol{\beta}\right\|_2^2 + \left\|(\mathbf{I} - \mathbf{X}^P (\mathbf{X}^{P \top} \mathbf{X}^P)^{-1} \mathbf{X}^P)\mathbf{y}\right\|_2^2\right] \\
    & = \inf_{\boldsymbol{\beta}} \left\|\mathbf{X}^S \boldsymbol{\beta}\right\|_2^2 + \left\|(\mathbf{I} - \mathbf{X}^P (\mathbf{X}^{P \top} \mathbf{X}^P)^{-1} \mathbf{X}^P)\mathbf{y}\right\|_2^2 \\
\end{align}
The minimum is attained at $\boldsymbol{\beta} = \mathbf{0}$, yielding:
\begin{align}
    \mathrm{MSE}_{\mathrm{misaligned}}
    & = \left\|(\mathbf{I} - \mathbf{X}^P (\mathbf{X}^{P \top} \mathbf{X}^P)^{-1} \mathbf{X}^P)\mathbf{y}\right\|_2^2 \\
    & = \inf_{\boldsymbol{\alpha} \in \mathbb{R}^{m^P}, \boldsymbol{\beta} = \mathbf{0}} \left\|\mathbf{y} - \mathbf{X}^P \boldsymbol{\alpha} - \mathbf{X}^S \boldsymbol{\beta}\right\|_2^2 \\
\end{align}
Comparing with Equation~\ref{eq:mse_aligned}, we conclude:
\begin{equation}
    \mathrm{MSE}_{\mathrm{misaligned}} \geq \inf_{\boldsymbol{\alpha} \in \mathbb{R}^{m^P}, \boldsymbol{\beta} \in \mathbb{R}^{m^S}} \left\|\mathbf{y} - \mathbf{X}^P \boldsymbol{\alpha} - \mathbf{X}^S \boldsymbol{\beta}\right\|_2^2 = \mathrm{MSE}_{\mathrm{aligned}}
\end{equation}
\end{proof}





















\subsection{Approximation Capacity of Cluster Sampler}\label{subsec:proof-cluster-sampler}

\begin{definition}[Definition of optimal cluster sampler]
    Assume the inputs are uniformly bounded by some constant $B$. 
    The optimal cluster sampler is defined by the uniform equi-continuous cluster sampler function which achieves the minimal optimization loss for the prediction task in \cref{fig:leal-framework}.
    \begin{equation}
        \textrm{Optimal cluster sampler} := \arginf_{\textrm{Uniform equi-continuous cluster sampler}} \textrm{Loss}(\textrm{cluster sampler})
    \end{equation}
    The cluster sampler is defined over bounded inputs ($|X^P|_{\infty} \leq B, |X^S|_{\infty} \leq B$) from $\mathbb{R}^{m^P} \times \mathbb{R}^{n^S \times m^S}$ and output in $\mathbb{R}^{n^S}$.
\end{definition}

\begin{remark}
    The existence of such optimal cluster sampler is guaranteed by the boundedness and uniform equi-continuity of the set of cluster sampler functions. 
\end{remark}


\thmclustersampler*

\begin{proof}
    We just need to prove the statement for small $\epsilon \leq 6$.

    The input of cluster sampler is $1 \times m^P$ and output is $n^S \times m^S$, the final prediction is to generate a sample probabilities:
    \begin{equation}
        (n^S * m^S, 1 * m^P) \to (n^S * d, 1 * C) \to (n^S * C, 1 * C) \to n^S * 1. 
    \end{equation}

    Also, since there is no weight depends on dimension $n_2$, we can reduce the approximation statement to that there exists trainable weight such that the continuous function $h$ can be approximated:
    \begin{equation}
        (1 * m^S, 1 * m^P) \to (n^S * d, 1 * C) \to (n^S * C, 1 * C) \to 1 * 1. 
    \end{equation}

    Notice that the layer operation of secondary embedding and trainable centroids weights $(C \times d)$ is continuous and the pretrained encoder as a neural network (which is a universal approximator) can approximates any continuous function $f$ composited with inverse embedding. 
    For simplicity, we will consider $m^P = m^S = 1$. 
    For any continuous function $h(p, s) \in [0, 1]$,
    we just need to show there exists trainable weight $\theta_1$, $\theta_2$ such that 
    \begin{equation}
        f(p; \theta_1) \odot g(s; \theta_2) = \sum_{i=1}^C f_i(p; \theta_1) \odot g_i(s; \theta_2). 
    \end{equation}
    Here $f(p; \theta_1) \in \mathbb{R}^C$ is a function of $p$ parameterized by $\theta_1$ and $g(s; \theta_1) \in \mathbb{R}^C$ is a function of $s$ parameterized by $\theta_2$.  
    As any continuous function $f(p, s)$ has a corresponding Taylor series expansion, it means for any $\epsilon > 0$, there exists $C$ which depends on error $\epsilon$ such that
    \begin{equation}
        \sup_p \sup_s |h(p, s) -\sum_{i=1}^C pol_{1,i}(p) pol_{2,i}(s)| \leq \frac{\epsilon}{2}. 
    \end{equation}
    Furthermore, as polynomial functions are continuous function, therefore $f_i$ can be used to approximate the polynomial function $pol_{1, i}$ and $g$ can be used to approximate the polynomial function $pol_{2, i}$.
    \begin{align}
        \sup_p |pol_{1,i}(p) - f_i(p; \theta_1)| & \leq \frac{\epsilon}{6B} \\ 
        \sup_s |pol_{2,i}(s) - g_i(s; \theta_2)| & \leq \frac{\epsilon}{6B}. 
    \end{align}
    Here $B := \max(1, \sup_p \max_{i} |pol_{1, i}(p)|, \sup_s \max_{i} |pol_{2, i}(s)|).$ 
    We show that the cluster sampler is capable to approximate any desirable continuous cluster sampler. 
    \begin{equation}
        \sup_p \sup_s |h(p, s) -\sum_{i=1}^C f_i(p; \theta_1) g_i(s; \theta_2)| \leq \frac{\epsilon}{2} + \frac{\epsilon}{6B} * B + \frac{\epsilon}{6B} (B + \frac{\epsilon}{6B}) = \frac{5}{6} \epsilon + \frac{\epsilon^2}{36B^2} < \epsilon. 
    \end{equation}
    The last inequality comes from $\epsilon < 6$. 
    The universal approximation capacity of the cluster sampler is proved. 
\end{proof}

\begin{remark}
    Since we are working with a cluster sampler with specific manually designed structure, it mainly comes from the fact the student's t-kernel introduce a suitable implicit bias to more efficiently learn the cluster sample probability $(n_2 \times 1)$. 
\end{remark}

\section{Experimental Settings \label{app:implementation}}

We follow the model architecture design and hyperparameter settings of \citet{luo2024classic}. In order to accommodate the computational requirements for our extensive experiments, we harness a variety of high-capacity GPU resources. This includes: Tesla V100 32Gb, NVIDIA RTX A6000
48Gb, NVIDIA RTX A5000 24Gb, and Quadro RTX 8000 48Gb.

\section{Datasets}

In this paper, we adopt $12$ datasets from different domains.

\noindent{\small$\bullet$} \textbf{[Citation network]}. \emph{Cora}, \emph{Citeseer}, and \emph{Pubmed}~\citep{Yang2016RevisitingSL} are citation graphs where each node corresponds to a scientific paper. In these graphs, nodes are characterized by bag-of-words feature vectors, and each one is assigned a label that indicates its research field. It is important to note that all three datasets are examples of homophilous graphs.

\noindent{\small$\bullet$} \textbf{[Amazon network]}. \citet{Shchur2018PitfallsOG} In this network, products are nodes, and an edge signifies that two products are often bought together.  Each product has associated reviews, which are treated as a bag of words.  The task is to determine the product category for each item in the network.

\noindent{\small$\bullet$} \textbf{[Coauthor network]}. \citet{Shchur2018PitfallsOG} The network represents authors connected by co-authorship.  Using the keywords from their published papers, we aim to classify each author according to their research field.

\noindent{\small$\bullet$} \textbf{[Wikics network]}. WikiCS~\citep{mernyei2020wiki} is a hyperlink network in the field of computer science on Wikipedia. The categories correspond to different research directions in computer science, such as artificial intelligence, computer vision, network security, etc.


\noindent{\small$\bullet$} \textbf{[Wikipedia network]}.  Squirrel and Chameleon~\citep{platonov2023a} represent two distinct portions of the Wikipedia web.  The objective is to categorize each individual webpage (node) within these portions into one of five traffic-based classifications, determined by their respective average monthly page views.

\noindent{\small$\bullet$} \textbf{[Heterophilous network]}. These networks are from \citet{platonov2023a} Amazon Ratings, A co-purchasing network of products with reviews used to predict product category. Minesweeper, a synthetic grid-based graph where nodes represent cells, and the task is to identify mines using information about neighboring mines. Roman Empire, a word dependency graph from a Wikipedia article, where nodes are words, and edges represent sequential or syntactic relationships, with the task of classifying words by their syntactic roles.



\end{document}
\endinput
%%
%% End of file `sample-sigconf-authordraft.tex'.
