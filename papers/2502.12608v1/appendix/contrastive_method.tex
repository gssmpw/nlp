\section{Datasets}

In this paper, we adopt $12$ datasets from different domains.

\noindent{\small$\bullet$} \textbf{[Citation network]}. \emph{Cora}, \emph{Citeseer}, and \emph{Pubmed}~\citep{Yang2016RevisitingSL} are citation graphs where each node corresponds to a scientific paper. In these graphs, nodes are characterized by bag-of-words feature vectors, and each one is assigned a label that indicates its research field. It is important to note that all three datasets are examples of homophilous graphs.

\noindent{\small$\bullet$} \textbf{[Amazon network]}. \citet{Shchur2018PitfallsOG} In this network, products are nodes, and an edge signifies that two products are often bought together.  Each product has associated reviews, which are treated as a bag of words.  The task is to determine the product category for each item in the network.

\noindent{\small$\bullet$} \textbf{[Coauthor network]}. \citet{Shchur2018PitfallsOG} The network represents authors connected by co-authorship.  Using the keywords from their published papers, we aim to classify each author according to their research field.

\noindent{\small$\bullet$} \textbf{[Wikics network]}. WikiCS~\citep{mernyei2020wiki} is a hyperlink network in the field of computer science on Wikipedia. The categories correspond to different research directions in computer science, such as artificial intelligence, computer vision, network security, etc.


\noindent{\small$\bullet$} \textbf{[Wikipedia network]}.  Squirrel and Chameleon~\citep{platonov2023a} represent two distinct portions of the Wikipedia web.  The objective is to categorize each individual webpage (node) within these portions into one of five traffic-based classifications, determined by their respective average monthly page views.

\noindent{\small$\bullet$} \textbf{[Heterophilous network]}. These networks are from \citet{platonov2023a} Amazon Ratings, A co-purchasing network of products with reviews used to predict product category. Minesweeper, a synthetic grid-based graph where nodes represent cells, and the task is to identify mines using information about neighboring mines. Roman Empire, a word dependency graph from a Wikipedia article, where nodes are words, and edges represent sequential or syntactic relationships, with the task of classifying words by their syntactic roles.

