\section{Introduction}

Research has shown that, perhaps paradoxically, text plays an important role in how readers interpret the data depicted by a visualization. These descriptions can explain how a chart is constructed, summarize statistical features (e.g., the minimum and maximum values), describe cognitive and perceptual phenomena (e.g., complex trends and patterns), or offer broader contextual and domain-specific explanations~\cite{lundgard2021accessible}.
When well-authored, textual descriptions can helpfully reinforce visual features of the chart (e.g., high prominence characteristics) to reduce a reader's cognitive load while interpreting the chart~\cite{kim2021towards}\,---\, indeed, in such cases, readers favor heavily annotated charts over simply charts or textual descriptions alone~\cite{stokes2022striking}. However, poorly written text can also slant the takeaway message~\cite{kong2018frames} in ways that impact readers' trust and recall~\cite{kong2019trust}.

Existing tools, however, provide almost no support for authoring text alongside the visualization. At best, a nascent body of work has begun to explore automated methods for generating titles~\cite{liu2023autotitle} and descriptions~\cite{choi2022intentable,obeid2020chart,hsu2024scicapenter} or, for pre-authored visualizations and textual descriptions, aligning emphasis~\cite{kim2023emphasischecker}, or constructing links and cross-references~\cite{latif2021kori}. While valuable, these approaches leave unexplored a rich design space for \emph{concurrently} authoring the two modalities together\,---\, that is, how might interactive mechanisms help users write a textual description while designing a chart; conversely, how might users update a chart's design to better reflect a textual description; and, how might a user iterate between the two modalities when authoring artifacts for data-driven communication?

To address this gap, we introduce \pluto\footnote{Inspired by the celestial narrative where Pluto represents an entity once marginalized in astronomical classification, this work seeks to underscore the often-overlooked importance of text in data visualization. The name also aligns with the tradition of naming advances in visualization after celestial bodies, such as Polaris~\cite{polaris} and Vega~\cite{vega2014visualization}.}, a novel mixed-initiative tool designed to enable authors to craft semantically rich textual content for a variety of popular charts, including bar charts, histograms, line charts, and scatterplots.
\pluto~facilitates an interactive composition process, allowing authors to leverage GPT for automatically generating chart descriptions (Figure~\ref{fig:teaser}A), use direct manipulation-based interactions with the chart to scope and guide text generation (Figure~\ref{fig:teaser}B), and even update the chart's design based on the composed text (Figure~\ref{fig:teaser}C). To operationalize this interactive authoring experience, we model a conceptual schema that captures key components of the chart (i.e., data, specification, selections, visual embellishments) and text (i.e., title, description, textual annotations along with the semantic information they convey~\cite{lundgard2021accessible}).
We leverage this schema as part of a pipeline that blends heuristics and large language models (LLMs) to generate authoring recommendations in \pluto. 


Through a preliminary user evaluation, we explore the utility of \pluto{} as a tool for helping authors create semantically relevant content when integrating text with the corresponding chart that it describes. We observe that the recommendations collectively afford a breadth of authoring workflows along a spectrum of using automatically generated text to leveraging recommendations only to augment manually authored content. Participant feedback also suggests that certain forms of assistance---such as generating text based on chart selections or suggesting chart design changes based on entered text---are deemed more useful than automated text generation and text editing suggestions. Our findings contribute to a growing body of literature on the integration of textual and visual elements for data-driven communication, offering insights into how future visualization authoring tools can better leverage the semantics and functional role of text with visualizations.