\section{Discussion}

\begin{figure*}[t!]
    \centering    \includegraphics[width=.6\linewidth]{figures/pdf/extensibility-examples.pdf}
    \caption{Examples illustrating the extensibility afforded by the proposed conceptual schema. Here, \pluto's text generation and recommendation modules are used \emph{as-is} to author text for a geographic map and an adjacency matrix. In (A), the system generates a description based on the map and subsequently highlights six states in the chart that are emphasized in the text. In (B), \pluto~suggests a sentence completion based on multimodal input of the previously drafted description and a user selection on the chart.}
    \Description[Two examples showing how the underlying schema used to develop Pluto can be extended to other charts.]{The first example illustrates the system is able to generate a textual description based on a choropleth map. The example also shows that some states are highlighted in the map because they are referenced in the generated text. The second example shows a sentence completion recommendation generated in response to clicking on a cell in a heatmap, further illustrating the system's ability to support charts that were beyond the initial set of visualizations implemented in the tool.}
    \label{fig:extensibility-examples}
\end{figure*}

\subsection{Grammar-based Approaches for Chart \& Text Interfaces}

While we currently implement only a set of charts and recommendations in \pluto, the underlying schema (Figure~\ref{fig:schema}) is not limited to this set.
Since the framework is based on the underpinning data and conceptual elements of the chart and text, the presented schema can be generalized to other cases.
Figure~\ref{fig:extensibility-examples} highlights examples of this generalization by illustrating \pluto's~text and embellishment suggestions for a geographic map and a graph dataset represented as a network matrix.

Furthermore, the idea of breaking individual \schemaPrimary{Statements} into \schemaPrimary{Text}, \schemaPrimary{StatementType}, and \schemaPrimary{DataItems} enables interactive text-chart linking (Figure~\ref{fig:scenario-1}A) as well as targeted recommendations for description statements (Figure~\ref{fig:statement-recommendations}).
Beyond these recommendations, however, this breakdown can also be leveraged more generally to design better linting systems for writing text for charts.
For example, by inspecting the \schemaPrimary{Text} and \schemaPrimary{StatementType}s in a given description and applying findings from studies on accessible visualization descriptions~\cite{yaneva2015accessible,lundgard2021accessible,jung2021communicating,kim2023explain}, visualization systems can flag inaccessible descriptions and suggest improvements to make the descriptions more accessible.

While these are just examples, leveraging logical concepts such as those in the presented schema (Figure~\ref{fig:schema}) affords a compelling opportunity to create a unifying grammar to capture the multimodal nature of chart + text authoring interfaces.
Besides streamlining the interface and interaction design of future tools, such a grammar that is centered around abstract concepts underpinning text and charts can also facilitate effective communication with LLMs by allowing systems to only share required task-specific context with the models in a structured representation.

\subsection{Design Considerations}

Based on the study observations and feedback, we derive four design considerations for future systems exploring AI-based suggestions to assist unified text and chart authoring.

\vspace{.5em}
\noindent\textbf{Format and phrasing of text are as important as its content.}
Participants recognized the value in text generation and noted that the generated text often picked up on the salient data points and trends in the data. However, the two participants who frequently used text and charts for communication, in particular, critiqued the verbosity and tone of the generated text, indicating that the text was ``\textit{too formal or complex}'' for their consumers. Future systems should explore providing users with control over the configuration of the properties of the generated text (e.g., choosing between paragraphs and bullets, adjusting the level of verbosity, and setting the tone or writing style)~\cite{louis2014}.


\vspace{.5em}
\noindent\textbf{Leverage multimodal context during text generation.}
Participants particularly appreciated text suggestions that were based on chart selections or built upon previously authored text.
$P5$, for instance, referred to the interaction experience of having some text and then using the chart in tandem to guide the system (Figure~\ref{fig:teaser}B) as being a ``\textit{smooth and controlled authoring flow.}'' To support similar fluid and coherent authoring experiences, future systems should consider multimodal context from both the chart and previously written text when suggesting new text.

\vspace{.5em}
\noindent\textbf{Include techniques to help verify the text with the chart.}
Participants appreciated the interactive highlighting and statement verification features in \pluto~(Figure~\ref{fig:scenario-1}A), noting that the features encouraged them be more critical of the text (regardless of whether it was written by them or was system-generated).
With the growing prevalence of generated text, future systems should continue incorporating such interactive verification features to mitigate false statements and enhance synergy between the text and the chart.

\vspace{.5em}
\noindent\textbf{Adjust chart design to align with the text description.}
Unlike prior systems that focus on the unidirectional task of generating text from charts, \pluto~introduces a bidirectional flow by also recommending changes to the chart design based on the text (Figure~\ref{fig:chart-recommendations}).
We observed that participants extensively used and appreciated these suggestions, commenting that the chart design recommendations helped them gain a better reader's perspective.
To this end, future systems should continue to explore techniques to leverage the bidirectional flow of information between text and charts
and generate chart design suggestions for an integrated reading experience.

\section{Limitations and Future Work}
\noindent\textbf{Incorporate data context into generated text.}
\pluto~currently only considers the active chart and data fields while suggesting text.
However, since the recommendations are provided within a general-purpose visualization specification tool, the system can access other data fields not displayed in the active chart (Figure~\ref{fig:interface}A).
For instance, during the study, viewing the generated text for the bird strikes chart (Figure~\ref{fig:scenario-2}A), $P8$ said, ``\textit{I wish it could generate some text explaining why the costs were high based on the number of incidents.}''
Although $P8$ was able to work around this limitation by creating the second chart, inspecting the new visualization, and returning to the original chart, automatically considering other fields as part of the generated text is an open area for future work.
Such dataset-level text (as opposed to chart-level text) can also help make the authored descriptions semantically rich by including more \schemaSecondary{domain-specific} statements.

\vspace{.5em}
\noindent\textbf{Text formatting recommendations.}
The current implementation leans heavily on content generation; however, future iterations should also focus on providing sophisticated formatting suggestions that can add expressivity to the text based on the semantic levels.
For example, titles can be rendered more prominently with annotations and footnotes shown to support the chart. Complementing chart design suggestions and providing structural suggestions for the description, such as including line breaks or using bullet points instead of paragraphs, can also help improve readability.

\vspace{.5em}
\noindent\textbf{Supporting multiple views and articles.} \pluto~currently supports authoring a single chart and assumes the resulting chart and text are static.
While such an approach covers a popular scenario for data-driven communication as evidenced by prior research focusing on this setup (e.g.,~\cite{choi2022intentable,obeid2020chart,liu2020autocaption,tang2023vistext,hsu2021scicap,stokes2022striking}), people also use dashboards and interactive storytelling articles when communicating with charts~\cite{sarikaya2018we,segel2010narrative}. Investigating support for multiple views and operationalizing \pluto's recommendations in more expressive interactive data-driven article authoring tools such as Idyll Studio~\cite{conlen2021idyll} is an open topic for future work.


\vspace{.5em}
\noindent\textbf{Manage data sharing and combine heuristics with LLMs.}
As discussed in \S\ref{sec:reco-generation}, we currently pprovide the context of the chart's specification and the data to the LLM.
However, depending on the size of the data and or the usage context,  sharing data directly with the LLM may not always be feasible.
Exploring alternatives to generate text without sharing the underlying data is an important direction for future work.
For instance, one possible approach could be to use heuristics-based approaches and prior knowledge of analytic tasks (e.g.,~\cite{manelski-krulee-1965-heuristic,snowy2021,kim2023emphasischecker}) to identify key trends from a chart and then leverage the LLM for appropriately consolidating the extracted information into a coherent text narrative.


\vspace{.5em}
\noindent\textbf{Conduct longitudinal evaluation across data domains.}
The combination of two expert interviews, three pilots, and an evaluation study with ten participants helped us validate \pluto~as a proof-of-concept for the general idea of a mixed-initiative interface for authoring semantically aligned charts and text.
\new{However, this evaluation is only preliminary, and assessing the practical value of a tool like \pluto~deems more longitudinal studies spanning different data domains and individuals with varying levels of expertise~\cite{lam2012,shneiderman2006strategies}.}

\vspace{.5em}
\noindent\new{\textbf{Incorporating safeguards for machine failures.}
With the current implementation of \pluto, we utilize a pretrained LLM to generate text and a heuristic parser to analyze the generated text for additional suggestions.
To make users aware of potential errors in either of these steps, we provide interactive visual previews that can be viewed before accepting system suggestions (e.g., Figure~\ref{fig:scenario-1}A and Figure~\ref{fig:chart-recommendations}).
While \pluto~serves as a proof-of-concept, developing a more generalized system requires a deeper investigation of the types of errors that might be generated both at the LLM and the parser level.
Categorizing and understanding the distribution of such errors can ultimately help develop better recommendation modules, but also design interface and interaction strategies to mitigate system errors.
}