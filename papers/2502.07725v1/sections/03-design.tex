\begin{figure*}[t!]
    \centering
    \includegraphics[width=\linewidth]{figures/interface.png}
    \caption{\pluto's user interface. The key components include a data panel (A), chart editor (B), chart title (C), main chart canvas (D), and a chart description (E).
    Here, the user has manually entered a description and clicked the {\small{\faIcon[regular]{lightbulb}}} \textbf{Suggest} button to get ideas on improving the chart and text for communication purposes.
    This results in the system suggesting a title and adding a highlight annotation for \annotation{\textit{Single Family}} homes, while also generating a chart design recommendation (F) and a set of description editing recommendations (G).}
    \Description[Pluto's user interface.]{From left to right, the system contains of: 1) a data pane showing the available attributes, 2) a pane to specify encodings, annotations, and filters to create a chart, 3) the chart along with text boxes for the title and description above and below it, respectively, and 4) a right panel where the system presents recommendations to edit the chart and description.}
    \label{fig:interface}
\end{figure*}

\section{Design}

The central idea of our work is exploring a unified visualization system for authoring well-integrated charts and text.
Designing such a system, however, requires considering several open questions about the type of assistance the system should provide, and when and how system suggestions should be surfaced.

In exploring the chart-and-text authoring experience, several questions arise that warrant exploration. We must first discern which chart elements, ranging from axis labels and ticks to titles, descriptions, and annotations, necessitate the most authoring support. Additionally, we would need to determine the appropriate level of system assistance—whether to generate entire descriptions, fill in partially written text, or refine user-authored drafts—and how this assistance might vary with different types of text. Given the non-mutual exclusivity of text types, such as descriptions influencing titles, we must also consider if and how the system should sequence its suggestions. The timing of these suggestions is another critical factor: should they be offered immediately following the creation of a chart or once the user has initiated the authoring process? Deciding whether these suggestions should be proactive or solicited on-demand, along with the specific user actions that should trigger them, is also a direction worth considering. Lastly, we must explore the potential for a synergistic relationship between the chart and text, i.e., how interactions with each can be leveraged to enhance the other and what mechanisms would facilitate this interplay.

\subsection{Design Goals}
\label{sec:design-goals}

With the aforementioned considerations in mind, we iteratively compiled a list of design goals to guide our system's development.
These goals were informed by prior research and systems focusing on authoring text for visualizations (e.g.,~\cite{kim2023emphasischecker,latif2021kori,liu2023autotitle,he2024leveraging,tang2023vistext,singh2024figura11y}), general principles of mixed-initiative user interfaces~\cite{horvitz1999}, as well as formative interviews with two experts on authoring text and charts for data-driven communication.

The two experts were a practitioner and a researcher who both author and critique text for data visualizations. 
Furthermore, they also regularly interact with end-users in the creation of text and charts.
\new{Both experts voluntarily participated in the interviews and were not financially compensated.
We interviewed each expert twice over a span of three weeks.
Each session lasted for 30-60 minutes.
}
\new{During the first set of interviews, we asked the experts about the key challenges users generally encounter during the authoring of text with charts. Additionally, to guide our design and identify critical features, we also presented an early version of our prototype with a basic set of functionality including generating titles and descriptions for a chart and supporting interactive highlighting of chart elements based on the text.
Based on the initial feedback, we incorporated additional types of recommendations and refined the system design before the second meeting where the experts provided feedback on the overall utility and perceived usability of the different features.
We subsequently developed the final version of \pluto~by iterating on this feedback and leveraging findings from related work on text+chart authoring systems~(e.g.,~\cite{latif2021kori,kim2023emphasischecker,sultanum2023datatales,lin2023inksight,choi2022intentable}).
}

\vspace{.5em}
\noindent\textbf{DG1. Leverage the textual narrative to guide chart design.}
In line with prior work~\cite{stokes2022striking,ottley2019curious,kim2021towards}, the experts also stressed that the text should not only convey the right levels of information but also be well-aligned with the chart for a smooth reading experience.
As text has an inherent narrative flow, we noted that the system should \new{incorporate techniques from prior work on updating chart specification based on narrative text~\cite{wang2022towards,chen2022crossdata,shen2024data} to} inspect the flow of information in the text and leverage it to augment the chart.
This augmentation could involve making data transformation changes (e.g., sorting) or adding annotations to highlight portions of the chart that are emphasized in the text.

\vspace{.5em}
\noindent\textbf{DG2. Support direct manipulation interactions with the chart for text generation.}
Visualizations make it easy to perceive trends in the data and identify points of interest.
Phrasing something visually interesting as text can be challenging, however. For instance, one of the experts noted, ``\textit{sometimes I notice something potentially interesting on the chart and want some quick text to verify what I'm seeing and get ideas for how to talk about it.}'' Given this multimodal nature of charts and text, \new{in line with prior chart-and-text authoring systems (e.g.,~\cite{chen2022crossdata,lin2023inksight})}, we noted that the system should allow leveraging direct interactions with the chart (e.g., brushing a region or mark selection) to generate corresponding text.

\vspace{.5em}
\noindent\textbf{DG3. Provide varying levels of assistance for text authoring.}
Both experts noted that users need different levels of assistance when writing text depending on their goals and experience level.
For instance, novice and intermediate users may need auto-generated text to jump-start their authoring process, whereas domain experts may benefit from fine-tuning suggestions to improve manually written text.
Combining this comment with prior work on text-chart authoring~\cite{stokes2022striking,kim2023emphasischecker}, we noted that the system should not only recommend text for chart authors to add but also recommend editing actions (e.g., reordering sentence) or flag potential factual errors in the text (e.g., incorrectly stated trends).

\vspace{.5em}
\noindent\textbf{DG4. Incorporate context-sensitive recommendations near their relevant targets to facilitate easier interpretation.}
System recommendations in the context of a unified text and chart authoring process could apply to different targets (e.g., chart, title, or description) and focus on either adding new content or editing existing content.
Interpreting this broad set of recommendations can be challenging, however.
For instance, in our early prototypes, we explored listing all recommendations in a side panel, but both experts noted that this was overwhelming and distracted them from the main content.
Iterating on the designs, we noted that for improved usability, the system recommendations should be placed close to the targets they apply to and should also be presented differently (e.g., in-place overlays vs. suggested actions) based on the type of recommendation.

\vspace{.5em}
\noindent\textbf{DG5. Recommendations should be unobtrusive during targeted authoring.}
While the recommendations are designed to help craft cohesive text and charts, there may be instances where the chart authors have clear authoring goals in mind.
In such targeted authoring scenarios, the recommendations should not interfere with the users' flow but still be available on demand if users want ideas for text content or chart design.
Authors should have full control over the final content, however, and should be able to edit/update any suggestions made by the system.
\newline

\noindent{}Note that these goals are not exhaustive or mutually exclusive, nor are they meant to be prescriptive.
For instance, we primarily focus on content suggestions and do not deeply consider operations like formatting as part of the recommendation space. Rather, \textbf{DG1}-\textbf{DG5} are only meant to be an initial set of goals to help ground our design and enable us to develop and test a viable prototype.