%%
%% This is file `sample-manuscript.tex',
%% generated with the docstrip utility.
%%
%% The original source files were:
%%
%% samples.dtx  (with options: `manuscript')
%% 
%% IMPORTANT NOTICE:
%% 
%% For the copyright see the source file.
%% 
%% Any modified versions of this file must be renamed
%% with new filenames distinct from sample-manuscript.tex.
%% 
%% For distribution of the original source see the terms
%% for copying and modification in the file samples.dtx.
%% 
%% This generated file may be distributed as long as the
%% original source files, as listed above, are part of the
%% same distribution. (The sources need not necessarily be
%% in the same archive or directory.)
%%
%% Commands for TeXCount
%TC:macro \cite [option:text,text]
%TC:macro \citep [option:text,text]
%TC:macro \citet [option:text,text]
%TC:envir table 0 1
%TC:envir table* 0 1
%TC:envir tabular [ignore] word
%TC:envir displaymath 0 word
%TC:envir math 0 word
%TC:envir comment 0 0
%%
%%
%% The first command in your LaTeX source must be the \documentclass command.
%%%% Small single column format, used for CIE, CSUR, DTRAP, JACM, JDIQ, JEA, JERIC, JETC, PACMCGIT, TAAS, TACCESS, TACO, TALG, TALLIP (formerly TALIP), TCPS, TDSCI, TEAC, TECS, TELO, THRI, TIIS, TIOT, TISSEC, TIST, TKDD, TMIS, TOCE, TOCHI, TOCL, TOCS, TOCT, TODAES, TODS, TOIS, TOIT, TOMACS, TOMM (formerly TOMCCAP), TOMPECS, TOMS, TOPC, TOPLAS, TOPS, TOS, TOSEM, TOSN, TQC, TRETS, TSAS, TSC, TSLP, TWEB.
% \documentclass[acmsmall]{acmart}

%%%% Large single column format, used for IMWUT, JOCCH, PACMPL, POMACS, TAP, PACMHCI
% \documentclass[acmlarge,screen]{acmart}

%%%% Large double column format, used for TOG
% \documentclass[acmtog, authorversion]{acmart}

%%%% Generic manuscript mode, required for submission
%%%% and peer review
\documentclass[sigconf]{acmart}
% \documentclass[sigconf,review,anonymous]{acmart}
% \documentclass[manuscript,review,anonymous]{acmart}
% \documentclass[manuscript,screen,review]{acmart}
%% Fonts used in the template cannot be substituted; margin 
%% adjustments are not allowed.
%%

\usepackage{fontawesome5}
\usepackage{xcolor}
\usepackage{soul}
% \usepackage[most]{tcolorbox}
\usepackage{wrapfig}
\usepackage{booktabs}
\usepackage{tabularx}
% \usepackage[normalem]{ulem}
\usepackage{enumitem}
\usepackage{verbatim}
\usepackage{syntax}
\usepackage{parcolumns}
\usepackage{listings}
\usepackage[frozencache,cachedir=.]{minted}
\usepackage[linesnumbered,ruled,vlined]{algorithm2e}
\usepackage{algorithmicx}
\usepackage{algpseudocode}
\usepackage{amsmath}
\usepackage{caption}

\newcommand{\pluto}{\textsc{Pluto}}

\newcommand{\arjun}[1]{\textcolor{orange}{AJ: #1}}
\newcommand{\vidya}[1]{\textcolor{blue}{(VS: #1)}}
\newcommand{\arvind}[1]{\textcolor{red}{(AS: #1)}}

\newcommand{\new}[1]{\textcolor{black}{#1}}

\DeclareCaptionType{InfoBox}

\definecolor{explicitBlue}{HTML}{095A94}
\definecolor{customCodeOrange}{HTML}{AB4C0C}
\definecolor{customGreen}{HTML}{54AA54}
\definecolor{customGold}{HTML}{FFD601}


\newcommand{\schemaPrimary}[1]{\texttt{\color{explicitBlue}{#1}}}
\newcommand{\schemaSecondary}[1]{\texttt{\color{customCodeOrange}{#1}}}

\newenvironment{tightItemize}{\begin{itemize}[leftmargin=.15in]\itemsep
-2.1pt}{\end{itemize}}

% Define a new tcbox with a red dashed border
% \newtcbox{\mybox}[1][]{%
%   nobeforeafter,
%   colframe=red, % Color of the frame
%   colback=white, % Background color
%   boxrule=0.5pt, % Frame thickness
%   boxsep=0pt, % Separation between frame and content
%   left=1mm, % Left padding
%   right=1mm, % Right padding
%   top=1mm, % Top padding
%   bottom=1mm, % Bottom padding
%   arc=2mm, % Corner rounding
%   borderline={2pt}{2pt}{red, dashed}, % Border line (width, separation, style)
%   % #1 % For specifying additional options
% }

% \newtcbox{\recobox}{nobeforeafter, colback=white, colframe=white, boxrule=0.5pt, arc=1pt, boxsep=0pt,left=2pt,right=2pt,top=1.75pt,bottom=1.5pt,tcbox raise base,enhanced, borderline={0.5pt}{.1pt}{red, dashed}}

% \newcommand{\reco}[1]{\recobox{\small{#1}}}
\newcommand{\reco}[1]{\fcolorbox{red}{white}{{\small{#1}}}}

% \newtcbox{\keybox}{nobeforeafter, colback=gray!10, colframe=gray!50, boxrule=0.5pt, arc=1pt, boxsep=0pt,left=2pt,right=2pt,top=1.75pt,bottom=1.5pt,tcbox raise base}

% \newcommand{\key}[1]{\keybox{\small{\texttt{#1}}}}
\newcommand{\key}[1]{\fcolorbox{gray!50}{gray!10}{{\small{#1}}}}

% \newtcbox{\annotationbox}{nobeforeafter, colback=gray!05, colframe=customGold, boxrule=1pt, arc=1pt, boxsep=0pt,left=2pt,right=2pt,top=1.75pt,bottom=1.5pt,tcbox raise base}

% \newcommand{\annotation}[1]{\annotationbox{\small{#1}}}
\newcommand{\annotation}[1]{\fcolorbox{customGold}{gray!05}{{\small{#1}}}}

%% \BibTeX command to typeset BibTeX logo in the docs
\AtBeginDocument{%
  \providecommand\BibTeX{{%
    \normalfont B\kern-0.5em{\scshape i\kern-0.25em b}\kern-0.8em\TeX}}}

%% Rights management information.  This information is sent to you
%% when you complete the rights form.  These commands have SAMPLE
%% values in them; it is your responsibility as an author to replace
%% the commands and values with those provided to you when you
%% complete the rights form.
\setcopyright{acmlicensed}
\copyrightyear{2025}
\acmYear{2025}
\setcopyright{cc}
\setcctype{by}
\acmConference[IUI '25]{30th International Conference on Intelligent User Interfaces}{March 24--27, 2025}{Cagliari, Italy}
\acmBooktitle{30th International Conference on Intelligent User Interfaces (IUI '25), March 24--27, 2025, Cagliari, Italy}\acmDOI{10.1145/3708359.3712122}
\acmISBN{979-8-4007-1306-4/25/03}



%% These commands are for a PROCEEDINGS abstract or paper.
% \acmConference[IUI '25]{The ACM Conference on Intelligent User Interfaces}{March 24--27, 2025}{Cagliari, Italy}
% %
% %  Uncomment \acmBooktitle if th title of the proceedings is different
% %  from ``Proceedings of ...''!
% %
% \acmBooktitle{The ACM Conference on Intelligent User Interfaces (IUI '25), March 24--27, 2026, Cagliari, Italy}
% \acmISBN{978-1-4503-XXXX-X/18/06}

% \settopmatter{printacmref=false}
% \setcopyright{none}
% \renewcommand\footnotetextcopyrightpermission[1]{}
\pagestyle{plain}

%%
%% Submission ID.
%% Use this when submitting an article to a sponsored event. You'll
%% receive a unique submission ID from the organizers
%% of the event, and this ID should be used as the parameter to this command.
%%\acmSubmissionID{123-A56-BU3}

%%
%% For managing citations, it is recommended to use bibliography
%% files in BibTeX format.
%%
%% You can then either use BibTeX with the ACM-Reference-Format style,
%% or BibLaTeX with the acmnumeric or acmauthoryear sytles, that include
%% support for advanced citation of software artefact from the
%% biblatex-software package, also separately available on CTAN.
%%
%% Look at the sample-*-biblatex.tex files for templates showcasing
%% the biblatex styles.
%%

%%
%% The majority of ACM publications use numbered citations and
%% references.  The command \citestyle{authoryear} switches to the
%% "author year" style.
%%
%% If you are preparing content for an event
%% sponsored by ACM SIGGRAPH, you must use the "author year" style of
%% citations and references.
%% Uncommenting
%% the next command will enable that style.
%%\citestyle{acmauthoryear}

%%
%% end of the preamble, start of the body of the document source.
\begin{document}

%% The "title" command has an optional parameter,
%% allowing the author to define a "short title" to be used in page headers.

\title{\pluto: Authoring Semantically Aligned Text and Charts\\ for Data-Driven Communication}

%%
%% The "author" command and its associated commands are used to define
%% the authors and their affiliations.
%% Of note is the shared affiliation of the first two authors, and the
%% "authornote" and "authornotemark" commands
%% used to denote shared contribution to the research.
\author{Arjun Srinivasan}
\affiliation{%
  \institution{Tableau Research}
  \city{Seattle}
  \state{Washington}
  \country{USA}
}
\email{arjunsrinivasan@tableau.com}

\author{Vidya Setlur}
\affiliation{%
  \institution{Tableau Research}
  \city{Palo Alto}
  \state{California}
  \country{USA}
}
\email{vsetlur@tableau.com}

\author{Arvind Satyanarayan}
\affiliation{%
  \institution{Massachusetts Institute of Technology}
  \city{Cambridge}
  \state{Massachusetts}
  \country{USA}
}
\email{arvindsatya@mit.edu}

%%
%% By default, the full list of authors will be used in the page
%% headers. Often, this list is too long, and will overlap
%% other information printed in the page headers. This command allows
%% the author to define a more concise list
%% of authors' names for this purpose.
\renewcommand{\shortauthors}{Srinivasan, et al.}

%%
%% The abstract is a short summary of the work to be presented in the
%% article.


During the early stages of interface design, designers need to produce multiple sketches to explore a design space.  Design tools often fail to support this critical stage, because they insist on specifying more details than necessary. Although recent advances in generative AI have raised hopes of solving this issue, in practice they fail because expressing loose ideas in a prompt is impractical. In this paper, we propose a diffusion-based approach to the low-effort generation of interface sketches. It breaks new ground by allowing flexible control of the generation process via three types of inputs: A) prompts, B) wireframes, and C) visual flows. The designer can provide any combination of these as input at any level of detail, and will get a diverse gallery of low-fidelity solutions in response. The unique benefit is that large design spaces can be explored rapidly with very little effort in input-specification. We present qualitative results for various combinations of input specifications. Additionally, we demonstrate that our model aligns more accurately with these specifications than other models. 

% OLD ABSTRACT
%When sketching Graphical User Interfaces (GUIs), designers need to explore several aspects of visual design simultaneously, such as how to guide the user’s attention to the right aspects of the design while making the intended functionality visible. Although current Large Language Models (LLMs) can generate GUIs, they do not offer the finer level of control necessary for this kind of exploration. To address this, we propose a diffusion-based model with multi-modal conditional generation. In practice, our model optionally takes semantic segmentation, prompt guidance, and flow direction to generate multiple GUIs that are aligned with the input design specifications. It produces multiple examples. We demonstrate that our approach outperforms baseline methods in producing desirable GUIs and meets the desired visual flow.

% Designing visually engaging Graphical User Interfaces (GUIs) is a challenge in HCI research. Effective GUI design must balance visual properties, like color and positioning, with user behaviors to ensure GUIs easy to comprehend and guide attention to critical elements. Modern GUIs, with their complex combinations of text, images, and interactive components, make it difficult to maintain a coherent visual flow during design.
% Although current Large Language Models (LLMs) can generate GUIs, they often lack the fine control necessary for ensuring a coherent visual flow. To address this, we propose a diffusion-based model that effectively handles multi-modal conditional generation. Our model takes semantic segmentation, optional prompt guidance, and ordered viewing elements to generate high-fidelity GUIs that are aligned with the input design specifications.
% We demonstrate that our approach outperforms baseline methods in producing desirable GUIs and meets the desired visual flow. Moreover, a user study involving XX designers indicates that our model enhances the efficiency of the GUI design ideation process and provides designers with greater control compared to existing methods.    



% %%%%%%%%%%%%%%%%%%%%%%%%%%%%%%%%%%%%%%%%%%%%%%%%%%%%%%
% % Writing Clinic Comments:
% %%%%%%%%%%%%%%%%%%%%%%%%%%%%%%%%%%%%%%%%%%%%%%%%%%%%%%
% % Define: Effective UI design
% % Motivate GANs and write in full form.
% % LLMs vs ControlNet vs GANs
% % Say something about the Figma plugin?
% % Write the work is novel or what has been done before
% % What is desirable UI and how to evalutate that?
% % Visual Flow - main theme (center around it)
% % Re-Title: use word Flow!
% % Use ControlNet++ & SPADE for abstract.
% % Write about input/output. 
% % Why better than previous work?
% %%%%%%%%%%%%%%%%%%%%%%%%%%%%%%%%%%%%%%%%%%%%%%%%%%%%%

% % v2:
% % \noindent \textcolor{red}{\textbf{NEW Abstract!} (Post Writing Clinic 1 - 25-Jun)}

% % \noindent \textcolor{red}{----------------------------------------------------------------------}

% % \noindent Designing user interfaces (UIs) is a time-consuming process, particularly for novice designers. 
% % Creating UI designs that are effective in market funneling or any other designer defined goal requires a good understanding of the visual flow to guide users' attention to UI elements in the desired order. 
% % While current Large Language Models (LLMs) can generate UIs from just prompts, they often lack finer pixel-precise control and fail to consider visual flow. 
% % In this work, we present a UI synthesis method that incorporates visual flow alongside prompts and semantic layouts. 
% % Our efficient approach uses a carefully designed Generative Adversarial Network (GAN) optimized for scenarios with limited data, making it more suitable than diffusion-based and large vision-language models.
% % We demonstrate that our method produces more "desirable" UIs according to the well-known contrast, repetition, alignment, and proximity principles of design. 
% % We further validate our method through comprehensive automatic non-reference, human-preference aligned network scoring and subjective human evaluations.
% % Finally, an evaluation with xx non-expert designers using our contributed Figma plugin shows that <method-name> improves the time-efficiency as well as the overall quality of the UI design development cycle.

% % \noindent \textcolor{red}{----------------------------------------------------------------------}


% \noindent \textcolor{blue}{\textbf{NEW Abstract!} (Pre Writing Clinic 9-July)}

% \noindent \textcolor{blue}{----------------------------------------------------------------------}

% \noindent Exploring different graphical user interface (GUI) design ideas is time-consuming, particularly for novice designers. 
% Given the segmentation masks, design requirement as prompt, and/or preferred visual flow, we aim to facilitate creative exploration for GUI design and generate different UI designs for inspiration.
% While current Vision Language Models (VLMs) can generate GUIs from just prompts, they often lack control over visual concepts and flow that are difficult to convey through language during the generation process. 
% In this work, we present FlowGenUI, a semantic map-guided GUI synthesis method that optionally incorporates visual flow information based on the user's choice alongside language prompts. 
% We demonstrate that our model not only creates more realistic GUIs but also creates "predictable" (how users pay attention to and order of looking at GUI elements) GUIs.
% Our approach uses Stable Diffusion (SD), a large paired image-text pretrained diffusion model with a rich latent space that we steer toward realistic GUIs using a trainable copy of SD's encoder for every condition (segmentation masks, prompts, and visual flow). 
% We further provide a semantic typography feature to create custom text-fonts and styles while also alleviating SD's inherent limitations in drawing coherent, meaningful and correct aspect-ratio text. 
% Finally, a subjective evaluation study of XX non-expert and expert designers demonstrates the efficiency and fidelity of our method.


% This process encourages creativity and prevents designers from falling into habitual patterns.


% ------------------------------------------------------------------
% Joongi Why is it important to create realistic GUI?
% I do not see how the Visual Flow given on the left hand side is reflected in the results on the right hand side. 
% I’d avoid making unsubstantiated claims about designers (falling into habitual patterns).
% The UIs you generate do not “align with users’ attention patterns” but rather try to control it (that’s what visual flow means)
% ------------------------------------------------------------------
% Comments - Writing Clinic - 9th July:
% Improve title. More names: FlowGen
% Figure 1: Use an inference time hand-drawn mask
% Figure 1: Show both workflows. Add a designer --> Input.
% Figure 1: Make them more diverse
% ------------------------------------------------------------------
% Designing graphical user interfaces (GUIs) requires human creativity and time. Designers often fall into habitual patterns, which can limit the exploration of new ideas. 
% To address this, we introduce FlowGenUI, a method that facilitates creative exploration and generates diverse GUI designs for inspiration. By using segmentation masks, design requirements as prompts, and/or selected visual flows, our approach enhances control over the visual concepts and flows during the generation process, which current Vision Language Models (VLMs) often lack.
% FlowGenUI uses Stable Diffusion (SD), a largely pretrained text-to-image diffusion model, and guides it to create realistic GUIs. 
% We achieve this by using a trainable copy of SD's encoder for each condition (segmentation masks, prompts, and visual flow). 
% This method enables the creation of more realistic and predictable GUIs that align with users' attention patterns and their preferred order of viewing elements.
% We also offer a semantic typography feature that creates custom text fonts and styles while addressing SD's limitations in generating coherent, meaningful, and correctly aspect-ratio text.
% Our approach's efficiency and fidelity are evaluated through a subjective user study involving XX designers. 
% The results demonstrate the effectiveness of FlowGenUI in generating high-quality GUI designs that meet user requirements and visual expectations.

% ---------------------------------------


%A critical and general issue remains while using such deep generative priors: creating coherent, meaningful and correct aspect-ratio text. 
%We tackle this issue within our framework and additionally provide a semantic typography feature to create custom text-fonts and styles. 


% %Creating UI designs that are effective in market funneling or any other designer-defined goal requires a good understanding of the visual flow to guide users' attention to UI elements in the desired order. 
% %While current largely pre-trained Vision Language Models (VLMs) can generate GUIs from just prompts, they often lack finer or pixel-precise control which can be crucial for many easy-to-understand visual concepts but difficult to convey through language. 
% % However, obtaining such pixe-level labels is an extremely expensive so we
% % For example - overlaying text on images with certain aspect ratios and two equally separated buttons 
% Additionally, all prior GUI generation work fails to consider visual flow information during the generation process. 
% We demonstrate that visual flow-informed generation not only creates more realistic and human-friendly GUIs but also creates "predictable" (how users pay attention to and order of looking at GUI elements) UIs that could be beneficial for designers for tasks like creating effective market funnels.
% In this work, we present a semantic map-guided GUI synthesis method that optionally incorporates visual flow information based on the user's choice alongside language prompts. 
% Our approach uses Stable Diffusion, a large (billions) paired image-text pretrained diffusion model with a rich latent space that we steer toward realistic GUIs using an ensemble of ControlNets. 
% % TODO: Mention it in 1 sentence:
% A critical and general issue remains while using such deep generative priors: creating coherent, meaningful and correct aspect-ratio text. 
% We tackle this issue within our framework and additionally provide a semantic typography feature to create custom text-fonts and styles. 
% To evaluate our method, we demonstrate that our method produces more "desirable" UIs according to the well-known contrast, repetition, alignment, and proximity principles of design. 
% % We further validate our method through comprehensive automatic non-reference and human-preference aligned scores. (TODO: Maybe Unskip if we get UIClip from Jason!)
% % TODO: Re-word this and only keep ideation cycles and time-efficiency.
% Finally, a subjective evaluation study of XX non-expert and expert designers demonstrates the efficiency and fidelity of our method.
% % improves the time-efficiency by quick iterations of the UI design ideation process.
% %Finally, an evaluation with xx non-expert designers using our contributed <method-name> improves the time-efficiency by quick iterations of the UI design ideation cycle.

%\noindent \textcolor{blue}{----------------------------------------------------------------------}


%In an evaluation with xx designers, we found that GenerativeLayout: 1) enhances designers' exploration by expanding the coverage of the design space, 2) reduces the time required for exploration, and 3) maintains a perceived level of control similar to that of manual exploration.



% Present-day graphical user interfaces (GUIs) exhibit diverse arrangements of text, graphics, and interactive elements such as buttons and menus, but representations of GUIs have not kept up. They do not encapsulate both semantic and visuo-spatial relationships among elements. %\color{red} 
% To seize machine learning's potential for GUIs more efficiently, \papername~ exploits graph neural networks to capture individual elements' properties and their semantic—visuo-spatial constraints in a layout. The learned representation demonstrated its effectiveness in multiple tasks, especially generating designs in a challenging GUI autocompletion task, which involved predicting the positions of remaining unplaced elements in a partially completed GUI. The new model's suggestions showed alignment and visual appeal superior to the baseline method and received higher subjective ratings for preference. 
% Furthermore, we demonstrate the practical benefits and efficiency advantages designers perceive when utilizing our model as an autocompletion plug-in.


% Overall pipeline: Maybe drop semantic typography / visual flow?

%%
%% The code below is generated by the tool at http://dl.acm.org/ccs.cfm.
%% Please copy and paste the code instead of the example below.
%%
\begin{CCSXML}
<ccs2012>
   <concept>
       <concept_id>10003120.10003145.10003151</concept_id>
       <concept_desc>Human-centered computing~Visualization systems and tools</concept_desc>
       <concept_significance>500</concept_significance>
       </concept>
   <concept>
       <concept_id>10003120.10003121.10003129</concept_id>
       <concept_desc>Human-centered computing~Interactive systems and tools</concept_desc>
       <concept_significance>500</concept_significance>
       </concept>
 </ccs2012>
\end{CCSXML}

\ccsdesc[500]{Human-centered computing~Visualization systems and tools}
\ccsdesc[500]{Human-centered computing~Interactive systems and tools}

%%
%% Keywords. The author(s) should pick words that accurately describe
%% the work being presented. Separate the keywords with commas.
\keywords{Visualization, description, caption, mixed-initiative, recommendation.}

%% A "teaser" image appears between the author and affiliation
%% information and the body of the document, and typically spans the
%% page.
\begin{teaserfigure}
  \includegraphics[width=\textwidth]{figures/pdf/teaser.pdf}
  \caption{Examples of text and chart suggestions in \pluto. (A) Coherent title and description that are auto-generated based on the chart. (B) Sentence completion is suggested based on multimodal input from the preceding text and a selection on the chart. (C) The chart is sorted and annotated to enhance coherence with the description.}
  \Description[Three examples of text and chart suggestions in Pluto.]{The first example shows a case where the system generates a title and description for a multi-series line chart. The second example shows how the system can complete a sentence in the description when users click on a mark in the chart. The third example illustrates the system's ability to suggest chart design changes, in this case, to short and annotate a chart based on text in the description.}
  \label{fig:teaser}
  \vspace{1em}
\end{teaserfigure}


% \received{9 October 2024}
% \received[revised]{Day Month 2025}
% \received[accepted]{Day Month 2025}

%%
%% This command processes the author and affiliation and title
%% information and builds the first part of the formatted document.
\maketitle


% humans are sensitive to the way information is presented.

% introduce framing as the way we address framing. say something about political views and how information is represented.

% in this paper we explore if models show similar sensitivity.

% why is it important/interesting.



% thought - it would be interesting to test it on real world data, but it would be hard to test humans because they come already biased about real world stuff, so we tested artificial.


% LLMs have recently been shown to mimic cognitive biases, typically associated with human behavior~\citep{ malberg2024comprehensive, itzhak-etal-2024-instructed}. This resemblance has significant implications for how we perceive these models and what we can expect from them in real-world interactions and decisionmaking~\citep{eigner2024determinants, echterhoff-etal-2024-cognitive}.

The \textit{framing effect} is a well-known cognitive phenomenon, where different presentations of the same underlying facts affect human perception towards them~\citep{tversky1981framing}.
For example, presenting an economic policy as only creating 50,000 new jobs, versus also reporting that it would cost 2B USD, can dramatically shift public opinion~\cite{sniderman2004structure}. 
%%%%%%%% 图1:  %%%%%%%%%%%%%%%%
\begin{figure}[t]
    \centering
    \includegraphics[width=\columnwidth]{Figs/01.pdf}
    \caption{Performance comparison (Top-1 Acc (\%)) under various open-vocabulary evaluation settings where the video learners except for CLIP are tuned on Kinetics-400~\cite{k400} with frozen text encoders. The satisfying in-context generalizability on UCF101~\cite{UCF101} (a) can be severely affected by static bias when evaluating on out-of-context SCUBA-UCF101~\cite{li2023mitigating} (b) by replacing the video background with other images.}
    \label{fig:teaser}
\end{figure}


Previous research has shown that LLMs exhibit various cognitive biases, including the framing effect~\cite{lore2024strategic,shaikh2024cbeval,malberg2024comprehensive,echterhoff-etal-2024-cognitive}. However, these either rely on synthetic datasets or evaluate LLMs on different data from what humans were tested on. In addition, comparisons between models and humans typically treat human performance as a baseline rather than comparing patterns in human behavior. 
% \gabis{looks good! what do we mean by ``most studies'' or ``rarely'' can we remove those? or we want to say that we don't know of previous work doing both at the same time?}\gili{yeah the main point is that some work has done each separated, but not all of it together. how about now?}

In this work, we evaluate LLMs on real-world data. Rather than measuring model performance in terms of accuracy, we analyze how closely their responses align with human annotations. Furthermore, while previous studies have examined the effect of framing on decision making, we extend this analysis to sentiment analysis, as sentiment perception plays a key explanatory role in decision-making \cite{lerner2015emotion}. 
%Based on this, we argue that examining sentiment shifts in response to reframing can provide deeper insights into the framing effect. \gabis{I don't understand this last claim. Maybe remove and just say we extend to sentiment analysis?}

% Understanding how LLMs respond to framing is crucial, as they are increasingly integrated into real-world applications~\citep{gan2024application, hurlin2024fairness}.
% In some applications, e.g., in virtual companions, framing can be harnessed to produce human-like behavior leading to better engagement.
% In contrast, in other applications, such as financial or legal advice, mitigating the effect of framing can lead to less biased decisions.
% In both cases, a better understanding of the framing effect on LLMs can help develop strategies to mitigate its negative impacts,
% while utilizing its positive aspects. \gabis{$\leftarrow$ reading this again, maybe this isn't the right place for this paragraph. Consider putting in the conclusion? I think that after we said that people have worked on it, we don't necessarily need this here and will shorten the long intro}


% If framing can influence their outputs, this could have significant societal effects,
% from spreading biases in automated decision-making~\citep{ghasemaghaei2024understanding} to reducing public trust in AI-generated content~\citep{afroogh2024trust}. 
% However, framing is not inherently negative -- understanding how it affects LLM outputs can offer valuable insights into both human and machine cognition.
% By systematically investigating the framing effect,


%It is therefore crucial to systematically investigate the framing effect, to better understand and mitigate its impact. \gabis{This paragraph is important - I think that right now it's saying that we don't want models to be influenced by framing (since we want to mitigate its impact, right?) When we talked I think we had a more nuanced position?}




To better understand the framing effect in LLMs in comparison to human behavior,
we introduce the \name{} dataset (Section~\ref{sec:data}), comprising 1,000 statements, constructed through a three-step process, as shown in Figure~\ref{fig:fig1}.
First, we collect a set of real-world statements that express a clear negative or positive sentiment (e.g., ``I won the highest prize'').
%as exemplified in Figure~\ref{fig:fig1} -- ``I won the highest prize'' positive base statement. (2) next,
Second, we \emph{reframe} the text by adding a prefix or suffix with an opposite sentiment (e.g., ``I won the highest prize, \emph{although I lost all my friends on the way}'').
Finally, we collect human annotations by asking different participants
if they consider the reframed statement to be overall positive or negative.
% \gabist{This allows us to quantify the extent of \textit{sentiment shifts}, which is defined as labeling the sentiment aligning with the opposite framing, rather then the base sentiment -- e.g., voting ``negative'' for the statement ``I won the highest prize, although I lost all my friends on the way'', as it aligns with the opposite framing sentiment.}
We choose to annotate Amazon reviews, where sentiment is more robust, compared to e.g., the news domain which introduces confounding variables such as prior political leaning~\cite{druckman2004political}.


%While the implications of framing on sensitive and controversial topics like politics or economics are highly relevant to real-world applications, testing these subjects in a controlled setting is challenging. Such topics can introduce confounding variables, as annotators might rely on their personal beliefs or emotions rather than focusing solely on the framing, particularly when the content is emotionally charged~\cite{druckman2004political}. To balance real-world relevance with experimental reliability, we chose to focus on statements derived from Amazon reviews. These are naturally occurring, sentiment-rich texts that are less likely to trigger strong preexisting biases or emotional reactions. For instance, a review like ``The book was engaging'' can be framed negatively without invoking specific cultural or political associations. 

 In Section~\ref{sec:results}, we evaluate eight state-of-the-art LLMs
 % including \gpt{}~\cite{openai2024gpt4osystemcard}, \llama{}~\cite{dubey2024llama}, \mistral{}~\cite{jiang2023mistral}, \mixtral{}~\cite{mistral2023mixtral}, and \gemma{}~\cite{team2024gemma}, 
on the \name{} dataset and compare them against human annotations. We find  that LLMs are influenced by framing, somewhat similar to human behavior. All models show a \emph{strong} correlation ($r>0.57$) with human behavior.
%All models show a correlation with human responses of more than $0.55$ in Pearson's $r$ \gabis{@Gili check how people report this?}.
Moreover, we find that both humans and LLMs are more influenced by positive reframing rather than negative reframing. We also find that larger models tend to be more correlated with human behavior. Interestingly, \gpt{} shows the lowest correlation with human behavior. This raises questions about how architectural or training differences might influence susceptibility to framing. 
%\gabis{this last finding about \gpt{} stands in opposition to the start of the statement, right? Even though it's probably one of the largest models, it doesn't correlate with humans? If so, better to state this explicitly}

This work contributes to understanding the parallels between LLM and human cognition, offering insights into how cognitive mechanisms such as the framing effect emerge in LLMs.\footnote{\name{} data available at \url{https://huggingface.co/datasets/gililior/WildFrame}\\Code: ~\url{https://github.com/SLAB-NLP/WildFrame-Eval}}

%\gabist{It also raises fundamental philosophical and practical questions -- should LLMs aim to emulate human-like behavior, even when such behavior is susceptible to harmful cognitive biases? or should they strive to deviate from human tendencies to avoid reproducing these pitfalls?}\gabis{$\leftarrow$ also following Itay's comment, maybe this is better in the dicsussion, since we don't address these questions in the paper.} %\gabis{This last statement brings the nuance back, so I think it contradicts the previous parapgraph where we talked about ``mitigating'' the effect of framing. Also, I think it would be nice to discuss this a bit more in depth, maybe in the discussion section.}






\section{Related Work}
Our work builds on several lines of research: exploring the role of text with visualizations, visualization and text systems, and image and text authoring interfaces.

\subsection{The role of text with visualizations}
The interplay between text and visual elements in data visualization has been a significant area of interest with increased advocacy for treating text as co-equal to visualization~\cite{stokesgive, lundgard2021accessible}. Kim et al.~\cite{kim2021towards} conducted a study to understand how readers integrate charts and captions in line charts. The study findings indicated that when both the chart and text emphasize the same prominent features, readers take away insights from both modalities. Their research underscores the importance of coherence between visual and textual elements and how external context provided by captions can enhance the reader's comprehension of the chart's message. Building on these insights, Lundgard and Satyanarayan~\cite{lundgard2021accessible} proposed a four-level model for content conveyed by natural language descriptions of visualizations. Their model delineates semantic content into four distinct levels: elemental and encoded properties (Level 1), statistical concepts (Level 2), perceptual and cognitive phenomena (Level 3), and contextual insights (Level 4).

Focusing on the role of textual annotations in visualization, Stokes et al.~\cite{stokes2022striking} observed that readers favored heavily annotated charts over less annotated charts or text alone. This preference highlights the added value of textual annotations in aiding data interpretation, with specific emphasis on how different types of semantic content impact the takeaways drawn by readers. Further contributions by Quadri et al.~\cite{quadri2024you} and Fan et al.~\cite{fan2024understanding} explored high-level visualization comprehension and the impact of text details and spatial autocorrelation on reader takeaways in thematic maps. These studies collectively underline the critical role of textual elements in shaping viewer perceptions, understanding, and accessibility of visual data. Ottley et al.~\cite{ottley2019curious} and Stokes et al.~\cite{stokes2023role} have also contributed to this body of research, focusing on how annotations influence perceptions of bias and predictions, reinforcing the multifaceted impact of text on visual data interpretation.

Our work further explores how text and charts can be better aligned with one another by offering a mixed-initiative authoring interface. Specifically, \pluto~allows leveraging both direct manipulation interactions and user-drafted text to generate recommendations for communicative text and chart design. Furthermore, \pluto's text recommendations explicitly incorporate Lundgard and Satyanarayan's model~\cite{lundgard2021accessible} for semantic information conveyed by visualization descriptions.
In doing so, the system ensures that the generated text has good semantic coverage and structure (e.g., generated descriptions start by conveying the chart's encodings and then list high-level trends) and is appropriate for the intended communicative use (e.g., the semantic information conveyed by titles is different from descriptions accompanying a chart or annotations on the chart).


\subsection{Visualization and text systems}

The integration of visualization and text has led to the development of various systems designed to facilitate the creation, interpretation, and enhancement of data visualizations with textual elements. He et al.~\cite{he2024leveraging} surveyed the leveraging of large models for crafting narrative visualizations, highlighting the potential of AI in supporting the narrative aspect of data visualization. This is complemented by AutoTitle, an interactive title generator for visualizations~\cite{liu2023autotitle}, and Vistext, a benchmark for semantically rich chart captioning~\cite{tang2023vistext}. VizFlow demonstrates the effectiveness of facilitating author-reader interaction by dynamically connecting text segments to corresponding chart elements to help enrich the storytelling experience~\cite{sultanum2021}. This body of research highlights the need for tools to support more nuanced integration of text and visualization.

Supporting the co-authoring of text and charts, Latif et al. introduced Kori~\cite{latif2021kori}, an interactive system for synthesizing text and charts in data documents, emphasizing the seamless integration of visual and textual data for enhanced communication.
\new{CrossData~\cite{chen2022crossdata} presents an interactive coupling between text and data in documents, enabling actions based on the document text and adjusting data values in the text through direct manipulation on the chart.
Such systems illustrate the potential for the bidirectional linking between text and charts to assist rich authoring of data-driven narratives.
}
Furthermore, systems like EmphasisChecker~\cite{kim2023emphasischecker}, Intentable~\cite{choi2022intentable}, Chart-to-text~\cite{obeid2020chart}, DataDive~\cite{kim2024datadive},
\new{InkSight~\cite{lin2023inksight}},
and FigurA11y~\cite{singh2024figura11y} focus on guiding chart and caption creation, supporting readers' contextualization of statistical statements, and assisting in writing scientific alt text. Recent work like SciCapenter supports the composition of scientific figure captions using AI-generated content and quality ratings \cite{hsu2024scicapenter}.
DataTales~\cite{sultanum2023datatales} is another example of a recent system using a large language model for authoring data-driven articles, indicating the growing interest in AI-assisted data storytelling.
These systems collectively illustrate the expanding scope of text integration into visualization, from enhancing data document creation to improving accessibility and data-driven communication. Reviewing the aforementioned tools and the use of generative AI for visualization more broadly, Basole and Major~\cite{basole2024generative} discuss how generative AI methods and tools offer creativity assistance and automation within the visualization workflow, specifically highlighting a shift towards ``human-led AI-assisted'' paradigms, where generative AI not only augments the creative process but also becomes a co-creator.

Aligned with this paradigm shift, \pluto~adopts a mixed-initiative approach that leverages the capabilities of generative AI to help create semantic alignment between the chart and its corresponding text for effective data-driven communication.
However, \pluto~differs from existing chart-and-text authoring tools in three significant ways.
First, going beyond existing systems that primarily leverage unimodal information from the chart to generate text, \pluto~supports multimodal authoring combining information from both the chart (including any direct interactions with marks) and user-drafted text.
Furthermore, unlike prior tools that focus exclusively on generating complete descriptions/captions or titles, \pluto's recommendations can be leveraged in flexible ways to author not only titles and descriptions but also more fine-grained annotations and sentence completions. Second, while existing tools primarily recommend text for a given chart, \pluto's recommendations are bidirectional.
Specifically, the system suggests chart design changes like sorting or adding embellishments based on the authored text, resulting in artifacts that more clearly communicate takeaways via a combination of text and charts. Lastly, unlike existing tools that primarily rely on pre-trained knowledge in generative AI models, \pluto's recommendations are grounded in a theoretical research-based model of semantic information conveyed in visualization text~\cite{lundgard2021accessible}, ensuring the generated text covers the appropriate level of detail and is effective for communication \emph{alongside} the chart.
%%%%%%%%%%%%%%%%%%%%%%%%%%%%%%%%%%%%%%%%%%%%%%%%
\begin{figure*}[t]
    \centering
    \begin{subfigure}[b]{0.49\textwidth}
        \centering
        \includegraphics[width=\textwidth]{figures/criu-cuda-checkpoint-arch.pdf}
        \caption{Checkpoint/restore with CUDA plugin.}
        \label{fig:criu-cuda-checkpoint-arch}
    \end{subfigure}
    \hspace{0.005em}
    \rulesep
    \hspace{0.005em}
    \begin{subfigure}[b]{0.49\textwidth}
        \centering
        \includegraphics[width=\textwidth]{figures/criu-amdgpu-checkpoint-arch.pdf}
        \caption{Checkpoint/restore with AMD GPU plugin.}
        \label{fig:criu-amdgpu-checkpoint-arch}
    \end{subfigure}
    \vspace{-0.5em}
    \caption{An overview of the transparent checkpoint/restore mechanisms with CUDA and AMD GPU plugins for CRIU.}
    \label{fig:criu-gpu-checkpoint-arch}
    \vspace{-1em}
\end{figure*}
%%%%%%%%%%%%%%%%%%%%%%%%%%%%%%%%%%%%%%%%%%%%%%%%
\section{Transparent GPU Checkpointing}\label{sec:design}%
Several open-source tools enable transparent checkpointing of Linux processes running on the CPU~\cite{hhargrove2006berkeley,ansel2009dmtcp,criu}, of which Checkpoint/Restore in Userspace (CRIU) is the most widely used and actively maintained. However, a key limitation of CRIU is that, out of the box, it does not support saving and restoring the state of external hardware devices such as GPUs~\cite{shukla2022singularity,eiling2022cricket}. To address this limitation, we extend the functionality of CRIU with \textit{plugins} (\autoref{sec:gpu-plugins}) that handle GPU state. In comparison to the previous work (utilizing device-proxy mechanisms to intercept, log, and replay API calls~\cite{eiling2022cricket,eiling2023cricket,shukla2022singularity,gupta2024just}), we leverage recently introduced driver capabilities to enable transparent GPU checkpointing~\cite{bhardwaj2021drm,gurfinkel2024checkpointing}. Our aim is to enable \textit{fully transparent} checkpointing that supports a wide range of GPU devices and avoids the performance overheads and limitations of API interception (\autoref{sec:background}).

%%%%%%%%%%%%%%%%%%%%%%%%%%%%%%%%%%%%%%%%%%%%%%%%%%%%%%%%%
\subsection{GPU Plugins}\label{sec:gpu-plugins}
%
Checkpointing of CUDA~\cite{gurfinkel2024checkpointing} and ROCm~\cite{bhardwaj2021fast} applications is achieved through driver capabilities that capture and restore the GPU state (e.g., memory) associated with the target processes. Since this functionality is specific to GPU-accelerated applications and not required for other (e.g., CPU-only) workloads, we implement it as dynamically loadable shared libraries (\textit{plugins}), which can be optionally installed. When these plugins are installed, they are loaded during CRIU's initialization phase and utilized to handle GPU resources. ~\Cref{fig:criu-gpu-checkpoint-arch} illustrates the checkpoint/restore mechanisms with CUDA (\autoref{sec:cuda-plugin}) and AMD GPU (\autoref{sec:amd-gpu-plugin}) plugins. These plugins implement callbacks that are executed at specific stages (known as \textit{hooks}; \autoref{sec:plugin-hooks}) during the checkpoint and restore operations. In addition, each plugin defines \textit{initialization} and \textit{exit} callback functions. The initialization function is called when the plugin is loaded, with an argument specifying the current CRIU operation (\textit{dump}, \textit{pre-dump}, or \textit{restore}). Similarly, the plugin's exit function is invoked at the end of the checkpoint/restore operation, with an argument indicating whether the operation has been successful. This allows the plugins to perform cleanup tasks or, in the event of an error, restore the target processes to their original state.

%%%%%%%%%%%%%%%%%%%%%%%%%%%%%%%%%%%%%%%%%%%%%%%%%%%%%%%%%
\subsubsection{CUDA Plugin}\label{sec:cuda-plugin}%
The CUDA plugin utilizes a checkpointing utility called \texttt{cuda-checkpoint}~\cite{cuda-checkpoint} to perform a set of actions (\textit{lock}, \textit{checkpoint}, \textit{restore}, \textit{unlock}) for all tasks running on NVIDIA GPUs.
%
In particular, these actions are used to enable transparent GPU checkpointing as follows:
\begin{enumerate}[label=\itshape(\roman*\upshape),nosep]
    \item \textit{Locking} all CUDA APIs affecting the GPU state of the target processes and waiting for active operations (e.g., stream callbacks) to complete. \sys uses a timeout (10 seconds by default) with this action to avoid indefinite blocking. If the timeout expires, \sys attempts to restore all CPU and GPU tasks to their original state.

    \item \textit{Checkpointing} the GPU state of CUDA tasks into host memory allocations managed by the driver, and releasing all GPU resources held by the application.
\end{enumerate}
%
Executing these steps results in the CUDA tasks entering a \textit{checkpointed} state without direct reference to GPU hardware.
%
This allows to perform checkpoint/restore operations with CRIU similar to a CPU-only workloads.
%
It is important to note that a standalone invocation of the \texttt{cuda-checkpoint} tool does not handle the state of processes and threads running on the CPU, which can result in undefined behavior.
%
For example, multi-threaded workloads, such as Ollama~\cite{morgan2023ollama}, use error-handling mechanisms that detect unresponsive GPU tasks and restart them.
% 
To prevent inconsistencies and undefined behavior, \sys ensures that all CPU and GPU tasks are suspended (locked) before checkpointing their state.
% 
This is achieved through the Linux ptrace seize with interrupt mechanism~\cite{linux-ptrace}, which halts the execution of relevant processes and threads running on the CPU, allowing \sys to capture their state in a unified CPU-GPU snapshot.
%
Restoring the state of CUDA applications has the following steps:
%
\begin{enumerate}[label=\itshape(\roman*\upshape),nosep]
    \item \textit{Restore} resources such as device memory back to the GPU, memory mappings to their original addresses, and reconstruct CUDA objects (e.g., streams and contexts).
    \item \textit{Unlock} driver APIs, allowing the CUDA application to resume execution on the GPU.
\end{enumerate}
% 
In addition, a boolean flag is set in the inventory image of the snapshot indicating whether it contains GPU state, allowing for compatibility checks and optimizations during restore.

%%%%%%%%%%%%%%%%%%%%%%%%%%%%%%%%%%%%%%%%%%%%%%%%%%%%%%%%%
\begin{figure*}[t]
    \centering
    \begin{subfigure}[]{0.45\textwidth}
        \centering
        \includegraphics[width=\textwidth]{figures/cuda-checkpoint-flow.pdf}
        \caption{Sequence of interactions between CRIU, cuda-checkpoint, and NVIDIA driver.}
        \label{fig:cuda-checkpoint-flow}
    \end{subfigure}
    \hspace{0.95em}
    \rulesep
    \hspace{0.05em}
    \begin{subfigure}[]{0.45\textwidth}
        \centering
        \includegraphics[width=.85\textwidth]{figures/amdgpu-checkpoint-flow.pdf}
        \caption{Sequence of interactions between CRIU and KFD.}
        \label{fig:amdgpu-checkpoint-flow}
    \end{subfigure}
    \vspace{-0.5em}
    \caption{Sequence diagrams of CRIU interactions with NVIDIA and AMD drivers.}
    \label{fig:plugins-checkpoint-workflow}
    \vspace{-1em}
\end{figure*}
%%%%%%%%%%%%%%%%%%%%%%%%%%%%%%%%%%%%%%%%%%%%%%%%%%%%%%%%%
\subsubsection{AMD GPU Plugin}\label{sec:amd-gpu-plugin}
The AMD GPU plugin enables transparent checkpoint/restore using input/output control (\texttt{ioctl}) operations with the Kernel Fusion Driver (KFD). These operations are used to pause and resume the execution of GPU processes, as well as to capture and restore their state, which consist of memory buffer objects (BOs), queues, events, and topology.

GPU-accessible BOs are kernel-managed device (VRAM) and system (graphics translation table) memory, user-managed memory (userptr), and special apertures for signaling (doorbell) and control registers (MMIO). The saved BO properties include buffer type, handle, size, virtual address, device file offset for CPU mapping, and memory contents.

GPU work is typically submitted through user-mode queues with associated user- and kernel-managed memory buffers.
Checkpointing requires preempting and saving the state of all queues belonging to the process.
This includes queue type (compute or DMA), kernel-managed control stack, memory queue descriptor, read/write pointers, doorbell offset, and architected queueing language (AQL) pointer.
The state stored in user-managed BOs includes ring buffer (commands), AQL queue, completion tracking (end-of-processing) buffer, and context save area (preempted shader state).
%
For checkpoint/restore of GPU-to-host signaling events, the allocated event IDs and their signaling state are saved and restored, while the event slot contents are included in the memory data.
%
During checkpointing, the plugin performs the following \texttt{ioctl} operations:
\begin{enumerate}[label=\itshape(\roman*\upshape), nosep]
    \item \texttt{PROCESS\_INFO} -- collecting metadata about the process, pausing its execution, and evicting all queues

    \item \texttt{CHECKPOINT} -- capturing the GPU state described above

    \item \texttt{UNPAUSE} -- restores the evicted queues
\end{enumerate}
%
For security reasons, KFD allows these \texttt{ioctl} calls to be performed only by the same process that opened the \texttt{/dev/kfd} file descriptor, and requires \texttt{CAP\_CHECKPOINT\_RESTORE} or \texttt{CAP\_SYS\_ADMIN} capability.
% 
The plugin performs the following operations during restore:
\begin{enumerate}[label=\itshape(\roman*\upshape),nosep]
    \item \texttt{RESTORE}: reinstates the checkpointed state of processes
    \item \texttt{RESUME}: resumes execution of processes on the GPU
\end{enumerate}

Checkpointed applications can only be restored on systems with compatible GPU topology with the same number, type, memory size, VRAM accessibility by the host, and connectivity between GPUs.
When restoring on a different system or with different subset of GPUs on the same system, the unique GPU identifiers (GPUIDs) might be different during restore. These identifiers are based on properties like the instruction set and compute units. To address this, the plugin performs a translation of the GPUIDs used by the restored processes that applies to all KFD ioctl calls.


%%%%%%%%%%%%%%%%%%%%%%%%%%%%%%%%%%%%%%%%%%%%%%%%%%%%%%%%%
\subsubsection{Plugin Hooks}\label{sec:plugin-hooks}
CRIU provides a set of hooks for checkpointing external resources such as UNIX sockets, file descriptors, mountpoints, and network devices. These hooks serve as an API that can be used with plugins to extend the existing functionality.

\stitle{AMD GPU Plugin Hooks.} CRIU provides two hooks for handling checkpoint and restore operations with device files: \texttt{DUMP\_EXT\_FILE} and \texttt{RESTORE\_EXT\_FILE}. When checkpointing ROCm applications, these hooks are invoked for the \texttt{/dev/kfd} and \texttt{/dev/dri/renderD*} device nodes. The obtained KFD file descriptor is used by the plugin to perform \texttt{ioctl} calls to manage memory, queues, and signals, while per-GPU device render node files are utilized to handle CPU mapping of VRAM and GTT BOs. Two additional plugin hooks have been introduced to enable checkpoint/restore of AMD GPU device virtual memory areas (VMA): \texttt{HANDLE\_DEVICE\_VMA} and \texttt{UPDATE\_VMA\_MAP}. These hooks allow the plugin to translate device file names and mmap offsets to newly allocated ones during restore. In particular, these offsets identify BOs within a render node device file, and the translation mechanism allows a process to be restored on a different GPU. A \texttt{RESUME\_DEVICES\_LATE} hook has been introduced to finalize the restore of userptr mappings and resume execution on the GPU for each restored process, after CRIU's restorer PIE code has restored all VMAs.

\stitle{CUDA Plugin Hooks.} Similarly, two additional plugin hooks have been introduced to invoke \textit{lock} and \textit{checkpoint} actions with the \texttt{cuda-checkpoint} utility for processes running on NVIDIA GPUs: \texttt{PAUSE\_DEVICES} and \texttt{CHECKPOINT\_DEVICES}. The \textit{pause} hook is called immediately before the target CPU processes are frozen. This hook is used by the CUDA plugin to place the corresponding GPU tasks in a \textit{locked} state, halting any pending work and preparing them to be checkpointed. Following this, the \textit{checkpoint} hook is called after all CPU and GPU processes are frozen/locked state to checkpoint their GPU state into host memory. The CUDA plugin also utilizes the \texttt{RESUME\_DEVICES\_LATE} hook to \textit{restore} the state of processes from host memory to the GPU and perform the \textit{unlock} action to resume their execution.

%%%%%%%%%%%%%%%%%%%%%%%%%%%%%%%%%%%%%%%%%%%%%%%%%%%%%%%%%
\subsection{Checkpoint/Restore Workflow}
The sequence of operations described above for AMD GPU and CUDA plugins is illustrated in \Cref{fig:plugins-checkpoint-workflow}. Each plugin uses a different method for checkpointing the GPU state of applications. For CUDA applications, \circled{1} performs a \textit{lock} action that halts the execution of device API calls. Similarly, the AMD GPU plugin invokes a KFD ioctl call to collect metadata, pause execution, and the evict queues for the target ROCm application. \circled{2} checkpoints the GPU state to host memory for CUDA applications. In contrast, at this stage the AMD GPU plugin saves the GPU state into a set of checkpoint files. \circled{3} continues with traditional checkpoint operations for CUDA applications as the GPU state is included in host memory. The AMD GPU plugin at this stage invokes a KFD ioctl call to resume the state of queues. The restore functionality has analogous sequence operations as described in \Cref{sec:gpu-plugins}.
\section{Pluto}

Incorporating these design goals, we implemented \pluto~as a prototype system for authoring semantically-aligned text and charts.

\subsection{Example Usage Scenarios}
\label{sec:scenarios}

\begin{figure}[t!]
    \centering
    \includegraphics[width=.5\textwidth]{figures/pdf/scenario-1.pdf}
    \caption{Upon processing a description, \pluto~flags statements that require manual verification (A) and automatically \annotation{annotates} the chart based on data references in the description (B).}
    \Description[Two examples shows Pluto's suggestions for statement verification and chart annotation.]{In the first case, the system highlights a potentially incorrect statement in the user's description of the chart, allowing the user to inspect and verify whether the statement is correct. The second example illustrates how the system annotates (here, by adding a gold stroke) a portion of the chart by inspecting the data value references in the description.}
    \label{fig:scenario-1}
\end{figure}

Figure~\ref{fig:interface} shows \pluto's interface.
Users can drag and drop data fields onto visual encoding channels to create charts.
To underscore a unified experience for authoring charts and text, the system also presents an explicit title and description region just above and below the chart.
Users can also annotate the chart by creating text callouts via a context menu invoked on the chart, or by adding visual embellishments using the chart editor (e.g., borders to highlight marks).
System recommendations are either directly applied to the title, chart, or description or displayed to the right of the chart and description (Figure~\ref{fig:interface}F, G) for authors to review (\textbf{DG4}, \textbf{DG5}).

To illustrate how \pluto's interface and features collectively enable unified authoring of text and charts for data-driven communication, we now describe three vignettes\footnote{These usage scenarios are modeled on examples of how participants used \pluto~during the study described in Section~\ref{sec:study}.}.
These examples are also illustrated in the supplementary video.
% Pluto's friends from Disney cartoons: Dinah, Ronnie, Fifi

\vspace{.5em}\noindent\textbf{Augmenting generated text with built-in safeguards and chart annotations.}
Consider Dinah, an analyst at a movie production company.
Dinah is tasked with summarizing a chart showing movie earnings across genres (Figure~\ref{fig:teaser}A) to share as part of a report her company plans to publish.

Dinah is unsure about how to start her description, so she uses the {\small\faIcon{feather-alt}} \textbf{Generate} feature to bootstrap her authoring process (\textbf{DG3}).
In response, \pluto~inspects the chart and returns a description for Dinah to review (Figure~\ref{fig:teaser}A-bottom).
Dinah peruses the generated text and manually edits it for conciseness.
She then clicks {\small{\faIcon[regular]{lightbulb}}} \textbf{Suggest} to get ideas for using a combination of the description and the chart for better communication.

Analyzing the description, \pluto~makes three changes.
The system flags
% \reco{description statements with ambiguous takeaways}
description statements with ambiguous takeaways
for review using a dashed border, suggesting that Dinah manually verifies the text with the chart before sharing it with others (Figure~\ref{fig:scenario-1}A) (\textbf{DG3}).
Using \pluto's interactive highlighting feature, Dinah hovers over the statement in the description to see portions of the chart it refers to.
Reflecting on the flagged text, Dinah updates it to remove the modifier ``\textit{significant}'' and make her description more objective for the readers' interpretation.

\pluto~also suggests a title, ``\textit{Action and Animation Dominate: Gross Earnings by Genre (2010-2019)}'' based on both the narrative in the description and the underlying trends in the chart (Figure~\ref{fig:teaser}A-top).

Finally, besides suggestions for the text, \pluto~also adds an annotation to the chart highlighting the key regions the text describes (\textbf{DG1}).
In this case, detecting the emphasis on the \textit{Action} and \textit{Animation} genres and their trends between 2013 and 2017, the system adds a \annotation{gold stroke} around the corresponding lines in the chart (Figure~\ref{fig:scenario-1}B).

Satisfied with her changes based on the system suggestions, Dinah shares the title, chart, and description with her colleagues for review.

\begin{figure}[t!]
    \centering
    \includegraphics[width=\linewidth]{figures/pdf/scenario-2.pdf}
    \caption{Examples of \pluto's recommendations including an in-place sentence completion (A), \annotation{annotations} based on a chart's description (B), and a text callout generated based on marks selected on a chart (C).}
    \Description[Three examples of Pluto's recommendations.]{In the first case, the system completes the user's sentence when the user presses the tab key on a partially typed statement. In the second example, the system adds a gold stroke to marks on the chart to highlight that they are referenced in the text. In the third example, the user clicks two marks on the chart and asks Pluto to generate a text callout - this results in the system adding a text bubble with content focusing on the selected marks.}
    \label{fig:scenario-2}
\end{figure}

\vspace{.5em}
\noindent\textbf{Steering text competition through data-driven narratives and multimodal input.}
Imagine Ronnie, a financial analyst, writing a report on the monetary impact of bird strikes across the US based on the chart shown in Figure~\ref{fig:scenario-2}A.

Ronnie notices that most states, with the exception of Texas and New Jersey, have tall orange bars corresponding to costs incurred by bird strikes during the day.
Noting this observation, Ronnie types, ``\textit{Across all states, most bird strikes happen during the day.}''
Wanting to emphasize the exception of Texas and New Jersey, Ronnie types ``\textit{However, }'' and presses the \key{Tab} key to ask the system to finish the sentence.
Parsing the preceding sentence and the data trends from the chart, \pluto~generates the completion ``\textit{Texas breaks this trend by incurring the highest costs from birdstrikes at dawn.}'' (Figure~\ref{fig:scenario-2}A)
Ronnie accepts this completion but edits it to include New Jersey.

Next, to emphasize the high cost incurred by incidents in New York at night, Ronnie types ``\textit{It is also interesting that}," clicks on the teal bar showing the total cost for \textit{New York} at \textit{Night}, and again invokes a sentence completion.
Using the multimodal input from the chart selection and the existing description text (\textbf{DG2}), \pluto~suggests the text ``\textit{New York experiences its highest costs from birdstrikes at night, again deviating from the predominant trend of daytime incidents.}'' (Figure~\ref{fig:teaser}B)

Content with his description, Ronnie uses {\small{\faIcon[regular]{lightbulb}}} \textbf{Suggest} to see how he can further improve his text and the chart.
\pluto~processes the description and adds a \annotation{stroke} to visually highlight the states \textit{Texas}, \textit{New York}, and \textit{New Jersey} and the times \textit{Dawn} and \textit{Night} based on their high data values and the emphasis in the description (Figure~\ref{fig:scenario-2}B) (\textbf{DG1}).
Seeing this annotation gives Ronnie an idea to explicitly call out the striking differences in values between Texas and New York.
He selects the two tall bars within Texas and New York and uses {\small{\faIcon[regular]{feather-alt}}} \textbf{Generate Callout} to create a textual annotation directly overlaid onto the chart (Figure~\ref{fig:scenario-2}C) (\textbf{DG2}).
Manually refining the generated callout and visual embellishments (\textbf{DG5}), Ronnie saves the annotated chart and description for his report.

\vspace{.5em}
\noindent\textbf{Guiding manual authoring via system recommendations.}
Imagine Fifi, a realtor who is using the grouped bar chart shown in Figure~\ref{fig:interface}D to author an email blast on house pricing trends for her clients.
Analyzing the chart, Fifi manually writes a description with three sentences, shown in Figure~\ref{fig:interface}E.
With this initial text, she invokes the {\small{\faIcon[regular]{lightbulb}}} \textbf{Suggest} feature to see how she can improve her text and chart for communication.

Parsing the description, \pluto~detects that it lacks a summary of the chart's encodings and also detects that there is no higher-level statement encompassing a trend across multiple home types.
Translating these into recommendations, \pluto~suggests adding a brief statement about the chart's layout and a statement talking about the general impact of garage types across house types, respectively (Figure~\ref{fig:interface}G) (\textbf{DG3}).
Acknowledging these might be useful as overview statements for her readers, Fifi previews what her description would read like with the suggested text by hovering on the recommendations and subsequently accepting them.
As with the other examples, \pluto~also suggests a title (\textit{Home Type and Garage Influence on Property Prices}), but Fifi finds this too formal and manually adjusts it to make it more catchy: ``\textit{Can I afford both a car and a home?: The Influence of Garage Type on Property Prices}.''

\begin{figure}[t!]
    \centering
    \section{Problem Studied}\label{sec:def}
We first present Fixed-Radius Near Neighbor (FRNN) queries and then formalize Aggregation Queries over Nearest Neighbors (AQNNs) that build on them. We then state our problem.

\subsection{Nearest Neighbor Queries}\label{subsec:FRNN}
We build on generalized Fixed-Radius Near Neighbor (FRNN) queries \cite{FRNNSurvey}. Given a dataset \( D \), a query object \( q \), a radius \( r \), and a distance function \( dist \), a generalized FRNN query retrieves all nearest neighbors of \( q \) within radius \( r \). More formally:
\[
NN_D(q, r) = \{x \in D \mid dist(x, q) \leq r\},
\]
where \(x\) is any data point in \(D\) and \(dist(x, q)\) denotes the distance between them. We use \(|NN_D(q,r)|\) to denote the neighborhood size of \(q\). As shown in Fig. \ref{fig:framework}, given a radius \(r\) and a target patient \(q\), patients in the dotted circle are nearest neighbors, and the neighborhood size is 6.

\subsection{Aggregation Queries over Nearest Neighbors}\label{subsec:AQNN} 
Given an FRNN query object \(q\) in dataset \(D\), a radius \(r\), and an attribute \(\texttt{attr}\), an Aggregation Query over Nearest Neighbors (AQNN) is defined as:
\[ \text{agg}(NN_D(q,r)[\texttt{attr}]) \]
where agg is an aggregation function, such as $\mathtt{AVG}$, $\mathtt{SUM}$, and $\mathtt{PCT}$, and \(NN_D(q,r)[\texttt{attr}]\) denotes the bag of values of attribute \texttt{attr} of all FRNN results of \(q\) within radius \(r\). 
% \end{definition}

An AQNN expresses aggregation operations to capture key insights about the neighborhood of a query object. For example, \(\mathtt{AVG}\) can be used to reflect the average heart rate or systolic blood pressure of patients in the neighborhood, providing a measure of typical health conditions. \(\mathtt{SUM}\) is useful for assessing cumulative effects, such as the total cost of treatments in the neighborhood that instructs public policy in terms of health. Similarly, $\mathtt{PCT}$ can be used to find the proportion of patients in the neighborhood of a patient of interest, relative to the population in the dataset.
%\laks{Why is finding the total \#meds to NNs or the total treatment cost of everyone in the NN interesting?}

% \texttt{MIN} and \texttt{MAX} are not included in the aggregation functions because they only capture extreme values, which may not represent the typical characteristics of the nearest neighbors and are more sensitive to outliers. 
% \laks{AVG is also sensitive to outliers, but we still allow it. isn't the real reason we don't consider MIN/MAX because they are amenable to estimation via sampling?} We choose \texttt{PCT} instead of \texttt{COUNT} in order to provide a normalized measure that remains comparable across different neighborhood sizes. It allows for more consistent interpretation of relative popularity \cite{moore1989introduction}.


Fig. \ref{fig:framework} illustrates an example of an AQNN: ``\textit{Find the average systolic blood pressure of patients similar to an insomnia patient \(q\)}''. The aggregation function is \(\mathtt{AVG}\) and the target attribute of interest is systolic blood pressure. Exact query evaluation requires consulting physicians (or predicting embeddings by an expensive machine learning model) for all 500 patients in \(D\) and calculate \(q\)'s nearest neighbors wrt \(r\) \cite{DBLP:journals/isci/RodriguesGSBA21}. We refer to such highly accurate but computationally expensive models as \textit{oracle models}, denoted as \(O\), including deep learning models trained on domain-specific data or human expert annotations \cite{DBLP:conf/sigmod/LuCKC18}. Using oracle models is very expensive \cite{sze2017efficient, DujianPQA, DBLP:journals/pvldb/KangGBHZ20}. To address that, we seek an approximate solution by \textit{proxy models}, denoted as \(P\), that are at least one order of magnitude cheaper than oracle models. In the example, if consulting physicians for one patient incurs one cost unit, calling a cheap machine learning model instead incurs at most \(0.1\) cost unit. Once the similar patients are identified, their systolic blood pressure values are averaged and returned as  output. The use of a proxy model may reduce the accuracy of the neighborhood prediction and hence, we should judiciously call oracle and proxy models to minimize the error of aggregate results.

Note that the values of the target attribute \texttt{attr} are \textit{not} predicted but are instead known quantities.

\subsection{Problem Statement}
Given an AQNN, our goal is to return an approximate aggregate result by leveraging both oracle and proxy models while reducing error and cost.


    \caption{Conceptual schema representing the key text and chart elements in \pluto's interface.}
    \Description[Conceptual schema representing the key text and chart elements in Pluto.]{Conceptual schema representing the key text and chart elements in Pluto}
    \label{fig:schema}
\end{figure}

Besides the text suggestions, \pluto~also detects that the description emphasizes the home types with highest and lowest values, whereas the home types in the chart are sorted alphabetically.
To resolve this disparity, the system provides a chart design recommendation to sort the home types by price ranging from the highest to lowest (Figure~\ref{fig:interface}F) (\textbf{DG1}).
Fifi accepts this recommendation as it can give her clients a glanceable summary of some key takeaways in her text (Figure~\ref{fig:teaser}C).

\subsection{Conceptual Model}

To enable the aforementioned workflows and recommendations, we model the various components across the text and the chart in \pluto~as a \emph{conceptual schema} summarized in Figure~\ref{fig:schema}.
We use this schema in the subsequent sections to detail how the system tracks user input and generates recommendations.

Specifically, a \schemaPrimary{Chart} is represented by mapping \schemaPrimary{Data} onto specific visual encodings (in this case, a Vega-Lite \schemaPrimary{Specification}~\cite{satyanarayan2016vega}).
The \schemaPrimary{ActiveSelection} enumerates the data items that have been selected through direct manipulation interaction (e.g., in Figures~\ref{fig:teaser}B,~\ref{fig:scenario-2}C).
Additionally, the chart can also have one or more \schemaPrimary{Annotations}.
These can be textual comments on the chart, visual embellishments applied to individual marks (e.g., Figures~\ref{fig:interface}D,~\ref{fig:scenario-1}B), or overlays like a regression line on a scatterplot or a line marking the average value across all bars in a bar chart.

The chart's \schemaPrimary{Title} is \schemaPrimary{Text} that may include references to \schemaPrimary{DataItems} (e.g., \textit{Genre}: [\textit{Action}, \textit{Animation}] in Figure~\ref{fig:teaser}A).

\begin{figure}[t!]
    \centering    
    \includegraphics[width=\linewidth]{figures/architecture.png}
    \caption{\pluto's system architecture overview}
    \Description[Pluto's system architecture.]{Given the active chart and text from the interface, Pluto uses a heuristic parser process the information and to generate recommendations. When providing recommendations involving text suggestions, the system also uses an LLM in parallel to generate the recommended text.}
    \label{fig:architecture}
\end{figure}

The \schemaPrimary{Description} is represented as a collection of \schemaPrimary{Statements}.
Each statement maps to one of the four semantic statement types proposed by Lundgard and Satyanarayan~\cite{lundgard2021accessible}---namely, \schemaSecondary{encoding}, \schemaSecondary{perceptual-trend}, \schemaSecondary{data-fact}, \schemaSecondary{domain-specific}, or  \schemaSecondary{other} (e.g., a statement about the data source for a chart).
Additionally, similar to the title, statements in the description may also contain references to specific \schemaPrimary{DataItems}.

Note that this schema is not exhaustive (e.g., there may be additional types of annotations, statement types, or chart selections) and was primarily designed to operationalize the recommendations in \pluto.

However, we hope that the idea of formalizing not only the chart but also its associated text can inspire future work on grammars and systems for data-driven communication through a \emph{combination} of text and charts.

\subsection{System Overview}

\pluto~is implemented as a web-based application and is developed using Python, HTML/CSS, and JavaScript.
Visualizations in the tool are created using Vega-Lite~\cite{satyanarayan2016vega}.
The system currently supports three encoding channels (\texttt{x}, \texttt{y}, \texttt{color}) and three mark types (\texttt{bar}, \texttt{line}, \texttt{point}).
Collectively, this combination of encoding and mark types enables specifying several visualizations, including single- and multi-series bar charts, line charts, histograms, and scatterplots, covering a breadth of visualizations explored in prior systems~\cite{kim2023emphasischecker,latif2021kori,sultanum2023datatales,kim2024datadive,choi2022intentable,obeid2020chart,liu2020autocaption,hsu2021scicap,alam2023seechart}.

Figure~\ref{fig:architecture} depicts a high-level overview of the system architecture.
Specifically, \pluto~uses a combination of an LLM (GPT-4~\cite{achiam2023gpt}) and a heuristics-based approach for generating suggestions.
Specifically, requests like generating an entire description or a title from a chart are directly fulfilled using the LLM.
In other cases, a custom parser extracts information from the text and chart and also classifies statements in the description based on their semantic levels~\cite{lundgard2021accessible}.
This extracted information is leveraged by a heuristics-based recommendation engine to generate recommendations, including adding/editing text, adding mark annotations, and suggesting chart design changes such as sorting, among others.
In cases where the recommendations involve generated text suggestions, the recommendation engine either uses its built-in templates or interacts with the LLM to pass it the required context for the text generation.
In the subsequent sections, we detail these components and \pluto's recommendation generation process.

\subsection{Text and Chart Parsing}
\label{sec:parser}

The parser extracts a number of features from the text and the chart that are used to determine system recommendations.

\textbf{Text.}
The parser analyzes text in the description, title, and annotations to identify \schemaPrimary{DataItems}.
The system uses a combination of a lexicon- and grammar-based approach adapted from prior natural language interfaces for visualization (e.g.,~\cite{gao2015datatone,setlur2016eviza,narechania2020nl4dv}) to detect data item references.
Specifically, given an input text, the parser extracts a list of N-grams and compares the N-grams to available data fields and values, looking for both syntactic (e.g., misspellings) and semantic similarities (e.g., synonyms) employing Levenshtein distance~\cite{yujian2007normalized} and the Wu-Palmer similarity score~\cite{wu1994verb}, respectively.
The extracted items are subsequently used to support features like adding mark annotations (Figures~\ref{fig:scenario-1}B,~\ref{fig:scenario-2}B) and highlighting relevant portions of the chart while hovering over statements in the description (Figure~\ref{fig:scenario-1}A).

In addition to detecting data item references, the parser also classifies description statements into one of the five statement types.
We use a random forest classifier with BERT~\cite{sanh2019distilbert} to match a statement to one of the four semantic levels of text---\schemaSecondary{encoding}, \schemaSecondary{perceptual-trend}, \schemaSecondary{data-fact}, or \schemaSecondary{domain-specific}~\cite{lundgard2021accessible}.
If the classification probability for all four types is below 60\% \new{(an empirically set threshold)}, a statement is labeled as \schemaSecondary{other}.
The classifier is trained on a dataset of $2147$ chart description statements curated by Lundgard and Satyanarayan~\cite{lundgard2021accessible}.
Our choice for the classifier was based on comparing the results of 10-fold cross-validation between different techniques, including support vector machines~\cite{Cortes1995SupportVectorN}, random forests~\cite{breiman2001}, logistic regression~\cite{strother1967}, and na\"{i}ve Bayes~\cite{Duda1974PatternCA}.

\textbf{Chart.}
The parser also detects salient \schemaPrimary{DataItems} in the chart.
For instance, for bar charts, the parser shortlists up to three categories with the highest and lowest values.
For line charts, the system records time periods or specific timestamps with the most significant peaks and drops based on computing the smoothed z-scores, and so on.
These chart-specific heuristics to determine salient targets are derived from prior ``auto-insight'' generating visualization systems (e.g.,~\cite{cui2019datasite,wang2019datashot,srinivasan2018augmenting,demiralp2017foresight}) and research on mappings between analytic tasks and visualizations (e.g.,~\cite{amar2005low,schulz2013design,saket2018task}).
The salient items detected from the chart are subsequently used to suggest potential annotations and to generate text suggestions for verifying the description statements (e.g., Figure~\ref{fig:scenario-1}A).

\subsection{Recommendation Generation}
\label{sec:reco-generation}
\pluto~uses a combination of heuristics, text templates, and an LLM to suggest changes to the text and the chart.
The vignettes in \S\ref{sec:scenarios} illustrate the breadth of \pluto's recommendations, which can broadly be categorized into three groups: 1) \textit{full-text recommendations} to populate descriptions, titles, or text annotations, 2) \textit{description statement recommendations} to fine-tune or update an existing description, and 3) \textit{chart design recommendations} to ensure the chart is structurally aligned to its corresponding text.

\begin{figure*}[t!]
    \centering
    \includegraphics[width=.96\textwidth]{figures/pdf/recommendations-full-text.pdf}
    \caption{Overview of full-text recommendation generation. Given the context of the chart, data, and any existing text, \pluto~generates new text for the description, title, or annotations. \schemaPrimary{Input parameters} with a \schemaPrimary{?} are optional only used if available.}
    \Description[Overview of full-text recommendation generation.]{Given the context of the chart, data, and any existing text, Pluto uses an LLM to suggest the title, description, and text annotations.}
    \label{fig:full-text-recommendation}
\end{figure*}

\vspace{.5em}
\noindent{\large{\textbf{Full-text Recommendations}}
\vspace{.5em}

\noindent{}These recommendations are invoked using the {\small\faIcon{feather-alt}} \textbf{Generate} button and suggest text for the title, description, or text annotations (e.g., Figure~\ref{fig:teaser}A and Figure~\ref{fig:scenario-2}C).
All recommendations in this category are generated using the LLM, and Figure~\ref{fig:full-text-recommendation} presents an overview of the input/output for the recommendations.
The LLM prompts are provided as part of the supplementary material.

\textbf{Description.}
We use \schemaPrimary{StatementTypes} to systematically generate descriptions in \pluto.
Specifically, we provide the LLM with examples of the four statement types from Lundgard and Satyanarayan's dataset~\cite{lundgard2021accessible}.
Following the findings from Tang et al.'s qualitative analysis of the VisText chart caption dataset~\cite{tang2023vistext}, we prompt the LLM to generate a description with a constraint that the text should start with an \schemaSecondary{encoding} statement and is followed by at least one \schemaSecondary{preceptual-trend}.
This pattern follows the classic \textit{``Overview first''} mantra for visualization design~\cite{shneiderman2003eyes} and ensures the description talks about the chart and high-level takeaways before listing details of individual items and values.
An example of this constraint in play can be noticed in the generated description in Figure~\ref{fig:full-text-recommendation} where the first statement, ``\textit{This line chart...}'' describes the chart's encodings and the second highlights how ``\textit{...certain genres like Animation and Action consistently outperform others...}'' before talking about other lower-level observations from the chart.

By default, the LLM only uses the chart type and \schemaPrimary{Data} to generate a description.
However, if the chart contains an \schemaPrimary{ActiveSelection}, has \schemaPrimary{TextAnnotations}, or the user has entered a \schemaPrimary{Title}, these are also used as context for generating the description.

\textbf{Title.}
By default, the system uses a chart's \schemaPrimary{Specification} and \schemaPrimary{Data} to generate a title.
However, similar to generating descriptions, if there is additional context in the form of an \schemaPrimary{ActiveSelection}, \schemaPrimary{TextAnnotations}, or a \schemaPrimary{Description},\\ \pluto~leverages that information to generate a title that highlights the key message across the chart and previously added text.
For instance, the suggested title in Figure~\ref{fig:full-text-recommendation} contains \textit{Action} and \textit{Animation} since these are called out as focal entities in the \schemaPrimary{Description}.

\textbf{Text annotations.}
\pluto~also allows users to directly select items of interest on the chart and generate annotations based on the \schemaPrimary{ActiveSelection} (\textbf{DG2}).
An example of this type of text generation is shown in Figure~\ref{fig:full-text-recommendation}-bottom where the \schemaPrimary{TextAnnotation} is created based on the two selected bars for the states of \textit{New York} and \textit{Texas}, respectively.
The example also illustrates the effect of including previously entered text as context for the generation.
Specifically, notice that because the \schemaPrimary{Description} talks about Texas and New York deviating from the general trend, the generated annotation text also adopts that framing and phrasing (e.g., ``\textit{...contrary to the norm...}'') for consistency.
\newline

\noindent{}Since all the above recommendations leverage the chart type (inferred via the \schemaPrimary{Specification}) and \schemaPrimary{Data}, from an implementation standpoint, we pass the chart type and the data to the LLM only once in an initial context setting prompt when a chart is created.
Our choice to include the chart type as part of the context was motivated by our initial testing, during which we found that including the chart type improved the LLM's performance in terms of detecting the most relevant data patterns (e.g., trends for line charts, extremes for bar charts, correlation for scatterplots).

\vspace{.5em}
\noindent{\large\textbf{Description Statement Recommendations}}
\vspace{.5em}

\noindent{}Besides suggesting text from scratch, \pluto~also recommends adding or editing \schemaPrimary{Statements} within an existing \schemaPrimary{Description} (\textbf{DG3}).
Statement recommendations take different forms, including in-place suggestions to verify statement correctness (Figure~\ref{fig:scenario-1}A), statement addition/reordering recommendations presented to the side of an existing description (Figure~\ref{fig:interface}G), and in-place text completions (Figure~\ref{fig:teaser}B and Figure~\ref{fig:scenario-2}A).

\begin{figure*}[t!]
    \centering
    \includegraphics[width=\textwidth]{figures/pdf/recommendations-sentence-level.pdf}
    \caption{Overview of the description statement recommendations in \pluto. The system uses a combination of the chart's specification, data, the active description, and selections on the chart to recommend changes to the description.}
    \Description[Overview of description statement recommendations in Pluto.]{To generate statement addition/reordering suggestions, the system uses a combination of a text parser and a heuristic recommendation engine that inspects the statement types (e.g., encoding, perceptual-trend) to suggest content. A LLM is used when perceptual trend statements are suggested as part of the recommendations. For statement verification recommendations, the system uses a parser to identify statement items, extract data references, and subsequently checks these against the underlying data to flag a statement as needs verification or not. Lastly, for statement completion recommendations, the system passes the context of the current chart, description, and any active selections to the LLM to have it generate the statement text.}
    \label{fig:statement-recommendations}
\end{figure*}

\textbf{Statement addition and reordering.}
These recommendations are designed to ensure the description is semantically rich and has a good narrative structure.
Figure~\ref{fig:statement-recommendations} summarizes the logic used to generate statement recommendations.
Note that because the rules used to generate these recommendations are already baked into the description generation prompt described above, the statement addition/reordering recommendations typically appear only for manually entered descriptions.

To generate the recommendations, we follow the same guidelines applied to generate a description from scratch. \pluto~first checks for the presence of an \schemaSecondary{encoding} statement and at least one \schemaSecondary{perceptual-trend}.
If these are absent or placed after other statements, the system recommends adding or reordering these statements.
For instance, consider the example in Figure~\ref{fig:statement-recommendations}A.
Detecting that the input description lacks both \schemaSecondary{encoding} and \schemaSecondary{precentual-trend} statements, \pluto~suggests adding one statement of each type.
We initially also explored suggesting adding \schemaSecondary{data-facts}.
However, the formative studies and our testing revealed that these recommendations quickly became mundane and merely listed data values from the chart, leading to us subsequently disabling them.
Drawing on prior work~\cite{tang2023vistext}, we use a template-based approach to suggest \schemaSecondary{encoding} statements and invoke the LLM to suggest \schemaSecondary{perceptual-trends}.

To avoid overwriting the existing description, the recommendations are presented next to the description area instead of being directly applied to the text (Figure~\ref{fig:interface}G) (\textbf{DG5}).
Authors can preview the updated description with the suggested changes by hovering on the recommendations.
Furthermore, because these recommendations are heuristically generated, \pluto~also provides an explanation for why a recommendation was shown.
The output in Figure~\ref{fig:statement-recommendations}A shows examples of these explanations accompanying an \schemaSecondary{encoding} statement suggestion and a \schemaSecondary{preceptual-trend} statement suggestion generated when the input description only contains \schemaSecondary{data-facts}.

\textbf{Statement verification.}
Prior research has shown that both manually-written and LLM-generated descriptions can contain erroneous mentions of data trends or values~\cite{kim2023emphasischecker,tang2023vistext}. For instance, a category stated to have the highest value in the text may not actually be the category with the highest value in the chart.
Motivated by this prior work and the experts' feedback on earlier prototypes, we check for potentially incorrect data references in the text and flag them for authors to manually verify (\textbf{DG3}).

Figure~\ref{fig:statement-recommendations}B gives an overview of how \pluto~flags statements for verification.
Specifically, the system first checks if the statement contains one or more \schemaPrimary{DataItems} (typically found in \schemaSecondary{data-fact} and \schemaSecondary{perceptual-trend} statements) and the type of takeaway the statement calls out (e.g., min/max, trend, correlation).
\pluto~then uses this extracted information to validate the mentioned items and values against the underlying \schemaPrimary{Data}, flagging the statement for review if it fails to detect a match.
An example of this is shown in Figure~\ref{fig:statement-recommendations}B, where the sentence is flagged because the system is unable to confirm if the fluctuation in values is ``significant''.
Authors can click on a flagged statement to {\small\faIcon{check}} \textit{Confirm} its correctness.

\textbf{Statement text completion.}
In addition to retrospective recommendations on the description text, \pluto~also allows users to request text completion suggestions while writing their descriptions by pressing the \key{Tab} key.
To generate these completions, the system sends the current \schemaPrimary{Description} along with any \schemaPrimary{ActiveSelections} on the chart to the LLM and prompts it to complete or suggest the last sentence.
Examples of these suggestions can be seen in Figure~\ref{fig:scenario-2}A, where the system generates a text completion based on the \schemaPrimary{Description} alone and Figures~\ref{fig:teaser}B and~\ref{fig:statement-recommendations}C, where the completion is generated based on multimodal input, including the \schemaPrimary{description} text and the \schemaPrimary{ActiveSelection} of \textit{New York} on the chart.
If the text completion recommendation is invoked with an empty description, the system follows the same rules as it does when generating descriptions from scratch and starts by suggesting an \schemaSecondary{encoding} statement followed by a \schemaSecondary{perceptual-trend} before other statements.

\vspace{.5em}
\noindent{\large\textbf{Chart Design Recommendations}}
\vspace{.5em}

\begin{figure*}[t!]
    \centering
    \includegraphics[width=\textwidth]{figures/pdf/recommendations-chart-design.pdf}
    \caption{Summary of \pluto's process for recommending chart design changes based on the authored text. Given a chart and accompanying text, the system extracts data references from both the chart and the text, and compares the references to suggest potential design changes to make the chart more structurally aligned to the text.}
    \Description[Overview of the recommendation logic for suggesting chart design changes.]{To suggest design changes based on a given description, the system first inspects the chart to identify key data items (e.g., categories with highest values in bar charts). Comparing the data references in the input text to chart, the system suggests annotating the references in the chart, prioritizing items that have a higher salience in the chart. Additionally, the system also compares the order of the items in the text and the chart and if there is a difference, suggests sorting the chart to match the order of data references in the text.}
    \label{fig:chart-recommendations}
\end{figure*}

\noindent{}Following \textbf{DG1}, \pluto's recommendations are geared not only to improve the text but also to align the chart with the text, resulting in a better-combined reading experience.
Specifically, once a description is entered, the system generates two types of recommendations for updating the chart.

\textbf{Annotations.}
\pluto~recommends visual embellishments based on the description to help emphasize the key takeaways from the text in the chart following the approach summarized in Figure~\ref{fig:chart-recommendations}.
The recommended annotations are applied by default since they do not impact the chart's structure/layout, but authors are provided with controls to refine or remove the applied annotations (e.g., Figures~\ref{fig:interface}B and \ref{fig:scenario-1}B).

The system first extracts a list of potential \schemaPrimary{DataItems} from both the text and the chart and assigns a saliency score to these items based on saliency or ``interestingness'' metrics~\cite{demiralp2017foresight,wang2019datashot,srinivasan2018augmenting,lundgard2021accessible} (e.g., categories with extreme values in bar charts have higher saliency scores, time ranges in line charts with more variability have higher saliency than those with lower variability, targets mentioned in \schemaSecondary{perceptual-trend} statements are considered more salient than those referenced in \schemaSecondary{data-facts}).

Next, \pluto~checks for overlaps in \schemaPrimary{DataItems} extracted from the chart and the text to shortlist candidates for annotation based on a combined saliency score.
Checking for the combined saliency scores across the text and chart ensures that the emphasized items are important in both the underlying data and the author's interpretation of the chart conveyed via the text.
In cases where there are no overlaps between \schemaPrimary{DataItems} in the chart and the text, \pluto~defaults to adjusting the chart to match the author's description and adds an annotation for the most salient \schemaPrimary{DataItems} in the text.
An example of this is shown in Figure~\ref{fig:chart-recommendations} where the system highlights \textit{Home Type}$=$\textit{Single Family} since it has the highest average value (i.e., it is a salient data item in the chart) and is also explicitly called out in the description.

Besides annotating specific marks or regions on the chart, the system also adds overlay annotations based on references to aggregate values (e.g., adding a line to highlight the average value across categories in a bar chart or a regression line to emphasize the correlation between fields on a scatterplot).

\textbf{Sorting.}
During our initial testing and formative interviews, we noted that there was often a disparity in the order in which data items or marks appear on a chart and the order in which they are discussed in the text.
For example, consider the bar chart in Figure~\ref{fig:interface}C.
Although the bars are sorted alphabetically by home type, the description highlights takeaways starting with the home type having the highest value (i.e., \textit{Single Family} homes).
To enable a more aligned chart and text reading experience, \pluto~checks for such disparities and recommends sorting order changes for alignment.

Specifically, as summarized in Figure~\ref{fig:chart-recommendations}, the system extracts an ordered list of \schemaPrimary{DataItems} in the description and compares it to the items in the chart.
If the description starts by focusing on the highest or lowest category and the chart is not ordered to match that narrative, \pluto~suggests sorting the chart in a descending or ascending order, respectively.
An example of the sorting recommendation in action can be seen in Figure~\ref{fig:chart-recommendations} where applying the sort aligns the chart and text with both emphasizing \textit{Single Family} homes having the highest values and \textit{Condos} having the lowest values.

Unlike the annotation recommendations, however, sorting recommendations are presented next to a chart (Figure~\ref{fig:interface}F) and not applied by default to prevent an abrupt visual layout change by the system.
As shown in Figure~\ref{fig:chart-recommendations}, authors can preview the suggested order by hovering on the recommendation and clicking to apply that recommendation.
\section{Preliminary User Study}
\label{sec:study}

Using \pluto~as a design probe, we conducted a preliminary user study to assess the utility of the proposed interactive experience for authoring semantically aligned text and charts for data-driven communication.
Specifically, we focused on two goals: 1) understand if and how the system suggestions aid the joint authoring of text and charts, and 2) gather feedback on \pluto's current recommendations and features.

\subsection{Participants and Setup}

We recruited ten participants ($P1$-$P10$) through mailing lists at a data analytics software company.
\new{The recruitment call sought for individuals who use a combination of charts and text for data-driven communication.
Participation in the study was voluntary and} participants were recruited on a first-come, first-serve basis.
Seven participants rated themselves as visualization experts, and three participants had a moderate level of expertise in visualization. 
Regarding participants' professional backgrounds, six participants were solution architects, two participants worked as visualization consultants, and two were data analysts.
Of the ten participants, six participants reported that they used a combination of text and charts to communicate data once in a few weeks, two frequently used text and charts as part of their jobs, and two participants had only recently started using charts and text for communication as part of their new roles.
This mix of participant backgrounds w.r.t. chart and text authoring ensured that the study captured holistic feedback on \pluto~from varying users, including novices, moderate-level users, and experts.

Speaking about their existing authoring experience, all participants noted they manually inspected the chart and wrote a corresponding text blurb while sharing.
Five participants said they sometimes annotated a chart's screenshot with text during communication.
The two participants who frequently communicated using text and charts also noted using the dashboard authoring features in tools like Tableau.

All sessions were conducted remotely via the Zoom video conferencing software. The prototype was hosted on a local server on the experimenter's laptop. Participants were granted control over the experimenter's screen during the session, and all studies followed a think-aloud protocol. The audio, video, and on-screen actions were recorded for all sessions with permission from the participants.

\subsection{Procedure}

We initially considered an evaluation of \pluto~against an existing chart-and-text authoring tool. However, we did not find a freely available baseline that provided equivalent features to \pluto~in terms of supporting multimodal authoring, bidirectional editing of text and charts, and using the semantic structure of descriptions~\cite{lundgard2021accessible} during text recommendation.
We also considered an ablation study to compare the system's output to that from the underlying GPT-4 model.
However, we did not see value in this approach as \pluto's utility stems from the integration of multimodal interactions and recommendations, and not one standalone feature focused on text generation.

We ultimately decided on a qualitative study where all participants interact with \pluto~and perform the same set of tasks as this would allow us to observe usage patterns and assess the utility of the recommendations.
Sessions lasted between 44-57 minutes ($\mu$: 53 min., \new{$\sigma$: 4 min.}) and were organized as follows:

\vspace{.5em}
\noindent\textbf{Introduction} [$\sim$10min]:
Participants were given an overview of the study and asked to share their background information. This briefing was followed by an introduction to \pluto's interface and key features. We used a simple bar chart about US college costs as a running example for the introduction.

\vspace{.5em}
\noindent\textbf{Tasks} [$\sim$30min]:
Participants were presented with three charts and were asked to write complementary text (including title, description, and annotations) for the chart so they could share their findings with others who may be interested in the data.
To ensure the tasks were realistic and of appropriate difficulty, we consulted the two experts from the formative study (Section~\ref{sec:design-goals}) and conducted three pilot studies with participants from academic and journalism backgrounds.

The three charts included a multi-series line chart about movie earnings by genre, a stacked bar chart about costs incurred by airplane bird strikes in the US, and a grouped bar chart about US housing prices shown in Figures~\ref{fig:teaser}A, \ref{fig:scenario-2}A, and \ref{fig:interface}D, respectively.
\new{We selected these chart types based on their frequency of use in prior chart-and-text authoring systems (e.g.,~\cite{sultanum2023datatales,chen2022crossdata,choi2022intentable,kim2023emphasischecker,lin2023inksight}) as well as the general prevalence of bar and line charts across visualization platforms (e.g.,~\cite{battle2018beagle,purich2023toward}) that are commonly leveraged for data-driven communication.}
The order of charts was randomized across participants. 
We asked the participants to use the system as they saw fit (e.g., start with auto-generated text, manually write text and use the suggestions to edit, or manually compose the chart and text without using system recommendations).

\vspace{.5em}
\noindent\textbf{Debrief} [$\sim$10min]:
Sessions concluded with a semi-structured interview discussing the overall experience, support for different authoring tasks, and areas for improvement. Participants also filled out a questionnaire rating the quality and utility \pluto's recommendations and features.

\subsection{Results}

\begin{figure}[t!]
    \centering
    \includegraphics[width=\linewidth]{figures/pdf/results-wo-means.pdf}
    \caption{Participant responses to post-session questions about \pluto's recommendations.}
    \Description[Participant responses to post-session questionnaire]{Overall, participants were generally positive about Pluto's recommendations, particularly commending the use of selection to guide suggestions and the recommendations for updating the chart design based on the text. Detailed responses are as follows. Question: Overall, how likely are you to use such a system in practice for authoring text and charts for data-driven communication? Responses: 4 participants said "very likely", 4 said "likely", 1 said "moderately likely" and 1 said "unlikely".
    Question: How helpful were the text generation suggestions? Responses: 3 participants said "very helpful", 6 said "helpful", and 1 said "unhelpful". Question: How helpful was it to use selections on the chart to guide the text generation features? Responses: 3 participants said "very helpful", 5 said "helpful", and 2 said "moderately helpful". Question: How helpful were the system recommendations for editing the description? Responses: 1 participant said "very helpful", 2 said "helpful", 6 said "moderately helpful", and 1 said "unhelpful". Question: How helpful were the system recommendations for updating the chart? Responses: 4 participants said "very helpful", 4 said "helpful", and 2 said "moderately helpful".}
    \label{fig:responses}
\end{figure}

Overall, participants were receptive to \pluto's features and recommendations, noting they would use such a system in practice for data-driven communication (Figure~\ref{fig:responses}). We summarize general themes from participant feedback.


\vspace{.5em}
\noindent\textbf{Full-text recommendations are helpful for bootstrapping.}
Participants generally found the text generation recommendations useful, with nine participants rating them as `helpful' or `very helpful' (Figure~\ref{fig:responses}).
Participants noted that these suggestions were particularly useful for bootstrapping the authoring process. For instance, commenting on the generated description, $P1$ said, ``\textit{I would love to use these as a first draft. Just helps get my juices flowing.}''
However, participants' feedback on the quality of generated descriptions was mixed, with some participants (P2, P4, P7) finding the suggested text too verbose.
Both $P4$ and $P7$, for instance, noted that they would prefer the system generate a bullet list of key ``insights," allowing them to use the insights to manually craft a narrative.
$P2$ and $P4$ (both frequent users of text and charts for communication) also stated they would like more control over the generated text in terms of its verbosity and writing style (e.g., configuring the text to have a more ``casual'' versus ``formal'' tone depending on the communication context).

Feedback on the suggested titles was unanimously positive, however.
For example, commending a suggested title for the house pricing chart, $P8$ said, ``\textit{Terms like `Influence of' really makes chart feel less templated and more informative.}''
Upon seeing the title ``Action and Adventure Dominate:...", $P5$ noted, ``\textit{I am very impressed with the title. It's almost like it took all my changes and gave me a summary."}
Participants were also pleasantly surprised by the quality of text completions for individual sentences (e.g., Figure~\ref{fig:scenario-2}A) stating ``\textit{the text completion followed my lead very well}" ($P6$).


\vspace{.5em}
\noindent\textbf{Using selections to guide recommendations saves time and instills confidence.}
Overall, participants appreciated the ability to directly select items on the chart to drive text recommendations, with 8/10 participants noting this feature was `helpful' or `very helpful' (Figure~\ref{fig:responses}).
The positive feedback for the multimodal text generation feature often stemmed from its ability to assist in faster writing and to adjust the scope of the generated text.

$P8$, for instance, particularly appreciated the ability to use chart selections to drive auto-complete and annotations (e.g., Figures~\ref{fig:teaser}B and \ref{fig:scenario-2}C) and said, ``\textit{It was nice to be able to point at things and have the system give the text. It saved me a lot of time.}"
P2 and P4, who found the description generation feature too verbose, switched to using selection-based text generation, with P4 stating that ``\textit{the selection at least ensured the generated text was about something I want to talk about.}''

\vspace{.5em}
\noindent\textbf{Description editing recommendations are primarily useful for validation.}
Participants' reactions to recommendations for verifying and editing an entered description (e.g., Figures~\ref{fig:scenario-1}A and \ref{fig:interface}G) were more neutral (Figure~\ref{fig:responses}, Q4).

Seven participants explicitly noted that they appreciated that the system flagged text that needed verification.
P6, for instance, said, ``\textit{Humans are lazy. Having that extra step is going to save a lot of people a lot of embarrassment.}''
Noting that the verification suggestions made him more critical, $P3$ said, ``\textit{It's important that we think about what charts are saying...and it's referring back and making sure. It's definitely a step in the right direction.}''
The remaining participants either felt the suggestions should not appear for manually written text ($P4$) or that statements could be flagged more conservatively, reducing the work on the authors' part ($P7$, $P8$).

Participants used recommendations to add or reorder text in the description only three times across all sessions. Participants commented that these recommendations were too ``obvious'' (particularly referring to the \schemaSecondary{encoding} statement suggestions).
$P9$, for instance, ``\textit{This [\schemaSecondary{encoding} statement] suggestion is too generic and not as interesting as the others that tell me key points about the data.}''
However, all participants noted that the recommendations were appropriately placed on the side and did not impede their workflow (\textbf{DG5}).

\vspace{.5em}
\noindent\textbf{Chart annotation and design recommendations foster an integrated reading experience.}
Participants were generally very impressed by \pluto's suggestions to annotate or sort the chart based on an entered description, with 8/10 participants rating this feature as `helpful' or `very helpful.'

For example, even $P7$, who was generally critical about the other features, exclaimed, ``\textit{Loved that just loved that! It's easy to forget the chart when writing because I know what I should be focusing on, but someone else looking at the chart may not.}''
Appreciating the ability to adjust the system-suggested annotations further (\textbf{DG5}), P1 commented, ``\textit{It's great that it highlighted some elements in the chart based on my text. It made me see things from a reader's perspective and go back and make additional changes.}''

\vspace{.5em}
\noindent\textbf{Usage patterns.}
As typical with mixed-initiative interfaces, there was a constant back and forth between the participants' authoring actions and the system recommendations. We observed four high-level usage strategies around how participants started the authoring process by writing descriptions. We summarize these strategies below, as they can help inform the user experience of future systems\footnote{Note that some participants adopted different strategies across tasks, resulting in the participant count across strategies adding up to more than 10}.

\begin{tightItemize}
    \item \textit{Generate then edit.}
    The most common strategy across participants (7/10) was to start with auto-generated descriptions ({\small\faIcon{feather-alt}} \textbf{Generate}).
    Once the description text was generated, participants would either first peruse through it and make edits or directly request suggestions for improvements to the text and the chart ({\small{\faIcon[regular]{lightbulb}}} \textbf{Suggest}).
    
    \vspace{.5em}
    \item \textit{Guide text completion.}
    Four participants started writing their descriptions leveraging the in-place text competition suggestions (via the \key{Tab} key).
    Participants would typically mention a data entity (e.g., New York) or a narrative hook (e.g., ``\textit{however},'' ``\textit{but},'' ``\textit{sadly}'') in their text and have the system suggest the remaining statement that they would use as-is or edit further.
    The {\small{\faIcon[regular]{lightbulb}}} \textbf{Suggest} feature was primarily used to verify the statements and get chart annotation recommendations to complement the text.

    \vspace{.5em}
    \item \textit{Clipboard text generation.}
    On two occasions, participants used the system text suggestions to create a clipboard of ideas.
    Specifically, participants started by selecting visually salient entities on the chart and asking \pluto~to generate a series of text annotations. Subsequently, the participants went through these annotations and either edited and kept them on the chart, moved them to the description, or deleted them.
    The {\small{\faIcon[regular]{lightbulb}}} \textbf{Suggest} feature was used after this initial drafting of the description and text annotations to get further editing suggestions for the chart.

    \vspace{.5em}
    \item \textit{Manual writing with chart design recommendations.}
    Two participants (both experts at communicating data using text and charts) typically started their process by manually drafting the description and then asking the system to {\small{\faIcon[regular]{lightbulb}}} \textbf{Suggest} improvements. These participants also noted that they utilized the suggest feature to obtain chart annotations and design suggestions based on their input text (\textbf{DG1}).
\end{tightItemize}
\section{Discussion}

\begin{figure*}[t!]
    \centering    \includegraphics[width=.6\linewidth]{figures/pdf/extensibility-examples.pdf}
    \caption{Examples illustrating the extensibility afforded by the proposed conceptual schema. Here, \pluto's text generation and recommendation modules are used \emph{as-is} to author text for a geographic map and an adjacency matrix. In (A), the system generates a description based on the map and subsequently highlights six states in the chart that are emphasized in the text. In (B), \pluto~suggests a sentence completion based on multimodal input of the previously drafted description and a user selection on the chart.}
    \Description[Two examples showing how the underlying schema used to develop Pluto can be extended to other charts.]{The first example illustrates the system is able to generate a textual description based on a choropleth map. The example also shows that some states are highlighted in the map because they are referenced in the generated text. The second example shows a sentence completion recommendation generated in response to clicking on a cell in a heatmap, further illustrating the system's ability to support charts that were beyond the initial set of visualizations implemented in the tool.}
    \label{fig:extensibility-examples}
\end{figure*}

\subsection{Grammar-based Approaches for Chart \& Text Interfaces}

While we currently implement only a set of charts and recommendations in \pluto, the underlying schema (Figure~\ref{fig:schema}) is not limited to this set.
Since the framework is based on the underpinning data and conceptual elements of the chart and text, the presented schema can be generalized to other cases.
Figure~\ref{fig:extensibility-examples} highlights examples of this generalization by illustrating \pluto's~text and embellishment suggestions for a geographic map and a graph dataset represented as a network matrix.

Furthermore, the idea of breaking individual \schemaPrimary{Statements} into \schemaPrimary{Text}, \schemaPrimary{StatementType}, and \schemaPrimary{DataItems} enables interactive text-chart linking (Figure~\ref{fig:scenario-1}A) as well as targeted recommendations for description statements (Figure~\ref{fig:statement-recommendations}).
Beyond these recommendations, however, this breakdown can also be leveraged more generally to design better linting systems for writing text for charts.
For example, by inspecting the \schemaPrimary{Text} and \schemaPrimary{StatementType}s in a given description and applying findings from studies on accessible visualization descriptions~\cite{yaneva2015accessible,lundgard2021accessible,jung2021communicating,kim2023explain}, visualization systems can flag inaccessible descriptions and suggest improvements to make the descriptions more accessible.

While these are just examples, leveraging logical concepts such as those in the presented schema (Figure~\ref{fig:schema}) affords a compelling opportunity to create a unifying grammar to capture the multimodal nature of chart + text authoring interfaces.
Besides streamlining the interface and interaction design of future tools, such a grammar that is centered around abstract concepts underpinning text and charts can also facilitate effective communication with LLMs by allowing systems to only share required task-specific context with the models in a structured representation.

\subsection{Design Considerations}

Based on the study observations and feedback, we derive four design considerations for future systems exploring AI-based suggestions to assist unified text and chart authoring.

\vspace{.5em}
\noindent\textbf{Format and phrasing of text are as important as its content.}
Participants recognized the value in text generation and noted that the generated text often picked up on the salient data points and trends in the data. However, the two participants who frequently used text and charts for communication, in particular, critiqued the verbosity and tone of the generated text, indicating that the text was ``\textit{too formal or complex}'' for their consumers. Future systems should explore providing users with control over the configuration of the properties of the generated text (e.g., choosing between paragraphs and bullets, adjusting the level of verbosity, and setting the tone or writing style)~\cite{louis2014}.


\vspace{.5em}
\noindent\textbf{Leverage multimodal context during text generation.}
Participants particularly appreciated text suggestions that were based on chart selections or built upon previously authored text.
$P5$, for instance, referred to the interaction experience of having some text and then using the chart in tandem to guide the system (Figure~\ref{fig:teaser}B) as being a ``\textit{smooth and controlled authoring flow.}'' To support similar fluid and coherent authoring experiences, future systems should consider multimodal context from both the chart and previously written text when suggesting new text.

\vspace{.5em}
\noindent\textbf{Include techniques to help verify the text with the chart.}
Participants appreciated the interactive highlighting and statement verification features in \pluto~(Figure~\ref{fig:scenario-1}A), noting that the features encouraged them be more critical of the text (regardless of whether it was written by them or was system-generated).
With the growing prevalence of generated text, future systems should continue incorporating such interactive verification features to mitigate false statements and enhance synergy between the text and the chart.

\vspace{.5em}
\noindent\textbf{Adjust chart design to align with the text description.}
Unlike prior systems that focus on the unidirectional task of generating text from charts, \pluto~introduces a bidirectional flow by also recommending changes to the chart design based on the text (Figure~\ref{fig:chart-recommendations}).
We observed that participants extensively used and appreciated these suggestions, commenting that the chart design recommendations helped them gain a better reader's perspective.
To this end, future systems should continue to explore techniques to leverage the bidirectional flow of information between text and charts
and generate chart design suggestions for an integrated reading experience.

\section{Limitations and Future Work}
\noindent\textbf{Incorporate data context into generated text.}
\pluto~currently only considers the active chart and data fields while suggesting text.
However, since the recommendations are provided within a general-purpose visualization specification tool, the system can access other data fields not displayed in the active chart (Figure~\ref{fig:interface}A).
For instance, during the study, viewing the generated text for the bird strikes chart (Figure~\ref{fig:scenario-2}A), $P8$ said, ``\textit{I wish it could generate some text explaining why the costs were high based on the number of incidents.}''
Although $P8$ was able to work around this limitation by creating the second chart, inspecting the new visualization, and returning to the original chart, automatically considering other fields as part of the generated text is an open area for future work.
Such dataset-level text (as opposed to chart-level text) can also help make the authored descriptions semantically rich by including more \schemaSecondary{domain-specific} statements.

\vspace{.5em}
\noindent\textbf{Text formatting recommendations.}
The current implementation leans heavily on content generation; however, future iterations should also focus on providing sophisticated formatting suggestions that can add expressivity to the text based on the semantic levels.
For example, titles can be rendered more prominently with annotations and footnotes shown to support the chart. Complementing chart design suggestions and providing structural suggestions for the description, such as including line breaks or using bullet points instead of paragraphs, can also help improve readability.

\vspace{.5em}
\noindent\textbf{Supporting multiple views and articles.} \pluto~currently supports authoring a single chart and assumes the resulting chart and text are static.
While such an approach covers a popular scenario for data-driven communication as evidenced by prior research focusing on this setup (e.g.,~\cite{choi2022intentable,obeid2020chart,liu2020autocaption,tang2023vistext,hsu2021scicap,stokes2022striking}), people also use dashboards and interactive storytelling articles when communicating with charts~\cite{sarikaya2018we,segel2010narrative}. Investigating support for multiple views and operationalizing \pluto's recommendations in more expressive interactive data-driven article authoring tools such as Idyll Studio~\cite{conlen2021idyll} is an open topic for future work.


\vspace{.5em}
\noindent\textbf{Manage data sharing and combine heuristics with LLMs.}
As discussed in \S\ref{sec:reco-generation}, we currently pprovide the context of the chart's specification and the data to the LLM.
However, depending on the size of the data and or the usage context,  sharing data directly with the LLM may not always be feasible.
Exploring alternatives to generate text without sharing the underlying data is an important direction for future work.
For instance, one possible approach could be to use heuristics-based approaches and prior knowledge of analytic tasks (e.g.,~\cite{manelski-krulee-1965-heuristic,snowy2021,kim2023emphasischecker}) to identify key trends from a chart and then leverage the LLM for appropriately consolidating the extracted information into a coherent text narrative.


\vspace{.5em}
\noindent\textbf{Conduct longitudinal evaluation across data domains.}
The combination of two expert interviews, three pilots, and an evaluation study with ten participants helped us validate \pluto~as a proof-of-concept for the general idea of a mixed-initiative interface for authoring semantically aligned charts and text.
\new{However, this evaluation is only preliminary, and assessing the practical value of a tool like \pluto~deems more longitudinal studies spanning different data domains and individuals with varying levels of expertise~\cite{lam2012,shneiderman2006strategies}.}

\vspace{.5em}
\noindent\new{\textbf{Incorporating safeguards for machine failures.}
With the current implementation of \pluto, we utilize a pretrained LLM to generate text and a heuristic parser to analyze the generated text for additional suggestions.
To make users aware of potential errors in either of these steps, we provide interactive visual previews that can be viewed before accepting system suggestions (e.g., Figure~\ref{fig:scenario-1}A and Figure~\ref{fig:chart-recommendations}).
While \pluto~serves as a proof-of-concept, developing a more generalized system requires a deeper investigation of the types of errors that might be generated both at the LLM and the parser level.
Categorizing and understanding the distribution of such errors can ultimately help develop better recommendation modules, but also design interface and interaction strategies to mitigate system errors.
}
\section{Conclusion}\label{sec:conclusion}
This work introduces a novel approach to TOT query elicitation, leveraging LLMs and human participants to move beyond the limitations of CQA-based datasets. Through system rank correlation and linguistic similarity validation, we demonstrate that LLM- and human-elicited queries can effectively support the simulated evaluation of TOT retrieval systems. Our findings highlight the potential for expanding TOT retrieval research into underrepresented domains while ensuring scalability and reproducibility. The released datasets and source code provide a foundation for future research, enabling further advancements in TOT retrieval evaluation and system development.
\section{Acknowledgements}


%%
%% The next two lines define the bibliography style to be used, and
%% the bibliography file.
\bibliographystyle{ACM-Reference-Format}
\bibliography{references}

\end{document}
\endinput