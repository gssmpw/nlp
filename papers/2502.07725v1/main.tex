%%
%% This is file `sample-manuscript.tex',
%% generated with the docstrip utility.
%%
%% The original source files were:
%%
%% samples.dtx  (with options: `manuscript')
%% 
%% IMPORTANT NOTICE:
%% 
%% For the copyright see the source file.
%% 
%% Any modified versions of this file must be renamed
%% with new filenames distinct from sample-manuscript.tex.
%% 
%% For distribution of the original source see the terms
%% for copying and modification in the file samples.dtx.
%% 
%% This generated file may be distributed as long as the
%% original source files, as listed above, are part of the
%% same distribution. (The sources need not necessarily be
%% in the same archive or directory.)
%%
%% Commands for TeXCount
%TC:macro \cite [option:text,text]
%TC:macro \citep [option:text,text]
%TC:macro \citet [option:text,text]
%TC:envir table 0 1
%TC:envir table* 0 1
%TC:envir tabular [ignore] word
%TC:envir displaymath 0 word
%TC:envir math 0 word
%TC:envir comment 0 0
%%
%%
%% The first command in your LaTeX source must be the \documentclass command.
%%%% Small single column format, used for CIE, CSUR, DTRAP, JACM, JDIQ, JEA, JERIC, JETC, PACMCGIT, TAAS, TACCESS, TACO, TALG, TALLIP (formerly TALIP), TCPS, TDSCI, TEAC, TECS, TELO, THRI, TIIS, TIOT, TISSEC, TIST, TKDD, TMIS, TOCE, TOCHI, TOCL, TOCS, TOCT, TODAES, TODS, TOIS, TOIT, TOMACS, TOMM (formerly TOMCCAP), TOMPECS, TOMS, TOPC, TOPLAS, TOPS, TOS, TOSEM, TOSN, TQC, TRETS, TSAS, TSC, TSLP, TWEB.
% \documentclass[acmsmall]{acmart}

%%%% Large single column format, used for IMWUT, JOCCH, PACMPL, POMACS, TAP, PACMHCI
% \documentclass[acmlarge,screen]{acmart}

%%%% Large double column format, used for TOG
% \documentclass[acmtog, authorversion]{acmart}

%%%% Generic manuscript mode, required for submission
%%%% and peer review
\documentclass[sigconf]{acmart}
% \documentclass[sigconf,review,anonymous]{acmart}
% \documentclass[manuscript,review,anonymous]{acmart}
% \documentclass[manuscript,screen,review]{acmart}
%% Fonts used in the template cannot be substituted; margin 
%% adjustments are not allowed.
%%

\usepackage{fontawesome5}
\usepackage{xcolor}
\usepackage{soul}
% \usepackage[most]{tcolorbox}
\usepackage{wrapfig}
\usepackage{booktabs}
\usepackage{tabularx}
% \usepackage[normalem]{ulem}
\usepackage{enumitem}
\usepackage{verbatim}
\usepackage{syntax}
\usepackage{parcolumns}
\usepackage{listings}
\usepackage[frozencache,cachedir=.]{minted}
\usepackage[linesnumbered,ruled,vlined]{algorithm2e}
\usepackage{algorithmicx}
\usepackage{algpseudocode}
\usepackage{amsmath}
\usepackage{caption}

\newcommand{\pluto}{\textsc{Pluto}}

\newcommand{\arjun}[1]{\textcolor{orange}{AJ: #1}}
\newcommand{\vidya}[1]{\textcolor{blue}{(VS: #1)}}
\newcommand{\arvind}[1]{\textcolor{red}{(AS: #1)}}

\newcommand{\new}[1]{\textcolor{black}{#1}}

\DeclareCaptionType{InfoBox}

\definecolor{explicitBlue}{HTML}{095A94}
\definecolor{customCodeOrange}{HTML}{AB4C0C}
\definecolor{customGreen}{HTML}{54AA54}
\definecolor{customGold}{HTML}{FFD601}


\newcommand{\schemaPrimary}[1]{\texttt{\color{explicitBlue}{#1}}}
\newcommand{\schemaSecondary}[1]{\texttt{\color{customCodeOrange}{#1}}}

\newenvironment{tightItemize}{\begin{itemize}[leftmargin=.15in]\itemsep
-2.1pt}{\end{itemize}}

% Define a new tcbox with a red dashed border
% \newtcbox{\mybox}[1][]{%
%   nobeforeafter,
%   colframe=red, % Color of the frame
%   colback=white, % Background color
%   boxrule=0.5pt, % Frame thickness
%   boxsep=0pt, % Separation between frame and content
%   left=1mm, % Left padding
%   right=1mm, % Right padding
%   top=1mm, % Top padding
%   bottom=1mm, % Bottom padding
%   arc=2mm, % Corner rounding
%   borderline={2pt}{2pt}{red, dashed}, % Border line (width, separation, style)
%   % #1 % For specifying additional options
% }

% \newtcbox{\recobox}{nobeforeafter, colback=white, colframe=white, boxrule=0.5pt, arc=1pt, boxsep=0pt,left=2pt,right=2pt,top=1.75pt,bottom=1.5pt,tcbox raise base,enhanced, borderline={0.5pt}{.1pt}{red, dashed}}

% \newcommand{\reco}[1]{\recobox{\small{#1}}}
\newcommand{\reco}[1]{\fcolorbox{red}{white}{{\small{#1}}}}

% \newtcbox{\keybox}{nobeforeafter, colback=gray!10, colframe=gray!50, boxrule=0.5pt, arc=1pt, boxsep=0pt,left=2pt,right=2pt,top=1.75pt,bottom=1.5pt,tcbox raise base}

% \newcommand{\key}[1]{\keybox{\small{\texttt{#1}}}}
\newcommand{\key}[1]{\fcolorbox{gray!50}{gray!10}{{\small{#1}}}}

% \newtcbox{\annotationbox}{nobeforeafter, colback=gray!05, colframe=customGold, boxrule=1pt, arc=1pt, boxsep=0pt,left=2pt,right=2pt,top=1.75pt,bottom=1.5pt,tcbox raise base}

% \newcommand{\annotation}[1]{\annotationbox{\small{#1}}}
\newcommand{\annotation}[1]{\fcolorbox{customGold}{gray!05}{{\small{#1}}}}

%% \BibTeX command to typeset BibTeX logo in the docs
\AtBeginDocument{%
  \providecommand\BibTeX{{%
    \normalfont B\kern-0.5em{\scshape i\kern-0.25em b}\kern-0.8em\TeX}}}

%% Rights management information.  This information is sent to you
%% when you complete the rights form.  These commands have SAMPLE
%% values in them; it is your responsibility as an author to replace
%% the commands and values with those provided to you when you
%% complete the rights form.
\setcopyright{acmlicensed}
\copyrightyear{2025}
\acmYear{2025}
\setcopyright{cc}
\setcctype{by}
\acmConference[IUI '25]{30th International Conference on Intelligent User Interfaces}{March 24--27, 2025}{Cagliari, Italy}
\acmBooktitle{30th International Conference on Intelligent User Interfaces (IUI '25), March 24--27, 2025, Cagliari, Italy}\acmDOI{10.1145/3708359.3712122}
\acmISBN{979-8-4007-1306-4/25/03}



%% These commands are for a PROCEEDINGS abstract or paper.
% \acmConference[IUI '25]{The ACM Conference on Intelligent User Interfaces}{March 24--27, 2025}{Cagliari, Italy}
% %
% %  Uncomment \acmBooktitle if th title of the proceedings is different
% %  from ``Proceedings of ...''!
% %
% \acmBooktitle{The ACM Conference on Intelligent User Interfaces (IUI '25), March 24--27, 2026, Cagliari, Italy}
% \acmISBN{978-1-4503-XXXX-X/18/06}

% \settopmatter{printacmref=false}
% \setcopyright{none}
% \renewcommand\footnotetextcopyrightpermission[1]{}
\pagestyle{plain}

%%
%% Submission ID.
%% Use this when submitting an article to a sponsored event. You'll
%% receive a unique submission ID from the organizers
%% of the event, and this ID should be used as the parameter to this command.
%%\acmSubmissionID{123-A56-BU3}

%%
%% For managing citations, it is recommended to use bibliography
%% files in BibTeX format.
%%
%% You can then either use BibTeX with the ACM-Reference-Format style,
%% or BibLaTeX with the acmnumeric or acmauthoryear sytles, that include
%% support for advanced citation of software artefact from the
%% biblatex-software package, also separately available on CTAN.
%%
%% Look at the sample-*-biblatex.tex files for templates showcasing
%% the biblatex styles.
%%

%%
%% The majority of ACM publications use numbered citations and
%% references.  The command \citestyle{authoryear} switches to the
%% "author year" style.
%%
%% If you are preparing content for an event
%% sponsored by ACM SIGGRAPH, you must use the "author year" style of
%% citations and references.
%% Uncommenting
%% the next command will enable that style.
%%\citestyle{acmauthoryear}

%%
%% end of the preamble, start of the body of the document source.
\begin{document}

%% The "title" command has an optional parameter,
%% allowing the author to define a "short title" to be used in page headers.

\title{\pluto: Authoring Semantically Aligned Text and Charts\\ for Data-Driven Communication}

%%
%% The "author" command and its associated commands are used to define
%% the authors and their affiliations.
%% Of note is the shared affiliation of the first two authors, and the
%% "authornote" and "authornotemark" commands
%% used to denote shared contribution to the research.
\author{Arjun Srinivasan}
\affiliation{%
  \institution{Tableau Research}
  \city{Seattle}
  \state{Washington}
  \country{USA}
}
\email{arjunsrinivasan@tableau.com}

\author{Vidya Setlur}
\affiliation{%
  \institution{Tableau Research}
  \city{Palo Alto}
  \state{California}
  \country{USA}
}
\email{vsetlur@tableau.com}

\author{Arvind Satyanarayan}
\affiliation{%
  \institution{Massachusetts Institute of Technology}
  \city{Cambridge}
  \state{Massachusetts}
  \country{USA}
}
\email{arvindsatya@mit.edu}

%%
%% By default, the full list of authors will be used in the page
%% headers. Often, this list is too long, and will overlap
%% other information printed in the page headers. This command allows
%% the author to define a more concise list
%% of authors' names for this purpose.
\renewcommand{\shortauthors}{Srinivasan, et al.}

%%
%% The abstract is a short summary of the work to be presented in the
%% article.

Humor is a social binding agent. It is an act of creativity that can provoke emotional reactions on a broad range of topics. Humor has long been thought to be “too human” for AI to generate. However, humans are complex, and humor requires our complex set of skills: cognitive reasoning, social understanding, a broad base of knowledge, creative thinking, and audience understanding. We explore whether giving AI such skills enables it to write humor. We target one audience: Gen Z humor fans. We ask people to rate meme caption humor from three sources: highly upvoted human captions, 2) basic LLMs, and 3) LLMs captions with humor skills. We find that users like LLMs captions with humor skills more than basic LLMs and almost on par with top-rated humor written by people. We discuss how giving AI human-like skills can help it generate communication that resonates with people. 


%%
%% The code below is generated by the tool at http://dl.acm.org/ccs.cfm.
%% Please copy and paste the code instead of the example below.
%%
\begin{CCSXML}
<ccs2012>
   <concept>
       <concept_id>10003120.10003145.10003151</concept_id>
       <concept_desc>Human-centered computing~Visualization systems and tools</concept_desc>
       <concept_significance>500</concept_significance>
       </concept>
   <concept>
       <concept_id>10003120.10003121.10003129</concept_id>
       <concept_desc>Human-centered computing~Interactive systems and tools</concept_desc>
       <concept_significance>500</concept_significance>
       </concept>
 </ccs2012>
\end{CCSXML}

\ccsdesc[500]{Human-centered computing~Visualization systems and tools}
\ccsdesc[500]{Human-centered computing~Interactive systems and tools}

%%
%% Keywords. The author(s) should pick words that accurately describe
%% the work being presented. Separate the keywords with commas.
\keywords{Visualization, description, caption, mixed-initiative, recommendation.}

%% A "teaser" image appears between the author and affiliation
%% information and the body of the document, and typically spans the
%% page.
\begin{teaserfigure}
  \includegraphics[width=\textwidth]{figures/pdf/teaser.pdf}
  \caption{Examples of text and chart suggestions in \pluto. (A) Coherent title and description that are auto-generated based on the chart. (B) Sentence completion is suggested based on multimodal input from the preceding text and a selection on the chart. (C) The chart is sorted and annotated to enhance coherence with the description.}
  \Description[Three examples of text and chart suggestions in Pluto.]{The first example shows a case where the system generates a title and description for a multi-series line chart. The second example shows how the system can complete a sentence in the description when users click on a mark in the chart. The third example illustrates the system's ability to suggest chart design changes, in this case, to short and annotate a chart based on text in the description.}
  \label{fig:teaser}
  \vspace{1em}
\end{teaserfigure}


% \received{9 October 2024}
% \received[revised]{Day Month 2025}
% \received[accepted]{Day Month 2025}

%%
%% This command processes the author and affiliation and title
%% information and builds the first part of the formatted document.
\maketitle


The increasing reliance on LLMs for multimodal tasks across far-reaching sectors such as healthcare, finance, and manufacturing underscores the need to assess the accuracy and reliability of the information they generate. Vision-Language Models (VLM) have achieved state-of-the-art (SoTA) performance on Visual Question-Answering (VQA) benchmarks, and these models often utilize Retrieval-Augmented Generation (RAG) to maintain factual accuracy and relevance in a dynamic information environment. However, this has led to uncertainty in the information the LLM bases its answer on, as it may choose between parametric memory and retrieved sources. When models rely on memorized information instead of dynamically retrieving information, they may inadvertently propagate outdated or incorrect information, causing serious legal and ethical risks and undermining trust and reliability in AI systems \citep{huang2023survey}.
% The ability to strike a balance between generalization and specialization in AI systems is therefore crucial for ensuring the safe, reliable use of these technologies in real-world applications.

Despite these concerns, the way that Vision-Language models (VLMs) memorize and retrieve information, particularly in complex multimodal tasks, remains under-explored. Current research often focuses on either the general capabilities of large language models (LLMs) or the specialized retrieval mechanisms in retrieval augmented generation systems (RAG) \citep{incontext_rag,chen_murag_2022,liu_universal_2023}. Particularly in the context of multimodal retrieval and multihop reasoning, few studies analyze the tradeoff between finetuning for specialized tasks and zero-shot prompting for general-purpose vision-language capabilities. A lack of consensus on how to approach this tradeoff motivates the development of measures to quantify reliance on parametric memory, as well as metrics for quantifying the potential performance impact of extending LLMs with RAG systems.

To address this gap, we investigate how multimodal QA models balance accuracy with memorization on the WebQA benchmark. We compare finetuned multimodal systems against zero-shot VLMs, analyzing how retrieval performance influences QA accuracy. In particular, we focus on cases where retrieval fails, allowing us to measure reliance on parametric memory through two proposed metrics---the \ppr (\PPR) which quantifies how much model accuracy is influenced by retrieval quality, contrasting performance in best-case versus worst-case retrieval scenarios, and the \ucr (\UCR) which measures how often correct QA responses are generated when the retriever fails, providing a proxy for memorization.

To enable this analysis, we make several methodological contributions. For the finetuned QA models, we investigate Vision-Transformer (ViT) architectures, which allow for multihop reasoning over multiple sources. To investigate the impact of retrieval performance on trained LMs, we propose a variable-input Fusion-in-Decoder (FiD) model \cite{tanaka_slidevqa_2023, nlvr2}, building upon the VoLTA architecture \citep{pramanick_volta_2023}. For the zero-shot case, we build upon previous research on In-Context Retrieval \citep{incontext_rag} by demonstrating that LLMs such as GPT-4o are capable of performing the final ranking step of the retrieval process. In doing so, we find that GPT-4o, a general-purpose LLM, achieves SoTA performance on the WebQA task, outperforming existing finetuned RAG models by a significant margin (7\% higher accuracy). 

Crucially, our results reveal that while retrieval-augmented models reduce memorization, the training paradigm plays an important role. Finetuned models exhibit higher reliance on parametric memory, whereas zero-shot RAG approaches have lower memorization scores at the cost of accuracy. This suggests that while retrieval modules may mitigate the risks associated with outdated or incorrect information, SoTA performance requires that they be coupled with specialized QA models. Our memorization measures contribute to the development of transparent and reliable AI systems, particularly in applications where the sourcing of up-to-date, factual information is critical.



% We investigate the impact of question complexity on the ability of these models to integrate multiple data sources—such as images, text, and external retrievers—and produce coherent and accurate answers. We also explore whether in-context retrieval can be a viable alternative to traditional retrieval-augmented systems, offering a more streamlined approach to multimodal QA.

% To achieve this, we first compare zero-shot prompting multimodal LLMs with finetuned multimodal systems. We evaluate both types of models on the WebQA benchmark, a dataset designed for complex question answering that requires reasoning across both image and text sources. For the finetuned models, we use a Fusion-in-Decoder (FiD) architecture, which allows for multihop reasoning over multiple sources. Additionally, we introduce the concept of In-Context Retrieval Language Modeling (RLM), where the LLM itself performs retrieval tasks without the need for external retrievers. This method builds upon existing research in in-context learning  and aims to explore the viability of LLMs retrieving relevant sources and generating accurate answers directly from their context window.

% In order to investigate source utilization in finetuned multimodal models and LLMs, three lines of inquiry are established; 
% \begin{itemize}
%     \item Study 1: retrieval vs QA performance on webQA (motivating example, does QA answer correctly even with incorrect sources?)
%     \item Study 2: performance on adversarial examples where parametric knowledge would be incorrect by design
%     \item Study 3: improving performance on adversarial examples by fine-tuning (i.e model robustness)
% \end{itemize}

% Note, there is one weakness in this plan which is tying in the work we've already done. 
% If we added something from adversarial generation to the retrieval experiment (like a combination of study 1 + 3) it would be complete. So for instance we could try fine-tuning the retriever with adversarial examples (and not just the QA model)

% \begin{figure}
%     \centering
%     \includegraphics[width=0.95\linewidth]{figures/segmentation/webqa_segment_infill.png}
%     \caption{Example of the segmentation substitution pipeline from the WebQA task.}
%     % d5c76d760dba11ecb1e81171463288e9
%     \label{fig:seg_sub_pipeline}
% \end{figure}



% Retrieval augmented generation (RAG) with zero-shot prompting and fine-tuning Large Language Models (LLMs) have become the go-to methods for tasks relying on information retrieval and text generation. In many cases the LLMs parametric memory can sufficiently generalize to answer questions without being provided with retrieval mechanisms for out-of-domain knowledge. However, LLMs often hallucinate and provide wrong information in certain scenarios. This problem is amplified even further on open-domain Question Answering (QA) tasks involving multiple modalities. Grounded text generation using retrieved sources \citep{lewis2021retrievalaugmented} has been extensively studied for text-to-text QA tasks, but its application in multimodal settings has not been studied as much.


% Multimodal reasoning and question answering have gained prominence in recent research endeavors, with an increasing emphasis on handling various forms of data, particularly text and images. In this study, we address a specific gap in the existing literature by focusing on the development of a versatile multihop model capable of accommodating varying numbers of input images.

% Our motivation for this research lies in the growing complexity of answering questions using information on the web, where the challenge of navigating the open-domain setting is further complicated by the presence of multiple modalities and sometimes requires reasoning over multiple sources. WebQA is an ideal dataset on which to compare performance of finetuned RAG systems against general purpose LLMs; it is multimodal, with correct answers requiring reasoning over image and text sources. It is multihop, requiring a complex reasoning process over multiple sources. Finally, WebQA questions from different categories can be broken down into subdomains to analyze performance over domains of varying cardinality.

% Motivated by the real-world challenges of building retrieval and question answering (QA) systems, we design and finetune a closed domain, multimodal, multihop QA model, that is capable of reasoning over a varying number of sources taken as input from an external retriever module. This research contributes to the relatively underexplored domain of multihop reasoning across various input sources and modalities. Our goal is to explore the challenges posed by these scenarios and develop strategies that enable QA models to retrieve relevant information, conduct logical or numerical reasoning across diverse modalities, and generate coherent responses in natural language. To our knowledge, this is the first application of the Fusion-in-Decoder (FiD) architecture \cite{tanaka_slidevqa_2023, nlvr2} that is shown to work with a variable number of inputs, enabling multi-hop reasoning over sources.

% In-Context Learning refers to the ability of LLMs to perform any task by simply providing examples in the input prompt \citep{dong2022survey,min2022rethinking}. Inspired by this research, we propose a method to use the LLM itself as a multimodal retriever, potentially eschewing the requirement of a distinct retrieval module, thereby allowing the design of simpler retrieval-augmented QA systems. We dub this method In-Context Retrieval Language Modeling (RLM). To the best of the authors knowledge, In-Content RLM is disparate from other retrieval augmented approaches which utilize external retrieval modules \citep{incontext_rag,chen_murag_2022,liu_universal_2023}. Despite being a natural extension of In-Context learning, In-Context RLM has not yet been studied empirically.

% To expand on our contribution of In-Context Retrieval, this stems from the well-researched in-context learning of LLMs. In-context learning is the ability of a model to perform any task given a sufficient context window \citep{dong2022survey,min2022rethinking}. Such tasks could include retrieval and ranking, but typically, the go-to solution for tasks requiring retrieval has been RAG. To the best of the authors knowledge, In-Context Retrieval is distinct from In-Context Retrieval Augmented Language Modelling (RALM), and despite being a natural extension of In-Context learning, In-Context Retrieval has not yet been shown empirically.

% Finally, we explore the tradeoff between using zero-shot prompting LLMs and the fine-tuning approach. While we find that, overall, GPT-4o obtains SoTA performance on the WebQA task, outperforming the accuracy of existing finetuned RAG approaches by 7\%, finetuned approaches still perform better on more restricted subdomains\footnote{``In-Context RLM" @ \url{https://eval.ai/web/challenges/challenge-page/1255/leaderboard/3168}}. Finally, we validate that GPT-4o is relying on retrieval abilities to solve the task; we find that GPT-4o is capable of retrieving relevant sources in the presence of distractors and furthermore, when GPT-4o fails to retrieve correct sources, it answers incorrectly 75\% of the time, meaning that it is not relying on parametric memory for this task.

% \paragraph{Contributions}
% Based on our experimentation and analysis on the WebQA benchmark, we make the following contributions:
% \begin{itemize}
%     \item Propose a new architecture for multimodal multihop QA that takes variable number of input sources inspired by the Fusion-in-Decoder method.
%     \item Comparison of general purpose LLMs vs specialized models on the WebQA benchmark.
%     \item Observation of In-Context Multimodal Retrieval abilities of GPT-4o and that it does not rely on parametric memory for multimodal QA.
%     \item Analysis of relationship between retrieval and QA task performance.
%     \item Analysis of task and query complexity on the performance of retrieval and QA tasks.
% \end{itemize}
















% Throughout this paper, we will present our methodology, experiments, and findings, emphasizing our approach to multihop reasoning over varying numbers of input images. We believe that our work contributes to a deeper understanding of multimodal reasoning and has the potential to enhance the capabilities of question-answering systems in the intricate, multimodal landscape of web-based information.
\section{Related Work}

Our work is informed by foundational paradigms in visual analytics including exploratory data analysis and exploratory search (Sec. \ref{sec:related-eda}). 
We also build on many prior methods for subgroup analysis from data mining and machine learning, the design space of which we describe in Sec. \ref{sec:related-subgroup-analysis}.

\subsection{Exploratory Data Analysis and Search}
\label{sec:related-eda}

\citeauthor{tukey_exploratory_1970} describes \textit{exploratory data analysis} (EDA) as ``looking at data to see what it seems to say''~\cite{tukey_exploratory_1970}.
EDA is therefore distinct from hypothesis testing, or confirmatory data analysis, in its emphasis on generating insight from the \textit{data} rather than prior knowledge and expectations.
Many systems for EDA are informed by interaction techniques for \textit{exploratory search}, in which people navigate through and query information resources to build understanding about some latent concept of interest~\cite{white_exploratory_2009}.
In these interactive search settings, features such as sorting, filtering, and faceted searches~\cite{yee_2003_faceted} play a key role in helping users uncover useful information.
Applied to EDA, these techniques can enable steerable recommendations of how to visualize data features~\cite{wongsuphasawat_voyager_2016,lee_2021_lux} or efficient overviews of text data~\cite{felix_texttile_2017}.
We draw inspiration from these search techniques in the design of Divisi.

A wide variety of EDA techniques have been developed for different types of data, including small-scale tabular settings~\cite{wongsuphasawat_voyager_2016,lee_2021_lux}, high-dimensional data~\cite{Liu2017}, text data~\cite{felix_texttile_2017}, and general unstructured data~\cite{Smilkov2016}.
It is often easiest to find useful insights in EDA on tabular data because the features are generally intrinsically interpretable. 
In contrast, for text or image data the ``features'' (words or pixels) may not have any meaning on their own, making it difficult to interpret what instances have in common.
As datasets grow larger, there may also be many different subtypes within the dataset, limiting the insight provided by top-level metrics and distributions.
For this reason, many prior works aim to mitigate the complexity of large, high-dimensional datasets by automatically deriving semantically meaningful features or ``concepts'' to bootstrap the analysis process~\cite{suresh_kaleidoscope_2023,kim_interpretability_2018}.
Alternatively, some systems allow the user to define their own features of interest~\cite{wu_errudite_2020,cabrera_zeno_2023}.
However, these methods require the user to already know roughly what concept they are looking for, limiting their opportunities to explore and find unexpected patterns.
Our work relies on the presence of interpretable tabular features for every instance; however, we design for use cases in which the data scientist wants to find the relevant features out of a large set of potentially-meaningful set of descriptors.
This can afford the simplicity of working with tabular data while not restricting the analysis to the user's prior hypotheses.

% \begin{enumerate}
%     \item EDA \cite{tukey_exploratory_1970}, exploratory search \cite{white_exploratory_2009,marchionini_exploratory_2006} - what are the activities involved in each?
%     \item Faceted browsing~\cite{yee_2003_faceted}, sort and filter
%     \item More modern notions of EDA: text exploration, image exploration, embedding analysis
%     \item Benefits of traditional EDA
%     \item Challenges in extending the traditional notions of EDA to modern, large-scale datasets: multiple driving phenomena or subtypes, many variables (possibly more than can be reasoned about), uninterpretable variables
% \end{enumerate}

\subsection{Tools for Subgroup Analysis}
\label{sec:related-subgroup-analysis}

Sometimes called slice discovery, cluster analysis, or rule mining, subgroup analysis is an important part of data science that can help people understand phenomena in a dataset~\cite{liu_exploratory_2020,gamberger_active_2003}, help model builders diagnose and fix issues~\cite{piorkowski_aimee_2023,zhang_drml_2022,cabrera2021deblinder,robertson_angler_2023,zhang_sliceteller_2022, jain_distilling_2022}, explain model predictions~\cite{ribeiro_anchors_2018}, or even be used in place of a model~\cite{lavrac_decision_2004}.
However, it is usually all but impossible to define clear-cut, interpretable subgroups that exactly capture the outcome of interest (e.g., model errors), creating a design space of trade-offs for how to produce useful insights.
A wide array of subgroup analysis techniques have been developed, varying across several dimensions:

\textit{Conceptualization of a subgroup.} Differences in data types, user needs, and algorithm formulations give rise to different definitions of what a subgroup is. 
At the most subjective level, subgroups can be any semantic human-readable description of instances, regardless of whether it is encoded in the data, such as ``images of people with glasses''~\cite{cabrera2021deblinder}. 
They can also be defined by numerical proximity to some conceptual entity, such as a direction or neighborhood around an instance in an embedding space~\cite{eyuboglu_domino_2022,kim_interpretability_2018,ahn_escape_2023}. 
Finally, subgroups can be defined more precisely by constructing rules for membership, such as textual patterns~\cite{wu_errudite_2020,robertson_angler_2023} or predicates on tabular features~\cite{kwon_rmexplorer_2022,hurley_interactive_2022}. 
While the latter results in the clearest subgroup definitions, it also requires crafting or mining high-quality rules.

\textit{Source of initiative.} Many subgroup discovery methods require the data scientist to define subgroups themselves, using the affordances of the various subgroup concepts described above~\cite{cabrera_zeno_2023,wu_errudite_2020,kwon_rmexplorer_2022}. 
These methods are flexible and often provide useful insights on known areas of interest, but it can be difficult to find \textit{new} subgroups without spending time perusing individual instances. 
Algorithm-initiated approaches can provide a strong initial set of subgroups to explore~\cite{chung_slice_2020,zhang_sliceteller_2022}; however, these techniques heavily focus on producing the most relevant set of subgroups in the initial query.
There is currently a lack of \textit{mixed-initiative} subgroup analysis approaches that allow the user to interactively steer the algorithm's output to produce more relevant slices.
When subgroup analysis tools do offer mixed-initiative interactions, it is typically to \textit{refine} the subgroup definitions~\cite{slyman_vlslice_2023} or to characterize and assess their validity~\cite{hurley_interactive_2022}, both of which are supported in Divisi within our broader mixed-initiative workflow.

\textit{Visualization.} Designs to visualize and compare subgroup-level data characteristics are largely dependent on the way the subgroups are conceptualized.
For example, most clustering-based tools use dimensionality reduction scatter plots, which provide a valuable overview of the dataset but are difficult to map to data features~\cite{Liu2019,slyman_vlslice_2023,xuan_attributionscanner_2024,suresh_kaleidoscope_2023,sivaraman_emblaze_2022}.
For handcrafted subgroups on tabular data, brushable histograms can serve as controls to define predicates that are then visualized in strip plots~\cite{cabrera_fairvis_2019} or domain-specific visualizations~\cite{kwon_rmexplorer_2022}.
To visualize rule-based subgroups generated by an algorithm, table representations with sparkline charts or glyphs are often preferred as they can efficiently present summary statistics over many subgroups~\cite{kahng_visual_2016,kerrigan_slicelens_2023,zhang_sliceteller_2022}.
Similarly, UpSet plots~\cite{2014_infovis_upset} provide a dense visual representation of metrics within multiple set intersections.
Divisi combines several of these elements, including the scatter plot and the subgroup table with sparklines, with novel adaptations for tasks such as assessing overlap and coverage.

\textit{Algorithmic approach.} We can divide prior algorithms for subgroup discovery into four broad classes: lattice search, frequent itemsets, classification, and clustering.
Lattice search methods, such as Slice Finder~\cite{chung_slice_2020,sagadeeva_sliceline_2021}, \textsc{Premise}~\cite{hedderich_label-descriptive_2022}, and the Nugget Browser~\cite{guo_nugget_2011}, perform combinatorial search of a space of discrete rules to find those that most satisfy the algorithm's desirability criteria.
These methods can result in easily-interpretable subgroups, but they tend to scale poorly to datasets with hundreds or thousands of features due to combinatorial explosion.
Frequent itemset-based methods, such as DivExplorer~\cite{pastor_looking_2021} and the method developed by \citeauthor{suzuki_rule_2023}~\cite{suzuki_rule_2023}, draw on efficient algorithms from data mining such as FPgrowth, then score and rank the returned subgroups.
Similarly, these methods work best with a relatively small number of possible feature combinations.
Classification-based methods can overcome some of the performance considerations of lattice search and frequent itemset approaches \cite{yuan_isea_2022,yuan_visual_2022}, but their results often require significant work to interpret.
Finally, clustering-based methods aim to group together instances by similarity in a high-dimensional space such as a learned embedding~\cite{zhang_manifold_2019,eyuboglu_domino_2022,kim_interpretability_2018}.
Though these methods can provide insight into unstructured data, they often require a trained model, sometimes one that is jointly trained with natural-language representations, limiting their applicability.
Moreover, like classification methods, the resulting clusters and concepts are not always straightforward to interpret because of their reliance on learned embeddings.
Divisi builds on this extensive space of previous algorithms, adopting a modified lattice search approach that addresses scalability issues using approximation.
While it is most directly applicable to tabular datasets as a result, we propose ways to use it in unstructured data contexts in Sec. \ref{sec:use-case}.

Because there are so many alternative techniques for subgroup analysis, each with their own specific associated data types and challenges, there is not a clear consensus of what approach should be applied to a given problem.
As a result, data scientists may not typically include subgroup analysis in the exploratory phase of their workflows.
Our work aims to make it easier to perform subgroup analyses interactively within a typical programming environment, and we assess in our study whether they might find such capabilities useful in their daily work.

% \begin{enumerate}
%     \item Why subgroup analysis? It can be used in place of classifiers ~\cite{lavrac_decision_2004}, it is useful for experts to understand phenomena in a dataset \cite{liu_exploratory_2020,gamberger_active_2003} or explain predictions~\cite{ribeiro_anchors_2018}, and it can help model builders diagnose and fix issues in their models~\cite{piorkowski_aimee_2023,zhang_drml_2022,cabrera2021deblinder,robertson_angler_2023,zhang_sliceteller_2022, jain_distilling_2022}.
%     \item There are many different subgroup analysis approaches, which vary in how the subgroup is conceptualized:
%     \begin{itemize}
%         \item Conceptualization of a subgroup: can be defined by a semantic human-readable description~\cite{cabrera2021deblinder}, defined by a pattern for text data~\cite{wu_errudite_2020,robertson_angler_2023,hedderich_label-descriptive_2022}, or a rule based on tabular features~\cite{kwon_rmexplorer_2022,hurley_interactive_2022}, based on proximity to some concept or a direction in an embedding space~\cite{suresh_kaleidoscope_nodate}, or based on clusters~\cite{Cavallo2019}
        
%     \end{itemize}
%     \item Source of initiative: often entirely human-initiated~\cite{cabrera_zeno_2023,wu_errudite_2020,kwon_rmexplorer_2022}, or algorithm-initiated. Slice discovery involves approaches to automatically generate the subgroups of interest by mining them from patterns in the data. Algorithmic approaches vary:
%     \begin{itemize}
%         \item rule mining - enumerate possible combinations of features and score them~\cite{chung_slice_2020,sagadeeva_sliceline_2021}
%         \item frequent itemsets~\cite{pastor_looking_2021,zhang_sliceteller_2022}
%         \item embedding-representation approaches~\cite{eyuboglu_domino_2022,kim_interpretability_2018}. For unstructured data we can also use cross-modal representation spaces to label clusters~\cite{slyman_vlslice_2023,eyuboglu_domino_2022}
%     \end{itemize}
%     \item Literature gap: mixed-initiative systems for subgroup discovery. Some interactive systems incorporating subgroup discovery allow users to refine the subgroup definitions~\cite{slyman_vlslice_2023,} or to investigate the characteristics of the subgroups and assess their validity~\cite{hurley_interactive_2022}. few systems have been developed that allow 
% \end{enumerate}
% rule-based explanations
\begin{figure*}[t!]
    \centering
    \includegraphics[width=\linewidth]{figures/interface.png}
    \caption{\pluto's user interface. The key components include a data panel (A), chart editor (B), chart title (C), main chart canvas (D), and a chart description (E).
    Here, the user has manually entered a description and clicked the {\small{\faIcon[regular]{lightbulb}}} \textbf{Suggest} button to get ideas on improving the chart and text for communication purposes.
    This results in the system suggesting a title and adding a highlight annotation for \annotation{\textit{Single Family}} homes, while also generating a chart design recommendation (F) and a set of description editing recommendations (G).}
    \Description[Pluto's user interface.]{From left to right, the system contains of: 1) a data pane showing the available attributes, 2) a pane to specify encodings, annotations, and filters to create a chart, 3) the chart along with text boxes for the title and description above and below it, respectively, and 4) a right panel where the system presents recommendations to edit the chart and description.}
    \label{fig:interface}
\end{figure*}

\section{Design}

The central idea of our work is exploring a unified visualization system for authoring well-integrated charts and text.
Designing such a system, however, requires considering several open questions about the type of assistance the system should provide, and when and how system suggestions should be surfaced.

In exploring the chart-and-text authoring experience, several questions arise that warrant exploration. We must first discern which chart elements, ranging from axis labels and ticks to titles, descriptions, and annotations, necessitate the most authoring support. Additionally, we would need to determine the appropriate level of system assistance—whether to generate entire descriptions, fill in partially written text, or refine user-authored drafts—and how this assistance might vary with different types of text. Given the non-mutual exclusivity of text types, such as descriptions influencing titles, we must also consider if and how the system should sequence its suggestions. The timing of these suggestions is another critical factor: should they be offered immediately following the creation of a chart or once the user has initiated the authoring process? Deciding whether these suggestions should be proactive or solicited on-demand, along with the specific user actions that should trigger them, is also a direction worth considering. Lastly, we must explore the potential for a synergistic relationship between the chart and text, i.e., how interactions with each can be leveraged to enhance the other and what mechanisms would facilitate this interplay.

\subsection{Design Goals}
\label{sec:design-goals}

With the aforementioned considerations in mind, we iteratively compiled a list of design goals to guide our system's development.
These goals were informed by prior research and systems focusing on authoring text for visualizations (e.g.,~\cite{kim2023emphasischecker,latif2021kori,liu2023autotitle,he2024leveraging,tang2023vistext,singh2024figura11y}), general principles of mixed-initiative user interfaces~\cite{horvitz1999}, as well as formative interviews with two experts on authoring text and charts for data-driven communication.

The two experts were a practitioner and a researcher who both author and critique text for data visualizations. 
Furthermore, they also regularly interact with end-users in the creation of text and charts.
\new{Both experts voluntarily participated in the interviews and were not financially compensated.
We interviewed each expert twice over a span of three weeks.
Each session lasted for 30-60 minutes.
}
\new{During the first set of interviews, we asked the experts about the key challenges users generally encounter during the authoring of text with charts. Additionally, to guide our design and identify critical features, we also presented an early version of our prototype with a basic set of functionality including generating titles and descriptions for a chart and supporting interactive highlighting of chart elements based on the text.
Based on the initial feedback, we incorporated additional types of recommendations and refined the system design before the second meeting where the experts provided feedback on the overall utility and perceived usability of the different features.
We subsequently developed the final version of \pluto~by iterating on this feedback and leveraging findings from related work on text+chart authoring systems~(e.g.,~\cite{latif2021kori,kim2023emphasischecker,sultanum2023datatales,lin2023inksight,choi2022intentable}).
}

\vspace{.5em}
\noindent\textbf{DG1. Leverage the textual narrative to guide chart design.}
In line with prior work~\cite{stokes2022striking,ottley2019curious,kim2021towards}, the experts also stressed that the text should not only convey the right levels of information but also be well-aligned with the chart for a smooth reading experience.
As text has an inherent narrative flow, we noted that the system should \new{incorporate techniques from prior work on updating chart specification based on narrative text~\cite{wang2022towards,chen2022crossdata,shen2024data} to} inspect the flow of information in the text and leverage it to augment the chart.
This augmentation could involve making data transformation changes (e.g., sorting) or adding annotations to highlight portions of the chart that are emphasized in the text.

\vspace{.5em}
\noindent\textbf{DG2. Support direct manipulation interactions with the chart for text generation.}
Visualizations make it easy to perceive trends in the data and identify points of interest.
Phrasing something visually interesting as text can be challenging, however. For instance, one of the experts noted, ``\textit{sometimes I notice something potentially interesting on the chart and want some quick text to verify what I'm seeing and get ideas for how to talk about it.}'' Given this multimodal nature of charts and text, \new{in line with prior chart-and-text authoring systems (e.g.,~\cite{chen2022crossdata,lin2023inksight})}, we noted that the system should allow leveraging direct interactions with the chart (e.g., brushing a region or mark selection) to generate corresponding text.

\vspace{.5em}
\noindent\textbf{DG3. Provide varying levels of assistance for text authoring.}
Both experts noted that users need different levels of assistance when writing text depending on their goals and experience level.
For instance, novice and intermediate users may need auto-generated text to jump-start their authoring process, whereas domain experts may benefit from fine-tuning suggestions to improve manually written text.
Combining this comment with prior work on text-chart authoring~\cite{stokes2022striking,kim2023emphasischecker}, we noted that the system should not only recommend text for chart authors to add but also recommend editing actions (e.g., reordering sentence) or flag potential factual errors in the text (e.g., incorrectly stated trends).

\vspace{.5em}
\noindent\textbf{DG4. Incorporate context-sensitive recommendations near their relevant targets to facilitate easier interpretation.}
System recommendations in the context of a unified text and chart authoring process could apply to different targets (e.g., chart, title, or description) and focus on either adding new content or editing existing content.
Interpreting this broad set of recommendations can be challenging, however.
For instance, in our early prototypes, we explored listing all recommendations in a side panel, but both experts noted that this was overwhelming and distracted them from the main content.
Iterating on the designs, we noted that for improved usability, the system recommendations should be placed close to the targets they apply to and should also be presented differently (e.g., in-place overlays vs. suggested actions) based on the type of recommendation.

\vspace{.5em}
\noindent\textbf{DG5. Recommendations should be unobtrusive during targeted authoring.}
While the recommendations are designed to help craft cohesive text and charts, there may be instances where the chart authors have clear authoring goals in mind.
In such targeted authoring scenarios, the recommendations should not interfere with the users' flow but still be available on demand if users want ideas for text content or chart design.
Authors should have full control over the final content, however, and should be able to edit/update any suggestions made by the system.
\newline

\noindent{}Note that these goals are not exhaustive or mutually exclusive, nor are they meant to be prescriptive.
For instance, we primarily focus on content suggestions and do not deeply consider operations like formatting as part of the recommendation space. Rather, \textbf{DG1}-\textbf{DG5} are only meant to be an initial set of goals to help ground our design and enable us to develop and test a viable prototype.
\section{Pluto}

Incorporating these design goals, we implemented \pluto~as a prototype system for authoring semantically-aligned text and charts.

\subsection{Example Usage Scenarios}
\label{sec:scenarios}

\begin{figure}[t!]
    \centering
    \includegraphics[width=.5\textwidth]{figures/pdf/scenario-1.pdf}
    \caption{Upon processing a description, \pluto~flags statements that require manual verification (A) and automatically \annotation{annotates} the chart based on data references in the description (B).}
    \Description[Two examples shows Pluto's suggestions for statement verification and chart annotation.]{In the first case, the system highlights a potentially incorrect statement in the user's description of the chart, allowing the user to inspect and verify whether the statement is correct. The second example illustrates how the system annotates (here, by adding a gold stroke) a portion of the chart by inspecting the data value references in the description.}
    \label{fig:scenario-1}
\end{figure}

Figure~\ref{fig:interface} shows \pluto's interface.
Users can drag and drop data fields onto visual encoding channels to create charts.
To underscore a unified experience for authoring charts and text, the system also presents an explicit title and description region just above and below the chart.
Users can also annotate the chart by creating text callouts via a context menu invoked on the chart, or by adding visual embellishments using the chart editor (e.g., borders to highlight marks).
System recommendations are either directly applied to the title, chart, or description or displayed to the right of the chart and description (Figure~\ref{fig:interface}F, G) for authors to review (\textbf{DG4}, \textbf{DG5}).

To illustrate how \pluto's interface and features collectively enable unified authoring of text and charts for data-driven communication, we now describe three vignettes\footnote{These usage scenarios are modeled on examples of how participants used \pluto~during the study described in Section~\ref{sec:study}.}.
These examples are also illustrated in the supplementary video.
% Pluto's friends from Disney cartoons: Dinah, Ronnie, Fifi

\vspace{.5em}\noindent\textbf{Augmenting generated text with built-in safeguards and chart annotations.}
Consider Dinah, an analyst at a movie production company.
Dinah is tasked with summarizing a chart showing movie earnings across genres (Figure~\ref{fig:teaser}A) to share as part of a report her company plans to publish.

Dinah is unsure about how to start her description, so she uses the {\small\faIcon{feather-alt}} \textbf{Generate} feature to bootstrap her authoring process (\textbf{DG3}).
In response, \pluto~inspects the chart and returns a description for Dinah to review (Figure~\ref{fig:teaser}A-bottom).
Dinah peruses the generated text and manually edits it for conciseness.
She then clicks {\small{\faIcon[regular]{lightbulb}}} \textbf{Suggest} to get ideas for using a combination of the description and the chart for better communication.

Analyzing the description, \pluto~makes three changes.
The system flags
% \reco{description statements with ambiguous takeaways}
description statements with ambiguous takeaways
for review using a dashed border, suggesting that Dinah manually verifies the text with the chart before sharing it with others (Figure~\ref{fig:scenario-1}A) (\textbf{DG3}).
Using \pluto's interactive highlighting feature, Dinah hovers over the statement in the description to see portions of the chart it refers to.
Reflecting on the flagged text, Dinah updates it to remove the modifier ``\textit{significant}'' and make her description more objective for the readers' interpretation.

\pluto~also suggests a title, ``\textit{Action and Animation Dominate: Gross Earnings by Genre (2010-2019)}'' based on both the narrative in the description and the underlying trends in the chart (Figure~\ref{fig:teaser}A-top).

Finally, besides suggestions for the text, \pluto~also adds an annotation to the chart highlighting the key regions the text describes (\textbf{DG1}).
In this case, detecting the emphasis on the \textit{Action} and \textit{Animation} genres and their trends between 2013 and 2017, the system adds a \annotation{gold stroke} around the corresponding lines in the chart (Figure~\ref{fig:scenario-1}B).

Satisfied with her changes based on the system suggestions, Dinah shares the title, chart, and description with her colleagues for review.

\begin{figure}[t!]
    \centering
    \includegraphics[width=\linewidth]{figures/pdf/scenario-2.pdf}
    \caption{Examples of \pluto's recommendations including an in-place sentence completion (A), \annotation{annotations} based on a chart's description (B), and a text callout generated based on marks selected on a chart (C).}
    \Description[Three examples of Pluto's recommendations.]{In the first case, the system completes the user's sentence when the user presses the tab key on a partially typed statement. In the second example, the system adds a gold stroke to marks on the chart to highlight that they are referenced in the text. In the third example, the user clicks two marks on the chart and asks Pluto to generate a text callout - this results in the system adding a text bubble with content focusing on the selected marks.}
    \label{fig:scenario-2}
\end{figure}

\vspace{.5em}
\noindent\textbf{Steering text competition through data-driven narratives and multimodal input.}
Imagine Ronnie, a financial analyst, writing a report on the monetary impact of bird strikes across the US based on the chart shown in Figure~\ref{fig:scenario-2}A.

Ronnie notices that most states, with the exception of Texas and New Jersey, have tall orange bars corresponding to costs incurred by bird strikes during the day.
Noting this observation, Ronnie types, ``\textit{Across all states, most bird strikes happen during the day.}''
Wanting to emphasize the exception of Texas and New Jersey, Ronnie types ``\textit{However, }'' and presses the \key{Tab} key to ask the system to finish the sentence.
Parsing the preceding sentence and the data trends from the chart, \pluto~generates the completion ``\textit{Texas breaks this trend by incurring the highest costs from birdstrikes at dawn.}'' (Figure~\ref{fig:scenario-2}A)
Ronnie accepts this completion but edits it to include New Jersey.

Next, to emphasize the high cost incurred by incidents in New York at night, Ronnie types ``\textit{It is also interesting that}," clicks on the teal bar showing the total cost for \textit{New York} at \textit{Night}, and again invokes a sentence completion.
Using the multimodal input from the chart selection and the existing description text (\textbf{DG2}), \pluto~suggests the text ``\textit{New York experiences its highest costs from birdstrikes at night, again deviating from the predominant trend of daytime incidents.}'' (Figure~\ref{fig:teaser}B)

Content with his description, Ronnie uses {\small{\faIcon[regular]{lightbulb}}} \textbf{Suggest} to see how he can further improve his text and the chart.
\pluto~processes the description and adds a \annotation{stroke} to visually highlight the states \textit{Texas}, \textit{New York}, and \textit{New Jersey} and the times \textit{Dawn} and \textit{Night} based on their high data values and the emphasis in the description (Figure~\ref{fig:scenario-2}B) (\textbf{DG1}).
Seeing this annotation gives Ronnie an idea to explicitly call out the striking differences in values between Texas and New York.
He selects the two tall bars within Texas and New York and uses {\small{\faIcon[regular]{feather-alt}}} \textbf{Generate Callout} to create a textual annotation directly overlaid onto the chart (Figure~\ref{fig:scenario-2}C) (\textbf{DG2}).
Manually refining the generated callout and visual embellishments (\textbf{DG5}), Ronnie saves the annotated chart and description for his report.

\vspace{.5em}
\noindent\textbf{Guiding manual authoring via system recommendations.}
Imagine Fifi, a realtor who is using the grouped bar chart shown in Figure~\ref{fig:interface}D to author an email blast on house pricing trends for her clients.
Analyzing the chart, Fifi manually writes a description with three sentences, shown in Figure~\ref{fig:interface}E.
With this initial text, she invokes the {\small{\faIcon[regular]{lightbulb}}} \textbf{Suggest} feature to see how she can improve her text and chart for communication.

Parsing the description, \pluto~detects that it lacks a summary of the chart's encodings and also detects that there is no higher-level statement encompassing a trend across multiple home types.
Translating these into recommendations, \pluto~suggests adding a brief statement about the chart's layout and a statement talking about the general impact of garage types across house types, respectively (Figure~\ref{fig:interface}G) (\textbf{DG3}).
Acknowledging these might be useful as overview statements for her readers, Fifi previews what her description would read like with the suggested text by hovering on the recommendations and subsequently accepting them.
As with the other examples, \pluto~also suggests a title (\textit{Home Type and Garage Influence on Property Prices}), but Fifi finds this too formal and manually adjusts it to make it more catchy: ``\textit{Can I afford both a car and a home?: The Influence of Garage Type on Property Prices}.''

\begin{figure}[t!]
    \centering
    \section{Problem Studied}\label{sec:def}
We first present Fixed-Radius Near Neighbor (FRNN) queries and then formalize Aggregation Queries over Nearest Neighbors (AQNNs) that build on them. We then state our problem.

\subsection{Nearest Neighbor Queries}\label{subsec:FRNN}
We build on generalized Fixed-Radius Near Neighbor (FRNN) queries \cite{FRNNSurvey}. Given a dataset \( D \), a query object \( q \), a radius \( r \), and a distance function \( dist \), a generalized FRNN query retrieves all nearest neighbors of \( q \) within radius \( r \). More formally:
\[
NN_D(q, r) = \{x \in D \mid dist(x, q) \leq r\},
\]
where \(x\) is any data point in \(D\) and \(dist(x, q)\) denotes the distance between them. We use \(|NN_D(q,r)|\) to denote the neighborhood size of \(q\). As shown in Fig. \ref{fig:framework}, given a radius \(r\) and a target patient \(q\), patients in the dotted circle are nearest neighbors, and the neighborhood size is 6.

\subsection{Aggregation Queries over Nearest Neighbors}\label{subsec:AQNN} 
Given an FRNN query object \(q\) in dataset \(D\), a radius \(r\), and an attribute \(\texttt{attr}\), an Aggregation Query over Nearest Neighbors (AQNN) is defined as:
\[ \text{agg}(NN_D(q,r)[\texttt{attr}]) \]
where agg is an aggregation function, such as $\mathtt{AVG}$, $\mathtt{SUM}$, and $\mathtt{PCT}$, and \(NN_D(q,r)[\texttt{attr}]\) denotes the bag of values of attribute \texttt{attr} of all FRNN results of \(q\) within radius \(r\). 
% \end{definition}

An AQNN expresses aggregation operations to capture key insights about the neighborhood of a query object. For example, \(\mathtt{AVG}\) can be used to reflect the average heart rate or systolic blood pressure of patients in the neighborhood, providing a measure of typical health conditions. \(\mathtt{SUM}\) is useful for assessing cumulative effects, such as the total cost of treatments in the neighborhood that instructs public policy in terms of health. Similarly, $\mathtt{PCT}$ can be used to find the proportion of patients in the neighborhood of a patient of interest, relative to the population in the dataset.
%\laks{Why is finding the total \#meds to NNs or the total treatment cost of everyone in the NN interesting?}

% \texttt{MIN} and \texttt{MAX} are not included in the aggregation functions because they only capture extreme values, which may not represent the typical characteristics of the nearest neighbors and are more sensitive to outliers. 
% \laks{AVG is also sensitive to outliers, but we still allow it. isn't the real reason we don't consider MIN/MAX because they are amenable to estimation via sampling?} We choose \texttt{PCT} instead of \texttt{COUNT} in order to provide a normalized measure that remains comparable across different neighborhood sizes. It allows for more consistent interpretation of relative popularity \cite{moore1989introduction}.


Fig. \ref{fig:framework} illustrates an example of an AQNN: ``\textit{Find the average systolic blood pressure of patients similar to an insomnia patient \(q\)}''. The aggregation function is \(\mathtt{AVG}\) and the target attribute of interest is systolic blood pressure. Exact query evaluation requires consulting physicians (or predicting embeddings by an expensive machine learning model) for all 500 patients in \(D\) and calculate \(q\)'s nearest neighbors wrt \(r\) \cite{DBLP:journals/isci/RodriguesGSBA21}. We refer to such highly accurate but computationally expensive models as \textit{oracle models}, denoted as \(O\), including deep learning models trained on domain-specific data or human expert annotations \cite{DBLP:conf/sigmod/LuCKC18}. Using oracle models is very expensive \cite{sze2017efficient, DujianPQA, DBLP:journals/pvldb/KangGBHZ20}. To address that, we seek an approximate solution by \textit{proxy models}, denoted as \(P\), that are at least one order of magnitude cheaper than oracle models. In the example, if consulting physicians for one patient incurs one cost unit, calling a cheap machine learning model instead incurs at most \(0.1\) cost unit. Once the similar patients are identified, their systolic blood pressure values are averaged and returned as  output. The use of a proxy model may reduce the accuracy of the neighborhood prediction and hence, we should judiciously call oracle and proxy models to minimize the error of aggregate results.

Note that the values of the target attribute \texttt{attr} are \textit{not} predicted but are instead known quantities.

\subsection{Problem Statement}
Given an AQNN, our goal is to return an approximate aggregate result by leveraging both oracle and proxy models while reducing error and cost.


    \caption{Conceptual schema representing the key text and chart elements in \pluto's interface.}
    \Description[Conceptual schema representing the key text and chart elements in Pluto.]{Conceptual schema representing the key text and chart elements in Pluto}
    \label{fig:schema}
\end{figure}

Besides the text suggestions, \pluto~also detects that the description emphasizes the home types with highest and lowest values, whereas the home types in the chart are sorted alphabetically.
To resolve this disparity, the system provides a chart design recommendation to sort the home types by price ranging from the highest to lowest (Figure~\ref{fig:interface}F) (\textbf{DG1}).
Fifi accepts this recommendation as it can give her clients a glanceable summary of some key takeaways in her text (Figure~\ref{fig:teaser}C).

\subsection{Conceptual Model}

To enable the aforementioned workflows and recommendations, we model the various components across the text and the chart in \pluto~as a \emph{conceptual schema} summarized in Figure~\ref{fig:schema}.
We use this schema in the subsequent sections to detail how the system tracks user input and generates recommendations.

Specifically, a \schemaPrimary{Chart} is represented by mapping \schemaPrimary{Data} onto specific visual encodings (in this case, a Vega-Lite \schemaPrimary{Specification}~\cite{satyanarayan2016vega}).
The \schemaPrimary{ActiveSelection} enumerates the data items that have been selected through direct manipulation interaction (e.g., in Figures~\ref{fig:teaser}B,~\ref{fig:scenario-2}C).
Additionally, the chart can also have one or more \schemaPrimary{Annotations}.
These can be textual comments on the chart, visual embellishments applied to individual marks (e.g., Figures~\ref{fig:interface}D,~\ref{fig:scenario-1}B), or overlays like a regression line on a scatterplot or a line marking the average value across all bars in a bar chart.

The chart's \schemaPrimary{Title} is \schemaPrimary{Text} that may include references to \schemaPrimary{DataItems} (e.g., \textit{Genre}: [\textit{Action}, \textit{Animation}] in Figure~\ref{fig:teaser}A).

\begin{figure}[t!]
    \centering    
    \includegraphics[width=\linewidth]{figures/architecture.png}
    \caption{\pluto's system architecture overview}
    \Description[Pluto's system architecture.]{Given the active chart and text from the interface, Pluto uses a heuristic parser process the information and to generate recommendations. When providing recommendations involving text suggestions, the system also uses an LLM in parallel to generate the recommended text.}
    \label{fig:architecture}
\end{figure}

The \schemaPrimary{Description} is represented as a collection of \schemaPrimary{Statements}.
Each statement maps to one of the four semantic statement types proposed by Lundgard and Satyanarayan~\cite{lundgard2021accessible}---namely, \schemaSecondary{encoding}, \schemaSecondary{perceptual-trend}, \schemaSecondary{data-fact}, \schemaSecondary{domain-specific}, or  \schemaSecondary{other} (e.g., a statement about the data source for a chart).
Additionally, similar to the title, statements in the description may also contain references to specific \schemaPrimary{DataItems}.

Note that this schema is not exhaustive (e.g., there may be additional types of annotations, statement types, or chart selections) and was primarily designed to operationalize the recommendations in \pluto.

However, we hope that the idea of formalizing not only the chart but also its associated text can inspire future work on grammars and systems for data-driven communication through a \emph{combination} of text and charts.

\subsection{System Overview}

\pluto~is implemented as a web-based application and is developed using Python, HTML/CSS, and JavaScript.
Visualizations in the tool are created using Vega-Lite~\cite{satyanarayan2016vega}.
The system currently supports three encoding channels (\texttt{x}, \texttt{y}, \texttt{color}) and three mark types (\texttt{bar}, \texttt{line}, \texttt{point}).
Collectively, this combination of encoding and mark types enables specifying several visualizations, including single- and multi-series bar charts, line charts, histograms, and scatterplots, covering a breadth of visualizations explored in prior systems~\cite{kim2023emphasischecker,latif2021kori,sultanum2023datatales,kim2024datadive,choi2022intentable,obeid2020chart,liu2020autocaption,hsu2021scicap,alam2023seechart}.

Figure~\ref{fig:architecture} depicts a high-level overview of the system architecture.
Specifically, \pluto~uses a combination of an LLM (GPT-4~\cite{achiam2023gpt}) and a heuristics-based approach for generating suggestions.
Specifically, requests like generating an entire description or a title from a chart are directly fulfilled using the LLM.
In other cases, a custom parser extracts information from the text and chart and also classifies statements in the description based on their semantic levels~\cite{lundgard2021accessible}.
This extracted information is leveraged by a heuristics-based recommendation engine to generate recommendations, including adding/editing text, adding mark annotations, and suggesting chart design changes such as sorting, among others.
In cases where the recommendations involve generated text suggestions, the recommendation engine either uses its built-in templates or interacts with the LLM to pass it the required context for the text generation.
In the subsequent sections, we detail these components and \pluto's recommendation generation process.

\subsection{Text and Chart Parsing}
\label{sec:parser}

The parser extracts a number of features from the text and the chart that are used to determine system recommendations.

\textbf{Text.}
The parser analyzes text in the description, title, and annotations to identify \schemaPrimary{DataItems}.
The system uses a combination of a lexicon- and grammar-based approach adapted from prior natural language interfaces for visualization (e.g.,~\cite{gao2015datatone,setlur2016eviza,narechania2020nl4dv}) to detect data item references.
Specifically, given an input text, the parser extracts a list of N-grams and compares the N-grams to available data fields and values, looking for both syntactic (e.g., misspellings) and semantic similarities (e.g., synonyms) employing Levenshtein distance~\cite{yujian2007normalized} and the Wu-Palmer similarity score~\cite{wu1994verb}, respectively.
The extracted items are subsequently used to support features like adding mark annotations (Figures~\ref{fig:scenario-1}B,~\ref{fig:scenario-2}B) and highlighting relevant portions of the chart while hovering over statements in the description (Figure~\ref{fig:scenario-1}A).

In addition to detecting data item references, the parser also classifies description statements into one of the five statement types.
We use a random forest classifier with BERT~\cite{sanh2019distilbert} to match a statement to one of the four semantic levels of text---\schemaSecondary{encoding}, \schemaSecondary{perceptual-trend}, \schemaSecondary{data-fact}, or \schemaSecondary{domain-specific}~\cite{lundgard2021accessible}.
If the classification probability for all four types is below 60\% \new{(an empirically set threshold)}, a statement is labeled as \schemaSecondary{other}.
The classifier is trained on a dataset of $2147$ chart description statements curated by Lundgard and Satyanarayan~\cite{lundgard2021accessible}.
Our choice for the classifier was based on comparing the results of 10-fold cross-validation between different techniques, including support vector machines~\cite{Cortes1995SupportVectorN}, random forests~\cite{breiman2001}, logistic regression~\cite{strother1967}, and na\"{i}ve Bayes~\cite{Duda1974PatternCA}.

\textbf{Chart.}
The parser also detects salient \schemaPrimary{DataItems} in the chart.
For instance, for bar charts, the parser shortlists up to three categories with the highest and lowest values.
For line charts, the system records time periods or specific timestamps with the most significant peaks and drops based on computing the smoothed z-scores, and so on.
These chart-specific heuristics to determine salient targets are derived from prior ``auto-insight'' generating visualization systems (e.g.,~\cite{cui2019datasite,wang2019datashot,srinivasan2018augmenting,demiralp2017foresight}) and research on mappings between analytic tasks and visualizations (e.g.,~\cite{amar2005low,schulz2013design,saket2018task}).
The salient items detected from the chart are subsequently used to suggest potential annotations and to generate text suggestions for verifying the description statements (e.g., Figure~\ref{fig:scenario-1}A).

\subsection{Recommendation Generation}
\label{sec:reco-generation}
\pluto~uses a combination of heuristics, text templates, and an LLM to suggest changes to the text and the chart.
The vignettes in \S\ref{sec:scenarios} illustrate the breadth of \pluto's recommendations, which can broadly be categorized into three groups: 1) \textit{full-text recommendations} to populate descriptions, titles, or text annotations, 2) \textit{description statement recommendations} to fine-tune or update an existing description, and 3) \textit{chart design recommendations} to ensure the chart is structurally aligned to its corresponding text.

\begin{figure*}[t!]
    \centering
    \includegraphics[width=.96\textwidth]{figures/pdf/recommendations-full-text.pdf}
    \caption{Overview of full-text recommendation generation. Given the context of the chart, data, and any existing text, \pluto~generates new text for the description, title, or annotations. \schemaPrimary{Input parameters} with a \schemaPrimary{?} are optional only used if available.}
    \Description[Overview of full-text recommendation generation.]{Given the context of the chart, data, and any existing text, Pluto uses an LLM to suggest the title, description, and text annotations.}
    \label{fig:full-text-recommendation}
\end{figure*}

\vspace{.5em}
\noindent{\large{\textbf{Full-text Recommendations}}
\vspace{.5em}

\noindent{}These recommendations are invoked using the {\small\faIcon{feather-alt}} \textbf{Generate} button and suggest text for the title, description, or text annotations (e.g., Figure~\ref{fig:teaser}A and Figure~\ref{fig:scenario-2}C).
All recommendations in this category are generated using the LLM, and Figure~\ref{fig:full-text-recommendation} presents an overview of the input/output for the recommendations.
The LLM prompts are provided as part of the supplementary material.

\textbf{Description.}
We use \schemaPrimary{StatementTypes} to systematically generate descriptions in \pluto.
Specifically, we provide the LLM with examples of the four statement types from Lundgard and Satyanarayan's dataset~\cite{lundgard2021accessible}.
Following the findings from Tang et al.'s qualitative analysis of the VisText chart caption dataset~\cite{tang2023vistext}, we prompt the LLM to generate a description with a constraint that the text should start with an \schemaSecondary{encoding} statement and is followed by at least one \schemaSecondary{preceptual-trend}.
This pattern follows the classic \textit{``Overview first''} mantra for visualization design~\cite{shneiderman2003eyes} and ensures the description talks about the chart and high-level takeaways before listing details of individual items and values.
An example of this constraint in play can be noticed in the generated description in Figure~\ref{fig:full-text-recommendation} where the first statement, ``\textit{This line chart...}'' describes the chart's encodings and the second highlights how ``\textit{...certain genres like Animation and Action consistently outperform others...}'' before talking about other lower-level observations from the chart.

By default, the LLM only uses the chart type and \schemaPrimary{Data} to generate a description.
However, if the chart contains an \schemaPrimary{ActiveSelection}, has \schemaPrimary{TextAnnotations}, or the user has entered a \schemaPrimary{Title}, these are also used as context for generating the description.

\textbf{Title.}
By default, the system uses a chart's \schemaPrimary{Specification} and \schemaPrimary{Data} to generate a title.
However, similar to generating descriptions, if there is additional context in the form of an \schemaPrimary{ActiveSelection}, \schemaPrimary{TextAnnotations}, or a \schemaPrimary{Description},\\ \pluto~leverages that information to generate a title that highlights the key message across the chart and previously added text.
For instance, the suggested title in Figure~\ref{fig:full-text-recommendation} contains \textit{Action} and \textit{Animation} since these are called out as focal entities in the \schemaPrimary{Description}.

\textbf{Text annotations.}
\pluto~also allows users to directly select items of interest on the chart and generate annotations based on the \schemaPrimary{ActiveSelection} (\textbf{DG2}).
An example of this type of text generation is shown in Figure~\ref{fig:full-text-recommendation}-bottom where the \schemaPrimary{TextAnnotation} is created based on the two selected bars for the states of \textit{New York} and \textit{Texas}, respectively.
The example also illustrates the effect of including previously entered text as context for the generation.
Specifically, notice that because the \schemaPrimary{Description} talks about Texas and New York deviating from the general trend, the generated annotation text also adopts that framing and phrasing (e.g., ``\textit{...contrary to the norm...}'') for consistency.
\newline

\noindent{}Since all the above recommendations leverage the chart type (inferred via the \schemaPrimary{Specification}) and \schemaPrimary{Data}, from an implementation standpoint, we pass the chart type and the data to the LLM only once in an initial context setting prompt when a chart is created.
Our choice to include the chart type as part of the context was motivated by our initial testing, during which we found that including the chart type improved the LLM's performance in terms of detecting the most relevant data patterns (e.g., trends for line charts, extremes for bar charts, correlation for scatterplots).

\vspace{.5em}
\noindent{\large\textbf{Description Statement Recommendations}}
\vspace{.5em}

\noindent{}Besides suggesting text from scratch, \pluto~also recommends adding or editing \schemaPrimary{Statements} within an existing \schemaPrimary{Description} (\textbf{DG3}).
Statement recommendations take different forms, including in-place suggestions to verify statement correctness (Figure~\ref{fig:scenario-1}A), statement addition/reordering recommendations presented to the side of an existing description (Figure~\ref{fig:interface}G), and in-place text completions (Figure~\ref{fig:teaser}B and Figure~\ref{fig:scenario-2}A).

\begin{figure*}[t!]
    \centering
    \includegraphics[width=\textwidth]{figures/pdf/recommendations-sentence-level.pdf}
    \caption{Overview of the description statement recommendations in \pluto. The system uses a combination of the chart's specification, data, the active description, and selections on the chart to recommend changes to the description.}
    \Description[Overview of description statement recommendations in Pluto.]{To generate statement addition/reordering suggestions, the system uses a combination of a text parser and a heuristic recommendation engine that inspects the statement types (e.g., encoding, perceptual-trend) to suggest content. A LLM is used when perceptual trend statements are suggested as part of the recommendations. For statement verification recommendations, the system uses a parser to identify statement items, extract data references, and subsequently checks these against the underlying data to flag a statement as needs verification or not. Lastly, for statement completion recommendations, the system passes the context of the current chart, description, and any active selections to the LLM to have it generate the statement text.}
    \label{fig:statement-recommendations}
\end{figure*}

\textbf{Statement addition and reordering.}
These recommendations are designed to ensure the description is semantically rich and has a good narrative structure.
Figure~\ref{fig:statement-recommendations} summarizes the logic used to generate statement recommendations.
Note that because the rules used to generate these recommendations are already baked into the description generation prompt described above, the statement addition/reordering recommendations typically appear only for manually entered descriptions.

To generate the recommendations, we follow the same guidelines applied to generate a description from scratch. \pluto~first checks for the presence of an \schemaSecondary{encoding} statement and at least one \schemaSecondary{perceptual-trend}.
If these are absent or placed after other statements, the system recommends adding or reordering these statements.
For instance, consider the example in Figure~\ref{fig:statement-recommendations}A.
Detecting that the input description lacks both \schemaSecondary{encoding} and \schemaSecondary{precentual-trend} statements, \pluto~suggests adding one statement of each type.
We initially also explored suggesting adding \schemaSecondary{data-facts}.
However, the formative studies and our testing revealed that these recommendations quickly became mundane and merely listed data values from the chart, leading to us subsequently disabling them.
Drawing on prior work~\cite{tang2023vistext}, we use a template-based approach to suggest \schemaSecondary{encoding} statements and invoke the LLM to suggest \schemaSecondary{perceptual-trends}.

To avoid overwriting the existing description, the recommendations are presented next to the description area instead of being directly applied to the text (Figure~\ref{fig:interface}G) (\textbf{DG5}).
Authors can preview the updated description with the suggested changes by hovering on the recommendations.
Furthermore, because these recommendations are heuristically generated, \pluto~also provides an explanation for why a recommendation was shown.
The output in Figure~\ref{fig:statement-recommendations}A shows examples of these explanations accompanying an \schemaSecondary{encoding} statement suggestion and a \schemaSecondary{preceptual-trend} statement suggestion generated when the input description only contains \schemaSecondary{data-facts}.

\textbf{Statement verification.}
Prior research has shown that both manually-written and LLM-generated descriptions can contain erroneous mentions of data trends or values~\cite{kim2023emphasischecker,tang2023vistext}. For instance, a category stated to have the highest value in the text may not actually be the category with the highest value in the chart.
Motivated by this prior work and the experts' feedback on earlier prototypes, we check for potentially incorrect data references in the text and flag them for authors to manually verify (\textbf{DG3}).

Figure~\ref{fig:statement-recommendations}B gives an overview of how \pluto~flags statements for verification.
Specifically, the system first checks if the statement contains one or more \schemaPrimary{DataItems} (typically found in \schemaSecondary{data-fact} and \schemaSecondary{perceptual-trend} statements) and the type of takeaway the statement calls out (e.g., min/max, trend, correlation).
\pluto~then uses this extracted information to validate the mentioned items and values against the underlying \schemaPrimary{Data}, flagging the statement for review if it fails to detect a match.
An example of this is shown in Figure~\ref{fig:statement-recommendations}B, where the sentence is flagged because the system is unable to confirm if the fluctuation in values is ``significant''.
Authors can click on a flagged statement to {\small\faIcon{check}} \textit{Confirm} its correctness.

\textbf{Statement text completion.}
In addition to retrospective recommendations on the description text, \pluto~also allows users to request text completion suggestions while writing their descriptions by pressing the \key{Tab} key.
To generate these completions, the system sends the current \schemaPrimary{Description} along with any \schemaPrimary{ActiveSelections} on the chart to the LLM and prompts it to complete or suggest the last sentence.
Examples of these suggestions can be seen in Figure~\ref{fig:scenario-2}A, where the system generates a text completion based on the \schemaPrimary{Description} alone and Figures~\ref{fig:teaser}B and~\ref{fig:statement-recommendations}C, where the completion is generated based on multimodal input, including the \schemaPrimary{description} text and the \schemaPrimary{ActiveSelection} of \textit{New York} on the chart.
If the text completion recommendation is invoked with an empty description, the system follows the same rules as it does when generating descriptions from scratch and starts by suggesting an \schemaSecondary{encoding} statement followed by a \schemaSecondary{perceptual-trend} before other statements.

\vspace{.5em}
\noindent{\large\textbf{Chart Design Recommendations}}
\vspace{.5em}

\begin{figure*}[t!]
    \centering
    \includegraphics[width=\textwidth]{figures/pdf/recommendations-chart-design.pdf}
    \caption{Summary of \pluto's process for recommending chart design changes based on the authored text. Given a chart and accompanying text, the system extracts data references from both the chart and the text, and compares the references to suggest potential design changes to make the chart more structurally aligned to the text.}
    \Description[Overview of the recommendation logic for suggesting chart design changes.]{To suggest design changes based on a given description, the system first inspects the chart to identify key data items (e.g., categories with highest values in bar charts). Comparing the data references in the input text to chart, the system suggests annotating the references in the chart, prioritizing items that have a higher salience in the chart. Additionally, the system also compares the order of the items in the text and the chart and if there is a difference, suggests sorting the chart to match the order of data references in the text.}
    \label{fig:chart-recommendations}
\end{figure*}

\noindent{}Following \textbf{DG1}, \pluto's recommendations are geared not only to improve the text but also to align the chart with the text, resulting in a better-combined reading experience.
Specifically, once a description is entered, the system generates two types of recommendations for updating the chart.

\textbf{Annotations.}
\pluto~recommends visual embellishments based on the description to help emphasize the key takeaways from the text in the chart following the approach summarized in Figure~\ref{fig:chart-recommendations}.
The recommended annotations are applied by default since they do not impact the chart's structure/layout, but authors are provided with controls to refine or remove the applied annotations (e.g., Figures~\ref{fig:interface}B and \ref{fig:scenario-1}B).

The system first extracts a list of potential \schemaPrimary{DataItems} from both the text and the chart and assigns a saliency score to these items based on saliency or ``interestingness'' metrics~\cite{demiralp2017foresight,wang2019datashot,srinivasan2018augmenting,lundgard2021accessible} (e.g., categories with extreme values in bar charts have higher saliency scores, time ranges in line charts with more variability have higher saliency than those with lower variability, targets mentioned in \schemaSecondary{perceptual-trend} statements are considered more salient than those referenced in \schemaSecondary{data-facts}).

Next, \pluto~checks for overlaps in \schemaPrimary{DataItems} extracted from the chart and the text to shortlist candidates for annotation based on a combined saliency score.
Checking for the combined saliency scores across the text and chart ensures that the emphasized items are important in both the underlying data and the author's interpretation of the chart conveyed via the text.
In cases where there are no overlaps between \schemaPrimary{DataItems} in the chart and the text, \pluto~defaults to adjusting the chart to match the author's description and adds an annotation for the most salient \schemaPrimary{DataItems} in the text.
An example of this is shown in Figure~\ref{fig:chart-recommendations} where the system highlights \textit{Home Type}$=$\textit{Single Family} since it has the highest average value (i.e., it is a salient data item in the chart) and is also explicitly called out in the description.

Besides annotating specific marks or regions on the chart, the system also adds overlay annotations based on references to aggregate values (e.g., adding a line to highlight the average value across categories in a bar chart or a regression line to emphasize the correlation between fields on a scatterplot).

\textbf{Sorting.}
During our initial testing and formative interviews, we noted that there was often a disparity in the order in which data items or marks appear on a chart and the order in which they are discussed in the text.
For example, consider the bar chart in Figure~\ref{fig:interface}C.
Although the bars are sorted alphabetically by home type, the description highlights takeaways starting with the home type having the highest value (i.e., \textit{Single Family} homes).
To enable a more aligned chart and text reading experience, \pluto~checks for such disparities and recommends sorting order changes for alignment.

Specifically, as summarized in Figure~\ref{fig:chart-recommendations}, the system extracts an ordered list of \schemaPrimary{DataItems} in the description and compares it to the items in the chart.
If the description starts by focusing on the highest or lowest category and the chart is not ordered to match that narrative, \pluto~suggests sorting the chart in a descending or ascending order, respectively.
An example of the sorting recommendation in action can be seen in Figure~\ref{fig:chart-recommendations} where applying the sort aligns the chart and text with both emphasizing \textit{Single Family} homes having the highest values and \textit{Condos} having the lowest values.

Unlike the annotation recommendations, however, sorting recommendations are presented next to a chart (Figure~\ref{fig:interface}F) and not applied by default to prevent an abrupt visual layout change by the system.
As shown in Figure~\ref{fig:chart-recommendations}, authors can preview the suggested order by hovering on the recommendation and clicking to apply that recommendation.
\section{Preliminary User Study}
\label{sec:study}

Using \pluto~as a design probe, we conducted a preliminary user study to assess the utility of the proposed interactive experience for authoring semantically aligned text and charts for data-driven communication.
Specifically, we focused on two goals: 1) understand if and how the system suggestions aid the joint authoring of text and charts, and 2) gather feedback on \pluto's current recommendations and features.

\subsection{Participants and Setup}

We recruited ten participants ($P1$-$P10$) through mailing lists at a data analytics software company.
\new{The recruitment call sought for individuals who use a combination of charts and text for data-driven communication.
Participation in the study was voluntary and} participants were recruited on a first-come, first-serve basis.
Seven participants rated themselves as visualization experts, and three participants had a moderate level of expertise in visualization. 
Regarding participants' professional backgrounds, six participants were solution architects, two participants worked as visualization consultants, and two were data analysts.
Of the ten participants, six participants reported that they used a combination of text and charts to communicate data once in a few weeks, two frequently used text and charts as part of their jobs, and two participants had only recently started using charts and text for communication as part of their new roles.
This mix of participant backgrounds w.r.t. chart and text authoring ensured that the study captured holistic feedback on \pluto~from varying users, including novices, moderate-level users, and experts.

Speaking about their existing authoring experience, all participants noted they manually inspected the chart and wrote a corresponding text blurb while sharing.
Five participants said they sometimes annotated a chart's screenshot with text during communication.
The two participants who frequently communicated using text and charts also noted using the dashboard authoring features in tools like Tableau.

All sessions were conducted remotely via the Zoom video conferencing software. The prototype was hosted on a local server on the experimenter's laptop. Participants were granted control over the experimenter's screen during the session, and all studies followed a think-aloud protocol. The audio, video, and on-screen actions were recorded for all sessions with permission from the participants.

\subsection{Procedure}

We initially considered an evaluation of \pluto~against an existing chart-and-text authoring tool. However, we did not find a freely available baseline that provided equivalent features to \pluto~in terms of supporting multimodal authoring, bidirectional editing of text and charts, and using the semantic structure of descriptions~\cite{lundgard2021accessible} during text recommendation.
We also considered an ablation study to compare the system's output to that from the underlying GPT-4 model.
However, we did not see value in this approach as \pluto's utility stems from the integration of multimodal interactions and recommendations, and not one standalone feature focused on text generation.

We ultimately decided on a qualitative study where all participants interact with \pluto~and perform the same set of tasks as this would allow us to observe usage patterns and assess the utility of the recommendations.
Sessions lasted between 44-57 minutes ($\mu$: 53 min., \new{$\sigma$: 4 min.}) and were organized as follows:

\vspace{.5em}
\noindent\textbf{Introduction} [$\sim$10min]:
Participants were given an overview of the study and asked to share their background information. This briefing was followed by an introduction to \pluto's interface and key features. We used a simple bar chart about US college costs as a running example for the introduction.

\vspace{.5em}
\noindent\textbf{Tasks} [$\sim$30min]:
Participants were presented with three charts and were asked to write complementary text (including title, description, and annotations) for the chart so they could share their findings with others who may be interested in the data.
To ensure the tasks were realistic and of appropriate difficulty, we consulted the two experts from the formative study (Section~\ref{sec:design-goals}) and conducted three pilot studies with participants from academic and journalism backgrounds.

The three charts included a multi-series line chart about movie earnings by genre, a stacked bar chart about costs incurred by airplane bird strikes in the US, and a grouped bar chart about US housing prices shown in Figures~\ref{fig:teaser}A, \ref{fig:scenario-2}A, and \ref{fig:interface}D, respectively.
\new{We selected these chart types based on their frequency of use in prior chart-and-text authoring systems (e.g.,~\cite{sultanum2023datatales,chen2022crossdata,choi2022intentable,kim2023emphasischecker,lin2023inksight}) as well as the general prevalence of bar and line charts across visualization platforms (e.g.,~\cite{battle2018beagle,purich2023toward}) that are commonly leveraged for data-driven communication.}
The order of charts was randomized across participants. 
We asked the participants to use the system as they saw fit (e.g., start with auto-generated text, manually write text and use the suggestions to edit, or manually compose the chart and text without using system recommendations).

\vspace{.5em}
\noindent\textbf{Debrief} [$\sim$10min]:
Sessions concluded with a semi-structured interview discussing the overall experience, support for different authoring tasks, and areas for improvement. Participants also filled out a questionnaire rating the quality and utility \pluto's recommendations and features.

\subsection{Results}

\begin{figure}[t!]
    \centering
    \includegraphics[width=\linewidth]{figures/pdf/results-wo-means.pdf}
    \caption{Participant responses to post-session questions about \pluto's recommendations.}
    \Description[Participant responses to post-session questionnaire]{Overall, participants were generally positive about Pluto's recommendations, particularly commending the use of selection to guide suggestions and the recommendations for updating the chart design based on the text. Detailed responses are as follows. Question: Overall, how likely are you to use such a system in practice for authoring text and charts for data-driven communication? Responses: 4 participants said "very likely", 4 said "likely", 1 said "moderately likely" and 1 said "unlikely".
    Question: How helpful were the text generation suggestions? Responses: 3 participants said "very helpful", 6 said "helpful", and 1 said "unhelpful". Question: How helpful was it to use selections on the chart to guide the text generation features? Responses: 3 participants said "very helpful", 5 said "helpful", and 2 said "moderately helpful". Question: How helpful were the system recommendations for editing the description? Responses: 1 participant said "very helpful", 2 said "helpful", 6 said "moderately helpful", and 1 said "unhelpful". Question: How helpful were the system recommendations for updating the chart? Responses: 4 participants said "very helpful", 4 said "helpful", and 2 said "moderately helpful".}
    \label{fig:responses}
\end{figure}

Overall, participants were receptive to \pluto's features and recommendations, noting they would use such a system in practice for data-driven communication (Figure~\ref{fig:responses}). We summarize general themes from participant feedback.


\vspace{.5em}
\noindent\textbf{Full-text recommendations are helpful for bootstrapping.}
Participants generally found the text generation recommendations useful, with nine participants rating them as `helpful' or `very helpful' (Figure~\ref{fig:responses}).
Participants noted that these suggestions were particularly useful for bootstrapping the authoring process. For instance, commenting on the generated description, $P1$ said, ``\textit{I would love to use these as a first draft. Just helps get my juices flowing.}''
However, participants' feedback on the quality of generated descriptions was mixed, with some participants (P2, P4, P7) finding the suggested text too verbose.
Both $P4$ and $P7$, for instance, noted that they would prefer the system generate a bullet list of key ``insights," allowing them to use the insights to manually craft a narrative.
$P2$ and $P4$ (both frequent users of text and charts for communication) also stated they would like more control over the generated text in terms of its verbosity and writing style (e.g., configuring the text to have a more ``casual'' versus ``formal'' tone depending on the communication context).

Feedback on the suggested titles was unanimously positive, however.
For example, commending a suggested title for the house pricing chart, $P8$ said, ``\textit{Terms like `Influence of' really makes chart feel less templated and more informative.}''
Upon seeing the title ``Action and Adventure Dominate:...", $P5$ noted, ``\textit{I am very impressed with the title. It's almost like it took all my changes and gave me a summary."}
Participants were also pleasantly surprised by the quality of text completions for individual sentences (e.g., Figure~\ref{fig:scenario-2}A) stating ``\textit{the text completion followed my lead very well}" ($P6$).


\vspace{.5em}
\noindent\textbf{Using selections to guide recommendations saves time and instills confidence.}
Overall, participants appreciated the ability to directly select items on the chart to drive text recommendations, with 8/10 participants noting this feature was `helpful' or `very helpful' (Figure~\ref{fig:responses}).
The positive feedback for the multimodal text generation feature often stemmed from its ability to assist in faster writing and to adjust the scope of the generated text.

$P8$, for instance, particularly appreciated the ability to use chart selections to drive auto-complete and annotations (e.g., Figures~\ref{fig:teaser}B and \ref{fig:scenario-2}C) and said, ``\textit{It was nice to be able to point at things and have the system give the text. It saved me a lot of time.}"
P2 and P4, who found the description generation feature too verbose, switched to using selection-based text generation, with P4 stating that ``\textit{the selection at least ensured the generated text was about something I want to talk about.}''

\vspace{.5em}
\noindent\textbf{Description editing recommendations are primarily useful for validation.}
Participants' reactions to recommendations for verifying and editing an entered description (e.g., Figures~\ref{fig:scenario-1}A and \ref{fig:interface}G) were more neutral (Figure~\ref{fig:responses}, Q4).

Seven participants explicitly noted that they appreciated that the system flagged text that needed verification.
P6, for instance, said, ``\textit{Humans are lazy. Having that extra step is going to save a lot of people a lot of embarrassment.}''
Noting that the verification suggestions made him more critical, $P3$ said, ``\textit{It's important that we think about what charts are saying...and it's referring back and making sure. It's definitely a step in the right direction.}''
The remaining participants either felt the suggestions should not appear for manually written text ($P4$) or that statements could be flagged more conservatively, reducing the work on the authors' part ($P7$, $P8$).

Participants used recommendations to add or reorder text in the description only three times across all sessions. Participants commented that these recommendations were too ``obvious'' (particularly referring to the \schemaSecondary{encoding} statement suggestions).
$P9$, for instance, ``\textit{This [\schemaSecondary{encoding} statement] suggestion is too generic and not as interesting as the others that tell me key points about the data.}''
However, all participants noted that the recommendations were appropriately placed on the side and did not impede their workflow (\textbf{DG5}).

\vspace{.5em}
\noindent\textbf{Chart annotation and design recommendations foster an integrated reading experience.}
Participants were generally very impressed by \pluto's suggestions to annotate or sort the chart based on an entered description, with 8/10 participants rating this feature as `helpful' or `very helpful.'

For example, even $P7$, who was generally critical about the other features, exclaimed, ``\textit{Loved that just loved that! It's easy to forget the chart when writing because I know what I should be focusing on, but someone else looking at the chart may not.}''
Appreciating the ability to adjust the system-suggested annotations further (\textbf{DG5}), P1 commented, ``\textit{It's great that it highlighted some elements in the chart based on my text. It made me see things from a reader's perspective and go back and make additional changes.}''

\vspace{.5em}
\noindent\textbf{Usage patterns.}
As typical with mixed-initiative interfaces, there was a constant back and forth between the participants' authoring actions and the system recommendations. We observed four high-level usage strategies around how participants started the authoring process by writing descriptions. We summarize these strategies below, as they can help inform the user experience of future systems\footnote{Note that some participants adopted different strategies across tasks, resulting in the participant count across strategies adding up to more than 10}.

\begin{tightItemize}
    \item \textit{Generate then edit.}
    The most common strategy across participants (7/10) was to start with auto-generated descriptions ({\small\faIcon{feather-alt}} \textbf{Generate}).
    Once the description text was generated, participants would either first peruse through it and make edits or directly request suggestions for improvements to the text and the chart ({\small{\faIcon[regular]{lightbulb}}} \textbf{Suggest}).
    
    \vspace{.5em}
    \item \textit{Guide text completion.}
    Four participants started writing their descriptions leveraging the in-place text competition suggestions (via the \key{Tab} key).
    Participants would typically mention a data entity (e.g., New York) or a narrative hook (e.g., ``\textit{however},'' ``\textit{but},'' ``\textit{sadly}'') in their text and have the system suggest the remaining statement that they would use as-is or edit further.
    The {\small{\faIcon[regular]{lightbulb}}} \textbf{Suggest} feature was primarily used to verify the statements and get chart annotation recommendations to complement the text.

    \vspace{.5em}
    \item \textit{Clipboard text generation.}
    On two occasions, participants used the system text suggestions to create a clipboard of ideas.
    Specifically, participants started by selecting visually salient entities on the chart and asking \pluto~to generate a series of text annotations. Subsequently, the participants went through these annotations and either edited and kept them on the chart, moved them to the description, or deleted them.
    The {\small{\faIcon[regular]{lightbulb}}} \textbf{Suggest} feature was used after this initial drafting of the description and text annotations to get further editing suggestions for the chart.

    \vspace{.5em}
    \item \textit{Manual writing with chart design recommendations.}
    Two participants (both experts at communicating data using text and charts) typically started their process by manually drafting the description and then asking the system to {\small{\faIcon[regular]{lightbulb}}} \textbf{Suggest} improvements. These participants also noted that they utilized the suggest feature to obtain chart annotations and design suggestions based on their input text (\textbf{DG1}).
\end{tightItemize}
\section{Discussion}

\begin{figure*}[t!]
    \centering    \includegraphics[width=.6\linewidth]{figures/pdf/extensibility-examples.pdf}
    \caption{Examples illustrating the extensibility afforded by the proposed conceptual schema. Here, \pluto's text generation and recommendation modules are used \emph{as-is} to author text for a geographic map and an adjacency matrix. In (A), the system generates a description based on the map and subsequently highlights six states in the chart that are emphasized in the text. In (B), \pluto~suggests a sentence completion based on multimodal input of the previously drafted description and a user selection on the chart.}
    \Description[Two examples showing how the underlying schema used to develop Pluto can be extended to other charts.]{The first example illustrates the system is able to generate a textual description based on a choropleth map. The example also shows that some states are highlighted in the map because they are referenced in the generated text. The second example shows a sentence completion recommendation generated in response to clicking on a cell in a heatmap, further illustrating the system's ability to support charts that were beyond the initial set of visualizations implemented in the tool.}
    \label{fig:extensibility-examples}
\end{figure*}

\subsection{Grammar-based Approaches for Chart \& Text Interfaces}

While we currently implement only a set of charts and recommendations in \pluto, the underlying schema (Figure~\ref{fig:schema}) is not limited to this set.
Since the framework is based on the underpinning data and conceptual elements of the chart and text, the presented schema can be generalized to other cases.
Figure~\ref{fig:extensibility-examples} highlights examples of this generalization by illustrating \pluto's~text and embellishment suggestions for a geographic map and a graph dataset represented as a network matrix.

Furthermore, the idea of breaking individual \schemaPrimary{Statements} into \schemaPrimary{Text}, \schemaPrimary{StatementType}, and \schemaPrimary{DataItems} enables interactive text-chart linking (Figure~\ref{fig:scenario-1}A) as well as targeted recommendations for description statements (Figure~\ref{fig:statement-recommendations}).
Beyond these recommendations, however, this breakdown can also be leveraged more generally to design better linting systems for writing text for charts.
For example, by inspecting the \schemaPrimary{Text} and \schemaPrimary{StatementType}s in a given description and applying findings from studies on accessible visualization descriptions~\cite{yaneva2015accessible,lundgard2021accessible,jung2021communicating,kim2023explain}, visualization systems can flag inaccessible descriptions and suggest improvements to make the descriptions more accessible.

While these are just examples, leveraging logical concepts such as those in the presented schema (Figure~\ref{fig:schema}) affords a compelling opportunity to create a unifying grammar to capture the multimodal nature of chart + text authoring interfaces.
Besides streamlining the interface and interaction design of future tools, such a grammar that is centered around abstract concepts underpinning text and charts can also facilitate effective communication with LLMs by allowing systems to only share required task-specific context with the models in a structured representation.

\subsection{Design Considerations}

Based on the study observations and feedback, we derive four design considerations for future systems exploring AI-based suggestions to assist unified text and chart authoring.

\vspace{.5em}
\noindent\textbf{Format and phrasing of text are as important as its content.}
Participants recognized the value in text generation and noted that the generated text often picked up on the salient data points and trends in the data. However, the two participants who frequently used text and charts for communication, in particular, critiqued the verbosity and tone of the generated text, indicating that the text was ``\textit{too formal or complex}'' for their consumers. Future systems should explore providing users with control over the configuration of the properties of the generated text (e.g., choosing between paragraphs and bullets, adjusting the level of verbosity, and setting the tone or writing style)~\cite{louis2014}.


\vspace{.5em}
\noindent\textbf{Leverage multimodal context during text generation.}
Participants particularly appreciated text suggestions that were based on chart selections or built upon previously authored text.
$P5$, for instance, referred to the interaction experience of having some text and then using the chart in tandem to guide the system (Figure~\ref{fig:teaser}B) as being a ``\textit{smooth and controlled authoring flow.}'' To support similar fluid and coherent authoring experiences, future systems should consider multimodal context from both the chart and previously written text when suggesting new text.

\vspace{.5em}
\noindent\textbf{Include techniques to help verify the text with the chart.}
Participants appreciated the interactive highlighting and statement verification features in \pluto~(Figure~\ref{fig:scenario-1}A), noting that the features encouraged them be more critical of the text (regardless of whether it was written by them or was system-generated).
With the growing prevalence of generated text, future systems should continue incorporating such interactive verification features to mitigate false statements and enhance synergy between the text and the chart.

\vspace{.5em}
\noindent\textbf{Adjust chart design to align with the text description.}
Unlike prior systems that focus on the unidirectional task of generating text from charts, \pluto~introduces a bidirectional flow by also recommending changes to the chart design based on the text (Figure~\ref{fig:chart-recommendations}).
We observed that participants extensively used and appreciated these suggestions, commenting that the chart design recommendations helped them gain a better reader's perspective.
To this end, future systems should continue to explore techniques to leverage the bidirectional flow of information between text and charts
and generate chart design suggestions for an integrated reading experience.

\section{Limitations and Future Work}
\noindent\textbf{Incorporate data context into generated text.}
\pluto~currently only considers the active chart and data fields while suggesting text.
However, since the recommendations are provided within a general-purpose visualization specification tool, the system can access other data fields not displayed in the active chart (Figure~\ref{fig:interface}A).
For instance, during the study, viewing the generated text for the bird strikes chart (Figure~\ref{fig:scenario-2}A), $P8$ said, ``\textit{I wish it could generate some text explaining why the costs were high based on the number of incidents.}''
Although $P8$ was able to work around this limitation by creating the second chart, inspecting the new visualization, and returning to the original chart, automatically considering other fields as part of the generated text is an open area for future work.
Such dataset-level text (as opposed to chart-level text) can also help make the authored descriptions semantically rich by including more \schemaSecondary{domain-specific} statements.

\vspace{.5em}
\noindent\textbf{Text formatting recommendations.}
The current implementation leans heavily on content generation; however, future iterations should also focus on providing sophisticated formatting suggestions that can add expressivity to the text based on the semantic levels.
For example, titles can be rendered more prominently with annotations and footnotes shown to support the chart. Complementing chart design suggestions and providing structural suggestions for the description, such as including line breaks or using bullet points instead of paragraphs, can also help improve readability.

\vspace{.5em}
\noindent\textbf{Supporting multiple views and articles.} \pluto~currently supports authoring a single chart and assumes the resulting chart and text are static.
While such an approach covers a popular scenario for data-driven communication as evidenced by prior research focusing on this setup (e.g.,~\cite{choi2022intentable,obeid2020chart,liu2020autocaption,tang2023vistext,hsu2021scicap,stokes2022striking}), people also use dashboards and interactive storytelling articles when communicating with charts~\cite{sarikaya2018we,segel2010narrative}. Investigating support for multiple views and operationalizing \pluto's recommendations in more expressive interactive data-driven article authoring tools such as Idyll Studio~\cite{conlen2021idyll} is an open topic for future work.


\vspace{.5em}
\noindent\textbf{Manage data sharing and combine heuristics with LLMs.}
As discussed in \S\ref{sec:reco-generation}, we currently pprovide the context of the chart's specification and the data to the LLM.
However, depending on the size of the data and or the usage context,  sharing data directly with the LLM may not always be feasible.
Exploring alternatives to generate text without sharing the underlying data is an important direction for future work.
For instance, one possible approach could be to use heuristics-based approaches and prior knowledge of analytic tasks (e.g.,~\cite{manelski-krulee-1965-heuristic,snowy2021,kim2023emphasischecker}) to identify key trends from a chart and then leverage the LLM for appropriately consolidating the extracted information into a coherent text narrative.


\vspace{.5em}
\noindent\textbf{Conduct longitudinal evaluation across data domains.}
The combination of two expert interviews, three pilots, and an evaluation study with ten participants helped us validate \pluto~as a proof-of-concept for the general idea of a mixed-initiative interface for authoring semantically aligned charts and text.
\new{However, this evaluation is only preliminary, and assessing the practical value of a tool like \pluto~deems more longitudinal studies spanning different data domains and individuals with varying levels of expertise~\cite{lam2012,shneiderman2006strategies}.}

\vspace{.5em}
\noindent\new{\textbf{Incorporating safeguards for machine failures.}
With the current implementation of \pluto, we utilize a pretrained LLM to generate text and a heuristic parser to analyze the generated text for additional suggestions.
To make users aware of potential errors in either of these steps, we provide interactive visual previews that can be viewed before accepting system suggestions (e.g., Figure~\ref{fig:scenario-1}A and Figure~\ref{fig:chart-recommendations}).
While \pluto~serves as a proof-of-concept, developing a more generalized system requires a deeper investigation of the types of errors that might be generated both at the LLM and the parser level.
Categorizing and understanding the distribution of such errors can ultimately help develop better recommendation modules, but also design interface and interaction strategies to mitigate system errors.
}
\section{Conclusion}\label{sec:conclusion}
This work introduces a novel approach to TOT query elicitation, leveraging LLMs and human participants to move beyond the limitations of CQA-based datasets. Through system rank correlation and linguistic similarity validation, we demonstrate that LLM- and human-elicited queries can effectively support the simulated evaluation of TOT retrieval systems. Our findings highlight the potential for expanding TOT retrieval research into underrepresented domains while ensuring scalability and reproducibility. The released datasets and source code provide a foundation for future research, enabling further advancements in TOT retrieval evaluation and system development.
\section*{Acknowledgments}
{\textcopyright}2025 All rights reserved. The research described in this paper was carried out at the Jet Propulsion Laboratory, California Institute of Technology, under a contract with the National Aeronautics and Space Administration (80NM0018D0004).

%%
%% The next two lines define the bibliography style to be used, and
%% the bibliography file.
\bibliographystyle{ACM-Reference-Format}
\bibliography{references}

\end{document}
\endinput