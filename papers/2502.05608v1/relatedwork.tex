\section{Background and Related Works}
Managing networks has evolved into Zero Touch Networks, an automated process by introducing a management domain to help with the life-cycle of VNFs, NFV orchestration, and hardware resource management. This allows for automatic changes to be made to networks which increases the reliability and quality of the provided service. The utilization of ZTNs also introduced automation using AI which brings advantages as well as challenges. Some of the challenges raise concerns with the decisions being made by the system, the data that is used to train the model, and the protection of the user from rouge systems. Some governance has been introduced but not fully introduced and integrated with many countries leaving these AI systems to operate without any standards or governance. 



\subsection{Zero Touch Networks}

Virtualization of core network components has helped shift more to software and less hardware making them easier to deploy than hardware-dependent legacy models. It is this new modern networking standard \cite{3gpp_2022} that led to the development of a more advanced software-centric called Zero Touch Networks. The framework presented by the European Telecommunications Standards Institute (ETSI), shows the new management components for the network or individual service. The framework introduced a management domain thatX communicates with the service domain via a cross-domain integration fabric \cite{etsi_2019}. The management domain features a closed loop for management functions that deal with the physical and virtual aspects of the service.

Many current works focus on incorporating AI into the management functions of the ZTN to help manage the VNF, orchestrate NVF, and manage the physical resources of the network. In \cite{ns_energy} the authors use AI to minimize energy consumption for network slices and VNF installation, this work considers overall network cost with respect to computing resources as well as network traffic. 

Another advantage of ZTN is self-healing, where the network can automatically detect poor performance and make changes accordingly. The authors of \cite{CDR_Analysis}, create an AI-driven approach to classify and identify faults in cells of a network through the analysis of Call Detail Records. The solution creates a report with figures and accurate root-cause analysis which reduces an expert's job. 


The separate domains of the ZTN architecture allow for new and advanced AI management tools to analyze and make changes to the network. This allows for a higher QoE to be achieved for the user without the need to add more computational resources to the core network. 

\subsection {AI governance}

ZTN brings a better QoE for the end user through the use of AI automation tools, making management more automatic with less human intervention. This brings forward challenges and concerns as important tasks that used to be designated for someone with certain expertise and knowledge are now being delegated to AI. Many concerns come into question regarding the validity of the decisions an AI can make.

To address the concerns presented, understanding how AI and its subsets of ML and RL are modeled is essential. For ML, the learning that goes on is done primarily by algorithms comparing features within a data set which can be both labeled and unlabeled data. The data is the foundation for an ML model's success, however, things such as incomplete data, biased data, and incorrect algorithms can lead to bad models. For RL, the driving force for training the model is an agent operating in a mathematically modeled environment. A poorly modeled environment can result in a model making poor decisions. These bad models can account for a model making unfair decisions.


Several governing bodies have tried to address this by creating some kind of regulatory standard that ensures that the models being used will be fair, by listing a set of regulations to ensure the creation of fair models \cite{AIGOV-EU, AIGOV-CAD, AIGOV-USA}.



\subsection{Responsibility Gap}


Currently, there exists a gap in responsibility with accountability and ethics with AI and its subsets, making it difficult to ensure fair use of them. The field of robotics has faced similar issues, especially concerns with human-robot interaction (HRI). To ensure the safety of the users, the ethics of the robot's programmed decision-making was called into question. Above all the robots must always be programmed with the intent of never harming a human\cite{robotics-1}, and making sure everyone receives the same treatment \cite{robotics-2}. Both of these core concepts of ethics must also be present within AI.