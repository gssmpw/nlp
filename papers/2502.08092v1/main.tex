%%
%% This is file `sample-sigconf.tex',
%% generated with the docstrip utility.
%%
%% The original source files were:
%%
%% samples.dtx  (with options: `all,proceedings,bibtex,sigconf')
%% 
%% IMPORTANT NOTICE:
%% 
%% For the copyright see the source file.
%% 
%% Any modified versions of this file must be renamed
%% with new filenames distinct from sample-sigconf.tex.
%% 
%% For distribution of the original source see the terms
%% for copying and modification in the file samples.dtx.
%% 
%% This generated file may be distributed as long as the
%% original source files, as listed above, are part of the
%% same distribution. (The sources need not necessarily be
%% in the same archive or directory.)
%%
%%
%% Commands for TeXCount
%TC:macro \cite [option:text,text]
%TC:macro \citep [option:text,text]
%TC:macro \citet [option:text,text]
%TC:envir table 0 1
%TC:envir table* 0 1
%TC:envir tabular [ignore] word
%TC:envir displaymath 0 word
%TC:envir math 0 word
%TC:envir comment 0 0
%%
%%
%% The first command in your LaTeX source must be the \documentclass
%% command.
%%
%% For submission and review of your manuscript please change the
%% command to \documentclass[manuscript, screen, review]{acmart}.
%%
%% When submitting camera ready or to TAPS, please change the command
%% to \documentclass[sigconf]{acmart} or whichever template is required
%% for your publication.
%%
%%
\documentclass[sigconf]{acmart}

\usepackage{amsthm}
\usepackage{amsmath}
\usepackage{algorithm}
\usepackage{enumitem}
\usepackage{picinpar}
\usepackage{lineno}
\usepackage{graphicx}
% \usepackage{caption}
% \usepackage{subcaption}
% \usepackage{multirow}

\usepackage{algorithmicx}
\usepackage[noend]{algpseudocode}
\usepackage{subfigure}
\usepackage{multirow}
\usepackage{color}
\usepackage{balance}
\usepackage{enumitem}
\usepackage{hhline}
\usepackage[normalem]{ulem}
\usepackage{booktabs}
\usepackage{wrapfig}
\usepackage{cancel}
\usepackage{hyperref}
\usepackage{makecell}


\newtheorem{theorem}{Theorem}
\newtheorem{lemma}{Lemma}
\newtheorem{corollary}{Corollary}
\newtheorem{assumption}{Assumption}
\newtheorem{definition}{Definition}

\newcommand{\bL}{\ensuremath{\mathcal{L}}}
\newcommand{\bU}{\ensuremath{\mathcal{U}}}
\newcommand{\bS}{\ensuremath{\mathcal{S}}}
\newcommand{\bH}{\ensuremath{\mathcal{H}}}
\newcommand{\bW}{\ensuremath{\mathcal{W}}}
\newcommand{\bV}{\ensuremath{\mathcal{V}}}
\newcommand{\bG}{\ensuremath{\mathcal{G}}}
\newcommand{\bN}{\ensuremath{\mathcal{N}}}
\newcommand{\bE}{\ensuremath{\mathcal{E}}}
\newcommand{\bT}{\ensuremath{\mathcal{T}}}
\newcommand{\bY}{\ensuremath{\mathcal{Y}}}
\newcommand{\bM}{\ensuremath{\mathcal{M}}}
\newcommand{\bC}{\ensuremath{\mathcal{C}}}
\newcommand{\bP}{\ensuremath{\mathcal{P}}}
\newcommand{\bD}{\ensuremath{\mathcal{D}}}

\newcommand{\cls}{\text{cls}}
\newcommand{\str}{\text{str}}
\newcommand{\con}{\text{con}}

\newcommand{\fang}[1]{\textcolor{red}{[FANG: #1]}}
\newcommand{\kien}[1]{\textcolor{purple}{[Kien: #1]}}

\renewcommand{\vec}[1]{\ensuremath{\mathbf{#1}}}

\newcommand{\stitle}[1]{\vspace{1mm} \noindent {\bf #1}}


\newcommand{\eg}{{\it e.g.}}
\newcommand{\etal}{{\it et al.}}
\newcommand{\ie}{{\it i.e.}}
\newcommand{\etc}{{\it etc.}}
\newcommand{\wrt}{w.r.t. }
\newcommand{\vs}{{\it vs.}}
\newcommand{\tabincell}[2]{\begin{tabular}{@{}#1@{}}#2\end{tabular}}

%\newcommand{\method}[1]{\scalebox{0.9}{\textsf{#1}}}
\newcommand{\method}[1]{\textsc{#1}}
% \newcommand{\method}{}
\newcommand{\model}{\method{GCoT}{}}
\newcommand{\concept}{\method{homophily sensitivity}{}}
\newcommand{\uconcept}{\method{homophily independence}{}}
\newcommand{\modelS}[1]{\method{SOTA-{#1}}}

\newcommand{\eat}[1]{}

\renewcommand{\algorithmicrequire}{\textbf{Input:}}
\renewcommand{\algorithmicensure}{\textbf{Output:}}

\newcommand\xingtong[1]{\textcolor{green}{#1}}
\newcommand\xingtongC[1]{\begin{CJK*}{UTF8}{gbsn}\textcolor{green}{[\textbf{Xingtong:} #1]}\end{CJK*}}
\newcommand{\stkout}[1]{\ifmmode\text{\sout{\ensuremath{#1}}}\else\sout{#1}\fi}

%%
%% \BibTeX command to typeset BibTeX logo in the docs
\AtBeginDocument{%
  \providecommand\BibTeX{{%
    Bib\TeX}}}

%% Rights management information.  This information is sent to you
%% when you complete the rights form.  These commands have SAMPLE
%% values in them; it is your responsibility as an author to replace
%% the commands and values with those provided to you when you
%% complete the rights form.
% \copyrightyear{2025}
% \acmYear{2025}
% \setcopyright{acmlicensed}
% \acmConference[KDD '25] {Proceedings of the 31st ACM SIGKDD Conference on Knowledge Discovery and Data Mining V.1}{August 3--7, 2025}{Toronto, ON, Canada.}
% \acmBooktitle{Proceedings of the 31st ACM SIGKDD Conference on Knowledge Discovery and Data Mining V.1 (KDD '25), August 3--7, 2025, Toronto, ON, Canada}
% \acmISBN{979-8-4007-1245-6/25/08}
% \acmDOI{10.1145/XXXXXX.XXXXXX}

\renewcommand{\shortauthors}{Trovato et al.}


%%
%% Submission ID.
%% Use this when submitting an article to a sponsored event. You'll
%% receive a unique submission ID from the organizers
%% of the event, and this ID should be used as the parameter to this command.
%%\acmSubmissionID{123-A56-BU3}

%%
%% For managing citations, it is recommended to use bibliography
%% files in BibTeX format.
%%
%% You can then either use BibTeX with the ACM-Reference-Format style,
%% or BibLaTeX with the acmnumeric or acmauthoryear sytles, that include
%% support for advanced citation of software artefact from the
%% biblatex-software package, also separately available on CTAN.
%%
%% Look at the sample-*-biblatex.tex files for templates showcasing
%% the biblatex styles.
%%

%%
%% The majority of ACM publications use numbered citations and
%% references.  The command \citestyle{authoryear} switches to the
%% "author year" style.
%%
%% If you are preparing content for an event
%% sponsored by ACM SIGGRAPH, you must use the "author year" style of
%% citations and references.
%% Uncommenting
%% the next command will enable that style.
%%\citestyle{acmauthoryear}


%%
%% end of the preamble, start of the body of the document source.
\begin{document}

%%
%% The "title" command has an optional parameter,
%% allowing the author to define a "short title" to be used in page headers.
%\title{Beyond Homophily and Heterophily: Non-Homophilic Graph Pre-Training and Prompt Learning}
\title{GCoT: Chain-of-Thought Prompt Learning for Graphs}

%%
%% The "author" command and its associated commands are used to define
%% the authors and their affiliations.
%% Of note is the shared affiliation of the first two authors, and the
%% "authornote" and "authornotemark" commands
%% used to denote shared contribution to the research.
\author{Xingtong Yu}
\affiliation{%
 \institution{Singapore Management University}
  \country{Singapore}}
\email{xingtongyu@smu.edu.sg}

\author{Chang Zhou}
\affiliation{%
 \institution{University of Science and Technology of China}
  \country{China}}
\email{zhouchang21sy@mail.ustc.edu.cn}

\author{Zhongwei Kuai}
\affiliation{%
 \institution{University of Science and Technology of China}
  \country{China}}
\email{asagiri@mail.ustc.edu.cn
}

\author{Xinming Zhang$^{\dagger}$}
\affiliation{%
 \institution{University of Science and Technology of China}
  \country{China}}
\email{xinming@ustc.edu.cn}


\author{Yuan Fang$^{\dagger}$}
\affiliation{%
  \institution{Singapore Management University}
  \country{Singapore}}
\email{yfang@smu.edu.sg}


\thanks{
    $^{\dagger}$Corresponding authors.
}

%%
%% By default, the full list of authors will be used in the page
%% headers. Often, this list is too long, and will overlap
%% other information printed in the page headers. This command allows
%% the author to define a more concise list
%% of authors' names for this purpose.


%%
%% The abstract is a short summary of the work to be presented in the
%% article.
\begin{abstract}
Chain-of-thought (CoT) prompting has achieved remarkable success in natural language processing (NLP). However, its vast potential remains largely unexplored for graphs. This raises an interesting question: How can we design CoT prompting for graphs to guide graph models to learn step by step?
On one hand, unlike natural languages, graphs are non-linear and characterized by complex topological structures. On the other hand, many graphs lack textual data, making it difficult to formulate language-based CoT prompting. 
%Therefore we cannot directly adopt the CoT prompting methods used in the language domain. 
In this work, we propose the first CoT prompt learning framework for text-free graphs, \model. Specifically, we decompose the adaptation process for each downstream task into a series of inference steps, with each step consisting of prompt-based inference, ``thought'' generation, and thought-conditioned prompt learning. While the steps mimic CoT prompting in NLP, the exact mechanism differs significantly. Specifically, at each step, an input graph, along with a prompt, is first fed into a pre-trained graph encoder for prompt-based inference. We then aggregate the hidden layers of the encoder to construct a ``thought'', which captures the working state of each node in the current step.  Conditioned on this thought, we learn a prompt specific to each node based on the current state. These prompts are fed into the next inference step, repeating the cycle.
%Note that, in the initial step, we employ learnable initial prompts for all nodes, while subsequent steps will utilize the thought-conditioned prompts generated by the preceding step. 这个有点太细节,introduction/method里说就行了。
%which produces the corresponding node embeddings. We generate a thought for each inference step by aggregating the hidden embeddings from each layer of the pre-trained graph encoder. These thoughts are then leveraged to generate a series of prompts that guide the model to the subsequent inference step in node-specific learning pattern. 
To evaluate and analyze the effectiveness of \model, we conduct comprehensive experiments on eight public datasets, which demonstrate the advantage of our approach. 
\end{abstract}


%%
%% The code below is generated by the tool at http://dl.acm.org/ccs.cfm.
%% Please copy and paste the code instead of the example below.
%%
\begin{CCSXML}
<ccs2012>
   <concept>
       <concept_id>10002951.10003260.10003277</concept_id>
       <concept_desc>Information systems~Web mining</concept_desc>
       <concept_significance>500</concept_significance>
       </concept>
   <concept>
       <concept_id>10002951.10003227.10003351</concept_id>
       <concept_desc>Information systems~Data mining</concept_desc>
       <concept_significance>500</concept_significance>
       </concept>
   <concept>
       <concept_id>10010147.10010257.10010293.10010319</concept_id>
       <concept_desc>Computing methodologies~Learning latent representations</concept_desc>
       <concept_significance>500</concept_significance>
       </concept>
 </ccs2012>
\end{CCSXML}

\ccsdesc[500]{Information systems~Web mining}
\ccsdesc[500]{Information systems~Data mining}
\ccsdesc[500]{Computing methodologies~Learning latent representations}

%%
%% Keywords. The author(s) should pick words that accurately describe
%% the work being presented. Separate the keywords with commas.
\keywords{Graph mining, chain-of-thought, prompt learning, pre-training, few-shot learning.}

%% A "teaser" image appears between the author and affiliation
%% information and the body of the document, and typically spans the
%% page.
% \begin{teaserfigure}
%   \includegraphics[width=\textwidth]{sampleteaser}
%   \caption{Seattle Mariners at Spring Training, 2010.}
%   \Description{Enjoying the baseball game from the third-base
%   seats. Ichiro Suzuki preparing to bat.}
%   \label{fig:teaser}
% \end{teaserfigure}

% \received{20 February 2007}
% \received[revised]{12 March 2009}
% \received[accepted]{5 June 2009}

%%
%% This command processes the author and affiliation and title
%% information and builds the first part of the formatted document.
\maketitle

%!TEX root = main.tex


\section{Introduction}

Games play a central role in AI research. In the early $20^{th}$ century, \cite{zermelo1913} showed that perfect information games in extensive form can be solved by a bottom-up traversal of the game tree. Despite the fact that this does not readily provide efficient ways to solve large games such as Chess or Go in practice, this has indeed 
laid the foundation for the dramatic progress in the field of perfect information games, with computer programs being able to challenge human experts. Solving games becomes more intricate when the players (agents) have incomplete information about the state of the game -- Poker for instance, where a player does not know the cards of the others. One of the remarkable imperfect information games where computer programs have been able to defeat professional human players is Texas Hold'em Poker~\cite{libratus-poker,deepstack,pluribus-poker}. A main technique used in these algorithms is the abstraction of large games into smaller \emph{imperfect recall} games. 

Perfect recall is the ability of a player to remember her own actions. Poker is an imperfect information game played by several players. However, ideally one would assume that the players have a perfect recall of their actions. An imperfect recall player does not remember the sequence of her own actions. Imperfect recall allows for a structured mechanism to forget the information history and as \cite{ijcai2024p332} argues, it is particularly suited for AI agents.

From a modeling perspective, imperfect recall has been used to describe teams of agents, where each team can be represented as a single agent with imperfect recall~\cite{VONSTENGEL1997309,DBLP:conf/aaai/Celli018} or to describe agents modeling multiple nodes which do not share information between each other due to privacy reasons~\cite{DBLP:conf/aaai/Conitzer19}. Moreover~\cite{LAMBERT2019164} argues that imperfect recall is a model of bounded rationality. Given the limited memory of players, it is not realistic to assume that the players remember all their actions. We refer the reader to~\cite{ijcai2024p332} for an excellent introduction to different uses of imperfect recall.  
From a practical perspective, the most prominent use of imperfect recall is in abstracting games~\cite{PracticalUseImperfect,DBLP:conf/aaai/GanzfriedS14,DBLP:conf/atal/BrownGS15, CERMAK2020103248}. The state space generated by usual games is typically very large and abstractions are crucial for solving such games. Abstractions that preserve perfect recall force a player to distinguish the current information gained, in all later rounds, even if it is not relevant. Abstractions using players with imperfect recall have been shown to outperform those using players with perfect recall~\cite{ DBLP:conf/sara/WaughZJKSB09, johanson2013evaluating, DBLP:conf/atal/BrownGS15,DBLP:conf/ijcai/CermakBL17}.



From a complexity perspective, imperfect recall games are known to be $\NP$-hard

~\cite{KollerMegiddo::1992,Cermak::2018} even when there is a single player, whereas perfect recall games can be solved in polynomial-time~\cite{KollerMegiddo::1992,vonStengel::1996}. Recent studies have aligned the complexity of different solution concepts for imperfect recall games to the modern complexity classes~\cite{GPS20,tewolde-et-al:2023,ijcai2024p332}. The hardness of imperfect recall games has motivated the search for subclasses which are polynomial-time solvable~\cite{kline2002minimum,kaneko1995behavior}, or where algorithms similar to the perfect recall case can be applied~\cite{DBLP:conf/icml/LanctotGBB12,DBLP:conf/sigecom/KroerS16}. The class of \emph{A-loss recall}~\cite{kline2002minimum,kaneko1995behavior} is a special kind of imperfect recall, where the loss of information can be traced back to a player forgetting her own action at a point in the past -- the player remembers \emph{where} it was played, but forgets \emph{what} was played. We consider A-loss recall games to be \emph{simple} since there are polynomial-time algorithms for solving them. To the best of our knowledge, A-loss recall games are the biggest known class of imperfect recall with a polynomial-time solution. This has led to research towards finding A-loss recall abstractions~\cite{Cermak::2018}. 

\emph{Contributions.} Our broad goal in this work is to find efficient ways to solve imperfect recall games in extensive-form. We do so by simplifying them into A-loss recall games. We focus on games where the players are not absent-minded: a player is absent-minded if she even forgets whether a decision point was previously seen or not. Here are our major contributions.
\begin{enumerate}\item We first identify a class of one-player games where the player's information structure is more complex than A-loss recall, but shuffling the order of actions results in an equivalent A-loss recall game. This leads to a new $\mathsf{PTIME}$ solvable class of imperfect recall games, that extends A-loss recall (\cref{thm:1p-shuffle-ptime}, \cref{cor:effic-solv-class}, \cref{cor:2-effic-solv-class}). Furthermore, these classes themselves can be tested in $\mathsf{PTIME}$. 

\item We show that every game with \emph{non-absentminded} players can be transformed into an equivalent A-loss recall game (\cref{thm:existence-alr-span}). We present an algorithm to generate an equivalent A-loss recall game with the smallest size.
\end{enumerate}
















 


The caveat in the second result above is that the resulting A-loss recall game could be exponentially bigger. This is expected, since solving imperfect recall games is $\NP$-hard, whereas A-loss recall games can be solved in polynomial-time. The result however shows that in order to solve imperfect recall games, one could either use a worst-case exponential-time algorithm on the original game, or apply our transformation to a worst-case exponential-sized game and run a polynomial-time algorithm on it. From a conceptual point of view, our result shows that as long as there is no absentmindedness, imperfect recall can be transformed into one where the information loss can be attributed to forgetting own actions at a past point.


\emph{Organization of the document.} Section~\ref{sec:an-example} introduces a modification of the popular matching pennies game that will be used as a running example to illustrate our results. Section~\ref{sec:background} recalls necessary preliminaries on extensive-form games. Section~\ref{sec:shuffled-loss-recall} presents the new polynomial-time class of shuffled A-loss recall. Section~\ref{sec:span} generalizes the idea of shuffling to incorporate a ``linear combination'' of action sequences, and presents the second result mentioned above. Section~\ref{sec:two-player} extends the results to the two-player setting. 

  






  



\section{An example}
\label{sec:an-example}



Let us start with a one-player game called the \emph{single team matching-unmatching pennies game}, which will be used as a running example. A team of players with the same goal can be interpreted as a single player. 
In this case, the team consists of two players Alice and Bob, each possessing a coin with two sides, Head (H) and Tail (T) and each of them must choose a side for their respective coins independently. 
The game unfolds in the following manner : a fair $n$-faced die with outcomes from $\{0, \dots, n-1 \}$ is rolled; then Alice chooses a side from $\{H,T\}$, followed by Bob choosing from $\{H,T\}$. Winning or losing depends on the parity of the die outcome. If the outcome of the die is even, then they win if and only if they match their sides. If the outcome is odd, they win if and only if their sides do not match. We consider three variants depending on what Alice and Bob can observe, and model it in extensive form in \cref{fig:match-penny-3-die} for $n=3$. An informal description of the figures follows after this paragraph.
\begin{description}
  \item[I.] Both Alice and Bob observe nothing (\cref{fig:match-penny-3-die-a}).
  \item[II.] Alice can't distinguish between die outcome $2i$ and $2i+1$ for $i \geq 0$,  but Bob observes nothing (\cref{fig:match-penny-3-die-b}).
   \item[III.] Alice can't distinguish between die outcome $2i$ and $2i+1$ for $i \geq 0$, Bob only observes coin of Alice but not outcome of die (\cref{fig:match-penny-3-die-c}). 
 \end{description} 
Alice and Bob want to maximize their \emph{expected payoff}. We will see their possible strategies in Section~\ref{sec:background}.  
Later, we will see that game \textbf{I} falls under the simple class of A-loss recall. 
In Section~\ref{sec:shuffled-loss-recall} and \cref{sec:span} we will see how to simplify games \textbf{II} and \textbf{III} respectively. 
%!TEX root = ../main.tex

\begin{figure}

\begin{subfigure}{0.45\columnwidth}
\centering
\tikzset{
triangle/.style = {regular polygon,regular polygon sides=3,draw,inner sep = 2},
circ/.style = {circle,fill=cyan!10,draw,inner sep = 3},
term/.style = {circle,draw,inner sep = 1.5,fill=black},
sq/.style = {rectangle,fill=gray!20, draw, inner sep = 4}
}
\begin{tikzpicture}[scale=0.8]

\tikzstyle{level 1}=[level distance=18mm,sibling distance=24mm]
\tikzstyle{level 2}=[level distance=11mm,sibling distance=12mm]
\tikzstyle{level 3}=[level distance=11mm,sibling distance=5mm]

\begin{scope}[->, >=stealth]
 \node(0)[triangle]{}
    child{  
    node(00)[circ,draw=black]{}
        child{
        node(000)[circ]{}
            child{
            node(0000)[term,label=below:{\scriptsize $1$}]{} 
            edge from parent node[left,pos = 0.3,inner sep=1.5]{\scriptsize H}
            }
            child{
            node(0001)[term,label=below:{\scriptsize $0$}]{}
            edge from parent node[right,pos = 0.3,inner sep=1.5]{\scriptsize T}                
            }
        edge from parent node[left,pos = 0.2]{{\scriptsize H}}                
        }
        child{
        node(001)[circ]{}
            child{
            node(0010)[term,label=below:{\scriptsize $0$}]{} 
            edge from parent node[left,pos = 0.3,inner sep=1.5]{\scriptsize H}
            }
            child{
            node(0011)[term,label=below:{\scriptsize $1$}]{}
            edge from parent node[right,pos = 0.3,inner sep=1.5]{\scriptsize T}                
            }
        edge from parent node[right,pos = 0.2]{{\scriptsize T}}
        }
    edge from parent node[above,pos = 0.8]{0}
    edge from parent node[above,pos = 0.4]{\scriptsize $\frac{1}{3}$} 
    }
    child{
    node(01)[circ]{}
        child{
        node(010)[circ]{}
            child{
            node(0100)[term,label=below:{\scriptsize $0$}]{}
            edge from parent node[left,pos = 0.3,inner sep=1.5]{\scriptsize H}
            }
            child{
            node(0101)[term,label=below:{\scriptsize $1$}]{}
            edge from parent node[right,pos = 0.3,inner sep=1.5]{\scriptsize T}                
            }
        edge from parent node[left,pos = 0.2]{\scriptsize H}
        }
        child{
        node(011)[circ]{}
            child{
            node(0110)[term,label=below:{\scriptsize $1$}]{} 
            edge from parent node[left,pos = 0.3,inner sep=1.5]{\scriptsize H}
            }
            child{
            node(0111)[term,label=below:{\scriptsize $0$}]{}
            edge from parent node[right,pos = 0.3,inner sep=1.5]{\scriptsize T}                
            } 
        edge from parent node[right,pos = 0.2]{\scriptsize T}
        }
    edge from parent node[right,pos = 0.8]{1}
    edge from parent node[left,pos = 0.4]{\scriptsize $\frac{1}{3}$}
    }
    child{
    node(02)[circ]{}
        child{
        node(020)[circ]{}
            child{
            node(0200)[term,label=below:{\scriptsize $1$}]{}
            edge from parent node[left,pos = 0.3,inner sep=1.5]{\scriptsize H}
            }
            child{
            node(0201)[term,label=below:{\scriptsize $0$}]{}
            edge from parent node[right,pos = 0.3,inner sep=1.5]{\scriptsize T}                
            }
        edge from parent node[left,pos = 0.2]{\scriptsize H}
        }
        child{
        node(021)[circ]{}
            child{
            node(0210)[term,label=below:{\scriptsize $0$}]{} 
            edge from parent node[left,pos = 0.3,inner sep=1.5]{\scriptsize H}
            }
            child{
            node(0211)[term,label=below:{\scriptsize $1$}]{}
            edge from parent node[right,pos = 0.3,inner sep=1.5]{\scriptsize T}                
            } 
        edge from parent node[right,pos = 0.2]{\scriptsize T}
        }
    edge from parent node[above,pos = 0.8]{2}
    edge from parent node[above,pos = 0.4]{\scriptsize $\frac{1}{3}$}
    }
    ;

\end{scope}
\draw [dashed,thick,blue,out=22,in=158] (00) to (01); 
\draw [dashed,thick,blue,out=22,in=158] (01) to (02); 
\draw [dashed,ForestGreen,thick,out=22,in=158] (001) to (010);
\draw [dashed,ForestGreen,thick,out=22,in=158] (000) to (001);
\draw [dashed,ForestGreen,thick,out=22,in=158] (010) to (011);
\draw [dashed,ForestGreen,thick,out=22,in=158] (011) to (020);
\draw [dashed,ForestGreen,thick,out=22,in=158] (020) to (021);

%\node[fit=(1),dashed,thick,blue, draw, circle,inner sep=1pt] {};


%\node[fit=(2),dashed,thick,black, draw, circle,inner sep=1pt] {};

\end{tikzpicture}
\caption{Alice and Bob, both observe nothing}
\label{fig:match-penny-3-die-a}
\end{subfigure}
\vspace{6mm}
\begin{subfigure}{0.48\columnwidth}
\centering
\tikzset{
triangle/.style = {regular polygon,regular polygon sides=3,draw,inner sep = 2},
circ/.style = {circle,fill=cyan!10,draw,inner sep = 3},
term/.style = {circle,draw,inner sep = 1.5,fill=black},
sq/.style = {rectangle,fill=gray!20, draw, inner sep = 4}
}
\begin{tikzpicture}[scale=0.8]

\tikzstyle{level 1}=[level distance=18mm,sibling distance=24mm]
\tikzstyle{level 2}=[level distance=11mm,sibling distance=12mm]
\tikzstyle{level 3}=[level distance=11mm,sibling distance=5mm]

\begin{scope}[->, >=stealth]
 \node(0)[triangle]{}
    child{  
    node(00)[circ,draw=black]{}
        child{
        node(000)[circ]{}
            child{
            node(0000)[term,label=below:{\scriptsize $1$}]{} 
            edge from parent node[left,pos = 0.3,inner sep=1.5]{\scriptsize H}
            }
            child{
            node(0001)[term,label=below:{\scriptsize $0$}]{}
            edge from parent node[right,pos = 0.3,inner sep=1.5]{\scriptsize T}                
            }
        edge from parent node[left,pos = 0.2]{{\scriptsize H}}                
        }
        child{
        node(001)[circ]{}
            child{
            node(0010)[term,label=below:{\scriptsize $0$}]{} 
            edge from parent node[left,pos = 0.3,inner sep=1.5]{\scriptsize H}
            }
            child{
            node(0011)[term,label=below:{\scriptsize $1$}]{}
            edge from parent node[right,pos = 0.3,inner sep=1.5]{\scriptsize T}                
            }
        edge from parent node[right,pos = 0.2]{{\scriptsize T}}
        }
    edge from parent node[above,pos = 0.8]{0} 
    edge from parent node[above,pos = 0.4]{\scriptsize $\frac{1}{3}$}
    }
    child{
    node(01)[circ]{}
        child{
        node(010)[circ]{}
            child{
            node(0100)[term,label=below:{\scriptsize $0$}]{}
            edge from parent node[left,pos = 0.3,inner sep=1.5]{\scriptsize H}
            }
            child{
            node(0101)[term,label=below:{\scriptsize $1$}]{}
            edge from parent node[right,pos = 0.3,inner sep=1.5]{\scriptsize T}                
            }
        edge from parent node[left,pos = 0.2]{\scriptsize H}
        }
        child{
        node(011)[circ]{}
            child{
            node(0110)[term,label=below:{\scriptsize $1$}]{} 
            edge from parent node[left,pos = 0.3,inner sep=1.5]{\scriptsize H}
            }
            child{
            node(0111)[term,label=below:{\scriptsize $0$}]{}
            edge from parent node[right,pos = 0.3,inner sep=1.5]{\scriptsize T}                
            } 
        edge from parent node[right,pos = 0.2]{\scriptsize T}
        }
    edge from parent node[right,pos = 0.8]{1}
    edge from parent node[left,pos = 0.4]{\scriptsize $\frac{1}{3}$}
    }
    child{
    node(02)[circ]{}
        child{
        node(020)[circ]{}
            child{
            node(0200)[term,label=below:{\scriptsize $1$}]{}
            edge from parent node[left,pos = 0.3,inner sep=1.5]{\scriptsize H}
            }
            child{
            node(0201)[term,label=below:{\scriptsize $0$}]{}
            edge from parent node[right,pos = 0.3,inner sep=1.5]{\scriptsize T}                
            }
        edge from parent node[left,pos = 0.2]{\scriptsize H}
        }
        child{
        node(021)[circ]{}
            child{
            node(0210)[term,label=below:{\scriptsize $0$}]{} 
            edge from parent node[left,pos = 0.3,inner sep=1.5]{\scriptsize H}
            }
            child{
            node(0211)[term,label=below:{\scriptsize $1$}]{}
            edge from parent node[right,pos = 0.3,inner sep=1.5]{\scriptsize T}                
            } 
        edge from parent node[right,pos = 0.2]{\scriptsize T}
        }
    edge from parent node[above,pos = 0.8]{2}
    edge from parent node[above,pos = 0.4]{\scriptsize $\frac{1}{3}$}
    }
    ;

\end{scope}
\node[fit=(02),dashed,thick,red, draw, circle,inner sep=1pt] {};
\draw [dashed,thick,blue,out=22,in=158] (00) to (01); 
\draw [dashed,ForestGreen,thick,out=22,in=158] (001) to (010);
\draw [dashed,ForestGreen,thick,out=22,in=158] (000) to (001);
\draw [dashed,ForestGreen,thick,out=22,in=158] (010) to (011);
\draw [dashed,ForestGreen,thick,out=22,in=158] (011) to (020);
\draw [dashed,ForestGreen,thick,out=22,in=158] (020) to (021);
%\node[fit=(1),dashed,thick,blue, draw, circle,inner sep=1pt] {};


%\node[fit=(2),dashed,thick,black, draw, circle,inner sep=1pt] {};

\end{tikzpicture}
\caption{Alice can't distinguish between $2i$ and $2i+1$ for $i \geq 0$, Bob observes nothing}
\label{fig:match-penny-3-die-b}
\end{subfigure}

\begin{subfigure}{\columnwidth}
\centering
\tikzset{
triangle/.style = {regular polygon,regular polygon sides=3,draw,inner sep = 2},
circ/.style = {circle,fill=cyan!10,draw,inner sep = 3},
term/.style = {circle,draw,inner sep = 1.5,fill=black},
sq/.style = {rectangle,fill=gray!20, draw, inner sep = 4}
}
\begin{tikzpicture}[scale=0.8]

\tikzstyle{level 1}=[level distance=18mm,sibling distance=24mm]
\tikzstyle{level 2}=[level distance=11mm,sibling distance=12mm]
\tikzstyle{level 3}=[level distance=11mm,sibling distance=5mm]

\begin{scope}[->, >=stealth]
 \node(0)[triangle]{}
    child{  
    node(00)[circ,draw=black]{}
        child{
        node(000)[circ]{}
            child{
        node(0000)[term,label=below:{\scriptsize $1$}]{} 
            edge from parent node[left,pos = 0.2]{\scriptsize H}
            }
            child{
            node(0001)[term,label=below:{\scriptsize $0$}]{}
            edge from parent node[right,pos = 0.2]{\scriptsize T}                
            }
        edge from parent node[left,pos = 0.2]{{\scriptsize H}}                
        }
        child{
        node(001)[circ]{}
            child{
            node(0010)[term,label=below:{\scriptsize $0$}]{} 
            edge from parent node[left,pos = 0.2]{\scriptsize H}
            }
            child{
            node(0011)[term,label=below:{\scriptsize $1$}]{}
            edge from parent node[right,pos = 0.2]{\scriptsize T}                
            }
        edge from parent node[right,pos = 0.2]{{\scriptsize T}}
        }
    edge from parent node[above,pos = 0.8]{0} 
    edge from parent node[above,pos = 0.4]{\scriptsize $\frac{1}{3}$}
    }
    child{
    node(01)[circ]{}
        child{
        node(010)[circ]{}
            child{
            node(0100)[term,label=below:{\scriptsize $0$}]{}
            edge from parent node[left,pos = 0.2]{\scriptsize H}
            }
            child{
            node(0101)[term,label=below:{\scriptsize $1$}]{}
            edge from parent node[right,pos = 0.2]{\scriptsize T}                
            }
        edge from parent node[left,pos = 0.2]{\scriptsize H}
        }
        child{
        node(011)[circ]{}
            child{
            node(0110)[term,label=below:{\scriptsize $1$}]{} 
            edge from parent node[left,pos = 0.2]{\scriptsize H}
            }
            child{
            node(0111)[term,label=below:{\scriptsize $0$}]{}
            edge from parent node[right,pos = 0.2]{\scriptsize T}                
            } 
        edge from parent node[right,pos = 0.2]{\scriptsize T}
        }
    edge from parent node[right,pos = 0.8]{1}
    edge from parent node[left,pos = 0.4]{\scriptsize $\frac{1}{3}$}
    }
    child{
    node(02)[circ]{}
        child{
        node(020)[circ]{}
            child{
            node(0200)[term,label=below:{\scriptsize $1$}]{}
            edge from parent node[left,pos = 0.2]{\scriptsize H}
            }
            child{
            node(0201)[term,label=below:{\scriptsize $0$}]{}
            edge from parent node[right,pos = 0.2]{\scriptsize T}                
            }
        edge from parent node[left,pos = 0.2]{\scriptsize H}
        }
        child{
        node(021)[circ]{}
            child{
            node(0210)[term,label=below:{\scriptsize $0$}]{} 
            edge from parent node[left,pos = 0.2]{\scriptsize H}
            }
            child{
            node(0211)[term,label=below:{\scriptsize $1$}]{}
            edge from parent node[right,pos = 0.2]{\scriptsize T}                
            } 
        edge from parent node[right,pos = 0.2]{\scriptsize T}
        }
    edge from parent node[above,pos = 0.8]{2}
    edge from parent node[above,pos = 0.4]{\scriptsize $\frac{1}{3}$}
    }
    ;

\end{scope}
\node[fit=(02),dashed,thick,red, draw, circle,inner sep=1pt] {};
\draw [dashed,thick,blue,out=22,in=158] (00) to (01);
\draw [dashed,thick,ForestGreen,out=28,in=152] (000) to (010);
\draw [dashed,ForestGreen,thick,out=28,in=152] (010) to (020);
\draw [dashed,brown,thick,out=28,in=152] (001) to (011);
\draw [dashed,brown,thick,out=28,in=152] (011) to (021);
%\node[fit=(1),dashed,thick,blue, draw, circle,inner sep=1pt] {};


%\node[fit=(2),dashed,thick,black, draw, circle,inner sep=1pt] {};

\end{tikzpicture}
\caption{Bob only observes Alice's coin}
\label{fig:match-penny-3-die-c}
\end{subfigure}


% \begin{subfigure}{.48\columnwidth}
% \centering
% \tikzset{
% triangle/.style = {regular polygon,regular polygon sides=3,draw,inner sep = 2},
% circ/.style = {circle,fill=cyan!10,draw,inner sep = 3},
% term/.style = {circle,draw,inner sep = 1.5,fill=black},
% sq/.style = {rectangle,fill=gray!20, draw, inner sep = 4}
% }
% \begin{tikzpicture}[scale=0.8]

% \tikzstyle{level 1}=[level distance=15mm,sibling distance=23mm]
% \tikzstyle{level 2}=[level distance=10mm,sibling distance=12mm]
% \tikzstyle{level 3}=[level distance=12mm,sibling distance=4mm]

% \begin{scope}[->, >=stealth]
%  \node(0)[triangle]{}
%     child{  
%     node(00)[circ,draw=black]{}
%         child{
%         node(000)[circ]{}
%             child{
%             node(0000)[term,label=below:{\scriptsize $1$}]{} 
%             edge from parent node[left,pos = 0.2]{\scriptsize H}
%             }
%             child{
%             node(0001)[term,label=below:{\scriptsize $0$}]{}
%             edge from parent node[right,pos = 0.2]{\scriptsize T}                
%             }
%         edge from parent node[left,pos = 0.2]{{\scriptsize H}}                
%         }
%         child{
%         node(001)[circ]{}
%             child{
%             node(0010)[term,label=below:{\scriptsize $0$}]{} 
%             edge from parent node[left,pos = 0.2]{\scriptsize H}
%             }
%             child{
%             node(0011)[term,label=below:{\scriptsize $1$}]{}
%             edge from parent node[right,pos = 0.2]{\scriptsize T}                
%             }
%         edge from parent node[right,pos = 0.2]{{\scriptsize T}}
%         }
%     edge from parent node[above,pos = 0.8]{0} 
%     }
%     child{
%     node(01)[circ]{}
%         child{
%         node(010)[circ]{}
%             child{
%             node(0100)[term,label=below:{\scriptsize $0$}]{}
%             edge from parent node[left,pos = 0.2]{\scriptsize H}
%             }
%             child{
%             node(0101)[term,label=below:{\scriptsize $1$}]{}
%             edge from parent node[right,pos = 0.2]{\scriptsize T}                
%             }
%         edge from parent node[left,pos = 0.2]{\scriptsize H}
%         }
%         child{
%         node(011)[circ]{}
%             child{
%             node(0110)[term,label=below:{\scriptsize $1$}]{} 
%             edge from parent node[left,pos = 0.2]{\scriptsize H}
%             }
%             child{
%             node(0111)[term,label=below:{\scriptsize $0$}]{}
%             edge from parent node[right,pos = 0.2]{\scriptsize T}                
%             } 
%         edge from parent node[right,pos = 0.2]{\scriptsize T}
%         }
%     edge from parent node[right,pos = 0.8]{1}
%     }
%     child{
%     node(02)[circ]{}
%         child{
%         node(020)[circ]{}
%             child{
%             node(0200)[term,label=below:{\scriptsize $1$}]{}
%             edge from parent node[left,pos = 0.2]{\scriptsize H}
%             }
%             child{
%             node(0201)[term,label=below:{\scriptsize $0$}]{}
%             edge from parent node[right,pos = 0.2]{\scriptsize T}                
%             }
%         edge from parent node[left,pos = 0.2]{\scriptsize H}
%         }
%         child{
%         node(021)[circ]{}
%             child{
%             node(0210)[term,label=below:{\scriptsize $0$}]{} 
%             edge from parent node[left,pos = 0.2]{\scriptsize H}
%             }
%             child{
%             node(0211)[term,label=below:{\scriptsize $1$}]{}
%             edge from parent node[right,pos = 0.2]{\scriptsize T}                
%             } 
%         edge from parent node[right,pos = 0.2]{\scriptsize T}
%         }
%     edge from parent node[above,pos = 0.8]{2}
%     }
%     ;

% \end{scope}
% \draw [dashed,thick,blue,out=22,in=158] (00) to (01); 
% \draw [dashed,ForestGreen,thick,out=22,in=158] (000) to (021);
% \draw [dashed,brown,thick,out=22,in=158] (001) to (011);
% \draw [dashed,brown,thick,out=22,in=158] (010) to (020);
% %\node[fit=(1),dashed,thick,blue, draw, circle,inner sep=1pt] {};


% %\node[fit=(2),dashed,thick,black, draw, circle,inner sep=1pt] {};

% \end{tikzpicture}
% \caption{Bob can't distinguish between $(i,\text{C})$ and $(i+1 \mod n,\text{C})$ for $i \geq 0 $, $C \in \{H,T\}$ and $n=3$}
% \label{fig:match-penny-3-die-d}
% \end{subfigure}

\caption{Three versions of the single team matching-unmatching pennies game for $n=3$}
\label{fig:match-penny-3-die}
\end{figure}

Before we delve into the background and results, here is a description of the extensive-form model. 
The root node, marked with a triangle, is the event of rolling the die. The triangle nodes are called $\chance$ nodes, and the
edges out of them associate probabilities to each of the outcomes. For this game, the distribution is uniform. The circle nodes denote decision nodes of the team. The nodes in the second level (root being the first level) belong to Alice whereas the nodes in the third level belong to Bob. The actions labelled in edges out of these nodes denote the actions available to the corresponding players. 
A leaf node indicates an end state, and a path from root to leaf denotes
a play from start to end. The number associated with a leaf gives the
payoff that the team receives at the end of the corresponding play. E.g., in \cref{fig:match-penny-3-die-a} in the play resulting from the path $0, H, T$ the payoff is $0$ because the team loses. It is $1$ when they win. 

Imperfect information is expressed using a dotted line: a player cannot distinguish between two nodes joined by a dotted line.
For e.g., in \cref{fig:match-penny-3-die-a} the dotted red line joining all of Alice's nodes indicates that Alice cannot observe the die outcome. Similarly, the blue dotted line for Bob indicates, he neither observes the outcome of the die, nor the side of the coin chosen by Alice. These sets of indistinguishable nodes are called \emph{information sets}.





\section{Background and notations}
\label{sec:background}

%!TEX root = ../main.tex

\begin{figure}
%\begin{center}
\tikzset{
triangle/.style = {regular polygon,regular polygon sides=3,draw,inner sep = 2},
circ/.style = {circle,fill=cyan!10,draw,inner sep = 3},
term/.style = {circle,draw,inner sep = 1.5,fill=black},
sq/.style = {rectangle,fill=gray!20, draw, inner sep = 4}
}

\begin{subfigure}{.45\columnwidth}
\centering
\begin{tikzpicture}[scale=0.9]
\tikzstyle{level 1}=[level distance=9mm,sibling distance = 22mm]
\tikzstyle{level 2}=[level distance=7mm,sibling distance=10mm]
\tikzstyle{level 3}=[level distance=7mm,sibling distance=6mm]
\tikzstyle{level 4}=[level distance=7mm,sibling distance=5mm]

%node (ij) is the j th node in i th level

\begin{scope}[->, >=stealth]
\node (0) [circ] {}
child {
  node (00) [triangle] {}
  child {
    node (000) [circ] {}
    child {
      node (0000) [term, label=below:{}] {}
      edge from parent node [left] {\scriptsize $c$}
    }
    child {
      node (0001) [term, label=below:{}] {}
      edge from parent node [right] {\scriptsize $d$}
      }
    edge from parent node [left] {}
  }
  child {
    node (001) [circ] {}
    child {
      node (0010) [term, label=below:{}] {}
      edge from parent node [left] {\scriptsize $c$}
    }
    child {
      node (0011) [term, label=below:{}] {}
      edge from parent node [right] {\scriptsize $d$}
      }
    edge from parent node [right] {} 
  }
  edge from parent node [above] {\scriptsize$a$}
}
child {
  node (01) [triangle] {}
   child {
     node (010) [circ] {}
     child {
      node (0100) [term, label=below:{}] {}
      edge from parent node [left] {\scriptsize $e$}
    }
    child {
      node (0101) [term, label=below:{}] {}
      edge from parent node [right] {\scriptsize $f$}
      }
    edge from parent node [left] {}
  }
  child {
    node (011) [circ] {}
    child {
      node (0110) [term, label=below:{}] {}
      edge from parent node [left] {\scriptsize $e$}
    }
    child {
      node (0111) [term, label=below:{}] {}
      edge from parent node [right] {\scriptsize $f$}
      }
    edge from parent node [right] {} 
  }
  edge from parent node [above] {\scriptsize$b$}
}
;
\end{scope}

%observations
%\draw [dashed, thick, red, in=150,out=30](00) to (01) ;

  \node[fit=(0),dashed,thick,red, draw, circle,inner sep=1pt] {};
\draw [dashed, thick, blue, in=150,out=30] (000) to (001) ;
\draw [dashed, thick, ForestGreen, in=150,out=30] (010) to (011);

%node labels
\node [black] at (0,0.35) {\scriptsize $r$};
\node [black] at (-1,-0.55) {\scriptsize $u_1$};
\node [black] at (1, -0.55) {\scriptsize $u_2$};
\node [black] at (-2, -1.5) {\scriptsize $u_3$};
\node [black] at (-.25, -1.5) {\scriptsize $u_4$};

\node [black] at (0.25, -1.5) {\scriptsize $u_5$};
\node [black] at (2, -1.5) {\scriptsize $u_6$};

%obs labels
\node [red] at (0,-.5) {\scriptsize $I_1$};
\node [blue] at (-1.1,-1.6) {\scriptsize $I_2$};
\node [ForestGreen] at (1.1,-1.6) {\scriptsize $I_3$};



\end{tikzpicture}

\caption{$\Max$ with perfect recall}
\label{fig-allexmp-pftrec}
\end{subfigure}
\quad
\begin{subfigure}{.45\columnwidth}
\centering
\begin{tikzpicture}
\tikzstyle{level 1}=[level distance=7mm,sibling distance = 10mm]
\tikzstyle{level 2}=[level distance=7mm,sibling distance=10mm]
\tikzstyle{level 3}=[level distance=7mm,sibling distance=15mm]
\tikzstyle{level 4}=[level distance=7mm,sibling distance=8mm]

%\draw [help lines, step=0.5] (-3,-3) grid (3,0);

\begin{scope}[->, >=stealth]
\node (0) [circ] {}
child{
  node (1) [circ] {}
  child{
    node (3) [term, label=below:{}] {}
    edge from parent node [left] {\scriptsize $a$}
  }
  child{
    node (4) [term,label=below:{}] {}
    edge from parent node [right] {\scriptsize $b$}
  }
  edge from parent node [left] {\scriptsize $a$}
}
child{
  node (2) [term, label=below:{}] {}
  edge from parent node [right] {\scriptsize $b$}
}
;
\end{scope}

\draw [dashed, thick, blue, in=10,out=-100] (0) to (1);



\node [black] at (0,0.25) {\scriptsize $r$};
\node [black] at (-.9,-0.6) {\scriptsize $u_1$};



\node [blue] at (.1,-.6) {\scriptsize $I_1$};

\end{tikzpicture}
\caption{$\Max$ with absentmindedness}
\label{fig-allexmp-absentm}
\end{subfigure}


%\end{center}
\caption{Recalls of $\Max$}
\label{fig:recall-examples}
\end{figure}

This section presents the formal definitions. The single team matching-unmatching pennies game has only one player and chance nodes, but in general we will talk about zero-sum two player games. As in \cref{fig:2-p-shuffle}, there are two players $\Max$ (circle nodes) and $\Min$ (square nodes). The payoff at the leaf, is the amount $\Min$ loses and $\Max$ gains. The goal of $\Max$ is to maximize the expected payoff whereas $\Min$ wishes to minimize it. In \cref{fig:match-penny-3-die} $\Max$ was the team consisting of Alice and Bob.







In this paper, we mainly work with \emph{game-structures} and not games
themselves. Game-structures are essentially games sans the numerical
quantities. Any game on a game structure can be represented symbolically as in shown \cref{fig:alossSpan-a} with symbolic payoffs $z_i$s and symbolic chance probabilities $p_i$s (with constraints on $p_i$'s). An extensive
form game can be obtained from a game structure by plugging in values for $z_i$s and $p_i$s.  We work with game structures because the
notions of perfect recall and imperfect recall can be determined
simply by looking at the game-structure.


Formally, a game-structure $\Tt$ is a tuple $(V, L, r, A, E, \Ii)$
where $V$ is a finite set of non-terminal nodes partitioned as
$V_{\Max}$, $V_{\Min}$ and $V_{\chance}$; $L$ is a finite set of leaves;
$r \in V$ is a root node; $A = A_{\Max} \cup A_{\Min}$ is a finite set
of actions; $E \incl V \times (V \cup L)$ is an edge
relation that induces a directed tree; edges originating from $V_{\Max} \cup V_{\Min}$ are labelled with actions from $A$; we write $u \xra{a} v$ if
$(u, v)$ is labelled with $a$, and assume that there is no incoming edge
$u \xra{} r$ to the root node $r$; $\Ii = \Ii_{\Max} \cup \Ii_{\Min}$
is a set of information sets for $i \in \{ \Max, \Min \}$, each
information set $I \in \Ii_i$ is a subset of vertices belonging to
$i$, i.e. $I \incl V_i$, and moreover, the set of information sets
$\Ii_i$ partitions $V_i$. E.g., in \cref{fig-allexmp-pftrec},
$\Ii_{\Max} = \{I_1, I_2, I_3\}$ and $I_1 = \{r\}, I_2 = \{u_3, u_4\}$
and $I_3 = \{u_5, u_6\}$. We can understand these information sets as a signal that the player receives when she reaches a node in it. On receiving the signal, the player knows the actions that are available to play at that position. 

An information set models the fact that a player cannot distinguish
between the nodes within it. Therefore, the set of outgoing actions
from each node in an information set is required to be the same. This
allows us to define $\act(I)$ as the set of actions available at
information set $I$. E.g., in \cref{fig-allexmp-pftrec},
$\act(I_2) = \{c, d\}$. For technical convenience, we make a second
assumption: for all $I, I' \in \Ii$ with $I \neq I'$, we have
$\act(I) \cap \act(I') = \emptyset$. Therefore, the actions identify
the information sets. With this assumption, in \cref{fig:match-penny-3-die}, the actions of Alice should be seen as $H_A, T_A$ and those of Bob's as $H_B, T_B$. But we omit the subscripts in the figure for clarity. 
\begin{definition}[Extensive form games]\label{def:ext-form-games}
  A two-player zero-sum game in extensive form is a tuple
  $(\Tt,\d, \Uu)$ where $\Tt$ is a game-structure, $\d$ is the
  \emph{chance probability} associating to each $\chance$ node, a
  probability distribution on the outgoing actions, and
  $\Uu : L \mapsto \Rat $ is the utility function associating a payoff
  to each leaf.
\end{definition}

The \emph{size} of a game is the sum of the bit-lengths of all chance probabilities and leaf
payoffs in it. A \emph{behavioral strategy} for player $\Max$ ($\Min$ resp.) assigns a probability
distribution to $\act(I)$ for each $I \in \Ii_{\Max}$ ($\Ii_{\Min}$ resp.). Once we fix behavioral strategies $\sigma$ and $\tau$ for $\Max$ and $\Min$ respectively,
each edge in the game has an associated probability of being taken,
given by the corresponding strategy or $\chance$. The probability of reaching a leaf $u \in L$ is given by the product of all the numbers along the path to the leaf. Consider \cref{fig:shuffle-a}.
Let $\sigma$ assign $\frac{1}{4}$ to $b$ and $\frac{3}{4}$ to
$\bar{b}$; $0$ and $1$ to $c$ and $\bar{c}$, and $\frac{1}{3}$ to $a$
and $\frac{2}{3}$ to $\bar{a}$. The probability of reaching the leaf $b \bar{a}$ is 
then: $p_1 \times \frac{1}{4} \times \frac{2}{3}$. For a leaf $u$, we denote this quantity by
$\prob_{\sigma, \tau}(u)$. The \emph{expected payoff} $\Ee(\s, \t)$
when $\Max$ plays $\sigma$ and $\Min$ plays $\tau$, then equals
$\sum_{u \in L} \prob_{\s, \t} (u) \Uu(u)$. The solution concept that we
will consider in this paper is the notion of maxmin.
The \emph{maxmin value} of a game is the
following: \[\max\limits_{\s}\min\limits_{\t}\Ee(\s,\t)\] where
$\s,\t$ are behavioral strategies of $\Max$ and $\Min$ respectively. A
strategy of $\Max$ which provides the maxmin value is called a
\emph{maxmin strategy}. In one-player games, we only have $\Max$ player and the maxmin value of the game is $\max\limits_{\s}\Ee(\s)$. For one-player non-absentminded games, the maxmin value can be in fact obtained by a \emph{pure strategy} -- pure strategies are special cases of behavioural strategies which assign either $0$ or $1$ to each action~\cite{KollerMegiddo::1992}.

The maxmin value of the game in \cref{fig:match-penny-3-die-a} is $\frac{2}{3}$ since Alice and Bob can win at most in 2 of the 3 die rolls by playing matching sides. Another way to see this is to consider the four possible pure strategies $HH, HT, TH, TT$, which induce payoffs $\frac{2}{3}$, $\frac{1}{3}$, $\frac{1}{3}$ and $\frac{2}{3}$ respectively. Now since, in the rest of the following two versions, the team has more information \footnote{This can be observed by the fact that information sets in each version are refinements of the previous versions.} they can guarantee at least $\frac{2}{3}$ by playing the same strategy. Interestingly, one can observe (by enumerating all pure strategies) that they cannot do better than that in any version. 
\paragraph*{Histories and recalls.} We now move on to describing the
various types of imperfect information, based on what the player
remembers about her history. A node $w \in V$ is reached by a unique
path from the root: $r = v_0 \xra{} v_1 \xra{} \cdots \xra{} v_n =
w$. Let $v_{i_1}, v_{i_2}, \dots, v_{i_k}$ be the vertices in this
sequence which do not belong to $\chance$. Then,
$\his(w) = a_{1} a_{2} \cdots a_{k-1}$, where $v_{i_j} \xra{a_j} v_{i_{j + 1}}$.
For a player $i \in \{\Max, \Min\}$ the history of $i$ at $w$, denoted
by $\his_i(w)$, is the sequence of player $i$'s actions in the path to
$w$, which is simply the sub-sequence of $\his(w)$ restricted to
actions from $A_i$. E.g.: in \cref{fig-allexmp-pftrec},
$\his_{\Max}(u_3) = \his_{\Max}(u_4) = a$; in
\cref{fig:shuffle-a}, $\his_{\Max}(u_3) = b$ and $\his_{\Max}(u_2) = \epsilon$, the empty sequence. It is important to remark that this definition uses the assumption that actions determine information sets -- otherwise, we would need to incorporate the information sets that were visited along the way, into the history.



Let $\Hh$ denote the set of all histories and $\Hh_i$ be the set of
all histories of player $i$. For an information set $I \in \Ii_i$ let
$\Hh(I) = \{ \his(u) \mid u \in I\}$ be the set of histories of all
nodes in $I$. Similarly, we can define $\Hh_i(I)$ with respect to
$\Hh_i$. Let $\Hh(L)$ denote the set of all leaf histories.
When $\Hh_i(I)$ has multiple histories, at a node $v \in I$ the player
does not remember which history she traversed to reach $v$. Hence the
player loses information. For two
nodes $u$ and $v$ in $I$, comparing $\his_i(u)$ and $\his_i(v)$
reveals the loss or retention of previously withheld information at
the respective nodes. To capture this there are different notions of
\emph{recall}.

\emph{Perfect recall.} Player $i$ is said to have \emph{perfect
  recall} ($\pfr$) if for every $I \in \Ii_i$, and every pair of
distinct vertices $u, v \in I$, we have $\his_i(u) = \his_i(v)$,
i.e. $|\Hh_i(I)| = 1$.  Otherwise, the player is said to have imperfect
recall.  \cref{fig-allexmp-pftrec} is an example of a perfect recall
game.
\emph{Imperfect recall.} \cref{fig:shuffle-a} gives an example of
a game-structure that has imperfect recall. Notice that states $u_3$
and $u_4$ lie in the same information set $I_3$, but the sequence of
the player's actions leading to these states is different: history at
$u_3$ is $b$, whereas at $u_4$ it is $\bar{b}$. 
Within imperfect recall,
there are distinctions. The imperfect recall in
~\cref{fig:shuffle-b} and the one in ~\cref{fig:shuffle-a}
are in some sense different: in ~\cref{fig:shuffle-b}, the
inability to distinguish between the two nodes in $I_1$ can be traced back to
a point in the past where she forgets her own action from some
information set ($I_3$ in this case), whereas in \cref{fig:shuffle-a}, the player has been able to
distinguish between the two outcomes of the $\chance$ node, but later
forgets at $I_3$ where she started from, leading to four histories $b,\bar{b},c$ and $\bar{c}$ at $I_3$.


\emph{A-loss recall.} Game-structures as in \cref{fig:shuffle-b} are said to have \emph{A-loss
  recall}. A consequence of having A-loss recall is that a player always remembers any new information
gained from $\chance$ outcomes, which is not the case
in~\cref{fig:shuffle-a}. Player $i$ has \emph{A-loss recall}
($\alr$) if for all $I \in \Ii_i$, and every pair of distinct vertices
$u, v \in I$, either $\his_i(u) = \his_i(v)$, or $\his_i(u)$ is of the
form $s a s_1$, and $\his_i(v)$ of the form $s b s_2$, where
$a, b \in \act(I')$ for some $I' \in \Ii$, with $a \neq b$. The game in \cref{fig:match-penny-3-die-a} has A-loss recall, whereas the others, \cref{fig:match-penny-3-die-b} and \cref{fig:match-penny-3-die-c} do not.  

Finally, player $i$ is said to be
\emph{non-absentminded} ($\nam$) if $\forall u, v \in V_i$ with $u$
lying on the path to $v$, the information set that $u$ belongs to is
different from the information set that $v$ belongs
to, i.e. all nodes of $i$ on a path from $r$ to leaf node lie in distinct
information sets. \cref{fig-allexmp-absentm} is an example where
$\Max$ is absentminded, since both $r$ and $u_1$ lie in the same
information set. Notice that $\pfr$ implies $\alr$, which in turn implies implies $\nam$. 

When $\Max$ and $\Min$ have recalls $R_{\Max}, R_{\Min} \in \{ $\pfr$,~$\alr$,~$\nam$ \} $
respectively we will denote the game as a
$(R_{\Max}, R_{\Min})$-game. A one-player game with recall $R$ is denoted as $R$-game. In this paper we are only concerned with one-player $\nam$-games and two-player $(\nam,\nam)$-games. Let us now recall some known
results.
\begin{itemize}\item A maxmin solution in a $(\pfr,\alr)$-game can be computed in
  polynomial- time~\\\cite{KollerMegiddo::1992,vonStengel::1996,kaneko1995behavior}. As a corollary, an optimal solution in a one-player $\alr$-game can be computed in
  polynomial-time ~\cite{kaneko1995behavior}.

\item The maxmin decision problem for $(\nam,\nam)$-games is both
  $\NP$-hard~\cite{KollerMegiddo::1992} and

  
  $\sqsum$-hard~\cite{GPS20} \footnote{$\sqsum$ is the decision problem of checking
    if the sum of the square roots of $k$ positive integers is less
    than another positive number}.
   The $\NP$-hardness and the $\sqsum$-hardness hold even for
  $(\alr,\pfr)$-games~\cite{Cermak::2018,GPS20}. The maxmin decision problem for one-player $\nam$-games is
  $\NP$-complete~\cite{KollerMegiddo::1992}.
\end{itemize}


Our core idea is to view game structures through the polynomials they generate. 
\paragraph*{Leaf monomials.} In a game structure, assigning variable $x_a$ to each action
$a$, the monomial obtained by taking the product of all $x_a$ along the path to each leaf $t$ is called
a \emph{leaf monomial}, and denoted as $\mu(t)$. E.g., the leaf
monomials of the game-structure in \cref{fig-allexmp-pftrec} are
$\{ x_ax_c, x_ax_d, x_bx_e, x_bx_f\}$. For a game structure $\Tt$, we will write $X(\Tt)$ for the set of leaf monomials. For a game $G$, let
$\prob_{\chance}(t)$ denote the product of $\chance$ probabilities in
the path to $t$. The polynomial given by
$\sum\limits_{t \in L} \prob_{\chance}(t) \cdot \Uu(t) \cdot \mu(t)$
is called the \emph{payoff polynomial} of a game. 
A constraint of the form $\sum\limits_{a \in \act(I)} x_a = 1$ for an
information set $I$ will be called a \emph{strategy constraint}. Any non-negative valuation satisfying these constraints gives a behavioral strategy to the players.
The maxmin value in a game can be given by the maxmin of the payoff polynomial over all possible values satisfying the strategy constraints.


\paragraph*{Overview of our work}
In this work, our mantra for simplifying games is to find simpler games with same payoff polynomials (upto renaming of variables). Leaf monomials are the building blocks of payoff polynomials. We give methods to generate from a given game-structure $\Tt$, a transformed game-structure $\Tt'$ with A-loss recall such that: either $\Tt'$ has the same set of leaf monomials (Section~\ref{sec:shuffled-loss-recall}), or each leaf monomial of $\Tt$ is a linear combination of the leaf monomials of $\Tt'$ (Section~\ref{sec:span}).


\endinput



















%!TEX root = ../main.tex

\begin{figure}
\begin{subfigure}{0.5\columnwidth}
\centering
\tikzset{
triangle/.style = {regular polygon,regular polygon sides=3,draw,inner sep = 2},
circ/.style = {circle,fill=cyan!10,draw,inner sep = 3},
term/.style = {circle,draw,inner sep = 1.5,fill=black},
sq/.style = {rectangle,fill=gray!20, draw, inner sep = 4}
}

\begin{tikzpicture}[scale=0.85]
\tikzstyle{level 1}=[level distance=9mm,sibling distance = 22mm]
\tikzstyle{level 2}=[level distance=7mm,sibling distance=10mm]
\tikzstyle{level 3}=[level distance=7mm,sibling distance=6mm]
\tikzstyle{level 4}=[level distance=7mm,sibling distance=5mm]

%node (ij) is the j th node in i th level

\begin{scope}[->, >=stealth]
\node (0) [triangle] {}
child {
  node (00) [circ] {}
  child {
    node (000) [circ] {}
    child {
      node (0000) [term, label=below:{\scriptsize $z_1$}] {}
      edge from parent node [left] {\scriptsize $a$}
    }
    child {
      node (0001) [term, label=below:{\scriptsize $z_2$}] {}
      edge from parent node [right] {\scriptsize $\bar{a}$}
      }
    edge from parent node [left] {\scriptsize $b$}
  }
  child {
    node (001) [circ] {}
    child {
      node (0010) [term, label=below:{\scriptsize $z_3$}] {}
      edge from parent node [left] {\scriptsize $a$}
    }
    child {
      node (0011) [term, label=below:{\scriptsize $z_4$}] {}
      edge from parent node [right] {\scriptsize $\bar{a}$}
      }
    edge from parent node [right] {\scriptsize $\bar{b}$} 
  }
  edge from parent node [above] {\scriptsize $p_1$}
}
child {
  node (01) [circ] {}
   child {
     node (010) [circ] {}
     child {
      node (0100) [term, label=below:{\scriptsize $z_5$}] {}
      edge from parent node [left] {\scriptsize $a$}
    }
    child {
      node (0101) [term, label=below:{\scriptsize $z_6$}] {}
      edge from parent node [right] {\scriptsize $\bar{a}$}
      }
    edge from parent node [left] {\scriptsize $c$}
  }
  child {
    node (011) [circ] {}
    child {
      node (0110) [term, label=below:{\scriptsize $z_7$}] {}
      edge from parent node [left] {\scriptsize $a$}
    }
    child {
      node (0111) [term, label=below:{\scriptsize $z_8$}] {}
      edge from parent node [right] {\scriptsize $\bar{a}$}
      }
    edge from parent node [right] {\scriptsize $\bar{c}$} 
  }
  edge from parent node [above] {\scriptsize $p_2$}
}
;
 \node[fit=(00),dashed,thick,blue, draw, circle,inner sep=1pt] {};
  \node[fit=(01),dashed,thick,red, draw, circle,inner sep=1pt] {};
\end{scope}

\draw [dashed, thick, ForestGreen, in=150,out=30] (000) to (001);
\draw [dashed, thick, ForestGreen, in=150,out=30] (001) to (010);
\draw [dashed, thick, ForestGreen, in=150,out=30] (010) to (011);
%\draw [dashed, thick, blue, in=150,out=30] (000) to (001);
%\draw [dashed, thick, red, in=150,out=30] (010) to (011);

\node [black] at (0,0.35) {\scriptsize $r$};
\node [black] at (-1.5,-0.55) {\scriptsize $u_1$};
\node [black] at (1.5, -0.55) {\scriptsize $u_2$};
\node [black] at (-2, -1.6) {\scriptsize $u_3$};
\node [black] at (-.25, -1.7) {\scriptsize $u_4$};

\node [black] at (0.25, -1.7) {\scriptsize $u_5$};
\node [black] at (2, -1.6) {\scriptsize $u_6$};

%obs labels
\node [ForestGreen] at (0,-1.1) {\scriptsize $I_3$};
\node [blue] at (-.55,-.9) {\scriptsize $I_1$};
\node [red] at (.55,-.9) {\scriptsize $I_2$};

\end{tikzpicture}
\caption{$\Max$ without $\alr$ but has $\salr$}
\label{fig:shuffle-a}
\end{subfigure}
\begin{comment}
\begin{subfigure}{0.45\columnwidth}
%\centering
\tikzset{
triangle/.style = {regular polygon,regular polygon sides=3,draw,inner sep = 2},
circ/.style = {circle,fill=cyan!10,draw,inner sep = 3},
term/.style = {circle,draw,inner sep = 1.5,fill=black},
sq/.style = {rectangle,fill=gray!20, draw, inner sep = 4}
}

\begin{tikzpicture}[scale=0.85]
\tikzstyle{level 1}=[level distance=9mm,sibling distance = 22mm]
\tikzstyle{level 2}=[level distance=7mm,sibling distance=10mm]
\tikzstyle{level 3}=[level distance=7mm,sibling distance=6mm]
\tikzstyle{level 4}=[level distance=7mm,sibling distance=5mm]

%node (ij) is the j th node in i th level

\begin{scope}[->, >=stealth]
\node (0) [circ] {}
child {
  node (00) [triangle] {}
  child {
    node (000) [circ] {}
    child {
      node (0000) [term, label=below:{}] {}
      edge from parent node [left] {\scriptsize $b$}
    }
    child {
      node (0001) [term, label=below:{}] {}
      edge from parent node [right] {\scriptsize $\bar{b}$}
      }
    edge from parent node [left] {}
  }
  child {
    node (001) [circ] {}
    child {
      node (0010) [term, label=below:{}] {}
      edge from parent node [left] {\scriptsize $b$}
    }
    child {
      node (0011) [term, label=below:{}] {}
      edge from parent node [right] {\scriptsize $\bar{b}$}
      }
    edge from parent node [right] {} 
  }
  edge from parent node [above] {\scriptsize $a$}
}
child {
  node (01) [triangle] {}
   child {
     node (010) [circ] {}
     child {
      node (0100) [term, label=below:{}] {}
      edge from parent node [left] {\scriptsize $c$}
    }
    child {
      node (0101) [term, label=below:{}] {}
      edge from parent node [right] {\scriptsize $\bar{c}$}
      }
    edge from parent node [left] {}
  }
  child {
    node (011) [circ] {}
    child {
      node (0110) [term, label=below:{}] {}
      edge from parent node [left] {\scriptsize $c$}
    }
    child {
      node (0111) [term, label=below:{}] {}
      edge from parent node [right] {\scriptsize $\bar{c}$}
      }
    edge from parent node [right] {} 
  }
  edge from parent node [above] {\scriptsize $\bar{a}$}
}
;
\end{scope}

%\draw [dashed, thick, ForestGreen, in=150,out=30] (00) to (01);

\node[fit=(0),dashed,thick,ForestGreen, draw, circle,inner sep=1pt] {};
\draw [dashed, thick, blue, in=150,out=30] (000) to (010);
\draw [dashed, thick, red, in=150,out=30] (001) to (011);


\node [black] at (0,0.45) {\scriptsize $r$};
\node [black] at (-1.1,-0.45) {\scriptsize $u_1$};
\node [black] at (1.1, -0.45) {\scriptsize $u_2$};
\node [black] at (-2, -1.65) {\scriptsize $u_3$};
\node [black] at (-.2, -1.65) {\scriptsize $u_4$};

\node [black] at (0.25, -1.65) {\scriptsize $u_5$};
\node [black] at (2, -1.65) {\scriptsize $u_6$};

%obs labels
\node [ForestGreen] at (0,-0.6) {\scriptsize $I_3$};
\node [blue] at (-0.35,-1) {\scriptsize $I_1$};
\node [red] at (0.35,-1) {\scriptsize $I_2$};

\end{tikzpicture}
\caption{}
\label{fig:shuffle-c}
\end{subfigure}%
\end{comment}
\begin{subfigure}{0.48\columnwidth}
\centering
\tikzset{
triangle/.style = {regular polygon,regular polygon sides=3,draw,inner sep = 2},
circ/.style = {circle,fill=cyan!10,draw,inner sep = 3},
term/.style = {circle,draw,inner sep = 1.5,fill=black},
sq/.style = {rectangle,fill=gray!20, draw, inner sep = 4}
}

\begin{tikzpicture}[scale=0.85]
\tikzstyle{level 1}=[level distance=9mm,sibling distance = 22mm]
\tikzstyle{level 2}=[level distance=7mm,sibling distance=10mm]
\tikzstyle{level 3}=[level distance=7mm,sibling distance=6mm]
\tikzstyle{level 4}=[level distance=7mm,sibling distance=5mm]

%node (ij) is the j th node in i th level

\begin{scope}[->, >=stealth]
\node (0) [circ] {}
child {
  node (00) [triangle] {}
  child {
    node (000) [circ] {}
    child {
      node (0000) [term, label=below:{\scriptsize $z_1$}] {}
      edge from parent node [left] {\scriptsize $b$}
    }
    child {
      node (0001) [term, label=below:{\scriptsize $z_3$}] {}
      edge from parent node [right] {\scriptsize $\bar{b}$}
      }
    edge from parent node [left] {}
  }
  child {
    node (001) [circ] {}
    child {
      node (0010) [term, label=below:{\scriptsize $z_5$}] {}
      edge from parent node [left] {\scriptsize $c$}
    }
    child {
      node (0011) [term, label=below:{\scriptsize $z_7$}] {}
      edge from parent node [right] {\scriptsize $\bar{c}$}
      }
    edge from parent node [right] {} 
  }
  edge from parent node [above] {\scriptsize $a$}
}
child {
  node (01) [triangle] {}
   child {
     node (010) [circ] {}
     child {
      node (0100) [term, label=below:{\scriptsize $z_2$}] {}
      edge from parent node [left] {\scriptsize $b$}
    }
    child {
      node (0101) [term, label=below:{\scriptsize $z_4$}] {}
      edge from parent node [right] {\scriptsize $\bar{b}$}
      }
    edge from parent node [left] {}
  }
  child {
    node (011) [circ] {}
    child {
      node (0110) [term, label=below:{\scriptsize $z_6$}] {}
      edge from parent node [left] {\scriptsize $c$}
    }
    child {
      node (0111) [term, label=below:{\scriptsize $z_8$}] {}
      edge from parent node [right] {\scriptsize $\bar{c}$}
      }
    edge from parent node [right] {} 
  }
  edge from parent node [above] {\scriptsize $\bar{a}$}
}
;
\end{scope}

%\draw [dashed, thick, ForestGreen, in=150,out=30] (00) to (01);

\node[fit=(0),dashed,thick,ForestGreen, draw, circle,inner sep=1pt] {};
\draw [dashed, thick, blue, in=150,out=30] (000) to (010);
\draw [dashed, thick, red, in=150,out=30] (001) to (011);

%\node [black] at (0,0.45) {\scriptsize $r$};
%\node [black] at (-1.1,-0.45) {\scriptsize $u_1$};
%\node [black] at (1.1, -0.45) {\scriptsize $u_2$};
%\node [black] at (-2, -1.65) {\scriptsize $u_3$};
%\node [black] at (-.2, -1.65) {\scriptsize $u_4$};
%
%\node [black] at (0.25, -1.65) {\scriptsize $u_5$};
%\node [black] at (2, -1.65) {\scriptsize $u_6$};

%obs labels
\node [ForestGreen] at (0,-0.6) {\scriptsize $I_3$};
\node [blue] at (-0.35,-1) {\scriptsize $I_1$};
\node [red] at (0.35,-1) {\scriptsize $I_2$};

\node[black] at (-1.5,-.95) {\scriptsize $p_1$};
\node[black] at (-.73,-.95) {\scriptsize $p_2$};

\node[black] at (1.5,-.95) {\scriptsize $p_2$};
\node[black] at (.73,-.95) {\scriptsize $p_1$};
\end{tikzpicture}
\caption{$\Max$ with $\alr$}
\label{fig:shuffle-b}
\end{subfigure}
\caption{Equivalent $\alr$ game using $\salr$ for game without $\alr$ }
\label{fig:shuffle}
\end{figure}


\paragraph*{Shuffled A-loss recall}
The game-structure in \cref{fig:shuffle-a} does not have A-loss recall. This is because the player knows about $\chance$ outcomes $I_1$ and $I_2$ which she forgets at $I_3$.  Now, consider the game-structure in~\cref{fig:shuffle-c}, obtained by \emph{shuffling} the actions ($a$ goes above $b$ and $c$). This game-structure has A-loss recall. The crucial observation is that both the game-structures, \cref{fig:shuffle-a} and \cref{fig:shuffle-c}, lead to the same ``leaf monomials'': on assigning variable $x_a$ to an action labeled $a$, the product of the variables along the path to each leaf produces a leaf monomial. For instance, the leaf monomials for the game-structures in ~\cref{fig:shuffle-a} and ~\cref{fig:shuffle-c} respectively are $\{x_ax_b,x_ax_{\bar{b}},x_{\bar{a}}x_b,x_{\bar{a}}x_{\bar{b}},x_ax_c, x_ax_{\bar{c}},x_{\bar{a}}x_c,x_{\bar{a}}x_{\bar{c}} \}$.
We say that the game-structure of ~\cref{fig:shuffle-a} has \emph{shuffled A-loss recall}. Even though the game originally does not have A-loss recall, it can be shuffled in some way to get an A-recall structure. 
Not every game-structure has shuffled A-loss recall.

Our results:
\begin{itemize}
\item We provide a polynomial-time algorithm to identify whether a game-structure has shuffled A-loss recall. If the answer is yes, the algorithm also computes the shuffled game-structure.

\item As a result, we are able to show that one-player shuffled A-loss recall games can be solved in polynomial-time. Similarly, we deduce that two player games between a perfect recall player and a shuffled A-loss recall player can be solved in polynomial-time.
\end{itemize}






%!TEX root = ../main.tex

\begin{figure}
\centering

\begin{subfigure}{.3\columnwidth}
%\centering
\tikzset{
triangle/.style = {regular polygon,regular polygon sides=3,draw,inner sep = 2},
circ/.style = {circle,fill=cyan!10,draw,inner sep = 3},
term/.style = {circle,draw,inner sep = 1.5,fill=black},
sq/.style = {rectangle,fill=gray!20, draw, inner sep = 4}
}

\begin{tikzpicture}[scale=0.85]
\tikzstyle{level 1}=[level distance=9mm,sibling distance = 22mm]
\tikzstyle{level 2}=[level distance=7mm,sibling distance=10mm]
\tikzstyle{level 3}=[level distance=7mm,sibling distance=6mm]
\tikzstyle{level 4}=[level distance=7mm,sibling distance=5mm]

%node (ij) is the j th node in i th level

\begin{scope}[->, >=stealth]
\node (0) [triangle] {}
child {
  node (00) [circ] {}
  child {
    node (000) [circ] {}
    child {
      node (0000) [term, label=below:{\scriptsize $z_1$}] {}
      edge from parent node [left] {\scriptsize $c$}
    }
    child {
      node (0001) [term, label=below:{\scriptsize $z_2$}] {}
      edge from parent node [right] {\scriptsize $\bar{c}$}
      }
    edge from parent node [left] {\scriptsize $a$}
  }
  child {
    node (001) [circ] {}
    child {
      node (0010) [term, label=below:{\scriptsize $z_3$}] {}
      edge from parent node [left] {\scriptsize $d$}
    }
    child {
      node (0011) [term, label=below:{\scriptsize $z_4$}] {}
      edge from parent node [right] {\scriptsize $\bar{d}$}
      }
    edge from parent node [right] {\scriptsize $\bar{a}$} 
  }
  edge from parent node [above] {\scriptsize $p_1$}
}
child {
  node (01) [circ] {}
   child {
     node (010) [circ] {}
     child {
      node (0100) [term, label=below:{\scriptsize $z_5$}] {}
      edge from parent node [left] {\scriptsize $c$}
    }
    child {
      node (0101) [term, label=below:{\scriptsize $z_6$}] {}
      edge from parent node [right] {\scriptsize $\bar{c}$}
      }
    edge from parent node [left] {\scriptsize $b$}
  }
  child {
    node (011) [circ] {}
    child {
      node (0110) [term, label=below:{\scriptsize $z_7$}] {}
      edge from parent node [left] {\scriptsize $d$}
    }
    child {
      node (0111) [term, label=below:{\scriptsize $z_8$}] {}
      edge from parent node [right] {\scriptsize $\bar{d}$}
      }
    edge from parent node [right] {\scriptsize $\bar{b}$} 
  }
  edge from parent node [above] {\scriptsize $p_2$}
}
;
 \node[fit=(00),dashed,thick,blue, draw, circle,inner sep=1pt] {};
  \node[fit=(01),dashed,thick,red, draw, circle,inner sep=1pt] {};
\end{scope}

\draw [dashed, thick, ForestGreen, in=150,out=30] (000) to (010);
\draw [dashed, thick, brown, in=150,out=30] (001) to (011);
\node [black] at (0,0.35) {\scriptsize $r$};
\node [black] at (-1.5,-0.55) {\scriptsize $u_1$};
\node [black] at (1.5, -0.55) {\scriptsize $u_2$};
\node [black] at (-2, -1.6) {\scriptsize $u_3$};
\node [black] at (-.25, -1.7) {\scriptsize $u_4$};

\node [black] at (0.25, -1.7) {\scriptsize $u_5$};
\node [black] at (2, -1.6) {\scriptsize $u_6$};

%obs labels

\node [blue] at (-1.7,-.9) {\scriptsize $I_1$};
\node [red] at (1.7,-.9) {\scriptsize $I_2$};
\node [ForestGreen] at (-0.2,-1) {\scriptsize $I_3$};
\node [brown] at (0.3,-1) {\scriptsize $I_4$};

\end{tikzpicture}
\caption{$\Max$ without $\salr$}
\label{fig:alossSpan-a}
\end{subfigure}

\begin{subfigure}{.6\columnwidth}
%\centering
\tikzset{
triangle/.style = {regular polygon,regular polygon sides=3,draw,inner sep = 2},
circ/.style = {circle,fill=cyan!10,draw,inner sep = 3},
term/.style = {circle,draw,inner sep = 1.5,fill=black},
sq/.style = {rectangle,fill=gray!20, draw, inner sep = 4}
}

\begin{tikzpicture}[scale=0.8]
\tikzstyle{level 1}=[level distance=9mm,sibling distance = 50mm]
\tikzstyle{level 2}=[level distance=5mm,sibling distance=25mm]
\tikzstyle{level 3}=[level distance=9mm,sibling distance=12mm]
\tikzstyle{level 4}=[level distance=10mm,sibling distance=6mm]

%node (ij) is the j th node in i th level

\begin{scope}[->, >=stealth]
\node (0) [circ] {}
child {
  node (00) [circ] {}
  child {
  node (000) [triangle] {}
   child {
     node (0000) [circ] {}
     child {
      node (00000) [term, label=below:{\scriptsize $w_1$}] {}
      edge from parent node [left] {\scriptsize $a$}
    }
    child {
      node (00001) [term, label=below:{\scriptsize $w_2$}] {}
      edge from parent node [right] {\scriptsize $\bar{a}$}
      }
    edge from parent node [left,pos=0.2] {\scriptsize $\frac{1}{2}$}
  }
  child {
    node (0001) [circ] {}
    child {
      node (00010) [term, label=below:{\scriptsize $w_3$}] {}
      edge from parent node [left] {\scriptsize $b$}
    }
    child {
      node (00011) [term, label=below:{\scriptsize $w_4$}] {}
      edge from parent node [right] {\scriptsize $\bar{b}$}
      }
    edge from parent node [right,pos=0.2] {\scriptsize $\frac{1}{2}$} 
     }
  edge from parent node [above] {\scriptsize $d$}
  }
  child {
  node (001) [triangle] {}
   child {
     node (0010) [circ] {}
     child {
      node (00100) [term, label=below:{\scriptsize $w_5$}] {}
      edge from parent node [left] {\scriptsize $a$}
    }
    child {
      node (00101) [term, label=below:{\scriptsize $w_6$}] {}
      edge from parent node [right] {\scriptsize $\bar{a}$}
      }
    edge from parent node [left,pos=0.2] {\scriptsize $\frac{1}{2}$}
  }
  child {
    node (0011) [circ] {}
    child {
      node (00110) [term, label=below:{\scriptsize $w_7$}] {}
      edge from parent node [left] {\scriptsize $b$}
    }
    child {
      node (00111) [term, label=below:{\scriptsize $w_8$}] {}
      edge from parent node [right] {\scriptsize $\bar{b}$}
      }
    edge from parent node [right,pos=0.2] {\scriptsize $\frac{1}{2}$} 
     }
  edge from parent node [above] {\scriptsize $\bar{d}$}
  }
  edge from parent node [above] {\scriptsize $c$}
  }
child {
  node (01) [circ] {}
  child {
  node (010) [triangle] {}
   child {
     node (0100) [circ] {}
     child {
      node (01000) [term, label=below:{\scriptsize $w_9$}] {}
      edge from parent node [left] {\scriptsize $a$}
    }
    child {
      node (01001) [term, label=below:{\scriptsize $w_{10}$}] {}
      edge from parent node [right] {\scriptsize $\bar{a}$}
      }
    edge from parent node [left,pos=0.2] {\scriptsize $\frac{1}{2}$}
  }
  child {
    node (0101) [circ] {}
    child {
      node (01010) [term, label=below:{\scriptsize $w_{11}$}] {}
      edge from parent node [left] {\scriptsize $b$}
    }
    child {
      node (01011) [term, label=below:{\scriptsize $w_{12}$}] {}
      edge from parent node [right] {\scriptsize $\bar{b}$}
      }
    edge from parent node [right,pos=0.2] {\scriptsize $\frac{1}{2}$} 
  }
  edge from parent node [above] {\scriptsize $d$}
}
  child {
  node (011) [triangle] {}
   child {
     node (0110) [circ] {}
     child {
      node (01100) [term, label=below:{\scriptsize $w_{13}$}] {}
      edge from parent node [left] {\scriptsize $a$}
    }
    child {
      node (01101) [term, label=below:{\scriptsize $w_{14}$}] {}
      edge from parent node [right] {\scriptsize $\bar{a}$}
      }
    edge from parent node [left,pos=0.2] {\scriptsize $\frac{1}{2}$}
  }
  child {
    node (0111) [circ] {}
    child {
      node (01110) [term, label=below:{\scriptsize $w_{15}$}] {}
      edge from parent node [left] {\scriptsize $b$}
    }
    child {
      node (01111) [term, label=below:{\scriptsize $w_{16}$}] {}
      edge from parent node [right] {\scriptsize $\bar{b}$}
      }
    edge from parent node [right,pos=0.2,pos=0.2] {\scriptsize $\frac{1}{2}$} 
     }
  edge from parent node [above] {\scriptsize $\bar{d}$}
  }
  edge from parent node [above] {\scriptsize $\bar{c}$}
}
;
\end{scope}

%\draw [dashed, thick, ForestGreen, in=150,out=30] (00) to (01);

\node[fit=(0),dashed,thick,ForestGreen, draw, circle,inner sep=1pt] {};
\draw [dashed, thick, brown, in=165,out=15] (00) to (01);
\draw [dashed, thick, blue, in=150,out=30] (0000) to (0010);
\draw [dashed, thick, blue, in=150,out=30] (0010) to (0100);
\draw [dashed, thick, blue, in=150,out=30] (0100) to (0110);
\draw [dashed, thick, red, in=150,out=30] (0001) to (0011);
\draw [dashed, thick, red, in=150,out=30] (0101) to (0111);
\draw [dashed, thick, red, in=150,out=30] (0011) to (0101);

%\node [black] at (0,0.45) {\scriptsize $r$};
%\node [black] at (-1.1,-0.45) {\scriptsize $u_1$};
%\node [black] at (1.1, -0.45) {\scriptsize $u_2$};
%\node [black] at (-2, -1.65) {\scriptsize $u_3$};
%\node [black] at (-.2, -1.65) {\scriptsize $u_4$};
%
%\node [black] at (0.25, -1.65) {\scriptsize $u_5$};
%\node [black] at (2, -1.65) {\scriptsize $u_6$};

%obs labels
\node [ForestGreen] at (0.55,0.1) {\scriptsize $I_3$};
\node [brown] at (0,-.8) {\scriptsize $I_4$};
\node [blue] at (-2.8,-1.6) {\scriptsize $I_1$};
\node [red] at (2.8,-1.6) {\scriptsize $I_2$};

%\node[black] at (-1.5,-.95) {\scriptsize $p_1$};
%\node[black] at (-.73,-.95) {\scriptsize $p_2$};

%\node[black] at (1.5,-.95) {\scriptsize $p_2$};
%\node[black] at (.73,-.95) {\scriptsize $p_1$};
\end{tikzpicture}
\caption{$\Max$ with $\alr$}
\label{fig:alossSpan-b}
\end{subfigure}
\caption{Equivalent $\alr$ game using $\alr$-span for game without $\salr$}
\label{fig:span}
\end{figure}


\paragraph*{Span}
We move on to another way of simplifying game-structures. The game-structure $\Tt_1$ \cref{fig:alossSpan-a} neither has perfect recall, nor A-loss recall. Using the characterization obtained in Section~\ref{}, we can show that it does not have  shuffled A-loss recall either. Now, consider the game-structure $\Tt'_1$ in \cref{fig:alossSpan-c}. It has A-loss recall. Each leaf monomial of $\Tt_1$ can be written as a linear combination of the monomials of $\Tt'_1$: for example, the leaf monomial $x_ax_{\bar{b}}$ or $\Tt_1$ is equal to $x_a x_{\bar{b}}x_c + x_a x_{\bar{b} \bar{c}}$, the sum of two leaf monomials of $\Tt'_1$.  The game-structure $\Tt_1$ is said to be \emph{spanned by} $\Tt'_1$. This property allows to solve games derived from the structure $\Tt_1$ by converting them into a game on $\Tt'_1$, and solving the resulting A-loss recall game. 
Our results:
\begin{itemize}
\item We show that every imperfect recall game without absent-mindedness~\cite{} is spanned by an A-loss recall game.

\item The caveat is that the smallest A-loss recall span may be of exponential size: we exhibit a family of game structures where this happens. 

\item Finally, we provide an algorithm to compute an A-loss recall span of smallest size. 
\end{itemize}

From a conceptual point of view, we provide the following novel outlook.
\begin{itemize}
	\item Solving every non-absentminded game is \emph{equivalent} to solving an A-loss recall game. 
\end{itemize}

Recall that imperfect recall games are $\NP$-hard in general. The above results show that in order to solve an imperfect recall game, one could either use an exponential-time algorithm on the game directly, or apply the above transformation into a potentially exponential-sized game, on which a polynomial-time algorithm can be used. 








\section{Related Work}

We review related literature on pre-training, cross-domain transfer learning, and multi-domain pre-training for graph data.

\stitle{Graph pre-training.}
Graph pre-training methods aim to extract inherent properties of graphs, often utilizing self-supervised learning approaches, which can be either generative \cite{hu2020gpt,li2023s,hou2022graphmae,jiang2023incomplete} or contrastive \cite{velivckovic2018deep,xia2022simgrace,xu2021self,li2022mining}. The pre-trained model is then employed to address downstream tasks through fine-tuning \cite{you2020graph,velivckovic2018deep,qiu2020gcc} or parameter-efficient adaptation methods, notably prompt-based learning \cite{sun2022gppt,liu2023graphprompt,yu2023generalized,fang2022universal}. However, these methods typically assume that the pre-training and downstream graphs originate from the same domain, such as different subgraphs of a large graph \cite{you2020graph,yu2023hgprompt} or collections of similar graphs within the same dataset \cite{hu2020gpt,qiu2020gcc}, failing to account for multiple domains in either pre-training or downstream graphs.

\stitle{Graph cross-domain transfer.}
This line of work aims to transfer single-source domain knowledge to a different target domain by leveraging domain-invariant properties across domains~\cite{ding2021cross,hassani2022cross,wang2021pre,wang2023cross}. However, they rely exclusively on a single source domain, failing to harness the extensive knowledge available across multiple domains. Additionally, these approaches are often tailored to specific tasks or domains \cite{ding2021cross,hassani2022cross,wang2021pre,wang2023cross}, limiting their generalization.

\stitle{Multi-domain graph pre-training.}
In the context of graphs from multiple domains, recent works \cite{liu2023one,tang2024higpt,xia2024opengraph} utilize large language models to align node features from different domains through textual descriptions, thereby limiting their applicability to text-attributed graphs \cite{zhaolearning,wen2023prompt,zhang2024text}. For graphs without textual attributes, GraphControl \cite{zhu2024graphcontrol} applies ControlNet~\cite{zhang2023adding} to incorporate target domain node features with the pre-trained model, while neglecting the alignment among multiple source domains. Another recent study proposes GCOPE~\cite{zhao2024all}, which employs domain-specific virtual nodes that interconnect nodes across domains, facilitating the alignment of feature distribution and homophily patterns. Meanwhile, MDGPT~\cite{yu2024text} pre-trains domain-specific tokens to align feature semantics across various domains. However, these studies do not account for structural variance across different domains, hindering their effectiveness in integrating multi-domain knowledge. On a related front, multi-task pre-training techniques \cite{wang2022multi,yu2023multigprompt} employ pretext tokens for each pre-training task. It is important to note that they address a distinct problem, aiming to overcome potential interference among multiple tasks within a single domain, rather than interference across multiple domains. 
%In our work, we propose structure tokens and dual prompts to overcome the limitations of current multi-domain graph pre-training methods. 
%Note that multi-task pre-training aims to reduce interference among different pre-training tasks within a single domain, distinct from our objective of multi-domain pre-training.


\section{Preliminaries}\label{sec.preliminaries}
\stitle{Graph.}
A graph is defined as \( G = (V, E) \), where \( V \) is the set of nodes and \( E \) is the set of edges. The nodes are associated with a feature matrix $\mathbf{X} \in \mathbb{R}^{|V| \times d}$, where \( \vec{x}_v \in \mathbb{R}^d \) is a row of $\mathbf{X}$ representing the feature vector for node \( v \in V \). For a collection of multiple graphs, we define it as \( \mathcal{G} = \{ G_1, G_2, \dots, G_N \} \).

\begin{figure*}[t]
\centering
\includegraphics[width=1\linewidth]{figures/framework.pdf}
\caption{Overall framework of \model.}
\label{fig.framework}
\end{figure*}


\stitle{Graph encoder.}
Towards graph representation learning, one of the most widely used families of graph encoders is graph neural networks (GNNs), which generally rely on message passing to capture structural knowledge \cite{wu2020comprehensive,zhou2020graph}. Each node updates its representation by aggregating information from its neighbors, and stacking multiple GNN layers enables iterative message propagation across the graph.
Formally, let $\vec{H}^l$ denote the embedding matrix at the $l$-th layer, where each row $\vec{h}_i^l$ represents the embedding of node $v_i$. This matrix is iteratively computed using the embeddings from the preceding layer:
\begin{equation}\label{eq.gnn}
\vec{H}^l = \textsc{MP}(\vec{H}^{l-1},G;\theta^l),
\end{equation}
where $\textsc{MP}(\cdot)$ is the message passing function, and $\theta^l$ represents the learnable parameters of the graph encoder at layer $l$. The initial embedding matrix, $\vec{H}^0$, is the input feature matrix, i.e., $\vec{H}^0=\vec{X}$. The output after a total of $L$ layers is then $\vec{H}^L$; for brevity we simply write $\vec{H}$. We abstract the multi-layer encoding process as 
\begin{align}
    \{\vec{H}^1, \vec{H}^2, \cdots, \vec{H}^L\} = \textsc{GraphEncoder}(\vec{X},G;\Theta),
\end{align}
where $\{\vec{H}^1, \vec{H}^2, \cdots, \vec{H}^L\}$ denotes the embedding matrix of the each layer of the graph encoder, respectively. $\Theta=(\theta^1,\ldots,\theta^L)$ is the collection of weights across the layers.

\stitle{Pre-training.}
As stated by prior studies \cite{yu2024generalized,yu2024non}, all contrastive pre-training task on graphs \cite{liu2023graphprompt,velickovic2019deep,you2020graph} can be unified under the task template of similarity calculation.
Formally, the unified pre-training objective is defined as follows:
\begin{equation}\label{eq:generalized_loss}
     \bL(\Theta)= -\sum_{o\in \bT_\text{pre}}\ln\frac{\sum_{a\in Pos_o}\exp(\text{sim}(\vec{h}_{a}, \vec{h}_{o})/\tau)}{\sum_{b\in Neg_o}\exp(\text{sim}(\vec{h}_{b}, \vec{h}_{o})/\tau)},
\end{equation}
where \( Pos_o \) and \( Neg_o \) denote the sets of positive and negative samples for a target instance \( o \), respectively. \( \vec{h}_o \) represents the embedding of the target instance, while \( \vec{h}_a \) and \( \vec{h}_b \) correspond to the embeddings of positive and negative samples. The hyperparameter \( \tau \) controls the temperature scaling in the similarity computation. In our framework, we follow previous work \cite{liu2023graphprompt,yu2024generalized} by employing similarity calculation as the task template and using link prediction as the pre-training task.

\stitle{Problem definition.}
In this work, we explore chain-of-thought (CoT) prompt learning framework for text-free graphs. We focus on two widely used tasks in graph learning: node classification and graph classification, in few-shot scenarios.
For node classification, given a graph \( G = (V, E) \) with a set of node classes \( Y \), each node \( v_i \in V \) is associated with a label \( y_i \in Y \). In contrast, graph classification considers a collection of graphs \( \mathcal{G} \), where each graph \( G_i \in \mathcal{G} \) is assigned a class label \( Y_i \in Y \).
In the few-shot setting, only \( m \) labeled examples per class are available (e.g., \( m \leq 10 \)), a paradigm defined as \( m \)-shot classification \cite{liu2023graphprompt,yu2024generalized}.

\section{Chain-of-Thought Graph Prompt Learning}

In this section, we propose our model, \model, starting with an overview of its framework. Then we detail its core components and conclude with a complexity analysis of the algorithm.

\subsection{Overall framework}
We illustrate the overall framework of \model\ in Fig.~\ref{fig.framework}, consisting of two key stages: pre-training and CoT prompting. 
First, we pre-train a graph encoder as shown in Fig.~\ref{fig.framework}(a). %Specifically, we adopt a universal task template based on similarity calculation \citep{liu2023graphprompt}, which provides a unified framework for diverse graph-based tasks, including node classification and graph classification.
Details of pre-training are provided in Sect.~\ref{sec.preliminaries}.
Second, given a pre-trained graph encoder, to guide the model take additional inference step before finalizing predictions, we propose CoT prompting, as shown in Fig.~\ref{fig.framework}(c). Specifically, we design an inference step with three substages: prompt-based inference, thought construction, and thought-conditioned prompt learning. We first feed the prompt modified query graph into the pre-trained graph encoder. Then we construct a thought by fusing embeddings of all hidden layers of the pre-trained graph encoder. Conditioned on the thought from the previous step, we employ a condition-net to generate a series of node-specific conditional prompts that capture individualized learning patterns for nodes in a fine-grained and parameter-efficient manner, as shown in in Fig.~\ref{fig.framework}(b). We repeat the inference steps until obtain a final answer, as shown in Fig.~\ref{fig.framework}(c). Moreover, to enhance alignment between downstream tasks and the pre-training objective, we incorporate an initial prompt to adjust features or embeddings, same as prior graph prompting methods \cite{liu2023graphprompt,fang2024universal}.

\subsection{Chain-of-Thought Prompting}
To bridge the gap between the objectives between pre-training and downstream tasks, we adhere to previous graph prompting methods, leveraging an initial prompt to modify node features \cite{sun2022gppt,fang2024universal} or output embeddings \cite{liu2023graphprompt,yu2024non}, as shown in Fig.~\ref{fig.framework}(c). Note that our framework is fully compatible with any standard graph prompting techniques and can further enhance their performance, as demonstrated in Sect.~\ref{sec.backbone-flexibility}. In our experiments, we employ an attention mechanism to generate prompts that modify the output embedding of the final step, as detailed in the \textit{Prompt tuning} section.

\stitle{Prompt-based inference}
For the \(k\)-th inference step, we first feed the prompt modified query graph into the pre-trained encoder:
\begin{equation}
    \{\vec{H}_k^1, \vec{H}_k^2, \cdots, \vec{H}_k^L\} = \textsc{GraphEncoder}(\vec{X}_{k},G;\Theta_0)
\end{equation}
where $\Theta_0$ is the pre-trained parameters in the graph encoder, $\vec{X}_{k}$ is the modified feature  by prompts, which will be introduced in the \textit{Thought conditioned prompt learning} section. 

\stitle{Thought construction}
% Previous graph learning methods, whether supervised learning approaches \cite{kipf2016semi,velivckovic2017graph}, pre-training and fine-tuning methods \cite{you2020graph,velickovic2019deep}, or graph prompting methods \cite{liu2023graphprompt,yu2024generalized}, typically produce an answer in a single inference step. Specifically, in the downstream encoding phase, they feed the query graphs into a (pre-trained) graph encoder and leverage adaptation mechanism to generate the final answer. Inspired by CoT prompting in the language domain, we decompose this single inference step into several steps to refine the answer incrementally. Specifically, 
To leverage the hierarchical knowledge across multiple layers of the graph encoder, we construct the thought  by fusing the hidden embeddings from each layer of the pre-trained graph encoder as follows:
\begin{equation}
    \vec{T}_{k} = \mathtt{Fuse}(\vec{H}_k^1, \vec{H}_k^2, \cdots, \vec{H}_k^L),
\end{equation}
where \(\vec{H}_k^l\) denotes the hidden embedding from the \(l\)-th layer during the \(k\)-th inference step. The $\mathtt{Fuse}(\cdot)$ function can be implemented in various type. In our experiments, we employ weighted summation as the fusion function:
\begin{equation}
    \vec{T}_{k} = w^1\cdot\vec{H}_k^1+w^2\cdot\vec{H}_k^2+\cdots+w^L\cdot\vec{H}_k^L,
\end{equation}
where $w^1,w^2,\cdots,w^L$ are learnable parameters. The thought $\vec{T}_{k}$ is then used to guide the the next inference step.

\stitle{Thought conditioned prompt learning}
The thought generated in the previous inference step stores hierarchical structural knowledge, which guides the learning in the subsequent step. Since different nodes may exhibit distinct characteristics with respect to the downstream task, rather than learn all node's representation in the same manner, the pre-trained model should be steered to learn node-specific patterns. To this end, we propose leveraging a condition-net \cite{zhou2022conditional,yu2024dygprompt,yu2024non} to generate node-specific prompts, guiding node-specific adaptation. Specifically, conditioned on the thought from the previous step, \(\vec{T}_{k}\), the condition-net generates a series of thought prompts as follows:
\begin{align}\label{eq.prompt-generation}
    \vec{P}_{k} = \mathtt{CondNet}(\vec{T}_{k}; \phi),
\end{align}
where \(\mathtt{CondNet}\) is the condition-net parameterized by \(\phi\). Note that condition-net is a lightweight hypernetwork \cite{ha2022hypernetworks}, where our condition-net $\mathtt{CondNet}$ functions as an auxiliary network to generate the prompts conditioned on the thoughts parameter-efficiently. In our implementation, we utilize a simple multi-layer perceptron (MLP) with a bottleneck structure for improved efficiency \cite{wu2018reducing}.
The $i$-th row of \(\vec{P}_{k}\) represents a unique node-specific prompt vector \( \vec{p}_{i,k} \) for node $v_i$, which is derived from the thought \(\vec{T}_{k}\). This prompt vector then guides the next inference step by modifying the node features as follows:
\begin{align}\label{eq.thought-prompting}
    \vec{X}_{k+1} = \vec{P}_k \odot \vec{X},
\end{align}
where \(\odot\) denotes element-wise multiplication, and \(\vec{X}_{k+1}\) represents the input to the pre-trained graph encoder in the \(k+1\) step. After repeating \(K\) inference steps, the final step outputs the nodes embeddings \(\vec{H}_{K}\).


\stitle{Prompt tuning}\label{sec.prompt-tuning}
For initial prompts, all standard graph prompting methods can be applied within our framework. In our experiments, we adhere to GPF+ \cite{fang2024universal} to generate initial prompts. Specifically, we train \( N \) bias prompts $\{\vec{p}_\text{bias}^1, \vec{p}_\text{bias}^2, \ldots, \vec{p}_\text{bias}^N\}$, 
and leverage attention-based aggregation to generate node-specific prompts. However, unlike GPF+, which uses these prompts to modify graph features, we use them to modify the final embeddings. Specifically, the task prompt for node \( i \) is computed as:
\begin{equation}
    \vec{p}_\text{init}^i = \sum_{j=1}^{N} \alpha_{i,j} \vec{p}_\text{bias}^j,\quad \text{where} \quad \alpha_{i,j} = \frac{\exp\big(\vec{a}^{j}\vec{h}_{K}^i\big)}{\sum_{l=1}^{N} \exp\big(\vec{a}^{l}\vec{h}_{K}^i\big)}.
\end{equation}
Here, \( \vec{h}_{K}^i \) denotes the \( i \)-th row of the final embedding matrix \( \vec{H}_{K} \), which serves as the embedding for node \( i \), and \(\vec{A}= \{\vec{a}^1, \vec{a}^2, \ldots, \vec{a}^N\} \) are \( N \) learnable linear projection vectors. The generated initial prompts form the matrix \( \vec{P}_\text{init} \), which is then used to modify the final output embeddings as follows:
\begin{equation}\label{eq.initial-prompt}
    \vec{\Tilde{H}} = \vec{P}_\text{init} \odot \vec{H}_{K}.
\end{equation}
Note that for clarity and to maintain a unified representation of the various task prompt mechanisms, we use \(\vec{P}_\text{init}\) to denote the trainable parameters associated with these mechanisms throughout the remainder of this paper.


Finally, for a given task with a labeled training set 
\[
\mathcal{D}_t = \{(x_1, y_1), (x_2, y_2), \dots\},
\]
where each \( x_i \) represents either a node or a graph and \( y_i \in Y \) denotes its corresponding class label, the downstream loss function is defined as:
\begin{align}\label{eq.prompt-loss}
    \bL_{\text{down}}(\phi,\vec{P}_\text{init}) = -\sum_{(x_i, y_i) \in \mathcal{D}_t} \ln \frac{\exp\left(\text{sim}\left(\vec{\tilde{h}}_{x_i}, \vec{\bar{h}}_{y_i}\right)/\tau\right)}{\sum_{c \in Y} \exp\left( \text{sim}\left(\vec{\tilde{h}}_{x_i}, \vec{\bar{h}}_{c}/\right)\tau\right)},
\end{align}
where \( \vec{\tilde{h}}_{x_i} \) represents the final embedding of a node \( v \) or a graph \( G \). Specifically, for node classification, \( \vec{\tilde{h}}_{v} \) corresponds to a row in the answer matrix \( \vec{\tilde{H}} \), while for graph classification, we compute the graph embedding as
\[
    \vec{\tilde{h}}_{G} = \sum_{v \in V} \vec{\tilde{h}}_{v},
\]
incorporating an additional graph readout step.
The prototype embedding for each class \( c \), denoted as \( \vec{\bar{h}}_{c} \), is obtained by averaging the embeddings of all labeled nodes or graphs belonging to that class.
During prompt tuning, only the task prompts and lightweight parameters of the condition network (\( \phi \)) are updated, while the pre-trained GNN weights remain frozen. This parameter-efficient design makes our approach well-suited for few-shot learning, where the training set \( \mathcal{D}_t \) contains only a limited number of labeled examples.

\subsection{Algorithm and Complexity Analysis}
\stitle{Algorithm.}
We detail the main steps for Chain-of-Thought graph prompting in Algorithm~\ref{alg.prompt}. In lines 3--12, we iterate through \( K \) inference steps during the downstream adaptation phase while keeping the pre-trained weights \( \Theta_0 \) frozen. Specifically, in lines 4--7, we generate the thought for the \( k \)-th inference step by fusing the output embeddings from each layer of the pre-trained graph encoder. In lines 8--12, we leverage this thought to generate node-specific prompts that guide the subsequent inference step to capture node-specific patterns. In lines 13--14, we employ a standard task prompt to modify the output embedding of the last inference step. Note that, based on the task prompt mechanism described in the original paper, these prompts can also be applied to modify the input features at the first inference step. Finally, in lines 15--17, we update the embeddings for the prototypical nodes/graphs based on the few-shot labeled data.


\begin{algorithm}[tbp]
\small
\caption{\textsc{Chain-of-Thought Graph Prompt Learning}}
\label{alg.prompt}
\begin{algorithmic}[1]
    \Require Pre-trained graph encoder with parameters $\Theta_0$.
    \Ensure Optimized parameters $\phi$ of condition-net, and task prompt $\vec{P}_{\text{init}}$.
    \While{not converged} 
        \State $\phi_i \leftarrow$ initialization
        % \State \slash* Task prompt modify the input *\slash
        % \State \vec{\tilde{X}}\leftarrow \vec{p}_{\text{task}}\odot X
        % \State $\vec{H}_k^1, \vec{H}_k^2, \cdots, \vec{H}_k^L \leftarrow \textsc{GraphEncoder}(G,X;\Theta_0)$ 
        \State $\vec{X}_1\leftarrow \vec{X}$
        \State \slash* 1 to $K-1$ inference steps*\slash
        \While{inference step $1\leq k< K$}
            \State \slash* Encoding by the pre-trained graph encoder *\slash
            \State $\vec{H}_k^1, \vec{H}_k^2, \cdots, \vec{H}_k^L \leftarrow \textsc{GraphEncoder}(G,\vec{X}_k;\Theta_0)$
            \State \slash* Thought construction *\slash
            \State $\vec{T}_{k} \leftarrow \mathtt{Fuse}(\vec{H}_k^1, \vec{H}_k^2, \cdots, \vec{H}_k^L)$
            \State \slash* Thought prompting *\slash
            \State \slash*  Generate thought prompts by Eq.~\ref{eq.prompt-generation} *\slash
            \State $\vec{P}_{k} \leftarrow \mathtt{CondNet}(\vec{T}_{k}; \phi)$
            \State \slash* Thought prompts modification by Eq.~\ref{eq.thought-prompting} *\slash
            \State $\vec{X}_{k+1} \leftarrow \vec{P}_{k} \odot \vec{X}$
        \EndWhile
            \State \slash* $K$-th inference step *\slash
            \State $\{\vec{H}_K^1, \vec{H}_K^2, \cdots, \vec{H}_K^L\} \leftarrow \textsc{GraphEncoder}(G,\vec{X}_K;\Theta_0)$
            \State $\vec{H}_{K}\leftarrow \vec{H}_K^L$
            \State \slash* Initial prompt modification by Eq.~\ref{eq.initial-prompt}*\slash
            \State $\vec{\tilde{H}} \leftarrow \vec{P}_\text{init} \odot \vec{H}_{K}$
            \State \slash* Update prototypical instance *\slash
            \For{each class $c$} 
                \State ${\vec{\bar{h}}}_{c} \leftarrow \textsc{Average}(\vec{\tilde{h}}_{x}$: instance $x$ belongs to class $c$)
            \EndFor
            \State \slash* Optimizing the parameters in condition-net *\slash
            \State Calculate $\bL_\text{down}(\phi,\vec{P}_\text{init})$ by Eq.~\eqref{eq.prompt-loss}
            \State Update $\phi$ by backpropagating  $\bL_\text{down}(\phi,\vec{P}_\text{init})$
        \EndWhile    
    \State \Return $\{\phi,\vec{P}_\text{task}\}$
\end{algorithmic}
\end{algorithm}

\stitle{Complexity analysis.}
For a downstream graph \( G \), we perform \( K \) iterative inference steps. Each step comprises two key components: (1) thought generation using a pre-trained GNN (executed for \( K \) iterations) and (2) thought-based conditional prompt learning (executed for \( K-1 \) iterations).
The computational complexity of the first component is largely determined by the GNN architecture. In a standard GNN, each node aggregates information from up to \( D \) neighboring nodes per layer. Consequently, computing node embeddings over \( L \) layers incurs a complexity of \( O(D^L \cdot |V|) \), where \( |V| \) denotes the number of nodes. Aggregating the outputs from all \( L \) layers adds an extra complexity of \( O(L\cdot |V|) \). Since the thought generation process is repeated \( K \) times, its overall complexity is:$O(K\cdot (D^L+L \cdot |V| ))$.
The second component, thought-based conditional prompt learning, refines the learning process by leveraging the thought representation from the previous step. This component consists of two stages: prompt generation and prompt tuning. In the prompt generation stage, the thought is fed into the condition-net, with a complexity of \( O(|V|) \). In the prompt tuning stage, each node is adjusted via a prompt vector, also incurring a complexity of \( O(|V|) \). Since this process is executed for \( K-1 \) iterations, the total complexity for this phase is: $O(2(K-1) \cdot |V|)$.
Additionally, we employ a standard task prompt to modify node features or embeddings. The complexity of this process depends on the specific prompting mechanism. Here, we assign it a complexity of \(O(|V|)\), which is common among graph prompting methods \cite{liu2023graphprompt,fang2024universal}.
In summary, the overall computational complexity of \model\ is: $O((D^L+L + 2(K-1) + 1) \cdot |V|)$.
% Given that both \( L \) and \( K \) are small constants, the two computational components exhibit comparable complexity.

\subsection{Why \model\ Works}
% To explain why \model\ works, we compare it with CoT prompting in NLP and standard graph prompting methods, as shown in Table~\ref{table.compariion}. 
In \model, we adapt the following mechanism. (1) All three approaches leverage a universal task template to unify pre-training and downstream tasks, ensuring that \model\ can efficiently adapt to different downstream tasks, particularly in few-shot settings. (2) The chain-of-thought mechanism enables \model\ to perform multiple inference steps, thereby refining the final answer more effectively. (3) The thoughts in \model\ fuse hierarchical topological knowledge from graphs, allowing it to better capture structural information. (4) Conditioned on the thought from the previous step, \model\ generates a series of node-specific prompts that efficiently capture fine-grained node characteristics, thus effectively refining answer step by step. The combination of above mechanisms ensures the effectiveness of \model.



% \begin{table}[tbp]
% \center
% \small
% \addtolength{\tabcolsep}{-1mm}
% \caption{Comparison among CoT prompting in NLP, standard graph prompting, and \model. 
% \label{table.compariion}}
% \resizebox{0.6\columnwidth}{!}{%
% \begin{tabular}{@{}c|ccc@{}}
% \toprule
%    Component &\makecell{CoT \\in NLP} & \makecell{Standard\\GP} &\makecell{\model}\\
% \midrule
%      Task prompts & $\checkmark$ & $\checkmark$ & $\checkmark$ \\ 
%      Chain-of-thoughts & $\checkmark$ & $\times$ & $\checkmark$ \\ 
%      Thought prompts & $\times$ & $\times$ & $\checkmark$ \\
%  \bottomrule
% \end{tabular}}
% \end{table}

\section{Experiments}


The \textbf{MEDQA} dataset is a free-form, multiple-choice open-domain \gls{qa} data set specifically designed for medical \gls{qa}. Derived from professional medical board exams, this dataset presents a significant challenge as it requires both the retrieval of relevant evidence and sophisticated reasoning to answer questions accurately. Each question is accompanied by multiple-choice answers that demand a deep understanding of medical concepts and logical inference, often relying on evidence found in medical textbooks. For this study, the test partition of the MEDQA dataset, comprising approximately 1,200 samples, was used \cite{jin2021disease}.

The \textbf{MedMCQA} dataset is another multiple-choice question-answering dataset tailored for medical \gls{qa}. Unlike MEDQA, which is derived from board exam questions, MedMCQA offers a broader variety of question types, encompassing both foundational and clinical knowledge across diverse medical specialties. In this study, the MedMCQA development set, containing approximately 4,000 questions, was used to benchmark against other models \cite{pmlr-v174-pal22a}.



This study employed the MEDQA and MedMCQA datasets to benchmark and evaluate medical \gls{qa} systems. These datasets serve as challenging testbeds for open-domain \gls{qa} tasks due to their demands for multi-hop reasoning and the integration of domain-specific knowledge. The relevance of MEDQA in the real world, together with the diverse question styles and extensive development set of MedMCQA make them ideal for advancing the development of robust \gls{qa} models capable of addressing medical inquiries. We utilize \textit{GPT-4o-mini} as the backbone of the implementation for both \gls{mkg} and \gls{myrag}, leveraging its capabilities with approximately \(\sim 8B\) parameters. This model serves as the core component, enabling advanced reasoning, \gls{rag}, and structured knowledge integration.


\subsection{Medical Knowledge Graph}
The \gls{kg} was dynamically constructed by integrating search items, contextual information, and relationships derived from textbooks and search queries from the PubMed engine for each question in the dataset. This data was processed and stored in a Neo4j database. The key features of the knowledge graph include:

\begin{enumerate}
    \item \textbf{Dynamic Node and Relationship Creation}: Nodes are dynamically generated for search items, and relationships between these nodes are established based on their relevance and predefined relationship types.
    
    \item \textbf{Bidirectional Relationships}: To ensure a comprehensive representation, the graph includes both forward and reverse relationships between nodes, enhancing its utility for diverse queries.
    
    \item \textbf{Relevance Scoring}: Each relationship is enriched with descriptive annotations and a confidence score, quantifying the strength of the association and aiding in prioritizing relevant connections.
    
    \item \textbf{Summarization}: Concise summaries for each search item are included, derived from contextual data. A confidence score accompanies each summary to indicate its reliability.
    
    \item \textbf{Integration with Neo4j}: The entire graph is stored in a Neo4j database, leveraging its graph-based query capabilities for efficient analysis and retrieval.
\end{enumerate}

A snapshot of a portion of the knowledge graph is shown in Figure \ref{fig:model_schema}.B, illustrating its structure and relationships.

This \gls{mkg} serves as the foundational information source for the \gls{myrag} framework during the inference phase. The evaluation of \gls{mkg} confirmed its robustness and reliability, with experts \glspl{llm} such as GPT-4 achieving high precision (e.g. $~$9/10). These results underscore the effectiveness of \gls{mkg} in supporting medical reasoning and decision-making, as detailed in Appendix \ref{app:mkd-analysis}.

\subsection{Performance Comparison}
\begin{table*}[h!]
\small
\centering
\caption{Comparison of LLM models on the MEDQA Benchmark.}
\label{tab:medqa_comparison}
\renewcommand{\arraystretch}{0.9}
\resizebox{\textwidth}{!}{%
\begin{tabular}{@{}lcccccc@{}}
\toprule
\textbf{Model}            & \textbf{Model Size} & \textbf{Acc. (\%)} & \textbf{F1 (\%)} & \textbf{Fine-Tuned} & \textbf{Uses CoT} & \textbf{Uses Search} \\
\midrule
Med-Gemini \cite{saab2024capabilities}               & $\sim$1800B                                     & 91.1                   & 89.5                   & \yesmarker          & \yesmarker        & \yesmarker          \\
GPT-4 \cite{nori2023can}                     & $\sim$1760B                                     & 90.2                   & 88.7                   & \yesmarker          & \yesmarker        & \yesmarker          \\
Med-PaLM 2 \cite{singhal2025toward}               & $\sim$340B                                     & 85.4                   & 82.1                   & \yesmarker          & \yesmarker        & \nomarker           \\
Med-PaLM 2 (5-shot)       & $\sim$340B                                     & 79.7                   & 75.3                   & \nomarker           & \yesmarker        & \nomarker           \\
\gls{myrag}                 & $\sim$8B                                     & 73.9                   & 74.1                   & \nomarker          & \yesmarker        & \yesmarker           \\
Meerkat\cite{kim2024small}              & 7B                                       & 74.3                   & 70.4                   & \yesmarker          & \yesmarker        & \nomarker           \\
Meditron \cite{chen2023meditron}                 & 70B                                      & 70.2                   & 68.3                   & \yesmarker          & \yesmarker        & \yesmarker          \\
Flan-PaLM  \cite{singhal2023large}               & 540B                                     & 67.6                   & 65.0                   & \yesmarker          & \yesmarker        & \nomarker           \\
LLAMA-2 \cite{chen2023meditron}                  & 70B                                      & 61.5                   & 60.2                   & \yesmarker          & \yesmarker        & \nomarker           \\
Shakti-LLM \cite{shakhadri2024shakti}               & 2.5B                                     & 60.3                   & 58.9                   & \yesmarker          & \nomarker         & \nomarker           \\
Codex 5-shot CoT  \cite{lievin2024can}        & --                                     & 60.2                   & 57.7                   & \nomarker           & \yesmarker        & \yesmarker          \\
BioMedGPT  \cite{luo2023biomedgpt}               & 10B                                      & 50.4                   & 48.7                   & \yesmarker          & \nomarker         & \nomarker           \\
BioLinkBERT (base)  \cite{singhal2023large}      & --                                     & 40.0                   & 38.4                   & \yesmarker          & \nomarker         & \nomarker           \\
\bottomrule
\end{tabular}%
}
\end{table*}

Table~\ref{tab:medqa_comparison} presents a comprehensive comparison of state-of-the-art language models on the MEDQA benchmark. The results highlight the critical role of advanced reasoning strategies in achieving higher performance, such as CoT reasoning, fine-tuning, and the integration of search tools. While larger models like Med-Gemini and GPT-4 achieve the highest accuracy and F1 scores, their performance comes at the cost of significantly larger parameter sizes. These models exemplify the power of scaling combined with sophisticated reasoning and retrieval techniques.

Significantly, \gls{myrag}, despite having just 8 billion parameters, attains an F1 score of 74.1\% on the MEDQA benchmark, surpassing models like Meditron, which possess 70 billion parameters without needing any fine tuning. This highlights \gls{myrag}'s exceptional efficiency and proficiency in utilizing CoT reasoning and external evidence retrieval. The model leverages tools such as PubMedSearch and WikiSearch to dynamically integrate domain-specific knowledge dynamically, thereby improving its ability to address medical questions. Examples of \gls{qa} interactions, including detailed search items and reasoning for question samples, are provided in Appendix \ref{app:examples-medqa}. These examples are organized in Tables \ref{table:search_guidance_1}, \ref{table:search_guidance_2}, \ref{table:search_guidance_3}, and \ref{table:search_guidance_4}, drawn from the MEDQA benchmark.

On the MedMCQA benchmark, as shown in Table~\ref{tab:medmcqa_comparison}, \gls{myrag} achieves an accuracy of 66.34\%, even outperforming larger models like Meditron-70B and better than Codex 5-shot CoT. This result underscores \gls{myrag}'s adaptability and robustness, demonstrating that it can deliver competitive performance even against significantly larger models. Its ability to maintain high accuracy on diverse datasets further highlights the effectiveness of its design, which combines CoT reasoning with structured knowledge graph integration and retrieval mechanisms.



\begin{table}[h!]
\small
\centering
\caption{Comparison of Models on the MedMCQA.}
\label{tab:medmcqa_comparison}
\renewcommand{\arraystretch}{0.9}

\begin{tabular}{@{}lcc@{}}
\toprule
\textbf{Model}                              & \textbf{Model Size}           & \textbf{ Acc. (\%)} \\
\midrule
\gls{myrag}                 & $\sim$8B                                     & \textbf{66.34}\\
Meditron \citep{chen2023meditron}                     & 70B                          & 66.0 \\

Codex 5-shot \citep{lievin2024can}                            & --                           & 59.7 \\
VOD \citep{lievin2023variational}                           & --                           & 58.3 \\
Flan-PaLM \citep{singhal2022large}                        & 540B                         & 57.6 \\
PaLM                       & 540B                         & 54.5 \\

GAL                        & 120B                         & 52.9 \\
% Flan-PaLM                   & 62B                          & 46.2 \\
% PaLM                        & 62B                          & 43.4 \\
PubmedBERT \citep{gu2021domain}                & --                           & 40.0 \\
SciBERT \citep{pal2022medmcqa}              & --                           & 39.0 \\
BioBERT \citep{lee2020biobert}                  & --                           & 38.0 \\
BERT \citep{devlin2018bert}                  & --                           & 35.0 \\
% Flan-PaLM                    & 8B                           & 34.5 \\
% BLOOM                       & --                           & 32.5 \\
% OPT                          & --                           & 29.6 \\
% PaLM                         & 8B                           & 26.7 \\
\bottomrule
\end{tabular}%

\end{table}


Overall, \gls{myrag}'s results on MEDQA and MedMCQA benchmarks solidify its position as a highly efficient and effective model for medical \gls{qa}. By leveraging CoT reasoning, search tools, and external knowledge sources, \gls{myrag} not only closes the gap with much larger models but also sets a new standard for performance among smaller-sized models.
\subsection{Impact of Search Tools and CoT Reasoning on AMG-RAG Performance}

Figure~\ref{fig:cot_kg_comparison} and Table~\ref{tab:cot_kg_metrics} demonstrate the impact of integrating search tools such as PubMedSearch and WikiSearch on the performance of \gls{myrag} when applied to the MEDQA dataset. The inclusion of these search capabilities significantly improves accuracy and F1 scores by providing access to relevant external evidence, which is critical for addressing medical questions. Among the search tools, PubMedSearch outperforms WikiSearch, likely due to its more focused and domain-specific content, which better aligns with the nature of medical \gls{qa} tasks.

Additionally, the impact of CoT reasoning and \gls{mkg} integration on \gls{myrag} performance is highlighted in the same figure and table. The results reveal that the removal of either CoT reasoning or KG integration leads to a substantial drop in accuracy and F1 scores. This underscores the indispensable role of structured reasoning and domain-specific retrieval in enhancing the system’s ability to generate accurate and evidence-backed answers.



\begin{table}[h!]
\small
    \centering
    \caption{Performance metrics for \gls{myrag} model with and without CoT and Knowledge Graph integration with different search tools for MEDQA dataset.}
    \label{tab:cot_kg_metrics}
    \begin{tabular}{lccc}
    \hline
    \textbf{Model}                  & \textbf{Acc. (\%)} & \textbf{F1-Score} & \textbf{Recall} \\
    \hline
    PubMedSearch        & 73.92                  & 0.7410            & 0.7392          \\
    WikiSearch          & 70.62                  & 0.7067            & 0.7062          \\
    No Search           & 67.16                  & 0.6696            & 0.6716          \\

    No Search \& CoT    & 66.69                  & 0.6655            & 0.6669          \\
    \hline
    \end{tabular}
\end{table}
\begin{figure}[t]
    \centering
    \includegraphics[width=\columnwidth]{latex/Figures/wiki_conf_matrix.png}
    \caption{Confusion matrix for \gls{myrag} with and without CoT and Knowledge Graph integration on MEDQA dataset.}
    \label{fig:cot_kg_comparison}
\end{figure}
\subsection{Improving QA in Rapidly Changing Medical Domains}

Figure~\ref{fig:performance_clusters} illustrates the performance of various models across different question domains, including Neurology and Genetics. The \gls{myrag} model consistently outperforms other approaches, showcasing its superior adaptability and robustness in these rapidly evolving and knowledge-intensive fields. This exceptional performance stems from its ability to seamlessly integrate external sources of information and evidence. By leveraging PubMed searches, the \gls{myrag} model dynamically retrieves the latest medical research and continuously updates the \gls{mkg}, ensuring that it remains relevant and up-to-date. This dynamic updating process not only enhances the model's ability to reason across multiple domains but also allows it to address complex, multi-hop questions with greater accuracy and depth.

\begin{figure}[t]
    \centering
    \includegraphics[width=.8\columnwidth]{latex/Figures/compariong_clusters_medqa.png}
    \caption{Performance comparison across different question domains in the Neurology and Genetics fields.}
    \label{fig:performance_clusters}
\end{figure}

\section{Conclusions}
In this paper, we propose \model, the first CoT prompting framework for graphs. We define an inference step with three substages: prompt-based inference, thought construction, and thought conditioned prompt learning. Specifically, we first feed the prompt modified query into the pre-trained encoder, and then construct a thought by fusing the hidden embeddings from each layer of the pre-trained graph encoder. To guide the subsequent inference step, we generate a series of prompts conditioned on the thought from the previous step. By repeating the above inference steps, \model\ obtain the answer. Finally, we conduct extensive experiments on eight public datasets, demonstrating that \model\ significantly outperforms a range of state-of-the-art baselines.

\clearpage
\newpage

% \section*{Acknowledgments}
% This research / project is supported by the Ministry of Education, Singapore, under its Academic Research Fund Tier 2 (Proposal ID: T2EP20122-0041). Any opinions, findings and conclusions or recommendations expressed in this material are those of the author(s) and do not reflect the views of the Ministry of Education, Singapore. 

\bibliographystyle{ACM-Reference-Format}
\bibliography{references}

\clearpage
\newpage
% \newpage
%%
%% If your work has an appendix, this is the place to put it.
\appendix
\section*{Appendices}
\renewcommand\thesubsection{\Alph{subsection}}
\renewcommand\thesubsubsection{\thesubsection.\arabic{subsection}}
\appendix

\section*{Appendix}

\section{Prompts}\label{app:prompts}
\subsection{Textual Description}\label{app:img_to_text_prompt}
\begin{quote}
    {\small
    \texttt{Create a short, descriptive persona for the person in the image. Describe them using only the following details: their age, gender, facial expression or mood, attire, any tools or items they’re holding, their work environment, the nature of their job, and their connection to the area and location. Avoid taking creative liberties beyond these details, only using details that can be inferred from the image, while aiming for a realistic portrayal that gives insight into their daily life, professional dedication, and overall demeanor. For example: Meet a skilled construction worker in his late 30s, living in Sydney, Australia. Every day, he heads out to work in one of the city's bustling urban sites, often with a view of iconic landmarks like the Sydney Opera House and Sydney Harbour Bridge. Outfitted in essential safety gear—a hard hat, reflective vest, and a set of versatile tools—he’s well-prepared for a physically demanding role that demands focus and precision. His job involves a blend of construction and maintenance tasks, requiring him to pay close attention to safety protocols and collaborate with a team. Confident and professional in his work, he takes pride in contributing to the infrastructure and vibrant aesthetic of Sydney, adding to the city’s ever-evolving landscape with each project.
    }}
\end{quote}
    

\begin{table*}[t]
    \centering
    \caption{A complete list of personas annotated for their attribute categories.}
    \label{tab:personalist}
    \resizebox{1.0\linewidth}{!}{
    \begin{tabular}{l c c c c}
        \toprule
        Persona & Age & Gender & Occupation & Location \\
        \midrule
        A 25-year-old female nurse from Toronto & 25-34 & female & healthcare \& education & Strong Developed Economies \\ 
        A 41-year-old female electrician from Sydney & 35-44 & female & manual labor & Strong Developed Economies \\ 
        A 36-year-old male electrician from Houston & 35-44 & male & manual labor & Largest Global Economies \\ 
        A 29-year-old female police officer from New York & 25-34 & female & public safety & Largest Global Economies \\ 
        A 28-year-old female police officer from London & 25-34 & female & public safety & Largest Global Economies \\ 
        A 35-year-old male chef from Paris & 35-44 & male & hospitality & Largest Global Economies \\ 
        A 32-year-old female chef from Rome & 25-34 & female & hospitality & Strong Developed Economies \\ 
        A 50-year-old male farmer from Sao Paulo & 45-54 & male & manual labor & Emerging Markets \\ 
        A 40-year-old female farmer from Nairobi & 35-44 & female & manual labor & Emerging Markets \\ 
        A 27-year-old female mechanic from Berlin & 25-34 & female & manual labor & Largest Global Economies \\ 
        A 28-year-old female pilot from Los Angeles & 25-34 & female & transportation & Largest Global Economies \\ 
        A 28-year-old female pilot from Vancouver & 25-34 & female & transportation & Strong Developed Economies \\ 
        A 60-year-old female carpenter from Rome & 55-64 & female & manual labor & Strong Developed Economies \\ 
        A 45-year-old male carpenter from Auckland & 45-54 & male & manual labor & Emerging Markets \\ 
        A 44-year-old female cashier from Montreal & 35-44 & female & hospitality & Strong Developed Economies \\ 
        A 56-year-old male roofer from Brisbane & 55-64 & male & manual labor & Strong Developed Economies \\ 
        A 30-year-old female garbage collector from Toronto & 25-34 & female & manual labor & Strong Developed Economies \\ 
        A 63-year-old male miner from Johannesburg & 55-64 & male & manual labor & Emerging Markets \\ 
        A 24-year-old female lab technician from Shanghai & 18-24 & female & healthcare \& education & Largest Global Economies \\ 
        A 29-year-old male postal worker from Mexico City & 25-34 & male & transportation & Emerging Markets \\ 
        A 44-year-old female welder from Dubai & 35-44 & female & manual labor & Mid-Sized \& Regional Powers \\ 
        A 54-year-old male librarian from Amsterdam & 45-54 & male & healthcare \& education & Mid-Sized \& Regional Powers \\ 
        A 51-year-old female dentist from Seoul & 45-54 & female & healthcare \& education & Strong Developed Economies \\ 
        A 40-year-old female landscaper from Edinburgh & 35-44 & female & manual labor & Largest Global Economies \\ 
        A 24-year-old male hairdresser from Barcelona & 18-24 & male & hospitality & Strong Developed Economies \\ 
        A 19-year-old male janitor from Stockholm & 18-24 & male & manual labor & Mid-Sized \& Regional Powers \\ 
        A 53-year-old female bus driver from Copenhagen & 45-54 & female & transportation & Mid-Sized \& Regional Powers \\ 
        A 27-year-old female machinist from Frankfurt & 25-34 & female & manual labor & Largest Global Economies \\ 
        A 52-year-old male doctor from Madrid & 45-54 & male & healthcare \& education & Strong Developed Economies \\ 
        A 60-year-old male security guard from Lisbon & 55-64 & male & public safety & Mid-Sized \& Regional Powers \\ 
        A 42-year-old male firefighter from Sao Paulo & 35-44 & male & public safety & Emerging Markets \\ 
        A 36-year-old male pharmacist from Berlin & 35-44 & male & healthcare \& education & Largest Global Economies \\ 
        A 56-year-old female teacher from Melbourne & 55-64 & female & healthcare \& education & Strong Developed Economies \\ 
        A 42-year-old male taxi driver from Hong Kong & 35-44 & male & transportation & Largest Global Economies \\ 
        A 39-year-old female veterinarian from Nairobi & 35-44 & female & healthcare \& education & Emerging Markets \\ 
        A 25-year-old male baker from Lisbon & 25-34 & male & hospitality & Mid-Sized \& Regional Powers \\ 
        A 40-year-old male welder from Moscow & 35-44 & male & manual labor & Mid-Sized \& Regional Powers \\ 
        A 39-year-old male plumber from Melbourne & 35-44 & male & manual labor & Strong Developed Economies \\ 
        A 22-year-old male lab technician from Tokyo & 18-24 & male & healthcare \& education & Largest Global Economies \\ 
        A 20-year-old female security guard from Cape Town & 18-24 & female & public safety & Emerging Markets   \\
        \bottomrule
    \end{tabular}
    }
\end{table*}

\subsection{Effect of Safety Training}
\label{app:safety-training}


In our experiments, we observed that Llama 3.2 90B frequently refused to assume visual personas\footnote{Refusal detection was performed using a fine-tuned \texttt{distilroberta-base} model \citep{distilroberta-base-rejection-v1}}, refusing to engage with 76.7\% of all visual persona prompts (Figure \ref{fig:safety-training}). This behavior can be attributed to an overgeneralization of the model's safety training, as personas can create competing objectives between aligned models' safety measures and instruction-following directives \citep{wei2024jailbroken}. This vulnerability has frequently been exploited in adversarial attacks \citep{ma2024visual}, leading to unsafe outputs even when models assume benign personas \citep{zhao2024bias}. To address this issue, the development of Llama 3 incorporated targeted safety training specifically designed to handle persona-based interactions \citep{grattafiori2024llama3herdmodels}.

\begin{figure}[t]
        \centering
        \includegraphics[width=\linewidth]{refusal-graph.pdf}
        \caption{The rate and number of refusals in response to persona prompts. Llama 3.2 90B shows a strong aversion to multimodal persona prompts, while other models rarely refuse.}
    \label{fig:safety-training}
\end{figure}

\begin{table*}[t]
    \centering
    \caption{Direct testing question list}
    \label{tab:direct_questions_list}
    \resizebox{0.9\textwidth}{!}{
    \begin{tabular}{c|c}
    \toprule
    % \multicolumn{2}{c}{Direct testing}
    Attribute & Direct questions \\
    \midrule
        \multirow{10}{*}{Age} & What age-related milestone are you approaching or have recently celebrated, and how did you celebrate it? \\
            & Which television shows or movies were popular when you were a teenager? \\
            & What life experiences do you consider most defining for your generation? \\
            & What were some common trends or fashions during your college years? \\
            & At what age did you first use the internet regularly, and what activities did you engage in online? \\
            & What age were you when you first experienced a major economic event? \\
            & How old were you when you first started using social media, and which platform did you join first? \\
            & How did people in your age group typically meet and socialize in their younger years? \\
            & What music formats (vinyl, cassettes, CDs, etc.) did you grow up using? \\
            & What historical moments do people slightly older than you remember that you just missed? \\
        \midrule
        \multirow{10}{*}{Location} & 
            What are the top three universities or colleges in your area? \\
            & What is the most visited tourist attraction in your area? \\
            & How does the local climate influence your daily activities and lifestyle in your region? \\
            & What are the most frequented local cuisines where you live? \\
            & What are the main industries driving the economy in your area? \\
            & What natural features (mountains, rivers, coast) shape your local landscape? \\
            & What local sports teams unite your community? \\
            & What's the primary mode of public transportation in your area, if any? \\
            & What are the most popular local festivals or events in your area? \\
            & How has the demographic makeup of your area changed over the past decade? \\
        \midrule
        \multirow{10}{*}{Occupation} & 
            Can you outline your primary responsibilities in your current occupation? \\
            & What specific skills are essential for success in your profession? \\
            & What does a typical workday look like for you? \\
            & How do you stay updated with the latest developments in your industry? \\
            & What tools or technologies do you regularly use in your work? \\
            & What's the most significant change you've witnessed in your industry? \\
            & What emerging trends do you see impacting your profession? \\
            & What advice would you give to someone aspiring to enter your field? \\
            & Which legislation directly impacts the way you perform your job? \\
            & What safety protocols specific to your profession do you follow? \\
        \bottomrule
    \end{tabular}}
\end{table*}

\begin{table*}[t]
    \centering
    \footnotesize
    \caption{Scenarios for situational testing}
    \label{tab:direct_scenarios_list}
    \resizebox{0.9\textwidth}{!}{
    \begin{tabular}{c|p{13cm}}
        \toprule
        \textbf{Attribute} & \textbf{Scenarios} \\
        \midrule
        \multirow{20}{*}{\textbf{Age}} 
            & Your extended family is digitizing old home videos for a reunion. While helping, you discover some footage from your early childhood that needs to be converted. The deadline is next week. You \dots \\
            & During a basement cleanup, you find your old gaming systems and accessories from when you were 12. A local collector has shown interest in purchasing the set. You \dots \\
            & You're coordinating a playlist for your high school reunion after-party. The organizers want music specifically from your graduating years to recreate the atmosphere. You \dots \\
            & Your childhood bank is closing its local branch. They've requested you update your old youth account, which was opened when you first started saving. You \dots \\
            & A museum exhibit about educational evolution needs artifacts from your school years. They're particularly interested in how your grade did research projects, so you \dots \\
            & Your first protest/activism experience has become historically significant. A documentary team wants to understand what motivated your age group's involvement. You talk about \dots \\
            & While discussing financial education, younger relatives ask about your first experiences with digital payment apps and online banking. You say \dots \\
            & A market research team needs insight into how your age group first learned about major global events. They're comparing news consumption across generations. You share \dots \\
            & Your old elementary school is celebrating its anniversary. They're recreating typical classroom setups from different eras, including your years there, so you \dots \\
            & During a job interview, you're asked about your generation's unique approach to work-life balance and career expectations. You describe \dots \\
        \midrule
        \multirow{20}{*}{\textbf{Location}} 
            & A major sporting event has caused unusual traffic patterns downtown. You have an important appointment across the city in 45 minutes, so you \dots \\
            & Construction has blocked your usual route through downtown. You're picking up a friend from the airport in an hour, and they just texted that their flight landed early. To make it on time, you \dots \\
            & The biggest annual cultural celebration in your region coincides with a business meeting. International colleagues want to experience local traditions. You \dots \\
            & Severe weather typical for your region has disrupted normal operations. You need to help visitors understand local emergency procedures. You \dots \\
            & A local food shortage has affected typical ingredients in your area. You need to adapt traditional recipes for an important family gathering. You \dots \\
            & Regional housing policies have changed, affecting your rental situation. You need to explain local housing norms to potential roommates. Breaking it down, you explain \dots \\
            & A community center needs help designing weather-appropriate outdoor activities for children new to the area. To help, you \dots \\
            & Your area's unique rush hour patterns are affecting a planned event. You need to coordinate arrival times for out-of-town guests. You \dots \\
            & Local religious or cultural observations are affecting business hours. You need to plan essential errands accordingly. You \dots \\
            & A seasonal health advisory typical for your region has been issued. You need to adjust your outdoor workout routine. \\
        \midrule
        \multirow{20}{*}{\textbf{Occupation}} 
            & During a casual conversation at dinner, your aunt mentions an ongoing situation that raises red flags based on your background and training. You \dots \\
            & A friend's child is working on a school project related to your profession. They need help understanding basic industry concepts. To assist, you \dots \\
            & During a home renovation, you notice issues that relate to your professional expertise. The contractors seem unaware of potential complications. You \dots \\
            & A community workshop needs professionals to demonstrate how their job impacts daily life. Your industry's perspective would fill a key gap. You \dots \\
            & A community Facebook group is sharing advice that conflicts with principles you work with daily, so you \dots \\
            & A local news story misrepresents aspects of your industry. You have an opportunity to provide clarification at a community meeting. At the meeting, you \dots \\
            & Your hobby group encounters a challenge that relates to your professional expertise. They're unsure about proper procedures. You demonstrate \dots \\
            & A neighbor's insurance claim involves aspects of your profession. They're asking for general guidance about standard practices. \\
            & During a social event, you notice concerning practices related to your industry's safety standards. Others seem unaware of the risks, so you \dots \\
            & A local youth program needs career mentors. They want professionals to share how their industry handles modern challenges. You \dots \\
        \bottomrule
    \end{tabular}}
\end{table*}

\begin{table*}[t]
\centering
\small
\begin{tabular}{lcccc}
\toprule
\rowcolor{gray!25}
\multicolumn{5}{c}{\textbf{GPT-4o}} \\
\textbf{Modality} & \textbf{Linguistic Habits} & \textbf{Persona Consistency} & \textbf{Expected Action} & \textbf{Action Justification} \\
\midrule
\textbf{Text} & \begin{tabular}{@{}c@{}}1.68 $\pm$ {\scriptsize 0.04} \\ {\scriptsize (95\% CI: 1.61--1.75)}\end{tabular} & \begin{tabular}{@{}c@{}}3.00 $\pm$ {\scriptsize 0.06} \\ {\scriptsize (95\% CI: 2.87--3.12)}\end{tabular} & \begin{tabular}{@{}c@{}}3.25 $\pm$ {\scriptsize 0.05} \\ {\scriptsize (95\% CI: 3.16--3.34)}\end{tabular} & \begin{tabular}{@{}c@{}}3.91 $\pm$ {\scriptsize 0.04} \\ {\scriptsize (95\% CI: 3.83--3.99)}\end{tabular} \\
\textbf{Assisted Image} & \begin{tabular}{@{}c@{}}1.22 $\pm$ {\scriptsize 0.03} \\ {\scriptsize (95\% CI: 1.16--1.27)}\end{tabular} & \begin{tabular}{@{}c@{}}2.89 $\pm$ {\scriptsize 0.06} \\ {\scriptsize (95\% CI: 2.77--3.01)}\end{tabular} & \begin{tabular}{@{}c@{}}2.83 $\pm$ {\scriptsize 0.05} \\ {\scriptsize (95\% CI: 2.74--2.93)}\end{tabular} & \begin{tabular}{@{}c@{}}3.60 $\pm$ {\scriptsize 0.04} \\ {\scriptsize (95\% CI: 3.52--3.68)}\end{tabular} \\
\textbf{Image} & \begin{tabular}{@{}c@{}}1.05 $\pm$ {\scriptsize 0.02} \\ {\scriptsize (95\% CI: 1.00--1.10)}\end{tabular} & \begin{tabular}{@{}c@{}}2.70 $\pm$ {\scriptsize 0.06} \\ {\scriptsize (95\% CI: 2.58--2.82)}\end{tabular} & \begin{tabular}{@{}c@{}}2.75 $\pm$ {\scriptsize 0.05} \\ {\scriptsize (95\% CI: 2.66--2.84)}\end{tabular} & \begin{tabular}{@{}c@{}}3.56 $\pm$ {\scriptsize 0.04} \\ {\scriptsize (95\% CI: 3.48--3.64)}\end{tabular} \\
\textbf{Descriptive Image} & \begin{tabular}{@{}c@{}}1.17 $\pm$ {\scriptsize 0.03} \\ {\scriptsize (95\% CI: 1.12--1.23)}\end{tabular} & \begin{tabular}{@{}c@{}}3.67 $\pm$ {\scriptsize 0.06} \\ {\scriptsize (95\% CI: 3.56--3.79)}\end{tabular} & \begin{tabular}{@{}c@{}}3.26 $\pm$ {\scriptsize 0.05} \\ {\scriptsize (95\% CI: 3.17--3.35)}\end{tabular} & \begin{tabular}{@{}c@{}}3.87 $\pm$ {\scriptsize 0.04} \\ {\scriptsize (95\% CI: 3.79--3.95)}\end{tabular} \\
\midrule
\rowcolor{gray!25}
\multicolumn{5}{c}{\textbf{GPT-4o-mini}} \\
\textbf{Modality} & \textbf{Linguistic Habits} & \textbf{Persona Consistency} & \textbf{Expected Action} & \textbf{Action Justification} \\
\midrule
\textbf{Text} & \begin{tabular}{@{}c@{}}1.32 $\pm$ {\scriptsize 0.04} \\ {\scriptsize (95\% CI: 1.25--1.39)}\end{tabular} & \begin{tabular}{@{}c@{}}1.95 $\pm$ {\scriptsize 0.07} \\ {\scriptsize (95\% CI: 1.82--2.08)}\end{tabular} & \begin{tabular}{@{}c@{}}2.02 $\pm$ {\scriptsize 0.05} \\ {\scriptsize (95\% CI: 1.93--2.12)}\end{tabular} & \begin{tabular}{@{}c@{}}2.78 $\pm$ {\scriptsize 0.05} \\ {\scriptsize (95\% CI: 2.68--2.88)}\end{tabular} \\
\textbf{Assisted Image} & \begin{tabular}{@{}c@{}}1.17 $\pm$ {\scriptsize 0.03} \\ {\scriptsize (95\% CI: 1.11--1.23)}\end{tabular} & \begin{tabular}{@{}c@{}}2.17 $\pm$ {\scriptsize 0.06} \\ {\scriptsize (95\% CI: 2.04--2.30)}\end{tabular} & \begin{tabular}{@{}c@{}}2.16 $\pm$ {\scriptsize 0.05} \\ {\scriptsize (95\% CI: 2.06--2.25)}\end{tabular} & \begin{tabular}{@{}c@{}}2.88 $\pm$ {\scriptsize 0.05} \\ {\scriptsize (95\% CI: 2.78--2.97)}\end{tabular} \\
\textbf{Image} & \begin{tabular}{@{}c@{}}0.93 $\pm$ {\scriptsize 0.03} \\ {\scriptsize (95\% CI: 0.88--0.99)}\end{tabular} & \begin{tabular}{@{}c@{}}2.11 $\pm$ {\scriptsize 0.06} \\ {\scriptsize (95\% CI: 1.98--2.23)}\end{tabular} & \begin{tabular}{@{}c@{}}1.94 $\pm$ {\scriptsize 0.05} \\ {\scriptsize (95\% CI: 1.85--2.04)}\end{tabular} & \begin{tabular}{@{}c@{}}2.69 $\pm$ {\scriptsize 0.05} \\ {\scriptsize (95\% CI: 2.59--2.78)}\end{tabular} \\
\textbf{Descriptive Image} & \begin{tabular}{@{}c@{}}1.11 $\pm$ {\scriptsize 0.03} \\ {\scriptsize (95\% CI: 1.05--1.17)}\end{tabular} & \begin{tabular}{@{}c@{}}2.68 $\pm$ {\scriptsize 0.07} \\ {\scriptsize (95\% CI: 2.54--2.82)}\end{tabular} & \begin{tabular}{@{}c@{}}2.49 $\pm$ {\scriptsize 0.05} \\ {\scriptsize (95\% CI: 2.40--2.59)}\end{tabular} & \begin{tabular}{@{}c@{}}2.89 $\pm$ {\scriptsize 0.05} \\ {\scriptsize (95\% CI: 2.80--2.99)}\end{tabular} \\
\midrule
\rowcolor{gray!25}
\multicolumn{5}{c}{\textbf{Llama 3.2 11B}} \\
\textbf{Modality} & \textbf{Linguistic Habits} & \textbf{Persona Consistency} & \textbf{Expected Action} & \textbf{Action Justification} \\
\midrule
\textbf{Text} & \begin{tabular}{@{}c@{}}1.28 $\pm$ {\scriptsize 0.04} \\ {\scriptsize (95\% CI: 1.21--1.35)}\end{tabular} & \begin{tabular}{@{}c@{}}1.69 $\pm$ {\scriptsize 0.06} \\ {\scriptsize (95\% CI: 1.57--1.81)}\end{tabular} & \begin{tabular}{@{}c@{}}1.82 $\pm$ {\scriptsize 0.05} \\ {\scriptsize (95\% CI: 1.73--1.91)}\end{tabular} & \begin{tabular}{@{}c@{}}2.42 $\pm$ {\scriptsize 0.05} \\ {\scriptsize (95\% CI: 2.32--2.51)}\end{tabular} \\
\textbf{Assisted Image} & \begin{tabular}{@{}c@{}}0.67 $\pm$ {\scriptsize 0.02} \\ {\scriptsize (95\% CI: 0.63--0.71)}\end{tabular} & \begin{tabular}{@{}c@{}}1.31 $\pm$ {\scriptsize 0.05} \\ {\scriptsize (95\% CI: 1.21--1.41)}\end{tabular} & \begin{tabular}{@{}c@{}}1.19 $\pm$ {\scriptsize 0.04} \\ {\scriptsize (95\% CI: 1.12--1.26)}\end{tabular} & \begin{tabular}{@{}c@{}}1.73 $\pm$ {\scriptsize 0.04} \\ {\scriptsize (95\% CI: 1.65--1.81)}\end{tabular} \\
\textbf{Image} & \begin{tabular}{@{}c@{}}0.61 $\pm$ {\scriptsize 0.02} \\ {\scriptsize (95\% CI: 0.58--0.64)}\end{tabular} & \begin{tabular}{@{}c@{}}1.15 $\pm$ {\scriptsize 0.05} \\ {\scriptsize (95\% CI: 1.06--1.24)}\end{tabular} & \begin{tabular}{@{}c@{}}1.05 $\pm$ {\scriptsize 0.03} \\ {\scriptsize (95\% CI: 0.98--1.12)}\end{tabular} & \begin{tabular}{@{}c@{}}1.40 $\pm$ {\scriptsize 0.04} \\ {\scriptsize (95\% CI: 1.33--1.48)}\end{tabular} \\
\textbf{Descriptive Image} & \begin{tabular}{@{}c@{}}0.71 $\pm$ {\scriptsize 0.02} \\ {\scriptsize (95\% CI: 0.68--0.75)}\end{tabular} & \begin{tabular}{@{}c@{}}1.60 $\pm$ {\scriptsize 0.06} \\ {\scriptsize (95\% CI: 1.48--1.71)}\end{tabular} & \begin{tabular}{@{}c@{}}1.33 $\pm$ {\scriptsize 0.04} \\ {\scriptsize (95\% CI: 1.25--1.40)}\end{tabular} & \begin{tabular}{@{}c@{}}1.72 $\pm$ {\scriptsize 0.04} \\ {\scriptsize (95\% CI: 1.64--1.80)}\end{tabular} \\
\midrule
\rowcolor{gray!25}
\multicolumn{5}{c}{\textbf{Llama 3.2 90B}} \\
\textbf{Modality} & \textbf{Linguistic Habits} & \textbf{Persona Consistency} & \textbf{Expected Action} & \textbf{Action Justification} \\
\midrule
\textbf{Text} & \begin{tabular}{@{}c@{}}1.45 $\pm$ {\scriptsize 0.04} \\ {\scriptsize (95\% CI: 1.37--1.53)}\end{tabular} & \begin{tabular}{@{}c@{}}1.94 $\pm$ {\scriptsize 0.06} \\ {\scriptsize (95\% CI: 1.81--2.06)}\end{tabular} & \begin{tabular}{@{}c@{}}2.18 $\pm$ {\scriptsize 0.05} \\ {\scriptsize (95\% CI: 2.08--2.27)}\end{tabular} & \begin{tabular}{@{}c@{}}2.69 $\pm$ {\scriptsize 0.05} \\ {\scriptsize (95\% CI: 2.59--2.79)}\end{tabular} \\
\textbf{Assisted Image} & \begin{tabular}{@{}c@{}}0.40 $\pm$ {\scriptsize 0.03} \\ {\scriptsize (95\% CI: 0.35--0.45)}\end{tabular} & \begin{tabular}{@{}c@{}}1.01 $\pm$ {\scriptsize 0.08} \\ {\scriptsize (95\% CI: 0.86--1.16)}\end{tabular} & \begin{tabular}{@{}c@{}}0.87 $\pm$ {\scriptsize 0.06} \\ {\scriptsize (95\% CI: 0.76--0.97)}\end{tabular} & \begin{tabular}{@{}c@{}}0.98 $\pm$ {\scriptsize 0.06} \\ {\scriptsize (95\% CI: 0.86--1.09)}\end{tabular} \\
\textbf{Image} & \begin{tabular}{@{}c@{}}0.31 $\pm$ {\scriptsize 0.02} \\ {\scriptsize (95\% CI: 0.27--0.36)}\end{tabular} & \begin{tabular}{@{}c@{}}0.63 $\pm$ {\scriptsize 0.06} \\ {\scriptsize (95\% CI: 0.50--0.75)}\end{tabular} & \begin{tabular}{@{}c@{}}0.56 $\pm$ {\scriptsize 0.04} \\ {\scriptsize (95\% CI: 0.47--0.64)}\end{tabular} & \begin{tabular}{@{}c@{}}0.59 $\pm$ {\scriptsize 0.04} \\ {\scriptsize (95\% CI: 0.51--0.68)}\end{tabular} \\
\textbf{Descriptive Image} & \begin{tabular}{@{}c@{}}0.37 $\pm$ {\scriptsize 0.03} \\ {\scriptsize (95\% CI: 0.31--0.42)}\end{tabular} & \begin{tabular}{@{}c@{}}0.89 $\pm$ {\scriptsize 0.09} \\ {\scriptsize (95\% CI: 0.73--1.06)}\end{tabular} & \begin{tabular}{@{}c@{}}0.74 $\pm$ {\scriptsize 0.05} \\ {\scriptsize (95\% CI: 0.63--0.84)}\end{tabular} & \begin{tabular}{@{}c@{}}0.87 $\pm$ {\scriptsize 0.05} \\ {\scriptsize (95\% CI: 0.77--0.98)}\end{tabular} \\
\midrule
\rowcolor{gray!25}
\multicolumn{5}{c}{\textbf{Pixtral 12B}} \\
\textbf{Modality} & \textbf{Linguistic Habits} & \textbf{Persona Consistency} & \textbf{Expected Action} & \textbf{Action Justification} \\
\midrule
\textbf{Text} & \begin{tabular}{@{}c@{}}1.26 $\pm$ {\scriptsize 0.04} \\ {\scriptsize (95\% CI: 1.19--1.34)}\end{tabular} & \begin{tabular}{@{}c@{}}1.47 $\pm$ {\scriptsize 0.06} \\ {\scriptsize (95\% CI: 1.35--1.58)}\end{tabular} & \begin{tabular}{@{}c@{}}1.85 $\pm$ {\scriptsize 0.05} \\ {\scriptsize (95\% CI: 1.76--1.94)}\end{tabular} & \begin{tabular}{@{}c@{}}2.51 $\pm$ {\scriptsize 0.05} \\ {\scriptsize (95\% CI: 2.41--2.60)}\end{tabular} \\
\textbf{Assisted Image} & \begin{tabular}{@{}c@{}}1.08 $\pm$ {\scriptsize 0.03} \\ {\scriptsize (95\% CI: 1.02--1.14)}\end{tabular} & \begin{tabular}{@{}c@{}}1.43 $\pm$ {\scriptsize 0.05} \\ {\scriptsize (95\% CI: 1.32--1.54)}\end{tabular} & \begin{tabular}{@{}c@{}}1.65 $\pm$ {\scriptsize 0.04} \\ {\scriptsize (95\% CI: 1.56--1.73)}\end{tabular} & \begin{tabular}{@{}c@{}}2.32 $\pm$ {\scriptsize 0.05} \\ {\scriptsize (95\% CI: 2.22--2.41)}\end{tabular} \\
\textbf{Image} & \begin{tabular}{@{}c@{}}1.04 $\pm$ {\scriptsize 0.03} \\ {\scriptsize (95\% CI: 0.98--1.10)}\end{tabular} & \begin{tabular}{@{}c@{}}1.90 $\pm$ {\scriptsize 0.06} \\ {\scriptsize (95\% CI: 1.78--2.02)}\end{tabular} & \begin{tabular}{@{}c@{}}2.06 $\pm$ {\scriptsize 0.05} \\ {\scriptsize (95\% CI: 1.96--2.15)}\end{tabular} & \begin{tabular}{@{}c@{}}2.62 $\pm$ {\scriptsize 0.05} \\ {\scriptsize (95\% CI: 2.52--2.71)}\end{tabular} \\
\textbf{Descriptive Image} & \begin{tabular}{@{}c@{}}1.05 $\pm$ {\scriptsize 0.03} \\ {\scriptsize (95\% CI: 0.99--1.11)}\end{tabular} & \begin{tabular}{@{}c@{}}2.75 $\pm$ {\scriptsize 0.07} \\ {\scriptsize (95\% CI: 2.61--2.88)}\end{tabular} & \begin{tabular}{@{}c@{}}2.42 $\pm$ {\scriptsize 0.05} \\ {\scriptsize (95\% CI: 2.32--2.51)}\end{tabular} & \begin{tabular}{@{}c@{}}2.97 $\pm$ {\scriptsize 0.05} \\ {\scriptsize (95\% CI: 2.87--3.06)}\end{tabular} \\
\bottomrule
\end{tabular}
\caption{Evaluation Metrics by Model and Modality with \textbf{\underline{GPT-4o}} as the evaluator. Each cell shows mean $\pm$ {\scriptsize SEM} on the first line and 95\% CI on the second.}
\label{tab:eval-table-gpt-4o}
\end{table*}

\begin{table*}[t]
\centering
\small
\begin{tabular}{lcccc}
\toprule
\rowcolor{gray!25}
\multicolumn{5}{c}{\textbf{GPT-4o}} \\
\textbf{Modality} & \textbf{Linguistic Habits} & \textbf{Persona Consistency} & \textbf{Expected Action} & \textbf{Action Justification} \\
\midrule
\textbf{Text} & \begin{tabular}{@{}c@{}}2.47 $\pm$ {\scriptsize 0.03} \\ {\scriptsize (95\% CI: 2.41--2.53)}\end{tabular} & \begin{tabular}{@{}c@{}}3.88 $\pm$ {\scriptsize 0.04} \\ {\scriptsize (95\% CI: 3.79--3.97)}\end{tabular} & \begin{tabular}{@{}c@{}}4.46 $\pm$ {\scriptsize 0.03} \\ {\scriptsize (95\% CI: 4.41--4.52)}\end{tabular} & \begin{tabular}{@{}c@{}}4.34 $\pm$ {\scriptsize 0.03} \\ {\scriptsize (95\% CI: 4.28--4.40)}\end{tabular} \\
\textbf{Assisted Image} & \begin{tabular}{@{}c@{}}2.01 $\pm$ {\scriptsize 0.03} \\ {\scriptsize (95\% CI: 1.95--2.06)}\end{tabular} & \begin{tabular}{@{}c@{}}3.50 $\pm$ {\scriptsize 0.04} \\ {\scriptsize (95\% CI: 3.42--3.59)}\end{tabular} & \begin{tabular}{@{}c@{}}4.35 $\pm$ {\scriptsize 0.03} \\ {\scriptsize (95\% CI: 4.30--4.40)}\end{tabular} & \begin{tabular}{@{}c@{}}4.03 $\pm$ {\scriptsize 0.03} \\ {\scriptsize (95\% CI: 3.97--4.10)}\end{tabular} \\
\textbf{Image} & \begin{tabular}{@{}c@{}}1.96 $\pm$ {\scriptsize 0.03} \\ {\scriptsize (95\% CI: 1.90--2.02)}\end{tabular} & \begin{tabular}{@{}c@{}}3.36 $\pm$ {\scriptsize 0.04} \\ {\scriptsize (95\% CI: 3.28--3.45)}\end{tabular} & \begin{tabular}{@{}c@{}}4.36 $\pm$ {\scriptsize 0.03} \\ {\scriptsize (95\% CI: 4.31--4.41)}\end{tabular} & \begin{tabular}{@{}c@{}}3.93 $\pm$ {\scriptsize 0.04} \\ {\scriptsize (95\% CI: 3.86--4.00)}\end{tabular} \\
\textbf{Descriptive Image} & \begin{tabular}{@{}c@{}}2.01 $\pm$ {\scriptsize 0.03} \\ {\scriptsize (95\% CI: 1.95--2.07)}\end{tabular} & \begin{tabular}{@{}c@{}}4.14 $\pm$ {\scriptsize 0.04} \\ {\scriptsize (95\% CI: 4.07--4.21)}\end{tabular} & \begin{tabular}{@{}c@{}}4.62 $\pm$ {\scriptsize 0.02} \\ {\scriptsize (95\% CI: 4.58--4.66)}\end{tabular} & \begin{tabular}{@{}c@{}}4.12 $\pm$ {\scriptsize 0.03} \\ {\scriptsize (95\% CI: 4.05--4.19)}\end{tabular} \\
\midrule
\rowcolor{gray!25}
\multicolumn{5}{c}{\textbf{GPT-4o-mini}} \\
\textbf{Modality} & \textbf{Linguistic Habits} & \textbf{Persona Consistency} & \textbf{Expected Action} & \textbf{Action Justification} \\
\midrule
\textbf{Text} & \begin{tabular}{@{}c@{}}2.31 $\pm$ {\scriptsize 0.03} \\ {\scriptsize (95\% CI: 2.25--2.36)}\end{tabular} & \begin{tabular}{@{}c@{}}4.01 $\pm$ {\scriptsize 0.04} \\ {\scriptsize (95\% CI: 3.93--4.09)}\end{tabular} & \begin{tabular}{@{}c@{}}4.47 $\pm$ {\scriptsize 0.03} \\ {\scriptsize (95\% CI: 4.43--4.52)}\end{tabular} & \begin{tabular}{@{}c@{}}4.34 $\pm$ {\scriptsize 0.03} \\ {\scriptsize (95\% CI: 4.28--4.40)}\end{tabular} \\
\textbf{Assisted Image} & \begin{tabular}{@{}c@{}}2.06 $\pm$ {\scriptsize 0.03} \\ {\scriptsize (95\% CI: 2.01--2.12)}\end{tabular} & \begin{tabular}{@{}c@{}}3.79 $\pm$ {\scriptsize 0.04} \\ {\scriptsize (95\% CI: 3.70--3.87)}\end{tabular} & \begin{tabular}{@{}c@{}}4.42 $\pm$ {\scriptsize 0.02} \\ {\scriptsize (95\% CI: 4.37--4.47)}\end{tabular} & \begin{tabular}{@{}c@{}}4.07 $\pm$ {\scriptsize 0.03} \\ {\scriptsize (95\% CI: 4.00--4.14)}\end{tabular} \\
\textbf{Image} & \begin{tabular}{@{}c@{}}2.04 $\pm$ {\scriptsize 0.03} \\ {\scriptsize (95\% CI: 1.98--2.09)}\end{tabular} & \begin{tabular}{@{}c@{}}3.84 $\pm$ {\scriptsize 0.04} \\ {\scriptsize (95\% CI: 3.76--3.92)}\end{tabular} & \begin{tabular}{@{}c@{}}4.44 $\pm$ {\scriptsize 0.02} \\ {\scriptsize (95\% CI: 4.39--4.48)}\end{tabular} & \begin{tabular}{@{}c@{}}4.02 $\pm$ {\scriptsize 0.03} \\ {\scriptsize (95\% CI: 3.95--4.09)}\end{tabular} \\
\textbf{Descriptive Image} & \begin{tabular}{@{}c@{}}2.15 $\pm$ {\scriptsize 0.03} \\ {\scriptsize (95\% CI: 2.09--2.20)}\end{tabular} & \begin{tabular}{@{}c@{}}4.49 $\pm$ {\scriptsize 0.03} \\ {\scriptsize (95\% CI: 4.43--4.55)}\end{tabular} & \begin{tabular}{@{}c@{}}4.69 $\pm$ {\scriptsize 0.02} \\ {\scriptsize (95\% CI: 4.65--4.72)}\end{tabular} & \begin{tabular}{@{}c@{}}4.20 $\pm$ {\scriptsize 0.03} \\ {\scriptsize (95\% CI: 4.14--4.27)}\end{tabular} \\
\midrule
\rowcolor{gray!25}
\multicolumn{5}{c}{\textbf{Llama 3.2 11B}} \\
\textbf{Modality} & \textbf{Linguistic Habits} & \textbf{Persona Consistency} & \textbf{Expected Action} & \textbf{Action Justification} \\
\midrule
\textbf{Text} & \begin{tabular}{@{}c@{}}3.12 $\pm$ {\scriptsize 0.03} \\ {\scriptsize (95\% CI: 3.07--3.18)}\end{tabular} & \begin{tabular}{@{}c@{}}3.90 $\pm$ {\scriptsize 0.04} \\ {\scriptsize (95\% CI: 3.82--3.99)}\end{tabular} & \begin{tabular}{@{}c@{}}4.14 $\pm$ {\scriptsize 0.03} \\ {\scriptsize (95\% CI: 4.08--4.19)}\end{tabular} & \begin{tabular}{@{}c@{}}4.07 $\pm$ {\scriptsize 0.03} \\ {\scriptsize (95\% CI: 4.01--4.13)}\end{tabular} \\
\textbf{Assisted Image} & \begin{tabular}{@{}c@{}}1.93 $\pm$ {\scriptsize 0.03} \\ {\scriptsize (95\% CI: 1.87--1.99)}\end{tabular} & \begin{tabular}{@{}c@{}}3.02 $\pm$ {\scriptsize 0.04} \\ {\scriptsize (95\% CI: 2.93--3.11)}\end{tabular} & \begin{tabular}{@{}c@{}}3.36 $\pm$ {\scriptsize 0.03} \\ {\scriptsize (95\% CI: 3.30--3.43)}\end{tabular} & \begin{tabular}{@{}c@{}}3.15 $\pm$ {\scriptsize 0.04} \\ {\scriptsize (95\% CI: 3.08--3.23)}\end{tabular} \\
\textbf{Image} & \begin{tabular}{@{}c@{}}2.03 $\pm$ {\scriptsize 0.03} \\ {\scriptsize (95\% CI: 1.97--2.09)}\end{tabular} & \begin{tabular}{@{}c@{}}2.66 $\pm$ {\scriptsize 0.05} \\ {\scriptsize (95\% CI: 2.56--2.75)}\end{tabular} & \begin{tabular}{@{}c@{}}3.02 $\pm$ {\scriptsize 0.04} \\ {\scriptsize (95\% CI: 2.95--3.09)}\end{tabular} & \begin{tabular}{@{}c@{}}2.94 $\pm$ {\scriptsize 0.04} \\ {\scriptsize (95\% CI: 2.87--3.02)}\end{tabular} \\
\textbf{Descriptive Image} & \begin{tabular}{@{}c@{}}2.17 $\pm$ {\scriptsize 0.03} \\ {\scriptsize (95\% CI: 2.10--2.23)}\end{tabular} & \begin{tabular}{@{}c@{}}3.50 $\pm$ {\scriptsize 0.04} \\ {\scriptsize (95\% CI: 3.42--3.59)}\end{tabular} & \begin{tabular}{@{}c@{}}3.65 $\pm$ {\scriptsize 0.03} \\ {\scriptsize (95\% CI: 3.59--3.71)}\end{tabular} & \begin{tabular}{@{}c@{}}3.27 $\pm$ {\scriptsize 0.04} \\ {\scriptsize (95\% CI: 3.19--3.34)}\end{tabular} \\
\midrule
\rowcolor{gray!25}
\multicolumn{5}{c}{\textbf{Llama 3.2 90B}} \\
\textbf{Modality} & \textbf{Linguistic Habits} & \textbf{Persona Consistency} & \textbf{Expected Action} & \textbf{Action Justification} \\
\midrule
\textbf{Text} & \begin{tabular}{@{}c@{}}3.20 $\pm$ {\scriptsize 0.03} \\ {\scriptsize (95\% CI: 3.14--3.25)}\end{tabular} & \begin{tabular}{@{}c@{}}4.05 $\pm$ {\scriptsize 0.04} \\ {\scriptsize (95\% CI: 3.96--4.13)}\end{tabular} & \begin{tabular}{@{}c@{}}4.38 $\pm$ {\scriptsize 0.03} \\ {\scriptsize (95\% CI: 4.32--4.43)}\end{tabular} & \begin{tabular}{@{}c@{}}4.29 $\pm$ {\scriptsize 0.03} \\ {\scriptsize (95\% CI: 4.24--4.35)}\end{tabular} \\
\textbf{Assisted Image} & \begin{tabular}{@{}c@{}}2.09 $\pm$ {\scriptsize 0.07} \\ {\scriptsize (95\% CI: 1.96--2.22)}\end{tabular} & \begin{tabular}{@{}c@{}}2.24 $\pm$ {\scriptsize 0.08} \\ {\scriptsize (95\% CI: 2.09--2.40)}\end{tabular} & \begin{tabular}{@{}c@{}}2.00 $\pm$ {\scriptsize 0.06} \\ {\scriptsize (95\% CI: 1.89--2.12)}\end{tabular} & \begin{tabular}{@{}c@{}}2.36 $\pm$ {\scriptsize 0.07} \\ {\scriptsize (95\% CI: 2.21--2.50)}\end{tabular} \\
\textbf{Image} & \begin{tabular}{@{}c@{}}2.18 $\pm$ {\scriptsize 0.08} \\ {\scriptsize (95\% CI: 2.03--2.34)}\end{tabular} & \begin{tabular}{@{}c@{}}1.53 $\pm$ {\scriptsize 0.07} \\ {\scriptsize (95\% CI: 1.38--1.67)}\end{tabular} & \begin{tabular}{@{}c@{}}1.48 $\pm$ {\scriptsize 0.05} \\ {\scriptsize (95\% CI: 1.38--1.58)}\end{tabular} & \begin{tabular}{@{}c@{}}2.02 $\pm$ {\scriptsize 0.08} \\ {\scriptsize (95\% CI: 1.87--2.18)}\end{tabular} \\
\textbf{Descriptive Image} & \begin{tabular}{@{}c@{}}2.23 $\pm$ {\scriptsize 0.08} \\ {\scriptsize (95\% CI: 2.08--2.38)}\end{tabular} & \begin{tabular}{@{}c@{}}1.96 $\pm$ {\scriptsize 0.09} \\ {\scriptsize (95\% CI: 1.78--2.14)}\end{tabular} & \begin{tabular}{@{}c@{}}1.74 $\pm$ {\scriptsize 0.06} \\ {\scriptsize (95\% CI: 1.62--1.85)}\end{tabular} & \begin{tabular}{@{}c@{}}2.12 $\pm$ {\scriptsize 0.08} \\ {\scriptsize (95\% CI: 1.97--2.27)}\end{tabular} \\
\midrule
\rowcolor{gray!25}
\multicolumn{5}{c}{\textbf{Pixtral 12B}} \\
\textbf{Modality} & \textbf{Linguistic Habits} & \textbf{Persona Consistency} & \textbf{Expected Action} & \textbf{Action Justification} \\
\midrule
\textbf{Text} & \begin{tabular}{@{}c@{}}2.31 $\pm$ {\scriptsize 0.03} \\ {\scriptsize (95\% CI: 2.25--2.36)}\end{tabular} & \begin{tabular}{@{}c@{}}3.28 $\pm$ {\scriptsize 0.05} \\ {\scriptsize (95\% CI: 3.19--3.38)}\end{tabular} & \begin{tabular}{@{}c@{}}4.01 $\pm$ {\scriptsize 0.03} \\ {\scriptsize (95\% CI: 3.95--4.07)}\end{tabular} & \begin{tabular}{@{}c@{}}4.14 $\pm$ {\scriptsize 0.03} \\ {\scriptsize (95\% CI: 4.08--4.19)}\end{tabular} \\
\textbf{Assisted Image} & \begin{tabular}{@{}c@{}}2.18 $\pm$ {\scriptsize 0.03} \\ {\scriptsize (95\% CI: 2.12--2.24)}\end{tabular} & \begin{tabular}{@{}c@{}}3.22 $\pm$ {\scriptsize 0.05} \\ {\scriptsize (95\% CI: 3.13--3.31)}\end{tabular} & \begin{tabular}{@{}c@{}}4.05 $\pm$ {\scriptsize 0.03} \\ {\scriptsize (95\% CI: 3.99--4.11)}\end{tabular} & \begin{tabular}{@{}c@{}}3.82 $\pm$ {\scriptsize 0.03} \\ {\scriptsize (95\% CI: 3.76--3.89)}\end{tabular} \\
\textbf{Image} & \begin{tabular}{@{}c@{}}2.26 $\pm$ {\scriptsize 0.03} \\ {\scriptsize (95\% CI: 2.20--2.32)}\end{tabular} & \begin{tabular}{@{}c@{}}3.64 $\pm$ {\scriptsize 0.05} \\ {\scriptsize (95\% CI: 3.55--3.73)}\end{tabular} & \begin{tabular}{@{}c@{}}4.25 $\pm$ {\scriptsize 0.03} \\ {\scriptsize (95\% CI: 4.20--4.31)}\end{tabular} & \begin{tabular}{@{}c@{}}3.79 $\pm$ {\scriptsize 0.04} \\ {\scriptsize (95\% CI: 3.72--3.86)}\end{tabular} \\
\textbf{Descriptive Image} & \begin{tabular}{@{}c@{}}2.15 $\pm$ {\scriptsize 0.03} \\ {\scriptsize (95\% CI: 2.09--2.21)}\end{tabular} & \begin{tabular}{@{}c@{}}4.43 $\pm$ {\scriptsize 0.03} \\ {\scriptsize (95\% CI: 4.36--4.49)}\end{tabular} & \begin{tabular}{@{}c@{}}4.64 $\pm$ {\scriptsize 0.02} \\ {\scriptsize (95\% CI: 4.60--4.68)}\end{tabular} & \begin{tabular}{@{}c@{}}4.03 $\pm$ {\scriptsize 0.04} \\ {\scriptsize (95\% CI: 3.96--4.10)}\end{tabular} \\
\bottomrule
\end{tabular}
\caption{Evaluation Metrics by Model and Modality with \textbf{\underline{Gemini 2.0 Flash}} as the evaluator. Each cell shows mean $\pm$ {\scriptsize SEM} on the first line and 95\% CI on the second.}
\label{tab:eval-table-gemini-flash}
\end{table*}

\subsection{Human survey design}\label{app:human}
Figure~\ref{fig:survey} demonstrates our survey design that we conduct on $8$ independent annotators to evaluate the quality of LLM evaluators. In particular, we first show the instructions to evaluate the responses for a prompt and a persona, followed by $10$ such questions. 
\begin{figure*}[t]
    \centering
    \subfloat[Instruction]{\includegraphics[width=0.48\textwidth]{survey_start.png}}\hfill
    \subfloat[Question]{\includegraphics[width=0.48\textwidth]{survey_q1.png}}
    \caption{Human survey design}
    \label{fig:survey}
\end{figure*}


% \begin{table*}[t]
%     \centering
%     \caption{}
%     \label{tab:personalist}
%     \resizebox{1.0\linewidth}{!}{
%     \begin{tabular}{l c}
%     \toprule
%     Scenario & Target Attribute \\
%     \midrule
%     "Your extended family is digitizing old home videos for a reunion. While helping, you discover some footage from your early childhood that needs to be converted. The deadline is next week. You ..." & Age \\
%     "During a basement cleanup, you find your old gaming systems and accessories from when you were 12. A local collector has shown interest in purchasing the set. You ..." \\
%     "You're coordinating a playlist for your high school reunion after-party. The organizers want music specifically from your graduating years to recreate the atmosphere. You ..." \\
%     \bottomrule
%     \end{tabular}
% }
% \end{table*}


\end{document}
\endinput
%%
%% End of file `sample-sigconf.tex'.
