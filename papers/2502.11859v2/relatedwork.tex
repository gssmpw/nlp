\section{Related works}
% xwr润色中
% 1. 人类空间智能的定义、层级、维度(g、小g到五个维度的基础空间能力)和测试。
% 2. 已有大模型的空间能力的研究。——没关注基础空间能力,没我们全。


\subsection{Psychometric Studies on Human Spatial Intelligence}

Human spatial intelligence, defined as the ability to mentally model and manipulate spatial environments \cite{bornsteinFramesMindTheory1986}, has been studied since Spearman first identified it as an independent domain in 1904 \cite{spearmanGeneralIntelligenceObjectively1904}. Research diverges into two complementary streams \cite{poratSpatialAbilityUnderstanding2023}:

\textbf{Psychometric Classification.} The first stream establishes hierarchical intelligence models, from general intelligence to domain-specific intelligence like spatial intelligence \cite{gearySpatialAbilityDistinct2022, spearmanGeneralIntelligenceObjectively1904}. These frameworks enable systematic measurement of spatial abilities through standardized tests, continually refining ability subtypes and assessment methods.

\textbf{Interdisciplinary Mechanisms.} The second stream integrates evolutionary psychology, developmental studies, and cognitive neuroscience to explore the origins and mechanisms of spatial abilities, complementing psychometric taxonomies.

Psychometric consensus defines spatial intelligence through three hierarchical levels \cite{caemmererIndividualIntelligenceTests2020, mcgrewCHCTheoryHuman2009, johnsonStructureHumanIntelligence2005, linnEmergenceCharacterizationSex1985, maccobyPsychologySexDifferences1974, michaelDescriptionSpatialvisualizationAbilities1957, thurstonePrimaryAbilitiesVisual1950}:

\begin{enumerate}

\item General Intelligence (\textit{g}): cross-domain cognitive processes, such as attentional control \cite{kaneRolePrefrontalCortex2002}, neural integration \cite{jungParietofrontalIntegrationTheory2007}, and cellular processes \cite{gearyInclassAttentionSpatial2021}, which impact cross-domain learning and performance.

\item Domain-Specific Intelligence: Abilities in particular domains that share common features, with spatial intelligence representing a distinct set of abilities \cite{vernonAbilityFactorsEnvironmental1965}, as formalized in models such as the CHC theory \cite{carrollHumanCognitiveAbilities1993, hornOrganizationAbilitiesDevelopment1968, cattellTheoryFluidCrystallized1963}.

\item Basic Spatial Abilities (BSAs): Decomposing spatial intelligence into measurable subskills \cite{johnsonSexDifferencesMental2007, hegartySpatialAbilitiesDifferent2006, maierSpatialGeometrySpatial1996, voyerMagnitudeSexDifferences1995, halpernSexDifferencesCognitive1992, pellegrinoUnderstandingSpatialAbility1984}.

\end{enumerate}

This study focuses on Level 3, adopting established human experimental paradigms to evaluate VLMs' basic spatial abilities.

%Human Spatial Intelligence is a set of abilities to form mental models of the spatial world and to manipulate and use these models \cite{bornsteinFramesMindTheory1986}. Research on human SI began in 1904 with the identification of SI as an independent intelligence domain \cite{spearmanGeneralIntelligenceObjectively1904}. Since then, the studies can be broadly categorized into two main research lines \cite{poratSpatialAbilityUnderstanding2023}.
%Firstly, Psychometrics introduced hierarchical classifications of human intelligence \cite{gearySpatialAbilityDistinct2022} ranging from broad General Intelligence (\textit{g}) to subdomains (such as verbal and spatial intelligence) to more specific and basic abilities \cite{spearmanGeneralIntelligenceObjectively1904}. In addition, a substantial body of psychometric research has been devoted to developing tests and experiments for spatial abilities. This line of research continues to this day, refining the distinctions between subtypes of spatial ability and identifying the best methods for their measurement.
%Secondly, interdisciplinary research in evolutionary and developmental psychology, as well as cognitive neuroscience, has complemented traditional psychometric studies. These disciplines explore the influencing factors and internal mechanisms underlying spatial abilities, enriching our understanding of their development.
%Psychometric studies have established a hierarchical framework for defining human spatial intelligence, identifying at least three levels of abilities \cite{caemmererIndividualIntelligenceTests2020, mcgrewCHCTheoryHuman2009, johnsonStructureHumanIntelligence2005, linnEmergenceCharacterizationSex1985, maccobyPsychologySexDifferences1974, michaelDescriptionSpatialvisualizationAbilities1957, thurstonePrimaryAbilitiesVisual1950}. At the broadest level lies general intelligence (\textit{g}), which represents a system of multiple cognitive abilities, including factors such as attentional control \cite{kaneRolePrefrontalCortex2002}, neural networks \cite{jungParietofrontalIntegrationTheory2007}, and cellular processes \cite{gearyInclassAttentionSpatial2021} that impact cross-domain learning and performance. The second level captures the commonalities within specific domains of ability, with Spatial Intelligence being one of these domains \cite{vernonAbilityFactorsEnvironmental1965}. For instance, the integration of models by Cattell, Horn, and Carroll (CHC model) incorporates the \textit{g} factor alongside multiple broad cognitive domains, such as fluid reasoning, short-term memory, and quantitative knowledge \cite{carrollHumanCognitiveAbilities1993,hornOrganizationAbilitiesDevelopment1968,cattellTheoryFluidCrystallized1963}. The third level involves further decomposition of spatial abilities into subdomains \cite{johnsonSexDifferencesMental2007, hegartySpatialAbilitiesDifferent2006, maierSpatialGeometrySpatial1996, voyerMagnitudeSexDifferences1995, halpernSexDifferencesCognitive1992, pellegrinoUnderstandingSpatialAbility1984}, with different scholars offering varying interpretations and definitions.

%This study focuses specifically on these basic spatial abilities at the third level, utilizing established testing methods from prior human experiments to evaluate VLMs.


\subsection{Evaluation of LLMs and VLMs' Spatial Abilities}
% 用表格整理一下现有研究测定的能力

Recent advances in Visual Language Models (VLMs) have spurred evaluations of their spatial abilities, yet existing studies remain fragmented (Table \ref{tab:existing_studies}). Most prior work focuses on text-based LLMs, assessing abstract spatial relations through verbal descriptions \cite{yamadaEvaluatingSpatialUnderstanding2024}, inherently neglecting visual-spatial processing. Emerging VLM evaluations primarily target specialized 3D tasks (e.g., robotic trajectory labeling \cite{sharmaExploringImprovingSpatial2023}, indoor scene captioning \cite{fuSceneLLMExtendingLanguage2024}), which partially engage Spatial Perception, Spatial Relation, and Spatial Orientation. However, critical gaps persist: 

\begin{enumerate}

\item Theoretical Disconnect: Tasks lack grounding in theoretical frameworks, preventing direct comparison with human cognition. For instance, robotic trajectory tests \cite{sharmaExploringImprovingSpatial2023} conflate spatial reasoning with action planning.
\item Limited Scope: Most studies omit Mental Rotation and Spatial Visualization (Table \ref{tab:existing_studies}, MR/SV columns).
\item Benchmark Absence: No study systematically maps VLM performance to hierarchical BSAs or provides human baselines.

\end{enumerate}

This study addresses these limitations through a psychometric evaluation framework. Unlike prior studies testing subsets of BSAs (e.g., \citet{fuSceneLLMExtendingLanguage2024}: SP+SR+SO), we assess all five BSAs using standardized human experiments, enabling cross-model and human-AI comparisons.

%The development of Visual Large Language Models (Visual LLMs or VLMs) has been advancing rapidly. Recently, several studies have explored different dimensions of spatial abilities through targeted evaluations. As shown in Table \ref{tab:existing_studies}, most existing research focuses on assessing LLMs' understanding and reasoning of abstract spatial relations, often overlooking their ability to process spatial information from visual inputs. This limitation stems from the fact that most studies test text-in, text-out LLMs, relying primarily on text-based descriptions of spatial problems for evaluation \cite{yamadaEvaluatingSpatialUnderstanding2024}.

%More recently, some studies have begun evaluating Visual LLMs, but these efforts primarily focus on specific 3D spatial data and specialized tasks like interactive planning or 3D captioning. Within our framework, these studies partially test VLMs' spatial perception, spatial relation, and spatial orientation abilities, with limited attention given to other basic abilities like mental rotation and spatial visualization. Additionally, their experimental designs lack a solid theoretical foundation and fail to ensure comparability with similar human cognitive abilities. For example, \citet{sharmaExploringImprovingSpatial2023} investigated VLMs' capabilities in spatial orientation and spatial relation. Their evaluation included 2D path labeling tasks for directions (e.g., "up," "left") and shapes, 3D trajectory labeling tasks involving motions like "lifting," "rotating," and "sliding" (as well as cleaned versions), and identifying relationships between blocks in imagined setups using 3D robotic trajectory data. Similarly, \citet{fuSceneLLMExtendingLanguage2024} focused on VLMs' spatial abilities in indoor scene application tasks, such as describing scene details (dense captioning), identifying and describing objects (object captioning), and interactive planning.

\begin{table}[t]
\centering
\caption{Existing studies testing LLMs and VLMs' spatial abilities.}
\label{tab:existing_studies}
\resizebox{\linewidth}{!}{%
\begin{tabular}{lcccccc}
\hline
\multirow{2}{*}{Related Work} & \multirow{2}{*}{VLM} & \multicolumn{5}{c}{Tested Spatial Abilities} \\ \cline{3-7} 
 &  & SP & SR & SO & MR & SV \\ \hline
\citet{fuSceneLLMExtendingLanguage2024}          & Yes & \checkmark & \checkmark & \checkmark &            &  \\
\citet{tangSparkleMasteringBasic2024}            & Yes & \checkmark & \checkmark & \checkmark &            &  \\
\citet{sharmaExploringImprovingSpatial2023}      & Yes &            & \checkmark & \checkmark &            &  \\
\citet{hong3DLLMInjecting3D2023}                 & Yes & \checkmark & \checkmark & \checkmark &            &  \\
\citet{bangMultitaskMultilingualMultimodal2023}  & Yes &            & \checkmark & \checkmark &            &  \\
\citet{yamadaEvaluatingSpatialUnderstanding2024} & No  &            & \checkmark &            &            &  \\
\citet{momennejadEvaluatingCognitiveMaps2023}    & No  &            & \checkmark & \checkmark &            &  \\
\hyperref[section:bib]{Cohn et al. (2023)}       & No  &            & \checkmark &            & \checkmark &  \\
\hline
This study                                       & Yes & \checkmark & \checkmark & \checkmark & \checkmark & \checkmark \\
\hline
\end{tabular}%
}
\end{table}


%%%%%%%%%%%%%%%%%%%%%%%%%%%%%%%%%%%%%%%%%%