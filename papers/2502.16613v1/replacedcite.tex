\section{Related Works}
Previous research has shown that intelligent tutors can significantly improve learning outcomes, particularly in K-12 settings, where their use is often mandatory. For example, Koedinger et al. ____ found that high school students using Cognitive Tutor for Algebra I outperformed their peers receiving traditional instruction on standardized tests. Similarly, a meta-analysis by Ma et al. ____ concluded that students using intelligent tutors exhibited substantial gains in academic achievement compared to those receiving conventional classroom instruction. VanLehn's analysis ____ suggests that intelligent tutors are nearly as effective as human tutors and sometimes more effective than traditional classroom instruction. 

Although intelligent tutors have been extensively studied in K-12 contexts, adult learners represent a distinct demographic with unique learning characteristics. Adult learners differ from younger students in several key aspects, including greater self-direction, accumulated life experiences, and a focus on practical application ____. According to the principles of andragogy, adults are intrinsically motivated and prefer learning that is relevant to their personal and professional lives ____. Adult learners tend to be self-directed learners who value autonomy in the learning process ____. In addition, they often balance multiple responsibilities, such as work and family, which can affect their engagement with educational technologies ____.

Despite a consistent track record of improving student learning, tutors have not been widely adopted by adult learners across non-traditional educational environments, such as workplace training programs, online education platforms, and continuing professional development courses. However, there has been work exploring the use of tutoring systems to support adult learners in specialized contexts. For example, the GIFT (Generalized Intelligent  Framework for Tutoring) framework was developed to support tutor creation, with a specific consideration to the unique needs of military personnel, focusing on skills acquisition and decision-making under pressure ____. Furthermore, Lippert et al. ____ investigated intelligent tutors for adults, emphasizing adaptability and real-time feedback to enhance operational readiness and training effectiveness. Nevertheless, this prior work primarily targets adult learners in high-stakes environments and does not always directly translate to real-world classroom settings involving adult students pursuing non-military educational goals.

Kizilcec and Schneider ____ explored how adults engage with adaptive learning technologies in massive open online courses (MOOCs) and found that personalized feedback can enhance engagement and persistence. However, a gap in this work is the lack of  insight into how users engage in MOOCs, from a session adoption and retention standpoint. Further, this work does not fully address the influence of motivation and self-regulation on their learning behaviors and engagement. 

% In addition, Walkington ____ suggested that contextualizing the tutor content to align with the experiences of adult learners improves motivation and learning outcomes. However, this study was designed for younger learners and adult education is not yet explored. The unique needs of adult learners require tutors that support self-directed learning and adapt to their schedules and life contexts ____. 

When intelligent tutor usage is optional, adults' adoption and engagement can vary widely. The Technology Acceptance Model (TAM) proposed by Davis ____ indicates that perceived usefulness and ease of use are critical factors that influence technology adoption. Barriers such as time constraints, lack of technical skills, and low self-efficacy can hinder voluntary usage ____. Beyond such external factors, demographic factors such as age, gender, and ethnicity can impact the adoption and effectiveness of intelligent tutors among adult learners. Zawacki-Richter and Latchem ____ observed that older adults may be less inclined to use educational technologies due to lower digital literacy. Gender differences have also been reported; Ong and Lai ____ found that women may experience greater computer anxiety, affecting their engagement levels. 

Understanding these demographic influences is key to designing tutors that work well for diverse adult populations. We see an opportunity to extend current intelligent tutor research by combining critical data on tutor adoption, user demographics, and learning outcomes into a single study. While previous work often treats these data separately, integrating them provides a clearer picture of how these technologies impact users. This gap highlights the need for observational studies that analyze real-world usage data to understand how users voluntarily engage with educational technologies. By examining both usage patterns and academic performance, this research offers a more nuanced view of how intelligent tutors support adult learners in supplemental educational settings.

\begin{figure*}
    \centering
    \includegraphics[width=1.0\textwidth]{images/apprentice.pdf} % adjust width of image
    \caption{User interface of Apprentice Tutors platform with key features: (a) penalization through adaptive problem selection (b) real-time correctness feedback (c) four available tutors (d) hint box and multi-layer hints (e) user profile screen with progress bars corresponding to knowledge components.}
    \label{fig:apprentice}
\end{figure*}