\section{Related Works}
\label{sec:relatedworks}

In this section, we summarize the multi-contact modeling and solver algorithms that have been utilized in robotic simulation. See also Table~\ref{table:simulator_compare} for the comparison of widely used simulators in robotics.

\subsection{Direct Method}
The conventional approach to handling dynamics equations with multi-contact constraints involves formulating the equations as a linear complementarity problem (LCP) \cite{potra1997formulating} then applying Lemke's algorithms \cite{llyod2005fast} or Dantzig's pivoting algorithms. 
While these direct methods can guarantee accuracy, they often suffer from high computational complexity. Moreover, the LCP-based formulation necessitates polygonal friction cone approximation, leading to undesirable error in friction behavior. 
In robotic simulation software, DART \cite{lee2018dart}, ODE \cite{ode}, and Bullet \cite{bullet} provide implementations of Dantzig's method to solve the LCP problem.

\subsection{Per-Contact Iteration}

More widely used in recent years are iterative methods, which typically involve locally performing an impulse projection step to achieve global equilibrium. 
One of the most popular iteration schemes is projected Gauss-Seidel (PGS), which has been extensively developed and adopted in the game and graphics community \cite{macklin2014unified,macklin2016xpbd} as well as in robotics \cite{todorov2014convex,horak2019similarities}. 
These methods are known for being simple, robust, and advantageous in generating visually plausible results. 
However, they often experience slow convergence and limited efficiency, especially when the constraints are highly coupled. 
These weaknesses are particularly emphasized in robotic simulation, as the generalized coordinate representation (e.g., robot joint angles) is common, and over-specified contact (i.e., system DOF $<$ constraint DOF) is prevalent in manipulation tasks.
%and complex internal constraints (e.g., finite element).
Several research efforts have aimed to enhance the performance of impulse iteration methods. In \cite{hwangbo2018per}, the bisection method is presented as a potential replacement for the local projection scheme in PGS, demonstrating its effectiveness in quadruped locomotion simulation. Additionally, a substepping variant of PGS, named temporal Gauss-Seidel (TGS), is introduced in \cite{macklin2019small}, showing its better convergence in various situations.
Unlike direct methods, iterative methods can be applied to various types of problem modeling, including LCP, cone complementarity problems (CCP), nonlinear complementarity problems (NCP), and also their position-based dynamics (PBD) variants \cite{muller2020detailed}. 
As a result, they are employed in a wide range of simulation software, including Bullet \cite{bullet}, MuJoCo \cite{mujoco}, RaiSim \cite{RaiSim}, and Isaac Sim \cite{makoviychuk2021isaac}.

\subsection{Nonlinear Equation}

Another approach to dealing with multi-contact simulation is to express all required relations in nonlinear equation form and solve them using gradient descent iteration. 
Implicit penalty-based contact, often referred to as regularized contact, exhibits the most natural connection to this approach, as demonstrated in \cite{geilinger2020add,castro2020transition}. 
However, penalty methods have well-known weaknesses that they often necessitate parameter tuning to achieve plausible results, and high penalty gains can lead to numerical issues.
For the other direction, in \cite{howell2022dojo} construct and solve a nonlinear equation with complementarity smoothing, and \cite{macklin2019nonsmooth} we derived a nonsmooth equation using the complementarity function (e.g., Fischer-Burmeister).
While these methods typically exhibit superlinear convergence, the intricate nature of contact conditions frequently leads to lack of robustness or challenges in line search. 
Addressing this issue, the Newton-based techniques \cite{castro2022unconstrained} and conjugate gradient (CG) algorithm for regularized convex contact models aim to ensure algorithmic robustness, albeit at the potential expense of physical accuracy. 
Among current simulation software, MuJoCo and Drake \cite{drake} are incorporating nonlinear equation-based solvers.


\begin{table}[t]
\centering
\caption{Comparison of contact models and solvers used in popular robotic simulators.}
\renewcommand{\arraystretch}{2.0}{
\resizebox{8.0cm}{!}{
\begin{tabular}{|c|c|c|c|c|c|c|}
\hline
& Bullet & MuJoCo & DART & PhysX & Drake & ODE \\
\hline
\hline
Model & LCP & Convex & LCP & NCP & Convex & LCP \\
\hline 
Solver & \makecell{Direct \\ PGS} & \makecell{Newton \\ CG \\ PGS} & \makecell{Direct \\ PGS} & \makecell{PGS \\ TGS} & Newton & \makecell{Direct \\ PGS} \\
\hline 
\end{tabular}
}   
}
\label{table:simulator_compare}
\end{table}


\subsection{Augmented Lagrangian}
 
Proximal algorithms, which were possibly pioneered by Moreau \cite{moreau1962fonctions} comprise a class of methods designed to address constrained convex optimization problems by sequentially solving a series of subproblems. 
The augmented Lagrangian (AL) method can be viewed as a class of proximal algorithm \cite{parikh2014proximal}, as it formulates subproblems using the method of multipliers and a penalty term. 
Typically, the subproblems are addressed through simpler solutions or tailored designs, which has spurred the development of numerous open-source libraries that implement these strategies, thereby facilitating broader access to robust optimization tools. Notable examples include libraries for quadratic programming, such as OSQP \cite{stellato2020osqp} and QPALM \cite{hermans2019qpalm}, and for cone programming, such as SCS \cite{sopasakis2019superscs}.
In robotics, proximal algorithms have been effectively utilized to address constraints within computational structures, notably in applications such as factor graph optimization \cite{bazzana2024augmented} and differentiable dynamics programming \cite{howell2019altro}.
The utility in solving robot dynamics with equality constraints is presented in \cite{carpentier2021proximal}. 
Our previous work \cite{lee2023modular} presents a specific algorithm based on the Augmented Lagrangian (AL) method to achieve effective parallelization in contact simulations. Building upon this foundation, this article extends the general theory and variations of AL designed to handle robotic simulations involving contact.