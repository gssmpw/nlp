\section{Hawkes-process inspired methods}
\label{sec:hawkes_insprired_methods}

The original Hawkes process is described by the following formula for the conditional intensity function $\lambda_u(t)$:
\begin{align*}
\lambda_u(t) &=
\mu_u(t) + \sum_{v} b_{uv} \sum_{e_i \in \mathcal{H}_v(t)} \kappa(t-t_i) = \\
&= \mu_u(t) + \sum_{v} \sum_{e_i \in \mathcal{H}_v(t)} b_{uv}  \kappa(t-t_i),
\end{align*}
where the first term, $\mu_u(t)$, models the current event sequence behaviour, and the second term, with $f_{uv}$ models interaction between sequence $u$ and sequence $v$,  $\kappa(t - t_i)$ is a kernel function of time, usually exponential. $e_i \in \mathcal{H}_v(t)$ are all events from sequence $v$ before time $t$. 

In the original Hawkes process, events from one sequence have the same type. In our case, this is not true. Also, as we work not with original sequences but with their representations, we can simplify the equation:
\[
\lambda_u(t) = \vecmu_u + \sum_{v} b(\vecmu_u, \vecmu_v) \kappa(t - t_v).
\]
Here, $\vecmu_u$ is a representation vector of the sequence under consideration, $\vecmu_v$ is a representation vector of the sequence from the training dataset,  and $t_v$ is the time moment of the last event in the $v$ sequence at the current time point, and $b$ and $\kappa$ are learnable kernels. The second term in the equation is the target external representation. 

In our notation, the equation can be rewritten:
\[
     \mathbf{g}_t = b(X, \mathbf{h}_t)\cdot \kappa(t \cdot \mathbf{1} - \mathbf{T}),
\]
where $\mathbf{T}$ is a vector of  last event times in $X$ for the current time~$t$.

We simplify this general formulation to a set of different special cases. 
Mostly, this step is taken due to the instability of learning in a general case.
The different variations of the presented formulation of the problem are enumerated below.

\paragraph{Exponential Hawkes.} In this simple variation of the Hawkes method, we use an exponential kernel for time transformation and identical transformation for matrix $X$.
\begin{equation}
     \mathbf{g}_t = X \exp{(-(t \cdot \mathbf{1} - \mathbf{T}))}.
\end{equation}
In this case, we weigh the vectors from $X$ with exponential time-dependent weights.

\textbf{Learnable exponential Hawkes}

The learnable transformation was used in this method. Firstly, we concatenate each vector from  $X$ with $\mathbf{h}_t$ and get matrix $Y \in \mathbb{R}^{2m \times n}$. Secondly, we pass the concatenated matrix through feed-forward neural network $\phi_{NN}(\cdot)$ and get the matrix  $X'\in \mathbb{R}^{m \times n}$. 
\begin{align*}
        \mathbf{g}_t &= \phi_{\mathrm{NN}}(\mathrm{concatenate}(X, \mathbf{h}_t))\exp{(-(t \cdot \mathbf{1} - \mathbf{T}))} =\\ 
     &= \phi_{\mathrm{NN}}(Y)\exp{(-(t \cdot \mathbf{1} - \mathbf{T}))} = \\
     &= X'\exp{(-(t \cdot \mathbf{1} - \mathbf{T}))}.
\end{align*}

Here, we additionally consider the dependencies between the current embedding vector and embedding vectors from the dataset.

\paragraph{Learnable Hawkes.}
This variation of the method is learnable. The embeddings transformation works exactly the same way as in the previous exponential learnable Hawkes method, but here we also add a learnable transformation for times deltas using a feed-forward neural network~$\kappa_{NN}(\cdot)$.
\[
    \mathbf{g}_t = \phi_{\mathrm{NN}}(\text{concatenate}(X, \mathbf{h}_t))\kappa_{\mathrm{NN}}(t \mathbf{1} - \mathbf{T}). 
\]
This method shows instability during learning, as the loss tends to go to infinity. So, it requires more in-depth study, and its results are not presented in this work.

\paragraph{Attention Hawkes.} In this method, we combine the usual exponential Hawkes and the attention methods:
\begin{equation}
     \mathbf{g}_t = X\mathrm{softmax}(X^T \mathbf{h}_t)\exp{(-(t \cdot \mathbf{1} - \mathbf{T}))},
\end{equation}
This approach has double weighting: the first accounts for user similarity, and the second accounts for the time delta.

\section{Implementation details}~\label{sec:details}

\paragraph{Backbone pretrain.} In this research, the pipeline follows that of the \href{https://github.com/dllllb/pytorch-lifestream/tree/main}{pytorch-lifestream} package, which contains an implementation of the CoLES model. 
Following their original article, we use an LSTM block with a hidden layer dimension of $1024$ for the Churn and HSBC datasets.
For Default, absent in that article, we select GRU with a hidden size of $800$ that provides competitive results on numerical experiments.
The models were trained for $60$ epochs with a batch size of $128$.
The model takes two features as input: MCCs and amounts of transactions. 
As MCC is a categorical variable, we use its embedding of size $d_{\mathrm{mcc}} = 24$ for Churn and HSBC or $16$ for Default.


\paragraph{Learnable aggregations.}
The learnable attention matrix $A$ from the learnable attention method, the learnable matrix $S$ from the symmetrical attention method, and the learnable functions $\phi(*)$ and $\phi_{\mathrm{NN}}(*)$ from the kernel attention and exponential learnable Hawkes methods were trained in the CoLES learning pipeline with fixed encoder. 

For the  $S \in \mathbb{R}^{r \times m}$ matrix, internal size $r$ was taken to be equal to $100$. For the learnable functions $\phi(*)$ and  $\phi_{\mathrm{NN}}$, we use a two-layer neural network with a hidden size equal to $100$. 

In the CoLES pipeline, all the learnable elements were trained for $60$ epochs with a batch size $128$.

A model using external information requires considerable computing resources since it needs to store local representations of all users from the training set for all time points.
Due to available memory requirements, we use only a random part of the set of local representations: for the Churn dataset, the number of clients to train in all experiments was $1000$; for the Default dataset, $300$; and for HSBC, $1000$.


\paragraph{Validation.} 


For the global validation task, we train the LGBM model with $500$ estimators, learning rate equal to $0.02$, regularization coefficients for $l_1$ and $l_2$ regularization equal to $1$. For the local validation task, we train linear head over encoder. There, the hyperparameters are the following: window size equal to $32$, stride equal to $16$, batch size equal to $512$, and a maximum number of epochs equal to $20$ for Churn and Default datasets and $10$ for HSBC dataset, learning rate equals to $0.001$, optimizer is Adam. The local task validation was conducted in two modes: freeze and unfreeze. In freeze mode, we train only the local validation head without updating encoder weights. In unfrozen mode, we train both the encoder and head parts on the neural networks.

\section{Description of datasets}
\label{sec:app_datasets}

\emph{Churn} dataset contains transactional data of bank customers and has been used previously~\cite{babaev2022coles}.
For global validation, it is proposed that the problem of binary client classification be solved depending on whether the client left the bank (churn case).
The classes here are almost balanced.

\emph{Default} dataset~\cite{bazarova2024universal} also contains transactional data of bank customers.
For global validation, it is proposed that the problem of binary classification of clients be solved depending on whether the client could repay the loan to the bank.
There is a significant class imbalance in this sample.
Unlike most other open transaction data in this dataset, the target variable is close to a real business problem, namely future credit default identification in credit scoring.

\section{Dataset Generation}
\label{sec:dataset}
\revise{
To train the proposed GNN, we constructed a dataset of building structures and a subset of these structures were subjected to fire simulations using FEA. The dataset generation process is illustrated in \figref{fig:dataset_generation_procedure}. Initially, a total of 33,000 building structures with geometrical details, material properties, and gravity loads were created. Due to randomness in generating these structures, a filter is applied to remove unreasonable data after gravity load simulation, which included 15,377 structures. A trade-off between computational feasibility and model performance is made among the remaining 17,623 structures. As further labeling structures with MIDR requires resource-intensive fire simulations via OpenSeesRT, a large proportion of 16,050 structures is selected as unlabeled dataset. On the other hand, each of the other 1,573 structures was further subjected to 30 different fire simulations, forming the labeled dataset containing $1,573\times 30 = 47,190$ fire cases.} This section details the step-by-step process for generating the dataset, including geometry creation, material property assignment, and simulations due to gravity loads and fire scenarios. 
% To train the proposed neural network, we constructed a dataset comprising building structure data and a subset of fire scenario data. The dataset generation process is illustrated in \figref{fig:dataset_generation_procedure}. 
% A total of 33,000 building structures with geometric details, material properties, and gravity loads were initially created. Out of these, 3,000 structures were selected as labeled data, and the remaining 30,000 were designated as unlabeled data. Further, about half of them filtered out due to instability under gravity loads only. 
\begin{figure*}[h!]
    \centering
    \includegraphics[width=0.8\linewidth]{figures/dataset_filter_procedure.pdf}
    \caption{Workflow for dataset generation (geometry, material property, gravity loads, and fire scenarios).}
    \label{fig:dataset_generation_procedure}
\end{figure*}

\subsection{Geometry Generation}
\label{subsec:geometry_generation}
The geometry of the building structures forms the foundation of the dataset. Regular 
\revise{3D structures} resembling multi-story parking structures or shopping malls were generated, with parameters such as building floor dimensions and story heights selected randomly. Each building structure is composed of multiple rooms, which serve as the basic unit in this study. A room herein is a cuboid space defined by specific length, width, and height. Within a structure, rooms of the same dimensions are uniformly arranged along the length, width, and height, corresponding to the $x$-, $y$-, and $z$-axes, respectively. Structures vary in room size and number of rooms along each axis. Specifically, the room length, width, and height are independently sampled from a uniform distribution within the interval $[2, 5]$ meters along the three directions of the structure. Similarly, the room number along each axis is uniformly sampled independently as an integer within the interval $[2, 7]$, i.e., the maximum number of stories of the buildings simulated in this study is 7.

To introduce variability and simulate real-world scenarios, approximately $8\%$ of structural elements (beams or columns) are randomly removed after initial geometry creation. 
\revise{Such removal is not fire-induced damage, but reflects functional diversity often observed in real buildings, such as open spaces designed for activities in shopping malls, e.g., ice skating rinks. Examples of the generated geometries are illustrated in \figref{fig:example_generated_geometry}, showcasing the diversity and realism of the dataset. This element removal does not affect the definition of room's geometry in the structure and nor does it affect the number of considered fire scenarios.} 

\revise{A range of coefficient of variation values ($3.3\%$ to $17.5\%$) was derived from prior studies that investigated the statistics of geometrical and material properties of structural components of buildings (e.g., \cite{mirza1979variations, lee2004probabilistic}). These studies provide empirical data on the natural variability in parameters such as Young's modulus, yield strength, and dimensions of structural elements due to manufacturing tolerances and material inconsistencies. By selecting $8\%$ for the removal of structural elements in our database, we aimed to maintain a level of variability that is representative of real-world uncertainties while ensuring computational feasibility. This choice ensures that the database captures realistic deviations without introducing extreme cases that may not be commonly encountered in practice.}

\begin{figure*}[h!]
    \centering
    \includegraphics[width=\linewidth]{figures/example_generated_geometry.pdf}
    \caption{Examples of generated structural geometry of different sizes (all dimensions in meters).}
    \label{fig:example_generated_geometry} 
\end{figure*}

{\blockRevise

In this study, we opted for a deterministic square, dimension of $0.1$ m, solid cross-sectional steel elements due to their simplicity in modeling and analysis. Square sections exhibit uniform geometrical properties in all directions, simplifying the computation of structural responses and avoiding complications associated with more complex shapes, such as wide-flange sections, facilitating the computational efficiency and scalability to generate a large dataset. This choice also helps to mitigate issues related to stress concentrations and facilitates a more straightforward representation of structural behavior under thermal loads. 

\textit{Remark:} The selected cross-section provides a comparable flexural rigidity to a $W 130 \times 130 \times 28.1$ wide-flange section (metric units), albeit with significantly higher axial rigidity. This cross-section is acceptable for gravity-load-designed frames under service loading conditions where the models assume fully rigid, moment-resisting beam-column connections for the evaluation of the IDR under thermal loading. This assumption is reasonable in this computational study where the primary interest is to understand the global deformation response of frames under fire conditions. The selection of uniform square cross-sections for both beams and columns, rather than adherence to standard capacity design principles, was made here primarily for computational efficiency and to reduce design parameters in the database generation process. This choice allows for simplified and scalable approach to analyze the fire-induced response of generic steel frames without the need for large section variations, where this study mainly focuses on the fire vulnerability assessment using ML-based predictions. However, if additional loading conditions, e.g., seismic or wind loads, were to be considered, larger sections, strong-column/weak-beam principle, and ductile detailing would be required in the generated buildings for realistic structural behavior under combined loading conditions. Future studies may also consider investigating the influence of variable cross-sectional dimensions and semi-rigid connections on the structural performance under fire conditions. 
} % blockRevise

\subsection{Material Properties}
Steel is chosen as the material for the structures. To reflect real-world variations, we randomly assign one of five slightly different steel material types to each structural element. \revise{
The ranges of material properties are provided in \tabref{tab:material_property_ranges} and the properties are sampled from uniform distributions of the corresponding ranges. These variations simulate differences arising from manufacturing batches or regional material properties. That these properties are at ambient temperature and change when the temperature rises due to a fire. The selection of materials with varying properties is aimed at increasing the diversity of the data. Our goal is to represent as wide a range of data as possible with a limited amount of building structure data, thereby enhancing the generalization ability of the GNN. Our assumed material property ranges are expected to be wider than the real-world conditions based on findings in \cite{mirza1979variations, lee2004probabilistic}. Therefore, we are essentially tackling a more challenging and general task. If we can solve this problem, we are confident that our method will perform equally well or even better in real-world scenarios.
}
\begin{table}[h!]
    \centering
    \caption{Material properties ranges for considered steel structures.}
    \begin{tabular}{lc}
        \toprule
        Property & Range \\
        \midrule
        Young's modulus & [168, 252] GPa \\
        Yield strength & [220, 330] MPa \\
        Strain-hardening ratio & [0.8, 1.2] \% \\
        \bottomrule
    \end{tabular}
    \label{tab:material_property_ranges}
\end{table}

\subsection{Gravity Loads}
Gravity loads are applied to columns and beams based on their \revise{influence (tributary) areas as typically conducted in structural analysis. The considered ``service'' load conditions include the column self-weight and the additional loads directly supported on the beams from their self-weight and weights of the reinforced concrete slabs, people as live load, and building content. An edge beam typically carries approximately half the gravity load supported by a parallel interior beam}. The ranges of gravity loads are listed in \tabref{tab:gravity_load_ranges}. \revise{The loads are sampled from uniform distributions of the corresponding ranges.} Structures that failed to meet an MIDR threshold of $1\%$ under gravity loads were deemed unacceptable designs and filtered out, as such configurations of randomly chosen geometry, material, and gravity load combinations were considered unrealistic from a regulatory and practicality points of view.
\begin{table}[h!]
    \centering
    \caption{Gravity load ranges for considered beams and columns.}
    \begin{tabular}{lc}
        \toprule
        Element & Range (kN/m)  \\
        \midrule
        Column & [0.5, 1.0]  \\
        Edge beam & [1.5, 4.5]  \\
        Interior beam & [3.0, 7.5]  \\
        \bottomrule
    \end{tabular}
    \label{tab:gravity_load_ranges}
\end{table} 

\subsection{Rule-based Thermal Load Generation}
\label{subsec:thermal_load_generation}
To evaluate a building's structural response during a fire event, we employed a simplified rule-based approach for thermal load generation. 
% Previous studies \cite{nan_structuralfire_2023} have demonstrated that steel structures rapidly equilibrate with surrounding gases temperatures due to efficient heat exchange. Consequently, gas temperatures can be directly used as inputs for FEA tools, e.g., OpenSees, simplifying the process of modeling thermal loads. 
% Accurately simulating temperature fields in fire scenarios poses significant challenges. Advanced thermodynamic simulations, such as those performed using Fire Dynamics Simulator (FDS) \cite{mcgrattan_fire_2000}, provide precise temperature predictions. However, these methods are hindered by high computational costs, prolonging execution times, and limited scalability, making them impractical for generating large datasets. Additionally, real-world fire loads often display substantial spatial variability across different rooms \cite{dundar_fire_2023}, resulting in scenario-specific temperature fields with limited generalizability. For example, studies on bridge fires \cite{he_study_2024} have demonstrated that environmental factors, such as wind speeds, can significantly influence temperature distributions. Furthermore, even within identical scenarios, variations in fire modeling methodologies can produce distinctly different temperature fields \cite{zhang_temperature_2020, du_new_2012}. These challenges emphasize the need for efficient and adaptable methods to generate fire temperature data.
% To address these issues, we adopted a rule-based approach to model temperature variations. 
According to \cite{spearpoint_fire_2008}, a typical fire development follows a predictable pattern. During the {\em{growth stage}}, the temperature rises slowly and approximately linearly after ignition. This is followed by the {\em{flashover stage}}, where temperatures increase rapidly to peak values. After reaching the peak, the temperature either stabilizes or continues to rise slowly until the {\em{decay stage}} begins. Inspired by this fire development pattern, we describe the temperature evolution in time, $t$, prior to the decay stage in two distinct stages:
\begin{enumerate}
    \item {\bf{Initial linear increase stage}}: For $t \in [0, t_1)$, temperature increases gradually and linearly as the fire spreads through the building. This stage represents the time before the fire directly affects a structural element.  
    \item {\bf{ISO 834 fire curve stage}}: For $t \in [t_1, t_{\thre}]$, temperature rises rapidly following the ISO 834 curve \cite{ISO834}, modeling the direct impact of the fire on the structural element. 
\end{enumerate}
The slope of the linear temperature increase, $c$, and the transition time, $t_1$, are influenced by the spatial relationship between the fire source and the structural element. For the second stage of temperature evolution, we utilize the ISO 834 curve, a widely accepted standard for fire resistance testing. This standardized fire curve describes the temperature rise over time, enabling rapid and consistent thermal fields across various scenarios. The duration of fire simulation in this study is set to $t_{\thre}=60$ minutes. This value represents the upper limit for the temperature evolution of each structural element, providing a consistent basis for analyzing the structural response to fire.

Let $(x, y, z)$ represents the midpoint of a structural element and $(x_{\subfire}, y_{\subfire}, z_{\subfire})$ the fire source point. \revise{Integer parameters $h$ and $h_{\subfire}$ correspond to the respective floor levels of the element and the fire source}. The temperature evolution for each element is expressed as follows:
\begin{enumerate}
    \item Linear increase stage ($0 < t < t_1$):
    \begin{equation}
    T(t) = c \cdot t,
    \end{equation}
    where $c$, the rate of temperature increase ($^\circ\mathrm{C}/\mathrm{min}$), depends on the height difference between the element, $h$, and the fire source, $h_{\subfire}$:
    \begin{equation}
        c = 
        \begin{cases} 
        5\left/\left(h - h_{\subfire} + 1\right)\right., & h \geq h_{\subfire}, \\
        2\left/\left(h_{\subfire} - h\right)\right., & h < h_{\subfire}.
        \end{cases}
    \end{equation}
     \item ISO 834 stage ($t \geq t_1$):
\begin{equation}
    T(t) = c \cdot t_1 + 345 \log_{10} \left(8 \left(t - t_1\right) + 1\right).
\end{equation}
\end{enumerate}

The transition (arrival) time $t_1$, marking the end of the linear stage, depends on the spatial distance between the fire source and the element. We define the following two Euclidean distances $L_p$ in the $xy$ plane and $L_s$ in the $xyz$ space:
\begin{eqnarray}
L_p & \triangleq & \sqrt{(x - x_{\subfire})^2 + (y - y_{\subfire})^2}, \\
\label{eq:Lp}
L_s & \triangleq & \sqrt{(x - x_{\subfire})^2 + (y - y_{\subfire})^2 + (z - z_{\subfire})^2}.
\label{eq:Ls}
\end{eqnarray}
Accordingly, the transition time, $t_1$, is expressed as follows:
\begin{equation}
    t_1 = 
    \begin{cases}
    \beta_{1} \cdot \left(1 - \exp\left\{- L_s\left/\alpha_{1}\right.\right\}\right), & h > h_{\subfire}, \\
    \beta_{2} \cdot \left(1 - \exp\left\{- L_p\left/\alpha_{2}\right.\right\}\right), & h = h_{\subfire}, \\
    \beta_{3} \cdot \left(1 - \exp\left\{- L_s\left/\alpha_{3}\right.\right\}\right), & h < h_{\subfire} .
    \end{cases}
    \label{eq:t1}
\end{equation}
The parameters $\beta_i$ and $\alpha_i$ for determining $t_1$ are summarized in Table~\ref{tab:fire_spread_parameters}. In this study, we take $r_{\mathrm{up}}=0.95$ and $r_{\mathrm{down}}=0.97$.
\begin{table}[ht]
    \centering
    \caption{Fire spread parameters for $t_1$ calculations.}
    \begin{tabular}{lcc}
        \toprule
        Case  & $\beta_i$ & $\alpha_i$  \\
        \midrule
        $i=1$, Upward spread & $16 \left.\left(1-r_{\mathrm{up}}^{\left|h-h_{\subfire}\right|}\right)\right/\left(1-r_{\mathrm{up}}\right)$ & $10$  \\
        $i=2$, Horizontal spread & $18$ & $18$  \\
        $i=3$, Downward spread & $30 \left.\left(1-r_{\mathrm{down}}^{\left|h-h_{\subfire}\right|}\right)\right/\left(1-r_{\mathrm{down}}\right)$ & $5$  \\
        \bottomrule
    \end{tabular}
    \label{tab:fire_spread_parameters}
\end{table}

\figref{fig:t1_curve} illustrates the $t_1$ curves for various fire scenarios: (1) fire originating on the lower floor, $h-h_{\subfire}=1$ with rapid upward spread, (2) fire on the same floor, $h=h_{\subfire}$ with the fastest spread, and (3) fire on the upper floor, $h_{\subfire}-h=1$ with slow downward spread. The exponential decay in $t_1$ reflects the accelerating fire propagation speed as the distance increases. \figref{fig:t1_curve} also indicates that the employed simplified model is consistent with the Markov chain-based dynamic model given by \cite{cheng_dynamic_2011}, where the rooms at the same floor of the fire point start flashover slightly before the corresponding upper floors. Additionally, $\beta_{1}$ and $\beta_{3}$ are the summation of a geometric sequence, where story level $h$ is the index. The common ratios $r_{\mathrm{up}}<1$ in $\beta_{1}$ and $r_{\mathrm{down}}<1$ in $\beta_{3}$ indicate that the fire speeds up to spread through the next story, which is consistent with the real-world fire spread mechanism given in \cite{hokugo_mechanism_2000}. The temperature profile within the range $t \in [0, t_{\thre}]$ is subsequently used as the thermal load in OpenSeesRT simulations to compute displacements at each structural node at time $t_{\thre}$.
\begin{figure}[h!]
    \centering
    \includegraphics[width=0.8\linewidth]{figures/m204_t1_curve.pdf}
    \caption{Three examples for the $t_1$ curve.}
    \label{fig:t1_curve}
\end{figure}

\revise{
\textit{Remark:} The effects of structural elements, such as concrete floor slabs and partitions, are not explicitly modeled in our approach. Instead, their influence is implicitly captured through the careful selection of the parameters $ \alpha, \beta, r_\mathrm{up} $, and $ r_\mathrm{down} $. This parameterization provides a unified framework for generating temperature fields. Indeed, fire propagation is governed by a multitude of factors and remains an open research question. For instance, if the fire resistance of a floor slab is enhanced by fire protective coating, the corresponding model can account for this by decreasing $\alpha_1$ \& $\alpha_3$, increasing $\beta_1$ \& $\beta_3$, and adopting larger values for $r_\mathrm{up}$ \& $r_\mathrm{down}$, which collectively slow down the vertical spread of fire. Conversely, scenarios involving higher amounts of combustible materials would warrant the opposite adjustments. This flexible and integrated approach avoids the need to design separate models for different fire propagation scenarios while still capturing the essential effects.
}

\revise{
In conclusion, our rule-based approach is a computationally efficient method for approximating fire temperature fields, enabling large-scale dataset generation to train predictive models. By combining ISO 834 fire curves with spatial considerations and embedding structural effects through parameter calibration, the method achieves a balanced trade-off between accuracy and scalability, making it a practical solution for thermal load modeling in fire scenarios. After generating the temperature of each beam or column according to the middle point, the temperature is applied as uniform thermal load to the elements of the structure in question using OpenSeesRT. 
}

% In conclusion, this rule-based approach is a computationally efficient method to approximate fire temperature fields, enabling large-scale dataset generation to train predictive models. By combining ISO 834 fire curves with spatial considerations, the method balances accuracy and scalability, making it a practical solution for thermal load modeling in fire scenarios.

% \subsection{Interstory Drift Ratio}
\subsection{OpenSeesRT Simulation}
\label{subsec:opensees_simulation}

The thermal and mechanical responses of 3D frame structures under combined fire and gravity loads are simulated using OpenSeesRT \cite{perez2024openseesrt}. \revise{In the simulation, the IDR of each node at $t_{\thre}$ is computed using the computed nodal displacements. Each structural model features six degrees of freedom per node (3 translational  and 3 rotational), with linear geometrical transformations (\texttt{geomTransf: Linear}) defining how the element local coordinate systems are mapped to the global coordinate system and assuming small displacements and rotations. Although OpenSeesRT allows a variety of options for modeling finite deformations, in the present simulations and mainly for simplicity, we did not consider large deformations. All bottom nodes (nodes on the ground) are fully constrained in all six degrees of freedom, while degrees of freedom os all other nodes are free.} Material behavior is temperature-dependent and modeled with \texttt{Steel01Thermal}, while fiber-based sections (\texttt{FiberThermal}) capture nonlinear interactions between thermal and mechanical responses at the cross-section level. \revise{Structural elements are represented as displacement-based Euler-Bernoulli beam-columns (\texttt{dispBeamColumnThermal}). This element  formulation accounts for thermal strains (temperature gradients) in the section, which is discretized into fibers. Numerical integration is used along the length of each element using three integration (Gauss) points, one at each end and the third in the middle of the element.}

{\revise{Thermal expansion of steel members plays a crucial role in IDR development. In reality, reinforced concrete floor slabs heat at a different rate than steel members due to their higher thermal mass and lower thermal conductivity. This differential heating can lead to restrained thermal expansion, introducing axial compression in beams and affecting the overall structural response. In this study, explicit {\em{composite action}} between steel members and concrete slabs is not modeled. Instead, our approach focuses on isolating the response of the steel structural frame, which is often the critical load-bearing component in fire scenarios. This assumption aligns with prior studies \cite{Possidente_2024} demonstrating that steel structures reach thermal equilibrium with surrounding gases quickly, allowing the use of uniform thermal loading in fire analysis. Future work could enhance this framework by incorporating slab-beam interaction effects, through a refined FEA for an extended dataset where constraints imposed by floor slabs are explicitly considered.}

The analysis begins with the application of gravity loads, followed by incremental thermal loads simulating the fire exposure. A static nonlinear solver using  \texttt{ExpressNewton} algorithm ensures convergence, while the \texttt{NormDispIncr} test maintains accuracy. An incremental \texttt{LoadControl} scheme with small step sizes is employed to guarantee numerical stability, using 10\% for gravity loads and 1\% for thermal loads. 

\revise{
In the thermal load analysis, uniform thermal load is applied to each beam or column, i.e., the temperature of each element is set to be that at the middle point, according to \secref{subsec:thermal_load_generation}. The \texttt{Steel01Thermal} material allows the properties (e.g., Young's modulus and yield strength) to be adjusted at increasing temperatures according to \cite{EN1993} using its Table 3.1: Reduction factors for the stress-strain relationship of carbon steel at elevated temperatures. For example, if the Young’s modulus at ambient temperature is $E_0$, then as the temperature ($T$) increases, the modulus changes as $E(T) = \eta (T) \times E_0$. \cite{EN1993} directly provides the values of $\eta(T) \in \left[0,1\right] $ at every $100 ^\circ\mathrm{C}$ interval and recommends using linear interpolation to obtain $\eta(T)$ for intermediate values of $T$.
} OpenSeesRT documentation \cite{OpenSeesThermalExamples} provides several examples of thermal analyses.

This modeling framework accommodates variations in material properties, cross-sectional geometries, and temperature profiles, providing robust simulations of structural behavior under fire conditions. The primary settings and configurations for the OpenSeesRT simulations are summarized in \tabref{tab:ops_detail}.
\begin{table}[h!]
    \centering
        \caption{Key settings of OpenSeesRT simulations.}
    \begin{tabular}{l|>{\raggedright\arraybackslash}p{0.6\linewidth}} %
    \toprule
    Modeling Aspect     & Details \\
    \midrule
    Geometry            & 3D models; 6 degrees of freedom per node \\
    Transformation      & geomTransf: Linear \\ 
    Material            & Steel01Thermal \\
    Section             & FiberThermal; Cross-section: $0.1$ m $\times$ $0.1$ m \\ 
    Element type        & {dispBeamColumnThermal} \\ 
    Loading             & Gravity loads: {beamUniform}; Thermal loads: {beamThermal} \\
    Integration scheme  & Incremental {LoadControl}; Step size: $10\%$ (gravity analysis), $1\%$ (thermal analysis) \\
    Nonlinear solver    & {ExpressNewton} algorithm; {UmfPack} solver; Convergence test: {NormDispIncr} tolerance: $10^{-8}$; Maximum \# iterations per step: $1000$. \\ 
    \bottomrule
    \end{tabular}
    \label{tab:ops_detail}
\end{table}

For each structure in the labeled dataset, 30 fire points are selected using a dual-granularity approach, \revise{i.e., two-stage sampling strategy,} to ensure they are well-distributed. Specifically, rooms are sequentially selected, with one fire point randomly chosen within each selected room. If a building is large and contains more than 30 rooms, we randomly select 30 rooms without replacement, i.e., ensuring that no more than one fire point is located in the same room. Conversely, if the building is small and has fewer than 30 rooms, all rooms are initially selected, with one fire point randomly assigned to each room. Additionally, rooms are then selected with replacement until a total of 30 fire points are assigned. \revise{The room-level sampling prioritizes selecting distinct rooms to avoid spatial clustering of fire points, while the point-level sampling ensures intra-room variability. This approach aligns with stratified sampling principles commonly used for efficient spatial representation, where multi-stage sampling strategies optimize coverage and variability, e.g., \cite{arunachalam_generalized_2023}, and enables a more comprehensive characterizing of how the structures respond under fire conditions.}
% This selection method prevents fire points from clustering too closely while maintaining an element of randomness. By distributing fire points in this manner, the 30 fire scenarios are effectively utilized, enabling a more comprehensive characterizing of how the structures respond under fire conditions.

\subsection{Summary of the Dataset Generation}
As discussed in this section and related to  \figref{fig:dataset_generation_procedure}, three key steps were considered in the development of the dataset: 
\begin{enumerate}
    \item {\bf{Filtering process}}: Structures with MIDR exceeding $1\%$ under gravity loads were excluded,  resulting in $1,573$ labeled structures retained for fire simulation and $16,050$ unlabeled structures for training the MFSP predictor.
    \item {\bf{Fire simulations}}: For each retained labeled structure, 30 fire scenarios were simulated using OpenSeesRT, yielding $47,190$ fire cases.
    \item {\bf{Data distribution check}}: MIDR distributions for labeled and unlabeled data under gravity loads were highly similar, because both datasets were generated using the same method. Under fire conditions, the MIDR distribution shifted, reflecting significant structural deformation with values reaching a maximum of about 6\%, an average of 1.70\%, and a standard deviation of 1.12\%. This step ensured a diverse and comprehensive dataset for the proposed predictive framework.
\end{enumerate}
The statistical distribution histograms for MIDR (after applying the $1\%$ filtering threshold \revise{for gravity load responses}) under different loading conditions are plotted in \figref{fig:histogram_mdr}. Figures \ref{fig:histogram_mdr}(a) and \ref{fig:histogram_mdr}(b) show the MIDR distributions of the labeled and unlabeled data, respectively, under gravity loads only. \figref{fig:histogram_mdr}(c) shows the MIDR distribution of the labeled data under the combined effects of gravity and fire loads. Fire load causes the structures to significantly deform, leading to a noticeably \revise{right-skewed} MIDR distribution.

\begin{figure*}[h!]
    \centering
    \includegraphics[width=\linewidth]{figures/histogram_mdr.pdf}
    \caption{Histograms of MIDR for labeled and unlabeled structures with gravity loads and fire cases.}
    \label{fig:histogram_mdr}
\end{figure*}

\revise{
This dataset provides the basis for training and testing the performance of the GNN-based framework. Although we employed a simplified rule-based thermal load generation method compared with conventional CFD-based simulations, the temperature field, the changes of the material properties, and the response of the structures, are all still highly nonlinear and complex. Therefore, it is still a challenging task for the NN to predict the MIDRs based on this dataset.
}

% \begin{table}[!t]
% \caption{Results for Dunnhumby and Assessment datasets}
% \label{tab:dunnhumby}
% \centering
% \begin{tabular}{lcc}
% \hline
%     ~ & Dunnhumby & Assessment \\
%     ~ & Event type & Event type \\
%     ~ & ROC AUC $\uparrow$ & ROC AUC $\uparrow$ \\
%     \hline
% Without context & 0.762$\pm$0.002 & 0.859$\pm$0.002 \\
% Max & 0.763$\pm$0.001 & 0.858$\pm$0.002 \\
% Mean & \textbf{0.763$\pm$0.002} & 0.859$\pm$0.002 \\
% Exp Hawk. & 0.763$\pm$0.001 & 0.859$\pm$0.002 \\
% Learn. Exp Hawk. & 0.748$\pm$0.011 & 0.826$\pm$0.035 \\
% Attention & 0.760$\pm$0.001 & \textbf{0.860$\pm$0.003} \\
% Learnable atten. & \textbf{0.763$\pm$0.002} & 0.859$\pm$0.002 \\
% Symmetric atten. & \textbf{0.763$\pm$0.002} & 0.859$\pm$0.002 \\
% Attention Hawk. & 0.762$\pm$0.001 & 0.859$\pm$0.002 \\
% Kernel atten. & \textbf{0.763$\pm$0.002} & 0.859$\pm$0.002 \\
% \hline
% \multicolumn{3}{c}{Global target ROC AUC $\uparrow$} \\
% \hline
% Without context & \multicolumn{2}{c}{0.871$\pm$0.018} \\
% Max & \multicolumn{2}{c}{0.996$\pm$0.001} \\
% Mean & \multicolumn{2}{c}{0.996$\pm$0.001} \\
% Exp Hawk. & \multicolumn{2}{c}{0.996$\pm$0.001} \\
% Learn. Exp Hawk. & \multicolumn{2}{c}{0.996$\pm$0.001} \\
% Attention & \multicolumn{2}{c}{0.996$\pm$0.001} \\
% Learnable atten. & \multicolumn{2}{c}{0.996$\pm$0.001} \\
% Symmetric atten. & \multicolumn{2}{c}{0.996$\pm$0.001} \\
% Attention Hawk. & \multicolumn{2}{c}{0.996$\pm$0.001} \\
% Kernel atten. & \multicolumn{2}{c}{0.996$\pm$0.001} \\
% \hline
% \end{tabular}
% \end{table}


\begin{table}[!t]
\caption{ROC AUC $\uparrow$ metric for Dunnhumby and Assessment datasets}
\label{tab:dunnhumby}
\centering
\begin{tabular}{l|c|cc}
\hline
    % ~ & Dunnhumby & \multicolumn{2}{c}{Assessment } \\ 
    % \hline
    % ~ & \begin{tabular}{l} Event type\\ ROC AUC $\uparrow$ \end{tabular} & \begin{tabular}{l} Event type\\ ROC AUC $\uparrow$ \end{tabular}  & \begin{tabular}{l} Global target\\ ROC AUC $\uparrow$ \end{tabular} \\
    % \hline

& Dunnhumby & \multicolumn{2}{c}{Assessment } \\ 
& Event type & Event type & Global Target\\ 
\hline
Without context&0.762$\pm$0.002&0.859$\pm$0.002&0.871$\pm$0.018\\
Max&0.763$\pm$0.001&0.858$\pm$0.002&0.996$\pm$0.001\\
Mean&\textbf{0.763$\pm$0.002}&0.859$\pm$0.002&0.996$\pm$0.001\\
Exp Hawk.&0.763$\pm$0.001&0.859$\pm$0.002&0.996$\pm$0.001\\
Attention Hawk.&0.762$\pm$0.001&0.859$\pm$0.002&0.996$\pm$0.001\\
Attention&0.760$\pm$0.001&\textbf{0.860$\pm$0.003}&0.996$\pm$0.001\\
Learn. exp Hawk.*&0.748$\pm$0.011&0.826$\pm$0.035&0.996$\pm$0.001\\
Learnable atten.*&\textbf{0.763$\pm$0.002}&0.859$\pm$0.002&0.996$\pm$0.001\\
Symmetric atten.*&\textbf{0.763$\pm$0.002}&0.859$\pm$0.002&0.996$\pm$0.001\\
Kernel atten.*&\textbf{0.763$\pm$0.002}&0.859$\pm$0.002&0.996$\pm$0.001\\
\hline
\end{tabular}
\end{table}

\begin{table*}[!t]
\centering
% \scriptsize
\begin{tabular}{lcccccccc}\hline
Metric &\multicolumn{3}{c|}{ROC-AUC $\uparrow$} &\multicolumn{3}{c}{PR-AUC $\uparrow$} \\\hline
Dataset &Churn &Default &HSBC &Churn &Default &HSBC \\
\hline
\multicolumn{7}{c}{\textit{Global target}} \\
\hline
Without context & 0.7432 $\pm$ 0.009 & 0.5493 $\pm$ 0.026 & 0.693 $\pm$ 0.041 & 0.7922 $\pm$ 0.014 & 0.058 $\pm$ 0.02 & 0.154 $\pm$ 0.02 \\
  Mean & 0.7728 $\pm$ 0.004 & 0.5557 $\pm$ 0.004 & \underline{0.7335 $\pm$ 0.017} & 0.8277 $\pm$ 0.003 & 0.0564 $\pm$ 0.009 & \underline{0.2368 $\pm$ 0.038} \\
  Max & 0.7737 $\pm$ 0.021 & 0.5413 $\pm$ 0.018 & 0.7234 $\pm$ 0.02 & 0.8184 $\pm$ 0.032 & 0.0494 $\pm$ 0.005 & 0.2058 $\pm$ 0.022 \\
  Exp Hawkes & 0.7647 $\pm$ 0.008 & 0.5577 $\pm$ 0.029 & 0.6595 $\pm$ 0.031 & 0.814 $\pm$ 0.009 & \underline{0.0708 $\pm$ 0.039} & 0.1389 $\pm$ 0.018 \\
  Attention Hawkes & 0.7609 $\pm$ 0.007 & 0.5556 $\pm$ 0.022 & 0.6706 $\pm$ 0.022 & 0.7963 $\pm$ 0.009 & 0.0625 $\pm$ 0.024 & \textbf{0.331 $\pm$ 0.312} \\
  Attention & 0.7598 $\pm$ 0.014 & \textbf{0.5629 $\pm$ 0.008} & 0.6687 $\pm$ 0.013 & 0.8085 $\pm$ 0.017 & 0.0503 $\pm$ 0.005 & 0.1472 $\pm$ 0.021 \\
  Learnable exp Hawkes* & 0.7637 $\pm$ 0.008 & 0.5582 $\pm$ 0.024 & 0.6567 $\pm$ 0.019 & 0.812 $\pm$ 0.008 & 0.0613 $\pm$ 0.025 & 0.1282 $\pm$ 0.006 \\
  Learnable attention* & 0.7768 $\pm$ 0.013 & 0.5397 $\pm$ 0.019 & 0.7028 $\pm$ 0.008 & \underline{0.8303 $\pm$ 0.013} & \textbf{0.0727 $\pm$ 0.032 }& 0.2104 $\pm$ 0.014 \\
  Symmetric attention* & \textbf{0.7847 $\pm$ 0.01} & 0.5309 $\pm$ 0.004 & 0.7096 $\pm$ 0.022 & \textbf{0.8353 $\pm$ 0.005} & 0.055 $\pm$ 0.008 & 0.2318 $\pm$ 0.006 \\
  Kernel attention* & \underline{0.7745 $\pm$ 0.003} & \underline{0.5594 $\pm$ 0.003} & \textbf{0.7471 $\pm$ 0.018} & 0.8243 $\pm$ 0.002 & 0.0573 $\pm$ 0.003 & 0.2248 $\pm$ 0.035 \\
\hline
\multicolumn{7}{c}{\textit{Event type}} \\
\hline
Without context & 0.7004 $\pm$ 0.002 & \underline{0.7526 $\pm$ 0.002} & 0.898 $\pm$ 0.009 & 0.2033 $\pm$ 0.003 & 0.2644 $\pm$ 0.001 & 0.824 $\pm$ 0.007 \\
  Mean & \underline{0.7102 $\pm$ 0.003} & 0.7526 $\pm$ 0.001 & 0.9007 $\pm$ 0.003 & \underline{0.207 $\pm$ 0.003} & \underline{0.2656 $\pm$ 0.001} & \underline{0.8266 $\pm$ 0.004} \\
  Max & \textbf{0.7113 $\pm$ 0.003} & 0.7524 $\pm$ 0.001 & \textbf{0.9015 $\pm$ 0.003} & 0.2041 $\pm$ 0.003 & \underline{0.2656 $\pm$ 0.001} & 0.8241 $\pm$ 0.008 \\
  Exp Hawkes & 0.6812 $\pm$ 0.002 & 0.7523 $\pm$ 0.001 & 0.8978 $\pm$ 0.003 & 0.1919 $\pm$ 0.002 & \underline{0.2656 $\pm$ 0.001} & 0.826 $\pm$ 0.002 \\
  Attention Hawkes & 0.6848 $\pm$ 0.001 & 0.7523 $\pm$ 0.001 & 0.8978 $\pm$ 0.003 & 0.1905 $\pm$ 0.002 & \underline{0.2656 $\pm$ 0.001} & 0.8255 $\pm$ 0.002 \\
  Attention & 0.7095 $\pm$ 0.003 & 0.7487 $\pm$ 0.001 & 0.8996 $\pm$ 0.001 & 0.2055 $\pm$ 0.003 & 0.2613 $\pm$ 0.001 & 0.8274 $\pm$ 0.002 \\
  Learnable exp Hawkes* & 0.6806 $\pm$ 0.002 & 0.7522 $\pm$ 0.001 & 0.8985 $\pm$ 0.003 & 0.1898 $\pm$ 0.001 & \underline{0.2656 $\pm$ 0.001} & 0.8275 $\pm$ 0.002 \\
  Learnable attention* & \underline{0.7102 $\pm$ 0.002} & \textbf{0.7527 $\pm$ 0.001} & \underline{0.9009 $\pm$ 0.004} & \underline{0.207 $\pm$ 0.003} & \textbf{0.2658 $\pm$ 0.001} & 0.8272 $\pm$ 0.004 \\
  Symmetric attention* & \underline{0.7102 $\pm$ 0.003} & 0.7525 $\pm$ 0.001 & 0.8998 $\pm$ 0.004 & \textbf{0.2071 $\pm$ 0.003} & \underline{0.2654 $\pm$ 0.001} & 0.8257 $\pm$ 0.006 \\
  Kernel attention* & \underline{0.7102 $\pm$ 0.003} & \underline{0.7526 $\pm$ 0.001} & 0.9007 $\pm$ 0.003 & \underline{0.207 $\pm$ 0.003} & \underline{0.2656 $\pm$ 0.001} & \textbf{0.8268 $\pm$ 0.004} \\
\bottomrule
\end{tabular}
\caption{\selectfont Quality metrics for global and local embedding validation results. 
All metrics in the Table should be maximized. 
The results are averaged by three runs and are given in the format $mean \pm std$. 
The best values are \textbf{highlighted}, and the second-best values are \underline{underlined}.}
\label{tab:metrics}
\end{table*}


% \begin{table*}[ht]
% \centering
% % \scriptsize
% \begin{tabular}{lcccccccc}\hline
% Metric &\multicolumn{3}{c|}{ROC-AUC $\uparrow$} &\multicolumn{3}{c}{PR-AUC $\uparrow$} \\\hline
% Dataset &Churn &Default &HSBC &Churn &Default &HSBC \\
% \hline
% \multicolumn{7}{c}{\textit{Global target}} \\
% \hline
% Without context &0.7432 $\pm$ 0.0094 &0.5493 $\pm$ 0.026 &0.693 $\pm$ 0.041 &0.7922 $\pm$ 0.0144 &0.058 $\pm$ 0.0202 &0.154 $\pm$ 0.02 \\
% Max &0.7737 $\pm$ 0.0207 &0.5413 $\pm$ 0.0181 &0.7234 $\pm$ 0.0203 &0.8184 $\pm$ 0.032 &0.0494 $\pm$ 0.0054 &0.2058 $\pm$ 0.022 \\
% Mean &0.7728 $\pm$ 0.004 &0.5557 $\pm$ 0.0037 &\ul{0.7335 $\pm$ 0.0169} &0.8277 $\pm$ 0.0032 &0.0564 $\pm$ 0.0088 &\ul{0.2368 $\pm$ 0.0381} \\
% Exp Hawkes &0.7647 $\pm$ 0.0078 &0.5577 $\pm$ 0.0286 &0.6595 $\pm$ 0.0311 &0.814 $\pm$ 0.0092 &\ul{0.0708 $\pm$ 0.0386} &0.1389 $\pm$ 0.0176 \\
% Exp learnable Hawkes &0.7637 $\pm$ 0.0082 &0.5582 $\pm$ 0.0236 &0.6567 $\pm$ 0.019 &0.812 $\pm$ 0.0083 &0.0613 $\pm$ 0.0249 &0.1282 $\pm$ 0.0061 \\
% Attention &0.7598 $\pm$ 0.0135 &\textbf{0.5629 $\pm$ 0.0084} &0.6687 $\pm$ 0.0126 &0.8085 $\pm$ 0.0173 &0.0503 $\pm$ 0.0046 &0.1472 $\pm$ 0.0206 \\
% Learnable attention &0.7768 $\pm$ 0.0128 &0.5397 $\pm$ 0.019 &0.7028 $\pm$ 0.0077 &\ul{0.8303 $\pm$ 0.0129} &\textbf{0.0727 $\pm$ 0.0316} &0.2104 $\pm$ 0.0144 \\
% Symmetric attention &\textbf{0.7847 $\pm$ 0.0097} &0.5309 $\pm$ 0.0038 &0.7096 $\pm$ 0.0223 &\textbf{0.8353 $\pm$ 0.005} &0.055 $\pm$ 0.0083 &0.2318 $\pm$ 0.0059 \\
% Kernel attention &\ul{0.7745 $\pm$ 0.0026} &\ul{0.5594 $\pm$ 0.0034} &\textbf{0.7471 $\pm$ 0.0184} &0.8243 $\pm$ 0.0018 &0.0573 $\pm$ 0.0031 &0.2248 $\pm$ 0.0352 \\
% Attention Hawkes &0.7609 $\pm$ 0.007 &0.5556 $\pm$ 0.0217 &0.6706 $\pm$ 0.0218 &0.7963 $\pm$ 0.0093 &0.0625 $\pm$ 0.0239 &\textbf{0.331 $\pm$ 0.3117} \\
% \hline
% \multicolumn{7}{c}{\textit{Event type}} \\
% \hline
% Without context &0.6529 $\pm$ 0.0017 &0.7448 $\pm$ 0.0024 &0.898 $\pm$ 0.009 &0.1676 $\pm$ 0.0008 &0.26 $\pm$ 0.0023 &0.824 $\pm$ 0.007 \\
% Max &0.6921 $\pm$ 0.0035 &0.7459 $\pm$ 0.001 &\ul{0.9015 $\pm$ 0.0029} &0.1916 $\pm$ 0.001 &0.2603 $\pm$ 0.0013 &0.8241 $\pm$ 0.0075 \\
% Mean &0.6882 $\pm$ 0.0045 &\textbf{0.7464 $\pm$ 0.0007} &0.9007 $\pm$ 0.003 &0.193 $\pm$ 0.0032 &\ul{0.2614 $\pm$ 0.0009} &0.8266 $\pm$ 0.0045 \\
% Exp Hawkes &0.6355 $\pm$ 0.0032 &0.7452 $\pm$ 0.0013 &0.8978 $\pm$ 0.0027 &0.1615 $\pm$ 0.0017 &0.2608 $\pm$ 0.0009 &0.826 $\pm$ 0.0021 \\
% Exp learnable Hawkes &0.6129 $\pm$ 0.0121 &0.7452 $\pm$ 0.0013 &0.8985 $\pm$ 0.0027 &0.1531 $\pm$ 0.0068 &0.2609 $\pm$ 0.0009 &\textbf{0.8275 $\pm$ 0.0015} \\
% Attention &0.6914 $\pm$ 0.0041 &0.7401 $\pm$ 0.001 &0.8996 $\pm$ 0.0013 &\textbf{0.1941 $\pm$ 0.0039} &0.2533 $\pm$ 0.0023 &\ul{0.8274 $\pm$ 0.0016} \\
% Learnable attention &\textbf{0.6896 $\pm$ 0.0027} &\ul{0.7463 $\pm$ 0.0008} &\textbf{0.9009 $\pm$ 0.0035} &\ul{0.1935 $\pm$ 0.0036} &\textbf{0.2615 $\pm$ 0.0008} &0.8272 $\pm$ 0.0036 \\
% Symmetric attention &0.6893 $\pm$ 0.0029 &0.7462 $\pm$ 0.0006 &0.8998 $\pm$ 0.0036 &0.1932 $\pm$ 0.004 &0.2611 $\pm$ 0.0009 &0.8257 $\pm$ 0.0057 \\
% Attention Hawkes &0.6351 $\pm$ 0.0012 &0.7453 $\pm$ 0.0014 &0.8978 $\pm$ 0.0028 &0.1591 $\pm$ 0.0022 &0.2608 $\pm$ 0.0009 &0.8255 $\pm$ 0.0022 \\
% Kernel attention &\ul{0.6894 $\pm$ 0.0029} &\textbf{0.7464 $\pm$ 0.0007} &0.9007 $\pm$ 0.0031 &0.1931 $\pm$ 0.0039 &\ul{0.2614 $\pm$ 0.0008} &0.8268 $\pm$ 0.0045 \\
% \bottomrule
% \end{tabular}
% \caption{\selectfont Quality metrics for global and local embedding validation results. 
% All metrics in the Table should be maximized. 
% The results are averaged by three runs and are given in the format $mean \pm std$. 
% The best values are \textbf{highlighted}, and the second-best values are \underline{underlined}.}
% \label{tab:metrics}
% \end{table*}
% \begin{table*}[h]
% \caption{\linespread{1.5}\selectfont Average ranks (by ROC-AUC metric) for both datasets. 
% All metrics in the Table should be minimized. 
% The best values are \textbf{highlighted}, the second-best values are \underline{underlined}, and the third-best values are \doubleunderline{double underlined}}
% \label{tab:ranks}
% \resizebox{\textwidth}{!}{%
% \begin{tabular}{l|c|c|c|c|c|c|c}
% \hline
% & \multicolumn{2}{c}{\textit{Global target validation}} \vline & \multicolumn{2}{c}{\textit{Local target validation}} \vline & \multicolumn{2}{c}{\textit{Next event type validation}} \vline & \textit{Mean} \\ \hline
% & ~~~CoLES~~~ & AR & ~~~CoLES~~~ & AR & ~~~CoLES~~~ & AR\\ \hline
% Without context & 8.5 & 9.5 & 9.5 & 7 & 7.25 & 9.25 & 8.5\\
% Mean & \doubleunderline{5.25} & \doubleunderline{3.75} & \textbf{3.25} & 5 & 4.75 & \textbf{1.75} & \textbf{3.958}\\
% Max & 6 & \underline{2.5} & 6 & 6 & \textbf{2.25} & \underline{2.75} & \underline{4.25}\\
% Attention & 5 & 8.5 & \underline{4.5} & 7 & 6 & 6.25 & 6.208\\
% Learn. attention & 5.5 & 7.5 & \doubleunderline{4.75} & \underline{3.5} & \doubleunderline{3.25} & 6.75 & 5.208\\
% Sym. attention & 5.5 & \textbf{1.75} & 5.25 & 6 & 4 & 6.25 & \doubleunderline{4.792}\\
% Kernel attention & \textbf{2.5} & 8.5 & \doubleunderline{4.75} & 6 & \underline{2.75} & 5.25 & 4.958\\
% Exp Hawkes & \underline{4.75} & 4 & \underline{4.5} & \doubleunderline{4.5} & 8 & \doubleunderline{4.25} & 5\\
% Exp learn. Hawkes & \doubleunderline{5.25} & 4 & 8 & 7.5 & 8.75 & 7.5 & 6.833\\
% Attention Hawkes & 6.75 & 5 & \underline{4.5} & \textbf{2.5} & 8 & 5 & 5.292 \\
% \hline
% \end{tabular}}
% \end{table*}


\begin{table*}[!t]
\caption{\selectfont Summarized ranks ($\downarrow$) based on ROC-AUC metric for all the datasets. Smaller ranks are better. All metrics have been rounded to four decimal places before getting ranks. 
The best values are \textbf{highlighted}, the second-best values are \underline{underlined}.}
\label{tab:ranks}
%\resizebox{\textwidth}{!}
{%
\begin{tabular}{l|c|c|c|c|c|c|c}\toprule
Dataset &\multicolumn{2}{c}{Churn} \vline &\multicolumn{2}{c}{Default} \vline &\multicolumn{2}{c}{HSBC} \vline & \multirow{2}*{Mean} \\ 
Target & Event type & Global target & Event type & Global target & Event type & Global target & \\ \hline
Without context & 7.0 & 10.0 & \underline{3.0} & 7.0 & 8.0 & 6.0 & 6.833 \\
Mean & \underline{3.5} & 5.0 & \underline{3.0} & 5.0 & 3.5 & \underline{2.0} & 3.667 \\
Max & \textbf{1.0} & 4.0 & 6.0 & 8.0 & \textbf{1.0} & 3.0 & 3.833 \\
Exp Hawkes & 9.0 & 6.0 & 7.5 & 4.0 & 9.5 & 9.0 & 7.500 \\
Attention Hawkes & 8.0 & 8.0 & 7.5 & 6.0 & 9.5 & 7.0 & 7.667 \\
Attention & 6.0 & 9.0 & 10.0 & \textbf{1.0} & 6.0 & 8.0 & 6.667 \\
Learnable exp Hawkes* & 10.0 & 7.0 & 9.0 & 3.0 & 7.0 & 10.0 & 7.667 \\
Learnable attention* & \underline{3.5} & \underline{2.0} & \textbf{1.0} & 9.0 & \underline{2.0} & 5.0 & 3.750 \\
Symmetric attention* & \underline{3.5} & \textbf{1.0} & 5.0 & 10.0 & 5.0 & 4.0 & 4.750 \\
Kernel attention* & \underline{3.5} & 3.0 & \underline{3.0} & \underline{2.0} & 3.5 & \textbf{1.0} & 2.667 \\
\bottomrule
\end{tabular}}
\end{table*}


% \begin{table*}[ht]
% \caption{\selectfont Summarized ranks based on ROC-AUC metric for all the datasets. Smaller ranks are better. 
% The best values are \textbf{highlighted}, the second-best values are \underline{underlined}.}
% \label{tab:ranks}
% %\resizebox{\textwidth}{!}
% {%
% \begin{tabular}{l|c|c|c|c|c|c|c}\toprule
% Dataset &\multicolumn{2}{c}{Churn} \vline &\multicolumn{2}{c}{Default} \vline &\multicolumn{2}{c}{HSBC} \vline & Mean \\ \hline
% Target & Global target & Event type & Global target & Event type & Global target & Event type & \\ \hline
% Without context & 10.0 & 7.0 & 7.0 & 9.0 & 6.0 & 8.0 & 7.833 \\
% Max & 4.0 & \textbf{1.0} & 8.0 & 5.0 & 3.0 & \textbf{1.0} & \underline{3.667} \\
% Mean & 5.0 & 6.0 & 5.0 & \textbf{1.5} & \underline{2.0} & 3.5 & 3.833 \\
% Exp Hawkes & 6.0 & 8.0 & 4.0 & 7.5 & 9.0 & 9.5 & 7.333 \\
% Learnable Exp Hawkes & 7.0 & 10.0 & 3.0 & 7.5 & 10.0 & 7.0 & 7.417 \\
% Attention & 9.0 & \underline{2.0} & \textbf{1.0} & 10.0 & 8.0 & 6.0 & 6.000 \\
% Learnable attention & \underline{2.0} & 3.0 & 9.0 &\underline{3.0} &5.0 &\underline{2.0} & 4.000 \\
% Symmetric attention & \textbf{1.0} & 5.0 & 10.0 & 4.0 & 4.0 & 5.0 & 4.833 \\
% Attention Hawkes & 8.0 & 9.0 & 6.0 & 6.0 & 7.0 & 9.5 & 7.583 \\
% Kernel attention & 3.0 & 4.0 & \underline{2.0} & \textbf{1.5} & \textbf{1.0} & 3.5 & \textbf{2.500} \\
% \bottomrule
% \end{tabular}}
% \end{table*}


\emph{HSBC} dataset~\cite{bazarova2024universal} originates from a competition on the fraud identification task.
Each transaction in the dataset has a mark that indicates whether this transaction was made by the client personally or maliciously. 
We add the global target, which identifies whether at least one transaction in sequence was fake for each user.
With fewer identified fraudulent transactions, the target labels are imbalanced, similarly to \emph{Default}.

Table~\ref{tab:datasets} summarizes the main characteristics of the dataset.


\section{Detailed metrics for main experiments} 
\label{sec:AppMainTable}
% TODO: if we have time, make text nicer

Here, we provided an expanded version of the results for financial datasets with additional metrics.
ROC AUC and PR AUC values are available in Table~\ref{tab:metrics}.


Precise metrics and measurements of PR AUC values confirm our finding from the main part.
We also complement these results with ranks of the considered approach and their average in Table~\ref{tab:ranks}.
They also confirm the lead of \emph{Kernel attention}, while simpler approaches like \emph{Mean} provide a strong alternative.



\section{Non-financial domain dataset}
\label{sec:other-data-results}
We also conducted experiments for several non-transactional datasets. 

% % \begin{table}[!t]
% \caption{Results for Dunnhumby and Assessment datasets}
% \label{tab:dunnhumby}
% \centering
% \begin{tabular}{lcc}
% \hline
%     ~ & Dunnhumby & Assessment \\
%     ~ & Event type & Event type \\
%     ~ & ROC AUC $\uparrow$ & ROC AUC $\uparrow$ \\
%     \hline
% Without context & 0.762$\pm$0.002 & 0.859$\pm$0.002 \\
% Max & 0.763$\pm$0.001 & 0.858$\pm$0.002 \\
% Mean & \textbf{0.763$\pm$0.002} & 0.859$\pm$0.002 \\
% Exp Hawk. & 0.763$\pm$0.001 & 0.859$\pm$0.002 \\
% Learn. Exp Hawk. & 0.748$\pm$0.011 & 0.826$\pm$0.035 \\
% Attention & 0.760$\pm$0.001 & \textbf{0.860$\pm$0.003} \\
% Learnable atten. & \textbf{0.763$\pm$0.002} & 0.859$\pm$0.002 \\
% Symmetric atten. & \textbf{0.763$\pm$0.002} & 0.859$\pm$0.002 \\
% Attention Hawk. & 0.762$\pm$0.001 & 0.859$\pm$0.002 \\
% Kernel atten. & \textbf{0.763$\pm$0.002} & 0.859$\pm$0.002 \\
% \hline
% \multicolumn{3}{c}{Global target ROC AUC $\uparrow$} \\
% \hline
% Without context & \multicolumn{2}{c}{0.871$\pm$0.018} \\
% Max & \multicolumn{2}{c}{0.996$\pm$0.001} \\
% Mean & \multicolumn{2}{c}{0.996$\pm$0.001} \\
% Exp Hawk. & \multicolumn{2}{c}{0.996$\pm$0.001} \\
% Learn. Exp Hawk. & \multicolumn{2}{c}{0.996$\pm$0.001} \\
% Attention & \multicolumn{2}{c}{0.996$\pm$0.001} \\
% Learnable atten. & \multicolumn{2}{c}{0.996$\pm$0.001} \\
% Symmetric atten. & \multicolumn{2}{c}{0.996$\pm$0.001} \\
% Attention Hawk. & \multicolumn{2}{c}{0.996$\pm$0.001} \\
% Kernel atten. & \multicolumn{2}{c}{0.996$\pm$0.001} \\
% \hline
% \end{tabular}
% \end{table}


\begin{table}[!t]
\caption{ROC AUC $\uparrow$ metric for Dunnhumby and Assessment datasets}
\label{tab:dunnhumby}
\centering
\begin{tabular}{l|c|cc}
\hline
    % ~ & Dunnhumby & \multicolumn{2}{c}{Assessment } \\ 
    % \hline
    % ~ & \begin{tabular}{l} Event type\\ ROC AUC $\uparrow$ \end{tabular} & \begin{tabular}{l} Event type\\ ROC AUC $\uparrow$ \end{tabular}  & \begin{tabular}{l} Global target\\ ROC AUC $\uparrow$ \end{tabular} \\
    % \hline

& Dunnhumby & \multicolumn{2}{c}{Assessment } \\ 
& Event type & Event type & Global Target\\ 
\hline
Without context&0.762$\pm$0.002&0.859$\pm$0.002&0.871$\pm$0.018\\
Max&0.763$\pm$0.001&0.858$\pm$0.002&0.996$\pm$0.001\\
Mean&\textbf{0.763$\pm$0.002}&0.859$\pm$0.002&0.996$\pm$0.001\\
Exp Hawk.&0.763$\pm$0.001&0.859$\pm$0.002&0.996$\pm$0.001\\
Attention Hawk.&0.762$\pm$0.001&0.859$\pm$0.002&0.996$\pm$0.001\\
Attention&0.760$\pm$0.001&\textbf{0.860$\pm$0.003}&0.996$\pm$0.001\\
Learn. exp Hawk.*&0.748$\pm$0.011&0.826$\pm$0.035&0.996$\pm$0.001\\
Learnable atten.*&\textbf{0.763$\pm$0.002}&0.859$\pm$0.002&0.996$\pm$0.001\\
Symmetric atten.*&\textbf{0.763$\pm$0.002}&0.859$\pm$0.002&0.996$\pm$0.001\\
Kernel atten.*&\textbf{0.763$\pm$0.002}&0.859$\pm$0.002&0.996$\pm$0.001\\
\hline
\end{tabular}
\end{table}

\href{https://www.dunnhumby.com/source-files/}{\textbf{Dunnhumby-Carbo.}} This dataset includes transactional data of households in a retail store. It contains more than 500k transactions over two years.

For this dataset, we consider only the next event-type prediction task. Results are presented in the Table~\ref{tab:dunnhumby}. Mean, max and different variations of the learnable attention methods outperform the baseline, while differences are pretty small --- we assume, that the data nature prevents us from significantly benefiting from aggregation.

\href{https://www.kaggle.com/c/data-science-bowl-2019}{\textbf{Assessment.}}
The dataset contains the history of children’s gameplay data. 
It consists of 12M gameplay events combined into 330K gameplays representing 18K children, but only 17.7K gameplays are labelled. 
Each gameplay event is characterized by a timestamp, an event code, an incremental counter of events within a game session, the time since the start of the game session, etc.

For this dataset, we have considered two tasks. The first is the next-event type prediction, and the second is the prediction of the in-game assessment results. For the second task, the target is one of four grades, with proportions of $0.50$, $0.24$, $0.14$, and $0.12$. For the first task, we use the full dataset for validation, and for the second, only its labelled part.

For external information aggregation, we consider each gameplay as a separate user.

% \begin{figure}[!ht] 
%       \centering
%           \includegraphics[width=1.0\columnwidth]{figures/assessment.pdf}
%           \caption{\selectfont ROC AUC for next event prediction task for \textit{Assessment} dataset. }\label{fig:assessment}
% \end{figure}

Results are also presented in Table \ref{tab:dunnhumby}. All aggregation techniques provide a significant boost for global target validation but a very insignificant difference for the next event-type prediction task. This can be explained by the fact that different children play individually; that is, they do not interact in any way and do not influence each other during the game. Therefore, the results of the event type prediction task, which depends on the time component, remains unchanged. 
On the other hand, the change of metrics for the global target validation task may be due to the fact that this target is a grade for a game, and the addition of other people’s embeddings allows comparing them. If the model can highlight the difference between the grades of the current person and the average grade, then the assessment of absolute values becomes easier to predict.

% \begin{table}[!t]
% \caption{Results for Assessment dataset}
% \label{tab:assessment}
% \centering
% \begin{tabular}{lcc}
% \hline
% ~ & \begin{tabular}{l}
%      Event type\\
%      ROC AUC
% \end{tabular}  & \begin{tabular}{l}
%      Global target\\
%      ROC AUC
% \end{tabular} \\
% \hline
% Without context & 0.8588 $\pm$ 0.0020 & 0.8711 $\pm$ 0.0182 \\
% Max &   & \textbf{0.9962 $\pm$ 0.0012} \\
% Mean &  0.8586 $\pm$ 0.0024 & \textbf{0.9962 $\pm$ 0.0012} \\
% Exp Hawkes  &  0.8586 $\pm$ 0.0023 & \textbf{0.9962 $\pm$ 0.0012} \\
% Learnable Exp Hawkes &   & 0.9958 $\pm$ 0.0014 \\
% Attention & 0.8602 $\pm$ 0.0025 & 0.9958 $\pm$ 0.0013 \\
% Learnable attention &   & 0.9957 $\pm$ 0.0012 \\
% Symmetric attention &   & 0.9959 $\pm$ 0.0013 \\
% Attention Hawkes & 0.8593 $\pm$ 0.0016  & 0.9958 $\pm$ 0.0013 \\
% Kernel attention &   & 0.9960 $\pm$ 0.0013 \\

% \hline
% \end{tabular}
% \end{table}