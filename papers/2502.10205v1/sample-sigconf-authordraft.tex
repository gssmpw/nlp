%%
%% This is file `sample-sigconf-authordraft.tex',
%% generated with the docstrip utility.
%%
%% The original source files were:
%%
%% samples.dtx  (with options: `all,proceedings,bibtex,authordraft')
%% 
%% IMPORTANT NOTICE:
%% 
%% For the copyright see the source file.
%% 
%% Any modified versions of this file must be renamed
%% with new filenames distinct from sample-sigconf-authordraft.tex.
%% 
%% For distribution of the original source see the terms
%% for copying and modification in the file samples.dtx.
%% 
%% This generated file may be distributed as long as the
%% original source files, as listed above, are part of the
%% same distribution. (The sources need not necessarily be
%% in the same archive or directory.)
%%
%%
%% Commands for TeXCount
%TC:macro \cite [option:text,text]
%TC:macro \citep [option:text,text]
%TC:macro \citet [option:text,text]
%TC:envir table 0 1
%TC:envir table* 0 1
%TC:envir tabular [ignore] word
%TC:envir displaymath 0 word
%TC:envir math 0 word
%TC:envir comment 0 0
%%
%%
%% The first command in your LaTeX source must be the \documentclass
%% command.
%%
%% For submission and review of your manuscript please change the
%% command to \documentclass[manuscript, screen, review]{acmart}.
%%
%% When submitting camera ready or to TAPS, please change the command
%% to \documentclass[sigconf]{acmart} or whichever template is required
%% for your publication.
%%
%%
\documentclass[sigconf,authordraft]{acmart}

%%
%% \BibTeX command to typeset BibTeX logo in the docs
\AtBeginDocument{%
  \providecommand\BibTeX{{%
    Bib\TeX}}}

%% Rights management information.  This information is sent to you
%% when you complete the rights form.  These commands have SAMPLE
%% values in them; it is your responsibility as an author to replace
%% the commands and values with those provided to you when you
%% complete the rights form.
\setcopyright{acmlicensed}
\copyrightyear{2018}
\acmYear{2018}
\acmDOI{XXXXXXX.XXXXXXX}

%% These commands are for a PROCEEDINGS abstract or paper.
\acmConference[Conference acronym 'XX]{Make sure to enter the correct
  conference title from your rights confirmation emai}{June 03--05,
  2018}{Woodstock, NY}
%%
%%  Uncomment \acmBooktitle if the title of the proceedings is different
%%  from ``Proceedings of ...''!
%%
%%\acmBooktitle{Woodstock '18: ACM Symposium on Neural Gaze Detection,
%%  June 03--05, 2018, Woodstock, NY}
\acmISBN{978-1-4503-XXXX-X/18/06}


%%
%% Submission ID.
%% Use this when submitting an article to a sponsored event. You'll
%% receive a unique submission ID from the organizers
%% of the event, and this ID should be used as the parameter to this command.
%%\acmSubmissionID{123-A56-BU3}

%%
%% For managing citations, it is recommended to use bibliography
%% files in BibTeX format.
%%
%% You can then either use BibTeX with the ACM-Reference-Format style,
%% or BibLaTeX with the acmnumeric or acmauthoryear sytles, that include
%% support for advanced citation of software artefact from the
%% biblatex-software package, also separately available on CTAN.
%%
%% Look at the sample-*-biblatex.tex files for templates showcasing
%% the biblatex styles.
%%

%%
%% The majority of ACM publications use numbered citations and
%% references.  The command \citestyle{authoryear} switches to the
%% "author year" style.
%%
%% If you are preparing content for an event
%% sponsored by ACM SIGGRAPH, you must use the "author year" style of
%% citations and references.
%% Uncommenting
%% the next command will enable that style.
%%\citestyle{acmauthoryear}


%%
%% end of the preamble, start of the body of the document source.
\begin{document}

%%
%% The "title" command has an optional parameter,
%% allowing the author to define a "short title" to be used in page headers.
\title{Addition of external information for enhancement of local embeddings for event sequences data models}

%%
%% The "author" command and its associated commands are used to define
%% the authors and their affiliations.
%% Of note is the shared affiliation of the first two authors, and the
%% "authornote" and "authornotemark" commands
%% used to denote shared contribution to the research.
\author{Maria Kovaleva}
\email{M.Kovaleva@skoltech.ru}
\affiliation{%
  \institution{Skoltech}
  \city{Moscow}
  \country{Russia}}
\orcid{0000-0002-2633-8849}

\author{Petr Sokerin}
\affiliation{%
  \institution{Skoltech}
  \city{Moscow}
  \country{Russia}}
\email{P.Sokerin@skoltech.ru}
\orcid{0000-0002-2633-8849}

\author{Alexey Zaytsev}
\affiliation{%
  \institution{Skoltech}
  \city{Moscow}
  \country{Russia}}
\email{A.Zaytsev@skoltech.ru}
\orcid{0000-0002-1653-0204}


%%
%% By default, the full list of authors will be used in the page
%% headers. Often, this list is too long, and will overlap
%% other information printed in the page headers. This command allows
%% the author to define a more concise list
%% of authors' names for this purpose.
\renewcommand{\shortauthors}{Kovaleva et al.}

%%
%% The abstract is a short summary of the work to be presented in the
%% article.


\begin{abstract}
\begin{abstract}  
Test time scaling is currently one of the most active research areas that shows promise after training time scaling has reached its limits.
Deep-thinking (DT) models are a class of recurrent models that can perform easy-to-hard generalization by assigning more compute to harder test samples.
However, due to their inability to determine the complexity of a test sample, DT models have to use a large amount of computation for both easy and hard test samples.
Excessive test time computation is wasteful and can cause the ``overthinking'' problem where more test time computation leads to worse results.
In this paper, we introduce a test time training method for determining the optimal amount of computation needed for each sample during test time.
We also propose Conv-LiGRU, a novel recurrent architecture for efficient and robust visual reasoning. 
Extensive experiments demonstrate that Conv-LiGRU is more stable than DT, effectively mitigates the ``overthinking'' phenomenon, and achieves superior accuracy.
\end{abstract}  
\end{abstract}

%%
%% The code below is generated by the tool at http://dl.acm.org/ccs.cfm.
%% Please copy and paste the code instead of the example below.
%%
\begin{CCSXML}
<ccs2012>
 <concept>
  <concept_id>00000000.0000000.0000000</concept_id>
  <concept_desc>Do Not Use This Code, Generate the Correct Terms for Your Paper</concept_desc>
  <concept_significance>500</concept_significance>
 </concept>
 <concept>
  <concept_id>00000000.00000000.00000000</concept_id>
  <concept_desc>Do Not Use This Code, Generate the Correct Terms for Your Paper</concept_desc>
  <concept_significance>300</concept_significance>
 </concept>
 <concept>
  <concept_id>00000000.00000000.00000000</concept_id>
  <concept_desc>Do Not Use This Code, Generate the Correct Terms for Your Paper</concept_desc>
  <concept_significance>100</concept_significance>
 </concept>
 <concept>
  <concept_id>00000000.00000000.00000000</concept_id>
  <concept_desc>Do Not Use This Code, Generate the Correct Terms for Your Paper</concept_desc>
  <concept_significance>100</concept_significance>
 </concept>
</ccs2012>
\end{CCSXML}

\ccsdesc[500]{Do Not Use This Code~Generate the Correct Terms for Your Paper}
\ccsdesc[300]{Do Not Use This Code~Generate the Correct Terms for Your Paper}
\ccsdesc{Do Not Use This Code~Generate the Correct Terms for Your Paper}
\ccsdesc[100]{Do Not Use This Code~Generate the Correct Terms for Your Paper}

%%
%% Keywords. The author(s) should pick words that accurately describe
%% the work being presented. Separate the keywords with commas.
\keywords{Do, Not, Us, This, Code, Put, the, Correct, Terms, for,
  Your, Paper}
%% A "teaser" image appears between the author and affiliation
%% information and the body of the document, and typically spans the
%% page.


\received{20 February 2007}
\received[revised]{12 March 2009}
\received[accepted]{5 June 2009}

%%
%% This command processes the author and affiliation and title
%% information and builds the first part of the formatted document.
\maketitle

\begin{Introduction}
\section{Introduction}\label{sec:Intro} 


Novel view synthesis offers a fundamental approach to visualizing complex scenes by generating new perspectives from existing imagery. 
This has many potential applications, including virtual reality, movie production and architectural visualization \cite{Tewari2022NeuRendSTAR}. 
An emerging alternative to the common RGB sensors are event cameras, which are  
 bio-inspired visual sensors recording events, i.e.~asynchronous per-pixel signals of changes in brightness or color intensity. 

Event streams have very high temporal resolution and are inherently sparse, as they only happen when changes in the scene are observed. 
Due to their working principle, event cameras bring several advantages, especially in challenging cases: they excel at handling high-speed motions 
and have a substantially higher dynamic range of the supported signal measurements than conventional RGB cameras. 
Moreover, they have lower power consumption and require varied storage volumes for captured data that are often smaller than those required for synchronous RGB cameras \cite{Millerdurai_3DV2024, Gallego2022}. 

The ability to handle high-speed motions is crucial in static scenes as well,  particularly with handheld moving cameras, as it helps avoid the common problem of motion blur. It is, therefore, not surprising that event-based novel view synthesis has gained attention, although color values are not directly observed.
Notably, because of the substantial difference between the formats, RGB- and event-based approaches require fundamentally different design choices. %

The first solutions to event-based novel view synthesis introduced in the literature demonstrate promising results \cite{eventnerf, enerf} and outperform non-event-based alternatives for novel view synthesis in many challenging scenarios. 
Among them, EventNeRF \cite{eventnerf} enables novel-view synthesis in the RGB space by assuming events associated with three color channels as inputs. 
Due to its NeRF-based architecture \cite{nerf}, it can handle single objects with complete observations from roughly equal distances to the camera. 
It furthermore has limitations in training and rendering speed: 
the MLP used to represent the scene requires long training time and can only handle very limited scene extents or otherwise rendering quality will deteriorate. 
Hence, the quality of synthesized novel views will degrade for larger scenes. %

We present Event-3DGS (E-3DGS), i.e.,~a new method for novel-view synthesis from event streams using 3D Gaussians~\cite{3dgs} 
demonstrating fast reconstruction and rendering as well as handling of unbounded scenes. 
The technical contributions of this paper are as follows: 
\begin{itemize}
\item With E-3DGS, we introduce the first approach for novel view synthesis from a color event camera that combines 3D Gaussians with event-based supervision. 
\item We present frustum-based initialization, adaptive event windows, isotropic 3D Gaussian regularization and 3D camera pose refinement, and demonstrate that high-quality results can be obtained. %

\item Finally, we introduce new synthetic and real event datasets for large scenes to the community to study novel view synthesis in this new problem setting. 
\end{itemize}
Our experiments demonstrate systematically superior results compared to EventNeRF \cite{eventnerf} and other baselines. 
The source code and dataset of E-3DGS are released\footnote{\url{https://4dqv.mpi-inf.mpg.de/E3DGS/}}. 





\end{Introduction}

\begin{Literature_review}
\section{Related work}
\label{lit_rewiev}

We review the approaches to external context aggregation in various data modalities and methods for evaluating sequence representations. 
The review also considers existing representation learning methods for event sequences, including self-supervised methods for general and bank transaction domains, to provide further background.

\subsection{Event sequence representation learning}
Neural networks have been shown to perform well when faced with the task of event sequence representation learning~\cite{babaev2022coles}.
Obtained representations are helpful both for tasks describing local in-time properties, such as next event prediction~\cite{zhuzhel2023continuous} or change point detection~\cite{deldari2021time,ala2022deep} and whole sequence classification~\cite{bin2022review,jaiswal2020survey}.

\paragraph{Self-supervised learning} (SSL) is a solid representation learning paradigm that learns an encoder --- neural network, providing representation vector, without labelled data. 
This allows researchers to skip costly and possibly limited human expert annotation and leverage the large bodies of low-cost unlabeled data, leading to more universal representations~\cite{caron2021emerging}.
This paradigm is often implemented within contrastive and generative learning frameworks~\cite{zhang2020self,liu2023ssl}. 

In contrastive learning, the encoder learns to produce embeddings that are close if objects are labelled as similar and distant, and vice versa. 
%allow distinguishing between objects that are labelled as close or distant in a self-supervised manner.
It originated in computer vision in the form of Siamese loss~\cite{hoffer2015deep} and SimCLR~\cite{chen2020simple} with subsequent DINO \cite{caron2021emerging} and BarlowTwins~\cite{zbontar2021barlow} among others.
It also allows the production of meaningful representations for time series and event sequence data~\cite{zhang2024self}.
% These approaches are also gaining popularity in the field of multidimensional event sequences and time series~\cite{babaev2022coles, li_deep_2020, romanenkova2022similarity, moskvoretskii2024self, li2023new}. 
Our work also uses a contrastive representation model as it performs well~\cite{babaev2022coles} for event sequences data, producing universal representations. 
The authors of the method studied various ways to define positive and negative pairs during training. 
They came up with the split strategy, which considers two transaction subsequences from the same client to be a positive pair and from different clients to be negative. 
There, representations of the subsequences were obtained via a Long Short-Term Memory network~\cite{hochreiter1997long} (LSTM) as an encoder, as transformers don't always perform stronger for sequential data, similarly to~\cite{yue2022ts2vec}. We will use the CoLES~\cite{babaev2022coles} model with an RNN-based encoder as a basic encoder in our work. The limitations of CoLES are strongly connected to the method, which is used to obtain positive and negative pairs of subsequences. This algorithm explicitly encourages the model to ensure that the resulting representations show information about the user as a whole rather than about his local state at a specific time.
In addition, the authors use client classification as the main task in the article to demonstrate the quality of their model.
% TODO: add few articles about Neural Hawkes-based models as another form of SSL for event sequences data (we use these ideas later!) 

Another powerful tool for constructing representation vectors can be Hawkes processes~\cite{hawkes1971spectra, zhang2020self, hawkes1971spectra, hawkes2018hawkes,  hawkes2016de}. Hawkes processes, a class of self-exciting point processes, offer a powerful framework for modelling event sequences where the occurrence of one event increases the probability of subsequent events. This makes them well-suited for analyzing transactional data, where events such as purchases often exhibit temporal dependencies. Hawkes processes incorporate the temporal dynamics and self-exciting nature of transactional events.

\subsection{Accounting for external information}
All the methods mentioned above consider only one sequence at a time. It is sometimes beneficial to consider the overall context formed from the actions of other clients~\cite{ala2022deep}.

External context may consist of the behaviour of other specific clients, or it may reflect the current macroeconomic state. It has been shown that the two strongly correlate~\cite{thomas2000survey,begicheva2021bank}. This implies that the external context may contain useful information for model training.

Since choosing macroeconomic indicators is difficult without the proper education and thus requires bringing in a human expert in the field, it makes more sense to try different ways to aggregate the actions of all bank clients (mean, max, etc.). This approach allows the extraction of more data from the dataset without additional annotation.

Preliminary experiments in~\cite{bazarova2024universal} show that even naive aggregation within an external context provides superior results in some event sequences modelling problems, while their results lacked stability and were limited within two datasets.
For graph representation, learning links between objects is the essential part of a model, being implemented in SOTA models~\cite{kipf2022semi}.
This idea extends to temporal graphs where attention-based approaches~\cite{velivckovic2018graph} via self-supervised learning help to handle complex interactions between objects~\cite{liu2024self}.
Similarly, this line of thought has been developing in temporal point processes, Hawkes mutually-exciting point processes in particular, where one can either derive or use provided node links~\cite{dizaji2022comparative,passino2023mutually}. 
However, in most cases, exact information between connections even from graph data, is absent, and their restoration is of limited quality~\cite{shumovskaia2021linking}.

Other examples of simultaneous accounting for the behaviour of different users and characteristics of a particular moment in time occur in recommendation systems. While classical algorithms consider exclusively the similarity of users or items for recommendations, session-based~\cite{wang2021session} and time-based~\cite{ghiye2023timedecay, JAIN20231834, xia2010timedecay} recommendation systems integrates information from a flow of purchases from different users. Thus, we account for dynamic user preferences and ongoing external context.
The system in this case balances user and group recommendation, as a review~\cite{ceh2022performance} notes. 
Further papers included attention mechanism in this workflow~\cite{guo2020group}. 
Limited scope of this works is constrained by efficiency constraints, that existing works solve by considering recommendations for predefined group of multiple users.
However, a single sequence perspective can be crucial given the diversity of possible life paths.


\subsection{Sequence embedding evaluation}
Evaluation and comparison of proposed approaches are crucial for obtaining evidence on the topic. They allow the researcher to test the applicability of the considered methods.
For event sequences, we require tests that consider both local and global embedding properties.
The local ones reflect the ability of a model to capture an instant state.
For example, in this case, we can test whether the model can predict the next event type~\cite{shchur2021neural}.
Additionally, the global properties consider questions about a sequence as a whole. 
To test these properties, we make a model to predict a sequence label, following~\cite{babaev2022coles}, e.g. if a client will default or leave a bank.
While the papers~\cite{osin2024ebes,xue2023easytpp} use a wider selection of datasets, they consider either specific downstream targets or focus on the prediction of the next event type and time. 
We expand these statements by focusing on the universality of embedded embeddings.
So, our work expands the benchmark from~\cite{bazarova2024universal}, which simultaneously evaluates the quality of a representation for a simultaneous classification of a whole sequence and predicts upcoming event characteristics by using more relevant datasets. 


% On the other hand, CoLES and some of the other mentioned before models have a bunch of drawbacks. Firstly, CoLES, trained on whole sequences, performs poorly on smaller sequence slices. Secondly, it cannot be used to track the changes in the behavior of a single user: all slices from a common sequence will have close representations. Besides that, it could be argued that similar subsequences need to have close representations, even if they come from different users, which is definitely not the case for CoLES. All in all, CoLES representations show good performance when faced with global, inter-user tasks, but local in time tasks require a different training procedure.


% \textbf{Generative models} use different learning approaches to learn the distribution of hidden data.
% The knowledge they gain can then be used to generate plausible data. These approaches often originate from neural language processing (NLP)~\cite{kenton2019bert, radford2019language}.

% Autoencoder is a popular generative model due to its efficiency and simplicity in different domains~\cite{tschannen2018recent}. 
% The paper~\cite{mancisidor2021learning} learns bank client representations useful for downstream tasks using variational autoencoder. 
% Masked language models take the autoencoder idea one step further. 
% Such models recover randomly changed tokens and perform well on various benchmarks~\cite{kenton2019bert, he2022masked}.

% According to the latest research, generative methods outperform contrastive ones at missing value prediction~\cite{jaiswal2020survey}. 


% Autoregressive (AR) modeling is extensively used in CV~\cite{van2016pixel, chen2018pixelsnail, razavi2019generating, esser2021taming}, as well as in sequential domains like NLP~\cite{radford2018improving, brown2020language, black2022gpt} and Audio~\cite{oord2016wavenet, borsos2023audiolm}. 
% Such models are trained to predict the next item in a sequence. Unlike token embeddings in autoencoders, autoregressive models do not use information about future tokens.
% Because of this feature, autoregressive models can capture more complex patterns.  It's confirmed by their superior performance on text generation tasks~\cite{ethayarajh2019contextual}.

\end{Literature_review}

\begin{Methodology}
\section{Methodology}\label{sec:methods}

\begin{figure*}[!t]
     \centering
     \includegraphics[width=0.85\textwidth]{figures/Picture_pipeline_v2.pdf}
     \caption{General pipeline for external context generation to integrate it with a vanilla internal context followed by the considered validation procedures.}
     \label{fig:pipeline}
\end{figure*}

The general experiment pipeline is depicted in Figure~\ref{fig:pipeline} and structured as follows:
\begin{enumerate}
    \item Sequences of the transactions are preprocessed and passed through the pretrained SSL encoder to get embeddings.
    \item Various external context aggregation methods combine embeddings from different sequences and produce an external context vector at the current time moment. After that, the external context vector is concatenated with the embedding of the sequence under consideration. 
    \item The concatenated vector is the current representation of the sequence. It is used in two following downstream tasks. 
\end{enumerate}
The corresponding subsections discuss in detail all the presented steps of the general pipeline. 
We begin with describing transactional data and its preprocessing in general. 
Next, we describe the given models from the baseline approach. 
Then, we present the approaches used to account for external representations. 
Lastly, we discuss the methods of validating obtained representations.



\subsection{Transactional data description and its preprocessing}

In this work, we consider several samples of transactional data, all of which have common and different features. The dataset of transactions $D$ is a set of $n$ sequences with length $T_i$ for every sequence with available features $\vecx$ and timestamps $t$ for every timestamp.
Formally, each sequences $s_i = \{(\vecx_{ij}, t_{ij})\}_{j = 1}^{T_i}$ has a single label for the whole sequence $y_i$.
So, our sample $D = \{(s_i, y_i)\}_{i = 1}^n$.
For a set of event sequences with available labels the notation would be the~same.

To standardize the pipeline, we focus on two main features, accompanied by timestamps, that are presented for all events in all datasets: the transaction's merchant category code (MCC code --- the type of transaction, a categorical variable with $\sim 1000$ possible codes), its volume of money (Amount). 
The time marks of each transaction are also used later for representation construction, not being input to an encoder as proposed in ~\cite{babaev2022coles}. 
More preprocessing details can be found in Appendix~\ref{sec:details}, while Appendix~\ref{sec:app_datasets} provides more details on the used datasets.

\subsection{Models for building representations}

\paragraph{Input encoder.} A special transaction encoder processes a transaction sequence before being put into the main part of a parametric model.
The basic procedure involves obtaining representations of MCC codes of some fixed dimension $d_{\mathrm{mcc}}$, where each transaction type has its vector from a dictionary~\cite{pennington2014glove}. 
Simple preprocessing of numerical characteristics, such as normalization of the amount, is also performed.

Thus, after this preprocessing, we have a sequence of dimension $(T, d_{\mathrm{mcc}} + 1)$, where $1$ is added because of the Amount feature, and $T$ is the length of the transactions sequence.

\paragraph{Sequence encoder.} The resulting sequence of the encoded transactions is fed to the main model of sequential data processing (SeqEncoder), which determines the structure of the considered representations. Any neural network that operates in Sequence-to-Sequence mode can be used as a SeqEncoder. In other words, when the model receives a certain sequence of observations as input, it produces a sequence of representations of the same length of dimension $(T, m)$, where $m$ is the embedding size.
We use recurrent neural networks with a Long-Short-Term Memory (LSTM) or Gated Recurrent Unit (GRU) architecture, depending on the dataset. It was shown in~\cite{babaev2022coles}, RNN-based models outperform transformers for transaction data. 

\paragraph{Self-supervised representation learning for event sequences data.}
% \label{chap:methods}
Our basic representation learning method is the contrastive method CoLES. We also use it as a self-supervised baseline approach.
%and the autoregressive method. 

Contrastive Learning for Event Sequences with Self-Supervision, or shortly CoLES, is a contrastive method for representation learning, proposed in the work~\cite{babaev2022coles}.
CoLES shows high quality on several tasks of event sequence processing, including transactional data. 

In the case of CoLES, two subsets of transactions obtained from one bank client are used as positive pairs, and two subsets of transactions obtained from different clients are used as negative pairs.
A parametric encoder model is used to obtain representations of these subsequences. This encoder is trained using a specific contractive loss function. 
The concept allows the use of different encoders to obtain transaction representations. 



% \textbf{Autoregressive models} (AR)  predict the next item in a sequence. A recurrent neural network similar to the CoLES model is implemented in our work. The model's last hidden state represents the sequence because it contains complete knowledge about the whole sequence.
% The model is trained to predict the next transaction, encompassing the MCC code and transaction amount.

% We add two linear heads to the recurrent neural network output for amounts and MCC prediction to train the AR model. Since transactions contain both categorical (MCC codes) and continuous (amounts) information, the reconstruction loss is also divided into two parts. 

% We use cross-entropy for the categorical features and mean squared error for the continuous features. 
% The final loss function is a weighted sum of the two intermediate ones, with the weights set to five and one for the amount and MCC part, respectively.  

% We found that the preprocessing of transaction amounts defined below is essential for the model to train successfully:
% \begin{equation} \label{eq:ae_amount_transform}
%     f(a) = \mathrm{sign}(a) \ln{(1 + a)}.
% \end{equation}
% This transformation allows stabilization of the loss functions for amounts and MCCs without using dramatically different weights. It is also closer to human perception. People tend to focus on the order of magnitude, ignoring the exact value.

% Optimal complexity for the training objective is very important for self-supervised approaches ~\cite{he2022masked,kenton2019bert}. 
% In our case, we found out that prediction of rare MCC code is too complicated and unnecessary task.
% So, we reduce the number of unique MCC codes to 100 for all datasets, clipping all less frequent MCC codes.

\subsection{Usage of external information based on local representations}
\label{sec:global_context_methods}

An additional external context representation vector (or global embedding) is built from the local representation vectors from all or some selected users by aggregating them. 
The global embedding model works on top of the original encoder.

The procedure for constructing a context vector at a specific point in time is presented in Figure~\ref{fig:global_pooling} and is described as follows:
\begin{itemize}
     \item [1.] Collect a sample of all possible local customer representations for all users and each unique moment of the transaction.
     \item [2.] Select local representations that precede the current time point but are close to it in terms of temporal proximity.
     \item [3.] Apply aggregation to the resulting set of vectors. The resulting vector is the vector of external context.
\end{itemize}

\begin{figure}[t!]
     \centering
     \includegraphics[width=0.9\columnwidth]{figures/scheme_v5.pdf}     
     \caption{External context aggregation that outputs the external representation. Here we don't show encoding of external sequences}
     \label{fig:global_pooling}
\end{figure}

As mentioned earlier, accounting of external context vector  can improve the quality of models in applied problems. To check this, we concatenate the resulting context embedding vector with the user’s local embedding and validate the extended representation.

\subsubsection{Straightforward aggregation}

Averaging and maximization pooling provide a natural way to aggregate external information. 
In the first case, the vector of external representation is obtained by componentwise averaging the local representation vectors for all users in the stored dataset.
In the second case, the maximum value for each component is taken. 
They are close to Mean and Max Pooling operations in convolutional neural networks~\cite{boureau2010theoretical} and language models~\cite{xing2024comparative}. Like global pooling in convolutional neural networks, such aggregation methods are designed to generalize the environment before the following transformations, while taking a new perspective: instead of aggregating events for a specific sequence, we aggregate representations for a set of sequences.

\subsubsection{Methods based on the attention mechanism}

The aggregation methods described in the previous paragraph ignore the interaction between the local representations of all users from the dataset and the considered one and produce similar external context for all sequences.
However, some users in the dataset may behave more like the user than others. 
While we don't have a direct links like for spatial graphs~\cite{huang2024temporal},
we can try to extract these connections from embeddings itself.
Similar clients determine the user’s closest environment and, correspondingly, help describe her or his behaviour better.

Thus, we need an aggregation method that take into account the similarity of users. 
The attention mechanism, originally used to describe the similarity of the word embedding vectors~\cite{vaswani2017attention}, is natural for such a problem.

In this work, different variants of the attention mechanism are used.

\textbf{Simple attention without learnable attention matrix}

In the version without a training matrix, the external context vector for a given point in time has the form:
\begin{equation}
     \vecg_t = X \mathrm{softmax} (X^T \vech_t),
\end{equation}
where $\vech_t \in \mathbb{R}^m$ is a vector of local (internal) representation for the considered user, $\vecg_t \in \mathbb{R}^{m}$ is a vector of external (global) representation for the considered user, and $X \in \mathbb{R}^{m \times n}$ is a matrix, which rows are embeddings of size $m$ for all $n$ users from dataset at a given time point.
In this case, the user similarity metric is normalized by the softmax of the dot product.

\textbf{Attention with learnable attention matrix}

For the method with a trained matrix, the formula is similar:
\begin{equation}
     \vecg_t = X \mathrm{softmax} (X^T A \vech_t),
\end{equation}
here $A \in \mathbb{R}^{m \times m}$ -- is the matrix to be trained.
There, before calculating the scalar product, vectors of representations from the dataset are additionally passed through a trained linear layer. 

\textbf{Attention with symmetric attention matrix}

In this approach, we model the attention matrix as a product of one matrix with  its transposed version $A = S^T S$.
That makes the resulting formula similar to the calculation of kernel the <<dot product>> between the current representation vector and vectors from the train set with the linear kernel:
\begin{equation}
\begin{split}
     % \mathbf{b}_t = X \text{softmax} (X^T S^TS \mathbf{h}_t) = X \text{softmax} ((SX) S\mathbf{h}_t) =  \\ X \text{softmax} (<SX, S\mathbf{h}_t>),
      \mathbf{g}_t = X \mathrm{softmax} (X^T S^T S \mathbf{h}_t) = X \mathrm{softmax} (\langle S X, S \mathbf{h}_t \rangle),
\end{split}
\end{equation}
where $\langle \cdot, \cdot \rangle$ is the notation for the <<dot product>> between all the vectors from $SX$ and $S\mathbf{h}_t$ vector combined in one vector.

\textbf{Kernel attention}

We generalize the approach with a symmetric attention matrix and propose a kernel attention method. The main idea here is that we can use the general kernel dot product to calculate attention scores:
\begin{equation}
     \vecg_t = X \mathrm{softmax} (\langle \phi(X), \phi(\vech_t) \rangle),
\end{equation}
here $\phi(\cdot)$ is a learnable function, that is applied to a row or independently to all rows, if the input is the matrix. In our case, it is parameterized by a two-layer fully-connected neural network.

\subsubsection{Methods inspired by Hawkes process}
~

Previous proposed methods do not take into account the time of the embeddings explicitly. This factor can affect the result as it is natural that the events that happened a long time ago should have less influence on the current moment compared to recent events, as various temporal process models state. To consider this, we turn to the analogues to Multivariate Hawkes processes ~\cite{hawkes1971spectra,hawkes2018hawkes}. 
Formally, Alan Hawkes proposed a generalization of Poisson point process with the intensity that is conditional on past events.

We propose three options based on this intuition, \emph{Exponential Hawkes}, \emph{Learnable exponential Hawkes}, and \emph{Attention Hawkes}. 
They are described in more details in Appendix~\ref{sec:hawkes_insprired_methods}.


\subsection{Production-ready downstream problems} 
\label{sec:validation_methods}

\textbf{Global properties} of representations are validated using the following approach used in the paper~\cite{babaev2022coles}, in which the CoLES model was proposed.

All the datasets considered in this work contain the binary classification mark for each sequence. 
For example, this target mark can represent the clients who left the bank or did not repay the loan. This procedure consists of three steps, which are described below. 
For an initial sequence of transactions of length $T_{i}$ related to the $i$-th user, we obtain the representation $\mathbf{h}^{i}\in\mathbb{R}^{m}$, which characterizes the entire sequence of transactions as a whole. 
Given fixed representations $\mathbf{h}^{i}$, we predict the binary target label $y^{i}\in\{0, 1\}$ using gradient boosting. 
As a specific implementation, the LightGBM~\cite{ke2017lightgbm} model is used, which works fast enough for large data samples and allows obtaining results of sufficiently high quality. 
The specific hyperparameters of the gradient boosting model are fixed (corresponding to ~\cite{babaev2022coles}) and are the same for all base models under study. The quality of the solution of a binary classification problem is measured using a standard set of metrics.

This procedure allows us to evaluate how well the representations capture the client's "global" pattern across its history. 
We call this task \textit{Global target} prediction or \textit{Global validation} from here and below.

\textbf{Local properties} of sequences differ from global ones in that they change over time, even for one sequence. 
We use a sliding window procedure of size $w$ to obtain and evaluate local embeddings. To do this, for the $i$-th user at time $t_{j}\in[t_{w}, T_{i}]$ the subsequence of his transactions $\mathbf{S}_{j-w:j}^{i }$ is taken.
Next, this interval is passed through the encoder model under consideration to obtain a local representation $\mathbf{h}_{j}^{i}\in\mathbb{R}^{m}$. 
% An illustration of this approach can be found in Figure~\ref{fig:local_validation_coles}.

% \begin{figure}[!ht]
%      \centering
%      \includegraphics[width=\columnwidth]{figures/local_val.pdf}
%      \caption{Usage of the sliding window for local validation approaches}
%      \label{fig:local_validation_coles}
% \end{figure}

The longer the time interval the model uses, the greater the risk that the data it relies on will become outdated. Our artificial limitation allows us to reduce this effect for all models and also strengthen their local properties only due to the “relevance” of the data.

An obvious limitation of this approach is that it does not allow obtaining local representations at times $t\leq t_{w}^{i}$.
In addition, the window size $w$ is an additional hyperparameter that must be chosen, taking into account the fact that for small values of $w$, the resulting representations do not contain enough information to solve local problems and for large values of $w$ the representations lose local properties and become global in the limit $w\to T_{i}$.

As local validation problems, we explored the prediction of the next transaction's MCC code. We call this task \textit{Event type} prediction from here and below. This validation approach was inspired by the work~\cite{zhuzhel2023continuous}, in which it was proposed to predict the type of the next event based on the history of observations --- in our case, the MCC code of the next transaction. Formally, in this case, the multiclass classification problem is solved: given the local representation $\mathbf{h}_{j}^{i}$, we predict the MCC code of the transaction of the $i$-th client, completed at time $t_{j+1}$.

Note that there are many rare MCC types in datasets. From a business perspective, such categories are often less interesting and meaningful. Therefore, to simplify the task, in this procedure for testing local properties, it was decided to leave only transactions corresponding to the $100$ most popular codes.

\subsection{Efficient work in production} 
\label{sec:prod_efficient}

% In our work, we propose a solution for an applied problem in the Financial domain, like credit default prediction. Our approach increases the quality of a downstream task by enriching user representation vectors with an external context. We provide a description of the system for our solution and quantification of system performance in a special section of the paper. 

To integrate our method of external aggregation in the production system to conduct inference, 
we maintain a database of embedding vectors that allows on-the-fly aggregation.
Let us consider all the steps of the proposed procedure below.

The vector database consists of embedding vectors with actual users $D \in \mathbb{R}^{n_0 \times m}$: $n$ user vectors of size $m$, that also appear as a pre-computing embeddings database in the recommendation system domain~\cite{zanardi2011dynamic}. 
To reduce the computational time, we randomly select $n_0 \ll n$ that is significantly smaller than the whole set of users.
In our experiments, $n_0 = 100$ users for a dataset of $n = 5000$ users is enough for aggregation to perform well. 

An update of the database happens daily, to account for new events, such as transactions, as typical user performs no more than three transactions per day. 
An RNN-based model there can easily be updated without rerunning the model on the whole event sequence by using the current embedding as an input. 

The external context vector can be independent and same for all users (mean aggregation) or dependent from user embedding and different for all users (attention aggregation) of the specific user embeddings.  
In the first case, we easily aggregate vectors of selected users in a single vector $\mathbf{g}$ and use it throughout the day.
In the second case, we compute the necessary aggregation on request with a single attention layer.
In both cases, we have a linear in the $n_0$ procedure.

The concluding prediction takes a close time to the usual prediction, as we concatenate the external context embedding and a user-local embedding and process it with a linear or a gradient boosting layer.  

So, our method is easy to develop and fast to employ in real-life environments with little computational overhead.



%scenario is also not computationally expensive. If you need to get predictions for users from the external vector database, you don't need to rerun the model in offline mode. You can just take these users' embeddings from the vector database and concatenate them with external context embedding. It can even speed up computations for  offline model usage scenario with CPU computation for neural networks. If the user is not in the vector database, you need to run the encoder and just compute an actual external context vector or get it from the database with unified computational complexity and get a prediction with an NN head or another model without external aggregation.  


% The size of this database will include  plus one external context vector of the same size.


\end{Methodology}

\begin{Experiments}
% \section{Results and Discussion}

% We start this chapter with methods for validating the resulting embeddings in terms of global and local properties and discussion of used datasets and metrics.
% Next, the chapter provides obtained results of different methods used to take into account external representations. The discussion of results and the proposition for further development of the project are also presented in this part.

% All experiments were carried out in the Python programming language using the models and hyperparameters described in the methods chapter. 

\section{Results}\label{sec:results}

In this section, we describe the results of experiments on obtaining an external context representation and using it to improve existing~models.

\subsection{Methods}

In the experiments, we consider three types of methods: the one without external context inclusion and unlearnable and learnable approaches to aggregate information from other event sequences.
A conventional CoLES encoder~\cite{babaev2022coles} provides strong performance in the considered problems, so we use it in the case without adding external context information (\emph{Without context}).
For aggregation approaches, all learnable parts of these methods come from this CoLES pipeline with fixed original encoders. 
We investigated the following types of aggregation of representations to obtain a context~vector:
\begin{itemize}
    \item straightforward aggregation methods: Averaging (\emph{Mean}), Maximization (\emph{Max}) poolings;
    \item methods inspired by Hawkes process: Exponential Hawkes (\emph{Exp Hawkes}), Exponential learnable Hawkes (\emph{Learnable exp Hawkes*});
    \item attention-based aggregation methods: Attention mechanism without learning (\emph{Attention}), with learning (\emph{Learnable attention*}), Attention with symmetric matrix (\emph{Symmetrical attention*}), Kernel attention (\emph{Kernel attention*});
    \item combined methods: Hawkes with attention (\emph{Attention Hawkes}).
\end{itemize}
All learnable methods are marked with the asterisk sign~"*".



\begin{figure*}[!th]
      \centering
          \includegraphics[width=0.9\textwidth]{figures/star_plot.pdf}
          \caption{\selectfont Quality of the models regarding their global and local properties on the \textit{Churn} dataset (left), \textit{Default} dataset (central), and \textit{HSBC} dataset (right). The $x$-axis corresponds to the global validation ROC-AUC, while the $y$-axis shows the next event type prediction ROC-AUC. Thus, the upper and righter the dot is, the better the model is. A dot in the plot is mean, and lines are std for experiments with 3 seeds. }\label{fig:coles}
\end{figure*}


\begin{figure}[!th]
     \centering
     \includegraphics[width=\columnwidth]{figures/multi_user.pdf}
     \caption{Dependencies between the number of event sequences in external context and forthe  global target (left) and event type (right) ROC-AUC for \textit{Churn} dataset. }
     \label{fig:iters}
\end{figure}

\subsection{Validation}

\paragraph{Procedure.} We consider two types of downstream problems: local and global, defined for a current timestamp for a client or for an event sequences as a whole correspondingly. More detailed problem definitions are in Subsection~\ref{sec:validation_methods}. For global validation, we follow the procedure described in~\cite{babaev2022coles}.
There, self-supervised learning is followed by adding a gradient boosting on top of learned embedding features. 
We use the same hyperparameters for the boosting model.
A more common for computer vision~\cite{grill2020bootstrap} linear probing instead of boosting probing performs worse, so the experiments use the latter one. 
For local validation, we use the procedure described in the corresponding section~\ref{sec:validation_methods}.
All results were averaged across $3$ separate training runs for pre-trained encoder models. 

\paragraph{Datasets.} % \label{sec:data}
To compare the models and methods, we work with three open samples of transactional data: \href{https://boosters.pro/championship/rosbank1}{Churn}, \href{https://boosters.pro/championship/alfabattle2}{Default} and \href{https://www.kaggle.com/datasets/ashisparida/hsbc-ml-hackathon-2023}{HSBC}.
Details on the datasets are in Appendix~\ref{sec:app_datasets}





\subsection{Main results}

% All models were tested in the way described in section \ref{sec:methods} on two applied tasks. 
The results for the Churn, Default, and HSBC datasets are presented in Figure \ref{fig:coles}.
Also, Figure \ref{fig:mean_rank} summarizes these results by presenting ranks of all methods averaged by the mean ROC-AUC metric for all datasets.
Results for two additional non-financial datasets are provided in Appendix~\ref{sec:other-data-results}. We also provide more detailed experiment metrics and rank data results in Table~\ref{tab:metrics} in Appendix~\ref{sec:AppMainTable} and and Table~\ref{tab:metrics} in Appendix~\ref{sec:AppMainTable}.

External context improves metrics in most cases. This is especially noticeable in the balanced Churn sample. In the unbalanced Default and HSBC samples, contextual representations also help models solve local and global problems, but this is not as explicit as in the case of a balanced dataset. Figures also allows us to note, that aggregation methods provides less variation among different seeds, thus, the produced models are more robust. 

According to Figure~\ref{fig:mean_rank} and Table~\ref{tab:ranks}, the best method is the Kernel attention, which performs best or is near the best for all tasks.
However, all attention-based methods with learnable parts (Learnable attention, Symmetrical attention, Kernel attention) often end up among the leaders, especially in the event-type prediction task. 
It is natural, as attention-based probing often leads to stronger models that account for more intricate connections between users.

On the other hand, approaches inspired by the Hawkes process (Exp Hawkes, Exp learnable Hawkes, Attention Hawkes) show inferior results even though these methods take into account the temporal distance of sequence embeddings and, because of this, should respond better to local temporal changes in the data. 
So, accounting for diverse time differences in the context of conditional intensity modelling, given the history of events, remains open for the task of external context addition and requires further research.

Among the methods without learnable parts (Mean, Max, Attention, Exp Hawkes, Attention Hawkes), the Mean and Max methods perform best in local and global validation tasks --- in this case, we can capture average external context, which also should be helpful to specific prediction tasks. 
%This says a lot about society :)
So, default methods of aggregation can be reasonable in the absence of computational resources. 
However, the better choice is to build an external context accounting method with learnable parts, like \emph{Kernel attention}. 

% \usepackage{booktabs}


\begin{table}
	\centering
	\caption{Analysis of models' inference speed.}
	\begin{tabular}{c|ccc} 
		\toprule
		& Parms (MB) & Speed & Moderate  \\ 
		\hline
		IA-SSD     & 2.7        & 84    & 79.12     \\
		PDM-SSD(A) & 3.3        & 84    & 79.37     \\
		PDM-SSD(J) & 3.3        & 68    & 79.75     \\
		\bottomrule
	\end{tabular}
\label{tabel8}
\end{table}

\begin{figure*}[!t]
\centering
\begin{subfigure}{.46\textwidth}
  \centering
  \includegraphics[width=.97\linewidth]{figures/churn_ft.pdf}
  %\captionsetup{justification=raggedright,singlelinecheck=false}
  \caption{Churn dataset}
  \label{fig:scatter_fgsm}
\end{subfigure}
\begin{subfigure}{.46\textwidth}
  \centering
  \includegraphics[width=.97\linewidth]{figures/default_ft.pdf}
  %\captionsetup{justification=raggedright,singlelinecheck=false}
  \caption{Default dataset}
  \label{fig:scatter_kll2}
\end{subfigure}
\caption{ROC-AUC for the task next event type prediction with and without fine-tuning encoder.}
\label{fig:fine-tune}
\end{figure*}

\subsection{Dependence on the number of external sequences}

To show how the number of users influences the construction of external context aggregation, we measured how the quality of applied tasks changes depending on the change.
Considered context sizes included $10$, $50$, $100$, $500$ and $1000$ external sequences.
In this experiment, we select the users whose last event was the most recent for the considered time point.
There, the results are for the best approaches from each cohort: from the pooling methods, it was the Max; from the attention-based methods --- our best Kernel attention, and from methods inspired by the Hawkes method --- Exp Hawkes.  

The results for the Churn dataset are presented in Figure ~\ref{fig:iters}.
The increase in the number of users positively affects the quality of global target and event-type prediction tasks. This statement is not valid for the Exp Hawkes method in the event type prediction task, but as stated above, the Hawkes method group did not work for us. 
On the other hand, $500$ sequences in a context should be enough for the performance sufficiently close to the top one.

\subsection{Fine-tuning backbone for local validation}

To improve the quality of our aggregation methods, we unfreeze the backbone in the process of model head event type prediction training. 
This procedure runs only for the local validation because using a differentiable linear head instead of gradient boosting significantly decreases the quality of the global validation task. 
We compare training the neural network in two modes: training both the encoder and head (unfreeze mode) and training only the head with a freezed encoder, as in all experiments before. 

The results are presented in Figure \ref{fig:fine-tune}. 
Unfreezing of the encoder increases the quality of pooling aggregation, especially for the Churn dataset. 
Besides, the ROC-AUC of all pooling grew substantially compared the encoder without external context.  



\subsection{Computational overhead for the external aggregation}

As discussed in Section~\ref{sec:prod_efficient}, during Inference, the inclusion of external context leads to minor computational overhead if an overall procedure is implemented properly.
However, during training, additional expenses can outweigh potential benefits.

We measured the time required to complete four different training stages.
\begin{itemize}
    \item \textit{Pretrain CoLES SSL} is training encoder with self-supervised loss. In this stage, we get a pre-trained encoder for getting only local representation vectors.
    \item \textit{Pretrain Pool SSL} is training pooling learnable parameters only for learnable pooling methods with self-supervised loss. In this stage, we get a pre-trained encoder with pooling layers for getting both local and external representation vectors.
    \item \textit{Fine-tune, freeze} and \textit{Fine-tune, unfreeze} is training head over freezed or unfreezed backbone
 with pooling on the next event prediction task, respectively. In this stage, we train the head of the encoder for the local validation task.
\end{itemize}
The first stage \textit{Pretrain CoLES SSL} is exactly the training of a basic model, all others are additional costs that can be required to use external aggregation approaches.

The training time in minutes is in Table~\ref{tab:time}.
We see, that all training time are comparable, while additional training with unfreezing requires significant additional resource, while simultaneously boosting the model quality.




\end{Experiments}

\begin{Conclusion}
\section{Conclusion}
We introduced \Bench, the first ever IMTS forecasting benchmark.
\Bench's datasets are created with ODE models, that were defined in decades of research and published on
the Physiome Model Repository. Our experiments showed that LinODEnet and CRU are actually
better than previous evaluation on established datasets indicated. Nevertheless,
we also provided a few datasets, on which models are unable to outperform a
constant baseline model. We believe that our datasets, especially the very difficult ones,
can help to identify deficits of current architectures and support future research on
IMTS forecasting.

\end{Conclusion}

\begin{Acknowledgements}
\input{chapters/11_acknowledgements}
\end{Acknowledgements}

% \section{Related work}
\label{lit_rewiev}

We review the approaches to external context aggregation in various data modalities and methods for evaluating sequence representations. 
The review also considers existing representation learning methods for event sequences, including self-supervised methods for general and bank transaction domains, to provide further background.

\subsection{Event sequence representation learning}
Neural networks have been shown to perform well when faced with the task of event sequence representation learning~\cite{babaev2022coles}.
Obtained representations are helpful both for tasks describing local in-time properties, such as next event prediction~\cite{zhuzhel2023continuous} or change point detection~\cite{deldari2021time,ala2022deep} and whole sequence classification~\cite{bin2022review,jaiswal2020survey}.

\paragraph{Self-supervised learning} (SSL) is a solid representation learning paradigm that learns an encoder --- neural network, providing representation vector, without labelled data. 
This allows researchers to skip costly and possibly limited human expert annotation and leverage the large bodies of low-cost unlabeled data, leading to more universal representations~\cite{caron2021emerging}.
This paradigm is often implemented within contrastive and generative learning frameworks~\cite{zhang2020self,liu2023ssl}. 

In contrastive learning, the encoder learns to produce embeddings that are close if objects are labelled as similar and distant, and vice versa. 
%allow distinguishing between objects that are labelled as close or distant in a self-supervised manner.
It originated in computer vision in the form of Siamese loss~\cite{hoffer2015deep} and SimCLR~\cite{chen2020simple} with subsequent DINO \cite{caron2021emerging} and BarlowTwins~\cite{zbontar2021barlow} among others.
It also allows the production of meaningful representations for time series and event sequence data~\cite{zhang2024self}.
% These approaches are also gaining popularity in the field of multidimensional event sequences and time series~\cite{babaev2022coles, li_deep_2020, romanenkova2022similarity, moskvoretskii2024self, li2023new}. 
Our work also uses a contrastive representation model as it performs well~\cite{babaev2022coles} for event sequences data, producing universal representations. 
The authors of the method studied various ways to define positive and negative pairs during training. 
They came up with the split strategy, which considers two transaction subsequences from the same client to be a positive pair and from different clients to be negative. 
There, representations of the subsequences were obtained via a Long Short-Term Memory network~\cite{hochreiter1997long} (LSTM) as an encoder, as transformers don't always perform stronger for sequential data, similarly to~\cite{yue2022ts2vec}. We will use the CoLES~\cite{babaev2022coles} model with an RNN-based encoder as a basic encoder in our work. The limitations of CoLES are strongly connected to the method, which is used to obtain positive and negative pairs of subsequences. This algorithm explicitly encourages the model to ensure that the resulting representations show information about the user as a whole rather than about his local state at a specific time.
In addition, the authors use client classification as the main task in the article to demonstrate the quality of their model.
% TODO: add few articles about Neural Hawkes-based models as another form of SSL for event sequences data (we use these ideas later!) 

Another powerful tool for constructing representation vectors can be Hawkes processes~\cite{hawkes1971spectra, zhang2020self, hawkes1971spectra, hawkes2018hawkes,  hawkes2016de}. Hawkes processes, a class of self-exciting point processes, offer a powerful framework for modelling event sequences where the occurrence of one event increases the probability of subsequent events. This makes them well-suited for analyzing transactional data, where events such as purchases often exhibit temporal dependencies. Hawkes processes incorporate the temporal dynamics and self-exciting nature of transactional events.

\subsection{Accounting for external information}
All the methods mentioned above consider only one sequence at a time. It is sometimes beneficial to consider the overall context formed from the actions of other clients~\cite{ala2022deep}.

External context may consist of the behaviour of other specific clients, or it may reflect the current macroeconomic state. It has been shown that the two strongly correlate~\cite{thomas2000survey,begicheva2021bank}. This implies that the external context may contain useful information for model training.

Since choosing macroeconomic indicators is difficult without the proper education and thus requires bringing in a human expert in the field, it makes more sense to try different ways to aggregate the actions of all bank clients (mean, max, etc.). This approach allows the extraction of more data from the dataset without additional annotation.

Preliminary experiments in~\cite{bazarova2024universal} show that even naive aggregation within an external context provides superior results in some event sequences modelling problems, while their results lacked stability and were limited within two datasets.
For graph representation, learning links between objects is the essential part of a model, being implemented in SOTA models~\cite{kipf2022semi}.
This idea extends to temporal graphs where attention-based approaches~\cite{velivckovic2018graph} via self-supervised learning help to handle complex interactions between objects~\cite{liu2024self}.
Similarly, this line of thought has been developing in temporal point processes, Hawkes mutually-exciting point processes in particular, where one can either derive or use provided node links~\cite{dizaji2022comparative,passino2023mutually}. 
However, in most cases, exact information between connections even from graph data, is absent, and their restoration is of limited quality~\cite{shumovskaia2021linking}.

Other examples of simultaneous accounting for the behaviour of different users and characteristics of a particular moment in time occur in recommendation systems. While classical algorithms consider exclusively the similarity of users or items for recommendations, session-based~\cite{wang2021session} and time-based~\cite{ghiye2023timedecay, JAIN20231834, xia2010timedecay} recommendation systems integrates information from a flow of purchases from different users. Thus, we account for dynamic user preferences and ongoing external context.
The system in this case balances user and group recommendation, as a review~\cite{ceh2022performance} notes. 
Further papers included attention mechanism in this workflow~\cite{guo2020group}. 
Limited scope of this works is constrained by efficiency constraints, that existing works solve by considering recommendations for predefined group of multiple users.
However, a single sequence perspective can be crucial given the diversity of possible life paths.


\subsection{Sequence embedding evaluation}
Evaluation and comparison of proposed approaches are crucial for obtaining evidence on the topic. They allow the researcher to test the applicability of the considered methods.
For event sequences, we require tests that consider both local and global embedding properties.
The local ones reflect the ability of a model to capture an instant state.
For example, in this case, we can test whether the model can predict the next event type~\cite{shchur2021neural}.
Additionally, the global properties consider questions about a sequence as a whole. 
To test these properties, we make a model to predict a sequence label, following~\cite{babaev2022coles}, e.g. if a client will default or leave a bank.
While the papers~\cite{osin2024ebes,xue2023easytpp} use a wider selection of datasets, they consider either specific downstream targets or focus on the prediction of the next event type and time. 
We expand these statements by focusing on the universality of embedded embeddings.
So, our work expands the benchmark from~\cite{bazarova2024universal}, which simultaneously evaluates the quality of a representation for a simultaneous classification of a whole sequence and predicts upcoming event characteristics by using more relevant datasets. 


% On the other hand, CoLES and some of the other mentioned before models have a bunch of drawbacks. Firstly, CoLES, trained on whole sequences, performs poorly on smaller sequence slices. Secondly, it cannot be used to track the changes in the behavior of a single user: all slices from a common sequence will have close representations. Besides that, it could be argued that similar subsequences need to have close representations, even if they come from different users, which is definitely not the case for CoLES. All in all, CoLES representations show good performance when faced with global, inter-user tasks, but local in time tasks require a different training procedure.


% \textbf{Generative models} use different learning approaches to learn the distribution of hidden data.
% The knowledge they gain can then be used to generate plausible data. These approaches often originate from neural language processing (NLP)~\cite{kenton2019bert, radford2019language}.

% Autoencoder is a popular generative model due to its efficiency and simplicity in different domains~\cite{tschannen2018recent}. 
% The paper~\cite{mancisidor2021learning} learns bank client representations useful for downstream tasks using variational autoencoder. 
% Masked language models take the autoencoder idea one step further. 
% Such models recover randomly changed tokens and perform well on various benchmarks~\cite{kenton2019bert, he2022masked}.

% According to the latest research, generative methods outperform contrastive ones at missing value prediction~\cite{jaiswal2020survey}. 


% Autoregressive (AR) modeling is extensively used in CV~\cite{van2016pixel, chen2018pixelsnail, razavi2019generating, esser2021taming}, as well as in sequential domains like NLP~\cite{radford2018improving, brown2020language, black2022gpt} and Audio~\cite{oord2016wavenet, borsos2023audiolm}. 
% Such models are trained to predict the next item in a sequence. Unlike token embeddings in autoencoders, autoregressive models do not use information about future tokens.
% Because of this feature, autoregressive models can capture more complex patterns.  It's confirmed by their superior performance on text generation tasks~\cite{ethayarajh2019contextual}.

% \section{Methodology}\label{sec:methods}

\begin{figure*}[!t]
     \centering
     \includegraphics[width=0.85\textwidth]{figures/Picture_pipeline_v2.pdf}
     \caption{General pipeline for external context generation to integrate it with a vanilla internal context followed by the considered validation procedures.}
     \label{fig:pipeline}
\end{figure*}

The general experiment pipeline is depicted in Figure~\ref{fig:pipeline} and structured as follows:
\begin{enumerate}
    \item Sequences of the transactions are preprocessed and passed through the pretrained SSL encoder to get embeddings.
    \item Various external context aggregation methods combine embeddings from different sequences and produce an external context vector at the current time moment. After that, the external context vector is concatenated with the embedding of the sequence under consideration. 
    \item The concatenated vector is the current representation of the sequence. It is used in two following downstream tasks. 
\end{enumerate}
The corresponding subsections discuss in detail all the presented steps of the general pipeline. 
We begin with describing transactional data and its preprocessing in general. 
Next, we describe the given models from the baseline approach. 
Then, we present the approaches used to account for external representations. 
Lastly, we discuss the methods of validating obtained representations.



\subsection{Transactional data description and its preprocessing}

In this work, we consider several samples of transactional data, all of which have common and different features. The dataset of transactions $D$ is a set of $n$ sequences with length $T_i$ for every sequence with available features $\vecx$ and timestamps $t$ for every timestamp.
Formally, each sequences $s_i = \{(\vecx_{ij}, t_{ij})\}_{j = 1}^{T_i}$ has a single label for the whole sequence $y_i$.
So, our sample $D = \{(s_i, y_i)\}_{i = 1}^n$.
For a set of event sequences with available labels the notation would be the~same.

To standardize the pipeline, we focus on two main features, accompanied by timestamps, that are presented for all events in all datasets: the transaction's merchant category code (MCC code --- the type of transaction, a categorical variable with $\sim 1000$ possible codes), its volume of money (Amount). 
The time marks of each transaction are also used later for representation construction, not being input to an encoder as proposed in ~\cite{babaev2022coles}. 
More preprocessing details can be found in Appendix~\ref{sec:details}, while Appendix~\ref{sec:app_datasets} provides more details on the used datasets.

\subsection{Models for building representations}

\paragraph{Input encoder.} A special transaction encoder processes a transaction sequence before being put into the main part of a parametric model.
The basic procedure involves obtaining representations of MCC codes of some fixed dimension $d_{\mathrm{mcc}}$, where each transaction type has its vector from a dictionary~\cite{pennington2014glove}. 
Simple preprocessing of numerical characteristics, such as normalization of the amount, is also performed.

Thus, after this preprocessing, we have a sequence of dimension $(T, d_{\mathrm{mcc}} + 1)$, where $1$ is added because of the Amount feature, and $T$ is the length of the transactions sequence.

\paragraph{Sequence encoder.} The resulting sequence of the encoded transactions is fed to the main model of sequential data processing (SeqEncoder), which determines the structure of the considered representations. Any neural network that operates in Sequence-to-Sequence mode can be used as a SeqEncoder. In other words, when the model receives a certain sequence of observations as input, it produces a sequence of representations of the same length of dimension $(T, m)$, where $m$ is the embedding size.
We use recurrent neural networks with a Long-Short-Term Memory (LSTM) or Gated Recurrent Unit (GRU) architecture, depending on the dataset. It was shown in~\cite{babaev2022coles}, RNN-based models outperform transformers for transaction data. 

\paragraph{Self-supervised representation learning for event sequences data.}
% \label{chap:methods}
Our basic representation learning method is the contrastive method CoLES. We also use it as a self-supervised baseline approach.
%and the autoregressive method. 

Contrastive Learning for Event Sequences with Self-Supervision, or shortly CoLES, is a contrastive method for representation learning, proposed in the work~\cite{babaev2022coles}.
CoLES shows high quality on several tasks of event sequence processing, including transactional data. 

In the case of CoLES, two subsets of transactions obtained from one bank client are used as positive pairs, and two subsets of transactions obtained from different clients are used as negative pairs.
A parametric encoder model is used to obtain representations of these subsequences. This encoder is trained using a specific contractive loss function. 
The concept allows the use of different encoders to obtain transaction representations. 



% \textbf{Autoregressive models} (AR)  predict the next item in a sequence. A recurrent neural network similar to the CoLES model is implemented in our work. The model's last hidden state represents the sequence because it contains complete knowledge about the whole sequence.
% The model is trained to predict the next transaction, encompassing the MCC code and transaction amount.

% We add two linear heads to the recurrent neural network output for amounts and MCC prediction to train the AR model. Since transactions contain both categorical (MCC codes) and continuous (amounts) information, the reconstruction loss is also divided into two parts. 

% We use cross-entropy for the categorical features and mean squared error for the continuous features. 
% The final loss function is a weighted sum of the two intermediate ones, with the weights set to five and one for the amount and MCC part, respectively.  

% We found that the preprocessing of transaction amounts defined below is essential for the model to train successfully:
% \begin{equation} \label{eq:ae_amount_transform}
%     f(a) = \mathrm{sign}(a) \ln{(1 + a)}.
% \end{equation}
% This transformation allows stabilization of the loss functions for amounts and MCCs without using dramatically different weights. It is also closer to human perception. People tend to focus on the order of magnitude, ignoring the exact value.

% Optimal complexity for the training objective is very important for self-supervised approaches ~\cite{he2022masked,kenton2019bert}. 
% In our case, we found out that prediction of rare MCC code is too complicated and unnecessary task.
% So, we reduce the number of unique MCC codes to 100 for all datasets, clipping all less frequent MCC codes.

\subsection{Usage of external information based on local representations}
\label{sec:global_context_methods}

An additional external context representation vector (or global embedding) is built from the local representation vectors from all or some selected users by aggregating them. 
The global embedding model works on top of the original encoder.

The procedure for constructing a context vector at a specific point in time is presented in Figure~\ref{fig:global_pooling} and is described as follows:
\begin{itemize}
     \item [1.] Collect a sample of all possible local customer representations for all users and each unique moment of the transaction.
     \item [2.] Select local representations that precede the current time point but are close to it in terms of temporal proximity.
     \item [3.] Apply aggregation to the resulting set of vectors. The resulting vector is the vector of external context.
\end{itemize}

\begin{figure}[t!]
     \centering
     \includegraphics[width=0.9\columnwidth]{figures/scheme_v5.pdf}     
     \caption{External context aggregation that outputs the external representation. Here we don't show encoding of external sequences}
     \label{fig:global_pooling}
\end{figure}

As mentioned earlier, accounting of external context vector  can improve the quality of models in applied problems. To check this, we concatenate the resulting context embedding vector with the user’s local embedding and validate the extended representation.

\subsubsection{Straightforward aggregation}

Averaging and maximization pooling provide a natural way to aggregate external information. 
In the first case, the vector of external representation is obtained by componentwise averaging the local representation vectors for all users in the stored dataset.
In the second case, the maximum value for each component is taken. 
They are close to Mean and Max Pooling operations in convolutional neural networks~\cite{boureau2010theoretical} and language models~\cite{xing2024comparative}. Like global pooling in convolutional neural networks, such aggregation methods are designed to generalize the environment before the following transformations, while taking a new perspective: instead of aggregating events for a specific sequence, we aggregate representations for a set of sequences.

\subsubsection{Methods based on the attention mechanism}

The aggregation methods described in the previous paragraph ignore the interaction between the local representations of all users from the dataset and the considered one and produce similar external context for all sequences.
However, some users in the dataset may behave more like the user than others. 
While we don't have a direct links like for spatial graphs~\cite{huang2024temporal},
we can try to extract these connections from embeddings itself.
Similar clients determine the user’s closest environment and, correspondingly, help describe her or his behaviour better.

Thus, we need an aggregation method that take into account the similarity of users. 
The attention mechanism, originally used to describe the similarity of the word embedding vectors~\cite{vaswani2017attention}, is natural for such a problem.

In this work, different variants of the attention mechanism are used.

\textbf{Simple attention without learnable attention matrix}

In the version without a training matrix, the external context vector for a given point in time has the form:
\begin{equation}
     \vecg_t = X \mathrm{softmax} (X^T \vech_t),
\end{equation}
where $\vech_t \in \mathbb{R}^m$ is a vector of local (internal) representation for the considered user, $\vecg_t \in \mathbb{R}^{m}$ is a vector of external (global) representation for the considered user, and $X \in \mathbb{R}^{m \times n}$ is a matrix, which rows are embeddings of size $m$ for all $n$ users from dataset at a given time point.
In this case, the user similarity metric is normalized by the softmax of the dot product.

\textbf{Attention with learnable attention matrix}

For the method with a trained matrix, the formula is similar:
\begin{equation}
     \vecg_t = X \mathrm{softmax} (X^T A \vech_t),
\end{equation}
here $A \in \mathbb{R}^{m \times m}$ -- is the matrix to be trained.
There, before calculating the scalar product, vectors of representations from the dataset are additionally passed through a trained linear layer. 

\textbf{Attention with symmetric attention matrix}

In this approach, we model the attention matrix as a product of one matrix with  its transposed version $A = S^T S$.
That makes the resulting formula similar to the calculation of kernel the <<dot product>> between the current representation vector and vectors from the train set with the linear kernel:
\begin{equation}
\begin{split}
     % \mathbf{b}_t = X \text{softmax} (X^T S^TS \mathbf{h}_t) = X \text{softmax} ((SX) S\mathbf{h}_t) =  \\ X \text{softmax} (<SX, S\mathbf{h}_t>),
      \mathbf{g}_t = X \mathrm{softmax} (X^T S^T S \mathbf{h}_t) = X \mathrm{softmax} (\langle S X, S \mathbf{h}_t \rangle),
\end{split}
\end{equation}
where $\langle \cdot, \cdot \rangle$ is the notation for the <<dot product>> between all the vectors from $SX$ and $S\mathbf{h}_t$ vector combined in one vector.

\textbf{Kernel attention}

We generalize the approach with a symmetric attention matrix and propose a kernel attention method. The main idea here is that we can use the general kernel dot product to calculate attention scores:
\begin{equation}
     \vecg_t = X \mathrm{softmax} (\langle \phi(X), \phi(\vech_t) \rangle),
\end{equation}
here $\phi(\cdot)$ is a learnable function, that is applied to a row or independently to all rows, if the input is the matrix. In our case, it is parameterized by a two-layer fully-connected neural network.

\subsubsection{Methods inspired by Hawkes process}
~

Previous proposed methods do not take into account the time of the embeddings explicitly. This factor can affect the result as it is natural that the events that happened a long time ago should have less influence on the current moment compared to recent events, as various temporal process models state. To consider this, we turn to the analogues to Multivariate Hawkes processes ~\cite{hawkes1971spectra,hawkes2018hawkes}. 
Formally, Alan Hawkes proposed a generalization of Poisson point process with the intensity that is conditional on past events.

We propose three options based on this intuition, \emph{Exponential Hawkes}, \emph{Learnable exponential Hawkes}, and \emph{Attention Hawkes}. 
They are described in more details in Appendix~\ref{sec:hawkes_insprired_methods}.


\subsection{Production-ready downstream problems} 
\label{sec:validation_methods}

\textbf{Global properties} of representations are validated using the following approach used in the paper~\cite{babaev2022coles}, in which the CoLES model was proposed.

All the datasets considered in this work contain the binary classification mark for each sequence. 
For example, this target mark can represent the clients who left the bank or did not repay the loan. This procedure consists of three steps, which are described below. 
For an initial sequence of transactions of length $T_{i}$ related to the $i$-th user, we obtain the representation $\mathbf{h}^{i}\in\mathbb{R}^{m}$, which characterizes the entire sequence of transactions as a whole. 
Given fixed representations $\mathbf{h}^{i}$, we predict the binary target label $y^{i}\in\{0, 1\}$ using gradient boosting. 
As a specific implementation, the LightGBM~\cite{ke2017lightgbm} model is used, which works fast enough for large data samples and allows obtaining results of sufficiently high quality. 
The specific hyperparameters of the gradient boosting model are fixed (corresponding to ~\cite{babaev2022coles}) and are the same for all base models under study. The quality of the solution of a binary classification problem is measured using a standard set of metrics.

This procedure allows us to evaluate how well the representations capture the client's "global" pattern across its history. 
We call this task \textit{Global target} prediction or \textit{Global validation} from here and below.

\textbf{Local properties} of sequences differ from global ones in that they change over time, even for one sequence. 
We use a sliding window procedure of size $w$ to obtain and evaluate local embeddings. To do this, for the $i$-th user at time $t_{j}\in[t_{w}, T_{i}]$ the subsequence of his transactions $\mathbf{S}_{j-w:j}^{i }$ is taken.
Next, this interval is passed through the encoder model under consideration to obtain a local representation $\mathbf{h}_{j}^{i}\in\mathbb{R}^{m}$. 
% An illustration of this approach can be found in Figure~\ref{fig:local_validation_coles}.

% \begin{figure}[!ht]
%      \centering
%      \includegraphics[width=\columnwidth]{figures/local_val.pdf}
%      \caption{Usage of the sliding window for local validation approaches}
%      \label{fig:local_validation_coles}
% \end{figure}

The longer the time interval the model uses, the greater the risk that the data it relies on will become outdated. Our artificial limitation allows us to reduce this effect for all models and also strengthen their local properties only due to the “relevance” of the data.

An obvious limitation of this approach is that it does not allow obtaining local representations at times $t\leq t_{w}^{i}$.
In addition, the window size $w$ is an additional hyperparameter that must be chosen, taking into account the fact that for small values of $w$, the resulting representations do not contain enough information to solve local problems and for large values of $w$ the representations lose local properties and become global in the limit $w\to T_{i}$.

As local validation problems, we explored the prediction of the next transaction's MCC code. We call this task \textit{Event type} prediction from here and below. This validation approach was inspired by the work~\cite{zhuzhel2023continuous}, in which it was proposed to predict the type of the next event based on the history of observations --- in our case, the MCC code of the next transaction. Formally, in this case, the multiclass classification problem is solved: given the local representation $\mathbf{h}_{j}^{i}$, we predict the MCC code of the transaction of the $i$-th client, completed at time $t_{j+1}$.

Note that there are many rare MCC types in datasets. From a business perspective, such categories are often less interesting and meaningful. Therefore, to simplify the task, in this procedure for testing local properties, it was decided to leave only transactions corresponding to the $100$ most popular codes.

\subsection{Efficient work in production} 
\label{sec:prod_efficient}

% In our work, we propose a solution for an applied problem in the Financial domain, like credit default prediction. Our approach increases the quality of a downstream task by enriching user representation vectors with an external context. We provide a description of the system for our solution and quantification of system performance in a special section of the paper. 

To integrate our method of external aggregation in the production system to conduct inference, 
we maintain a database of embedding vectors that allows on-the-fly aggregation.
Let us consider all the steps of the proposed procedure below.

The vector database consists of embedding vectors with actual users $D \in \mathbb{R}^{n_0 \times m}$: $n$ user vectors of size $m$, that also appear as a pre-computing embeddings database in the recommendation system domain~\cite{zanardi2011dynamic}. 
To reduce the computational time, we randomly select $n_0 \ll n$ that is significantly smaller than the whole set of users.
In our experiments, $n_0 = 100$ users for a dataset of $n = 5000$ users is enough for aggregation to perform well. 

An update of the database happens daily, to account for new events, such as transactions, as typical user performs no more than three transactions per day. 
An RNN-based model there can easily be updated without rerunning the model on the whole event sequence by using the current embedding as an input. 

The external context vector can be independent and same for all users (mean aggregation) or dependent from user embedding and different for all users (attention aggregation) of the specific user embeddings.  
In the first case, we easily aggregate vectors of selected users in a single vector $\mathbf{g}$ and use it throughout the day.
In the second case, we compute the necessary aggregation on request with a single attention layer.
In both cases, we have a linear in the $n_0$ procedure.

The concluding prediction takes a close time to the usual prediction, as we concatenate the external context embedding and a user-local embedding and process it with a linear or a gradient boosting layer.  

So, our method is easy to develop and fast to employ in real-life environments with little computational overhead.



%scenario is also not computationally expensive. If you need to get predictions for users from the external vector database, you don't need to rerun the model in offline mode. You can just take these users' embeddings from the vector database and concatenate them with external context embedding. It can even speed up computations for  offline model usage scenario with CPU computation for neural networks. If the user is not in the vector database, you need to run the encoder and just compute an actual external context vector or get it from the database with unified computational complexity and get a prediction with an NN head or another model without external aggregation.  


% The size of this database will include  plus one external context vector of the same size.


% % \section{Results and Discussion}

% We start this chapter with methods for validating the resulting embeddings in terms of global and local properties and discussion of used datasets and metrics.
% Next, the chapter provides obtained results of different methods used to take into account external representations. The discussion of results and the proposition for further development of the project are also presented in this part.

% All experiments were carried out in the Python programming language using the models and hyperparameters described in the methods chapter. 

\section{Results}\label{sec:results}

In this section, we describe the results of experiments on obtaining an external context representation and using it to improve existing~models.

\subsection{Methods}

In the experiments, we consider three types of methods: the one without external context inclusion and unlearnable and learnable approaches to aggregate information from other event sequences.
A conventional CoLES encoder~\cite{babaev2022coles} provides strong performance in the considered problems, so we use it in the case without adding external context information (\emph{Without context}).
For aggregation approaches, all learnable parts of these methods come from this CoLES pipeline with fixed original encoders. 
We investigated the following types of aggregation of representations to obtain a context~vector:
\begin{itemize}
    \item straightforward aggregation methods: Averaging (\emph{Mean}), Maximization (\emph{Max}) poolings;
    \item methods inspired by Hawkes process: Exponential Hawkes (\emph{Exp Hawkes}), Exponential learnable Hawkes (\emph{Learnable exp Hawkes*});
    \item attention-based aggregation methods: Attention mechanism without learning (\emph{Attention}), with learning (\emph{Learnable attention*}), Attention with symmetric matrix (\emph{Symmetrical attention*}), Kernel attention (\emph{Kernel attention*});
    \item combined methods: Hawkes with attention (\emph{Attention Hawkes}).
\end{itemize}
All learnable methods are marked with the asterisk sign~"*".



\begin{figure*}[!th]
      \centering
          \includegraphics[width=0.9\textwidth]{figures/star_plot.pdf}
          \caption{\selectfont Quality of the models regarding their global and local properties on the \textit{Churn} dataset (left), \textit{Default} dataset (central), and \textit{HSBC} dataset (right). The $x$-axis corresponds to the global validation ROC-AUC, while the $y$-axis shows the next event type prediction ROC-AUC. Thus, the upper and righter the dot is, the better the model is. A dot in the plot is mean, and lines are std for experiments with 3 seeds. }\label{fig:coles}
\end{figure*}


\begin{figure}[!th]
     \centering
     \includegraphics[width=\columnwidth]{figures/multi_user.pdf}
     \caption{Dependencies between the number of event sequences in external context and forthe  global target (left) and event type (right) ROC-AUC for \textit{Churn} dataset. }
     \label{fig:iters}
\end{figure}

\subsection{Validation}

\paragraph{Procedure.} We consider two types of downstream problems: local and global, defined for a current timestamp for a client or for an event sequences as a whole correspondingly. More detailed problem definitions are in Subsection~\ref{sec:validation_methods}. For global validation, we follow the procedure described in~\cite{babaev2022coles}.
There, self-supervised learning is followed by adding a gradient boosting on top of learned embedding features. 
We use the same hyperparameters for the boosting model.
A more common for computer vision~\cite{grill2020bootstrap} linear probing instead of boosting probing performs worse, so the experiments use the latter one. 
For local validation, we use the procedure described in the corresponding section~\ref{sec:validation_methods}.
All results were averaged across $3$ separate training runs for pre-trained encoder models. 

\paragraph{Datasets.} % \label{sec:data}
To compare the models and methods, we work with three open samples of transactional data: \href{https://boosters.pro/championship/rosbank1}{Churn}, \href{https://boosters.pro/championship/alfabattle2}{Default} and \href{https://www.kaggle.com/datasets/ashisparida/hsbc-ml-hackathon-2023}{HSBC}.
Details on the datasets are in Appendix~\ref{sec:app_datasets}





\subsection{Main results}

% All models were tested in the way described in section \ref{sec:methods} on two applied tasks. 
The results for the Churn, Default, and HSBC datasets are presented in Figure \ref{fig:coles}.
Also, Figure \ref{fig:mean_rank} summarizes these results by presenting ranks of all methods averaged by the mean ROC-AUC metric for all datasets.
Results for two additional non-financial datasets are provided in Appendix~\ref{sec:other-data-results}. We also provide more detailed experiment metrics and rank data results in Table~\ref{tab:metrics} in Appendix~\ref{sec:AppMainTable} and and Table~\ref{tab:metrics} in Appendix~\ref{sec:AppMainTable}.

External context improves metrics in most cases. This is especially noticeable in the balanced Churn sample. In the unbalanced Default and HSBC samples, contextual representations also help models solve local and global problems, but this is not as explicit as in the case of a balanced dataset. Figures also allows us to note, that aggregation methods provides less variation among different seeds, thus, the produced models are more robust. 

According to Figure~\ref{fig:mean_rank} and Table~\ref{tab:ranks}, the best method is the Kernel attention, which performs best or is near the best for all tasks.
However, all attention-based methods with learnable parts (Learnable attention, Symmetrical attention, Kernel attention) often end up among the leaders, especially in the event-type prediction task. 
It is natural, as attention-based probing often leads to stronger models that account for more intricate connections between users.

On the other hand, approaches inspired by the Hawkes process (Exp Hawkes, Exp learnable Hawkes, Attention Hawkes) show inferior results even though these methods take into account the temporal distance of sequence embeddings and, because of this, should respond better to local temporal changes in the data. 
So, accounting for diverse time differences in the context of conditional intensity modelling, given the history of events, remains open for the task of external context addition and requires further research.

Among the methods without learnable parts (Mean, Max, Attention, Exp Hawkes, Attention Hawkes), the Mean and Max methods perform best in local and global validation tasks --- in this case, we can capture average external context, which also should be helpful to specific prediction tasks. 
%This says a lot about society :)
So, default methods of aggregation can be reasonable in the absence of computational resources. 
However, the better choice is to build an external context accounting method with learnable parts, like \emph{Kernel attention}. 

% \usepackage{booktabs}


\begin{table}
	\centering
	\caption{Analysis of models' inference speed.}
	\begin{tabular}{c|ccc} 
		\toprule
		& Parms (MB) & Speed & Moderate  \\ 
		\hline
		IA-SSD     & 2.7        & 84    & 79.12     \\
		PDM-SSD(A) & 3.3        & 84    & 79.37     \\
		PDM-SSD(J) & 3.3        & 68    & 79.75     \\
		\bottomrule
	\end{tabular}
\label{tabel8}
\end{table}

\begin{figure*}[!t]
\centering
\begin{subfigure}{.46\textwidth}
  \centering
  \includegraphics[width=.97\linewidth]{figures/churn_ft.pdf}
  %\captionsetup{justification=raggedright,singlelinecheck=false}
  \caption{Churn dataset}
  \label{fig:scatter_fgsm}
\end{subfigure}
\begin{subfigure}{.46\textwidth}
  \centering
  \includegraphics[width=.97\linewidth]{figures/default_ft.pdf}
  %\captionsetup{justification=raggedright,singlelinecheck=false}
  \caption{Default dataset}
  \label{fig:scatter_kll2}
\end{subfigure}
\caption{ROC-AUC for the task next event type prediction with and without fine-tuning encoder.}
\label{fig:fine-tune}
\end{figure*}

\subsection{Dependence on the number of external sequences}

To show how the number of users influences the construction of external context aggregation, we measured how the quality of applied tasks changes depending on the change.
Considered context sizes included $10$, $50$, $100$, $500$ and $1000$ external sequences.
In this experiment, we select the users whose last event was the most recent for the considered time point.
There, the results are for the best approaches from each cohort: from the pooling methods, it was the Max; from the attention-based methods --- our best Kernel attention, and from methods inspired by the Hawkes method --- Exp Hawkes.  

The results for the Churn dataset are presented in Figure ~\ref{fig:iters}.
The increase in the number of users positively affects the quality of global target and event-type prediction tasks. This statement is not valid for the Exp Hawkes method in the event type prediction task, but as stated above, the Hawkes method group did not work for us. 
On the other hand, $500$ sequences in a context should be enough for the performance sufficiently close to the top one.

\subsection{Fine-tuning backbone for local validation}

To improve the quality of our aggregation methods, we unfreeze the backbone in the process of model head event type prediction training. 
This procedure runs only for the local validation because using a differentiable linear head instead of gradient boosting significantly decreases the quality of the global validation task. 
We compare training the neural network in two modes: training both the encoder and head (unfreeze mode) and training only the head with a freezed encoder, as in all experiments before. 

The results are presented in Figure \ref{fig:fine-tune}. 
Unfreezing of the encoder increases the quality of pooling aggregation, especially for the Churn dataset. 
Besides, the ROC-AUC of all pooling grew substantially compared the encoder without external context.  



\subsection{Computational overhead for the external aggregation}

As discussed in Section~\ref{sec:prod_efficient}, during Inference, the inclusion of external context leads to minor computational overhead if an overall procedure is implemented properly.
However, during training, additional expenses can outweigh potential benefits.

We measured the time required to complete four different training stages.
\begin{itemize}
    \item \textit{Pretrain CoLES SSL} is training encoder with self-supervised loss. In this stage, we get a pre-trained encoder for getting only local representation vectors.
    \item \textit{Pretrain Pool SSL} is training pooling learnable parameters only for learnable pooling methods with self-supervised loss. In this stage, we get a pre-trained encoder with pooling layers for getting both local and external representation vectors.
    \item \textit{Fine-tune, freeze} and \textit{Fine-tune, unfreeze} is training head over freezed or unfreezed backbone
 with pooling on the next event prediction task, respectively. In this stage, we train the head of the encoder for the local validation task.
\end{itemize}
The first stage \textit{Pretrain CoLES SSL} is exactly the training of a basic model, all others are additional costs that can be required to use external aggregation approaches.

The training time in minutes is in Table~\ref{tab:time}.
We see, that all training time are comparable, while additional training with unfreezing requires significant additional resource, while simultaneously boosting the model quality.




% \section{Conclusion}
We introduced \Bench, the first ever IMTS forecasting benchmark.
\Bench's datasets are created with ODE models, that were defined in decades of research and published on
the Physiome Model Repository. Our experiments showed that LinODEnet and CRU are actually
better than previous evaluation on established datasets indicated. Nevertheless,
we also provided a few datasets, on which models are unable to outperform a
constant baseline model. We believe that our datasets, especially the very difficult ones,
can help to identify deficits of current architectures and support future research on
IMTS forecasting.

% \include{chapters/02_author_contrib}
% \include{chapters/03_publications} % optional
% \include{chapters/11_acknowledgements}


%%
%% The next two lines define the bibliography style to be used, and
%% the bibliography file.
\bibliographystyle{ACM-Reference-Format}
\bibliography{sample-base}


\end{document}