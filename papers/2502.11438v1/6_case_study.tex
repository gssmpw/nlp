\subsection{Analysis}


\begin{table}[h]
    \centering
    \small
    \begin{tabular}{lccc}
        \toprule
        Score & cos $\theta$ &\# of Generated EX & \%  Filtered EX \\
        \midrule
        \textbf{$\geq 0$} &0.581& 10340 & 0 \% \\ 
        \textbf{$\geq 2$} &0.625& 10185  & 1.50\% (-155) \\
        \textbf{$\geq 4$} &0.744& 9883 & 4.41\% (-457)  \\
        \textbf{$\geq 6$} &0.762 & 9378 & 9.30\% (-962)  \\
        \textbf{$\geq 8$}&0.765& 8606 & 16.76\% (-1734)\\
        \textbf{$\geq 10$} &0.769& 6795 & 34.28\% (-3545)  \\
        \bottomrule
    \end{tabular}
    %\caption{A summary of the data generation and filtering result, along with an embedding similarity analysis of the filtered examples, categorized by their respective scores.}
\caption{Summary of data generation, filtering results, and embedding similarity analysis by score.}
    \label{tab:number_of_generated}
    % \vspace{-4mm}
\end{table}

% & 0.581          & 0.625            &  0.744         & 0.762          & 0.765    &  \textbf{0.769}  

\begin{figure*}[t]
\centerline{\includegraphics[scale=0.48]{Pictures/corr_bin.pdf}}

\caption{(Left) Correlation between question embedding similarity and average EX, (Right) Average EX across embedding similarity bins}
% \vspace{-4mm}
\label{fig:corr_bin}
\end{figure*}

\paragraph{Number of generated and filtered examples per score, along with an embedding similarity analysis of the filtered examples}
For each test question in the Spider dev set, 10 examples are generated, resulting in a total of 10,340 examples. The quality of these examples is assessed using a relevance score ranging from 0 to 10. As shown in Table~\ref{tab:number_of_generated}, the 65.71\% of examples are assigned a score of 10, while the 0.59\% of examples are received a score of 0. This trend suggests that the LLM tends to assign high relevance to its own generated examples. The similarity is computed using cosine similarity, where higher scores indicate greater semantic alignment between the test questions and the retained examples. As the filtering threshold increases, the embedding similarity also increases, suggesting that higher-relevance examples exhibit stronger semantic consistency with the test questions. However, we also observe that overly strict filtering—selecting only examples with a perfect score of 10—leads to a decline in performance. This drop occurs because an excessively high threshold significantly reduces the number of available examples, limiting the diversity.


\paragraph{Effect of question embedding similarity on Execution Accuracy.}
In Figure~\ref{fig:corr_bin}, the left graph illustrates the correlation between embedding similarity and EX. Each point represents one of the 11 data points obtained by filtering examples based on different threshold scores (0 to 10). The data points follow an upward trend, suggesting that higher similarity tends to result in better EX. The red line indicates the overall correlation, with a coefficient of 0.82, showing a relatively strong positive relationship. Building on this analysis, the right graph provides a more fine-grained view by examining the execution accuracy of individual generated examples based on their embedding similarity with test questions. The x-axis represents the normalized similarity between the test question and the generated question, and the y-axis indicates EX. The results show that EX is lowest in the 0.0-0.1 similarity range, suggesting that examples with very low similarity to test questions tend to be less useful. As similarity increases, EX generally improves, peaking in the 0.7-0.8 range. This suggests that examples with a moderate to high similarity to test questions are more effective in generating executable SQL queries. However, accuracy drops slightly in the 0.8-0.9 range before rising again in the 0.9-1.0 range. This indicates that excessively high similarity can reduce diversity, potentially limiting the model’s generalization ability. 


\begin{figure}[t]
\centerline{\includegraphics[scale=0.36]{Pictures/Diff_threshold_GPT4o.png}}
\caption{Performance of GPT-4o at different relevance score thresholds.}
% \vspace{-5mm}
\label{tab:diff_thres}
\end{figure}


\paragraph{Effect of Relevance Scoring Thresholds on Performance.}

To further evaluate the effectiveness of SAFE-SQL, we conduct a detailed case study using varying thresholds for the relevance scoring mechanism as shown in Figure~\ref{tab:diff_thres}.  The self-generated examples are filtered based on relevance scores, with thresholds ranging from 0 to 10. For each test question, the number of high-scoring examples varied due to the specific content and schema structure (e.g., some test questions had six examples with scores $\geq 8$, while others had three). The selected examples are then used during the final inference stage to generate SQL queries. The $\geq 8$ threshold consistently produced the best results, validating the robustness of SAFE-SQL’s relevance score filtering. The results demonstrate that selecting high-quality examples plays a critical role in guiding LLMs to generate accurate SQL queries, regardless of the underlying model.


\begin{comment}
\begin{table*}[h]
    \centering
    \renewcommand{\arraystretch}{1.3}  % 행 간격 조정
    \begin{tabularx}{\textwidth}{p{4cm} p{6cm} p{4cm} p{6cm}}
        \toprule
        \textbf{Original SQL Question} & \textbf{Original SQL Query} & \textbf{Generated SQL Question} & \textbf{Generated Reasoning Path} \\
        \midrule
        What are all the flights that leave from Aberdeen? & 
        \lstinline|SELECT * FROM flights WHERE departure_city = 'Aberdeen'| & 
        What are all the flights departing from Aberdeen? & 
        Identify all flights with Aberdeen as the departure city. \\
        
        Of those, which land in Ashley? & 
        \lstinline|SELECT * FROM flights WHERE departure_city = 'Aberdeen' AND arrival_city = 'Ashley'| & 
        Which flights leave from Aberdeen and land in Ashley? & 
        Filter previous results to include only flights arriving in Ashley. \\
        
        How many are there? & 
        \lstinline|SELECT COUNT(*) FROM flights WHERE departure_city = 'Aberdeen' AND arrival_city = 'Ashley'| & 
        How many flights travel from Aberdeen to Ashley? & 
        Count the number of flights from the filtered list. \\
        \midrule
        
        What are all the airlines? & 
        \lstinline|SELECT DISTINCT airline FROM flights| & 
        What airlines operate flights? & 
        Retrieve distinct airline names from the flights table. \\
        
        Of these, which is JetBlue Airways? & 
        \lstinline|SELECT * FROM flights WHERE airline = 'JetBlue Airways'| & 
        Which flights are operated by JetBlue Airways? & 
        Filter flights to include only those operated by JetBlue Airways. \\
        
        What is the country corresponding it? & 
        \lstinline|SELECT country FROM airlines WHERE name = 'JetBlue Airways'| & 
        What country is JetBlue Airways based in? & 
        Retrieve the country associated with JetBlue Airways from the airlines table. \\
        \bottomrule
    \end{tabularx}
    \caption{Examples of original and generated SQL questions with reasoning paths.}
    \label{tab:sql_examples}
\end{table*}
\end{comment}

\begin{comment}
\begin{table*}[h]
    \centering
    \small
    \renewcommand{\arraystretch}{1.3}  % Adjust row spacing
    \begin{tabularx}{\textwidth}{X X X X X}
        \toprule
        \textbf{SQL Question} & \textbf{GOLD SQL Query} & \textbf{Augmented SQL Question} & \textbf{Generated Reasoning Path} & \textbf{Relevance Score} \\
        \midrule
        \hl{Question1:}
        What are the names, countries, and ages for every singer in descending order of age? & 
        \texttt{SELECT name, country, age FROM singer ORDER BY age DESC} & 
        \sethlcolor{lime!50}
        \hl{What are the names, ages, and countries of all singers from a specific country, sorted by age in descending order?} & 
        \sethlcolor{violet!20}
        \hl{1.Identify the desired columns: name, age, and country. 
        2.Specify the table: singer. 
        3.Sort the results by age in descending order.}& semantic similarity:3   Structure \& key word 
 score: 3  Reasoning patt score:4 Relevance score = 10
        \\
        \midrule
        \hl{Question2:}
        What is the number of car models that are produced by each maker and what is the id and full name of each maker?
        &  
        \texttt{SELECT Count(*), T2.FullName , T2.id FROM MODEL\_LIST AS T1 JOIN CAR\_MAKERS AS T2 ON T1.Maker = T2.id GROUP BY T2.id;} & 
               \sethlcolor{lime!50}
 \hl{Could you provide the count of car models produced by each manufacturer, along with the ID and full name of each manufacturer?} & 
 \sethlcolor{violet!20}
 \hl{1.Retrieve Required Information: Count car models per maker and get each maker's ID and full name. 2.Join Tables: Link MODEL\_LIST (T1) and CAR\_MAKERS (T2) using T1.Maker = T2.Id. 3.Group and Aggregate: Use COUNT(*) to count models and group by T2.id. 4.Select Output: Return the model count, maker’s full name, and ID.} & semantic similarity:1   Structure \& key word 
 score: 2  Reasoning patt score:3 Relevance score = 6 \\ 
        \midrule
        \hl{Question3:} Return the names and template ids for documents that contain the letter w in their description. & 
        \texttt{SELECT document\_name , template\_id FROM Documents WHERE Document\_Description LIKE "\%w\%"} & 
        \sethlcolor{lime!50}
        \hl{Retrieve the names and template IDs of documents whose descriptions include the letter 'w'.} & 
        \sethlcolor{violet!20}
        \hl{1.Extract the document\_name and template\_id columns.
        2.Search within the Documents table. 3.Filter results where Document\_Description contains the letter "w".
        Return the matching records.} & semantic similarity:3   Structure \& key word 
 score: 2  Reasoning path score:3 Relevance score = 8  \\

        \bottomrule
    \end{tabularx}
    \caption{Examples of original and Augmented SQL questions with reasoning paths by GPT-4o mini.}
    \label{tab:sql_examples}
\end{table*}
\end{comment}

\begin{table*}[t]
    \centering
    \scriptsize
    \renewcommand{\arraystretch}{1.2}
    \resizebox{1.97\columnwidth}{!}{% Adjust row spacing
    \begin{tabularx}{\textwidth}{X X X >{\raggedright\arraybackslash}p{0.3\textwidth} X}
        \toprule
        \textbf{SQL Question} & \textbf{GOLD SQL Query} & \textbf{Augmented SQL Question} & \textbf{Generated Reasoning Path} & \textbf{Relevance Score} \\
        \midrule
        \hl{Question1:} What are the names, countries, and ages for every singer in descending order of age? & 
        \texttt{SELECT name, country, age FROM singer ORDER BY age DESC} & 
        \sethlcolor{lime!50}\hl{What are the names, ages, and countries of all singers from a specific country, sorted by age in descending order?} & 
        \sethlcolor{violet!20}\hl{1. Identify the desired columns: name, age, and country. \\
        2. Specify the table: singer. \\
        3. Sort the results by age in descending order.} &  
        Semantic similarity = 10 
        Structural Similarity = 10  
        Reasoning path = 10  
        Relevance score = (10+10+10)/3 = 10
                \\ 
        \midrule
        \hl{Question2:} Return the names and template ids for documents that contain the letter w in their description. & 
        \texttt{SELECT document\_name, template\_id FROM Documents WHERE Document\_Description LIKE "\%w\%"} & 
        \sethlcolor{lime!50}\hl{Retrieve the titles and category IDs of articles whose summaries contain the word "data".} &
        \sethlcolor{violet!20}\hl{1. Identify the necessary columns: extract title and category\_id from the Articles table. \\
         2. Locate the relevant table: use the Articles table as it contains the required data. \\
        3. Define the filtering condition: apply a WHERE clause to check if the summary column contains the substring "data". \\
         4. Use the LIKE operator: employ LIKE '\%data\%' to search for any instance of "data" within the summary. \\
        5. Retrieve the results: return the title and category\_id values for all matching records.} & Semantic similarity = 7
                    Structural Similarity = 9
Reasoning path = 8
Relevance score = (7+9+8)/3 = 8
        \\
        \midrule
        \hl{Question3:} What is the number of car models that are produced by each maker and what is the id and full name of each maker? &  
        \texttt{SELECT Count(*), T2.FullName, T2.id FROM MODEL\_LIST AS T1 JOIN CAR\_MAKERS AS T2 ON T1.Maker = T2.id GROUP BY T2.id;} & 
        \sethlcolor{lime!50}\hl{List all employees who work in the IT department along with their employee ID and hire date.} & 
        \sethlcolor{violet!20}\hl{1. Identify required details: employee ID and hire date. \\
        2. Filter condition: find employees who work in IT. \\
        3. Retrieve data: select only emp\_id and hire\_date.} & 
        Semantic similarity = 6
        Structural Similarity = 3
        Reasoning path = 2
        Relevance score = (6+3+2)/3 = 3.67
  \\
        \bottomrule
    \end{tabularx}
    }
    %\vspace{-2mm}
    \caption{Examples of original and augmented SQL questions with reasoning paths by GPT-4o.}
    \label{tab:sql_examples}
    \vspace{-4mm}
\end{table*}


%This experiment is performed across multiple models, including GPT 4o Mini, Deepseek Coder 6.7B, %Llama 3.1 8B, and Starcoder 7B.
\begin{table}[t]
    \centering
    \small
    \resizebox{0.48\textwidth}{!}{
    \begin{tabular}{lcc||ccccc}
        \toprule
        \textbf{$\alpha$} & \textbf{$\beta$} &\textbf{$\gamma$}& \textbf{Easy}& \textbf{Medium}& \textbf{Hard} &\textbf{Extra}& \textbf{EX} \\
        \midrule
        0.33 & 0.33 & 0.33 & \textbf{93.4} & \textbf{89.3} & \underline{88.4} & \textbf{75.8} & \textbf{87.9} \\ 
        \midrule
        1 & 0 & 0 & 90.7& 84.2& 82.3& 68.3&  82.8 \\ 
        0 & 1 & 0 & 89.8& 85.6& 81.2& 69.2&  83.1 \\ 
        0 & 0 & 1 & 89.2& 85.1& 84.3& 71.7& 83.7  \\ 
        \midrule
        0.5 & 0.5 & 0& 91.2& 87.3& 82.5& 69.4& 84.4 \\ 
        0.5 & 0 & 0.5& 92.5& \underline{87.9}& 83.5& 70.3& 85.3 \\ 
        0 & 0.5 & 0.5& \underline{92.7}& 86.8& \textbf{88.5}& \underline{72.4}& \underline{86.1} \\ 
        \bottomrule
    \end{tabular}
    }
    %\vspace{-2mm}
    %\caption{Execution accuracy across different difficulty levels with varying weights of semantic similarity ($\alpha$), keyword \& structural similarity ($\beta$), and reasoning path quality ($\gamma$).}
    \caption{Execution accuracy across difficulty levels under different weights: semantic similarity ($\alpha$), Structural similarity ($\beta$), and reasoning path quality ($\gamma$).}
    % \vspace{-4mm}
    \label{tab:filtering_score_ablation}
\end{table} 

\paragraph{Effect of three measuring components on Performance.}

To assess the impact of the three measuring components—semantic similarity ($\alpha$), keyword \& structural similarity ($\beta$), and reasoning path quality ($\gamma$)—on EX, we conduct experiments by varying their respective weightings. The results, presented in Table~\ref{tab:filtering_score_ablation}, highlight distinct performance trends across different difficulty levels. Notably, the exclusion of reasoning path quality leads to a drop in EX, particularly in the Hard and Extra Hard. This suggests that a well-structured reasoning path is crucial for handling complex queries, as it provides essential logical steps that bridge the gap between natural language understanding and SQL formulation. Conversely, semantic similarity and structural SQL query similarity have a greater influence on the Easy and Medium levels. This is because these queries tend to be relatively straightforward, meaning that having structurally similar SQL questions in the example set often provides sufficient guidance for generating correct queries. In these cases, direct pattern matching and schema alignment play a larger role. Overall, the findings demonstrate that a balanced combination of all three components is essential for optimizing performance across different levels of query complexity.

% Simply maximizing similarity may not always yield the best results, and a balanced approach that considers both relevance and diversity could be more effective.


%: Qwen2.5-3B-instruct, Qwen2.5-7B-instruct, and Qwen2.5-14B-instruct 


\subsection{Case Study}
As shown in Table~\ref{tab:sql_examples}, test questions from the Spider dev set alongside their generated similar examples, evaluated based on semantic similarity, structural similarity, and the reasoning path score, which together determine the relevance score. The first example achieves a perfect relevance score of 10, as the generated question closely aligns with the original in meaning, structure, and reasoning. The SQL formulation remains nearly identical, and the reasoning path explicitly details each step, ensuring full alignment. The second example receives a relevance score of 8, with semantic similarity of 7 due to minor differences in terminology ("documents" vs. "articles" and "letter 'w'" vs. "word 'data'"). However, its structural similarity remains high, as the SQL structure is nearly identical. The reasoning path score of 8 reflects a clear explanation of query formulation, though slightly less detailed than the first example. The third example has the lowest relevance score due to significant differences. The generated question shifts focus from counting car models to listing IT employees, resulting in semantic similarity of 6 and structural similarity of 3. These results emphasize the importance of fine-grained example selection due to the varing quality of generated examples.
% Table 6번 언급되는 곳이 하나도 없었습니다. 맨뒤로 빼고 Case Study 만들어서 설명할 필요가 있습니다. 또한 Relevance Score 변경했는데 확인해주셔야합니다