\begin{table}[h]
\centering
\scalebox{0.85}{
\begin{tabular}{l}
\hline
\begin{tabular}[c]{@{}p{1.0\linewidth}@{}}
You are a powerful text-to-SQL reasoner. Your task is to generate ten similar questions, ten SQL queries, and ten reasoning paths for how the SQL queries are derived.
To ensure high-quality examples, focus on the following three key aspects:\\ \\ 
\textbf{Semantic Similarity}\\
Ensure that all generated questions have the same underlying meaning as the test question. Variations in wording, synonyms, and phrasing are allowed as long as they preserve the intended query objective.
Avoid introducing ambiguity or additional constraints that alter the intent.\\
\textbf{Structural Similarity}\\
While key terms (such as table names, column names, and numerical values) may vary, their functional roles and relationships should remain intact. \\
\textbf{Reasoning Path Similarity}\\
The logical reasoning required to construct the SQL query should remain consistent across examples.Clearly outline each step, including how key conditions are identified and mapped to SQL operations.Maintain coherence in how joins, aggregations, filters, and sorting operations are applied. 

Do not explain me about the result and just give me ten examples.
\\ \\
\textbf{\#\# Schema linking:} {schema\_linking[i]} \\
\textbf{\#\# Tables:} {test\_table[i]} \\
\textbf{\#\# Foreign keys:} {test\_foreign\_keys[i]} \\
\textbf{\#\# Question:} {test\_question[i]} \\ \\
\textbf{\#\# Similar Question:} \\
\textbf{\#\# SQL query:} \\
\textbf{\#\# Reasoning Path:} \\
\end{tabular}  \\ 
\hline
\end{tabular}
}
\caption{The zero-shot prompt used for example generation}
\label{tab:example_generation}
\end{table}


