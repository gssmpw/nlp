\begin{table}[]
\centering
\scalebox{0.85}{\begin{tabular}{l}
\hline
\begin{tabular}[c]{@{}p{1.0\linewidth}@{}}\\
You are a powerful text-to-SQL reasoner. Your task is to generate the final SQL query using a set of selected examples that provide guidance on query construction. Utilizing Selected Examples. Do not explain me about the answer, just give me SQL query. \\
A set of chosen examples, each containing:
A natural language question similar to the test question
A corresponding SQL query
A detailed reasoning path explaining how the SQL query was derived
These examples are selected based on three key criteria:
\\ \\ 
\textbf{Semantic Similarity of Questions}
The selected examples closely match the intent of the test question.
Variations in wording do not change the meaning.
\\
\textbf{Structural Similarity}
The database schema elements (tables, columns, joins) used in the examples align with the test question.
The SQL syntax and structure are relevant to the expected query.
\\
\textbf{Reasoning Path Similarity}
The logical steps used to construct the SQL query align with the reasoning required for the test question.
Key transformations, filtering conditions, and aggregation logic are similar.
\\
\textbf{Final SQL Query Construction} \\
Using the selected examples, generate the final SQL query that correctly retrieves the desired result for the given test question.
Follow the reasoning patterns observed in the examples.
Maintain correct table joins, filters, aggregations, and conditions based on schema constraints. Ensure that the final query accurately represents the intent of the test question while leveraging the insights from the selected examples. Now, generate the final SQL query for the given test question: \\ \\ 
\textbf{\#\#Tables:} test\_table[i]\\
\textbf{\#\#Foreign\_keys:} test\_foreign\_keys[i]\\
\textbf{\#\#Question:} text\_question[i] \\
\textbf{\#\#Filtered\_example:} filtered\_example[i] \\

\end{tabular}  \\ \hline
\end{tabular}}
\caption{The zero-shot prompt used for Final SQL query inference.}
\label{tab:final_prompt}
\end{table}

