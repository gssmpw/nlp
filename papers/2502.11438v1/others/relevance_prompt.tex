\begin{table}[]
\centering
\small
\scalebox{0.95}{
\begin{tabular}{l}
\hline
\begin{tabular}[c]{@{}p{1.0\linewidth}@{}}
You are a powerful text-to-SQL reasoner. Given a test question and a set of examples, compute the relevance score for each example based on the following criteria. Do not explain me about the answer, just give me scores.  \\ \\
\textbf{\*\*Semantic Similarity of Questions} \\
Compare the overall meaning of the test question and the example question.
Higher scores should be assigned if the two questions have the same intent, even if they are phrased differently.
Consider synonyms, paraphrasing, and minor wording variations that do not alter the fundamental meaning.
Assign lower scores if the test and example questions focus on different database operations (e.g., aggregation vs. filtering) or require fundamentally different types of information.(up to 10 points).\\
10: Almost identical meaning and intent.\\
7–9: Minor paraphrasing but highly relevant.\\
4–6: Some overlap but different focus.\\
1–3: Mostly unrelated meaning.\\
0: Completely different intent.
\\ \\
\textbf{\*\*Keyword \& Structural Similarity} \\ 
Evaluate the structural alignment between the test question and the example question by analyzing how key elements (such as entities, attributes, and numerical values) are connected. Even if individual nouns, verbs, or numbers differ, the overall relational structure should be considered. Focus on whether the dependencies between key components (e.g., how entities relate to each other in the database) remain consistent.(up to 10 points). \\
10: Nearly identical structural relationships and dependencies. \\
7–9: Mostly similar structure, with minor differences in entity connections. \\
4–6: Some overlap, but noticeable differences in how key components interact. \\
1–3: Few shared structural relationships, making alignment weak. \\
0: No meaningful structural similarities. 
\\ \\ 
\textbf{\*\*Reasoning Path Similarity} \\ 
Evaluate whether the logical steps needed to answer the example question align with those required for the test question. Consider whether the database operations (e.g., filtering, aggregation, joins, subqueries) are similar.A high score should be given if the example follows the same logical sequence to derive the SQL query.Lower scores should be assigned if the reasoning process differs significantly, even if the questions seem similar at a surface level.(up to 10  points). \\
10: Exact reasoning process to get right SQL query.\\
7–9: Mostly similar but with minor differences.\\
4–6: Some alignment but different key steps.\\
1–3: Largely different reasoning.\\
0: Completely unrelated logic. \\ \\

\textbf{\#\# Question:} {test\_question[i]} \\ 
\textbf{\#\# Similar Question:} {similar\_question[i]} \\
\textbf{\#\# Reasoning Path:} {reasoning\_path[i]} \\
\textbf{\#\# Relevance score:} 
\end{tabular}  \\ 
\hline
\end{tabular} 
}
\caption{The zero-shot prompt used for filtering examples.}
\label{tab:filtering_exmaples}
\end{table}

