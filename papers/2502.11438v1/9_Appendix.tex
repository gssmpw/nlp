\appendix
\newpage
\clearpage
\section{Appendix}

% \subsection{Number of generated examples per score.}
% \label{appen:Full_number_of_generated}
% Number of generated examples per score.
% \input{Tables/Full_number_of_generated_ex(APP)}





%\section{Performance based on generated examples across different model size}
%\begin{table}[t]
\centering
\small
\begin{tabularx}{\linewidth}{lXXXXX}
\toprule
\textbf{}  & \textbf{Easy} & \textbf{Med} & \textbf{Hard} & \textbf{Extra} & \textbf{All} \\ \midrule
\textbf{Qwen 2.5-3B}  & 62.4          &     61.2        &  58.6         & 48.8          & 59.1         \\
\textbf{Qwen 2.5-7B}  & 80.0         & 78.0           & 67.2         &   51.8        & 72.3        \\
\textbf{Qwen 2.5-14B} & \textbf{81.2}          & \textbf{80.3}            & \textbf{69.5}         & \textbf{56.4}          & \textbf{74.7}      \\ 
\bottomrule
\end{tabularx}
\caption{Execution accuracy performance of different size of models of Qwen series across difficulty levels of spider dev set.}
\label{tab:qwen_ab_models}
\end{table}
% \begin{table}[ht]
%     \centering
%     \small
%     \resizebox{0.48\textwidth}{!}{
%     \begin{tabular}{lcc||ccccc}
%         \toprule
%         EG & F &FA& \textbf{Easy}& \textbf{Medium}& \textbf{Hard} &\textbf{Extra}& \textbf{EX} \\
%         G & G & G & & & & &  - \\ 
%         \midrule
%         Q3 & G & G & & & & &  - \\ 
%         Q3 & Q3 & G & & & & &  - \\ 
%         Q3 & Q3 & Q3 & & & & &  - \\ 
%         \midrule
%         Q7 & G & G & & & & &  - \\ 
%         Q7 & Q7 & G & & & & &  - \\ 
%         Q7 & Q7 & Q7 & & & & &  - \\ 
%         \midrule
%         Q14 & G & G & & & & &  - \\ 
%         Q14 & Q14 & G & & & & &  - \\ 
%         Q14 & Q14 & Q14 & & & & &  - \\ 

%         \bottomrule
%     \end{tabular}
%     }
%     \caption{.}
%     \label{tab:qwen_ablation}
% \end{table}

\section{Prompts for SAFE-SQL}
\subsection{Prompt for example generation.}
\label{appen:example_prompt}
For example generation, we use zero shot prompt as shown in the figure~\ref{tab:example_generation}. 
\begin{table}[h]
\centering
\scalebox{0.85}{
\begin{tabular}{l}
\hline
\begin{tabular}[c]{@{}p{1.0\linewidth}@{}}
You are a powerful text-to-SQL reasoner. Your task is to generate ten similar questions, ten SQL queries, and ten reasoning paths for how the SQL queries are derived.
To ensure high-quality examples, focus on the following three key aspects:\\ \\ 
\textbf{Semantic Similarity}\\
Ensure that all generated questions have the same underlying meaning as the test question. Variations in wording, synonyms, and phrasing are allowed as long as they preserve the intended query objective.
Avoid introducing ambiguity or additional constraints that alter the intent.\\
\textbf{Structural Similarity}\\
While key terms (such as table names, column names, and numerical values) may vary, their functional roles and relationships should remain intact. \\
\textbf{Reasoning Path Similarity}\\
The logical reasoning required to construct the SQL query should remain consistent across examples.Clearly outline each step, including how key conditions are identified and mapped to SQL operations.Maintain coherence in how joins, aggregations, filters, and sorting operations are applied. 

Do not explain me about the result and just give me ten examples.
\\ \\
\textbf{\#\# Schema linking:} {schema\_linking[i]} \\
\textbf{\#\# Tables:} {test\_table[i]} \\
\textbf{\#\# Foreign keys:} {test\_foreign\_keys[i]} \\
\textbf{\#\# Question:} {test\_question[i]} \\ \\
\textbf{\#\# Similar Question:} \\
\textbf{\#\# SQL query:} \\
\textbf{\#\# Reasoning Path:} \\
\end{tabular}  \\ 
\hline
\end{tabular}
}
\caption{The zero-shot prompt used for example generation}
\label{tab:example_generation}
\end{table}






\subsection{Prompt for filtering examples.}
\label{appen:filtering_examples}
For example generation, we use zero shot prompt as shown in figure~\ref{tab:filtering_exmaples}. 
\begin{table}[]
\centering
\small
\scalebox{0.95}{
\begin{tabular}{l}
\hline
\begin{tabular}[c]{@{}p{1.0\linewidth}@{}}
You are a powerful text-to-SQL reasoner. Given a test question and a set of examples, compute the relevance score for each example based on the following criteria. Do not explain me about the answer, just give me scores.  \\ \\
\textbf{\*\*Semantic Similarity of Questions} \\
Compare the overall meaning of the test question and the example question.
Higher scores should be assigned if the two questions have the same intent, even if they are phrased differently.
Consider synonyms, paraphrasing, and minor wording variations that do not alter the fundamental meaning.
Assign lower scores if the test and example questions focus on different database operations (e.g., aggregation vs. filtering) or require fundamentally different types of information.(up to 10 points).\\
10: Almost identical meaning and intent.\\
7–9: Minor paraphrasing but highly relevant.\\
4–6: Some overlap but different focus.\\
1–3: Mostly unrelated meaning.\\
0: Completely different intent.
\\ \\
\textbf{\*\*Keyword \& Structural Similarity} \\ 
Evaluate the structural alignment between the test question and the example question by analyzing how key elements (such as entities, attributes, and numerical values) are connected. Even if individual nouns, verbs, or numbers differ, the overall relational structure should be considered. Focus on whether the dependencies between key components (e.g., how entities relate to each other in the database) remain consistent.(up to 10 points). \\
10: Nearly identical structural relationships and dependencies. \\
7–9: Mostly similar structure, with minor differences in entity connections. \\
4–6: Some overlap, but noticeable differences in how key components interact. \\
1–3: Few shared structural relationships, making alignment weak. \\
0: No meaningful structural similarities. 
\\ \\ 
\textbf{\*\*Reasoning Path Similarity} \\ 
Evaluate whether the logical steps needed to answer the example question align with those required for the test question. Consider whether the database operations (e.g., filtering, aggregation, joins, subqueries) are similar.A high score should be given if the example follows the same logical sequence to derive the SQL query.Lower scores should be assigned if the reasoning process differs significantly, even if the questions seem similar at a surface level.(up to 10  points). \\
10: Exact reasoning process to get right SQL query.\\
7–9: Mostly similar but with minor differences.\\
4–6: Some alignment but different key steps.\\
1–3: Largely different reasoning.\\
0: Completely unrelated logic. \\ \\

\textbf{\#\# Question:} {test\_question[i]} \\ 
\textbf{\#\# Similar Question:} {similar\_question[i]} \\
\textbf{\#\# Reasoning Path:} {reasoning\_path[i]} \\
\textbf{\#\# Relevance score:} 
\end{tabular}  \\ 
\hline
\end{tabular} 
}
\caption{The zero-shot prompt used for filtering examples.}
\label{tab:filtering_exmaples}
\end{table}



\subsection{Prompt for final inference.}
\label{appen:final_inference}
For final inference, we use zero shot prompt as shown in figure~\ref{tab:final_prompt}. 
\begin{table}[]
\centering
\scalebox{0.85}{\begin{tabular}{l}
\hline
\begin{tabular}[c]{@{}p{1.0\linewidth}@{}}\\
You are a powerful text-to-SQL reasoner. Your task is to generate the final SQL query using a set of selected examples that provide guidance on query construction. Utilizing Selected Examples. Do not explain me about the answer, just give me SQL query. \\
A set of chosen examples, each containing:
A natural language question similar to the test question
A corresponding SQL query
A detailed reasoning path explaining how the SQL query was derived
These examples are selected based on three key criteria:
\\ \\ 
\textbf{Semantic Similarity of Questions}
The selected examples closely match the intent of the test question.
Variations in wording do not change the meaning.
\\
\textbf{Structural Similarity}
The database schema elements (tables, columns, joins) used in the examples align with the test question.
The SQL syntax and structure are relevant to the expected query.
\\
\textbf{Reasoning Path Similarity}
The logical steps used to construct the SQL query align with the reasoning required for the test question.
Key transformations, filtering conditions, and aggregation logic are similar.
\\
\textbf{Final SQL Query Construction} \\
Using the selected examples, generate the final SQL query that correctly retrieves the desired result for the given test question.
Follow the reasoning patterns observed in the examples.
Maintain correct table joins, filters, aggregations, and conditions based on schema constraints. Ensure that the final query accurately represents the intent of the test question while leveraging the insights from the selected examples. Now, generate the final SQL query for the given test question: \\ \\ 
\textbf{\#\#Tables:} test\_table[i]\\
\textbf{\#\#Foreign\_keys:} test\_foreign\_keys[i]\\
\textbf{\#\#Question:} text\_question[i] \\
\textbf{\#\#Filtered\_example:} filtered\_example[i] \\

\end{tabular}  \\ \hline
\end{tabular}}
\caption{The zero-shot prompt used for Final SQL query inference.}
\label{tab:final_prompt}
\end{table}


\label{appen:Impact_size}





\section{Impact of model size}

\begin{table}[t]
\centering
\small
\begin{tabularx}{\linewidth}{lXXXXX}
\toprule
\textbf{}  & \textbf{Easy} & \textbf{Med} & \textbf{Hard} & \textbf{Extra} & \textbf{All} \\ \midrule
\textbf{Qwen 2.5-3B}  & 62.4          &     61.2        &  58.6         & 48.8          & 59.1         \\
\textbf{Qwen 2.5-7B}  & 80.0         & 78.0           & 67.2         &   51.8        & 72.3        \\
\textbf{Qwen 2.5-14B} & \textbf{81.2}          & \textbf{80.3}            & \textbf{69.5}         & \textbf{56.4}          & \textbf{74.7}      \\ 
\bottomrule
\end{tabularx}
\caption{Execution accuracy performance of different size of models of Qwen series across difficulty levels of spider dev set.}
\label{tab:qwen_ab_models}
\end{table}
% \begin{table}[ht]
%     \centering
%     \small
%     \resizebox{0.48\textwidth}{!}{
%     \begin{tabular}{lcc||ccccc}
%         \toprule
%         EG & F &FA& \textbf{Easy}& \textbf{Medium}& \textbf{Hard} &\textbf{Extra}& \textbf{EX} \\
%         G & G & G & & & & &  - \\ 
%         \midrule
%         Q3 & G & G & & & & &  - \\ 
%         Q3 & Q3 & G & & & & &  - \\ 
%         Q3 & Q3 & Q3 & & & & &  - \\ 
%         \midrule
%         Q7 & G & G & & & & &  - \\ 
%         Q7 & Q7 & G & & & & &  - \\ 
%         Q7 & Q7 & Q7 & & & & &  - \\ 
%         \midrule
%         Q14 & G & G & & & & &  - \\ 
%         Q14 & Q14 & G & & & & &  - \\ 
%         Q14 & Q14 & Q14 & & & & &  - \\ 

%         \bottomrule
%     \end{tabular}
%     }
%     \caption{.}
%     \label{tab:qwen_ablation}
% \end{table}

\paragraph{Performance based on generated examples across different model size}
As shown in Table~\ref{tab:qwen_ab_models}, We investigate the impact of model size on example generation with different variants of the Qwen2.5 Models. The results demonstrate that the 14B model achieves the highest overall performance, followed by the 7B and the 3B. This trend is consistent across all difficulty levels, with large model size generating higher-quality examples that lead to more accurate SQL query generation. The performance improvement with increasing model size can be attributed to the enhanced capacity of larger models to capture SQL question patterns and semantic relationships. Moreover, larger models possess more extensive information, allowing them to generate more appropriate questions and construct detailed reasoning paths, which contribute to the overall accuracy of SQL query generation.


\section{Spider dev training set embedding clusters.}
% \begin{figure}[h]
% \centerline{\includegraphics[scale=0.35]{Pictures/tsne_embedding_train.pdf}}
% \caption{t-SNE embedding clusters of the Spider dev training dataset categorized into 15 groups.}
% \end{figure}
\begin{figure}[h]
\centerline{\includegraphics[scale=0.35]{Pictures/tsne_embedding_train.pdf}}
\caption{Embedding of spider dev set training questions.}
\vspace{-5mm}
\label{tab:spider_dev_embedding}
\end{figure}
 Although questions within the same category share semantic similarities, they may belong to different clusters, leading to inconsistencies when retrieving examples from the training set. This highlights the limitations of training set retrieval in Text-to-SQL tasks.
 

\begin{table*}[t]
\centering
\begin{tabular}{llcccccc}
\toprule
\textbf{Methods}& \textbf{Model}& \textbf{Type}  & \textbf{Easy} & \textbf{Medium} & \textbf{Hard} & \textbf{Extra} & \textbf{All} \\  
\midrule 
\textit{\textbf{ \textit{SAFE-SQL}}}& Llama3.1-8B-Instruct& ICL & {73.2\%} & {76.1}\% & {63.2\%} & {59.4\%} & {70.5\%} 

\\
\textit{\textbf{ \textit{SAFE-SQL}}}& Deepseek-coder-6.7B & ICL & {88.8\%} & {65.5}\% & {63.8\%} & {25.3\%} & {64.2\%} 

\\
\textit{\textbf{ \textit{SAFE-SQL}}}& Qwen2.5-7B-Instruct & ICL & {83.6\%} & {80.7}\% & {78.7\%} & {69.4\%} & {79.2\%} 

\\ 
\textit{\textbf{ \textit{SAFE-SQL}}}& Starcoder-7B& ICL & {89.2\%} & {88.9}\% & {84.5\%} & {70.6\%} & {85.2\%} 
\\
\bottomrule

\end{tabular}
\caption{Execution accuracy performance of different methods across difficulty levels of spider dev set.}
\label{tab:another_main}
\end{table*}
\subsection{Additional model performance} 
To evaluate the impact of example generation quality on Text-to-SQL performance, we conducted experiments using different models for final inference. Examples generated by GPT-4o, followed by inference using the target model. Large language models', such as Qwen 2.5-7B and Deepseek-coder-6.7B, ability to generate high-quality, semantically relevant in-context examples is limited. To mitigate this, we first used GPT-4o to generate in-context examples and filtering examples, then performed final inference using the selected model. Our results show that leveraging GPT-4o for example generation and scoring improved overall execution accuracy by 6.9 points compared to fully relying on Qwen 2.5-7B for the entire process as shown in Table~\ref{tab:qwen_ab_models}. This confirms that high-quality, well-aligned in-context examples play a crucial role in enhancing Text-to-SQL performance, especially in complex queries.