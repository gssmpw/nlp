\maketitle
\begin{comment}
\begin{abstract}
Text-to-SQL is the task of transforming natural language questions into executable SQL queries, enabling seamless database interaction. 
While existing approaches, such as skeleton-masked selection, rely on retrieving similar examples from the training set to guide query generation, these methods often fail when similar examples are unavailable, a common challenge in real-world scenarios. 
To overcome this limitation, we propose SAL-SQL, a novel framework that empowers large language models (LLMs) to autonomously improve their SQL generation process. 
By iteratively refining inputs through self-generated examples and structured relevance assessments, SAL-SQL addresses the limitations of traditional retrieval-based methods. 
SAL-SQL outperforms previous zero-shot Text-to-SQL frameworks, achieving superior accuracy and robustness. 
Experimental results demonstrate significant improvements in SQL query accuracy, particularly in complex and unseen scenarios where conventional approaches struggle.
\end{abstract}
\end{comment}

\begin{abstract}
Text-to-SQL aims to convert natural language questions into executable SQL queries. While previous approaches, such as skeleton-masked selection, have demonstrated strong performance by retrieving similar training examples to guide large language models (LLMs), they struggle in real-world scenarios where such examples are unavailable. To overcome this limitation, we propose Self-Augmentation in-context learning with Fine-grained Example selection for Text-to-SQL (SAFE-SQL), a novel framework that improves SQL generation by generating and filtering self-augmented examples. SAFE-SQL first prompts an LLM to generate multiple Text-to-SQL examples relevant to the test input. Then SAFE-SQL filters these examples through three relevance assessments, constructing high-quality in-context learning examples. Using self-generated examples, SAFE-SQL surpasses the previous zero-shot, and few-shot Text-to-SQL frameworks, achieving higher execution accuracy. Notably, our approach provides additional performance gains in extra hard and unseen scenarios, where conventional methods often fail.
\end{abstract}