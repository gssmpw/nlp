% This must be in the first 5 lines to tell arXiv to use pdfLaTeX, which is strongly recommended.
\pdfoutput=1
% In particular, the hyperref package requires pdfLaTeX in order to break URLs across lines.

\documentclass[11pt]{article}
\usepackage[preprint]{acl} 
%\usepackage{acl}
\usepackage{times}
\usepackage{latexsym}
\usepackage[T1]{fontenc}
\usepackage[utf8]{inputenc}
\usepackage{microtype}
\usepackage{inconsolata}
\usepackage{graphicx}
\usepackage{booktabs}
\usepackage{amsmath}
\usepackage{array}
\usepackage{xcolor}
\usepackage{float}
\usepackage{amssymb}
\usepackage{comment}
\usepackage{soul}

\usepackage{booktabs}  % For professional table rules
\usepackage{tabularx}  % For flexible-width tables
\usepackage{listings}  % For inline code formatting
\usepackage{array}     % For better column customization
\usepackage{comment}
\usepackage{enumitem}
%\setlist[itemize]{leftmargin=0pt, labelindent=0pt}

\begin{document}

%\title{SAL-SQL: Self augmentation and refinement Learning in Text-to-SQL}
\title{SAFE-SQL: Self-Augmented In-Context Learning with Fine-grained Example Selection for Text-to-SQL }


\author{Jimin Lee, ~Ingeol Baek, ~Byeongjeong Kim, ~Hwanhee Lee\thanks{Corresponding author.}\\
{Department of Artificial Intelligence, Chung-Ang University, Seoul, Korea} \\
\texttt{\{ljm1690, ingeolbaek, michael97k, hwanheelee\}@cau.ac.kr} \\
}


\begin{abstract}

% Recent works to jointly reconstruct 3D human and object from a single RGB image, are mostly model-based, that fail to capture the fine details of the clothed human body and object surface. In this paper, we introduce ReCHOR, a novel, model-free, first-method to produce realistic clothed human-object reconstructions from a monocular view. This is extremely challenging due to human-object occlusions, diverse interactions and depth ambiguity, as it needs to infer both 3D spatial awareness and high resolution details. Our core idea is based on estimating neural implicit representations for human and object respectively by an attention-based neural implicit model that attends to pixel-aligned features from both the global human-object image for spatial awareness and  the local separate view of human and object images for high quality details. Additionally, the network is conditioned on semantic features from an initial estimated human-object pose prior and a generative diffusion model that inpaints occluded regions, thus enabling the retrieval of details from them.
% We also propose a synthetic dataset with rendered scenes of diverse, inter-occluded 3D human and object scans, to train our network. We evaluate our method on the synthetic and real world BEHAVE dataset. Our experiments show that our method outperforms the SOTA in achieving realistic clothed human-object reconstructions.
Recent approaches to jointly reconstruct 3D humans and objects from a single RGB image represent 3D shapes with template-based or coarse models, which fail to capture details of loose clothing on human bodies. In this paper, we introduce a novel implicit approach for jointly reconstructing realistic 3D clothed humans and objects from a monocular view. For the first time, we model both the human and the object with an implicit representation, allowing to capture more realistic details such as clothing. This task is extremely challenging due to human-object occlusions and the lack of 3D information in 2D images, often leading to poor detail reconstruction and depth ambiguity. To address these problems, we propose a novel attention-based neural implicit model that leverages image pixel alignment from both the input human-object image for a global understanding of the human-object scene and from local separate views of the human and object images to improve realism with, for example, clothing details. Additionally, the network is conditioned on semantic features derived from an estimated human-object pose prior, which provides 3D spatial information about the shared space of humans and objects. To handle human occlusion caused by objects, we use a generative diffusion model that inpaints the occluded regions, recovering otherwise lost details. For training and evaluation, we introduce a synthetic dataset featuring rendered scenes of inter-occluded 3D human scans and diverse objects. Extensive evaluation on both synthetic and real-world datasets demonstrates the superior quality of the proposed human-object reconstructions over competitive methods.
\end{abstract}
\section{Introduction}\label{sec:intro}

In computational finance, Monte Carlo simulations are used extensively to estimate the expected value of financial payoffs based on the solution of stochastic differential equations (SDEs) which model the evolution of stock prices, interest rates, exchange rates and other quantities \cite{glasserman04}.  Monte Carlo methods are very general and flexible, but for high accuracy it requires generating a large number of costly SDE path approximations, which has motivated research into a number of variance reduction or, equivalently, cost reduction techniques. One such method is
Multilevel Monte Carlo (MLMC), which was proposed in \cite{GILES2008} and was adapted for various applications that are summarised in \cite{Giles_overview17} and successfully combined with other methods such as quasi-Monte Carlo methods. The main idea of MLMC is to approximate the payoff using different time stepping resolutions when numerically solving the underlying SDE and to generate an optimal number of samples on each level, such that the overall computational cost is minimised subject to the desired bound on the variance. %, such that the total computational cost is minimised. 
The computational savings come from the fact that most samples are computed on the coarser levels and hence are less expensive while only a few samples from the finest levels are required \cite{GILES2008}.


Among the directions in which the computational cost 
of MLMC methods could further be reduced, an important avenue is the use of lower precision calculations, especially for the first Monte Carlo levels where the targeted accuracy is relatively low. 
 An overview of the research on mixed precision for the standard Monte Carlo (MC) framework is provided in \cite{ChowMixedPrecisionStandardMC} but only a few references study the potential of low precision computation in the MLMC framework \cite{Rounding_error_oliver}. To the best of our knowledge, the only MLMC framework with customised precision in the literature is \cite{brugger2014mixed}, but they use a uniform precision for all operations on each Monte Carlo level instead of optimising 
 the precision of each intermediary variable to reduce as much as possible the cost of path generation.
 
An important motivation for an MLMC framework with variable precision would be performing the low precision computations on reconfigurable hardware devices such as Field Programmable Gate Arrays (FPGAs). FPGAs contain customizable logic blocks and connectors that make it easy to adapt the digital circuit architecture for a specific application, leading to a highly parallel and optimised implementation. Therefore they are successfully exploited in applications that require high speed and have high computational workload, such as signal processing \cite{woods2008fpga}, and real time applications like high frequency trading \cite{HFT1,HFT2}. That is why a number of previous works in hardware architecture design implemented the MLMC algorithm to price financial options using FPGAs as accelerators, which resulted in improved speed and power efficiency compared to full CPU architectures \cite{Schryver2013AMM}. The paper \cite{lindsey2016domain} also proposed 
a Domain Specific Language to automate the configuration of FPGAs for this specific application. However, only \cite{brugger2014mixed} proposed a heuristic to reduce the precision in calculations.

In addition, all aforementioned works considered that the random number generation (RNG) is performed in single or double precision. Yet in most cases an important portion of the workload in the overall MLMC simulation comes from the RNG and in \cite{brugger2014mixed} this limited the total computational savings.
To reduce the cost of MLMC simulations in particular those based on the Geometric Brownian Motion (GBM), \cite{approximateICDF_Oliver, NestedOliver} have proposed to use approximate random numbers that are generated by applying an approximation of the inverse CDF to uniform random numbers. In \cite{NestedOliver}, the authors proposed a way to integrate these lower precision random variables into a \textit{nested} MLMC framework and completed a numerical analysis to bound the resulting error at each MC level by a product of the time step and the error in the random number approximation. The same authors show in \cite{approximateICDF_Oliver} that using approximate random variables reduces the cost of path generation by a factor 7.


In this paper we propose a nested MLMC framework that combines the use of approximate random normal variables and lower precision calculations to reduce the computational cost of MLMC even further than \cite{brugger2014mixed,NestedOliver}. We illustrate the efficiency of our framework in Matlab, after making several assumptions on the cost of operations and size of the errors that we carefully justify. We focus on the case of GBM and use the approximate RNG methods presented in \cite{approximateICDF_Oliver} as well as a new slightly modified method that combines CDF inversion and the central limit theorem. To choose the precision of the variables in the low precision path generation, we introduce a novel method to optimise the bit-widths. This optimisation is performed before the main path generation loop is executed and is based on a linear model of the payoff error  
due to rounding when computing in low precision. The error model relies on algorithmic differentiation in a similar manner to \cite{unifying-bwoptim,bitwidth-AD,ADAPT}. The bit-width optimisation procedure can be performed off-line, so this stage can be excluded from the on-line time complexity of our framework. The user specified desired accuracy is then enforced by calculating on-line the number of samples that need to be generated.

In terms of hardware design, we suggest implementing the low precision path generation on FPGAs and the full-precision ones on a CPU or GPU. 
The FPGA offers enough flexibility to define a separate bit-width for every variable in the low precision path generation, and can be reconfigured periodically to update the bit-widths when the market parameters have changed considerably. 


The paper is organized as follows : \Cref{sec:MLMC} introduces MLMC and nested MLMC to make clear the estimator that is implemented in our framework. Then in \Cref{sec:RNG} we detail the methods that could be used to obtain approximate random normally distributed numbers very cheaply for the low precision path generation. In \Cref{sec:error_model} and \Cref{sec:costModel} we propose an error model and a cost model (resp.) that we then use to formulate the optimisation problem that is solved to obtain the optimal bit-widths of fixed point variables in \Cref{sec:optimisation}. Finally we summarise our results and future directions in \Cref{sec:conclusion}.



\section{Related Work}
\label{sec:related_work}

The original investigation \cite{gibson1979ecological} on the relationship between visual perception and human action defines \emph{affordance} as the opportunities for interaction with the surrounding environment. Behavioral studies on regular and cognitively impaired persons have shown evidence that perception results in both visual and motor signals in the human brain. An extended study \cite{anderson2002attentional} shows that visual attention to the spatial characteristics of the perceived objects initiates automatic motor signals for different actions. In computer vision, human affordance learning involves novel pose prediction such that the estimated pose represents a valid human action within the scene context. The task is fundamental to many problems requiring robust semantic reasoning about the environment, such as human motion synthesis \cite{wang2021scene} and scene-aware human pose generation \cite{wang2017binge, roy2016multi, zhang2022inpaint, yao2023scene}.

Earlier methods of affordance learning have explored knowledge mining \cite{zhu2014reasoning} and multimodal feature cues \cite{roy2016multi} to address the problem. In \cite{zhu2014reasoning}, the authors use a Markov Logic Network for constructing a knowledge base by extracting several object attributes from different image and metadata sources, which can perform various downstream visual inference tasks without any additional classifier, including zero-shot affordance prediction. In \cite{roy2016multi}, the authors use depth map, surface normals, and segmentation map as multimodal cues to train a multi-scale convolutional neural network (CNN) for scene-level semantic label assignment associated with specific human actions. In \cite{do2018affordancenet}, the authors design a multi-branch end-to-end CNN with two separate pathways for object detection and affordance label assignment to achieve high real-time inference throughput. Researchers \cite{chuang2018learning} have also explored socially imposed constraints for affordance learning. In \cite{chuang2018learning}, the authors propose a graph neural network (GNN) to propagate contextual scene information from egocentric views for action-object affordance reasoning.

Probabilistic modeling of scene-aware human motion generation also involves semantic reasoning of human interaction with the environment. Initial works on human motion synthesis have taken different architectural approaches, such as sequence-to-sequence models \cite{barsoum2018hp}, generative adversarial networks (GAN) \cite{barsoum2018hp, cai2018deep, yang2018pose}, graph convolutional networks (GCN) \cite{yan2019convolutional}, and variational autoencoders (VAE) \cite{guo2020action2motion}. However, these methods have mostly ignored the role of environmental semantics. Due to potential uncertainty in human motion, in a recent approach \cite{wang2021scene}, the authors address such motion synthesis with a GAN conditioned on scene attributes and motion trajectory to predict probable body pose dynamics.

One key challenge of human affordance generation in 2D scenes is the lack of large-scale datasets with rich pose annotations. In \cite{wang2017binge}, the authors compile the only public dataset of annotated human body poses in complex 2D indoor scenes by extracting frames from sitcom videos. Aiming to generate a contextually valid human affordance at a user-defined location, the authors propose sampling the scale and deformation parameters for an existing human pose template using a VAE conditioned on the localized image patches as scene context. In \cite{zhang2022inpaint}, the authors introduce a two-stage GAN architecture for achieving a similar goal by estimating the affine bounding box parameters to localize a probable human in the scene and then generating a potential body pose at that location. The method uses the input scene, corresponding depth, and segmentation maps as semantic guidance. In \cite{yao2023scene}, the authors propose a transformer-based approach with knowledge distillation for generating human affordances in 2D indoor scenes.




\section{Methodology}
\paragraph{Preliminaries.}
We primarily focus on the homologous model merging, in which $\boldsymbol{\theta}_i$ all come from the same base model $\boldsymbol{\theta}_{\rm{base}}$. Given $K$ tasks $\{T_1,T_2,\cdots,T_K\}$ and $K$ corresponding fine-tuned models with parameters $\{\boldsymbol{\theta}_1,\boldsymbol{\theta}_2,\cdots,\boldsymbol{\theta}_K\}$, model merging aims to combine $K$ fine-tuned models into one single model simultaneously performing on $\{T_1,T_2,\cdots,T_K\}$ without post-training~\cite{method_p1_1,method_p1_2}.
Task vector~\cite{ilharco2023editing,yang2024adamerging} is a key element in merging method which could enhances the base model‘s ability or enable the model to handle other tasks. Specifically, for task $T_i$, the task vector $\boldsymbol\tau_i\in \mathbb{R}^D$ is defined as the vector obtained by subtracting the SFT weights $\boldsymbol{\theta}_i$ from the base model weight
$\boldsymbol{\theta}_{\rm{base}}$, \emph{i.e.}, $\boldsymbol\tau_i=\boldsymbol{\theta}_i-\boldsymbol{\theta}_{\rm{base}}$. The merged model could be denoted as $\boldsymbol{\theta}_m=\boldsymbol{\theta}_{\rm{base}}+\sum_i \lambda_i\boldsymbol{\tau}_i$, which $\lambda_i$ is the scaling factor measuring the importance of task vector. For clarification, we also denote the neuron set in $\boldsymbol{\theta}_i$ as $\mathcal{N}_i$, the neuron set in $\boldsymbol{\tau}_i$ as $\mathcal{T}_i$.



\begin{algorithm}[!ht]
    \caption{LED-Merging}
    \label{alg1}
    \begin{algorithmic}[1]
        \REQUIRE  base model $\boldsymbol{\theta}_{\rm{base}}$, SFT models $\{\boldsymbol{\theta}_{i}\mid i\in [K]\}$, mask ratios \{$r_{i} \mid i\in [K]\}$, scaling factors $\{\lambda_i\mid i\in[K]\}$, location datasets $\{\mathcal{X}_{i}\mid i\in[K]\}$
        \ENSURE merged parameter $\boldsymbol{\theta}_{m}$
        \STATE $\mathcal{M}\leftarrow\phi$
        \STATE $\boldsymbol{\theta}_{m}\leftarrow \boldsymbol{\theta}_{\rm{base}}$
        \FOR{$i\in [K]$}
        \STATE $I(\boldsymbol{\theta}_i)=\mathbb{E}_{x\sim \mathcal{X}_i}|\boldsymbol{\theta}_{i}\odot \nabla_{\boldsymbol{\theta}_i}\mathcal{L}(x)|$
        \STATE $I(\boldsymbol{\theta}_{\rm{base}})=\mathbb{E}_{x\sim \mathcal{X}_i}|\boldsymbol{\theta}_{\rm{base}}\odot \nabla_{\boldsymbol{\theta}_{\rm{base}}}\mathcal{L}(x)|$
        
        \STATE calculate $\mathcal{T}^{r_i}_{i}$ following Equation \ref{vote}
        \STATE  $\mathcal{M}\leftarrow \mathcal{M}\cup\{\mathcal{T}^{r_i}_i\}$
       
        
   
        
        
        \ENDFOR  
        \FOR{$i\in [K]$}
        
        \STATE calculate $\text{Disjoint}(\mathcal{T}_i^{r_i})$ use Equation~\ref{disjoint_safety}
        \STATE $\boldsymbol{m}_i \leftarrow \boldsymbol{0}$
        \FOR{$d\in \mathcal{T}_i^{r_i}$}
        \STATE $\boldsymbol{m}_{i,d}=1$
        \ENDFOR
        \STATE $\boldsymbol{\theta}_{m}\leftarrow \boldsymbol{\theta}_{m}+\lambda_i \boldsymbol{\tau}_i\odot \boldsymbol{m}_{i}$
        \ENDFOR
    \end{algorithmic}
\end{algorithm}
    %\vspace{-5pt}
\begin{figure*}[h!]
    \centering
    \includegraphics[width=\linewidth]{figs/pipeline_v2.pdf}
    \vspace{-40mm}
    \caption{Overview of our two-stage training pipeline {\ours}.}
    \label{fig:pipeline}
\end{figure*}


\paragraph{LED-Merging: Location, Election, and Disjoint Merging}
To address the neuron misidentification and interference issues in existing model merging methods, we propose LED-Merging (Location, Election, and Disjoint Merging). Specifically, previous studies \cite{modelstock, ilharco2023editing, tiesmerging} fail to accurately identify safety-related neurons in task vectors with a single magnitude score, namely \textit{neuron misidentification}. Meanwhile, there exists an interference between safety-related and utility-related task vector neurons during the merging process, namely \textit{neuron interference}. To address neuron misidentification, we first locate important neurons both in the base and fine-tuned models and then elect neurons from the task vector considering these two scores together. Subsequently, to mitigate the interference, we introduce a disjoint step, isolating these important neurons so that they influence different base neurons. The whole process is illustrated in Figure~\ref{fig:method}. 




In the location and election step, we consider the importance score from base and fine-tuned models simultaneously to locate task-specific neurons. In this way, it is more accurate than relying on the magnitude score alone because task-specific neurons with high importance score in the fine-tuned model may not necessarily score high in the base model, and vice versa.

{\textbf{Location}}.  We first calculate importance scores for each neuron in a base/fine-tuned model. Given a location dataset $\mathcal{X}_i=\{(x,y)_k\}$, where $x$ is the question and $y$ is the answer, we calculate the importance scores for the weight $\boldsymbol{\theta}_i\in\mathbb{R}^D$ in any  layer as follows~\cite{snip,spareseGPT,sun2024a}:
\begin{equation}
    I(\boldsymbol{\theta}_i)=\mathbb{E}_{x\sim \mathcal{X}_i}[\boldsymbol{\theta}_i\odot \nabla _{\boldsymbol{\theta}_i}\mathcal{L}(x)],
    \label{location}
\end{equation}
which $\mathcal{L}(x)=-\log p(y\mid x)$ is the conditional negative log-likelihood loss. We choose the SNIP score~\cite{snip} because it balances computational efficiency and performance~\cite{cq}. Please refer to Sec.~\ref{sec:ablation} for the comparison between different location methods. After computing importance scores, we choose top-$r_i$ neurons as the important neuron subset $\mathcal{N}_{i}^{r_i}$ from $I(\boldsymbol{\theta}_i)$.
 
 % After computing locating scores, we select the neurons scoring both high in base and fine-tuned models as important neurons in task vectors. Then in the disjoint step,  with preventing  polysemantic neurons  from receiving gradient updates towards different directions,
 % we use set difference to isolate the safety   and utility-related neurons  and construct corresponding masks for merging process,

{\textbf{Election}}. A natural question is how to select important neurons in the task vector $\boldsymbol{\tau}_i$ based on $I(\boldsymbol{\theta}_{\rm{base}})$ and $I(\boldsymbol{\theta}_{i})$. The important neurons in the base model may be different from neurons in the fine-tuned model. Therefore, we introduce the following election strategy to select neurons with high scores in both base and fine-tuned models:
\begin{equation}
    \mathcal{T}_i^{r_i}=\mathcal{N}_i^{r_i}\cap \mathcal{N}_{\rm{base}}^{r_i}.
    \label{vote}
\end{equation}
\emph{Remark}. We compare different choosing methods, including scoring low or high in base or fine-tuned model in Section~\ref{sec:ablation} and find that Equation \ref{vote} achieves the best performance.





{\textbf{Disjoint}}. As important neurons from different task vectors may conflict with each other at the same position, we use the set difference to disjoint the neurons from others to prevent interference:
\begin{equation}
    \text{Disjoint}(\mathcal{T}^{r_i}_{i})=\mathcal{T}^{r_i}_{i}-\mathop{\cup}\limits_{{J}\subsetneqq [K],|J|\geq 2}\mathop{\cap}\limits_{j\in {J}}\mathcal{T}^{r_j}_{j}.
    \label{disjoint_safety}
\end{equation}

Next, we construct a mask $\boldsymbol{m}_i\in\mathbb{R}^D$ to implement disjoint in the merging process. Specifically, this mask $\boldsymbol{m}_i$ is used to select neurons from $\mathcal{T}_i$. The mask ratio is $r_i$, where $r\in(0,1]$. The mask $\boldsymbol{m}_i$ can be derived from:
\begin{equation}
    \boldsymbol{m}_{i,d}=\begin{aligned} &\left\{ \begin{array}{ll} 1, & \text{if } d\in \text{Disjoint}(\mathcal{T}_{i}^{r_i}), \\ 0, & \text{otherwise}. \end{array} \right. \end{aligned}
    \label{mask_safety}
\end{equation}


% \subsection{Merging Models with Masks}
{\textbf{Merging}}. The final
merged task vector $\boldsymbol{\tau}_m$ is as follows:
\begin{equation}
    \boldsymbol{\tau}_m= \sum_i \lambda_i\boldsymbol{\tau}_{i}\odot\boldsymbol{m}_i.
    \label{merged_task_vector}
\end{equation}
We summarize the workflow in Algorithm \ref{alg1}.



\section{Experiments}
\label{sec:experiment}

Experiments are carried out on NVIDIA RTX4090 GPUs using PyTorch 2.2.0 \cite{paszke2019pytorch} and the rotation detection tool kits: MMRotate 1.0.0 \cite{zhou2022mmrotate}. All the experiments follow the same hyper-parameters (learning rate, batch size, optimizer, etc.).

Average precision (AP) is adopted as the primary metric. All the models are configured upon ResNet50 \cite{he2016deep} and trained with AdamW \cite{loshchilov2018decoupled}.
\textbf{1) Learning rate.} Initialized at 5e-5, warm-up for 500 iterations, and divided by ten at each decay step. 
\textbf{2) Epochs.} 72 for HRSC; 12 for the others.
\textbf{3) Augmentation.} Random rotation/flip for HRSC; random flip for the others.
\textbf{4) Image size.} Split into 1,024 $\times$ 1,024 with an overlap of 200 for DOTA/FAIR1M/STAR; scaled to 800 $\times$ 800 for others.
\textbf{5) Multi-scale.} All experiments evaluated without multi-scale technique \cite{zhou2022mmrotate}. 
\textbf{6) Datasets.} Six remote sensing and one retail scene datasets, covering all datasets used by the main counterparts \cite{yu2024point2rbox, luo2024pointobb, cao2023p2rbox}:

\begin{table*}[!tb]
\fontsize{8.5pt}{10pt}\selectfont
\setlength{\tabcolsep}{0.65mm}
\setlength{\aboverulesep}{0.4ex}
\setlength{\belowrulesep}{0.4ex}
\setlength{\abovecaptionskip}{1.5mm}
\centering
\begin{tabular}{l|c|c|c|c|c|c|c|c|c|c}
\toprule
{\textbf{Methods}} & {*} & {\textbf{\,DOTA-v1.0\,}} & {\textbf{\,DOTA-v1.5\,}} & {\textbf{\,DOTA-v2.0\,}} & {\textbf{~~DIOR~~}} & {\textbf{~~HRSC~~}} & {\textbf{\,FAIR1M\,}} & {\textbf{~~STAR~~}} & {\textbf{\,SKU110K\,}} & {\textbf{~~RSAR~~}} \\
\hline
\rowcolor{gray!20} \multicolumn{11}{l}{$\blacktriangledown$ \textit{RBox-supervised OOD}} \\ \hline
RetinaNet (2017) \cite{lin2017focal} & \checkmark & 68.69 & 60.57        & 47.00 & 54.96 & 84.49   & 37.67   & 21.80 & 78.50 & 57.67  \\
GWD (2021) \cite{yang2021rethinking} & \checkmark & 71.66 & 63.27        & 48.87 & 57.60 & 86.67   & 39.11   & 25.30 & 79.16 & 57.80 \\
FCOS (2019) \cite{tian2019fcos} & \checkmark & 72.44 & 64.53        & 51.77    &  59.83  & 88.99  & 41.25   & \textbf{28.10} & 80.09 & \textbf{66.66} \\
S$^2$A-Net (2022) \cite{han2022align} & \checkmark & \textbf{75.81} & \textbf{66.53} & \textbf{52.39} & \textbf{61.41} & \textbf{90.10} & \textbf{42.44}   & 27.30 & \textbf{80.36} & 66.47 \\
\hline
\rowcolor{gray!20} \multicolumn{11}{l}{$\blacktriangledown$ \textit{HBox-supervised OOD}} \\ \hline
Sun et al. (2021) \cite{sun2021oriented} & $\times$ & 38.60 & - & - & - & - & - & - & - & - \\
KCR (2023) \cite{zhu2023knowledge} & \checkmark & - & - & - & - &  79.10  & -  & - & - & -  \\
H2RBox (2023) \cite{yang2023h2rbox} & \checkmark & 70.05 & 61.70        & 48.68    & 57.80 &  7.03  & 35.94  & 17.20 & 57.15 & 49.92    \\
H2RBox-v2 (2023) \cite{yu2023h2rboxv2} & \checkmark & 72.31 & 64.76 & 50.33 & 57.64 & \textbf{89.66} & \textbf{42.27} & \textbf{27.30} & \textbf{70.70} & \textbf{65.16} \\
AFWS (2024) \cite{lu2024afws} & \checkmark & \textbf{72.55} & \textbf{65.92} & \textbf{51.73} & \textbf{59.07} & - & 41.80 & - & - & - \\
\hline
\rowcolor{gray!20} \multicolumn{11}{l}{$\blacktriangledown$ \textit{Point-supervised OOD}} \\ \hline
P2RBox (2024) \cite{cao2023p2rbox}$^\dagger$ & $\times$ & \underline{59.04} & -        & - & - & -   & -  & -  & - & -  \\
PointSAM (2024) \cite{liu2024pointsam}$^\dagger$ & $\times$ & - & - & - & \textbf{46.20} & -   & -  & -  & - & - \\
PointOBB (2024) \cite{luo2024pointobb} & $\times$ & 30.08 & 10.66        & 5.53     &  37.31  & -   & 11.19 & 9.19  & - & 13.80    \\
Point2RBox+SK (2024) \cite{yu2024point2rbox}$^\dagger$ & \checkmark & 40.27 & 30.51        & 23.43    & 27.34 & 79.40   & 20.03 & 7.86  & 3.41 & 27.81    \\
PointOBB-v2 (2025) \cite{ren2024pointobbv2} & $\times$ & 41.68 & 30.59        & 20.64    &  39.56  & -   & 13.36 & 9.00  & 56.63 & 18.99   \\
PointOBB-v3 (2025) \cite{zhang2025pointobbv3} & $\checkmark$ & 41.20 & 31.25 & 22.82 & 37.60 & - & 11.42  & 11.31 & - & 15.84 \\
PointOBB-v3 (2025) \cite{zhang2025pointobbv3} & $\times$ & 49.24 & 33.79 & 23.52 & 40.18 & - & 18.35 & \underline{12.85} & - & 22.60 \\
\rowcolor{gray!20} Point2RBox-v2 (ours) & \checkmark & 51.00 & \underline{39.45} & \underline{27.11} & 34.70 & \underline{82.67} & \underline{25.72} & 7.80 & \underline{64.00} & \underline{28.60}
 \\
\rowcolor{gray!20} Point2RBox-v2 (ours) & $\times$ & \textbf{62.61} & \textbf{54.06}        & \textbf{38.79}   & \underline{44.45}  & \textbf{86.15}   & \textbf{34.71}  & \textbf{14.20} & \textbf{65.64} & \textbf{30.90}    \\
\bottomrule
\specialrule{0pt}{2pt}{0pt}
\multicolumn{11}{l}{$^*$Comparison tracks: \checkmark = End-to-end training and testing; $\times$ = Generating pseudo labels to train the FCOS detector (two-stage training).} \\
\multicolumn{11}{l}{$^\dagger$Using additional priors. P2RBox/PointSAM: Pre-trained SAM model; Point2RBox+SK: One-shot sketches for each class.} \\
\bottomrule
\end{tabular}
\caption{Accuracy (AP$_{50}$) comparisons on the DOTA-v1.0/1.5/2.0, DIOR, HRSC, FAIR1M, STAR, SKU110K, and RSAR datasets.}
\label{tab:exp_other}
\vspace{-4pt}
\end{table*}

\begin{itemize}
    \item \textbf{DOTA \cite{xia2018dota}.} DOTA-v1.0 has 2,806 aerial images annotated with 15 categories, while DOTA-v1.5/2.0 are the extended versions with more small objects and categories.
    
    \item \textbf{DIOR \cite{cheng2022anchor}.} It is an aerial image dataset re-annotated with RBoxes based on its original HBox version \cite{li2020object}, with a high variation in object size and high intra‐class diversity. 

    \item \textbf{HRSC \cite{liu2017hrsc}.} It contains ship instances on the sea and inshore. The train/val/test set includes 436/181/444 images.

    \item \textbf{FAIR1M \cite{sun2022fair1m}.} It has more than 1 million instances and more than 40,000 images for fine-grained object recognition in remote sensing imagery, annotated with 37 categories. The results are evaluated on FAIR1M-1.0.

    \item \textbf{STAR \cite{li2024star}.} It is extensive for scene graph generation, covering more than 210,000 objects with diverse spatial resolutions, classified into 48 fine-grained categories and precisely annotated with oriented bounding boxes. 

    \item \textbf{SKU110K \cite{pan2020dynamic}.} It focuses on the detection of densely packed retail scenes with 110,712 objects in 11,762 images. The density reaches 86 instances per image. 

    \item \textbf{RSAR \cite{zhang2025rsar}.} It is a remote sensing dataset based on Synthetic Aperture Radar (SAR) imagery with 6 categories.

\end{itemize}

\begin{table*}[!tb]
\fontsize{8.5pt}{10pt}\selectfont
\setlength{\tabcolsep}{2.08mm}
\setlength{\aboverulesep}{0.4ex}
\setlength{\belowrulesep}{0.4ex}
\setlength{\abovecaptionskip}{1.5mm}
\hspace{1pt}
\begin{minipage}[t]{0.315\linewidth}
\centering
\begin{tabular}{c|cc|cc}
\toprule
\multirow{2}{*}{$w_\text{O}$} & \multicolumn{2}{c|}{\textbf{DOTA}} & \multicolumn{2}{c}{\textbf{HRSC}} \\
                  & {E2E} & {FCOS} & {E2E} & {FCOS} \\ \midrule
3  & 48.76 & 61.62 & 81.85 & 84.36 \\
5  & 49.81 & 62.44 & 82.46 & 85.76 \\
\rowcolor{gray!20} 10 & \textbf{51.00} & \textbf{62.61} & \textbf{82.67} & \textbf{86.15} \\
30 & 45.88 & 57.83 & 81.56 & 85.61 \\
\bottomrule
\end{tabular}
\caption{Ablation with the weight of $\mathcal{L}_\text{O}$.}
\label{tab:abl_lo}
\end{minipage}
\quad
\begin{minipage}[t]{0.315\linewidth}
\centering
\begin{tabular}{c|cc|cc}
\toprule
\multirow{2}{*}{$w_\text{W}$} & \multicolumn{2}{c|}{\textbf{DOTA}} & \multicolumn{2}{c}{\textbf{HRSC}} \\
                  & {E2E} & {FCOS} & {E2E} & {FCOS} \\ \midrule
3  & 50.85 & 56.78 & 78.42 & 83.49 \\
\rowcolor{gray!20} 5  & \textbf{51.00} & \textbf{62.61} & \textbf{82.67} & \textbf{86.15} \\
10 & 49.15 & 60.54 & 30.37 & 35.13 \\
30 & 42.84 & 52.53 & 23.89 & 25.91 \\
\bottomrule
\end{tabular}
\caption{Ablation with the weight of $\mathcal{L}_\text{W}$.}
\label{tab:abl_lw}
\end{minipage}
\quad
\begin{minipage}[t]{0.315\linewidth}
\setlength{\tabcolsep}{2.04mm}
\centering
\begin{tabular}{c|cc|cc}
\toprule
\multirow{2}{*}{$w_\text{E}$} & \multicolumn{2}{c|}{\textbf{DOTA}} & \multicolumn{2}{c}{\textbf{HRSC}} \\
                  & {E2E} & {FCOS} & {E2E} & {FCOS} \\ \midrule
0.1 & 48.75 & 57.62 & 34.71 & 39.45 \\
\rowcolor{gray!20} 0.3 & 51.00 & 62.61 & \textbf{82.67} & \textbf{86.15} \\
0.5 & \textbf{51.36} & \textbf{62.63} & 76.85 & 85.22 \\
1.0 & 49.05 & 60.63 & 56.59 & 59.59 \\
\bottomrule
\end{tabular}
\caption{Ablation with the weight of $\mathcal{L}_\text{E}$.}
\label{tab:abl_le}
\end{minipage}
\vspace{-4pt}
\end{table*}

\begin{table*}[!tb]
\fontsize{8.5pt}{10pt}\selectfont
\setlength{\tabcolsep}{2.04mm}
\setlength{\aboverulesep}{0.4ex}
\setlength{\belowrulesep}{0.4ex}
\setlength{\abovecaptionskip}{1.5mm}
\hspace{1pt}
\begin{minipage}[t]{0.315\linewidth}
\centering
\begin{tabular}{c|cc|cc}
\toprule
\multirow{2}{*}{$w_\text{ss}$} & \multicolumn{2}{c|}{\textbf{DOTA}} & \multicolumn{2}{c}{\textbf{HRSC}} \\
                  & {E2E} & {FCOS} & {E2E} & {FCOS} \\ \midrule
0.1 & 49.28 & 59.66 & 73.66 & 78.92 \\
\rowcolor{gray!20} 1.0 & \textbf{51.00} & \textbf{62.61} & \textbf{82.67} & \textbf{86.15} \\
3.0 & 49.15 & 59.20 & 1.30  & 1.65 \\
\bottomrule
\end{tabular}
\caption{Ablation with the weight of $\mathcal{L}_\text{ss}$.}
\label{tab:abl_lss}
\end{minipage}
\quad
\begin{minipage}[t]{0.647\linewidth}
\setlength{\tabcolsep}{3.05mm}
\centering
\begin{tabular}{c|c|c||c|c|c}
\toprule
{R / F / S} & {\textbf{DOTA}} & {\textbf{HRSC}} & {R / F / S} & {\textbf{DOTA}} & {\textbf{HRSC}} \\
 \midrule
90\% / 10\% / 0\% & 60.42 & 85.46 & 80\% / 20\% / 0\%  & 59.46 & 84.73 \\
75\% / 0\% / 25\% & 60.79 & 86.22 & 60\% / 15\% / 25\% & 62.38 & 84.21 \\
\cellcolor{gray!20}68\% / 7\% / 25\% & \cellcolor{gray!20}\textbf{62.61} & \cellcolor{gray!20}\textbf{86.15} & 38\% / 37\% / 25\% & 45.87 & 8.56  \\
45\% / 5\% / 50\% & 60.55 & 85.34 & 40\% / 10\% / 50\% & 60.49 & 10.74 \\
\bottomrule
\end{tabular}
\caption{Ablation with the proportion of augmented views in self-supervision.}
\label{tab:abl_pro}
\end{minipage}
\vspace{-10pt}
\end{table*}

\subsection{Main Results on DOTA-v1.0}
\label{sec:experiment-main}

Table \ref{tab:exp_dota} compares Point2RBox-v2 with the state-of-the-art methods, which can be categorized into two tracks: 

\textbf{1) End-to-end training.} These methods apply the trained weakly-supervised detector directly to the test set. Without relying on priors, our approach demonstrates an improvement of 16.93\% (51.00\% vs. 34.07\%) compared to Point2RBox. Even when compared to Point2RBox+SK, which incorporates additional data-side priors (i.e. one-shot examples for each class), our method still outperforms it by 10.73\% (51.00\% vs. 40.27\%).

\textbf{2) Two-stage training.} These methods generate RBox labels on train/val sets, with which the FCOS detector is trained. In this two-stage mode, Point2RBox-v2 achieves an accuracy of 62.61\%, considerably surpassing PointOBB series. Remarkably, it even outperforms the SAM-powered method P2RBox by 3.57\% (62.61\% vs. 59.04\%).

\textbf{Class-wise analysis.} The FCOS detector trained with labels generated by Point2RBox-v2 achieves accuracy nearly equivalent to RBox-supervised FCOS across six high-density categories: SH (86.9\% vs. 87.1\%), SV (79.6\% vs. 79.8\%), LV (76.3\% vs. 79.8\%), PL (88.0\% vs. 89.1\%), ST (82.9\% vs. 84.6\%), and TC (89.1\% vs. 90.4\%). Interestingly, these six high-density categories account for 88\% of DOTA instances. By annotating these categories with points and generating RBoxes using Point2RBox-v2 while labeling the other sparse categories with RBoxes, we can significantly reduce annotation labor without sacrificing much accuracy, highlighting the valuable role our method can play.

\begin{figure*}[t!]
\setlength{\abovecaptionskip}{1.2mm}
\centering
\includegraphics[width=0.96\linewidth]{figs/case.pdf}
\caption{Qualitative analysis on failed cases and overlap cases.}
\label{fig:case}
\vspace{-6pt}
\end{figure*}

\subsection{Results on More Datasets}

The results are displayed in Table \ref{tab:exp_other}.
On more challenging DOTA-v1.5/2.0, Point2RBox-v2 presents a similar trend, 23.47\%/18.15\% higher than PointOBB-v2 in the pseudo-generation track. 
On the ship detection dataset HRSC, the gap between Point2RBox-v2 and RBox-supervised FCOS is only 2.84\% (86.15\% vs. 88.99\%).
DIOR is relatively sparse, leading to less improvement with our methods---lower than PointSAM (44.45\% vs. 46.20\%) but still higher than methods that do not use SAM. 
Our method also provides competitive performance on fine-grained datasets FAIR1M and STAR. 
In addition to remote sensing scenarios, we carry out experiments on SKU110K for densely packed retail scenes. Existing point-supervised methods struggle in this case, whereas Point2RBox-v2 achieves performance on par with HBox-supervised H2RBox (65.64\% vs. 57.15\%).

\begin{table}[!tb]
\fontsize{8.5pt}{10pt}\selectfont
\setlength{\tabcolsep}{1.78mm}
\setlength{\aboverulesep}{0.4ex}
\setlength{\belowrulesep}{0.4ex}
\setlength{\abovecaptionskip}{1.5mm}
\centering
\begin{tabular}{ccccc|cc|cc}
\toprule
\multicolumn{5}{c|}{\textbf{Modules}} & \multicolumn{2}{c|}{\textbf{DOTA}} & \multicolumn{2}{c}{\textbf{HRSC}} \\
$\mathcal{L}_\text{O}$ & $\mathcal{L}_\text{W}$ & $\mathcal{L}_\text{ss}$ & $\mathcal{L}_\text{E}$ & \textit{CP} & {E2E} & {FCOS} & {E2E} & {FCOS} \\ \midrule
\checkmark & & & & & 0.00 & 0.00 & 0.00 & 0.00 \\
\checkmark & \checkmark & & & & 41.54 & 52.98 & 17.96 & 19.64 \\
\checkmark & \checkmark & \checkmark & & & 46.64 & 54.26 & 18.10 & 22.13 \\
\checkmark & \checkmark & \checkmark & \checkmark & & 49.55 & 61.88 & 78.79 & 83.79 \\
& \checkmark & \checkmark & \checkmark & \checkmark & 48.58 & 59.56 & 20.35 & 24.76 \\
\checkmark & & \checkmark & \checkmark & \checkmark & 38.94 & 48.44 & 11.64 & 14.93 \\
\checkmark & \checkmark & \checkmark & & \checkmark & 47.08 & 55.05 & 19.58 & 21.78 \\
\rowcolor{gray!20} \checkmark & \checkmark & \checkmark & \checkmark & \checkmark & \textbf{51.00} & \textbf{62.61} & \textbf{82.67} & \textbf{86.15} \\
\bottomrule
\end{tabular}
\caption{Ablation with incremental addition of modules.}
\label{tab:abl_mod}
\vspace{-4pt}
\end{table}

\begin{table}[!tb]
\fontsize{8.5pt}{10pt}\selectfont
\setlength{\tabcolsep}{2.85mm}
\setlength{\aboverulesep}{0.4ex}
\setlength{\belowrulesep}{0.4ex}
\setlength{\abovecaptionskip}{1.5mm}
\centering
\begin{tabular}{c|c|c||c|c|c}
\toprule
16 & \cellcolor{gray!20}$K\!=\!24$ & 32 & 1.2 & \cellcolor{gray!20}$\beta\!=\!1.6$ & 2.0 \\ \midrule
50.87 & \cellcolor{gray!20}\textbf{51.00} & 48.08 & 48.14 & \cellcolor{gray!20}51.00 & \textbf{51.33} \\
\bottomrule
\end{tabular}
\caption{Ablation with $K$ and $\beta$ in edge loss on DOTA (E2E).}
\label{tab:abl_edgeparam}
\vspace{-4pt}
\end{table}

\begin{table}[!tb]
\fontsize{8.5pt}{10pt}\selectfont
\setlength{\tabcolsep}{1.75mm}
\setlength{\aboverulesep}{0.4ex}
\setlength{\belowrulesep}{0.4ex}
\setlength{\abovecaptionskip}{1.5mm}
\centering
\begin{tabular}{c|cc|cc|cc}
\toprule
\multirow{2}{*}{$\sigma$} & \multicolumn{2}{c|}{Point2RBox} & \multicolumn{2}{c|}{PointOBB-v2} & \multicolumn{2}{c}{Point2RBox-v2} \\
 & {\textbf{DOTA}} & {\textbf{HRSC}} & {\textbf{DOTA}} & {\textbf{HRSC}} & {\textbf{DOTA}} & {\textbf{HRSC}} \\ \midrule
0\%  & 40.27 & 79.40 & 44.85 & - & 62.61 & 86.15 \\
10\% & 39.60 & 78.81 & 42.30 & - & 61.58 & 85.76 \\
30\% & 38.42 & 78.28 & 38.46 & - & 60.31 & 85.71 \\
\bottomrule
\end{tabular}
\caption{Ablation with the inaccuracy in point annotations.}
\label{tab:abl_noise}
\vspace{-10pt}
\end{table}

\subsection{Ablation Studies}
\label{sec:experiment-ablation}

Tables \ref{tab:abl_lo}-\ref{tab:abl_noise} display the ablation studies on DOTA-v1.0 and HRSC. ``E2E'' denotes end-to-end training; ``FCOS'' denotes two-stage training (i.e. generating pseudo labels to train FCOS). The final values adopted are highlighted in gray.

\textbf{Weight of each loss.} Tables \ref{tab:abl_lo}-\ref{tab:abl_le} determine the weights of the proposed losses. Based on these experiments, the weights $(w_\text{O},w_\text{W},w_\text{E},w_\text{ss})$ are set to $(10, 5, 0.3, 1)$.

\textbf{Proportion of augmented views.} Table \ref{tab:abl_pro} studies the proportion between rotation, flip, and scale. The results are reported with two-stage training (FCOS). Based on the results, the proportion is set to 68\%, 7\%, and 25\%.

\textbf{Incremental addition of modules.} Table \ref{tab:abl_mod} demonstrates the constraints from Gaussian and Voronoi achieve an accuracy of 52.98\% on DOTA. Adding consistency loss and edge loss further boosts it to 54.26\% and 61.88\%, respectively, whereas the improvement from copy-paste is 0.73\%. We also demonstrate the impact of omitting each core loss.

\textbf{Edge loss parameters.} We set $K=24$ and $\beta=1.6$ as they are observed to discern the correct edges during code development. Table \ref{tab:abl_edgeparam} provides a more precise ablation.

\textbf{Annotation inaccuracy.} We offset the annotated points by a noise from the uniform distribution $\left[-\sigma H, +\sigma H \right ]$, where $H$ is the height of objects. Table \ref{tab:abl_noise} shows that the AP$_{50}$ of Point2RBox-v2 decreases by less than 3\% when noise is added to point annotations, demonstrating the robustness of the proposed learning mechanisms.

\subsection{More Discussions}
\label{sec:experiment-discussions}

The qualitative analysis on the failed/overlap cases is shown in Fig. \ref{fig:case}. \textbf{1) Failed cases.} Although our method performs well overall, it struggles with certain categories that are sparse and not constrained by other objects. \textbf{2) Overlap cases.} 
Minimizing overlap as a soft constraint during training does not entirely eliminate overlap. Once trained, the model remains robust to some overlap during inference.



\subsection{Ablation study}
To assess the contribution of each key component in our model, we conduct an ablation study by systematically removing four critical modules: \textbf{Reasoning Path}, \textbf{Relevance Score}, \textbf{Schema Linking}, and \textbf{Similar Examples}.  We evaluate the resulting impact on performance using EX shown in Table~\ref{tab:ablation}. Our findings indicate that each component plays a crucial role in the model’s effectiveness. Removing the Reasoning Path leads to a 3.5-point drop in EX, highlighting its importance in guiding the model toward generating accurate SQL queries. The absence of the Relevance Score resulted in a 5.8-point decrease in EX, underscoring its contribution to overall performance. Eliminating Schema Linking causes a 7.5-point drop in EX, which demonstrates its critical role in similar example construction. Overall, each of the four components—Reasoning Path, Relevance Score, Schema Linking, and Similar Examples—is essential for achieving optimal performance in SQL generation.

\subsection{Analysis}


\begin{table}[h]
    \centering
    \small
    \begin{tabular}{lccc}
        \toprule
        Score & cos $\theta$ &\# of Generated EX & \%  Filtered EX \\
        \midrule
        \textbf{$\geq 0$} &0.581& 10340 & 0 \% \\ 
        \textbf{$\geq 2$} &0.625& 10185  & 1.50\% (-155) \\
        \textbf{$\geq 4$} &0.744& 9883 & 4.41\% (-457)  \\
        \textbf{$\geq 6$} &0.762 & 9378 & 9.30\% (-962)  \\
        \textbf{$\geq 8$}&0.765& 8606 & 16.76\% (-1734)\\
        \textbf{$\geq 10$} &0.769& 6795 & 34.28\% (-3545)  \\
        \bottomrule
    \end{tabular}
    %\caption{A summary of the data generation and filtering result, along with an embedding similarity analysis of the filtered examples, categorized by their respective scores.}
\caption{Summary of data generation, filtering results, and embedding similarity analysis by score.}
    \label{tab:number_of_generated}
    % \vspace{-4mm}
\end{table}

% & 0.581          & 0.625            &  0.744         & 0.762          & 0.765    &  \textbf{0.769}  

\begin{figure*}[t]
\centerline{\includegraphics[scale=0.48]{Pictures/corr_bin.pdf}}

\caption{(Left) Correlation between question embedding similarity and average EX, (Right) Average EX across embedding similarity bins}
% \vspace{-4mm}
\label{fig:corr_bin}
\end{figure*}

\paragraph{Number of generated and filtered examples per score, along with an embedding similarity analysis of the filtered examples}
For each test question in the Spider dev set, 10 examples are generated, resulting in a total of 10,340 examples. The quality of these examples is assessed using a relevance score ranging from 0 to 10. As shown in Table~\ref{tab:number_of_generated}, the 65.71\% of examples are assigned a score of 10, while the 0.59\% of examples are received a score of 0. This trend suggests that the LLM tends to assign high relevance to its own generated examples. The similarity is computed using cosine similarity, where higher scores indicate greater semantic alignment between the test questions and the retained examples. As the filtering threshold increases, the embedding similarity also increases, suggesting that higher-relevance examples exhibit stronger semantic consistency with the test questions. However, we also observe that overly strict filtering—selecting only examples with a perfect score of 10—leads to a decline in performance. This drop occurs because an excessively high threshold significantly reduces the number of available examples, limiting the diversity.


\paragraph{Effect of question embedding similarity on Execution Accuracy.}
In Figure~\ref{fig:corr_bin}, the left graph illustrates the correlation between embedding similarity and EX. Each point represents one of the 11 data points obtained by filtering examples based on different threshold scores (0 to 10). The data points follow an upward trend, suggesting that higher similarity tends to result in better EX. The red line indicates the overall correlation, with a coefficient of 0.82, showing a relatively strong positive relationship. Building on this analysis, the right graph provides a more fine-grained view by examining the execution accuracy of individual generated examples based on their embedding similarity with test questions. The x-axis represents the normalized similarity between the test question and the generated question, and the y-axis indicates EX. The results show that EX is lowest in the 0.0-0.1 similarity range, suggesting that examples with very low similarity to test questions tend to be less useful. As similarity increases, EX generally improves, peaking in the 0.7-0.8 range. This suggests that examples with a moderate to high similarity to test questions are more effective in generating executable SQL queries. However, accuracy drops slightly in the 0.8-0.9 range before rising again in the 0.9-1.0 range. This indicates that excessively high similarity can reduce diversity, potentially limiting the model’s generalization ability. 


\begin{figure}[t]
\centerline{\includegraphics[scale=0.36]{Pictures/Diff_threshold_GPT4o.png}}
\caption{Performance of GPT-4o at different relevance score thresholds.}
% \vspace{-5mm}
\label{tab:diff_thres}
\end{figure}


\paragraph{Effect of Relevance Scoring Thresholds on Performance.}

To further evaluate the effectiveness of SAFE-SQL, we conduct a detailed case study using varying thresholds for the relevance scoring mechanism as shown in Figure~\ref{tab:diff_thres}.  The self-generated examples are filtered based on relevance scores, with thresholds ranging from 0 to 10. For each test question, the number of high-scoring examples varied due to the specific content and schema structure (e.g., some test questions had six examples with scores $\geq 8$, while others had three). The selected examples are then used during the final inference stage to generate SQL queries. The $\geq 8$ threshold consistently produced the best results, validating the robustness of SAFE-SQL’s relevance score filtering. The results demonstrate that selecting high-quality examples plays a critical role in guiding LLMs to generate accurate SQL queries, regardless of the underlying model.


\begin{comment}
\begin{table*}[h]
    \centering
    \renewcommand{\arraystretch}{1.3}  % 행 간격 조정
    \begin{tabularx}{\textwidth}{p{4cm} p{6cm} p{4cm} p{6cm}}
        \toprule
        \textbf{Original SQL Question} & \textbf{Original SQL Query} & \textbf{Generated SQL Question} & \textbf{Generated Reasoning Path} \\
        \midrule
        What are all the flights that leave from Aberdeen? & 
        \lstinline|SELECT * FROM flights WHERE departure_city = 'Aberdeen'| & 
        What are all the flights departing from Aberdeen? & 
        Identify all flights with Aberdeen as the departure city. \\
        
        Of those, which land in Ashley? & 
        \lstinline|SELECT * FROM flights WHERE departure_city = 'Aberdeen' AND arrival_city = 'Ashley'| & 
        Which flights leave from Aberdeen and land in Ashley? & 
        Filter previous results to include only flights arriving in Ashley. \\
        
        How many are there? & 
        \lstinline|SELECT COUNT(*) FROM flights WHERE departure_city = 'Aberdeen' AND arrival_city = 'Ashley'| & 
        How many flights travel from Aberdeen to Ashley? & 
        Count the number of flights from the filtered list. \\
        \midrule
        
        What are all the airlines? & 
        \lstinline|SELECT DISTINCT airline FROM flights| & 
        What airlines operate flights? & 
        Retrieve distinct airline names from the flights table. \\
        
        Of these, which is JetBlue Airways? & 
        \lstinline|SELECT * FROM flights WHERE airline = 'JetBlue Airways'| & 
        Which flights are operated by JetBlue Airways? & 
        Filter flights to include only those operated by JetBlue Airways. \\
        
        What is the country corresponding it? & 
        \lstinline|SELECT country FROM airlines WHERE name = 'JetBlue Airways'| & 
        What country is JetBlue Airways based in? & 
        Retrieve the country associated with JetBlue Airways from the airlines table. \\
        \bottomrule
    \end{tabularx}
    \caption{Examples of original and generated SQL questions with reasoning paths.}
    \label{tab:sql_examples}
\end{table*}
\end{comment}

\begin{comment}
\begin{table*}[h]
    \centering
    \small
    \renewcommand{\arraystretch}{1.3}  % Adjust row spacing
    \begin{tabularx}{\textwidth}{X X X X X}
        \toprule
        \textbf{SQL Question} & \textbf{GOLD SQL Query} & \textbf{Augmented SQL Question} & \textbf{Generated Reasoning Path} & \textbf{Relevance Score} \\
        \midrule
        \hl{Question1:}
        What are the names, countries, and ages for every singer in descending order of age? & 
        \texttt{SELECT name, country, age FROM singer ORDER BY age DESC} & 
        \sethlcolor{lime!50}
        \hl{What are the names, ages, and countries of all singers from a specific country, sorted by age in descending order?} & 
        \sethlcolor{violet!20}
        \hl{1.Identify the desired columns: name, age, and country. 
        2.Specify the table: singer. 
        3.Sort the results by age in descending order.}& semantic similarity:3   Structure \& key word 
 score: 3  Reasoning patt score:4 Relevance score = 10
        \\
        \midrule
        \hl{Question2:}
        What is the number of car models that are produced by each maker and what is the id and full name of each maker?
        &  
        \texttt{SELECT Count(*), T2.FullName , T2.id FROM MODEL\_LIST AS T1 JOIN CAR\_MAKERS AS T2 ON T1.Maker = T2.id GROUP BY T2.id;} & 
               \sethlcolor{lime!50}
 \hl{Could you provide the count of car models produced by each manufacturer, along with the ID and full name of each manufacturer?} & 
 \sethlcolor{violet!20}
 \hl{1.Retrieve Required Information: Count car models per maker and get each maker's ID and full name. 2.Join Tables: Link MODEL\_LIST (T1) and CAR\_MAKERS (T2) using T1.Maker = T2.Id. 3.Group and Aggregate: Use COUNT(*) to count models and group by T2.id. 4.Select Output: Return the model count, maker’s full name, and ID.} & semantic similarity:1   Structure \& key word 
 score: 2  Reasoning patt score:3 Relevance score = 6 \\ 
        \midrule
        \hl{Question3:} Return the names and template ids for documents that contain the letter w in their description. & 
        \texttt{SELECT document\_name , template\_id FROM Documents WHERE Document\_Description LIKE "\%w\%"} & 
        \sethlcolor{lime!50}
        \hl{Retrieve the names and template IDs of documents whose descriptions include the letter 'w'.} & 
        \sethlcolor{violet!20}
        \hl{1.Extract the document\_name and template\_id columns.
        2.Search within the Documents table. 3.Filter results where Document\_Description contains the letter "w".
        Return the matching records.} & semantic similarity:3   Structure \& key word 
 score: 2  Reasoning path score:3 Relevance score = 8  \\

        \bottomrule
    \end{tabularx}
    \caption{Examples of original and Augmented SQL questions with reasoning paths by GPT-4o mini.}
    \label{tab:sql_examples}
\end{table*}
\end{comment}

\begin{table*}[t]
    \centering
    \scriptsize
    \renewcommand{\arraystretch}{1.2}
    \resizebox{1.97\columnwidth}{!}{% Adjust row spacing
    \begin{tabularx}{\textwidth}{X X X >{\raggedright\arraybackslash}p{0.3\textwidth} X}
        \toprule
        \textbf{SQL Question} & \textbf{GOLD SQL Query} & \textbf{Augmented SQL Question} & \textbf{Generated Reasoning Path} & \textbf{Relevance Score} \\
        \midrule
        \hl{Question1:} What are the names, countries, and ages for every singer in descending order of age? & 
        \texttt{SELECT name, country, age FROM singer ORDER BY age DESC} & 
        \sethlcolor{lime!50}\hl{What are the names, ages, and countries of all singers from a specific country, sorted by age in descending order?} & 
        \sethlcolor{violet!20}\hl{1. Identify the desired columns: name, age, and country. \\
        2. Specify the table: singer. \\
        3. Sort the results by age in descending order.} &  
        Semantic similarity = 10 
        Structural Similarity = 10  
        Reasoning path = 10  
        Relevance score = (10+10+10)/3 = 10
                \\ 
        \midrule
        \hl{Question2:} Return the names and template ids for documents that contain the letter w in their description. & 
        \texttt{SELECT document\_name, template\_id FROM Documents WHERE Document\_Description LIKE "\%w\%"} & 
        \sethlcolor{lime!50}\hl{Retrieve the titles and category IDs of articles whose summaries contain the word "data".} &
        \sethlcolor{violet!20}\hl{1. Identify the necessary columns: extract title and category\_id from the Articles table. \\
         2. Locate the relevant table: use the Articles table as it contains the required data. \\
        3. Define the filtering condition: apply a WHERE clause to check if the summary column contains the substring "data". \\
         4. Use the LIKE operator: employ LIKE '\%data\%' to search for any instance of "data" within the summary. \\
        5. Retrieve the results: return the title and category\_id values for all matching records.} & Semantic similarity = 7
                    Structural Similarity = 9
Reasoning path = 8
Relevance score = (7+9+8)/3 = 8
        \\
        \midrule
        \hl{Question3:} What is the number of car models that are produced by each maker and what is the id and full name of each maker? &  
        \texttt{SELECT Count(*), T2.FullName, T2.id FROM MODEL\_LIST AS T1 JOIN CAR\_MAKERS AS T2 ON T1.Maker = T2.id GROUP BY T2.id;} & 
        \sethlcolor{lime!50}\hl{List all employees who work in the IT department along with their employee ID and hire date.} & 
        \sethlcolor{violet!20}\hl{1. Identify required details: employee ID and hire date. \\
        2. Filter condition: find employees who work in IT. \\
        3. Retrieve data: select only emp\_id and hire\_date.} & 
        Semantic similarity = 6
        Structural Similarity = 3
        Reasoning path = 2
        Relevance score = (6+3+2)/3 = 3.67
  \\
        \bottomrule
    \end{tabularx}
    }
    %\vspace{-2mm}
    \caption{Examples of original and augmented SQL questions with reasoning paths by GPT-4o.}
    \label{tab:sql_examples}
    \vspace{-4mm}
\end{table*}


%This experiment is performed across multiple models, including GPT 4o Mini, Deepseek Coder 6.7B, %Llama 3.1 8B, and Starcoder 7B.
\begin{table}[t]
    \centering
    \small
    \resizebox{0.48\textwidth}{!}{
    \begin{tabular}{lcc||ccccc}
        \toprule
        \textbf{$\alpha$} & \textbf{$\beta$} &\textbf{$\gamma$}& \textbf{Easy}& \textbf{Medium}& \textbf{Hard} &\textbf{Extra}& \textbf{EX} \\
        \midrule
        0.33 & 0.33 & 0.33 & \textbf{93.4} & \textbf{89.3} & \underline{88.4} & \textbf{75.8} & \textbf{87.9} \\ 
        \midrule
        1 & 0 & 0 & 90.7& 84.2& 82.3& 68.3&  82.8 \\ 
        0 & 1 & 0 & 89.8& 85.6& 81.2& 69.2&  83.1 \\ 
        0 & 0 & 1 & 89.2& 85.1& 84.3& 71.7& 83.7  \\ 
        \midrule
        0.5 & 0.5 & 0& 91.2& 87.3& 82.5& 69.4& 84.4 \\ 
        0.5 & 0 & 0.5& 92.5& \underline{87.9}& 83.5& 70.3& 85.3 \\ 
        0 & 0.5 & 0.5& \underline{92.7}& 86.8& \textbf{88.5}& \underline{72.4}& \underline{86.1} \\ 
        \bottomrule
    \end{tabular}
    }
    %\vspace{-2mm}
    %\caption{Execution accuracy across different difficulty levels with varying weights of semantic similarity ($\alpha$), keyword \& structural similarity ($\beta$), and reasoning path quality ($\gamma$).}
    \caption{Execution accuracy across difficulty levels under different weights: semantic similarity ($\alpha$), Structural similarity ($\beta$), and reasoning path quality ($\gamma$).}
    % \vspace{-4mm}
    \label{tab:filtering_score_ablation}
\end{table} 

\paragraph{Effect of three measuring components on Performance.}

To assess the impact of the three measuring components—semantic similarity ($\alpha$), keyword \& structural similarity ($\beta$), and reasoning path quality ($\gamma$)—on EX, we conduct experiments by varying their respective weightings. The results, presented in Table~\ref{tab:filtering_score_ablation}, highlight distinct performance trends across different difficulty levels. Notably, the exclusion of reasoning path quality leads to a drop in EX, particularly in the Hard and Extra Hard. This suggests that a well-structured reasoning path is crucial for handling complex queries, as it provides essential logical steps that bridge the gap between natural language understanding and SQL formulation. Conversely, semantic similarity and structural SQL query similarity have a greater influence on the Easy and Medium levels. This is because these queries tend to be relatively straightforward, meaning that having structurally similar SQL questions in the example set often provides sufficient guidance for generating correct queries. In these cases, direct pattern matching and schema alignment play a larger role. Overall, the findings demonstrate that a balanced combination of all three components is essential for optimizing performance across different levels of query complexity.

% Simply maximizing similarity may not always yield the best results, and a balanced approach that considers both relevance and diversity could be more effective.


%: Qwen2.5-3B-instruct, Qwen2.5-7B-instruct, and Qwen2.5-14B-instruct 


\subsection{Case Study}
As shown in Table~\ref{tab:sql_examples}, test questions from the Spider dev set alongside their generated similar examples, evaluated based on semantic similarity, structural similarity, and the reasoning path score, which together determine the relevance score. The first example achieves a perfect relevance score of 10, as the generated question closely aligns with the original in meaning, structure, and reasoning. The SQL formulation remains nearly identical, and the reasoning path explicitly details each step, ensuring full alignment. The second example receives a relevance score of 8, with semantic similarity of 7 due to minor differences in terminology ("documents" vs. "articles" and "letter 'w'" vs. "word 'data'"). However, its structural similarity remains high, as the SQL structure is nearly identical. The reasoning path score of 8 reflects a clear explanation of query formulation, though slightly less detailed than the first example. The third example has the lowest relevance score due to significant differences. The generated question shifts focus from counting car models to listing IT employees, resulting in semantic similarity of 6 and structural similarity of 3. These results emphasize the importance of fine-grained example selection due to the varing quality of generated examples.
% Table 6번 언급되는 곳이 하나도 없었습니다. 맨뒤로 빼고 Case Study 만들어서 설명할 필요가 있습니다. 또한 Relevance Score 변경했는데 확인해주셔야합니다
\section{Conclusion and future directions} \label{sec:conclusion}

In this paper we proposed a nested MLMC framework that offers important computational savings by performing most calculations in low precision and exploiting approximate random normal variables for the low precision path calculations. The low precision calculations could be performed in fixed precision on an FPGA for greater efficiency, and we suggested a procedure to optimise the bit-widths of every variable at each Monte Carlo level. This is an important improvement over previous mixed precision MLMC frameworks which held the lower precision fixed \cite{Rounding_error_oliver} or defined uniform bit-width at every level heuristically \cite{brugger2014mixed}. Our numerical results suggest that for the first levels our procedure reduces the cost at these levels by a factor 5 or 7. Hence the overall savings are significant since most paths are calculated on the first levels. Our approach would be even more efficient for the Milstein scheme because its higher order strong convergence leads to a greater proportion of the computational costs being on the coarsest levels.

The next stage of the research project will be to implement the RNG methods and the nested framework on FPGAs to determine the hardware requirements and confirm the extent of the computational savings. It would also be good to compare the performance benefits to using half-precision floating point arithmetic on GPUs or CPUs for the low-accuracy computations.



Our study has several limitations. First, due to our capacity, we mainly focus on three programming languages—Python, Java, and JavaScript—missing the chance to include other languages like C and C\#. Additionally, given the fact that the input length restrictions of current LLMs make them unsuitable for handling larger projects in their entirety, 
%as they may miss some information and fail to generate sufficient or high-quality unit tests for extensive codebases. 
we selected moderate-sized projects, allowing us to explore issues like the robustness of LLMs in unit test generation (e.g., hallucinations or incorrect assertions) rather than focusing solely on their ability to handle long-context inputs. 
% However, this approach may not fully capture the challenges of applying LLMs to larger-scale projects.

\section*{Acknowledgement}
This research was supported by Institute for Information \& Communications Technology Planning \& Evaluation (IITP) through the Korea government (MSIT) under Grant No. 2021-0-01341 (Artificial Intelligence Graduate School Program (Chung-Ang University)).
%\bibliographystyle{plain}
\bibliography{10_ref}
\section{Related Works}
\label{apx:rw}

\textbf{Physics-Informed Neural Networks.}
Physics-Informed Neural Networks~\cite{raissi2019physics} are a class of deep learning models designed to solve problems governed by physical laws described in PDEs. 
    They integrate physics-based constraints directly into the training process in the loss function, allowing them to numerically solve many key physical equations, such as Navier-Stokes equations\cite{jin2021nsfnets}, Euler equations~\cite{mao2020physics}, heat equatuons~\cite{cai2021physics}. Several advanced learning schemes such as gPINN~\cite{kharazmi2019variational}, vPINN\cite{yu2022gradient}, and RoPINN\cite{wu2024ropinn}, model architectures such as QRes~\cite{bu2021quadratic}, FLS~\cite{wong2022learning}, PINNsFormer~\cite{zhao2024pinnsformer}, KAN~\cite{liu2024kan,liu2024kanw} are proposed in terms of convergence, optimization, and generalization.

\textbf{Failure Modes in PINNs.}
Despite these efforts, PINN still has some inherently intractable failure modes. 
\citet{krishnapriyan2021characterizing} identify several types of equations that are vulnerable to difficulties in solving by PINNs.  
    These equations are usually manifested by the presence of a parameter in them that makes their pattern behave as a high frequency or a complex state~\cite{pmlr-v235-cho24b}, failing to propagate the initial condition. 
        In such cases, an empirical loss constructed using a collection point can easily fall into an over-smooth solution (e.g. $\bar u(x,t)=0$ can make the loss of all collection points except whose $t=0$ descend to 0 for 1d-wave equations). Several methods regarding optimization~\cite{wu2024ropinn,wang20222}, sampling~\cite{gao2023failure,wu2023comprehensive}, model architecture~\cite{zhao2024pinnsformer,pmlr-v235-cho24b,pmlr-v235-nguyen24c}, transfer learning~\cite{xu2023transfer,pmlr-v235-cho24b} are proposed to mitigate such failure modes. 
            However, the above approaches do not focus on the fact that a PDE system should be modeled as a continuous dynamic, leading to difficulties in generalization over a wide range of problems.




\textbf{State Space Models.} The state space model~\cite{kalman1960new} is a mathematical representation of a physical system in terms of state variables. 
    Modern SSMs~\cite{gu2022efficiently,smith2023simplified,gu2023mamba} combine the representational power of neural networks with their own superior long-range dependency capturing and parallel computing capabilities and thus are widely used in many fields, such as language modeling~\cite{fu2023hungry,poli2023hyena,gu2023mamba,pmlr-v235-dao24a}, computer vision~\cite{pmlr-v235-zhu24f,liu2024vmamba}, and genomics~\cite{gu2023mamba,nguyen2024sequence}. Specifically, Structured SSMs~(S4)~\cite{gu2022efficiently} decomposing the structured state matrices as the sum of a low-rank
and normal terms to improve the efficiency of state-space-based deep models. Further, Selective SSMs (Mamba)~\cite{gu2023mamba} eliminates the Linear Time Invariance~\cite{sain1969invertibility} of SSMs by introducing a gating mechanism, allowing the model to selectively propagate or forget information and greatly enhancing the model performance. In physics, SSMs are used in conjunction with Neural Operator to form a data-driven solution to PDEs~\cite{zheng2024aliasfree,hu2024state}. 
However, these methods are data-driven which lack generalization ability in some scenarios where real data is not available. Unlike these methods, our approach, PINNMamba is fully physics-driven, relying only on residuals constructed using PDEs without any training data.

%



\section{Proof of Theorem \ref{thm:continuous-discrete}}
\label{apx:proof3_1}

We start with a function $v$ such that $\mathcal{M}(v)$ is non-zero almost everywhere. Such a function exists because $\mathcal{M}$ is a non-zero differential operator. For example, if $\mathcal{M}$ is the Laplacian, a non-harmonic function can be chosen.

\begin{lemma}[Existence of Base Function]
    Let $\mathcal{M}$ be a non-degenerate differential operator on $\Omega \times [0,T]$, where $\Omega \subset \mathbb{R}^n$ is a domain. There exists a function $v \in C^\infty(\Omega \times [0,T])$ such that:  
$$
\mathcal{M}(v) \neq 0 \quad \text{for almost every } (x,t) \in \Omega \times [0,T].
$$
\end{lemma}


\begin{proof}
    Since $\mathcal{M}$ is non-degenerate (i.e., not identically zero), there exists at least one function $w \in C^\infty(\Omega \times [0,T])$ and a point $(x_0, t_0) \in \Omega \times [0,T]$ such that:  
   $$
   \mathcal{M}(w)(x_0, t_0) \neq 0.
   $$  
   By continuity of $\mathcal{M}(w)$ (assuming smooth coefficients for $\mathcal{M}$), there is an open neighborhood $U \subset \Omega \times [0,T]$ around $(x_0, t_0)$ where $\mathcal{M}(w) \neq 0$.
   
   Construct a smooth bump function $\phi \in C^\infty(\Omega \times [0,T])$ with:  
   $\phi \equiv 1$ on a smaller neighborhood $V \subset U$,  
   and $\phi \equiv 0$ outside $U$.  
      Define $v_0 = \phi \cdot w$. Then $\mathcal{M}(v_0) = \mathcal{M}(\phi w)$ is non-zero on $V$ and smooth everywhere.  
   Let $\{(x_k, t_k)\}_{k=1}^\infty$ be a countable dense subset of $\Omega \times [0,T]$. For each $k$, repeat the above construction to obtain a function $v_k \in C^\infty(\Omega \times [0,T])$ such that: $\mathcal{M}(v_k) \neq 0$ in a neighborhood $U_k$ of $(x_k, t_k)$,  
   $\text{supp}(v_k) \subset U_k$,  
   and the supports $\{U_k\}$ are pairwise disjoint. 

   Define the function:  
   $$
   v = \sum_{k=1}^\infty \epsilon_k v_k,
   $$  
   where $\epsilon_k > 0$ are chosen such that the series converges in $C^\infty(\Omega \times [0,T])$ (e.g., $\epsilon_k = 2^{-k}/\max\{\|v_k\|_{C^k}, 1\}$).

   The set $\bigcup_{k=1}^\infty U_k$ is open and dense in $\Omega \times [0,T]$. Since $\mathcal{M}(v) \neq 0$ on this dense open set, the zero set $Z = \{(x,t) : \mathcal{M}(v)(x,t) = 0\}$ is contained in the complement of $\bigcup_{k=1}^\infty U_k$, which is nowhere dense and hence has Lebesgue measure zero. Therefore:  
   $$
   \mathcal{M}(v) \neq 0 \quad \text{for almost every } (x,t) \in \Omega \times [0,T].
   $$
\end{proof}

\begin{lemma}[Local Correction Functions]\label{lem:B2}
    Let $\mathcal{M}$ be a non-degenerate differential operator on $\Omega \times [0,T]$, and let $\chi^* = \{(x^*_1,t^*_1),\dots,(x^*_N,t^*_N)\} \subset \Omega \times [0,T]$. There exist smooth functions $\{w_i\}_{i=1}^N \subset C^\infty(\Omega \times [0,T])$ and radii $\epsilon_1, \dots, \epsilon_N > 0$ such that for each $i$:  
    
1. Compact Support: $\text{supp}(w_i) \subset B_{\epsilon_i}(x^*_i,t^*_i)$,  

2. Non-Vanishing Action: $\mathcal{M}(w_i)(x^*_i,t^*_i) \neq 0$, 

3. Disjoint Supports: $B_{\epsilon_i}(x^*_i,t^*_i) \cap B_{\epsilon_j}(x^*_j,t^*_j) = \emptyset$ for $i \neq j$. 
\end{lemma} 


\begin{proof}
    Let $d_{\text{min}} = \min_{i \neq j} \text{dist}\left((x^*_i,t^*_i), (x^*_j,t^*_j)\right)$ be the minimal distance between distinct points in $\chi^*$. For all $i$, choose radii $\epsilon_i > 0$ such that:  
$$
\epsilon_i < \frac{d_{\text{min}}}{2}.
$$  
This ensures the balls $B_{\epsilon_i}(x^*_i,t^*_i)$ are pairwise disjoint.  

For each $(x^*_i,t^*_i)$, since $\mathcal{M}$ is non-degenerate, there exists a smooth function $f_i \in C^\infty(\Omega \times [0,T])$ such that:  
$$
\mathcal{M}(f_i)(x^*_i,t^*_i) \neq 0.
$$  This is because, when $\mathcal{M}$ is non-degenerate, its action cannot vanish on all smooth functions at $(x^*_i,t^*_i)$. For instance, if $\mathcal{M}$ contains a derivative $\partial_{x_k}$, take $f_i = x_k$ near $(x^*_i,t^*_i)$.

Then for each $i$, construct a smooth bump function $\phi_i \in C^\infty(\Omega \times [0,T])$ satisfying:  

1. $\phi_i \equiv 1$ on $B_{\epsilon_i/2}(x^*_i,t^*_i)$, 

2. $\phi_i \equiv 0$ outside $B_{\epsilon_i}(x^*_i,t^*_i)$, 

3. $0 \leq \phi_i \leq 1$ everywhere.  

Therefore, define the localized function:  
$$
w_i = \phi_i \cdot f_i.
$$  
By construction:  

1. $\text{supp}(w_i) \subset B_{\epsilon_i}(x^*_i,t^*_i)$,

2. $w_i = f_i$ on $B_{\epsilon_i/2}(x^*_i,t^*_i)$, so  
$$
\mathcal{M}(w_i)(x^*_i,t^*_i) = \mathcal{M}(f_i)(x^*_i,t^*_i) \neq 0.
$$  

Since $\epsilon_i < \frac{d_{\text{min}}}{2}$, the distance between any two balls $B_{\epsilon_i}(x^*_i,t^*_i)$ and $B_{\epsilon_j}(x^*_j,t^*_j)$ is at least $d_{\text{min}} - 2\epsilon_i > 0$. Thus, the supports of $w_i$ and $w_j$ are disjoint for $i \neq j$. 

Therefore, the functions $\{w_i\}_{i=1}^N$ satisfy all required conditions.  


\end{proof}



We now state the one-dimensional case of Theorem~\ref{thm:continuous-discrete} here:

\begin{lemma}[One-Dimensional Case of Theorem~\ref{thm:continuous-discrete}]
\label{lemma:1d}
    Let $\chi^* = \{(x^*_1,t^*_1),\dots,(x^*_N,t^*_N)\}\subset \Omega\times[0,T]$. Then for differential operator $\mathcal M$ there exist infinitely many functions
$u_\theta : \Omega \to \mathbb{R}$ parametrized by $\theta$ , s.t.
$$ \mathcal{M}(u_\theta(x^*_i,t^*_i)) = 0 \quad \text{for } i=1,\dots,N,$$ $$ 
   \mathcal{M}(u_\theta(x,t)) \neq 0
   \quad \text{for a.e. } x \in \Omega\times[0,T] \setminus \chi^*.$$
\end{lemma}

\begin{proof}
    Define the corrected function:
$$
u_\theta = v + \sum_{i=1}^N \alpha_i w_i,
$$
where $w_i$ is the local correction function defined in Lemma~\ref{lem:B2}, $\alpha_i \in \mathbb{R}$ are scalars chosen such that:
$$
\mathcal{M}(u_\theta)(x_i^*, t_i^*) = \mathcal{M}(v)(x_i^*, t_i^*) + \alpha_i \mathcal{M}(w_i)(x_i^*, t_i^*) = 0.
$$

Since $\mathcal{M}(w_i)(x_i^*, t_i^*) \neq 0$, we can solve for $\alpha_i$:
$$
\alpha_i = -\frac{\mathcal{M}(v)(x_i^*, t_i^*)}{\mathcal{M}(w_i)(x_i^*, t_i^*)}.
$$

Outside the union of supports $\bigcup_{i=1}^N B_{\epsilon_i}(x_i^*, t_i^*)$, we have:
$$
\mathcal{M}(u_\theta) = \mathcal{M}(v) + \sum_{i=1}^N \alpha_i \mathcal{M}(w_i) = \mathcal{M}(v),
$$
since $w_i \equiv 0$ outside $B_{\epsilon_i}(x_i^*, t_i^*)$. By construction, $\mathcal{M}(v) \neq 0$ almost everywhere. 

The parameters $\theta = (\epsilon_1, \dots, \epsilon_N, \alpha_1, \dots, \alpha_N)$ can be varied infinitely by varying $w_i$: The bump functions $w_i$ can be scaled, translated, or reshaped (e.g., Gaussian vs. polynomial) while retaining the properties of Local Correction in Lemma~\ref{lem:B2} and varying $\epsilon_i$: For each $i$, choose $\epsilon_i$ from a continuum $(0, \delta_i)$, where $\delta_i$ ensures disjointness.

Thus, the family $\{u_\theta\}$ is uncountably infinite.

The set $\chi^*$ by definition has Lebesgue measure zero in $\Omega \times [0,T]$. The corrections $\sum_{i=1}^N \alpha_i w_i$ are confined to the measure-zero set $\bigcup_{i=1}^N B_{\epsilon_i}(x_i^*, t_i^*)$. Therefore:
$$
\mathcal{M}(u_\theta) \neq 0 \quad \text{for a.e. } (x,t) \in \Omega \times [0,T] \setminus \chi^*.
$$
\end{proof}


We now generalize Lemma~\ref{lemma:1d} to $m$-dimension, to get Theorem~\ref{thm:continuous-discrete}.

\begin{theorem}[Theorem~\ref{thm:continuous-discrete}]
    Let $\chi^* = \{(x^*_1,t^*_1),\dots,(x^*_N,t^*_N)\}\subset \Omega\times[0,T]$. Then for differential operator $\mathcal M$ there exist infinitely many functions
$u_\theta : \Omega \to \mathbb{R}^m$ parametrized by $\theta$ , s.t.
$$ \mathcal{M}(u_\theta(x^*_i,t^*_i)) = 0 \quad \text{for } i=1,\dots,N,$$ $$ 
   \mathcal{M}(u_\theta(x,t)) \neq 0
   \quad \text{for a.e. } x \in \Omega\times[0,T] \setminus \chi^*.$$
\end{theorem}

\begin{proof}
    It is trivial to generalize the Lemma~\ref{lemma:1d} to the case $u_\theta : \Omega \to \mathbb{R}^m$, by constructing:
    $$
   u_\theta = v + \sum_{i=1}^N \sum_{j=1}^m \alpha_{i,j} w_{i,j},
   $$
   where $ \alpha = (\alpha_{i,j}) \in \mathbb{R}^{N \cdot m} $. Adjust $ \alpha_{i,j} $ such that:
   $$
   \mathcal{M}(u_\theta)(x_i^*, t_i^*) = \mathcal{M}(v)(x_i^*, t_i^*) + \sum_{j=1}^m \alpha_{i,j} \mathcal{M}(w_{i,j})(x_i^*, t_i^*) = 0.
   $$
   This gives a linear system for $ \alpha $, which is solvable because the $ w_{i,j} $ are linearly independent.
\end{proof}


\section{Linear Time-Varying System}
\label{apx:LTI}

To adjust the given Linear Time-Invariant system to a Linear Time-Varying system, we replace the constant matrices $ \bar{A} $, $ \bar{B} $, and $ C $ with their time-varying counterparts $ \bar{A}(k) $, $ \bar{B}(k) $, and $ C(k) $. The state transition matrix $ \bar{A}^{k-i} $ in the LTI system becomes the product of time-varying matrices from time $ i $ to $ k-1 $. The resulting time-varying output equation is:

\begin{equation}
    \mathbf{u}_k = C(k) \Phi(k, 0) \mathbf{h}_0 + C(k) \sum_{i=0}^k \Phi(k, i) \bar{B}(i) \mathbf{x}_i,
\end{equation}



where $ \Phi(k, i) $ is the state transition matrix from time $ i $ to $ k $, defined as:
\begin{equation}
      \Phi(k, i) = \begin{cases} 
    \bar{A}(k-1) \bar{A}(k-2) \cdots \bar{A}(i) & \text{if } k > i, \\
    I & \text{if } k = i.
  \end{cases}
\end{equation}


  
and the term $ \Phi(k, 0) \mathbf{h}_0 $ represents the free response due to the initial condition $ \mathbf{h}_0 $.

The summation $ \sum_{i=0}^k \Phi(k, i) \bar{B}(i) \mathbf{x}_i $ includes contributions from all inputs $ \mathbf{x}_i $ up to time $ k $, with $ \Phi(k, i) \bar{B}(i) $ capturing the time-varying dynamics.

To adjust the Eq.~\ref{equ:timeloss} to a Time-Varying system The state transition term $ \bar{A}^{k-i} $ becomes the time-ordered product $ \Phi(k, i) $, and the output $ \mathbf{u}_k $ now explicitly depends on time-varying dynamics. The adjusted equation becomes:

\begin{equation}
    \sum_{i=1}^M \mathcal{L}_{\mathcal{F}}(u(x_i, k\Delta t)) = \frac{1}{M} \left\| \mathcal{F}\left( \mathbf{1}_M \cdot \mathbf{u}_k \right) \right\|^2= \frac{1}{M} \left\| \mathcal{F}\left( \mathbf{1}_M \cdot \mathbf{u}_k = C(k) \Phi(k, 0) \mathbf{h}_0 + C(k) \sum_{i=0}^k \Phi(k, i) \bar{B}(i) \mathbf{x}_i\right) \right\|^2.
\end{equation}




This modification ensures consistency with the Time-Varying system’s time-dependent parameters while preserving the structure of the original loss function.

\section{PDEs Setups}
\label{apx:setup}

\subsection{1-D Convection}

The 1-D convection equation, also known as the 1-D advection equation, is a partial differential equation that models the transport of a scalar quantity $ u(x,t) $ (such as temperature, concentration, or momentum) due to fluid motion at a constant velocity $ c $. It is a fundamental equation in fluid dynamics and transport phenomena. The equation is given by:
\begin{gather}
    \frac{\partial u}{\partial t} + \beta \frac{\partial u}{\partial x} = 0,\; \forall x \in[0,2\pi], t\in [0,1],\nonumber\\
    u(x,0) = \sin x,\;\forall x \in[0,2\pi],\\
    u(0,t)=u(2\pi,t),\;\forall  t\in [0,1],\nonumber
\end{gather}
where $\beta$ is the constant convection (advection) speed. As $\beta$ increases, the equation will be harder for PINN to approximate. It is a well-known equation with failure mode for PINN. We set $\beta=50$ following common practice~\cite{zhao2024pinnsformer,wu2024ropinn}.

The 1-D convection equation's analytical solution is given by:
\begin{equation}
    u_\text{ana}(x,t) = \sin(x-\beta t).
\end{equation}


\subsection{1-D Reaction}

The 1-D reaction equation is a partial differential equation that models how a chemical species reacts over time and (optionally) varies along a single spatial dimension. The equation is given by:
\begin{gather}
    \frac{\partial u}{\partial t} -\rho u(1-u) = 0,\; \forall x \in[0,2\pi], t\in [0,1],\nonumber\\
    u(x,0) = \exp(-\frac{(x-\pi)^2}{2(\pi/4)^2}),\;\forall x \in[0,2\pi],\\
    u(0,t)=u(2\pi,t),\; \forall  t\in [0,1],\nonumber
\end{gather}
where $\rho$ is the growth rate coefficient. As $\rho$ increases, the equation will be harder for PINN to approximate. It is a well-known equation with failure mode for PINN. We set $\rho=5$ following common practice~\cite{zhao2024pinnsformer,wu2024ropinn}.

The 1-D reaction equation's analytical solution is given by:
\begin{equation}
    u_\text{ana}=\frac{\exp(-\frac{(x-\pi)^2}{2(\pi/4)^2})\exp(\rho t)}{\exp(-\frac{(x-\pi)^2}{2(\pi/4)^2})(\exp(\rho t)-1)+1}.
\end{equation}

\subsection{1-D Wave}

The 1-D wave equation is a partial differential equation that describes how a wave propagates through a medium, such as a vibrating string.  We consider such an equation given by:
\begin{gather}
    \frac{\partial^2 u}{\partial t^2} - 4\frac{\partial^2 u}{\partial x^2} = 0,\; \forall x \in[0,1], t\in [0,1],\nonumber\\
    u(x,0) = \sin(\pi x)+\frac{1}{2}\sin(\beta \pi x), \;\forall x\in[0,1],\\
    \frac{\partial u(x,0)}{\partial t} = 0, \;\forall x\in[0,1],\nonumber\\
    u(0,t)=u(1,t) = 0, \; \forall  t\in [0,1],\nonumber
\end{gather}
where $\beta$ is a wave frequency coefficient. We set $\beta$ as 3 following common practice~\cite{zhao2024pinnsformer,wu2024ropinn}. The wave equation contains second-order derivative terms in the equation and first-order derivative terms in the initial condition, which is considered to be hard to optimize~\cite{wu2024ropinn}. This example illustrates that PINNMamba can better capture the time continuum because its differentiation for time is directly defined by the matrix, whose differential scale is uniform for multiple orders.

The 1-D wave equation's analytical solution is given by:
\begin{equation}
    u_\text{ana}(x,t)=\sin(\pi x)\cos(2\pi t)+\sin(\beta \pi x)\cos(2\beta \pi t).
\end{equation}

\subsection{2-D Navier-Stokes}

The 2-D Navier-Stokes equation describes the motion of fluid in two spatial dimensions $x$ and $y$. It is fundamental in fluid dynamics and is used to model incompressible fluid flows. We consider such an equation given by:
\begin{gather}
    \frac{\partial u}{\partial t} + \lambda_1 (u\frac{\partial u}{\partial x} + v \frac{\partial u}{\partial y}) = - \frac{\partial p}{\partial x} + \lambda_2 (\frac{\partial^2 u}{\partial x^2} + \frac{\partial^2 u}{\partial v^2}), \nonumber \\
    \frac{\partial v}{\partial t} + \lambda_1 (u\frac{\partial v}{\partial x} + v \frac{\partial v}{\partial y}) = - \frac{\partial p}{\partial y} + \lambda_2 (\frac{\partial^2 u}{\partial x^2} + \frac{\partial^2 u}{\partial v^2}),
\end{gather}
where $u(x,y,t)$, $v(x,y,t)$, and $p(x,y,t)$ are the x-coordinate velocity field, y-coordinate velocity field, and pressure field, respectively. We set $\lambda_1 = 1$ and $\lambda_2 = 0.01$ following common practice~\cite{zhao2024pinnsformer,raissi2019physics}. 

The 2-dimensional Navier-Stokes equation doesn't have an analytical solution that can be described by existing mathematical symbols, we take~\citet{raissi2019physics}'s finite-element numerical simulation as ground truth. 

\subsection{PINNNacle}
PINNacle~\cite{hao2023pinnacle} contains 16 hard PDE problems, which can be classified as Burges, Poisson, Heat, Navier-Stokes, Wave, Chaotic, and other High-dimensional problems. We only test PINNmamba on 6 problems, because solving the remaining problems with a sequence-based PINN model will cause an out-of-memory issue, even on the most advanced NVIDIA H100 GPU. Please refer to the original paper of PINNacle~\cite{hao2023pinnacle} for the details of the benchmark.

\section{Training Details}
\label{apx:hyperparam}

\textbf{Hyperparameters.} We provide the training hyperparameters of the main experiments in Table~\ref{tab:hyperpara}.


\begin{figure*}[!ht]
    \centering
    \includegraphics[width=\linewidth]{rebuttal-figures-src/hyperparams.pdf}
    \vspace{-1.5em}
    \caption{Concept Sliders Comparison \& Hyperparameter analysis: (Left) Impact of PCA directions: SliderSpace with 10 directions matches the FID of 64 Concept Sliders. More directions, upto 40, leads to improved FID. (Right) Effect of LoRA rank: Given a fixed training budget rank-one sliders are efficient than higher rank versions and outperforms Concept Sliders}
    \vspace{-0.3em}
    \label{fig:reb-hyperparam}
\end{figure*}
 

\textbf{Computation Overhead.} We report the training time and memory consumption of baseline models and PINNMamba on the convection equation in Table~\ref{tab:training}. 


% \section{With what \emph{data}?}\label{sec:data}
\section{How do we \emph{train models}?}\label{sec:model_training}

% Which scales? How? 
%  - [DONE] Too small may not hold
%  - [DONE] number of models can affect confidence intervals
%  - [DONE] counting size also matters data (tokens vs bits) and flop counts kaplan vs hoffman CITE; non-embedding vs embedding 
%  - DITCH? how you scale architecture matters; for eg - clark shows scaling N and E is needed to separate; enc and decoder scaling can be different for architectures; in the 
%  - [DONE] 6ND is a common approximation - does not hold for very long context (these days 128k to 1M)


In order to fit a scaling law, one needs to train a range of models across multiple orders of magnitude in model size and/or dataset size. Researchers must first decide the range and distribution of $N$ and $D$ values for their training runs, in order to achieve stable convergence to a solution with high confidence, while limiting the total compute budget of all experiments. Many papers did not specify the number of data points used to fit each scaling law; those that did range from 4 to several hundred, but most used fewer than 50 data points. The specific $N$ and $D$ values also skew the optimization process towards a certain range of $N/D$ ratios, which may be too narrow to include the true optimum. Some approaches, such as using IsoFLOPs \citep{hoffmann2022training}, additionally dictate rules for choosing $N$ and $D$ values. Moreover, using a minimum $N$ or $D$ value may result in outlier values that may need to be dropped \citep{porian2024resolving,shin2023scaling,henighan2020scaling}. We investigate this choice in Section \S\ref{sec:repl-model_training}

The definition of $N$, $D$, or compute cost $C$ can affect the results of a scaling study. For example, if a study studies variation in tokenizers, a definition of training data size based on character count may be more appropriate than one based on token count \citep{tao2024scaling}. The inclusion or exclusion of embedding layer compute and parameters, may also skew the results of a study - a major factor in the different in optimal ratios determined by \cite{kaplan2020scaling} and \cite{hoffmann2022training} has been attributed to not factoring embedding FLOPs into the final compute cost \citep{pearce2024reconcilingkaplanchinchillascaling, porian2024resolving}. Given the increase in extremely long context models (128k-1M) \cite{reid2024gemini}, the commonly used training FLOPs approximation $C = 6 ND$ (see Appendix \ref{app:full-details}) may not hold for such models, given the additional cost proportional to the context length and model dimension - \citet{bi2024deepseek} introduce a new terms non-embedding FLOPs/token to account for this.

% \ml{we've gotta decide where this discussion should go }

% \luke{I agree we could probably cut some of the next few paragraphs for space if needed. The last paragraph ends well though I think.}
% \srk{ DITCH? how you scale architecture matters; for eg - clark shows scaling N and E is needed to separate; enc and decoder scaling can be different for architectures}

 % - counting size also matters data (tokens vs bits) and flop counts kaplan vs hoffman CITE; non-embedding vs embedding 

% The goal of scaling laws is generally to extrapolate findings to larger compute budgets. It is unclear which $N$ and $D$ values should be included in the data in order to predict loss at a larger scale. 


% \srk{discuss outliers being dropped and therefore, need to be sure about the scale of training}



% The scaling law form identifies the relevant variables (e.g., $N$ and $D$). However, there remain many decisions affecting the way in which data is selected to for scaling law optimization.

%  - [DONE] knowing data composition matters because knowing data quality can change exponent across different studies ofc CITE


% Moreover, hyperparams can matter
%  - [DONE] For example, learning rate schedule can changes results CITE
%  - [DONE] batch size can change
%  - [DONE] optimal hparams change with scale so determining those matters
%  - [DONE] embedding size has been shown to matter - part of scaling law 



Scaling law fit depends on the performance of each individual checkpoint, which is highly dependent on factors such as training data source, architecture and hyperparameter choice. \citet{bansal2022data} and \citet{goyal2024scaling}, for instance, discuss the effect of data quality and composition on power law exponents and constants. Repeating data has also been found to yield different scaling patterns in large language models \citep{muennighoff2024scaling,goyal2024scaling}. 

Researchers have also studied the effect of architecture choice on scaling - \citet{hestness2017deep} find that architectural improvements only shift the irreducible loss, while \citet{poli2024mechanistic} suggest that these improvements may be more significant. The way in which a model is scaled can also affect results. Within the same architecture family, \citet{clark2022unified} show that increasing the number of experts in a routed language model has diminishing returns beyond a point, while \citet{ghorbani2021scaling} find that scaling the encoder and decoder have different effects on model performance. Scaling embedding size can also drastically change scaling trends \citep{tao2024scaling}.

The optimal hyperparameters to train a model changes with scale. Changing batch size, for example, can change model performance \cite{mccandlish2018empirical, kaplan2020scaling}. Optimal learning rate is another hyperparameter shown to change with scale, though techniques such as those proposed in Tensor Programs series of papers \citep{yang2022tensor} can keep this factor constant with simple changes to initialization. More specifically, changing the learning rate schedule from a cosine decay to a constant learning rate with a cooldown (or even changing the learning rate hyperparameters) has been found to greatly affect the results of scaling laws studies \citep{hu2024minicpm,porian2024resolving,hagele2024scaling, hoffmann2022training}.

% \ml{add paragraph about max model params}

% \srk{talk abt }

% - LR \citep{hu2024minicpm}.
% - batch size

% Some factors, such as training data distribution, do not have a single optimal value and may simply be held constant for all experiments, but they change the absolute value of all power law parameters, e.g. it is not meaningful to compare power laws fit to experiments on different training data distributions. Other factors, such as batch size, may have optimal values depending on other hyperparameters, data budget, architecture, etc. It is sometimes possible to fix these factors in relation to others (e.g., scale batch size with the parameter count). There are other factors for which there is no smooth interpolation between points. For example, the width and depth of a transformer are limited to integer values, and further limited by the convention of using widths equal to small multiples of powers of 2.

% \srk{we should concretely discuss learning rate schedules and cite miniCPM, the LR schedule rate paper}
One common motivation for fitting a scaling laws is extrapolation to higher compute budgets. However, there is no consensus on the orders of magnitude up that one can project a scaling law and still find it accurate, nor on the breadth of compute budgets that should be covered by the data. We find that the range of model size $N$ and dataset size $D$ greatly varies, with the maximum value of $N$ in each paper ranging from 10M parameters to around 7B and that of $D$ being as large as 400B tokens. 
% \textcolor{blue}{
For most papers we survey, the scales are relatively modest: 13 of 51 papers train models beyond 2B parameters; most only train models smaller than 1B parameters.
% }
% \srk{idk what this refers to: Some papers show projected results at as much as 6 or 7 orders of magnitude greater than the compute budgets they experimented with. } 
It has been shown, with some controversy \cite{schaeffer2023emergent}, that scaling to significantly larger scales can result in new abilities that did not appear in smaller models \citep{wei2022emergent}. Forecasting limits to extrapolation and the appearance of new abilities at new scales is an open question.

% 
% - while SOTA models go till 100s of B; 
% - scaling laws often only go until 1b
% - it remains unclear what the corrcet scale is

% \srk{mention emergent abilities; and counterpt}

% \srk{Training data [DONE] and downstream metrics chosen may completely change exponent?}

\section{Sensitivity Analysis}

PINNMamba can be further improved by hyper-parameters tuning, we test the sub-sequence length, interval and activation selection in this section.

\label{apx:sense}

\textbf{Sub-sequence Length.} We test the effect of different sub-sequence lengths on model performance. As shown in Table~\ref{tab:length}, we test the length of 3, 5, 7, 9, 21.  Length $k =7$ achieves the best performance on reaction and wave equations, while  $k =5$ achieves the best performance on convection equation.

\begin{table}[!t]
% \small
    \centering
    \resizebox{\linewidth}{!}{
    \begin{tabular}{l|cc|c}
    \toprule
    \textbf{Methods} & \textbf{Length} & \textbf{ASR} & \textbf{Length-to-ASR Ratio (\%$\uparrow$)} \\
    \midrule
    \textbf{Pre-generated} & 111.2 & 0.80 & 0.72 \\
    \midrule
    \textbf{Simple Truncation-i} & 72.15 & 0.66 & 0.91 \\
    \textbf{Simple Truncation-ii} & 50.46 & 0.57 & 1.13 \\
    \textbf{Word Frequency} & 97.43 & 0.49 & 0.50 \\
    \textbf{LLM-based} & 52.12 & 0.68 & \textbf{1.30} \\
    \bottomrule
    \end{tabular}}
    \caption{Comparison of methods of length declining. \textbf{Length} refers to the token size of the prompt, \textbf{ASR} denotes the attack success rate, and \textbf{length-to-ASR ratio} indicates the ratio of length to ASR.}
    \label{tab:length}
\end{table}

\textbf{Sub-Sequence Interval.}  We test the effect of different sub-sequence intervals on model performance. As shown in Table~\ref{tab:interval}, we test the intervals of $2e-3$, $5e-3$, $1e-2$, $1e-1$. The interval $\Delta t =1e-2$ achieves the best performance on convection and wave equations, while $\Delta t = 5e-3$ achieves the best performance on reaction. Note that, when $\Delta t = 1e-1$, we cannot build the sub-sequence contrastive alignment.
\begin{table}[H]
\vspace{-3mm}
  \caption{Results with different Sub-Sequence Interval of PINNmamba, $k$ is set to 7.}
  
  \centering
    \small
  \begin{tabular}{c|cc|cc|cc}

    \toprule 
      &\multicolumn{2}{c}{Convection }&\multicolumn{2}{c}{Reaction}&\multicolumn{2}{c}{Wave}\\
    \cmidrule(lr){2-3}\cmidrule(lr){4-5}\cmidrule(lr){6-7}
   Interval & rMAE & rRMSE & rMAE & rRMSE & rMAE & rRMSE\\
   \midrule
   2e-3 &0.0249& 0.0257 & 0.0739 & 0.1389 &0.1693 &0.1903  \\
 5e-3 & 0.0243& 0.0287& \textbf{0.0083} & \textbf{0.0185} & 0.2492 & 0.2690 \\
 1e-2 &\textbf{0.0188} & \textbf{0.0201} &0.0094 &0.0217 & \textbf{0.0197} &  \textbf{0.0199}\\
 1e-1 & 1.2169 &1.3480 &0.4324& 0.5034   &0.0666 &  0.0703\\

    

   

    \bottomrule
  \end{tabular}
  \normalsize
  \label{tab:interval}

\end{table}


\textbf{Activation Function.} We test the activation function's effect on the performance of PINNMamba. We report the results of ReLU~\cite{nair2010rectified}, Tanh~\cite{fan2000extended}, and Wavelet~\cite{zhao2024pinnsformer} in Table~\ref{tab:activation}.
\begin{table}[H]
\vspace{-3mm}
  \caption{Results with different activation function in PINNmamba.}
  
  \centering
    \small
  \begin{tabular}{c|cc|cc|cc}

    \toprule 
      &\multicolumn{2}{c}{Convection }&\multicolumn{2}{c}{Reaction}&\multicolumn{2}{c}{Wave}\\
    \cmidrule(lr){2-3}\cmidrule(lr){4-5}\cmidrule(lr){6-7}
   Activation & rMAE & rRMSE & rMAE & rRMSE & rMAE & rRMSE\\
   \midrule
   ReLU & 0.4695& 0.4722 & 0.0865 & 0.1583 &0.4139 &0.4203  \\
 Tanh & 0.4531& 0.4601& 0.0299 & 0.0568  & 0.3515 & 0.3539  \\
Wavelet &0.0188 & 0.0201 &0.0094 &0.0217 & 0.0197 &  0.0199\\


    

   

    \bottomrule
  \end{tabular}
  \normalsize
  \label{tab:activation}

\end{table}

\section{Complex Problem Results}
\label{apx:comp}

\subsection{2D Navier-Stokes Equations}

Although PINN can already handle Navier-Stokes equations well, we still tested the performance of PINN Mamba on Navier-Stokes equations to check the generalization performance of our method on high-dimensional problems. As shown in Fig.~\ref{fig:nss}, our method achieves good results on Navier-Stokes pressure prediction. Since there is no initial condition information for the N-S equation for pressure, we took the data from the only collection point for pattern alignment.

\begin{figure*}[t]
    \centering
    \includegraphics[width=\textwidth]{_fig/nss}
    \vspace{-3mm}
    \caption{The ground truth solution, prediction (top), and absolute error (bottom) on Navier-Stokes equations.}
    \label{fig:nss}
    %\vspace{-5mm}
  %  \vspace{-1mm}
\end{figure*}

\subsection{PINNacle Benchmark}

Like PINNsFormer, PINNMamba is a sequence model. The sequence model suffers from Out-of-Memory problems when dealing with some of the problems in the PINNacle Benchmark~\cite{hao2023pinnacle}, even when running on the advanced Nvidia H100 GPU. We report here the results of the sub-problems for which results can be obtained in Table~\ref{tab:pinnacle}. PINNMamba can solve the Out-of-Memory problem by distributed training over multiple cards, which we leave as a follow-up work.

\begin{table}[H]
\vspace{-3mm}
  \caption{Results on PINNacle. Baseline results are from RoPINN paper~\cite{wu2024ropinn}. OOM means Out-of-Memroy.}
  
  \centering
    \small
  \begin{tabular}{c|cc|cc|cc}

    \toprule 
      &\multicolumn{2}{c}{PINN }&\multicolumn{2}{c}{PINNsFormer}&\multicolumn{2}{c}{PINNMamba}\\
    \cmidrule(lr){2-3}\cmidrule(lr){4-5}\cmidrule(lr){6-7}
   Equation & rMAE & rRMSE & rMAE & rRMSE & rMAE & rRMSE\\
   \midrule
   Burgers 1d-C &1.1e-2& 3.3e-2 & 9.3e-3 & 1.4e-2 & 3.7e-3 & 1.1e-3 \\
 Burgers 2d-C & 4.5e-1& 5.2e-1&  OOM & OOM & OOM & OOM \\
 Poisson 2d-C & 7.5e-1 & 6.8e-1 & 7.2e-1 & 6.6e-1 & 6.2 e-1 & 5.7e-1 \\
Poisson 2d-CG & 5.4e-1 & 6.6e-1 & 5.4e-1& 6.3e-1 & 1.2e-1 & 1.4e-1 \\
Poisson 3d-CG & 4.2e-1 & 5.0e-1 & OOM& OOM & OOM & OOM \\
Poisson 2d-MS & 7.8e-1 & 6.4e-1 & 1.3e+0& 1.1e+0 & 7.2e-1& 6.0e-1 \\
Heat 2d-VC & 1.2e+0 & 9.8e-1 & OOM& OOM &OOM &OOM  \\
Heat 2d-MS & 4.7e-2 & 6.9e-2 &OOM & OOM &OOM &OOM  \\
Heat 2d-CG & 2.7e-2 & 2.3e-2 & OOM& OOM &OOM &OOM  \\
NS 2d-C & 6.1e-2 & 5.1e-2 & OOM& OOM & OOM& OOM \\
NS 2d-CG & 1.8e-1 & 1.1e-1 & 1.0e-1& 7.0e-2 & 1.1e-2& 7.8e-3  \\
Wave 1d-C & 5.5e-1 & 5.5e-1 & 5.0e-1 & 5.1e-1 & 1.0e-1 & 1.0e-1 \\ 
Wave 2d-CG & 2.3e+0 & 1.6e+0 &OOM & OOM  &OOM &OOM  \\
Chaotic GS & 2.1e-2 & 9.4e-2 & OOM& OOM & OOM &OOM  \\
High-dim PNd & 1.2e-3 & 1.1e-3 &OOM &OOM  &OOM &OOM  \\
High-dim HNd & 1.2e-2 & 5.3e-3 &OOM &OOM  &OOM &OOM  \\
   

    \bottomrule
  \end{tabular}
  \normalsize
  \label{tab:pinnacle}

\end{table}




\end{document}
\EOD


% \maketitle
% \begin{abstract}
% Text-to-SQL is the task of transforming natural language questions into executable SQL queries, enabling seamless database interaction. Existing approaches, such as skeleton-masked selection, rely on retrieving similar examples from the training set to guide query generation. However, such methods struggle when no similar examples exist in the training data, which is a common challenge in real-world scenarios. To address this limitation, we propose SAL-SQL, a novel approach that enables large language models (LLMs) to self-correct and refine their SQL generation process without relying on example similarity. SAL-SQL incorporates predefined error warnings and self-teaching signals, guiding LLMs to avoid common pitfalls in Text-to-SQL tasks. Our approach improves model robustness and accuracy, particularly in unseen or complex cases where traditional methods fail. 
% \end{abstract}




% \section{Introduction}


% Text-to-SQL generation, the process of translating natural language questions into SQL queries, plays a crucial role in enabling intuitive database interactions. Traditional methods, such as skeleton-masked selection, rely heavily on retrieving similar examples from training data to guide query generation. However, these methods face significant challenges in real-world scenarios where similar examples are often unavailable in the training set.
% To overcome these limitations, we introduce SAL-SQL, an approach that leverages the generative capabilities of large language models (LLMs) to create synthetic examples, consisting of SQL queries, their corresponding natural language questions, and reasoning paths. Unlike traditional methods, It enables the model to generate and learn from its own examples, guided by predefined error warnings and self-teaching signals. This mechanism allows LLMs to iteratively improve their inference capabilities, enhancing both robustness and accuracy. By relying on LLM-generated examples, our method demonstrates superior performance, particularly in complex or unseen scenarios where traditional retrieval-based approaches fail.

% \begin{figure}[t]
% \centerline{\includegraphics[scale=0.3]{Pictures/ss1.png}}
% \caption{Our proposed SQL example generation flowchart.}
% \vspace{-5mm}
% \end{figure}

% \begin{figure*}[!ht]
% \centerline{\includegraphics[scale=0.2]{Pictures/ss.png}}
% \caption{Our proposed SAL-SQL flowchart.}
% \vspace{-5mm}
% \end{figure*}


% \section{Related work}

% \subsection{Rule based system}
% Early approaches to Text-to-SQL relied on rule-based systems and semantic parsing frameworks, which required extensive domain expertise and manual feature engineering. Rule-based methods involved designing syntactic and semantic rules to map natural language to SQL, often utilizing predefined templates or handcrafted grammars. Systems focusing on translating restricted natural language inputs into SQL queries. 


% \subsection{Supervised Fine-Tuning}
% Supervised Fine-Tuning enables models to learn domain-specific nuances and schema alignments, significantly improving performance on specialized tasks. Techniques such as schema linking, constraint-based decoding, and execution-guided generation have further enhanced the robustness of fine-tuned models in handling domain-specific challenges. There is a method that synthesizes text-to-SQL data from weak and strong LLMs~\cite{synthesize}. This method utilizes preference learning from the weak data from small LLMs and strong data from Large LLMs. Supervised fine-tuning in Text-to-SQL is a time-consuming task and requires an enormous amount of computational resources.



% \subsection{In-context learning}
% In-context learning has emerged as another influential method, leveraging the ability of large language models to perform text to sql task by conditioning on a few examples provided in the input prompt, without requiring explicit parameter updates.

% \subsubsection{Schema Linking}
% Schema Linking is a crucial process for learning associations between database schema elements (e.g., tables, columns, and values) and natural language questions. This technique identifies keywords or concepts in a question and links them to specific schema components in the database, clarifying which parts of the schema the question refers to. It typically employs word embeddings, edit distances, or pre-trained language models to measure the similarity between question tokens and schema elements. Schema Linking plays a vital role in improving the accuracy of Text-to-SQL models, particularly by interpreting complex references in user queries and mapping them to the correct SQL components.
% DIN-SQL~\cite{din} utilizes pre-sql with schema linking which related tables and column entities which boosts the execution accuracy compared with the traditional method.  



% \subsubsection{Skeleton masked similarity}
% AST-sql~\cite{ast} introduces using an abstract syntax tree algorithm to select similar examples.
% Skeleton Masked Similarity is an approach that emphasizes structural similarity between natural language questions and SQL queries. This method involves extracting the skeleton of a SQL query from the given question and masking unnecessary details to focus on its essential structure. By preserving key structural patterns, such as SELECT-FROM-WHERE clauses, this approach facilitates learning the correspondence between natural language expressions and SQL query elements. It moves beyond simple word-level matching to capture deeper structural correlations, which is particularly effective in handling complex SQL queries or questions with diverse linguistic expressions. This method is not applicable without a training set.  
% \subsubsection{Classification and decomposition}
% The Classification and Decomposition Method simplifies the generation of SQL queries from natural language questions by breaking down the task into sequential steps. This process begins with classifying the input question based on its structure or intent (e.g., single-table queries, multi-table joins, or nested queries). Following this, the question is decomposed into smaller subtasks, such as identifying specific clauses and resolving their individual components. PTD-SQL~\cite{ptd} decomposes and categorizes SQL questions to enhance the LLM inference. This step-wise approach allows models to address each sub-problem independently, reducing complexity and improving overall accuracy.
% By modularizing the SQL generation process, the Classification and Decomposition Method enables better handling of complex, multi-intent queries or those involving nested and hierarchical relationships. This method also enhances robustness when dealing with ambiguous questions by isolating and resolving individual components before combining them into a cohesive SQL query. However, this method is hard to apply when the SQL question is not in the predefined categories. 

% \subsubsection{Self correction}
% MAGIC~\cite{magic} introduces self-correction guidelines for in-context Text-to-SQL. Magic consists of three agents(manager, feedback, and correction agents) to self-correct the generated SQL queries. This method also requires training set to define the self-correction guideline. Furthermore, it requires more than five LLM iterations to generate one final SQL query. Text-to-SQL needs to be more simple and precise.
% \section{SAL-SQL}
% SAL-SQL introduces a novel framework for enabling large language models (LLMs) to autonomously improve SQL query generation. Unlike prior methods that depend on retrieving similar questions or predefined structures (e.g., skeleton-masked selection), SAL-SQL focuses on self-generating relevant examples and reasoning paths dynamically to guide the model toward producing correct SQL queries.


% \subsection{Self generating example}
% Given a test question and database table, SAL-SQL generates a set of three similar SQL questions and their corresponding queries. These similar examples are generated by:
% Utilizing the LLM's capability to reason about question intent and database schema.
% Adjusting entities, conditions, and structures of the test question while maintaining a comparable reasoning path.
% \subsection{Reasoning Path generation}
% For each of the generated questions and SQL queries, SAL-SQL produces a reasoning path that explains the thought process behind mapping the question to the SQL query. The reasoning path includes:
% Question Analysis: Identifying the intent, target columns, tables, and filters.
% Schema Alignment: Mapping question components to database schema elements. 
% Query Construction Steps: Breaking down the SQL query into logical steps.
% \subsection{Predifined error warning}
% SAL-SQL further improves accuracy by incorporating predefined error warnings that address common pitfalls in Text-to-SQL tasks, such as Missing clauses.
% Incorrect column or table selection.
% Logical inconsistencies (e.g., mismatched aggregations and filters).
% These warnings serve as self-learning signals, enabling the model to detect and correct errors autonomously during the generation process.

% \begin{table*}[t]
% \centering
% \begin{tabular}{llccccc}
% \toprule
% \textbf{Methods} & \textbf{Model} & \textbf{Easy} & \textbf{Medium} & \textbf{Hard} & \textbf{Extra Hard} & \textbf{All} \\  
% \midrule
% \textit{Few-shot} & CodeX-davinci & 84.7\% & 67.3\% & 47.1\% & 26.5\% & 61.5\% \\  
% \textit{Few-shot} & GPT-4o         & 86.7\% & 73.1\% & 59.2\% & 31.9\% & 67.4\% \\  
% \textit{DIN-SQL}  & CodeX-davinci & 89.1\% & 75.6\% & 58.0\% & 38.6\% & 69.9\% \\  
% \textit{DIN-SQL}  & GPT-4o         & 91.1\% & 79.8\% & 63.9\% & 43.4\% & 74.2\% \\  
% \textit{DAIL-SQL}  & CodeX-davinci & 90.3\% & 79.6\% & 62.0\% & 38.6\% & 69.9\% \\  
% \textit{DAIL-SQL}  & GPT-4o         & 92.1\% & 80.8\% & 63.9\% & 48.4\% & 75.2\% \\  
% \textit{MAGIC-SQL}  & GPT-4o          & 93.1\% & \textbf{85.8\%} & 69.9\% & 49.4\% & 77.2\% \\  
% \textit{\textbf{ \textit{SAL-SQL}}} & GPT-4o-mini & \textbf{93.6\%} & 87.5\% & \textbf{90.08\%}& \textbf{74.7\%} & \textbf{87.4\%} \\  
% \bottomrule
% \end{tabular}
% \caption{Execution accuracy performance of different methods and models on Text-to-SQL tasks across difficulty levels.}
% \label{tab:sql_comparison}
% \end{table*}

% \subsection{Final SQL query generation}
% The LLM then examines the generated SQL examples and their reasoning paths to self-reflect and correct its initial SQL query. By analyzing patterns and logical consistencies across the generated examples, the model identifies and rectifies any errors in its own reasoning process.

% \subsection{Cross consistency}
% PET-SQL~\cite{pet} introduces cross consistency which leverages outputs from multiple models to select the final SQL query. We implement three independent models (e.g., GPT-4, Codex, and Llama3.1) to generate SQL queries for the same input question. The outputs from the three models are compared to identify the most frequent SQL query. Cross-consistency is determined by matching the generated SQL queries at the structural level rather than exact token-level matching. If two or more models produce identical or highly similar SQL queries, that query is selected as the final output. In case of conflicts or divergence, a majority-voting mechanism is applied based on logical consistency and execution outcomes.


% % Bibliography entries for the entire Anthology, followed by custom entries
% %\bibliography{anthology,custom}
% % Custom bibliography entries only

% \section{Experiment}

% \begin{table}[t]
%     \centering
%     \begin{tabular}{lcc}
%         \toprule
%         Models & EX & VES \\
%         \midrule
%         GPT-4o + SAL & 37.94 & 42.15 \\
%         w/o 3 examples & 35.21 (-2.73) & 40.03 (-1.95) \\
%         w/o R path & 37.23 (-0.71) & 41.30 (-0.85) \\
%         w/o P warning & 36.25 (-1.69) & 40.46 (-2.07) \\
%         w/o cross consist & 36.91 (-1.03) & 41.12 (-1.55) \\
%         \bottomrule
%     \end{tabular}
%     \caption{This table shows the performance with different methods}
%     \label{tab:models}
% \end{table}



% \subsection{Setting}
% For our experiments, we utilized GPT-4o as the backbone model to evaluate the performance of the Text-to-SQL task. The evaluation was conducted on the Spider Dev Dataset, which is a widely used benchmark for Text-to-SQL systems.
% The Spider dataset is a large-scale, cross-domain benchmark specifically designed to assess the generalization capabilities of Text-to-SQL models. It contains 7,000 training samples spanning 166 databases across various domains and 
% 1,034 evaluation samples (referred to as the “Dev split”) from 20 databases.
% This dataset was chosen for its diversity and ability to evaluate the model’s performance in generating SQL queries across unseen database schemas, making it a suitable testbed for assessing the generalizability and robustness of the proposed approach.The experiments were conducted under controlled conditions to ensure consistency across all evaluations.
% \subsection{Evaluation}
% Execution Accuracy(EX) measures whether the SQL query generated by the model produces the same results as the ground truth query when executed on a database. This metric is sensitive to the state and schema of the database, making it essential to maintain a consistent testing environment for reliable evaluation. Valid Efficient Score(VES) considering both validity and efficiency. 

% \subsection{Performance for different SQL difficulty level}
% In our analysis, we evaluate the efficacy of SQL-PaLM against a spectrum of SQL difficulty levels, which are categorized based on the number of SQL keywords used, the presence of nested subqueries, and the application of column selections or aggregations. The results in Table 2 highlight STL performance in comparison with the standard few-shot prompting approach using GPT-4 and CodeX-Davinci,
% as well as the advanced prompting approach DIN-SQL. Our findings reveal that
% SAL-SQL overall surpasses the other approaches across all evaluated difficulty levels.

% \section{Ablation study}
% To analyze the contribution of key components in our model, we conducted an ablation study by removing the Reasoning Path, Predefined Warning, and Cross Consistency modules individually. The results, evaluated using Execution Accuracy (EX) and VES, are summarized in the table above. The removal of the Reasoning Path resulted in a decrease in Execution Accuracy (EX) by 0.71 and VES by 0.85. This demonstrates the importance of reasoning pathways for guiding the model in generating accurate SQL queries. Without the Predefined Warning mechanism, EX dropped by 1.69 and VES decreased by 2.07. This indicates that predefined warnings help the model avoid common pitfalls or ambiguities during SQL generation, leading to improved performance. Eliminating the Cross Consistency mechanism caused EX to drop by 1.03 and VES by 1.55. Cross Consistency plays a significant role in ensuring that the generated SQL queries remain coherent and accurate across varying conditions. These results highlight that all three components—Reasoning Path, Predefined Warning, and Cross Consistency—are critical for achieving optimal performance. The full model (GPT-4o + SAL-SQL) outperforms its ablated versions, demonstrating the synergistic effect of these components in improving Execution Accuracy and VES scores.



% \section{Conclusion}
% In this paper, we propose SAL-SQL which enhances SQL execution accuracy with self-augmented examples and reasoning paths. Through extensive experiments and an ablation study, we show that critical components such as Reasoning Path, Predefined Warning, and cross-consistency contribute significantly to the overall performance. Specifically, the full model (GPT-4o + SAL) achieves state-of-the-art results, with notable improvements over ablated versions, highlighting the importance of these modules in generating accurate and semantically valid SQL queries. Our findings underscore the capability of large language models when carefully structured and enhanced, to address complex Text-to-SQL tasks. Future work will focus on further improving performance in low-resource settings, handling more complex SQL queries, and expanding evaluation to larger and more diverse real-world datasets.

