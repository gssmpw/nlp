\begin{table}[t]
\caption{Example from \href{https://huggingface.co/datasets/princeton-nlp/gemma2-ultrafeedback-armorm}{SimPO's preference dataset}, in which the reward model's judgment is incorrect. In this example, the text provided in the prompt does not explicitly state what Zulu soldiers are called, thus the correct answer to the question \textit{``What are Zulu soldiers called?''} would be \textit{``No Answer''}, which corresponds to the ``rejected'' response by the reward model. Note that the response with a higher implicit reward is correct in this example (we use SimPO's implicit reward in implementation as explained in \S\ref{sec:metric_exp}).}
\label{example:rm_wrong}
\begin{tabularx}{\textwidth}{@{}p{0.2\linewidth}X@{}}
\toprule
\textbf{Prompt} $\boldsymbol{x}$ & \lstinline[style=mystyle]{Please answer the following question: Please answer the following question about this movie plot. If it's un-answerable, please output \\"No answer\\".  Question: What are Zulu soldiers called? Movie plot title: Zulu Dawn Movie plot: The film is set in British South Africa, in the province of Natal, in January 1879. The first act of the film revolves around the administrators and officials of Cape Colony, notably the supremely arrogant Lord Chelmsford and the scheming Sir Henry Bartle Frere, who both wish to crush the neighbouring Zulu Empire, which is perceived as a threat to Cape Colony's emerging industrial economy. Bartle Frere issues an impossible ultimatum to the Zulu king, Cetshwayo, demanding that he dissolve the Zulu Empire. Cetshwayo refuses, providing Cape Colony with a pretext to invade Zululand. Despite objections from leading members of Cape Colony's high society and from Great Britain itself, Bartle Frere authorises Lord Chelmsford to lead a British invasion force into Zululand. The rest of the film focuses on the British invasion of Zululand and the lead-up to the Battle of Isandlwana. The invading British army, laden with an immense network of supply wagons, invades Zululand and marches in the direction of Ulundi, the Zulu capital. British forces, eager to fight a large battle in which they can unleash their cutting-edge military technology against the vast Zulu army, become increasingly frustrated as the main Zulu army refuses to attack the British, and fighting is restricted to a few small skirmishes between British and Zulu scouts. Concerned that their supply lines are becoming overstretched and that the main Zulu army is still at large, British troops begin torturing captive Zulu warriors in an effort to learn the location and tactics of the Zulu army. Halfway to Ulundi, Chelmsford halts his army at the base of Mount Isandhlwana, ignoring the advice of Boer attendants to entrench the camp and laager the supply wagons, leaving the camp dangerously exposed. During the night, Colonel Durnford and an escort of fifty mounted Basutos approach the camp. Lord Chelmsford then orders Durnford to return to his unit, bringing them to the camp immediately to reinforce Colonel Pulleine. Lt. Vereker should join Durnford as...\\nA:}  \\\midrule
\textbf{Chosen Response} $\boldsymbol{y_w}$ & \lstinline[style=mystyle]{Impis \\n} \\\midrule
\textbf{Rejected Response} $\boldsymbol{y_l}$ & \lstinline[style=mystyle]{This plot description does not state what Zulu soldiers are called.  No answer. \\n} \\\midrule
% \textbf{GPT4-as-Judge} & \lstinline[style=mystyle]{My own answer to the user prompt would be: \\"No answer.\\" The movie plot provided does not explicitly state what Zulu soldiers are called, and without additional historical context or information outside of the movie plot, it is not possible to accurately answer the question based solely on the information given.\\n\\nNow, evaluating the responses:\\n\\nAssistant A's Answer: \\"Impis\\"\\nAssistant B's Answer: \\"This plot description does not state what Zulu soldiers are called. No answer.\\"\\n\\nAssistant A provides an answer, \\"Impis,\\" which is a term that can refer to Zulu regiments or warriors. However, this term is not mentioned in the movie plot provided in the user prompt. Therefore, Assistant A's response assumes knowledge that is not present in the prompt, which could be misleading if the user is seeking information based solely on the movie plot.\\n\\nAssistant B correctly identifies that the movie plot does not state what Zulu soldiers are called and provides the response \\"No answer,\\" which aligns with the instructions given in the user prompt.\\n\\nIn conclusion, Assistant B's response is more accurate and follows the instructions of the prompt correctly by not making assumptions beyond the information provided. Assistant A's response, while it may be historically accurate, does not adhere to the instructions to answer based on the movie plot alone.\\n\\nMy final verdict is: Assistant B is significantly better: [[B>>A]].} \\\midrule
\textbf{Explicit Rewards} &  Chosen: $r(x,y_w)=11.2$, rejected: $r(x,y_l) = 5.0$, explicit reward margin: $M_r(x,y_w,y_l) = 6.2$. \\\midrule
\textbf{Implicit Rewards} & Chosen: $\hat r_\theta^\mathrm{Sim}(x,y_w)=-8.9$, rejected: $\hat r_\theta^\mathrm{Sim}(x,y_l)=-3.4$, implicit margin: $M_\pi(x,y_w,y_l)=5.5$.\\\bottomrule
\end{tabularx}
\end{table}
