\begin{table}[t]
\caption{Prompts to evolve instructions, adapted from EvolInstruct \citep{EvolInstuct}.}
\label{example:evol_prompts}
\begin{tabularx}{\textwidth}{@{}p{0.21\linewidth}X@{}}
\toprule
\textbf{In-Depth Base Prompt} with an adaptive ``strategy'' identifier: \lstinline[style=mystyle]{<STRATEGY>}, which will be replaced by the following 4 strategies  & \lstinline[style=mystyle]{I want you act as a Prompt Rewriter.\\n 
Your objective is to rewrite a \{Given Prompt\} into a **more complex version** to make those famous AI systems (e.g., ChatGPT) a bit harder to handle.\\n 
**Requirements**:\\n
- The \{Rewritten Prompt\} cannot omit the input in the \{Given Prompt\}. \\n 
- You SHOULD complicate the given prompt using the following method: \\n
<STRATEGY> \\n
- The \{Rewritten Prompt\} must be reasonable and must be understood and responded to by humans.\\n
**Constraints that must be followed**:\\n
- The \{Rewritten Prompt\} can only add 10 to 20 words into the \{Given Prompt\}. \\n
- The \{Rewritten Prompt\} should be self-contained, with **all necessary information** provided, so that it can be responded to without needing to refer back to the \{Given Prompt\}.\\n
- Your response should contain **only** the \{Rewritten Prompt\}, **without any** additional formatting or introductory phrases such as 'Here is the rewritten prompt:' or 'The rewritten prompt is:'.\\n
The \{Given Prompt\}: \\n<PROMPT>\\n
========\\nBased on the prompt above, rewrite a prompt:\\n\{Rewritten Prompt\}:\\n}  \\\midrule
\textbf{-- Adding Constraints} & The strategy is: \lstinline[style=mystyle]{Please add one more constraint/requirements into the \{Given Prompt\}}  \\\midrule
\textbf{-- Deppening} & The strategy is: \lstinline[style=mystyle]{If the \{Given Prompt\} contains inquiries about certain issues, the depth and breadth of the inquiry can be increased.}  \\\midrule
\textbf{-- Concretizing} & The strategy is: \lstinline[style=mystyle]{Please replace general concepts with more specific concepts.}  \\\midrule
\textbf{-- Increasing Reasoning} & The strategy is: \lstinline[style=mystyle]{If the \{Given Prompt\} can be solved with just a few simple thinking processes, you can rewrite it to explicitly request multiple-step reasoning.}  \\\midrule
\textbf{In-Breadth Prompt} & \lstinline[style=mystyle]{I want you to act as a Prompt Creator.\\n
Your objective is to take inspiration from the \{Given Prompt\} to create **one** brand new prompt.\\n
**Reqiuirements**:\\n
- This new \{Created Prompt\} should belong to the same domain as the \{Given Prompt\} but with different details.\\n
- The LENGTH and complexity of the \{Created Prompt\} should be similar to that of the \{Given Prompt\}.\\n
- The \{Created Prompt\} must be reasonable and must be understood and responded by humans.\\n
- If the \{Given Prompt\} includes a specific input as part of its instructions, create a new input for your \{Created Prompt\} when applicable.\\n
**Constraints that must be followed**:\\n
- The \{Created Prompt\} should be self-contained, with **all necessary information** provided, so that it can be responded to without needing to refer back to the \{Given Prompt\}.\\n
- Your response should contain **only** the \{Created Prompt\}, **without any** additional formatting or introductory phrases such as 'Here is the created prompt:' or 'The created prompt is:'.\\n
The \{Given Prompt\}: \\n<PROMPT>\\n
========\\nBased on the prompt above, create your prompt:\\n\{Created Prompt\}:\\n}  \\\bottomrule
\end{tabularx}
\end{table}