\section{Related Work}
\subsection{In-context Learning}
In-context Learning (ICL) ____ is a technique that uses few-shot in-context learning samples to guide the pre-traind autoregressive LLM to produce satisfying results, without additional training or fine-tuning. The in-context learning samples are usually specifically designed for designated downstream task and can serve as auxiliary parameters attached to the model to guide the generation process. ____ explored the way to better design the context sample through distance metrics, and impact of different kinds of the distance metrics like Euclidean distance and so on. ____ proposed a method to reinforce generalization ability through sample diversification. ____ introduced a perplexity based method for designing in-context samples. ____ conducted cross-lingual contextal learning experiments using clustering methods. ____ used ICL techniques to perform the alignment task of different languages.

\subsection{Automatic Depression Detection}
Plenty of works have been dedicated to automating the process of depression detection by implementing methods like natural language processing, machine learning, multimodal model LLM. To start with, researchers used traditional methods to tackle with the issue. For instance, ____ introduced a Convolutional Neural Network (CNN) and Bidirectional Long Short-Term Memory (BiLSTM) based deep learning model to perform depression detecting tasks upon social media posts. ____ and ____ have posed time series based LSTM to detect suicide risk. Meanwhile, there are also some works (e.g. ____), combining feature engineering with machine learning algorithms and deep neural network methods for diagnosing mental disorder. Multimodal methods have also been generally applied in depression detection task. ____ leveraged audio and text modalities to analyze sentiment and mental health, with a method of prompt engineering. ____ extracted features of audio and video respectively and used the technique of mamba to fuse them and made collaborative classification for depression detection. ____ proposed HiQuE, hierarchically modeled the question and answer series in the interview dialogues for depression detection. Furthermore, LLMs like BERT, LLaMA and GPT have also been implemented by many works due to its reasoning ability. ____, ____, ____ implemented some prevailing LLMs and got decent performance on social media based dataset. ____ developed a chat based method using GPT for interactive depression. ____ proposed MentaLLaMA, a fine tuned version of LLaMA-2, which concentrated on interpretability of issues upon mental disorder. 

To the best of our knowledge, interactive depression detection methods based on explicit theme learning have not yet been explored in the field of automated depression detection. Unlike existing related work, our model is the first to learn theme correlation and design an interactive strategy to incorporate clinical feedback for preference learning.
% Nevertheless, aforementioned works all have their defects, such as outdated performance, lack of interpretability, extremely sophisticated architecture, to name a few. To tackle with this problem, we propose our method established on the combination of transformer based LLM and embedding neural network. Our model makes utilization of LLM to the maximum extent, which allows it to achieve the state-of-the-art (SOTA) performance on DAIC-WOZ dataset even by using only text modality.

\begin{figure*}[ht]
\centering
\begin{center}
\centerline{\includegraphics[width=\textwidth]{img/model.pdf}}
\caption{
Schematic illustration of the proposed PDIMC framework with three components. The theme-oriented in-context learning technique leverages the LLM to learn theme content from clinical interview dialogues. Theme correlation learning captures the inter- and intra-theme semantics related to depressive states. The interactive theme adjustment strategy utilizes the LLM to simulate clinical feedback, dynamically adjusting theme importance.
}
\label{fig:model}
\end{center}
\end{figure*}