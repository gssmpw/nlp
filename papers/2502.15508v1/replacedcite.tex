\section{Related work}
\label{sec::related}

The combination of three elements of energy efficient path reconfiguration for industrial field data, namely, the industrial field communication (wireless, multi-radio, multi-hop), the industrial requirements (delay constraints, energy restrictions, and data pieces in need to be accessed by consumers), and the dynamic insertion and deletion of nodes in the network at any time, is closely related to industrial routing protocols. Proposed mechanisms for distributed path reconfiguration can act complementarily and on top of industrial routing protocols. For this reason, we provide a brief overview on the recent state of the art and on some classic industrial routing alternatives. The typical routing alternative for industrial low power deployments is IETF RPL ____. RPL aims to generate routing links between the network devices. Within RPL, the coordinating network controller has all the necessary information to incorporate different devices into the network. Each set of nodes within a network is part of one or more directed acyclic graphs. In each graph, once a root node is defined, the graph will be oriented to it, propagating the generated data towards the root node. In ____ and ____, the authors present the reliable real-time flooding-based routing protocol REALFLOW. Contrary to the RPL case, instead of traditional routing tables, related node lists are generated in a distributed manner, serving for packet forwarding. A controlled flooding mechanism is applied to improve both reliability and real-time performance. The authors of ____ propose a routing scheme that enhances energy consumption and end-to-end delay for large-scale industrial deployments based on IEEE 802.15.4a. The proposed scheme targets networks where data are aggregated through different clusters on their way to the sink. Moreover, a hierarchical system framework is employed to promote scalability, by estimating the residual energy and hop counts for each path. ____ contributes towards maintaining quality of service in industrial networks with a multi-path routing protocol that identifies and establishes the necessary redundant routes between any pair of nodes of a wireless network in order to satisfy the reliability and delay quality of service levels demanded by industrial applications.