\section{Related Work}
\label{sec:ref}
This section reviews key advancements in outage modeling, decision-focused learning (DFL), and grid operation optimization, with an emphasis on their practical applications in power system resilience. Despite significant progress, existing methods separate forecasting from optimization, causing inefficiencies in decision-making. This review underscores the need for a unified framework that aligns predictive models with  resilience objectives to enhance grid reliability in extreme natural hazards.

\vspace{.05in}
\noindent\emph{Grid Operation Optimization}.
Grid optimization has been widely studied to improve power system resilience, covering areas such as distributed generator (DG) placement ____, infrastructure reinforcement ____, and dynamic power scheduling ____. However, traditional approaches often follow a two-stage predict-and-mitigate paradigm—first forecasting system conditions and then optimizing responses ____. This disconnect between prediction and optimization results in suboptimal grid operations, particularly under high uncertainty, where even small forecasting errors can cause significant deviations from the optimal response ____.

% Two critical problems in this domain highlight the limitations of existing approaches. The generator deployment problem focuses on optimal DG placement and sizing to mitigate outages from extreme events ____. Traditional methods rely on optimization techniques such as genetic algorithms and particle swarm optimization to reduce power losses and enhance voltage stability ____. Similarly, the power line undergrounding problem aims to strengthen grid infrastructure by relocating transmission lines underground to withstand hazards like hurricanes and earthquakes ____. Conventional approaches emphasize static risk assessments based on historical data, lacking adaptability to evolving threats ____. While these methods improve resilience, they fall short in addressing outage dynamics, limiting their effectiveness during extreme events.

% To address these limitations, this paper proposes integrating decision-focused learning (DFL) with predictive modeling. By embedding decision-making objectives directly into the learning process, our approach aligns predictions with optimization goals, improving resource allocation and grid reinforcement strategies. This integrated approach enables more adaptive and proactive decision-making, enhancing grid resilience against escalating natural-hazard-related risks.
To overcome these limitations, we propose integrating decision-focused learning (DFL) with predictive modeling. By embedding decision objectives directly into the learning process, our approach aligns predictions with optimization goals, enabling more adaptive and proactive resource allocation and grid reinforcement. This integration enhances grid resilience amid escalating natural hazard risks.

\vspace{.05in}
\noindent\emph{Power Outage Modeling.}
Accurate power outage forecasting is crucial for enhancing grid reliability and resilience ____. Various machine learning and statistical methods have been employed to predict outages under different conditions. 
% For example, some studies have focused on forecasting outage durations using historical natural language processing (NLP) data ____. 
These approaches incorporate neural networks enriched with environmental factors and semantic analysis of field reports, providing real-time updates and enhancing predictive performance through text analysis.

Additionally, ordinary differential equations have been widely used to model dynamic systems, such as outage propagation, capturing evolving disruptions under various conditions ____. For instance, adaptations of the Susceptible-Infected-Recovered (SIR) model from epidemiology have been applied to simulate outage propagation, drawing parallels between power failures and disease spread ____.

While these models provide valuable insights, they often lack the granularity needed for city- or county-level decision-making, limiting their practical application to localized resilience planning. By integrating local weather forecasts and socio-economic data into compartmental neural ODE models, our approach offers forecasting of local outage dynamics, enabling more targeted and effective interventions.

\vspace{.05in}
\noindent\emph{Decision-Focused Learning}.
Decision-focused learning (DFL) integrates predictive machine learning models with optimization, aligning training objectives with decision-making rather than purely maximizing predictive accuracy. Unlike traditional two-stage approaches, where predictions are first generated and then used as inputs for optimization, DFL enables end-to-end learning by backpropagating gradients through the optimization process. This is achieved via implicit differentiation of optimality conditions such as KKT constraints ____ or fixed-point methods ____. For nondifferentiable optimizations, approximation techniques such as surrogate loss functions ____, finite differences ____, and noise perturbations ____ have been developed. Recent work has also explored integrating differential equation constraints directly into optimization models, enabling end-to-end gradient-based learning while ensuring compliance with system dynamic constraints ____.

A well-studied class of DFL problems involves linear programs (LPs), where the Smart Predict-and-Optimize (SPO) framework ____ introduced a convex upper bound for gradient estimation, enabling cost-sensitive learning for optimization. Subsequent work has extended DFL to combinatorial settings, including mixed-integer programs (MIPs), using LP relaxations ____. 
Recent advances, such as decision-focused generative learning (Gen-DFL) ____, tackle the challenge of applying DFL in high-dimensional setting by using generative models to adaptively model uncertainty.

\vspace{.05in}
\noindent\emph{Differentiable Optimization}.
A key enabler of DFL is differentiable optimization (DO), which facilitates gradient propagation through differentiable optimization problems, aligning predictive models with decision-making objectives ____. Recent advances extend DO to distributionally robust optimization (DRO) for handling uncertainty in worst-case scenarios, improving decision quality under data scarcity ____. Beyond predictive modeling, DO has advanced combinatorial and nonlinear optimization through implicit differentiation of KKT conditions ____, fixed-point methods ____, and gradient approximations via noise perturbation ____ and smoothing ____. These techniques bridge forecasting with optimization, ensuring decision-aware learning. In this work, DO enables the backpropagation of resilience strategy losses, aligning the spatio-temporal outage prediction model with grid optimization objectives.