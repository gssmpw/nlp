\documentclass[lettersize,journal]{IEEEtran}
\usepackage[dvipsnames]{xcolor}
% \newcommand*{\etc}{\emph{etc}{}}
% \usepackage{cite}
% \usepackage{amsmath,amssymb,amsfonts}
% \usepackage{algorithmic}
% \usepackage{graphicx}
% \usepackage{textcomp}
% \usepackage{xcolor}
% \usepackage{bm}
% \usepackage[flushleft]{threeparttable}
% \usepackage{multirow}
% \usepackage{algorithmic}
% \usepackage{booktabs}   
% \usepackage{bbm}
% \usepackage[caption=false,font=normalsize,labelfont=sf,textfont=sf]{subfig}
% % \usepackage{natbib}
% \usepackage{hyperref}   % For hyperlinks

% \newcommand{\todo}[1]{\textcolor{red}{\textsf{TODO: #1}}}
% \newcommand{\woody}[1]{\textcolor{blue}{Woody: #1}}
% \newcommand{\nando}[1]{\textcolor{red}{Nando: #1}}
% \newcommand{\james}[1]{\textcolor{orange}{James: #1}}
% \newcommand{\ryan}[1]{\textcolor{blue}{Ryan: #1}}

% \IEEEoverridecommandlockouts
% The preceding line is only needed to identify funding in the first footnote. If that is unneeded, please comment it out.
\usepackage{cite}
\usepackage{amsmath,amssymb,amsfonts}
\usepackage{algorithmic}
\usepackage{algorithm}
\usepackage{graphicx}
% \usepackage[caption=false]{subfig}
\usepackage{textcomp}
\usepackage{xcolor}
\usepackage{bm}
\usepackage[flushleft]{threeparttable}
\usepackage{multirow}
\usepackage{algorithmic}
\usepackage{booktabs}   
\usepackage{bbm}
\usepackage{subfigure}
\usepackage{enumitem}
\usepackage{mathtools}
% \usepackage{natbib}
\usepackage{hyperref}   % For hyperlinks
\usepackage{subcaption} % Make sure this package is loaded
\newcommand*{\eg}{\emph{e.g.}{}}
\newcommand*{\ie}{\emph{i.e.}{}}
\newcommand*{\etc}{\emph{etc}{}}
\renewcommand{\d}{\mathrm{d}}
% \newcommand\harvardurl[1]{\textbf{URL:} \textit{#1}}
\usepackage{url}
\def\UrlBreaks{\do\/\do-}
\newtheorem{prop}{Proposition}
\newtheorem{lemma}{Lemma}
\newtheorem{thm}{Theorem}
\newtheorem{cor}{Corollary}
\newtheorem{rmk}{Remark}

\def\BibTeX{{\rm B\kern-.05em{\sc i\kern-.025em b}\kern-.08em
    T\kern-.1667em\lower.7ex\hbox{E}\kern-.125emX}}
    
\newcommand{\todo}[1]{\textcolor{red}{\textsf{TODO: #1}}}
\newcommand{\woody}[1]{\textcolor{red}{Woody: #1}}
\newcommand{\nando}[1]{\textcolor{purple}{[Nando: #1]}}
\newcommand{\james}[1]{\textcolor{orange}{James: #1}}
\newcommand{\ryan}[1]{\textcolor{blue}{Ryan: #1}}


\makeatletter
\newcommand{\linebreakand}{%
  \end{@IEEEauthorhalign}
  \hfill\mbox{}\par
  \mbox{}\hfill\begin{@IEEEauthorhalign}
}
\makeatother


\hyphenation{op-tical net-works semi-conduc-tor IEEE-Xplore}
% updated with editorial comments 8/9/2021

\begin{document}

\title{
Global-Decision-Focused Neural ODEs for\\Proactive Grid Resilience Management
% Decision-Focused Neural ODE for Power Outage
% Prediction and Resilience Planning
}
% \author{Shuyi Chen,~\IEEEmembership{Student Member,~IEEE,} Ferdinando Fioretto, Feng Qiu,~\IEEEmembership{Senior Member,~IEEE,} Shixiang Zhu
% }
        % <-this % stops a space
% \thanks{This paper was produced by the IEEE Publication Technology Group. They are in Piscataway, NJ.}% <-this % stops a space
% \thanks{Manuscript received April 19, 2021; revised August 16, 2021.}

\author{Shuyi Chen,
        Ferdinando Fioretto,
        Feng Qiu, and
        Shixiang Zhu
}

% The paper headers
\markboth{}%
{Shell \MakeLowercase{\textit{et al.}}: A Sample Article Using IEEEtran.cls for IEEE Journals}

% \IEEEpubid{0000--0000/00\$00.00~\copyright~2021 IEEE}
% Remember, if you use this you must call \IEEEpubidadjcol in the second
% column for its text to clear the IEEEpubid mark.

\maketitle

\begin{abstract}
Extreme hazard events such as wildfires and hurricanes increasingly threaten power systems, causing widespread outages and disrupting critical services.
Recently, predict-then-optimize approaches have gained traction in grid operations, where system functionality forecasts are first generated and then used as inputs for downstream decision-making. However, this two-stage method often results in a misalignment between prediction and optimization objectives, leading to suboptimal resource allocation.
% \nando{The reader may not know what is the predict-then-optimize framewrok at this point} 
To address this, we propose predict-all-then-optimize-globally (PATOG), a framework that integrates outage prediction with globally optimized interventions. At its core, our global-decision-focused (\texttt{GDF}) neural ODE model captures outage dynamics while optimizing resilience strategies in a decision-aware manner. Unlike conventional methods, our approach ensures spatially and temporally coherent decision-making, improving both predictive accuracy and operational efficiency. Experiments on synthetic and real-world datasets demonstrate significant improvements in outage prediction consistency and grid resilience.
\end{abstract}

\begin{IEEEkeywords} 
Outage prediction, proactive decision making, neural ODEs, decision-focused learning, 
\end{IEEEkeywords}

\section{Introduction}

% Similarly, in wildfire-prone areas, preemptive power shutoffs are often used to prevent electrical infrastructure from igniting fires. 
% If forecasts fail to consistently capture fire risks across multiple grid segments, an overly conservative approach may unnecessarily cut power to low-risk areas, causing economic losses, while a failure to shut off power in high-risk areas can lead to devastating consequences.

% In recent years, the increasing frequency and severity of natural hazards have posed significant threats to power systems worldwide \cite{Handmer2012, Kenward2014} 
% \nando{this first sentence can be removed and you can start directly with the second one}. 
Extreme hazard events such as wildfires, winter storms, hurricanes, and earthquakes can trigger widespread power outages, disrupting economic activity, threatening public safety, and complicating the delivery of critical services \cite{Handmer2012, Kenward2014}. 
For instance, the January 2025 wildfires in Los Angeles County, fueled by strong Santa Ana winds, led to the destruction of critical power infrastructure across vast areas, resulting in prolonged outages for over 400,000 customers \cite{reuters2024hurricane}. 
Similarly, Hurricane Milton in October 2024 made landfall in Florida, bringing extreme winds and flooding that damaged the electrical grid and left more than 3 million homes and businesses without power for weeks \cite{reuters2025wildfires}, illustrating the catastrophic consequences of extreme weather on energy resilience.


\begin{figure}[!t]
% \begin{wrapfigure}{r}{0.5\textwidth} % Adjust the alignment and width as needed
        \centering
        % \includegraphics[width=0.45\textwidth]{new_plots/MA_total_outage.pdf}
        \includegraphics[width=\columnwidth]{new_plots/MA_total_outage_combined.pdf}
        \caption{The number of outaged customers and the meteorological factors, including wind speed and turbulent kinetic energy.
        }
        \label{fig:vis}
% \end{wrapfigure}
% \vspace{-.1in}
\end{figure}

A fundamental challenge in strengthening power system resilience lies in the immense difficulty of restoring electricity after a severe event has caused widespread damage. The recent Los Angeles wildfires highlight how extreme winds can overwhelm firefighting efforts, making containment impossible even with abundant resources, let alone the rapidly repairing damaged transmission lines \cite{CNN2025Wildfires}.
The aftermath of major storms presents similar challenges -- damaged infrastructure can take weeks or even months to restore, leading to severe economic and social repercussions. However, the impact of such events can be significantly mitigated, and in some cases, entirely avoided through proactive resilience planning. For example, preemptive de-energization strategies have been successfully implemented in California to reduce wildfire risks \cite{wiki:2019_CA_power_shutoffs}, while grid hardening and strategic upgrades have improved system resilience in hurricane-prone regions \cite{HoustonChronicle2025Grid}.

To proactively mitigate power grid disruptions, it is crucial to anticipate the long-term impact of extreme events on power systems and implement preemptive measures accordingly. Accurate forecasting enables utility companies and policymakers to assess risks, prioritize interventions, optimize grid reinforcement, and allocate resources efficiently. By identifying vulnerable regions in advance, decision makers can develop strategies that not only reduce power disruptions but also strengthen overall grid resilience.

A widely used approach in such planning is the predict-then-optimize (PTO) framework, where forecasts inform downstream decisions \cite{elmachtoub_smart_2022}.
However, conventional PTO approaches often fall short in high-stake decision making problems \cite{mandi_decision-focused_2024}. A high-quality forecast does not necessarily translate into effective decision-making, as small errors in predictive models can propagate into significant inefficiencies in downstream optimization tasks \cite{fernandez2022causal, mandi_decision-focused_2024}. This issue arises because conventional models are trained independently of decision-making objectives, meaning they may prioritize minimizing forecasting errors without considering how those errors impact critical decisions. As a result, even small deviations in predictions can lead to costly misallocations of resources, delayed responses, and suboptimal grid resilience strategies \cite{10855832,10711393}. Addressing this gap requires a more integrated approach that directly aligns predictive modeling with decision-making objectives, ensuring that forecasts are not only accurate but also actionable for proactive grid management.

% \woody{Illustrate a toy example using a figure to demonstrate the challenge mentioned below here. (5th paragraph)}

This challenge becomes even more pronounced in system-wide decision-making, where resilience planning depends on aggregating independent predictions across multiple service units. 
Ensuring consistency across these independent predictions is inherently challenging, as errors in any individual unit’s forecast can propagate and compound, ultimately undermining the global decision quality. 
For instance, consider a utility company preparing for an approaching hurricane: if outage risks are underestimated in one region and overestimated in another, resources such as backup generators, repair crews, and grid reinforcements may be misallocated. Low-risk areas may be overprepared, while high-risk areas may lack sufficient resources, worsening service disruptions. 
These inconsistencies underscore the critical need for a holistic, system-wide approach that integrates predictions for multiple units with a global decision-making objective, ensuring coherent and globally optimized resilience strategies rather than fragmented, unit-specific responses. 


To address the challenges of proactive resilience planning, we first propose
% \nando{this sentence below does not add anything IMO to the reader's understanding; you could start directly with "a new decision-making paradigm, [...]}
% \textcolor{gray}{a novel framework that directly optimizes predictive models with respect to system-wide decision objectives, ensuring that local inaccuracies do not propagate into globally suboptimal resilience strategies.
% At the core of our approach is} 
a new decision-making paradigm, referred to as predict-all-then-optimize-globally (PATOG), which integrates predictive outage modeling across all service units with a global decision-making process for grid resilience enhancement. 
To solve PATOG problems, we then introduce a global-decision-focused (\texttt{GDF}) neural ordinary differential equation (ODE) model, which simultaneously predicts the long-term evolution of system functionality (quantified by the number of power outages) and optimizes global resilience decisions. Inspired by epidemiological models, our method conceptualizes power outage progression within each service unit as a dynamical system, where the failure and restoration processes evolve based on transmission line conditions, local weather, and socioeconomic factors. These transition dynamics are parameterized using neural networks, allowing the model to adapt to different hazard scenarios and geographic conditions. 
Instead of treating each service unit independently, our method holistically forecasts the outage evolution across the power grid and derives a globally coordinated intervention strategy. This enables utilities to allocate resources more effectively, mitigating service disruptions in a spatially and temporally coherent manner.
% \ryan{TBD: here might make reader think it's a real-time model which integrates the real time weather condition, but actually we predicting all. Is this a good time to mention the assumption that data collection is slow and impossible in real-time for hazards?}

% \woody{Introduce the dataset in the following paragraph and cite the data figure here. \ref{fig:vis}}

To assess our framework, we conduct experiments on both real-world and synthetic datasets. The real dataset captures outage events from a 2018 Nor’easter in Massachusetts, combining county-level outage reports \cite{maema2020power} with meteorological data from NOAA’s HRRR model \cite{noaa_hrrr} and socioeconomic indicators from the U.S. Census Bureau \cite{USCensusBureau2017ACS} (Fig.~\ref{fig:vis}). The synthetic dataset simulates real outage propagation using a simplified SIR model, enabling controlled evaluation under varying weather conditions and grid configurations. These datasets provide a robust testbed for assessing \texttt{GDF} in realistic resilience planning scenarios.
% We evaluate our framework through extensive numerical experiments on two grid resilience management problems, utilizing both synthetic and real-world datasets. 
Our results demonstrate significant improvements in forecast consistency, decision efficiency, and overall grid resilience, showcasing the potential of our approach to enable smarter, more proactive energy infrastructure management in the face of increasing hazard-induced disruptions.

% \woody{summarize managerial insights here.}
% Our research delivers practical insights for enhancing grid resilience, highlighting the critical role of proactive scheduling and early intervention in mitigating the impact of natural hazards. By embedding global decision-focused objectives into outage forecasts, the proposed \texttt{GDF} framework enhances system-wide planning and enables more effective, timely resource allocation. This approach is demonsdtrated to be particularly valuable in contexts where transportation costs or logistical delays are significant, empowering operators to prioritize and deploy limited resources where they are needed most. Ultimately, our method reduces costs, lowers regret, and enhances the grid’s capacity to withstand and recover from extreme events.

We summarize our main contributions as follows: 
\begin{itemize}[left=0pt,parsep=0pt,itemsep=0pt]
    \item Propose a new predict-all-then-optimize-globally (PATOG) paradigm for system-wide decision-making problems in grid resilience management;
    \item Develop a novel method, referred to as global-decision-focused (\texttt{GDF}) neural ODEs, for solving PATOG problems;
    \item Demonstrate that \texttt{GDF} neural ODEs outperform baseline methods across multiple grid resilience management tasks through experiments on synthetic data.
    \item Evaluate our approach on a unique real-world customer-level outage dataset, uncovering key insights that inform more effective grid resilience strategies.
\end{itemize}

% The paper is organized as follows: Section~\ref{sec:ref} provides a review of related work on outage prediction, grid optimization, and decision-focused learning. Section~\ref{sec:patog} formalizes the PATOG framework for decision-making in grid resilience management. Section~\ref{sec:gdf} introduces the \texttt{GDF} framework, leveraging Neural ODEs for spatio-temporal outage modeling. Section~\ref{sec:mobile} presents a case study on the mobile generator deployment problem. Section~\ref{sec:exp} evaluates the proposed framework using both synthetic and real-world datasets. Finally, Section~\ref{sec:conclusion} summarizes the key findings and discusses policy implications.


\subsection{Related Work}
\label{sec:ref}
This section reviews key advancements in outage modeling, decision-focused learning (DFL), and grid operation optimization, with an emphasis on their practical applications in power system resilience. Despite significant progress, existing methods separate forecasting from optimization, causing inefficiencies in decision-making. This review underscores the need for a unified framework that aligns predictive models with  resilience objectives to enhance grid reliability in extreme natural hazards.

\vspace{.05in}
\noindent\emph{Grid Operation Optimization}.
Grid optimization has been widely studied to improve power system resilience, covering areas such as distributed generator (DG) placement \cite{8668633,QIN2024752,9117454}, infrastructure reinforcement \cite{8273985}, and dynamic power scheduling \cite{5963580,990600}. However, traditional approaches often follow a two-stage predict-and-mitigate paradigm—first forecasting system conditions and then optimizing responses \cite{7080837}. This disconnect between prediction and optimization results in suboptimal grid operations, particularly under high uncertainty, where even small forecasting errors can cause significant deviations from the optimal response \cite{10711393,10855832}.

% Two critical problems in this domain highlight the limitations of existing approaches. The generator deployment problem focuses on optimal DG placement and sizing to mitigate outages from extreme events \cite{QIN2024752,9117454,8668633}. Traditional methods rely on optimization techniques such as genetic algorithms and particle swarm optimization to reduce power losses and enhance voltage stability \cite{9117454}. Similarly, the power line undergrounding problem aims to strengthen grid infrastructure by relocating transmission lines underground to withstand hazards like hurricanes and earthquakes \cite{8278121,7922545}. Conventional approaches emphasize static risk assessments based on historical data, lacking adaptability to evolving threats \cite{AbiSamra2013,Shea2018}. While these methods improve resilience, they fall short in addressing outage dynamics, limiting their effectiveness during extreme events.

% To address these limitations, this paper proposes integrating decision-focused learning (DFL) with predictive modeling. By embedding decision-making objectives directly into the learning process, our approach aligns predictions with optimization goals, improving resource allocation and grid reinforcement strategies. This integrated approach enables more adaptive and proactive decision-making, enhancing grid resilience against escalating natural-hazard-related risks.
To overcome these limitations, we propose integrating decision-focused learning (DFL) with predictive modeling. By embedding decision objectives directly into the learning process, our approach aligns predictions with optimization goals, enabling more adaptive and proactive resource allocation and grid reinforcement. This integration enhances grid resilience amid escalating natural hazard risks.

\vspace{.05in}
\noindent\emph{Power Outage Modeling.}
Accurate power outage forecasting is crucial for enhancing grid reliability and resilience \cite{9160513, 7752978}. Various machine learning and statistical methods have been employed to predict outages under different conditions. 
% For example, some studies have focused on forecasting outage durations using historical natural language processing (NLP) data \cite{su12041525}. 
These approaches incorporate neural networks enriched with environmental factors and semantic analysis of field reports, providing real-time updates and enhancing predictive performance through text analysis.

Additionally, ordinary differential equations have been widely used to model dynamic systems, such as outage propagation, capturing evolving disruptions under various conditions \cite{zhang2024recurrent,chen2018neural}. For instance, adaptations of the Susceptible-Infected-Recovered (SIR) model from epidemiology have been applied to simulate outage propagation, drawing parallels between power failures and disease spread \cite{9646114}.

While these models provide valuable insights, they often lack the granularity needed for city- or county-level decision-making, limiting their practical application to localized resilience planning. By integrating local weather forecasts and socio-economic data into compartmental neural ODE models, our approach offers forecasting of local outage dynamics, enabling more targeted and effective interventions.

\vspace{.05in}
\noindent\emph{Decision-Focused Learning}.
Decision-focused learning (DFL) integrates predictive machine learning models with optimization, aligning training objectives with decision-making rather than purely maximizing predictive accuracy. Unlike traditional two-stage approaches, where predictions are first generated and then used as inputs for optimization, DFL enables end-to-end learning by backpropagating gradients through the optimization process. This is achieved via implicit differentiation of optimality conditions such as KKT constraints \cite{amos2017optnet, gould2021deep} or fixed-point methods \cite{kotary2023folded, wilder2019end}. For nondifferentiable optimizations, approximation techniques such as surrogate loss functions \cite{elmachtoub_smart_2022}, finite differences \cite{vlastelica2019differentiation}, and noise perturbations \cite{berthet2020learning} have been developed. Recent work has also explored integrating differential equation constraints directly into optimization models, enabling end-to-end gradient-based learning while ensuring compliance with system dynamic constraints \cite{Fioretto:ICLR25, jacquillat2024branch}.

A well-studied class of DFL problems involves linear programs (LPs), where the Smart Predict-and-Optimize (SPO) framework \cite{elmachtoub_smart_2022} introduced a convex upper bound for gradient estimation, enabling cost-sensitive learning for optimization. Subsequent work has extended DFL to combinatorial settings, including mixed-integer programs (MIPs), using LP relaxations \cite{mandi2020smart, wilder2019melding}. 
Recent advances, such as decision-focused generative learning (Gen-DFL) \cite{wang2025gendfldecisionfocusedgenerativelearning}, tackle the challenge of applying DFL in high-dimensional setting by using generative models to adaptively model uncertainty.

\vspace{.05in}
\noindent\emph{Differentiable Optimization}.
A key enabler of DFL is differentiable optimization (DO), which facilitates gradient propagation through differentiable optimization problems, aligning predictive models with decision-making objectives \cite{cvx}. Recent advances extend DO to distributionally robust optimization (DRO) for handling uncertainty in worst-case scenarios, improving decision quality under data scarcity \cite{zhu2022distributionally, chen2025uncertainty}. Beyond predictive modeling, DO has advanced combinatorial and nonlinear optimization through implicit differentiation of KKT conditions \cite{amos2017optnet}, fixed-point methods \cite{kotary2023folded}, and gradient approximations via noise perturbation \cite{berthet2020learning} and smoothing \cite{vlastelica2019differentiation}. These techniques bridge forecasting with optimization, ensuring decision-aware learning. In this work, DO enables the backpropagation of resilience strategy losses, aligning the spatio-temporal outage prediction model with grid optimization objectives.

\section{Predict All Then Optimize Globally}
\label{sec:patog}
In this section, we first provide an overview of decision-focused learning (DFL) and its application to solving predict-then-optimize (PTO) problems. We then introduce a new class of decision-making problems termed \emph{predict-all-then-optimize-globally} (PATOG). 
Unlike traditional PTO approaches, which generate instantaneous or overly aggregated forecasts and optimize decisions independently, PATOG explicitly accounts for how predictions evolve over time and space, integrating them into a single, system-wide optimization framework.

PATOG is particularly useful for grid resilience management, where decisions must consider complex interactions across all service units. A key example is the mobile generator distribution problem. In a conventional PTO setting, potential damage from an extreme weather event is first forecasted for each unit independently. Decisions, such as scheduling power generator deployments, are then made in isolation, without considering the evolving conditions of other units. This localized approach often leads to resource misallocation and suboptimal resilience outcomes.
In contrast, PATOG embeds these interdependencies into a global optimization problem, enabling system-wide decision-making that improves predictive models by incorporating cross-unit interactions. This results in more robust and effective resilience planning.

% \nando{I don't really understand this point. Can you elaborate? I don't think that the ability to proactive forecast + optimization should be contrasted with predicting individual units -- these are two different scales (time vs size). I think we need to make the differences of PATOG w.r.t. traditional PTO and DFL much clearer here.}

% \nando{The way we are selling this is a bit strange. I think the key novelty is not in the fact that we can handle proactive optimization, this is something that you could address with an increase in problem size, even in classic PTO, but that these predictions depends on system dynamics and that they interact with the decisions in non trivial ways so that modeling the predictions and decisions dijointly would bring to substantial suboptimalities. This also addresses the novelty with respect to classical DFL approaches, that do not model system dynamics. }

\subsection{Preliminaries: Decision-Focused Learning}

The predict-then-optimize (PTO) has been extensively studied across a wide range of applications \cite{7080837, mandi_decision-focused_2024}.  
It follows a two-step process: First, predicting the unknown parameters $\mathbf{c}$ using a model $f_\theta$ based on the input $\boldsymbol{z}$, denoted as $\hat{\boldsymbol{c}} \coloneqq f_\theta(\boldsymbol{z})$. Second, solving an optimization problem: 
\begin{equation}
    \boldsymbol{x}^*(\hat{\mathbf{c}}) = \arg \min_{\boldsymbol{x}} g(\boldsymbol{x}, \hat{\mathbf{c}}),
    \label{eq:pto}
\end{equation}
where $g$ is the objective function and $\boldsymbol{x}^*(\hat{\mathbf{c}})$ represents the optimal decision given the predicted parameters. 
This framework has numerous practical applications in power grid operations. 
For example, PTO is used in power grid operations to predict system stress using synchrophasor data, optimize outage management by forecasting disruptions and proactively dispatching restoration crews, and enhance renewable integration by predicting fluctuations in wind and solar generation to improve scheduling and grid balancing \cite{7080837}.
These predictive insights enable system operators to make informed strategic decisions, enhancing grid reliability and resilience.

However, the conventional two-stage approach does not always lead to high-quality decisions. 
In this approach, the model parameter $\theta$ is first trained to minimize a predictive loss, such as mean squared error. Then the predicted parameters $\hat{\mathbf{c}}$ are used to solve the downstream optimization. 
This separation between prediction and optimization can result in suboptimal decisions, as the prediction model is not directly optimized for decision quality \cite{kotary2021end}. 

To address this limitation, the decision-focused learning (DFL) integrates prediction with the downstream optimization process \cite{mandi_decision-focused_2024}.
Instead of optimizing for predictive accuracy, DFL trains the model parameter $\theta$ by directly minimizing the \emph{decision regret} \cite{mandi_decision-focused_2024}:
\[
\theta^* = \arg \min_{\theta} \mathbb{E} \left[ g(\boldsymbol{x}^*(f_\theta(\boldsymbol{z})), \mathbf{c}) - g(\boldsymbol{x}^*(\mathbf{c}), \mathbf{c}) \right].
\]
This approach ensures that the model is learned with the ultimate goal of improving decision quality, making it particularly effective for PTO problems.
Note that we assume that the constraints on decision variable $\boldsymbol{x}$ are fully known and do not depend on the uncertain parameters $\mathbf{c}$ in this study. This assumption simplifies the problem by ensuring that all feasible solutions $\boldsymbol{x}$ remain valid regardless of the parameter estimates.


\subsection{Proposed PATOG Framework}

The objective of PATOG in this work is to develop proactive global recourse actions that enable system operators to better prepare for natural hazards. 
These actions may include preemptive dispatch of mobile generators, strategic load shedding, grid reconfiguration, or reinforcement of critical infrastructure.
The PATOG consists of two steps:
($i$) Predicting the temporal evolution of unit functionality across all the service units in the network throughout the duration of a hazard event. 
($ii$) Deriving system-wide strategies that minimize overall loss based on all the predictions, enabling optimized resource allocation by anticipating critical failures before they occur.

Consider a power network consisting of $K$ geographical units, where each unit $k$ serves $N_k$ customers.
We define the global recourse actions as $\boldsymbol{x} \coloneqq \{x_k\}_{k=1}^K$, where $x_k$ represents the action taken for unit $k$. 
A key challenge in designing effective actions is understanding how the system will respond to  an impending hazard event. 
To this end, we use the number of customer power outages, which is publicly accessible via utility websites, as a measure of system functionality \cite{7080837}. 

To model the outage dynamics, we represent the outage state of each unit $k$ using a dynamical system over the time horizon $[0, T]$ during a hazard event.
During the event, the state of each unit $k$ at time $t$ is represented by three quantities: 
\begin{itemize}[left=0pt,itemsep=0pt]
    \item $Y_k(t) \in \mathbb{Z}_*$: the number of customers experiencing outages;
    \item $U_k(t) \in \mathbb{Z}_*$: the number of unaffected customers;
    \item $R_k(t) \in \mathbb{Z}_*$: the number of recovered customers.
\end{itemize}
The total number of customers, $N_k$, in the unit $k$ remains constant throughout the event, satisfying the following constraint: 
\[
    Y_k(t) + U_k(t) + R_k(t) = N_k, ~\forall t \in [0, T].
\]
For compact representation, we define the state vector for unit $k$ as $\mathbf{S}_k(t) \coloneqq [U_k(t), R_k(t), Y_k(t)]^\top$. 
To simplify notation, we collectively represent the outage states across all units over the time horizon as $\mathbf{S} \coloneqq \{\mathbf{S}_k(t) \mid t \in [0, T] \}_{k=1}^K$.

Formally, we are tasked to search for the optimal action:
\begin{align}
    \boldsymbol{x}^*(\mathbf{S}) = &~\arg \min_{\boldsymbol{x}}~g(\boldsymbol{x}, \mathbf{S}) \label{eq:opt}\\
    \text{s.t.} \quad \frac{\d \mathbf{S}_k(t)}{\d t} = &~ f_\theta(\mathbf{S}_k(t), \boldsymbol{z}_k),~\forall k, \label{eq:ode}\\
    \mathbf{S}_k(0) = &~ [N_k, 0, 0]^\top,~\forall k, \label{eq:init-ode}
\end{align}
where $g$ quantifies the decision loss of the action $\boldsymbol{x}$ based on the predicted future evolution of outage states $\mathbf{S}$. 
The transition function $f_\theta$ models the progression of unaffected, recovered, and outaged customers in each unit over time, influenced by both local weather conditions and socioeconomic factors, jointly represented as covariates $\boldsymbol{z}_k \in \mathbb{R}^p$.
We note that most power outages during extreme weather stem from localized transmission line damage, with cascading failures being rare \cite{zhu2021quantifying,10479971}. Thus, we assume each unit evolves independently under its local conditions.

We emphasize that \eqref{eq:opt} extends the traditional PTO framework by integrating predictions across multiple units over an extended future horizon to derive a single, globally optimized solution. Unlike PTO, which focuses only on instantaneous or localized dynamics, PATOG captures both temporal and spatial outage states while modeling how decisions influence outage dynamics across the entire system. 
This comprehensive approach enables proactive, system-wide high-resolution resilience strategies that adapt to evolving conditions. 
% \nando{same considerations as above and we should also emphasize the differences with respect to DFL.}

\section{Global-Decision-Focused Neural ODEs}
\label{sec:gdf}
This section presents a novel decision-focused neural ordinary differential equations (ODE) model tailored for solving PATOG problems in grid resilience management. The proposed neural ODE model predicts outage progression at the unit level while simultaneously optimizing global operational decisions by learning model parameters in a decision-aware manner. We refer to this approach as global-decision-focused (\texttt{GDF}) Neural ODEs.
Fig.~\ref{fig:arc} provides an overview of the proposed framework.

\begin{figure}[!t] \centering \includegraphics[width=\linewidth]{new_plots/model_structure_conference.pdf} 
\caption{Overview of the proposed \texttt{GDF} framework. Given covariates $\boldsymbol{z}_k$ and initial states $\mathbf{S}_k(0)$ for all $K$ service units, a model parameterized by $\theta$ predicts the system states $\hat{\mathbf{S}}_k$ for all units. These predictions inform global decision-making, where optimal actions $\boldsymbol{x}^*(\hat{\mathbf{S}})$ minimize the global decision loss $g(\boldsymbol{x},\hat{\mathbf{S}})$. The framework optimizes $\theta$ by minimizing a global-decision-focused loss, regularized by a prediction-focused loss (\eg, MSE loss) to enhance predictive interpretability. Red arrows denote the backpropagation through $\nabla \ell_\texttt{GDF}(\theta)$, ensuring that the model learns both system-level decision quality and region-specific prediction accuracy. 
}
\label{fig:arc}
% \vspace{-.1in}
\end{figure}

\subsection{Neural ODEs for Power Outages}

Assume that we have observations of $I$ natural hazards (\eg, hurricanes, and winter storms) in the history. For the $i$-th event, the observations in unit $k$ are represented by a data tuple, denoted by $(\boldsymbol{z}_k^i, \{y_k^i(t)\})$, where $\boldsymbol{z}_k^i$ represents the covariates for unit $k$ during the $i$-th event and $y_k^i(t)$ is the number of customers experiencing power outages at time $t$. 
The outage trajectory $\{y_k^i(t) \mid t\in[0,T^i]\}$ is recorded at $15$-minute intervals.
A significant challenge in modeling outage dynamics is the lack of detailed observations for the underlying failure and restoration processes. 
Specifically, while $y_k^i(t)$ provides the number of customers experiencing outages at time $t$, we do not directly observe the failure and restoration states, $U_k(t)$ and $R_k(t)$, respectively. 

Drawing inspiration from the Susceptible Infectious Recovered (SIR) models commonly used in epidemic modeling \cite{kosma2023neural}, we conceptualize power outages within a unit as the spread of a ``virus''. 
In this analogy, outages propagate among customers due to local transmission line failures, while restorations provide lasting resistance to subsequent outages.
We formalize this analogy with the following three assumptions:
($i$) The number of unaffected customers $U_k(t)$ decreases over time as some customers transition from being unaffected to experiencing outages. 
This transition is governed by the \emph{failure transmission rate}, denoted by $\phi(\boldsymbol{z}_k; \theta_U)$.
% , which represents the probability of an outage per contact multiplied by the rate of such contacts, depending on its connectivity in the grid network.
Inspired by epidemiological transmission rates, this rate quantifies how local conditions -- such as weather patterns and other regional factors encapsulated by covariates $\boldsymbol{z}_k$ -- influence the rate at which outages spread within the grid.
($ii$) Conversely, the number of customers with restored power ($R_k(t)$) gradually increases as the system operator repairs transmission lines and restores service. 
This process is captured by the \emph{restoration rate}, denoted $\phi(\boldsymbol{z}_k; \theta_R)$.
Both transmission and restoration rates, $\phi(\boldsymbol{z}_k; \theta_U)$ and $\phi(\boldsymbol{z}_k; \theta_R)$, are functions of the local covariates $\boldsymbol{z}_k$, and are modeled using deep neural networks.
($iii$) The total number of customers within each unit remains constant throughout the studied period. 
% \ryan{i know assumption 3 is mentioned before in PATGO, but for completeness for formulation I think we could mention again here.}
Based on these assumptions, we model the outage state transition for each unit $k$ in \eqref{eq:ode}, \ie, $\d \mathbf{S}_k(t) / \d t$, as follows:
\begin{equation}
\begin{cases}
\begin{aligned}
    \d U_k(t) / \d t &= -\phi(\boldsymbol{z}_k; \theta_U) Y_k(t) U_k(t), \\
    \d R_k(t) / \d t &= \phi(\boldsymbol{z}_k; \theta_R) Y_k(t), \\
    \d Y_k(t) / \d t &= -\d U_k(t) / \d t - \d R_k(t) / \d t.
\end{aligned}
\end{cases}
\label{eq:deterministic sys}
\end{equation}
These ODEs capture the dynamic evolution of power outage states within each unit $k$. For notational simplicity, we use $\theta \coloneqq \{\theta_U, \theta_R\}$ to denote their parameters jointly. 

In practice, we work with discrete observations $\{t_j\}_{j=0}^T$ and adopt the Euler method to approximate the solution to the ODE model \cite{chen2018neural}, \ie, 
\[
    \mathbf{S}_k(t_{j+1}) = \mathbf{S}_k(t_j) +  f_\theta(\mathbf{S}_k(t_j), \boldsymbol{z}_k) \Delta t_j,\quad\forall k,
\]
where $\mathbf{S}_k(t_{j+1})$ and $\mathbf{S}_k(t_{j})$ are the history states at times $t_{j+1}$ and $t_j$, respectively, 
and $\Delta t_j \coloneqq t_{j+1} - t_j$ is time interval.

\subsection{Global-Decision-Focused Learning}

The training loss for \texttt{GDF} Neural ODEs is formulated as the aggregated regret across all unit $k$, with an additional regularization term to ensure stability in prediction:
\begin{equation} 
\begin{aligned}
\ell_\texttt{GDF}(\theta) \coloneqq 
&~ \frac{1}{I} \sum_{i} \left [ g\Big(\boldsymbol{x}^*(\hat{\mathbf{S}}^i), {\mathbf{S}}^i \Big) -g\Big(\boldsymbol{x}^*({\mathbf{S}}^i), \mathbf{S}^i \Big) \right ] +\\
& ~ \lambda \cdot \frac{1}{IKT} \sum_{i,k,j} \left[ y_k^i(t_j) - \hat{Y}_k^i(t_j) \right]^2,
\end{aligned}
\label{eq:opt_dfl_ERM}
\end{equation}
where ${\hat{\mathbf{S}}^i}$ is the predicted outages states across all units for event $i$.

The objective function \eqref{eq:opt_dfl_ERM} consists of two key components:
($i$) Global-decision-focused loss (first term): This term evaluates the quality of the optimal action based on predictions, capturing the impact of prediction errors on operational decisions. The loss and its gradient are computed over all geographical units affected by the event, ensuring that learning is guided by system-wide decision quality. However, this loss alone does not provide direct insights into the structure of outage trajectories, which may limit the interpretability of the learned model.
($ii$) Prediction-focused loss (second term): To address this limitation, a prediction-based penalty is introduced to minimize discrepancies between observed outage trajectories $y^i_k(t)$ and predicted values $\hat{Y}^i_k(t)$. This term refines the model's ability to capture outage dynamics without explicitly observing the failure and restoration processes.
A user-specified hyperparameter $\lambda$ governs the trade-off between prediction accuracy and the regret associated with suboptimal operational decisions.

The most salient feature of the proposed \texttt{GDF} neural ODEs method is its incorporation of both global and local perspectives. The global-decision-focused loss aggregates errors across all geographical regions and time steps, directly linking prediction quality to system-wide resilience measures and operational strategies (\eg, resource dispatch, outage management, and service restoration). Meanwhile, the prediction-focused component refines local accuracy by penalizing deviations at each service unit. 
By incorporating predictive regularization, the model empirically improves generalization to new events with unknown distributional shifts, mitigating overfitting, particularly given the limited availability of extreme event data. 
% \nando{I don't know whether this sentence below adds anything, especially when it tries to allude to a comparison (more effective policy reccomendations). Consider removing it, or link to the experimental result with practical insights. }
% This integrated approach ensures that the model learns both system-level decision quality and region-specific predictive accuracy, ultimately leading to more effective policy recommendations that enhance overall power grid reliability while preserving the granularity required for targeted interventions.

% Also, considering predictive loss can also enhance the generalization of the model to new event with significant distributional shift and avoid overfit as the extreme event data is scarce oftentimes.

\subsection{Model Estimation}

The learning of \texttt{GDF} Neural ODEs is carried out through stochastic gradient descent, where the gradient is calculated using a novel algorithm based on differentiable optimization techniques \cite{amos2017optnet,cvx, zhu2022distributionally}.
% \woody{Mention that we directly minimize $g$ instead of the regret in practice, which is equivalent.}
% Note that minimizing regret is equivalent to minimizing the value of $g(\boldsymbol{x}^*(\hat{\mathbf{S}}), \mathbf{S})$, since the term $g(\boldsymbol{x}^*(\mathbf{S}), \mathbf{S})$ is constant with respect to the prediction model.
To enable differentiation through the $\arg \min$ operator in \eqref{eq:opt} embedded in the global-decision-focused loss, we relax the decision variable $\boldsymbol{x}$ from a potentially discrete space to a continuous space. 
For combinatorial optimization problems, the problem is reformulated as a differentiable quadratic program, and a small quadratic regularization term is added to ensure continuity and strong convexity \cite{wilder2019melding}.
Formally, we replace the original objective in \eqref{eq:opt} with the following:
\begin{equation} 
\min_{\boldsymbol{x}} g(\boldsymbol{x}, \mathbf{S}) + \rho\|\boldsymbol{x}\|_2^2,
\label{eq:regularized-LP}
\end{equation}
where $\rho > 0$ ensures differentiability. More implementation details can be found in the supplementary material.
% where $\rho > 0$ ensures differentiability. More implementation details can be found in Appendix~\ref{append:implement}.


\begin{algorithm}[!t]
\caption{Learning of \texttt{GDF} Neural ODEs}
\label{alg:gdfl}
\textbf{Input:} Data $\mathcal{D}=\{(\boldsymbol{z}_k^i, \{y_k^i(t)\})\}$, initial parameter $\theta_0$, initial states $\mathbf{S}^i(0)$, learning rate $\eta$, trade-off parameter $\lambda$, epochs $N$.\\
\textbf{Output:} $\theta^*$
\begin{algorithmic}[1]
\label{alg:alg1}
\FOR{epoch $n=1,\dots,N$}
    \FOR{$i = 1$ to $I$}
\STATE Initialize: $\hat{\mathbf{S}}^i(0) \leftarrow \mathbf{S}^i(0)$.
        \FOR{$j=1$ to $T_i$}
            \STATE $\hat{\mathbf{S}}^i(t_{j}) \leftarrow \hat{\mathbf{S}}^i(t_{j-1}) + f_\theta(\hat{\mathbf{S}}^i(t_{j-1}),\boldsymbol{z}_k^i)\Delta t_{j-1}$
        \ENDFOR
        \STATE $\ell_{\texttt{GDF}}^i(\theta) \leftarrow g\big(\boldsymbol{x}^*(\hat{\mathbf{S}}^i),\mathbf{S}^i\big)-g\big(\boldsymbol{x}^*(\mathbf{S}^i),\mathbf{S}^i\big)$
        \STATE $\theta \leftarrow \theta - \eta\, \nabla_\theta \ell_{\texttt{GDF}}^i(\theta)$.
    \ENDFOR
    \FOR{each mini-batch $\mathcal{B}\subset\mathcal{D}$}
        \STATE 
        $
        \ell_\text{Pred}(\theta) \leftarrow \frac{1}{|\mathcal{B}|KT}\sum_{(i,k,t)\in\mathcal{B}}\Big[y_k^i(t)-\hat{Y}_k^i(t)\Big]^2
        $
        \STATE $\theta \leftarrow \theta - \eta\, \lambda\,\nabla_\theta \ell_\text{Pred}(\theta)$.
    \ENDFOR
\ENDFOR
\RETURN $\theta$
\end{algorithmic}
\end{algorithm}
\vspace{-0.2in}


% \begin{algorithm}[!t]
% \caption{Global Decision-Focused Learning Algorithm \woody{double check the algorithm and fix the notations accordingly.}\ryan{have not updated; will fix later}}
% \label{alg:alg1}
% \textbf{Input:} Historical data $\mathcal{D} = (\boldsymbol{z}_k^i, \{y_k^i(t)\})$, initial parameter $\theta_0$, learning rate $\eta$, trade-off parameter $\lambda$, epochs $T$.
% \begin{algorithmic}[1]
% \FOR{epoch $t = 1$ to $T$}
%     \FOR{$i = 1$ to $I$}
%         \STATE $\displaystyle \hat{\mathbf{S}}^i \leftarrow f_{\theta}(\boldsymbol{z}^{i})$ \quad \textit{// forward pass for event $i$ \woody{Inaccurate. Use Euler method.}}
%         \STATE $\theta \;\leftarrow\;$ Decision-focused gradient update;
%     \ENDFOR
%     \FOR{each mini-batch $\mathcal{B}$ in $\mathcal{D}$}
%         \STATE $\displaystyle \hat{\mathbf{S}}_{\mathcal{B}} \leftarrow f_{\theta}(\boldsymbol{z}_{\mathcal{B}})$; \quad \textit{// forward pass for batch $\mathcal{B}$}
%         \STATE $\theta \;\leftarrow\;$ Mini-batch MSE gradient update;
%     \ENDFOR
% \ENDFOR
% \RETURN $\theta$
% \end{algorithmic}
% \end{algorithm}


\section{Application: Mobile Generator Deployment}
\label{sec:mobile}

% \begin{figure}[t!]
% \centering
% \includegraphics[width=\linewidth]{new_plots/plot_3rows_seed0_tc400_gamma2.0_G5_tau1_Ng100.0_lambda0.pdf}
% % \multicolumn{3}{c}{\small Predicted Outages with \textit{Decision Focused Learning}} \\
% \caption{A synthetic example of the mobile generator deployment problem for a system with three cities and five generators. The $y$-axis represents the number of outaged households. Three decision-making methods are evaluated on out-of-sample data. The bottom row shows the optimal solution derived from the true data, achieving the lowest total cost. The proposed \texttt{GDF} framework (top row) incurs a regret of $94.22$, while the online baseline approach (second row) obtains a substantially higher regret of $8033.84$. 
% Details of the synthetic data setup are provided in Section~\ref{sec:dataset}.
% \ryan{add x*(S),S, Y, plot decomposition of SIR}
% }
% \label{fig:diesel}
% \end{figure}


\begin{figure}[t!]
\centering
\includegraphics[width=\linewidth]{new_plots/SIR_plot_4rows_seed0_tc400_gamma2.0_G5_tau1_Ng100.0_lambda0_dt10_lag5.pdf}
\caption{A synthetic example of the mobile generator deployment problem for a system with three cities and five generators ($Q_w = 5$). The $y$-axis represents the number of outaged households. In this example, the uniform travel time $\delta_t = 10$, transportation cost is set to $c = 400$, the customer interruption cost is $\tau = 1$, and the operational cost is $\gamma = 2$. Four methods are compared on out‑of‑sample data: the proposed \texttt{GDF} framework (regret = $183.47$), a two‑stage approach (regret = $488.45$), an online baseline with observation lag of 5 (regret = $2846.13$), and the optimal solution with the groundtruth. Details of the synthetic data setup are provided in Section~\ref{sec:dataset}.}
% \woody{Can you add green and orange curves to the two-stage as well? I guess there might be a more significant difference among these two curves comparing to GDF. }
\label{fig:diesel}
% \vspace{-.1in}
\end{figure}

% \ryan{use the stylized warehouse example, but include more complete one in appendix}
Our method is broadly applicable to various proactive decision-making problems in grid operations, particularly in response to potential hazard events. 
% To demonstrate \texttt{GDF}'s versatility, this section examines a classic power grid resilience management problem, mobile generator deployment \cite{QIN2024752,9117454,6847750,868775}, as a representative PATOG task, where we apply the proposed \texttt{GDF} Neural ODEs to optimize deployment strategies.
% \woody{Emphasize the versatility of the model, rather the optimization problems themselves.}
This section highlights the adaptability of \texttt{GDF} by applying it to a mobile generator deployment task \cite{8668633,QIN2024752,9117454}, a representative \texttt{PATOG} problem. 

The objective of the mobile generator deployment problem is to strategically deploy mobile (\eg, diesel) generators across a network of potentially affected sites before a large-scale power outage to minimize associated costs. Due to the time-sensitive nature of power restoration, it is crucial to anticipate the spatiotemporal dynamics of outages while accounting for operational constraints such as generator capacity, fuel availability, and transportation logistics. \texttt{GDF} is well-suited for this task as it jointly learns outage patterns across cities over the planning horizon while optimizing deployment decisions, enabling proactive decision-making that adapts to evolving conditions.
% Unlike traditional approaches that treat forecasting and decision-making as separate stages, \texttt{GDF} integrates both by jointly learning outage predictions across cities over the planning horizon and optimizing generator allocation decisions. This ensures that predictive modeling is directly aligned with deployment strategies, leading to more effective and cost-efficient operations.


% \ryan{The proposed framework supports strategic pre-positioning and deployment of mobile generators to minimize downtime and operational costs across a network of critical sites. By integrating real-time predictions with system-wide optimization, the model enhances both preparedness and response efficiency. Figure~\ref{fig:diesel} illustrates an example of mobile generator deployment across multiple service units.
% Fig.~\ref{fig:diesel} illustrates a stylized example of a generator deployment problem involving three service units and five mobile generators.}
% \ryan{To describe Figure~\ref{fig:diesel} here.}

Formally, let $Q_w \in \mathbb{Z}_+$ denote the initial inventory of generators at warehouse $w$.
For simplicity, we assume that warehouses also function as staging areas, where generators are initially stored and returned after deployment.
We assume that each generator has a fixed maximum capacity, capable of supplying electricity to $N_g$ customers, and that a single extreme event is anticipated.
The planning horizon is discretized into uniform time periods $\mathcal{T} = \{1, 2, \dots, T\}$, where ${T}$ represents the expected duration of the outage.
Let $\mathcal{K} = \{1, 2, \dots, K\}$ be the set of $K$ service units (\eg, cities or counties) where generators can be deployed, and $\mathcal{W} = \{1, \dots, W\}$ be the set of warehouses where all generators are initially stored. 
We denote the stock of generators in unit $k$ at time $t$ as $q_{tk}$. 

The mobile generator deployment problem is therefore defined on a directed graph denoted by $(\mathcal{V}, \mathcal{E})$, where $\mathcal{V} \coloneqq \mathcal{K} \cup \mathcal{W}$ is the vertex set, and $\mathcal{E}$ represents the edge set defining feasible transportation routes.
Each edge $(k, k') \in \mathcal{E}$ incurs a transportation cost $c_{kk'}$, and an associated travel time $\delta_{kk'} \in \mathbb{Z}_+$, representing the number of time periods required for generators to be transported from location $k$ to location $k'$.
% a travel time $t_{ij}$.
The decision variables, ${\boldsymbol{x} = \{x_{tkk'}\}}$, $(k, k') \in \mathcal{E}$, $t \in \mathcal{T}$, specifies the transportation schedule, where $x_{tkk'} \in \mathbb{Z}_+$ represents the number of generators transported from location $k$ to $k'$ at time $t$. 
For simplicity, we assume negligible deployment time, allowing generators to become operational immediately upon arrival to supply power to $N_g$ customers.

The cost function of generator deployment problem is:
\begin{equation}
\begin{aligned}
    & g(\boldsymbol{x}, \mathbf{S}) \coloneqq \\
    & \underbrace{\sum_{(k,k') \in \mathcal{E}} \sum_{t=1}^T c_{kk'} x_{tkk'}}_{\text{Transportation Cost}} 
    + \underbrace{\gamma \sum_{k \in \mathcal{K}} \sum_{t=1}^T q_{tk} \vphantom{\sum_{(k,k') \in \mathcal{E}} \sum_{t=1}^T c_{kk'} x_{tkk'}}}_{\text{Operation Cost}}
    + \underbrace{h(\boldsymbol{x}, \mathbf{S}) 
    \vphantom{\sum_{(k,k') \in \mathcal{E}} \sum_{t=1}^T c_{kk'} x_{tkk'} + \gamma \sum_{k \in \mathcal{K}} \sum_{t=1}^T q_{tk}}}_{\text{Outage Cost}},
\end{aligned}
\label{eq:objective_Generator Distribution Problem}
\end{equation}
which consists of three components: 
($i$) Transportation Cost: The cost of delivering generators from warehouses to service units.
($ii$) Operation Cost: The fixed operational cost per unit time ($\gamma$) for deployed generators.
($iii$) Outage Cost: Economic loss due to power outages, determined by the number of customers experiencing outages given the generator deployment decision $\boldsymbol{x}$.
For simplicity, we define the outage cost as:
\[
    h(\boldsymbol{x}, \mathbf{S}) = \tau \sum_{k \in \mathcal{K}, t \in \mathcal{T}} \max \Big \{Y_k(t) - q_{tk} N_g, 0 \Big \},
\]
where $\tau$ denotes unit customer interruption cost, an adapted version of the Value of Lost Load (VoLL) \cite{6672826}, representing the economic impact per household outage per day, and $Y_k(t)$ denotes the number of outages in service unit $k$ at time $t$, extracted from the state $\mathbf{S}$ according to \eqref{eq:deterministic sys}.
This framework primarily evaluates system functionality based on the number of customers experiencing outages ($Y$). It is worth emphasizing that it is flexible to incorporate additional metrics, including the number of recovered customers ($R$) and unaffected customers ($U$), if needed.
% \woody{We note that in this problem, only $Y$ is used. But more generally, other problems might need to use $R$ and $U$ if needed.}
% \ryan{improve expression for value of load}

The transportation and inventory constraints of the mobile genereator deployment are as follows:
\begin{align}
& q_{0k} = 0, \quad \forall k \in \mathcal{K} 
\label{eq:initial_stock_cities} \\
& q_{0w} = Q_w, \quad \forall w \in \mathcal{W} \label{eq:initial_stock_warehouse} \\
% & q_{tk} = \sum_{t}(\sum_{k' \in \mathcal{K}} x_{tk'k} - \sum_{k' \in \mathcal{K}} x_{tkk'}), \quad \forall k \in \mathcal{V} \label{eq:stock_dynamics_warehouse} \\
& q_{tk} = \sum_{k' \in \mathcal{V}} \sum_{\substack{t' \in \mathcal{T} \\ t' + \delta_{k'k} \leq t}} x_{t'k'k} - \sum_{k' \in \mathcal{V}} x_{tkk'}, \quad \forall k \in \mathcal{V}, \, t \in \mathcal{T}  \label{eq:stock_dynamics_warehouse} \\
% & x_{tkk'} \leq 
%     \begin{cases} 
%         C, & \text{if } (k,k') \in \mathcal{E} \\ 
%         0, & \text{otherwise} 
%     \end{cases}, \quad \forall t \in \mathcal{T} \label{eq:flow_between_cities}\\ 
& x_{tkk'} \leq C, \quad \forall t \in \mathcal{T}, (k,k') \in \mathcal{E} \label{eq:flow_between_cities} \\
& \sum_{k \in \mathcal{V}/{\{k'\}} , t \in \mathcal{T}} x_{tkk'} = \sum_{k \in \mathcal{V}/{\{k'\}}, t \in \mathcal{T}} x_{tk'k}, \quad \forall k' \in \mathcal{V}, \label{eq:flow_constraint}\\
& q_{kk'}, x_{tkk'} \in \mathbb{Z}_+, \, \quad  \forall (k,k') \in \mathcal{E}, \, \forall t \in \mathcal{T} \label{eq:integer_constraint}.
\end{align}
Constraints \eqref{eq:initial_stock_cities} and \eqref{eq:initial_stock_warehouse} establish the initial generator stock at each location.
Equation \eqref{eq:stock_dynamics_warehouse} tracks the generator stock at each location, accounting for the travel time $\delta$ taken for incoming shipments.
Inequality constraint \eqref{eq:flow_between_cities} limits the flow between service units or warehouses in $\mathcal{E}$ to a maximum capacity $C$.
Finally, equation \eqref{eq:flow_constraint} ensures flow conservation, requiring that the total inflow of generators equals the total outflow over the planning horizon for each node in the network. As a result, all generators return to the staging areas after the events.

To implement the deployment strategy based on our \texttt{GDF} framework, we first predict outage levels $\hat{Y}_k(t)$ for all service units using the neural ODE model specified in \eqref{eq:deterministic sys}. 
The model is learned by minimizing the \texttt{GDF} loss defined in \eqref{eq:opt_dfl_ERM}. These global-deision-focused predictions are then integrated into the objective function \eqref{eq:objective_Generator Distribution Problem} of the mobile generator deployment problem, and the decisions are derived by solving a mixed-integer linear programming (MILP). 
% The efficacy of our approach is assessed by evaluating the resulting deployment decisions against ground truth outage values $Y_k(t)$ using the objective function \eqref{eq:objective_Generator Distribution Problem}. 
% The following section presents more numerical evaluations on synthetic and real-world datasets of this application to validate the performance of our method.
Figure~\ref{fig:diesel} illustrates the optimal transportation schedules derived from the \texttt{GDF} predictions using an MILP solver, compared to the schedules produced by the Two-stage and reactive online approaches in a stylized example of the mobile generator deployment problem.

% As shown, \texttt{GDF} significantly reduces regret compared to two-stage and online baselines.
% The online algorithm takes a reactive approach, frequently relocating generators and incurring high operational costs, while the two-stage method fails to capture system-wide dynamics, leading to suboptimal decisions. In contrast, \texttt{GDF}'s globally optimized approach ensures more effective and cost-efficient resource allocation, enhancing overall grid resilience with minimal cost.

% \woody{Need a paragraph discussing how to operationalize our algorithm solve this problem. First predict ... then solve ... we will present the numerical results of this problem using both synthetic and real data in the next section.}  

% Following the formulation in \eqref{eq:opt_dfl_ERM}, the Global Decision-Focused (GDF) learning objective for generator distribution problem can be defined using the specified cost function $g(\boldsymbol{x}, \mathbf{S})$ \eqref{eq:objective_Generator Distribution Problem}, thus enabling end-to-end learning.
% The experimental results for the application of the GDF framework to generator distribution problem are presented in Section \ref{sec:Generator Distribution Problem}.

\section{Experiments}
\label{sec:exp}

In this section, we evaluate the proposed \texttt{GDF} on two grid resilience management problems—mobile generator deployment and power line undergrounding. The numerical results demonstrate its superior performance compared to conventional Two-stage methods, enabling better decision-making in the face of natural hazards.

\subsection{Dataset Overview}
\label{sec:dataset}
We evaluate \texttt{GDF} using both real and synthetic datasets to assess its effectiveness in outage prediction and resilience planning. The real dataset records the number of customers affected by outages during the 2018 Nor’easter in Massachusetts \cite{wiki:March_11-15_2018_nor'easter}, while the synthetic outage trajectories are generated using a simplified SIR model.
% \ryan{mention ANL somethere for outage data source?}
% simulating outage propagation under varying weather conditions.
\subsubsection{Real Dataset}
The real dataset used in this study comprises county-level customer outage counts \cite{maema2020power}, combined with meteorological measurements and socioeconomic indicators from regions affected by a Nor’easter snowfall event in Massachusetts in 2018 (Fig.~\ref{fig:vis}). Meteorological variables—such as wind speed, temperature, and pressure—are sourced from NOAA’s High-Resolution Rapid Refresh (HRRR) model \cite{noaa_hrrr}.
Socioeconomic and demographic data are collected from the U.S. Census Bureau’s American Community Survey \cite{USCensusBureau2017ACS} and include median household income, median age, the number of food stamp recipients, the unemployment rate, the poverty rate, college enrollment, mean travel time to work, and average household size. These variables capture economic and mobility factors that may influence outage recovery dynamics and the effectiveness of emergency response efforts.
% \woody{Only cite one ref here and describe what feature has been used in text.} \ryan{according to the citation guidance of US census bureau, different data features have different authors, so I have to cite them all. But I will omit them in future reference.}
% \cite{Census2022ACSST5Y2022.S0101,Census2022ACSDP5Y2022.DP03,Census2022ACSDT5Y2022.B25103,Census2022ACSST5Y2022.S2201,Census2022ACSST5Y2022.S2701,Census2022ACSST5Y2022.S1101,Census2022ACSST5Y2022.S1401,Census2022ACSST5Y2022.S1701,Census2022ACSST5Y2022.S1901}.
% These sources provide valuable insights into community resilience through metrics such as median household income, real estate taxes, employment and poverty levels, age distribution, health insurance coverage, educational enrollment, household size, and SNAP participation.

A key advantage of using outage data from Massachusetts during the 2018 Nor’easter event \cite{wiki:March_11-15_2018_nor'easter} is that three consecutive snowstorms impacted the power system within a short 15-day period, during which the local infrastructure remained largely unchanged. This allows us to reasonably assume that the outage patterns from these storms follow the same underlying data distribution, making them suitable for widely used train-test evaluation. Accordingly, we use the first storm for training and the second for testing the effects of different frameworks, while excluding the third storm due to its relatively minor impact. Fig.~\ref{fig:vis} illustrates the spatiotemporal dynamics of the real dataset.

\subsubsection{Synthetic Dataset}
To augment these real-world observations and enable experiments under varying conditions, synthetic outage trajectories are generated using a simplified SIR model. In this model, each county is treated as an independent population that experiences outages and eventually recovers, in accordance with the dynamics specified in \eqref{eq:deterministic sys}. Simulated weather conditions are incorporated to modulate the transmission rate, thereby capturing the variability and severity of extreme events. This synthetic dataset provides a realistic and flexible testbed for systematically evaluating the proposed decision-focused learning framework under different scenarios.

% The dataset used in this study comprises township-level customer power outage numbers, meteorological data, and socioeconomic metrics from regions impacted by extreme weather events, focusing on data from Hurricane Ian, which made landfall in Florida in 2022. Meteorological data, including wind speed, temperature, and pressure, was obtained from the NOAA’s High-Resolution Rapid Refresh (HRRR) model \cite{noaa_hrrr}.
% The local census data is drawn from the U.S. Census Bureau’s American Community Survey (ACS) and includes metrics such as median household income \cite{Census2022ACSST5Y2022.S1901}, real estate taxes \cite{Census2022ACSDT5Y2022.B25103}, and economic characteristics like employment and poverty status \cite{Census2022ACSDP5Y2022.DP03}. Demographic factors, including median age \cite{Census2022ACSST5Y2022.S0101}, health insurance coverage \cite{Census2022ACSST5Y2022.S2701}, educational enrollment \cite{Census2022ACSST5Y2022.S1401}, household size \cite{Census2022ACSST5Y2022.S1101}, and SNAP participation \cite{Census2022ACSST5Y2022.S2201}, provide additional context on community resilience. 


% \subsubsection{Synthetic Data Generation}
% \label{sec:synthetic_data}
% To augment the real-world observations and enable experiments under varying conditions, we generate synthetic outage trajectories using a simplified SIR framework. In this approach, each township is treated as an independent population that experiences outages and eventually recovers, following the dynamics in \eqref{eq:deterministic sys}. Additionally, simulated weather factors are introduced to modulate the transmission rate, thereby capturing the variability and severity of extreme events. This controlled synthetic data provides a realistic yet flexible testbed for evaluating our decision-focused learning framework.

\subsection{Experimental Setup}
\label{sec:experiment_setup}
% \ryan{to mention the assumption of lag and hard to collect data in the exp setup here.}
% \ryan{will also mention use of Gurobi as the MILP solver here?}
% \ryan{find literature on reported disaster and time lag}
% \woody{Model configurations; discussion on baselines; evaluation metrics; training details. }
% \ryan{mention the simplified setup that all generators go back and forth between cities and warehourse, but not between cities}
% \ryan{go back to central warehouse for refuel; relax limit of x binary } 
% \woody{Discuss how the start time is chosen in practice.} \ryan{also initial conditions}

% To evaluate the effectiveness of the proposed Global Decision-Focused (\texttt{GDF}) Neural ODE framework, we conduct experiments on two application domains: the power line undergrounding and the mobile generator deployment problem as introduced in Section~\ref{sec:mobile}.
% For both problems, the decision-making process is initiated when the outage ratio at a location exceeds a predefined threshold (set to 2\% in our experiments), which determines the initial conditions of the ODE model and the start time for model prediction. 
To evaluate the effectiveness of the proposed Global Decision-Focused (\texttt{GDF}) Neural ODE framework, we conduct experiments on two application domains: power line undergrounding \cite{8278121, 7922545,AbiSamra2013,Shea2018} and mobile generator deployment as introduced in Section~\ref{sec:mobile}. For both problems, predictions for the entire time horizon are made before the occurrence of the hazard, reflecting the practical challenge of collecting data in real time during a developing hazard \cite{9803820}. Consequently, for both problems, 
when the hazard surpasses a predefined threshold (1\% in our experiments), the ODE model’s initial conditions are set, and decisions are made accordingly.
% all predictions and subsequent decisions are made when the hazard reaches the predefined threshold (set to 1\% in our experiments), which determines the initial conditions of the ODE prediction model.
% Each county is characterized by local covariates (e.g., weather conditions and census data) that influence outage dynamics.


% For the generator deployment problem formulation, we assume a single centralized warehouse that serves as the sole source of generators, thereby precluding inter-county transfers. Generators are dispatched from the warehouse to affected counties and must eventually return for refueling, following the setup in Section~\ref{sec:Generator Distribution Problem}. 
In the generator deployment problem, we assume a single centralized warehouse as the sole source of generators, with no inter-county transfers. Generators are dispatched from the warehouse to affected counties and must eventually return for refueling, following the setup described in Section~\ref{sec:Generator Distribution Problem}. We also assume a uniform travel time parameter, $\delta_{kk'} = \delta_t, \; \forall (k, k') \in \mathcal{E}$, to capture the impact of transportation delays. The online baseline method assumes an observation lag, making greedy observe-then-optimize decisions based on data from previous data, reflecting a realistic delay in data availability for decision makers without predictive insights. 
% While we report results for different lag values, all online methods shown in the plots use a fixed lag of one day.

% Our experiments leverage both synthetic and real-world datasets. The synthetic dataset is generated using a simplified SIR model that simulates outage trajectories for three cities. In this model, each county is treated as an independent population that experiences outages and eventually recovers, following the dynamics specified in \eqref{eq:deterministic sys}. Simulated weather factors modulate the transmission rate to capture the variability and severity of extreme events, thereby providing a realistic yet flexible testbed for evaluating our decision-focused learning framework under various scenarios and cost settings.

% For real-world evaluations, we utilize historical data from Hurricane Ian, which includes county-level outage trajectories, meteorological measurements, and socioeconomic indicators. Notably, the Massachusetts dataset features three consecutive storms over a 15-day period during which the local infrastructure remained essentially unchanged. This temporal consistency permits us to treat these events as samples from a single stationary distribution under an ERM framework. Accordingly, we use the first storm for training and the second for testing, while excluding the third storm due to its minor impact.

% Performance is assessed using three key metrics: (1) prediction accuracy, measured by the mean squared error (MSE) between predicted and observed outages; (2) decision quality, quantified by the total cost of generator deployment or grid hardening interventions; and (3) regret, defined as the cost difference between the optimal allocation under ground truth and the allocation produced by each model. 

The training process consists of two stages. Initially, the neural ODE model is trained using a standard mean squared error (MSE) loss, providing a predictive model baseline. Optimal decisions based on these baseline predictions serve as a Two-stage reference. In the second stage, the baseline model is then used to initialize end-to-end training with the combined loss function in \eqref{eq:opt_dfl_ERM}, which integrates both decision-focused and prediction-focused objectives, as described in Algorithm~\ref{alg:alg1}.

% Performance is measured by MSE, total cost, and regret (defined as the cost difference between the optimal allocation under ground truth and the model’s allocation). Baseline methods include SIR and spatio-temporal neural ODE models trained solely with MSE loss, while the decision-focused variants (\(\texttt{SIR-ODE}_{\texttt{DFL}}\) and \(Two-stage_{\texttt{DFL}}\)) integrate the GDF framework.

We compare the proposed model, \texttt{GDF}, with spatio-temporal neural ODE models trained solely on MSE loss (Two-stage) and the ground truth optimal decision baseline (True Optimal). We also include results from an online method for the mobile generator deployment problem. Evaluation is based on three metrics: ($i$) prediction accuracy, measured as the mean squared error between predicted and observed outages; ($ii$) decision loss, represented by the total cost for generator deployment or the total System Average Interruption Duration Index (SAIDI) for grid hardening; and ($iii$) regret, defined as the cost difference between the ground truth optimal decision and the decision based on the model’s predictions. For all experiments on synthetic datasets, reported results and standard deviations are calculated with three different random seeds, which generate varying outage curves and weather conditions.

% \woody{TBD: There is another policy we should compare, which is ``observe-then-optimize'' or online with an observational delay and we need to create ablation study to investigate the impact of the delay.} \nando{Agreed!}


% To evaluate the effectiveness of the proposed Global Decision-Focused (GDF) Neural ODE framework, we conduct experiments on two application domains: the Power Hardening Problem and the Generator Distribution Problem (Generator Distribution Problem). In the Power Hardening Problem, the objective is to determine optimal interventions for reinforcing local power grids—typically modeled as binary decisions—to enhance grid resilience. The total budget for power lind undergrdounign is limited by a global constraint. And the Generator Distribution Problem focuses on the optimal allocation of mobile generators during extreme weather events, with the goal of minimizing overall outage-related costs.

% For the neural ODE model used for both problems, the start time is determined when the outage ratio at a location exceeds a predefined threshold (set to 2\% in our experiments). In addition, each county is characterized by local covariates (e.g., weather and census data) that influence outage dynamics.

% For the Generator Distribution Problem, we assume a single centralized warehouse that serves as the exclusive source of generators. Generators are dispatched from the warehouse to affected counties and are required to return to the warehouse for refueling, thereby precluding inter-county transfers. 
% We also introduce a myopic, online baseline stregy to sovle the Generator Distribution problem. In the \emph{Online} baseline, only the generators required to cover the demand observed in the previous time step are shipped, provided that sufficient stock is available. Full details of this method are provided in the online Appendix.

% Our experiments utilize both synthetic and real-world datasets. The synthetic dataset is generated using a simplified SIR model that simulates outage trajectories for three cities. In this model, each county is treated as an independent population that experiences outages and eventually recovers, following the dynamics specified in \eqref{eq:deterministic sys}. Simulated weather factors are incorporated to modulate the transmission rate, thereby capturing the variability and severity of extreme events. This controlled synthetic dataset offers a realistic yet flexible testbed for systematically evaluating our decision-focused learning framework under varying scenarios and cost settings.

% For real-world evaluations, we employ historical data from Hurricane Ian, which provides county-level outage trajectories, meteorological measurements, and socioeconomic indicators. Notably, the Massachusetts dataset includes three consecutive storms over a 15-day period, during which the local infrastructure remained essentially unchanged. This temporal consistency enables us to treat these events as samples from a single stationary distribution under an ERM framework. Accordingly, we use the first storm for training and the second for testing, while excluding the third storm due to its minor impact.

% Performance is evaluated using three key metrics: (1) prediction accuracy, measured by the mean squared error (MSE) between predicted and observed outages; (2) decision quality, assessed via the total cost of generator deployment or grid hardening interventions; and (3) regret, defined as the cost difference between the optimal allocation (under ground truth) and the allocation determined by each model. Baseline methods include traditional ARIMA and RNN models for outage prediction (coupled with conventional optimization for generator allocation), as well as SIR and spatio-temporal neural ODE models trained with MSE loss. The decision-focused variants, namely \(\texttt{SIR-ODE}_{\texttt{DFL}}\) and \(Two-stage_{\texttt{DFL}}\), integrate the GDF framework to align predictive training with downstream decision objectives.

% \woody{Rewrite with more details about the data. I simply move it from the experiment section. }
% \woody{Description on the synthetic data.}

% \woody{Add one or a few images for data.}

% To evaluate the effectiveness of the proposed Global Decision-Focused (GDF) Neural ODE framework, we conducted experiments on the described Generator Distribution Problem (Generator Distribution Problem). The objective is to optimize the deployment of mobile generators across a network of affected geographical units during extreme weather events, minimizing the total cost while addressing power outage impacts.

% The starting time of these processes is chosen when the outage ratio reaches a predefined threshold $\eta$, which is 2\% in our experiments. 
% They are introduced to describe the speed at which a blackout occurs or is resolved at location $i$. 
% In our setting, we model each location $i$ with $k$ covariates—such as weather conditions and census data—that may impact outages, represented as  $\boldsymbol{z}_i = (w_{1}, \dots, w_{I})$ (e.g., metrics like wind speed, temperature, and population).

% The experiments were conducted using both synthetic and real-world datasets. The synthetic dataset simulates outage scenarios in three cities with varying population sizes and weather conditions using an SIR model. Each city is independently modeled to experience distinct outage dynamics influenced by local weather variables, including wind speed, pressure, and temperature, along with socioeconomic factors such as population. Travel time and transportation costs between cities and warehouses are predefined to reflect realistic constraints.
% For real-world datasets, we utilized historical data from Hurricane Ian. This dataset includes power outage trajectories recorded at a county-level resolution, along with meteorological data (e.g., wind speed, temperature, and pressure) and census data (e.g., population, median household income). The ground truth data provides outage durations and magnitudes for benchmarking model performance.

% In these experiments, we compared the proposed framework with several baselines. Baseline methods include traditional ARIMA and RNN models for outage prediction, followed by standard optimization techniques for generator allocation. We also included a compartmental SIR model trained with standard mean squared error (MSE) loss and a spatio-temporal neural ODE (STNODE) model trained similarly. The decision-focused models, SIR-ODE\textsubscript{DFL} and STNODE\textsubscript{DFL}, integrate the GDF framework to align prediction training with decision objectives.

% Three key metrics were used to evaluate performance: (1) prediction accuracy, measured by MSE between predicted and observed outages; (2) decision quality, evaluated using the total cost of generator deployment; and (3) regret, defined as the difference in cost between the optimal allocation under ground truth data and the allocation determined by each model.



\begin{table*}[!t]
\centering
\caption{Out-of-Sample Performance for Generator Distribution Problem with Synthetic Data. Results are averaged over 3 repeated experiments with standard error (SE) in brackets.}
\resizebox{\linewidth}{!}{%
\begin{tabular}{llllllllll}
\toprule
\multirow{2}{*}{Model} &
\multicolumn{3}{c}{$\delta_t$ = 1} &
\multicolumn{3}{c}{$\delta_t$ = 5} &
\multicolumn{3}{c}{$\delta_t$ = 10} \\
\cmidrule(lr){2-4} \cmidrule(lr){5-7} \cmidrule(lr){8-10}
& MSE & Cost & Regret & MSE & Cost & Regret & MSE & Cost & Regret \\
\midrule
Ground Truth  &  / & 10316.9 (73.9) & 0  & / & 11031.7 (69.0)  & 0  & / & 12473.7 (63.8) & 0 \\
Online (lag = 1) & /      & 63663.2 (3762.5) & 53346.3 (3769.9)  & /      & 18862.8 (965.7) & 7831.1 (916.1)  & /      & 19681.8 (2069.3) & 7208.1 (2121.0) \\
Online (lag = 3) & /      & 84769.4 (122.8) & 74452.6 (7246.8)  & /      & 19816.9 (73.6) & 8785.2 (1216.9) & /      & 38378.6 (3020.2) & 4962.5 (792.7) \\
% Online (lag = 5) & /      & 9618.9 (57.3)  & 4939.9 (111.1)  & /      & 22152.2 (498.5)  & 16497.9 (557.8)  & /      & 37818.9 (1074.7) & 32164.5 (1135.0) \\
Two-stage  & 9336.0 (975.6)  &  10459.8 (90.6) & 142.9 (90.3)  & 9336.0 (975.6)    &  11506.6 (98.1)  & 474.9 (95.7)    & 9336.0 (975.6)     & 12746.2 (196.7) & 272.4 (207.2) \\ 
\texttt{GDF}       & \textbf{7204.1 (1804.8)} & 10409.9 (79.7) & \textbf{93.0 (11.8)} & \textbf{7531.5 (1805.8)} & 11452.1 (80.3)  & \textbf{420.3 (24.3)} & \textbf{8907.7 (2645.2)} & 12601.9 (58.1) & \textbf{128.2 (49.3)} \\
\bottomrule
\end{tabular}
}
\label{table:Generator_Distribution_DeltaT}
% \vspace{-.1in}
\end{table*}


\begin{table}[!t]
\centering
\caption{Out-of-Sample Performance of Power Line Undergrounding Problem with Real and Synthetic Data}
\resizebox{\linewidth}{!}{%
\begin{tabular}{lcccccc}
\toprule
\multirow{2}{*}{Models} & \multicolumn{3}{c}{Nor'easter, MA, 2018} & \multicolumn{3}{c}{Synthetic} \\
\cmidrule(lr){2-4} \cmidrule(lr){5-7}
& {MSE ($\times 10^3)$} & {SAIDI} & {Regret} & {MSE ($\times 10^3)$} & {SAIDI} & {Regret} \\
\midrule
True Optimal    & /      & 218.2 & /   & / & 15.6 & / \\
Two-stage            & \textbf{117.4} & 328.3 & 112.1   & 13.5 & 16.1 & 0.5 \\
$\texttt{GDF}$ & 165.6  & \textbf{312.7} & \textbf{94.5} & \textbf{13.1} & \textbf{15.6} & \textbf{0} \\
\bottomrule
\end{tabular}
}
\label{table:real}
% \vspace{-.1in}
\end{table}

% Table 1: Out-of-Sample Regret for Synthetic Data
% \begin{table}[!t]
% \centering
% \caption{Out-of-Sample Regret for Generator Distribution problem (Synthetic Data)}
% \resizebox{\linewidth}{!}{%
% \begin{tabular}{lcccccc}
% \toprule
% Travel Cost & {Ground Truth}& \texttt{Online} & \texttt{SIR-ODE} & Two-stage & \texttt{SIR-ODE}$_{\texttt{DFL}}$ & \texttt{GDF} \\
% \midrule
% 1 & 4632 & 4658 & 5863  &  &  5464\\
% 10 & 4761 & 6457 & 5990 &  & 5759 \\
% 100 & 6048 & 35789 & 6892 &  & 6438\\
% \bottomrule
% \end{tabular}
% }
% \label{table:Generator Distribution Problem_synthetic_regret}
% \end{table}

% % Table 2: Out-of-Sample Regret for Nor'easter (MA, 2018)
% \begin{table}[!t]
% \centering
% \caption{Out-of-Sample Regret for Generator Distribution problem (Nor'easter, MA, 2018)}
% \resizebox{\linewidth}{!}{%
% \begin{tabular}{lcccccc}
% \toprule
% Travel Cost & \texttt{Ground Truth} & \texttt{Online} & \texttt{SIR-ODE} & Two-stage & \texttt{SIR-ODE}$_{\texttt{DFL}}$ & \texttt{GDF} \\
% \midrule
% Low &  &  & &  &  &  \\
% Medium &  & &  &  &  &  \\
% High &  &  & &  &  &  \\
% \bottomrule
% \end{tabular}
% }
% \label{table:Generator Distribution Problem_noreaster_regret}
% \end{table}


% \begin{table}[!t]
% \caption{Out-of-Sample Regret for Generator Distribution problem}
% \centering
% \resizebox{\linewidth}{!}{%
% \begin{tabular}{llcccccc}
% \toprule
% \textbf{Dataset} & \textbf{Travel Cost} & \textbf{Ground Truth} & \textbf{\texttt{Online}} & \textbf{\texttt{SIR-ODE}} & \textbf{Two-stage} & \textbf{\texttt{SIR-ODE}$_{\texttt{DFL}}$} & \textbf{\texttt{GDF}} \\
% \midrule
% \multirow{3}{*}{\textbf{Synthetic}} 
% & 1 & 4632 & 4658 & 5863  &  &  5464 \\
% & 10 & 4761 & 6457 & 5990 &  & 5759 \\
% & 100 & 6048 & 35789 & 6892 &  & 6438 \\
% \midrule
% \multirow{3}{*}{\textbf{Nor'easter}} 
% & 1 &  &  &  &  &  \\
% & 10 &  &  &  &  &  \\
% & 100 &  &  &  &  &  \\
% \bottomrule
% \end{tabular}
% }
% \label{table:Generator Distribution Problem_combined}
% \end{table}

% % Table 1: Out-of-Sample Regret for Synthetic Data
% \begin{table}[!t]
% \centering
% \caption{Out-of-Sample Regret for Generator Distribution problem (Synthetic Data)}
% \resizebox{\linewidth}{!}{%
% \begin{tabular}{lcccccc}
% \toprule
% Travel Cost & {Ground Truth}& \texttt{Online} & \texttt{SIR-ODE} & Two-stage & \texttt{SIR-ODE}$_{\texttt{DFL}}$ & \texttt{GDF} \\
% \midrule
% 1 & 4632 & 4658 & 5863  &  &  5464\\
% 10 & 4761 & 6457 & 5990 &  & 5759 \\
% 100 & 6048 & 35789 & 6892 &  & 6438\\
% \bottomrule
% \end{tabular}
% }
% \label{table:Generator Distribution Problem_synthetic_regret}
% \end{table}

% % Table 2: Out-of-Sample Regret for Nor'easter (MA, 2018)
% \begin{table}[!t]
% \centering
% \caption{Out-of-Sample Regret for Generator Distribution problem (Nor'easter, MA, 2018)}
% \resizebox{\linewidth}{!}{%
% \begin{tabular}{lcccccc}
% \toprule
% Travel Cost & \texttt{Ground Truth} & \texttt{Online} & \texttt{SIR-ODE} & Two-stage & \texttt{SIR-ODE}$_{\texttt{DFL}}$ & \texttt{GDF} \\
% \midrule
% Low &  &  & &  &  &  \\
% Medium &  & &  &  &  &  \\
% High &  &  & &  &  &  \\
% \bottomrule
% \end{tabular}
% }
% \label{table:Generator Distribution Problem_noreaster_regret}
% \end{table}




% \begin{table}[!t]
% \centering
% \caption{Out-of-Sample Performance for Generator Distribution problem (Nor’easter, MA, 2018)}
% \resizebox{\linewidth}{!}{%
% \begin{tabular}{lccccccccc}
% \toprule
% \multirow{2}{*}{\texttt{Model}} &
% \multicolumn{3}{c}{Travel Cost = 1} &
% \multicolumn{3}{c}{Travel Cost = 10} &
% \multicolumn{3}{c}{Travel Cost = 100} \\
% \cmidrule(lr){2-4} \cmidrule(lr){5-7} \cmidrule(lr){8-10}
% & {MSE} & {Cost} & {Regret} & {MSE} & {Cost} & {Regret} & {MSE} & {Cost} & {Regret} \\
% \midrule
% \texttt{Ground Truth} &  &  &  &  &  &  &  &  &  \\
% \texttt{Online} &  &  &  &  &  &  &  &  &  \\
% % \texttt{SIR-ODE}       &  &  &  &  &  &  &  &  &  \\
% \texttt{GDF} &  &  &  &  &  &  &  &  &  \\
% % \texttt{SIR-ODE}$_{\texttt{DFL}}$ &  &  &  &  &  &  &  &  &  \\
% \texttt{GDF} &  &  &  &  &  &  &  &  &  \\
% \bottomrule
% \end{tabular}
% }
% \label{table:Generator Distribution Problem_noreaster}
% \end{table}

% \subsection{Results}
% \begin{table}[!t]
% \centering
% \caption{Out-of-Sample Performance of Generator Distribution Problem}
% \resizebox{\linewidth}{!}{%
% \begin{tabular}{lcccccc}
% \toprule
% \multirow{2}{*}{Models} &
% \multicolumn{3}{c}{Synthetic Data} & \multicolumn{3}{c}{Nor'easter (MA, 2018)} \\
% \cmidrule(lr){2-4} \cmidrule(lr){5-7}
% & {MSE} & {Total Cost} & {Regret} & {MSE} & {Total Cost} & {Regret} \\
% \midrule
% $\texttt{SIR-ODE}$ & \multicolumn{1}{c}{} & \multicolumn{1}{c}{} & \multicolumn{1}{c}{} & \multicolumn{1}{c}{} & \multicolumn{1}{c}{} & \multicolumn{1}{c}{} \\
% Two-stage & \multicolumn{1}{c}{2710} & \multicolumn{1}{c}{6705} & \multicolumn{1}{c}{248} & \multicolumn{1}{c}{} & \multicolumn{1}{c}{} & \multicolumn{1}{c}{} \\
% $\texttt{SIR-ODE}_{\texttt{DFL}}$ & \multicolumn{1}{c}{} & \multicolumn{1}{c}{} & \multicolumn{1}{c}{} & \multicolumn{1}{c}{} & \multicolumn{1}{c}{} & \multicolumn{1}{c}{} \\
% $\texttt{GDF}$ & \multicolumn{1}{c}{\textbf{2628}} & \multicolumn{1}{c}{\textbf{6632}} & \multicolumn{1}{c}{\textbf{175}} & \multicolumn{1}{c}{} & \multicolumn{1}{c}{} & \multicolumn{1}{c}{} \\
% $\texttt{Online}$ & \multicolumn{1}{c}{-} & \multicolumn{1}{c}{33942} & \multicolumn{1}{c}{27485} & \multicolumn{1}{c}{} & \multicolumn{1}{c}{} & \multicolumn{1}{c}{} \\
% $\texttt{Online+}$ & \multicolumn{1}{c}{} & \multicolumn{1}{c}{} & \multicolumn{1}{c}{} & \multicolumn{1}{c}{} & \multicolumn{1}{c}{} & \multicolumn{1}{c}{} \\
% \bottomrule
% \end{tabular}
% }
% \label{table:Generator Distribution Problem}
% \end{table}



% \begin{table}[!t]
% \centering
% \caption{Out-of-Sample Performance for Generator Distribution problem (Real Data)}
% \resizebox{\linewidth}{!}{%
% \begin{tabular}{lccccccccc}
% \toprule
% \multirow{2}{*}{\texttt{Model}} &
% \multicolumn{3}{c}{Travel Cost = 1} &
% \multicolumn{3}{c}{Travel Cost = 10} &
% \multicolumn{3}{c}{Travel Cost = 100} \\
% \cmidrule(lr){2-4} \cmidrule(lr){5-7} \cmidrule(lr){8-10}
% & {MSE} & {Cost} & {Regret} & {MSE} & {Cost} & {Regret} & {MSE} & {Cost} & {Regret} \\
% \midrule
% \texttt{Ground Truth} & / & 4632 & 0 & / & 5272 & 0 & / & 6457 & 0 \\
% \texttt{Online} &  &  &  &  & 5610 & 338 & / & 8490 & 2033 \\
% Two-stage       & 8992 & 5863 & 1231 & 8992 & 5514 & 242 & 8992 & 6648 & 191 \\
% % Two-stage &  &  &  &  &  &  &  &  &  \\
% \texttt{GDF} & 5464 & 5464 & 832 & 9147 & 5626 & 354 & 9147 & 6642 & 185 \\
% % \texttt{GDF} &  &  &  &  &  &  &  &  &  \\
% \bottomrule
% \end{tabular}
% }
% \label{table:Generator Distribution Problem_real}
% \end{table}
% \begin{figure}[!t]
% % \begin{wrapfigure}{r}{0.5\textwidth} % Adjust the alignment and width as needed
%         \centering
%         \includegraphics[width=0.5\textwidth]{new_plots/Generator Distribution Problem_syn_travel_cost.png}
%         \caption{\ryan{Placeholder. Currently gdf model is undertrained so not performing well.} \woody{Try higher transportation cost with more evaluation points or using log scale, such as $10^{0}$, $10^{1}$, $10^{2}$, so on so forth.}}
%         \label{fig:Generator Distribution Problem_frontier}
%     \label{fig:main}
% % \end{wrapfigure}
% \end{figure}
% \begin{figure}[!t]
% \centering
% \subfigure[Total Cost vs. Transportation Cost Factor \ryan{take exponential}]{%
%   \includegraphics[width=0.49\columnwidth]{new_plots/Total_Cost_vs_Transportation_Cost_Factor.pdf}}
% \hfil
% \includegraphics[width=0.9\columnwidth]{new_plots/Total_Regret_vs_Transportation_Cost_Factor_boxplot.pdf}
% \caption{Synthetic mobile generator deployment problem performance across transportation costs
% \ryan{border}
% }
% \label{fig:extensive}
% \end{figure}

% \begin{figure}[!t]
% \centering
% % \subfigure[Total Cost vs. Transportation Cost Factor \ryan{take exponential}]{%
% %   \includegraphics[width=0.49\columnwidth]{new_plots/Total_Cost_vs_Transportation_Cost_Factor.pdf}}
% % \hfil
% \includegraphics[width=0.9\columnwidth]{new_plots/Total_Regret_vs_gamma_split.pdf}
% \caption{Synthetic mobile generator deployment problem performance across operation costs.}
% \label{fig:extensive_lambda}
% \end{figure}

% \begin{figure}[!t]
% \centering
% % \subfigure[Total Cost vs. Transportation Cost Factor \ryan{take exponential}]{%
% %   \includegraphics[width=0.49\columnwidth]{new_plots/Total_Cost_vs_Transportation_Cost_Factor.pdf}}
% % \hfil
% \includegraphics[width=0.9\columnwidth]{new_plots/Total_Regret_vs_Tau_boxplot.pdf}
% \caption{Synthetic mobile generator deployment problem performance across customer interruption costs \woody{Make two-stage and GDF at $\tau=0.1$ visible.}}
% \label{fig:extensive_tau}
% \end{figure}

% \begin{figure}[!t]
% \centering
% % \subfigure[Total Cost vs. Transportation Cost Factor \ryan{take exponential}]{%
% %   \includegraphics[width=0.49\columnwidth]{new_plots/Total_Cost_vs_Transportation_Cost_Factor.pdf}}
% % \hfil
% \includegraphics[width=0.9\columnwidth]{new_plots/Total_Regret_vs_G_boxplot.pdf}
% \caption{Synthetic mobile generator deployment problem performance across different number of generators. \woody{Consider condensing fig 4,5,6,7 and fitting them into a single column, sharing the same legend (\eg, $2 \times 2$).}}
% \label{fig:extensive_q_w}
% \end{figure}


\begin{figure}[!t]
\centering
% \subfigure[Total Cost vs. Transportation Cost Factor \ryan{take exponential}]{%
%   \includegraphics[width=0.49\columnwidth]{new_plots/Total_Cost_vs_Transportation_Cost_Factor.pdf}}
% \hfil
\includegraphics[width=\columnwidth]{new_plots/Combined_2x2_Plots.pdf}
\caption{Performance Comparison for Synthetic Mobile Generator Deployment: A detailed comparison of regret outcomes for \texttt{GDF}, Two-Stage, and Online methods under varying customer interruption costs ($\tau$), transportation cost factors ($c$), operational costs ($\gamma$), and numbers of generators ($Q_{w}$).}
\label{fig:extensive}
% \vspace{-.1in}
\end{figure}

% \begin{figure}[!t]
%     \centering
%     % Total Cost Plot
%     \begin{subfigure}
%         \centering
%         \includegraphics[width=0.2\linewidth]{new_plots/total.pdf}
%         \caption{Total Cost}
%         \label{fig:total_cost}
%     \end{subfigure}
%     \hfill
%     % Economic Cost Plot
%     \begin{subfigure}
%         \centering
%         \includegraphics[width=0.2\linewidth]{new_plots/econ.pdf}
%         \caption{Economic Cost}
%         \label{fig:economic_cost}
%     \end{subfigure}
    
%     \vspace{0.5cm}
    
%     % Transportation Cost Plot
%     \begin{subfigure}
%         \centering
%         \includegraphics[width=0.2\linewidth]{new_plots/trans.pdf}
%         \caption{Transportation Cost}
%         \label{fig:transportation_cost}
%     \end{subfigure}
%     \hfill
%     % Operational Cost Plot
%     \begin{subfigure}
%         \centering
%         \includegraphics[width=0.2\linewidth]{new_plots/ops.pdf}
%         \caption{Operational Cost}
%         \label{fig:operational_cost}
%     \end{subfigure}
    
%     \caption{Comparison of cost metrics versus transportation cost factor for different models: (a) Total cost, (b) Economic cost, (c) Transportation cost, and (d) Operational cost.}
%     \label{fig:cost_comparison}
% \end{figure}

% \begin{figure}[!t]
% \centering \includegraphics[width=.7\linewidth]{example-image-a} \caption{Power outages and Pressure (Pa) during Hurricane Ian (2022)
%  \ryan{placeholder}}
% \label{fig:vis} 
% \vspace{-.1in}
% \end{figure}

% The combination of dynamic weather data and static census data provides a comprehensive representation of each location’s vulnerability and recovery dynamics under extreme weather conditions. This diverse set of inputs enables the proposed model to capture intricate, location-specific patterns in power outage processes, enhancing the precision of decision-making for resource allocation and grid hardening.


% \vspace{.1in}
% \centering
% \caption{Comparison of MSE, SAIDI, and Regret With Synthetic Data}
% \resizebox{\linewidth}{!}{%
% \begin{tabular}{llllllll}
% \toprule
% \multirow{2}{*}{Models} &
% \multicolumn{3}{c}{Training set} & \multicolumn{3}{c}{Testing Set} \\
% & {MSE} & {SAIDI} & {Regret} & {MSE} & {SAIDI} & {Regret} \\
% \midrule
% $\texttt{ARIMA}$ &  &  &  &  &  &  \\
% $\texttt{RNN}$ &  &  &  &  &  &  \\
% $\texttt{SIR-ODE}$ &  &  &  &  &  &  \\
% Two-stage &  & &  &  &  &  \\
% \midrule
% $\texttt{SIR-ODE}_{\texttt{DFL}}$ &  &  &  &  &  &  \\
% $\texttt{GDF}$ &  &  &  &  &  &  \\
% \bottomrule
% \end{tabular}
% }
% \label{table:synthetic}


% \begin{table}[!t]
% \centering
% \caption{Effect of $\lambda$ on prediction accuracy and decision quality}
% \resizebox{0.7\linewidth}{!}{%
% \begin{tabular}{lcccc}
% \toprule
% \multirow{2}{*}{Metrics} & \multicolumn{4}{c}{$\gamma$} \\
% \cmidrule(lr){2-5}
% & {0} & {1e-3} & {1e-2} & {1e-1} \\
% \midrule
% MSE & & & & \\
% SAIDI & & & & \\
% Regret & 263.44 & 147.53 & 79.50 & \\
% \bottomrule
% \end{tabular}
% }
% \label{table:real}
% \end{table}


% \begin{figure}[!t]
% \centering \includegraphics[width=0.8\linewidth]{new_plots/output.png} \caption{\ryan{will include results with sir and stnode}
% }
% \label{fig:res} 
% \vspace{-.1in}
% \end{figure}

% \woody{We need another baseline that makes decisions based on the real-time observations rather than prediction, to show the benefit of proactive decision making. In this baseline, we can come up with a simple online strategy to distribute the resources and compare its regret with our framework.}
% \ryan{basically, greedy decision at every time.}

% \ryan{add baseline, a model only predict saidi, to show framework good compaiblity}



\subsection{Results}
This section presents the results of \texttt{GDF} on mobile generator deployment and power line undergrounding, evaluated in terms of predictive accuracy and decision quality. Results on both synthetic and real datasets demonstrate that \texttt{GDF} improves decision-making compared to conventional Two-stage methods, enabling a more effective response to natural hazards.

\subsubsection{Mobile Generator Deployment Problem}
\label{sec:Generator Distribution Problem}
Table~\ref{table:Generator_Distribution_DeltaT} summarizes the out-of-sample performance for the generator deployment problem on synthetic data under different travel time settings ($\delta_t = 1, 5, 10$). The proposed method consistently delivers the best decision quality, as reflected in the lowest regret. Notably, as $\delta_t$ increases—representing longer transportation delays—the performance improvement of \texttt{GDF} over the two-stage approach becomes increasingly significant. A similar trend is observed under higher transportation cost factors, as shown in Fig.~\ref{fig:extensive}, until transportation costs become so high that the best strategy is effectively to remain idle, at which point the improvement from proactive actions with \texttt{GDF} diminishes.

These findings highlight the importance of proactive scheduling and early interventions when facing longer deployment delays or higher transportation costs. It also demonstrates the advantage of \texttt{GDF} in optimizing resource allocation under more challenging, time-sensitive conditions in the face of extreme hazards that threaten grid resilience. 
As visualized in Fig.~\ref{fig:diesel}, the \texttt{GDF} model slightly overestimate outages in critical regions first hit by outage, enabling the reallocation of additional resources to mitigate potential disruptions. While this adjustment in the forecast is subtle, it leads to a significant reduction in regret. 
Remarkably, the \texttt{GDF} models also achieve MSE comparable to that of the MSE-trained model in Table~\ref{table:Generator_Distribution_DeltaT}, albeit with higher variance. 
This indicates that improved decision quality was not achieved at the expense of predictive accuracy but rather through targeted and meaningful adjustments in the forecasts.

% Furthermore, we introduce a uniform travel time parameter, $\delta_{kk'} = \delta_t, \; \forall (k, k') \in \mathcal{E}$, to capture the impact of non-negligible transit delays between generator dispatch and their arrival at affected service units. This parameter models realistic logistical constraints that influence resource availability during extreme events.
% As shown in Table~\ref{table:Generator_Distribution_DeltaT}, incorporating $\delta_t$ highlights the effectiveness of our \texttt{GDF} approach in optimizing resource allocation under practical conditions. Notably, as travel times increase, the \texttt{GDF} method demonstrates a greater improvement in regret compared to two-stage and baseline strategies.

% To understand how \texttt{GDF} impacts MSE-trained predictive models, see Fig.\ref{fig:diesel}. By slightly increasing the predicted outage for city 3, \texttt{GDF} proactively allocated more resources to city 3’s anticipated outage, drawing from the stock allocated to city 1. In contrast, the Two-stage method did not make this adjustment. While this change in prediction is subtle, it results in significantly lower regret. 

% \ryan{this kind of argueement is shakey. Can argue it's the MSE improvement results in decision quality.}

% These results underscore the importance of scheduling and proactive actions based on a GDF model. As illustrated in Fig.\ref{fig:diesel}, the \texttt{GDF} model tended to underestimate the outage in city1, thereby diverting more resources to mitigate outages in city~3. Although this change in prediction was subtle, it resulted in significantly lower regret. Remarkably, the MSE achieved by the \texttt{GDF} model was comparable to that of the MSE-trained model, demonstrating that the improved decision quality was not achieved at the expense of predictive accuracy.

Additionally, we conducted extensive experiments on synthetic data for the mobile generator deployment problem under various settings, as shown in Fig.~\ref{fig:extensive}. We observed that the performance advantage of \texttt{GDF} over the Two-stage and greedy methods narrows as customer interruption costs, operational costs, or the number of available generators increase. This behavior is expected, as higher system resources or reduced flexibility diminish the need for globally optimized interventions—resulting in less improvement with \texttt{GDF}.
% The decision quality improvement of \texttt{GDF} relative to the Two-stage method initially increases with transportation cost before eventually declining. 
% This behavior is expected: higher transportation costs highlight the importance of proactive actions.
% Additionally, we observed that the greedy method exhibits zero regret when the transportation cost is zero, as it allocates generators nearly optimally. However, as transportation costs rise, its regret increases dramatically due to the more frequent repositioning of generators required to meet real-time demand without proactive planning.

\subsubsection{Power Line Undergrounding}
\label{sec:power-line-undergrounding}
% For both hurricane scenarios, Irma and Ian, the proposed \texttt{ODE\textsubscript{DFL}} model achieved significantly lower SAIDI values compared to the traditional MSE-optimized models. In the case of Hurricane Irma, \texttt{ODE\textsubscript{DFL}} demonstrated a notable improvement in decision quality, reducing SAIDI from 381.57 to 147.53: a reduction of over 60\% in decision regret. This outcome indicates that integrating DFL directly into the prediction framework allows for more effective prioritization of undergrounding locations, particularly under constraints that limit the subset of locations that can receive interventions.

% In terms of MSE, while the DFL-integrated models (\texttt{SIR-ODE\textsubscript{DFL}} and \texttt{STNODE\textsubscript{DFL}}) generally exhibited slightly higher prediction errors than the MSE-optimized counterparts, their decision quality, as measured by regret, was substantially improved. This trade-off suggests that pure predictive accuracy may not always correlate with optimal decision-making outcomes. The DFL model, by balancing prediction and decision objectives, effectively aligns the model’s outputs with the power line undergrounding goals, demonstrating the framework’s capacity to support robust grid resilience strategies.

Table~\ref{table:real} presents the out-of-sample performance on both the Nor’easter, MA, 2018 event and the synthetic dataset. For the real dataset, the proposed \texttt{GDF} model—despite a slightly higher MSE—delivers improved decision quality, achieving a lower SAIDI and reduced regret compared to the Two-stage baseline. On the synthetic dataset, \texttt{GDF} similarly outperforms the Two-stage method in decision quality while maintaining a low MSE, possibly due to the relatively simple prediction tasks. These results confirm that \texttt{GDF} leads to better overall decision performance in real and synthetic settings.

Overall, the experimental results on both real and synthetic data demonstrate that \texttt{GDF} achieves substantial improvements in decision quality compared to the traditional two-stage approach. Although both models exhibit comparable prediction accuracy, \texttt{GDF} consistently yields lower regret values, highlighting the benefits of proactive scheduling and decision-focused training. 
% These findings underscore that integrating decision-focused learning into predictive models leads to more effective resource allocation decisions under varying cost constraints.

% Table~\ref{table:Generator Distribution Problem_synthetic} shows the out-of-sample performance. The decision-focused models outperform these baselines and other references in terms of lower total cost and lower regret. On synthetic data, \texttt{GDF} achieves a notably lower regret compared to the MSE-trained Two-stage, demonstrating the benefit of aligning training objectives with decision quality. Although the decision-focused models exhibit slightly higher MSE on average, their superior downstream performance highlights how GDF effectively trades off pure prediction accuracy for better decisions.

% The results are presented in Table~\ref{table:Generator Distribution Problem}. They indicate that decision-focused models outperform the baselines in terms of decision quality, as measured by lower total costs and regret. On the synthetic dataset, STNODE\textsubscript{DFL} achieved a significant reduction in regret compared to its MSE-trained counterpart, highlighting the benefit of aligning prediction training with decision-making objectives. While the decision-focused models exhibited slightly higher MSE than the baseline models, their ability to improve downstream decision-making demonstrates that GDF effectively trades off prediction accuracy for better decision quality.



% This result underscores the importance of tuning $\gamma$ to optimize DFL training, especially when decision quality is prioritized over pure prediction accuracy, enabling more effective resilience-focused power line undergrounding.

% \section{Conclusion}

% This paper presents a novel approach integrating Decision-Focused Learning (DFL) with Neural Ordinary Differential Equations (ODEs) to address power grid resilience under extreme weather events. By modeling the failure-restoration dynamics and embedding decision-making objectives into the learning process, our method bridges the gap between outage prediction and actionable grid-hardening strategies. Results from case studies show improved decision-making effectiveness, offering valuable insights for policymakers and utility companies. In sum, this research offers a promising solution for strengthening power grid resilience against escalating climate risks.


\section{Conclusion}
\label{sec:conclusion}
This paper presented a novel framework for grid resilience management that jointly optimized prediction accuracy and global decision quality. Experimental results on both real and synthetic datasets demonstrated that, while achieving comparable MSE results to the conventional two-stage approach, \texttt{GDF} consistently yielded lower regret and better decisions. These findings suggest that integrating decision-focused learning enhances proactive scheduling and resource allocation, enabling system operators to more effectively mitigate outages. Future work will explore integrating real-time data streams to improve responsiveness, incorporating renewable energy forecasting, and adapting the framework to larger network systems.

Our research provides actionable insights for grid resilience practitioners, highlighting the importance of proactive scheduling and early interventions when facing the threat of natural hazards to power systems. It also shows that embedding global-decision-focused learning objectives into predictive models can meaningfully shift forecasting to improve system-wide outcomes. Specifically, by strategically adjusting predicted outages—such as preemptively amplifying forecasts for high-risk or critical regions—the \texttt{GDF} framework enables earlier and more targeted resource allocation decisions. This leads to substantial reductions in overall costs and, consequently, in regret. The benefits are especially pronounced in settings with high transportation costs or delays, where globally optimized forecasts are key for operators to prioritize and allocate limited resources proactively and more effectively across the entire network. In such scenarios, the forecasting model, empowered by decision-aware optimization, acts not just as a passive predictor but as a key lever in optimizing strategic resource deployment, ultimately improving grid resilience and reducing the economic burden of extreme events.

\section*{Acknowledgments}
We sincerely appreciate Dr. James Kotary for his valuable insights on implementing quadratic regularization in decision-focused learning and for enhancing the literature review.
% His expertise in decision-focused learning has been instrumental in improving the quality of this work.
% We also acknowledge Baoni Li, a master's student in Heinz, for her dedicated work in processing the NOAA data set, which greatly supported this study.

% \woody{Fix references with url links}

% \bibliographystyle{plain}
\bibliographystyle{IEEEtran}
\bibliography{ref}

% \newpage

% \section{Biography Section}
% If you have an EPS/PDF photo (graphicx package needed), extra braces are
%  needed around the contents of the optional argument to biography to prevent
%  the LaTeX parser from getting confused when it sees the complicated
%  $\backslash${\tt{includegraphics}} command within an optional argument. (You can create
%  your own custom macro containing the $\backslash${\tt{includegraphics}} command to make things
%  simpler here.)
 
% \vspace{11pt}

% \bf{If you include a photo:}\vspace{-33pt}
% \begin{IEEEbiography}[{\includegraphics[width=1in,height=1.25in,clip,keepaspectratio]{fig1}}]{Michael Shell}
% Use $\backslash${\tt{begin\{IEEEbiography\}}} and then for the 1st argument use $\backslash${\tt{includegraphics}} to declare and link the author photo.
% Use the author name as the 3rd argument followed by the biography text.
% \end{IEEEbiography}

% \vspace{11pt}

% \bf{If you will not include a photo:}\vspace{-33pt}
% \begin{IEEEbiographynophoto}{John Doe}
% Use $\backslash${\tt{begin\{IEEEbiographynophoto\}}} and the author name as the argument followed by the biography text.
% \end{IEEEbiographynophoto}


% \vfill
\clearpage
\appendix

\subsection{Ablation Study}
\label{sec:alb}
To evaluate the impact of the prediction error weight $\lambda$ in \eqref{eq:opt_dfl_ERM} on balancing prediction accuracy and decision quality, we conducted an ablation study with various $\lambda$ values on synthetic dataset. Results in Table~\ref{table:alb} indicate that a lower $\lambda$ shifts the model’s focus toward decision quality, reducing decision regret by aligning predictions with resilience goals, albeit with a slight trade-off in MSE, compared to Two-stage method. We note that such a design offers better flexibility and interpretability for the \texttt{GDF}-trained decision-making models.


% \begin{table}[!t]
% \centering
% \caption{Effect of $\lambda$ and Accuracy-Decision Quality Tradeoff on Mobile Generator Deployment Problem and Power Hardening Problems. \woody{Kind reminder: There are missing entries.}}
% \resizebox{\linewidth}{!}{%
% \begin{tabular}{lcccccc}
% \toprule
% \multirow{2}{*}{} & \multicolumn{3}{c}{Generator Distribution Problem} & \multicolumn{3}{c}{Power Hardening Problem} \\
% \cmidrule(lr){2-4} \cmidrule(lr){5-7}
% & {$\lambda = \;$1e-4} & {1e-3} & {1e-2} & {$\lambda = \;$0} & {0.1} & {10} \\
% \midrule
% MSE ($\times 10^3$) & 165.6 & \textbf{158.7} &  & 9454 & 8660 & \textbf{8094} \\
% Decsion Cost & 334.7 & \textbf{325.8} &  & 8031 & 8048 & 9448 \\
% Regret & 88.4 & \textbf{79.50} &  & 347 & \textbf{343} & 555 \\
% \bottomrule
% \end{tabular}
% }
% \label{table:alb}
% \end{table}

% \begin{table}[!h]
% \centering
% \caption{Effect of $\lambda$ on Mobile Generator Deployment Problem.}
% \resizebox{0.9\linewidth}{!}{%
% \begin{tabular}{lcccccc}
% \toprule
% \multirow{2}{*}{Metric} & \multicolumn{4}{c}{$\lambda$} & \multicolumn{2}{c}{Baselines} \\
% \cmidrule(lr){2-5} \cmidrule(lr){6-7}
%                       & 0  & 1   & 10 & 20  & Two-stage & Online \\
% \midrule
% MSE ($\times 10^3$)    & 9.7 & 9.5 &  9.6  &   &  9.3 & /      \\
% Cost         & 6668.7 & 6609.5 & 6611.8  & & 6631.7 &  8163.7   \\
% Regret         & 153.0 & 135.7 &  138.0  & &  155.8 &   1648.1   \\
% \bottomrule
% \end{tabular}
% }
% \label{table:alb}
% \end{table}

\begin{table}[ht]
\centering
\caption{Effect of $\lambda$ on Mobile Generator Deployment Problem.}
\resizebox{0.9\linewidth}{!}{%
\begin{tabular}{lccccc}
\toprule
\multirow{2}{*}{Metric} & \multicolumn{3}{c}{$\lambda$} & \multicolumn{2}{c}{Baselines} \\
\cmidrule(lr){2-4} \cmidrule(lr){5-6}
                      & 0  & 1   & 10   & Two-stage & Online \\
\midrule
MSE ($\times 10^3$)    & 9.7 & 9.5 &  9.6  &  9.3 & /      \\
Cost         & 6668.7 & 6609.5 & 6611.8  & 6631.7 &  8163.7   \\
Regret         & 153.0 & 135.7 &  138.0  &  155.8 &   1648.1   \\
\bottomrule
\end{tabular}
}
\label{table:alb}
\end{table}

\subsection{Detailed Description on Online Algorithms for Mobile Generator Deployment Problem}
We include pesudocode for the proposed online baseline allocation methods in Alg.~\ref{alg:sms}.

\begin{algorithm}[H]
\caption{Observe-then-optimize algorithm for mobile generator deployment problem}
\label{alg:sms}
\textbf{Input:} Observations or forecasts $Y_{t,i}$, parameters $(\tau,\,N_g,\,\gamma)$, warehouse stock $s_w$, etc.\\
\textbf{Output:} Shipping decisions $\{x^{\text{to}}_{t,i}, x^{\text{back}}_{t,i}\}$ for each time $t$ and city $i$.
\begin{algorithmic}[1]
\FOR{$t = 1$ to $T$}
    \STATE \textit{---\;update warehouse and city stocks from previous shipments\;---}
    \FOR{each city $i$}
        \STATE Compute demand shortfall $d_{t,i} \leftarrow \max(0,\ \lceil Y_{t,i} / N_H\rceil - q_{t,i})$.
        \STATE $x^{\text{to}}_{t,i} \leftarrow \min(s_w(t),\, d_{t,i})$ \quad \textit{// send enough to cover shortfall}
        \STATE $x^{\text{back}}_{t,i} \leftarrow 0$ \quad  \textit{//  no return shipments}
    \ENDFOR
\ENDFOR
\end{algorithmic}
\end{algorithm}

% \begin{algorithm}[H]
% \caption{Marginal-Cost Myopic Shipping (MCMS)}
% \label{alg:mcms}
% \textbf{Input:} Observations or forecasts $Y_{t,i}$, cost parameters $(\tau,\,N_g,\,\gamma,\,c_{\text{transport}})$, etc.\\
% \textbf{Output:} Shipping decisions $\{x^{\text{to}}_{t,i}, x^{\text{back}}_{t,i}\}$ for each time $t$ and city $i$.
% \begin{algorithmic}[1]
% \FOR{$t = 1$ to $T$}
%     % \STATE \textit{update warehouse and city stocks from previous shipments}
%     \STATE \textit{---\;update warehouse and city stocks from previous shipments\;---}
%     \FOR{each city $i$}
%         \STATE Consider shipping \emph{back} surplus if $\gamma > c_{\text{transport}}$ or demand is predicted lower next step.
%         \STATE For each needed generator, compare $\text{marginal\_benefit} = \tau \times N_g$ with $\text{marginal\_cost} = c_{\text{transport}} + \gamma$.
%         \STATE Ship only if $\text{marginal\_benefit} > \text{marginal\_cost}$ and if warehouse has stock.
%     \ENDFOR
% \ENDFOR
% \end{algorithmic}
% \end{algorithm}
% \bibliographystyle{plain}
% \bibliographystyle{unsrt}
% \bibliography{ref}

% \newpage
% \appendix
% \ryan{
% We could simplify the formulation of task loss (\ref{eq:opt_power}) by constructing a diagonal matrix of number of customers at location $i$,
% \[
%   \mathbf{N} =
%   \begin{bmatrix}
%     \frac{1}{N_{1}} & & \\
%     & \ddots & \\
%     & & \frac{1}{N_{I}}
%   \end{bmatrix},
% \]
% then,
% \begin{align*}
%     g(\hat{\mathbf{S}}, \boldsymbol{x}) & = \frac{1}{I} \sum_{i=1}^I \frac{1}{n_i} \int_{t=0}^\infty \left( \mathbbm{1}(x_i = 0) \hat{Y}_i(t) + \mathbbm{1}(x_i = 1) \bar{Y}_i(t)\right)~dt \\
%     & = \frac{1}{I} \int_{t=0}^\infty \left(\mathbf{N}^{T}\bar{\mathbf{Y}}(t)^{T}\boldsymbol{x} - \mathbf{N}^{T}\hat{\mathbf{Y}}(t)^{T}(\mathbf{1}-\boldsymbol{x}) \right)~dt \\
%     & = \frac{1}{I} \int_{t=0}^\infty \left(\mathbf{N}^{T}(\bar{\mathbf{Y}}(t)- \hat{\mathbf{Y}}(t))^{T}\boldsymbol{x} + \mathbf{N}^{T}\hat{\mathbf{Y}}(t)^{T}\mathbf{1} \right)~dt \\
%     &  = \int_{t=0}^\infty \left(\mathbf{N}^T(\bar{\mathbf{Y}}(t)- \hat{\mathbf{Y}}(t))^{T}\boldsymbol{x} \right)~dt + \frac{1}{I} \int_{t=0}^\infty  \mathbf{N}^{T}\hat{\mathbf{Y}}(t)^{T}\mathbf{1} ~dt \\
%     & \propto \int_{t=0}^\infty \left(\mathbf{N}^T(\bar{\mathbf{Y}}(t)- \hat{\mathbf{Y}}(t))^{T}\boldsymbol{x} \right)~dt,
% \end{align*}
% which is in the form $\mathbf{c}^{T}\boldsymbol{x}$ since $x$ is not related to the integral with $t$, with $\mathbf{c} = \int_{t=0}^\infty \mathbf{N}^T(\bar{\mathbf{Y}}(t)- \hat{\mathbf{Y}}(t))^{T}~dt$
% }

% \begin{thebibliography}{00}
% \bibitem{b1} G. Eason, B. Noble, and I. N. Sneddon, ``On certain integrals of Lipschitz-Hankel type involving products of Bessel functions,'' Phil. Trans. Roy. Soc. London, vol. A247, pp. 529--551, April 1955.
% \bibitem{b2} J. Clerk Maxwell, A Treatise on Electricity and Magnetism, 3rd ed., vol. 2. Oxford: Clarendon, 1892, pp.68--73.
% \bibitem{b3} I. S. Jacobs and C. P. Bean, ``Fine particles, thin films and exchange anisotropy,'' in Magnetism, vol. III, G. T. Rado and H. Suhl, Eds. New York: Academic, 1963, pp. 271--350.
% \bibitem{b4} K. Elissa, ``Title of paper if known,'' unpublished.
% \bibitem{b5} R. Nicole, ``Title of paper with only first word capitalized,'' J. Name Stand. Abbrev., in press.
% \bibitem{b6} Y. Yorozu, M. Hirano, K. Oka, and Y. Tagawa, ``Electron spectroscopy studies on magneto-optical media and plastic substrate interface,'' IEEE Transl. J. Magn. Japan, vol. 2, pp. 740--741, August 1987 [Digests 9th Annual Conf. Magnetics Japan, p. 301, 1982].
% \bibitem{b7} M. Young, The Technical Writer's Handbook. Mill Valley, CA: University Science, 1989.
% \end{thebibliography}
% \vspace{12pt}
% \color{red}
% IEEE conference templates contain guidance text for composing and formatting conference papers. Please ensure that all template text is removed from your conference paper prior to submission to the conference. Failure to remove the template text from your paper may result in your paper not being published.


% {\appendices
% \section*{Proof of the First Zonklar Equation}
% Appendix one text goes here.
% You can choose not to have a title for an appendix if you want by leaving the argument blank
% \section*{Proof of the Second Zonklar Equation}
% Appendix two text goes here.}


% \section{References Section}
% You can use a bibliography generated by BibTeX as a .bbl file.
%  BibTeX documentation can be easily obtained at:
%  http://mirror.ctan.org/biblio/bibtex/contrib/doc/
%  The IEEEtran BibTeX style support page is:
%  http://www.michaelshell.org/tex/ieeetran/bibtex/
 
 % argument is your BibTeX string definitions and bibliography database(s)
%\bibliography{IEEEabrv,../bib/paper}
%
% \section{Simple References}
% You can manually copy in the resultant .bbl file and set second argument of $\backslash${\tt{begin}} to the number of references
%  (used to reserve space for the reference number labels box).

% \begin{algorithm}[!t]
% \caption{Global Decision-Focused Learning Algorithm}
% \textbf{Input:} Historical data 
% \(\mathcal{D} = \{(\boldsymbol{z}_k^i, \{y_k^i(t_j)\}_{j=0}^{T_i})\}_{i=1}^I\),  
% initial parameters \(\theta_0\), learning rate \(\eta\), trade-off parameter \(\lambda\), 
% number of epochs \(T\), and time intervals \(\{\Delta t_j\}\).\\[1mm]
% \textbf{Output:} Learned parameters \(\theta\).
% \begin{algorithmic}[1]
% \STATE Initialize \(\theta \gets \theta_0\).
% \FOR{epoch \(=1,\dots,T\)}
%     \STATE \textbf{// Decision-Focused Update}
%     \FOR{each event \(i=1,\dots,I\)}
%         \STATE Compute predicted trajectory \(\hat{\mathbf{S}}^i\) via Euler integration:
%         \[
%             \hat{\mathbf{S}}^i(t_{j+1}) = \hat{\mathbf{S}}^i(t_j) + f_\theta\big(\hat{\mathbf{S}}^i(t_j),\boldsymbol{z}_k^i\big)\Delta t_j,\quad j=0,\dots,T_i-1,
%         \]
%         with initial state \(\hat{\mathbf{S}}^i(0)\) given.
%         \STATE Compute decision loss for event \(i\):
%         \[
%          L_{\text{dec}}^i = g\Big(\boldsymbol{x}^*(\hat{\mathbf{S}}^i), \mathbf{S}^i\Big) - g\Big(\boldsymbol{x}^*(\mathbf{S}^i), \mathbf{S}^i\Big).
%         \]
%     \ENDFOR
%     \STATE Aggregate decision loss:
%     \[
%        L_{\text{dec}} = \frac{1}{I}\sum_{i=1}^I L_{\text{dec}}^i.
%     \]
%     \STATE Update parameters by decision-focused gradient:
%     \[
%        \theta \gets \theta - \eta\, \nabla_\theta L_{\text{dec}}.
%     \]
    
%     \STATE \textbf{// Prediction-Focused Update (MSE loss)}
%     \FOR{each mini-batch \(\mathcal{B}\subset \mathcal{D}\)}
%         \STATE Compute predicted trajectories \(\hat{\mathbf{S}}_{\mathcal{B}}\) via Euler integration.
%         \STATE Compute prediction loss:
%         \[
%            L_{\text{pred}} = \frac{1}{|\mathcal{B}|KT} \sum_{(i,k,j)\in \mathcal{B}} \Big[y_k^i(t_j) - \hat{Y}_k^i(t_j)\Big]^2.
%         \]
%         \STATE Update parameters by prediction-focused gradient:
%         \[
%            \theta \gets \theta - \eta\, \nabla_\theta L_{\text{pred}}.
%         \]
%     \ENDFOR
% \ENDFOR
% \RETURN \(\theta\).
% \end{algorithmic}
% \end{algorithm}



\subsection{Power Line Undergrounding}
\label{exp:power hardening}
Power line undergrounding is a grid-hardening measure against extreme meteorological events (such as hurricanes and heavy snowfalls) \cite{8278121, 7922545,AbiSamra2013,Shea2018}. Although effective, it involves substantial costs for the authority and causes significant disruptions to local communities \cite{9220164,328395}. The objective of the power line undergrounding problem is to select an optimal subset of locations for underground interventions under budget constraints \cite{7922545, 
AbiSamra2013, Shea2018}, in anticipation of an incoming hazard.

To formalize the decision-making problem, let $x_k$ be a binary variable indicating whether city~$k$ is selected for undergrounding. 
With $K$ cities in total, the decision vector is $\boldsymbol{x}=[x_1,\dots,x_K]^\top$, and $\hat{\mathbf{S}}$ denotes the predicted outage states. 
We then define:
\begin{equation}
\label{eq:opt_power}
\begin{aligned}
\min_{\boldsymbol{x}} \quad & g(\boldsymbol{x}, \hat{\mathbf{S}}),\\
\text{s.t.}\quad
  & \sum_{k=1}^{K} x_k \,\leq\, C,\\
  & x_k \,\in\, \{0,1\},\quad k = 1,\dots,K,
\end{aligned}
\end{equation}
where $g(\boldsymbol{x}, \hat{\mathbf{S}})$ is the decision loss that quantifies the impact of outages given the chosen undergrounding plan $\boldsymbol{x}$.
% % \ryan{Should I introduce the formal definition of the two problems, follow by gdf formulation?}
% % \ryan{remove redundant equation 16 and 17}
% Assuming only one foreseeable extreme weather events, the power line undergrounding problem is formally defined as follows:
% \begin{equation} 
% \begin{aligned}
% \boldsymbol{x}  = & \arH\min_{\boldsymbol{x}} \;\; g(\boldsymbol{x}, {\mathbf{\hat{M}}}), \\
% \text{s.t.} & \quad \boldsymbol{x}_k \in \{0, 1\} \\  
% % & \hat{Y}_k^i(t) = \left[ \hat{\mathbf{S}}_k^i(t) \right]_3 \\
% & \sum_{k = 1}^{K}{\boldsymbol{x}_k} - C \leq 0,  \\
% % & \boldsymbol{x}_i \in \{0, 1\} \;\; \forall i, \\
% \end{aligned}
% \label{eq:opt_power}
% \end{equation}


% Following the formulation of \eqref{eq:opt_dfl_ERM}, the GDF learning objective for power line undergrounding optimization problem is:
% \begin{equation} 
% \begin{aligned}
% \min_{\theta} \;\;  & g(\boldsymbol{x}^*, \mathbf{S}) + \lambda \cdot \text{MSE}(\hat{Y}, Y) \\
% \text{s.t.} \quad & \boldsymbol{x}^* = \arg \min_{g(\boldsymbol{x}) \leq 0,\boldsymbol{x}_k \in \{0, 1\} } g(\boldsymbol{x}, \mathbf{\hat{M}}), \\  
% & h(\boldsymbol{x}) = \sum_{k = 1}^{I}{\boldsymbol{x}_k} - C.  \\
% % & \boldsymbol{x}_i \in \{0, 1\} \;\; \forall i, \\
% \end{aligned}
% \label{eq:opt_power_gdf}
% \end{equation}
% where $C$ is the given resource constraint limiting the number of cities that can have power line udnergrounded, $\hat{Y}$ is the prediction of the number of outage customers based on states prediction $\hat{\mathbf{S}}$, and $g$ is the decision loss function.

We adopt the System Average Interruption Duration Index (SAIDI)~\cite{9955492} to measure how outages affect the population. 
Let $Y_k(t)$ be the (true) number of outages at city $k$ and time $t$, and $N_k$ be the total number of customers in city~$k$. 
Since undergrounding is assumed fully effective, a city $k$ with $x_k=1$ incurs no further outages from the event. 
Hence, the decision loss is:
\begin{equation}
\label{eq:saidi}
g(\boldsymbol{x}, \mathbf{S}) 
 \;=\; \frac{1}{K}\,\sum_{k=1}^{K} 
  \frac{1}{N_k} \int_{0}^{\infty} 
    \Bigl[\bigl(1 - x_k\bigr)\,Y_k(t)\Bigr]\,
  dt.
\end{equation}
The optimal solution $\boldsymbol{x}^*$ to~\eqref{eq:opt_power} is then the subset of cities to be undergrounded in order to minimize the total outage impact. 
Its performance is evaluated via $g(\boldsymbol{x}^*, \mathbf{S})$ using true outage data $\mathbf{S}$.
% In our study, we adopt System Average Interruption Duration Index (SAIDI) \cite{6209381, 9955492} as the decision loss function:
% \begin{align}
%     g(\boldsymbol{x}, \mathbf{S}) = \frac{1}{K} \sum_{k=1}^K \frac{1}{N_k} \int_{t=0}^\infty &\left(  \mathbbm{1}({x}_i = 0) {Y}_k(t) \right. \nonumber \\
%     & \left. + \mathbbm{1}({x}_k = 1) {Y}^{'}_k(t)\right)~dt,
%     \label{eq:saidi}
% \end{align}
% where ${Y}_k(t)$ denote the number of outages at city $k$ and time $t$ based on outage stages $\mathbf{S}$, and ${Y}^{'}_k(t)$ represents the predicted number of outages under the undergrounding intervention. We further assume all undergrounding measures are effective, setting ${Y}^{'}_k(t)$ to 0 in \eqref{eq:saidi}.
% In this case, the optimal solution to \eqref{eq:opt_power}, $\boldsymbol{x}^*$, is the optimal set of cities that should have power line undergrounded to minimize the total impact of the predicted outages. 
% The decision $\boldsymbol{x}^*$ can be evaluated with SAIDI using the actual outage data, $g(\boldsymbol{x}^*, \mathbf{S})$. 

% \subsection{Reformulation}

% Following the Global Decision-Focused (GDF) learning paradigm in~\eqref{eq:opt_dfl_ERM}, we embed the optimization problem~\eqref{eq:opt_power} with the SAIDI-based loss~\eqref{eq:saidi} into the downstream training objective. 
% This end-to-end approach ensures that the predictor $\hat{\mathbf{S}}$ is learned to directly minimize the decision loss of power line undergrounding. 
% We report the detailed GDF procedure and its numerical results in Section~\ref{sec:power-line-undergrounding}.

\subsection{Implementation Details of \texttt{GDF}}
\label{append:implement}

To get the gradient of \eqref{eq:regularized-LP}, let matrix $H$ encode coefficients for all the linear constraints $\quad H\boldsymbol{x} \leq \boldsymbol{a}$, and ${\xi^i}$ represents a reformulation of the ground truth cost factors derived from $\mathbf{S}^i$, such that $g(\boldsymbol{x},\mathbf{S}^i) = {\xi^i}^T \boldsymbol{x} + \boldsymbol{b}$. Using the KKT conditions of the Lagrangian of the problem, the gradient of the QP in \eqref{eq:regularized-LP} is:
\begin{equation}
\nabla_{\theta} L_{\text{QP}} = {\xi^i}^T K^{-1} \nabla_{\theta} \hat{\mathbf{S}}^i, \quad
K =
\begin{bmatrix}
2\rho_x I & H^T \\
H & 0
\end{bmatrix},
\label{eq:event_decision_gradient}
\end{equation}

% \ryan{$\theta$ used in \eqref{eq:ode}, will change}
% Theoretical results by \cite{wilder2019melding} demonstrated guarantees the differentiability of the optimal solution to \eqref{eq:regularized-LP} under mild conditions, the approximation error introduced by regularization is bounded. 
% Therefore, the gradient for \eqref{eq:regularized-LP} can be approximated with \eqref{eq:event_decision_gradient}
% \begin{equation}
% \nabla_{\beta, \gamma} L_{\text{GDF}} \approx \sum_i {\theta^i}^T \cdot K^{-1} \cdot \nabla_{\beta, \gamma} \hat{\mathbf{S}}^i,
% \label{eq:quadratic}
% \end{equation}
And $\nabla_{\beta, \gamma} \hat{\mathbf{S}}^i$ can obtained via backpropagation through the neural ODE model parameters $\{\theta_U, \theta_R\}$ \cite{chen2018neural}.
This gradient aims to improve decision quality across all cities and all events $i \in \{1, \dots, I\}$. Therefore, we refer to it as \emph{decision-focused gradient}.


To be more specific about derivation of \eqref{eq:event_decision_gradient}, the optimal solution $\boldsymbol{x}^*$ must satisfy the KKT conditions. We define the Lagrangian:
\begin{equation}
\mathcal{L}(\boldsymbol{x}, \lambda) = g(\boldsymbol{x}, \mathbf{S}) + \rho \|\boldsymbol{x}\|_2^2 + \lambda^T (H \boldsymbol{x} - \boldsymbol{a}).
\end{equation}

The stationarity of $x^*$ for optimality gives:
\begin{equation}
\nabla_x g(\boldsymbol{x}, \mathbf{S}) + 2\rho \boldsymbol{x} + H^T \lambda = 0.
\end{equation}

Since the optimal decision $\boldsymbol{x}^*$ satisfies the above KKT system, we can apply implicit differentiation. Taking the total derivative with respect to the predicted system state $\hat{\mathbf{S}}$:

\begin{equation}
\begin{bmatrix}
\nabla^2_{xx} \mathcal{L} & H^T \\
H & 0
\end{bmatrix}
\begin{bmatrix}
\frac{d\boldsymbol{x}^*}{d\hat{\mathbf{S}}} \\
\frac{d\lambda}{d\hat{\mathbf{S}}}
\end{bmatrix}
=
\begin{bmatrix}
\frac{d (-\nabla_x g(\boldsymbol{x}, \mathbf{S}))}{d\hat{\mathbf{S}}} \\
0
\end{bmatrix}.
\label{eq:kkt_diff}
\end{equation}

where
\begin{equation}
\frac{d\boldsymbol{x}^*}{d\hat{\mathbf{S}}} = - K^{-1} \frac{d \nabla_x g}{d\hat{\mathbf{S}}},
\end{equation}
and the KKT matrix $K$ is defined as:
\begin{equation}
K =
\begin{bmatrix}
2\rho I & H^T \\
H & 0
\end{bmatrix}.
\end{equation}
since $\nabla^2_{xx} \mathcal{L} = 2\rho I$ etc.

This allows us to compute the gradient of the loss function with respect to model parameters:
\begin{equation}
\nabla_{\theta} L_{\text{QP}} = {\xi^i}^T K^{-1} \nabla_{\theta} \hat{\mathbf{S}}^i.
\end{equation}

In practice, we regularize the \texttt{GDF} loss with prediction-focused gradient,
% , we employ mini-batch stochastic gradient descent (SGD), 
as the prediction error is localized and requires fine-grained information for each individual sample unit. 
Specifically, we construct mini-batches $\mathcal{B} \subset \mathcal{D}$ from the training dataset $\mathcal{D} = \{\boldsymbol{z}_k^i, y_k^i(t)\}$.
% Each mini-batch $\mathcal{B}$ consists of randomly sampled training instances $(\boldsymbol{z}_k^i, y_k^i(t))$.

The neural ODE model generates forecasts $\hat{\mathbf{S}}_{\mathcal{B}} = f_{\beta, \gamma}(\boldsymbol{z}_{\mathcal{B}})$ for each sample in the batch. From these forecasts, we extract the predicted values $\hat{Y}_{\mathcal{B}}$, which are then used to compute the MSE loss:
\begin{equation}
L_{\text{MSE}} = \sum_{\mathcal{B} \subset \mathcal{D}}\frac{1}{|\mathcal{B}|} \sum_{(i,k,t)\in \mathcal{B}} \bigl(y_k^i(t) - \hat{Y}_k^i(t)\bigr)^2.
\end{equation}

Finally, for each epoch, the model parameters $\theta$ is updated using a combination of both loss gradients, balanced by a hyperparameter $\lambda$:
\begin{equation}
\nabla_{\theta} L_{\text{total}} = \nabla_{\theta} L_{\text{GDF}} + \lambda \nabla_{\theta} L_{\text{MSE}}.
\end{equation}

\begin{figure}[!t]
\centering
% \subfigure[Total Cost vs. Transportation Cost Factor \ryan{take exponential}]{%
%   \includegraphics[width=0.49\columnwidth]{new_plots/Total_Cost_vs_Transportation_Cost_Factor.pdf}}
% \hfil
\includegraphics[width=0.9\columnwidth]{new_plots/Total_Regret_vs_Transportation_lambda_mse.pdf}
% \caption{Synthetic mobile generator deployment problem performance across transportation cost factors.
% }
\caption{Regret performance of the \texttt{GDF} method with varying $\lambda$ values across different transportation cost factors in synthetic data for mobile generator deployment problem, benchmarked against the Two-stage method.}
\label{fig:Total_Regret_vs_Transportation_lambda_mse}
\vspace{-.1in}
\end{figure}


\subsection{Additional Results for the Mobile Generator Deployment Problem}

This section presents additional results and visualizations for the mobile generator deployment problem.

As demonstrated in the ablation study, when $\lambda$ is large, the MSE dominates model training, reducing the advantage of \texttt{GDF} over MSE-trained models in decision quality. For more detailed ablation results in the mobile generator deployment problem, see Fig.~\ref{fig:Total_Regret_vs_Transportation_lambda_mse}, which shows that as $\lambda$ increases, the \texttt{GDF} results become similar to those of the Two-stage method, resulting in larger regret and higher variance.

Further more, Table~\ref{table:Generator Distribution Problem_synthetic} summarizes the out-of-sample performance for the generator deployment problem on synthetic data across three transportation cost factors (100, 500, and 1000). As the transportation cost increases, the improvement in decision quality for \texttt{GDF} compared to the Two-stage methods becomes more apparent. This highlights the importance of scheduling and proactive actions when transportation costs are high, demonstrating the clear advantage of \texttt{GDF}.

We also provide additional visualizations of the deployment schemes under varying conditions. Comparing  Fig.~\ref{fig:SIR_plot_4rows_seed0_tc10_gamma2.0_G10_tau1_Ng100.0_lambda0} and Fig.~\ref{fig:SIR_plot_4rows_seed0_tc1000_gamma2.0_G10_tau1_Ng100.0_lambda0.1}, we observe that when travel costs are low, the online strategy closely approximates the optimal strategy, resulting in small regret. This is because low travel costs allow the online method to frequently move generators based on previous day data at minimal expense, leading to near-optimal regret, whereas the Two-stage and \texttt{GDF} methods rely more on predictions, and the associated noise can diminish the benefits of prediction or proactive allocation under these conditions.

% Comparing Fig.\ref{fig:SIR_plot_4rows_seed0_tc1000_gamma2.0_G3_tau1_Ng100.0_lambda0.1} and Fig.\ref{fig:SIR_plot_4rows_seed0_tc1000_gamma2.0_G10_tau1_Ng100.0_lambda0.1}, we find that limited resources make the online strategy even less effective than scenarios where resources are abundant. Without proactive action based on prediction, the strategy must reposition the limited number of generators more frequently to respond to real-time outages, leading to higher transportation costs and, consequently, greater overall regret, compared to \texttt{GDF} and Two-stage methods.

In contrast, comparing Fig.\ref{fig:SIR_plot_4rows_seed0_tc1000_gamma2.0_G3_tau1_Ng100.0_lambda0.1} and Fig.\ref{fig:SIR_plot_4rows_seed0_tc1000_gamma2.0_G10_tau1_Ng100.0_lambda0.1}, we find that limited resources degrade the online strategy’s performance. Without proactive planning, fewer generators must be relocated more frequently with an online method, incurring higher transportation costs and overall regret compared to \texttt{GDF} and Two-stage approaches.

\begin{table*}[!t]
\centering
\caption{Out-of-Sample Performance for Generator Distribution problem with Synthetic Data. Results are averaged over 3 repeated
experiments with standard error (SE) in the brackets.}
\resizebox{\linewidth}{!}{%
\begin{tabular}{llllllllll}
\toprule
\multirow{2}{*}{{Model}} &
\multicolumn{3}{c}{Transportation Cost = 100} &
\multicolumn{3}{c}{Transportation Cost = 500} &
\multicolumn{3}{c}{Transportation Cost = 1000} \\
\cmidrule(lr){2-4} \cmidrule(lr){5-7} \cmidrule(lr){8-10}
& {MSE} & {Cost} & {Regret} & {MSE} & {Cost} & {Regret} & {MSE} & {Cost} & {Regret} \\
\midrule
Ground Truth & / & 6515.6 (74.4) & 0 & / & 11271 & 0 & / & 16271 & 0 \\
Online (lag = 1) & / & 8163.7 (293.4) & 1648.1 (358.6) & / & 19630.4 (1514.8) & 8358.7 (1576.8) & / & 33963.7 (3042.2) & 17692.1 (3104.4) \\
Online (lag = 3) & / & 8978.6 (277.6) & 2462.9 (313.8) & / & 22045.2 (1493.3) & 10773.6 (1522.9) & / & 38378.6 (3020.2) & 22106.9 (3050.4) \\
Online (lag = 5) & / & 9618.9 (57.3)  & 4939.9 (111.1) & / & 22152.2 (498.5) & 16497.9 (557.8) & / & 37818.9 (1074.7) & 32164.5 (1135.0) \\
Two-stage       & \textbf{9336.0 (975.6)} & 6671.5 (4328.7) & 155.8 (33.3)  & 9336.0 (975.6) & 11465.8 (94.6) & 194.2 (107.3) & 9336.0 (975.6) & 16517.8 (133.9) & 246.1 (112.7) \\
% Two-stage &  &  &  &  &  &  &  &  &  \\
\texttt{GDF} & 9709.6 (3055.6) & 6668.7 (4365.9) & \textbf{153.0 (19.8)} & \textbf{8671.2 (1553.7)} & 11428.1 (87.1) & \textbf{156.4 (44.2)} & \textbf{6981.1 (1874.0)} & 16395.9 (89.0) & \textbf{124.2 (36.3)} \\
% \texttt{GDF} &  &  &  &  &  &  &  &  &  \\
\bottomrule
\end{tabular}
}
\label{table:Generator Distribution Problem_synthetic}
\vspace{-.1in}
\end{table*}


% \begin{figure}[t!]
% \centering
% % \subfigure[Total Cost vs. Transportation Cost Factor \ryan{take exponential}]{%
% %   \includegraphics[width=0.49\columnwidth]{new_plots/Total_Cost_vs_Transportation_Cost_Factor.pdf}}
% % \hfil
% \includegraphics[width=0.9\columnwidth]{new_plots/Total_Regret_vs_Transportation_Cost_Factor.pdf}
% \caption{Synthetic mobile generator deployment problem performance across transportation cost factors.}
% \end{figure}




% \begin{figure}[t!]
% \centering
% \includegraphics[width=\linewidth]{new_plots/plot_4rows_seed1_tc2000_gamma2.0_G5.0_tau1.0_Ng100.0_lambda0.pdf}
% % \multicolumn{3}{c}{\small Predicted Outages with \textit{Decision Focused Learning}} \\
% \caption{Travel cost = 2000, economic cost = 1
% % \woody{As we discussed earlier, we may also want to show other optimization results like this figure under different settings, like generators are insufficient, high vs low transportation costs, etc.}
% % \ryan{Sure, I will include more results in the appendix first and pick some to demo here. This plot takes space so we have to think about the best to present here.}
% }
% % \vspace{-.1in}
% \end{figure}


% \begin{figure}[t!]
% \centering
% \includegraphics[width=\linewidth]{new_plots/plot_4rows_seed0_tc1000_gamma2.0_G5.0_tau1.0_Ng100.0_lambda10.pdf}
% % \multicolumn{3}{c}{\small Predicted Outages with \textit{Decision Focused Learning}} \\
% \caption{Travel cost = 1000, economic cost = 1, lambda = 10
% % \woody{As we discussed earlier, we may also want to show other optimization results like this figure under different settings, like generators are insufficient, high vs low transportation costs, etc.}
% % \ryan{Sure, I will include more results in the appendix first and pick some to demo here. This plot takes space so we have to think about the best to present here.}
% }
% % \vspace{-.1in}
% \end{figure}


% \begin{figure}[t!]
% \centering
% \includegraphics[width=\linewidth]{new_plots/plot_4rows_seed1_tc1000_gamma2.0_G5.0_tau1.0_Ng100.0_lambda0.pdf}
% % \multicolumn{3}{c}{\small Predicted Outages with \textit{Decision Focused Learning}} \\
% \caption{Travel cost = 1000, economic cost = 1
% % \woody{As we discussed earlier, we may also want to show other optimization results like this figure under different settings, like generators are insufficient, high vs low transportation costs, etc.}
% % \ryan{Sure, I will include more results in the appendix first and pick some to demo here. This plot takes space so we have to think about the best to present here.}
% }
% % \vspace{-.1in}
% \end{figure}


% \begin{figure}[t!]
% \centering
% \includegraphics[width=\linewidth]{new_plots/plot_4rows_seed0_tc400_gamma2.0_G5.0_tau1.0_Ng100.0_lambda0.pdf}
% % \multicolumn{3}{c}{\small Predicted Outages with \textit{Decision Focused Learning}} \\
% \caption{Travel cost = 400, economic cost = 1
% }
% % \vspace{-.1in}
% \end{figure}


% \begin{figure}[t!]
% \centering
% \includegraphics[width=\linewidth]{new_plots/plot_4rows_seed0_tc400_gamma2.0_G5_tau5.0_Ng100.0_lambda0.pdf}
% % \multicolumn{3}{c}{\small Predicted Outages with \textit{Decision Focused Learning}} \\
% \caption{
% }
% % \label{fig:diesel}
% % \vspace{-.1in}
% \end{figure}

% \begin{figure}[t!]
% \centering
% \includegraphics[width=\linewidth]{new_plots/plot_4rows_seed1_tc200_gamma2.0_G5.0_tau1.0_Ng100.0_lambda0.pdf}
% % \multicolumn{3}{c}{\small Predicted Outages with \textit{Decision Focused Learning}} \\
% \caption{Travel cost = 200, economic cost = 1
% }
% % \vspace{-.1in}
% \end{figure}

% \woody{Consider adding a figure showing a service territory containing the locations of units and warehouses...}


% \begin{figure}[t!]
% \centering
% \includegraphics[width=\linewidth]{new_plots/SIR_plot_4rows_seed0_tc400_gamma2.0_G5_tau1_Ng100.0_lambda0.pdf}
% \caption{A synthetic instance of the mobile generator deployment problem for a system with three cities and five generators ($Q_w = 5$). The $y$-axis shows the number of households experiencing outages over time. In this example, the transportation cost is set to $c = 400$, the customer interruption cost to $\tau = 1$, and the operational cost to $\gamma = 2$. The online method operates with a one-day observation lag. A uniform travel time of $\delta_t = 0$ is assumed.}
% % \woody{Can you add green and orange curves to the two-stage as well? I guess there might be a more significant difference among these two curves comparing to GDF. }
% \label{fig:diesel}
% \vspace{-.1in}
% \end{figure}

\begin{figure}[!t]
\centering
\includegraphics[width=\linewidth]{new_plots/SIR_plot_4rows_seed0_tc10_gamma2.0_G10_tau1_Ng100.0_lambda0.pdf}
% \multicolumn{3}{c}{\small Predicted Outages with \textit{Decision Focused Learning}} \\
% \caption{A synthetic example of the mobile generator deployment problem for a system with three cities and five generators ($Q_w = 10$). The $y$-axis represents the number of outaged households. In this example, the transportation cost is set to $c = 10$, the customer interruption cost is $\tau = 1$, and the operational cost is $\gamma = 2$. Observation lag of online method is set to $1$.
% }
\caption{A synthetic instance of the mobile generator deployment problem for a system with three cities and five generators ($Q_w = 10$). The $y$-axis shows the number of households experiencing outages over time. In this example, the transportation cost is set to $c = 10$, the customer interruption cost to $\tau = 1$, and the operational cost to $\gamma = 2$. The online method operates with a one-day observation lag. Travel time $\delta_t = 0$ is neglected in this case.}
\label{fig:SIR_plot_4rows_seed0_tc10_gamma2.0_G10_tau1_Ng100.0_lambda0}
\vspace{-.1in}
\end{figure}

\begin{figure}[!]
\centering
\includegraphics[width=\linewidth]{new_plots/SIR_plot_4rows_seed0_tc1000_gamma2.0_G3_tau1_Ng100.0_lambda0.1.pdf}
% \multicolumn{3}{c}{\small Predicted Outages with \textit{Decision Focused Learning}} \\
\caption{
% A synthetic instance of the mobile generator deployment problem for a system with three cities and five generators ($Q_w = 3$). The $y$-axis shows the number of households experiencing outages over time. In this example, the transportation cost is set to $c = 1000$, the customer interruption cost to $\tau = 1$, and the operational cost to $\gamma = 2$. The online method operates with a one-day observation lag. A uniform travel time of $\delta_t = 0$ is assumed.
A synthetic instance of the mobile generator deployment problem for a system with three cities and five generators ($Q_w = 3$). The $y$-axis shows the number of households experiencing outages over time. In this example, the transportation cost is set to $c = 1000$, the customer interruption cost to $\tau = 1$, and the operational cost to $\gamma = 2$. The online method operates with a one-day observation lag. Travel time $\delta_t = 0$ is neglected in this case.
}
% \vspace{-.1in}
\label{fig:SIR_plot_4rows_seed0_tc1000_gamma2.0_G3_tau1_Ng100.0_lambda0.1}
\vspace{-.1in}
\end{figure}

\begin{figure}[ht]
\centering
\includegraphics[width=\linewidth]{new_plots/SIR_plot_4rows_seed0_tc1000_gamma2.0_G10_tau1_Ng100.0_lambda0.1.pdf}
% \multicolumn{3}{c}{\small Predicted Outages with \textit{Decision Focused Learning}} \\
\caption{
A synthetic instance of the mobile generator deployment problem for a system with three cities and five generators ($Q_w = 10$). The $y$-axis shows the number of households experiencing outages over time. In this example, the transportation cost is set to $c = 1000$, the customer interruption cost to $\tau = 1$, and the operational cost to $\gamma = 2$. The online method operates with a one-day observation lag. Travel time $\delta_t = 0$ is neglected in this case.
}
\label{fig:SIR_plot_4rows_seed0_tc1000_gamma2.0_G10_tau1_Ng100.0_lambda0.1}
\vspace{-.1in}
\end{figure}



\subsection{Additional Results for Power Line Undergrounding}
Fig.~\ref{fig:hardenining_real_ma} shows the predicted outage trajectories for all Massachusetts counties using the \texttt{GDF} model compared with groundtruth and Two-stage. Notably, the model exaggerates outages for selected counties—a deliberate strategy to prioritize resource allocation. This controlled overestimation, while slightly increasing MSE. Overall, it effectively reduces SAIDI and regret compared to the Two-stage baseline, demonstrating that decision-focused training can enhance overall scheduling performance.
\begin{figure}[!t]
\centering
% \subfigure[Total Cost vs. Transportation Cost Factor \ryan{take exponential}]{%
%   \includegraphics[width=0.49\columnwidth]{new_plots/Total_Cost_vs_Transportation_Cost_Factor.pdf}}
% \hfil
\includegraphics[width=\columnwidth]{new_plots/test_set_county_predictions.png}
\caption{\texttt{GDF} is overestimating outage in certain counties to prioritize resource allocation at the cost of MSE. 
}
\label{fig:hardenining_real_ma}
\vspace{-.1in}
\end{figure}


\subsection{Extended Literature Review for DFL and Differentiable Optimization}
\noindent\emph{Decision-Focused Learning}.
Decision-focused learning (DFL) has emerged as a powerful framework for integrating predictive models with downstream optimization tasks. Unlike traditional two-stage approaches, which first train standalone prediction models and then use their predictions as input parameters to optimal decision models, DFL aligns the prediction model's training loss with the objective function of the downstream optimization. This concept is enabled in gradient descent training by backpropagating gradients through the solution to an optimization problem. When the optimization is a differentiable function of its parameters, this can be implemented via implicit differentiation of optimality conditions such as KKT conditions \cite{amos2017optnet, gould2021deep} or fixed-point conditions \cite{kotary2023folded, wilder2019end}.  When the optimization is nondifferentiable, it can instead be implemented by means of various approximation techniques \cite{kotary2021end, mandi_decision-focused_2024}.  
Unlike traditional two-stage approaches, which first train standalone prediction models and then use their predictions as inputs for decision-making, DFL directly embeds the optimization problem within the learning process. This allows the learning model to focus on the variables that matter most for the final decision \cite{mandi_decision-focused_2024}.

Elmachtoub and Grigas \cite{elmachtoub_smart_2022} first proposed the \textit{Smart Predict-and-Optimize} (SPO) framework, which introduced a novel method for formulating optimization problems in the prediction process. SPO essentially bridges the gap between predictive modeling and optimization by constructing a decision-driven loss function that reflects the downstream task. However, the SPO framework only addresses linear optimization problems and does not extend well to more complex combinatorial tasks. 

The most-studied class of nondifferentiable optimization problems in decision-focused learning (DFL) involves linear programs (LPs). Notably, the Smart Predict-and-Optimize (SPO) framework by Elmachtoub and Grigas \cite{elmachtoub_smart_2022} introduced a convex surrogate upper bound to approximate subgradients for minimizing the suboptimality of LP solutions based on predicted cost coefficients. 
The most-studied class of nondifferentiable optimizations are linear programs (LPs). Elmachtoub and Grigas \cite{elmachtoub_smart_2022} proposed the Smart Predict-and-Optimize (SPO) framework for minimizing the suboptimality of solutions to a linear program  as a function of its predicted cost coefficients.
Despite this function being inherently non-differentiable, a convex surrogate upper-bound is used to derive informative subgradients. Wilder et al. \cite{wilder2019melding} propose to smooth linear programs by augmenting their objectives with small quadratic terms  \cite{amos2017optnet} and differentiating the resulting KKT conditions.
% Vlastelica et al. \cite{vlastelica2019differentiation} proposed a method of finite differences for estimating gradients through LP's, which was later generalized as a form of proximal gradient descent training \cite{paulus2024lpgd}.
A method of smoothing LP's by noise perturbations was proposed in \cite{berthet2020learning}.
Differentiation through combinatorial problems, such as mixed-integer programs (MIPs), is generally performed by adapting the approaches proposed for LP's, either directly or on their LP relaxations. For example, Mandi et al. \cite{mandi2020smart} demonstrated the effectiveness of the SPO  method in  predicting cost coefficients to MIPs. Vlastelica et al. \cite{vlastelica2019differentiation} demonstrated their method directly on MIPs, and Wilder et al. \cite{wilder2019melding} evaluated their approach on LP relaxations of MIPs.

%This method improves upon traditional predict-then-optimize approaches by training the model to prioritize decision quality rather than prediction accuracy alone. Their work demonstrated how incorporating the decision-making problem directly into the learning phase can yield significant improvements in optimization-driven tasks, especially when data quality is limited or noisy.
%\james{resume here}
%Vlastelica et al. \cite{vlastelica2019differentiation} introduced a method to differentiate through combinatorial optimization problems by approximating the non-differentiable optimization process using perturbations. %This enables learning models to be optimized directly for tasks that involve discrete decision-making problems, improving performance in settings where the exact form of the optimization problem affects the predictions.

%Wilder et al. \cite{wilder2019melding} address the challenges posed by the combinatorial nature of such problems with a quadratic relaxation, which transforms the original combinatorial problem into a convex one. This relaxation enables the use of efficient optimization techniques while preserving the key structural properties of the problem. Their work demonstrated the effectiveness of DFL in real-world applications, such as resource allocation, where uncertainty in the data influences both predictions and downstream decision-making.
Building on these works, we extend DFL to spatio-temporal decision-making for power grid resilience management. Our approach employs quadratic relaxations to enable gradient backpropagation through MIPs \cite{wilder2019melding}, thereby integrating a spatio-temporal ODE model for power outage forecasting directly into the optimization process. Additionally, we introduce a Global Decision-Focused Framework that combine prediciton error with decision losses across geophysical units, improving grid resilience against extreme natural events and bridging the gap between localized predictions and system-wide decisions.

% This not only ensures precise outage predictions but also tailors those predictions to i

\vspace{.1in}
\noindent\emph{Differentiable Optimization}.
Differentiable optimization (DO) techniques have demonstrated significant potential in integrating predictive models with optimization problems. By enabling the computation of gradients through optimization processes, DO facilitates the seamless incorporation of complex system objectives into machine learning models, thereby enhancing decision-making capabilities \cite{cvx}.
Recent extensions of DO methods have tackled challenges beyond standard optimization tasks. For example, distributionally robust optimization (DRO) problems have been addressed using differentiable frameworks to handle prediction tasks under worst-case scenarios. For instance, \cite{zhu2022distributionally, chen2025uncertainty} employed DO-based techniques to improve uncertainty quantification and robust learning, effectively addressing data scarcity and enhancing resilience modeling.
% \ryan{I think the DRO introduction is abrupt here. Consider a new paragraph}. 
% These approaches highlight the importance of seamlessly integrating predictive and optimization models, particularly in high-stakes scenarios where decision quality directly impacts resilience outcomes.

Beyond predictive modeling, DO has advanced solutions in combinatorial and nonlinear optimization. Techniques such as implicit differentiation of KKT conditions \cite{amos2017optnet} and fixed-point conditions \cite{kotary2023folded} address differentiable constraints, while approximation methods, including noise perturbation \cite{berthet2020learning} and smoothing techniques \cite{vlastelica2019differentiation}, enable gradient computation for nondifferentiable tasks.

These advancements underscore DO’s pivotal role in bridging predictive modeling and optimization, especially where decision quality critically affects system resilience. In this work, DO is employed to align spatio-temporal outage predictions with grid optimization objectives, enabling robust strategies for generator deployment and power line undergrounding. 
% By embedding optimization within neural networks, our approach addresses uncertainties across diverse resilience challenges, providing actionable insights to enhance power system reliability and preparedness. 
% ====== Appendix References ======
\end{document}


