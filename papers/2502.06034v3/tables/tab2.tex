\begin{table}[t]
    \centering
    \begin{tabular}{llrrr}
   \toprule
      \hspace{-2mm}  Model & $\# \theta$ & FG-Acc & FG-IoU & FG-Loss \\
    \midrule
       \hspace{-2mm} U-Net 2c \hspace{-2mm}    & 31K      & 0.66 ± 0.27 & 0.60 ± 0.28 & 1.19 ± 0.56 \\
       \hspace{-2mm} U-Net 3c  \hspace{-2mm}   & 69K      & 0.90 ± 0.10 & 0.87 ± 0.13 & 0.56 ± 0.26 \\
     \midrule
       \hspace{-2mm} NWM \hspace{-3mm} & 55K      & 0.96 ± 0.01 & 0.93 ± 0.01 & 0.15 ± 0.02 \\
    \midrule
       \hspace{-2mm} U-Net 4c  \hspace{-2mm}   & 122K     & 0.97 ± 0.00 & 0.96 ± 0.01 & 0.24 ± 0.04 \\
       \hspace{-2mm} U-Net 5c  \hspace{-2mm}   & 190K     & 0.98 ± 0.00 & 0.97 ± 0.00 & 0.17 ± 0.05 \\
     \bottomrule
    \end{tabular}
    \vspace{-1mm}
    \caption{
    \textbf{Wave-based models outperform comparably sized U-Net models on more challenging Multi-MNIST segmentation.} 
    U-Net \#c refers to the number of feature maps output by the first layer, doubling each layer thereafter, and $\# \theta$ refers to the number of parameters. Rows sorted by $\# \theta$.
    Results are from 12 random seeds, displayed as $mean \pm std$.}
    \vspace{-4mm}
    \label{tab:multi-mnist}
\end{table}