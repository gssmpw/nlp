\section{Motivation: Hearing the Shape of a Drum}
To build intuition for how traveling waves may integrate information over space, we take inspiration from the famous mathematical question `\emph{Can one hear the shape of a drum?}' posed by Mark \cite{kac_can_1966}. Simply put, this question asks whether the boundary conditions of an idealized drum head are uniquely identified by the frequencies at which the drum head will vibrate. 
Intuitively, when one strikes a drum head, this initial disturbance will propagate outwards as a transient traveling wave until it reaches the fixed boundary conditions where it will reflect with a phase shift. This reflected wave will thus have \emph{collected information} about the boundary, and serves to bring it back towards the center. These waves will eventually collide with other reflected waves from all edges of the shape, and combine in a superposition of wavefronts, eventually resulting in a solution which is a superposition of discrete normal modes which are constrained by the boundary of the shape itself.

At a high level, Kac's question investigates one specific mechanism by which traveling waves may be used to integrate global information, by evolving a spatially-local dynamical system (a wave equation) to a steady-state solution determined by global conditions (the fixed drum boundary). While there exist many other mechanisms by which traveling waves can be considered to transfer and combine information, such as through delay-line mechanisms \citep{Jeffress1948APT} known to exist in the brainstems of owls \citep{owl_time}, or through more complex mechanisms such as interfering wave fronts \citep{gong2009distributed, pwc}, the steady-state solution mechanism yields a powerful and well-understood starting point for us to begin building models.

Formally, a ``drum'' in this problem is considered to be a perfectly elastic two-dimensional membrane whose vertical displacement over space and time is denoted $u(x, y, t)$. The drum head is considered to be stretched under uniform tension to a boundary of shape $\Omega$, such that its dynamics satisfy the two-dimensional wave equation with constant wave-speed $c$:
\begin{equation}
\label{eq:wave}
\frac{\partial^2 u}{\partial t^2} = c^2 \nabla^2 u  = c^2 (\frac{\partial^2 u}{\partial x^2} + \frac{\partial^2 u}{\partial y^2}).
\end{equation}
In the original problem, the drum head is constrained to be `clamped' to 0 displacement on the boundary, known as a Dirichlet boundary condition. This is often written as $u|_{\partial \Omega} = 0$. Since we are interested in steady-state oscillatory solutions, we can assert they must take the form of `normal modes' $\phi_k(x, y)$ with associated oscillation frequencies $\omega_k$:
\begin{equation}
    u(x, y, t) = \phi_k(x, y) \cos(\omega_k t).
\end{equation}
Plugging this into Equation \ref{eq:wave}, we see the solutions must satisfy:
\begin{equation}
\label{eqn:helmholtz}
     \nabla^2  \phi_k(x,y) = -\lambda_k^2  \phi_k(x,y), \ \ \ \text{where}\ \ \ \lambda^2_k = \frac{\omega_k^2}{c^2}.
\end{equation}
In this form, it is clear that $\lambda_k$ is an eigenvalue of the Laplacian operator acting on the surface $u$. Succinctly then, the question posed by \cite{kac_can_1966}, is if full set of eigenvalues $\{\lambda_1, \lambda_2, \ldots \}$ (called the eigenspectrum) of the Laplacian operating on a given boundary is sufficient to uniquely identify all two-dimensional boundaries. At the time of posing the question, it was known that the area of a drum-head could be deduced from its eigenspectrum uniquely; however, it took more than 25 years for researchers to find counter examples of drum heads that could not be distinguished by their eigenspectra \citep{Gordon1992}, while later work from \cite{zelditch1999spectraldeterminationanalyticaxisymmetric} was able to precisely characterize a class of shapes which are uniquely identifiable. Overall, these results demonstrated that the amount of unique geometric information in spectral representations is significant, and in most cases, aside from the pathological examples, most shapes are not `isospectral'.


\paragraph{The Solution for Square Drums}
For a square drum of side length $L$, the above boundary value problem has a well-known simple solution \citep{Berard2014}. Specifically, the normal modes and corresponding Laplacian eigenvalues are: 
\begin{equation}
  \phi_{m,n}(x,y) 
  \;=\; 
  \sin\!\Bigl(\tfrac{m\pi}{L}\,x\Bigr)\,
  \sin\!\Bigl(\tfrac{n\pi}{L}\,y\Bigr),
  \quad
  m,n = 1,2,3,\dots
\end{equation}
\begin{equation}
      \lambda_{m,n}
  \;=\;
  \sqrt{\Bigl(\tfrac{m\pi}{L}\Bigr)^2
        + \Bigl(\tfrac{n\pi}{L}\Bigr)^2}.
\end{equation}
We can quickly verify that indeed, these modes are all precisely zero at the boundary locations of the square since $\sin(\frac{n \pi}{L}x) = \sin(0) = 0$ when $x = 0$, and $\sin(\frac{n \pi}{L}x) = \sin(n \pi) = 0 $ when $x = L$. From Equation \ref{eqn:helmholtz}, the oscillation frequencies in the wave equation 
are $\omega_{m,n} = c\,\lambda_{m,n} = c \tfrac{\pi}{L}\,\sqrt{m^2 + n^2}.$ Therefore, the \emph{lowest} resonant frequency of a square drum is 
\begin{equation}    
  \omega_{1,1} 
  \;=\;
  c\frac{\pi}{L}\,\sqrt{2},
\end{equation}
measured in radians per second, following our intuition that larger drums (larger $L$) produce lower pitches (smaller $\omega_{1,1}$). 


To validate our main idea that this mechanism for global information integration may be simulated reasonably in a recurrent neural network, in the following we implement a simple RNN model which emulates wave dynamics, and measure if the Fourier transform of the resulting hidden state dynamics inside the drum exhibits these fundamental frequencies. 

\paragraph{Emulation in a Recurrent Neural Network} To simulate the above equation in an RNN, we observe that the wave equation (Equation \ref{eq:wave}) can be discretized over space and time to yield a set of equations which are very reminiscent of an RNN. This is the same approach taken by \cite{wrnn} for the first order one-way wave equation, and \cite{cornn} for a network of coupled oscillations, but adapted to the standard 2D wave equation. Explicitly, we construct an RNN to accurately numerically integrate the wave equation using Verlet integration, yielding the following set of updates for the hidden state $\mathbf{h}$ and the associated coupled velocity state $\mathbf{v}$: 
\begin{align}
    \label{eq:wrnn}
    \mathbf{v}_{t+1/2} &= \mathbf{v}_t + \frac{1}{2}\Delta t \cdot K_{\nabla^2} \star \mathbf{h}_t \\
        \label{eq:wrnn2}
    \mathbf{h}_{t+1} &= \mathbf{h}_t + \Delta t \cdot \mathbf{v}_{t+1/2} \\
        \label{eq:wrnn3}
    \mathbf{v}_{t+1} &= \mathbf{v}_{t+1/2} + \frac{1}{2}\Delta t \cdot K_{\nabla^2} \star \mathbf{h}_{t+1}
\end{align}
Where $\mathbf{h} \in \mathbb{R}^{H \times W}$ is defined to have 2 spatial dimensions, $\star$ denotes convolution over these dimensions, and $K_{\nabla^2}$ is the five-point stencil for the discrete Laplacian operator in 2D:
\begin{equation}
    \label{eq:fd_laplacian}
    K_{\nabla^2} =  
\begin{bmatrix}
0 & 1 & 0 \\
1 & -4 & 1 \\
0 & 1 & 0
\end{bmatrix}.
\end{equation}
The most straightforward manner to then provide `the drum as input' to the RNN, is to treat it as existing on a discretized grid (like an image), and map each spatial location (x,y) to a corresponding neuron $h_{x,y}$. We then emulate an idealized learned encoder by clamping the values of neurons at the boundary of the square (and outside) to zero. Explicitly, $h_{x,y} = 0 \ \ \forall\ \  \{x,y\} \in \Omega$. We can then provide an initial condition by setting the hidden state at the center of the drum ($h_{c_x, c_y}$) to a displacement of $1$ with all other locations set to $0$, and allow the dynamics above to unfold over time.

\begin{figure}[t]
    \centering
    \includegraphics[width=\linewidth]{figures/theory_barchart.png} 
    \vspace{-5mm}
    \caption{\textbf{Waves-RNNs generate theoretical frequencies.} Theoretical fundamental frequencies in Hz ($\tfrac{cycles}{sec}$) for square drum heads of different side lengths L, compared with the measured lowest peak frequencies of a wave-based RNN which uses the square input to determine it's recurrent dynamics.}
    \label{fig:theory}
    \vspace{-2mm}
\end{figure}

In Figure \ref{fig:theory}, we present the results of this experiment. Along the x-axis, we vary the square side length \(L\) from 13 to 21, and for each \(L\), we compute the theoretical fundamental frequency (in Hz) as \(\omega_{1,1} = c \sqrt{2}/(2L)\). The corresponding lowest peak frequency is then measured from the Fourier transform of the hidden state dynamics at \(h_{c_x, c_y}\) over 40,000 timesteps (\(\Delta t = 0.025\)) and plotted on the y-axis. The wave-based RNN’s results align almost perfectly with theoretical predictions, with minor deviations likely due to numerical integration limitations.


\begin{figure*}[t]
    \centering
    \includegraphics[width=\linewidth]{figures/polygons_vid.png} 
    \vspace{-6mm}
    \caption{\textbf{Waves propagate differently inside and outside shapes, integrating global shape information to the interior.} Sequence of hidden states of an oscillator model (NWM) trained to classify pixels of polygon images based on the number of sides using only local encoders and recurrent connections. We see the model has learned to use differing natural frequencies inside and outside the shape to induce soft boundaries, causing reflection, thereby yielding different internal dynamics based on shape.}
    \label{fig:polygons_vid}    
\end{figure*}