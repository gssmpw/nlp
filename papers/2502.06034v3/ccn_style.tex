\documentclass[10pt,letterpaper]{article}

\usepackage{ccn}
\usepackage{pslatex}
\usepackage{apacite}
\usepackage{hyperref}
\usepackage{natbib}
\usepackage{lineno}
\usepackage{url}
\usepackage{booktabs}
\usepackage{microtype}
\usepackage{graphicx}
\usepackage{subfigure}
\usepackage{multirow} 
\usepackage[title]{appendix}
\usepackage{caption}
\usepackage{subcaption}
\usepackage{wrapfig}

\usepackage[T1]{fontenc}

%%%%% NEW MATH DEFINITIONS %%%%%

% \usepackage{amsmath,amsfonts,bm}
\usepackage{amsmath,amsfonts}

\usepackage{pifont}


\newcommand{\R}{\mathbb{R}}


\def\va{{\mathbf{a}}}
\def\vg{{\mathbf{g}}}

% Sets
\def\sR{\mathbb{R}}
\def\sC{\mathbb{C}}
\def\sZ{\mathbb{Z}}
\def\sN{\mathbb{N}}
\def\sQ{\mathbb{Q}}

\def\sS{\mathcal{S}}



% Vectors
\def\vzero{{\mathbf{0}}}
\def\vone{{\mathbf{1}}}
\def\vmu{{\mathbf{\mu}}}
\def\vtheta{{\mathbf{\theta}}}
\def\va{{\mathbf{a}}}
\def\vb{{\mathbf{b}}}
\def\vc{{\mathbf{c}}}
\def\vd{{\mathbf{d}}}
\def\ve{{\mathbf{e}}}
\def\vf{{\mathbf{f}}}
\def\vg{{\mathbf{g}}}
\def\vh{{\mathbf{h}}}
\def\vi{{\mathbf{i}}}
\def\vj{{\mathbf{j}}}
\def\vk{{\mathbf{k}}}
\def\vl{{\mathbf{l}}}
\def\vm{{\mathbf{m}}}
\def\vn{{\mathbf{n}}}
\def\vo{{\mathbf{o}}}
\def\vp{{\mathbf{p}}}
\def\vq{{\mathbf{q}}}
\def\vr{{\mathbf{r}}}
\def\vs{{\mathbf{s}}}
\def\vt{{\mathbf{t}}}
\def\vu{{\mathbf{u}}}
\def\vv{{\mathbf{v}}}
\def\vw{{\mathbf{w}}}
\def\vx{{\mathbf{x}}}
\def\vy{{\mathbf{y}}}
\def\vz{{\mathbf{z}}}
\def\vzeta{{\mathbf{\zeta}}}

% Matrix
\def\mA{{\mathbf{A}}}
\def\mB{{\mathbf{B}}}
\def\mC{{\mathbf{C}}}
\def\mD{{\mathbf{D}}}
\def\mE{{\mathbf{E}}}
\def\mF{{\mathbf{F}}}
\def\mG{{\mathbf{G}}}
\def\mH{{\mathbf{H}}}
\def\mI{{\mathbf{I}}}
\def\mJ{{\mathbf{J}}}
\def\mK{{\mathbf{K}}}
\def\mL{{\mathbf{L}}}
\def\mM{{\mathbf{M}}}
\def\mN{{\mathbf{N}}}
\def\mO{{\mathbf{O}}}
\def\mP{{\mathbf{P}}}
\def\mQ{{\mathbf{Q}}}
\def\mR{{\mathbf{R}}}
\def\mS{{\mathbf{S}}}
\def\mT{{\mathbf{T}}}
\def\mU{{\mathbf{U}}}
\def\mV{{\mathbf{V}}}
\def\mW{{\mathbf{W}}}
\def\mX{{\mathbf{X}}}
\def\mY{{\mathbf{Y}}}
\def\mZ{{\mathbf{Z}}}
\def\mBeta{{\mathbf{\beta}}}
\def\mPhi{{\mathbf{\Phi}}}
\def\mLambda{{\mathbf{\Lambda}}}
\def\mSigma{{\mathbf{\Sigma}}}


% Expectation
% \def\eE{\mathop{\mathbb{E}}\limits}
\def\eE{\mathbb{E}}

% Probability
\def\pP{\mathbb{P}}

% Tilde
\def\tf{\tilde{f}}
\def\tS{\tilde{S}}
\def\wtF{\widetilde{\mathcal{F}}}
\def\whR{\widehat{R}}
\def\tvx{\tilde{\mathbf{x}}}
\def\ty{\tilde{y}}


\def\defeq{\overset{\textup{def}}{=}}
% \def\defeq{\overset{.}{=}}
\def\defone{\overset{\text{\ding{172}}}{=}}
\def\deftwo{\overset{\text{\ding{173}}}{=}}
\def\leqone{\overset{\text{\ding{172}}}{\leq}}
\def\leqtwo{\overset{\text{\ding{173}}}{\leq}}
\def\leqthree{\overset{\text{\ding{174}}}{\leq}}
\def\leqfour{\overset{\text{\ding{175}}}{\leq}}
\def\eqone{\overset{\text{\ding{172}}}{=}}
\def\eqtwo{\overset{\text{\ding{173}}}{=}}
\def\eqthree{\overset{\text{\ding{174}}}{=}}
\def\eqfour{\overset{\text{\ding{175}}}{=}}
\def\geqfive{\overset{\text{\ding{176}}}{\geq}}


\newcommand{\theHalgorithm}{\arabic{algorithm}}
\newcommand{\fix}{\marginpar{FIX}}
\newcommand{\new}{\marginpar{NEW}}
\newcommand{\fixme}[1]{\textcolor{red}{#1}}


\title{Traveling Waves Integrate Spatial Information Through Time}
 
\author{
    {\large \bf Mozes Jacobs\textsuperscript{1} \quad 
    Roberto C. Budzinski\textsuperscript{2} \quad 
    Lyle Muller\textsuperscript{2} \quad 
    Demba Ba\textsuperscript{1} \quad 
    T. Anderson Keller\textsuperscript{1}} \\
    \textsuperscript{1}The Kempner Institute for the Study of Natural and Artificial Intelligence, Harvard University \\
    \textsuperscript{2}Western University, Department of Mathematics, London, Ontario, Canada \\
}


\begin{document}

\maketitle

\section{Abstract}
{
\bf
Traveling waves of neural activity are widely observed in the brain, but their precise computational function remains unclear. One prominent hypothesis is that they enable the transfer and integration of spatial information across neural populations. However, few computational models have explored how traveling waves might be harnessed to perform such integrative processing. Drawing inspiration from the famous “\emph{Can one hear the shape of a drum?}” problem -- which highlights how normal modes of wave dynamics encode geometric information -- we investigate whether similar principles can be leveraged in artificial neural networks. Specifically, we introduce convolutional recurrent neural networks that learn to produce traveling waves in their hidden states in response to visual stimuli, enabling spatial integration. By then treating these wave-like activation sequences as visual representations themselves, we obtain a powerful representational space that outperforms local feed-forward networks on tasks requiring global spatial context. In particular, we observe that traveling waves effectively expand the receptive field of locally connected neurons, supporting long-range encoding and communication of information. We demonstrate that models equipped with this mechanism solve visual semantic segmentation tasks demanding global integration, significantly outperforming local feed-forward models and rivaling non-local U-Net models with fewer parameters. As a first step toward traveling-wave-based communication and visual representation in artificial networks, our findings suggest wave-dynamics may provide efficiency and training stability benefits, while simultaneously offering a new framework for connecting models to biological recordings of neural activity.
}
\begin{quote}
\small
\textbf{Keywords:} 
Traveling Waves; Oscillation; Information Integration
\end{quote}

\section{Introduction}\label{sec:intro}

In computational finance, Monte Carlo simulations are used extensively to estimate the expected value of financial payoffs based on the solution of stochastic differential equations (SDEs) which model the evolution of stock prices, interest rates, exchange rates and other quantities \cite{glasserman04}.  Monte Carlo methods are very general and flexible, but for high accuracy it requires generating a large number of costly SDE path approximations, which has motivated research into a number of variance reduction or, equivalently, cost reduction techniques. One such method is
Multilevel Monte Carlo (MLMC), which was proposed in \cite{GILES2008} and was adapted for various applications that are summarised in \cite{Giles_overview17} and successfully combined with other methods such as quasi-Monte Carlo methods. The main idea of MLMC is to approximate the payoff using different time stepping resolutions when numerically solving the underlying SDE and to generate an optimal number of samples on each level, such that the overall computational cost is minimised subject to the desired bound on the variance. %, such that the total computational cost is minimised. 
The computational savings come from the fact that most samples are computed on the coarser levels and hence are less expensive while only a few samples from the finest levels are required \cite{GILES2008}.


Among the directions in which the computational cost 
of MLMC methods could further be reduced, an important avenue is the use of lower precision calculations, especially for the first Monte Carlo levels where the targeted accuracy is relatively low. 
 An overview of the research on mixed precision for the standard Monte Carlo (MC) framework is provided in \cite{ChowMixedPrecisionStandardMC} but only a few references study the potential of low precision computation in the MLMC framework \cite{Rounding_error_oliver}. To the best of our knowledge, the only MLMC framework with customised precision in the literature is \cite{brugger2014mixed}, but they use a uniform precision for all operations on each Monte Carlo level instead of optimising 
 the precision of each intermediary variable to reduce as much as possible the cost of path generation.
 
An important motivation for an MLMC framework with variable precision would be performing the low precision computations on reconfigurable hardware devices such as Field Programmable Gate Arrays (FPGAs). FPGAs contain customizable logic blocks and connectors that make it easy to adapt the digital circuit architecture for a specific application, leading to a highly parallel and optimised implementation. Therefore they are successfully exploited in applications that require high speed and have high computational workload, such as signal processing \cite{woods2008fpga}, and real time applications like high frequency trading \cite{HFT1,HFT2}. That is why a number of previous works in hardware architecture design implemented the MLMC algorithm to price financial options using FPGAs as accelerators, which resulted in improved speed and power efficiency compared to full CPU architectures \cite{Schryver2013AMM}. The paper \cite{lindsey2016domain} also proposed 
a Domain Specific Language to automate the configuration of FPGAs for this specific application. However, only \cite{brugger2014mixed} proposed a heuristic to reduce the precision in calculations.

In addition, all aforementioned works considered that the random number generation (RNG) is performed in single or double precision. Yet in most cases an important portion of the workload in the overall MLMC simulation comes from the RNG and in \cite{brugger2014mixed} this limited the total computational savings.
To reduce the cost of MLMC simulations in particular those based on the Geometric Brownian Motion (GBM), \cite{approximateICDF_Oliver, NestedOliver} have proposed to use approximate random numbers that are generated by applying an approximation of the inverse CDF to uniform random numbers. In \cite{NestedOliver}, the authors proposed a way to integrate these lower precision random variables into a \textit{nested} MLMC framework and completed a numerical analysis to bound the resulting error at each MC level by a product of the time step and the error in the random number approximation. The same authors show in \cite{approximateICDF_Oliver} that using approximate random variables reduces the cost of path generation by a factor 7.


In this paper we propose a nested MLMC framework that combines the use of approximate random normal variables and lower precision calculations to reduce the computational cost of MLMC even further than \cite{brugger2014mixed,NestedOliver}. We illustrate the efficiency of our framework in Matlab, after making several assumptions on the cost of operations and size of the errors that we carefully justify. We focus on the case of GBM and use the approximate RNG methods presented in \cite{approximateICDF_Oliver} as well as a new slightly modified method that combines CDF inversion and the central limit theorem. To choose the precision of the variables in the low precision path generation, we introduce a novel method to optimise the bit-widths. This optimisation is performed before the main path generation loop is executed and is based on a linear model of the payoff error  
due to rounding when computing in low precision. The error model relies on algorithmic differentiation in a similar manner to \cite{unifying-bwoptim,bitwidth-AD,ADAPT}. The bit-width optimisation procedure can be performed off-line, so this stage can be excluded from the on-line time complexity of our framework. The user specified desired accuracy is then enforced by calculating on-line the number of samples that need to be generated.

In terms of hardware design, we suggest implementing the low precision path generation on FPGAs and the full-precision ones on a CPU or GPU. 
The FPGA offers enough flexibility to define a separate bit-width for every variable in the low precision path generation, and can be reconfigured periodically to update the bit-widths when the market parameters have changed considerably. 


The paper is organized as follows : \Cref{sec:MLMC} introduces MLMC and nested MLMC to make clear the estimator that is implemented in our framework. Then in \Cref{sec:RNG} we detail the methods that could be used to obtain approximate random normally distributed numbers very cheaply for the low precision path generation. In \Cref{sec:error_model} and \Cref{sec:costModel} we propose an error model and a cost model (resp.) that we then use to formulate the optimisation problem that is solved to obtain the optimal bit-widths of fixed point variables in \Cref{sec:optimisation}. Finally we summarise our results and future directions in \Cref{sec:conclusion}.



\section{Background}
\label{sec:background}


\subsection{Preliminaries}

{\color{red}[TODO: LLMs? in-context learning?]}

\subsection{Problem Definition}

{\color{red}[TODO: define the problem of citation intent]}



\section{Methodology}
\paragraph{Preliminaries.}
We primarily focus on the homologous model merging, in which $\boldsymbol{\theta}_i$ all come from the same base model $\boldsymbol{\theta}_{\rm{base}}$. Given $K$ tasks $\{T_1,T_2,\cdots,T_K\}$ and $K$ corresponding fine-tuned models with parameters $\{\boldsymbol{\theta}_1,\boldsymbol{\theta}_2,\cdots,\boldsymbol{\theta}_K\}$, model merging aims to combine $K$ fine-tuned models into one single model simultaneously performing on $\{T_1,T_2,\cdots,T_K\}$ without post-training~\cite{method_p1_1,method_p1_2}.
Task vector~\cite{ilharco2023editing,yang2024adamerging} is a key element in merging method which could enhances the base model‘s ability or enable the model to handle other tasks. Specifically, for task $T_i$, the task vector $\boldsymbol\tau_i\in \mathbb{R}^D$ is defined as the vector obtained by subtracting the SFT weights $\boldsymbol{\theta}_i$ from the base model weight
$\boldsymbol{\theta}_{\rm{base}}$, \emph{i.e.}, $\boldsymbol\tau_i=\boldsymbol{\theta}_i-\boldsymbol{\theta}_{\rm{base}}$. The merged model could be denoted as $\boldsymbol{\theta}_m=\boldsymbol{\theta}_{\rm{base}}+\sum_i \lambda_i\boldsymbol{\tau}_i$, which $\lambda_i$ is the scaling factor measuring the importance of task vector. For clarification, we also denote the neuron set in $\boldsymbol{\theta}_i$ as $\mathcal{N}_i$, the neuron set in $\boldsymbol{\tau}_i$ as $\mathcal{T}_i$.



\begin{algorithm}[!ht]
    \caption{LED-Merging}
    \label{alg1}
    \begin{algorithmic}[1]
        \REQUIRE  base model $\boldsymbol{\theta}_{\rm{base}}$, SFT models $\{\boldsymbol{\theta}_{i}\mid i\in [K]\}$, mask ratios \{$r_{i} \mid i\in [K]\}$, scaling factors $\{\lambda_i\mid i\in[K]\}$, location datasets $\{\mathcal{X}_{i}\mid i\in[K]\}$
        \ENSURE merged parameter $\boldsymbol{\theta}_{m}$
        \STATE $\mathcal{M}\leftarrow\phi$
        \STATE $\boldsymbol{\theta}_{m}\leftarrow \boldsymbol{\theta}_{\rm{base}}$
        \FOR{$i\in [K]$}
        \STATE $I(\boldsymbol{\theta}_i)=\mathbb{E}_{x\sim \mathcal{X}_i}|\boldsymbol{\theta}_{i}\odot \nabla_{\boldsymbol{\theta}_i}\mathcal{L}(x)|$
        \STATE $I(\boldsymbol{\theta}_{\rm{base}})=\mathbb{E}_{x\sim \mathcal{X}_i}|\boldsymbol{\theta}_{\rm{base}}\odot \nabla_{\boldsymbol{\theta}_{\rm{base}}}\mathcal{L}(x)|$
        
        \STATE calculate $\mathcal{T}^{r_i}_{i}$ following Equation \ref{vote}
        \STATE  $\mathcal{M}\leftarrow \mathcal{M}\cup\{\mathcal{T}^{r_i}_i\}$
       
        
   
        
        
        \ENDFOR  
        \FOR{$i\in [K]$}
        
        \STATE calculate $\text{Disjoint}(\mathcal{T}_i^{r_i})$ use Equation~\ref{disjoint_safety}
        \STATE $\boldsymbol{m}_i \leftarrow \boldsymbol{0}$
        \FOR{$d\in \mathcal{T}_i^{r_i}$}
        \STATE $\boldsymbol{m}_{i,d}=1$
        \ENDFOR
        \STATE $\boldsymbol{\theta}_{m}\leftarrow \boldsymbol{\theta}_{m}+\lambda_i \boldsymbol{\tau}_i\odot \boldsymbol{m}_{i}$
        \ENDFOR
    \end{algorithmic}
\end{algorithm}
    %\vspace{-5pt}
\begin{figure*}[h!]
    \centering
    \includegraphics[width=\linewidth]{figs/pipeline_v2.pdf}
    \vspace{-40mm}
    \caption{Overview of our two-stage training pipeline {\ours}.}
    \label{fig:pipeline}
\end{figure*}


\paragraph{LED-Merging: Location, Election, and Disjoint Merging}
To address the neuron misidentification and interference issues in existing model merging methods, we propose LED-Merging (Location, Election, and Disjoint Merging). Specifically, previous studies \cite{modelstock, ilharco2023editing, tiesmerging} fail to accurately identify safety-related neurons in task vectors with a single magnitude score, namely \textit{neuron misidentification}. Meanwhile, there exists an interference between safety-related and utility-related task vector neurons during the merging process, namely \textit{neuron interference}. To address neuron misidentification, we first locate important neurons both in the base and fine-tuned models and then elect neurons from the task vector considering these two scores together. Subsequently, to mitigate the interference, we introduce a disjoint step, isolating these important neurons so that they influence different base neurons. The whole process is illustrated in Figure~\ref{fig:method}. 




In the location and election step, we consider the importance score from base and fine-tuned models simultaneously to locate task-specific neurons. In this way, it is more accurate than relying on the magnitude score alone because task-specific neurons with high importance score in the fine-tuned model may not necessarily score high in the base model, and vice versa.

{\textbf{Location}}.  We first calculate importance scores for each neuron in a base/fine-tuned model. Given a location dataset $\mathcal{X}_i=\{(x,y)_k\}$, where $x$ is the question and $y$ is the answer, we calculate the importance scores for the weight $\boldsymbol{\theta}_i\in\mathbb{R}^D$ in any  layer as follows~\cite{snip,spareseGPT,sun2024a}:
\begin{equation}
    I(\boldsymbol{\theta}_i)=\mathbb{E}_{x\sim \mathcal{X}_i}[\boldsymbol{\theta}_i\odot \nabla _{\boldsymbol{\theta}_i}\mathcal{L}(x)],
    \label{location}
\end{equation}
which $\mathcal{L}(x)=-\log p(y\mid x)$ is the conditional negative log-likelihood loss. We choose the SNIP score~\cite{snip} because it balances computational efficiency and performance~\cite{cq}. Please refer to Sec.~\ref{sec:ablation} for the comparison between different location methods. After computing importance scores, we choose top-$r_i$ neurons as the important neuron subset $\mathcal{N}_{i}^{r_i}$ from $I(\boldsymbol{\theta}_i)$.
 
 % After computing locating scores, we select the neurons scoring both high in base and fine-tuned models as important neurons in task vectors. Then in the disjoint step,  with preventing  polysemantic neurons  from receiving gradient updates towards different directions,
 % we use set difference to isolate the safety   and utility-related neurons  and construct corresponding masks for merging process,

{\textbf{Election}}. A natural question is how to select important neurons in the task vector $\boldsymbol{\tau}_i$ based on $I(\boldsymbol{\theta}_{\rm{base}})$ and $I(\boldsymbol{\theta}_{i})$. The important neurons in the base model may be different from neurons in the fine-tuned model. Therefore, we introduce the following election strategy to select neurons with high scores in both base and fine-tuned models:
\begin{equation}
    \mathcal{T}_i^{r_i}=\mathcal{N}_i^{r_i}\cap \mathcal{N}_{\rm{base}}^{r_i}.
    \label{vote}
\end{equation}
\emph{Remark}. We compare different choosing methods, including scoring low or high in base or fine-tuned model in Section~\ref{sec:ablation} and find that Equation \ref{vote} achieves the best performance.





{\textbf{Disjoint}}. As important neurons from different task vectors may conflict with each other at the same position, we use the set difference to disjoint the neurons from others to prevent interference:
\begin{equation}
    \text{Disjoint}(\mathcal{T}^{r_i}_{i})=\mathcal{T}^{r_i}_{i}-\mathop{\cup}\limits_{{J}\subsetneqq [K],|J|\geq 2}\mathop{\cap}\limits_{j\in {J}}\mathcal{T}^{r_j}_{j}.
    \label{disjoint_safety}
\end{equation}

Next, we construct a mask $\boldsymbol{m}_i\in\mathbb{R}^D$ to implement disjoint in the merging process. Specifically, this mask $\boldsymbol{m}_i$ is used to select neurons from $\mathcal{T}_i$. The mask ratio is $r_i$, where $r\in(0,1]$. The mask $\boldsymbol{m}_i$ can be derived from:
\begin{equation}
    \boldsymbol{m}_{i,d}=\begin{aligned} &\left\{ \begin{array}{ll} 1, & \text{if } d\in \text{Disjoint}(\mathcal{T}_{i}^{r_i}), \\ 0, & \text{otherwise}. \end{array} \right. \end{aligned}
    \label{mask_safety}
\end{equation}


% \subsection{Merging Models with Masks}
{\textbf{Merging}}. The final
merged task vector $\boldsymbol{\tau}_m$ is as follows:
\begin{equation}
    \boldsymbol{\tau}_m= \sum_i \lambda_i\boldsymbol{\tau}_{i}\odot\boldsymbol{m}_i.
    \label{merged_task_vector}
\end{equation}
We summarize the workflow in Algorithm \ref{alg1}.



\section{Experiments}
\label{sec:experiments}

\begin{figure*}[t]
\vspace{-6mm}
    \centering
    \includegraphics[width=0.8\linewidth]{figs/compare.pdf}
    \vspace{-4mm}
    \caption{\textbf{Qualitative comparison} with the baseline for generating a sequence of novel view images.  
    The results demonstrate that our method synthesizes more consistent multi-view images compared to our baseline model (Zero123). In addition, compared to SyncDreamer, our method visually maintains better similarity to the conditioned image and appears more natural.}
    \label{fig:sota_compare}
\vspace{-5mm}
\end{figure*}

\subsection{Experimental Setups}
\textbf{Dataset.}
Following previous work~\cite{zero123, SyncDreamer}, we evaluate our work on the Google Scanned Object (GSO)~\cite{GSO} dataset to verify the zero-shot novel view image synthesis capability. 
We also provide results for additional datasets in the Supplementary Material.
Specifically, we randomly select 30 objects from the GSO dataset with various object categories. 
Unlike recent approaches~\cite{mvdream, SyncDreamer} that aim to enhance the consistency of novel view synthesis models by generating multiple fixed-view images, our method can generate images from any camera pose and any number of views. Therefore, we conduct experiments under different camera pose settings to validate our approach:
specifically, 
1) \textit{16-views with free camera pose}: for each object, we circularly render 16 views with the elevation angles ranging in $[-10\degree, 40\degree]$ and the azimuth angles are evenly distributed in $[0\degree, 360\degree]$. 
2) \textit{16-views with fixed camera pose}: We maintain a constant elevation angle of $30\degree$ and uniformly sample azimuth angles (same as SyncDreamer~\cite{SyncDreamer}).
3) \textit{32-views with free camera pose}: Similar to the first setting, but we sample 32 views.
It's important to note that our method does not require additional training or fine-tuning on any datasets.

\noindent\textbf{Metrics.}
To validate the effectiveness of our method, we mainly evaluate it based on three criteria:
1) \textit{Quality Score}. We evaluate the image quality of synthesized multi-view images by measuring their similarity with ground truth images. Following prior research~\cite{zero123, sparsefusion}, we report the similarity between the synthesized images and the ground truth images with standard metrics: PSNR, SSIM~\cite{ssim}, and LPIPS~\cite{lpips}.
2) \textit{Multi-view Consistency Score}. As the primary goal of our work is to improve the consistency of generated images, we also employ the 3D consistency score~\cite{3dim} to verify the consistency among the synthesized images. Specifically, we train an Instant-NGP~\cite{instant_ngp} with the input image and part of the synthesized novel view images of our model and evaluate the similarity between the remaining synthesized images and the rendered images of Instant-NGP. For the synthesized multi-view images of each object, we allocate $3/4$ for training and reserve the remaining $1/4$ for validation.
Intuitively, if the consistency of synthesized images is improved, the NeRF-like model will train a better object representation, and the re-rendered images will agree more with the validation images.
3) \textit{Input Consistency Score}. To assess the faithfulness of synthesized images in preserving the identity of the input condition image, we introduce the input consistency score. This score calculates the similarity of each synthesized image with the input condition image, utilizing the LPIPS metric.

In addition, we use synthesized multi-view images to train a neural 3D reconstruction model (NeuS~\cite{neus}) and report commonly used Chamfer Distances (CD) and Volume IoUs between the trained 3D model and the ground truth.

\noindent\textbf{Baselines.}
Given that our main goal is to improve the consistency of the trained baseline model without further fine-tuning, we mainly compare our approach with the used baseline model Zero123~\cite{zero123}. Additionally, we compare our method to the SOTA approaches such as PGD~\cite{tseng2023consistent} and SyncDreamer~\cite{SyncDreamer} using the same Zero123 base model.

\noindent\textbf{Implementation Details.}
We use the official checkpoint provided by Zero123~\cite{zero123}, which is trained on objaverse~\cite{objaverse} for 165,000 steps. We inject our epipolar attention layer after step $T=4$ and layer $L=10$ by default. We find that feature fusion weight $\alpha=0.5$, and the number of context views $M=2$ work better.

\begin{table}[t]
\centering
\caption{Comparison of multi-view consistency, image quality, and input consistency of synthesized multi-view images at the 16-view setting with free camera pose.}
\label{tab:view16_free_compare}
\vspace{-2mm}
\scalebox{0.6}{
\begin{tabular}{c ccc ccc c}
\toprule
              & \multicolumn{3}{c}{Multi-view Consistency} & \multicolumn{3}{c}{Quality Score} & \multicolumn{1}{c}{Input Consis.} \\
              \cmidrule(lr){2-4} \cmidrule(lr){5-7} \cmidrule(lr){8-8}
              & PSNR$\uparrow$  & SSIM$\uparrow$ & LPIPS$\downarrow$ 
              & PSNR$\uparrow$  & SSIM$\uparrow$ & LPIPS$\downarrow$ 
              & LPIPS$\downarrow$ 
              \\ \midrule

Zero123
& 15.225        & 0.645       & 0.408
& 14.255        & 0.747       &	0.208
& 0.303         
\\
SyncDreamer
& 14.830        & 0.626       & 0.434
& 12.650        & 0.713       &	0.254
& 0.317         
\\
Ours 
& \best{18.300}	& \best{0.734}	& \best{0.355}
& \best{14.947}	& \best{0.763}	& \best{0.191}
& \best{0.282}
\\

\bottomrule
\end{tabular}
}
\end{table}

\begin{table}[t]
\vspace{-1mm}
\centering
\caption{Comparison of multi-view consistency, image quality, and input consistency at the 16-view setting with fixed camera pose as SyncDreamer~\cite{SyncDreamer}.}
\label{tab:view16_fxied_compare}
\vspace{-3mm}
\scalebox{0.6}{
\begin{tabular}{c ccc ccc c}
\toprule
              & \multicolumn{3}{c}{Multi-view Consistency} & \multicolumn{3}{c}{Quality Score} & \multicolumn{1}{c}{Input Consis.} \\
              \cmidrule(lr){2-4} \cmidrule(lr){5-7} \cmidrule(lr){8-8}
              & PSNR$\uparrow$  & SSIM$\uparrow$ & LPIPS$\downarrow$ 
              & PSNR$\uparrow$  & SSIM$\uparrow$ & LPIPS$\downarrow$ 
              & LPIPS$\downarrow$ 
              \\ \midrule

Zero123
& 16.556        & 0.682       & 0.378
& 14.592        & 0.750       &	0.207
& 0.305         
\\
SyncDreamer
& \best{22.424}        & \best{0.812}       & \best{0.268}
& 15.269        & 0.749       &	0.196
& 0.300         
\\
Ours 
& 21.151	& 0.780	& 0.302
& \best{15.293}	& \best{0.764}	& \best{0.184}
& \best{0.287}
\\

\bottomrule
\end{tabular}
}
\vspace{-4mm}
\end{table}


\subsection{Comparison With Baseline Models}
The quantitative comparison on three settings are shown in Tab.~\ref{tab:view16_free_compare}, Tab.~\ref{tab:view16_fxied_compare}, and Tab.~\ref{tab:view32_free_compare}. The qualitative comparison is shown in Fig.~\ref{fig:sota_compare}.

\begin{table}[t]
\centering
\caption{Comparison of multi-view consistency and image quality scores of synthesized multi-view images at the 32-view setting with free camera pose.}
\vspace{-3mm}
\label{tab:view32_free_compare}
\scalebox{0.7}{
\begin{tabular}{c ccc ccc}
\toprule
              & \multicolumn{3}{c}{Multi-view Consistency} & \multicolumn{3}{c}{Quality Score} \\
              \cmidrule(lr){2-4} \cmidrule(lr){5-7}
              & PSNR$\uparrow$  & SSIM$\uparrow$ & LPIPS$\downarrow$ 
              & PSNR$\uparrow$  & SSIM$\uparrow$ & LPIPS$\downarrow$ 
              \\ \midrule

Zero123
& 16.515        & 0.694       & 0.378
& 15.142        & 0.733       &	0.211
\\
PGD~\cite{tseng2023consistent}
& 18.481        & 0.720       & 0.343
& 15.281        & 0.739       &	0.205
\\
Ours 
& \best{20.655}	& \best{0.792}	& \best{0.305}
& \best{15.268}	& \best{0.742}	& \best{0.203}
\\

\bottomrule
\end{tabular}
}
\vspace{-3mm}
\end{table}

\begin{table*}
  [t]
  \centering
  \resizebox{\textwidth}{!}{%
  \begin{tabular}{cccccccccccc}
    \toprule \multicolumn{2}{c}{Components}                                                             & \multicolumn{5}{c}{Re-executability Rate (\%)} & \multicolumn{5}{c}{Readability (\#)} \\
    \cmidrule(lr){1-2} \cmidrule(lr){3-7} \cmidrule(lr){8-12}        \hspace{8pt}\labelemoji\hspace{8pt}                                                                & \hspace{8pt}\toolemoji\hspace{8pt}                                      & O0                                 & O1             & O2             & O3             & AVG            & O0             & O1             & O2             & O3             & AVG            \\
    \hline
    \rowcolor[rgb]{0.93,0.93,0.93}\multicolumn{12}{c}{\textbf{Initialize with LLM4Decompile-End-6.7B~\citep{llm4decompile}}}   \\
    \xmark                                                                                              & \xmark                                    & 69.51                              & 46.95          & 50.61          & 46.34          & 53.35          & 3.98 & 3.41 & 3.44 & 3.38 & 3.55 \\
    \cmark                                                                                              & \xmark                                    & 75.61                              & 50.61          & 50.00          & 50.00          & 56.55          & 4.01 & 3.44 & 3.39 & \textbf{3.49} & 3.58 \\
    \xmark                                                                                              & \cmark                                    & 83.54                     & \textbf{56.10}          & 51.22          & 50.61 & 60.37 & 4.05 & 3.51 & 3.51 & 3.42 & 3.62 \\
    \cmark                                                                                              & \cmark                                    & \textbf{85.37}                            & \textbf{56.10}                     & \textbf{51.83} & \textbf{52.43}          & \textbf{61.43} & \textbf{4.13} & \textbf{3.60} & \textbf{3.54} & \textbf{3.49} & \textbf{3.69} \\

    \rowcolor[rgb]{0.93,0.93,0.93}\multicolumn{12}{c}{\textbf{Initialize with Deepseek-Coder-6.7B-base~\citep{deepseekcoder}}} \\
    \xmark                                                                                              & \xmark                                    & 59.15                              & 35.98          & 39.02          & 37.80          & 42.99          & 3.71 & 3.05 & 3.16 & 3.05 & 3.24 \\
    \cmark                                                                                              & \xmark                                    & 66.46                              & 41.46          & 38.41          & 36.59          & 45.73          & 3.76 & 3.17 & \textbf{3.21} & 3.08 & 3.31 \\
    \xmark                                                                                              & \cmark                                    & 70.73                              & 39.63          & 39.02          & 40.24          & 47.41          & 3.90 & 3.17 & 3.08 & 3.11 & 3.31 \\
    \cmark                                                                                              & \cmark                                    & \textbf{79.88}                     & \textbf{45.73} & \textbf{43.90} & \textbf{42.68} & \textbf{53.05} & \textbf{3.96} & \textbf{3.21} & 3.18 & \textbf{3.19} & \textbf{3.38} \\
    \bottomrule
  \end{tabular}%
  }
  \caption{The ablation study of different methods across four optimization levels
  (O0, O1, O2, O3), as well as their average scores (AVG). The results in bold represent the optimal performance. The ~\labelemoji~ and ~\toolemoji~ means Relabedling and Function Call. \textbf{Bold} denotes the best performance.}
  \label{tab:ablation}
\end{table*}



\begin{figure*}[ht]
    \centering
    \begin{minipage}{0.65\textwidth}
        \centering
        \includegraphics[width=0.95\linewidth]{figs/ablation.pdf}
        \vspace{-2mm}
        \captionof{figure}{Qualitative Comparison for different design choices. Our method, employing multi-view epipolar attention, demonstrates the best consistency.}
        \label{fig:ablation}
    \end{minipage}\hfill
    \begin{minipage}{0.33\textwidth}
        \centering
        \includegraphics[width=0.8\linewidth]{figs/neus_ver.pdf}
        \vspace{-3mm}
        \caption{Our method shows better direct 3D reconstruction~\cite{neus}.}
        \label{fig:neus}
    \end{minipage}
    \vspace{-5mm}
\end{figure*}

\noindent\textbf{Multi-view Consistency.}
Tab.~\ref{tab:view16_fxied_compare} presents the 3D consistency scores compared to our baseline model (Zero123) and SyncDreamer. The results indicate a significant improvement across all three metrics achieved by our method when compared with Zero123.
While our method exhibits a marginally lower numerical consistency score compared to SyncDreamer, it enables the synthesis of images with arbitrary camera poses.	
This capability is illustrated in Tab.~\ref{tab:view16_free_compare}, where our method consistently enhances consistency with changes in camera pose settings, whereas SyncDreamer fails to do so and exhibits inferior results compared to Zero123.
Furthermore, our method facilitates the synthesis of multi-view images with any number of camera views. This versatility is demonstrated in Tab.~\ref{tab:view32_free_compare}, where our method continues to achieve significant improvements in consistency scores, while SyncDreamer is unable to operate under such conditions.	

Meanwhile, Fig.~\ref{fig:sota_compare} provides a qualitative comparison with the baseline. While both our method and SyncDreamer enhance consistency, our method visually preserves better similarity to the input image, including color and texture details. The input consistency score further corroborates this.

\noindent\textbf{Image Quality.}
While our primary goal centers around enhancing the consistency of synthesized multi-view images, we also evaluate the image quality by comparing the similarity with the ground truth images. The results shown in Tab.~\ref{tab:view16_free_compare}, Tab.~\ref{tab:view16_fxied_compare}, and Tab.~\ref{tab:view32_free_compare} indicate that our method also enhances the image quality under different settings besides improving the consistency.
Moreover, our method shows better image quality compared with SyncDreamer even in the 16-view setting with fixed camera pose.

\noindent\textbf{Input Consistency.}
Input consistency terms whether the results align with the input image.
Fig.~\ref{fig:sota_compare} illustrates that both our method and SyncDreamer enhance multi-view consistency. However, the color and texture details of SyncDreamer's results diverge from the input image and appear visually unnatural.
This discrepancy is evident in the input consistency score presented in Tab.~\ref{tab:view16_fxied_compare}, indicating lower similarity with the condition image in the SyncDreamer results.	

\subsection{Ablation Study}
The overall quantitative results are shown in Tab.~\ref{tab:ablation}, and the qualitative comparisons are shown in Fig.~\ref{fig:ablation}.

\noindent \textbf{Full Attention \vs Epipolar Attention.}
The results presented in Tab.\ref{tab:ablation} and Fig.\ref{fig:ablation} demonstrate that our epipolar attention mechanism can synthesize more consistent multi-view images compared with full attention. Furthermore, our epipolar attention achieves a greater performance improvement compared to full attention when using multiple reference images. This could be attributed to the fact that our epipolar attention more effectively localizes target information, as depicted in Fig.~\ref{fig:full_attn_compare}, thereby reducing noise from the reference images. In the multi-view setting, where multiple reference images are utilized, this noise reduction becomes particularly crucial.
Moreover, it is noteworthy that the epipolar attention mechanism consumes less GPU memory compared to our baseline, as discussed in Sec.~\ref{sec:attn_analysis}.

\noindent \textbf{Attending Single-View \vs Multi-View.}
Applying the epipolar attention significantly improves the consistency between the input and target views. However, the consistency between different views in the unobserved regions of the input view is not well preserved.
After implementing our epipolar attention in the multi-view setting, the consistency across the generated multi-view images is further improved. The last row in Tab.~\ref{tab:ablation} shows that after applying our multi-view epipolar attention, the consistency score is further improved compared with the single-view setting. Besides, the qualitative result in Fig.~\ref{fig:ablation} also shows better consistency among different target views.



\begin{table}[t]
\centering
\vspace{-1mm}
\caption{Comparison of 3D reconstruction results. Our method significantly improves the reconstruction quality.}
\vspace{-3mm}
\label{tab:neus}
\scalebox{0.7}{
\begin{tabular}{c cc}
\toprule
              &  Chamfer Dist.$\downarrow$  & Volume IoU$\uparrow$
\\ \midrule

            Zero123         & 0.017         & 0.819    \\
            SyncDreamer     & \best{0.013}         & \best{0.847}    \\
            Ours            & 0.014	& 0.842 \\

\bottomrule
\end{tabular}
}
\vspace{-5mm}
\end{table}


\vspace{-2mm}
\subsection{Downstream Application}
\vspace{-2mm}
To demonstrate the effectiveness of our method, we also applied it to the downstream 3D reconstruction task. Specifically, we trained the NeuS model~\cite{neus} directly using images synthesized by our method, Zero123, and SyncDreamer, respectively.
The quantitative results in Tab.~\ref{tab:neus} show that the consistent multi-view images synthesized by our method can significantly improve the 3D reconstruction quality.
Additionally, our method exhibits similar performance to SyncDreamer which requires time-consuming re-training.
The qualitative results in Fig.~\ref{fig:neus} show that it is challenging to train the NeuS model directly due to the lack of consistency in the images generated by Zero123. In contrast, our method generates more consistent multi-view images and, therefore, better reconstructs the geometry and texture details.
We show improvements on other downstream applications such as image-to-3D in the Supplementary Material.


\section{Related Work}

% Reaction Diffusion
\paragraph{Wave-based Computing}
While prior work on wave-based computing in trainable task-oriented neural networks remains scarce, there is a rich history of using wave-like or other spatiotemporal field dynamics generally for computation.  
Early work studied the ability for waves to perform simple logical operations and thereby compute in a distributed manner \citep{pwc, wave_compute}, while other work has studied the ability for physical water waves to act as literal instantiations of classic `reservoir computers' \citep{maksymov2023analoguephysicalreservoircomputing}. Classically, the domain of `Neural Field Theory' has studied the role of spatiotemporal field dynamics in neural computation from a rigorous mathematical standpoint, although to-date these models have not been adapted to deep-neural network task-oriented performance. We refer readers to \cite{nft} for a thorough review of such models. 

More recently, \cite{hughes2019wave} have noted the analogy between the wave equation and recurrent neural networks, as we have done here, and used this to suggest that wave-based RNNs with learnable wave speeds may perform a type of analog computation. The authors use this to perform acoustic signal classification in a simplified setting, similar to our study in spirit, but differing in how waves are used and their computational purpose. Most related to the present study, \cite{BALKENHOL20244288} use an architecture similar to ours, with a Laplacian recurrent operator, damping, and gating, to show that when provided with an audio signal at a specific spatial location of the network, neurons at more distant locations can perfectly reconstruct the signal. The authors also show that this network is able to reproduce electrical recordings from macaque monkeys in response to simple grating stimuli, hypothesizing that their detection of high frequency waves is highly related to the transfer of information over large cortical distances.  

\looseness=-1
In terms of task-oriented wave-based models, recent work by \cite{felix} extensively studies the computational abilities of oscillatory neural networks, and specifically notes the emergence of traveling waves in these models in response to visual stimuli. Similarly, work by \cite{nwm, wrnn} studies wave-based RNNs for sequence processing and prediction. Our work fundamentally differs from these in the precise study of how these waves may be utilized for the spatial integration of visual information, as is hypothesized to happen in the visual cortex. Furthermore, our work uniquely demonstrates that a timeseries based readout is crucial for performing this type of integration, inspired by Kac's question, opening the door for future novel applications of these models. 

\vspace{-4mm}
\paragraph{Recurrence vs. Depth}
Another relevant line of research concerns the ability to trade off depth for recurrence in CNNs. 
Early work in this area was performed by \citet{liao2020bridginggapsresiduallearning}, with a more extensive recent study performed by \citet{schwarzschild2022the}. The authors demonstrate how iterating a single convolutional layer in a deep CNN yields similar performance to equivalently deep fully untied CNNs. Our work differs from these in that we demonstrate the advantage of a timeseries readout mechanism, inspired by Kac's question, whereas prior work can be seen as using the 'last' hidden state mechanism, that we see underperforms in this work. Interestingly, our findings thus suggest a potential novel method to improve the performance of these recurrent alternatives to deep networks through the use of our readout, a direction we intend to study in future work. Other more machine learning focused work has studied the impact of various weight-sharing schemes in deep convolutional networks \citep{eigen2014understandingdeeparchitecturesusing, jastrzębski2018residualconnectionsencourageiterative, boulch2017sharesnetreducingresidualnetwork}, however these share the same distinction with the present study in terms of their readout mechanism, while our proposed timeseries readouts appear to be uniquely linked to the wave dynamics that emerge in our models. 


\subsubsection{Binding By Synchrony}
Finally, we believe our work shares an interesting connection with the ``binding by synchrony'' concept \citep{Singer:2007} from early neuroscience research. Specifically, while our model's `binding' of parts into wholes does not rely on precise zero-lag synchrony—where oscillators within an object are perfectly in phase, as in the original framework; our method does rely on traveling waves of activity within objects that can be interpreted as a type of phase-lag synchrony. The ``binding operation'' then involves a transformation of the time signal using a suitable linear projection (our proposed timeseries readout). We believe this connection is valuable precisely since it enables a connection with the extensive historical literature on this concept, while simultaneously forming novel predictions on how such phenomena might manifest in natural neural systems. 
On the machine learning side of this concept, our work shares a strong connection with a class of object-centric learning methods which leverage a notion of synchrony of neural activations to define `bound' visual units for computational purposes. This includes models such as complex autoencoders \citep{lowe_complex-valued_2022, lowe_rotating_2024, stanic_contrastive_2024, gopalakrishnan_recurrent_2024} and recent Artificial Kuramoto Oscillatory Neurons (AKOrN) \cite{miyato_artificial_2024}. 
Unlike our method, the waves in the AKOrN model are not used directly as a representation themselves, but instead are neglected through the use of the `last hidden state' readout method. Perhaps most related to our work, \cite{liboni_image_2023} use a complex-valued recurrent neural network designed to generate traveling waves for image segmentation, with binding information encoded in the temporal phase sequence of these waves. This method can indeed be seen as using traveling waves to integrate information spatially, but contains no trainable components, offering a more theoretical exposition to the problem, as opposed to the task-oriented empirical study presented here. 
\section{Concluding Remarks}
In this paper, we proposed a novel approach utilizing multimodal LLMs to generate gesture-aware speech recognition transcripts for patients with language disorders. Our framework integrates verbal speech and iconic gestures, enabling the generation of enriched transcripts that capture the latent meaning conveyed through both modalities. Through extensive experimentation, we demonstrated that the proposed method effectively contextualizes incomplete or disfluent speech by incorporating gesture information, leading to more accurate and meaningful representations of the speaker's intent. These findings highlight the potential of our approach to significantly contribute to the field of speech and language therapy, offering innovative tools that can enhance the quality of life for individuals with language disorders by facilitating better communication and assessment methods.

\subsection{Ethical Statement} 
Our dataset was obtained from AphasiaBank with the approval of the Institutional Review Board (IRB) and adheres to the data sharing guidelines set by TalkBank\footnote{https://talkbank.org/share/ethics.html}. This includes complying with the Ground Rules for all TalkBank databases, which are based on the American Psychological Association Code of Ethics~\cite{american2002ethical}.

\subsection{Limitation \& Future Work} 
%This study represents a preliminary investigation into using multimodal LLMs to generate gesture-aware speech recognition transcripts. 
While the results are promising, we recognize several limitations and outline our plans to extend this work further.

One primary limitation is the absence of a definitive ground truth for quantitative evaluation. Since our model generates transcripts by synthesizing speech and gesture data from scratch, traditional benchmarks, such as comparisons with standard speech recognition outputs, are insufficient. Moreover, existing original transcripts lack gesture annotations, making direct comparisons challenging. In future work, we aim to address this gap by collaborating with certified pathologists to conduct qualitative assessments, such as A-B preference tests, to evaluate the effectiveness of gesture-enriched transcripts in accurately conveying the speaker's intentions.

To support quantitative evaluations, we plan to develop novel metrics that assess transcript quality, including grammar accuracy, semantic consistency, and the integration of multimodal information. Such metrics will provide a more objective basis for assessing our model's performance and facilitate comparisons with other multimodal and unimodal approaches.

Another limitation of this study is its focus on structured gestures from a specific task, the Peanut Butter Sandwich Task. While this task offers a controlled context for testing our approach, it does not encompass the diversity of gestures and communication patterns seen in everyday scenarios. As part of our future work, we plan to expand the scope of our model to include tasks such as the Cinderella Story Recall Task~\cite{bird1996cinderella}, which involves unstructured and complex narrative gestures. This expansion will allow us to evaluate the adaptability and robustness of our model in handling varied linguistic and gestural contexts.

In summary, while this study establishes a strong foundation for gesture-aware speech recognition, we aim to refine and extend our methods through collaborative qualitative evaluations, the development of robust quantitative metrics, and broader task applications. These efforts will ensure that our approach continues to evolve, ultimately contributing to more effective communication tools and interventions for individuals with language disorders.






\newpage

\bibliographystyle{ccn_style}


\bibliography{ccn_style}
\appendix
\onecolumn
\begin{appendices}
\section{Supplementary Material}
\subsection{Experimental Details}

This section provides details on the training and evaluation procedures for the models presented in this paper. The full code for reproducing results and visualizations from the main text is available at: \url{https://github.com/anonymous123-user/Traveling_Waves_Integrate}.

Each model is trained for 300 epochs on the MNIST, Tetrominoes, and Multi-MNIST datasets. We evaluate the validation loss at the end of every epoch and retain the model with the lowest validation loss throughout training. The training process employs the Adam optimizer \citep{kingma_adam_2017} with a learning rate of 0.001 and a batch size of 64.

For dataset partitioning, we use 51,000 images for training, 9,000 for validation, and 10,000 for testing in MNIST. The Tetrominoes dataset consists of 10,000 images for training, 1,000 for validation, and 1,000 for testing. Similarly, the Multi-MNIST dataset comprises 10,000 images for training, 1,000 for validation, and 1,000 for testing. Table \ref{tab:merged_appendix} reports pixel-wise accuracy, IoU, and loss for both foreground and background.

Each model is trained using multiple random seeds. For MNIST and Tetrominoes, we train each model using 10 different random seeds, while for Multi-MNIST, we use 12 seeds. For example, the NWM with a linear readout is trained on MNIST 10 times, each with a different random seed. After training, we evaluate each model individually and present the aggregated results, including the mean and standard deviation, in Tables \ref{tab:merged}, \ref{tab:multi-mnist}, and \ref{tab:merged_appendix}. In total, we train 150 models on MNIST, 150 models on Tetrominoes, and 60 models on Multi-MNIST, leading to a total of 360 models.

We train the Conv-LSTMs for 20 timesteps. The NWM model is trained for 100 timesteps on the MNIST, Tetrominoes, and Multi-MNIST datasets, while for the polygons dataset, the NWM runs for 500 timesteps. In the FFT readout, we use the real component of the discrete Fourier transform. This results in 50 bins for the NWM on MNIST, Tetrominoes, and Multi-MNIST, and 250 bins for the polygons dataset. The LSTM model outputs 10 Fourier bins for MNIST, Tetrominoes, and Multi-MNIST.

All convolutional models are trained with 16 channels. To ensure a fair comparison with the NWM, a linear layer operating channelwise outputs 100 channels. The readout MLP used for MNIST and Tetrominoes consists of four layers. Its input size is given by the number of Fourier bins multiplied by 2, followed by two hidden layers of 256 neurons each, with ReLU activation between layers, and a final output layer producing logits for classification. On Multi-MNIST, the NWM employs a six-layer readout with 32 neurons in each hidden layer.

For the U-Net architecture, the Arch parameter in Tables \ref{tab:multi-mnist} and \ref{tab:merged_appendix} refers to the number of feature maps output by the first convolutional layer. Each U-Net begins with stacked convolutions that maintain spatial resolution, producing $Arch$ feature maps (e.g., 3). The following four layers apply multiple convolutional operations per layer, each reducing spatial resolution by half while doubling the number of feature maps. If the initial layer outputs 3 feature maps, the subsequent layers modify the number as follows:
\[
3 \rightarrow 6 \rightarrow 12 \rightarrow 24 \rightarrow 48.
\]
The final number of feature maps, determined by $Arch$, follows:
\[
2 \rightarrow 32, \quad 3 \rightarrow 48, \quad 4 \rightarrow 64, \quad 5 \rightarrow 80.
\]

The U-Net decoder progressively upsamples the spatial resolution over four layers while simultaneously reducing the number of feature maps by half. By the fourth layer, it restores the feature map count to the value specified by $Arch$. Finally, 1×1 convolutions are used to project the feature maps into logits for pixel-wise classification.

\newpage

\subsection{Extended Semantic Segmentation Results}
Below we include the accuracies, IoU, and Loss values computed over the full set of classes (including the background class) for all models presented in the main text. While these numbers are artificially inflated from the inclusion of the background class (which dominates the majority of pixels) we include them here for competeness. 
\begin{table*}[h!]
    \centering
    \begin{tabular}{lllrrr}
    \toprule
    Dataset & Model & Architecture & Acc & IOU & Loss \\
    \midrule
    MNIST & CNN & 2 & 0.89 $\pm$ 0.01 & 0.20 $\pm$ 0.07 & 0.34 $\pm$ 0.13 \\
      &     & 4 & 0.89 $\pm$ 0.01 & 0.22 $\pm$ 0.12 & 0.35 $\pm$ 0.18 \\
      &     & 8 & 0.92 $\pm$ 0.02 & 0.40 $\pm$ 0.14 & 0.25 $\pm$ 0.16 \\
      &     & \textbf{\boldmath 16} & \textbf{\boldmath 0.94 $\pm$ 0.05} & \textbf{\boldmath 0.55 $\pm$ 0.38} & \textbf{\boldmath 0.27 $\pm$ 0.29} \\
      &     & 32 & 0.90 $\pm$ 0.06 & 0.27 $\pm$ 0.43 & 0.50 $\pm$ 0.31 \\
    \hline
      & LSTM & Max   & 0.92 $\pm$ 0.01 & 0.43 $\pm$ 0.04 & 0.21 $\pm$ 0.02 \\
      &      & Mean  & 0.92 $\pm$ 0.01 & 0.43 $\pm$ 0.07 & 0.21 $\pm$ 0.03 \\
      &      & Last  & 0.91 $\pm$ 0.03 & 0.32 $\pm$ 0.23 & 0.35 $\pm$ 0.24 \\
      &      & FFT   & 0.96 $\pm$ 0.04 & 0.71 $\pm$ 0.26 & 0.14 $\pm$ 0.20 \\
      &      & \textbf{\boldmath Linear} 
                        & \textbf{\boldmath 0.97 $\pm$ 0.04} 
                        & \textbf{\boldmath 0.76 $\pm$ 0.27} 
                        & \textbf{\boldmath 0.12 $\pm$ 0.20} \\
    \hline
      & \textbf{CoRNN} & Max   & 0.93 $\pm$ 0.01 & 0.47 $\pm$ 0.07 & 0.19 $\pm$ 0.02 \\
      &                & Mean  & 0.93 $\pm$ 0.01 & 0.46 $\pm$ 0.07 & 0.20 $\pm$ 0.03 \\
      &                & Last  & 0.93 $\pm$ 0.01 & 0.49 $\pm$ 0.08 & 0.18 $\pm$ 0.03 \\
      &                & FFT   & 0.97 $\pm$ 0.01 & 0.72 $\pm$ 0.06 & 0.10 $\pm$ 0.02 \\
      &                & \textbf{\boldmath Linear} 
                        & \textbf{\boldmath 0.99 $\pm$ 0.00} 
                        & \textbf{\boldmath 0.88 $\pm$ 0.03} 
                        & \textbf{\boldmath 0.04 $\pm$ 0.01} \\
    \hline\hline
    Tetrominoes & CNN & 2 & 0.89 $\pm$ 0.01 & 0.24 $\pm$ 0.09 & 0.27 $\pm$ 0.13 \\
            &     & 4 & 0.90 $\pm$ 0.02 & 0.27 $\pm$ 0.14 & 0.28 $\pm$ 0.19 \\
            &     & \textbf{\boldmath 8} 
                        & \textbf{\boldmath 0.96 $\pm$ 0.04} 
                        & \textbf{\boldmath 0.66 $\pm$ 0.23} 
                        & \textbf{\boldmath 0.11 $\pm$ 0.19} \\
            &     & 16 & 0.91 $\pm$ 0.07 & 0.40 $\pm$ 0.51 & 0.39 $\pm$ 0.33 \\
            &     & 32 & 0.90 $\pm$ 0.07 & 0.33 $\pm$ 0.47 & 0.41 $\pm$ 0.31 \\
    \hline
            & LSTM & Max  & 0.94 $\pm$ 0.04 & 0.56 $\pm$ 0.30 & 0.16 $\pm$ 0.18 \\
            &      & Mean & 0.95 $\pm$ 0.04 & 0.57 $\pm$ 0.24 & 0.15 $\pm$ 0.18 \\
            &      & Last & 0.94 $\pm$ 0.06 & 0.54 $\pm$ 0.39 & 0.24 $\pm$ 0.28 \\
            &      & FFT  & 0.99 $\pm$ 0.01 & 0.93 $\pm$ 0.06 & 0.03 $\pm$ 0.02 \\
            &      & \textbf{\boldmath Linear} 
                        & \textbf{\boldmath 0.99 $\pm$ 0.00}
                        & \textbf{\boldmath 0.96 $\pm$ 0.02}
                        & \textbf{\boldmath 0.02 $\pm$ 0.01} \\
    \hline
            & \textbf{CoRNN} & Max  & 0.98 $\pm$ 0.02 & 0.85 $\pm$ 0.14 & 0.05 $\pm$ 0.03 \\
            &                & Mean & 0.99 $\pm$ 0.01 & 0.89 $\pm$ 0.10 & 0.04 $\pm$ 0.02 \\
            &                & Last & 0.99 $\pm$ 0.01 & 0.92 $\pm$ 0.08 & 0.03 $\pm$ 0.02 \\
            &                & FFT  & 1.00 $\pm$ 0.00 & 0.98 $\pm$ 0.01 & 0.01 $\pm$ 0.00 \\
            &                & \textbf{\boldmath Linear} 
                        & \textbf{\boldmath 1.00 $\pm$ 0.00} 
                        & \textbf{\boldmath 0.98 $\pm$ 0.01}
                        & \textbf{\boldmath 0.01 $\pm$ 0.00} \\
    \bottomrule
    \end{tabular}
    % }
    \caption{Supervised segmentation performance of various models and spectral methods.
    Models with the lowest foreground loss are in bold.
    Arch (architecture) for the CNN refers to the number of layers, while for the LSTM and NWM refers to the type of recurrent readout used.
    Each model is trained with 10 random seeds, and the results are displayed as $mean \pm standard \text{ }deviation$ over the 10 seeds.}
    \vspace{-3mm}
    \label{tab:merged_appendix}
\end{table*}
\begin{table*}[h!]
    \centering
    \begin{tabular}{llllrrr}
    \toprule
        & Model & Arch. & Parameters & Acc & IoU & Loss \\
    \midrule
        & U-Net & 2     & 30745      & 0.98 ± 0.01 & 0.66 ± 0.27 & 0.06 ± 0.03 \\
        &       & 3     & 68834      & 1.00 ± 0.00 & 0.91 ± 0.08 & 0.03 ± 0.01 \\
        &       & 4     & 122071     & 1.00 ± 0.00 & 0.97 ± 0.01 & 0.01 ± 0.00 \\
        &       & 5     & 190456     & 1.00 ± 0.00 & 0.98 ± 0.00 & 0.01 ± 0.00 \\
    \midrule
        & NWM   & Linear & 54855      & 1.00 ± 0.00 & 0.94 ± 0.01 & 0.01 ± 0.00 \\
    \bottomrule
    \end{tabular}
    % }
    \caption{Supervised segmentation performance of UNet and NWM with Linear Time Projection on Multi-MNIST. Arch for the U-Net refers to the number of feature maps output by the first layer. The number of feature maps doubles between each layer (e.g. 3 means 3 $\rightarrow$ 6 $\rightarrow$ 12 $\rightarrow$ 24 $\rightarrow$ 48 by the final layer). For the NWM, Arch (architecture) refers to the type of recurrent readout used.
    Each model is trained with 12 random seeds, and the results are displayed as $mean \pm standard \text{ }deviation$ over the 12 seeds.}
    \vspace{-3mm}
    \label{tab:multi-mnist_appendix}
\end{table*}

Below, we include the minimum, maximum, and median accuracies, IoU, and loss values for MNIST and Tetrominoes.
%--------------------------------------------------------------
% Table A: Acc, IoU, Loss
%--------------------------------------------------------------
\begin{table*}[ht!]
\centering
\begin{minipage}{0.99\linewidth}
\centering
\begin{tabular}{lllccc}
\toprule
 &  &  & Acc & IoU & Loss \\
\midrule
\multirow{15}{*}{MNIST} 
 & \multirow{5}{*}{CNN} 
 & 2 
 & 0.89 / 0.89 / 0.87 
 & 0.25 / 0.23 / 0.00 
 & 0.70 / 0.29 / 0.29 
\\
 &  & 4 
 & 0.90 / 0.90 / 0.87 
 & 0.31 / 0.28 / 0.00 
 & 0.70 / 0.27 / 0.26 
\\
 &  & 8 
 & 0.93 / 0.93 / 0.87 
 & 0.47 / 0.45 / 0.00 
 & 0.70 / 0.20 / 0.19 
\\
 &  & 16 
 & 0.98 / 0.97 / 0.87 
 & 0.81 / 0.78 / 0.00 
 & 0.70 / 0.09 / 0.07 
\\
 &  & 32 
 & 0.99 / 0.87 / 0.87 
 & 0.90 / 0.00 / 0.00 
 & 0.70 / 0.69 / 0.05 
\\
\cline{2-6}
 & \multirow{5}{*}{LSTM} 
 & Linear 
 & 0.99 / 0.98 / 0.87 
 & 0.91 / 0.86 / 0.00 
 & 0.70 / 0.05 / 0.03 
\\
 &  & FFT 
 & 0.98 / 0.97 / 0.87 
 & 0.87 / 0.78 / 0.00 
 & 0.70 / 0.07 / 0.05 
\\
 &  & Last 
 & 0.94 / 0.92 / 0.87 
 & 0.57 / 0.41 / 0.00 
 & 0.70 / 0.21 / 0.16 
\\
 &  & Mean 
 & 0.94 / 0.92 / 0.90 
 & 0.52 / 0.43 / 0.31 
 & 0.26 / 0.20 / 0.17 
\\
 &  & Max 
 & 0.93 / 0.92 / 0.91 
 & 0.48 / 0.43 / 0.34 
 & 0.24 / 0.20 / 0.19 
\\
\cline{2-6}
 & \multirow{5}{*}{CoRNN} 
 & Linear 
 & 0.99 / 0.99 / 0.98 
 & 0.93 / 0.88 / 0.83 
 & 0.06 / 0.04 / 0.03 
\\
 &  & FFT 
 & 0.98 / 0.97 / 0.95 
 & 0.81 / 0.74 / 0.61 
 & 0.14 / 0.09 / 0.07 
\\
 &  & Last 
 & 0.95 / 0.94 / 0.91 
 & 0.60 / 0.51 / 0.34 
 & 0.23 / 0.17 / 0.14 
\\
 &  & Mean 
 & 0.94 / 0.93 / 0.91 
 & 0.56 / 0.45 / 0.36 
 & 0.24 / 0.20 / 0.16 
\\
 &  & Max 
 & 0.94 / 0.93 / 0.92 
 & 0.57 / 0.45 / 0.38 
 & 0.21 / 0.20 / 0.15 
\\
\midrule
\multirow{15}{*}{Tetrominoes} 
 & \multirow{5}{*}{CNN} 
 & 2 
 & 0.89 / 0.89 / 0.86 
 & 0.29 / 0.26 / 0.00 
 & 0.65 / 0.22 / 0.22 
\\
 &  & 4 
 & 0.91 / 0.91 / 0.86 
 & 0.36 / 0.34 / 0.00 
 & 0.65 / 0.19 / 0.18 
\\
 &  & 8 
 & 0.97 / 0.97 / 0.86 
 & 0.76 / 0.73 / 0.00 
 & 0.65 / 0.06 / 0.05 
\\
 &  & 16 
 & 1.00 / 0.86 / 0.86 
 & 1.00 / 0.00 / 0.00 
 & 0.65 / 0.65 / 0.00 
\\
 &  & 32 
 & 1.00 / 0.86 / 0.86 
 & 1.00 / 0.00 / 0.00 
 & 0.65 / 0.63 / 0.00 
\\
\cline{2-6}
 & \multirow{5}{*}{LSTM} 
 & Linear 
 & 1.00 / 1.00 / 0.99 
 & 0.98 / 0.97 / 0.91 
 & 0.04 / 0.02 / 0.01 
\\
 &  & FFT 
 & 1.00 / 0.99 / 0.98 
 & 0.97 / 0.96 / 0.79 
 & 0.06 / 0.02 / 0.01 
\\
 &  & Last 
 & 0.99 / 0.98 / 0.86 
 & 0.91 / 0.74 / 0.00 
 & 0.65 / 0.06 / 0.04 
\\
 &  & Mean 
 & 0.99 / 0.95 / 0.86 
 & 0.95 / 0.58 / 0.00 
 & 0.65 / 0.10 / 0.03 
\\
 &  & Max 
 & 0.99 / 0.94 / 0.86 
 & 0.94 / 0.48 / 0.00 
 & 0.65 / 0.13 / 0.03 
\\
\cline{2-6}
 & \multirow{5}{*}{CoRNN} 
 & Linear 
 & 1.00 / 1.00 / 1.00 
 & 0.99 / 0.98 / 0.97 
 & 0.01 / 0.01 / 0.00 
\\
 &  & FFT 
 & 1.00 / 1.00 / 0.99 
 & 0.99 / 0.98 / 0.97 
 & 0.02 / 0.01 / 0.01 
\\
 &  & Last 
 & 1.00 / 0.99 / 0.97 
 & 0.98 / 0.96 / 0.76 
 & 0.08 / 0.02 / 0.01 
\\
 &  & Mean 
 & 1.00 / 0.99 / 0.97 
 & 0.99 / 0.91 / 0.69 
 & 0.07 / 0.03 / 0.01 
\\
 &  & Max 
 & 1.00 / 0.99 / 0.95 
 & 0.98 / 0.90 / 0.59 
 & 0.12 / 0.04 / 0.01 
\\
\bottomrule
\end{tabular}
\caption{Segmentation performance (only Acc, IoU, and Loss). 
Values are max / median / min over 10 seeds.}
\label{tab:max_min}
\end{minipage}
\end{table*}


%--------------------------------------------------------------
% Table B: FG-Acc, FG-IoU, FG-Loss
%--------------------------------------------------------------
\begin{table*}[ht!]
\centering
\begin{minipage}{0.99\linewidth}
\centering
\begin{tabular}{lllccc}
\toprule
 &  &  & FG-Acc & FG-IoU & FG-Loss \\
\midrule
\multirow{15}{*}{MNIST} 
 & \multirow{5}{*}{CNN} 
 & 2 
 & 0.18 / 0.17 / 0.00 
 & 0.12 / 0.10 / 0.00 
 & 4.35 / 2.21 / 2.20 
\\
 &  & 4 
 & 0.28 / 0.25 / 0.00 
 & 0.18 / 0.17 / 0.00 
 & 4.37 / 2.00 / 1.93 
\\
 &  & 8 
 & 0.49 / 0.47 / 0.00 
 & 0.36 / 0.33 / 0.00 
 & 4.31 / 1.47 / 1.42 
\\
 &  & 16 
 & 0.83 / 0.81 / 0.00 
 & 0.75 / 0.72 / 0.00 
 & 4.35 / 0.63 / 0.55 
\\
 &  & 32 
 & 0.91 / 0.00 / 0.00 
 & 0.86 / 0.00 / 0.00 
 & 4.37 / 4.29 / 0.33 
\\
\cline{2-6}
 & \multirow{5}{*}{LSTM} 
 & Linear 
 & 0.93 / 0.88 / 0.00 
 & 0.88 / 0.82 / 0.00 
 & 4.33 / 0.35 / 0.24 
\\
 &  & FFT 
 & 0.89 / 0.81 / 0.00 
 & 0.83 / 0.72 / 0.00 
 & 4.39 / 0.56 / 0.35 
\\
 &  & Last 
 & 0.58 / 0.40 / 0.00 
 & 0.46 / 0.28 / 0.00 
 & 4.34 / 1.58 / 1.18 
\\
 &  & Mean 
 & 0.53 / 0.43 / 0.27 
 & 0.41 / 0.32 / 0.20 
 & 1.97 / 1.53 / 1.30 
\\
 &  & Max 
 & 0.47 / 0.44 / 0.32 
 & 0.36 / 0.32 / 0.22 
 & 1.83 / 1.52 / 1.42 
\\
\cline{2-6}
 & \multirow{5}{*}{CoRNN} 
 & Linear 
 & 0.94 / 0.90 / 0.86 
 & 0.91 / 0.85 / 0.78 
 & 0.42 / 0.31 / 0.20 
\\
 &  & FFT 
 & 0.84 / 0.78 / 0.65 
 & 0.76 / 0.67 / 0.52 
 & 1.04 / 0.68 / 0.51 
\\
 &  & Last 
 & 0.61 / 0.52 / 0.35 
 & 0.49 / 0.41 / 0.25 
 & 1.74 / 1.30 / 1.07 
\\
 &  & Mean 
 & 0.57 / 0.46 / 0.34 
 & 0.46 / 0.34 / 0.24 
 & 1.78 / 1.48 / 1.16 
\\
 &  & Max 
 & 0.59 / 0.46 / 0.41 
 & 0.47 / 0.34 / 0.31 
 & 1.55 / 1.47 / 1.11 
\\
\midrule
\multirow{15}{*}{Tetrominoes} 
 & \multirow{5}{*}{CNN} 
 & 2 
 & 0.26 / 0.26 / 0.00 
 & 0.16 / 0.16 / 0.00 
 & 3.50 / 1.55 / 1.54 
\\
 &  & 4 
 & 0.40 / 0.39 / 0.00 
 & 0.27 / 0.25 / 0.00 
 & 3.55 / 1.30 / 1.28 
\\
 &  & 8 
 & 0.83 / 0.82 / 0.00 
 & 0.72 / 0.70 / 0.00 
 & 3.55 / 0.38 / 0.33 
\\
 &  & 16 
 & 1.00 / 0.00 / 0.00 
 & 1.00 / 0.00 / 0.00 
 & 3.61 / 3.50 / 0.01 
\\
 &  & 32 
 & 1.00 / 0.00 / 0.00 
 & 1.00 / 0.00 / 0.00 
 & 3.60 / 3.40 / 0.01 
\\
\cline{2-6}
 & \multirow{5}{*}{LSTM} 
 & Linear 
 & 0.98 / 0.98 / 0.92 
 & 0.97 / 0.96 / 0.87 
 & 0.25 / 0.10 / 0.08 
\\
 &  & FFT 
 & 0.98 / 0.97 / 0.84 
 & 0.96 / 0.94 / 0.74 
 & 0.45 / 0.14 / 0.09 
\\
 &  & Last 
 & 0.92 / 0.84 / 0.00 
 & 0.86 / 0.73 / 0.00 
 & 3.55 / 0.43 / 0.29 
\\
 &  & Mean 
 & 0.96 / 0.68 / 0.00 
 & 0.93 / 0.53 / 0.00 
 & 3.61 / 0.72 / 0.20 
\\
 &  & Max 
 & 0.96 / 0.57 / 0.00 
 & 0.92 / 0.42 / 0.00 
 & 3.44 / 0.93 / 0.21 
\\
\cline{2-6}
 & \multirow{5}{*}{CoRNN} 
 & Linear 
 & 0.99 / 0.99 / 0.98 
 & 0.99 / 0.98 / 0.96 
 & 0.09 / 0.05 / 0.03 
\\
 &  & FFT 
 & 0.99 / 0.98 / 0.97 
 & 0.98 / 0.97 / 0.95 
 & 0.11 / 0.06 / 0.04 
\\
 &  & Last 
 & 0.98 / 0.97 / 0.81 
 & 0.97 / 0.95 / 0.70 
 & 0.53 / 0.15 / 0.10 
\\
 &  & Mean 
 & 0.99 / 0.94 / 0.78 
 & 0.98 / 0.89 / 0.65 
 & 0.50 / 0.23 / 0.06 
\\
 &  & Max 
 & 0.99 / 0.92 / 0.69 
 & 0.98 / 0.86 / 0.54 
 & 0.82 / 0.27 / 0.09 
\\
\bottomrule
\end{tabular}
\caption{Segmentation performance (only FG-Acc, FG-IoU, and FG-Loss). 
Values are max / median / min over 10 seeds.}
\label{tab:fg_max_min}
\end{minipage}
\end{table*}


\begin{figure}[h!] % 'htbp' specifies how the figure should be placed
    \centering
    \includegraphics[width=\linewidth]{figures/fft_mulitpolygon.png} % Width can be adjusted as needed
    % \vspace{-5mm}
    \caption{Visualization of all frequency bins for an example of the Polygons dataset. We see that the background and different shapes appear in separate frequency bins, allowing the model to easily segment the shapes semantically in frequency space. }
    \label{fig:all_fft}
\end{figure}



\begin{figure}[h!] % 'htbp' specifies how the figure should be placed
    \centering
    \includegraphics[width=0.7\linewidth]{figures/shape_combinations.png} % Width can be adjusted as needed
    % \vspace{-5mm}
    \caption{Visualization of the impact of different combinations of shapes in the same image on the frequency space representation of the other shape for the Polygons dataset. We see that while there is a minor impact on the frequency space representation of each shape when another shape appears nearby, the overall frequency spectrum is relatively invariant. This implies that each neuron indeed has global information about all shapes present in the image, but mainly represents the shape which it is currently `located within'.}
    \label{fig:shape_combo}
\end{figure}
\end{appendices}

\end{document}
