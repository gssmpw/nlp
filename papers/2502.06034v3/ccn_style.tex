\documentclass[10pt,letterpaper]{article}

\usepackage{ccn}
\usepackage{pslatex}
\usepackage{apacite}
\usepackage{hyperref}
\usepackage{natbib}
\usepackage{lineno}
\usepackage{url}
\usepackage{booktabs}
\usepackage{microtype}
\usepackage{graphicx}
\usepackage{subfigure}
\usepackage{multirow} 
\usepackage[title]{appendix}
\usepackage{caption}
\usepackage{subcaption}
\usepackage{wrapfig}

\usepackage[T1]{fontenc}

%%%%% NEW MATH DEFINITIONS %%%%%

\usepackage{amsmath,amsfonts,bm}
\usepackage{derivative}
% Mark sections of captions for referring to divisions of figures
\newcommand{\figleft}{{\em (Left)}}
\newcommand{\figcenter}{{\em (Center)}}
\newcommand{\figright}{{\em (Right)}}
\newcommand{\figtop}{{\em (Top)}}
\newcommand{\figbottom}{{\em (Bottom)}}
\newcommand{\captiona}{{\em (a)}}
\newcommand{\captionb}{{\em (b)}}
\newcommand{\captionc}{{\em (c)}}
\newcommand{\captiond}{{\em (d)}}

% Highlight a newly defined term
\newcommand{\newterm}[1]{{\bf #1}}

% Derivative d 
\newcommand{\deriv}{{\mathrm{d}}}

% Figure reference, lower-case.
\def\figref#1{figure~\ref{#1}}
% Figure reference, capital. For start of sentence
\def\Figref#1{Figure~\ref{#1}}
\def\twofigref#1#2{figures \ref{#1} and \ref{#2}}
\def\quadfigref#1#2#3#4{figures \ref{#1}, \ref{#2}, \ref{#3} and \ref{#4}}
% Section reference, lower-case.
\def\secref#1{section~\ref{#1}}
% Section reference, capital.
\def\Secref#1{Section~\ref{#1}}
% Reference to two sections.
\def\twosecrefs#1#2{sections \ref{#1} and \ref{#2}}
% Reference to three sections.
\def\secrefs#1#2#3{sections \ref{#1}, \ref{#2} and \ref{#3}}
% Reference to an equation, lower-case.
\def\eqref#1{equation~\ref{#1}}
% Reference to an equation, upper case
\def\Eqref#1{Equation~\ref{#1}}
% A raw reference to an equation---avoid using if possible
\def\plaineqref#1{\ref{#1}}
% Reference to a chapter, lower-case.
\def\chapref#1{chapter~\ref{#1}}
% Reference to an equation, upper case.
\def\Chapref#1{Chapter~\ref{#1}}
% Reference to a range of chapters
\def\rangechapref#1#2{chapters\ref{#1}--\ref{#2}}
% Reference to an algorithm, lower-case.
\def\algref#1{algorithm~\ref{#1}}
% Reference to an algorithm, upper case.
\def\Algref#1{Algorithm~\ref{#1}}
\def\twoalgref#1#2{algorithms \ref{#1} and \ref{#2}}
\def\Twoalgref#1#2{Algorithms \ref{#1} and \ref{#2}}
% Reference to a part, lower case
\def\partref#1{part~\ref{#1}}
% Reference to a part, upper case
\def\Partref#1{Part~\ref{#1}}
\def\twopartref#1#2{parts \ref{#1} and \ref{#2}}

\def\ceil#1{\lceil #1 \rceil}
\def\floor#1{\lfloor #1 \rfloor}
\def\1{\bm{1}}
\newcommand{\train}{\mathcal{D}}
\newcommand{\valid}{\mathcal{D_{\mathrm{valid}}}}
\newcommand{\test}{\mathcal{D_{\mathrm{test}}}}

\def\eps{{\epsilon}}


% Random variables
\def\reta{{\textnormal{$\eta$}}}
\def\ra{{\textnormal{a}}}
\def\rb{{\textnormal{b}}}
\def\rc{{\textnormal{c}}}
\def\rd{{\textnormal{d}}}
\def\re{{\textnormal{e}}}
\def\rf{{\textnormal{f}}}
\def\rg{{\textnormal{g}}}
\def\rh{{\textnormal{h}}}
\def\ri{{\textnormal{i}}}
\def\rj{{\textnormal{j}}}
\def\rk{{\textnormal{k}}}
\def\rl{{\textnormal{l}}}
% rm is already a command, just don't name any random variables m
\def\rn{{\textnormal{n}}}
\def\ro{{\textnormal{o}}}
\def\rp{{\textnormal{p}}}
\def\rq{{\textnormal{q}}}
\def\rr{{\textnormal{r}}}
\def\rs{{\textnormal{s}}}
\def\rt{{\textnormal{t}}}
\def\ru{{\textnormal{u}}}
\def\rv{{\textnormal{v}}}
\def\rw{{\textnormal{w}}}
\def\rx{{\textnormal{x}}}
\def\ry{{\textnormal{y}}}
\def\rz{{\textnormal{z}}}

% Random vectors
\def\rvepsilon{{\mathbf{\epsilon}}}
\def\rvphi{{\mathbf{\phi}}}
\def\rvtheta{{\mathbf{\theta}}}
\def\rva{{\mathbf{a}}}
\def\rvb{{\mathbf{b}}}
\def\rvc{{\mathbf{c}}}
\def\rvd{{\mathbf{d}}}
\def\rve{{\mathbf{e}}}
\def\rvf{{\mathbf{f}}}
\def\rvg{{\mathbf{g}}}
\def\rvh{{\mathbf{h}}}
\def\rvu{{\mathbf{i}}}
\def\rvj{{\mathbf{j}}}
\def\rvk{{\mathbf{k}}}
\def\rvl{{\mathbf{l}}}
\def\rvm{{\mathbf{m}}}
\def\rvn{{\mathbf{n}}}
\def\rvo{{\mathbf{o}}}
\def\rvp{{\mathbf{p}}}
\def\rvq{{\mathbf{q}}}
\def\rvr{{\mathbf{r}}}
\def\rvs{{\mathbf{s}}}
\def\rvt{{\mathbf{t}}}
\def\rvu{{\mathbf{u}}}
\def\rvv{{\mathbf{v}}}
\def\rvw{{\mathbf{w}}}
\def\rvx{{\mathbf{x}}}
\def\rvy{{\mathbf{y}}}
\def\rvz{{\mathbf{z}}}

% Elements of random vectors
\def\erva{{\textnormal{a}}}
\def\ervb{{\textnormal{b}}}
\def\ervc{{\textnormal{c}}}
\def\ervd{{\textnormal{d}}}
\def\erve{{\textnormal{e}}}
\def\ervf{{\textnormal{f}}}
\def\ervg{{\textnormal{g}}}
\def\ervh{{\textnormal{h}}}
\def\ervi{{\textnormal{i}}}
\def\ervj{{\textnormal{j}}}
\def\ervk{{\textnormal{k}}}
\def\ervl{{\textnormal{l}}}
\def\ervm{{\textnormal{m}}}
\def\ervn{{\textnormal{n}}}
\def\ervo{{\textnormal{o}}}
\def\ervp{{\textnormal{p}}}
\def\ervq{{\textnormal{q}}}
\def\ervr{{\textnormal{r}}}
\def\ervs{{\textnormal{s}}}
\def\ervt{{\textnormal{t}}}
\def\ervu{{\textnormal{u}}}
\def\ervv{{\textnormal{v}}}
\def\ervw{{\textnormal{w}}}
\def\ervx{{\textnormal{x}}}
\def\ervy{{\textnormal{y}}}
\def\ervz{{\textnormal{z}}}

% Random matrices
\def\rmA{{\mathbf{A}}}
\def\rmB{{\mathbf{B}}}
\def\rmC{{\mathbf{C}}}
\def\rmD{{\mathbf{D}}}
\def\rmE{{\mathbf{E}}}
\def\rmF{{\mathbf{F}}}
\def\rmG{{\mathbf{G}}}
\def\rmH{{\mathbf{H}}}
\def\rmI{{\mathbf{I}}}
\def\rmJ{{\mathbf{J}}}
\def\rmK{{\mathbf{K}}}
\def\rmL{{\mathbf{L}}}
\def\rmM{{\mathbf{M}}}
\def\rmN{{\mathbf{N}}}
\def\rmO{{\mathbf{O}}}
\def\rmP{{\mathbf{P}}}
\def\rmQ{{\mathbf{Q}}}
\def\rmR{{\mathbf{R}}}
\def\rmS{{\mathbf{S}}}
\def\rmT{{\mathbf{T}}}
\def\rmU{{\mathbf{U}}}
\def\rmV{{\mathbf{V}}}
\def\rmW{{\mathbf{W}}}
\def\rmX{{\mathbf{X}}}
\def\rmY{{\mathbf{Y}}}
\def\rmZ{{\mathbf{Z}}}

% Elements of random matrices
\def\ermA{{\textnormal{A}}}
\def\ermB{{\textnormal{B}}}
\def\ermC{{\textnormal{C}}}
\def\ermD{{\textnormal{D}}}
\def\ermE{{\textnormal{E}}}
\def\ermF{{\textnormal{F}}}
\def\ermG{{\textnormal{G}}}
\def\ermH{{\textnormal{H}}}
\def\ermI{{\textnormal{I}}}
\def\ermJ{{\textnormal{J}}}
\def\ermK{{\textnormal{K}}}
\def\ermL{{\textnormal{L}}}
\def\ermM{{\textnormal{M}}}
\def\ermN{{\textnormal{N}}}
\def\ermO{{\textnormal{O}}}
\def\ermP{{\textnormal{P}}}
\def\ermQ{{\textnormal{Q}}}
\def\ermR{{\textnormal{R}}}
\def\ermS{{\textnormal{S}}}
\def\ermT{{\textnormal{T}}}
\def\ermU{{\textnormal{U}}}
\def\ermV{{\textnormal{V}}}
\def\ermW{{\textnormal{W}}}
\def\ermX{{\textnormal{X}}}
\def\ermY{{\textnormal{Y}}}
\def\ermZ{{\textnormal{Z}}}

% Vectors
\def\vzero{{\bm{0}}}
\def\vone{{\bm{1}}}
\def\vmu{{\bm{\mu}}}
\def\vtheta{{\bm{\theta}}}
\def\vphi{{\bm{\phi}}}
\def\va{{\bm{a}}}
\def\vb{{\bm{b}}}
\def\vc{{\bm{c}}}
\def\vd{{\bm{d}}}
\def\ve{{\bm{e}}}
\def\vf{{\bm{f}}}
\def\vg{{\bm{g}}}
\def\vh{{\bm{h}}}
\def\vi{{\bm{i}}}
\def\vj{{\bm{j}}}
\def\vk{{\bm{k}}}
\def\vl{{\bm{l}}}
\def\vm{{\bm{m}}}
\def\vn{{\bm{n}}}
\def\vo{{\bm{o}}}
\def\vp{{\bm{p}}}
\def\vq{{\bm{q}}}
\def\vr{{\bm{r}}}
\def\vs{{\bm{s}}}
\def\vt{{\bm{t}}}
\def\vu{{\bm{u}}}
\def\vv{{\bm{v}}}
\def\vw{{\bm{w}}}
\def\vx{{\bm{x}}}
\def\vy{{\bm{y}}}
\def\vz{{\bm{z}}}

% Elements of vectors
\def\evalpha{{\alpha}}
\def\evbeta{{\beta}}
\def\evepsilon{{\epsilon}}
\def\evlambda{{\lambda}}
\def\evomega{{\omega}}
\def\evmu{{\mu}}
\def\evpsi{{\psi}}
\def\evsigma{{\sigma}}
\def\evtheta{{\theta}}
\def\eva{{a}}
\def\evb{{b}}
\def\evc{{c}}
\def\evd{{d}}
\def\eve{{e}}
\def\evf{{f}}
\def\evg{{g}}
\def\evh{{h}}
\def\evi{{i}}
\def\evj{{j}}
\def\evk{{k}}
\def\evl{{l}}
\def\evm{{m}}
\def\evn{{n}}
\def\evo{{o}}
\def\evp{{p}}
\def\evq{{q}}
\def\evr{{r}}
\def\evs{{s}}
\def\evt{{t}}
\def\evu{{u}}
\def\evv{{v}}
\def\evw{{w}}
\def\evx{{x}}
\def\evy{{y}}
\def\evz{{z}}

% Matrix
\def\mA{{\bm{A}}}
\def\mB{{\bm{B}}}
\def\mC{{\bm{C}}}
\def\mD{{\bm{D}}}
\def\mE{{\bm{E}}}
\def\mF{{\bm{F}}}
\def\mG{{\bm{G}}}
\def\mH{{\bm{H}}}
\def\mI{{\bm{I}}}
\def\mJ{{\bm{J}}}
\def\mK{{\bm{K}}}
\def\mL{{\bm{L}}}
\def\mM{{\bm{M}}}
\def\mN{{\bm{N}}}
\def\mO{{\bm{O}}}
\def\mP{{\bm{P}}}
\def\mQ{{\bm{Q}}}
\def\mR{{\bm{R}}}
\def\mS{{\bm{S}}}
\def\mT{{\bm{T}}}
\def\mU{{\bm{U}}}
\def\mV{{\bm{V}}}
\def\mW{{\bm{W}}}
\def\mX{{\bm{X}}}
\def\mY{{\bm{Y}}}
\def\mZ{{\bm{Z}}}
\def\mBeta{{\bm{\beta}}}
\def\mPhi{{\bm{\Phi}}}
\def\mLambda{{\bm{\Lambda}}}
\def\mSigma{{\bm{\Sigma}}}

% Tensor
\DeclareMathAlphabet{\mathsfit}{\encodingdefault}{\sfdefault}{m}{sl}
\SetMathAlphabet{\mathsfit}{bold}{\encodingdefault}{\sfdefault}{bx}{n}
\newcommand{\tens}[1]{\bm{\mathsfit{#1}}}
\def\tA{{\tens{A}}}
\def\tB{{\tens{B}}}
\def\tC{{\tens{C}}}
\def\tD{{\tens{D}}}
\def\tE{{\tens{E}}}
\def\tF{{\tens{F}}}
\def\tG{{\tens{G}}}
\def\tH{{\tens{H}}}
\def\tI{{\tens{I}}}
\def\tJ{{\tens{J}}}
\def\tK{{\tens{K}}}
\def\tL{{\tens{L}}}
\def\tM{{\tens{M}}}
\def\tN{{\tens{N}}}
\def\tO{{\tens{O}}}
\def\tP{{\tens{P}}}
\def\tQ{{\tens{Q}}}
\def\tR{{\tens{R}}}
\def\tS{{\tens{S}}}
\def\tT{{\tens{T}}}
\def\tU{{\tens{U}}}
\def\tV{{\tens{V}}}
\def\tW{{\tens{W}}}
\def\tX{{\tens{X}}}
\def\tY{{\tens{Y}}}
\def\tZ{{\tens{Z}}}


% Graph
\def\gA{{\mathcal{A}}}
\def\gB{{\mathcal{B}}}
\def\gC{{\mathcal{C}}}
\def\gD{{\mathcal{D}}}
\def\gE{{\mathcal{E}}}
\def\gF{{\mathcal{F}}}
\def\gG{{\mathcal{G}}}
\def\gH{{\mathcal{H}}}
\def\gI{{\mathcal{I}}}
\def\gJ{{\mathcal{J}}}
\def\gK{{\mathcal{K}}}
\def\gL{{\mathcal{L}}}
\def\gM{{\mathcal{M}}}
\def\gN{{\mathcal{N}}}
\def\gO{{\mathcal{O}}}
\def\gP{{\mathcal{P}}}
\def\gQ{{\mathcal{Q}}}
\def\gR{{\mathcal{R}}}
\def\gS{{\mathcal{S}}}
\def\gT{{\mathcal{T}}}
\def\gU{{\mathcal{U}}}
\def\gV{{\mathcal{V}}}
\def\gW{{\mathcal{W}}}
\def\gX{{\mathcal{X}}}
\def\gY{{\mathcal{Y}}}
\def\gZ{{\mathcal{Z}}}

% Sets
\def\sA{{\mathbb{A}}}
\def\sB{{\mathbb{B}}}
\def\sC{{\mathbb{C}}}
\def\sD{{\mathbb{D}}}
% Don't use a set called E, because this would be the same as our symbol
% for expectation.
\def\sF{{\mathbb{F}}}
\def\sG{{\mathbb{G}}}
\def\sH{{\mathbb{H}}}
\def\sI{{\mathbb{I}}}
\def\sJ{{\mathbb{J}}}
\def\sK{{\mathbb{K}}}
\def\sL{{\mathbb{L}}}
\def\sM{{\mathbb{M}}}
\def\sN{{\mathbb{N}}}
\def\sO{{\mathbb{O}}}
\def\sP{{\mathbb{P}}}
\def\sQ{{\mathbb{Q}}}
\def\sR{{\mathbb{R}}}
\def\sS{{\mathbb{S}}}
\def\sT{{\mathbb{T}}}
\def\sU{{\mathbb{U}}}
\def\sV{{\mathbb{V}}}
\def\sW{{\mathbb{W}}}
\def\sX{{\mathbb{X}}}
\def\sY{{\mathbb{Y}}}
\def\sZ{{\mathbb{Z}}}

% Entries of a matrix
\def\emLambda{{\Lambda}}
\def\emA{{A}}
\def\emB{{B}}
\def\emC{{C}}
\def\emD{{D}}
\def\emE{{E}}
\def\emF{{F}}
\def\emG{{G}}
\def\emH{{H}}
\def\emI{{I}}
\def\emJ{{J}}
\def\emK{{K}}
\def\emL{{L}}
\def\emM{{M}}
\def\emN{{N}}
\def\emO{{O}}
\def\emP{{P}}
\def\emQ{{Q}}
\def\emR{{R}}
\def\emS{{S}}
\def\emT{{T}}
\def\emU{{U}}
\def\emV{{V}}
\def\emW{{W}}
\def\emX{{X}}
\def\emY{{Y}}
\def\emZ{{Z}}
\def\emSigma{{\Sigma}}

% entries of a tensor
% Same font as tensor, without \bm wrapper
\newcommand{\etens}[1]{\mathsfit{#1}}
\def\etLambda{{\etens{\Lambda}}}
\def\etA{{\etens{A}}}
\def\etB{{\etens{B}}}
\def\etC{{\etens{C}}}
\def\etD{{\etens{D}}}
\def\etE{{\etens{E}}}
\def\etF{{\etens{F}}}
\def\etG{{\etens{G}}}
\def\etH{{\etens{H}}}
\def\etI{{\etens{I}}}
\def\etJ{{\etens{J}}}
\def\etK{{\etens{K}}}
\def\etL{{\etens{L}}}
\def\etM{{\etens{M}}}
\def\etN{{\etens{N}}}
\def\etO{{\etens{O}}}
\def\etP{{\etens{P}}}
\def\etQ{{\etens{Q}}}
\def\etR{{\etens{R}}}
\def\etS{{\etens{S}}}
\def\etT{{\etens{T}}}
\def\etU{{\etens{U}}}
\def\etV{{\etens{V}}}
\def\etW{{\etens{W}}}
\def\etX{{\etens{X}}}
\def\etY{{\etens{Y}}}
\def\etZ{{\etens{Z}}}

% The true underlying data generating distribution
\newcommand{\pdata}{p_{\rm{data}}}
\newcommand{\ptarget}{p_{\rm{target}}}
\newcommand{\pprior}{p_{\rm{prior}}}
\newcommand{\pbase}{p_{\rm{base}}}
\newcommand{\pref}{p_{\rm{ref}}}

% The empirical distribution defined by the training set
\newcommand{\ptrain}{\hat{p}_{\rm{data}}}
\newcommand{\Ptrain}{\hat{P}_{\rm{data}}}
% The model distribution
\newcommand{\pmodel}{p_{\rm{model}}}
\newcommand{\Pmodel}{P_{\rm{model}}}
\newcommand{\ptildemodel}{\tilde{p}_{\rm{model}}}
% Stochastic autoencoder distributions
\newcommand{\pencode}{p_{\rm{encoder}}}
\newcommand{\pdecode}{p_{\rm{decoder}}}
\newcommand{\precons}{p_{\rm{reconstruct}}}

\newcommand{\laplace}{\mathrm{Laplace}} % Laplace distribution

\newcommand{\E}{\mathbb{E}}
\newcommand{\Ls}{\mathcal{L}}
\newcommand{\R}{\mathbb{R}}
\newcommand{\emp}{\tilde{p}}
\newcommand{\lr}{\alpha}
\newcommand{\reg}{\lambda}
\newcommand{\rect}{\mathrm{rectifier}}
\newcommand{\softmax}{\mathrm{softmax}}
\newcommand{\sigmoid}{\sigma}
\newcommand{\softplus}{\zeta}
\newcommand{\KL}{D_{\mathrm{KL}}}
\newcommand{\Var}{\mathrm{Var}}
\newcommand{\standarderror}{\mathrm{SE}}
\newcommand{\Cov}{\mathrm{Cov}}
% Wolfram Mathworld says $L^2$ is for function spaces and $\ell^2$ is for vectors
% But then they seem to use $L^2$ for vectors throughout the site, and so does
% wikipedia.
\newcommand{\normlzero}{L^0}
\newcommand{\normlone}{L^1}
\newcommand{\normltwo}{L^2}
\newcommand{\normlp}{L^p}
\newcommand{\normmax}{L^\infty}

\newcommand{\parents}{Pa} % See usage in notation.tex. Chosen to match Daphne's book.

\DeclareMathOperator*{\argmax}{arg\,max}
\DeclareMathOperator*{\argmin}{arg\,min}

\DeclareMathOperator{\sign}{sign}
\DeclareMathOperator{\Tr}{Tr}
\let\ab\allowbreak



\newcommand{\theHalgorithm}{\arabic{algorithm}}
\newcommand{\fix}{\marginpar{FIX}}
\newcommand{\new}{\marginpar{NEW}}
\newcommand{\fixme}[1]{\textcolor{red}{#1}}


\title{Traveling Waves Integrate Spatial Information Through Time}
 
\author{
    {\large \bf Mozes Jacobs\textsuperscript{1} \quad 
    Roberto C. Budzinski\textsuperscript{2} \quad 
    Lyle Muller\textsuperscript{2} \quad 
    Demba Ba\textsuperscript{1} \quad 
    T. Anderson Keller\textsuperscript{1}} \\
    \textsuperscript{1}The Kempner Institute for the Study of Natural and Artificial Intelligence, Harvard University \\
    \textsuperscript{2}Western University, Department of Mathematics, London, Ontario, Canada \\
}


\begin{document}

\maketitle

\section{Abstract}
{
\bf
Traveling waves of neural activity are widely observed in the brain, but their precise computational function remains unclear. One prominent hypothesis is that they enable the transfer and integration of spatial information across neural populations. However, few computational models have explored how traveling waves might be harnessed to perform such integrative processing. Drawing inspiration from the famous “\emph{Can one hear the shape of a drum?}” problem -- which highlights how normal modes of wave dynamics encode geometric information -- we investigate whether similar principles can be leveraged in artificial neural networks. Specifically, we introduce convolutional recurrent neural networks that learn to produce traveling waves in their hidden states in response to visual stimuli, enabling spatial integration. By then treating these wave-like activation sequences as visual representations themselves, we obtain a powerful representational space that outperforms local feed-forward networks on tasks requiring global spatial context. In particular, we observe that traveling waves effectively expand the receptive field of locally connected neurons, supporting long-range encoding and communication of information. We demonstrate that models equipped with this mechanism solve visual semantic segmentation tasks demanding global integration, significantly outperforming local feed-forward models and rivaling non-local U-Net models with fewer parameters. As a first step toward traveling-wave-based communication and visual representation in artificial networks, our findings suggest wave-dynamics may provide efficiency and training stability benefits, while simultaneously offering a new framework for connecting models to biological recordings of neural activity.
}
\begin{quote}
\small
\textbf{Keywords:} 
Traveling Waves; Oscillation; Information Integration
\end{quote}

\section{Introduction}

Video generation has garnered significant attention owing to its transformative potential across a wide range of applications, such media content creation~\citep{polyak2024movie}, advertising~\citep{zhang2024virbo,bacher2021advert}, video games~\citep{yang2024playable,valevski2024diffusion, oasis2024}, and world model simulators~\citep{ha2018world, videoworldsimulators2024, agarwal2025cosmos}. Benefiting from advanced generative algorithms~\citep{goodfellow2014generative, ho2020denoising, liu2023flow, lipman2023flow}, scalable model architectures~\citep{vaswani2017attention, peebles2023scalable}, vast amounts of internet-sourced data~\citep{chen2024panda, nan2024openvid, ju2024miradata}, and ongoing expansion of computing capabilities~\citep{nvidia2022h100, nvidia2023dgxgh200, nvidia2024h200nvl}, remarkable advancements have been achieved in the field of video generation~\citep{ho2022video, ho2022imagen, singer2023makeavideo, blattmann2023align, videoworldsimulators2024, kuaishou2024klingai, yang2024cogvideox, jin2024pyramidal, polyak2024movie, kong2024hunyuanvideo, ji2024prompt}.


In this work, we present \textbf{\ours}, a family of rectified flow~\citep{lipman2023flow, liu2023flow} transformer models designed for joint image and video generation, establishing a pathway toward industry-grade performance. This report centers on four key components: data curation, model architecture design, flow formulation, and training infrastructure optimization—each rigorously refined to meet the demands of high-quality, large-scale video generation.


\begin{figure}[ht]
    \centering
    \begin{subfigure}[b]{0.82\linewidth}
        \centering
        \includegraphics[width=\linewidth]{figures/t2i_1024.pdf}
        \caption{Text-to-Image Samples}\label{fig:main-demo-t2i}
    \end{subfigure}
    \vfill
    \begin{subfigure}[b]{0.82\linewidth}
        \centering
        \includegraphics[width=\linewidth]{figures/t2v_samples.pdf}
        \caption{Text-to-Video Samples}\label{fig:main-demo-t2v}
    \end{subfigure}
\caption{\textbf{Generated samples from \ours.} Key components are highlighted in \textcolor{red}{\textbf{RED}}.}\label{fig:main-demo}
\end{figure}


First, we present a comprehensive data processing pipeline designed to construct large-scale, high-quality image and video-text datasets. The pipeline integrates multiple advanced techniques, including video and image filtering based on aesthetic scores, OCR-driven content analysis, and subjective evaluations, to ensure exceptional visual and contextual quality. Furthermore, we employ multimodal large language models~(MLLMs)~\citep{yuan2025tarsier2} to generate dense and contextually aligned captions, which are subsequently refined using an additional large language model~(LLM)~\citep{yang2024qwen2} to enhance their accuracy, fluency, and descriptive richness. As a result, we have curated a robust training dataset comprising approximately 36M video-text pairs and 160M image-text pairs, which are proven sufficient for training industry-level generative models.

Secondly, we take a pioneering step by applying rectified flow formulation~\citep{lipman2023flow} for joint image and video generation, implemented through the \ours model family, which comprises Transformer architectures with 2B and 8B parameters. At its core, the \ours framework employs a 3D joint image-video variational autoencoder (VAE) to compress image and video inputs into a shared latent space, facilitating unified representation. This shared latent space is coupled with a full-attention~\citep{vaswani2017attention} mechanism, enabling seamless joint training of image and video. This architecture delivers high-quality, coherent outputs across both images and videos, establishing a unified framework for visual generation tasks.


Furthermore, to support the training of \ours at scale, we have developed a robust infrastructure tailored for large-scale model training. Our approach incorporates advanced parallelism strategies~\citep{jacobs2023deepspeed, pytorch_fsdp} to manage memory efficiently during long-context training. Additionally, we employ ByteCheckpoint~\citep{wan2024bytecheckpoint} for high-performance checkpointing and integrate fault-tolerant mechanisms from MegaScale~\citep{jiang2024megascale} to ensure stability and scalability across large GPU clusters. These optimizations enable \ours to handle the computational and data challenges of generative modeling with exceptional efficiency and reliability.


We evaluate \ours on both text-to-image and text-to-video benchmarks to highlight its competitive advantages. For text-to-image generation, \ours-T2I demonstrates strong performance across multiple benchmarks, including T2I-CompBench~\citep{huang2023t2i-compbench}, GenEval~\citep{ghosh2024geneval}, and DPG-Bench~\citep{hu2024ella_dbgbench}, excelling in both visual quality and text-image alignment. In text-to-video benchmarks, \ours-T2V achieves state-of-the-art performance on the UCF-101~\citep{ucf101} zero-shot generation task. Additionally, \ours-T2V attains an impressive score of \textbf{84.85} on VBench~\citep{huang2024vbench}, securing the top position on the leaderboard (as of 2025-01-25) and surpassing several leading commercial text-to-video models. Qualitative results, illustrated in \Cref{fig:main-demo}, further demonstrate the superior quality of the generated media samples. These findings underscore \ours's effectiveness in multi-modal generation and its potential as a high-performing solution for both research and commercial applications.
\section{Background} \label{section:LLM}

% \subsection{Large Language Model (LLM)}   

Figure~\ref{fig:LLaMA_model}(a) shows that a decoder-only LLM initially processes a user prompt in the “prefill” stage and subsequently generates tokens sequentially during the “decoding” stage.
Both stages contain an input embedding layer, multiple decoder transformer blocks, an output embedding layer, and a sampling layer.
Figure~\ref{fig:LLaMA_model}(b) demonstrates that the decoder transformer blocks consist of a self attention and a feed-forward network (FFN) layer, each paired with residual connection and normalization layers. 

% Differentiate between encoder/decoder, explain why operation intensity is low, explain the different parts of a transformer block. Discuss Table II here. 

% Explain the architecture with Llama2-70B.

% \begin{table}[thb]
% \renewcommand\arraystretch{1.05}
% \centering
% % \vspace{-5mm}
%     \caption{ML Model Parameter Size and Operational Intensity}
%     \vspace{-2mm}
%     \small
%     \label{tab:ML Model Parameter Size and Operational Intensity}    
%     \scalebox{0.95}{
%         \begin{tabular}{|c|c|c|c|c|}
%             \hline
%             & Llama2 & BLOOM & BERT & ResNet \\
%             Model & (70B) & (176B) & & 152 \\
%             \hline
%             Parameter Size (GB) & 140 & 352 & 0.17 & 0.16 \\
%             \hline
%             Op Intensity (Ops/Byte) & 1 & 1 & 282 & 346 \\
%             \hline
%           \end{tabular}
%     }
% \vspace{-3mm}
% \end{table}

% {\fontsize{8pt}{11pt}\selectfont 8pt font size test Memory Requirement}

\begin{figure}[t]
    \centering
    \includegraphics[width=8cm]{Figure/LLaMA_model_new_new.pdf}
    \caption{(a) Prefill stage encodes prompt tokens in parallel. Decoding stage generates output tokens sequentially.
    (b) LLM contains N$\times$ decoder transformer blocks. 
    (c) Llama2 model architecture.}
    \label{fig:LLaMA_model}
\end{figure}

Figure~\ref{fig:LLaMA_model}(c) demonstrates the Llama2~\cite{touvron2023llama} model architecture as a representative LLM.
% The self attention layer requires three GEMVs\footnote{GEMVs in multi-head attention~\cite{attention}, narrow GEMMs in grouped-query attention~\cite{gqa}.} to generate query, key and value vectors.
In the self-attention layer, query, key and value vectors are generated by multiplying input vector to corresponding weight matrices.
These matrices are segmented into multiple heads, representing different semantic dimensions.
The query and key vectors go though Rotary Positional Embedding (RoPE) to encode the relative positional information~\cite{rope-paper}.
Within each head, the generated key and value vectors are appended to their caches.
The query vector is multiplied by the key cache to produce a score vector.
After the Softmax operation, the score vector is multiplied by the value cache to yield the output vector.
The output vectors from all heads are concatenated and multiplied by output weight matrix, resulting in a vector that undergoes residual connection and Root Mean Square layer Normalization (RMSNorm)~\cite{rmsnorm-paper}.
The residual connection adds up the input and output vectors of a layer to avoid vanishing gradient~\cite{he2016deep}.
The FFN layer begins with two parallel fully connections, followed by a Sigmoid Linear Unit (SiLU), and ends with another fully connection.
\section{Study Design}
% robot: aliengo 
% We used the Unitree AlienGo quadruped robot. 
% See Appendix 1 in AlienGo Software Guide PDF
% Weight = 25kg, size (L,W,H) = (0.55, 0.35, 06) m when standing, (0.55, 0.35, 0.31) m when walking
% Handle is 0.4 m or 0.5 m. I'll need to check it to see which type it is.
We gathered input from primary stakeholders of the robot dog guide, divided into three subgroups: BVI individuals who have owned a dog guide, BVI individuals who were not dog guide owners, and sighted individuals with generally low degrees of familiarity with dog guides. While the main focus of this study was on the BVI participants, we elected to include survey responses from sighted participants given the importance of social acceptance of the robot by the general public, which could reflect upon the BVI users themselves and affect their interactions with the general population \cite{kayukawa2022perceive}. 

The need-finding processes consisted of two stages. During Stage 1, we conducted in-depth interviews with BVI participants, querying their experiences in using conventional assistive technologies and dog guides. During Stage 2, a large-scale survey was distributed to both BVI and sighted participants. 

This study was approved by the University’s Institutional Review Board (IRB), and all processes were conducted after obtaining the participants' consent.

\subsection{Stage 1: Interviews}
We recruited nine BVI participants (\textbf{Table}~\ref{tab:bvi-info}) for in-depth interviews, which lasted 45-90 minutes for current or former dog guide owners (DO) and 30-60 minutes for participants without dog guides (NDO). Group DO consisted of five participants, while Group NDO consisted of four participants.
% The interview participants were divided into two groups. Group DO (Dog guide Owner) consisted of five participants who were current or former dog guide owners and Group NDO (Non Dog guide Owner) consisted of three participants who were not dog guide owners. 
All participants were familiar with using white canes as a mobility aid. 

We recruited participants in both groups, DO and NDO, to gather data from those with substantial experience with dog guides, offering potentially more practical insights, and from those without prior experience, providing a perspective that may be less constrained and more open to novel approaches. 

We asked about the participants' overall impressions of a robot dog guide, expectations regarding its potential benefits and challenges compared to a conventional dog guide, their desired methods of giving commands and communicating with the robot dog guide, essential functionalities that the robot dog guide should offer, and their preferences for various aspects of the robot dog guide's form factors. 
For Group DO, we also included questions that asked about the participants' experiences with conventional dog guides. 

% We obtained permission to record the conversations for our records while simultaneously taking notes during the interviews. The interviews lasted 30-60 minutes for NDO participants and 45-90 minutes for DO participants. 

\subsection{Stage 2: Large-Scale Surveys} 
After gathering sufficient initial results from the interviews, we created an online survey for distributing to a larger pool of participants. The survey platform used was Qualtrics. 

\subsubsection{Survey Participants}
The survey had 100 participants divided into two primary groups. Group BVI consisted of 42 blind or visually impaired participants, and Group ST consisted of 58 sighted participants. \textbf{Table}~\ref{tab:survey-demographics} shows the demographic information of the survey participants. 

\subsubsection{Question Differentiation} 
Based on their responses to initial qualifying questions, survey participants were sorted into three subgroups: DO, NDO, and ST. Each participant was assigned one of three different versions of the survey. The surveys for BVI participants mirrored the interview categories (overall impressions, communication methods, functionalities, and form factors), but with a more quantitative approach rather than the open-ended questions used in interviews. The DO version included additional questions pertaining to their prior experience with dog guides. The ST version revolved around the participants' prior interactions with and feelings toward dog guides and dogs in general, their thoughts on a robot dog guide, and broad opinions on the aesthetic component of the robot's design. 

\section{Experiments}

\subsection{Setups}
\subsubsection{Implementation Details}
We apply our FDS method to two types of 3DGS: 
the original 3DGS, and 2DGS~\citep{huang20242d}. 
%
The number of iterations in our optimization 
process is 35,000.
We follow the default training configuration 
and apply our FDS method after 15,000 iterations,
then we add normal consistency loss for both
3DGS and 2DGS after 25000 iterations.
%
The weight for FDS, $\lambda_{fds}$, is set to 0.015,
the $\sigma$ is set to 23,
and the weight for normal consistency is set to 0.15
for all experiments. 
We removed the depth distortion loss in 2DGS 
because we found that it degrades its results in indoor scenes.
%
The Gaussian point cloud is initialized using Colmap
for all datasets.
%
%
We tested the impact of 
using Sea Raft~\citep{wang2025sea} and 
Raft\citep{teed2020raft} on FDS performance.
%
Due to the blurriness of the ScanNet dataset, 
additional prior constraints are required.
Thus, we incorporate normal prior supervision 
on the rendered normals 
in ScanNet (V2) dataset by default.
The normal prior is predicted by the Stable Normal 
model~\citep{ye2024stablenormal}
across all types of 3DGS.
%
The entire framework is implemented in 
PyTorch~\citep{paszke2019pytorch}, 
and all experiments are conducted on 
a single NVIDIA 4090D GPU.

\begin{figure}[t] \centering
    \makebox[0.16\textwidth]{\scriptsize Input}
    \makebox[0.16\textwidth]{\scriptsize 3DGS}
    \makebox[0.16\textwidth]{\scriptsize 2DGS}
    \makebox[0.16\textwidth]{\scriptsize 3DGS + FDS}
    \makebox[0.16\textwidth]{\scriptsize 2DGS + FDS}
    \makebox[0.16\textwidth]{\scriptsize GT (Depth)}

    \includegraphics[width=0.16\textwidth]{figure/fig3_img/compare3/gt_rgb/frame_00522.jpg}
    \includegraphics[width=0.16\textwidth]{figure/fig3_img/compare3/3DGS/frame_00522.jpg}
    \includegraphics[width=0.16\textwidth]{figure/fig3_img/compare3/2DGS/frame_00522.jpg}
    \includegraphics[width=0.16\textwidth]{figure/fig3_img/compare3/3DGS+FDS/frame_00522.jpg}
    \includegraphics[width=0.16\textwidth]{figure/fig3_img/compare3/2DGS+FDS/frame_00522.jpg}
    \includegraphics[width=0.16\textwidth]{figure/fig3_img/compare3/gt_depth/frame_00522.jpg} \\

    % \includegraphics[width=0.16\textwidth]{figure/fig3_img/compare1/gt_rgb/frame_00137.jpg}
    % \includegraphics[width=0.16\textwidth]{figure/fig3_img/compare1/3DGS/frame_00137.jpg}
    % \includegraphics[width=0.16\textwidth]{figure/fig3_img/compare1/2DGS/frame_00137.jpg}
    % \includegraphics[width=0.16\textwidth]{figure/fig3_img/compare1/3DGS+FDS/frame_00137.jpg}
    % \includegraphics[width=0.16\textwidth]{figure/fig3_img/compare1/2DGS+FDS/frame_00137.jpg}
    % \includegraphics[width=0.16\textwidth]{figure/fig3_img/compare1/gt_depth/frame_00137.jpg} \\

     \includegraphics[width=0.16\textwidth]{figure/fig3_img/compare2/gt_rgb/frame_00262.jpg}
    \includegraphics[width=0.16\textwidth]{figure/fig3_img/compare2/3DGS/frame_00262.jpg}
    \includegraphics[width=0.16\textwidth]{figure/fig3_img/compare2/2DGS/frame_00262.jpg}
    \includegraphics[width=0.16\textwidth]{figure/fig3_img/compare2/3DGS+FDS/frame_00262.jpg}
    \includegraphics[width=0.16\textwidth]{figure/fig3_img/compare2/2DGS+FDS/frame_00262.jpg}
    \includegraphics[width=0.16\textwidth]{figure/fig3_img/compare2/gt_depth/frame_00262.jpg} \\

    \includegraphics[width=0.16\textwidth]{figure/fig3_img/compare4/gt_rgb/frame00000.png}
    \includegraphics[width=0.16\textwidth]{figure/fig3_img/compare4/3DGS/frame00000.png}
    \includegraphics[width=0.16\textwidth]{figure/fig3_img/compare4/2DGS/frame00000.png}
    \includegraphics[width=0.16\textwidth]{figure/fig3_img/compare4/3DGS+FDS/frame00000.png}
    \includegraphics[width=0.16\textwidth]{figure/fig3_img/compare4/2DGS+FDS/frame00000.png}
    \includegraphics[width=0.16\textwidth]{figure/fig3_img/compare4/gt_depth/frame00000.png} \\

    \includegraphics[width=0.16\textwidth]{figure/fig3_img/compare5/gt_rgb/frame00080.png}
    \includegraphics[width=0.16\textwidth]{figure/fig3_img/compare5/3DGS/frame00080.png}
    \includegraphics[width=0.16\textwidth]{figure/fig3_img/compare5/2DGS/frame00080.png}
    \includegraphics[width=0.16\textwidth]{figure/fig3_img/compare5/3DGS+FDS/frame00080.png}
    \includegraphics[width=0.16\textwidth]{figure/fig3_img/compare5/2DGS+FDS/frame00080.png}
    \includegraphics[width=0.16\textwidth]{figure/fig3_img/compare5/gt_depth/frame00080.png} \\



    \caption{\textbf{Comparison of depth reconstruction on Mushroom and ScanNet datasets.} The original
    3DGS or 2DGS model equipped with FDS can remove unwanted floaters and reconstruct
    geometry more preciously.}
    \label{fig:compare}
\end{figure}


\subsubsection{Datasets and Metrics}

We evaluate our method for 3D reconstruction 
and novel view synthesis tasks on
\textbf{Mushroom}~\citep{ren2024mushroom},
\textbf{ScanNet (v2)}~\citep{dai2017scannet}, and 
\textbf{Replica}~\citep{replica19arxiv}
datasets,
which feature challenging indoor scenes with both 
sparse and dense image sampling.
%
The Mushroom dataset is an indoor dataset 
with sparse image sampling and two distinct 
camera trajectories. 
%
We train our model on the training split of 
the long capture sequence and evaluate 
novel view synthesis on the test split 
of the long capture sequences.
%
Five scenes are selected to evaluate our FDS, 
including "coffee room", "honka", "kokko", 
"sauna", and "vr room". 
%
ScanNet(V2)~\citep{dai2017scannet}  consists of 1,613 indoor scenes
with annotated camera poses and depth maps. 
%
We select 5 scenes from the ScanNet (V2) dataset, 
uniformly sampling one-tenth of the views,
following the approach in ~\citep{guo2022manhattan}.
To further improve the geometry rendering quality of 3DGS, 
%
Replica~\citep{replica19arxiv} contains small-scale 
real-world indoor scans. 
We evaluate our FDS on five scenes from 
Replica: office0, office1, office2, office3 and office4,
selecting one-tenth of the views for training.
%
The results for Replica are provided in the 
supplementary materials.
To evaluate the rendering quality and geometry 
of 3DGS, we report PSNR, SSIM, and LPIPS for 
rendering quality, along with Absolute Relative Distance 
(Abs Rel) as a depth quality metrics.
%
Additionally, for mesh evaluation, 
we use metrics including Accuracy, Completion, 
Chamfer-L1 distance, Normal Consistency, 
and F-scores.




\subsection{Results}
\subsubsection{Depth rendering and novel view synthesis}
The comparison results on Mushroom and 
ScanNet are presented in \tabref{tab:mushroom} 
and \tabref{tab:scannet}, respectively. 
%
Due to the sparsity of sampling 
in the Mushroom dataset,
challenges are posed for both GOF~\citep{yu2024gaussian} 
and PGSR~\citep{chen2024pgsr}, 
leading to their relative poor performance 
on the Mushroom dataset.
%
Our approach achieves the best performance 
with the FDS method applied during the training process.
The FDS significantly enhances the 
geometric quality of 3DGS on the Mushroom dataset, 
improving the "abs rel" metric by more than 50\%.
%
We found that Sea Raft~\citep{wang2025sea}
outperforms Raft~\citep{teed2020raft} on FDS, 
indicating that a better optical flow model 
can lead to more significant improvements.
%
Additionally, the render quality of RGB 
images shows a slight improvement, 
by 0.58 in 3DGS and 0.50 in 2DGS, 
benefiting from the incorporation of cross-view consistency in FDS. 
%
On the Mushroom
dataset, adding the FDS loss increases 
the training time by half an hour, which maintains the same
level as baseline.
%
Similarly, our method shows a notable improvement on the ScanNet dataset as well using Sea Raft~\citep{wang2025sea} Model. The "abs rel" metric in 2DGS is improved nearly 50\%. This demonstrates the robustness and effectiveness of the FDS method across different datasets.
%


% \begin{wraptable}{r}{0.6\linewidth} \centering
% \caption{\textbf{Ablation study on geometry priors.}} 
%         \label{tab:analysis_prior}
%         \resizebox{\textwidth}{!}{
\begin{tabular}{c| c c c c c | c c c c}

    \hline
     Method &  Acc$\downarrow$ & Comp $\downarrow$ & C-L1 $\downarrow$ & NC $\uparrow$ & F-Score $\uparrow$ &  Abs Rel $\downarrow$ &  PSNR $\uparrow$  & SSIM  $\uparrow$ & LPIPS $\downarrow$ \\ \hline
    2DGS&   0.1078&  0.0850&  0.0964&  0.7835&  0.5170&  0.1002&  23.56&  0.8166& 0.2730\\
    2DGS+Depth&   0.0862&  0.0702&  0.0782&  0.8153&  0.5965&  0.0672&  23.92&  0.8227& 0.2619 \\
    2DGS+MVDepth&   0.2065&  0.0917&  0.1491&  0.7832&  0.3178&  0.0792&  23.74&  0.8193& 0.2692 \\
    2DGS+Normal&   0.0939&  0.0637&  0.0788&  \textbf{0.8359}&  0.5782&  0.0768&  23.78&  0.8197& 0.2676 \\
    2DGS+FDS &  \textbf{0.0615} & \textbf{ 0.0534}& \textbf{0.0574}& 0.8151& \textbf{0.6974}&  \textbf{0.0561}&  \textbf{24.06}&  \textbf{0.8271}&\textbf{0.2610} \\ \hline
    2DGS+Depth+FDS &  0.0561 &  0.0519& 0.0540& 0.8295& 0.7282&  0.0454&  \textbf{24.22}& \textbf{0.8291}&\textbf{0.2570} \\
    2DGS+Normal+FDS &  \textbf{0.0529} & \textbf{ 0.0450}& \textbf{0.0490}& \textbf{0.8477}& \textbf{0.7430}&  \textbf{0.0443}&  24.10&  0.8283& 0.2590 \\
    2DGS+Depth+Normal &  0.0695 & 0.0513& 0.0604& 0.8540&0.6723&  0.0523&  24.09&  0.8264&0.2575\\ \hline
    2DGS+Depth+Normal+FDS &  \textbf{0.0506} & \textbf{0.0423}& \textbf{0.0464}& \textbf{0.8598}&\textbf{0.7613}&  \textbf{0.0403}&  \textbf{24.22}& 
    \textbf{0.8300}&\textbf{0.0403}\\
    
\bottomrule
\end{tabular}
}
% \end{wraptable}



The qualitative comparisons on the Mushroom and ScanNet dataset 
are illustrated in \figref{fig:compare}. 
%
%
As seen in the first row of \figref{fig:compare}, 
both the original 3DGS and 2DGS suffer from overfitting, 
leading to corrupted geometry generation. 
%
Our FDS effectively mitigates this issue by 
supervising the matching relationship between 
the input and sampled views, 
helping to recover the geometry.
%
FDS also improves the refinement of geometric details, 
as shown in other rows. 
By incorporating the matching prior through FDS, 
the quality of the rendered depth is significantly improved.
%

\begin{table}[t] \centering
\begin{minipage}[t]{0.96\linewidth}
        \captionof{table}{\textbf{3D Reconstruction 
        and novel view synthesis results on Mushroom dataset. * 
        Represents that FDS uses the Raft model.
        }}
        \label{tab:mushroom}
        \resizebox{\textwidth}{!}{
\begin{tabular}{c| c c c c c | c c c c c}
    \hline
     Method &  Acc$\downarrow$ & Comp $\downarrow$ & C-L1 $\downarrow$ & NC $\uparrow$ & F-Score $\uparrow$ &  Abs Rel $\downarrow$ &  PSNR $\uparrow$  & SSIM  $\uparrow$ & LPIPS $\downarrow$ & Time  $\downarrow$ \\ \hline

    % DN-splatter &   &  &  &  &  &  &  &  & \\
    GOF &  0.1812 & 0.1093 & 0.1453 & 0.6292 & 0.3665 & 0.2380  & 21.37  &  0.7762  & 0.3132  & $\approx$1.4h\\ 
    PGSR &  0.0971 & 0.1420 & 0.1196 & 0.7193 & 0.5105 & 0.1723  & 22.13  & 0.7773  & 0.2918  & $\approx$1.2h \\ \hline
    3DGS &   0.1167 &  0.1033&  0.1100&  0.7954&  0.3739&  0.1214&  24.18&  0.8392& 0.2511 &$\approx$0.8h \\
    3DGS + FDS$^*$ & 0.0569  & 0.0676 & 0.0623 & 0.8105 & 0.6573 & 0.0603 & 24.72  & 0.8489 & 0.2379 &$\approx$1.3h \\
    3DGS + FDS & \textbf{0.0527}  & \textbf{0.0565} & \textbf{0.0546} & \textbf{0.8178} & \textbf{0.6958} & \textbf{0.0568} & \textbf{24.76}  & \textbf{0.8486} & \textbf{0.2381} &$\approx$1.3h \\ \hline
    2DGS&   0.1078&  0.0850&  0.0964&  0.7835&  0.5170&  0.1002&  23.56&  0.8166& 0.2730 &$\approx$0.8h\\
    2DGS + FDS$^*$ &  0.0689 &  0.0646& 0.0667& 0.8042& 0.6582& 0.0589& 23.98&  0.8255&0.2621 &$\approx$1.3h\\
    2DGS + FDS &  \textbf{0.0615} & \textbf{ 0.0534}& \textbf{0.0574}& \textbf{0.8151}& \textbf{0.6974}&  \textbf{0.0561}&  \textbf{24.06}&  \textbf{0.8271}&\textbf{0.2610} &$\approx$1.3h \\ \hline
\end{tabular}
}
\end{minipage}\hfill
\end{table}

\begin{table}[t] \centering
\begin{minipage}[t]{0.96\linewidth}
        \captionof{table}{\textbf{3D Reconstruction 
        and novel view synthesis results on ScanNet dataset.}}
        \label{tab:scannet}
        \resizebox{\textwidth}{!}{
\begin{tabular}{c| c c c c c | c c c c }
    \hline
     Method &  Acc $\downarrow$ & Comp $\downarrow$ & C-L1 $\downarrow$ & NC $\uparrow$ & F-Score $\uparrow$ &  Abs Rel $\downarrow$ &  PSNR $\uparrow$  & SSIM  $\uparrow$ & LPIPS $\downarrow$ \\ \hline
    GOF & 1.8671  & 0.0805 & 0.9738 & 0.5622 & 0.2526 & 0.1597  & 21.55  & 0.7575  & 0.3881 \\
    PGSR &  0.2928 & 0.5103 & 0.4015 & 0.5567 & 0.1926 & 0.1661  & 21.71 & 0.7699  & 0.3899 \\ \hline

    3DGS &  0.4867 & 0.1211 & 0.3039 & 0.7342& 0.3059 & 0.1227 & 22.19& 0.7837 & 0.3907\\
    3DGS + FDS &  \textbf{0.2458} & \textbf{0.0787} & \textbf{0.1622} & \textbf{0.7831} & 
    \textbf{0.4482} & \textbf{0.0573} & \textbf{22.83} & \textbf{0.7911} & \textbf{0.3826} \\ \hline
    2DGS &  0.2658 & 0.0845 & 0.1752 & 0.7504& 0.4464 & 0.0831 & 22.59& 0.7881 & 0.3854\\
    2DGS + FDS &  \textbf{0.1457} & \textbf{0.0679} & \textbf{0.1068} & \textbf{0.7883} & 
    \textbf{0.5459} & \textbf{0.0432} & \textbf{22.91} & \textbf{0.7928} & \textbf{0.3800} \\ \hline
\end{tabular}
}
\end{minipage}\hfill
\end{table}


\begin{table}[t] \centering
\begin{minipage}[t]{0.96\linewidth}
        \captionof{table}{\textbf{Ablation study on geometry priors.}}
        \label{tab:analysis_prior}
        \resizebox{\textwidth}{!}{
\begin{tabular}{c| c c c c c | c c c c}

    \hline
     Method &  Acc$\downarrow$ & Comp $\downarrow$ & C-L1 $\downarrow$ & NC $\uparrow$ & F-Score $\uparrow$ &  Abs Rel $\downarrow$ &  PSNR $\uparrow$  & SSIM  $\uparrow$ & LPIPS $\downarrow$ \\ \hline
    2DGS&   0.1078&  0.0850&  0.0964&  0.7835&  0.5170&  0.1002&  23.56&  0.8166& 0.2730\\
    2DGS+Depth&   0.0862&  0.0702&  0.0782&  0.8153&  0.5965&  0.0672&  23.92&  0.8227& 0.2619 \\
    2DGS+MVDepth&   0.2065&  0.0917&  0.1491&  0.7832&  0.3178&  0.0792&  23.74&  0.8193& 0.2692 \\
    2DGS+Normal&   0.0939&  0.0637&  0.0788&  \textbf{0.8359}&  0.5782&  0.0768&  23.78&  0.8197& 0.2676 \\
    2DGS+FDS &  \textbf{0.0615} & \textbf{ 0.0534}& \textbf{0.0574}& 0.8151& \textbf{0.6974}&  \textbf{0.0561}&  \textbf{24.06}&  \textbf{0.8271}&\textbf{0.2610} \\ \hline
    2DGS+Depth+FDS &  0.0561 &  0.0519& 0.0540& 0.8295& 0.7282&  0.0454&  \textbf{24.22}& \textbf{0.8291}&\textbf{0.2570} \\
    2DGS+Normal+FDS &  \textbf{0.0529} & \textbf{ 0.0450}& \textbf{0.0490}& \textbf{0.8477}& \textbf{0.7430}&  \textbf{0.0443}&  24.10&  0.8283& 0.2590 \\
    2DGS+Depth+Normal &  0.0695 & 0.0513& 0.0604& 0.8540&0.6723&  0.0523&  24.09&  0.8264&0.2575\\ \hline
    2DGS+Depth+Normal+FDS &  \textbf{0.0506} & \textbf{0.0423}& \textbf{0.0464}& \textbf{0.8598}&\textbf{0.7613}&  \textbf{0.0403}&  \textbf{24.22}& 
    \textbf{0.8300}&\textbf{0.0403}\\
    
\bottomrule
\end{tabular}
}
\end{minipage}\hfill
\end{table}




\subsubsection{Mesh extraction}
To further demonstrate the improvement in geometry quality, 
we applied methods used in ~\citep{turkulainen2024dnsplatter} 
to extract meshes from the input views of optimized 3DGS. 
The comparison results are presented  
in \tabref{tab:mushroom}. 
With the integration of FDS, the mesh quality is significantly enhanced compared to the baseline, featuring fewer floaters and more well-defined shapes.
 %
% Following the incorporation of FDS, the reconstruction 
% results exhibit fewer floaters and more well-defined 
% shapes in the meshes. 
% Visualized comparisons
% are provided in the supplementary material.

% \begin{figure}[t] \centering
%     \makebox[0.19\textwidth]{\scriptsize GT}
%     \makebox[0.19\textwidth]{\scriptsize 3DGS}
%     \makebox[0.19\textwidth]{\scriptsize 3DGS+FDS}
%     \makebox[0.19\textwidth]{\scriptsize 2DGS}
%     \makebox[0.19\textwidth]{\scriptsize 2DGS+FDS} \\

%     \includegraphics[width=0.19\textwidth]{figure/fig4_img/compare1/gt02.png}
%     \includegraphics[width=0.19\textwidth]{figure/fig4_img/compare1/baseline06.png}
%     \includegraphics[width=0.19\textwidth]{figure/fig4_img/compare1/baseline_fds05.png}
%     \includegraphics[width=0.19\textwidth]{figure/fig4_img/compare1/2dgs04.png}
%     \includegraphics[width=0.19\textwidth]{figure/fig4_img/compare1/2dgs_fds03.png} \\

%     \includegraphics[width=0.19\textwidth]{figure/fig4_img/compare2/gt00.png}
%     \includegraphics[width=0.19\textwidth]{figure/fig4_img/compare2/baseline02.png}
%     \includegraphics[width=0.19\textwidth]{figure/fig4_img/compare2/baseline_fds01.png}
%     \includegraphics[width=0.19\textwidth]{figure/fig4_img/compare2/2dgs04.png}
%     \includegraphics[width=0.19\textwidth]{figure/fig4_img/compare2/2dgs_fds03.png} \\
      
%     \includegraphics[width=0.19\textwidth]{figure/fig4_img/compare3/gt05.png}
%     \includegraphics[width=0.19\textwidth]{figure/fig4_img/compare3/3dgs03.png}
%     \includegraphics[width=0.19\textwidth]{figure/fig4_img/compare3/3dgs_fds04.png}
%     \includegraphics[width=0.19\textwidth]{figure/fig4_img/compare3/2dgs02.png}
%     \includegraphics[width=0.19\textwidth]{figure/fig4_img/compare3/2dgs_fds01.png} \\

%     \caption{\textbf{Qualitative comparison of extracted mesh 
%     on Mushroom and ScanNet datasets.}}
%     \label{fig:mesh}
% \end{figure}












\subsection{Ablation study}


\textbf{Ablation study on geometry priors:} 
To highlight the advantage of incorporating matching priors, 
we incorporated various types of priors generated by different 
models into 2DGS. These include a monocular depth estimation
model (Depth Anything v2)~\citep{yang2024depth}, a two-view depth estimation 
model (Unimatch)~\citep{xu2023unifying}, 
and a monocular normal estimation model (DSINE)~\citep{bae2024rethinking}.
We adapt the scale and shift-invariant loss in Midas~\citep{birkl2023midas} for
monocular depth supervision and L1 loss for two-view depth supervison.
%
We use Sea Raft~\citep{wang2025sea} as our default optical flow model.
%
The comparison results on Mushroom dataset 
are shown in ~\tabref{tab:analysis_prior}.
We observe that the normal prior provides accurate shape information, 
enhancing the geometric quality of the radiance field. 
%
% In contrast, the monocular depth prior slightly increases 
% the 'Abs Rel' due to its ambiguous scale and inaccurate depth ordering.
% Moreover, the performance of monocular depth estimation 
% in the sauna scene is particularly poor, 
% primarily due to the presence of numerous reflective 
% surfaces and textureless walls, which limits the accuracy of monocular depth estimation.
%
The multi-view depth prior, hindered by the limited feature overlap 
between input views, fails to offer reliable geometric 
information. We test average "Abs Rel" of multi-view depth prior
, and the result is 0.19, which performs worse than the "Abs Rel" results 
rendered by original 2DGS.
From the results, it can be seen that depth order information provided by monocular depth improves
reconstruction accuracy. Meanwhile, our FDS achieves the best performance among all the priors, 
and by integrating all
three components, we obtained the optimal results.
%
%
\begin{figure}[t] \centering
    \makebox[0.16\textwidth]{\scriptsize RF (16000 iters)}
    \makebox[0.16\textwidth]{\scriptsize RF* (20000 iters)}
    \makebox[0.16\textwidth]{\scriptsize RF (20000 iters)  }
    \makebox[0.16\textwidth]{\scriptsize PF (16000 iters)}
    \makebox[0.16\textwidth]{\scriptsize PF (20000 iters)}


    % \includegraphics[width=0.16\textwidth]{figure/fig5_img/compare1/16000.png}
    % \includegraphics[width=0.16\textwidth]{figure/fig5_img/compare1/20000_wo_flow_loss.png}
    % \includegraphics[width=0.16\textwidth]{figure/fig5_img/compare1/20000.png}
    % \includegraphics[width=0.16\textwidth]{figure/fig5_img/compare1/16000_prior.png}
    % \includegraphics[width=0.16\textwidth]{figure/fig5_img/compare1/20000_prior.png}\\

    % \includegraphics[width=0.16\textwidth]{figure/fig5_img/compare2/16000.png}
    % \includegraphics[width=0.16\textwidth]{figure/fig5_img/compare2/20000_wo_flow_loss.png}
    % \includegraphics[width=0.16\textwidth]{figure/fig5_img/compare2/20000.png}
    % \includegraphics[width=0.16\textwidth]{figure/fig5_img/compare2/16000_prior.png}
    % \includegraphics[width=0.16\textwidth]{figure/fig5_img/compare2/20000_prior.png}\\

    \includegraphics[width=0.16\textwidth]{figure/fig5_img/compare3/16000.png}
    \includegraphics[width=0.16\textwidth]{figure/fig5_img/compare3/20000_wo_flow_loss.png}
    \includegraphics[width=0.16\textwidth]{figure/fig5_img/compare3/20000.png}
    \includegraphics[width=0.16\textwidth]{figure/fig5_img/compare3/16000_prior.png}
    \includegraphics[width=0.16\textwidth]{figure/fig5_img/compare3/20000_prior.png}\\
    
    \includegraphics[width=0.16\textwidth]{figure/fig5_img/compare4/16000.png}
    \includegraphics[width=0.16\textwidth]{figure/fig5_img/compare4/20000_wo_flow_loss.png}
    \includegraphics[width=0.16\textwidth]{figure/fig5_img/compare4/20000.png}
    \includegraphics[width=0.16\textwidth]{figure/fig5_img/compare4/16000_prior.png}
    \includegraphics[width=0.16\textwidth]{figure/fig5_img/compare4/20000_prior.png}\\

    \includegraphics[width=0.30\textwidth]{figure/fig5_img/bar.png}

    \caption{\textbf{The error map of Radiance Flow and Prior Flow.} RF: Radiance Flow, PF: Prior Flow, * means that there is no FDS loss supervision during optimization.}
    \label{fig:error_map}
\end{figure}




\textbf{Ablation study on FDS: }
In this section, we present the design of our FDS 
method through an ablation study on the 
Mushroom dataset to validate its effectiveness.
%
The optional configurations of FDS are outlined in ~\tabref{tab:ablation_fds}.
Our base model is the 2DGS equipped with FDS,
and its results are shown 
in the first row. The goal of this analysis 
is to evaluate the impact 
of various strategies on FDS sampling and loss design.
%
We observe that when we 
replace $I_i$ in \eqref{equ:mflow} with $C_i$, 
as shown in the second row, the geometric quality 
of 2DGS deteriorates. Using $I_i$ instead of $C_i$ 
help us to remove the floaters in $\bm{C^s}$, which are also 
remained in $\bm{C^i}$.
We also experiment with modifying the FDS loss. For example, 
in the third row, we use the neighbor 
input view as the sampling view, and replace the 
render result of neighbor view with ground truth image of its input view.
%
Due to the significant movement between images, the Prior Flow fails to accurately 
match the pixel between them, leading to a further degradation in geometric quality.
%
Finally, we attempt to fix the sampling view 
and found that this severely damaged the geometric quality, 
indicating that random sampling is essential for the stability 
of the mean error in the Prior flow.



\begin{table}[t] \centering

\begin{minipage}[t]{1.0\linewidth}
        \captionof{table}{\textbf{Ablation study on FDS strategies.}}
        \label{tab:ablation_fds}
        \resizebox{\textwidth}{!}{
\begin{tabular}{c|c|c|c|c|c|c|c}
    \hline
    \multicolumn{2}{c|}{$\mathcal{M}_{\theta}(X, \bm{C^s})$} & \multicolumn{3}{c|}{Loss} & \multicolumn{3}{c}{Metric}  \\
    \hline
    $X=C^i$ & $X=I^i$  & Input view & Sampled view     & Fixed Sampled view        & Abs Rel $\downarrow$ & F-score $\uparrow$ & NC $\uparrow$ \\
    \hline
    & \ding{51} &     &\ding{51}    &    &    \textbf{0.0561}        &  \textbf{0.6974}         & \textbf{0.8151}\\
    \hline
     \ding{51} &           &     &\ding{51}    &    &    0.0839        &  0.6242         &0.8030\\
     &  \ding{51} &   \ding{51}  &    &    &    0.0877       & 0.6091        & 0.7614 \\
      &  \ding{51} &    &    & \ding{51}    &    0.0724           & 0.6312          & 0.8015 \\
\bottomrule
\end{tabular}
}
\end{minipage}
\end{table}




\begin{figure}[htbp] \centering
    \makebox[0.22\textwidth]{}
    \makebox[0.22\textwidth]{}
    \makebox[0.22\textwidth]{}
    \makebox[0.22\textwidth]{}
    \\

    \includegraphics[width=0.22\textwidth]{figure/fig6_img/l1/rgb/frame00096.png}
    \includegraphics[width=0.22\textwidth]{figure/fig6_img/l1/render_rgb/frame00096.png}
    \includegraphics[width=0.22\textwidth]{figure/fig6_img/l1/render_depth/frame00096.png}
    \includegraphics[width=0.22\textwidth]{figure/fig6_img/l1/depth/frame00096.png}

    % \includegraphics[width=0.22\textwidth]{figure/fig6_img/l2/rgb/frame00112.png}
    % \includegraphics[width=0.22\textwidth]{figure/fig6_img/l2/render_rgb/frame00112.png}
    % \includegraphics[width=0.22\textwidth]{figure/fig6_img/l2/render_depth/frame00112.png}
    % \includegraphics[width=0.22\textwidth]{figure/fig6_img/l2/depth/frame00112.png}

    \caption{\textbf{Limitation of FDS.} }
    \label{fig:limitation}
\end{figure}


% \begin{figure}[t] \centering
%     \makebox[0.48\textwidth]{}
%     \makebox[0.48\textwidth]{}
%     \\
%     \includegraphics[width=0.48\textwidth]{figure/loss_Ignatius.pdf}
%     \includegraphics[width=0.48\textwidth]{figure/loss_family.pdf}
%     \caption{\textbf{Comparison the photometric error of Radiance Flow and Prior Flow:} 
%     We add FDS method after 2k iteration during training.
%     The results show
%     that:  1) The Prior Flow is more precise and 
%     robust than Radiance Flow during the radiance 
%     optimization; 2) After adding the FDS loss 
%     which utilize Prior 
%     flow to supervise the Radiance Flow at 2k iterations, 
%     both flow are more accurate, which lead to
%     a mutually reinforcing effects.(TODO fix it)} 
%     \label{fig:flowcompare}
% \end{figure}






\textbf{Interpretive Experiments: }
To demonstrate the mutual refinement of two flows in our FDS, 
For each view, we sample the unobserved 
views multiple times to compute the mean error 
of both Radiance Flow and Prior Flow. 
We use Raft~\citep{teed2020raft} as our default optical flow model
for visualization.
The ground truth flow is calculated based on 
~\eref{equ:flow_pose} and ~\eref{equ:flow} 
utilizing ground truth depth in dataset.
We introduce our FDS loss after 16000 iterations during 
optimization of 2DGS.
The error maps are shown in ~\figref{fig:error_map}.
Our analysis reveals that Radiance Flow tends to 
exhibit significant geometric errors, 
whereas Prior Flow can more accurately estimate the geometry,
effectively disregarding errors introduced by floating Gaussian points. 

%





\subsection{Limitation and further work}

Firstly, our FDS faces challenges in scenes with 
significant lighting variations between different 
views, as shown in the lamp of first row in ~\figref{fig:limitation}. 
%
Incorporating exposure compensation into FDS could help address this issue. 
%
 Additionally, our method struggles with 
 reflective surfaces and motion blur,
 leading to incorrect matching. 
 %
 In the future, we plan to explore the potential 
 of FDS in monocular video reconstruction tasks, 
 using only a single input image at each time step.
 


\section{Conclusions}
In this paper, we propose Flow Distillation Sampling (FDS), which
leverages the matching prior between input views and 
sampled unobserved views from the pretrained optical flow model, to improve the geometry quality
of Gaussian radiance field. 
Our method can be applied to different approaches (3DGS and 2DGS) to enhance the geometric rendering quality of the corresponding neural radiance fields.
We apply our method to the 3DGS-based framework, 
and the geometry is enhanced on the Mushroom, ScanNet, and Replica datasets.

\section*{Acknowledgements} This work was supported by 
National Key R\&D Program of China (2023YFB3209702), 
the National Natural Science Foundation of 
China (62441204, 62472213), and Gusu 
Innovation \& Entrepreneurship Leading Talents Program (ZXL2024361)
% !TEX root = ../main.tex

\section{Related work}
\label{sec:related_work}

\subsection{Visual unsupervised anomaly localization}

% In recent years the creation of the MVTec AD benchmark~\cite{mvtec} has given impetus to the development of new methods for visual unsupervised anomaly detection and localization. We review several main approaches which have representatives among top-5 methods on the localization track of the MVTec AD leaderboard
% The MVTec AD benchmark~\cite{mvtec}, developed in recent years, has been instrumental in propelling research towards new methods in visual unsupervised anomaly detection and localization.
In this section, we review several key approaches, each represented among the top five methods on the localization track of the MVTec AD benchmark~\cite{mvtec}, developed to stir progress in visual unsupervised anomaly detection and localization. 
% \footnote{\url{https://paperswithcode.com/sota/anomaly-detection-on-mvtec-ad}}.
% \paragraph{Synthetic anomalies} In unsupervised setting, real anomalies are either not present or not labeled in the training images. Some methods~\cite{memseg,mood_top1}, however, propose synthetic procedures that corrupt random regions in the images and train a segmentation model to predict the corrupted regions' masks.

\paragraph{Synthetic anomalies.} In unsupervised settings, real anomalies are typically absent or unlabeled in training images. To simulate anomalies, researchers synthetically corrupt random regions by replacing them with noise, random patterns from a special set~\cite{memseg}, or parts of other training images~\cite{mood_top1}. A segmentation model is trained to predict binary masks of corrupted regions, providing well-calibrated anomaly scores for individual pixels. While straightforward to train, these models may overfit to synthetic anomalies and struggle with real ones.
% . Unlabeled real anomalies in training images cannot be included in the binary masks, leading the model to predict zero scores for these regions and resulting in false negatives.

% One limitation of this approach is that the models may overfit to synthetic anomalies and generalize poorly to real anomalies. Another limitation is that training images may contain real anomalies which are unlabeled and cannot be included in the training binary masks. Thus, segmentation model is trained to predict zero scores for these regions which leads to false negatives.

% \paragraph{Reconstruction-based} Reconstruction-based methods build a generative model that takes an image $x$ as input and generates its normal (anomaly-free) version $\hat{x}$. Then anomaly scores are obtained as pixel-wise reconstruction errors between $x$ and $\hat{x}$. SotA methods from this family, e.g. DRAEM~\cite{draem}, DiffusionAD~\cite{diffusionad}, POUTA~\cite{pouta}, present a combination of reconstruction-based and synthetic-based approaches. First, they train a generative model to reconstruct synthetically corrupted image regions. Then, they train a segmentation model that takes a corrupted image and its reconstructed version as input and predicts the mask of the corrupted regions.

\paragraph{Reconstruction-based.} 
% In reconstruction-based methods, anomaly scores are obtained as reconstruction errors between the input image $x$ and generated normal (anomaly-free) counterpart $\hat{x}$.
% Reconstruction-based methods build a generative model that takes an image $x$ as input and generates its normal (anomaly-free) version $\hat{x}$. Then anomaly scores are obtained as reconstruction errors between $x$ and $\hat{x}$.
Trained solely on normal images, reconstruction-based approaches~\cite{autoencoder, vae, fanogan}, poorly reconstruct anomalous regions, allowing pixel-wise or feature-wise discrepancies to serve as anomaly scores. Later generative approaches~\cite{draem, diffusionad, pouta} integrate synthetic anomalies. The limitation stemming from anomaly-free train set assumption still persists -- if anomalous images are present, the model may learn to reconstruct anomalies as well as normal regions, undermining the ability to detect anomalies through differences between $x$ and $\hat{x}$.
% Early approaches, such as Autoencoders~\cite{autoencoder} and Variational Autoencoders~\cite{vae}, are trained solely on normal images. During inference, these models poorly reconstruct anomalous regions, allowing pixel-wise squared errors ${(x - \hat{x})^2}$ to serve as anomaly scores. Methods like f-AnoGAN~\cite{fanogan} enhance this by training W-GAN~\cite{wgan} $g$ to generate normal images and an encoder $f$ to map images to the GAN's latent space, ensuring ${\hat{x} = g(f(x)) \approx x}$. Anomalies are detected using a weighted average of reconstruction errors in pixel space and discrepancies in feature maps from GAN discriminator.

% State-of-the-art methods such as DRAEM~\cite{draem}, DiffusionAD~\cite{diffusionad}, and POUTA~\cite{pouta} integrate synthetic anomalies into the reconstruction process. They first train a generative model (autoencoder / diffusion model) to reconstruct synthetically corrupted regions. Then, they train a segmentation model that takes both the corrupted image and its reconstruction as input to predict masks of the corrupted regions.

% A major limitation of reconstruction-based methods is the assumption that the training set contains only normal images. If anomalous images are present, the generative model may learn to reconstruct anomalies as well as normal regions, undermining the ability to detect anomalies through differences between $x$ and $\hat{x}$.

% The earliest methods from this family are based on Autoencoder~\cite{autoencoder} or Variational Autoencoder~\cite{vae}, which are trained on anomaly-free images. At the inference stage, when it takes an image $x$ with anomalies it is intended to badly reconstruct the anomalous regions in $\hat{x}$, so that pixel-wise squared errors $(x - \hat{x})^2$ can be used as anomaly scores.

% Another method, f-AnoGAN~\cite{fanogan} at the first step trains W-GAN~\cite{wgan}, consisting of generator $g$ and discriminator $d$, to generate anomaly-free images $x \sim g(z)$ from latent variables $z \sim \mathcal{N}(0, I)$. Then, at the second step, it trains encoder $f$ to map anomaly-free images $x$ to the GAN's latent space, s.t. $\hat{x} = g(f(x)) \approx x$. At the inference stage, when $x$ is anomalous image, generator is assumed to generate its anomaly-free version $\hat{x}$, as it is trained only on normal images. Anomaly score are then obtained as a weighted average of reconstruction errors $(x - \hat{x})^2$ in pixel space and squared differences $(\varphi_d(x) - \varphi_d(x'))^2$ between feature maps $\varphi_d(x)$ and $\varphi_d(x')$ taken intermediate layers of GAN discriminator $d$.

% The SotA reconstruction-based methods, e.g. DRAEM~\cite{draem}, DiffusionAD~\cite{diffusionad}, POUTA~\cite{pouta}, present a combination with the approach based on synthetic anomalies. First, they train a generative model, e.g. autoencoder~\cite{draem,pouta} or diffusion model~\cite{diffusionad}, to reconstruct synthetically corrupted image regions. Then, they train a segmentation model that takes a corrupted image and its reconstructed version as input and predicts the mask of the corrupted regions.

% The main limitation of reconstruction-based methods is that they assume that training set does not contain anomalous images. Otherwise, generative model may learn to reconstruct anomalous regions as well as normal ones, which does not allow to detect anomalies by comparison of $x$ and $\hat{x}$.

\paragraph{Density-based.} Density-based methods for anomaly detection model the distribution of the training image patterns. As modeling of the joint distribution of raw pixel values is infeasible, these methods usually model the marginal or conditional distribution of pixel-wise deep feature vectors.

Some methods~\cite{ttr, pni} perform a non-parametric density estimation using memory banks. More scalable flow-based methods~\cite{fastflow,cflow,msflow}, leverage normalizing flows to assign low likelihoods to anomalies. From this family, we selected MSFlow as a representative baseline, because it is simpler than PNI, and yields similar top-5 results on the MVTec AD. 


\subsection{Medical unsupervised anomaly localization}
While there's no standard benchmark for pathology localization on CT images, MOOD~\cite{mood} offers a relevant benchmark with synthetic target anomalies. Unfortunately, at the time of preparing this work, the benchmark is closed for submissions, preventing us from evaluating our method on it. We include the top-performing method from MOOD~\cite{mood_top1} in our comparison, that relies on synthetic anomalies.

Other recognized methods for anomaly localization in medical images are reconstruction-based: variants of AE / VAE~\cite{autoencoder, dylov}, f-AnoGAN~\cite{fanogan}, and diffusion-based~\cite{latent_diffusion}. These approaches highly rely on the fact that the the training set consists of normal images only. However, it is challenging and costly to collect a large dataset of CT images of normal patients. While these methods work acceptable in the domain of 2D medical images and MRI, the capabilities of the methods have not been fully explored in a more complex CT data domain. We have adapted these methods to 3D.

\paragraph{Summary}
Our findings provide significant insights into the influence of correctness, explanations, and refinement on evaluation accuracy and user trust in AI-based planners. 
In particular, the findings are three-fold: 
(1) The \textbf{correctness} of the generated plans is the most significant factor that impacts the evaluation accuracy and user trust in the planners. As the PDDL solver is more capable of generating correct plans, it achieves the highest evaluation accuracy and trust. 
(2) The \textbf{explanation} component of the LLM planner improves evaluation accuracy, as LLM+Expl achieves higher accuracy than LLM alone. Despite this improvement, LLM+Expl minimally impacts user trust. However, alternative explanation methods may influence user trust differently from the manually generated explanations used in our approach.
% On the other hand, explanations may help refine the trust of the planner to a more appropriate level by indicating planner shortcomings.
(3) The \textbf{refinement} procedure in the LLM planner does not lead to a significant improvement in evaluation accuracy; however, it exhibits a positive influence on user trust that may indicate an overtrust in some situations.
% This finding is aligned with prior works showing that iterative refinements based on user feedback would increase user trust~\cite{kunkel2019let, sebo2019don}.
Finally, the propensity-to-trust analysis identifies correctness as the primary determinant of user trust, whereas explanations provided limited improvement in scenarios where the planner's accuracy is diminished.

% In conclusion, our results indicate that the planner's correctness is the dominant factor for both evaluation accuracy and user trust. Therefore, selecting high-quality training data and optimizing the training procedure of AI-based planners to improve planning correctness is the top priority. Once the AI planner achieves a similar correctness level to traditional graph-search planners, strengthening its capability to explain and refine plans will further improve user trust compared to traditional planners.

\paragraph{Future Research} Future steps in this research include expanding user studies with larger sample sizes to improve generalizability and including additional planning problems per session for a more comprehensive evaluation. Next, we will explore alternative methods for generating plan explanations beyond manual creation to identify approaches that more effectively enhance user trust. 
Additionally, we will examine user trust by employing multiple LLM-based planners with varying levels of planning accuracy to better understand the interplay between planning correctness and user trust. 
Furthermore, we aim to enable real-time user-planner interaction, allowing users to provide feedback and refine plans collaboratively, thereby fostering a more dynamic and user-centric planning process.



\newpage

\bibliographystyle{ccn_style}


\bibliography{ccn_style}
\appendix
\onecolumn
\begin{appendices}
\section{Supplementary Material}
\subsection{Experimental Details}

This section provides details on the training and evaluation procedures for the models presented in this paper. The full code for reproducing results and visualizations from the main text is available at: \url{https://github.com/anonymous123-user/Traveling_Waves_Integrate}.

Each model is trained for 300 epochs on the MNIST, Tetrominoes, and Multi-MNIST datasets. We evaluate the validation loss at the end of every epoch and retain the model with the lowest validation loss throughout training. The training process employs the Adam optimizer \citep{kingma_adam_2017} with a learning rate of 0.001 and a batch size of 64.

For dataset partitioning, we use 51,000 images for training, 9,000 for validation, and 10,000 for testing in MNIST. The Tetrominoes dataset consists of 10,000 images for training, 1,000 for validation, and 1,000 for testing. Similarly, the Multi-MNIST dataset comprises 10,000 images for training, 1,000 for validation, and 1,000 for testing. Table \ref{tab:merged_appendix} reports pixel-wise accuracy, IoU, and loss for both foreground and background.

Each model is trained using multiple random seeds. For MNIST and Tetrominoes, we train each model using 10 different random seeds, while for Multi-MNIST, we use 12 seeds. For example, the NWM with a linear readout is trained on MNIST 10 times, each with a different random seed. After training, we evaluate each model individually and present the aggregated results, including the mean and standard deviation, in Tables \ref{tab:merged}, \ref{tab:multi-mnist}, and \ref{tab:merged_appendix}. In total, we train 150 models on MNIST, 150 models on Tetrominoes, and 60 models on Multi-MNIST, leading to a total of 360 models.

We train the Conv-LSTMs for 20 timesteps. The NWM model is trained for 100 timesteps on the MNIST, Tetrominoes, and Multi-MNIST datasets, while for the polygons dataset, the NWM runs for 500 timesteps. In the FFT readout, we use the real component of the discrete Fourier transform. This results in 50 bins for the NWM on MNIST, Tetrominoes, and Multi-MNIST, and 250 bins for the polygons dataset. The LSTM model outputs 10 Fourier bins for MNIST, Tetrominoes, and Multi-MNIST.

All convolutional models are trained with 16 channels. To ensure a fair comparison with the NWM, a linear layer operating channelwise outputs 100 channels. The readout MLP used for MNIST and Tetrominoes consists of four layers. Its input size is given by the number of Fourier bins multiplied by 2, followed by two hidden layers of 256 neurons each, with ReLU activation between layers, and a final output layer producing logits for classification. On Multi-MNIST, the NWM employs a six-layer readout with 32 neurons in each hidden layer.

For the U-Net architecture, the Arch parameter in Tables \ref{tab:multi-mnist} and \ref{tab:merged_appendix} refers to the number of feature maps output by the first convolutional layer. Each U-Net begins with stacked convolutions that maintain spatial resolution, producing $Arch$ feature maps (e.g., 3). The following four layers apply multiple convolutional operations per layer, each reducing spatial resolution by half while doubling the number of feature maps. If the initial layer outputs 3 feature maps, the subsequent layers modify the number as follows:
\[
3 \rightarrow 6 \rightarrow 12 \rightarrow 24 \rightarrow 48.
\]
The final number of feature maps, determined by $Arch$, follows:
\[
2 \rightarrow 32, \quad 3 \rightarrow 48, \quad 4 \rightarrow 64, \quad 5 \rightarrow 80.
\]

The U-Net decoder progressively upsamples the spatial resolution over four layers while simultaneously reducing the number of feature maps by half. By the fourth layer, it restores the feature map count to the value specified by $Arch$. Finally, 1×1 convolutions are used to project the feature maps into logits for pixel-wise classification.

\newpage

\subsection{Extended Semantic Segmentation Results}
Below we include the accuracies, IoU, and Loss values computed over the full set of classes (including the background class) for all models presented in the main text. While these numbers are artificially inflated from the inclusion of the background class (which dominates the majority of pixels) we include them here for competeness. 
\begin{table*}[h!]
    \centering
    \begin{tabular}{lllrrr}
    \toprule
    Dataset & Model & Architecture & Acc & IOU & Loss \\
    \midrule
    MNIST & CNN & 2 & 0.89 $\pm$ 0.01 & 0.20 $\pm$ 0.07 & 0.34 $\pm$ 0.13 \\
      &     & 4 & 0.89 $\pm$ 0.01 & 0.22 $\pm$ 0.12 & 0.35 $\pm$ 0.18 \\
      &     & 8 & 0.92 $\pm$ 0.02 & 0.40 $\pm$ 0.14 & 0.25 $\pm$ 0.16 \\
      &     & \textbf{\boldmath 16} & \textbf{\boldmath 0.94 $\pm$ 0.05} & \textbf{\boldmath 0.55 $\pm$ 0.38} & \textbf{\boldmath 0.27 $\pm$ 0.29} \\
      &     & 32 & 0.90 $\pm$ 0.06 & 0.27 $\pm$ 0.43 & 0.50 $\pm$ 0.31 \\
    \hline
      & LSTM & Max   & 0.92 $\pm$ 0.01 & 0.43 $\pm$ 0.04 & 0.21 $\pm$ 0.02 \\
      &      & Mean  & 0.92 $\pm$ 0.01 & 0.43 $\pm$ 0.07 & 0.21 $\pm$ 0.03 \\
      &      & Last  & 0.91 $\pm$ 0.03 & 0.32 $\pm$ 0.23 & 0.35 $\pm$ 0.24 \\
      &      & FFT   & 0.96 $\pm$ 0.04 & 0.71 $\pm$ 0.26 & 0.14 $\pm$ 0.20 \\
      &      & \textbf{\boldmath Linear} 
                        & \textbf{\boldmath 0.97 $\pm$ 0.04} 
                        & \textbf{\boldmath 0.76 $\pm$ 0.27} 
                        & \textbf{\boldmath 0.12 $\pm$ 0.20} \\
    \hline
      & \textbf{CoRNN} & Max   & 0.93 $\pm$ 0.01 & 0.47 $\pm$ 0.07 & 0.19 $\pm$ 0.02 \\
      &                & Mean  & 0.93 $\pm$ 0.01 & 0.46 $\pm$ 0.07 & 0.20 $\pm$ 0.03 \\
      &                & Last  & 0.93 $\pm$ 0.01 & 0.49 $\pm$ 0.08 & 0.18 $\pm$ 0.03 \\
      &                & FFT   & 0.97 $\pm$ 0.01 & 0.72 $\pm$ 0.06 & 0.10 $\pm$ 0.02 \\
      &                & \textbf{\boldmath Linear} 
                        & \textbf{\boldmath 0.99 $\pm$ 0.00} 
                        & \textbf{\boldmath 0.88 $\pm$ 0.03} 
                        & \textbf{\boldmath 0.04 $\pm$ 0.01} \\
    \hline\hline
    Tetrominoes & CNN & 2 & 0.89 $\pm$ 0.01 & 0.24 $\pm$ 0.09 & 0.27 $\pm$ 0.13 \\
            &     & 4 & 0.90 $\pm$ 0.02 & 0.27 $\pm$ 0.14 & 0.28 $\pm$ 0.19 \\
            &     & \textbf{\boldmath 8} 
                        & \textbf{\boldmath 0.96 $\pm$ 0.04} 
                        & \textbf{\boldmath 0.66 $\pm$ 0.23} 
                        & \textbf{\boldmath 0.11 $\pm$ 0.19} \\
            &     & 16 & 0.91 $\pm$ 0.07 & 0.40 $\pm$ 0.51 & 0.39 $\pm$ 0.33 \\
            &     & 32 & 0.90 $\pm$ 0.07 & 0.33 $\pm$ 0.47 & 0.41 $\pm$ 0.31 \\
    \hline
            & LSTM & Max  & 0.94 $\pm$ 0.04 & 0.56 $\pm$ 0.30 & 0.16 $\pm$ 0.18 \\
            &      & Mean & 0.95 $\pm$ 0.04 & 0.57 $\pm$ 0.24 & 0.15 $\pm$ 0.18 \\
            &      & Last & 0.94 $\pm$ 0.06 & 0.54 $\pm$ 0.39 & 0.24 $\pm$ 0.28 \\
            &      & FFT  & 0.99 $\pm$ 0.01 & 0.93 $\pm$ 0.06 & 0.03 $\pm$ 0.02 \\
            &      & \textbf{\boldmath Linear} 
                        & \textbf{\boldmath 0.99 $\pm$ 0.00}
                        & \textbf{\boldmath 0.96 $\pm$ 0.02}
                        & \textbf{\boldmath 0.02 $\pm$ 0.01} \\
    \hline
            & \textbf{CoRNN} & Max  & 0.98 $\pm$ 0.02 & 0.85 $\pm$ 0.14 & 0.05 $\pm$ 0.03 \\
            &                & Mean & 0.99 $\pm$ 0.01 & 0.89 $\pm$ 0.10 & 0.04 $\pm$ 0.02 \\
            &                & Last & 0.99 $\pm$ 0.01 & 0.92 $\pm$ 0.08 & 0.03 $\pm$ 0.02 \\
            &                & FFT  & 1.00 $\pm$ 0.00 & 0.98 $\pm$ 0.01 & 0.01 $\pm$ 0.00 \\
            &                & \textbf{\boldmath Linear} 
                        & \textbf{\boldmath 1.00 $\pm$ 0.00} 
                        & \textbf{\boldmath 0.98 $\pm$ 0.01}
                        & \textbf{\boldmath 0.01 $\pm$ 0.00} \\
    \bottomrule
    \end{tabular}
    % }
    \caption{Supervised segmentation performance of various models and spectral methods.
    Models with the lowest foreground loss are in bold.
    Arch (architecture) for the CNN refers to the number of layers, while for the LSTM and NWM refers to the type of recurrent readout used.
    Each model is trained with 10 random seeds, and the results are displayed as $mean \pm standard \text{ }deviation$ over the 10 seeds.}
    \vspace{-3mm}
    \label{tab:merged_appendix}
\end{table*}
\begin{table*}[h!]
    \centering
    \begin{tabular}{llllrrr}
    \toprule
        & Model & Arch. & Parameters & Acc & IoU & Loss \\
    \midrule
        & U-Net & 2     & 30745      & 0.98 ± 0.01 & 0.66 ± 0.27 & 0.06 ± 0.03 \\
        &       & 3     & 68834      & 1.00 ± 0.00 & 0.91 ± 0.08 & 0.03 ± 0.01 \\
        &       & 4     & 122071     & 1.00 ± 0.00 & 0.97 ± 0.01 & 0.01 ± 0.00 \\
        &       & 5     & 190456     & 1.00 ± 0.00 & 0.98 ± 0.00 & 0.01 ± 0.00 \\
    \midrule
        & NWM   & Linear & 54855      & 1.00 ± 0.00 & 0.94 ± 0.01 & 0.01 ± 0.00 \\
    \bottomrule
    \end{tabular}
    % }
    \caption{Supervised segmentation performance of UNet and NWM with Linear Time Projection on Multi-MNIST. Arch for the U-Net refers to the number of feature maps output by the first layer. The number of feature maps doubles between each layer (e.g. 3 means 3 $\rightarrow$ 6 $\rightarrow$ 12 $\rightarrow$ 24 $\rightarrow$ 48 by the final layer). For the NWM, Arch (architecture) refers to the type of recurrent readout used.
    Each model is trained with 12 random seeds, and the results are displayed as $mean \pm standard \text{ }deviation$ over the 12 seeds.}
    \vspace{-3mm}
    \label{tab:multi-mnist_appendix}
\end{table*}

Below, we include the minimum, maximum, and median accuracies, IoU, and loss values for MNIST and Tetrominoes.
%--------------------------------------------------------------
% Table A: Acc, IoU, Loss
%--------------------------------------------------------------
\begin{table*}[ht!]
\centering
\begin{minipage}{0.99\linewidth}
\centering
\begin{tabular}{lllccc}
\toprule
 &  &  & Acc & IoU & Loss \\
\midrule
\multirow{15}{*}{MNIST} 
 & \multirow{5}{*}{CNN} 
 & 2 
 & 0.89 / 0.89 / 0.87 
 & 0.25 / 0.23 / 0.00 
 & 0.70 / 0.29 / 0.29 
\\
 &  & 4 
 & 0.90 / 0.90 / 0.87 
 & 0.31 / 0.28 / 0.00 
 & 0.70 / 0.27 / 0.26 
\\
 &  & 8 
 & 0.93 / 0.93 / 0.87 
 & 0.47 / 0.45 / 0.00 
 & 0.70 / 0.20 / 0.19 
\\
 &  & 16 
 & 0.98 / 0.97 / 0.87 
 & 0.81 / 0.78 / 0.00 
 & 0.70 / 0.09 / 0.07 
\\
 &  & 32 
 & 0.99 / 0.87 / 0.87 
 & 0.90 / 0.00 / 0.00 
 & 0.70 / 0.69 / 0.05 
\\
\cline{2-6}
 & \multirow{5}{*}{LSTM} 
 & Linear 
 & 0.99 / 0.98 / 0.87 
 & 0.91 / 0.86 / 0.00 
 & 0.70 / 0.05 / 0.03 
\\
 &  & FFT 
 & 0.98 / 0.97 / 0.87 
 & 0.87 / 0.78 / 0.00 
 & 0.70 / 0.07 / 0.05 
\\
 &  & Last 
 & 0.94 / 0.92 / 0.87 
 & 0.57 / 0.41 / 0.00 
 & 0.70 / 0.21 / 0.16 
\\
 &  & Mean 
 & 0.94 / 0.92 / 0.90 
 & 0.52 / 0.43 / 0.31 
 & 0.26 / 0.20 / 0.17 
\\
 &  & Max 
 & 0.93 / 0.92 / 0.91 
 & 0.48 / 0.43 / 0.34 
 & 0.24 / 0.20 / 0.19 
\\
\cline{2-6}
 & \multirow{5}{*}{CoRNN} 
 & Linear 
 & 0.99 / 0.99 / 0.98 
 & 0.93 / 0.88 / 0.83 
 & 0.06 / 0.04 / 0.03 
\\
 &  & FFT 
 & 0.98 / 0.97 / 0.95 
 & 0.81 / 0.74 / 0.61 
 & 0.14 / 0.09 / 0.07 
\\
 &  & Last 
 & 0.95 / 0.94 / 0.91 
 & 0.60 / 0.51 / 0.34 
 & 0.23 / 0.17 / 0.14 
\\
 &  & Mean 
 & 0.94 / 0.93 / 0.91 
 & 0.56 / 0.45 / 0.36 
 & 0.24 / 0.20 / 0.16 
\\
 &  & Max 
 & 0.94 / 0.93 / 0.92 
 & 0.57 / 0.45 / 0.38 
 & 0.21 / 0.20 / 0.15 
\\
\midrule
\multirow{15}{*}{Tetrominoes} 
 & \multirow{5}{*}{CNN} 
 & 2 
 & 0.89 / 0.89 / 0.86 
 & 0.29 / 0.26 / 0.00 
 & 0.65 / 0.22 / 0.22 
\\
 &  & 4 
 & 0.91 / 0.91 / 0.86 
 & 0.36 / 0.34 / 0.00 
 & 0.65 / 0.19 / 0.18 
\\
 &  & 8 
 & 0.97 / 0.97 / 0.86 
 & 0.76 / 0.73 / 0.00 
 & 0.65 / 0.06 / 0.05 
\\
 &  & 16 
 & 1.00 / 0.86 / 0.86 
 & 1.00 / 0.00 / 0.00 
 & 0.65 / 0.65 / 0.00 
\\
 &  & 32 
 & 1.00 / 0.86 / 0.86 
 & 1.00 / 0.00 / 0.00 
 & 0.65 / 0.63 / 0.00 
\\
\cline{2-6}
 & \multirow{5}{*}{LSTM} 
 & Linear 
 & 1.00 / 1.00 / 0.99 
 & 0.98 / 0.97 / 0.91 
 & 0.04 / 0.02 / 0.01 
\\
 &  & FFT 
 & 1.00 / 0.99 / 0.98 
 & 0.97 / 0.96 / 0.79 
 & 0.06 / 0.02 / 0.01 
\\
 &  & Last 
 & 0.99 / 0.98 / 0.86 
 & 0.91 / 0.74 / 0.00 
 & 0.65 / 0.06 / 0.04 
\\
 &  & Mean 
 & 0.99 / 0.95 / 0.86 
 & 0.95 / 0.58 / 0.00 
 & 0.65 / 0.10 / 0.03 
\\
 &  & Max 
 & 0.99 / 0.94 / 0.86 
 & 0.94 / 0.48 / 0.00 
 & 0.65 / 0.13 / 0.03 
\\
\cline{2-6}
 & \multirow{5}{*}{CoRNN} 
 & Linear 
 & 1.00 / 1.00 / 1.00 
 & 0.99 / 0.98 / 0.97 
 & 0.01 / 0.01 / 0.00 
\\
 &  & FFT 
 & 1.00 / 1.00 / 0.99 
 & 0.99 / 0.98 / 0.97 
 & 0.02 / 0.01 / 0.01 
\\
 &  & Last 
 & 1.00 / 0.99 / 0.97 
 & 0.98 / 0.96 / 0.76 
 & 0.08 / 0.02 / 0.01 
\\
 &  & Mean 
 & 1.00 / 0.99 / 0.97 
 & 0.99 / 0.91 / 0.69 
 & 0.07 / 0.03 / 0.01 
\\
 &  & Max 
 & 1.00 / 0.99 / 0.95 
 & 0.98 / 0.90 / 0.59 
 & 0.12 / 0.04 / 0.01 
\\
\bottomrule
\end{tabular}
\caption{Segmentation performance (only Acc, IoU, and Loss). 
Values are max / median / min over 10 seeds.}
\label{tab:max_min}
\end{minipage}
\end{table*}


%--------------------------------------------------------------
% Table B: FG-Acc, FG-IoU, FG-Loss
%--------------------------------------------------------------
\begin{table*}[ht!]
\centering
\begin{minipage}{0.99\linewidth}
\centering
\begin{tabular}{lllccc}
\toprule
 &  &  & FG-Acc & FG-IoU & FG-Loss \\
\midrule
\multirow{15}{*}{MNIST} 
 & \multirow{5}{*}{CNN} 
 & 2 
 & 0.18 / 0.17 / 0.00 
 & 0.12 / 0.10 / 0.00 
 & 4.35 / 2.21 / 2.20 
\\
 &  & 4 
 & 0.28 / 0.25 / 0.00 
 & 0.18 / 0.17 / 0.00 
 & 4.37 / 2.00 / 1.93 
\\
 &  & 8 
 & 0.49 / 0.47 / 0.00 
 & 0.36 / 0.33 / 0.00 
 & 4.31 / 1.47 / 1.42 
\\
 &  & 16 
 & 0.83 / 0.81 / 0.00 
 & 0.75 / 0.72 / 0.00 
 & 4.35 / 0.63 / 0.55 
\\
 &  & 32 
 & 0.91 / 0.00 / 0.00 
 & 0.86 / 0.00 / 0.00 
 & 4.37 / 4.29 / 0.33 
\\
\cline{2-6}
 & \multirow{5}{*}{LSTM} 
 & Linear 
 & 0.93 / 0.88 / 0.00 
 & 0.88 / 0.82 / 0.00 
 & 4.33 / 0.35 / 0.24 
\\
 &  & FFT 
 & 0.89 / 0.81 / 0.00 
 & 0.83 / 0.72 / 0.00 
 & 4.39 / 0.56 / 0.35 
\\
 &  & Last 
 & 0.58 / 0.40 / 0.00 
 & 0.46 / 0.28 / 0.00 
 & 4.34 / 1.58 / 1.18 
\\
 &  & Mean 
 & 0.53 / 0.43 / 0.27 
 & 0.41 / 0.32 / 0.20 
 & 1.97 / 1.53 / 1.30 
\\
 &  & Max 
 & 0.47 / 0.44 / 0.32 
 & 0.36 / 0.32 / 0.22 
 & 1.83 / 1.52 / 1.42 
\\
\cline{2-6}
 & \multirow{5}{*}{CoRNN} 
 & Linear 
 & 0.94 / 0.90 / 0.86 
 & 0.91 / 0.85 / 0.78 
 & 0.42 / 0.31 / 0.20 
\\
 &  & FFT 
 & 0.84 / 0.78 / 0.65 
 & 0.76 / 0.67 / 0.52 
 & 1.04 / 0.68 / 0.51 
\\
 &  & Last 
 & 0.61 / 0.52 / 0.35 
 & 0.49 / 0.41 / 0.25 
 & 1.74 / 1.30 / 1.07 
\\
 &  & Mean 
 & 0.57 / 0.46 / 0.34 
 & 0.46 / 0.34 / 0.24 
 & 1.78 / 1.48 / 1.16 
\\
 &  & Max 
 & 0.59 / 0.46 / 0.41 
 & 0.47 / 0.34 / 0.31 
 & 1.55 / 1.47 / 1.11 
\\
\midrule
\multirow{15}{*}{Tetrominoes} 
 & \multirow{5}{*}{CNN} 
 & 2 
 & 0.26 / 0.26 / 0.00 
 & 0.16 / 0.16 / 0.00 
 & 3.50 / 1.55 / 1.54 
\\
 &  & 4 
 & 0.40 / 0.39 / 0.00 
 & 0.27 / 0.25 / 0.00 
 & 3.55 / 1.30 / 1.28 
\\
 &  & 8 
 & 0.83 / 0.82 / 0.00 
 & 0.72 / 0.70 / 0.00 
 & 3.55 / 0.38 / 0.33 
\\
 &  & 16 
 & 1.00 / 0.00 / 0.00 
 & 1.00 / 0.00 / 0.00 
 & 3.61 / 3.50 / 0.01 
\\
 &  & 32 
 & 1.00 / 0.00 / 0.00 
 & 1.00 / 0.00 / 0.00 
 & 3.60 / 3.40 / 0.01 
\\
\cline{2-6}
 & \multirow{5}{*}{LSTM} 
 & Linear 
 & 0.98 / 0.98 / 0.92 
 & 0.97 / 0.96 / 0.87 
 & 0.25 / 0.10 / 0.08 
\\
 &  & FFT 
 & 0.98 / 0.97 / 0.84 
 & 0.96 / 0.94 / 0.74 
 & 0.45 / 0.14 / 0.09 
\\
 &  & Last 
 & 0.92 / 0.84 / 0.00 
 & 0.86 / 0.73 / 0.00 
 & 3.55 / 0.43 / 0.29 
\\
 &  & Mean 
 & 0.96 / 0.68 / 0.00 
 & 0.93 / 0.53 / 0.00 
 & 3.61 / 0.72 / 0.20 
\\
 &  & Max 
 & 0.96 / 0.57 / 0.00 
 & 0.92 / 0.42 / 0.00 
 & 3.44 / 0.93 / 0.21 
\\
\cline{2-6}
 & \multirow{5}{*}{CoRNN} 
 & Linear 
 & 0.99 / 0.99 / 0.98 
 & 0.99 / 0.98 / 0.96 
 & 0.09 / 0.05 / 0.03 
\\
 &  & FFT 
 & 0.99 / 0.98 / 0.97 
 & 0.98 / 0.97 / 0.95 
 & 0.11 / 0.06 / 0.04 
\\
 &  & Last 
 & 0.98 / 0.97 / 0.81 
 & 0.97 / 0.95 / 0.70 
 & 0.53 / 0.15 / 0.10 
\\
 &  & Mean 
 & 0.99 / 0.94 / 0.78 
 & 0.98 / 0.89 / 0.65 
 & 0.50 / 0.23 / 0.06 
\\
 &  & Max 
 & 0.99 / 0.92 / 0.69 
 & 0.98 / 0.86 / 0.54 
 & 0.82 / 0.27 / 0.09 
\\
\bottomrule
\end{tabular}
\caption{Segmentation performance (only FG-Acc, FG-IoU, and FG-Loss). 
Values are max / median / min over 10 seeds.}
\label{tab:fg_max_min}
\end{minipage}
\end{table*}


\begin{figure}[h!] % 'htbp' specifies how the figure should be placed
    \centering
    \includegraphics[width=\linewidth]{figures/fft_mulitpolygon.png} % Width can be adjusted as needed
    % \vspace{-5mm}
    \caption{Visualization of all frequency bins for an example of the Polygons dataset. We see that the background and different shapes appear in separate frequency bins, allowing the model to easily segment the shapes semantically in frequency space. }
    \label{fig:all_fft}
\end{figure}



\begin{figure}[h!] % 'htbp' specifies how the figure should be placed
    \centering
    \includegraphics[width=0.7\linewidth]{figures/shape_combinations.png} % Width can be adjusted as needed
    % \vspace{-5mm}
    \caption{Visualization of the impact of different combinations of shapes in the same image on the frequency space representation of the other shape for the Polygons dataset. We see that while there is a minor impact on the frequency space representation of each shape when another shape appears nearby, the overall frequency spectrum is relatively invariant. This implies that each neuron indeed has global information about all shapes present in the image, but mainly represents the shape which it is currently `located within'.}
    \label{fig:shape_combo}
\end{figure}
\end{appendices}

\end{document}
