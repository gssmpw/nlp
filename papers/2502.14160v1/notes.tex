\section{Intro Outline}

\amy{
\begin{itemize}
\item introduce game theory, including eqm definition. need to discuss how general this def'n should be. 
%
\begin{itemize}
\item do we really need CF deviations this early? might be too specific/detailed for first paragraph.
%
\item otoh, we could introduce parameterized-payoff this early, since they are fundamental to our story
\end{itemize}
%
\item define inverse game theory: payoffs are unknown. (again prob don't need to worry about Nash vs. CE vs. CCE, etc.)
%
\item four versions of the problem are known/have been studied: planning vs.\ learning, depending on whether oracle is exact or stochastic. and inverse vs. apprenticeship (planning or learning), depending on whether exact policies are known, or only samples from those policies are given.
%
\item we consider a further breakdown of the apprenticeship learning problem in particular. in the usual version, the samples are fine-grained. we also consider a more realistic version, in which the samples are coarser. while the solution to the former problem is an equilibrium, we call the solution to this latter problem a *simulacrum*. we call the problem itself, high-fidelity inverse MARL (MAYBE!!!/placeholder).
%
\item first contribution: solve inverse game, given eqm policies, assuming an exact oracle. Inverse MAP. 
%
\item second contribution: solve inverse game, given eqm policies, assuming stochastic oracle. Inverse MAL. actual application is to stochastic games, so we solve Inverse MARL.
%
\item third contribution: solve inverse game, given sample trajectories of eqm policies, assuming stochastic oracle. this problem is called multiagent apprenticeship learning. in fact, we solve a generalization of multiagent apprenticeship learning, namely ???, in which the sample trajectories are COARSE-grained/low-resolution.
%
\item N.B. i remain uncomfortable equating behavioral models and payoffs...still thinking about this point. (and wrote to a behavioral economist to inquire.)\deni{We should not use the term behavioral model anywhere. I removed any mention. It is not necessary for what we are doing.}
\end{itemize}
}


\section{Notes}

% \begin{enumerate}
%     % \item Was the GAES method computing a Bayesian coarse correlated equilibrium \deni{There seems to be a bug in the def'n of CCE for Bayesian games, need to ask Gianluca or someone else.}
%     % \item Relation between Bayesian coarse correlated equilibrium and Nash eqm.?
%     % \item Auction example
%     % \item Major difference between Vasilis' paper and this is that Vasilis works in a discretized setting.
%     % \begin{enumerate}
%     %     \item In particular: Correlated in continuous action space does not make sense, instead one would be better off considering Nash in mixed strategies which would be better and as easy to solve. The whole thing does not make too much sense.
%     % \end{enumerate}
% \end{enumerate}