\subsection{Additional Preliminary Definitions}\label{sec_app:prelims}

\paragraph{Notation}
We use caligraphic uppercase letters to denote sets (e.g., $\calX$), bold uppercase letters to denote matrices (e.g., $\allocation$), bold lowercase letters to denote vectors (e.g., $\price$), lowercase letters to denote scalar quantities (e.g., $x$), and uppercase letters to denote random variables (e.g., $X$).
We denote the $i$th row vector of a matrix (e.g., $\allocation$) by the corresponding bold lowercase letter with subscript $i$ (e.g., $\allocation[\buyer])$ and the $j$th entry of a vector (e.g., $\price$ or $\allocation[\buyer]$) by the corresponding Roman lowercase letter with subscript $j$ (e.g., $\price[\good]$ or $\allocation[\buyer][\good]$).
We denote functions by a letter determined by the value of the function: e.g., $f$ if the mapping is scalar-valued, $\f$ if the mapping is vector-valued, and $\calF$ if the mapping is set-valued. We denote the set of natural numbers by $\N$ and the set of real numbers by $\R$.
We denote the positive and strictly positive elements of a set by a $+$ and $++$ subscript, respectively, e.g., $\R_+$ and $\R_{++}$.
For any set $\calC$, we denote its diameter $\max_{\c, \c^\prime \in \calC} \|\c - \c^\prime \|$ by $\diam(\calC)$.