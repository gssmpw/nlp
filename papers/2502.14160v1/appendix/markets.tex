\subsection{Markets Experiments}\label{sec_app:markets}

\subsubsection{Static Fisher Markets} 
A \mydef{(one-shot) Fisher market} consists of $\numbuyers$ buyers and $\numgoods$ divisible goods with unit supply\citep{brainard2000compute}.
Each buyer $\buyer \in \buyers$ is endowed with a budget $\budget[\buyer] \in \budgetspace[\buyer] \subseteq \mathbb{R}_{+}$ and a utility function $\util[\buyer]: \mathbb{R}_{+}^{\numgoods} \times \typespace[\buyer] \to \mathbb{R}$, which is parameterized by a type $\type[\buyer] \in \typespace[\buyer]$ that defines a preference relation over the consumption space $\R^\numgoods_+$. An instance of a Fisher market is then a tuple $\calM \doteq (\numbuyers, \numgoods, \util, \type, \budget)$, where $\util \doteq \left(\util[1], \hdots, \util[\numbuyers] \right)$ is a vector-valued function of all utility functions and $\budget \doteq (\budget[1], \hdots, \budget[\numbuyers]) \in \R_{+}^{\numbuyers}$ is the vector of buyer budgets.
When clear from context, we simply denote $\calM$
by $(\type, \budget)$.

Given a Fisher market $(\type, \budget)$, an \mydef{allocation} $\allocation = \left(\allocation[1], \hdots, \allocation[\numbuyers] \right)^T \in \R_+^{\numbuyers \times \numgoods}$ is a map from goods to buyers, represented as a matrix, s.t. $\allocation[\buyer][\good] \ge 0$ denotes the amount of good $\good \in \goods$ allocated to buyer $\buyer \in \buyers$. Goods are assigned \mydef{prices} $\price = \left(\price[1], \hdots, \price[\numgoods] \right)^T \in \mathbb{R}_+^{\numgoods}$. A tuple $(\allocation[][][][*], \price^*)$ is said to be a \mydef{competitive equilibrium (CE)} \citep{arrow-debreu, walras} if 
1.~buyers are utility maximizing, constrained by their budget, i.e., $\forall \buyer \in \buyers, \allocation[\buyer][][][*] \in \argmax_{\allocation[ ] : \allocation[ ] \cdot \price^* \leq \budget[\buyer]} \util[\buyer](\allocation[ ], \type[\buyer])$;
and 2.~the market clears, i.e., $\forall \good \in \goods,  \price[\good]^* > 0 \Rightarrow \sum_{\buyer \in \buyers} \allocation[\buyer][\good][][*] = \supply[\good]$ and $\price[\good]^* = 0 \Rightarrow\sum_{\buyer \in \buyers} \allocation[\buyer][\good][][*] \leq \supply[\good]$.

The set of CE of any Fisher market $(\type, \budget)$ with continuous, concave, and homogeneous\footnote{A function $f: \R^m \to \R$ is called \mydef{homogeneous of degree $k$} if $\forall \allocation[ ] \in \R^m, \lambda > 0, f(\lambda \allocation[ ]) = \lambda^k f(\allocation[ ])$.} utility functions is equal to the set of Nash equilibria of the \mydef{Eisenberg-Gale min-max game},%
\footnote{This min-max game corresponds to the Lagrangian saddle-point formulation of the Eisenberg-Gale program \cite{gale1989theory, jain2005market}.} 
a convex-concave min-max game between a seller who chooses prices $\price \in \R_+^{\numgoods}$ and buyers who collectively choose allocations
$\allocation \in \R_+^{\numbuyers \times \numgoods}$:
the objective function of this game comprises two sums: the first is the logarithmic Nash social welfare of the buyers' utility \ssadie{per budget, i.e., bang-per-buck}{}, while the second is the profit of a fictional auctioneer who sells the goods in the market:
\begin{align}
    \min_{\price \in \R_+^{\numgoods}} \max_{\allocation \in \R_+^{\numbuyers \times \numgoods}} 
    & \obj (\price, \allocation; \type, \budget) \doteq \sum_{\buyer \in \buyers} \budget[\buyer] \log \left( \util[\buyer](\allocation[\buyer], \type[\buyer]) \right) + \sum_{\good \in \goods} \left( \price[\good] - \price[\good] 
    \sum_{\buyer \in \buyers}\allocation[\buyer][\good] \right)
    \label{eq:eg_game}
\end{align}


Therefore, for any Fisher market $\calM \doteq (\numbuyers, \numgoods, \util, \typetrue, \budgettrue)$, 
we can construct a inverse game $\game[][-1] \doteq 
(\game[][\paramtrue]/\paramtrue , \truestrat)$ where $\game[][\paramtrue]$ is the corresponding Eisenberg-Gale min-max game (\Cref{eq:eg_game}) parameterized by the true types and budgets $\paramtrue=(\typetrue, \budgettrue)$, and $\truestrat=(\allocation[][][][*], \price^*)$ is not only a NE of the game $\game[][\paramtrue]$ but also a CE of market $\calM$. Our goal is to recover the true market parameters $\paramtrue=(\typetrue, \budgettrue)$ given the observed $\truestrat=(\allocation[][][][*], \price^*)$, by solving this inverse game problem using \Cref{thm:inverse_NE} and \Cref{alg:gda}. 

We ran two different experiments\footnote{We include a detailed description of our experimental setup in the appendix.}. First, we solved a simpler inverse game problem where the true type $\typetrue$ is given, and we just need to retrieve the true budgets $\budgettrue$; then, we attempt to recover both true type and true budgets simultaneously. For both experiments, we created 500 markets with each of these three (standard) classes of utility functions parameterized by types:
1.~\mydef{linear}: $\util[\buyer](\allocation[\buyer]; \type[\buyer]) = \sum_{\good \in \goods} \type[\buyer][\good] \allocation[\buyer][\good]$; 2.~\mydef{Cobb-Douglas (CD)}:  $\util[\buyer](\allocation[\buyer]; \type[\buyer]) = \prod_{\good \in \goods} {\allocation[\buyer][\good]}^{\type[\buyer][\good]}$; and 3.~\mydef{Leontief}:  $\util[\buyer](\allocation[\buyer]; \type[\buyer]) = \min_{\good \in \goods} \left\{ \frac{\allocation[\buyer][\good]}{\type[\buyer][\good]}\right\}$. 
Then, we ran \Cref{alg:gda} on min-max optimization problem \sadie{I may write out the specific min-max problem for fisher, but I don't think we have enough space. Maybe in the Appendix.} defined in \Cref{eq:min_max_gen_sim} to compute the inverse Nash Equilibrium of the inverse game $\game[][-1]$ defined above.
Finally, for each utility type, we recorded the percentage of markets that we could recover the true parameters, i.e., the markets for which our computed parameters is close enough to the true parameters, and the average exploitability\sadie{refer appendix?} of the observed equilibrium evaluated under the computed parameters across all markets.
% To compute the true budgets, we solve the min-max optimization problem \begin{align}\label{eq:fisher_min_max_budgets}
%         \min_{\substack{\budget \in \R_+^{\numbuyers}}} \max_{\substack{\allocation\in \R_+^{\numbuyers\times\numgoods\\
%         \price\in \R_+^{\numgoods}  }}}
%         \underbrace{
%         \left( -\obj(\price, \allocation[][][][*]; \typetrue, \budget)+ \obj(\price^*, \allocation[][][][*]; \typetrue, \budget)
%         \right) + \left(\obj(\price^*, \allocation; \typetrue, \budget)- \obj(\price^*, \allocation[][][][*]; \typetrue, \budget)\right)
%         }_{ = \cumulregret[] ((\allocation[][][][*], \price^*), (\allocation,\price); \budget)}
% \end{align}
% using \Cref{alg:gda}.

\Cref{table:both} shows that when only retrieving budgets,we were able to recover all the parameters and minimize exploitability in markets with Linear and Cobb-Douglas utilities, but we hardly do so in Leontief markets. The difficulty in this case likely arises from two aspects: first, Leontief utility function is not differentiable, so the min-max optimization problems associated to Leontief
markets are not smooth; moreover, for any Leontief Fisher market, the CE is not guaranteed to be unique. 
When computing budgets and types at once, while our algorithm can still minimize exploitability for both Linear nad Cobb-Douglas markets, it cannot really retrive true parameters in Linear markets. This may due to the fact that, in Linear markets, though competitive equilibrium prices are unique, the competitive allocations are not guaranteed to be unique; by contrast, the CEs are always unique in Leontief markets. 

\paragraph{Hyperparameters}
We randomly initialized 500 different linear, Leontief, Cobb-Douglas Fisher markets, each with 3 buyers and 2 goods. Buyer $\buyer$’s budget $\budget[\buyer]$ was drawn randomly from a uniform distribution ranging from 0 to 10 (i.e., $U[0,10]$), and each buyer $\buyer$’s type for good $\good$, $\type[\buyer][\good]$, was drawn randomly from $U[0,10]$. 

For Fisher markets with all three class of utilities, we ran our algorithm for 5000 iterations with learning rate $\learnrate[\param]=0.01$. Moreover, we stop our algorithm when our computed inverse Nash equilibrium $\param$ is closed enough to the original parameter $\paramtrue$: $||\frac{\param-\paramtrue}{\paramtrue}||_2\leq \epsilon$ where we set $\epsilon=0.1$.



\subsubsection{Cournot Competition and Bertrand Competition}
A \mydef{Cournot competition model} $\calC \doteq (\numfirms, \mcost, \pricefunc)$ consists of $n$ firms that produce a homogeneous product, and each firm $i$ chooses a quantity level of production $\Cprod[i]$ that maximizes its profits. All firms face a marginal cost $\mcost$. That is, for a given firm $i$, the cost of producing $\Cprod[i]$ unit of good is $\mcost\Cprod[i]$. The price function $\pricefunc$ takes the total production of all firms $Q_{total}=\sum_{i\in \firms}\Cprod[i]$ as input and outputs the unit prices for the good. Thus, the profit function for firm $i$ is $f_i(\Cprod[i], \Cprod[-i]; \mcost)=\Cprod[i](\pricefunc(\sum_{i\in \firms}\Cprod[i])-\mcost)$. $\Cprod^*$ is a Nash equilibria of the Cournot game if and only if $\Cprod[i]^* \in \argmax_{\Cprod[i]in \R_+} f_i(\Cprod[i], \Cprod[-i]^*; \mcost)$ for all $i\in \firms$.


A \mydef{Bertrand competition model} $\calB \doteq (\numfirms, \mcost, \demandfunc)$ is also a competition model that consists of $n$ firms producing a homogeneous product, but this time, each firm $i$ set prices $\Bprices[i]$ to maximize its profits. All firms face a marginal cost $\mcost$. That is, for a given firm $i$, the cost of producing $\Cprod[i]$ unit of good is $\mcost\Cprod[i]$. The demand function $\demandfunc$ takes the minimum price proposed by the firms $\Bprices[\min]=\min_{i\in \firms} \Bprices[i]$ as input and outputs the demand for that good in the whole market. Firm $i$’s individual demand function is a function of the price set by each firm: 
\begin{align}
    D_{i}(\Bprices[i],\Bprices[-i])=\begin{cases}
        D(\Bprices[\min])  & \Bprices[i]=\Bprices[\min], \Bprices[j]\geq \Bprices[\min] \forall j\neq i\\
        \frac{D(\Bprices[\min])}{n} &\Bprices[i]=\Bprices[\min],\text{$n=$\# of $j\in \firms$ with $\Bprices[j]=\Bprices[\min]$}\\
        0 & \Bprices[i]\neq \Bprices[\min]
    \end{cases}
\end{align}

% \begin{align}
%     D_{i}(\Bprices[i],\Bprices[-i])&=D(\Bprices[i])*\mathrm{softmin}_i(\Bprices)\\
%     &D(\Bprices[i]) * \setindic[{\argmin_i(\Bprices[i])}](i)
% \end{align}

Thus, the profit function for the firm $\i$ is $f_i(\Bprices[i], \Bprices[-i]; \mcost)=D_{i}(\Bprices[i],\Bprices[-i])(\Bprices[i]-\mcost)$. $\Bprices^*$ is a Nash equilibria of the Bertrand game if and only if $\Bprices[i]^* \in \argmax_{\Bprices[i]in \R_+} f_i(\Bprices[i], \Bprices[-i]^*; \mcost)$ for all $i\in \firms$.

In experiments, we generated 500 Cournot competition models and 500 Bertrand competition models and attempted to retrieve their true parameter, i.e., marginal costs, given equilibrium productions/equilibrium prices respectively. \Cref{table:both} shows that our algorithm can effectively recover the true parameters in Cournot games and minimize the exploitability of the observed equilibrium evaluated under the computed parameters. In the Bertrand games, though we could only recover $78\%$ true parameters, the average exploitability of the observed equilibrium evaluated under the computed inverse Nash equilibrium is mostly minimized. 

\paragraph{Hyperparameters}
We randomly initialized 500 different duopoly Cournot competitions and duopoly Bertrand competitions. Marginal costs was drawn randomly from a uniform distribution ranging from 2 to 20 (i.e., $U[2,20]$) for both Cournot and Bertrand. Moreover, we define the price functions in Cournot competitions as $P(\Cprod[total]) = a + b\Cprod[total]$ where $a\sim U[10,100]$ and $b\sim U[-10, -0.01]$. We define the demand functions in Bertrand competitions as $D(\Bprices[\min])=c+d\Bprices[\min]$ where $c\sim U[10,100]$ and $d\sim U[-10, -0.01]$.

For Cournot competitions, we ran our algorithm for 10000 iterations with learning rate $\learnrate[\param]=0.01$, and for Bertrand competitions, we ran our algorithm for 250 iterations with learning rate $\learnrate[\param]=0.3$. Moreover, we stop our algorithm when our computed inverse Nash equilibrium $\param$ is closed enough to the original parameter $\paramtrue$: $||\frac{\param-\paramtrue}{\paramtrue}||_2\leq \epsilon$ where we set $\epsilon=0.1$.

\subsubsection{Stochastic Fisher Markets}
\mydef{A (static) Fisher market} $(\numbuyers, \numgoods, \consumptions, \util, \supply, \type, \budget)$, $(\supply, \type, \budget)$ when clear from context, consists of $\numbuyers \in \N_{++}$ buyers and $\numgoods \in \N_{++}$ divisible goods \cite{brainard2000compute}.
Each buyer $\buyer \in \buyers$ is represented by a tuple $(\consumptions[\buyer], \util[\buyer],\type[\buyer], \budget[\buyer])$, $(\type[\buyer], \budget[\buyer])$ when clear from context, which consists of a \mydef{budget} $\budget[\buyer] \in \budgetspace[\buyer] \subseteq \R_+$ of some num\'eraire good it is endowed with, a \mydef{utility function}
% \footnote{\deni{This footnote or the next has to go.}Without loss of generality, we assume that utility function are positive-valued since any real-valued utility function can be made positive-valued by passing it through the monotone transformation $u \mapsto e^u$, without modifying the underlying preference relation.}
$\util[\buyer]: \consumptions[\buyer] \times \typespace[\buyer] \to \mathbb{R}_+$, which is parameterized by a \mydef{type} $\type[\buyer] \in \typespace[\buyer]$ s.t. $\util[\buyer](\ \cdot \ ; \type[\buyer])$ defines a preference relation over the \mydef{consumption space} $\consumptions[\buyer] \subseteq \R^\numgoods_+$.
Each good is characterized by a supply $\supply[\good] \in \supplyspace[\good] \subseteq \R_+$. We denote the collection of all utility functions $\util \doteq \left(\util[1], \hdots, \util[\numbuyers] \right)$, the collection of buyer types $\type \doteq \left(\type[1], \hdots, \type[\numbuyers]\right)$, the collection of buyer budgets $\budget \doteq (\budget[1], \hdots, \budget[\numbuyers]) \in \R_{+}^{\numbuyers}$, the collection of all good supplies $\supply \doteq (\supply[1], \hdots, \supply[\numgoods]) \in \R_{+}^{\numgoods}$, the joint space of consumptions $\consumptions \doteq \bigtimes_{\buyer \in \buyers} \consumptions[\buyer]$, the joint space of  types $\typespace \doteq \bigtimes_{\buyer \in \buyers} \typespace[\buyer]$, and the joint space of budgets $\budgetspace \doteq \bigtimes_{\buyer \in \buyers} \budgetspace[\buyer]$.


% Solutions to a stochastic Fisher market can be modelled as a Markov game, which can be shown to correspond to recursive competitive equilibria \cite{goktas2022zero}:

\begin{definition}[Stochastic Fisher Market Game]
    Given a stochastic Fisher market $\fishermkt \doteq (\numbuyers, \numgoods, \numassets, \states, \util, \trans, \discount, \initstates)$, we define the Stochastic Fisher Market Game $\mgame[][\fishermkt] \doteq (\numplayers, \states, \outeractions, \inneractions, \reward, \trans, \discount, \initstates)$ where:
    \begin{alignat}{3}
        &\states \doteq \worldstates \times \supplyspace \times \budgetspace \times \typespace
        \qquad \qquad&\outeractions \doteq \pricespace
        \qquad \qquad&\inneractions(\state) \doteq \consumptions \times \savings(\state)
    \end{alignat}
    \begin{align}
        \reward((\worldstate, \supply, \type, \budget), \price, (\allocation, \saving)) \doteq  \price \cdot \left( \supply - \sum_{\buyer \in \buyers} \allocation[\buyer] \right) + \sum_{\buyer \in \buyers} \left(\budget[\buyer] + \saving[\buyer] \right) \log \left( \frac{\util[\buyer](\allocation[\buyer]; \type[\buyer])}{\budget[\buyer] + \saving[\buyer] }\right)
    \end{align}
\end{definition}

% where $\simplex[m] = \{\x \in \R_+^m \mid \sum_{i = 1}^n x_i = 1 \}$ denote the unit simplex in $\R^m$.

\paragraph{Experimental setup}
We use Jax, and Haiku to traint he simulacrum policies and use a feedforward neural network with 4 layers with 200 nodes. We run our experiments with 5 random seed and report the best results.
% Given a stochastic Fisher market with savings $\fishermkt[0]$
% %$(\states, \util, \state[0], \trans, \discount)$, 
% a \mydef{recursive competitive equilibrium (recCE)} \citep{mehra1977recursive} is a tuple $(\allocation[][][][*], \saving[][][*], \price^*) \in \R_+^{\numbuyers \times \numgoods \times \states} \times \R_+^{\numbuyers \times \states} \times \R_+^{\numgoods \times \states}$, which consists of stationary \mydef{allocation}, \mydef{savings}, and \mydef{pricing policies} s.t.\ 1) the buyers are expected utility maximizers, constrained by their savings and spending constraints, i.e., for all buyers $\buyer \in \buyers$, $(\allocation[\buyer][][][*], \saving[\buyer][][*])$ is an optimal policy that, for all states $\state \doteq (\type, \budget, \supply) \in \states$, solves the \mydef{consumption-savings problem}, defined by the following Bellman equations: for all $\state \in \states$
% \begin{align}
%     \budgetval[\buyer](\state) = \max_{(\allocation[\buyer], \saving[\buyer]) \in \R^{\numgoods + 1}_+: \allocation[\buyer] \cdot \price^*(\state) + \saving[\buyer] \leq \budget[\buyer]} \left\{ \util[\buyer]\left(\allocation[\buyer], \type[\buyer]\right)     + \discount \Ex_{({\type}^\prime, {\budget}^\prime, {\supply}^\prime) \sim \trans(\cdot \mid \state, (\allocation[\buyer], \allocation[][][][*]_{-\buyer}(\state)), 
%     (\saving[\buyer], \saving[][][*]_{-\buyer}(\state)))} 
%     \left[ \budgetval[\buyer]({\type}^\prime, {\budget}^\prime + (\saving[\buyer], \saving[][][*]_{-\buyer}(\state)), {\supply}^\prime) \right]  \right\},    
% \end{align}
%     %\budgetval[\buyer](\type, \budget, \supply) =
    

% %for all states $(\type, \budget, \supply) \in \states$ 
% where $\allocation[][][][*]_{-\buyer}$, $\saving[][][*]_{-\buyer}$ denote the 
% %competitive equilibrium 
% allocation and saving policies excluding buyer $\buyer$; and
% 2) the market clears in each state so that unallocated goods in each state are priced at 0, i.e., for all $\good \in \goods$ and $\state \in \states$,
% $    
%         \price[\good]^*(\state) > 0 \implies \sum_{\buyer \in \buyers} \allocation[\buyer][\good][][*](\state) = \supply[\good]
% $ and
% $
%         \price[\good]^*(\state) \geq 0 \implies \sum_{\buyer \in \buyers} \allocation[\buyer][\good][][*](\state) \leq \supply[\good]
% $.
% % \samy{By analogy with Markov perfect equilibrium, we can view a}
% A recCE is Markov perfect, as it is a CE regardless of initial  state,%
% \footnote{Just as any stochastic game can be (non-compactly) represented as a one-shot game, any stochastic Fisher market with savings can be represented as a one-shot Fisher market comprising the same buyers (with utility functions given by their discounted cumulative expected utility) 
% % $\Ex_{\staterv[\iter]} \left[\discount^\iter \util[\buyer]\left(\allocation[\buyer][][\iter], \type[\buyer][][\iter]\right) \right]$
% and the same goods, enhanced with time-stamps.
% It has been shown that competitive equilibria exist in such markets~\citep{prescott1972note}.
% In particular, a recCE is a CE in this market; the time stamps ensure that the market clears at every time step, as required.}
% %\amy{does whoever writes about the time-stamped construction prove that CE exist in these one-shot markets? also, would a recursive CE of the stochastic Fisher market also be a CE of this one-shot market?}
% i.e., buyers are allocated expected discounted cumulative utility-maximizing goods regardless of initial state, and the aggregate demand for each good is equal to its aggregate supply \emph{at all states}.

% We show that the computation of a recursive competitive equilibrium in stochastic Fisher markets is equivalent to the computation of nash equilibrium in zero-sum Markov game.