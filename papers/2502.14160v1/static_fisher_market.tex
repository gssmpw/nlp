\subsubsection{Static Fisher Markets\sadie{one-shot?}}

A \mydef{(one-shot) Fisher market} $\calM \doteq (\numbuyers, \numgoods, \util, \type, \budget)$ consists of $\numbuyers$ buyers and $\numgoods$ divisible goods with unit supply\citep{brainard2000compute}. The set of \mydef{Competitive Equilibrium (CE)} of any Fisher market $(\numbuyers, \numgoods, \util, \type, \budget)$ with continuous, concave, and homogeneous\footnote{A function $f: \R^m \to \R$ is called \mydef{homogeneous of degree $k$} if $\forall \allocation[ ] \in \R^m, \lambda > 0, f(\lambda \allocation[ ]) = \lambda^k f(\allocation[ ])$.} utility functions is equal to the set of Nash equilibria of the \mydef{Eisenberg-Gale min-max game} (defined in \Cref{eq:eg_game}),%
\footnote{This min-max game corresponds to the Lagrangian saddle-point formulation of the Eisenberg-Gale program \cite{gale1989theory, jain2005market}.} 
a convex-concave min-max game between a seller who chooses prices $\price \in \R_+^{\numgoods}$ and buyers who collectively choose allocations
$\allocation \in \R_+^{\numbuyers \times \numgoods}$ (more details see \Cref{sec_app:markets}).


Therefore, for any Fisher market $\calM \doteq (\numbuyers, \numgoods, \util, \typetrue, \budgettrue)$, 
we can construct a inverse game $\game[][-1] \doteq 
(\game[][\paramtrue]/\paramtrue , \truestrat)$ where $\game[][\paramtrue]$ is the corresponding Eisenberg-Gale min-max game (\Cref{eq:eg_game}) parameterized by the true types and budgets $\paramtrue=(\typetrue, \budgettrue)$, and $\truestrat=(\allocation[][][][*], \price^*)$ is not only a NE of the game $\game[][\paramtrue]$ but also a CE of market $\calM$. Our goal is to recover the true market parameters $\paramtrue=(\typetrue, \budgettrue)$ given the observed $\truestrat=(\allocation[][][][*], \price^*)$, by solving this inverse game problem using \Cref{thm:inverse_NE} and \Cref{alg:gda}. 



We ran two different experiments\footnote{We include a detailed description of our experimental setup in the appendix.}. First, we solved a simpler inverse game problem where the true type $\typetrue$ is given, and we just need to retrieve the true budgets $\budgettrue$; then, we attempt to recover both true type and true budgets simultaneously. For both experiments, we created 500 markets with each of these three (standard) classes of utility functions parameterized by types:
1.~\mydef{linear}: $\util[\buyer](\allocation[\buyer]; \type[\buyer]) = \sum_{\good \in \goods} \type[\buyer][\good] \allocation[\buyer][\good]$; 2.~\mydef{Cobb-Douglas (CD)}:  $\util[\buyer](\allocation[\buyer]; \type[\buyer]) = \prod_{\good \in \goods} {\allocation[\buyer][\good]}^{\type[\buyer][\good]}$; and 3.~\mydef{Leontief}:  $\util[\buyer](\allocation[\buyer]; \type[\buyer]) = \min_{\good \in \goods} \left\{ \frac{\allocation[\buyer][\good]}{\type[\buyer][\good]}\right\}$. 
Then, we ran \Cref{alg:gda} on min-max optimization problem \sadie{I may write out the specific min-max problem for fisher, but I don't think we have enough space. Maybe in the Appendix.} defined in \Cref{eq:min_max_gen_sim} to compute the inverse Nash Equilibrium of the inverse game $\game[][-1]$ defined above.
Finally, for each utility type, we recorded the percentage of markets that we could recover the true parameters, i.e., the markets for which our computed parameters is close enough to the true parameters, and the average exploitability\sadie{refer appendix?} of the observed equilibrium evaluated under the computed parameters across all markets.
% To compute the true budgets, we solve the min-max optimization problem \begin{align}\label{eq:fisher_min_max_budgets}
%         \min_{\substack{\budget \in \R_+^{\numbuyers}}} \max_{\substack{\allocation\in \R_+^{\numbuyers\times\numgoods\\
%         \price\in \R_+^{\numgoods}  }}}
%         \underbrace{
%         \left( -\obj(\price, \allocation[][][][*]; \typetrue, \budget)+ \obj(\price^*, \allocation[][][][*]; \typetrue, \budget)
%         \right) + \left(\obj(\price^*, \allocation; \typetrue, \budget)- \obj(\price^*, \allocation[][][][*]; \typetrue, \budget)\right)
%         }_{ = \cumulregret[] ((\allocation[][][][*], \price^*), (\allocation,\price); \budget)}
% \end{align}
% using \Cref{alg:gda}.

\Cref{table:fisher_results} shows that when only retrieving budgets,we were able to recover all the parameters and minimize exploitability in markets with Linear and Cobb-Douglas utilities, but we hardly do so in Leontief markets. The difficulty in this case likely arises from two aspects: first, Leontief utility function is not differentiable, so the min-max optimization problems associated to Leontief
markets are not smooth; moreover, for any Leontief Fisher market, the CE is not guaranteed to be unique. 
When computing budgets and types at once, while our algorithm can still minimize exploitability for both Linear nad Cobb-Douglas markets, it cannot really retrive true parameters in Linear markets. This may due to the fact that, in Linear markets, though competitive equilibrium prices are unique, the competitive allocations are not guaranteed to be unique; by contrast, the CEs are always unique in Leontief markets. 