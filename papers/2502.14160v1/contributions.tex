%
%\sdeni{Earlier methods to compute inverse equilibria primarily concern normal-form games with discrete action spaces, although a small literature on inverse MARL aims to computing inverse equilibria in discrete state and action space Markov games.}{}
%
\vspace{-2.5mm}
\paragraph{Contributions}    

% In this paper, we introduce a new method for solving the inverse game theory problem using generative learning. Our method is based on a generative game simulator. The generative game simulator is a game \amy{i don't think the GGS "IS" a game. i think it solves a game. we may need a name for the game it solves. in the abstract, i referred to it as a meta-game.} in which a learner chooses the parameters of the payoff functions, and a simulator computes the equilibrium strategy associated with the ensuing payoffs. The learner then uses the \samy{}{simulated} equilibrium to improve its estimate of the types.

The algorithms introduced in this paper extend the class of games for which an inverse Nash equilibrium can be computed efficiently (i.e., in polynomial-time) to the class of normal-form concave games (which includes normal-form finite action games), finite state and action Markov games, and a large class of continuous state and action Markov games. 
While our focus is on Markov games in this paper, the results apply to normal-form \citep{nash1950existence}, Bayesian \citep{harsanyi1967games, harsanyi1968bayesian}, and extensive-form games \citep{zermelo1913anwendung}.
The results also extend to other equilibrium concepts, beyond Nash, such as (coarse) correlated \cite{aumann1974subjectivity, moulin1978strategically}, and more generally, $\Phi$-equilibrium \citep{greenwald2003general} \emph{mutatis mutandis}.

% \sdeni{}{Our contributions are as follows.}

% \alec{I think contributions should be an itemized list. Much easier to read.}
%\amy{please have a go at this, Alec!!! thanks!} 
% \alec{Try to underscore much more succinctly and forcefully (i) the theoretical results are new; and (ii) what is innovative or weird/unusual about the experimental results and how they differ from capabilities of prior algorithms for inverse games}

% \deni{A word on equilibrium selection?}

%\deni{Add theorem numbers in following para when done.}\amy{don't need 'em}

First, regarding inverse multiagent planning, we provide a min-max characterization of the set of inverse Nash equilibria of any inverse game for which the set of inverse Nash equilibria is non-empty, assuming an exact oracle (\Cref{thm:inverse_NE}).
%, in spite of possibly stochastic equilibrium policies.
We then show that for any inverse concave game, when the regret of each player is convex in the parameters of the inverse game, an assumption satisfied by a large class of inverse games such as inverse normal-form games, this min-max optimization problem is convex-concave, and can thus be solved in polynomial time (\Cref{thm:concave_game_inverse_NE}) via standard first-order methods.
This characterization also shows that the set of inverse Nash equilibria can be convex, even when the set of Nash equilibria is not.

Second, we generalize our min-max characterization to inverse multiagent learning, in particular inverse MARL, where we are given an inverse Markov game, and correspondingly, a \emph{stochastic\/} oracle, and we seek a first-order simulacrum (\Cref{thm:inverse_stoch_NE}).
%that can only be accessed via \samy{}{a simulator that responds to queries with} sample \samy{queries}{trajectories} from the state-action history distribution.
% This setting covers inverse Nash equilibrium problems where the players' action might be noisy, as well as inverse multiagent reinforcement learning, when we are given a sample trajectory of play.  
We show that under standard assumptions, which are satisfied by a large class of inverse Markov games (e.g., all finite state and action Markov games and a class of continuous state and action Markov games), the ensuing min-max optimization problem is convex-gradient dominated, and thus an inverse Nash equilibrium can be computed once again via standard first-order methods in polynomial time (\Cref{thm:online_sgda}).

Third, we provide an extension of our min-max characterization to (second-order) simulacral learning (\Cref{thm:inverse_simulacrum}).
%, and we seek to compute a Nash simulacrum, i.e., parameters and associated Nash equilibrium policies, which replicate the observed state-action trajectories in expectation \sdeni{. 
We once again characterize the problem as a solution to a min-max optimization problem, for which standard first-order methods compute a first-order stationary \citep{lin2020gradient} solution in polynomial-time, using a number of observations (i.e., unfaithful samples of histories of play) that is polynomial in the size of the inverse simulation (\Cref{thm:apprenticeship_thm}).

\if 0
\amy{short version. DELETE if there is space for the full paragraph below!!!}
Finally, we run experiments and find that this algorithm outperforms the widely-used ARIMA method in predicting prices in Spanish electricity markets based on time-series data.
\fi

% \amy{will there be space for synthetic settings?} 

Finally, we include two sets of experiments.
In the first, we show that our method is effective in synthetic economic settings where the goal is to recover buyers' valuations from observed competitive equilibria (which, in this market, coincide with Nash equilibria).
Second, using real-world time-series data, we apply our method to predict prices in Spanish electricity markets, 
%%% SPACE
%by modeling the market as a Markov game, 
and find that it outperforms the widely-used ARIMA method in predicting prices on this real-world data set.

\begin{table}[t]
% \centering
\begin{table*}[h!]
\centering
\small
% \resizebox{2\columnwidth}{!}{
\begin{tabular}{@{}p{0.5cm}@{\hskip 1mm}@{}p{1.4cm}p{5.6cm}p{6.9cm}@{\hskip 2mm}c@{}}
\toprule
\textbf{1st} & \textbf{Intention} & \textbf{Definition} & \textbf{An example action} & \textbf{Prop.}\\
\midrule

% (1) Planning
 % \multirow{3}{*}{\textbf{Planning}} 
\parbox[t]{2mm}{\multirow{10}{*}{\rotatebox[origin=c]{90}{\textsc{\colorbox{planningcolor}{Planning}}}}}
&  Idea Generation & Formulate and record initial thoughts and concepts. & 
    writing keywords or notes (e.g., ``\textit{..\%[Comment out] main point: artifacts lack in human subjectivity..}'') & 7.0\% \\
\cmidrule(r){2-5}
& {Idea Organization} & Select the most useful materials and demarcate those generated ideas in a visually formatted way.  & Linking the generated ideas into a logical sequence and spacing out between ideas (e.g., ``..\% (1) need diff. stress testing...\%\%[Spacing] (2) exp. setup? '') & 0.5\% \\
\cmidrule(r){2-5}
& {Section Planning} & Initially create sections and sub-level structures. & Putting section-related LaTeX commands (e.g., \texttt{\textbackslash section}, \texttt{\textbackslash paragraph}) & 2.2\% \\
\toprule

% (2) Implementation
% \multirow{5}{*}{\textbf{Implementation}} 
\parbox[t]{2mm}{\multirow{15}{*}{\rotatebox[origin=c]{90}{\textsc{\colorbox{implementationcolor}{Implementation}}}}}
& {Text Production} & Translate their ideas into full languages, either from the writers' language or borrowed sentences from an external source. & Generating subsequent sentences with the author's own idea (e.g., ``... GPT-4 (OpenAI, 2023) explains the data ... Our approach is built on top of GPT-4...'') & 57.4\% \\
\cmidrule(r){2-5}
& {Object Insertion} & Insert visual claims of their arguments (e.g., figures, tables, equations, footnotes, lists) & e.g., \texttt{\textbackslash begin\{figure\}[h] \textbackslash centering \textbackslash includegraphics\{figure\_A.pdf\} \textbackslash end\{figure\}} & 4.6\% \\
\cmidrule(r){2-5}
& {Citation Integration} & Incorporate bibliographic references into a document and systematically link these references using citation commands. & Inserting a new BibTeX object in the bibliography file and adding the object name to an existing \texttt{\textbackslash cite\{\}} on the Related Work section & 1.7\% \\
\cmidrule(r){2-5}
& {Cross-reference} & Link different sections, figures, tables, or other elements within a document through referencing commands. & Putting a command \texttt{\textbackslash label\{figure-1\}} to a figure and referencing it in the main body by calling \texttt{\textbackslash ref\{figure-1\}} & 1.1\% \\
\cmidrule(r){2-5}
& {Macro Insertion} & Incorporate predefined commands or packages into a LaTeX document to alter its formatting. & Putting a \texttt{\textbackslash usepackage\{minted\}} for formatting a LLM prompt & 0.2\% \\
\toprule

% (3) Revision
\parbox[t]{2mm}{\multirow{20}{*}{\rotatebox[origin=c]{90}{\textsc{\colorbox{revisioncolor}{Revision}}}}}
& {Fluency} & Fix grammatical or syntactic errors in the text or LaTeX commands. & ``We desig{\textcolor{red}{\sout{ining}}}{\textcolor{teal}{ned}} several experiment setups for 
{\textcolor{teal}{the}} LLM evaluations as described in Figure \texttt{\textbackslash r{\textcolor{teal}{e}f}\{figure-A\}}.'' & 1.4\%  \\
\cmidrule(r){2-5} 
& {Coherence} & Logically link (1) any of the two or multiple sentences within the same paragraph; (2) any two subsequent paragraphs; or (3) objects to be consistent as a whole.  & ``Each comment was annotated by three different annotators \textcolor{red}{\sout{, which}} \textcolor{teal}{and we} achieved high inter-annotator agreement.'' & 3.3\% \\
\cmidrule(r){2-5} 
& {Clarity} & Improve the semantic relationships between texts to be more straightforward and concise. & ``..relevant studies have examined \textcolor{red}{\sout{one of the several textual styles}} \textcolor{teal}{one aspect of texts}, the formality, ....'' & 11.5\% \\
\cmidrule(r){2-5}
& {Structural} & Improve the flow of information by modifying the location of texts and objects. & ``We calculate Pearson's $r$ correlation for human alignment \textcolor{teal}{to compare the alignment between lexicon-based preferences and humans' original preferences.} First, we calculated the score from each human participant \textcolor{red}{\sout{to compare the alignment between lexicon-based preferences and humans' original preferences.}}'' & 3.7\% \\
\cmidrule(r){2-5}
& {Linguistic Style} & Modify texts with the writer’s writing preferences regarding styles and word choices, etc. & ``We \textcolor{red}{\sout{believe}} \textcolor{teal}{posit} that ...'' & 1.6\% \\
\cmidrule(r){2-5}
& {Scientific Accuracy} & Update or correct scientific evidence (e.g., numbers, equations) for more accurate claims. & ``..Pearson's $r$ correlation (\textcolor{red}{\sout{0.78}}\textcolor{teal}{0.68}; \textcolor{teal}{p < 0.01})'' & 0.7\% \\
\cmidrule(r){2-5}
& {Visual Formatting} & Modify the stylistic formatting of texts, objects, and citations & \texttt{\textbackslash cite} $\rightarrow$ \texttt{\textbackslash citet}, \texttt{\textbackslash textbf} $\rightarrow$ \texttt{\textbackslash textsc}, etc. & 3.2\% \\
\bottomrule
\end{tabular}
% }
\caption{The developed taxonomy of Scholarly Writing Process in \textsc{ScholaWrite}}
\label{table:taxonomy-full}
\end{table*}
\begin{subtable}[t]{0.5\textwidth}
    \centering
    \resizebox{\columnwidth}{!}{%
    \begin{tabular}{|l|l|l|c|}
        \hline
        Reference & Game Type & Solution Concept & \makecell{Polytime?} \\
        \hline
        \hline \citep{fu2021evaluating} & Finite Markov & Nash & \xmark \\
        \hline \citep{yu2019multi} & Finite Markov & Quantal Response & \xmark \\
        \hline \citep{lin2019multi} & Finite Zero-sum Markov & Various & \xmark\\
        \hline 
        \citep{song2018multi} & Finite Markov &
        Quantal Response & \xmark \\
        \hline  \citep{syrgkanis2017inference} & Finite Bayesian & Bayes-Nash & \cmark \\
        \hline \citep{kuleshov2015inverse} & Finite Normal-Form & Correlated & \cmark \\
        \hline \citep{waugh2013computational} & Finite Normal-Form & Correlated &  \cmark \\
        \hline \citep{bestick2013inverse} & Finite Normal-Form & Correlated & \xmark \\  
        \hline \citep{natarajan2010multi} & Finite Markov & Cooperative & \xmark \\
        \hline \rowcolor{orange!50} This work & \makecell[l]{Finite/Concave Normal-form\\ Finite/Concave Markov} & 
        % \makecell[l]{Nash /Any}
        \makecell[l]{Nash/Correlated\\ Any Other \quad \quad \quad \quad }
        & \cmark \\
        \hline
    \end{tabular}
    }
    \caption{A comparison of our work and prior work on inverse game theory and inverse MARL.}
    \label{tab:summary_lit}
    \vspace{-2em}
\end{subtable}

\end{table}
