\section{Computing Competitive Equilibria in Arrow-Debreu Markets}
\label{sec:arrow_debreu}

\deni{Explain why do you think this was hard.}

A \mydef{Fisher market} $(\numbuyers, \numgoods, \valuation, \budget)$, $(\valuation, \budget)$ when clear from context, consists of $\numbuyers$ buyers and $\numgoods$ divisible goods with unit supply.
Each buyer $\buyer \in \buyers$ has a budget $\budget[\buyer] \in \mathbb{R}_{+}$ and a utility function $\valuation[\buyer]: \consumptions[\buyer] \to \mathbb{R}_+$, where $\consumptions[\buyer] \subseteq \mathbb{R}^{\numgoods}_+$ is buyer $\buyer$'s consumption space \cite{brainard2000compute}.
We denote
%%% SPACE
%the set of 
joint consumptions by $\consumptions = \bigtimes_{\buyer \in \buyers} \consumptions[\buyer]$.

A \mydef{Fisher market competitive equilibrium (FMCE)} is a tuple comprising allocations $\allocation \in \consumptions$ and prices $\price \in \simplex[\numgoods]$ s.t.\ 1.~all buyers $\buyer \in \buyers$ maximize utility constrained by their budget: $\allocation[\buyer]^* \in \argmax_{\allocation[\buyer] \in \consumptions[\buyer]: \allocation[\buyer] \cdot \price \leq \budget[\buyer]} \valuation[\buyer](\allocation[\buyer])$, 2.~the markets clear, meaning the allocation is feasible and Walras' law holds, i.e., $ \sum_{\buyer \in \buyers} \allocation[\buyer]^*  \leq \ones[\numgoods]$ and $\price^* \cdot \left( \sum_{\buyer \in \buyers} \allocation[\buyer]^*  - \ones[\numgoods] \right) = 0$. 
% \amy{i thought we decided that the first condition was feasbility, and the second, Walras' law.}

% Without loss of generality, we assume that there is one divisible unit of each good in the market \cite{AGT-book}.

% The seller's utility $\valuation[0]: \mathbb{R}_{+}^{\numgoods} \to \mathbb{R}_+$ is given by $\valuation[0](\allocation[0]) = \allocation[0][\numbuyers+1]$, while the 

%\amy{i am not sure where this next paragraph is coming from. there is no reference? and there is no proof. is this something we have already shown, b/c i don't recall us talking about FMCE as Nash eqa, but only as SE.}

A function $f: \R^m \to \R$ is called \mydef{homogeneous of degree $k$} if $\forall \allocation[ ] \in \R^m, \lambda > 0, f(\lambda \allocation[ ]) = \lambda^k f(\allocation[ ])$.
The set of FMCE of any Fisher market $(\valuation, \budget)$ with continuous, concave, and homogeneous utility functions is equal to the set of Nash equilibria of the \mydef{Eisenberg-Gale min-max game},%
\footnote{This min-max game corresponds to the Lagrangian saddle-point formulation of the Eisenberg-Gale program \cite{gale1989theory, jain2005market}.} 
a convex-concave min-max game between a seller who chooses prices $\price \in \simplex[\numgoods]$ and buyers who collectively choose allocations %\amy{consumptions?} 
$\allocation \in \consumptions$, the joint consumption space. 
The objective function of this game comprises two sums: the first is the logarithmic Nash social welfare of the buyers' utility per budget, i.e., bang-per-buck, while the second is the profit of a fictional auctioneer who sells the goods in the market:
% \amy{this formula was not compiling. it was incomplete. it now compiles, but please complete. thanks.}
%
%\begin{align}
%\label{eq:fisher_market_game}
    $\min_{\price \in \simplex[\numgoods]} \max_{\allocation \in \consumptions} \obj (\price, \allocation; \budget) \doteq \sum_{\buyer \in \buyers} \budget[\buyer] \log \left( \valuation[\buyer](\allocation[\buyer]) \right) + \sum_{\good \in \goods} \left( \price[\good] - \price[\good] 
    \sum_{\buyer \in \buyers}\allocation[\buyer][\good] \right)$.
%\end{align}
%
% Note that for all $\price \in \simplex[\numgoods]$, $\obj (\price, \allocation; \budget)$ is jointly concave in $(\allocation, \budget)$.
% To see this, observe that the first sum in $\obj$, which depends on $(\allocation, \budget)$, corresponds to the negated KL-divergence between the budgets and the utilities of the buyers.
% Since the KL divergence is convex in both arguments \cite{boyd2004convex}, and the KL-divergence is a non-decreasing function in its second argument, the KL divergence between the budgets and the utilities of the buyers is convex in $(\allocation, \budget)$.
% Additionally, for any $(\allocation, \budget) \in \consumptions \times \simplex[\numbuyers]$, $\obj (\price, \allocation; \budget)$ is affine, i.e., convex, \amy{why is affine the same as convex here?} in $\price$.

An \mydef{exchange economy} $(\numbuyers, \numgoods, \valuation, \consendow)$,  $(\valuation, \consendow)$ when clear from context, consists of a finite set of $\numgoods \in \N_+$ goods and $\numbuyers \in \N_+$ consumers (or traders, or buyers).
Each consumer $\buyer \in \buyers$ has a set of possible consumptions $\consumptions[\buyer] \subseteq \mathbb{R}^{\numgoods}_+$, an endowment of divisible goods $\consendow[\buyer] = \left(\consendow[\buyer][1], \dots, \consendow[\buyer][\numgoods] \right) \in \R^\numgoods$, and a utility function $\valuation[\buyer]: \mathbb{R}^{\numgoods } \to \mathbb{R}$.
We define $\consendow \doteq \left(\consendow[1], \hdots, \consendow[\numbuyers] \right)^T$, and assume without loss of generality that there is one unit of supply of each good, i.e., $\sum_{\buyer \in \buyers} \consendow[\buyer] = \ones[\numgoods]$. 
% An exchange market $(\valuation, \budget)$ is said to be \mydef{homothetic} if, for all buyers $\buyer \in \buyers$, $\valuation[\buyer]$ is a continuous and homogeneous of degree 1, i.e., for all $\lambda \in \R_+$ $\valuation[\buyer](\lambda \allocation[\buyer]) = \lambda \valuation[\buyer](\allocation[\buyer])$\footnote{Without loss of generality, complete, transitive, continuous, and homothetic preference relations can be represented via a homogeneous utility function of degree 1, since any homogeneous utility function of degree $k$ can be made homogeneous of degree 1 without affecting the underlying preference relation by passing the utility function through the monotonic transformation $x \mapsto \sqrt[k]{x}$.}.

% \deni{To add somewhere in a non-confusing way, rn I think it is confusing.}
% \sdeni{}{A Fisher economy $(\valuation, \budget)$ is an exchange market $(\numbuyers + 1, \numgoods + 1, \valuation[][\prime], \consendow)$ where the first consumer called the seller is endowed with the first $\numgoods$ goods, i.e., $\forall \good \in \goods$, $\consendow[1][\good] = 1$, and gets utility $\valuation[1][\prime](\allocation[1]) = \allocation[1,][\numgoods + 1]$, while the $\numbuyers$ last consumers called the buyers are only endowed with good $\numbuyers + 1$ such that for all $\buyer \in [1, \numbuyers + 1]$, $\consendow[\buyer][\numgoods + 1] = \budget[\buyer]$ and have utility functions $\valuation[\buyer][\prime](\allocation[\buyer]) = \valuation[\buyer](\allocation[\buyer, -\numgoods + 1])$.}

An \mydef{Arrow-Debreu competitive equilibrium (ADCE)} is a tuple comprising allocations $\allocation \in \R_+^{\numgoods \times \numbuyers}$ and prices $\price \in \simplex[\numgoods]$ s.t.\ 1.~all traders $\buyer \in \buyers$ maximize utility constrained by the value of their endowment: $\allocation[\buyer]^* \in \argmax_{\allocation[\buyer] \in \consumptions[\buyer]: \allocation[\buyer] \cdot \price \leq \consendow[\buyer] \cdot \price} \valuation[\buyer](\allocation[\buyer])$, 2.~the markets clear, meaning 
% \amy{same edits as above!: feasibility and Walras' law}
the allocation is feasible and Walras' law holds, i.e., $\sum_{\buyer \in \buyers} \allocation[\buyer]^*  \leq \ones[\numgoods]$ and $\price^* \cdot \left( \sum_{\buyer \in \buyers} \allocation[\buyer]^*  - \ones[\numgoods] \right) = 0$.

%In the rest of this section, we restrict our attention to exchange markets with homogeneous utility functions for which an ADCE \samy{}{necessarily} exists. 
Our main result concerns \mydef{homothetic} exchange economies, namely those in which utilities functions are homogeneous.
Modulo the homogeneity assumption, the following assumptions are identical to those introduced by \citeauthor{arrow-debreu} \cite{arrow-debreu} to establish the existence of ADCE in exchange economies.
We adopt these assumptions in the sequel.
% \citeauthor{arrow-debreu} also introduce alternative assumptions; our results hold under those assumptions as well. 

\begin{assumption}[Homogeneous Market Assumption]
\label{assum:ad_exist}
Given exchange economy $(\util, \consendow)$, for all consumers $\buyer \in \buyers$, assume 1.~$\util[\buyer]$ is continuous, concave,%
%%% SPACE
%\footnote{Technically speaking, \citeauthor{arrow-debreu} \cite{arrow-debreu} assume payoffs are quasi-concave, a generalization of concavity. Nonetheless, for ease of presentation, we state our results under the stronger assumption of concavity.} 
locally non-satiated, and homogeneous; 2.~$\consumptions[\buyer]$ is non-empty, compact, and convex; and 3.~there exists $\consumption[\buyer] \in \consumptions[\buyer]$, such that $\consumption[\buyer] < \consendow[\buyer]$.
\end{assumption}

%\amy{note to self: existence before observation, so that observation is not vacuous}

The following lemma characterizes a key relationship between the competitive equilibria of Arrow-Debreu exchange economies and those of a particular Fisher market.

\begin{observation}
\label{obs:ad_fisher_eq_equiv}
    $(\allocation^*, \price^*)$ is an %competitive equilibrium 
    ADCE of an exchange economy $(\valuation, \consendow)$ iff it is an FMCE
    %competitive equilibrium
    of the Fisher market $(\valuation, \consendow \price^*)$.
\end{observation}

\vspace{-2.5mm}
%\begin{proof}
\emph{Proof.}
$(\allocation^*, \price^*)$ is an ADCE
%competitive equilibrium 
of the exchange market $(\valuation, \consendow)$ iff
%    \begin{align}
        for all $\buyer \in \buyers$, $\allocation[\buyer]^* \in  \argmax_{\allocation[\buyer] \in \consumptions[\buyer]: \allocation[\buyer] \cdot \price^* \leq \consendow[\buyer] \cdot \price^*} \valuation[\buyer](\allocation[\buyer])$;
        $\sum_{\buyer = 1}^{\numagents} \allocation[\buyer]^* \leq \ones[\numgoods]$; and 
        $\price^* \cdot \left( \sum_{\buyer = 1}^{\numagents} \allocation[\buyer]^*  - \ones[\numgoods] \right) = 0$
%    \end{align} 
    iff $(\allocation^*, \price^*)$ is an FMCE
    %competitive equilibrium 
    of the Fisher market $(\valuation, \consendow \price^*)$.
%\end{proof}

\if 0
\begin{proof}
    $(\Rightarrow)$
    If $(\allocation^*, \price^*)$ is a competitive equilibrium of the Arrow-Debreu exchange market $(\valuation, \consendow)$, then
    \begin{align}
        \allocation[\buyer]^* \in  \argmax_{\allocation[\buyer] \in \consumptions[\buyer]: \allocation[\buyer] \cdot \price^* \leq \consendow[\buyer] \cdot \price^*} \valuation[\buyer](\allocation[\buyer]), \forall \buyer \in \buyers,
        \qquad
        \sum_{\buyer = 1}^{\numagents} \allocation[\buyer]^*  \leq \ones[\numgoods],
        \qquad
        \price^* \cdot \left( \sum_{\buyer = 1}^{\numagents} \allocation[\buyer]^*  - \ones[\numgoods] \right) = 0
    \end{align} 
    which correspond to the competitive equilibrium conditions for the Fisher market $(\valuation, \consendow \price^*)$.
    % Similarly, suppose that 
    % However,
    % % \fi 
    % Under \Cref{assum:ad_exist}, the utilities of the consumers are locally non-satiated, which implies that they \sdeni{consume}{spend} \sdeni{their entire budget}{the entire value of their endowment}.
    % As a result, for $\allocation^*$ to be utility maximizing at prices $\price^*$, $\allocation^* \cdot \price^* = \consendow^* \cdot \price^*$ (see, for instance, \cite{mas-colell}). 
    % % \amy{what is $\budget^*$ in this proof? it seems to be unquantified. i guess it will lead to the magic of the leader's objective in the CP. but it has not been adequately explained/motivated here, yet.}
    % Since $(\allocation^*(\budget^*), \price^*(\budget^*))$ is a competitive equilibrium of $(\valuation, \consendow)$, it holds that allocations $\allocation^*(\budget^*)$  are utility maximizing at budgets $\consendow\price^*(\budget^*)$.
    % Hence, by local non-satiation, for all consumers $\buyer \in \buyers$, $\allocation[\buyer]^* \cdot \price^* = \consendow[\buyer] \cdot \price^*(\budget^*)$.
    % Combining these two budget constraints yields \samy{$\budget^* = \consendow \price^*(\budget^*)$}{$\budget^* = \consendow \price^*$}.
    % Thus $(\allocation^*, \price^*)$ is a competitive equilibrium of the Fisher market $(\valuation, \consendow \price^*)$.
    
    $(\Leftarrow)$
    If $(\allocation^*, \price^*)$ is a competitive equilibrium of the Fisher market $(\valuation, \consendow \price^*)$, then
    \begin{align}
    \allocation[\buyer]^* \in \argmax_{\allocation[\buyer] \in \consumptions[\buyer]: \allocation[\buyer] \cdot \price^* \leq \consendow[\buyer] \cdot \price^*} \valuation[\buyer](\allocation[\buyer]), \forall \buyer \in \buyers, \qquad
    \sum_{\buyer \in \buyers} \allocation[\buyer]^* \leq \ones[\numgoods], \qquad
    \price^* \cdot \left( \sum_{\buyer = 1}^{\numagents} \allocation[\buyer]^*  - \ones[\numgoods] \right) = 0
    \enspace ,
    \end{align}
    which correspond to the competitive equilibrium conditions for the Arrow-Debreu economy $(\valuation, \consendow)$.
\end{proof}
\fi

% Define $\obj (\price, \allocation; \budget) = \sum_{\buyer \in \buyers} \budget[\buyer] \log(\frac{\valuation[\buyer](\allocation[\buyer])}{\budget[\buyer]}) + \sum_{\good \in \goods}  \left( \price[\good] - \price[\good] 
    % \sum_{\buyer \in \buyers} \allocation[\buyer][\good] \right)$ which corresponds to the objective of the Fisher market min-max game.

\begin{corollary}
\label{cor:ad_fixed_point}
To find ADCE prices of an exchange economy $(\valuation, \consendow)$, it suffices to find budgets $\budget^* \in \simplex[\numbuyers]$ s.t.\ $\budget^* = \consendow \price^*$ and $\price^*$ are FMCE prices of the Fisher market $(\valuation, \budget^*)$.
\end{corollary}

\if 0

Define the operator $\budgetspace (\budget) \doteq \{\consendow \price \mid \price \in \argmin_{\price \in \simplex[\numgoods]} \max_{\allocation \in \consumptions} \obj (\price, \allocation; \budget) \}$.
\amy{the operator solves for FMCE prices, and then multiplies them by endowments, outputs these as consumers' budgets at these eqm prices.}
\amy{learning dynamic: given budgets (like income), solve for eqm = spend income, observes actual income Ep, adjusts next time period income accordingly.}

Given an exchange economy $(\valuation, \consendow)$, prices $\price^*$ for which the value of the buyers' endowments $\budget^* \doteq \consendow \price^*$ are fixed points of $\budgetspace$, i.e., $\budget^* \in \budgetspace (\budget^*)$, are ADCE prices of $(\valuation, \consendow)$.
By \Cref{obs:ad_fisher_eq_equiv}, such prices are also FMCE prices of $(\valuation, \consendow \price^*)$.
Consequently, if we can find budgets $\budget^* \in \simplex[\numbuyers]$ such that $\budget^* = \consendow \price^*$ and $\price^*$ are FMCE prices of $(\valuation, \budget^*)$, then $\consendow \price^* \in \budgetspace (\consendow \price^*)$, meaning we have also found ADCE prices of $(\valuation, \consendow)$.

\fi

%\deni{This should go in the related works section imo}\amy{i don't agree. it could be repeated in the intro, if you want. but it's purpose here is to motivate/explain the direction we are going with all these definitions. telling the reader why they should be paying attention.}

Prior work has proposed first-order methods to solve Fisher markets by defining a Stackelberg game whose equilibria correspond to FMCE \cite{goktas2021minmax}.
We employ a similar strategy: we solve exchange economies by defining a Stackelberg-Nash game whose SSNE correspond to ADCE.
We call this Stackelberg-Nash game the budget-balance game, and note that it can represent any exchange economy $(\valuation, \consendow)$ with continuous, concave, and homogeneous utility functions.

In the budget-balance game, the leader chooses budgets for the followers, who then choose prices and allocations simultaneously \amy{why did this become a Nash game, instead of a Stackelberg game?} in the Eisenberg-Gale min-max game determined by these budgets.
%%% SPACE
%(\Cref{eq:fisher_market_game}).
The lower-level game is thus a Fisher market, whose set of Nash equilibria is precisely the set of FMCE.
%(Theorem 2, \citeauthor{goktas2022cch} (\citeyear{goktas2022cch})).
The leader's objective is to choose budgets for which $\budget^* = \consendow^* \price$, which by \Cref{cor:ad_fixed_point} yield a competitive equilibrium in $(\valuation, \consendow)$. 

%\amy{want to find $\budget^*$ s.t.\ $\budget^* = \consendow \price^*(\budget^*)$.}

\begin{definition}[Budget-Balance Game]\label{def:budget_balance}
Given an exchange economy $(\valuation, \consendow)$, a \mydef{budget-balance game} is a Stackelberg-Nash game $(3, \outerset, \innerset, \util)$ with one leader and two followers, with leader action space $\outerset \doteq \simplex[\numbuyers]$, and action spaces $\innerset[1] \doteq \simplex[\numgoods]$ and $\innerset[2] \doteq \consumptions$ for the two followers. 
In this game, the leader first chooses a budget $\budget \in \simplex[\numbuyers]$ for each of the followers, after which the first follower chooses prices $\price \in \simplex[\numgoods]$ and the second follower chooses allocations $\allocation \in \consumptions$, simultaneously.
%After all players choose their actions, the players receive the following payoffs:
The payoffs are:
%%% SPACE
%as follows:
%\begin{align}
$\util[0](\budget, \price, \allocation) = -\frac{1}{2}\left\| \budget - \consendow \price\right\|_2^2$,
$\util[1](\budget, \price, \allocation) = - \obj (\budget, \price, \allocation)$, and 
$\util[2](\budget, \price, \allocation) = \obj (\budget, \price, \allocation)$.
%\end{align}
\end{definition}

% \begin{theorem}
\begin{restatable}{theorem}{thmadssneequiv}
\label{thm:ad_ssne_equiv}
Given an exchange economy $(\valuation, \consendow)$ that satisfies Assum.~\ref{assum:ad_exist},
% }{} \amy{can drop, i think, if we point out somewhere earlier that we've adopted this assumption in what follows.}
$(\budget^*, \price^*, \allocation^*)$ is an SSNE of the budget-balance game $(3, \outerset, \innerset, \util)$ associated with $(\valuation, \consendow)$ iff $(\allocation^*, \price^*)$ is an ADCE of $(\valuation, \consendow)$. 
% \end{theorem}
\end{restatable}
\if 0
%%% APPENDIX
\begin{proof}
    $(\Rightarrow):$ Assume $(\budget^*, \allocation^*, \price^*)$ is a SSNE of the budget-balance game $(3, \outerset, \innerset, \util)$ associated with the homothetic exchange economy $(\valuation, \consendow)$. 
    Then, $(\allocation^*, \price^*)$ is a FMCE of the homothetic Fisher market $(\valuation, \budget^*)$, since $(\allocation^*, \price^*)$ is a Nash equilibrium of the associated Fisher market min-max game. 
    
    % \samy{By way of contradiction, assume $\budget^* \neq \consendow \price^*$.}{} \amy{i'm not quite seeing how this contradicts $(\allocation^*, \price^*)$ being an ADCE of $(\valuation, \consendow)$. but i am not sure we need this, anyway.}
    
    Under \Cref{assum:ad_exist}, an ADCE $(\allocation^\prime, \price^\prime)$ exists.
    By \Cref{obs:ad_fisher_eq_equiv}, $(\allocation^\prime, \price^\prime)$ is a FMCE of $(\valuation, \consendow \price^\prime)$, and hence a Nash equilibrium of the associated Eisenberg-Game min-max game at budgets $\budget^\prime = \consendow \price^\prime$.
    Now, setting $\budget^\prime = \consendow \price^\prime$, it follows that $(\budget^\prime, \allocation^\prime, \price^\prime)$ is a SSNE of $(3, \outerset, \innerset, \util)$, because the leader's payoff at this SSNE is $0$, which is the lower bound of its value.
    %\amy{this part of the argument is the same as in the other direction.} 
    
    However, if the leader's payoff at any SSNE is equal to 0, then it must be so at any other SSNE, since the equilibrium payoff of the leader is unique. That is, we must have that that $\budget^* = \consendow \price^*$. Since $(\allocation^*, \price^*)$ is a FMCE of $(\valuation, \consendow \price^*)$, by \Cref{obs:ad_fisher_eq_equiv}, $(\allocation^*, \price^*)$ is likewise an ADCE of $(\valuation, \consendow)$.

    $(\Leftarrow)$: Assume $(\allocation^*, \price^*)$ is an ADCE of the homothetic exchange economy $(\valuation, \consendow)$. Then, by \Cref{obs:ad_fisher_eq_equiv}, $(\allocation^*, \price^*)$ is a FMCE of the Fisher market $(\valuation, \consendow \price^*)$. Hence, $(\allocation^*, \price^*)$ is a Nash equilibrium of the associated Eisenberg-Gale min-max game at budgets $\budget^* \doteq \consendow \price^*$. However, at budgets $\budget^* = \consendow \price^*$, the payoff of the leader is minimized in the budget-balance game $(3, \outerset, \innerset, \util)$, i.e., its value is the lower bound, namely 0. As the leader's action is optimal, and the followers' actions comprise a Nash equilibrium, $(\budget^*, \allocation^*, \price^*)$ is a SSNE of the budget-balance game.
\end{proof}
\fi

\amy{add motivating text: e.g.,}

Having reduced ADCE in homothetic exchange economies to SSNE in budget-balance games, a class of Stackelberg-Nash games, it remains to apply our method 
%(\Cref{alg:npam}} 
to solving these games. 
\sdeni{}{However, 
%as mentioned previously, 
falling short of the leader's action being affine in the followers' payoff, the exploitability is not guaranteed to be convex in all players' actions in the convex-concave min-max lower-level game.}
%%% FIX ME !!!
\amy{add more about: however, convexity, weak convexity, etc., to motivate the rest of this section?!}

Given a budget-balance Stackelberg-Nash game together with the leader's choice of budgets $\budget$, the follower cumulative regret of the lower-level Eisenberg-Gale min-max game is given by: $\cumulregret[\budget] ((\price, \allocation), (\otherprice, \otherallocation); \budget) \doteq \obj (\otherprice, \otherallocation; \budget) - \obj (\price, \allocation; \budget)$.
\amy{i added a $\budget$ subscript to $\cumulregret$. i think you need it for consistency.}\deni{Consistency with what? the regularized cumulative regret? Becuse this is now in conflict with previous section's notation.}
\if 0
\begin{align}
&\obj (\otherprice, \otherallocation; \budget) - \obj (\price, \allocation; \budget) \\
=& \sum_{\buyer \in \buyers} \budget[\buyer] \log \left( \valuation[\buyer](\otherallocation[\buyer]) \right) + \sum_{\good \in \goods} \left( \otherprice[\good] - \otherprice[\good] 
    \sum_{\buyer \in \buyers} \otherallocation[\buyer][\good] \right) - \sum_{\buyer \in \buyers} \budget[\buyer] \log \left( \valuation[\buyer](\allocation[\buyer]) \right) + \sum_{\good \in \goods} \left( \price[\good] - \price[\good] 
    \sum_{\buyer \in \buyers}\allocation[\buyer][\good] \right)
\end{align}
\fi
Under \Cref{assum:ad_exist}, this cumulative regret is concave in $(\otherprice, \otherallocation)$ and convex in $(\price, \allocation)$, but it is not convex in $(\budget, \price, \allocation)$. 
However, if $\consumptions$ is bounded away from zero, then $\obj$ is twice continuously differentiable, in which case, the leader-regularized cumulative regret $\regulcumulregret[\budget]$ is Lipschitz-smooth over $\simplex[\numgoods] \times \consumptions \times \simplex[\numgoods] \times \consumptions$,
%\amy{good use of ``over''. clarifying.} 
and further, by \Cref{lemma:reg_expl_prop}, the leader-regularized exploitability $\regulexploit[\budget]$ is convex in $(\budget, \price, \allocation)$. 

Nonetheless, 
%%% FIX ME !!!
as mentioned previously, \amy{please remind me where? thanks.}\deni{Line 401.}\amy{line numbers changed before i saw this!}
using an entropy rather than Euclidean regularizer can make it easier to achieve convexity in $(\budget, \price, \allocation)$.
%%% FIX ME !!!
\amy{but we already achieved convexity, so i am not sure i follow?}
We thus define the leader-regularized cumulative regret as the cumulative regret with an additional entropy term for the budgets:
%%% FIX ME !!!
\amy{why do you add the regularization term only once? that is, why do you add it to $\regulcumulregret[\budget]((\price, \allocation), (\otherprice, \otherallocation); \budget)$ instead of to $\obj (\otherprice, \otherallocation; \budget)$ and $\obj (\price, \allocation; \budget)$? if you added it to $\obj$, you would have what i've written below in red, i think. but this term is not negated, which may be why you cannot do it this way?}\deni{The regularization is added to the cumulative regret not the payoffs, if we add it to the payoffs the regularization terms cancel out for the cumulative regret.} \amy{i'm not seeing that. you could incorporate each of the regularization into each of the $\val[\buyer]$ terms, rather than cancel them out.}
$\regulcumulregret[\budget]((\price, \allocation), (\otherprice, \otherallocation); \budget) \doteq \cumulregret[\budget]((\price, \allocation), (\otherprice, \otherallocation); \budget) + \sum_{\buyer \in \buyers} \budget[\buyer] \log(\budget[\buyer]) =$
%\begin{align}
    $\sum_{\buyer \in \buyers} \budget[\buyer] \log \left( \valuation[\buyer](\otherallocation[\buyer]) \right) + \sum_{\good \in \goods} \left( \otherprice[\good] - \otherprice[\good] 
    \sum_{\buyer \in \buyers} \otherallocation[\buyer][\good] \right) - \sum_{\buyer \in \buyers} \budget[\buyer] \log \left( \frac{\valuation[\buyer](\allocation[\buyer])}{\budget[\buyer]} \right) + \sum_{\good \in \goods} \left( \price[\good] - \price[\good] 
    \sum_{\buyer \in \buyers}\allocation[\buyer][\good] \right)$.
    %%% FIX ME !!!
    \amy{$\samy{}{\sum_{\buyer \in \buyers} \budget[\buyer] \log \left( \frac{\valuation[\buyer](\otherallocation[\buyer])}{\budget[\buyer]} \right) + \sum_{\good \in \goods} \left( \otherprice[\good] - \otherprice[\good] \sum_{\buyer \in \buyers} \otherallocation[\buyer][\good] \right) - \sum_{\buyer \in \buyers} \budget[\buyer] \log \left( \frac{\valuation[\buyer](\allocation[\buyer])}{\budget[\buyer]} \right) + \sum_{\good \in \goods} \left( \price[\good] - \price[\good] \sum_{\buyer \in \buyers}\allocation[\buyer][\good] \right)}$}
%\end{align}

Under \Cref{assum:ad_exist}, the leader-regularized cumulative regret $\regulcumulregret[\budget]$ is 
%%% FIX ME !!!
convex in $(\budget, \price, \allocation)$, for all $(\otherprice, \otherallocation) \in \simplex[\numgoods] \times \consumptions$. 
The convexity of the first two and the last terms is straightforward, since these terms are linear, fixing $(\budget, \price, \allocation)$.
%%% FIX ME !!!
\amy{i'm not seeing this. let's slow down. (too much info in one sentence.) last term is fine. it is a constant if $(\budget, \price, \allocation)$ is fixed. second term? not sure. first term, linear in $\budget$, but not in $\otherallocation$.}
To see the convexity of the third term, observe that it corresponds to the negated KL-divergence between the budgets and the utilities of the buyers, which is concave in both arguments; since the negated KL divergence is non-decreasing
%%% FIX ME !!!
% \amy{non-decreasing?} 
in its second argument, namely $\allocation[\buyer]$, its composition with a concave utility function is concave; hence the negation of this term is convex in $(\budget, \allocation)$.
% Since the KL divergence is convex in both arguments \cite{boyd2004convex}, and the KL-divergence is a non-decreasing function in its second argument, the KL divergence between the budgets and the utilities of the buyers is convex in $(\allocation, \budget)$.
%
The following corollary now follows from \Cref{thm:ad_ssne_equiv}.

%%% SPACE
%using this entropy-regularized definition of leader cumulative regret.

% \begin{corollary}
\begin{restatable}{corollary}{corrconvexad}
    There exists an $\varepsilon \geq \log(\numbuyers)$ \amy{is this the sense in which we are approximate?} s.t.\ the solutions to the following convex program correspond to the ADCE of $(\valuation, \consendow)$:
\if 0
    %\begin{align}
        $\min_{(\budget, \price, \allocation) \in \simplex[\numbuyers] \times \simplex[\numgoods] \times \consumptions,  \max\limits_{\otherprice, \otherallocation \in \consumptions} \regulcumulregret[\budget]((\price, \allocation), (\otherprice, \otherallocation); \budget) \leq \varepsilon} \frac{1}{2}\left\| \budget - \consendow \price\right\|_2^2$,
        %\enspace ,
    %\end{align}
\fi
    %\begin{align}
        $\min_{(\budget, \price, \allocation) \in \simplex[\numbuyers] \times \simplex[\numgoods] \times \consumptions, \regulexploit (\allocation, \price; \budget) \le \varepsilon} \frac{1}{2} \left\| \budget - \consendow \price\right\|_2^2$, 
        %\enspace ,
    %\end{align}    
where $\regulexploit (\allocation, \price; \budget) \doteq \max\limits_{\otherprice, \otherallocation \in \consumptions} \regulcumulregret[\budget]((\price, \allocation), (\otherprice, \otherallocation); \budget)$.
% \end{corollary}
\end{restatable}
    
\if 0
%%% APPENDIX
\begin{proof}
    By \Cref{obs:ssne_max}, we can express the SSNE of the budget-balance game as the following maximization problem, where $\max\limits_{\allocation^\prime \in \consumptions} \obj (\price, \allocation^\prime, \budget) - \min\limits_{\price^\prime \in \simplex[\numgoods]} \obj (\price^\prime, \allocation, \budget)$ describes the \samy{}{follower} exploitability in the lower-level Eisenberg-Gale min-max game:\amy{i think the first of these should be a maximization problem, not a min}
%
\begin{align}
\label{eq:final convex program}
    \min_{\substack{(\budget, \price, \allocation) \in \simplex[\numbuyers] \times \simplex[\numgoods] \times \consumptions\\ \max\limits_{\otherprice, \otherallocation \in \consumptions} \regulcumulregret[\budget]((\price, \allocation), (\otherprice, \otherallocation); \budget) \leq \varepsilon}} - \frac{1}{2}\left\| \budget - \consendow \price\right\|_2^2
    &= \min_{\substack{(\budget, \price, \allocation) \in \simplex[\numbuyers] \times \simplex[\numgoods] \times \consumptions\\ \max\limits_{\otherprice, \otherallocation \in \consumptions} \regulcumulregret[\budget]((\price, \allocation), (\otherprice, \otherallocation); \budget) \leq \varepsilon}} \frac{1}{2}\left\| \budget - \consendow \price\right\|_2^2
\end{align}

\sdeni{}{\Cref{eq:final convex program} is a convex optimization problem with convex constraints.
In particular, the \samy{}{follower} exploitability, i.e., $\regulexploit (\allocation, \price; \budget) \doteq \max\limits_{\otherprice, \otherallocation \in \consumptions} \regulcumulregret[\budget]((\price, \allocation), (\otherprice, \otherallocation); \budget)$, is convex in $(\budget, \allocation, \price)$. To see this, simply notice that $\obj$ is convex } \amy{you trail off...}
\end{proof}
\fi

% \begin{lemma}\label{lemma:fixed_point_ad}
% % and suppose that \Cref{assum:ad_exist} holds 
% % Let $(3, \outerset, \innerset, \util)$ be the budget balance game associated with $(\numbuyers, \numgoods, \valuation, \consendow)$. 
% For any budget $\budget \in \simplex$, let $(\allocation^*(\budget), \price^*(\budget))$ be the competitive equilibrium of the homothetic Fisher market $(\valuation, \budget), i.e., \price^*(\budget) \in  \min\limits_{\price \in \simplex[\numgoods]} \max\limits_{\allocation \in \consumptions} \obj (\price, \allocation, \budget)$ and $\allocation^*(\budget) \in \argmax\limits_{\allocation \in \consumptions} \obj (\price^*(\budget), \allocation, \budget)$. 
% Consider a homothetic exchange economy $(\valuation, \consendow)$ and suppose that \Cref{assum:ad_exist} holds. Then, $(\price^*(\budget^*), \allocation^*(\budget^*))$ is a competitive equilibrium of the Arrow-Debreu economy $(\valuation, \consendow)$ iff $\budget^* = \consendow \price^*(\budget^*)$.
% \end{lemma}
% \begin{proof}
%     $(\Rightarrow)$
%     Since $(\allocation^*(\budget^*), \price^*(\budget^*))$ is a competitive equilibrium of the Fisher market $(\valuation, \budget^*)$ then allocations $\allocation^*(\budget^*)$  are utility maximizing at budgets $\budget^*$, i.e., for all buyers $\buyer \in \buyers$:
%     \begin{align}
%         \allocation[\buyer]^*(\budget^*) \in  \argmax_{\allocation[\buyer] \in \consumptions[\buyer]: \allocation[\buyer] \cdot \price^*(\budget^*) \leq \budget^*} \valuation[\buyer](\allocation[\buyer])
%     \end{align}
%     However, under \Cref{assum:ad_exist}, the utilities of the buyers are locally non-satiated. As a result, for $\allocation^*$ to be utility maximizing, we must have that $\allocation[\buyer]^*(\budget^*) \cdot \price^*(\budget^*) = \budget^*$ (see for instance \cite{mas-colell}).
%     Similarly, since $(\allocation^*(\budget^*), \price^*(\budget^*))$ is a competitive equilibrium of $(\numbuyers, \numgoods, \valuation, \consendow)$, we also have that allocations $\allocation^*(\budget^*)$  are utility maximizing at budgets $\consendow\price^*(\budget^*)$. Hence, by local non-satiation, we must have that for all consumers $\buyer \in \buyers$, $\allocation[\buyer]^*(\budget^*) \cdot \price^*(\budget^*) = \consendow[\buyer] \cdot \price^*(\budget^*)$. Combining the two budget constraint equations for all buyers, we then obtain $\budget^* = \consendow \price^*(\budget^*)$.
    
%     $(\Leftarrow)$
%     Since $(\allocation^*(\budget^*), \price^*(\budget^*))$ is a competitive equilibrium of the Fisher market $(\valuation, \budget^*)$ and $\budget^* = \consendow \price^*(\budget^*)$, we have that:
%     \begin{align}
%         \allocation[\buyer]^* \in \argmax_{\allocation[\buyer] \in \consumptions[\buyer]: \allocation[\buyer] \cdot \price \leq \budget[\buyer]^*} \valuation[\buyer](\allocation[\buyer]) = \argmax_{\allocation[\buyer] \in \consumptions[\buyer]: \allocation[\buyer] \cdot \price \leq \consendow[\buyer] \cdot \price^*} \valuation[\buyer](\allocation[\buyer])
%     \end{align}
%     \begin{align}
%         \sum_{\buyer \in \buyers} \allocation[\buyer]^* &\leq \ones[\numgoods]\\
%         \price^* \cdot \left( \sum_{\buyer = 1}^{\numagents} \allocation[\buyer]^*  - \ones[\numgoods] \right) &= 0
%     \end{align}
%     which correspond to the competitive equilibrium conditions for the Arrow-Debreu economy $(\valuation, \consendow)$.
% \end{proof}

% \begin{lemma}
%     Under \Cref{assum:ad_exist}, the leader's utility at a SSNE $(\budget^*, \allocation^*, \price^*)$ of the budget balance game $(3, \outerset, \innerset, \util)$ associated with the homothetic exchange economy $(\valuation, \consendow)$ is equal to $0$, i.e., $\left\| \budget^* - \consendow \price^*\right\|_2^2$, iff $(\allocation^*, \price^*)$ is a competitive equilibrium $(\numbuyers, \numgoods, \valuation, \consendow)$. 
% \end{lemma}

% \begin{theorem}
% Consider a homothetic exchange economy $(\valuation, \consendow)$ and suppose that \Cref{assum:ad_exist} holds. Consider the budget balance game $(3, \outerset, \innerset, \util)$ associated with $(\numbuyers, \numgoods, \valuation, \consendow)$. Then, $(\budget^*, \price^*, \allocation^*)$ is an SSNE of the budget balance game $(3, \outerset, \innerset, \util)$ iff $(\allocation^*, \price^*)$ is a competitive equilibrium of $(\numbuyers, \numgoods, \valuation, \consendow)$.
% \end{theorem}
% \begin{proof}
%     $(\Rightarrow)$:
%     Let $(\budget^*, \allocation^*, \price^*)$ be an SSNE of the budget balance game $(3, \outerset, \innerset, \util)$. 
%     Note that under \Cref{assum:ad_exist}, an Arrow-Debreu competitive equilibrium $(\allocation^\prime, \price^\prime)$ exists. Note that $(\allocation^\prime, \price^\prime)$ is a Nash equilibrium for the followers under budgets $\budget^\prime = \consendow \price^*\prime$ 
    
%     Since $\nicefrac{1}{2} \left\| \budget^* - \consendow\price^*\right\|_2^2 = 0$, we must have that $\budget^* = \price^*$. Additionally, by the constraints of the optimization problem, we have that $(\price^*, \allocation^*)$ is a Nash equilibrium of the convex-concave min-max game $\min_{\price \in \simplex[\numgoods]} \max_{\allocation \in \consumptions} \obj (\price, \allocation, \budget^*)$. In other words,  $(\price^*, \allocation^*)$ is competitive equilibrium of the Fisher market$(\valuation, \budget^*)$ meaning that:
%     \begin{align}
%         \allocation[\buyer]^* \in \argmax_{\allocation[\buyer] \in \consumptions[\buyer]: \allocation[\buyer] \cdot \price \leq \budget[\buyer]^*} \valuation[\buyer](\allocation[\buyer]) = \argmax_{\allocation[\buyer] \in \consumptions[\buyer]: \allocation[\buyer] \cdot \price \leq \consendow[\buyer] \cdot \price^*} \valuation[\buyer](\allocation[\buyer])
%     \end{align}
%     \begin{align}
%         \sum_{\buyer \in \buyers} \allocation[\buyer]^* &\leq \ones[\numgoods]\\
%         \price^* \cdot \left( \sum_{\buyer = 1}^{\numagents} \allocation[\buyer]^*  - \ones[\numgoods] \right) &= 0
%     \end{align}
%     $(\Leftarrow)$:
%     Since $(\price^*, \allocation^*)$ is competitive equilibrium of the Exchange market we have that $(\valuation, \consendow)$ meaning that:
%     \begin{align}
%         \allocation[\buyer]^* \in \argmax_{\allocation[\buyer] \in \consumptions[\buyer]: \allocation[\buyer] \cdot \price \leq \consendow[\buyer] \cdot \price^*} \valuation[\buyer](\allocation[\buyer])
%     \end{align}
%     \begin{align}
%         \sum_{\buyer \in \buyers} \allocation[\buyer]^* &\leq \ones[\numgoods]\\
%         \price^* \cdot \left( \sum_{\buyer = 1}^{\numagents} \allocation[\buyer]^*  - \ones[\numgoods] \right) &= 0
%     \end{align}
% \end{proof}