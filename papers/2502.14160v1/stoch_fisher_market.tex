\subsubsection{Stochastic Fisher Markets}

% A \mydef{(one-shot) Fisher market} consists of $\numbuyers$ buyers and $\numgoods$ divisible goods \citep{brainard2000compute}.
% Each buyer $\buyer \in \buyers$ is endowed with a budget $\budget[\buyer] \in \budgetspace[\buyer] \subseteq \mathbb{R}_{+}$ and a utility function $\util[\buyer]: \mathbb{R}_{+}^{\numgoods} \times \typespace[\buyer] \to \mathbb{R}$, which is parameterized by a type $\type[\buyer] \in \typespace[\buyer]$ that defines a preference relation over the consumption space $\R^\numgoods_+$.
% % \amy{you should introduce types here, if it is types that define utility functions!}
% % \footnote{\sdeni{WLOG, we assume that all budgets are strictly positive; otherwise, we could remove a buyer with a zero budget without affecting the equilibrium.}{}} 
% %As is standard in the literature, we assume that there is one divisible unit of each good in the market.
% Each good is characterized by a supply $\supply[\good] \in \supplyspace[\good] \subset \R_+$.


% An instance of a Fisher market is then a tuple $\calM \doteq (\numbuyers, \numgoods, \util, \type, \budget, \supply)$, where $\util \doteq \left(\util[1], \hdots, \util[\numbuyers] \right)$ is a vector-valued function of all utility functions,
% %\amy{why do you need utility functions in this market spec if you have types?}\deni{Type for me is the valuations that parameterize for instance linear utilities, utility functions are the actual form, i.e., they \amy{the utility functions?} describe whether if we have linear, CD, or Leontief utilities, then the type describes the utility function's parameters.} \sadie{I think in the static Fisher market model, we assumed that different buyers may have different class (linear, leontief, cd, or so) of utility, while type gives the specific valuations. That is, $u_i\in \{linear, leontief, cd\}$, while $t_i$ is some array of coefficients? But I thought the name ``type" is a little misleading, and maybe ``valuation" or ``coefficient" is more straight forward?}
% $\budget \doteq (\budget[1], \hdots, \budget[\numbuyers]) \in \R_{+}^{\numbuyers}$ is the vector of buyer budgets, and $\supply \doteq (\supply[1], \hdots, \supply[\numgoods]) \in \R_{+}^{\numgoods}$ is the vector of supplies.
% When clear from context, we simply denote $\calM$
% %$(\numbuyers, \numgoods, \util, \type, \budget, \supply)$ 
% by $(\type, \budget, \supply)$.

% Given a Fisher market $(\type, \budget, \supply)$, an \mydef{allocation} $\allocation = \left(\allocation[1], \hdots, \allocation[\numbuyers] \right)^T \in \R_+^{\numbuyers \times \numgoods}$ is a map from goods to buyers, represented as a matrix, s.t. $\allocation[\buyer][\good] \ge 0$ denotes the amount of good $\good \in \goods$ allocated to buyer $\buyer \in \buyers$. Goods are assigned \mydef{prices} $\price = \left(\price[1], \hdots, \price[\numgoods] \right)^T \in \mathbb{R}_+^{\numgoods}$. A tuple $(\allocation[][][][*], \price^*)$ is said to be a \mydef{competitive equilibrium (CE)} \citep{arrow-debreu, walras} if 
% 1.~buyers are utility maximizing, constrained by their budget, i.e., $\forall \buyer \in \buyers, \allocation[\buyer][][][*] \in \argmax_{\allocation[ ] : \allocation[ ] \cdot \price^* \leq \budget[\buyer]} \util[\buyer](\allocation[ ], \type[\buyer])$;
% and 2.~the market clears, i.e., $\forall \good \in \goods,  \price[\good]^* > 0 \Rightarrow \sum_{\buyer \in \buyers} \allocation[\buyer][\good][][*] = \supply[\good]$ and $\price[\good]^* = 0 \Rightarrow\sum_{\buyer \in \buyers} \allocation[\buyer][\good][][*] \leq \supply[\good]$.

% \amy{move to appendix!}
%\samy{We denote the type space of all buyers $\typespace = \bigtimes_{\buyer \in \buyers} \typespace[\buyer]$, and the space of all possible budgets for all buyers $\budgetspace = \bigtimes_{\buyer \in \buyers} \budgetspace[\buyer]$. We denote the set of all goods' possible supplies $\supplyspace = \bigtimes_{\good \in \goods} \supplyspace[\good]$.}{}\deni{This should not be removed, it is used in the appendix, and is also our state space, it's important!}\amy{if it is not used in the main body, just define it in the appendix. if you want to use it here, it goes above the previous paragraph so you can then say $(t,b,q) \in T \times B \times Q$.}


A \mydef{stochastic Fisher market with savings} $\fishermkt[0] \doteq (\numbuyers, \numgoods, \util, \states, \initstates, \trans, \discount)$ is a repeated market that travels through a sequence of states in which each state corresponds to a (one-shot) Fisher market (more details see \Cref{sec_app:markets}).
The market is initialized at state $\staterv[0] \sim \initstates$, and for each state $\staterv[\iter] \in \states$ encountered at time step $\iter \in \N_+$, the market determines the prices $\price[][\iter]$ of the goods, while the buyers choose their allocations $\allocation[][][\iter]$ and set aside some \mydef{savings} $\saving[\buyer][\iter] \in [0, \budget[\buyer]]$ to potentially spend at some future state.
Once allocations, savings, and prices have been determined, the market terminates with probability $1-\discount$, or it transitions to a new state $\staterv[\iter + 1] = \state^\prime$ with probability $\discount \trans(\state^\prime \mid \state, \saving)$, depending on the buyers' saving decisions.%
We denote a stochastic Fisher market by $\fishermkt[0] \doteq (\numbuyers, \numgoods, \util, \states, \initstates, \trans, \discount)$.

Given a stochastic Fisher market with savings $\fishermkt[0]$
%$(\states, \util, \state[0], \trans, \discount)$, 
a \mydef{recursive competitive equilibrium (recCE)} \citep{mehra1977recursive} is a tuple $(\allocation[][][][*], \saving[][][*], \price^*) \in \R_+^{\numbuyers \times \numgoods \times \states} \times \R_+^{\numbuyers \times \states} \times \R_+^{\numgoods \times \states}$, which consists of stationary \mydef{allocation}, \mydef{savings}, and \mydef{pricing policies} s.t.\ 1) the buyers are expected utility maximizers, constrained by their savings and spending constraints, i.e., for all buyers $\buyer \in \buyers$, $(\allocation[\buyer][][][*], \saving[\buyer][][*])$ is an optimal policy that, for all states $\state \doteq (\type, \budget, \supply) \in \states$, solves the \mydef{consumption-savings problem}, defined by the following Bellman equations: for all $\state \in \states$
\begin{align}
    \budgetval[\buyer](\state) = \max_{(\allocation[\buyer], \saving[\buyer]) \in \R^{\numgoods + 1}_+: \allocation[\buyer] \cdot \price^*(\state) + \saving[\buyer] \leq \budget[\buyer]} \left\{ \util[\buyer]\left(\allocation[\buyer], \type[\buyer]\right)     + \discount \Ex_{({\type}^\prime, {\budget}^\prime, {\supply}^\prime) \sim \trans(\cdot \mid \state, (\allocation[\buyer], \allocation[][][][*]_{-\buyer}(\state)), 
    (\saving[\buyer], \saving[][][*]_{-\buyer}(\state)))} 
    \left[ \budgetval[\buyer]({\type}^\prime, {\budget}^\prime + (\saving[\buyer], \saving[][][*]_{-\buyer}(\state)), {\supply}^\prime) \right]  \right\},    
\end{align}
    %\budgetval[\buyer](\type, \budget, \supply) =
    

%for all states $(\type, \budget, \supply) \in \states$ 
where $\allocation[][][][*]_{-\buyer}$, $\saving[][][*]_{-\buyer}$ denote the 
%competitive equilibrium 
allocation and saving policies excluding buyer $\buyer$; and
2) the market clears in each state so that unallocated goods in each state are priced at 0, i.e., for all $\good \in \goods$ and $\state \in \states$,
$    
        \price[\good]^*(\state) > 0 \implies \sum_{\buyer \in \buyers} \allocation[\buyer][\good][][*](\state) = \supply[\good]
$ and
$
        \price[\good]^*(\state) \geq 0 \implies \sum_{\buyer \in \buyers} \allocation[\buyer][\good][][*](\state) \leq \supply[\good]
$.
% \samy{By analogy with Markov perfect equilibrium, we can view a}
A recCE is Markov perfect, as it is a CE regardless of initial  state,%
\footnote{Just as any stochastic game can be (non-compactly) represented as a one-shot game, any stochastic Fisher market with savings can be represented as a one-shot Fisher market comprising the same buyers (with utility functions given by their discounted cumulative expected utility) 
% $\Ex_{\staterv[\iter]} \left[\discount^\iter \util[\buyer]\left(\allocation[\buyer][][\iter], \type[\buyer][][\iter]\right) \right]$
and the same goods, enhanced with time-stamps.
It has been shown that competitive equilibria exist in such markets~\citep{prescott1972note}.
In particular, a recCE is a CE in this market; the time stamps ensure that the market clears at every time step, as required.}
%\amy{does whoever writes about the time-stamped construction prove that CE exist in these one-shot markets? also, would a recursive CE of the stochastic Fisher market also be a CE of this one-shot market?}
i.e., buyers are allocated expected discounted cumulative utility-maximizing goods regardless of initial state, and the aggregate demand for each good is equal to its aggregate supply \emph{at all states}.

We show that the computation of a recursive competitive equilibrium in stochastic Fisher markets is equivalent to the computation of nash equilibrium in zero-sum Markov game