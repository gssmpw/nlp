\begin{subtable}[t]{0.5\textwidth}
    \centering
    \resizebox{\columnwidth}{!}{%
    \begin{tabular}{|l|l|l|c|}
        \hline
        Reference & Game Type & Solution Concept & \makecell{Polytime?} \\
        \hline
        \hline \citep{fu2021evaluating} & Finite Markov & Nash & \xmark \\
        \hline \citep{yu2019multi} & Finite Markov & Quantal Response & \xmark \\
        \hline \citep{lin2019multi} & Finite Zero-sum Markov & Various & \xmark\\
        \hline 
        \citep{song2018multi} & Finite Markov &
        Quantal Response & \xmark \\
        \hline  \citep{syrgkanis2017inference} & Finite Bayesian & Bayes-Nash & \cmark \\
        \hline \citep{kuleshov2015inverse} & Finite Normal-Form & Correlated & \cmark \\
        \hline \citep{waugh2013computational} & Finite Normal-Form & Correlated &  \cmark \\
        \hline \citep{bestick2013inverse} & Finite Normal-Form & Correlated & \xmark \\  
        \hline \citep{natarajan2010multi} & Finite Markov & Cooperative & \xmark \\
        \hline \rowcolor{orange!50} This work & \makecell[l]{Finite/Concave Normal-form\\ Finite/Concave Markov} & 
        % \makecell[l]{Nash /Any}
        \makecell[l]{Nash/Correlated\\ Any Other \quad \quad \quad \quad }
        & \cmark \\
        \hline
    \end{tabular}
    }
    \caption{A comparison of our work and prior work on inverse game theory and inverse MARL.}
    \label{tab:summary_lit}
    \vspace{-2em}
\end{subtable}
