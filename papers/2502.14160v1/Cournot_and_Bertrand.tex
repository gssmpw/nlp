\subsubsection{Cournot Competition and Bertrand Competition}

A \mydef{Cournot competition model} $\calC \doteq (\numfirms, \mcost, \pricefunc)$ consists of $n$ firms that produce a homogeneous product, and each firm chooses a quantity level of production that maximizes its profits. The set of \mydef{Cournot Equilibrium} of a Cournot competition model is equal to the set of Nash Equilibrium of a one-shot general-sum game, \mydef{Cournot game} (see more details in \Cref{sec_app:markets}). 

A \mydef{Bertrand competition model} $\calB \doteq (\numfirms, \mcost, \demandfunc)$ is also a competition model that consists of $n$ firms producing a homogeneous product, but this time, each firm set prices to maximize its profits. Similarly, we can compute the \mydef{Bertrand equilibrium} of the model by computing the NE of the one-shot general-sum game, \mydef{Bertrand game} (see more details in \Cref{sec_app:markets}). 

In experiments, we generated 500 Cournot competition models and 500 Bertrand competition models and attempted to retrieve their true parameter, i.e., marginal costs, given equilibrium productions/equilibrium prices respectively. \Cref{table:cournot+bertrand_results} shows that our algorithm can effectively recover the true parameters in Cournot games and minimize the exploitability of the observed equilibrium evaluated under the computed parameters. In the Bertrand games, though we could only recover $78\%$ true parameters, the average exploitability of the observed equilibrium evaluated under the computed inverse Nash equilibrium is mostly minimized. 


