\begin{subtable}[t]{0.5\textwidth}
\resizebox{\columnwidth}{!}{%
% \renewcommand{\arraystretch}{2} 
\begin{tabular}{|l||c|c|}
\hline
Equilibrium Access & Exact Oracle & Stochastic Oracle \\
\hline \hline
Direct & Inverse Multiagent Planning & Inverse Multiagent Learning \\
\hline 
\makecell[l]{Faithful Samples} & First-order Simulacral Planning & First-order Simulacral Learning \\
\hline
\makecell[l]{Nonfaithful Samples} & Second-order Simulacral Planning & Second-order Simulacral Learning \\
\hline
\end{tabular}
}
\caption{Taxonomy of inverse 
%(multiagent) 
game theory problems.
First-order simulacral learning is more commonly known as multiagent apprenticeship learning \citep{abbeel2004apprenticeship, yang2020inferring}.}
\label{table:inverse_gt_summary}
\vspace{-1em}
% Noisy oracles are applicable in Markov games.
\end{subtable}