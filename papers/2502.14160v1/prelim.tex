\section{Preliminaries}
\vspace{-1em}
\textbf{Notation. } 
All notation for variable types, e.g., vectors, should be clear from context; if any confusion arises, see \Cref{sec_app:prelims}.
% We use caligraphic uppercase letters to denote sets (e.g., $\calX$), bold uppercase letters to denote matrices (e.g., $\allocation$), bold lowercase letters to denote vectors (e.g., $\price$), lowercase letters to denote scalar quantities (e.g., $x$), and uppercase letters to denote random variables (e.g., $X$).
% We denote the $i$th row vector of a matrix (e.g., $\allocation$) by the corresponding bold lowercase letter with subscript $i$ (e.g., $\allocation[\buyer])$. 
% Similarly, we denote the $j$th entry of a vector (e.g., $\price$ or $\allocation[\buyer]$) by the corresponding Roman lowercase letter with subscript $j$ (e.g., $\price[\good]$ or $\allocation[\buyer][\good]$).
% We denote functions by a letter determined by the value of the function: e.g., $f$ if the mapping is scalar-valued, $\f$ if the mapping is vector-valued, and $\calF$ if the mapping is set-valued.
% We denote the vector of ones of size $\numbuyers$ by $\ones[\numbuyers]$.
We denote 
% a set of integers $\left\{0, \hdots, n\right\}$ by $[n^*]$,
by $[n]$ the set of integers $\left\{1, \hdots, n\right\}$.
%
Let $\calX$ be any set and $(\calX, \calF)$ any associated measurable space, where the $\sigma$-algebra $\calF$ unless otherwise noted is assumed to be the $\sigma$-algebra of countable sets, i.e., $\calF \doteq \{\calE \subseteq \calX \mid \calE \text{ is countable } \}$.
%%%
% SPACE
%We denote by $\simplex[n] = \{\x \in \R_+^n \mid \sum_{i = 1}^n x_i = 1 \}$ the unit simplex in $\R^n$, 
We write $\simplex(\calX) \doteq \{\mu: (\calX, \calF) \to [0, 1] \}$ to denote the set of  \mydef{probability measures} on $(\calX, \calF)$.
%
Additionally, we denote the orthogonal projection operator onto a set $\calX$ by $\project[\calX](\x) \doteq \argmin_{\y \in \calX} \left\|\x - \y \right\|_2^2$.
 
% \deni{Add definition of diameter.}
% We define $\ball[\varepsilon][\x] = \{ \z \in \calZ \mid ||\z - \x || \leq \varepsilon \}$ to be the closed epsilon ball centered at $\x$; here $\calZ$ and $\|\cdot \|$ will be clear from context.
% We denote the indicator function of a set $\calC$ by $\setindic[\calC](\x) = 0$ if $\x \in \calC$ and $\setindic[\calC](\x) = \infty$ if $\x \notin \calC$.
% We abbreviate  $\mathop{\Ex}[f(\strat) \mid \policy(\state)  = \strat]$, the conditional expectation of $f(\strat)$ for a given function $f$ with respect policy profile $\policy$ at a given state $\state$, by $\mathop{\Ex}^{\policy}[f(\strat)]$.
%%% SPACE
%Given two vector $\x, \y \in \R^n$, we write $\x \geq \y$ or $\x > \y$ to mean component-wise $\ge$ or $>$ respectively. 
% Given any parameterized probability distribution function $f : \calX \to \simplex(\calY)$ such that for any $\x \in \calX$ $f(\x) \in \simplex(\calY)$, we denote $f(\x)$ by $f(\cdot \mid \x)$,\amy{??? which one is the abbreviation? and why?} and we denote the probability of an event $\y \in \calY$ occuring according to $f(\x)$ by $f( \y \mid \x)$.


\textbf{Mathematical Concepts. } 
%We define several mathematical concepts that are used in our convergence proofs.
Consider any normed space $(\calX, \left\| \cdot \right\|)$ where $\calX \subset \R^\outerdim$ and any function $\obj: \calX \to \R$.
$\obj$ is $\lipschitz[\obj]$-\mydef{Lipschitz-continuous} w.r.t.\@ norm (typically, Euclidean)
%\footnote{Unless otherwise noted, we assume $\left\| \cdot \right\|$ is the Euclidean norm, i.e., $\left\| \cdot \right\| = \left\| \cdot \right\|_2$.} 
$\left\| \cdot \right\|$ iff $\forall \x_1, \x_2 \in \calA, \left\| \obj(\x_1) - \obj(\x_2) \right\| \leq \lipschitz[\obj] \left\| \x_1 - \x_2 \right\|$.
If the gradient of $\obj$ is $\lipschitz[\grad \obj]$-Lipschitz-continuous, we refer to $\obj$ as $\lipschitz[\grad \obj]$-\mydef{Lipschitz-smooth}.
Furthermore, given $\scparam > 0$,  $\obj$ is said to be $\scparam$-\mydef{gradient-dominated} if $\min_{\x^\prime \in \calX} \obj(\x^\prime) \geq \obj(\x) + \scparam \cdot \min_{\x^\prime \in \calX} \left< \x^\prime - \x, \grad \obj(\x) \right>$ \citep{bhandari2019global}.
% /$\obj: \calA \to \R$ is said to be \mydef{invex} w.r.t.\ a function $\h: \calA \times \calA \to \calA$ if for all $\x, \y \in \calA$, 
% $$\obj(\x) - \obj(\y) \geq \h(\x, \y)  \cdot \grad \obj(\y)$$
% , for all $\x, \y \in \calA$. $\obj: \calA \to \R$ is said to be \mydef{incave} if $-\obj$ is invex.  
% $\obj$ is $\scparam$-strongly-invex w.r.t.\ a function $\h: \calA \times \calA \to \calA$ if for all $\x, \y \in \calA$,
% $$ \obj(\x) - \obj(\x)  \leq \nicefrac{\scparam}{2} \h(\x, \y)\cdot \grad \obj(\x)$$.
% Similarly, $\obj$ is $\scparam$-strongly-incave if $-\obj$ is $\scparam$-strongly-invex.
% $\obj$ is \mydef{convex} (resp. concave) if it is invex (incave) w.r.t.\ $\h(\x, \y) = \x - \y$.
% $\obj$ is $\mu$-\mydef{strongly convex (SC)} if $\obj(\x_1) \geq \obj(\x_2) + \left< \grad[\x] \obj(\x_2), \x_1 - \x_2 \right> + \nicefrac{\mu}{2} \left\| \x_1 - \x_1 \right\|^2$;
% $\obj$ is $\mu$-\mydef{strongly concave} if $-\obj$ is $\mu$-strongly convex;
% $\obj$ is $\wcparam$-\mydef{weakly convex} (resp.\ $\wcparam$-\mydef{weakly concave}) if $\obj(\x)+\frac{\wcparam}{2}\|\x\|^2$ (resp.\ $\obj(\x)-\frac{\wcparam}{2}\|\x\|^2$) is convex (resp.\ concave);
% $\obj$ is \mydef{invex} w.r.t.\ a function $\h: \calA \times \calA \to \calA$ if $\obj(\x) - \obj(\y) \geq \h(\x, \y) \cdot \grad \obj(\y)$, for all $\x, \y \in \calA$;
% $\obj$ is \mydef{incave} if $-\obj$ is invex;  
% $\obj$ is $\scparam$-strongly-invex w.r.t.\ a function $\h: \calA \times \calA \to \calA$ if for all $\x, \y \in \calA$,
% $$ \obj(\x) - \obj(\x)  \leq \nicefrac{\scparam}{2} \h(\x, \y)\cdot \grad \obj(\x)$$.
% Similarly, $\obj$ is $\scparam$-strongly-incave if $-\obj$ is $\scparam$-strongly-invex.
% and $\obj$ is \mydef{convex} (resp.\ \mydef{concave}) if it is invex (incave) w.r.t.\ $\h(\x, \y) = \x - \y$. 
% \deni{Add definition of a convex graph correspondence and maybe a few more correspondence definitions.}


\amy{all throughout, in algos, appendix, etc., we have to check for $\actions$ vs.\@ $\actionspace$.}







