\paragraph{Related Work}

\paragraph{Min-max optimization}
When $\obj = -\otherobj$, a bilevel optimization problem reduces to a min-max optimization problem, or a zero-sum Stackelberg game. \sdeni{}{A variety of algorithm have been developed to solve such games in recent years (for a summary, see for instance, \cite{lin2020near, lin2020gradient}).}
Such games have also found many applications in machine learning, including but not limited to generative adversarial networks (GANs) \cite{goodfellow2014gan}, statistical learning \cite{xu2009robustness, shafieezadeh2015distributionally, dai2019kernel}, online learning \cite{cesa2006prediction}, deep learning \cite{sinha2017certifiable, sinha2020certifying}, distributed computing \cite{shamma2008cooperative, mateos2010distributed}, fairness in machine learning \cite{dai2019kernel, edwards2016censoring, madras2018learning, sattigeri2018fairness, xu2018fairgan}, and generative adversarial imitation learning \cite{cai2019global, hamedani2018iteration}.


\paragraph{Bilevel Optimization.} 
Many algorithms to solve bilevel optimization problems have been introduced in recent years.
\citeauthor{ghadimi2018approximation} (\citeyear{ghadimi2018approximation}) proposed a nested gradient descent ascent, which they call the bilevel approximation method, and which converges to a Stackelberg equilibrium in $O(\nicefrac{1}{\varepsilon})$ under the assumption that the leader's payoff is concave and the followers' payoff is strongly concave and unconstrained.
Under similar assumptions \citeauthor{ji2021bilevel} (\citeyear{ji2021bilevel}) proved tighter convergence bounds,
%to a Stackelberg equilibrium for the nested gradient descent ascent method, 
while \citeauthor{yang2021achieving} (\citeyear{yang2021achieving}) derived an
%nested gradient descent ascent 
algorithm that converges \amy{similar convergence guarantees?} 
%to a Stackelberg equilibrium
in a decentralized setting.
\citeauthor{hong2020two} (\citeyear{hong2020two}) subsequently provided a two time-scale algorithm, which consists of only a single gradient descent ascent loop, and which converges \amy{same rate?} under the same assumptions.\amy{which assumptions? also decentralized?}
\citeauthor{chen2021closing} (\citeyear{chen2021closing}) provided a tighter analysis of this latter algorithm, albeit under the assumption that the leader and followers' problems are unconstrained.
\citeauthor{chen2021closing} (\citeyear{chen2021closing})'s method improves upon previous work by using an additive correction term in the follower update step (beyond a basic SGD step).\amy{wait, is this the correct reference? the previous sentence said they analyze the same algo? not that they introduce a new method?}

There also exists a parallel line of work within game theory, which has focused on computing variants of Stackelberg equilibrium in single-leader-single-follower Stackelberg games.
These works include methods to compute first-order Stackelberg equilibria in games where the best-response of the follower is \amy{not?} unique \cite{fiez2019convergence, fiez2020implicit}, strong Stackelberg equilibria in Stackelberg games with discrete-action spaces, i.e., Stackelberg security games \cite{an2011refinement, sinha2018stackelberg, an2017stackelberg, blum2019computing, conitzer2006computing, gan2018stackelberg}.

\begin{table}
\renewcommand{\arraystretch}{1.3}
    \centering
    \begin{tabular}{|c|c|c|}
    \hline Assumptions & Authors & Iteration Complexity \\
    \hline
    \multicolumn{3}{|c|}{Bilevel Optimization}\\
    \hline \hline 
    $\outerset = \R^\outerdim$, $\innerset = \R^\innerdim$, $\obj$ concave, $\otherobj$ strongly concave in $\inner$ & \citeauthor{ghadimi2018approximation} (\citeyear{ghadimi2018approximation}) & $\tilde{O}(\nicefrac{\kappa^3}{\varepsilon^2})$ \\
    \hline
    $\innerset = \R^\innerdim$, $\obj$ concave, $\otherobj$ strongly concave in $\inner$ & \citeauthor{ji2021bilevel} (\citeyear{ji2021bilevel}) & $\tilde{O}(\nicefrac{\kappa^4}{\varepsilon^2})$ \\
    \hline
    \hline
    \multicolumn{3}{|c|}{Bilevel optimization with multiobjective lower-level problem}\\
    \hline
    \hline
\end{tabular}
    \caption{Comparison of the best known computational complexities for bilevel optimization problems.}
    \label{tab:my_label}
\end{table}


\paragraph{Bilevel Multiobjective Optimization.}
A limited number of papers \cite{sinha2015multi, deb2009multi, ji2018multiobjective} have analyzed bilevel multiobjective optimization problems in which the upper- and lower-level problems \emph{both\/} have multiple objectives, i.e., Stackelberg games with multiple leaders and multiple followers.
These works focus on Stackelberg equilibria in which the leaders first, and the followers second, each play a Pareto-optimal solution, and do not provide polynomial-time computation guarantees. 

There also exists a parallel line of work within game theory on computing Stackelberg-Nash equilibrium in Stackelberg-Nash games \cite{nakamura2015one, liu1998stackelberg}.
These papers provide convergence guarantees to various notions of first-order Stackelberg-Nash equilibria, as it is not possible to compute a Stackelberg-Nash equilibrium in polynomial-time in the games the authors consider \cite{basilico2020bilevel, coniglio2020computing}.
A more recent line of work provides a first-order method to compute a Stackelberg-Nash equilibrium when the followers are assumed \samy{}{(by fiat)} \amy{i want that this assumption is totally unrealistic!} to play a unique Nash equilibrium, by using variants of the implicit function theorem and the envelope theorem \cite{li2020end, li2022solving, sinha2019using, wang2022coordinating}.

A more recent line of work has turned its attention to multiple leader-multiple follower Stackelberg games.
These works have once again focused on computing stationary solution concepts rather than Nash-Stackelberg-Nash equilibria, \amy{did you make up this name? or did Solis or Sinha?}\deni{These guys did not make it up but other people did. see \url{https://arxiv.org/abs/2202.11880}} i.e., Stackelberg equilibria at which the leaders play a Nash assuming the followers play a Nash \cite{solis2016modeling, sinha2014finding}. 
Finally, \citeauthor{gu2023min} (\citeyear{gu2023min}) provide polynomial-time computational guarantees for robust solutions to %multiobjective bilevel optimization problems with multiple leaders and multiple followers,
Nash-Stackelberg-Nash games, which can be seen as variants of weak Nash-Stackelberg-Nash equilibrium, under the assumption that the followers' payoffs are strongly convex.


\paragraph{Computation of competitive equilibrium}
The study of the computational complexity of competitive equilibria was initiated by \citet{devanur2002market}, who provided a polynomial-time method for computing competitive equilibrium in a special case of the Arrow-Debreu (exchange) market model, namely Fisher markets, when buyers utilities are linear.
\citet{jain2005market} subsequently showed that a large class of Fisher markets with homogeneous utility functions could be solved in polynomial-time using interior point methods.
Recently, \citet{gao2020polygm} studied an alternative family of first-order methods for solving Fisher markets, assuming linear, quasilinear, and Leontief utilities, as such methods can be more efficient when markets are large.

Devising algorithms for the computation of competitive equilibrium in general Arrow-Debreu markets is still an active area of research.
While the computation of competitive equilibrium is PPAD-complete in general, \amy{i cannot parse the rest of this sentence. it's a run-on, for sure!} Arrow-Debreu markets with additively separable, piecewise-linear and concave utilities \cite{chen2006settling}, and the computation of competitive equilibrium in Arrow-Debreu markets with Leontief buyers being equivalent to the computation of Nash equilibrium in bimatrix games \cite{codenotti2006leontief, deng2008computation}, there exists polynomial-time algorithms to compute competitive equilibrium for special cases of Arrow-Debreu markets, including markets whose excess demand satisfies the weak gross substitutes (WGS) condition \cite{codenotti2005market, bei2015tatonnement}, Arrow-Debreu markets with buyers whose utilities are linear \cite{garg2004auction, branzei2021proportional} or CES utilities in the weak gross substitutes range \cite{branzei2021proportional}.
