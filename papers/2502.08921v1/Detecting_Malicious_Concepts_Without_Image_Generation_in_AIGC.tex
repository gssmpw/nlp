
%% bare_jrnl_compsoc.tex
%% V1.4b
%% 2015/08/26
%% by Michael Shell
%% See:
%% http://www.michaelshell.org/
%% for current contact information.
%%
%% This is a skeleton file demonstrating the use of IEEEtran.cls
%% (requires IEEEtran.cls version 1.8b or later) with an IEEE
%% Computer Society journal paper.
%%
%% Support sites:
%% http://www.michaelshell.org/tex/ieeetran/
%% http://www.ctan.org/pkg/ieeetran
%% and
%% http://www.ieee.org/

%%*************************************************************************
%% Legal Notice:
%% This code is offered as-is without any warranty either expressed or
%% implied; without even the implied warranty of MERCHANTABILITY or
%% FITNESS FOR A PARTICULAR PURPOSE! 
%% User assumes all risk.
%% In no event shall the IEEE or any contributor to this code be liable for
%% any damages or losses, including, but not limited to, incidental,
%% consequential, or any other damages, resulting from the use or misuse
%% of any information contained here.
%%
%% All comments are the opinions of their respective authors and are not
%% necessarily endorsed by the IEEE.
%%
%% This work is distributed under the LaTeX Project Public License (LPPL)
%% ( http://www.latex-project.org/ ) version 1.3, and may be freely used,
%% distributed and modified. A copy of the LPPL, version 1.3, is included
%% in the base LaTeX documentation of all distributions of LaTeX released
%% 2003/12/01 or later.
%% Retain all contribution notices and credits.
%% ** Modified files should be clearly indicated as such, including  **
%% ** renaming them and changing author support contact information. **
%%*************************************************************************


% *** Authors should verify (and, if needed, correct) their LaTeX system  ***
% *** with the testflow diagnostic prior to trusting their LaTeX platform ***
% *** with production work. The IEEE's font choices and paper sizes can   ***
% *** trigger bugs that do not appear when using other class files.       ***                          ***
% The testflow support page is at:
% http://www.michaelshell.org/tex/testflow/


\documentclass[10pt,journal,compsoc]{IEEEtran}

%
% If IEEEtran.cls has not been installed into the LaTeX system files,
% manually specify the path to it like:
% \documentclass[10pt,journal,compsoc]{../sty/IEEEtran}





% Some very useful LaTeX packages include:
% (uncomment the ones you want to load)


% *** MISC UTILITY PACKAGES ***
%
%\usepackage{ifpdf}
% Heiko Oberdiek's ifpdf.sty is very useful if you need conditional
% compilation based on whether the output is pdf or dvi.
% usage:
% \ifpdf
%   % pdf code
% \else
%   % dvi code
% \fi
% The latest version of ifpdf.sty can be obtained from:
% http://www.ctan.org/pkg/ifpdf
% Also, note that IEEEtran.cls V1.7 and later provides a builtin
% \ifCLASSINFOpdf conditional that works the same way.
% When switching from latex to pdflatex and vice-versa, the compiler may
% have to be run twice to clear warning/error messages.






% *** CITATION PACKAGES ***
%
\ifCLASSOPTIONcompsoc
  % IEEE Computer Society needs nocompress option
  % requires cite.sty v4.0 or later (November 2003)
  \usepackage[nocompress]{cite}
\else
  % normal IEEE
  \usepackage{cite}
\fi
% cite.sty was written by Donald Arseneau
% V1.6 and later of IEEEtran pre-defines the format of the cite.sty package
% \cite{} output to follow that of the IEEE. Loading the cite package will
% result in citation numbers being automatically sorted and properly
% "compressed/ranged". e.g., [1], [9], [2], [7], [5], [6] without using
% cite.sty will become [1], [2], [5]--[7], [9] using cite.sty. cite.sty's
% \cite will automatically add leading space, if needed. Use cite.sty's
% noadjust option (cite.sty V3.8 and later) if you want to turn this off
% such as if a citation ever needs to be enclosed in parenthesis.
% cite.sty is already installed on most LaTeX systems. Be sure and use
% version 5.0 (2009-03-20) and later if using hyperref.sty.
% The latest version can be obtained at:
% http://www.ctan.org/pkg/cite
% The documentation is contained in the cite.sty file itself.
%
% Note that some packages require special options to format as the Computer
% Society requires. In particular, Computer Society  papers do not use
% compressed citation ranges as is done in typical IEEE papers
% (e.g., [1]-[4]). Instead, they list every citation separately in order
% (e.g., [1], [2], [3], [4]). To get the latter we need to load the cite
% package with the nocompress option which is supported by cite.sty v4.0
% and later. Note also the use of a CLASSOPTION conditional provided by
% IEEEtran.cls V1.7 and later.





% *** GRAPHICS RELATED PACKAGES ***
%
\ifCLASSINFOpdf
  % \usepackage[pdftex]{graphicx}
  % declare the path(s) where your graphic files are
  % \graphicspath{{../pdf/}{../jpeg/}}
  % and their extensions so you won't have to specify these with
  % every instance of \includegraphics
  % \DeclareGraphicsExtensions{.pdf,.jpeg,.png}
\else
  % or other class option (dvipsone, dvipdf, if not using dvips). graphicx
  % will default to the driver specified in the system graphics.cfg if no
  % driver is specified.
  % \usepackage[dvips]{graphicx}
  % declare the path(s) where your graphic files are
  % \graphicspath{{../eps/}}
  % and their extensions so you won't have to specify these with
  % every instance of \includegraphics
  % \DeclareGraphicsExtensions{.eps}
\fi
% graphicx was written by David Carlisle and Sebastian Rahtz. It is
% required if you want graphics, photos, etc. graphicx.sty is already
% installed on most LaTeX systems. The latest version and documentation
% can be obtained at: 
% http://www.ctan.org/pkg/graphicx
% Another good source of documentation is "Using Imported Graphics in
% LaTeX2e" by Keith Reckdahl which can be found at:
% http://www.ctan.org/pkg/epslatex
%
% latex, and pdflatex in dvi mode, support graphics in encapsulated
% postscript (.eps) format. pdflatex in pdf mode supports graphics
% in .pdf, .jpeg, .png and .mps (metapost) formats. Users should ensure
% that all non-photo figures use a vector format (.eps, .pdf, .mps) and
% not a bitmapped formats (.jpeg, .png). The IEEE frowns on bitmapped formats
% which can result in "jaggedy"/blurry rendering of lines and letters as
% well as large increases in file sizes.
%
% You can find documentation about the pdfTeX application at:
% http://www.tug.org/applications/pdftex





\usepackage{bbding}

\usepackage{amsfonts}

\usepackage{amsmath}

%\usepackage{caption}
%\captionsetup[table]{justification=centering, font=small, labelformat=simple, labelsep=newline, textfont=sc}
%
%\usepackage{caption}
%\captionsetup[figure]{font=small, labelformat=simple}
%
%\usepackage{caption}
%\captionsetup[subfigure]{font=small, labelformat=simple}


\usepackage{graphicx}

\usepackage{float}  

\usepackage{subfigure}

\usepackage{bm}

\usepackage{courier}

\usepackage{booktabs}

\usepackage{multirow}

\newcommand{\tabincell}[2]{\begin{tabular}{@{}#1@{}}#2\end{tabular}}

\usepackage{rotating}

\usepackage{amsthm}

\usepackage{footnote}

\usepackage{color}

\usepackage{stfloats}

\usepackage{threeparttable}

\usepackage{adjustbox}

\newtheorem{proposition}{Proposition}

\newtheorem{definition}{Definition}

\newtheorem{property}{Property}

\newtheorem{note}{Note}


\usepackage{orcidlink}
\usepackage{colortbl, booktabs}
\usepackage{pifont}
\usepackage{textcomp}
\usepackage{stfloats}
\usepackage{url}
\usepackage{verbatim}
\usepackage{graphicx}
\usepackage{algorithm}
\usepackage{array}
\usepackage{amsmath,amsfonts}
\hypersetup{hidelinks}
\usepackage[noend]{algorithmic}
\usepackage{booktabs}
\usepackage{xcolor}
\usepackage{enumitem}


\usepackage{hyperref}
\hypersetup{
	colorlinks=true,
	citecolor=blue,
	linkcolor=red,
}

\usepackage[most]{tcolorbox}

% *** MATH PACKAGES ***
%
%\usepackage{amsmath}
% A popular package from the American Mathematical Society that provides
% many useful and powerful commands for dealing with mathematics.
%
% Note that the amsmath package sets \interdisplaylinepenalty to 10000
% thus preventing page breaks from occurring within multiline equations. Use:
%\interdisplaylinepenalty=2500
% after loading amsmath to restore such page breaks as IEEEtran.cls normally
% does. amsmath.sty is already installed on most LaTeX systems. The latest
% version and documentation can be obtained at:
% http://www.ctan.org/pkg/amsmath





% *** SPECIALIZED LIST PACKAGES ***
%
%\usepackage{algorithmic}
% algorithmic.sty was written by Peter Williams and Rogerio Brito.
% This package provides an algorithmic environment fo describing algorithms.
% You can use the algorithmic environment in-text or within a figure
% environment to provide for a floating algorithm. Do NOT use the algorithm
% floating environment provided by algorithm.sty (by the same authors) or
% algorithm2e.sty (by Christophe Fiorio) as the IEEE does not use dedicated
% algorithm float types and packages that provide these will not provide
% correct IEEE style captions. The latest version and documentation of
% algorithmic.sty can be obtained at:
% http://www.ctan.org/pkg/algorithms
% Also of interest may be the (relatively newer and more customizable)
% algorithmicx.sty package by Szasz Janos:
% http://www.ctan.org/pkg/algorithmicx




% *** ALIGNMENT PACKAGES ***
%
%\usepackage{array}
% Frank Mittelbach's and David Carlisle's array.sty patches and improves
% the standard LaTeX2e array and tabular environments to provide better
% appearance and additional user controls. As the default LaTeX2e table
% generation code is lacking to the point of almost being broken with
% respect to the quality of the end results, all users are strongly
% advised to use an enhanced (at the very least that provided by array.sty)
% set of table tools. array.sty is already installed on most systems. The
% latest version and documentation can be obtained at:
% http://www.ctan.org/pkg/array


% IEEEtran contains the IEEEeqnarray family of commands that can be used to
% generate multiline equations as well as matrices, tables, etc., of high
% quality.




% *** SUBFIGURE PACKAGES ***
%\ifCLASSOPTIONcompsoc
%  \usepackage[caption=false,font=footnotesize,labelfont=sf,textfont=sf]{subfig}
%\else
%  \usepackage[caption=false,font=footnotesize]{subfig}
%\fi
% subfig.sty, written by Steven Douglas Cochran, is the modern replacement
% for subfigure.sty, the latter of which is no longer maintained and is
% incompatible with some LaTeX packages including fixltx2e. However,
% subfig.sty requires and automatically loads Axel Sommerfeldt's caption.sty
% which will override IEEEtran.cls' handling of captions and this will result
% in non-IEEE style figure/table captions. To prevent this problem, be sure
% and invoke subfig.sty's "caption=false" package option (available since
% subfig.sty version 1.3, 2005/06/28) as this is will preserve IEEEtran.cls
% handling of captions.
% Note that the Computer Society format requires a sans serif font rather
% than the serif font used in traditional IEEE formatting and thus the need
% to invoke different subfig.sty package options depending on whether
% compsoc mode has been enabled.
%
% The latest version and documentation of subfig.sty can be obtained at:
% http://www.ctan.org/pkg/subfig




% *** FLOAT PACKAGES ***
%
%\usepackage{fixltx2e}
% fixltx2e, the successor to the earlier fix2col.sty, was written by
% Frank Mittelbach and David Carlisle. This package corrects a few problems
% in the LaTeX2e kernel, the most notable of which is that in current
% LaTeX2e releases, the ordering of single and double column floats is not
% guaranteed to be preserved. Thus, an unpatched LaTeX2e can allow a
% single column figure to be placed prior to an earlier double column
% figure.
% Be aware that LaTeX2e kernels dated 2015 and later have fixltx2e.sty's
% corrections already built into the system in which case a warning will
% be issued if an attempt is made to load fixltx2e.sty as it is no longer
% needed.
% The latest version and documentation can be found at:
% http://www.ctan.org/pkg/fixltx2e


%\usepackage{stfloats}
% stfloats.sty was written by Sigitas Tolusis. This package gives LaTeX2e
% the ability to do double column floats at the bottom of the page as well
% as the top. (e.g., "\begin{figure*}[!b]" is not normally possible in
% LaTeX2e). It also provides a command:
%\fnbelowfloat
% to enable the placement of footnotes below bottom floats (the standard
% LaTeX2e kernel puts them above bottom floats). This is an invasive package
% which rewrites many portions of the LaTeX2e float routines. It may not work
% with other packages that modify the LaTeX2e float routines. The latest
% version and documentation can be obtained at:
% http://www.ctan.org/pkg/stfloats
% Do not use the stfloats baselinefloat ability as the IEEE does not allow
% \baselineskip to stretch. Authors submitting work to the IEEE should note
% that the IEEE rarely uses double column equations and that authors should try
% to avoid such use. Do not be tempted to use the cuted.sty or midfloat.sty
% packages (also by Sigitas Tolusis) as the IEEE does not format its papers in
% such ways.
% Do not attempt to use stfloats with fixltx2e as they are incompatible.
% Instead, use Morten Hogholm'a dblfloatfix which combines the features
% of both fixltx2e and stfloats:
%
% \usepackage{dblfloatfix}
% The latest version can be found at:
% http://www.ctan.org/pkg/dblfloatfix




%\ifCLASSOPTIONcaptionsoff
%  \usepackage[nomarkers]{endfloat}
% \let\MYoriglatexcaption\caption
% \renewcommand{\caption}[2][\relax]{\MYoriglatexcaption[#2]{#2}}
%\fi
% endfloat.sty was written by James Darrell McCauley, Jeff Goldberg and 
% Axel Sommerfeldt. This package may be useful when used in conjunction with 
% IEEEtran.cls'  captionsoff option. Some IEEE journals/societies require that
% submissions have lists of figures/tables at the end of the paper and that
% figures/tables without any captions are placed on a page by themselves at
% the end of the document. If needed, the draftcls IEEEtran class option or
% \CLASSINPUTbaselinestretch interface can be used to increase the line
% spacing as well. Be sure and use the nomarkers option of endfloat to
% prevent endfloat from "marking" where the figures would have been placed
% in the text. The two hack lines of code above are a slight modification of
% that suggested by in the endfloat docs (section 8.4.1) to ensure that
% the full captions always appear in the list of figures/tables - even if
% the user used the short optional argument of \caption[]{}.
% IEEE papers do not typically make use of \caption[]'s optional argument,
% so this should not be an issue. A similar trick can be used to disable
% captions of packages such as subfig.sty that lack options to turn off
% the subcaptions:
% For subfig.sty:
% \let\MYorigsubfloat\subfloat
% \renewcommand{\subfloat}[2][\relax]{\MYorigsubfloat[]{#2}}
% However, the above trick will not work if both optional arguments of
% the \subfloat command are used. Furthermore, there needs to be a
% description of each subfigure *somewhere* and endfloat does not add
% subfigure captions to its list of figures. Thus, the best approach is to
% avoid the use of subfigure captions (many IEEE journals avoid them anyway)
% and instead reference/explain all the subfigures within the main caption.
% The latest version of endfloat.sty and its documentation can obtained at:
% http://www.ctan.org/pkg/endfloat
%
% The IEEEtran \ifCLASSOPTIONcaptionsoff conditional can also be used
% later in the document, say, to conditionally put the References on a 
% page by themselves.




% *** PDF, URL AND HYPERLINK PACKAGES ***
%
%\usepackage{url}
% url.sty was written by Donald Arseneau. It provides better support for
% handling and breaking URLs. url.sty is already installed on most LaTeX
% systems. The latest version and documentation can be obtained at:
% http://www.ctan.org/pkg/url
% Basically, \url{my_url_here}.





% *** Do not adjust lengths that control margins, column widths, etc. ***
% *** Do not use packages that alter fonts (such as pslatex).         ***
% There should be no need to do such things with IEEEtran.cls V1.6 and later.
% (Unless specifically asked to do so by the journal or conference you plan
% to submit to, of course. )


% correct bad hyphenation here
\hyphenation{op-tical net-works semi-conduc-tor}


\begin{document}
%
% paper title
% Titles are generally capitalized except for words such as a, an, and, as,
% at, but, by, for, in, nor, of, on, or, the, to and up, which are usually
% not capitalized unless they are the first or last word of the title.
% Linebreaks \\ can be used within to get better formatting as desired.
% Do not put math or special symbols in the title.
\title{Detecting Malicious Concepts Without Image Generation in AIGC}
%
%
% author names and IEEE memberships
% note positions of commas and nonbreaking spaces ( ~ ) LaTeX will not break
% a structure at a ~ so this keeps an author's name from being broken across
% two lines.
% use \thanks{} to gain access to the first footnote area
% a separate \thanks must be used for each paragraph as LaTeX2e's \thanks
% was not built to handle multiple paragraphs
%
%
%\IEEEcompsocitemizethanks is a special \thanks that produces the bulleted
% lists the Computer Society journals use for "first footnote" author
% affiliations. Use \IEEEcompsocthanksitem which works much like \item
% for each affiliation group. When not in compsoc mode,
% \IEEEcompsocitemizethanks becomes like \thanks and
% \IEEEcompsocthanksitem becomes a line break with idention. This
% facilitates dual compilation, although admittedly the differences in the
% desired content of \author between the different types of papers makes a
% one-size-fits-all approach a daunting prospect. For instance, compsoc 
% journal papers have the author affiliations above the "Manuscript
% received ..."  text while in non-compsoc journals this is reversed. Sigh.

\author{Kun Xu$^{\orcidlink{0000-0002-1866-4433}}$, Yushu Zhang$^{\orcidlink{0000-0001-8183-8435}}$,~\IEEEmembership{Senior Member,~IEEE}, Shuren Qi$^{\orcidlink{0000-0003-0574-2313}}$, Tao Wang$^{\orcidlink{0000-0001-5532-3999}}$, Wenying Wen$^{\orcidlink{0000-0002-3098-4640}}$,~\IEEEmembership{Member,~IEEE}, Yuming Fang$^{\orcidlink{0000-0002-6946-3586}}$,~\IEEEmembership{Senior Member,~IEEE}
        % <-this % stops a space
\thanks{\textcolor{red}{\textbf{Notice:} the illustrations in this paper contain some Not Safe/Suitable For Work (NSFW) images, which have been covered with black blocks, but please view them with caution.}}
\thanks{
Kun Xu is with the College of Computer Science and Technology, Nanjing University of Aeronautics and Astronautics, Nanjing 210016, China (e-mail: xukun930@nuaa.edu.cn).

Yushu Zhang is with the College of Computer Science and Technology, Nanjing University of Aeronautics and Astronautics, Nanjing 210016, China (e-mail: yushu@nuaa.edu.cn).

Shuren Qi is with the Department of Mathematics, Chinese University of Hong Kong, Hong Kong, China (e-mail: shurenqi@cuhk.edu.hk).

Tao Wang is with the College of Computer Science and Technology, Nanjing University of Aeronautics and Astronautics, Nanjing 210016, China (e-mail: wangtao21@nuaa.edu.cn).

Wenying Wen is with the School of Computing and Artificial Intelligence, Jiangxi University of Finance and Economics, Nanchang 330013, China (e-mail: wenyingwen@sina.cn).

Yuming Fang is with the School of Computing and Artificial Intelligence, Jiangxi University of Finance and Economics, Nanchang 330013, China (e-mail: fa0001ng@e.ntu.edu.sg).
}% <-this % stops a space
% \thanks{Manuscript received XXXX XX, XXXX; revised XXXX XX, XXXX.}}
% \thanks{Corresponding author: Yushu Zhang (e-mail: yushu@nuaa.edu.cn).}
}

%%%%%%%%%%%%%%%%%%%%%%%%%%%%%%%%%%%%%%
% \author{Shuren~Qi,~Yushu~Zhang,~\IEEEmembership{Member,~IEEE,}~Chao~Wang,\\~Jiantao~Zhou,~\IEEEmembership{Senior~Member,~IEEE},~and~Xiaochun~Cao,~\IEEEmembership{Senior~Member,~IEEE}%
% %\author{Shuren~Qi,~Yushu~Zhang,~Chao~Wang,~Jiantao~Zhou,~and~Xiaochun~Cao% <-this % stops a space
% \IEEEcompsocitemizethanks{
% 	\IEEEcompsocthanksitem S. Qi is with the College of Computer Science and Technology, Nanjing University of Aeronautics and Astronautics, Nanjing, China, and also with the Institute of Information Engineering, Chinese Academy of Sciences, Beijing, China (e-mail: shurenqi@nuaa.edu.cn).
% 	\IEEEcompsocthanksitem Y. Zhang is with the College of Computer Science and Technology, Nanjing University of Aeronautics and Astronautics, Nanjing, China, and also with the Guangxi Key Laboratory of Trusted Software, Guilin University of Electronic Technology, Guilin, China (e-mail: yushu@nuaa.edu.cn).
% 	\IEEEcompsocthanksitem C. Wang is with the College of Computer Science and Technology, Nanjing University of Aeronautics and Astronautics, Nanjing, China (e-mail: c.wang@nuaa.edu.cn).
% 	\IEEEcompsocthanksitem J. Zhou is with the State Key Laboratory of Internet of Things for Smart City, and also with the Department of Computer and Information Science, Faculty of Science and Technology, University of Macau, Macau, China (e-mail: jtzhou@umac.mo). 
% 	\IEEEcompsocthanksitem X. Cao is with the School of Cyber Science and Technology, Shenzhen Campus of Sun Yat-sen University, Shenzhen, China (e-mail: caoxiaochun@mail.sysu.edu.cn).
% 	\IEEEcompsocthanksitem Corresponding author: Y. Zhang (e-mail: yushu@nuaa.edu.cn).}% <-this % stops an unwanted space
% \thanks{IEEE Transactions on Pattern Analysis and Machine Intelligence, 2022, doi: 10.1109/TPAMI.2022.3204971, link: ieeexplore.ieee.org/document/9881995/.}
% %\thanks{(Corresponding author: Yushu Zhang.)}}
% }

%%%%%%%%%%%%%%%%%%%%%%%%%%%%%%%%%%%%%%

% note the % following the last \IEEEmembership and also \thanks - 
% these prevent an unwanted space from occurring between the last author name
% and the end of the author line. i.e., if you had this:
% 
% \author{....lastname \thanks{...} \thanks{...} }
%                     ^------------^------------^----Do not want these spaces!
%
% a space would be appended to the last name and could cause every name on that
% line to be shifted left slightly. This is one of those "LaTeX things". For
% instance, "\textbf{A} \textbf{B}" will typeset as "A B" not "AB". To get
% "AB" then you have to do: "\textbf{A}\textbf{B}"
% \thanks is no different in this regard, so shield the last } of each \thanks
% that ends a line with a % and do not let a space in before the next \thanks.
% Spaces after \IEEEmembership other than the last one are OK (and needed) as
% you are supposed to have spaces between the names. For what it is worth,
% this is a minor point as most people would not even notice if the said evil
% space somehow managed to creep in.



% The paper headers
\markboth{K. Xu \MakeLowercase{\textit{et al.}}: Detecting Malicious Concepts Without Image Generation in AIGC} {K. Xu \MakeLowercase{\textit{et al.}}: Detecting Malicious Concepts Without Image Generation in AIGC}
% The only time the second header will appear is for the odd numbered pages
% after the title page when using the twoside option.
% 
% *** Note that you probably will NOT want to include the author's ***
% *** name in the headers of peer review papers.                   ***
% You can use \ifCLASSOPTIONpeerreview for conditional compilation here if
% you desire.



% The publisher's ID mark at the bottom of the page is less important with
% Computer Society journal papers as those publications place the marks
% outside of the main text columns and, therefore, unlike regular IEEE
% journals, the available text space is not reduced by their presence.
% If you want to put a publisher's ID mark on the page you can do it like
% this:
%\IEEEpubid{0000--0000/00\$00.00~\copyright~2015 IEEE}
% or like this to get the Computer Society new two part style.
%\IEEEpubid{\makebox[\columnwidth]{\hfill 0000--0000/00/\$00.00~\copyright~2015 IEEE}%
%\hspace{\columnsep}\makebox[\columnwidth]{Published by the IEEE Computer Society\hfill}}
% Remember, if you use this you must call \IEEEpubidadjcol in the second
% column for its text to clear the IEEEpubid mark (Computer Society jorunal
% papers don't need this extra clearance.)



% use for special paper notices
%\IEEEspecialpapernotice{(Invited Paper)}



% for Computer Society papers, we must declare the abstract and index terms
% PRIOR to the title within the \IEEEtitleabstractindextext IEEEtran
% command as these need to go into the title area created by \maketitle.
% As a general rule, do not put math, special symbols or citations
% in the abstract or keywords.
\IEEEtitleabstractindextext{%
\begin{abstract}
The task of text-to-image generation has achieved tremendous success in practice, with emerging concept generation models capable of producing highly personalized and customized content. Fervor for concept generation is increasing rapidly among users, and platforms for concept sharing have sprung up. The concept owners may upload malicious concepts and disguise them with non-malicious text descriptions and example images to deceive users into downloading and generating malicious content. The platform needs a quick method to determine whether a concept is malicious to prevent the spread of malicious concepts. However, simply relying on concept image generation to judge whether a concept is malicious requires time and computational resources. Especially, as the number of concepts uploaded and downloaded on the platform continues to increase, this approach becomes impractical and poses a risk of generating malicious content. In this paper, we propose Concept QuickLook, the first systematic work to incorporate malicious concept detection into research, which performs detection based solely on concept files without generating any images. We define malicious concepts and design two work modes for detection: concept matching and fuzzy detection. Extensive experiments demonstrate that the proposed Concept QuickLook can detect malicious concepts and demonstrate practicality in concept sharing platforms. We also design robustness experiments to further validate the effectiveness of the solution. We hope this work can initiate malicious concept detection tasks and provide some inspiration.
\end{abstract}

% Note that keywords are not normally used for peerreview papers.
\begin{IEEEkeywords}
Malicious Concept Detection, Concept QuickLook, Text-to-image Generation, AIGC.
\end{IEEEkeywords}}


% make the title area
\maketitle


% To allow for easy dual compilation without having to reenter the
% abstract/keywords data, the \IEEEtitleabstractindextext text will
% not be used in maketitle, but will appear (i.e., to be "transported")
% here as \IEEEdisplaynontitleabstractindextext when the compsoc 
% or transmag modes are not selected <OR> if conference mode is selected 
% - because all conference papers position the abstract like regular
% papers do.
\IEEEdisplaynontitleabstractindextext
% \IEEEdisplaynontitleabstractindextext has no effect when using
% compsoc or transmag under a non-conference mode.



% For peer review papers, you can put extra information on the cover
% page as needed:
% \ifCLASSOPTIONpeerreview
% \begin{center} \bfseries EDICS Category: 3-BBND \end{center}
% \fi
%
% For peerreview papers, this IEEEtran command inserts a page break and
% creates the second title. It will be ignored for other modes.
\IEEEpeerreviewmaketitle



\IEEEraisesectionheading{\section{Introduction}\label{sec:intro}}
% Computer Society journal (but not conference!) papers do something unusual
% with the very first section heading (almost always called "Introduction").
% They place it ABOVE the main text! IEEEtran.cls does not automatically do
% this for you, but you can achieve this effect with the provided
% \IEEEraisesectionheading{} command. Note the need to keep any \label that
% is to refer to the section immediately after \section in the above as
% \IEEEraisesectionheading puts \section within a raised box.




% The very first letter is a 2 line initial drop letter followed
% by the rest of the first word in caps (small caps for compsoc).
% 
% form to use if the first word consists of a single letter:
% \IEEEPARstart{A}{demo} file is ....
% 
% form to use if you need the single drop letter followed by
% normal text (unknown if ever used by the IEEE):
% \IEEEPARstart{A}{}demo file is ....
% 
% Some journals put the first two words in caps:
% \IEEEPARstart{T}{his demo} file is ....
% 
% Here we have the typical use of a "T" for an initial drop letter
% and "HIS" in caps to complete the first word.



\IEEEPARstart{T}{here} have been immense advances in generative models that use text as an input condition and control the direction of generation in recent years, especially the text-to-image (T2I) generation models like Stable Diffusion (SD) \cite{sdpaper}. Diffusion models are extremely realistic in generating images that give a detailed description through text \cite{ramesh2022}, \cite{NEURIPS2022_ec795aea}, \cite{bar2023multidiffusion}, \cite{10841434}. With the rapid development of these generative models, there is a demand for the model-generated content to have personalization and customization \cite{ma2023subject, shi2023instantbooth}. This requires the models to be subject-driven as in Textual Inversion \cite{gal2022image} and Dreambooth \cite{Ruiz_2023_CVPR}, as opposed to the normal text-image generation models. Currently, expanding on the above generation models is defined as \textit{concept generation model} \cite{liu2023cones}, \cite{Kumari_2023_CVPR}, \cite{liu2023cones2}, \cite{li2024blip}, \cite{10.1145/3659578}, which abstracts the subject to be personalized into a concept that makes the model generate content that is difficult to describe in text.

\textbf{Generating and Applying Concepts.} The workflow of the concept generation model based on SD can be divided into two parts: \textit{generating concepts} and \textit{applying concepts}. The former is the process of extracting the concept from several image examples of the concept \cite{gal2022image, Ruiz_2023_CVPR, 10489849}, while the latter focuses on how to create new content from the extracted concept \cite{ma2023subject, safaee2023clic}. With the development of related community platforms such as the \textit{Civitai} and \textit{Hugging Face}, concept owners can easily upload their extracted concept files and share them with other users for creation.

\textbf{Understandable and Non-understandable Concepts.} During the process of uploading a concept to the platform by the owner, the concept file cannot be visualized to the user by itself because it is an embedding consisting of one or more vectors. Usually, it is necessary to attract users to download and utilize it in the form of descriptive texts and example diagrams. The users build up a ``concept'' of an object in their brains through the example diagrams and text descriptions presented, which are described in this paper as \textit{user understandable concepts} (UUC). In addition, users will make a natural correspondence between this ``concept'' and the concept files provided by the platform for download. Without example diagrams and text descriptions, it would be difficult to understand and differentiate between the concepts, and this paper describes them as \textit{user non-understandable concepts} (UNC).
\begin{figure*}[h]
    \centering
    \includegraphics[width=\linewidth]{fig1-overview}
    \caption{Overview. \textit{\textbf{Top}}: The left is the special case, where the actual concept file is malicious, but it is presented in a harmless form after disguise and embellishment, it will generate harmful content. The right is the general case, where the actual concept file mismatches the concept descriptions, generating images that are not user required. \textit{\textbf{Bottom}}: The left shows the inefficient method of determining by \underline{generating images at least once}. On the right is Concept QuickLook, which achieve \underline{directly judge without generating any images}.}
    \label{figure1-overview}
\end{figure*}

\textbf{Security Risks and Mitigation Strategies.} Concept generation models are developing rapidly, there is a widespread concern about the security risks involved in both the processes of generating concepts and applying concepts \cite{Gandikota_2024_WACV}, \cite{Van_Le_2023_ICCV}, \cite{zhang2023backdooring}, \cite{Kumari_2023_ICCV}, \cite{lyu2023onedimensional}. Some security issues are gradually exposed \cite{Degeneration-Tuning}, \cite{feng2023catch}, \cite{tsai2023ring}. While extracting concepts, on one hand, malicious concepts that violate ethics and laws may be extracted, coupled with the owner's malicious disguise as innocuous concepts uploaded to the platform. On the other hand, users are confused by normal text descriptions and example diagrams when applying concepts and are prone to generate images containing malicious content. A security concern regarding the concept is illustrated in Figure \ref{figure1-overview} (\textit{Top}). We explain the implementation logic for disguising and embellishing malicious concepts, as well as the malicious mismatch of concepts.

\textbf{Our Work.} We assume a scenario in which the owner is malicious, extracts the offending concepts, and generates a malicious concept file that is then uploaded to the platform. The user downloads the malicious concept file for generating personalized images based on misleading text descriptions and example diagrams, but the user cannot effectively determine whether the concept file generates harmful content before the image is generated. Moreover, it may take more than one generation to determine whether a concept file is malicious. This can be potentially jeopardized for the user, and also significantly undermines the rights and credibility of the platform. It is necessary to detect malicious concepts, i.e., the platform side needs to provide users with a way to distinguish between them. To address these issues, we propose the \textbf{Concept QuickLook}.

Concept QuickLook is a solution that informs the user’s judgment regarding the potential risks of a concept file (Figure \ref{figure1-overview} (\textit{Bottom})). It also assesses whether the generated image may contain malicious content before the user utilizes the concept file for personalized creative generation. Our work is the first to clearly define malicious concepts and propose targeted detection solutions. We define two cases of the malicious concepts: \textit{special} and \textit{general}.
(\romannumeral 1) \textit{The first case involves concepts that are inherently malicious and generate harmful content, but are disguised and embellished}.
(\romannumeral 2) \textit{The second case involves concepts that are maliciously mismatched with their text descriptions and example diagrams, preventing the generation of the desired content for the user}. Concept QuickLook employs two detection workflows: one that detects if the concept itself is malicious by combining the concept’s description to provide a judgment, and another that judges whether the concept belongs to a known concept class without relying on the description. To evaluate the performance of Concept QuickLook, we conducted extensive experiments, covering model training, detection performance of the two detection modes, and robustness assessment. We carried out targeted case studies for both special and general malicious concepts. In addition to using quantitative metrics for evaluation, we also introduced human evaluation methods to better align with user needs. The experimental results demonstrate the effectiveness of our work. The major contributions of this paper are summarized as follows:
\begin{itemize}
\item We are the first to define the malicious concept in the concept sharing process, and propose a solution called Concept QuickLook to achieve quickly malicious concept detection.

\item We analyze the generation mechanism of the concept generation model and the whole process of concept file sharing. The embedding vectors in the concept file are found to be the primary factor controlling the content of the generated subjects, and it can serve to detect whether the personalized generated content is malicious or not.

\item We design two work modes for the QuickLook model: concept matching and fuzzy detection. This paper demonstrates that these two modes effectively meet the requirements for malicious concept detection in the current concept sharing platform scenario.

\item We conducted extensive experiments, including effectiveness evaluation, baseline comparisons, manual scoring, and robustness testing. The results demonstrate that our proposed method implements the ability to distinguish malicious concepts without even one generation, which effectively protects concept sharing platforms and users.
\end{itemize}

\section{Related Works}


We briefly review the closely related text-to-image generation models, generating concepts and applying concepts.

\textbf{Text-to-image Generation Models.} AI-generated content (AIGC) \cite{cao2023comprehensive}, \cite{xu2024unleashing}, \cite{wang2023security}, \cite{10230895} has experienced significant expansion in recent years. Image generation has evolved from the pursuit of high resolution and more realism to a more flexible and broadly applicable to a variety of tasks, which allows for customized generation to fit user needs. GAN-based (Generative Adversarial Network \cite{goodfellow2014generative}) generation models have previously achieved better generation quality, and the images generated by GAN are usually highly realistic and lifelike \cite{10.1145/3439723}. GAN-INT-CLS \cite{reed2016generative} was the first to realize GAN-based text-to-image generation, and many subsequent works \cite{Cheng9156682}, \cite{Huang2021Unifying}, \cite{Ruan9710042} have demonstrated its powerful generative capabilities. However, GAN faces limitations such as poor performance during complex scenes and unstable training.

With diffusion probabilistic model (DM) \cite{pmlr-v37-sohl-dickstein15}, denoising diffusion probabilistic model (DDPM) \cite{NEURIPS2020_4c5bcfec}, latent diffusion model (LDM) \cite{sdpaper} and further development to SD, diffusion models have received much attention in image generation due to their powerful capabilities \cite{Croitoru10081412, Yang2023Diffusion}. An OpenAI study shows that diffusion models beat GANs in image synthesis \cite{NEURIPS2021_49ad23d1}. VQ-Diffusion \cite{Gu9879180} is a T2I generation model based on a vector quantized variational autoencoder (VQ-VAE) with a latent space modeled by a conditional variant of DDPM, which can handle more complex scenes and dramatically improve image quality. Imagen \cite{NEURIPS2022_ec795aea} achieves remarkable photorealism and a profound level of language understanding, with unprecedented realism and ensures precise text alignment. InstaFlow \cite{liu2024instaflow} is a proposed novel T2I generation model to turn SD into an ultra-fast text-conditioned pipeline. A recent research work HIVE \cite{zhang2023hive} attempts to combine T2I modeling with human feedback to control the output of generated images through editorial commands. Further, a comprehensive benchmark for open-world compositional T2I generation, T2I-CompBench \cite{NEURIPS2023_f8ad010c}, has been proposed.

\textbf{Generating Concepts.} Generating concepts is often described as a process where multiple images of shared concepts are used for concept extraction through well-designed models, such as Textual Inversion \cite{gal2022image}, DreamBooth \cite{Ruiz_2023_CVPR} and Lora-DreamBooth \cite{hu2022lora}. It is considered as an extension of the T2I generation task. The generative concepts technique is first directed toward extracting the concept of an entity \cite{10.1145/3618315}, e.g., a person, an automobile, a ragdoll toy, and so on. Cones \cite{liu2023cones, liu2023cones2}, Break-A-Scene \cite{Avrahami2023Break} and Custom Diffusion \cite{Kumari_2023_CVPR} implement multi-concept extraction. CatVersion \cite{zhao2023catversion} learns the personalization concept by concatenating embeddings on the feature-dense space of a text encoder in the diffusion model. For the extraction of style concepts, StyleDrop \cite{NEURIPS2023_d33b177b} is a method that faithfully follows a specific style and effectively learns new ones. InST \cite{Zhang_2023_CVPR} perceives styles as learnable text descriptions of paintings, which can efficiently and accurately learn key artistic style information about an image. Specialist Diffusion \cite{Lu_2023_CVPR} is a plug-and-play framework for style concept extraction with further performance improvements.

\textbf{Applying Concepts.} Those working on how to better extract concepts usually demonstrate the excellent generative power of their methods. CLiC \cite{safaee2023clic} performs contextual concept learning on the concept within the inclusion mask and the surrounding image region to acquire local visual concept. Subject-Diffusion \cite{ma2023subject} is a novel open-domain personalized image generation model that requires only a single reference image to support single-subject or multi-subject personalization generation with no need for test-time fine-tuning. Moreover, one kind of application phase of concepts can be summarized as concept assisting. Anti-DreamBooth \cite{Van_Le_2023_ICCV} destroys the generation quality of any DreamBooth model trained against malicious use by adding perturbations to the images. Concept censorship \cite{zhang2023backdooring} proposes to inject backdoors into Textual Inversion embeddings and select some sensitive words as triggers during training so that the models do not generate malicious images.

\section{Problem Formulation}
In this section, we first introduce the preliminary work, which mainly includes the implementation logic of concept extraction and the introduction of the three roles involved in the concept sharing process. Next, we define malicious concepts and present the special and general cases of malicious concepts. Finally, we introduce the threat model from two aspects: the malicious concept owner’s capabilities and objectives, and the concept sharing platform’s capabilities and requirements.
\begin{table}
    \centering
    \caption{Notations and Definitions}
    \scalebox{1}{
    \begin{tabular}{|l||l|}
        \hline
 		Notation & Definition \\
        \hline
 	% \midrule
            $C, C_n, C_m$ & The concept, the normal/malicious concept \\
            $C_f$ & The concept file \\
		$C_T$ & The set of confirmed concept \\
            $e, e^c$ & The concept embedding vector \\
            $\mathcal{R}_{cls}$ & The concept class \\
            $\mathcal{E}$ & The embedding space \\
		$\epsilon_\theta$ & The conditional denoising autoencoder \\ 
		$T_d, E_d$ & The text description, the example diagram  \\
		$C_{f_u}, C_{f_n}, C_{f_m}$ & The unknown/normal/malicious concept file \\
		$V^{*}$ & The pseudo-word \\
		$\mathcal{D}$ & The training set \\
		$\sigma$ & The QuickLook model \\
            $\Gamma _{ext}$ & The concept extraction model \\
            $\rm{Q}$ & The configuration information in concept file \\
            \hline
		$\mathcal{O}$ & The concept owner \\
		$\mathcal{P}$ & The concept sharing platform \\
		$\mathcal{U}$ & The user of concept sharing platform \\
     \hline
    \end{tabular}
    }
    \label{tab:ND}
\end{table}
\subsection{Preliminary}
LDM utilizes an image encoder $\varphi$ to convert images $x\in \mathbb{R}^{H \times W \times 3} $ into a spatial latent code $z=\varphi(x)$. The decoder $D_e$ learns to map such latents back to images. A conditional denoising autoencoder $\epsilon_\theta(z_t,t,c_\theta(y))$ generates images given a specific text $y$ as a condition. $c_\theta(y)$ is a model that maps a conditioning input $y$ into a conditioning vector. The loss of conditional LDM is designed as:
\begin{equation}\label{eq1}
L_{LDM}:=\mathbb{E}_{z\sim\varphi(x),y,\epsilon\sim\mathcal{N}(0,1),t}\Big[\|\epsilon-\epsilon_\theta(z_t,t,c_\theta(y))\|_2^2\Big],
\end{equation}
where the time steps $t \in \{ 1,...,\mathcal{T} \}$, $z_t$ is the latent noised to time $t$, $\epsilon$ is the unscaled noise sample from the Gaussian distribution $\mathcal{N}(0,1)$.

The goal of Textual Inversion is to convert new concepts into pseudo-words in the text encoder embedding. For Textual Inversion, the direct optimization of the LDM loss is used:
\begin{equation}\label{eq2}
e^{*}=\mathop{\arg\min}_{e} \mathbb{E}_{z\sim \varphi(x),y,\epsilon\sim \mathcal{N}(0,1),t}{\Big[}||\epsilon-\epsilon_{\theta}(z_{t},t,c_{\theta}(y))||_{2}^{2}{\Big]},
\end{equation}
where $e^{*}$ is the embedding vector. $e^{*}$ is found through direct optimization by minimizing the LDM loss. In conducting general and intuitive conditional text editing, we designate a placeholder string $V^{*}$ (pseudo-word) to represent the newly learned concept. Table \ref{tab:ND} lists the notations and their definitions for this paper.

The embedding space $\mathcal{E}$ is a high-dimensional vector space where each vector represents a specific concept, text description, or image feature \cite{pmlr-v139-radford21a, gal2022image}. By operating within this space, connections and transformations between different modalities can be achieved. We align the characterization with practical scenarios, and the interaction process of the three parties involved in concept sharing is illustrated in Figure \ref{figure2-roles}. We show a scenario where a concept owner uploads the concept file with a text description and example diagram to the platform, and a user downloads the concept file in interest to local area and generates a concept generative image. The three roles shown work together to advance the sharing and exchange of concepts \cite{feng2023catch}.
\begin{figure}[t]
    \centering
    \includegraphics[width=\linewidth]{fig2-roles}
    \caption{Introduction to the three roles of owner, platform and user in concept sharing.}
    \label{figure2-roles}
\end{figure}

\textbf{Role} \ding{182}: the concept owner $\mathcal{O}$. We assume that the owner has a certain level of technical proficiency, enabling them to extract concepts locally, upload and share concept files, and provide the text descriptions $T_d$ and example images $E_d$ for the concepts. The process of concept extraction can be formulated as:
\begin{equation}\label{eq-1}
e^c = {\Gamma _{ext}}(\varphi ({\rm{\mathbf{x}}})),{\rm{\mathbf{x}}} = \{ ({x_1},{x_2},...,{x_i}) \}^c ,
\end{equation}
where ${\Gamma _{ext}}$ is the concept extraction model, optimized through Equation (\ref{eq2}). Ultimately, the concept file shared by the owner can be represented as ${C_f} = \{ ({e^c},\rm{Q})\}$. $e^c$ represents a particular concept embedding. $\rm{Q}$ represents other configuration information included in the concept file. We set the superscript $c$ to the symbol to indicate that the symbol is associated with a defined concept.

\textbf{Role} \ding{183}: the concept sharing platform $\mathcal{P}$. We assume that the platform to receive the owner's concept file $C_f$, as well as text descriptions $T_d$ and example images $E_d$. The task of the platform is to make $T_d$ and $E_d$ (corresponding to UUC) available for showcase, and the corresponding $C_f$ (corresponding to UNC) available for download, in addition to providing pseudo-word $V^{*}$. Therefore, a concept can be expressed as $C \leftarrow ({C_f},{T_d},{E_d})$.

\textbf{Role} \ding{184}: the concept sharing platform user $\mathcal{U}$. We assume that the user browses concepts on the sharing platform by reviewing text descriptions $T_d$ and example images $E_d$, downloads concept file $C_f$ to their local devices, utilizes them for personalized concept image generation.
\begin{figure}
    \centering
    \includegraphics[width=\linewidth]{fig3-cases}
    \caption{Comparison of two cases for the malicious concept.}
    \label{figure3-cases}
\end{figure}
\subsection{Malicious Concept Definition}
We mentioned that concepts are sharing on the platform in the form of concept files (``.pt'' and ``.safetensors'' files). We are aware that files may contain malicious scripts, etc., but this is not the case for the malicious concepts defined. Concept files $\{C_f\}=\{C_{f_n}, C_{f_m} \}$, where $C_{f_n}$ is normal, $C_{f_m}$ is malicious. The pseudo-word represents a ``concept'' that is subjectively portrayed by the user through $T_d$ and $E_d$. The normal concept can be expressed as $C_n \leftarrow (C_{f_n}^c,{T_d^c},{E_d^c})$, and the malicious concept can be expressed as $C_m \leftarrow (C_{f_m}^{cm},{T_d^c},{E_d^c})$, where the superscript $cm$ indicates association with the malicious concept.

We are concerned with whether the concept files produce malicious or incorrect content when they are used for personalized and customized image generation. Therefore, we establish two cases for the definition of malicious concepts.
\begin{tcolorbox}
	[breakable,		                    %支持跨页
	%drop shadow southeast, enhanced,    % 阴影面积颜色调整 shadow={4mm}{-3mm}{0mm}{black!50!white},
	% colback= orange!10!white,		            % 背景颜色,!20表示百分比
	% colframe=yellow!30,					% 边框颜色
	%width=5cm,							% 边框的宽度,可自行调整
	arc=0mm, auto outer arc,            % 圆角的大小
	boxrule= 0pt,                        % 边框的厚度 bottomrule
	boxsep = 0mm,                       % 文字与盒子的距离 
	left = 1mm, right = 1mm, top = 1mm, bottom = 1mm, 
	]%\textcolor{black}                   % 文字颜色
	{\textbf{Definition Case 1: \textit{Special Malicious Concept}.} The concept owners extract NSFW concepts such as pornography, violence, and gore, and upload them to the concept sharing platform after they have been embellished and disguised.}
 \end{tcolorbox}

\begin{tcolorbox}
	[breakable,		                    %支持跨页
	%drop shadow southeast, enhanced,    % 阴影面积颜色调整 shadow={4mm}{-3mm}{0mm}{black!50!white},
	% colback= orange!10!white,		            % 背景颜色,!20表示百分比
	%colframe=yellow!30,					% 边框颜色
	%width=5cm,							% 边框的宽度,可自行调整
	arc=0mm, auto outer arc,            % 圆角的大小
	boxrule= 0pt,                        % 边框的厚度 bottomrule
	boxsep = 0mm,                       % 文字与盒子的距离 
	left = 1mm, right = 1mm, top = 1mm, bottom = 1mm, 
	]%\textcolor{black}                   % 文字颜色
	{\textbf{Definition Case 2: \textit{General Malicious Concept}.} The concept owners maliciously mismatch the concept files with their text descriptions and example diagrams when uploading to the platform, but the concepts themselves are innocuous and no malicious content will be generated.}
 \end{tcolorbox}

The comparison of the two cases of malicious concept generation as set in our framework is shown in Figure \ref{figure3-cases}. It should be noted that in Case 1, the concept extracted itself is malicious, and the owner embellishes their concept with innocuous $T_d$ and $E_d$. This case of malicious concepts emerge from the process of concept extraction. As for Case 2, it represents a broader definition of malicious concepts, where users cannot generate the required images. This approach is cost-effective and more widespread. This type of malicious concepts emerge after the process of concept extraction, where the owner maliciously mismatches the concept with $T_d$ and $E_d$.

There has been extensive discussion about the SD safe filter on enthusiast forums, with some users complaining about false triggers and considering it a nuisance. In these forums, users also share methods to bypass these safety checks, and our experiments confirm that such checks can indeed be circumvented. Additionally, the malicious concepts defined in Case 2 are inherently difficult for the safe filter to detect, as they do not contain NSFW content.
\begin{figure*}[t]
    \centering
    \includegraphics[width=\linewidth]{fig4-outline}
    \caption{Outline of the Concept QuickLook. The top part of the illustration presents the Concept QuickLook workflow and the QuickLook model training process. The bottom part provides a detailed and intuitive description.}
    \label{figure4-outline}
\end{figure*}

\subsection{Threat Model}
In this work, we focus on the scenario of malicious concept detection. This is a completely new endeavor. We specify our threat model from two aspects: the malicious concept owner’s capabilities and objectives \& the concept sharing platform’s capabilities and requirements.

\textbf{Malicious Concept Owner’s Capabilities and Objectives.} A concept upload is abstracted into three parts, whose constituent datasets can be denoted as $ \{ (C_f, T_d, E_d) \} $. In $C_f$ carrying the concept information is the embedding vector $e^c$. In this paper, we consider the case where the concept owner is malicious, i.e., they upload a file containing a malicious concept and mislead the platform and users through normal text descriptions $T_d$ and example diagrams $E_d$. Additionally, another general type of malicious concept involves the owner maliciously mismatching the concept file $C_f$ with $T_d$ and $E_d$, preventing users from generating the required concept content. In detail, since the concept file is not directly observable for its content, the malicious owner extracts the concept and disguises it as a normal upload. Malicious concept owners share disguised concepts, causing users to generate images containing malicious content, thereby undermining the credibility of the concept sharing platform, eroding user trust in the platform.

\textbf{Concept Sharing Platform’s Capabilities and Requirements.} The challenge previously was the lack of a definition for what constitutes malicious concepts and the absence of a solution for detecting them. Users could only judge whether a concept was malicious by using the generated images. We assume that the platform is trustworthy and committed to maintaining its reputation by defending the rights of all parties. The platform would like to pre-determine whether an uploaded concept file will generate malicious content, and accordingly provide risk alerts to users who need to download it or simply delete the concept file. We assume that the platform needs to automatically perform malicious concept detection tasks, which cannot be quickly achieved through image generation from concept embeddings. The platform requires a technical solution to rapidly complete the detection of concept files with minimal resource consumption. Moreover, we attempted to contact the support teams of the current mainstream concept sharing platforms (\textit{Civitai} and \textit{Hugging Face}) to express our concerns about the potential spread of malicious concepts on their platforms and to inquire about existing solutions. However, we did not receive any response.

\section{Concept QuickLook}
\subsection{QuickLook Model Implementation}
Our data acquisition methods are mainly divided into: local training \cite{gal2022image} and platform downloads (\textit{Civitai} and \textit{Hugging Face}). We perform content generation checks on the downloaded concept files and label all data with concept classes, symbolically denoted as $\mathcal{D}=\{ (C_f, \mathcal{R}_{cls})_N \}$. Additionally, it should be noted that the number of embedding vectors for concepts during the extraction process is a parameter that is set manually. We analyzed many concept files on the concept sharing platform and found that the number of vectors is indeed a variable value. We symbolically represent it as:
\begin{equation}\label{eq4}
C \to {C_f} = ( e^c_n,\rm{Q}) , n \in {\mathbb{Z}^ {+}},
\end{equation}
where $e^c_n$ represents the embedding of $n$ vectors for the concept. The outline of Concept QuickLook is illustrated in Figure \ref{figure4-outline}. The core functionality of Concept QuickLook relies on the QuickLook model $\sigma $ and the embedding vector-concept class pairs $\langle {e^c} \odot {{\cal R}_{cls}}\rangle$, as shown in Equation (\ref{eq-map}). In practice, the embedding vector-concept class pairs are derived from our constructed dataset $\mathcal{D}$. Algorithm \ref{algo_QL} shows the QuickLook model training. The $\sigma$ learns these pairs through training, enabling it to output the corresponding concept class when given an input embedding vector. Specifically, the actual composition of dataset $\mathcal{D}$ consists of concept files $C_f$ and labeled concepts class $\mathcal{R}_{cls}$. It is necessary to extract cf and convert it into embedding vectors $e^c$, and then pair $e^c$ with $\mathcal{R}_{cls}$,
\begin{equation}\label{eq-map}
\langle {e_i^c} \odot {{\cal R}_{cls}}\rangle  = Map_{\mathcal{E}}({e_i^c},{{\cal R}_{cls}}).
\end{equation}

\begin{algorithm}[t]
    \caption{QuickLook Model Training}
    \label{algo_QL}
    \begin{algorithmic}[1]
        \REQUIRE Concept file $C_f$ and concept class $\mathcal{R}_{cls}$, where $C_f, \mathcal{R}_{cls} \in \mathcal{D}$; Gradient descent steps $M$
        \ENSURE Trained QuickLook model $\sigma$
        \STATE \textbf{Procedure:} \textsc{QuickLookTrain}($C_f, \mathcal{R}_{cls}$)
        \STATE $\sigma \gets$ InitializeModel(), $S \gets \{\}$
        \FOR{each data point in $\mathcal{D}$}
            \IF{$C_f = \emptyset$}
                \RETURN NULL
            \ENDIF
            \IF{ConceptFileCheck($C_f$)}
                \STATE $(e^c, \mathcal{R}_{cls}) \gets$ Concept2Embedding($C_f, \mathcal{R}_{cls}$)
                \STATE $S \gets S \cup \{ (e^c, \mathcal{R}_{cls}) \}$
            \ENDIF
        \ENDFOR
        \FOR{$i \gets 1$ to $M$}
            \FOR{each $(e^c, \mathcal{R}_{cls})$ in $S$}
                \STATE $\hat{\mathcal{R}}_{cls} \gets \sigma(e^c)$
                \STATE $\mathcal{L} \gets$ LossFunction($\hat{\mathcal{R}}_{cls}, \mathcal{R}_{cls}$)
                \STATE UpdateModel($\sigma, \mathcal{L}$)
            \ENDFOR
        \ENDFOR
        \RETURN $\sigma$
    \end{algorithmic}
\end{algorithm}

In this approach, our objective is to train a model that can output the nearest embedding vector and its corresponding concept class given an input embedding vector. To achieve this, we will optimize the model's parameters to perform well on both cosine similarity and cross-entropy loss. The cosine similarity is calculated as shown in Equation (\ref{eq-eij}), where $e^c_i$ and $e^c_j$ are two embedding vectors,
\begin{equation}\label{eq-eij}
\psi_{cs} (e^c_i, e^c_j) = \frac{e^c_i \cdot e^c_j}{\|e^c_i\| \|e^c_j\|}.
\end{equation}

To train the model, we define an appropriate loss function aimed at maximizing the similarity between embedding vectors of the same class while minimizing the similarity between embedding vectors of different classes. The following equation defines our loss function.
\begin{multline}\label{eq8}
\mathcal{L} = \mathop{\arg \min}\limits_{e_i^c, e_j^c} {\mathbb{E}_{e_i^c,e_j^c \sim \mathcal{E}}} {\frac{1}{\mathcal{W}}} \sum\limits_{i,j}^\mathcal{W} \Bigg[ \mathcal{Y}_{ij} \left(1 - \psi_{cs}(e_i^c, e_j^c)\right)^2 \\
+ \left(1 - \mathcal{Y}_{ij}\right) \max\left(0, \psi_{cs}(e_i^c, e_j^c) - \kappa\right)^2 \Bigg],
\end{multline}
where $\mathcal{W}$ is the total number of $\langle {e^c} \odot {{\cal R}_{cls}}\rangle$ pairs, and $\mathcal{Y}_{ij}$ is an indicator variable that denotes whether the samples $i$ and $j$ belong to the same concept class. If they belong to the same class $\mathcal{Y}_{ij}=1$; otherwise, $\mathcal{Y}_{ij}=0$. $\kappa$ is a hyperparameter representing the minimum distance boundary between vectors of different classes.

\subsection{Concept QuickLook Workflows}
QuickLook model is designed to make on-demand judgments about concept embeddings. For malicious concepts, we have defined two cases. Accordingly, Concept QuickLook has designed two types of workflows to address the threats posed by these malicious concepts. Figure \ref{figure5-workflows} shows two workflows for Concept QuickLook. The upper part illustrates the fuzzy detection of owner-uploaded concept files, which allows for quick screening to determine if the file is malicious, thereby protecting the platform's interests. The bottom part shows the process of determining whether an unknown concept belongs to a specific known concept class.
\begin{figure}[t]
    \centering
    \includegraphics[width=\linewidth]{fig5-workflows}
    \caption{Two workflows of Concept QuickLook. TYPE 1: \textit{Concept Matching} is aimed at known concepts' text descriptions and example images, requiring detection of whether the concept files match. TYPE 2: \textit{Fuzzy Detection} is aimed at cases where the text descriptions and example images of concepts are absent, and the requirement is to detect whether the unknown concept belongs to a specific concept class.}
    \label{figure5-workflows}
\end{figure}
\begin{tcolorbox}
	[breakable,		                    %支持跨页
	%drop shadow southeast, enhanced,    % 阴影面积颜色调整 shadow={4mm}{-3mm}{0mm}{black!50!white},
	% colback= green!10!white,		            % 背景颜色,!20表示百分比
	%colframe=yellow!30,					% 边框颜色
	%width=5cm,							% 边框的宽度,可自行调整
	arc=0mm, auto outer arc,            % 圆角的大小
	boxrule= 0pt,                        % 边框的厚度 bottomrule
	boxsep = 0mm,                       % 文字与盒子的距离 
	left = 1mm, right = 1mm, top = 1mm, bottom = 1mm, 
	]%\textcolor{black}                   % 文字颜色
	{\textbf{TYPE 1: \textit{Concept Matching}.} The essence is to find the closest vector in the embedding space for the input concept embedding vectors using the QuickLook model, and the output text depiction is labeled from the model training.}
 \end{tcolorbox}

For concepts that need to be quickly checked, output fuzzy text descriptions (concept classes) such as human beings, animals, etc. to provide a basis for the judgment. The main focus here is on the consistency of the concepts with the text description $T_d$ and the example diagram $E_d$. As shown in Equation (\ref{eq-match}), the symbol $g$ is used to denote the consistency relation, $\mathcal{R}_{cls}^{+}$ denotes a concept that is consistent with $h=\{ ({T_d},{E_d})^c\}$, and $\mathcal{R}_{cls}^{-}$ denotes a concept that is inconsistent with $h$.
\begin{equation}\label{eq-match}
g = {\rm{Matching}}_\sigma (h ,\mathcal{C}_{cls}),
\end{equation}

\begin{tcolorbox}
	[breakable,		                    %支持跨页
	%drop shadow southeast, enhanced,    % 阴影面积颜色调整 shadow={4mm}{-3mm}{0mm}{black!50!white},
	% colback= green!10!white,		            % 背景颜色,!20表示百分比
	%colframe=yellow!30,					% 边框颜色
	%width=5cm,							% 边框的宽度,可自行调整
	arc=0mm, auto outer arc,            % 圆角的大小
	boxrule= 0pt,                        % 边框的厚度 bottomrule
	boxsep = 0mm,                       % 文字与盒子的距离 
	left = 1mm, right = 1mm, top = 1mm, bottom = 1mm, 
	]%\textcolor{black}                   % 文字颜色
	{\textbf{TYPE 2: \textit{Fuzzy Detection}.} The essence is to judge the distance between concept embedding vectors, but it requires that the concept $e_T$ information is provided to the model in advance, which can be a single or a set of concepts of the same class.}
 \end{tcolorbox}

To determine whether a concept belongs to a certain class of concepts, e.g., to determine whether a concept claiming to be ``\textit{Taylor}'' is true or not. It is formally represented as $C_{f_u} \in \rm{or} \notin $ $\{ C_{T_1},C_{T_2},...,C_{T_n}\}$ , $n \in {\mathbb{Z}^ {+}}$, where the matching problem of concepts is essentially to determine the class attribution of concepts. Equation (\ref{eq-d}) and (\ref{eq-fd}) describe this process in detail:
\begin{equation}\label{eq-d}
d(e^c_i, e^c_j) = 1 - \psi_{cs} (e^c_i, e^c_j),
\end{equation}
\begin{equation}\label{eq-fd}
\mathcal{F}_d = \left\{
\begin{aligned}
    &{C_{{f_u}}} \in {C_T}, \quad \text{if } d \leq \delta \text{ and } d \geq 0 \\
    &{C_{{f_u}}} \notin {C_T}, \quad \text{otherwise}
    &,
\end{aligned}
\right.
\end{equation}
where $\delta$ represents the threshold. $C_{f_u}$ in the formula denotes the unknown concept, and $C_T=\{ {C_{{T_1}}},{C_{T_2}},...,{C_{T_n}}\}$ is the set of confirmed concepts, which is the set of similar concepts, and the number of concepts in the set is more than or equal to one. Algorithm \ref{algo-cmfd} shows the implementation of concept matching and fuzzy detection.

\begin{algorithm}[t]
    \caption{Concept Matching and Fuzzy Detection}
    \label{algo-cmfd}
    \begin{algorithmic}[1]
        \REQUIRE Concept file $C_f (C_{f_u}, C_T)$; Concept embedding vector $e^c$; QuickLook model $\sigma$; Text description $T_d$; Threshold $\delta$
        \ENSURE Concept matching text description $\mathcal{C}_{desc}$; Concept class attribution result
        \STATE \textbf{Procedure:} \textsc{ConceptMatching}($e^c, \sigma, T_d, \delta$)
        \STATE $\hat{\mathcal{R}_{cls}} \gets \sigma(e^c)$, $h \gets \{ (T_d, \cdot)^c \}$, $g \gets (h, \hat{\mathcal{R}_{cls}})$
        \IF {$g = \mathcal{R}_{cls}^{+}$}
            \STATE $\mathcal{R}_{desc} \gets \hat{\mathcal{R}_{cls}}$
        \ENDIF      
        \IF {$g = \mathcal{R}_{cls}^{-} \land \hat{\mathcal{R}_{cls}} \in \mathcal{R}_{cls}^{NSFW}$}
            \STATE $\mathcal{R}_{desc} \gets \hat{\mathcal{R}_{cls}}$
        \ENDIF        
        \STATE \textbf{Return} $\mathcal{R}_{desc}$
        
        \STATE \textbf{Procedure:} \textsc{FuzzyDetection}($C_{f_u}, \sigma, C_T, \delta$)
        \STATE $e_u \gets \text{Concept2Embedding}(C_{f_u})$       
        \FOR{each $C_{T_i} \in C_T$}
            \STATE $e_{T_i} \gets \text{Concept2Embedding}(C_{T_i})$
            \STATE $d_i \gets \text{CompareDistance}(\sigma(e_u, e_{T_i}))$
            \IF{$d_i \leq \delta$}
                \STATE \textbf{Return} $C_{f_u} \in C_T$
            \ENDIF
        \ENDFOR      
        \STATE \textbf{Return} $C_{f_u} \notin C_T$
    \end{algorithmic}
\end{algorithm}

\section{Experiments}
\subsection{Experimental Settings and Implementation}
This section describes the experimental settings of this paper in terms of models, dataset, metrics, baselines and implementation details.

\subsubsection{Experimental Settings}
\textbf{Models.} We make use of Stable Diffusion V1.5 and V2.0 (SD1.5 and SD2.0). Since Concept QuickLook does not directly utilize the SD model, the primary role of the model is to determine the embedding space and the validation of the concept file used to customize the dataset. Furthermore, SD1.5 and SD2.0 adopt different CLIP encoders \cite{pmlr-v139-radford21a}, and accordingly the encoders are different for concept extraction. As a result, the embedding space corresponding to concept generation is also different, and in practice it is necessary to focus on the specific model used.

\textbf{Datasets.} We did not have an existing dataset to utilize. Therefore, we constructed the required dataset by collecting as well as extracting the concepts from the platform itself. Specifically, there are two sources of experimental data. On the one hand, we download concept files from the Civitai and Hugging Face, record their pseudo-words and text descriptions, and categorize them (\textit{Downloads}). In addition, we use concepts to generate images to verify whether there are any inconsistencies with the example diagrams and text descriptions. On the other hand, we use the concept extraction model to generate concept files, and similarly, set pseudo-words and text descriptions, and organize them into classes (\textit{Customizations}). In the experiments, the dataset contains a total of 800 samples, with 500 Downloads and 300 Customizations. All concept classes in the dataset are manually labeled. The NSFW concept constitutes 15\% of the dataset, with 5\% originating from platform downloads and 10\% from concept extraction. NSFW concepts from Downloads are distinguished by whether malicious content is generated through the concept images, while the NSFW concepts from Customizations are differentiated by the input images prior to concept extraction.

It should be noted that NSFW concepts are not readily and publicly available because of their intrinsic peculiarities, so we use images from the dataset Gore-Blood-Dataset-v1.0 \cite{Gore-Blood-Dataset-v1.0} and Gore Dataset \cite{gore-kfldh_dataset} to extract the concepts.

\textbf{Metrics.} To evaluate Concept QuickLook, we use the following metrics: \underline{Accuracy}, \underline{FPR} (False Positive Rate), \underline{FNR} (False Negative Rate) and \underline{F1-score}. For the evaluation of the Concept Matching, we use Manual Scoring (\underline{MS}). Specifically, we use the \textit{Five-Point Likert Scale} as a scheme for manual scoring. We set the degree of conformity between the detection results of the text description output by the model and the actuality of the concept, where Strongly Conforms (5 points), Conforms (4 points), Neutral Conforms (3 points), Disconforms (2 points) and Strongly Disconforms (1 point). We received 123 valid questionnaires in the experimental evaluation of this paper.

Fuzzy Detection Score (\underline{FDS})  is a metric designed to evaluate the performance of a model in detecting concept ambiguity.
\begin{equation}
\rm{FDS} = 1 + \beta log_{2}(\hat{\beta}) - (1 - \beta)log_{2}(1-\hat{\beta}),
\end{equation}
where the final layer output of the model is $\hat \beta$, representing the probability that the input concept vector belongs to a confirmed concept class, $\beta$, which is the confirmed concept class. When the model's detection is accurate, $\beta$ is close to $\hat \beta$, resulting in a lower cross-entropy loss and an FDS value close to 1. Conversely, when the detection is inaccurate, the FDS value approaches 0.

\subsubsection{Baselines and Implementation Details}
\textbf{Baselines.} As this is the first paper addressing malicious concept detection, there are no existing works for comparison. We use nearest neighbor search (NNS) to replace the QuickLook model as the baseline. NNS is implemented using Faiss \cite{8733051}, the most advanced library for approximate nearest neighbor search, which provides efficient vector similarity search. The baseline experiments used the same data inputs and experimental environment as the Concept QuickLook. In addition, we simultaneously compared the detection method of concept generation images under the same conditions.

\textbf{Implementation Details.} Firstly, for the process of extracting concepts, We employ Textual Inversion \cite{gal2022image} as the concept extraction method $\Gamma _{ext}(\cdot)$, with the input set uniformly to four images. The required pre-trained models are SD1.5 and SD2.0. Secondly, the concept generation into images is a way to validate the concept files during the process of collecting our dataset, which requires a suitable SD model and relevant prompt. Thirdly, The performance, effectiveness, robustness etc. of malicious concept detection is described in the subsequent part of this paper. During the experimental evaluation of concept matching, we employed the MS metric, which involves using questionnaires to assess the results of concept matching. We conducted an ethical review of the questionnaire content, taking steps to minimize the inclusion of NSFW material by using necessary descriptions as substitutes and applying content masking to avoid psychologically distressing content. Prior to the evaluation, we fully informed participants of the purpose of the experiment and the precautions to be taken. Additionally, in the paper, we have implemented measures such as adding warnings and covering displayed content with black blocks. All results were produced on Nvidia GeForce RTX 3090Ti GPU with CUDA Version 12.0.

\begin{figure*}[t]
	\centering
	\includegraphics[width=\linewidth]{fig6-performance1}
	\caption{The illustration of QuickLook model detection results for the concept matching.}
	\label{figure-performance1}
\end{figure*}
\subsection{Concept Matching: Resolve Case 1}
To illustrate more intuitively the process of detecting malicious concepts by our QuickLook model, we conducted detailed experimental validation for TYPE 1. This section includes the effectiveness of Concept Matching and comparisons with baselines.

\begin{figure}[t]
	\centering
	\includegraphics[width=\linewidth]{fig7-twobar}
	\caption{The illustration of the MS points statistical distribution. (a) shows malicious concepts under Case 1, (b) shows non-malicious concepts.}
	\label{figure7-twobar}
\end{figure}
\subsubsection{Overview of Concept Matching Evaluation}
Due to the high degree of generalization (compression of information) in the extraction of the embedding vectors of the concepts, much of the information that characterizes the class of the target concepts is lost. Therefore, it is necessary for us to develop a concept matching function for the QuickLook model, which provides information about the concept's class quilt, so that by comparing it with the example diagram and text description of the concept, it can be determined whether the concept matches or not. For concept matching in the QuickLook model, we consider the scenario where it is necessary to quickly determine the \textit{concept class} to which a given concept belongs.

It is shown in Figure \ref{figure-performance1} the detection results of concept matching. The row of the concept example diagrams in the figure shows the content of a particular concept, which is a visualization of the concept. In relation to the reality of concept sharing platforms, its role is to give users a reference to the content of the images that may be generated using this concept. The concept example diagram is usually a set of images generated by the concept, and the images are chosen here for a clearer presentation of the set of images used in concept extraction (UUC). There are no concepts or concept files in Figure \ref{figure-performance1}, as they are not intuitively distinguishable. The concept matching results shown in the figure are class nouns of the concepts, such as person, dog, flower, etc., and the QuickLook model gives a broader class when it is not possible to directly output its specific concept class.
\begin{table*}[t]
    \centering
	\caption{Evaluation results on concept matching}
	\label{tab-ERCM}
        \scalebox{0.95}{
	\begin{tabular}{|c|c|c|c|l|c|c|}
		\hline
		Intuitional concept (UUC) & $V^{*}$ &  Actual concept (UNC) & $\mathcal{R}_{cls}$ & Concept text description (actual) & Detection results  & MS$\uparrow$ \\
  		\hline
            Taylor Swift & ts* & Gore & NSFW & ``a gore scene''  & NSFW &  4.1 ($\pm$0.1) \\
            Wangcai Dog & wc* & Terror & NSFW & ``a terror scene''  & NSFW &  4.3 ($\pm$0.1) \\
            Monster Toy & mt* & Gore \& Blood & NSFW & ``a gore and blood scene'' & NSFW &  4.0 ($\pm$0.3) \\
            Laevigata & la* & Terror & NSFW & ``a terror scene''  & NSFW &  4.1 ($\pm$0.2) \\
%%%%%%%%%%
            Chen Yao & cy* & Gore & NSFW & ``a gore and terror scene''  & NSFW &  4.0 ($\pm$0.1) \\
            Tesla Model Y & tmy* & Terror & NSFW & ``a terror scene''  & NSFW &  4.4 ($\pm$0.1) \\
            Black Mug & mug* & Gore \& Blood & NSFW & ``a gore and blood scene'' & NSFW &  3.9 ($\pm$0.2) \\
            Duck Toy & dt* & Terror & NSFW & ``a terror scene''  & NSFW &  4.1 ($\pm$0.2) \\

            Russian Blue & rb* & Gore & NSFW & ``a gore scene''  & NSFW &  4.3 ($\pm$0.1) \\
            Corgi & co* & Terror & NSFW & ``a terror scene''  & NSFW &  4.3 ($\pm$0.1) \\
            Chow Chow & cc* & Gore \& Blood & NSFW & ``a gore and blood scene'' & NSFW &  4.1 ($\pm$0.1) \\
            Shorthair & sh* & Terror & NSFW & ``a terror scene''  & NSFW &  4.5 ($\pm$0.2) \\
%%%%%%%%%%
\hline

            
		Taylor Swift & ts* & Monster Toy & Toy & ``a red monster toy''  & Toy &  3.1 ($\pm$0.3) \\
		Chen Yao & cy* & Tesla Model Y & Car & ``a Tesla car, the color is white''  & Car &  4.3 ($\pm$0.2) \\
		Wangcai Dog & wc* & Can & Can & ``a can of beverage''  & Can &  3.9 ($\pm$0.2) \\
		Backpack & bkpk* & Taylor Swift & Person & ``a beautiful lady with golden hair''  & Person &  4.5 ($\pm$0.1) \\
		% \hline
		Logi Mouse  & logi* & Logi Mouse  & Mouse & ``a Logitech mouse on the desktop''  & Mouse &  3.0 ($\pm$0.1) \\
		Car & su7* & Xiaomi SU7 & Car & ``a Xiaomi car, the color is Gulf Blue'' & Car &  4.2 ($\pm$0.1) \\
		Taylor Swift & ts* & Taylor Swift & Person & ``a beautiful lady with golden hair''  & Person & 3.4 ($\pm$0.1) \\
		Chen Yao & cy* & Chen Yao & Person & ``a beautiful lady with black hair'' & Person &  3.5 ($\pm$0.2) \\
		Backpack & bkpk* & Backpack & Backpack & ``a red backpack''  & Backpack &  4.1 ($\pm$0.1) \\
		Can & can* & Can & Can & ``a can of beverage'' & Can &  2.9 ($\pm$0.1) \\
		Wangcai Dog & wc* & Wangcai Dog & Dog & ``a yellow Shiba Inu''  & Dog & 4.8 ($\pm$0.2) \\
		Monster Toy & mt* & Monster Toy & Toy & ``a red monster toy''  & Toy &  3.1 ($\pm$0.3) \\

	\hline
	\end{tabular}}
\end{table*}

\subsubsection{Effectiveness of Concept Matching}
For the concept matching function of the QuickLook model, we perform the evaluation based on the effectiveness of the implementation of its function. The concept matching is a function of the model that provides the class to which the concept embedding vectors belong, and the concept class is typically broad because the concepts are extracted with only part of the key information and the concepts cannot be separated from the generative models (e.g., SD1.5 and SD2.0) and personalized for image generation alone. In this paper, manual scoring is chosen for evaluation in line with the reality, because the result of concept matching is to provide the users with a basis for judgment. We show the distribution of MS points in Figure \ref{figure7-twobar}. The manual scoring statistics for both malicious and non-malicious concepts give a high percentage of ratings for \textit{Neutral Conforms}, \textit{Conforms} and \textit{Strongly Conforms}.The MS point statistics highlight the effectiveness of concept matching.
\begin{table*}[t]
    \centering
	\caption{Comparison results of Concept QuickLook and baselines in the concept matching}
	\label{tab-CWB1}
        \scalebox{0.9}{
	\begin{tabular}{|l|c|c|c|c|c|c|c|c|}
		\hline
		Method & Accuracy & FPR & FNR & F1-score & MS $\uparrow$ & Time cost & By-products & Scalability \\
	\hline
  		Concept Generation  & 0.9001 ($\pm$0.08) & 0.1473 ($\pm$0.03) & 0.2398 ($\pm$0.05) & 0.5430  ($\pm$0.07) & 4.1 ($\pm$0.1) & 15.0s ($\pm$0.5) & Image & No \\
    
            Concept Generation (Unseen) & 0.7235 ($\pm$0.07) & 0.1589 ($\pm$0.03) & 0.2371 ($\pm$0.05) & 0.5032  ($\pm$0.07) & 3.8 ($\pm$0.3) & 15.0s ($\pm$0.5) & Image & No \\
            
		NNS  & 0.8834 ($\pm$0.05) & 0.1667 ($\pm$0.12) & 0.2553 ($\pm$0.09) & 0.4946 ($\pm$0.03)  & 3.8 ($\pm$0.2) & 1.0s ($\pm$0.1) & No & No \\
  
            NNS (Unseen)  & 0.6534 ($\pm$0.09) & 0.1967 ($\pm$0.10) & 0.2786 ($\pm$0.09) & 0.4611 ($\pm$0.03)  & 3.1 ($\pm$0.3) & 1.0s ($\pm$0.1) & No & No \\
            
            \rowcolor{gray!30}Concept QuickLook  & 0.9782  ($\pm$0.02) & 0.1126 ($\pm$0.03) & 0.2012 ($\pm$0.03) & 0.6012 ($\pm$0.02) & 4.7 ($\pm$0.1) & 1.5s ($\pm$0.2) & No & Yes \\
            
            \rowcolor{gray!30} Concept QuickLook (Unseen)  & 0.7751 ($\pm$0.04) & 0.1553 ($\pm$0.01) & 0.2133 ($\pm$0.05) & 0.5736 ($\pm$0.02) & 3.8 ($\pm$0.1) & 1.5s ($\pm$0.2) & No & Yes \\
  
		\hline
	\end{tabular}
     }
\end{table*}

We designed the Concept Matching evaluation experiment for user reviews. QuickLook model outputs a detection result for actual concepts (concepts that are input to the model), and the content of its result is the class of the concept. The users manually evaluated the fuzzy detection results of the model based on the Five-Point Likert Scale, and the score was calculated. As shown in Table \ref{tab-ERCM}, the concept matching evaluation results with manual scores are listed. For some statistics in the table, it should be pointed out that the first column represents the ``concept'' formed by the user through example diagrams and text descriptions. The third column represents the actual concepts (embedding vectors) that were actually detected. According to the user review results in the table, the output results of concept matching with the concepts meets and exceeds the ``Neutral Conforms'' in the MS. Some of the results that are not specific enough get low evaluation scores and those that accurately give the concept classes get high scores. Therefore, from the perspective of user review, the evaluation results of concept matching show their effectiveness.

\subsubsection{Comparisons with Baselines}
In our proposed approach, cosine similarity is used to calculate the distance between embedding vectors, and the QuickLook model learns the mapping relationship between embedding vectors and concept classes. Since this paper is the first to define and provide a detection method for malicious concepts, there are no existing works for comparison. Therefore, we establish a baseline by replacing the QuickLook model in our approach with NNS. Additionally, it is also compared with the concept image generation.

The NNS implemented with Faiss lacks learning capabilities, although it offers fast search speeds. We need to manually construct a mapping table between embedding vectors and concept classes. During concept matching, the embedding vector to be detected is searched through NNS to find the closest embedding vector in the table, and the concept class of the nearest embedding vector in the table is returned. Table \ref{tab-CWB1} presents the comparison results between Concept QuickLook and baselines in the concept matching scenario, demonstrating that our approach outperforms the baseline across all metrics. Moreover, our approach avoids generating image by-products.

% ---------------------------------------------

\subsection{Fuzzy Detection: Resolve Case 2}
This section evaluates the solution for malicious concept Case 2. To illustrate the process of our QuickLook model detecting malicious concepts more intuitively, we conducted detailed experimental validation on TYPE 2. This section includes the effectiveness of fuzzy detection and comparisons with the baselines.
\begin{figure*}[t]
	\centering
	\includegraphics[width=\linewidth]{fig8-performance2}
	\caption{The illustration of QuickLook model detection results for the fuzzy detection.}
	\label{figure8-performance2}
\end{figure*}
\subsubsection{Overview of Fuzzy Detection Evaluation}
For Fuzzy Detection in the QuickLook model, we consider the scenario when an unknown concept is obtained and it is required to know whether it belongs to a confirmed concept or set of concepts. We implement this need to judge an unknown concept as one of the functions in the QuickLook model. Specifically, the model has learned the features for a class of concepts and can output scores to provide a basis for judgment when unknown concepts need to be determined if they belong to that class. It is important to point out that the pseudo-words are in one-to-one correspondence with the extracted concepts, thus we only need to break this correspondence condition in our experiments to resolve Case 2, which produces malicious concepts.

The results of the visualization of the QuickLook model in the Fuzzy Detection are shown in Figure \ref{figure8-performance2}. The first row of the figure shows confirmed concept diagrams, where the represented concepts or concept sets have been identified, and the model has learned these concepts (UUC). The third row demonstrates the setup of pairs of same/different class concepts for each concept or concept set. The unknown concept example diagrams are displayed (in practice, the concept matching involves unknown concepts). Correspondingly, The second row describes the concept class consistency. Additionally, The bottom row displays the FDS, with the diagram showing one high-scoring same-class concept and one different-class concept in each set.
\begin{figure}[t]
	\centering
	\includegraphics[width=\linewidth]{fig9-lines}
	\caption{The illustration of the FDS distribution.}
	\label{figure9-lines}
\end{figure}

\subsubsection{Effectiveness of Fuzzy Detection}
Fuzzy detection is based on the unintuitive recognition problem that exists with concept files, where concepts are understood and recognized as dependent on the text descriptions and example diagrams provided by the owners. In this paper, we design an experiment to evaluate the effectiveness of fuzzy detection, aiming to achieve that by determining whether a concept vector belongs to a certain class of concepts and giving a score. Figure \ref{figure9-lines} shows the distribution of FDS, which is the proportion of FDS values between 0 and 1. The figure gives the evaluation results of the dataset and the unseen data in the case of concept class consistent/inconsistent. The FDS distribution statistics show the effectiveness of fuzzy detection.
\begin{table}[t]
    \centering
	\caption{Evaluation results on fuzzy detection. CC is short for class consistency, PWC stands for pseudo-word consistency}
	\label{tab-ERFD}
        \scalebox{0.83}{
	\begin{tabular}{|c|l|c|c|c|c|}
		\hline
		Confirmed $\mathcal{R}_{cls}$ &  Unknown $C$ & CC & Claimed $V^{*}$ & PWC & FDS \\ 
		\hline
		\multirow{4}{*}{Mouse} & Logi Mouse & \ding{51} & logi* & \ding{51} & 0.8523 ($\pm$0.02) \\
							   & Logi Mouse & \ding{51} & dell* & \ding{54} & 0.8445 ($\pm$0.02) \\
		 					   & Xiaomi SU7 & \ding{54} & - & - & 0.4362 ($\pm$0.03) \\
		 					   & Tesla Model Y & \ding{54} & - & - & 0.4578 ($\pm$0.03) \\
		\hline

            %%%%%%%%%%%%%%%%%%%
            \multirow{4}{*}{Flower} & Laevigata & \ding{51} & la* & \ding{51} & 0.8602 ($\pm$0.02) \\
							   & Laevigata & \ding{51} & pe* & \ding{54} & 0.8534 ($\pm$0.02) \\
		 					   & Taylor Swift & \ding{54} & - & - & 0.3526 ($\pm$0.02) \\
		 					   & Chen Yao & \ding{54} & - & - & 0.312 ($\pm$0.02) \\
		\hline

            \multirow{4}{*}{Backpack} & Red Backpack & \ding{51} & bkpk* & \ding{51} & 0.8943 ($\pm$0.02) \\
							   & Red Backpack & \ding{51} & cy* & \ding{54} & 0.8689 ($\pm$0.02) \\
		 					   & Dell Mouse & \ding{54} & - & - & 0.3332 ($\pm$0.03) \\
		 					   & Duck Toy & \ding{54} & - & - & 0.3476 ($\pm$0.02) \\
		\hline

            \multirow{4}{*}{Sneaker} & Nike Sneaker & \ding{51} & nike* & \ding{51} & 0.8532 ($\pm$0.03) \\
							   & Nike Sneaker & \ding{51} & bmug* & \ding{54} & 0.8645 ($\pm$0.03) \\
		 					   & Dell Mouse & \ding{54} & - & - & 0.4362 ($\pm$0.03) \\
		 					   & Tesla Model Y & \ding{54} & - & - & 0.4578 ($\pm$0.02) \\
		\hline

            \multirow{4}{*}{Glasses} & Sun Glasses & \ding{51} & sg* & \ding{51} & 0.9145 ($\pm$0.03) \\
							   & Sun Glasses & \ding{51} & tmy* & \ding{54} & 0.9140 ($\pm$0.03) \\
		 					   & Taylor Swift & \ding{54} & - & - & 0.3425 ($\pm$0.03) \\
		 					   & Corgi & \ding{54} & - & - & 0.3011 ($\pm$0.02) \\
		\hline

            \multirow{4}{*}{Cup} & Black Mug & \ding{51} & bmug* & \ding{51} & 0.8887 ($\pm$0.03) \\
							   & Black Mug & \ding{51} & su7* & \ding{54} & 0.8845 ($\pm$0.03) \\
		 					   & Monster Toy & \ding{54} & - & - & 0.4062 ($\pm$0.03) \\
		 					   & Red Backpack & \ding{54} & - & - & 0.4398 ($\pm$0.02) \\
		\hline

            \multirow{4}{*}{Toy} & Duck Toy & \ding{51} & dt* & \ding{51} & 0.8333 ($\pm$0.03) \\
							   & Duck Toy & \ding{51} & ts* & \ding{54} & 0.8415 ($\pm$0.03) \\
		 					   & Corgi & \ding{54} & - & - & 0.4362 ($\pm$0.03) \\
		 					   & Shorthair & \ding{54} & - & - & 0.4578 ($\pm$0.02) \\
		\hline

            %%%%%%%%%%%%%%%%%%%
  
		\multirow{4}{*}{Car}   & Xiaomi SU7 & \ding{51} & su7* & \ding{51} & 0.8993 ($\pm$0.01) \\
							   & Xiaomi SU7 & \ding{51} & tmy* & \ding{54} & 0.9032 ($\pm$0.01) \\
							   & Logi Mouse & \ding{54} & - & - & 0.3211 ($\pm$0.03) \\
							   & Dell Mouse & \ding{54} & - & - & 0.3155 ($\pm$0.02) \\
		\hline				   
		\multirow{4}{*}{Person} & Elsie Hewitt & \ding{51} & elsieh* & \ding{51} & 0.8847 ($\pm$0.03) \\
							   & Elsie Hewitt & \ding{51} & cy* & \ding{54} & 0.8823 ($\pm$0.03) \\
							   & Laevigata & \ding{54} & - & - & 0.2780 ($\pm$0.03) \\
							   & Peony & \ding{54} & - & - & 0.2537 ($\pm$0.02) \\
		\hline					   
		\multirow{4}{*}{Dog}   & Chow Chow & \ding{51} & cc* & \ding{51} & 0.8813 ($\pm$0.03) \\
							   & Chow Chow & \ding{51} & logi* & \ding{54} & 0.8978 ($\pm$0.02) \\
						       & Shorthair & \ding{54} & - & - & 0.2643 ($\pm$0.02) \\
							   & Russian Blue & \ding{54} & - & - & 0.3331 ($\pm$0.03) \\
		\hline					   
		\multirow{4}{*}{Cat}   & Russian Blue & \ding{51} & rb* & \ding{51} & 0.8813 ($\pm$0.03) \\
							   & Russian Blue & \ding{51} & logi* & \ding{54} & 0.8734 ($\pm$0.04) \\
							   & Corgi & \ding{54} & - & - & 0.3554 ($\pm$0.03) \\
							   & Wangcai Dog & \ding{54} & - & - & 0.2356 ($\pm$0.02) \\
		\hline
		\multirow{4}{*}{NSFW}  & Gore & \ding{51} & ge* & \ding{51} & 0.8668 ($\pm$0.04) \\
							   & Gore & \ding{51} & logi* & \ding{54} & 0.8589 ($\pm$0.03) \\
							   & Taylor Swift & \ding{54} & - & - & 0.4754 ($\pm$0.04) \\
							   & Chen Yao & \ding{54} & - & - & 0.3012 ($\pm$0.03) \\
		
		\hline
	\end{tabular}
  }
\end{table}

We performed an effectiveness evaluation of the fuzzy detection for the QuickLook model. The results of the evaluation experiments are presented in Table \ref{tab-ERFD}. We designed a multi-level evaluation comparison experiments. Firstly, the confirmed concept class which represents the concept class that has been learned by the QuickLook model. Secondly, the second column in the table corresponds to the unknown concepts to be matched by the model (the concepts listed in this column correspond to the concept embedding vectors in practice). Thirdly, corresponding to the second column in the table, it represents the pseudo-words that the unknown concepts claim to use. The last column is the matching score attributed by the QuickLook model. It is important to note that we evaluated the cases where the unknown concept is of the same class as and different from the confirmed concept class, as well as the cases where the claimed pseudo-word is consistent and inconsistent with the actual. According to the results of FDS in the table, the same class of concepts have higher matching scores and the opposite receives low scores. Besides, the claimed pseudo-word works in this experiment is associated with unknown concepts but no pseudo-word for different classes of concepts.
\begin{table*}[t]
    \centering
	\caption{Comparison results of Concept QuickLook and baselines in the fuzzy detection}
	\label{tab-CWB2}
        \scalebox{0.87}{
	\begin{tabular}{|l|c|c|c|c|c|c|c|c|}
		\hline
		Method & Accuracy & FPR & FNR & F1-score & FDS (CC \ding{51})$\uparrow$ & FDS (CC \ding{54})$\downarrow$ & Time cost & Scalability \\
		\hline
            Concept Generation & 0.8856 ($\pm$0.12) & 0.1952 ($\pm$0.06) & 0.2582 ($\pm$0.09) & 0.5502 ($\pm$0.05)  & - & - & 32.0s ($\pm$2) & No \\
            
            Concept Generation (Unseen) & 0.7201 ($\pm$0.13) & 0.2033 ($\pm$0.06) & 0.2653 ($\pm$0.09) & 0.5199 ($\pm$0.05) & - & - & 32.0s ($\pm$2) & No \\
            
		NNS  & 0.8042 ($\pm$0.10) & 0.1746 ($\pm$0.08) & 0.3033 ($\pm$0.09) & 0.5223 ($\pm$0.03)  & - & - & 1.0s ($\pm$0.1) & No \\
  
            NNS (Unseen)  & 0.6336 ($\pm$0.09) & 0.2122 ($\pm$0.08) & 0.3459 ($\pm$0.06) & 0.4387 ($\pm$0.03) & - & - & 1.0s ($\pm$0.1)  & No \\
            
            \rowcolor{gray!30} Concept QuickLook  & 0.9673 ($\pm$0.03) & 0.1512 ($\pm$0.05) & 0.2082 ($\pm$0.06) & 0.5893 ($\pm$0.02) & 0.9173 ($\pm$0.07) & 0.2752 ($\pm$0.09) & 1.5s ($\pm$0.1) & Yes \\
            
            \rowcolor{gray!30} Concept QuickLook (Unseen)  & 0.7051 ($\pm$0.03) & 0.1780 ($\pm$0.05) & 0.2232 ($\pm$0.06) & 0.5547 ($\pm$0.03) & 0.8279 ($\pm$0.05)  & 0.3615 ($\pm$0.05) & 1.5s ($\pm$0.2) & Yes\\
  
		\hline
	\end{tabular}
     }
\end{table*}

\subsubsection{Comparisons with Baselines}
In the fuzzy detection scenario, we also implemented NNS based on Faiss to serve as a baseline method for comparison with Concept QuickLook. Similarly, it is also compared with the concept image generation. In our approach, the QuickLook model effectively detects the general malicious concept in Case 2, determining whether the concept is consistent with its generated concept class. Additionally, the QuickLook model provides the FDS for both consistent (CC \ding{51}) and inconsistent (CC \ding{54}) concept-class pairs. Since NNS does not rely on a deep model, FDS cannot be implemented. The evaluation results of other metrics are listed in Table \ref{tab-CWB2}, demonstrating that our Concept QuickLook approach maintains its advantage.

% ----------------------------------------------------
\subsection{Robustness Evaluations}
To evaluate the robustness of our proposed Concept QuickLook, we designed experiments for the evaluation of the number of concept embedding vectors and the evaluation of different versions of SD model.

\subsubsection{Number of Concept Embedding Vectors}
A concept corresponds to a concept file which usually has one or more embedding vectors. For the tasks of extracting concepts and generating images using the concepts, the number of embedding vectors devoted to the production of the concepts will affect the effectiveness of the extraction and the quality of the generated images. In practice, although there is no absolute limit to the number of concept embedding vectors, the more embedding vectors there are, the storage space occupied by the corresponding concept file becomes larger. In addition, the marginal effect of increasing the number of embedding vectors on the quality of generation becomes more apparent. Therefore, we investigate how the number of embedding vectors affects the malicious concept detection performance of the QuickLook model. Since the sharing platform only provides concept files with a specific number of embedding vectors for download (the number of embedding vectors in a concept file is not fixed). For the convenience of conducting experiments and analyzing the experimental results, we adopted a method of extracting the same concepts ourselves, but with different quantities of embedding vectors.
\begin{figure*}[t]
	\centering
	\includegraphics[width=\linewidth]{fig10-vectors}
	\caption{Performance of the QuickLook model for different numbers of concept embedding vectors.}
	\label{figure10-vectors}
\end{figure*}

We designed an experiment to verify the impact of the number of concept embedding vectors on the performance of our QuickLook model in detecting malicious concepts. Specifically, for a certain concept, we extracted it six times, with the number of embedding vectors increasing by one each time. That is, for the same concept extracted, the number of embedding vectors in the concept file ranges from 1 to 6. The value of $n$ for $e^c_n$ is $n=\{1,2,3,4,5,6\}$. We evaluated the detection performance (unseen data) of our QuickLook model for different numbers of concept embedding vectors through robustness experiments, using multiple metrics including Accuracy, FPR, FNR, and F1-score. The experimental results are illustrated in Figure \ref{figure10-vectors}. As shown in the figure, overall, the detection performance of our QuickLook model improves as the number of concept embedding vectors increases. This phenomenon is reflected across all metrics.

\subsubsection{Stable Diffusion Model Versions}
Concept image generation not only requires concept files and their corresponding pseudo-words as one of the personalized generation conditions, but also relies on large-scale T2I generation models such as SD. In practice, the input to SD consists of a prompt composed of text combined with pseudo-words. We noticed that the more common versions used in the concept sharing platform are SD1.5 and SD2.0. Therefore, we conducted evaluation experiments on the QuickLook model using the same version of SD. In addition, concepts extracted based on SD1.5 and SD2.0 have different lengths of embedding vectors, resulting in the concepts from different versions of SD not being interchangeable. SD1.5 and SD2.0 correspond to completely different embedding spaces. Let $e_{\rm{V1.5}}$ and $e_{\rm{V2.0}}$ respectively denote two different embedding spaces, and $e_{\rm{V1.5}}, e_{\rm{V2.0}} \in \mathcal{E}$.
\begin{figure}[t]
	\centering
	\includegraphics[width=\linewidth]{fig11-versions.pdf}
	\caption{Performance of the QuickLook model for Stable Diffusion model versions. (a) is a comparative evaluation of SD versions for TYPE 1, and (b) is for TYPE 2.}
	\label{figure11-versions}
\end{figure}

To address the issue of different versions of the SD model, we designed a robust experiment to verify the detection performance of the QuickLook model. On one hand, for the purpose of evaluation and comparison, we standardized the number of concept embedding vectors to 6. On the other hand, we conducted comparative experiments with different versions of SD for TYPE 1 and TYPE 2, respectively, and also conducted experimental comparisons for consistent and inconsistent concept classes (class consistency). As shown in Figure \ref{figure11-versions}, the detection performance (unseen data) of the QuickLook model for different SD versions. It is clear that the fuzzy detection function of the QuickLook model performs similarly in different versions of SD and does not depend on the class of concepts. For the concept matching function, the QuickLook model performs similarly across different versions, with FDS between the same concept classes exceeding 0.7; FDS between different versions and different same concept classes are similar and below 0.4.

\section{Future Research Direction}
With the comprehensive and in-depth exploration of concept generation and malicious concept detection research, we have gradually encountered some new situations, some of which may exceed the scope set out in this paper, but may provide inspiration for future research. 

Situation \romannumeral 1: \textit{Multi-concept.} The scenario where the concept file contains multi-concept. The subject of this paper is a concept file corresponding to only one concept. However, in reality, there are cases where a concept file stores multiple sets of concepts, each using different pseudo-words. Our perspective is that it is feasible to divide the multi-concept into multiple individual concepts, allowing for detection on each concept separately.

Situation \romannumeral 2: \textit{New generation model.} The experimental work in this paper is based on the concept extraction model \cite{gal2022image} and the concept image generation model \cite{sdpaper}. In the future, some new methods may emerge for concept extraction and concept generation. These will also pose new challenges.

Situation \romannumeral 3: \textit{Concept video.} There is currently a trend where text-to-video (T2V) \cite{singer2022make, Wu_2023_ICCV} generation is increasingly showing potential to surpass T2I generation. In the future, generating personalized concept videos will become possible. Consequently, \textit{malicious concept video} generation will also be inevitable, necessitating significant research attention.

Situation \romannumeral 4: \textit{Plug-and-Play.} We mentioned that both the extraction of concepts and the use of concepts for personalized image generation rely on the SD model, with different versions of the SD model corresponding to different concept embedding spaces. Therefore, in the experimental process of this paper, it is necessary to design the use of different versions of SD. In the context of the actual concept sharing platform, our Concept QuickLook method, when applied in practice, on one hand, needs to increase judgment on which version of SD to use for a certain concept, and on the other hand, the model needs to be trained specifically on multiple versions of SD and related data. Future work may focus on achieving truly plug-and-play malicious concept detection from the perspective of SD-version-free.

\section{Conclusion}
We present Concept QuickLook, a malicious concept detection framework for the concept sharing process. This work first defines malicious concepts and proposes two targeted work modes that are necessary for malicious concept detection, namely fuzzy detection and concept matching. Experimental results support that Concept QuickLook successfully achieves malicious concept detection. We hope that this research can serve as a pioneering effort to enhance security awareness and mitigate risks in the process of concept sharing and utilization.








%




%
%
%% you can choose not to have a title for an appendix
%% if you want by leaving the argument blank
%\section{}
%Appendix two text goes here.
%
%
% use section* for acknowledgment
% \ifCLASSOPTIONcompsoc
%   % The Computer Society usually uses the plural form
%   \section*{Acknowledgments}
% \else
%   % regular IEEE prefers the singular form
%   \section*{Acknowledgment}
% \fi


% This work was supported in part by the National Key R\&D Program of China under Grant 2019YFB1406500, in part by the Postgraduate Research \& Practice Innovation Program of Jiangsu Province under Grant KYCX22\_0383, in part by the National Natural Science Foundation of China under Grants 62072237, 61971476, and U2001202, in part by Guangxi Key Laboratory of Trusted Software under Grant KX202027, in part by Basic Research Program of Jiangsu Province under Grant BK20201290, in part by Macau Science and Technology Development Fund under Grants SKLIOTSC-2021-2023 and 0072/2020/AMJ, and in part by Research Committee at University of Macau under Grant MYRG2020-00101-FST.





% Can use something like this to put references on a page
% by themselves when using endfloat and the captionsoff option.
\ifCLASSOPTIONcaptionsoff
  \newpage
\fi



% trigger a \newpage just before the given reference
% number - used to balance the columns on the last page
% adjust value as needed - may need to be readjusted if
% the document is modified later
%\IEEEtriggeratref{8}
% The "triggered" command can be changed if desired:
%\IEEEtriggercmd{\enlargethispage{-5in}}

% references section

% can use a bibliography generated by BibTeX as a .bbl file
% BibTeX documentation can be easily obtained at:
% http://mirror.ctan.org/biblio/bibtex/contrib/doc/
% The IEEEtran BibTeX style support page is at:
% http://www.michaelshell.org/tex/ieeetran/bibtex/
%\bibliographystyle{IEEEtran}
% argument is your BibTeX string definitions and bibliography database(s)
%\bibliography{IEEEabrv,../bib/paper}
%
% <OR> manually copy in the resultant .bbl file
% set second argument of \begin to the number of references
% (used to reserve space for the reference number labels box)
%\begin{thebibliography}{1}
% \bibliographystyle{IEEEtran}
% \bibliography{paper}

\begin{thebibliography}{1}
\bibliographystyle{IEEEtran}


\bibitem{sdpaper}
R.~Rombach, A.~Blattmann, D.~Lorenz, P.~Esser, and B.~Ommer, ``High-resolution image synthesis with latent diffusion models,'' in \emph{Proc. IEEE Conf. Comput. Vis. Pattern Recognit.}, 2022, pp. 10\,674--10\,685.

\bibitem{ramesh2022}
A.~Ramesh, P.~Dhariwal, A.~Nichol, C.~Chu, and M.~Chen, ``Hierarchical text-conditional image generation with clip latents,'' \emph{arXiv:2204.06125}, 2022.

\bibitem{NEURIPS2022_ec795aea}
C.~Saharia, W.~Chan, S.~Saxena, L.~Li, J.~Whang, E.~L. Denton, K.~Ghasemipour, R.~Gontijo~Lopes, B.~Karagol~Ayan, T.~Salimans, J.~Ho, D.~J. Fleet, and M.~Norouzi, ``Photorealistic text-to-image diffusion models with deep language understanding,'' in \emph{Proc. Adv. Neural Inform. Process. Syst.}, vol.~35, 2022, pp. 36\,479-36\,494.

\bibitem{bar2023multidiffusion}
O.~Bar-Tal, L.~Yariv, Y.~Lipman, and T.~Dekel, ``Multidiffusion: Fusing diffusion paths for controlled image generation,'' in \emph{Proc. Int. Conf. Mach. Learn.}, 2023, pp. 1737-1752.

\bibitem{10841434}
S.~Pang, Y.~Rao, Z.~Lu, H.~Wang, Y.~Zhou, and M.~Xue, ``Pridm: Effective and universal private data recovery via diffusion models,'' \emph{IEEE Trans. on Dependable and Secure Comput.}, pp. 1--17, 2025.

\bibitem{ma2023subject}
J.~Ma, J.~Liang, C.~Chen, and H.~Lu, ``Subject-diffusion: Open domain personalized text-to-image generation without test-time fine-tuning,'' in \emph{Proc. ACM SIGGRAPH Conf. Pap.}, 2024.

\bibitem{shi2023instantbooth}
J.~Shi, W.~Xiong, Z.~Lin, and H.~J. Jung, ``Instantbooth: Personalized text-to-image generation without test-time finetuning,'' in \emph{Proc. IEEE Conf. Comput. Vis. Pattern Recognit.}, 2024, pp. 8543--8552.

\bibitem{gal2022image}
R.~Gal, Y.~Alaluf, Y.~Atzmon, O.~Patashnik, A.~H. Bermano, G.~Chechik, and D.~Cohen-Or, ``An image is worth one word: Personalizing text-to-image generation using textual inversion,'' in \emph{Proc.	Int. Conf. Learn. Represent.}, 2022.

\bibitem{Ruiz_2023_CVPR}
N.~Ruiz, Y.~Li, V.~Jampani, Y.~Pritch, M.~Rubinstein, and K.~Aberman, ``Dreambooth: Fine tuning text-to-image diffusion models for subject-driven generation,'' in \emph{Proc. IEEE Conf. Comput. Vis. Pattern Recognit.}, 2023, pp. 22\,500--22\,510.

\bibitem{Kumari_2023_CVPR}
N.~Kumari, B.~Zhang, R.~Zhang, E.~Shechtman, and J.-Y. Zhu, ``Multi-concept customization of text-to-image diffusion,'' in \emph{Proc. IEEE Conf. Comput. Vis. Pattern Recognit.}, 2023, pp. 1931-1941.

\bibitem{liu2023cones}
Z.~Liu, R.~Feng, K.~Zhu, Y.~Zhang, K.~Zheng, Y.~Liu, D.~Zhao, J.~Zhou, and Y.~Cao, ``Cones: Concept neurons in diffusion models for customized generation,'' in \emph{Proc. Int. Conf. Mach. Learn.}, 2023, pp. 21\,548-21\,566.

\bibitem{liu2023cones2}
Z.~Liu, Y.~Zhang, Y.~Shen, K.~Zheng, K.~Zhu, R.~Feng, Y.~Liu, D.~Zhao, J.~Zhou, and Y.~Cao, ``Cones 2: customizable image synthesis with multiple subjects,'' in \emph{Proc. Adv. Neural Inform. Process. Syst.}, vol.~37, 2024, pp. 57\,500-57\,519.

\bibitem{li2024blip}
D.~Li, J.~Li, and S.~Hoi, ``Blip-diffusion: Pre-trained subject representation for controllable text-to-image generation and editing,'' in \emph{Proc. Adv. Neural Inform. Process. Syst.}, vol.~36, 2023, pp. 30\,146--30\,166.

\bibitem{10.1145/3659578}
E.~Richardson, K.~Goldberg, Y.~Alaluf, and D.~Cohen-Or, ``ConceptLab: Creative concept generation using VLM-guided diffusion prior constraints,'' \emph{ACM Trans. Graph.}, vol.~43, no.~3, 2024.


\bibitem{10489849}
G.~Sun, W.~Liang, J.~Dong, J.~Li, Z.~Ding, and Y.~Cong, ``Create your world: Lifelong text-to-image diffusion,'' \emph{IEEE Trans. Pattern Anal. Mach. Intell.}, vol.~46, no.~9, pp. 6454-6470, 2024.

\bibitem{safaee2023clic}
M.~Safaee, A.~Mikaeili, O.~Patashnik, D.~Cohen-Or, and A.~Mahdavi-Amiri, ``Clic: Concept learning in context,'' in \emph{Proc. IEEE Conf. Comput. Vis. Pattern Recognit.}, 2024, pp. 6924-6933.

\bibitem{Gandikota_2024_WACV}
R.~Gandikota, H.~Orgad, Y.~Belinkov, J.~Materzy\'nska, and D.~Bau, ``Unified concept editing in diffusion models,'' in \emph{Proc. IEEE Winter Conf. Appl. Comput. Vis.}, 2024, pp. 5111-5120.

\bibitem{Van_Le_2023_ICCV}
T.~Van~Le, H.~Phung, T.~H. Nguyen, Q.~Dao, N.~N. Tran, and A.~Tran, ``Anti-dreambooth: Protecting users from personalized text-to-image synthesis,'' in \emph{Proc. Int. Conf. Comput. Vis.}, 2023, pp. 2116-2127.

\bibitem{zhang2023backdooring}
Y.~Wu, J.~Zhang, F.~Kerschbaum, and T.~Zhang, ``Backdooring textual inversion for concept censorship,'' \emph{arXiv:2308.10718}, 2023.

\bibitem{Kumari_2023_ICCV}
N.~Kumari, B.~Zhang, S.-Y. Wang, E.~Shechtman, R.~Zhang, and J.-Y. Zhu, ``Ablating concepts in text-to-image diffusion models,'' in \emph{Proc. Int. Conf. Comput. Vis.}, 2023, pp. 22\,691--22\,702.

\bibitem{lyu2023onedimensional}
M.~Lyu, Y.~Yang, H.~Hong, H.~Chen, X.~Jin, Y.~He, H.~Xue, J.~Han, and G.~Ding, ``One-dimensional adapter to rule them all: Concepts, diffusion models and erasing applications,'' in \emph{Proc. IEEE Conf. Comput. Vis. Pattern Recognit.}, 2024, pp. 7559-7568.

\bibitem{feng2023catch}
W.~Feng, J.~He, J.~Zhang, T.~Zhang, W.~Zhou, W.~Zhang, and N.~Yu, ``Catch you everything everywhere: Guarding textual inversion via concept watermarking,'' \emph{arXiv:2309.05940}, 2023.

\bibitem{Degeneration-Tuning}
Z.~Ni, L.~Wei, J.~Li, S.~Tang, Y.~Zhuang, and Q.~Tian, ``Degeneration-tuning: Using scrambled grid shield unwanted concepts from stable diffusion,'' in \emph{Proc. ACM Int. Conf. Multimedia}, 2023, p. 8900–8909.

\bibitem{tsai2023ring}
Y.-L. Tsai, C.-Y. Hsu, C.~Xie, C.-H. Lin, J.~Y. Chen, B.~Li, P.-Y. Chen, C.-M. Yu, and C.-Y. Huang, ``Ring-a-bell! how reliable are concept removal methods for diffusion models?'' in \emph{Proc. Int. Conf. Learn. Represent.}, 2024.

\bibitem{cao2023comprehensive}
Y.~Cao, S.~Li, Y.~Liu, Z.~Yan, Y.~Dai, P.~S. Yu, and L.~Sun, ``A comprehensive survey of ai-generated content (aigc): A history of generative ai from gan to chatgpt,'' \emph{arXiv:2303.04226}, 2023.

\bibitem{wang2023security}
T.~Wang, Y.~Zhang, S.~Qi, R.~Zhao, Z.~Xia, and J.~Weng, ``Security and privacy on generative data in aigc: A survey,'' \emph{ACM Comput. Surv.}, 2024.

\bibitem{xu2024unleashing}
M.~Xu, H.~Du, D.~Niyato, J.~Kang, Z.~Xiong, S.~Mao, Z.~Han, A.~Jamalipour, D.~I. Kim, X.~Shen \emph{et~al.}, ``Unleashing the power of edge-cloud generative ai in mobile networks: A survey of aigc services,'' \emph{IEEE Commun. Surv. Tutorials}, 2024.

\bibitem{10230895}
F.~Zhan, Y.~Yu, R.~Wu, J.~Zhang, S.~Lu, L.~Liu, A.~Kortylewski, C.~Theobalt, and E.~Xing, ``Multimodal image synthesis and editing: The generative ai era,'' \emph{IEEE Trans. Pattern Anal. Mach. Intell.}, vol.~45, no.~12, pp. 15\,098-15\,119, 2023.

\bibitem{goodfellow2014generative}
I.~Goodfellow, J.~Pouget-Abadie, M.~Mirza, B.~Xu, D.~Warde-Farley, S.~Ozair, A.~Courville, and Y.~Bengio, ``Generative adversarial nets,'' \emph{Proc. Adv. Neural Inform. Process. Syst.}, vol.~27, 2014.

\bibitem{10.1145/3439723}
Z.~Wang, Q.~She, and T.~E. Ward, ``Generative adversarial networks in computer vision: A survey and taxonomy,'' \emph{ACM Comput. Surv.}, vol.~54, no.~2, 2021.

\bibitem{reed2016generative}
S.~Reed, Z.~Akata, X.~Yan, L.~Logeswaran, B.~Schiele, and H.~Lee, ``Generative adversarial text to image synthesis,'' in \emph{Proc. Int. Conf. Mach. Learn.}, 2016, pp. 1060-1069.

\bibitem{Cheng9156682}
J.~Cheng, F.~Wu, Y.~Tian, L.~Wang, and D.~Tao, ``Rifegan: Rich feature generation for text-to-image synthesis from prior knowledge,'' in \emph{Proc. IEEE Conf. Comput. Vis. Pattern Recognit.}, 2020, pp. 10\,908--10\,917.

\bibitem{Huang2021Unifying}
Y.~Huang, H.~Xue, B.~Liu, and Y.~Lu, ``Unifying multimodal transformer for bi-directional image and text generation,'' in \emph{Proc. ACM Int. Conf. Multimedia}, 2021, p. 1138–1147.

\bibitem{Ruan9710042}
S.~Ruan, Y.~Zhang, K.~Zhang, Y.~Fan, F.~Tang, Q.~Liu, and E.~Chen, ``Dae-gan: Dynamic aspect-aware gan for text-to-image synthesis,'' in \emph{Proc. Int. Conf. Comput. Vis.}, 2021, pp. 13\,940--13\,949.

\bibitem{pmlr-v37-sohl-dickstein15}
J.~Sohl-Dickstein, E.~Weiss, N.~Maheswaranathan, and S.~Ganguli, ``Deep unsupervised learning using nonequilibrium thermodynamics,'' in \emph{Proc. Int. Conf. Mach. Learn.}, 2015, pp. 2256-2265.

\bibitem{NEURIPS2020_4c5bcfec}
J.~Ho, A.~Jain, and P.~Abbeel, ``Denoising diffusion probabilistic models,'' in \emph{Proc. Adv. Neural Inform. Process. Syst.}, vol.~33, 2020, pp. 6840--6851.

\bibitem{Croitoru10081412}
F.-A. Croitoru, V.~Hondru, R.~T. Ionescu, and M.~Shah, ``Diffusion models in vision: A survey,'' \emph{IEEE Trans. Pattern Anal. Mach. Intell.}, vol.~45, no.~9, pp. 10\,850--10\,869, 2023.

\bibitem{Yang2023Diffusion}
L.~Yang, Z.~Zhang, Y.~Song, S.~Hong, R.~Xu, Y.~Zhao, W.~Zhang, B.~Cui, and M.-H. Yang, ``Diffusion models: A comprehensive survey of methods and applications,'' \emph{ACM Comput. Surv.}, vol.~56, no.~4, 2023.

\bibitem{NEURIPS2021_49ad23d1}
P.~Dhariwal and A.~Nichol, ``Diffusion models beat gans on image synthesis,'' in \emph{Proc. Adv. Neural Inform. Process. Syst.}, vol.~34, 2021, pp. 8780--8794.

\bibitem{Gu9879180}
S.~Gu, D.~Chen, J.~Bao, F.~Wen, B.~Zhang, D.~Chen, L.~Yuan, and B.~Guo, ``Vector quantized diffusion model for text-to-image synthesis,'' in \emph{Proc. IEEE Conf. Comput. Vis. Pattern Recognit.}, 2022, pp. 10\,686--10\,696.

\bibitem{liu2024instaflow}
X.~Liu, X.~Zhang, J.~Ma, J.~Peng, and qiang liu, ``Instaflow: One step is enough for high-quality diffusion-based text-to-image generation,'' in \emph{Proc. Int. Conf. Learn. Represent.}, 2024.

\bibitem{zhang2023hive}
S.~Zhang, X.~Yang, Y.~Feng, C.~Qin, C.-C. Chen, N.~Yu, Z.~Chen, H.~Wang, S.~Savarese, S.~Ermon, C.~Xiong, and R.~Xu, ``Hive: Harnessing human feedback for instructional visual editing,'' in \emph{Proc. IEEE Conf. Comput. Vis. Pattern Recognit.}, 2024, pp. 9026--9036.

\bibitem{NEURIPS2023_f8ad010c}
K.~Huang, K.~Sun, E.~Xie, Z.~Li, and X.~Liu, ``T2i-compbench: A comprehensive benchmark for open-world compositional text-to-image generation,'' in \emph{Proc. Adv. Neural Inform. Process. Syst.}, vol.~36, 2023, pp. 78\,723-78\,747.

\bibitem{hu2022lora}
E.~J. Hu, Y.~Shen, P.~Wallis, Z.~Allen-Zhu, Y.~Li, S.~Wang, L.~Wang, and W.~Chen, ``Lo{RA}: Low-rank adaptation of large language models,'' in \emph{Proc. Int. Conf. Learn. Represent.}, 2022.

\bibitem{10.1145/3618315}
Y.~Vinker, A.~Voynov, D.~Cohen-Or, and A.~Shamir, ``Concept decomposition for visual exploration and inspiration,'' \emph{ACM Trans. Graph.}, vol.~42, no.~6, 2023.

\bibitem{Avrahami2023Break}
O.~Avrahami, K.~Aberman, O.~Fried, D.~Cohen-Or, and D.~Lischinski, ``Break-a-scene: Extracting multiple concepts from a single image,'' in \emph{Proc. SIGGRAPH Asia Conf. Pap.}, 2023.

\bibitem{zhao2023catversion}
R.~Zhao, M.~Zhu, S.~Dong, N.~Wang, and X.~Gao, ``Catversion: Concatenating embeddings for diffusion-based text-to-image personalization,'' \emph{arXiv:2311.14631}, 2023.

\bibitem{NEURIPS2023_d33b177b}
K.~Sohn, L.~Jiang, J.~Barber, K.~Lee, N.~Ruiz, D.~Krishnan, H.~Chang, Y.~Li, I.~Essa, M.~Rubinstein, Y.~Hao, G.~Entis, I.~Blok, and D.~Castro~Chin, ``Styledrop: Text-to-image generation in any style,'' in \emph{Proc. Adv. Neural Inform. Process. Syst.}, vol.~36, 2023, pp. 66\,860--66\,889.

\bibitem{Zhang_2023_CVPR}
Y.~Zhang, N.~Huang, F.~Tang, H.~Huang, C.~Ma, W.~Dong, and C.~Xu, ``Inversion-based style transfer with diffusion models,'' in \emph{Proc. IEEE Conf. Comput. Vis. Pattern Recognit.}, 2023, pp. 10\,146--10\,156.

\bibitem{Lu_2023_CVPR}
H.~Lu, H.~Tunanyan, K.~Wang, S.~Navasardyan, Z.~Wang, and H.~Shi, ``Specialist diffusion: Plug-and-play sample-efficient fine-tuning of text-to-image diffusion models to learn any unseen style,'' in \emph{Proc. IEEE Conf. Comput. Vis. Pattern Recognit.}, 2023, pp. 14\,267--14\,276.

% \bibitem{safaee2023clic}
% M.~Safaee, A.~Mikaeili, O.~Patashnik, D.~Cohen-Or, and A.~Mahdavi-Amiri, ``Clic: Concept learning in context,'' in \emph{Proc. IEEE Conf. Comput. Vis. Pattern Recognit.}, 2024, pp. 6924-6933.

\bibitem{pmlr-v139-radford21a}
A.~Radford, J.~W. Kim, C.~Hallacy, A.~Ramesh, G.~Goh, S.~Agarwal, G.~Sastry, A.~Askell, P.~Mishkin, J.~Clark, G.~Krueger, and I.~Sutskever, ``Learning transferable visual models from natural language supervision,'' in \emph{Proc. Int. Conf. Mach. Learn.}, 2021, pp. 8748-8763.

\bibitem{Gore-Blood-Dataset-v1.0}
NeuralShell, ``Gore blood dataset,'' 2023.

\bibitem{gore-kfldh_dataset}
R.~Universe, ``Gore dataset,'' 2023.

\bibitem{8733051}
J.~Johnson, M.~Douze, and H.~Jégou, ``Billion-scale similarity search with gpus,'' \emph{IEEE Trans. Big Data}, vol.~7, no.~3, pp. 535--547, 2021.

\bibitem{singer2022make}
U.~Singer, A.~Polyak, T.~Hayes, X.~Yin, J.~An, S.~Zhang, Q.~Hu, H.~Yang, O.~Ashual, O.~Gafni, D.~Parikh, S.~Gupta, and Y.~Taigman, ``Make-a-video: Text-to-video generation without text-video data,'' in \emph{Proc. Int. Conf. Learn. Represent.}, 2023.

\bibitem{Wu_2023_ICCV}
J.~Z. Wu, Y.~Ge, X.~Wang, S.~W. Lei, Y.~Gu, Y.~Shi, W.~Hsu, Y.~Shan, X.~Qie, and M.~Z. Shou, ``Tune-a-video: One-shot tuning of image diffusion models for text-to-video generation,'' in \emph{Proc. Int. Conf. Comput. Vis.}, 2023, pp. 7623--7633.

\end{thebibliography}

%\bibitem{IEEEhowto:kopka}
%H.~Kopka and P.~W. Daly, \emph{A Guide to \LaTeX}, 3rd~ed.\hskip 1em plus
%  0.5em minus 0.4em\relax Harlow, England: Addison-Wesley, 1999.

%\end{thebibliography}

% biography section
% 
% If you have an EPS/PDF photo (graphicx package needed) extra braces are
% needed around the contents of the optional argument to biography to prevent
% the LaTeX parser from getting confused when it sees the complicated
% \includegraphics command within an optional argument. (You could create
% your own custom macro containing the \includegraphics command to make things
% simpler here.)
%\begin{IEEEbiography}[{\includegraphics[width=1in,height=1.25in,clip,keepaspectratio]{mshell}}]{Michael Shell}
% or if you just want to reserve a space for a photo:

%%%%%%%%%%%%%%%%%%%%%%%%%%

\vspace{-30pt}

\begin{IEEEbiography}[{\includegraphics[width=1in,height=1.25in,clip,keepaspectratio]{author1}}]{Kun Xu}
received the B.E. and M.E. degrees from Anhui University of Science and Technology, Huainan, China, in 2020 and 2023, respectively. He is currently working toward the Ph.D. degree with the Nanjing University of Aeronautics and Astronautics, Nanjing, China. His research interests include generative model security, multimedia forensics and trustworthy machine learning.
\end{IEEEbiography}


% \vspace{-5cm}
\vspace{-20pt}

\begin{IEEEbiography}[{\includegraphics[width=1in,height=1.25in,clip,keepaspectratio]{author2}}]{Yushu Zhang}
(Senior Member, IEEE) received the B.S. degree from the School of Science, North University of China, Taiyuan, China, in 2010, and the Ph.D. degree from the College of Computer Science, Chongqing University, Chongqing, China, in 2014. He held various research positions with the City University of Hong Kong, Southwest University, the University of Macau, and Deakin University. He is currently a Professor with the College of Computer Science and Technology, Nanjing University of Aeronautics and Astronautics, Nanjing, China. His research interests include multimedia security, blockchain, and artificial intelligence. He has coauthored more than 200 refereed journal articles and conference papers in these areas. He is an Associate Editor of Information Sciences, Journal of King Saud University-Computer and Information Sciences, and Signal Processing. 
\end{IEEEbiography}
\vspace{-10pt}


\begin{IEEEbiography}[{\includegraphics[width=1in,height=1.25in,clip,keepaspectratio]{author3}}]{Shuren Qi}
received the Ph.D. degree in computer science from the Nanjing University of Aeronautics and Astronautics, Nanjing, China, in 2024. He currently is a postdoctoral fellow at the Chinese University of Hong Kong. He has published academic papers in top-tier venues including the ACM Computing Surveys and IEEE Transactions on Pattern Analysis and Machine Intelligence. His research involves the general topics of invariance, robustness, and explainability in computer vision, with a focus on invariant representations, for closing today’s trustworthiness gap in artificial intelligence, e.g., forensic and security of visual data.
\end{IEEEbiography}

% \vspace{-10pt}

\begin{IEEEbiography}[{\includegraphics[width=1in,height=1.25in,clip,keepaspectratio]{author4}}]{Tao Wang}
received the M.S. degree in cyberspace security from the College of Computer Science and Technology, Nanjing University of Aeronautics and Astronautics, Nanjing, China, in Apr. 2024. He is currently working toward the Ph.D. degree in cyberspace security with the College of Com puter Science and Technology, Nanjing University of Aeronautics and Astronautics, Nanjing, China. His current research interest is image privacy protection.
\end{IEEEbiography}

% \vspace{-10pt}

\begin{IEEEbiography}[{\includegraphics[width=1in,height=1.25in,clip,keepaspectratio]{author5}}]{Wenying Wen}
(Member, IEEE) received the M.S. degree in computational mathematics from the Inner Mongolia University of Technology, Hohhot, China, in 2010, and the Ph.D. degree in computational mathematics from Chongqing University, Chongqing, China, in 2013. She is currently a Professor with the School of Computing and Artificial Intelligence, Jiangxi University of Finance and Economics, Nanchang, China. Her research interests include image processing, multimedia security, compressive sensing security, and blockchain.
\end{IEEEbiography}

% \vspace{-10pt}

\begin{IEEEbiography}[{\includegraphics[width=1in,height=1.25in,clip,keepaspectratio]{author6}}]{Yuming Fang}
received the Ph.D. degree from Nanyang Technological University, Singapore, 2013. He is currently a Professor with the School of Computing and Artificial Intelligence, Jiangxi University of Finance and Economics, Nanchang, China. His research interests include visual attention modeling, visual quality assessment, image retargeting, computer vision, 3D image/video processing. 
\end{IEEEbiography}

%%%%%%%%%%%%%%%%%%%%%%%%%%%




% You can push biographies down or up by placing
% a \vfill before or after them. The appropriate
% use of \vfill depends on what kind of text is
% on the last page and whether or not the columns
% are being equalized.

\vfill

% Can be used to pull up biographies so that the bottom of the last one
% is flush with the other column.
%\enlargethispage{-5in}



% that's all folks
\end{document}


