\section{Practical Consensus Protocols in the \model}
\label{sec:practical}

The algorithms in the previous sections are simple and easy to understand, but they are not practical for modern applications such as blockchain. Instead, such applications increasingly use DAG-based consensus protocols~\cite{mysticeti, bullshark, narwhal}. 

Here, we briefly describe how to obtain a practical DAG-based deterministic consensus protocol for the \model.

One such protocol can be easily derived from the partially synchronous version of Bullshark~\cite{PSBullshark}, by removing timeouts. The original algorithm uses timeouts to ensure liveness, as follows: 
\begin{enumerate}
    \item In even rounds, upon receiving $n-f$ messages that do not contain a message from the round's anchor (leader), a correct process additionally waits for a message from the leader, or a timeout.
    \item In odd rounds, correct processes wait for $f+1$ messages that vote for the previous round's anchor, or for $n-f$ messages that do not vote for the anchor, or for a timeout.
\end{enumerate} 

Without the timeouts, the original algorithm would not satisfy liveness in the partially synchronous model, since the adversary can always delay messages such that, for example, no correct process ever receives the anchor message of any round among the first $n-f$ messages, which would prevent progress. Note that, even with the timeout, it is possible, before GST, for some anchors not to be committed.

However, in the \model, we can remove the timeouts, thus obtaining a deterministic asynchronous consensus protocol, and still guarantee liveness (with probability 1). 

\alberto{
We can discuss the composition of \model with synchrony to obtain a "partial random async model", and indicate that they inherit the best of both worlds
}

