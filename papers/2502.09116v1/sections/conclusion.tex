\section{Conclusion}

We introduce the random asynchronous model, a novel relaxation of the classic asynchronous model that replaces adversarial message scheduling with a randomized scheduler. By eliminating the adversary’s ability to indefinitely delay messages, our model circumvents traditional impossibility results in asynchronous Byzantine consensus while preserving unbounded message delays and tolerating Byzantine faults. Our approach avoids the need for synchronized periods (as in partial synchrony) or cryptographic randomness (as in randomized consensus), offering a foundation for practical alternatives to existing asynchronous systems. 
%
We demonstrated that this relaxation enables new feasibility results across different resilience thresholds: deterministic safety and probabilistic termination for $n=3f+1$, deterministic termination with safety holding with high probability (whp) for $n=2f+1$, and weak validity with whp agreement for $n=f+2$. These results are complemented by impossibility bounds, showing our protocols achieve near-optimal guarantees under the model. 

Future work could explore extensions of this model to other distributed computing problems, such as state machine replication, and investigate empirical performance trade-offs in real-world deployments. By bridging the gap between theoretical impossibility and practical assumptions, we believe our model opens avenues for efficient, resilient consensus protocols.
