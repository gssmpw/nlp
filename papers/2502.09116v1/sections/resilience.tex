\section{$n = 2f+1$: Deterministic termination, safety whp}\label{sec:2f+1}

In this section, we pose $n=2f+1$ and solve strong Byzantine consensus such that safety (validity and agreement) holds with high probability and termination is deterministic.
% In other words, the fair model enables higher resilience at the cost of relaxed safety.

\Cref{alg:resilient-consensus} shows our proposed protocol. The main idea is as follows: There are $f+1$ phases, each consisting of $R$ communication rounds, where $R$ is a parameter. $R$ is large enough so that correct processes hear from each other at least once within $R$ rounds with high probability. In each round, a correct process sends its set of \textit{accepted values} ($V_i$ in \Cref{alg:resilient-consensus}) to all other processes. A correct process $i$ \textit{accepts} a value from process $j$ at phase $p$ ($p = 1,\ldots,f+1$) if (1) $j$ is the origin of $v$ (i.e., the first signature on $v$ is by $j$), (2) $i$ has not already accepted a value from $j$, and (3) $v$ has valid signatures from $p$ distinct processes. We say that $i$ accepts a value $v$ at phase $p$ if $p$ is the earliest phase at which $i$ accepts $v$ (from any process).

After the $f+1$ phases are over, a correct process decides on the majority value within its set of accepted values (i.e., the value that appears most often). Note that at the end of the communication phases, each correct process must have at least one accepted value, since correct processes sign and send their input value in phase $1$ and the network is reliable.

% Each correct process $p$ performs a predetermined number $R$ of communication rounds. In the first round, $p$ sends its input value to all processes. In each subsequent round, $p$ sends its history (the set of all messages it has received in previous rounds) to all processes. After $R$ rounds of communication, $p$ decides based on the input values it has received: if $p$ has observed $0$ as an input value from at least $f+1$ processes, then it decides $0$; otherwise it decides $1$.

The main intuition behind this protocol is that, if a correct process accepts a value $v$, then all correct processes will accept $v$ by the end of the execution, and thus all correct processes will have the same set of accepted processes. Therefore, all correct processes can use a deterministic rule to decide the same value.

\begin{algorithm}
\caption{Binary Byzantine consensus for $n = 2f+1$; pseudocode for process $i$}
\label{alg:resilient-consensus}
\begin{algorithmic} [1]
\State \textbf{Local variables:}
\State \hskip1em $V_i = \varnothing$, a map from processes to signed values
\bigskip
\State \textbf{procedure} \textsc{Propose}$(v)$:
\State \hskip1em $V_i[i] \gets \textsc{Sign}(v)$
\State \hskip1em \textbf{for} $phase$ in $1\ldots f+1$:
\State \hskip2em \textbf{for} $round$ in $1\ldots R$:
\State \hskip3em Send $(V_i, phase, round)$ to all processes
\State \hskip3em $valid \gets 0$
\State \hskip3em \textbf{while} $valid < n-f$:
\State \hskip4em Receive a $(V_j, p, r)$ message\BlueComment{receive single message per process, phase and round}
\State \hskip4em \textbf{if} $V_j$ contains a value $v$ with valid signatures from $phase$ distinct processes \textbf{and} $V_i[\textsc{Origin}(v)] = \varnothing$
\State \hskip5em $V_i[\textsc{Origin}(v)] \gets \textsc{Sign}(v)$
\State \hskip4em \textbf{if} $(p,r) = (phase, round)$
\State \hskip5em $valid \gets valid + 1$

\State \hskip1em \textbf{decide} $\textsc{MajorityValue}(V_i)$
\end{algorithmic}
\end{algorithm}

We now prove that \Cref{alg:resilient-consensus} satisfies the consensus properties in \Cref{sec:model}.

\begin{lemma}\label{lem:all-hear}
    With high probability, every correct process receives at least one message from every other correct process in each phase.
\end{lemma}
\begin{proof}
We start by fixing two correct processes $p$ and $q$ and upper bounding the probability that $q$ does not receive any message from $p$ after $R$ iterations. The probability of not delivering $p$'s round-$1$ message to $q$ is at most $(1-\Cnf)$ at each time step. There must be at least $R(n-f)^2$ time steps for $q$ to complete $R$ iterations, so the probability of $q$ not observing $p$'s message is at most $$(1-\Cnf)^{R(n-f)^2} \approx e^{-R\Cnf(n-f)^2}.$$

To finish the proof, we upper-bound the probability of any correct process not observing the input value of some other correct process. We first compute the expected number of (ordered) pairs of processes $(p,q)$ such that $q$ does not observe the input value of $p$ at the end of the $R$ iterations. There are $n(n-1)$ possible pairs, so this expected value is at most $E = n(n-1)e^{-R\Cnf(n-f)^2}$. Now, by Markov's inequality, we have that 
$$\Pr(\text{at least one unreachable pair}) \leq E = n(n-1)e^{-R\Cnf(n-f)^2}.$$
For fixed $n$ and $f$, this probability approaches $0$ exponentially in the number of iterations $R$.
\end{proof}

\begin{lemma}\label{lem:all-accept}
    With high probability, every correct process accepts the input values of every other correct process.
\end{lemma}
\begin{proof}
    By \Cref{lem:all-hear}, every correct process receives at least one message from every other correct process in the first phase, whp. Since messages from correct processes always contain their input values with a valid signature, every correct processes $i$ will accept the input value of another correct process $j$ when $i$ receives the first message from $j$. 
\end{proof}

\begin{theorem}
    With $n=2f+1$, \Cref{alg:resilient-consensus} satisfies strong validity whp.
\end{theorem}
\begin{proof}
    Assume that all correct processes propose the same value $v$. Then, by \Cref{lem:all-accept}, all correct processes will accept at least $n-f=f+1$ $v$ values whp. Since $f+1$ is a majority out of a maximum of $n=2f+1$ accepted values, correct processes decide $v$ whp.
\end{proof}

\begin{lemma}\label{lem:accept-early}
    If a value $v$ is accepted by a correct process $i$ at phase $p \leq f$, then all correct processes accept $v$ by the end of phase $p+1$, whp.
\end{lemma}
\begin{proof}
    If $i$ accepts $v$ at phase $p$, then $v$ must have signatures from at least $p$ processes. Process $i$ is not one of the $p$ processes, otherwise $i$ would have accepted $v$ at an earlier phase. Since $i$ accepts $v$, $i$ adds its signature to $v$ and will send $v$, as part of $V_i$, to all processes in phase $p+1$. By \Cref{lem:all-hear}, all correct processes will thus receive $v$ by the end of phase $p+1$ whp, and will accept $v$ (if they haven't already), since $v$ has the required number of signatures.
\end{proof}

\begin{lemma}\label{lem:accept-last}
    If a value $v$ is accepted by a correct process $i$ at phase $f+1$, then all correct processes accept $v$ by the end of phase $f+1$, whp.
\end{lemma}
\begin{proof}
    If $i$ accepts $v$ at phase $f+1$, then $v$ must have signatures from at least $f+1$ processes; $i$ is not among these processes, otherwise $i$ would have accepted $v$ at an earlier phase. Since there are at most $f$ faulty processes, $v$ must have at least one signature from a correct process $j \ne i$.

    So $j$ must have accepted $v$ at an earlier phase $p \leq f$ and therefore, by \Cref{lem:accept-early}, all correct processes will accept $v$ by the end of phase $f+1$ whp.
\end{proof}

\begin{theorem}\label{thm:2f+1-agreement}
    With $n=2f+1$, \Cref{alg:resilient-consensus} satisfies agreement whp.
\end{theorem}
\begin{proof}
    By \Cref{lem:accept-early} and \Cref{lem:accept-last}, with high probability, correct processes have the same set of accepted values by the end of phase $f+1$, and thus decide the same value.
\end{proof}

\begin{theorem}
    With $n=2f+1$, \Cref{alg:resilient-consensus} satisfies deterministic termination.
\end{theorem}
\begin{proof}
    Follows immediately from the algorithm: correct processes only execute for $R(f+1)$ rounds. In each round, a correct process waits to receive $n-f$ messages from that round, which is guaranteed to occur since there are at least $n-f$ correct processes and the network is reliable (no-loss property).
\end{proof}



% \begin{lemma}
%     Let $A = [n]$ and $h\leq n$. Let $A_1,...,A_k$ be subsets of $A$ of size $h$, drawn uniformly and independently at random. Then $A_1,...,A_k$ cover $[h]$ with probability at least $1-h\left(\frac{n-h}{n}\right)^k$.
% \end{lemma}
% \begin{proof}
%     Let $B$ be the event that $A_1,...,A_k$ cover $[h]$. Let $B_{i,j}$ be the event that $A_i$ contains $j$ for $i \in [k]$, $j \in [h]$. Then 
%     $$B = \bigcap_{j\in[h]}\bigcup_{i \in [k]}B_{i,j}.$$
%     By inclusion-exclusion,
%     $$B^c = \bigcup_{j\in[h]}\bigcap_{i \in [k]}B^c_{i,j}.$$
%     By Boole's inequality, 
%     $$P(B^c) \leq \sum_{j\in[h]}\bigcap_{i \in [k]}B^c_{i,j}.$$
%     By the independence of the $A_i$'s, 
%     $$P\left( \bigcap_{i \in [k]}B^c_{i,j} \right) = P(B^c_{1,j})^k.$$
%     Now, 
%     $$P(B^c_{1,j}) = P(A_1 \cup \{j\}) = \frac{{n-1 \choose h}}{{n \choose h}} = \frac{n-h}{n}.$$
%     Putting it all together:
%     $$P(B) = 1-P(B^c) \geq 1 - \sum_{j\in[h]}\left(\frac{n-h}{n}\right)^k = 1 - h\left(\frac{n-h}{n}\right)^k$$
% \end{proof}

\section{$n = f+2$: Deterministic Termination and Weak Validity, Agreement Whp}

Interestingly, if we pose $n=f+2$, we can solve weak Byzantine consensus with deterministic validity and termination, and agreement \textit{whp}, using the same protocol in \Cref{alg:resilient-consensus}.

Weak validity is clearly preserved: if all processes are correct and have the same input value $v$, no other value is received by any process, and thus all processes decide $v$.


\begin{theorem}
    With $n=f+2$, \Cref{alg:resilient-consensus} satisfies deterministic weak validity.
\end{theorem}
\begin{proof}
    If all processes are correct and propose the same value $v$, then all $(V_i, phase, round)$ messages will have $v$ as their value, so no process can decide any other value.
\end{proof}

\begin{theorem}
    With $n=f+2$, \Cref{alg:resilient-consensus} satisfies agreement whp.
\end{theorem}
\begin{proof}
    Lemmas~\ref{lem:all-hear}, \ref{lem:all-accept}, \ref{lem:accept-early}, and \ref{lem:accept-last} still hold: their proofs are also valid if $n=f+2$. Thus the proof of this theorem is the same as the proof of \Cref{thm:2f+1-agreement}: By \Cref{lem:accept-early} and \Cref{lem:accept-last}, with high probability, correct processes have the same set of accepted values by the end of phase $f+1$, and thus decide the same value.
\end{proof}

\begin{theorem}
    With $n=f+2$, \Cref{alg:resilient-consensus} satisfies deterministic termination.
\end{theorem}
\begin{proof}
    Correct processes only execute for $R(f+1)$ rounds. In each phase and round, a correct process waits to receive $n-f$ valid messages from that phase and round. This wait is guaranteed to terminate since there are at least $n-f$ correct processes, correct processes can always produce a valid message (a message $(V_i, p, r)$ is valid if $p$ and $r$ are equal to the current phase and round, respectively), and the network is reliable (no-loss property).
\end{proof}