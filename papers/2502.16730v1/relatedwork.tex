\section{Related Work}
\label{sec:relatedwork}

\subsection{LLM-Based Penetration Testing}

\subsubsection{PentestGPT – Task Tree-Driven AI Pentesting}
Recent research has explored using large language models (LLMs) to automate penetration testing. \textbf{PentestGPT} \cite{pentestgpt2024} is a notable example: it leverages an LLM (GPT-3.5/GPT-4) to guide the pentest process via a \textit{Pentesting Task Tree (PTT)} structure. Inspired by attack trees, the PTT decomposes engagements into sub-tasks (e.g., port scanning, service enumeration, exploitation), allowing the LLM to maintain context throughout testing. PentestGPT operates using three coordinated modules: a \emph{Reasoning} module (the "lead tester") that updates the task tree and determines next steps, a \emph{Generation} module (the "junior tester") that proposes specific commands, and a \emph{Parsing} module to summarize tool output.

While PentestGPT automates attack planning, it requires a human-in-the-loop to execute suggested commands and correct errors. Users must review and refine commands before execution, limiting its autonomy. Thus, PentestGPT functions more as a guided assistant rather than a fully autonomous pentesting tool.

\subsubsection{Other LLM-Driven Pentesting Tools}
Beyond PentestGPT, several emerging tools utilize LLMs for penetration testing, each focusing on different aspects of the workflow. These tools can be categorized as follows:

\paragraph{Tools that Automate Initial Access but Focus on Broad Vulnerability Scanning Rather than Speed}
\begin{itemize}
    \item \textbf{PenHeal} \cite{penheal2023} – an AI agent that operates without direct human involvement, designed to identify a broad range of vulnerabilities and propose mitigation strategies. Although the paper does not explicitly confirm automation of initial foothold attacks, it is possible that PenHeal’s capabilities overlap with RapidPen in terms of initial access. However, no evaluation is provided regarding the time and cost required to achieve initial access. In contrast, RapidPen focuses on demonstrating the most immediate security risk—namely, gaining unauthorized shell access as quickly as possible—before handing over control to established tools designed for post-exploitation. While RapidPen does not yet provide broad vulnerability coverage or automated remediation, incorporating such features remains an area for future exploration.
\end{itemize}

\paragraph{Tools That Focus on Post-Exploitation and Are Complementary to RapidPen}
\begin{itemize}
    \item \textbf{BLADE} \cite{blade2024} – \textbf{B}reaking \textbf{L}imits, \textbf{A}utomate \textbf{D}eep \textbf{E}xploitation – an AI-driven pentesting agent built on an autonomous agent framework (Microsoft’s AutoGen~\cite{autogen2023}). BLADE autonomously orchestrates exploitation tasks by leveraging external tools and dynamic script generation. For example, it uses pre-configured tools like LinPEAS for privilege escalation and John the Ripper for credential cracking to achieve deeper system compromise. Additionally, it includes agents for network scanning and lateral movement, showcasing how multi-agent AI systems can enhance penetration testing workflows.
    \item \textbf{AutoAttacker} \cite{autoattacker2024} – an LLM-guided system designed to implement automated “hands-on-keyboard” cyber-attacks in post-breach scenarios.
    \item \textbf{Wintermute} \cite{happe2023wintermute} – an LLM-driven Linux privilege escalation tool that evaluates model performance in fully automated exploit scenarios. It highlights strengths and weaknesses in autonomous security workflows, focusing on post-exploitation.
\end{itemize}

\subsection{Reinforcement Learning-Based Penetration Testing Approaches}

\textit{Deep reinforcement learning (RL)} has also been explored for autonomous pentesting. RL-based systems learn attack sequences by interacting with an environment and optimizing for successful exploits. Key contributions include:
\begin{itemize}
    \item \textbf{Hu et al.} \cite{hu2020deeprl} developed a deep RL framework for automated penetration testing, modeling scanning, exploitation, and lateral movement as a reinforcement learning problem.
    \item \textbf{Garrad and Unnikrishnan} \cite{garrad2023vanet} applied RL to vehicular ad-hoc network (VANET) penetration testing, demonstrating AI-driven attack sequence learning.
    \item \textbf{Liu et al.} \cite{liu2024hierarchical} proposed a hierarchical RL agent for large-scale network penetration, improving efficiency by splitting attack planning into multiple levels.
    \item \textbf{DeepExploit} \cite{takaesu2018deepexploit,deepexploit2019}, an early RL-powered pentesting tool integrated with Metasploit, demonstrated full automation of initial access but suffered from overfitting to training environments.
\end{itemize}
While RL-based pentesting can autonomously uncover attack paths, its major drawback is \textbf{poor generalization} beyond training data, requiring extensive retraining for new environments.

\subsection{Comparison with RapidPen}

\paragraph{Degree of Automation:} RapidPen is designed for full automation of initial access, requiring no human intervention once launched. This sets it apart from PentestGPT, which requires users to review and execute commands manually. In contrast, RL-based systems like DeepExploit require extensive training and tuning before deployment, making RapidPen a more practical choice for real-world pentesting with minimal setup overhead.

\paragraph{Scope of Initial Access Techniques:} RapidPen focuses on achieving unauthorized shell access as quickly as possible, covering a broad range of network and system-level exploitation techniques. Unlike PentestGPT, which primarily provides recommendations, RapidPen directly executes exploits. Meanwhile, tools like BLADE and AutoAttacker specialize in post-exploitation rather than initial access, making them complementary rather than competing solutions.

\paragraph{Usability for Non-Experts:} RapidPen is explicitly designed for usability by non-experts, enabling security assessments without deep penetration testing expertise. Unlike PentestGPT, which still requires expert validation of generated commands, RapidPen autonomously performs the entire attack process. Additionally, tools like Autonomous Web Exploitation target a different domain (web applications), leaving gaps in usability for broader infrastructure pentesting.

Overall, RapidPen distinguishes itself by combining \textbf{full automation, speed, and accessibility}. It provides a \textbf{highly practical and deployable solution} for automated initial access testing, making it a valuable tool for security practitioners and organizations looking to assess their exposure to real-world attack scenarios.