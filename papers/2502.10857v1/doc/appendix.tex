\section{Details of Hybrid Instruction Tuning}

\label{sec:appendix1}
We show the proportion of MathInstruct~\cite{yu2023metamath}, CodeInstruct~\cite{wei2023magicoder}, and EDAInstruct~\cite{wu2024chateda} datasets for hybrid instruction tuning of ChipLlama models in \Cref{table:datasets}.

Firstly, the MathInstruct dataset focuses on CoT reasoning, strengthening the logical reasoning skills required for EDA task planning. 
Then, the  CodeInstruct dataset presents complex coding tasks and solutions, offering coding skills for EDA script generation. 
Finally, we utilize the EDAInstruct dataset which provides the EDA domain-specific knowledge and instructions for EDA flow automation.
The composition of these datasets creates a hybrid corpus meticulously designed to expand the capabilities of ChipLlama models. 

\begin{table}[!htbp]
\centering
\resizebox{0.928\linewidth}{!}{
\begin{tabular}{cccc}
\toprule
& MathInstruct & CodeInstruct & EDAInstruct \\ 
\midrule
Proportion & 80K & 100K & 8K \\
\bottomrule
\end{tabular}}
\caption{The proportion of MathInstruct, CodeInstruct and EDAInstruct datasets for hybrid instruction tuning.}
\label{table:datasets}
\end{table}

\begin{figure*}[!tb]
    \centering
    \includegraphics[width=0.975\linewidth]{figs/figure8.pdf} 
    \caption{Case Studies of EDA flow automation with our EDAid powered by ChipLlama models. Each case provides an EDA task, its corresponding task planning pathway and the generated EDA script.}
    \label{fig:cases}
\end{figure*}

\section{Agents in EDAid}
\label{sec:appendix2}
In our experiments, we use three agents to generate divergent thoughts and one agent to compute the probabilities of these thoughts to make the final decision. 
When the number of agents generating divergent thoughts is less than three, the stability and accuracy of EDA flow automation with EDAid improve with the addition of more agents. 
However, once the number of agents reaches three, the performance tends to saturate, meaning that further increases in the number of agents do not necessarily lead to significant improvements in performance. 
In practical applications, considering the costs associated with real-world EDA flows, as we mentioned before, an appropriate increase in the number of agents can still be acceptable if it enhances the system's overall stability and reliability.

\section{More Case Studies}
\label{sec:appendix3}
We provide more case studies to figure out how our EDAid resolves the given EDA task.
% We outline the EDA task and then present the task planning pathway.
As illustrated in \Cref{fig:cases}, we provide two EDA tasks and their corresponding task planning pathways and generated EDA scripts.
Both EDA tasks require the system to provide a parameter-tuning solution.
Our system appropriately grasps the need for the given EDA task and shows an excellent understanding of the details of each API interface parameter.