\section{Related Works}

\minisection{Chain-of-Thought}
The CoT paradigm~\cite{wei2022cot} encourages LLMs to break down complex problems into several intermediate steps, emulating the way humans reason through a problem. 
Rather than directly outputting the final answer, LLMs are required to generate a step-by-step reasoning process, which can help models handle more complex tasks.
Self-consistency with CoT (CoT-SC)~\cite{wang2022selfcon} improves upon CoT by generating different thought processes for the same problem and the output decision can be more reliable by exploring a richer set of thoughts.
In this paper, we utilize the CoT-SC paradigm to collaborate with various agents in EDAid to perform EDA task planning and script generation for EDA flow automation. 

\minisection{In-Context Learning}
In-context learning \cite{min2022metaicl} has emerged as a transformative paradigm in machine learning, characterized by the meticulous training of models to perform specific tasks through examples and directives provided within an interactive, conversational framework. 
The ability of in-context learning emerges~\cite{wei2022emergent} in large-scale, versatile LLMs~\cite{openai2023gpt4, 2024claude, dubey2024llama3}. 
These models demonstrate an impressive ability to leverage their broad knowledge across various downstream tasks through in-context learning~\cite{brown2020gpt3}.  
In our research, we apply few-shot prompts to enhance the performance and reliability of each agent in EDAid based on in-context learning.

\minisection{LLM-powered Agent System}
Single-agent systems driven by LLMs have demonstrated remarkable cognitive capabilities~\cite{sumers2023cognitive, wang2024agentsurvey, xi2023llmagentsurvey, wu2024chateda}. 
These LLM-powered agents can decompose complex tasks into manageable subtasks~\cite{khot2022decomposed} and methodically think through each component to make better decisions. 
Moreover, the tool-calling capability~\cite{qin2023toolllm} of LLMs enables agent systems to leverage external resources and tools, allowing them to operate more effectively in various scenarios.
Developed based on the single-agent system, several studies~\cite{hong2024metagpt, wang2023unleashing, du2023improving, hao2023chatllm} have enhanced the problem-solving abilities of LLMs by integrating discussions among multiple agents.
The collaboration of multiple autonomous agents, each equipped with unique strategies, can address more dynamic and complex tasks. 
In this work, multiple agents in EDAid collaborate with divergent thoughts to automate EDA flow via complex long-chain EDA tool-calling.
