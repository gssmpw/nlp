\section{Introduction}
Electronic design automation (EDA) is indispensable for the design of integrated circuits (ICs).
EDA tools are integrated into a complex design flow and utilize programming interfaces to control the design process.
EDA platforms such as OpenROAD~\cite{ajayi2019openroad} and iEDA~\cite{li2024ieda}, consist of complex procedures with various configurations.
Circuit design engineers utilize EDA tools iteratively to fulfill design targets, relying on tailored scripts that manipulate these tools via programming interfaces.
However, interacting with EDA tools through scripting~\cite{chen2001scripteda} is often laborious and error-prone. 
This complexity is further intensified when design teams employ tools from various vendors in the circuit design process.

Large language models (LLMs)~\cite{openai2023gpt4, 2024claude, dubey2024llama3} have demonstrated profound instruction comprehension, planning, and reasoning capabilities. 
The potential of LLMs to interact with diverse tools for executing complex tasks has gained increasing recognition~\cite{qin2023toolllm}. 
Researchers have explored the automation of complex EDA flows by interfacing with EDA tools via LLMs~\cite{wu2024chateda, liu2023chipnemo}.
Specifically, ChatEDA~\cite{wu2024chateda} uses LLMs as ``brains'' of the agent to generate EDA scripts, automating EDA tool utilization, reducing the workload of circuit design engineers, and minimizing errors.

Although LLMs have shown potential in EDA flow automation, significant challenges remain.
Firstly, although LLMs excel at understanding natural language, they lack specialized knowledge of EDA tool usage.
These tools are designed for specific tasks such as logic synthesis, floorplanning, placement, and routing, each requiring detailed domain-specific knowledge and familiarity with various EDA tool interfaces.
To solve these problems, LLMs can be fine-tuned on datasets containing tutorials and EDA scripts specific to EDA tools~\cite{wu2024chateda}.
However, this approach presents its problems.
Each EDA platform has a unique set of commands and flows that must be mastered for effective use. 
If LLMs only focus on a specific EDA tool or platform during the instruction tuning process will limit their cross-platform utility, potentially reducing the effectiveness in practical scenarios.
Furthermore, EDA flows typically involve a sequence of intermediate steps. 
A single error in any of these steps can lead to failure in the overall process.
This risk is compounded by the probabilistic nature of LLMs, which may produce varying solutions to the same task. 
Moreover, errors may occur in intermediate steps during a long-chain tool-calling process, introducing instability into the EDA flow automation.

%Contribution
To address these challenges, we introduce a multi-agent collaboration system, EDAid, designed for EDA flow automation through script generation in response to natural language instructions. 
This system is characterized by the collaboration of multiple agents, which are powered by the LLMs.
As mentioned earlier, EDA flow automation is challenging even for the greatest LLM such as GPT-4~\cite{openai2023gpt4}. 
Consequently, we develop ChipLlama models, expert LLMs fine-tuned for EDA flow automation. 
We specifically focus on improving the understanding of overall EDA flow rather than simple EDA tool usage during the fine-tuning process.
We also employ few-shot chain-of-thought (CoT) prompts for each agent to further develop the performance and portability through the in-context learning capability of LLMs. 
Based on multiple agents, we propose the multi-agent system, EDAid, to ensure stability and avoid erroneous intermediate steps during the long-chain EDA tool calling process.
In this system, multiple agents collaborate with divergent thoughts following different few-shot contexts and then make the final decision based on divergent thoughts, working in concert to automate the EDA process.
Our EDAid can interpret human instructions, plan EDA tasks, and interface with EDA tools through APIs, serving as a valuable assistant in automating EDA flows and eliminating the need for manual intervention.
In summary, our contributions are as follows:
\begin{itemize}[itemsep=0pt,topsep=0pt,parsep=0pt]
	\item We develop the ChipLlama-powered agent, collaborating with few-shot CoT prompts, to develop the performance and portability of the single-agent system;
    \item Propose EDAid, a multi-agent system that collaborates multiple agents including divergent-thoughts agents and a decision-making agent for EDA flow automation;
    \item Perform extensive evaluations, which demonstrate the SOTA performance of ChipLlama models, the effectiveness of the few-shot CoT prompting, and the superior performance of our EDAid for EDA flow automation. 
\end{itemize}

