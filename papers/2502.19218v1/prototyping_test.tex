\section{CPG-Based Manipulations on Prototype}
\label{sec:prototype}

This section describes the laboratory experiments conducted with the Ori-Pixel robotic surface to manipulate various objects using the optimized CPG motions from~\sectionref{sec:motion_optimization}.

\begin{figure}[t]
    \centering
    \includegraphics[width = \columnwidth]{lab_test.pdf}
    \caption{\textbf{Experimental validation.} (a) Comparison of simulated and prototype object motions for translational and rotational manipulations; (b) lab test setup.
    }
    \label{fig:prototype}
\end{figure}

\subsection{Experiment Setup}
\label{sec:exp_setup}
\begin{figure*}[t]
    \centering
    \includegraphics[width = \textwidth]{sim_results.pdf}
    \caption{\textbf{Simulation results for system robustness analysis.} (a) Averaged manipulation velocity from simulation for various object masses and widths using the Ori-pixel module spacing of 120\,mm as a reference unit; (b) averaged manipulation velocity from simulation with various contact friction coefficients.
    }
    \label{fig:dis}
\end{figure*}
The Ori-Pixel robotic surface, comprising 25 modules each actuated by three Dynamixel XL-320 servos operating at 20\,Hz in position mode, is used for laboratory experiments (\figureref{fig:prototype}(b)). A Vicon Vero motion tracking system provides real-time object pose data for module activation control following \figureref{fig:alg}(a). Experiments are conducted with objects of varying size, weight, shape, and stiffness, as listed in \tableref{tab:exp}.

\subsection{CPG-Based Manipulation Experiments on Ori-pixel}
\label{sec:lab_exp}

The optimized CPG parameters are applied to manipulate a spectrum of objects as listed in~\tableref{tab:exp}. The system successfully manipulated objects ranging from a small plate (ii) to a much larger plate (iv), demonstrating the optimized motion's ability to handle a wide size range. The system also successfully manipulated objects with flexible materials and irregular shapes (v, vi) where position tracking becomes challenging. The platform operates in open-loop mode, with all modules activated following optimized CPG motions without position feedback. These experiments demonstrate that motions derived using the proposed framework can successfully handle objects of varying sizes, shapes, and stiffness, showcasing the robustness and versatility of the proposed manipulation strategy. Complete documentation is available in supplementary video~2.

\begin{table}[t]
\centering
\caption{Properties of Tested Objects}
\renewcommand{\arraystretch}{1.2}
\setlength{\tabcolsep}{4pt}
\begin{tabular}{c l l l c c}
\hline
\textbf{Index} & \textbf{Shape} & \textbf{Material} & \textbf{Size (mm)} & \textbf{Mass (g)} & \textbf{Tested Modes} \\
\hline
i   & Plate      & Acrylic  & 300$\times$300 & 254  & Fast, Smooth \\
ii  & Plate      & Wood  & 200$\times$200 & 172  & Fast, Smooth \\
iii & Cylinder   & Foam     & $\diameter$36$\times$140 & 9  & Smooth \\
iv  & Plate      & Acrylic  & 1000$\times$300 & 1000 & Fast \\
v   & Polo shirt & Fabric   & 400$\times$400 & 280  & Fast \\
vi  & Trilby hat & Straw   & 270$\times$250 & 55  & Fast \\
\hline
\end{tabular}
\label{tab:exp}
\end{table}

We evaluated the fast and smooth manipulation modes using object i as depicted in \figureref{fig:prototype}(a). Fast manipulation achieved higher velocities (30\,mm/s Y-direction, 25\,mm/s X-direction) compared to smooth manipulation (20\,mm/s Y-direction, 17\,mm/s X-direction). However, smooth manipulation demonstrated superior stability with lower Z-direction displacement (averaged standard deviation: 3.03\,mm vs 7.05\,mm) and rotation angles (averaged standard deviation: 0.0091\,rad vs 0.0133\,rad). To demonstrate stability, we successfully manipulated object i while supporting an unrestrained object iii (Figure~\ref{fig:concept}(b)(iii)). Pure rotational tests achieved average angular velocities of 0.079\,rad/s clockwise and 0.063\,rad/s counterclockwise. All experiments are documented in supplementary video~3.

\subsection{Sim-to-real Analysis}
\label{sec:sim_real}
To analyze the sim-to-real gap, we conducted simulations using object i with optimized CPG-based control parameters and compared them with experimental results, as shown in \figureref{fig:prototype}(a). Our simulation demonstrates strong alignment with the actual dynamic behavior during manipulation, though discrepancies were observed in rotation data during translation modes and position data during rotation modes. These gaps mainly stem from actuation delays between servos in the physical platform and natural variations in object placement during trials. Despite these differences, the control parameters optimized in simulation transferred effectively to real-world implementation, validating our sim-to-real approach.


\subsection{Conclusion}

This section evaluated the CPG-based manipulation motions derived from the simulation-based optimization process through prototype experiments. The experiments demonstrated high-fidelity sim-to-real transfer and validated the proposed framework by successfully manipulating objects of various size, shape, and stiffness, while executing fast and smooth manipulation modes to meet different performance requirements.