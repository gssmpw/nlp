\section{CPG Motion Planning and Optimization for Manipulations}
\label{sec:motion_optimization}

\begin{figure*}[t]
    \centering
    \includegraphics[width = \textwidth]{CPG.pdf}
    \caption{\textbf{Single-module CPG motion plan and inter-group motion plan. }(a) Single module motion plan; (b) multi-module manipulation motion plan. (c) inter-group motion planning for translation manipulations; (d) motion planning for clock-wise rotation manipulation.
    }
    \label{fig:CPG}
\end{figure*}

This section presents a novel CPG-based motion generation framework with simulation-based optimizations that, while demonstrated on Ori-Pixel platform, offer broad applicability across robotic systems. The proposed framework represents the first implementation of surface manipulations capable of handling diverse object geometries and stiffness. The following details the CPG parameter design, and the optimization framework across different manipulation modes.

\subsection{CPG-based Manipulation Motion Planning}
\label{sec:CPG_plan}
The CPG-based manipulation motion of a single 3-DoF Canfield origami robot mimics walking gait generation, consisting of three steps: object engagement through top plate lifting, object pushing through plate tilting, and plate retraction for disengagement, as shown in~\figureref{fig:CPG}(a). 

\begin{figure*}[t]
    \centering
    \includegraphics[width = \textwidth]{algorithms.pdf}
    \caption{\textbf{Control and Optimization Frameworks.} (a) Control framework for CPG-based manipulation; (b) optimization framework for CPG parameters.
    }
    \label{fig:alg}
\end{figure*}

A CPG-based controller is designed to produce synchronized manipulation movements at the single module's top plate. These movements are governed by coupled sinusoidal functions in height $H$ and inclination angle $\psi$, as illustrated in~\figureref{fig:modeling}(a). The variation of $H$ and $\psi$ over time can be described as:
\begin{align}
    H(t) &= h_\text{amp} \sin(2\pi f \cdot t + \phi) + h_0\label{eq:height}, \\
    \psi(t) &= \psi_\text{amp} \sin(2\pi f \cdot t + \phi+ \sigma) + \psi_0\label{eq:psi},
\end{align}
where $h_\text{amp}$ denotes the amplitude of the height variation, $f$ is the frequency of motion, $\sigma$ represents the phase shift coupling between $H$ and $\psi$, $h_0$ and $\psi_0$ indicate the height and the inclination angle, respectively, of the top plate at its natural resting position, as depicted in~\figureref{fig:CPG}(a). Additionally, $\phi$ represents the inter-group phase shift used to coordinate multiple groups of modules for effective manipulations.

As shown in~\figureref{fig:CPG}(a), Eqs.~\eqref{eq:height} and \eqref{eq:psi} together define the motion pattern. The phase shift $\sigma$ determines the inclination angle $\psi$ when the top plate engages with the object, which dictates the manipulation direction. Specifically, if the first contact occurs during the tilt push I phase, where $\sigma \in [0, \pi)$, the manipulation direction is toward the left-hand side; if the first contact occurs during the tilt push II phase, where $\sigma \in [-\pi, 0)$, the direction shifts to the right-hand side. Furthermore, the phase shift $\sigma$ influences both the floating and tilting ranges of the object during manipulation, playing a key role in balancing the trade-off between speed and smoothness in the overall performance. This trade-off will be further studied during the optimization process in~\sectionref{sec:motion_plan}.

While single modules can complete manipulation cycles (\figureref{fig:CPG}(a)), additional module support during plate retraction is needed to prevent backward slippage. Thus, modules are divided into two groups using Eqs.~\eqref{eq:height} and \eqref{eq:psi} with identical parameters except for an inter-group phase shift $\phi$. As depicted in \figureref{fig:CPG}(b), this coordination enables simultaneous pushing and supporting motions for effective manipulations.


\subsection{Collective motion planning for manipulations using multi-module robotic surface}
\label{sec:motion_plan}
The inter-group motion plan developed for translational manipulations on the Ori-Pixel platform divides the modules into two diagonal groups, with their motion generated by the CPG described in~\sectionref{sec:CPG_plan}, as depicted in~\figureref{fig:CPG}(c). These groups are synchronized through the inter-group phase shift term, $\phi$. The diagonal symmetric configuration maintains object orientation during movement by applying forces without rotational torque, ensuring robust and stable motion.

The direction of translational manipulation is determined by two parameters: the azimuth angle $\delta$ (which defines the manipulation axis: $\delta = 0$\,degree for Y-axis and $\delta=90$\,degree for X-axis as shown in~\figureref{fig:modeling}(a) and \figureref{fig:CPG}(c)), and the in-group phase shift $\sigma$ (which determines direction along the chosen axis: $\sigma \in [0, \pi)$ for positive and $\sigma \in [-\pi, 0)$ for negative direction, as detailed in~\sectionref{sec:CPG_plan}). Together, these parameters enable omni-directional planar manipulation.

The rotational manipulation plan coordinates modules' translational movements in different directions. Objects must contact at least two by two top plates. Two module groups operate with $\phi=\pi$ phase shift. As shown in~\figureref{fig:CPG}(d), Group 1 moves along X-axis (positive in top-left, negative in bottom-right), while Group 2 moves along Y-axis (negative in top-right, positive in bottom-left) for clockwise rotation. Counter-clockwise rotation reverses these directions while maintaining $\phi$, ensuring continuous rotational manipulation.

In addition to the motion plans, the top plate's effective contact ratio ($S_{\textit{contact}} / S_{\textit{plate}}$) is crucial for manipulations, where $S_{\textit{contact}}$ represents the object-covered area and $S_{\textit{plate}}$ the total plate area. Modules are activated when this ratio exceeds a threshold $\epsilon$, otherwise returning to rest. This threshold, analyzed during optimization, ensures effective manipulation without obstruction. The complete control framework is illustrated in \figureref{fig:alg}(a).

\subsection{Optimization Framework for Manipulation Motions}
\label{sec:optimization}
A simulation-based optimization framework is developed to identify the optimal CPG parameters for different manipulation modes and objects, using the motion planned in~\sectionref{sec:motion_plan}. The complete process is elaborated in \figureref{fig:alg}(b).

The proposed framework utilizes the simulation model from Section~\ref{sec:simulation}, integrated with an evolutionary Bayesian hyperparameter optimizer~\cite{Cowen-Rivers2022-HEBO} to identify optimal parameter sets for different manipulation modes. The optimization search space spans eight parameters ${h_\text{amp}, \psi_\text{amp}, f, h_\text{0}, \psi_\text{0}, \phi, \delta, \epsilon}$ from Equations~\eqref{eq:height} and \eqref{eq:psi}. The amplitude and frequency ranges in Table~\ref{tab:opt_space} were bounded by our servo motors' physical limits. The initial height $h_0$ and orientation $\psi_0$ ranges were determined by the prototype's geometric design and kinematic workspace analysis as in~\sectionref{sec:kinematic}, while the phase parameters were constrained to ensure smooth transitions between motion states.
During the optimization process, the object is positioned at the center of one side of the robotic surface. The modules are commanded to move for 5\,seconds following the control protocol outlined in \figureref{fig:alg}(a), using the parameters suggested by the optimizer. The object's travel distance, Z-axis displacement, and rotation angles during the movement are evaluated using a cost function, which serves as reward feedback for the optimizer.

\begin{table}[htbp]
\begin{center}
\caption{Optimization Search Space}
\setlength{\extrarowheight}{1pt}
    \begin{tabular}{  c  c  c  c  p{5cm} }
    \hline
   \textbf{Parameter} & \textbf{Symbol} & \textbf{Search Space} & \textbf{Unit}\\
    \hline
    Height amplitude & $h_\text{amp}$ & $$[0.005, 0.04]$$ & m\\
    Inclination angle amplitude & $\psi_\text{amp}$ & $[0.35, 0.79]$ & radian\\
    Frequency & $f$ & $[0.1, 0.8]$ & Hz\\
    Resting height & $h_\text{0}$ & $[0.02, 0.04]$ & m\\
    Resting inclination angle & $\psi_\text{0}$ & $[-0.26, 0.26]$ & radian\\
    Height-inclination phase shift & $\sigma$ & $[0, \pi]$ or $[\pi, 2\pi]$ & radian\\
    Inter-group phase shift & $\phi$ & $[0, 2\pi]$ & radian\\
    Top plate contact threshold & $\epsilon$ & $[0.1, 0.5]$ & - \\
    \hline
    \end{tabular}
    \label{tab:opt_space}
\end{center}
\end{table}

A generalized cost function $J$ is constructed to assess manipulation performance, incorporating various manipulation objectives. As shown in Eq.~\eqref{eq:cost}, the cost function accounts for the object's absolute averaged translational speed $v$, the absolute averaged yaw speed $\omega$, the max roll ($\eta$) and pitch ($\rho$) angles, as well as the max displacement in z-direction throughout the manipulation process. The weights $\{\alpha, \beta, \gamma, \varsigma\}$ can be tuned to prioritize different manipulation objectives.

\begin{equation}
\begin{split}
J = &\ \alpha \cdot v + \beta \cdot \omega\\
    &\ + \gamma \cdot \left( \max_{t} \eta(t) + \max_{t} \rho(t) \right) + \varsigma \cdot \max_{t} z(t).
\label{eq:cost}
\end{split}
\end{equation}

For fast manipulations, where only the object's average manipulation speed $v$ is considered, the weights are set to $\{\alpha, \beta, \gamma, \varsigma \} = \{-1, 0, 0, 0\}$, ensuring that the optimizer focuses on maximizing the manipulation speed. In contrast, for smooth manipulations aimed at minimizing rotation, tilting, and shaking, the weights are adjusted to $\{\alpha, \beta, \gamma, \varsigma \} = \{-0.2, 0.3, 0.3, 0.3\}$, which directs the optimizer to prioritize reducing pose variations during the manipulation while still maintaining a reasonable speed. For rotational manipulations, the weights are set to $\{\alpha, \beta, \gamma, \varsigma \} = \{1, -1, 0, 0\}$, penalizing translational motions while rewarding yaw rotations. 

Using the optimization framework and simulation environments from~\sectionref{sec:simulation}, we optimized CPG parameters (\tableref{tab:opt_space}) for various manipulation motions (omnidirectional planar translations in fast/smooth modes and pure rotations). Due to the asymmetric workspace as discussed in \sectionref{sec:kinematic}, each case was optimized separately. For fast modes, we fixed $\phi=\pi$ to maximize contact time, while for smooth modes, $\phi$ was optimized to control contact transitions and minimize vertical displacement. The framework converged in 30 minutes on an AMD Ryzen Threadripper 7960X with 128GB RAM. Results are shown in supplementary video~1.
