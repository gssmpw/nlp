\section{Discussion}
\label{sec:discussion}
This section analyzes experimental results to evaluate our CPG-based manipulation framework. We demonstrate the framework's robustness across varying object properties (mass and size) and contact friction conditions, followed by a discussion of its key assumptions and limitations.

\subsection{Robustness Analysis}
\label{sec:robustness}
For robustness analysis of our proposed CPG-based manipulation framework, we conducted a series of simulations with the Ori-pixel platform serving as our experimental testbed. We analyzed box-shaped objects with a fixed height of 50\,mm but varying mass and width. The mass ranged from 50\,g to 950\,g in 150\,g increments, and using the Ori-pixel module spacing of 120\,mm as a reference unit, we varied object widths from 1.25 to 8.75 module spans in 1.25-span increments. We simulated translational manipulations in all directions using fast and smooth modes, with results shown in \figureref{fig:dis}(a). The analysis reveals that manipulation performance strongly correlates with module coverage, where objects spanning 2$\times$2 modules result in manipulation that is more sensitive to object properties and achieves lower velocities, while coverage of 3$\times$3 modules or more enables robust, high-velocity manipulation. This improved performance with larger coverage stems from better load distribution across modules, which helps mitigate the velocity reduction effects from increasing object mass. These results demonstrate the framework's robustness across a range of object masses and sizes.

We then investigated the effect of contact friction between the object and the platform. Using the same simulation setup with object i as in \tableref{tab:exp}, we varied the friction coefficient from 0.02 to 1 in 0.02 increments, testing all directional translational manipulations in both fast and smooth modes. The average velocities are shown in \figureref{fig:dis}(b). At friction coefficients below 0.3, manipulation velocity shows unstable saturation behavior. Above 0.3, the velocity stabilizes, indicating robust performance. The green shaded region (0.3-0.9) highlights that our proposed manipulation method works effectively with common materials ranging from acrylic (friction coefficient 0.4) to rubber (friction coefficient 0.9).

\subsection{Assumptions and Limitations}
\label{sec:sim_to_real}
The manipulation method presented here demonstrates robust performance across objects with diverse shapes, sizes, weights, and materials. For implementation on the current Ori-pixel platform, we assume objects have a flat contact surface, are larger than 150\,mm to effectively cover more than 2$\times$2 tiles and prevent falling into gaps between modules, and weigh less than 1500\,g due to actuator capabilities. 

As for limitations of the proposed framework, the robust manipulation performance requires friction coefficients above 0.3, though this encompasses most common materials from acrylic to rubber. Additionally, the framework has a resolution limitation requiring objects to span at least 2$\times$2 tiles to maintain consistent manipulation forces.