\section{Kinematic Modeling and Dynamic Simulation}
\label{sec:modeling}
\begin{figure}[t]
    \centering
    \includegraphics[width = \columnwidth]{model.pdf}
    \caption{\textbf{Kinematic model, workspace, and simulation setups. }(a) Kinematic model of the Canfield origami structure; (b) non-monotonic behavior of end-effector's workspace from lower to higher Z-height configurations, first expanding then contracting; (c) single module model for simulation; (d) simulation contact model; (e) 5$\times$5 multi-module model for simulation.
    }
    \label{fig:modeling}
\end{figure}

This section discusses the kinematic modeling and workspace analysis of the Ori-Pixel platform, emphasizing its application in generating manipulation motion patterns. A dynamic model is then developed to simulate the dynamic behavior of the origami robot surface using MuJoCo~\cite{todorov2012mujoco}.

\subsection{Kinematic model and workspace analysis}
\label{sec:kinematic}
This work uses a 5$\times$5 grid of 25 3-DoF Canfield origami robots. Based on prior kinematic analyses~\cite{Canfield1998, 10122040}, the structure (\figureref{fig:modeling}(a)) consists of revolute joints $(B_1, B_2, B_3, R_1, R_2, R_3)$ and ball joints $(M_1, M_2, M_3)$. All linkages $(l_1, l_2, l_3, l_1^{'}, l_2^{'}, l_3^{'})$ are 30\,mm long, with $B_{1,2,3}$ and $R_{1,2,3}$ equidistant (20.21\,mm) from centers $O_B$ and $O_R$ respectively. Actuation angles ${\theta_i}, i\in{1,2,3}$ are determined from the top plate's pose parameters $(\delta, \psi, H)$ through:
\begin{equation}
\label{eq:kinematic}
    \theta_i=2\cdot arctan(t_i),  \theta_i \in[0, \frac{\pi}{2}],
\end{equation}
where:
\begin{equation}
    \begin{split}
        t_i &=\frac{-b_i\pm \sqrt{b_i^2-4a_ic_i}}{2a_i},\\
        a_i &=(r-l)(sin(\frac{\psi}{2})\cdot cos(\delta-\theta_i))-\frac{r_0}{2},\\
        b_i &=2l\cdot cos(\frac{\psi}{2}),\\
        c_i &=(r+l)(sin(\frac{\psi}{2})\cdot cos(\delta-\theta_i))-\frac{r_0}{2},\\
        r_0 &=\frac{H}{sin(\frac{\pi}{2}-\frac{\psi}{2})}.
 \end{split}
\end{equation}

The end-effector's inclination angle $(\psi)$ and height $(H)$ are key parameters constrained by the system's kinematics. As depicted in~\figureref{fig:modeling}(b), analysis of their workspace across three height configurations ($h\in[10, 25]$, $[25, 40]$ and $[40, 55]$ mm) reveals that the workspace first expands from lower to medium heights, then contracts at higher configurations. The workspace shows asymmetry across $\psi$ and $\delta$ ranges, necessitating separate optimizations for each direction of manipulations.


\subsection{Dynamic Simulation}
\label{sec:simulation}

The single-module model derived from the kinematic analysis is then developed for dynamic simulations in MuJoCo with a timestep of $5\times 10^{-4}$\,s using the default semi-implicit Euler integrator. As illustrated in \figureref{fig:modeling}(c), the revolute joints $B_1, B_2, B_3$ are connected to the base of each lower linkage, with their axes offset by $60$ degrees from one another. The ball joints $M_1, M_2, M_3$ link the lower linkages to the upper linkages, while the revolute joints $R_1, R_2, R_3$ connect the upper linkages to the top plate, sharing the same axis orientation as the joints $B_1, B_2, B_3$. A spring-damper model is applied to each joint, with a spring stiffness of $k_p = 0.2$ $N\cdot\text{m}/\text{rad}$ and a damping coefficient of $d = 0.1$ $N\cdot\text{m}\cdot\text{s}/\text{rad}$. The dimensions and masses of the linkages and top plates are derived from the same design parameters used in the prototype, as presented in~\cite{Oripixel}. Three motor actuators are implemented in position control mode with position feedback gain $k_p = 5$ and connected to the joints $B_1, B_2, B_3$. 

The single-module model is then replicated to form the 5$\times$5 module grid surface with identical distributions and dimensions as the prototype design presented in~\cite{Oripixel}. The multi-module MuJoCo model is depicted in~\figureref{fig:modeling}(e). 
This model includes 75 motor actuators, and all top plates feature contact models to simulate interactions with objects using soft contact dynamics with a solver tolerance of $10^{-6}$ and a maximum 30 iterations per timestep. As illustrated in~\figureref{fig:modeling}(d), the contact model incorporates sliding and rolling friction along the X and Y axes, and torsional friction along the Z axis. The sliding friction coefficient, $\mu_{slide}$, and the rolling friction coefficient, $\mu_{roll}$, are calibrated to $0.5$ and $0.01$, respectively. 
