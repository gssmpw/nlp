\section{Introduction}
\label{sec:intro}

\begin{figure}[t]
    \centering
    \includegraphics[width = \columnwidth]{figure1.pdf}
    \caption{\textbf{Conceptual overview.} 
    (a) Conceptual illustration of the proposed 3-DoF CPG-based manipulation framework using a multi-module origami robot surface;
    (b) experiments demonstrating the system's versatility for manipulating various objects: 
    (i) 300$\times$300\,mm acrylic plate, 
    (ii) 200$\times$200\,mm wood plate, 
    (iii) 300$\times$300\,mm acrylic plate with a slender foam cylinder loosely positioned on top,
    (iv) 1000$\times$300\,mm acrylic plate weighing 1\,kg, 
    (v) 400$\times$400\,mm Polo shirt weighing 280\,g, and
    (vi) 250$\times$270\,mm Trilby hat weighing 55\,g.
    }
    \label{fig:concept}
\end{figure}


Robotic manipulation has made significant strides in recent years, leveraging advanced control and planning algorithms to demonstrate a variety of automated and precise tasks. Traditional robotic manipulators, typically employing robotic arms and grippers, have shown remarkable versatility in handling objects of different materials and shapes~\cite{shintake_versatile_2016, shintake_soft_2018, khurana_motion_2024}. When combined with advanced learning-based control strategies~\cite{cui_toward_2021}, these systems can perform intricate tasks such as in-hand manipulations~\cite{abondance_dexterous_2020, andrychowicz_learning_2020, liu_modeling_2020}, teleoperation~\cite{aldaco_aloha_nodate, 10035484}, dynamic stabilization~\cite{9811752}, dynamic throwing~\cite{liu_tube_2024}, and dressing~\cite{zhang_learning_2022}.

Traditional robotic grippers excel in their designed applications but often face scalability challenges when handling objects of varying types and sizes \cite{shintake2018soft}. For instance, ``Gripping by Actuation'' approaches effectively handle convex objects but show limitations with deformable materials \cite{crooks2016fin, crooks2017passive}. While controlled-stiffness grippers \cite{6225373,7086317} and grippers with integrated adhesion \cite{shintake2016,hawkes2015grasping} offer unique advantages for specific object types, achieving versatile manipulation remains an open challenge, particularly for objects at meter-scale and with diverse material properties.

To address these challenges, researchers have explored alternative approaches such as dynamic planar robotic surfaces. These surfaces, often using arrays of 1-DoF pins or more complex mechanisms like delta robots, have shown promise in manipulating various objects~\cite{leithinger_shape_2015, barr_smart_2013, follmer_inform_2013, xue_arraybot_2023, thompson_towards_2021, patil_linear_2023}. However, such systems usually require a significant number of actuators and sophisticated control methods, which limit their applications. Other novel actuators, including soft pneumatic actuators~\cite{deng_novel_2016, robertson_compact_2019}, ciliary actuators~\cite{ataka_design_2009}, and liquid crystal elastomers~\cite{liu_robotic_2021}, have also been investigated to address these limitations. Nevertheless, these approaches have yet to fully overcome the challenges posed by larger objects or flexible materials.

Central Pattern Generators (CPGs) have been extensively studied for generating locomotion in robots~\cite{ijspeert_central_2008}. By producing rhythmic signals, CPG-based controllers have proven effective in simplifying control requirements, thereby reducing actuation complexity in various multi-legged robotic systems, including bipedal robots~\cite{badri-sprowitz_birdbot_2022, 10499824}, quadrupedal robots~\cite{Cohen2003,cheetah2013,10175020}, and swimming robots~\cite{Porez2014,4459741}. Despite their prevalence in robotic locomotion, CPGs have seen limited application in robotic manipulation.

In this letter, we introduce a novel framework for manipulating objects of diverse sizes and stiffness, ranging from centimeters to meters, using the previously developed multi-module origami robotic surface - Ori-Pixel~\cite{Oripixel}. This approach combines a collective CPG-based manipulation motion generator with simulation-based optimizations. As shown in~\figureref{fig:concept}(a), our method utilizes the Canfield parallel origami robot, which offers three degrees of freedom: Z-axis translation and rotation around the X and Y axes. By arranging these robots in a 5$\times$5 multi-module array, we enable versatile manipulations including fast and smooth translations and rotations for objects of varying scales and stiffness.

The key challenge in controlling this platform lies in its high dimensionality, with 75 degrees of freedom (DoF) across the array. While this high-DoF configuration provides exceptional flexibility and precision for complex, localized manipulation tasks, it also presents significant challenges for control synthesis. Traditional control methods struggle with the complex kinematics and actuation coordination, while learning-based approaches face difficulties due to the vast search space, challenges in collecting comprehensive training data, and limited adaptability to hardware modifications. For instance, adding or removing a row of modules would typically necessitate complete model retraining in learning-based methods. To address these challenges, we introduce a CPG-based method that strategically groups the modules and represents end-effector motions using synchronized sinusoidal functions, effectively reducing the control optimization targets from 75 individual actuator positions to only 8 parameters. This reduction dramatically simplifies the optimization space, improving both the search efficiency for optimal control parameters and the system's real-world applicability. Moreover, our CPG-based approach offers inherent flexibility to platform modifications, as the underlying control principles remain valid regardless of the specific module configuration.

The proposed framework employs simulation-based optimization of the CPG parameters to generate effective motion patterns across the robotic surface. Through dynamic simulations and prototype experiments, we demonstrate the framework's capability to translate and rotate objects of varying sizes and materials, from rigid wood and acrylic to flexible fabrics. Fine-tuning objective functions allows our CPG-based controller to operate in two distinct modes - fast manipulation and smooth, stable manipulation - with the latter being particularly beneficial for delicate items sensitive to sudden positional and orientational changes, making it an adaptable solution for a broad spectrum of manipulation tasks.


The contributions of this letter are summarized as follows:
\begin{enumerate}
    \item  A novel CPG-based motion generator is developed for manipulations using multi-module robot surface, enabling various collective manipulation modes for objects of diverse sizes and stiffness.
    \item  A simulation-based optimization framework is then proposed to guide selecting optimal CPG parameters across a range of object settings and manipulation modes.
    \item  Dynamic simulations and prototype experiments are conducted to validate the proposed motion design and optimization framework, demonstrating effective manipulations of objects with varying sizes, shapes, and stiffness.
\end{enumerate}



