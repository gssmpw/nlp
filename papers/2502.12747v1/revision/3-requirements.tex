
\section{Design Considerations}
With \toolkit, we set out to develop an exoskeleton specifically for the arms with up to 3 active degrees of freedom~(DoF) on each arm. We extend beyond existing HCI exoskeleton prototypes, which have largely focused on the hands and elbows to date (e.g.,~\cite{teng_2022,gu_2016,homola_2022, nishida_2022}). By targeting the arms including the shoulder joint, we aim to open up a wider range of applications that involve larger body movements, such as motion guidance in sports, physical rehabilitation, and experiences in VR. 
\toolkit's objective is to facilitate the prototyping of human-exoskeleton interactions for novice roboticists in the early design stages.
Its design considerations are guided by the five goals of HCI toolkit research~\cite{ledo_2018} and informed by literature on wearables~\cite{gemperle_1998, saberpour_2023} and exoskeletons~\cite{souza_2016,sarac_2019}:

\paragraphB{Ease of development}
Programming the interactive behavior of robots is a challenging task for novices~\cite{ajaykumar_2021}. To ease development and empower designers to focus early on on the prototyping of novel interactions, \toolkit~should
\begin{itemize}[leftmargin=*, noitemsep, topsep=3pt]
\item provide \textit{powerful functional abstractions} that offer relevant functionalities for programming versatile interactive behaviors in few lines of code. In line with principles of end-user robot programming (e.g.,~\cite{desantis_2008}), these should abstract away the details of low-level motion control and instead allow the developer to focus on higher-level augmentation strategies,  
\item enable \textit{interactive reactions from sensing to actuation}. As a body-centred technology, it is essential to provide functionalities that sense and interpret body motions in real-time, such as a limb's motion direction, position, or speed. This allows designers to employ these as conditional triggers for specific exoskeleton actuation, creating complex interactive behaviors.
\item should provide an infrastructure that allows to \textit{experience the implemented interactions with the prototype in real-time}, thus enabling designers to iteratively improve the design -- a key principle in interaction design~\cite{preece_2015}.
\end{itemize}

\paragraphB{Fabrication}
In line with design principles discussed in other physical toolkit papers in HCI, \toolkit~should
\begin{itemize}[leftmargin=*, noitemsep, topsep=3pt]
\item use \textit{accessible materials and fabrication processes}. To empower new audiences and  integrate with existing practices~\cite{ledo_2018}, we intend to use technologies widely used in HCI, design and making~(e.g., 3D printing, Arduino) for building and implementing the prototype. This allows the community to use, extend and tailor the toolkit to their needs.
\item consist of \textit{modular components} that facilitate easy assembly of the prototype and rapid exchange of functional components after fabrication. This aligns with the iterative nature of interaction design and prior approaches of robot toolkits in HCI~\cite{cui_2023,cui_2024}.
\item be \textit{customizable to individual body characteristics}, an essential consideration in wearable robotic toolkits~\cite{saberpour_2023}. As different body characteristics can affect an exoskeleton's fit, function and wearability, \toolkit~must offer mechanisms to easily adjust the prototype dimensions to different arm lengths and circumferences. 
\end{itemize}

\paragraphB{Safety}
\toolkit~must involve \textit{safety measures}, which are essential for physical HRI~\cite{desantis_2008, saberpour_2023}. Actuating the human body, we must provide mechanisms that ensure that the applied forces and ranges of motions are safe and comfortable~\cite{souza_2016}, even in case of inexperienced developers.  This also involves, for instance, to accommodate motors with different maximum torques to let designers decide what is a safe range required for their application. 