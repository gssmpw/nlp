\section{Introduction}
Exoskeletons create unique opportunities for novel interactions which assist, modify or constrain physical body movement. For instance, recent work in the HCI community has leveraged hand-based or elbow-based exoskeletons for diverse applications including motion capture~\cite{gu_2016}, haptic feedback in VR~\cite{gu_2016, teng_2022}, or as a means for skill transfer between people~\cite{nishida_2022}.

Despite the great potential of exoskeletons, and despite decades of research in other communities, human-exoskeleton interaction is still a largely underinvestigated field in HCI. 
A key challenge contributing to this gap is the lack of affordable exoskeleton technologies and the lack of easily customizable and easily programmable commercial exoskeletons. The degree of expertise currently required for engineering an exoskeleton and for controlling its motion presents high barriers of entry for novices. This is further aggravated by a certain disciplinary gap in goals between HCI and robotics: while exoskeletons from robotics tend to address specific applications with sophisticated high-end technology~\cite{gull_2020,bogue_2018,kapsalyamov_2020}%\js{is this a safe claim? can you add a few references to back this}
, interaction designers need support for creatively exploring design opportunities in the early design phases. This requires low-fidelity, easily customizable and easy-to-iterate-on prototypes~\cite{preece_2015}.

We address this gap with \toolkit, an \textbf{open-source toolkit for rapid prototyping of low-fidelity yet functional exoskeleton prototypes}. \toolkit~is targeted at researchers, hobbyists, and makers with basic electronics and programming skills (e.g., gained through Arduino-based projects). Expertise with robotics, including their design and control, is not required\footnote{We refer to this target group as \textit{novice roboticists} for the remainder of the paper.}. \toolkit~aims to support them in creatively designing and exploring novel human-exoskeleton interactions in the early design phases. 
\toolkit~is informed by a set of considerations that we derived from the literature on HCI toolkit research, wearables and exoskeletons. 
The key benefits of \toolkit~are: (a)~\textit{ease of development for novice roboticists}, achieved with a DIY approach that uses 3D printed components, off-the-shelf servo motors, and a dedicated Arduino firmware library, and (b)~\textit{customizability in design and behavior for rapid prototyping}, achieved through modularization of the hardware and software required for prototyping human-exoskeleton interactions. The resulting building blocks can be easily exchanged or combined into new functionalities, promoting the exploratory and iterative character of interaction design for a wide range of applications. 

\toolkit's \textbf{hardware components} realize a low-fidelity exoskeleton prototype that can actuate one or two arms, with up to two active degrees-of-freedom (DoF) at the shoulder and one active DoF at the elbow. 
Arms hold promise for novel interactions in versatile application areas, such as physical motion guidance or strength augmentation; however, they present challenges in the mechanical design~\cite{tiseni_2019}. 
To tailor the prototype to the needs of diverse applications, the toolkit offers modular hardware components, which allow for hands-on reconfigurability of the exoskeleton's functionality, enable adjusting the size to accommodate various body sizes, and deploy important safety mechanisms.

\toolkit's \textbf{software library} provides functional abstractions which free the novice roboticist from detailed low-level motion control. We first identified frequently used augmentation strategies from state-of-the-art exoskeleton literature~\cite{proietti_2016,gasperina_2021} and organized them in a two-dimensional conceptual space. 
For each identified augmentation strategy, \toolkit~provides pre-implemented functions that the developer can use off-the-shelf for rapid experimentation and ideation. Amongst others, these comprise functions that interactively amplify or resist user motion, modulate the style of a motion, transfer motion from one exoskeleton to another, or guide the user's motion through real-time haptic feedback. For the ease of development, \toolkit~ provides simplified access to these functions through a command-line interface, a GUI, a Processing library, or directly through the Arduino firmware.
The firmware library enables the novice roboticist to compose the offered functional abstractions into more complex and meaningful interactions.
We make the hardware design and software libraries openly available\footnote{\url{https://github.com/HCI-Lab-Saarland/ExoKit}}.

To confirm the versatility of the toolkit for rapid prototyping, we demonstrate application cases that have been successfully realized with the framework, discuss their iterative design process and lessons learned. 
Second, we present results from two usage studies. In our first study, we collect feedback of \toolkit~in use, and discuss user's opinions on the provided functional abstractions, programmability, and wearability. With our second user study, we provide insights into how users approached \toolkit~to create their own applications, their workflows and encountered challenges.
We conclude by discussing implications for the design of human-exoskeleton interactions, limitations, and future directions.


In summary, the main contribution of this paper is \toolkit, a toolkit for rapid prototyping of human-exoskeleton interaction, targeted at novice roboticists in the early phases of interaction design. We
\begin{itemize}
    \item 
    conceptually identify functional abstractions that encapsulate relevant augmentation strategies, implemented in an Arduino firmware library. A designer can readily access these through a command-line interface, GUI, Processing library, or further customize the implemented strategies directly through the firmware.
    \item modularize the exoskeleton into hardware components that can be easily exchanged or combined into new functionalities, adjusted in size, and used to implement physical safety mechanisms. The resulting prototype can support the user with up to 3 DoF per arm.
    \item  demonstrate \toolkit's versatility and utility through application examples and two usage studies. 
\end{itemize}

We hope that this work will inspire and empower HCI researchers, designers, and makers alike to start engaging with the exciting area of human-exoskeleton interaction and exploring its potential for innovative applications.
