
\begin{figure*}[tb]
    \centering
    \includegraphics[width=0.78\linewidth]{figures/anatomy3.pdf}
    \caption{Degrees of freedom supported by ExoKit and the joints' ranges of motion (RoM) according to~\cite{norkin_2016}. The colored areas indicate the RoMs supported by ExoKit.}
    \label{fig:biomechanics}
    \Description{ExoKit supports shoulder abduction-adduction, shoulder extension-flexion and elbow extension-flexion. According to prior work, an adult's range of motion for shoulder adduction-abduction ranges from 0 to 180 degree, for shoulder extension-flexion from -60 to 180 degree and for elbow extension-flexion from 0 to 150 degree. ExoKit supports shoulder adduction-abduction in a range from 0 to 90 degree, shoulder extension-flexion from -20 to 115 degree and elbow extension-flexion from 0 to 115 degree.}
\end{figure*}

\section{Related Work}
This work is informed by prior work on exoskeletons, robot programming, and toolkit research in HCI.

\subsection{Exoskeletons}
Robots worn on the human body are gaining increasing attention in the HCI community~\cite{inami_2022, muehlhaus_2023,saberpour_2023}.
Important types include wearable robotic limbs~\cite{saraiji_2018,yamamura_2023}, prosthetic limbs~\cite{hofmann_2016,phelan_2021}, robots that roam on the user's body~\cite{dementyev_2016,kao_2017,sathya_2022}, and exoskeletons~\cite{nishida_2020, gu_2016,li_2023}. Actuating the user's body, exoskeletons integrate particularly close with the human body. The resulting embodied partnership opens up a unique potential for designing novel, embodied human-robot interactions~\cite{li_2023}.
However, to date, research on exoskeletons in HCI is rare and how to design for intertwined human–computer integrations is not well understood~\cite{mueller_2023}. One of the few works on exoskeletons within the HCI community is Dexmo, an inexpensive and lightweight hand-based exoskeleton, used for motion capturing and force feedback in VR~\cite{gu_2016}. DigituSync, another prominent example, employs a dual hand-based exoskeleton to share hand movements between two people for skill learning through interconnected links, showcasing the potential of exoskeletons for novel and creative applications in HCI~\cite{nishida_2022}.
Beyond hand-based exoskeletons, only little research investigated the elbow~\cite{teng_2022,homola_2022}. For instance, Homola et al. explored exoskeletons to aid game-based rehabilitation, leveraging a low-fidelity elbow-based exoskeleton built from Lego~\cite{homola_2022}. 

Exoskeletons have been extensively studied in the robotics community, but contrary to the HCI community, their primary focus lies on sophisticated mechanical designs whose objectives are stability, force, the support of complex joint types, and the prevention of misalignments with the human body, along dedicated control strategies~\cite{gull_2020}.
Such exoskeletons are often developed for specific applications~\cite{gull_2020}, such as  industry~\cite{bogue_2018}, military~\cite{proud_2022}, rehabilitation~\cite{proietti_2016}, and daily life assistance~\cite{jung_2018, kapsalyamov_2020}, rather than as a generic design that allows for a broader exploration of interactions and rapid prototypability.

Contrasting the two communities, it is apparent that HCI's requirements for exoskeleton hardware inherently differ from those in robotics research. While the former focuses on user-centered design objectives, requiring rather low-fidelity prototypes that facilitate the exploration of novel human-exoskeleton interactions, the latter primarily excels in the development of sophisticated mechanical and control aspects. 
Attempting the needs of educators and tinkerers, EduExo\footnote{EduExo Kit: \url{https://www.eduexo.com/eduexo-kit/}, last accessed July 3, 2024} provides a valuable platform to teach fundamentals of exoskeleton principles to novices~\cite{tashi_2024,bartenbach_2020}.
However, it intentionally engages users with low-level motion control and provides limited actuated DoFs, and thereof still leaves a barrier to novices interested in human-exoskeleton interaction design from a higher-level perspective. Hence, there is a need for a toolkit that provides a low-fidelity exoskeleton prototype which empowers users to easily and rapidly customize the exoskeleton's design and interactive behavior to their needs, without having to handle the details of mechanical design and low-level motion control. 
We deliberately focused our toolkit on the shoulder and elbow to complement prior work in HCI which has primarily focused on hand-based and, to a lesser extent, on elbow-based exoskeletons~\cite{gu_2016,teng_2022,nishida_2022,li_2023}. 

\paragraph{Biomechanics of the upper limbs}
Considering the biomechanics of the human upper limbs is crucial for the design of arm-based exoskeletons. Key properties that are particularly relevant are the joints’ degrees of freedom (DoFs), range of motion (RoM), and torque settings. 
The shoulder joint possesses three DoFs, enabling abduction-adduction (\autoref{fig:biomechanics}a), flexion-extension (\autoref{fig:biomechanics}b), and medial-lateral rotation. The elbow has two DoFs, allowing for flexion-extension (\autoref{fig:biomechanics}c) and supination-pronation, which is the rotation of the forearm.
The RoM defines the maximum arc through which the joint can move.
Figure~\ref{fig:biomechanics} provides an overview of the DoFs and RoMs supported by ExoKit, which lay within the typical RoMs of an adult's joints~\cite{norkin_2016}.
Another critical factor in the exoskeleton design and selection of suitable actuators is the torque required for the application, as different applications may require different torques; for example, rehabilitation exoskeletons may require less torque than industrial exoskeletons designed to support heavy loads. Furthermore, a user's torque perception is also influenced by various other factors such as movement direction or speed~\cite{kim_2021}. Therefore, the required torque settings need to be carefully determined for the specific use case.

\subsection{Robot Programming}
End-user robot programming focuses on enabling non-experts to control robot operations, broadening their deployment in real-world applications. To simplify robot programming, various approaches exist, such as API-based manual programming~\cite{diprose_2017, angerer_2013}, visual block-based programming~\cite{weintrop_2018}, or programming by demonstration~\cite{ajaykumar_2021}.
One interesting API-based approach is the Robotics API~\cite{angerer_2013}, an object-oriented framework for developing complex robotic applications. It abstracts key robotic actions and sensors as object proxies, which users can sequentialize and parallelize in an event-driven architecture. Such a ``Trigger-Action Programming'' approach has been widely used in HCI, too, enabling non-experts to program devices like smart homes~\cite{ur_2014} and humanoid robots~\cite{leonardi_2019} with relative ease. 
One crucial consideration in programming robots in this way is to identify essential functional abstractions which can be combined according to the tasks the robot should perform~\cite{desantis_2008}. 
Striking a balance between exposing low-level functions for flexibility and providing high-level abstractions for ease of use is critical. While prior work has investigated this in domains like industrial or social robotics~\cite{angerer_2013,leonardi_2019}, no such abstractions have been established for exoskeletons to the best of our knowledge. We argue that identifying a suitable level of functional abstractions for novice roboticists is one crucial aspect to facilitate rapid prototyping of human-exoskeleton interactions, empowering them to focus on high-level augmentation strategies rather than low-level motion control.

\subsection{Robotics Toolkit Research in HCI}
Toolkit research plays an important role in HCI by making new technologies accessible to broader audiences, reducing complexity, encouraging creative exploration, and integrating with existing practices~\cite{ledo_2018}. HCI toolkits typically achieve these goals by encapsulating low-level domain knowledge, lowering the barrier to entry for non-experts~(e.g.,~\cite{lei_2022, arabi_2022,li_2019}). However, despite being established in HCI, toolkits specifically addressing robot prototyping remain limited~\cite{suguitan_2019}. A few notable examples include Blossom~\cite{suguitan_2019}, Phybots~\cite{kato_2012}, and WRLKit~\cite{saberpour_2023}. Suguitan et al. identify three key requirements for their robotics toolkit, two of which are particularly relevant to \toolkit: accessibility for rapid prototyping and flexibility to customize both the robot's design and behavior~\cite{suguitan_2019}. 
Similarly, Phybots is a toolkit targeted at researchers and interaction designers to rapidly prototype with locomotive robots through hardware and a software API~\cite{kato_2012}.
WRLKit enables designers to prototype personalized wearable robotic limbs tailored to specific tasks and body locations through a computational design tool, thereby lowering the entry barrier to wearable robotics for the HCI community~\cite{saberpour_2023}. 
Inspired by these approaches, we address the gap in wearable robotics toolkits in HCI with \toolkit, a do-it-yourself toolkit which lowers the entry barrier for prototyping human-exoskeleton interactions.

