\begin{figure*}[t]
    \centering
    \includegraphics[width=0.8\linewidth]{figures/exo_configurations.pdf}
    \caption{\toolkit's modular components allow to freely configure the degrees-of-freedom (DoF). Some example configurations include: (a)~6 DoFs at both arms, (b)~3 DoFs at one arm, (c)~1 DoF at one elbow.}
    \label{fig:customizable_dof}
    \Description{Examples of ExoKit’s adjustable degrees-of-freedom. The image features three sub-images of a person wearin ExoKit configured with varying degrees-of-freedom. The caption provides additional details. In configurations that involve the shoulder, the person also wears a posture corrector, which is used to secure the exoskeleton to the body.}
\end{figure*}
\section{\toolkit~Hardware Components}\label{sec:hardware}
A key challenge in designing hardware components for an exoskeleton toolkit lies in balancing functional requirements, such as required degrees of freedom, stability, and safety, with designer's needs for rapid prototyping. Particularly in the early stages of interaction design, an important key characteristic for design is iteration~\cite{preece_2015}, in which a designer usually explores a wide range of possibilities before converging on the most promising solutions. 
While modularity, hands-on reconfigurability, and prototypability are well-established principles in HCI robotic toolkits (e.g.,~\cite{cui_2023,cui_2024}), these considerations are less prevalent in exoskeleton research so far. Traditionally, research on exoskeletons prioritizes functional requirements for high-fidelity solutions for specific applications, driven by the unique demands of each use case (e.g.,~\cite{sarac_2019}). 
We therefore note a disciplinary gap between the iterative and exploratory prototyping approach required in HCI and the function-focused demands of professional, specialised exoskeletons.

Building on the design considerations from Section 3, the design of \toolkit~'s hardware components is centered on three aspects that integrate exoskeleton-specific requirements~\cite{souza_2016, sarac_2019} with HCI principles: (1)~Modular components that facilitate hands-on reconfigurability of the exoskeleton's functionality, involving their DoFs, sensing and actuation capabilities, (2)~alignment strategies that facilitate adjusting the exoskeleton's fitting for wearability and rapid prototyping, and (3)~safety mechanisms suited for inexperienced developers.

All components offered by \toolkit~can be fabricated using commercially available 3D printers and ABS filament, addressing the trade-off between stability and accessible fabrication. The toolkit further relies on widely-used off-the-shelf components, notably Arduino and Dynamixel motors.
The current \toolkit~prototype supports all motors of the Dynamixel XM series, providing forces up to $10~Nm$ with the strongest motor, and a maximum speed of $77$ RPM with the fastest. The motors can be powered from 12 Volts using the supplied Dynamixel power supply or a wearable power bank. 
For torque sensing, the toolkit achieves rates of $80~Hz$. 
\toolkit~allows to configure exoskeletons with up to 6 active degrees of freedom -- 3 per arm. The supported ranges of motion are $0$ to $115$ degrees for elbow flexion-extension, $-20$ to $115$ degrees for shoulder flexion-extension, and $0$ to $90$ degrees for shoulder abduction-adduction, comparable to existing exoskeleton designs (cf.,~\cite{rahman_2015,bilancia_2021,kiguchi_2008}). More implementation details can be found in Appendix~\ref{subsec:implementation_hw}.

In the following, we detail on the individual hardware components offered in \toolkit~and their roles in enabling reconfigurable, size-adjustable and safe exoskeletons:

\subsection{Modular Components To (Re)Configure the Exoskeleton's Functionality}\label{subsec:reconfigurability}

In early stages of prototyping, designers might not yet have a clear idea of the exoskeleton's final functional requirements. Thus, it is essential to create exoskeletons in HCI that support iterative development by empowering designers to add, remove, and modify components as requirements and the understanding of the problem space evolve.
To address this, we propose a principled, modular architecture that is based on three key considerations:
(1)~In-line with modular exoskeleton designs~\cite{souza_2016}, the exoskeleton should allow designers to adjust the DoFs to suit different application scenarios. (2)~As exoskeletons serve a range of purposes---from motion capture~\cite{gu_2016} to delivering haptic feedback~\cite{teng_2022}, or applying stronger actuation forces~\cite{exoJacketIndustrial}---designers must be able to swap between sensing and actuating modules of varying strengths to match specific task requirements. (3)~Inspired by exoskeletons where sensors extend beyond position sensing in joints (e.g., load cells~\cite{gull_2020}), the architecture must be able to accommodate additional sensing elements, ensuring that the exoskeleton can be adapted to changing requirements. We now discuss each consideration:


\paragraphB{Customizable degrees-of-freedom}
Different applications require varying DoFs in exoskeletons.
The challenge for an exoskeleton toolkit with customizable DoFs lies in providing the necessary flexibility by freely configuring the DoFs after fabrication, without adding mechanical complexity, more components or reducing user comfort. Moreover, the provided components, even for more complex joints such as the shoulders, should be fabricable using accessible fabrication methods while providing a sufficient degree of stability.
The toolkit extends beyond current exoskeleton approaches in HCI by offering support for the elbow and the more complex shoulder joints. It supports shoulder flexion/extension and abduction/adduction through two DoFs which can be actuated and a passive DoF that enables the exoskeleton to adapt to complex movements of the shoulder joint. 
In addition, the elbow is supported with one DoF in flexion-extension. 
The designer can easily adjust the hardware prototype by detaching joints to remove DoFs not needed for a specific application. For instance, designers can construct exoskeletons that support both arms, only one arm or the elbow (see Figure~\ref{fig:customizable_dof}).

\begin{figure*}[tb]
    \centering
    \includegraphics[width=1\linewidth]{figures/joint_modules.pdf}
    \caption{\toolkit~features interchangeable joints based on a versatile joint base (a). This base can be fitted with (b)~a motor for actuated joints, (c)~a passive module to replace the motor, or (d)~a passive module with an integrated position encoder for sensing-only joints. (e)~Designers can insert mechanical restrictions to the base joint to further limit the offered range of motion by 15, 30 or 45° on each side.}
    \label{fig:joint_modules}
    \Description{The figure illustrates the different joint types employed in ExoKit, as detailed in the caption and accompanying subsection.}
\end{figure*}

\paragraphB{Modular joints}
As exoskeletons can be either actuated, act as a sensing device, or be passive, it is crucial for designers to have the flexibility to choose the functionality for their application themselves. To enable hands-on reconfigurability, \toolkit~offers three types of interchangeable joint modules: (1)~\textit{Actuated joints} comprise an actuator. A joint base, depicted in Figure~\ref{fig:joint_modules}a, serves for mounting a motor (Figure~\ref{fig:joint_modules}b). The base is designed to be compatible with all motors of the Dynamixel XM series. These motors inherently offer sensing capabilities (position, velocity). If desired, developers can modify the design of the joint base to include other motors with different dimensions. 
(2)~\textit{Sensing-only joints} measure the exoskeleton angle configuration, through which further motion properties, including velocity and acceleration, can be inferred. (3)~\textit{Passive joints} simply follow the user's motions, realizing a passive DoF. 
To realize sensing-only and passive joints, \toolkit~provides a module that can be mounted on top of the joint base instead of a motor~(Figure~\ref{fig:joint_modules}c). 
This module offers a space for an optional position encoder, hence can be repurposed as a sensing-only joint when the position encoder is attached~(Figure~\ref{fig:joint_modules}d). 

\paragraphB{Modular links}
Exoskeleton links are structural components that connect and transmit forces between joints. In \toolkit, we provide two types of 3D-printable passive links. The first is a simple rigid rod, used to stably connect the shoulder and elbow joints. The second is a flexible link made of 3d printed chain segments informed by the mechanism presented by Tiseni et al.~\cite{tiseni_2019}. The advantage of this approach is that the mechanism remains rigid in the direction where torque is applied while complying with user's complex body movements in other directions. In \toolkit, we employ this mechanism as a flexible link between the two shoulder joints, which helps to better comply with the shoulder's additional degrees of freedom, and as a link that connects the elbow joint with the lower arm cuff. 
If desired, both the rigid and chain links can be equipped with an additional load cell that captures the torques applied to the joints. As this is a commonly used control input in exoskeletons~\cite{gull_2020}, this empowers designers to quickly integrate the load cell into their design without having to manually redesign the hardware. The Arduino firmware provides complementary support, allowing users to register the load cell as an exoskeleton component, use the sensing function to stream load cell measures to external applications and deploy them as runtime triggers in the event-based architecture. 
The link modules are depicted in Figure~\ref{fig:loadcell}a and~\ref{fig:loadcell}b.

\begin{figure}[tb]
    \centering
    \includegraphics[width=0.96 \linewidth]{figures/loadcell.pdf}
    \caption{(a)~Rigid and (b)~flexible links without and with integrated load cells, respectively.}
    \label{fig:loadcell}
    \Description{The figure illustrates the rigid and flexible links once without and once with a load cell integrated, as detailed in the caption and accompanying subsections.}
\end{figure}


\subsection{Modular Components to Support Size-Adjustability}
Adjusting the exoskeleton to the user's body size is essential to prevent misalignments that could cause discomfort or hinder proper functionality through undesired torques~\cite{gull_2020}. There are two promising approaches to achieve better joint alignments: fabricating a personalized exoskeleton or integrating size-adjustable mechanisms post-fabrication~\cite{sarac_2019}. \toolkit~follows the latter, enabling hands-on customization which avoid the need to reprint new parts for every user, thereby facilitating rapid prototyping in collaborative settings. All of \toolkit's links are designed for size adjustability. In line with strategies presented in exoskeleton literature~\cite{bartenbach_2016,zhang_2023}, the rigid links on the upper arm and back use adjustable rail mechanisms to accommodate a range of body sizes within the 95th percentile~\cite{gordon_1988}~(see Figure~\ref{fig:size-adjustability}a). The chain link introduced in Subsection~\ref{subsec:reconfigurability} can be adjusted to different shoulder circumferences and lower arm lengths by simply attaching or detaching chain segments~(see Figure~\ref{fig:size-adjustability}b). Finally, the arm cuffs which connect the links to the human arm come in two different sizes, and can be stacked together to provide a larger area that holds the exoskeleton in place.

\begin{figure}[tb]
    \centering
    \includegraphics[width=0.92\linewidth]{figures/size-adjustability.pdf}
    \caption{(a)~The rigid joint can be adjusted in length through a rail system which allows to move the 3d printed rods further in and out. (b)~The flexible link can be easily adjusted in size by adding or removing chain segments.}%sensing_components
    \label{fig:size-adjustability}
    \Description{The figure illustrates the size-adjustability of the rigid and flexible links, as detailed in the caption and accompanying subsections.}
\end{figure}

\subsection{Physical Safety}
Safety is an important concern in exoskeletons, particularly if large forces are involved. 
Based on considerations in robotics literature~(e.g.,~\cite{souza_2016,zacharaki_2020}), \toolkit~comprises three layers of safety, which cover mechanical, software, and user-centered safety measures, specifically targeted at novice developers:

The first layer presents mechanical safety measures to maintain a safe range of motion, an important safety goal in exoskeletons~\cite{souza_2016}. Designers can add 3d printed mechanical restrictions to the base joints~(see Figure~\ref{fig:joint_modules}e). Through a simple plug-and-play mechanism, these mechanical restrictions physically prevent the motor from exceeding the defined range of motion. They are offered in 3 different sizes~(restricting by 15, 30 or 45° on both sides) and can be added, removed, or exchanged after fabrication as needed. 

The second layer realizes safety mechanisms in software, another essential building block of user safety in exoskeletons~\cite{souza_2016}. \toolkit~implements a routine continuously monitoring whether the user's configured range of motion is exceeded and if so terminates and shuts down the system. 

The third layer comprises user-initiated safety measures, to empower the user to actively stop the exoskeleton if required. \toolkit~incorporates a panic button, a common feature in human-robot interaction~\cite{zacharaki_2020}, which the user can either hold in the hands or attach somewhere on their body. Analogously to the prior layer, if triggered, the system immediately terminates the program and shuts down. This can be a particularly helpful measure to ensure safety during initial development or debugging of some new functionality. 
