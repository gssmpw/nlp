\subsection{Discussion, Limitations \& Future Work}
We now discuss lessons learnt from the implementation of the application cases along with insights from the usage studies. We also discuss limitations of our current implementation and derive implications for future work on exoskeleton toolkits in HCI. 

\paragraphB{Exploring human-exoskeleton interactions}
Our example applications and usage studies demonstrate how \toolkit~facilitates exoskeleton prototyping and design iteration for non-experts, and hence lowers the entry barrier for engaging with exoskeletons. While prototyping the application cases, we intensely leveraged the modular design of the exoskeleton, which allowed us to easily reconfigure prototypes by swapping electronics and functional parts. As demonstrated in the usage studies, the command-line interface and the GUI enabled rapidly exploring functions and iterating on their parameters. It became apparent that the developers needed to fine-tune the parameters of real-time modifications, which modify ongoing user motions, such that the combined motion of both user and exoskeleton felt smooth. Here, too, the provided interfaces proved helpful for rapid iterations. We also noted that the screws required for assembly slowed down prototyping. While other toolkits such as MiuraKit~\cite{cui_2023} 
deploy plug-and-play mechanisms which allow for rapid reconfigurability of the robotic structure, they must be carefully designed in case of \toolkit~to withstand the mechanical stresses on the exoskeleton, a point for future refinement.
Furthermore, \toolkit~opened up new possibilities for exploring human-exoskeleton interactions, beyond the initially anticipated use cases. Both the study participants and the authors came up with novel ideas for interaction designs, such as unconventional body remapping via motion transfer functions. This underlines \toolkit's potential as a platform for creative exploration for the HCI community. 

\paragraphB{Ease of programming interactive behavior}
Our usage studies showed that all users could rapidly grasp and modify the behavior of the augmentation strategies through the command-line interface and GUI.
Our example applications further demonstrate more complex use cases which were implemented with the Arduino firmware.
While the provided functional abstractions present a first essential step to making complex exoskeleton behaviors accessible to novice roboticists, we identified the following functionality to be desirable in future extensions: first, the possibility to control exoskeleton motion in 3D world coordinates, rather than only in joint angle space; second, adding more sophisticated force profiles for motion guidance~\cite{proietti_2016,gasperina_2021}, and third, a record-and-replay mechanism for asynchronous motion transfer that captures motions and replays them at a later point~\cite{proietti_2016,gasperina_2021}. 
We further found that the perceived ease of use of the provided programming interfaces depends on to the user's programming experiences. Drawing on these insights, an additional worthwhile direction to further ease programmability and make the toolkit accessible for even broader audiences is to extend beyond text-based programming to novice-friendly visual programming, such as block-based approaches that could be used instead of the Arduino library.
Finally, a simulation tool could provide further support to the designer, for simulating and testing interactions before deploying them on an exoskeleton. Yet, it is an open question how to most effectively simulate and communicate the effects of body-augmenting systems on the human body.

\paragraphB{Extending support to other body parts}
\toolkit~currently supports up to three active DoFs for each arm, opening up new opportunities for interaction design in the HCI community. However, we are far from supporting the whole body. Extending support to other body parts could unlock new interaction opportunities. For instance, future iterations could add more DoF at the arms, by including internal rotation. Another promising avenues is to extend support to the lower limbs, including the hip and knees. We expect that our fundamental design decisions~(modular components, exchangeable joint types, etc.) will generalize to the lower limbs. However, these pose additional safety challenges, as stronger actuation forces are required and if not done right, these could negatively affect the user's balance. In consequence, future work should also expand the conceptual space to address new augmentation strategies for the lower body, such as gait assistance for leg exoskeletons~\cite{baud_2021}. 


\paragraphB{Fabrication \& Cost}
The cost of realizing an exoskeleton with \toolkit~centrally depends on the cost of the motors (between 270 to 430 USD). In comparison, the cost of the fabrication materials is negligible. Relying on 3d printing filament is one factor that makes this technology accessible to a larger audience. While the resulting prototype is sufficient for practical use in early iterations, of note, it also inevitably involves compromises when compared to professional exoskeletons, which are often made from metal structures suited for bearing heavy loads.
Finally, while the overall cost is only a fraction of the price tag of a professional exoskeleton, this may be still too expensive for private users. It is a worthwhile direction to investigate how lower-end, lower-cost motors can be integrated with \toolkit. We invite the community to contribute to the advancement of \toolkit, adapting existing modules and sharing new ones. 
