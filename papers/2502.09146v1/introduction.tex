The growing complexity of software systems demands tools that allow for efficient and intuitive modeling~\cite{abrahao2017user}. MDE has emerged as a powerful paradigm to facilitate the transition from abstract models to concrete implementations \cite{3103551,10350721}. However, many existing modeling tools inadvertently add complexity~\cite{brooks1987essence}, especially in educational settings where the primary objective is to simplify learning and make modeling more accessible to students and newcomers \cite{Liebel2017}.

Modeling tools play a critical role in software development, offering designers and engineers a means to explore design spaces and communicate complex systems effectively to stakeholders~\cite{Kienzle2024}. Despite their importance, academic tools often fail to meet expectations, lacking maturity, robustness, and usability. Students are frequently confronted with issues such as cumbersome installation, configuration, and maintenance processes, which shift their focus from understanding modeling concepts to overcoming technical obstacles. In contrast, while industrial tools tend to be more polished and feature rich, they are often prohibitively expensive, difficult to learn, and misaligned with the specific needs of educational settings~\cite{Kienzle2024}.  

These challenges highlight the need for tools that simplify the modeling process without compromising capability. An ideal tool for education should be intuitive, easy to use and platform independent, while supporting various paradigms and DSLs~\cite{Kienzle2024}. Furthermore, modern tools must showcase the practical benefits of MDE by allowing students to create models that drive downstream development activities, rather than serving as static diagrams~\cite{Stahl2006}. 

\smallskip
%This article introduces \jjodel{}~\cite{di2021enhancing,di2023jjodel}\footnote{The \jjodel{} tool available online at \url{https://www.jjodel.io}. The source code can be accessed on GitHub at \url{https://github.com/MDEGroup/jjodel}. Furthermore, a set of video tutorials demonstrating various features and use cases of \jjodel{} are available at \url{https://www.jjodel.io/video-tutorials/}.}, a cloud-based reflective platform designed to address these challenges. Through a practical example, we demonstrate how \jjodel{} streamlines the modeling process by offering a low-code environment that balances simplicity and advanced functionality, enabling diverse modeling tasks while prioritizing accessibility and usability.  
%
%By emphasizing an intuitive and user-friendly approach, \jjodel{} removes technical barriers making it accessible to users of varying expertise while integrating advanced features such as syntax viewpoints, real-time validation, and collaborative modeling.Addressing these demands, \jjodel{} offers a low-code, web-based environment designed to remove technical barriers and streamline the modeling experience. Unlike many academic tools, \jjodel{} eliminates installation and configuration challenges, allowing students and educators to focus on modeling concepts and their practical applications. Features such as real-time collaboration, automated feedback, and support for multiple modeling paradigms enable \jjodel{} to bridge the gap between academic simplicity and the robustness required for professional software development. By making modeling more accessible and empowering, \jjodel{} provides an intuitive platform that enhances learning while illustrating the real-world value of MDE.  

%%
This article introduces \jjodel{}~\cite{di2021enhancing,di2023jjodel}\footnote{The \jjodel{} tool available online at \url{https://www.jjodel.io}. The source code can be accessed on GitHub at \url{https://github.com/MDEGroup/jjodel}. Furthermore, a set of video tutorials demonstrating various features and use cases of \jjodel{} are available at \url{https://www.jjodel.io/video-tutorials/}.}, a cloud-based reflective platform designed to address these challenges. Through a practical example, we demonstrate how \jjodel{} streamlines the modeling process by offering a low-code environment that balances simplicity and advanced functionality, enabling diverse modeling tasks while prioritizing accessibility and usability.  
%
Building on its emphasis on \textit{transparency in tools}\footnote{The idea of transparency in tools has been developed and explored by several scholars, particularly in the domains of philosophy, technology studies, and human-computer interaction (HCI). Although no single individual is solely credited with the concept, a key contribution has come from Martin Heidegger with his work \textit{Being and Time} (1927)~\cite{heidegger1927being}.}, \jjodel{} seeks to eliminate technical barriers that often disrupt the modeling process, ensuring that modelers can remain focused on their objectives rather than struggling with tool operation. By prioritizing accessibility for users of varying level of expertise, \jjodel{} integrates advanced features such as syntax viewpoints, real-time validation, and collaborative modeling, offering a seamless and efficient environment for diverse modeling tasks. 

Unlike many academic tools, \jjodel{} removes the challenges of installation and configuration, allowing modelers to focus on modeling concepts and their practical applications. Its features, including real-time collaboration, automated feedback, and support for multiple modeling paradigms, bridge the gap between the simplicity required for modeling and the robustness demanded by software development. By making modeling more accessible and empowering, \jjodel{} provides an intuitive platform that not only enhances learning, but also demonstrates the real-world value of MDE.
%%

In this article, we demonstrate how \jjodel{} transforms the language engineering process. Its innovative approach simplifies the learning curve, offering students an opportunity to focus on mastering MDE principles and applying them effectively in diverse scenarios. Through its integration of accessibility, functionality, and educational support, \jjodel{} emerges as a compelling solution to the challenges of teaching modeling in modern educational environments. 


\subsection{Objectives and Structure the Article}
%This article introduces \jjodel{}, a cloud-based reflective platform designed to address the challenges of Model-Driven Engineering (MDE) by integrating accessibility, advanced functionality, and pedagogical support. By reducing cognitive complexity and enabling real-time collaboration, automated co-evolution, and modular syntax viewpoints, \jjodel{} bridges the gap between theoretical research and practical applications in MDE.

This article introduces \jjodel{}, a cloud-based platform that tackles the challenges of MDE by combining accessibility, advanced features, and educational support. By simplifying complexity and enabling real-time collaboration, automated co-evolution, and modular syntax viewpoints, \jjodel{} connects more theoretical research with practical MDE applications.

%The article begins with an in-depth overview of \jjodel{}’s architecture and key features (Section \ref{sec:overview}), detailing its front-end and back-end components, object model, and dynamic viewpoints that collectively enhance the modeling process. To demonstrate its practical utility, a case study of an algebraic expression language is presented (Section \ref{sec:practice}), illustrating \jjodel{}’s capabilities in managing abstract and concrete syntax, validation, and event-driven workflows. A comparative analysis with existing tools (Section \ref{sec:rw}), highlights \jjodel{}'s unique strengths, particularly its support for dynamic collaboration and seamless co-evolution.

The discussion opens with an in-depth overview of \jjodel{}’s architecture and key features (Section \ref{sec:overview}), detailing its front-end and back-end components, object model, and dynamic viewpoints that collectively enhance the modeling process. To demonstrate its practical utility, a case study of an algebraic expression language is presented (Section \ref{sec:practice}), illustrating \jjodel{}’s capabilities in managing abstract and concrete syntax, validation, and event-driven workflows. 
The article reflects on the lessons learned during the development of \jjodel{}, emphasizing its broader implications for the MDE community (Section \ref{sec:lessons}).
%A analysis with existing tools (Section \ref{sec:rw}) highlights the unique strengths of \jjodel{}, particularly its support for dynamic collaboration and seamless co-evolution.
Section \ref{sec:rw} reviews related tools and highlights how \jjodel{} addresses specific challenges in MDE. Conclusions outline future directions (Section \ref{sec:conclusion}), with a focus on strategies to improve scalability, integration, and community participation. Furthermore, it highlights the opportunity to showcase a demonstrator, further solidifying \jjodel{}’s position as a versatile and impactful tool for addressing diverse MDE challenges.