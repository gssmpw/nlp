The design and implementation of \jjodel{} provided invaluable insights into the challenges and opportunities of creating a modern and user-friendly modeling platform. These lessons, which span architectural decisions, usability considerations, and technology integration, offer a foundation for advancing the field of MDE tools while reflecting on the successes and setbacks encountered during platform development.

\subsection{The Impact of Technology Choices}

The selection of foundational technologies such as Node.js and React played a key role in shaping the architecture, scalability, and usability of \jjodel{}. Node.js introduced a lightweight event-driven back-end optimized for real-time interactions, providing the performance and responsiveness necessary for a collaborative modeling platform. React’s component-based architecture (not to be confused with that of the EMF ecosystem)  enabled modularity and reusability, allowing dynamic syntax customization and intuitive user interface design. However, adopting these frameworks was not straightforward; it required a significant paradigm change in application development and iterative testing to align their architectural models with the goals of \jjodel{}.

The broader evolution of the full-stack web ecosystem underscores the growing influence of these technologies. Key advancements such as Node.js (2009), Angular.js (2010), React (2013), and Vue.js (2014) have transformed the way modern applications are designed and developed. By 2024, Node.js powers over 40\% of web applications, cementing its role as a preferred backend technology in low-code platforms. React, used by 39.5\% of front-end developers, integrates seamlessly with advanced interfaces, making it a leading choice for creating responsive, component-based UIs\footnote{\url{https://www.statista.com/statistics/1124699/worldwide-developer-survey-most-used-frameworks-web/?utm_source=chatgpt.com}, last accessed on January 9, 2025}. These frameworks highlight the stark contrast in user experience between traditional modeling tools (all first released before), often emphasizing technical rigor, and modern low-code development platforms (LCDPs)~\cite{sahay2020supporting,prinz2021low}, which prioritize usability, real-time collaboration, and seamless adaptability.

For \jjodel{}, the decision to adopt Node.js and React was driven by the need to balance accessibility, flexibility, and \textit{richness}—qualities often absent in traditional tools. Node.js provided the foundation for real-time interactions, while React’s modular architecture empowered the platform to deliver an intuitive and customizable user experience. Together, these technologies helped \jjodel{} align its design with the broader goals of scalability and usability, ensuring that the platform could meet the diverse needs of its users. This process underscored the importance of carefully integrating technological frameworks to shape not only the technical implementation but also the overall user experience.

Crucially, these insights were not evident at the beginning. They emerged through a process of trial and error, with one particularly challenging iteration spanning two years of development. This iterative approach, guided by the expertise and advocacy of one of the authors, ultimately enabled the team to make informed decisions and implement a modern, scalable architecture for \jjodel{}.

In summary, the development of \jjodel{} reinforced a critical realization: technology is not neutral. It defines both the possibilities and constraints of a platform, influencing its ability to innovate and its inherent limitations. 
%These lessons highlight the pivotal role of thoughtful technology choices in ensuring that platforms align with their goals and fulfill their potential.

\subsection{Balancing Flexibility and Simplicity}
Balancing the flexibility of the platform with its usability proved to be one of the most significant challenges in the development of \jjodel{}. Many features were essential to meeting the requirements we add in mind~\cite{di2023jjodel}, but their integration risked introducing unnecessary complexity. To address this, \jjodel{} adopted a progressive disclosure approach, where essential features were initially exposed and advanced functionalities could be revealed later. Like this, both novice and experienced users could interact with the platform more effectively.

The decision to prioritize simplicity without compromising flexibility was further informed by user feedback. Iterative testing demonstrated that users, particularly those new to MDE tools, found overwhelming interfaces to be a barrier to productivity. By simplifying initial interactions and allowing users to explore more advanced features, \jjodel{} demonstrated that even complex modeling platforms could cater to diverse audiences while retaining powerful capabilities. 

\subsection{Avoiding Accidental Complexity}

Reducing accidental complexity~\cite{atkinson2008reducing} was a cornerstone of the \jjodel{} project. Traditional MDE tools often require complex installation procedures and specialized technical expertise, which hinder accessibility for students and domain experts. By prioritizing a client-side intelligence model, \jjodel{} eliminated many of these barriers, enabling users to engage with the platform without requiring advanced setup or configuration.

This decision significantly improved the accessibility and usability of the platform, aligning with its mission to democratize modeling practices. In addition, the streamlined architecture and processes reduced the cognitive load for users, allowing them to focus on their tasks rather than the intricacies of the tool itself. This approach highlights the importance of designing platforms that prioritize usability without sacrificing flexibility or power.

\subsection{Rethinking Concrete Syntax}

The integration of React’s component-based architecture into \jjodel{} fundamentally transformed how concrete syntax was conceptualized and rendered. Unlike traditional modeling tools, which often rely on static and rigid representations, the proposed approach allowed dynamic customization and real-time interaction. This flexibility enabled users to define syntax viewpoints tailored to their specific needs, fostering a more intuitive modeling experience.

This innovation also demonstrated the potential of modern web technologies to drive advancements in traditionally static domains. Using React's state management and rendering capabilities, \jjodel{} bridged the gap between abstract modeling concepts and practical applications, providing users with a seamless and interactive experience.

\subsection{Iterative Design and Feedback}

The iterative development process~\cite{4668134} was central to the success of \jjodel{}. Regular testing with novice and experienced modelers revealed critical gaps in usability and functionality, informing refinements in interface design and platform features. Early feedback emphasized the importance of real-time feedback mechanisms and context-aware interfaces, leading to the implementation of progressive disclosure and other user-centric design principles.

This approach ensured that \jjodel{} evolved as a responsive and adaptive platform, capable of meeting the diverse needs of its user base. The iterative process also reinforced the value of engaging with stakeholders throughout development, ensuring that the platform remained aligned with user expectations and requirements.

\subsection{Navigating Trade-offs}

Departing from established standards, such as Eclipse GLSP~\cite{bork2023vision}, presented significant challenges and unique opportunities for \jjodel{}. This decision enabled the platform to align more closely with its meta-metamodel, a strict extension of Ecore, and to prioritize client-side intelligence. IAlthough it resulted in a less standardized architecture, the trade-off proved advantageous, making the platform more reactive and agile while eliminating cumbersome integration processes often associated with traditional frameworks.

This experience highlights the critical importance of carefully evaluating trade-offs in platform design. By prioritizing flexibility and usability over strict adherence to established frameworks, we took a calculated risk that ultimately paid off. The resulting architecture demonstrated that divergence from canonical approaches, when guided by clear objectives, can lead to innovative and effective solutions aligned with the vision and goals of the platform.

\subsection{Exploring Alternatives to OCL.js}
Our experience with OCL.js\footnote{\url{https://ocl.stekoe.de/}}, a JavaScript-based implementation of the Object Constraint Language (OCL), provided both opportunities and challenges during the development of \jjodel{}. Despite OCL.js offered full integration with \jjodel{} and initially appeared to be a natural fit, its limitations became apparent over time. The lack of active development of the project and its partial coverage of the OCL standard forced us to explore alternatives to define constraints and predicates within the platform.

The most natural alternative was to adopt the declarative and functional expressions of JSX, which \jjodel{} already uses to template. This approach allowed us to leverage the same notation for both templating and predicates, creating a consistent and simplified user experience. However, it also introduced new uncertainties. To the best of our knowledge, JSX has not been formally characterized and no direct comparison between JSX and OCL has been documented in the literature. This absence of formalization leaves open questions about the expressive power, correctness, and potential limitations of JSX as a substitute for OCL.

This experience underscores the importance of evaluating the trade-offs between adopting emerging notation like JSX and relying on established but limited standards like OCL.js. Although JSX has served \jjodel{} well in the short term, further research and validation is necessary to determine its suitability as a foundational element to define constraints and predicates.

\subsection{The Value of Learning Through Failure}

Some of the most valuable lessons emerged from the failures encountered during development. Iterative attempts to align the architectural design with the objectives of the platform revealed the complexity of creating a scalable, reactive system~\cite{lyytinen1999learning}. In one instance, a two-year iteration led to significant revisions, demonstrating the importance of persistence and openness in addressing setbacks.

This experience reinforced the value of accepting failure as an opportunity for growth. Learning from missteps, \jjodel{} was able to refine its approach and develop a platform that is both innovative and user-friendly. The development of \jjodel{} underscored the importance of balancing technical innovation with user-centric design. The lessons learned highlight the need for thoughtful technology choices, iterative development, and a willingness to embrace trade-offs and failures as opportunities for growth. These insights provide a foundation for advancing the field of MDE tools, ensuring that future platforms are both powerful and accessible.