In this article, we present \jjodel{}, a cloud-based reflective modeling platform designed to address the challenges of cognitive complexity and usability of model-driven engineering (MDE). By emphasizing a low-code, user-friendly approach and integrating advanced features such as syntax viewpoints, real-time validation, and collaborative modeling, \jjodel{} bridges the gap between conceptual research and practical application in MDE. Through a detailed case study and a comprehensive analysis of its architecture and functionalities, we demonstrate how \jjodel{} empowers educators, professionals, and researchers to efficiently create, refine, and adapt domain-specific languages and models.

A core objective of \jjodel{} is to simplify the teaching of MDE concepts by providing an interactive and intuitive environment. Instructors often face significant challenges in adopting existing tools, which are frequently based on outdated technology stacks, and struggle to align them with modern teaching practices. By integrating familiar front-end technologies, such as JSX templates, \jjodel{} lowers the entry barrier, making it more accessible to students and allowing MDE to be effectively introduced at the undergraduate level. In addition, the platform supports for agile workflows and its ability to adapt to dynamic requirements make it an excellent choice for developing robust and scalable modeling solutions in industrial contexts.

Key innovations, including progressive disclosure, modular syntax viewpoints, and real-time collaboration, establish \jjodel{} as a new benchmark for MDE tools. These features reduce cognitive barriers and improve flexibility and scalability, making \jjodel{} a versatile tool for a wide range of use cases.

The development of \jjodel{} provided critical insight into the trade-offs and challenges inherent in the building of modern MDE tools. The implementation of advanced features, such as real-time collaboration and undo/redo mechanisms, required sophisticated state management and synchronization protocols. Furthermore, the adoption of React’s unidirectional data flow architecture enabled powerful functionalities like semantic zooming and dynamic syntax customization, but also presented scalability challenges when applied to large-scale models. These experiences also identified the importance of interdisciplinary approaches, such as taking advantage of insights from related fields like online gaming, to tackle complex design challenges.

Looking ahead, \jjodel{}’s development will focus on addressing scalability for industry-scale models by adopting advanced optimization techniques from the broader communities of MDE and distributed systems. Key areas for future work include the following challenges:
% \begin{itemize}
%     \item improving rendering and synchronization mechanisms to efficiently support larger and more complex models;
%     \item expanding automation for metamodel and model co-evolution to handle a wider range of use cases and reduce manual intervention;
%     \item expanding automation for metamodel and model co-evolution to handle a wider range of use cases and reduce manual intervention;
%     \item developing more sophisticated conflict resolution mechanisms and version control systems for distributed teams;
%     \item exploring integration with existing MDE platforms and tools, such as the Epsilon Playground, to enable functionalities like executing model-to-model transformations directly within \jjodel{}’s environment;
%     \item integrating with existing MDE platforms and tools, such as the Epsilon Playground\footnote{https://eclipse.dev/epsilon/playground/}, to enable functionalities like executing model-to-model transformations directly within \jjodel{}’s environment;
%     \item consolidating the current experiments with LLM agents;
%     \item providing quality learning materials, including a textbook, presentations, and online tutorials; finally
%     \item growing the \jjodel{} community by releasing open-source components, fostering contributions, and establishing an organization capable of securing funding to sustain the project over the long term. 
% \end{itemize}
\begin{itemize}
    \item Enhancing rendering and synchronization mechanisms to efficiently support larger and more complex models.
    \item Advancing automation for metamodel and model co-evolution to accommodate a broader range of use cases and minimize manual intervention.
    \item Developing sophisticated conflict resolution mechanisms and version control systems tailored for distributed teams.
    \item Exploring integration with existing MDE platforms and tools, such as the Epsilon Playground\footnote{\url{https://eclipse.dev/epsilon/playground/}}, to enable functionalities such as the execution of model-to-model transformations directly within the environment of \jjodel{}.
    \item Consolidating ongoing experiments with large language model (LLM) agents to further enhance modeling capabilities;
%    \item Creating high-quality learning resources, including a comprehensive textbook, presentations, and interactive online tutorials, to support educators and learners;
    \item Expanding the \jjodel{} community by releasing open-source components, encouraging contributions, and establishing an organization dedicated to securing funding and ensuring the project’s sustainability over the long term.
\end{itemize}

By continuing to evolve and address these challenges, \jjodel{} aims to establish itself as a cornerstone tool for MDE, serving both as a practical resource for immediate application and as a foundation for innovation in the field. Its commitment to reducing complexity, fostering collaboration, and supporting adaptability ensures that \jjodel{} will remain a relevant and impactful tool as MDE demands continue to grow.

\subsection{A Roadmap for the Future}

The lessons learned during the development of \jjodel{} provide a roadmap for the future of MDE platforms. Reducing cognitive complexity, fostering collaboration, and supporting adaptability are essential to ensure widespread adoption of MDE practices. These principles make \jjodel{} a valuable resource for practitioners today and a foundation for future innovation.  

However, sustaining such advances requires more than technical expertise: it also demands robust community support and funding. Unlike in the early 2000s, when national and international research programs funded tool development, today’s funding landscape is less aligned with such objectives. This shift raises important questions about the responsibility for developing and maintaining high-quality modeling platforms.  

Although opinions within the MDE community vary, we advocate a shared commitment to advancing both foundational theories and practical solutions. High-quality tools are essential to demonstrate the effectiveness of MDE and ensure its continued relevance in various domains. By working together, the MDE community can build platforms that empower users, enhance education, and drive innovation across industries.  




 