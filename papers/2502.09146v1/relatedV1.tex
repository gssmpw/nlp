This section reviews various MDE platforms, discussing their features, strengths, and limitations to position \jjodel{} within the current landscape of MDE tools.

MetaEdit+ \cite{Kelly2013} is a mature and widely recognized platform for domain-specific modeling (DSM). It uses the GOPPRR meta-metamodeling language \cite{kelly2005domain} to define domain-specific concepts and relationships, offering extensive customization capabilities. Despite its introduction in 1995 and its reliance on an outdated technology stack, MetaEdit+ features a reflective and integrated architecture that provides robust operational support through built-in governance mechanisms. These capabilities simplify complex tasks, such as ensuring metamodel consistency and managing artifacts, challenges that remain prevalent in many other MDE tools.
%
In comparison, \jjodel{} inherits the strengths of MetaEdit +, such as its reflective and integrated architecture. Leveraging a native cloud architecture and cutting-edge front-end technologies, \jjodel{} offers a more \textit{transparent} user experience by minimizing the accidental complexity, alongside support for real-time collaboration and automated syntax co-evolution. 
%These features make \jjodel{} particularly well-suited for agile, iterative, and distributed development workflows.

GMF/Eugenia\footnote{\url{https://eclipse.dev/modeling/gmp/}}, built on the Eclipse Modeling Framework (EMF), facilitates the automated generation of graphical editors. These tools are powerful in defining graphical representations. Unfortunately, their dependence on intricate configurations and technical complexity poses significant challenges for non-expert users. Moreover, their generative approach and component-based architecture hinder their ability to support dynamic adaptability, an essential feature for iterative and agile development workflows, as well as real-time collaboration. In contrast, \jjodel{} overcomes these limitations with an intuitive low-code environment that combines dynamic syntax customization with seamless collaboration, making it a more practical and efficient choice for modern team-based modeling scenarios.

Sirius\footnote{\url{https://eclipse.dev/sirius/}}, also built on the Eclipse ecosystem, introduces viewpoint-based modeling, allowing users to define and work with multiple perspectives in a single model. While this approach reduces programming effort, Sirius remains configuration-intensive and lacks native support for real-time collaboration or dynamic syntax co-evolution, limiting its adaptability in iterative and distributed development workflows. In contrast, the reflective architecture of \jjodel{} incorporates built-in governance mechanisms, often absent in tools such as Sirius, that simplify complex tasks such as coevolution and validation of metamodels, providing a more streamlined and robust modeling experience.

Building on the concepts of its predecessor, Sirius Web \cite{Giraudet2024} transitions to a native web environment, using React to deliver a modern interface. It introduces collaborative modeling features and supports various representation types, including diagrams, forms, and Gantt charts. While these advancements address some limitations of Sirius Desktop, Sirius Web still suffers from significant configuration overhead and limited dynamic adaptability, which constrain its effectiveness in highly iterative and fast-paced projects. Furthermore, its component-based architecture, rooted in EMF, lacks the seamless integration of an all-in-one environment, leading to fragmentation and increased complexity for users. 
%In contrast, \jjodel{} offers a fully declarative and reflective architecture, improving usability and governance while simplifying complex modeling tasks, such as metamodel co-evolution. By providing a seamlessly integrated environment, \jjodel{} ensures greater agility and adaptability to rapidly changing requirements, making it particularly well-suited for collaborative and dynamic modeling scenarios.

Sprotty\footnote{\url{https://sprotty.org/}} and Gentleman \cite{3417990.3421998} address narrower roles in the MDE ecosystem. Sprotty is a web-based visualization framework focused on rendering lightweight models, while Gentleman provides a flexible projectional editing interface for directly modifying abstract syntax trees. Although these tools excel in specific use cases, they lack deeper integration with metamodel evolution and collaboration workflows. \jjodel{} combines the simplicity and lightweight design of tools like Sprotty and Gentleman with advanced capabilities such as automated syntax co-evolution, making it a more comprehensive solution.


A recent paper by Metin et al.~\cite{Metin2025} explores the challenges and lessons learned from working with GLSP and developing several GLSP-based modeling tools. The authors present a reusable reference architecture designed to facilitate the development and operation of GLSP-based web modeling tools. The reference architecture is then instantiated using BigUML as a running example. The paper highlights the procedural approach, key lessons learned, and critical reflections on the challenges and opportunities associated with using GLSP. In contrast to \jjodel{}, which natively supports cloud-based modeling, GLSP-based web modeling tools require the ad hoc implementation of the domain language, adding an additional layer of complexity.

HyperGraphOS \cite{Ceravola2024} represents an ambitious vision, integrating DSLs, modeling, and task execution into a unified system. Its infinite workspace concept and AI-enhanced adaptability demonstrate significant innovation, but these features come at the expense of usability for less technical users. 
%In contrast, \jjodel{} offers a more balanced approach, focusing on accessibility and iterative workflows, making it ideal for teams that prioritize usability and collaboration over extensive technical features.

Many of the platforms surveyed possess industrial-grade or commercial-technology readiness levels (TRL), making them well suited for high-stakes, large-scale projects. In contrast, \jjodel{} remains an academic endeavor, designed to explore and adapt the principles and solutions commonly found in large-budget projects and low-code platforms. Translating these into accessible and innovative approaches, \jjodel{} makes a unique contribution to the field. However, this academic focus also highlights its limitations in directly competing with fully commercialized solutions, particularly in terms of scalability, robustness, and enterprise-level support.
%
However, the initial adoption of \jjodel{} in academic settings has yielded promising results. Although a systematic evaluation of its impact on teaching and learning outcomes has not yet been conducted, preliminary feedback has been highly encouraging. Students have demonstrated significant engagement and understanding, even in the absence of extensive learning resources at the time of the course\footnote{For more details on the survey results, see \url{https://www.jjodel.io/student-survey/}.}. These early findings indicate that \jjodel{} has the potential to become a valuable educational tool, particularly in model-driven engineering courses.