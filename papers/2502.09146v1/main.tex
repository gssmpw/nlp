%%%%%%%%%%%%%%%%%%%%%%% file template.tex %%%%%%%%%%%%%%%%%%%%%%%%%
%
% This is a general template file for the LaTeX package SVJour3
% for Springer journals.          Springer Heidelberg 2010/09/16
%
% Copy it to a new file with a new name and use it as the basis
% for your article. Delete % signs as needed.
%
% This template includes a few options for different layouts and
% content for various journals. Please consult a previous issue of
% your journal as needed.
%
%%%%%%%%%%%%%%%%%%%%%%%%%%%%%%%%%%%%%%%%%%%%%%%%%%%%%%%%%%%%%%%%%%%
%
% First comes an example EPS file -- just ignore it and
% proceed on the \documentclass line
% your LaTeX will extract the file if required
% \begin{filecontents*}{example.eps}
% %!PS-Adobe-3.0 EPSF-3.0
% %%BoundingBox: 19 19 221 221
% %%CreationDate: Mon Sep 29 1997
% %%Creator: programmed by hand (JK)
% %%EndComments
% gsave
% newpath
%   20 20 moveto
%   20 220 lineto
%   220 220 lineto
%   220 20 lineto
% closepath
% 2 setlinewidth
% gsave
%   .4 setgray fill
% grestore
% stroke
% grestore
% \end{filecontents*}
%
\RequirePackage{fix-cm}
%
%\documentclass{svjour3}                     % onecolumn (standard format)
%\documentclass[smallcondensed]{svjour3}     % onecolumn (ditto)
\documentclass[smallextended]{svjour3}    
% onecolumn (second format)
%\documentclass[twocolumn]{svjour3}          % twocolumn
%
\smartqed  % flush right qed marks, e.g. at end of proof
%
\usepackage{adjustbox}
\usepackage{multirow}
\usepackage{appendix}
\usepackage{booktabs,caption}
\usepackage[flushleft]{threeparttable}
\usepackage{orcidlink}
\usepackage{comment}
\usepackage {drafts}
\usepackage{subfigure}
\usepackage{listings}
\usepackage{courier}
\usepackage{array}
\usepackage{longtable}
\usepackage{circledsteps}
\usepackage{float}

\usepackage{xurl} 


\usepackage{cite}
\usepackage{amsmath,amssymb,amsfonts}
\usepackage{algorithmic}
\usepackage{graphicx}
\usepackage{textcomp}
\usepackage{xcolor}
\usepackage{url}
\usepackage[nolist]{acronym}% consistently use acronyms
%%% REVIEW
\newcommand{\tocite}{{\color{red}CITE} }
\newcommand{\toref}{{\color{red}REF} }

%%% LOGO
\newcommand{\usc}{\raisebox{-1pt}{\includegraphics[height=0.8em]{figures/usc_logo.png}}}
\newcommand{\vuam}{\raisebox{-1pt}{\includegraphics[height=0.8em]{figures/vu_logo.png}}}

%%% SIGNS and SYMBOLS
\newcommand{\grad}{\texttt{grad-CROP}}
\newcommand{\att}{\texttt{att-CROP}}
\newcommand{\seg}{\texttt{seg}}
\newcommand{\clip}{\texttt{clip-CROP}}
\newcommand{\sam}{\texttt{sam-CROP}}
\newcommand{\yolo}{\texttt{yolo-CROP}}
\newcommand{\hc}{\texttt{human-CROP}}
\newcommand{\zsvqa}{\texttt{ZSVQA}}
\newcommand{\vic}{\textbf{ViCrop}}
\newcommand{\xmark}{\text{\ding{55}}}
\newcommand{\cmark}{\text{\ding{51}}}
\newcommand{\success}{\texttt{\color{green} \cmark}}
\newcommand{\failure}{\texttt{\color{red} \xmark}}
\newcommand{\rel}{\texttt{rel-att}}
\newcommand{\gra}{\texttt{grad-att}}
\newcommand{\pgra}{\texttt{pure-grad}}
\newcommand{\relh}{\texttt{rel-att$^h$}}
\newcommand{\grah}{\texttt{grad-att$^h$}}
\newcommand{\pgrah}{\texttt{pure-grad$^h$}}


%%% Text Abb.
\makeatletter
\DeclareRobustCommand\onedot{\futurelet\@let@token\@onedot}
\def\@onedot{\ifx\@let@token.\else.\null\fi\xspace}

\def\aka{\emph{a.k.a}\onedot} \def\Eg{\emph{E.g}\onedot}
\def\eg{\emph{e.g}\onedot} \def\Eg{\emph{E.g}\onedot}
\def\ie{\emph{i.e}\onedot} \def\Ie{\emph{I.e}\onedot}
\def\cf{\emph{c.f}\onedot} \def\Cf{\emph{C.f}\onedot}
\def\etc{\emph{etc}\onedot} \def\vs{\emph{vs}\onedot}
\def\wrt{w.r.t\onedot} \def\dof{d.o.f\onedot}
\def\etal{\emph{et al}\onedot}
\makeatletter



\definecolor{myred}{HTML}{FF8577}
\definecolor{mygreen}{HTML}{0FA958}
\definecolor{myblue}{HTML}{1982C4}
\definecolor{codegreen}{rgb}{0,0.5,0}
\definecolor{codegray}{rgb}{0.5,0.5,0.5}
\definecolor{codepurple}{rgb}{0.07,0,0.53}
\definecolor{codered}{RGB}{189,41,0}
\definecolor{codecomment}{RGB}{153,153,153}
\definecolor{backcolour}{rgb}{0.96,0.96,0.96}
\definecolor{royalblue}{rgb}{0.0, 0.14, 0.4}
\definecolor{egyptianblue}{rgb}{0.06, 0.2, 0.65}
\definecolor{royalazure}{rgb}{0.0, 0.22, 0.66}
\definecolor{portlandorange}{rgb}{1.0, 0.35, 0.21}
\definecolor{sienna}{RGB}{183,105,68}
\definecolor{saddlebrown}{RGB}{139,69,19}
\definecolor{mediumbrown}{RGB}{83,41,11}
\definecolor{darkbrown}{RGB}{58,28,7}
\hypersetup{
    colorlinks=true,
    linkcolor=sienna,
    urlcolor=royalblue,
    citecolor=royalblue,
}
\def\BibTeX{{\rm B\kern-.05em{\sc i\kern-.025em b}\kern-.08em
    T\kern-.1667em\lower.7ex\hbox{E}\kern-.125emX}}
%
% \usepackage{mathptmx}      % use Times fonts if available on your TeX system
%
% insert here the call for the packages your document requires
%\usepackage{latexsym}
% etc.
%
% please place your own definitions here and don't use \def but
% \newcommand{}{}
%
% Insert the name of "your journal" with
% \journalname{myjournal}
%
\begin{document}



\title{Modeling in \jjodel: Bridging Complexity and Usability in Model-Driven Engineering}
%\subtitle{Do you have a subtitle?\\ If so, write it here}

\titlerunning{Modeling in \jjodel{}}        % if too long for running head

\author{Antonio Bucchiarone \orcidlink{0000-0003-1154-1382}        \and
        Juri Di Rocco\orcidlink{0000-0002-7909-3902}     \and 
        Damiano Di Vincenzo \and
        Alfonso Pierantonio  \orcidlink{0000-0002-5231-3952} 
}


\institute{SWEN, Università degli Studi dell'Aquila, 67100, L'Aquila, Italy\\ \email{\{name.surname\}@univaq.it}\\ \email{damiano.divincenzo@studenti.univaq.it}
}

\authorrunning{A. Bucchiarone et al.} % if too long for running head

 

\date{Received: date / Accepted: date}
% The correct dates will be entered by the editor


\maketitle
\begin{abstract}

\jjodel{} is a cloud-based reflective platform designed to address the challenges of Model-Driven Engineering (MDE), particularly the cognitive complexity and usability barriers often encountered in existing model-driven tools. This article presents the motivation and requirements behind the design of \jjodel{} and demonstrates how it satisfies these through its key features. By offering a low-code environment with modular viewpoints for syntax, validation, and semantics, \jjodel{} empowers language designers to define and refine domain-specific languages (DSLs) with ease. Its innovative capabilities, such as real-time collaboration, live co-evolution support, and syntax customization, ensure adaptability and scalability for academic and industrial contexts. A practical case study of an algebraic expression language highlights the ability of \jjodel{} to manage positional semantics and event-driven workflows, illustrating its effectiveness in simplifying complex modeling scenarios. Built on modern front-end technologies, \jjodel{} bridges the gap between theoretical MDE research and practical application, providing a versatile and accessible solution for diverse modeling needs.

\end{abstract}

\keywords{Model-Driven Engineering (MDE) \and reflective platforms \and low-code development \and \jjodel{} \and collaborative modeling \and co-evolution}

\section{Introduction}\label{sec1}
\documentclass[../main.tex]{subfiles}
\graphicspath{{../images/}}
\makeatletter
\def\input@path{{../images/}}
\makeatother
\begin{document}
\section{Introduction}
\begin{figure}
\centering
\begin{tikzpicture}
\node[inner sep=0pt] (ws) at (0, 0) {
\includegraphics[height=.4\textwidth, trim={10cm 0 10cm 0},clip]{world_space.png}};
\node[inner sep=0pt] (cs) at (6,0) {\includegraphics[height=.4\textwidth, trim={10cm 1cm 10cm 4cm},clip]{conf_space.png}};
\end{tikzpicture}
\vspace{-5pt}
\label{fig:pbrm_intro}
\caption{\textbf{Left}: Shows world space obstacles as grey spheres. Robots start and goal configuration is colored red and green, respectively. Configurations along the computed path are colored transparent blue. \textbf{Right:} Mapped world space scenario to configuration space. Obstacle region is the grey mesh. Red spheres are collision-free regions computed by the neural SCDF. The optimized shortest path in the convex corridor is the blue curve.}
\vspace{-25pt}
\end{figure}
Motion planning is the problem of finding a collision-free trajectory that connects a given start and goal configuration. The planning takes place in the configuration space of the robot. For single body robots, like mobile robots or drones, the configuration space and the world space are usually the same. This simplifies the planning, since explicit obstacle representations are available which enables geometrical tools like separating hyperplanes, smallest distance to obstacles etc., to be used when designing motion planning algorithms. For multi-body robots like manipulators, the situation is completely different. The world space obstacles are usually mapped to non-convex regions, and to make the problem even harder, the mapping is usually not known. Forming explicit representations of the obstacle region in the configuration space is usually too expensive or intractable. Despite all of this, sampling based planners are used with great success, which mainly is due to their use of implicit representations of the obstacle region. The basic idea is to construct a graph in the configuration space that covers and connects the collision-free region. From this graph, a path can be extracted that connects a given start and goal configuration. The approach is computationally expensive, since the graph is constructed with the smallest geometrical building block available, points, which represents a collision-check. Furthermore, the extracted paths from the graph are non-smooth and jagged due to the stochastic nature of the approach. This adds an additional post-processing step to the process, where the paths are shortcutted and smoothened, before the path can be used for tracking. Clearly a lot of time is invested to form this graph and produce smooth paths. Thus, if the obstacles start to move, then all of this work is done in no use, since all points that make up this graph need to be re-verified, which is simply too time consuming to be done in real time.
\\\\
In this work, we want to address the existing drawbacks of the sampling based planners. Our main contribution is an improved motion planner where each vertex in the graph covers a collision-free region in the form of a sphere instead of a point and where the edges are formed with neighboring intersecting spheres. This representation has the advantage of instead of returning piecewise linear paths, returning a sequence of overlapping spheres, i.e. a convex corridor, that connects a given start and goal configuration, illustrated in Figure \ref{fig:pbrm_intro}. This convex corridor allows us to use convex optimization to produce smooth trajectories, instead of computationally expensive post-processing methods. The representation further allows us to estimate the coverage of the collision-free space, which gives us awareness and feedback in the offline roadmap construction phase. Finally, our representation is simple to adapt to moving obstacles, simply requery for the new radii and recheck for intersections. 
\\\\
The spherical collision-free regions are formed using a signed distance function (SDF), which is a function that returns the smallest distance from an arbitrary point to the boundary of an obstacle. As the name implies, the distance is signed, thus if the point is inside the obstacle it is negative otherwise positive. If the distance is positive, a sphere with radius equal to the distance is guaranteed to cover a collision-free region. Using an SDF in motion planning is not new, but what is novel about our approach is that we express the distance in the configuration space instead of the world space and by doing so allows us to form these convex collision-free regions. We refer to the resulting SDF as a signed configuration distance function (SCDF). Computing an SCDF analytically is non-trivial, our approach is therefore to parameterize the SCDF with a deep neural network and learn the mapping by supervised learning. Our resulting neural SCDF can compute distances for different parameter values of obstacle shapes and we also show how multiple distances can be combined, thus making our approach flexible.
\section{Related work}
Motion planning algorithms can roughly be divided into three families, grid-based, sampling based and optimization based methods. Grid-based methods (GBM) discretize the planning space from which a graph is then compiled. A standard search method is A$^\star$ \citep{a_star}, which is classified as an \textit{informed} search method, since it employs a heuristic function to speed up the search. A$^\star$ guarantees to return an optimal path at the level of discretization used. GBMs usually discretize the planning space by a regular lattice and this limits the GBMs to problems with low dimensionality due to the curse of dimensionality. Thus, GBMs are usually limited to single-body robots where the degrees of freedom (DOF) are low. To overcome the inherent scaling problem with the GBMs, stochastic methods are usually used for multi-body robots. These methods are termed as sampling-based methods (SBM) and core members within this family are the rapidly-exploring random trees (RRT) \citep{rrt} and the probabilistic roadmap (PRM) \citep{prm}. RRT grows a tree from the start configuration and explores the collision-free region in a rapid way until it is able to connect to the goal region. RRT is usually improved by bi-directional planning \citep{rrt_connect}, i.e. an additional tree is grown from the goal configuration and the trees are tested for connection after any tree has been expanded. RRT is a single-query method, thus it searches for a path from scratch each time it is queried. Contrary to this, PRM is a multi-query method, which solves for multiple queries without starting from scratch. PRM does this by creating a roadmap (graph) that covers the collision-free space as an offline step. The graph is then used to solve for multiple queries. PRMs are used in cases where the environment does not change since the extra offline step is too computationally costly and needs to be re-done if the environment is changed. In our work, we address this inherent issue by using a different roadmap representation. Our vertices in the graph cover a collision-free region in the form of spheres and we form the edges by checking for intersecting spheres. If something in the environment changes, we recompute the spheres radii and recheck the intersections, without relying on collision detection. We use a trained neural network to compute the sphere radius, therefore querying for the radius can be done fast, hence our representation enables the PRM for dynamic environments.
\\\\
In the recent decades, optimization based methods (OBM) \citep{chomp, schulman, itomp, stomp} have been introduced as an alternative to SBM for multi-body robots. Like the SBM, the OBMs scale well to higher dimensional problems and produce smoother motion. It is common to use a SDF in the optimization since it is a smooth function, thus enabling gradient-based methods. However, the standard way of expressing the SDF is in world space. The distance therefore needs to be mapped to the configuration space by the forward kinematics. This mapping makes the optimization problem a non-linear program (NLP), which is computationally expensive to solve. Recently, a different approach has been proposed. In \cite{mp_gcs} motion planning is formulated as a convex optimization problem by using the graph of convex sets framework \citep{gcs}. The underlying idea is to decompose the collision-free space into intersecting convex sets from which a convex optimization problem is formulated. In cases where an explicit representation of the obstacles in the configuration space exists, like for single-body robots, creating collision-free convex regions can be done fast \citep{iris}. For multi-body robots, this is non-trivial. Existing work does this successfully \citep{iris_nlp, iris_c} by an optimization based approach, but the methods are still too time consuming to be used in the presence of moving obstacles. Our approach is instead to use deep learning to learn an SDF expressed in the configuration space. With this, we can query for shortest distances to the collision boundary, which allows us to expand spherical regions which are collision-free. Our approach is fast and therefore enables our suggested roadmap planner to be used in dynamic environments.
\\\\
Recent research has focused on learning collision detection \citep{fk_kernel_distance, diffco, graphdistnet} by predicting the signed distance between the robot links and the surrounding obstacles in the world space. The learned SDF is used in trajectory optimization but since the distance is expressed in the world space, the problem becomes an NLP and therefore takes a long time to solve. We take a novel approach and suggest to instead express the signed distance in the configuration space. This allows us to improve the PRM at the same time as it enables convex optimization for trajectory optimization, which runs faster and is more reliable than NLP solvers. In \cite{cspf} a learned signed distance function in the configuration space is proposed similar to our approach. However, their approach is restricted to point cloud representations, while we propose to represent the obstacles as parameterized geometric shapes, e.g. spheres. Furthermore, we also show how to use our learned SCDF to improve an existing roadmap planner.
\section{Problem formulation}
A robot is located in the world space, $\W \subset \R^3 $. The unique location of the robot is given by its configuration $\q \in \C$, where $\C$ is the configuration space. The set of points covered by the robots bodies at a certain configuration is expressed as $\B(\q) \subset \W$. The robot is surrounded by $\NrObst$ obstacles $\O = \bigcup_{i=1}^{\NrObst} \O_i$, where  $\O_i \subset \W$. The representation of the obstacle in the configuration space is the set $\C\O_i = \{\q \in \C \: |\: \B(\q) \cap \O_i \neq \emptyset \}$. The obstacle space is formed as $\Co = \bigcup_{i=1}^{\NrObst} \C \O_i$. The complement is referred to as the free space, $\Cf = \C \setminus \Co$. The path planning problem is a tuple, ($\Cf$, $\qStart$, $\qGoal$), where we want to connect a query pair, consisting of a start, $\qStart$, and goal configuration, $\qGoal$, with a geometric path, $\q(s): [0, 1] \mapsto \Cf$, such that $\q(0)=\qStart$ and $\q(1)=\qGoal$, or report correctly when such a path does not exist.
\end{document}


\section{Overview of \jjodel{}}
\label{sec:overview}
\begin{figure*}[t]
\begin{center}
\includegraphics[width=.85\linewidth]{fig_overview_v3.pdf}
\end{center}
\caption{
FastAtlas Overview: In each frame, we compute charts spanning fully or partially visible triangles (a), determine texture space bounding boxes for the visible portions of the view-space projections of each chart, and tightly pack these boxes into atlases (b, here $2K \times 2K$). We simultaneously bijectively parameterize and shade the charts into their atlas boxes, obtaining high quality texture space shading (c), and use this shading to render the shaded frames (d).}
\label{fig:overview}
\label{fig:alg_overview}
\end{figure*}

\section{Overview}
\label{sec:overview}
Our work has two core contributions: a real-time, GPU-based algorithm for tight packing of general parameterized charts into compact atlases; and a real-time TSS method that
utilizes this packing.  

\paragraph*{FastAtlas Packing.}
FastAtlas runs entirely on the GPU as a series of compute shaders. It takes the bounding boxes of parameterized charts as input, and packs them into an atlas (Fig~\ref{fig:overview}b, Sec.~\ref{sec:pack}). As such, the only input it requires are the dimensions of the bounding boxes.
Its outputs are deterministic; identical input charts are packed into identical atlases. This is critical for TSS and similar applications, as it ensures that consecutive frames taken from the same camera view have the same shading. Even minute shading differences across such frames can cause sampling jitter, leading to undesirable flicker \cite{baker2012rock}. 
While prior methods such as \cite{mueller2018shading,hladky2019tessellated,hladky2021snakebinning,Neff2022MSA} cap the dimensions of the charts that can be packed as-is for a given atlas size, and scale down all charts that exceed these dimensions, we scale all charts by the same factor, and do so only when strictly necessary to achieve packing success (Figs~\ref{fig:atlas},~\ref{fig:sas_issues}). 

\paragraph*{TSS using FastAtlas.}
Our end-to-end TSS atlas generation method combines the packing method above with a novel approach for computing seamless per-frame charts. 
We define our charts as the connected components of the visible surfaces in each frame (Fig.~\ref{fig:overview}a), and efficiently compute them using a parallel union-find algorithm (Sec.~\ref{sec:visible}). Since the boundaries of these charts coincide with the contours of the rendered surface, they are {\em invisible} to the viewer. This approach 
eliminates the artifacts caused by shading discontinuities along visible seams (Fig.~\ref{fig:seams}). 

\begin{parWithWrapFigure}
\begin{wrapfigure}{l}{.27\columnwidth}%
\includegraphics[width=\linewidth]{fig_inset_view_plane.pdf}%
\end{wrapfigure}
We bijectively parametrize the {\em visible portions} of our charts by projecting them to view space (inset). This maps a constant number of texels to each pixel in the final rendered output, evenly distributing residual undersampling error across all image pixels. While conceptually straightforward, efficiently parameterizing charts containing partially visible triangles using viewspace projection is non-trivial, as the visible portions may no longer be triangular (e.g. green triangle in the inset); applying naive projection to triangles with vertices behind the camera may produce ill-posed results. Clipping triangles before projection is both computationally expensive and significantly complicates downstream operations. We avoid explicit clipping by observing that all that is required for atlas packing is the dimensions of, potentially conservative, bounding boxes of these projected visible portions. We compute such bounding boxes without explicit chart clipping by adapting a conservative screen coverage estimator \shortcite{Blinn:CalculatingScreenCoverage} (Sec.~\ref{sec:box}). We then pack the computed boxes using FastAtlas. 
\end{parWithWrapFigure}

Finally, we shade the visible portion of each chart into its corresponding atlas bounding box (Fig~\ref{fig:overview}c). 
The resulting texture is then used during rasterization as a standard texture map (Fig. ~\ref{fig:overview}d). 
Our framework is compatible with all existing approaches for texture space shading, including forward shading, raytraced illumination, or deferred shading in texture space \cite{baker:2016}. In the examples shown, we use the standard forward shading based rendering pipeline included in the G3D Innovation Engine \cite{G3D17}, a commercial grade renderer.



\section{Flexibility and Flexible Modeling}\label{sec:advanced}
% \draft{\begin{itemize}
%     \item Semantic Zooming
%     \item Projectional and Positional Editing
% \end{itemize}}

% \subsection{Flexible Modeling with Shapeless Objects and Bidirectional Semantic Mapping}

% \begin{itemize}
%     \item Explain how JJODEL’s shapeless objects allow users to create “in-progress” elements that can be refined later, supporting flexible, iterative modeling. 

% \item Example: Demonstrate creating a placeholder object representing an undefined UML element that can later be specified as a class or attribute. 

% \item Bidirectional Semantic Mapping: Discuss how JJODEL’s mapping allows changes to concrete representations to update the underlying model. 

% \item Takeaway: JJODEL’s flexible modeling approach enables users to explore ideas iteratively, without locking them into rigid structures early on. 
% \end{itemize}

% \subsection{ Collaborative Modeling}

% \begin{itemize}
%     \item  Multi-User Editing and Conflict Resolution: Describe JJODEL’s support for real-time collaboration, where multiple users can work on the same model or metamodel with instant updates. 

%     \item  Example: In a group project scenario, team members simultaneously edit a shared UML model, with JJODEL handling versioning and conflict management. 

%     \item  Takeaway: JJODEL fosters collaboration by allowing multiple users to contribute to a single model seamlessly, making it ideal for team environments and distributed projects. 
% \end{itemize}
A core objective of \jjodel{} is to offer flexibility, enabling end users and language engineers to navigate the complexity and richness of modern multifaceted domains. Flexibility is not just a feature, but a foundational principle embedded within its design and technology stack. This approach allows language designers to tailor the modeling environment to their specific needs without encountering the challenges typically associated with custom coding or inflexible extension mechanisms.

To show the flexibility of \jjodel{} in action, we present two illustrative examples: \textit{grid snapping} and \textit{semantic zooming}~\cite{frisch2008towards}. These examples highlight how \jjodel{}'s design facilitates effortless customization of modeling functionalities, demonstrating its adaptability to diverse user requirements and complex scenarios.

\subsection{Grid-snapping}\label{grid-snapping}

Grid snapping is an essential feature in modeling environments, enhancing precision, usability, and efficiency. Although its implementation can be challenging, requiring accurate layout information and dynamic adaptability, these challenges are outweighed by the benefits it provides. By automating alignment tasks, grid snapping allows users to concentrate on the creative and semantic dimensions of their models, ensuring both functional accuracy and aesthetic refinement.

\jjodel{} offers a flexible framework for extending modeling editors by integrating two powerful methodologies: (i) JSX templating and CSS for advanced visual customization, and (ii) Event-Condition-Action (ECA) rules~\cite{widom1995active} for introducing dynamic behavior. As an example, implementing a dot-based grid—a visually minimal yet effective tool for aligning and organizing elements within the modeling editor—can be accomplished through the following steps: 

\begin{enumerate}
\item Defining a CSS class to render the grid as a pattern of dots on the editor canvas.  
\item Adding a toggle command to dynamically enable or disable the grid.  
\item Specifying an ECA rule to snap elements to the nearest grid vertex during movement.  
\end{enumerate}

Each of these steps is elaborated in the following paragraphs.

\paragraph{Step 1: Defining the Grid Style.}

The dot-based grid style is implemented through a simple CSS class definition, as shown below:

\begin{lstlisting}[language=css]
 .grid {
    background-image: radial-gradient(silver 1px, transparent 0);
    background-size: 15px 15px; /* 15px x 15px grid */
 }
\end{lstlisting}

This class renders the grid as a pattern of evenly spaced dots. In the next step, we show how to dynamically apply or remove the \code{.grid} class in the model view.

\paragraph{Step 2: Adding a Toggle Command for Grid Control.}

\jjodel{} enables the extension of the editor functionalities using the \code{<Control/>} component within the model view\footnote{The model view is included in the default viewpoint but can be cloned and customized as needed.}. In this example, a \code{<Toggle/>} component is added to dynamically enable or disable the grid. The toggle operates on a Boolean parameter, \code{grid}, which is defined on the template page, as illustrated in Figure~\ref{fig:grid-parameter}.
%
\begin{figure}
\centering
\includegraphics[width=0.75\linewidth]{figs/gridparameter.png}
\caption{The user-defined \code{grid} parameter.}
\label{fig:grid-parameter}
\end{figure}
%
The parameter \code{grid} serves as a shortcut for the expression:\\
%\begin{center}
\begin{lstlisting}[language=JSX-template]
node.state.grid ?? false
\end{lstlisting}
%\end{center}
which initializes the parameter in the \code{node} submodel to \code{false}. The updated template for the model view is shown below:
%
\begin{lstlisting}[language=JSX-template]
 <View className={model ${grid && 'grid'}}>
    ...
    <Control title={'Workbench'} payoff={'Controls'}>
        <Toggle name={'grid'} title={'Grid'} node={node} />
    </Control>
 </View>
\end{lstlisting}
%
Here, the conditional expression \code{\$\{grid \&\& 'grid'\}} dynamically applies the \code{.grid} class whenever the value of the parameter \code{grid} is \code{true}, allowing the feature.

\paragraph{Step 3: Enabling Snap-to-Grid Behavior.}
To implement the snap-to-grid functionality, an ECA rule is defined, as shown in Figure~\ref{fig:eca}. This rule takes advantage of the \code{grid} parameter to ensure that the elements align with the nearest grid vertex when moved.
%
\begin{figure}[h]
\centering
\includegraphics[width=0.95\linewidth]{figs/ECA3.png}
\caption{ECA rule for the \textit{snap-to-grid} feature.}
\label{fig:eca}
\end{figure}
%
The event \code{onDataUpdate} \circled{1} is triggered when an element is moved. Condition \circled{2} verifies the state of the parameter \code{grid}, and if enabled, the action \circled{3} adjusts the coordinates \code{x} and \code{y} of the model element to align with the nearest grid vertex.  

Figure~\ref{fig:editor-grid} provides a screenshot of the editor with the grid enabled and \code{<Toggle/>} on.

\begin{figure}[h!]
\centering
\includegraphics[width=1\linewidth]{figs/editorgrid2.png}
\caption{Grid-enabled editor.}
\label{fig:editor-grid}
\end{figure}

\subsection{Semantic Zooming}
Semantic zooming~\cite{frisch2008towards} is an advanced interaction paradigm that dynamically adjusts the level of detail presented in a model based on the zoom level. Unlike traditional zooming, which merely magnifies or shrinks visual elements, semantic zooming alters the content itself to better align with the user’s context and focus. In modeling, this approach offers several benefits, including enhanced usability, reduced cognitive load, and more efficient navigation by presenting only the most relevant details at each zoom level~\cite{pirolli2001focus}. For example, at higher zoom levels, intricate details of specific model elements become visible, while at lower zoom levels, complexity is abstracted to provide an overview of the entire model.

In \jjodel{}, semantic zooming is achieved by linking different visual representations to predefined zoom thresholds. These representations are dynamically updated as users adjust the zoom level, ensuring that the displayed content remains contextually relevant and appropriately detailed. Although semantic zooming might seem more conceptually complex than features such as grid snapping, it is implemented seamlessly in \jjodel{} thanks to its flexible templating framework. This framework allows for the integration of controls, such as parameters similar to \code{grid} discussed above, that dynamically influence visualization without affecting the underlying models.

Semantic zooming in the default syntax is implemented through the following steps:  

\begin{enumerate}  
\item Define an integer parameter, such as \code{level}, to represent the current zoom level.  
\item Add a slider control to the user interface that allows users to interactively adjust the zoom level.  
\item Divide the template into sections corresponding to different zoom levels, ensuring that the level of detail displayed dynamically adjusts based on the parameter's value.  
\end{enumerate}  
Each of these steps is elaborated in the following paragraphs.

\paragraph{Step 1: Defining a new integer parameter.} This step mirrors the process of defining the parameter \code{grid} for the snap-to-grid feature discussed earlier. Here, a user-defined parameter serves as a dynamic control to adjust the visualization. For example, the \code{level} parameter can be introduced in the template as part of the dependency settings. It is initialized using an expression such as:  

%\begin{center}
\begin{lstlisting}[language=JSX-template]  
node.state.level ?? 3
\end{lstlisting}  
%\end{center}
depicted also in Figure~\ref{fig:level}. This ensures that the parameter is stored in the \code{node} submodel and defaults to 3 (or any specified starting zoom level) if the user does not explicitly set it.  

\begin{figure}  
    \centering  
\includegraphics[width=0.75\linewidth]{figs/level.png}  
    \caption{User-defined \code{level} parameter.}  
    \label{fig:level}  
\end{figure}  

The parameter \code{level} acts as a bridge between user input and the dynamic behavior of the semantic zooming feature. By making the zoom level accessible and modifiable, it allows the modeling editor to adjust the level of detail displayed dynamically based on the current zoom value. This foundational step ensures that the parameter is seamlessly integrated into the template and other relevant parts of the modeling environment, setting the stage for subsequent steps in the implementation of semantic zooming.  

\paragraph{Step 2: Adding a Slider Control for Zoom Level Adjustment.} 


\paragraph{Adding a Slider Control for Zoom Level Adjustment.} This step parallels the use of the \code{<Toggle/>} component for the snap-to-grid feature but utilizes a \code{<Slider/>} component to modify the \code{level} parameter.  

The component \code{<Slider/>} is added to the control panel in the template, providing an intuitive mechanism for users to adjust the zoom level. Dynamically updates the \code{level} parameter stored in the \code{node} submodel, ensuring real-time responsiveness. This parameter acts as a bridge between user input and the dynamic behavior of the template.  

For instance, the updated template might look as follows:  

\begin{lstlisting}[language=JSX-template]  
 <View className={`model`}>  
   ...  
   <Control title={'Workbench'} payoff={'Zoom Controls'}>  
      <Slider name={'level'} title={'Zoom level'} node={node} min={0} max={3} />  
   </Control>  
 </View>    
\end{lstlisting}  

In this example:  
\begin{itemize}
    \item In line 4, the \code{name} property links the slider to the \code{level} parameter, ensuring that any adjustments made via the slider directly update the parameter. 
    \item In line 5, additional properties such as \code{min} and \code{max} define the range of zoom levels (e.g., 0 to 3) and the granularity of adjustments.  
\end{itemize}


Similar to the grid-snapping feature, this slider control integrates seamlessly with the \jjodel{} framework, enabling users to dynamically modify visualizations without directly interacting with the model data. The \code{level} parameter remains synchronized with the slider's position, facilitating context-aware rendering based on the zoom level, which will be detailed in the next step.  

\paragraph{Step 3: Slicing the Template for Semantic Zooming.} This step specifies how content dynamically adapts as the zoom level changes, ensuring that the displayed information is contextually relevant and aligned with the user’s focus. The template is divided into sections, each corresponding to a specific zoom level. For example, at lower zoom levels (e.g. \code{level = 0}), the template displays high-level abstractions of the model, while at higher zoom levels (e.g., \code{level = 3}), more detailed information is revealed.  

The following is a simplified example of a template associated with a generic metaclass \code{<metaclass-name>}:  

\begin{lstlisting}[language=JSX-template]  
 <View className={'metaclass-name'}>  
    {level === 0 && (  
        <div className={'overview'}>  
            {/* Render high-level abstractions */}  
            Overview of the model...  
        </div>  
    )}  
    {level === 1 && (  
        <div className={'mid-detail'}>  
            {/* Render mid-level details */}  
            Model with basic elements...  
        </div>  
    )}  
    {level >= 2 && (  
        <div className="full-detail">  
            {/* Render detailed elements */}  
            Complete model details...  
        </div>  
    )}  
 </View>  
\end{lstlisting}  

In this example, lines 2, 8, and 14 use conditional rendering to determine which content is displayed based on the value of the \code{level} parameter. This approach ensures that the visualization dynamically adjusts to the user’s zoom level, providing the appropriate level of detail for their current focus.  

The described approach to slicing the template ensures that semantic zooming is both straightforward to implement and highly effective, creating a dynamic and user-friendly modeling environment. Using the \code{level} parameter and the \jjodel{} templating system, semantic zooming is achieved without the need for custom coding or complex extension mechanisms.  

It is worth noting that this is just one method of implementing semantic zooming, using a single template with multiple sections corresponding to different zoom levels. An alternative approach could involve using a separate view for each \code{level} value, which offers additional flexibility depending on the use case.


\section{\jjodel{} in Practice}
\label{sec:practice}
\section{Example}
\label{app:example}

Table~\ref{tab:example} shows a counseling session between a client agent and CAMI, with the client's profile shown in Table~\ref{tab:example profile}. During the session, CAMI explores Health and Relationships topics in the first 6 turns.  It subsequently performs Step Into operation from Relationships to Parenting topics in Turn 7, and from Parenting to Paternal Bond topics in Turn 9, before a Switch operation to Habituation topic in Turn 11.  The full exploration path is visualized in Figure~\ref{fig:exploration path}.

\begin{table*}[ht]
\centering
\begin{tabularx}{\textwidth}{c X}
\toprule
        {\bf Turn ID} & {\bf Utterances}                     \\ \midrule
1       & Counselor: Hello. How are you?                                                                                                                                                                                                                                                                                                                     \\ \hline
2       & Client: I am good. What about you?                                                                                                                                                                                                                                                                                                                 \\ \hline
3       & Counselor: I'm doing well, thanks for asking! Can you share how you've been feeling about your health lately, especially in relation to smoking?                                                                                                                                                                                                   \\ \hline
4       & Client: I've been pretty busy with work and juggling family stuff. Sometimes I just think about how nice it is to take a break with my coworkers outside, chatting and during those moments, life feels a bit more relaxed. You know how it is—it’s nice to unwind.                                                                                \\ \hline
5       & Counselor: It sounds like smoking is a comforting routine for you amidst your busy life with work and your daughter. I wonder how it might impact the time you spend with your family.                                                                                                                                                             \\ \hline
6       & Client: I get what you mean, but I really don't see my smoking as an issue. It helps me manage stress, and I don’t think it affects my time with my daughter. We still have good moments together, and I wouldn't want to change that.                                                                                                             \\ \hline
7       & Counselor: It sounds like you find comfort in smoking to manage stress, and that feels crucial to your parenting now. I wonder, though, how you might feel if your daughter started to see smoking as a way to unwind, too. What impact would that have on your relationship with her?                                                             \\ \hline
8       & Client: That’s a tough thought. I remember starting young and how hard it was to quit later. I wouldn’t want that for her. It makes me reflect on what I’m modeling.                                                                                                   \\ \hline
9       & Counselor: It's great that you're reflecting on your modeling for your daughter. Perhaps seeing smoking as something that may weaken your bond could help reshape how you approach stress—considering healthier ways to connect can actually strengthen your relationship and set a positive example for her.                                      \\ \hline
10      & Client: I appreciate that perspective. I do want to set a strong example for her. My experience with my own family and their struggles makes me cautious. I see the patterns and don’t want her to feel that smoking is a go-to solution for stress, like I did.                                                                                   \\ \hline
11      & Counselor: It sounds like your family’s struggles have really shaped your perspective on smoking. You want to protect your daughter from those same patterns, knowing that smoking can not only impact her stress management but also her overall learning and adaptation to challenges. That’s a powerful motivation.                             \\ \hline
12      & Client: Exactly. I want her to learn healthier coping strategies instead of following in my footsteps. I realize that I need to make a change for both our sakes. It’s not just about me; it’s about creating an environment where she can thrive without those habits. It's time to find better ways to manage stress that we can share together.   
\\ \bottomrule
\end{tabularx}
\caption{Counseling Session between CAMI and a Simulated Client.}
\label{tab:example}
\end{table*}



\begin{table*}[ht]
\centering
\begin{tabularx}{\textwidth}{l X}
\toprule
{\bf Profile Item}        & {\bf Description}                     \\ \midrule
Behavioral Problem  & Smoking \\ \hline
State of Mind       & Precontemplation \\ \hline
Personas            & You smoked when you were young.", "You have a 12-year-old daughter. \newline The friends of the your daughter have tried smoking. \newline When you are at work, you all go out and smoke and chit chat and talk about things. \newline You like to have a cigarette after eating. \\ \hline
Motivation          & You are motivated because of the risk of smoking in habituation for your daughter, as you don't want your daughter to think smoking is good thing.    \\ \hline
Beliefs             & You feel smoking enjoyable and relaxing. \newline You would be left out if you would hate to not smoke. \newline You use smoking as a way to manage weight and control eating habits. \newline You have been smoking for a long time and feels it's almost impossible to quit. \\ \hline
Motivation Topics          & Interpersonal Relationships, Parenting, Habituation
\\ \bottomrule
\end{tabularx}
\caption{Client's Profile in the Example.}
\label{tab:example profile}
\end{table*}


\begin{figure*}
    \centering
    \includegraphics[width=\linewidth]{figs/example_explore_path.pdf}
    \caption{Topic Exploration Path by the Counselor in the Example.}
    \label{fig:exploration path}
\end{figure*}



% \section{Lessons Learned and Community Impact}
% \label{sec:lessons}
% \section{The Road Ahead}
\label{sec:ll}

Based on our study, we now present a few observations with implications for the future design and deployment of rewriting tools.

\subsection{Rewrite Space Coverage by LLMs}

Given the decades-long research on database query optimization, we expected the potential for performance improvement via rewriting to be limited. What came as a surprise was the substantial scope for improvement still available, as showcased by the large \csgm and \tsgm values, even on commercial platforms. These results suggest that LLMs explore optimization spaces that are well outside the purview of contemporary database engines. Further, this enhanced space could be augmented, in a two-stage process, with the recent proposals for LLM-based ``plan hints'' that steer the optimizer in fruitful directions within a plan space~\cite{LLMplanhint}.

\subsection{Rewrite Migration to Optimizer}
The above demonstrated the potent exploratory power of LLMs. But from an overheads perspective, such rewrites should ideally be within the optimizer's native search space rather than recommended from outside. Therefore, it would be a useful exercise to try and distill fresh optimization rules from these instances, leveraging the extensibility features of contemporary optimizers~\cite{extoptbook} to facilitate their incorporation in existing systems.

\begin{figure}[t]
    \centering
    \includegraphics[width=0.5\linewidth]{Figures/q90-sql.png}
    \vspace{-0.1in}
    \caption{Rewritten TPC-DS Q90}
    \label{fig:q90-rewrite}
    \vspace{-0.1in}
\end{figure}

On the flip side, there appears to be an ``impedance mismatch'' against such integration for certain classes of rewrites. 
%
For example, consider the TPC-DS Q90 rewrite in Figure~\ref{fig:q90-rewrite}. 
The original query individually computed  AM (morning) sales and PM (evening) sales, which were then used to compute the AM to PM ratio. The rewrite, however, 
extracted all relevant rows in one shot and computed the ratio using CASE statements -- encoding such transformations as generic rules in the optimizer appears challenging, given the combinatorial ways in which such transformations can occur.

Therefore, a fruitful area of future research could be achieving a middle-ground between the disparate world-views of LLMs and traditional optimizers.
%

\subsection{Revisiting Optimizer Plan Costing}
\label{sec:cost-mismatch}

As highlighted in Section~\ref{sec:exp}, there were instances of substantive differences between the promised speedup and that delivered at run-time. In fact, to the extent that speedups could even turn out to be regressions! This is due to the brittleness of optimizer plan costing in the new spaces explored by the LLM. As a case in point, during our \lithe design process, the prefiltering in Rule R6 (Table~\ref{tab:basic-prompt-exp}) had initially \emph{not} been restricted to self-joins. It resulted in the number of {\cpr}s (on PostgreSQL) being as high as 65, with an astonishing \csgm of 30.6! However, upon execution, most of these rewrites turned out to be regressions, which led to our inclusion of the restriction. But note that we are artificially modulating the rewrite, rather than fixing the plan costing module, which is the principled solution.  In sum, while plan cost modeling has been a long-standing area of research, there is now even more reason given the new rewrite spaces to study this topic further -- for instance, the operator cost model could be extended using calibration techniques similar to those advocated in \cite{wucostmodel}, while the operator cardinality model could be improved with attention-based techniques~\cite{alece}.

\subsection{Agentic LLMs for Query Rewriting}

An effective way to extend \lithe is to use an agentic LLM that actively interacts with the database environment. 
This would allow a ``\lithe agent" to leverage a memory store to log one or more of: (1)~query execution times; (2)~memory usages of slow queries; (3)~prior interactions with users; and (4)~useful/new rewrite rules  learnt over time. Additionally, the agent could be given database engine access so that it can directly request metadata such as query execution plans, database statistics, live query analyzers, etc. 
%
By following a structured workflow -- planning, acting, observing, and refining -- the agent would then iteratively enhance query performance. This adaptive workflow supports continuous learning, so that query rewriting can become more efficient over time.

\subsection{Scope of Semantic Equivalence Tools}
As seen in the experiments section, the coverage of logic-based query equivalence testing on industrial-strength queries is rather limited.
On the other hand, while it is highly likely that the statistics-verified rewrites are valid, it still requires the DBA to make a final call on the correctness. 
%
This limitation restricts the use of \lithe in a fully automated scenario, i.e., as a direct preprocessor to the query engine. Therefore, a key challenge is to improve logic-based coverage. 


\section{Reflections and Lessons Learned}
\label{sec:lessons}
The design and implementation of \jjodel{} provided invaluable insights into the challenges and opportunities of creating a modern and user-friendly modeling platform. These lessons, which span architectural decisions, usability considerations, and technology integration, offer a foundation for advancing the field of MDE tools while reflecting on the successes and setbacks encountered during platform development.

\subsection{The Impact of Technology Choices}

The selection of foundational technologies such as Node.js and React played a key role in shaping the architecture, scalability, and usability of \jjodel{}. Node.js introduced a lightweight event-driven back-end optimized for real-time interactions, providing the performance and responsiveness necessary for a collaborative modeling platform. React’s component-based architecture (not to be confused with that of the EMF ecosystem)  enabled modularity and reusability, allowing dynamic syntax customization and intuitive user interface design. However, adopting these frameworks was not straightforward; it required a significant paradigm change in application development and iterative testing to align their architectural models with the goals of \jjodel{}.

The broader evolution of the full-stack web ecosystem underscores the growing influence of these technologies. Key advancements such as Node.js (2009), Angular.js (2010), React (2013), and Vue.js (2014) have transformed the way modern applications are designed and developed. By 2024, Node.js powers over 40\% of web applications, cementing its role as a preferred backend technology in low-code platforms. React, used by 39.5\% of front-end developers, integrates seamlessly with advanced interfaces, making it a leading choice for creating responsive, component-based UIs\footnote{\url{https://www.statista.com/statistics/1124699/worldwide-developer-survey-most-used-frameworks-web/?utm_source=chatgpt.com}, last accessed on January 9, 2025}. These frameworks highlight the stark contrast in user experience between traditional modeling tools (all first released before), often emphasizing technical rigor, and modern low-code development platforms (LCDPs)~\cite{sahay2020supporting,prinz2021low}, which prioritize usability, real-time collaboration, and seamless adaptability.

For \jjodel{}, the decision to adopt Node.js and React was driven by the need to balance accessibility, flexibility, and \textit{richness}—qualities often absent in traditional tools. Node.js provided the foundation for real-time interactions, while React’s modular architecture empowered the platform to deliver an intuitive and customizable user experience. Together, these technologies helped \jjodel{} align its design with the broader goals of scalability and usability, ensuring that the platform could meet the diverse needs of its users. This process underscored the importance of carefully integrating technological frameworks to shape not only the technical implementation but also the overall user experience.

Crucially, these insights were not evident at the beginning. They emerged through a process of trial and error, with one particularly challenging iteration spanning two years of development. This iterative approach, guided by the expertise and advocacy of one of the authors, ultimately enabled the team to make informed decisions and implement a modern, scalable architecture for \jjodel{}.

In summary, the development of \jjodel{} reinforced a critical realization: technology is not neutral. It defines both the possibilities and constraints of a platform, influencing its ability to innovate and its inherent limitations. 
%These lessons highlight the pivotal role of thoughtful technology choices in ensuring that platforms align with their goals and fulfill their potential.

\subsection{Balancing Flexibility and Simplicity}
Balancing the flexibility of the platform with its usability proved to be one of the most significant challenges in the development of \jjodel{}. Many features were essential to meeting the requirements we add in mind~\cite{di2023jjodel}, but their integration risked introducing unnecessary complexity. To address this, \jjodel{} adopted a progressive disclosure approach, where essential features were initially exposed and advanced functionalities could be revealed later. Like this, both novice and experienced users could interact with the platform more effectively.

The decision to prioritize simplicity without compromising flexibility was further informed by user feedback. Iterative testing demonstrated that users, particularly those new to MDE tools, found overwhelming interfaces to be a barrier to productivity. By simplifying initial interactions and allowing users to explore more advanced features, \jjodel{} demonstrated that even complex modeling platforms could cater to diverse audiences while retaining powerful capabilities. 

\subsection{Avoiding Accidental Complexity}

Reducing accidental complexity~\cite{atkinson2008reducing} was a cornerstone of the \jjodel{} project. Traditional MDE tools often require complex installation procedures and specialized technical expertise, which hinder accessibility for students and domain experts. By prioritizing a client-side intelligence model, \jjodel{} eliminated many of these barriers, enabling users to engage with the platform without requiring advanced setup or configuration.

This decision significantly improved the accessibility and usability of the platform, aligning with its mission to democratize modeling practices. In addition, the streamlined architecture and processes reduced the cognitive load for users, allowing them to focus on their tasks rather than the intricacies of the tool itself. This approach highlights the importance of designing platforms that prioritize usability without sacrificing flexibility or power.

\subsection{Rethinking Concrete Syntax}

The integration of React’s component-based architecture into \jjodel{} fundamentally transformed how concrete syntax was conceptualized and rendered. Unlike traditional modeling tools, which often rely on static and rigid representations, the proposed approach allowed dynamic customization and real-time interaction. This flexibility enabled users to define syntax viewpoints tailored to their specific needs, fostering a more intuitive modeling experience.

This innovation also demonstrated the potential of modern web technologies to drive advancements in traditionally static domains. Using React's state management and rendering capabilities, \jjodel{} bridged the gap between abstract modeling concepts and practical applications, providing users with a seamless and interactive experience.

\subsection{Iterative Design and Feedback}

The iterative development process~\cite{4668134} was central to the success of \jjodel{}. Regular testing with novice and experienced modelers revealed critical gaps in usability and functionality, informing refinements in interface design and platform features. Early feedback emphasized the importance of real-time feedback mechanisms and context-aware interfaces, leading to the implementation of progressive disclosure and other user-centric design principles.

This approach ensured that \jjodel{} evolved as a responsive and adaptive platform, capable of meeting the diverse needs of its user base. The iterative process also reinforced the value of engaging with stakeholders throughout development, ensuring that the platform remained aligned with user expectations and requirements.

\subsection{Navigating Trade-offs}

Departing from established standards, such as Eclipse GLSP~\cite{bork2023vision}, presented significant challenges and unique opportunities for \jjodel{}. This decision enabled the platform to align more closely with its meta-metamodel, a strict extension of Ecore, and to prioritize client-side intelligence. IAlthough it resulted in a less standardized architecture, the trade-off proved advantageous, making the platform more reactive and agile while eliminating cumbersome integration processes often associated with traditional frameworks.

This experience highlights the critical importance of carefully evaluating trade-offs in platform design. By prioritizing flexibility and usability over strict adherence to established frameworks, we took a calculated risk that ultimately paid off. The resulting architecture demonstrated that divergence from canonical approaches, when guided by clear objectives, can lead to innovative and effective solutions aligned with the vision and goals of the platform.

\subsection{Exploring Alternatives to OCL.js}
Our experience with OCL.js\footnote{\url{https://ocl.stekoe.de/}}, a JavaScript-based implementation of the Object Constraint Language (OCL), provided both opportunities and challenges during the development of \jjodel{}. Despite OCL.js offered full integration with \jjodel{} and initially appeared to be a natural fit, its limitations became apparent over time. The lack of active development of the project and its partial coverage of the OCL standard forced us to explore alternatives to define constraints and predicates within the platform.

The most natural alternative was to adopt the declarative and functional expressions of JSX, which \jjodel{} already uses to template. This approach allowed us to leverage the same notation for both templating and predicates, creating a consistent and simplified user experience. However, it also introduced new uncertainties. To the best of our knowledge, JSX has not been formally characterized and no direct comparison between JSX and OCL has been documented in the literature. This absence of formalization leaves open questions about the expressive power, correctness, and potential limitations of JSX as a substitute for OCL.

This experience underscores the importance of evaluating the trade-offs between adopting emerging notation like JSX and relying on established but limited standards like OCL.js. Although JSX has served \jjodel{} well in the short term, further research and validation is necessary to determine its suitability as a foundational element to define constraints and predicates.

\subsection{The Value of Learning Through Failure}

Some of the most valuable lessons emerged from the failures encountered during development. Iterative attempts to align the architectural design with the objectives of the platform revealed the complexity of creating a scalable, reactive system~\cite{lyytinen1999learning}. In one instance, a two-year iteration led to significant revisions, demonstrating the importance of persistence and openness in addressing setbacks.

This experience reinforced the value of accepting failure as an opportunity for growth. Learning from missteps, \jjodel{} was able to refine its approach and develop a platform that is both innovative and user-friendly. The development of \jjodel{} underscored the importance of balancing technical innovation with user-centric design. The lessons learned highlight the need for thoughtful technology choices, iterative development, and a willingness to embrace trade-offs and failures as opportunities for growth. These insights provide a foundation for advancing the field of MDE tools, ensuring that future platforms are both powerful and accessible.

\section{Related Work}
\label{sec:rw}
This section reviews various MDE platforms, discussing their features, strengths, and limitations to position \jjodel{} within the current landscape of MDE tools.

MetaEdit+ \cite{Kelly2013} is a mature and widely recognized platform for domain-specific modeling (DSM). It uses the GOPPRR meta-metamodeling language \cite{kelly2005domain} to define domain-specific concepts and relationships, offering extensive customization capabilities. Despite its introduction in 1995 and its reliance on an outdated technology stack, MetaEdit+ features a reflective and integrated architecture that provides robust operational support through built-in governance mechanisms. These capabilities simplify complex tasks, such as ensuring metamodel consistency and managing artifacts, challenges that remain prevalent in many other MDE tools.
%
In comparison, \jjodel{} inherits the strengths of MetaEdit +, such as its reflective and integrated architecture. Leveraging a native cloud architecture and cutting-edge front-end technologies, \jjodel{} offers a more \textit{transparent} user experience by minimizing the accidental complexity, alongside support for real-time collaboration and automated syntax co-evolution. 
%These features make \jjodel{} particularly well-suited for agile, iterative, and distributed development workflows.

GMF/Eugenia\footnote{\url{https://eclipse.dev/modeling/gmp/}}, built on the Eclipse Modeling Framework (EMF), facilitates the automated generation of graphical editors. These tools are powerful in defining graphical representations. Unfortunately, their dependence on intricate configurations and technical complexity poses significant challenges for non-expert users. Moreover, their generative approach and component-based architecture hinder their ability to support dynamic adaptability, an essential feature for iterative and agile development workflows, as well as real-time collaboration. In contrast, \jjodel{} overcomes these limitations with an intuitive low-code environment that combines dynamic syntax customization with seamless collaboration, making it a more practical and efficient choice for modern team-based modeling scenarios.

Sirius\footnote{\url{https://eclipse.dev/sirius/}}, also built on the Eclipse ecosystem, introduces viewpoint-based modeling, allowing users to define and work with multiple perspectives in a single model. While this approach reduces programming effort, Sirius remains configuration-intensive and lacks native support for real-time collaboration or dynamic syntax co-evolution, limiting its adaptability in iterative and distributed development workflows. In contrast, the reflective architecture of \jjodel{} incorporates built-in governance mechanisms, often absent in tools such as Sirius, that simplify complex tasks such as coevolution and validation of metamodels, providing a more streamlined and robust modeling experience.

Building on the concepts of its predecessor, Sirius Web \cite{Giraudet2024} transitions to a native web environment, using React to deliver a modern interface. It introduces collaborative modeling features and supports various representation types, including diagrams, forms, and Gantt charts. While these advancements address some limitations of Sirius Desktop, Sirius Web still suffers from significant configuration overhead and limited dynamic adaptability, which constrain its effectiveness in highly iterative and fast-paced projects. Furthermore, its component-based architecture, rooted in EMF, lacks the seamless integration of an all-in-one environment, leading to fragmentation and increased complexity for users. 
%In contrast, \jjodel{} offers a fully declarative and reflective architecture, improving usability and governance while simplifying complex modeling tasks, such as metamodel co-evolution. By providing a seamlessly integrated environment, \jjodel{} ensures greater agility and adaptability to rapidly changing requirements, making it particularly well-suited for collaborative and dynamic modeling scenarios.

Sprotty\footnote{\url{https://sprotty.org/}} and Gentleman \cite{3417990.3421998} address narrower roles in the MDE ecosystem. Sprotty is a web-based visualization framework focused on rendering lightweight models, while Gentleman provides a flexible projectional editing interface for directly modifying abstract syntax trees. Although these tools excel in specific use cases, they lack deeper integration with metamodel evolution and collaboration workflows. \jjodel{} combines the simplicity and lightweight design of tools like Sprotty and Gentleman with advanced capabilities such as automated syntax co-evolution, making it a more comprehensive solution.


A recent paper by Metin et al.~\cite{Metin2025} explores the challenges and lessons learned from working with GLSP and developing several GLSP-based modeling tools. The authors present a reusable reference architecture designed to facilitate the development and operation of GLSP-based web modeling tools. The reference architecture is then instantiated using BigUML as a running example. The paper highlights the procedural approach, key lessons learned, and critical reflections on the challenges and opportunities associated with using GLSP. In contrast to \jjodel{}, which natively supports cloud-based modeling, GLSP-based web modeling tools require the ad hoc implementation of the domain language, adding an additional layer of complexity.

HyperGraphOS \cite{Ceravola2024} represents an ambitious vision, integrating DSLs, modeling, and task execution into a unified system. Its infinite workspace concept and AI-enhanced adaptability demonstrate significant innovation, but these features come at the expense of usability for less technical users. 
%In contrast, \jjodel{} offers a more balanced approach, focusing on accessibility and iterative workflows, making it ideal for teams that prioritize usability and collaboration over extensive technical features.

Many of the platforms surveyed possess industrial-grade or commercial-technology readiness levels (TRL), making them well suited for high-stakes, large-scale projects. In contrast, \jjodel{} remains an academic endeavor, designed to explore and adapt the principles and solutions commonly found in large-budget projects and low-code platforms. Translating these into accessible and innovative approaches, \jjodel{} makes a unique contribution to the field. However, this academic focus also highlights its limitations in directly competing with fully commercialized solutions, particularly in terms of scalability, robustness, and enterprise-level support.
%
However, the initial adoption of \jjodel{} in academic settings has yielded promising results. Although a systematic evaluation of its impact on teaching and learning outcomes has not yet been conducted, preliminary feedback has been highly encouraging. Students have demonstrated significant engagement and understanding, even in the absence of extensive learning resources at the time of the course\footnote{For more details on the survey results, see \url{https://www.jjodel.io/student-survey/}.}. These early findings indicate that \jjodel{} has the potential to become a valuable educational tool, particularly in model-driven engineering courses.

\section{Conclusions and Future Work}
\label{sec:conclusion}
\section*{Conclusion}
This paper aims to enhance our understanding of the computational complexity of computing various Shapley value variants. We found that for various ML models --- including decision trees, regression tree ensembles, weighted automata, and linear regression --- both local and global interventional and baseline SHAP can be computed in polynomial time under HMM modeled distributions. This extends popular algorithms, such as TreeSHAP, beyond their empirical distributional scope. We also establish strict complexity gaps between the various SHAP variants (baseline, interventional, and conditional) and prove the intractability of computing SHAP for tree ensembles and neural networks in simplified scenarios. Overall, we present SHAP as a versatile framework whose complexity depends on four key factors: \begin{inparaenum}[(i)] \item model type, \item SHAP variant, \item distribution modeling approach, \item and local vs. global explanations\end{inparaenum}. We believe this perspective provides deeper insight into the computational complexity of SHAP, paving the way for future work.




%We believe that our framework provides a more intricate understanding of SHAP computation complexity across different models, distributions, and variants, paving the way for further research.

Our work opens promising directions for future research. First, expanding our computational analysis to other SHAP-related metrics, such as asymmetric SHAP~\citep{frye20} and SAGE~\citep{covert2020understanding}, would be valuable. Additionally, we aim to explore more expressive distribution classes and relaxed assumptions beyond those in Section \ref{sec:tractable} while maintaining tractable SHAP computation. Finally, when exact computation is intractable (Section \ref{sec:intractable}), investigating the approximability of SHAP metrics through approximation and parameterized complexity theory~\citep{downey2012parameterized} is an important direction.

%Our work opens several promising avenues for future research on the computational properties of explainable AI methods, with a particular focus on SHAP. First, it would be interesting to broaden the computational analysis conducted in this work to include other popular SHAP-related metrics in the literature, such as asymmetric SHAP \cite{frye20} and SAGE \cite{covert2020understanding}. Also, in the future, we aim to explore more expressive distribution classes and relaxed distributional assumptions—extending beyond those examined in Section \ref{sec:tractable} —that still yield tractable SHAP computation. Finally, when exact computation proves intractable (Section \ref{sec:intractable}), it is worthwhile to theoretically investigate the question of the approximability of computing the SHAP metrics across various configurations, through the lens of approximation and parametrized complexity theory \cite{arora2009computational}.

%This paper aims to deepen our understanding of the computational complexity involved in obtaining different Shapley value variants. We found that for a variety of ML models, including decision trees, tree ensembles for regression, weighted automata, and linear regression models — computing both local and global interventional and baseline SHAP can be done in polynomial time when distributions are modeled by HMMs. This extends the distributional scope of popular algorithms like TreeSHAP, which is limited to empirical distributions. Additionally, we demonstrate a strict complexity gap between SHAP variants, showing that interventional and baseline SHAP can be strictly easier to compute than conditional SHAP. Despite these positive results, we uncovered intractability for various SHAP variants in neural networks and tree ensembles. Finally, we provided generalized complexity relations across SHAP variants. We believe that our framework offers a deeper understanding of the complexity involved in computing SHAP across various variants, models, distributions, as well as in both local and global computations, laying the groundwork for future research.

\bibliographystyle{abbrv}  
\bibliography{bib.bib}   






\end{document}
% end of file template.tex

