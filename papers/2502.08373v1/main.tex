\documentclass[preprint,12pt,authoryear]{elsarticle}

\usepackage{amssymb}
\usepackage{amsmath}
\usepackage{algorithmicx}
\usepackage{algorithm}
\usepackage{array}
\usepackage{url}
\usepackage{multirow}
\usepackage{booktabs}
\usepackage{algpseudocode}
\usepackage{hyperref}

\journal{Expert Systems with Applications}

\begin{document}

\begin{frontmatter}


\title{Uncertainty Aware Human-machine Collaboration in Camouflaged Object Detection} %% Article title


\author[label1]{Ziyue Yang}\ead{zyyang@hdu.edu.cn}
\author[label1]{Kehan Wang}\ead{242050376@hdu.edu.cn}
\author[label1]{Yuhang Ming}\ead{yuhang.ming@hdu.edu.cn}
\author[label1]{Yong Peng}\ead{yongpeng@hdu.edu.cn}
\author[label1]{Han Yang}\ead{43008@hdu.edu.cn}
\author[label2]{Qiong Chen}\ead{chenqiong@cethik.com}
\author[label1]{Wanzeng Kong\corref{cor1}}\ead{kongwanzeng@hdu.edu.cn}


\affiliation[label1]{organization={College of Computer Science, Hangzhou Dianzi University},
             addressline={1158 Baiyang Road}, 
             city={Hangzhou},
             postcode={310018}, 
             state={Zhejiang},
             country={China}}

\affiliation[label2]{organization={Hangzhou Hikvision Digital Technology Company},
             addressline={555 Qianmo Road}, 
             city={Hangzhou},
             postcode={310051}, 
             state={Zhejiang},
             country={China}}

%% Abstract
\begin{abstract}
Camouflaged Object Detection (COD), the task of identifying objects concealed within their environments, has seen rapid growth due to its wide range of practical applications. A key step toward developing trustworthy COD systems is the estimation and effective utilization of uncertainty. In this work, we propose a human-machine collaboration framework for classifying the presence of camouflaged objects, leveraging the complementary strengths of computer vision (CV) models and noninvasive brain-computer interfaces (BCIs). Our approach introduces a multiview backbone to estimate uncertainty in CV model predictions, utilizes this uncertainty during training to improve efficiency, and defers low-confidence cases to human evaluation via RSVP-based BCIs during testing for more reliable decision-making. We evaluated the framework in the CAMO dataset, achieving state-of-the-art results with an average improvement of 4.56\% in balanced accuracy (BA) and 3.66\% in the F1 score compared to existing methods. For the best-performing participants, the improvements reached 7.6\% in BA and 6.66\% in the F1 score. Analysis of the training process revealed a strong correlation between our confidence measures and precision, while an ablation study confirmed the effectiveness of the proposed training policy and the human-machine collaboration strategy. In general, this work reduces human cognitive load, improves system reliability, and provides a strong foundation for advancements in real-world COD applications and human-computer interaction. Our code and data are available at: https://github.com/ziyuey/Uncertainty-aware-human-machine-collaboration-in-camouflaged-object-identification.
\end{abstract}

%%Graphical abstract
\begin{graphicalabstract}
\includegraphics[width=\textwidth]{Figure_1.pdf}
\end{graphicalabstract}

%%Research highlights
\begin{highlights}
\item Introduces a novel framework combining computer vision (CV) models with RSVP-based brain-computer interfaces (BCIs) to enhance camouflaged object detection (COD).
\item Develops a multiview backbone that estimates prediction confidence, leveraging strong and weak augmentations to assess uncertainty and guide model training.
\item Incorporates RSVP-based BCIs to improve detection accuracy, where low-confidence cases from the CV model are re-evaluated using human cognitive responses.
\item Achieves 4.56$\%$ higher balanced accuracy (BA) and 3.66$\%$ higher F1 score on the CAMO dataset compared to existing methods, with the best cases reaching 95.60$\%$ BA and 95.49$\%$ F1 score.
\item Proposes a selective human intervention strategy, deferring only uncertain cases to human evaluation, significantly reducing cognitive effort while maintaining high reliability.
\end{highlights}

%% Keywords
\begin{keyword}
Camouflaged Object Detection, Human-Machine Collaboration, Brain-Computer Interface, Uncertainty Estimation, Computer Vision.
\end{keyword}


\cortext[cor1]{Corresponding author.}

\end{frontmatter}

\section{Introduction}
\label{sec1}
Rapid adoption of deep learning has highlighted concerns about the robustness and transparency of neural network predictions. Addressing these concerns is critical for the advancement of trustworthy artificial intelligence (AI) \cite{kaur2022trustworthy}. A key aspect of building trust is to enable models to assess and communicate their own uncertainty \cite{10.1145/3675392,gawlikowski2023survey}, which can guide decisions to defer to human operators or seek additional data in complex scenarios. In particular, in tasks where humans and AI have complementary strengths, effective human-machine collaboration represents a promising trend in the evolution of digital societies. However, how to leverage uncertainty quantification for such collaboration, and its impact, remains underexplored.
Camouflaged Object Detection (COD) is a well-suited task for such synergy. COD focuses on identifying objects concealed within their environments, making it both challenging and intriguing \cite{lv2023toward}. This field has grown rapidly due to its practical applications, such as defect detection in manufacturing \cite{xiong2021attention}, pest control in agriculture \cite{rustia2020application}, segmentation of lesions in medical diagnosis \cite{fan2020pranet}, pedestrian detection in nighttime environments\cite{yang2024dual}, and even creative pursuits such as image blending \cite{suo2021neuralhumanfvv}. Advances in image-level camouflaged object segmentation have been driven by models such as DPSNet \cite{li2024dpsnet}, JCNet \cite{jiang2023camouflaged}, SINet-V2 \cite{fan2021concealed}, and DGNet \cite{ji2023deep}, supported by recognized datasets and benchmarks \cite{bi2021rethinking}. COD tasks involve two key components: identifying whether camouflaged objects are present and segmenting them when they are. However, much of the current research focuses on the segmentation stage, assuming the model can produce a continuous mask map where a black saliency map signifies the absence of a camouflaged object. However, since camouflaged objects are not guaranteed to be present, a critical first step is to detect their presence before proceeding to segmentation. While precise localization is vital for applications like medical diagnostics, in tasks such as rescue operations or pest monitoring, detecting the presence of camouflaged objects takes precedence. Therefore, our study centers on binary classification to determine the presence or absence of camouflaged objects.

Despite rapid progress in COD, concerns persist about the safety and reliability of computer vision (CV) models. Deep learning models often function as black boxes \cite{hassija2024interpreting} and can struggle with challenges such as small targets, incomplete objects, and complex backgrounds (e.g. noise, obstructions, or shadows) \cite{bi2021rethinking}. Teaching models to admit uncertainty, essentially saying 'I don't know', remains a significant challenge.

In contrast, the human brain excels at adapting to diverse environments, recognizing subtle patterns, and identifying hidden objects in complex scenarios, such as low-light conditions or cluttered backgrounds. This adaptability allows humans to detect hidden or camouflaged targets with high accuracy, even when features are partially occluded. However, manual search for such targets is time-consuming. Non-invasive brain-computer interfaces (BCIs) offer a novel solution to this challenge \cite{kim2019high}. When individuals encounter rare targets in visual sequences, their brain activity generates event-related potentials (ERPs), particularly the P300 component, which can help detect targets. Researchers have exploited this phenomenon by combining rapid serial visual presentation (RSVP) paradigms with BCIs technology to evoke ERPs in response to visual stimuli, allowing efficient classification of target images \cite{zhang2020benchmark}--even under challenging conditions where targets are camouflaged and hidden \cite{lian2023eeg, zhou2024rsvp}. 

In this study, we propose a novel framework for human-machine collaboration in COD, leveraging model uncertainty as the bridge between CV models and human intervention. Specifically, CV models handle the bulk of images, exploiting their ability to process large volumes of data in parallel, while humans, through BCIs, provide instinctive responses to challenging or unusual images flagged by model uncertainty. To the best of our knowledge, no existing research has been proposed that uses and evaluates the uncertainty-based human-machine partnership in COD. Our contributions include the following.

\begin{itemize}
    \item Multiview backbone for COD: We propose a backbone that evaluates the confidence across multiple views for each image.
    \item Uncertainty-aware training policy: Our training strategy improves CV models by effectively using information from the training set.
    \item Human-machine collaboration paradigm for COD: For samples with low confidence from the CV model, we fuse RSVP-based BCIs results with model predictions to achieve more reliable decision making.
\end{itemize}

\section{Related Work}
\subsection{Uncertainty Quantification}
Uncertainty estimation, or confidence assessment, in neural network predictions has emerged as a major research focus in the machine learning community \cite{smith2024uncertainty}. Current approaches to uncertainty modeling typically fall into three categories: Monte Carlo Dropout \cite{neal2012bayesian, Moreau_2022_WACV, gal2017concrete, kang2023active}, the Bootstrap model \cite{osband2016deep}, and the Gaussian Mixture Model \cite{9666964, zhang2019short}. These methods have been extensively explored in various domains, demonstrating promising results \cite{abdar2021review}. However, most models merely display uncertainty without leveraging it to guide further actions. To address this gap, recent advances have begun to incorporate uncertainty into training strategies \cite{li2023disc,cordeiro2023longremix}. Although promising, these cutting-edge efforts have focused primarily on tasks that involve noisy label learning. For standard supervised learning tasks, the effectiveness of uncertainty-informed training strategies remains unclear. In this work, we extend the confidence learning method based on two views introduced in \cite{li2023disc} and apply it to the COD problem. Our approach not only integrates uncertainty into the model's learning process, but also flexibly delegates uncertain samples to human-based RSVP systems for enhanced decision making.

\subsection{Camouflaged Object Detection (COD)}
In recent years, numerous deep learning-based COD models have been developed \cite{liang2024systematic}, while recent research has started to place more emphasis on uncertainty. Yi Zhang et al. introduced PUENet \cite{10159663}, which uses a Bayesian conditional variational auto-encoder for predictive uncertainty estimation. Yixuan Lyu et al. \cite{10183371} proposed the Uncertainty-Edge Dual Guide model, which combines probabilistic uncertainty with deterministic edge information for accurate COD. Jiawei Liu et al. \cite{9706783} developed a confidence-based COD framework with dynamic supervision, producing both camouflage masks and aleatoric uncertainty estimates, showing superior performance. Fan Yang et al. \cite{9710683} integrated Bayesian learning with Transformer reasoning, leveraging both deterministic and probabilistic information to improve detection accuracy. Furthermore, Aixuan Li et al. \cite{9578707} proposed an adversarial learning network for higher-order similarity measures and confidence estimation. Current research mainly addresses uncertainty in segmentation tasks, focusing on generating confidence maps for boundary distinction, while this study aims to model the uncertainty in identifying object presence to enhance classification accuracy.

\subsection{RSVP-based BCIs for Target Detection}
RSVP-based BCIs have received significant attention in recent years, particularly in the domain of target detection, due to their efficiency in processing rapid visual stimuli and eliciting robust neural responses such as potentials related to P300 events. Research advancements have focused on optimizing RSVP paradigms to enhance system performance, with notable contributions including the use of adaptive parameter tuning and hybrid paradigms that integrate steady-state visual evoked potentials to improve detection accuracy and user experience \cite{jalilpour2020novel}. Novel electroencephalogram(EEG) decoding algorithms, such as deep learning frameworks that take advantage of convolutional neural networks \cite{santamaria2020eeg} and attention mechanisms\cite{wang2020linking}, have further enhanced classification performance, while collaborative approaches of multiple users have demonstrated the potential for improved accuracy through collective neural signal analysis \cite{9931160}. The introduction of benchmark data sets has standardized the evaluation of algorithms and facilitated reproducible research \cite{zhang2020benchmark}. Furthermore, the integration of multimodal data, including EEG and eye tracking, has shown promise in addressing signal noise and enhancing target detection reliability \cite{mao2023cross}, cementing RSVP-BCIs as a crucial interface for bridging neuroscience and real-world applications. The most related work to ours is by Yujie Cui et al.\cite{cui2022dynamic}, who proposed a human-computer fusion method called Dynamic Probability Integration for nighttime vehicle detection. Their approach uses a probability assignment method to assign classification weights between different information sources, which requires full human participation in the EEG-based RSVP task. In contrast, our model reduces human effort by using uncertainty to guide collaboration, with humans only evaluating high-uncertainty samples from the CV model, thus improving efficiency.

\section{Method}
\begin{figure*}[ht]
    \centering
    \includegraphics[width=\textwidth]{Figure_1.pdf} % Replace with your image file name
    \caption{Overall framework of this study. During training, the dataset is split into high and low-confidence sets, with augmentations applied to low-confidence samples and the split updated per epoch; during testing, high-uncertainty samples are classified using the RSVP program, while high-confidence samples are handled by the CV model, enhancing COD performance.}
    \label{fig:overall}
\end{figure*}
The framework of this paper is illustrated in Figure
~\ref{fig:overall}. In the training stage, the CV model is pretrained on the complete dataset. The data is then divided into a high confidence set (\(X_{hc}\)) and a low confidence set (\(X_{lc}\)) based on uncertainty, the split being updated once per epoch. Weak and strong augmentations are applied to \(X_{lc}\), creating \(X_{lc'}\), while \(X_{hc}\) is oversampled to match the size of \(X_{lc'}\). In the testing phase, high-uncertainty samples are transferred to the EEG-based RSVP program for classification, while high-confidence samples are predicted using the CV model. This two-stage process combines the strengths of CV models and RSVP-based BCIs systems to enhance COD performance.

\subsection{Dataset}
The data set used in this study consists of 2,500 images, evenly divided into camouflage target images and background images. Our data set is sourced from the publicly available CAMO dataset, which includes 1,250 camouflage target images. Using the ground truth map for each image, we generated paired camouflage target and background images through a combination of manual and automated methods. The resulting data set is available on the provided GitHub link. Although the CAMO-COCO dataset was considered, its background images differ substantially from the camouflage target images, making it less aligned with our focus. This study emphasizes scenarios where camouflage targets are embedded within similar backgrounds, reflecting more realistic application contexts.

\subsection{CV Model}
\subsubsection{Multi-view based Uncertainty Estimation}
We propose a backbone model that evaluates confidence in multiple views for each image. For each sample \((x, y)\), we apply one weak augmentation to \(x\), resulting in \(x_w\), and \(n\) strong augmentations, resulting in \(\{x_{s1}, x_{s2}, \dots, x_{sn}\}\). To compute uncertainty, we used the cross-entropy (CE) between two distributions \(p\) and \(q\) defined as:
\begin{equation}
\text{CE}(p, q) = -\sum_{i=1}^{C} p_i \log(q_i),
\label{eq:ce}
\end{equation}
where \(C\) is the number of classes, \(p_i\) is the ground truth probability of class \(i\), and \(q_i\) is the predicted probability of class \(i\). Hence, uncertainty is calculated as the mean cross-entropy between the weakly augmented sample and each strongly augmented sample:
\begin{equation}
\text{Uncertainty} = \frac{1}{n} \sum_{j=1}^{n} \text{CE}(p_w, p_{sj}),
\label{eq:uncertainty}
\end{equation}
where $n$ denotes the number of strong augmentations, while $x_w$ and $x_{sj}$ are processed by the trained CV model to obtain $p_w$ and $p_{sj}$, respectively. This approach quantifies uncertainty by assessing the consistency between weakly and strongly augmented views of the same sample.

\subsubsection{Strong and weak augmentation}
In this study, strong and weak data augmentations serve two primary purposes: evaluating multiview-based uncertainty and enhancing the model's robustness during training.

Weak augmentations include operations such as random horizontal flipping, slight rotation, and random cropping. In contrast, strong augmentations involve more significant perturbations, such as cropping, color transformations, image quality adjustments, occlusion, and composite augmentations to simulate complex scene variations. The specific augmentation method for both weak and strong augmentations is selected randomly for each iteration.

\begin{algorithm}[H]
\caption{Uncertainty-Aware Training Policy}
\label{alg:training_procedure}
\begin{algorithmic}[0]
    \State \textbf{Input:} Dataset $W$, Total Epochs $E$, Batch Size $B$, Warm-up Epochs $\text{Warm}$, Ramp-up Length $\text{rampup\_length}$, Augmentation Strength $M$, Augmentation Times $N$, Clean Dataset Filtering Method $\text{Method\_c}$, Consecutive Clean Rounds $t$

    \State Load pre-trained weights from ImageNet
    \State Initialize epoch counter $e \gets 0$
    
    \While{$e < E$}
    
        \For{$i = 1$ \textbf{to} $|W|$}
            \State Perform warm-up and compute loss using cross-entropy loss: $l(x, y) = \text{CE}(f(x), y)$
        \EndFor
    
        \If{$e > \text{Warm}$}
            \State Build high-confidence $X_{hc}(K)$ and low-confidence $X_{lc}(K)$ sets
            \State $X_{hc}(K) \gets \text{Samples from last } t \text{ epochs of training}$
            \State $X_{lc}(K) \gets \text{All training samples} \setminus X_{hc}(K)$
            \State $num\_iter \gets \frac{|X_{hc}(K)|}{B}$
    
            \For{$iter = 1$ \textbf{to} $num\_iter$}
                \State Select high-confidence samples $\{(X, Y)\} \sim X(k)$
                \State Select low-confidence samples $\{(U, Y)\} \sim U(k)$
    
                \For{$b = 1$ \textbf{to} $B$}
                    \State $x_b  \gets \text{Batch samples for high-confidence set}$
                    \State $u_{b1} \gets  \text{StrongDataAugment}(u_b)$
                    \State $u_{b2} \gets  \text{WeakDataAugment}(u_b)$
                    \State $r \gets \text{clip}\left(\frac{e - \text{Warm}}{\text{rampup\_length}}, 0, 1\right) \times \lambda_u$
                    \State $loss = l(x_b, y_b) + [l(u_{b1}, y_b) + l(u_{b2}, y_b)] \times r$
                \EndFor
            \EndFor
        \EndIf
        
        \State Update high- and low-confidence sets using \texttt{Method\_c}
        \State Increment epoch: $e = e + 1$
    \EndWhile
    
    \State \textbf{Output:} Model Parameters $\theta$
\end{algorithmic}
\end{algorithm}

The uncertainty-aware training policy outlined in Algorithm 1 aims to enhance model robustness by leveraging high-confidence and low-confidence samples during training. Initially, during the warm-up phase, the model is trained using standard loss functions. Once the warm-up is complete, the training data is divided into high-confidence and low-confidence sets. High-confidence samples \(X_{hc}(K)\) are selected from the last training epochs $t$, while low-confidence samples \(X_{lc}(K)\) are derived from the remaining data. The model is then trained on these two sets, with low-confidence samples undergoing strong and weak data augmentations (detailed in Section 3.2.2 ) to improve generalization. For confident set selection, \texttt{Method\_c} includes five strategies based on uncertainty estimation: (1) splitting Low-confidence and High-confidence images in a 1:2 ratio (Ratio 1:2); (2) splitting in a 2:1 ratio (Ratio 2:1); (3) dynamically partitioning the samples using a threshold (details: dividing data into intervals using a 0.1 threshold, then traversing each interval in order of increasing accuracy, selecting the first interval with a sample and accuracy lower than the overall accuracy, and using the median of that interval as the dynamic threshold) (Dynamic Threshold); (4) labeling images with consistent predicted and ground-truth labels after augmentation as High confidence, otherwise labeling them as Low-confidence (Consistent Labeling); and (5) labeling images as High confidence if at least one prediction matches the ground truth after augmentation, otherwise labeling them as Low confidence (At Least One Match). A key feature of the policy is the gradual increase in weight loss for low-confidence samples, controlled by the factor $r$, which evolves throughout training. This dynamic weighting helps stabilize training by initially focusing on high-confidence samples and progressively incorporating low-confidence data. The subsets of high- and low-confidence samples are updated after each epoch using a filtering method to ensure data quality. This strategy effectively balances the influence of clean, high-confidence data with more uncertain, low-confidence data, leading to improved model performance and robustness.


\subsection{RSVP-based EEG Model}\label{stage-2-brain-machine-synergy}

\subsubsection{Participants and Data
Recording}
The study was reviewed and approved by the 	
Ethics Committee of Second Affiliated Hospital of Zhejiang University, College of Medicine and the protocol number is IRB-2024-1535. Signed informed consent was obtained from each participant.
A total of 8 participants (mean age: 24.10 years) were recruited for
the brain-machine collaborative RSVP target detection paradigm study.
All participants had normal or corrected vision and no neurological
problems. Before the experiment, each participant was informed of the
potential risks and signed a written informed consent form.\\
EEG data were recorded using the Synamps2 system (64 channels,
NeuroScan, Inc.) at a sampling rate of 1000 Hz. In this study, 62
electrodes were used to record EEG signals, following the
international 10-20 electrode placement system, with the reference
electrode placed at the vertex. Before data recording, the impedance
of the electrodes was measured and adjusted to ensure it remained below
25 k$\Omega$. EEG data were initially filtered using a finite impulse response filter with a frequency range between 0.1 and 40 Hz and
then resampled to 250 Hz for classification. Segments of events of
one second were extracted from the data starting from the onset of the
stimulus.

\subsubsection{RSVP Paradigm Design}
\begin{figure}[ht]
    \centering
    \includegraphics[width=\linewidth]{Figure_2.pdf} % Replace with your image
    \caption{RSVP Paradigm Design. At the start of each trial, a fixation
cross appeared in the center of the screen. The stimuli were then presented
at a frequency of 1 Hz. The participants were given ample rest time between
blocks. The experiment ensured that there were at least three nontarget
images between any two target images. All stimulus images were displayed
on a 512 × 512 resolution monitor with a refresh rate of 60 Hz.}
    \label{fig:RSVP}
\end{figure}
The experimental setup is shown in Figure~\ref{fig:RSVP}. In the experiment,
participants were seated in a quiet room and instructed to identify
camouflage targets within a sequence of stimuli. Original frames
were used to present the stimuli. An EEG amplifier was employed to
capture the participants\textquotesingle{} brain activity, with no
button pressing required when a target was detected.\\
The experimental procedure and paradigm parameters are depicted in the
figure. The experiment consisted of five blocks, each comprising 11 trials.
In each block, 75 target images and 530 untarget images were presented. Before the experiment began, a set
of images containing camouflage target and background images was
presented to the participants as a guideline, informing them of the
target they needed to identify. At the start of each trial, a fixation
cross appeared in the center of the screen. The stimuli were then presented
at a frequency of 1 Hz. The participants were given ample rest time between
blocks. The experiment ensured that there were at least three nontarget
images between any two target images. All stimulus images were displayed
on a 512 × 512 resolution monitor with a refresh rate of 60 Hz.

\subsubsection{Brain-Machine Collaboration}
In the test phase, samples with high uncertainty from the CV model are sent to human evaluation. These samples are treated as "potential targets." To construct the test target sequence, at least three nontarget images from the training set are inserted between each pair of target images, ensuring activation of the P300 component during stimulus presentation. An RSVP-based EEG model, trained in the training set, is then used to detect the presence of targets in the test set. The predictions for high-uncertainty samples are replaced with the output of the RSVP-based model, enabling effective human-machine collaboration.

\section{Results}
%-------------------------------------------------------------------------
In this section, we evaluate the effectiveness of the proposed method. The data set was divided into training and testing sets with a 9: 1 ratio and the training set was further divided into training and validation subsets, also with a 9: 1 ratio. During the experiments, we used early stopping with a patience parameter set to 10. To ensure the stability and reliability of the results, we performed experiments using five different random seeds: 37, 12, 6, 99, and 123. We applied five strong augmentations and one weak augmentation. The reported results are the mean and standard deviation for the five runs. All experiments were performed on a GeForce RTX 4090 GPU. The evaluation of our experimental results was based on balanced accuracy (BA) and F1 score, with their respective formulas as follows:

\begin{equation}
\text{BA} = \frac{1}{2} \left( \frac{TP}{TP + FN} + \frac{TN}{TN + FP} \right),
\end{equation}
where \( TP \) is the number of true positives, \( FN \) is the number of false negatives, \( TN \) is the number of true negatives and \( FP \) is the number of false positives.

\begin{equation}
\text{F1} = 2 \times \frac{\text{Precision} \times \text{Recall}}{\text{Precision} + \text{Recall}},
\end{equation}
where \(\text{Precision} = \frac{TP}{TP + FP}\) and \(\text{Recall} = \frac{TP}{TP + FN}\).


\subsection{Results for different training policies}

In designing the training strategy, we compared different data augmentation and confident set selection methods, using the Swin Transformer (SwinT) as the backbone network. For data augmentation, we tested three strategies: (1) applying strong and weak augmentations separately to high-confidence and low-confidence images; (2) augmenting only high-confidence images (H-only); and (3) augmenting only low-confidence images (L-only). Details of the confident set selection approach are provided in Section 3.2.3. The results, shown in Table~\ref{2}, reveal that augmenting only low-confidence samples, along with a fixed ratio of 2:1 between low- and high-confidence images, yielded the best performance. This configuration achieved a BA of 89.92\% and an F1 score of 90.40\%. These findings suggest that it is crucial to effectively mine and utilize low-confidence samples during training. Based on these results, we selected the L-only augmentation strategy combined with a fixed ratio of 2:1 for subsequent experiments.

\begin{table}[ht]
\centering
\footnotesize
\caption{Comparison of Different Data Augmentation Strategies and Confident Set Selection Policies}
\label{1}
\begin{tabular}{clcc}
\toprule
\textbf{\parbox{2.5cm}{Augmentation}} & \textbf{\parbox{2.5cm}{Selection}} & \textbf{BA (\%)} $\uparrow$ & \textbf{F1 (\%)} $\uparrow$ \\
\midrule
Both & Ratio 1:2 & 89.36 $\pm$ 1.15 & 89.73 $\pm$ 1.20 \\
Both & Ratio 2:1 & 88.08 $\pm$ 1.39 & 88.83 $\pm$ 1.13 \\
Both & Dynamic Threshold & 88.64 $\pm$ 2.21 & 89.48 $\pm$ 1.92 \\
Both & Consistent Labeling & 88.56 $\pm$ 1.51 & 88.80 $\pm$ 1.69 \\
Both & At Least One Match & 89.04 $\pm$ 1.40 & 89.51 $\pm$ 1.25 \\
H-only & Ratio 1:2 & 88.56 $\pm$ 1.66 & 89.25 $\pm$ 1.40 \\
H-only & Ratio 2:1 & 88.96 $\pm$ 0.60 & 89.46 $\pm$ 0.56 \\
H-only & Dynamic Threshold & 89.84 $\pm$ 1.56 & 90.27 $\pm$ 1.42 \\
H-only & Consistent Labeling & 88.08 $\pm$ 0.95 & 88.52 $\pm$ 1.01 \\
H-only & At Least One Match & 88.24 $\pm$ 1.45 & 88.62 $\pm$ 1.47 \\
L-only & Ratio 1:2 & 88.80 $\pm$ 1.16 & 89.25 $\pm$ 1.21 \\
\textbf{L-only} & \textbf{Ratio 2:1} & \textbf{89.92 $\pm$ 0.76} & \textbf{90.40 $\pm$ 0.64} \\
L-only & Dynamic Threshold & 89.68 $\pm$ 0.91 & 90.01 $\pm$ 0.93 \\
L-only & Consistent Labeling & 88.16 $\pm$ 0.73 & 88.78 $\pm$ 0.75 \\
L-only & At Least One Match & 87.52 $\pm$ 1.42 & 88.05 $\pm$ 1.31 \\
\bottomrule
\end{tabular}
\end{table}



\begin{table}[ht]
\centering
\footnotesize  % 使用 \footnotesize 而不是 \small
\caption{Comparison with Competing Methods on Our Dataset}  % 使用标题大小写
\label{2}
%与最先进技术(SOTA)方法在我们的数据集上的比较。表中选取了两个伪装目标检测(COD)模型,并使用白色像素点作为阈值进行图像分割。同时,选取了六个主流的计算机视觉(CV)模型进行对比分析。我们还展示了在本方法中,每个步骤对性能提升的贡献。
\begin{tabular}{lcc}
\toprule
\textbf{Model} & \textbf{BA(\%)} $\uparrow$ & \textbf{F1(\%)} $\uparrow$ \\
\midrule
DGNet \cite{ji2023gradient}& 78.80 & 76.10 \\
SINet-V2 \cite{fan2021concealed}& 72.40 & 69.10 \\
ResNet-18 \cite{he2016deep}& 86.00 $\pm$ 0.88 & 86.44 $\pm$ 0.72 \\
ResNeXt-50 \cite{xie2017aggregated}& 86.56 $\pm$ 1.51 & 87.16 $\pm$ 1.13 \\
DenseNet121 \cite{huang2017densely}& 87.76 $\pm$ 0.60 & 88.12 $\pm$ 0.84 \\
EfficientNetB0 \cite{tan2019efficientnet}& 84.40 $\pm$ 1.52 & 84.44 $\pm$ 1.47 \\
SwinT \cite{Liu_2021_ICCV}& 88.00 $\pm$ 1.77 & 88.83 $\pm$ 1.54 \\
ViT-B16 \cite{dosovitskiy2020image}& 86.40 $\pm$ 1.13 & 86.56 $\pm$ 1.25 \\
\textbf{SwinT + Training Policy} & \textbf{89.92 $\pm$ 0.76} & \textbf{90.40 $\pm$ 0.64} \\
\textbf{SwinT + Training Policy + RSVP(Mean)} & \textbf{92.56 $\pm$ 1.12} & \textbf{92.49 $\pm$ 1.17} \\
\textbf{SwinT + Training Policy + RSVP(Best)} & \textbf{95.60$\pm$0.15} & \textbf{95.49$\pm$0.16}\\
\bottomrule
\end{tabular}
\end{table}

\subsection{Comparison with Existing Methods and Ablation Study}
We evaluated our approach against two categories of existing methods, as summarized in Table~\ref{2}. The first category includes camouflaged object segmentation models, represented by DGNet and SINet-V2. These models generate completely black output maps for background images, requiring a dynamic threshold (optimized on the training set) to analyze the proportion of white pixels in the Ground Truth image and detect camouflaged targets. The second category comprises widely used CV models, including CNN-based architectures such as ResNet-18, ResNeXt-50, and DenseNet121; lightweight models such as EfficientNetB0; and transformer-based models, such as the Swin Transformer (SwinT) and Vision Transformer (ViT-B16). 

For these CV models, we perform the following steps: 1) load the models and use pre-trained weights from ImageNet1k; 2) freeze the pre-trained weights; 3) retrieve the number of input features for the classification head; 4) replace the original classification head with a new fully connected layer consisting of four linear transformation layers, which map to 256, 32, and 8 dimensions, and finally output two classes; 5) Unfreeze the parameters of the new classification head. The key findings of the experiments include the following.

\begin{itemize}
\item Performance of COD Models: COD models, such as DGNet and SINet-V2, achieved approximately 75\% balanced accuracy (BA) in detecting the presence or absence of camouflaged targets. Although effective in edge detection and segmentation, their reliance on local pixel-level features limits their ability to perform well in tasks that require a broader contextual understanding.

\item Performance of CV Models: Among the CV models tested, SwinT exhibited the best performance in our data set and was selected as the backbone model for our approach.

\item Impact of Training Policy: The confidence-based training policy applied to SwinT improved BA by 1.92\% and the F1 score by 1.57\%, demonstrating the ability of the policy to enhance model robustness and precision.

\item RSVP Integration: Combining RSVP with the training policy resulted in additional improvements of 2.64\% in BA and 2.09\% in F1 score on average. The best participant achieved a remarkable balanced accuracy of 95.60\% and an F1 score of 95.49\%, setting a new state-of-the-art in this evaluation.
\end{itemize}
%-------------------------------------------------------------------------
\subsection{Results for RSVP-based BCIs}
To determine the optimal EEG model, we compared several leading EEG analysis methods. EEGNet, designed specifically for EEG signal classification, demonstrated outstanding performance. PLNet leverages phase-locking features of Event-Related Potentials (ERPs) for spatio-temporal feature extraction, excelling in single-session RSVP EEG classification. PPNN, a pyramid-structured parallel neural network, captures multiscale spatio-temporal features, improving classification. EEG-Inception adapts computer vision concepts to ERP detection, improving classification accuracy for ERP-based BCIs. LMDA-Net (LMDA) combines channel and depth attention modules to improve classification by integrating multidimensional features. EEG-Conformer (Conformer) combines convolutional networks and transformers, capturing both local and long-range dependencies, thus improving classification performance.

Table~\ref{3} shows the BA and F1 scores for each model across all participants. As seen in Table~\ref{3}, EEGNet performed the best for six participants, while Conformer excelled for two. Table~\ref{4} provides the mean performance statistics for each model. Among all models, EEGNet achieved the highest performance in the RSVP-based BCIs task, with a BA of 76.76\% and the highest classification precision. Although PLNet, PPNN and EEG-Inception had a lower accuracy, LMDA and Conformer showed better F1 scores but did not outperform EEGNet in accuracy. Based on these results, we selected EEGNet as the optimal EEG model for our task.

\begin{table}[H]
\centering
\caption{Comparison of Classification Accuracy Across Different EEG Models for Various Participants}  % 使用标题大小写
\footnotesize  % 使用 \footnotesize 而不是 \small
\label{3}
\setlength{\tabcolsep}{1mm}  % 可调整列之间的距离
\begin{tabular}{ccccccc}  % 这里的 {ccccccc} 是列格式
\toprule
 & \multicolumn{2}{c}{Subject 1} & \multicolumn{2}{c}{Subject 2} & \multicolumn{2}{c}{Subject 3} \\
\cmidrule{2-3}\cmidrule{4-5}\cmidrule{6-7}
& BA(\%) $\uparrow$ & $F_1$(\%) $\uparrow$ & BA(\%) $\uparrow$ & $F_1$(\%) $\uparrow$ & BA(\%) $\uparrow$ & $F_1$(\%) $\uparrow$  \\ \midrule
EEGNet & 63.66$\pm$1.33 & 58.26$\pm$1.10 & 85.41$\pm$0.77 & 80.25$\pm$1.31 & 77.77$\pm$1.14 & 69.12$\pm$1.38 \\
PLNet & 60.63$\pm$2.05 & 56.62$\pm$1.61 & 84.02$\pm$1.24 & 78.31$\pm$3.51 & 74.79$\pm$1.55 & 68.80$\pm$2.08 \\
PPNN & 55.54$\pm$1.83 & 54.68$\pm$1.22 & 78.55$\pm$1.44 & 79.58$\pm$1.14 & 71.19$\pm$1.82 & 71.09$\pm$1.61 \\
EEGInception & 59.25$\pm$4.05 & 56.63$\pm$5.76 & 72.61$\pm$6.54 & 74.68$\pm$8.01 & 73.29$\pm$6.13 & 72.56$\pm$3.34 \\
LMDA & 58.41$\pm$1.99 & 59.03$\pm$2.18 & 83.89$\pm$2.50 & 84.48$\pm$1.86 & 69.64$\pm$1.05 & 71.38$\pm$1.20 \\
Conformer & 67.63$\pm$1.91 & 61.26$\pm$1.45 & 88.82$\pm$1.11 & 85.60$\pm$1.76 & 74.67$\pm$2.32 & 71.10$\pm$1.84 \\
\\
\toprule
 & \multicolumn{2}{c}{Subject 4} & \multicolumn{2}{c}{Subject 5} & \multicolumn{2}{c}{Subject 6} \\
\cmidrule{2-3}\cmidrule{4-5}\cmidrule{6-7}
& BA(\%) $\uparrow$ & $F_1$(\%) $\uparrow$ & BA(\%) $\uparrow$ & $F_1$(\%) $\uparrow$ & BA(\%) $\uparrow$ & $F_1$(\%) $\uparrow$ \\ \midrule
EEGNet & 83.17$\pm$1.47 & 75.21$\pm$1.74 & 80.57$\pm$1.62 & 69.59$\pm$1.64 & 79.98$\pm$1.47 & 66.76$\pm$2.18 \\
PLNet & 80.97$\pm$2.14 & 74.72$\pm$1.39 & 78.38$\pm$2.55 & 69.59$\pm$2.20 & 71.96$\pm$2.57 & 64.12$\pm$1.41 \\
PPNN & 77.70$\pm$3.07 & 77.64$\pm$1.65 & 74.02$\pm$1.77 & 71.93$\pm$1.78 & 66.66$\pm$3.04 & 66.32$\pm$2.17 \\
EEGInception & 76.03$\pm$5.85 & 74.61$\pm$4.84 & 72.91$\pm$5.52 & 70.31$\pm$3.91 & 76.40$\pm$6.20 & 66.89$\pm$9.17 \\
LMDA & 74.38$\pm$1.92 & 74.85$\pm$1.68 & 71.84$\pm$2.57 & 71.63$\pm$1.85 & 72.16$\pm$2.07 & 70.80$\pm$1.82 \\
Conformer & 79.09$\pm$1.59 & 77.04$\pm$0.98 & 76.97$\pm$1.92 & 71.85$\pm$1.48 & 75.35$\pm$3.20 & 69.11$\pm$1.96 \\
\\
\toprule
 & \multicolumn{2}{c}{Subject 7} & \multicolumn{2}{c}{Subject 8} \\
\cmidrule{2-3}\cmidrule{4-5}
& BA(\%) $\uparrow$ & $F_1$(\%) $\uparrow$ & BA(\%) $\uparrow$ & $F_1$(\%) $\uparrow$ \\ \cmidrule{1-5}
EEGNet & 70.56$\pm$1.31 & 63.13$\pm$1.31 & 72.98$\pm$1.62 & 65.21$\pm$1.38 \\
PLNet & 62.81$\pm$2.79 & 58.60$\pm$1.06 & 67.38$\pm$2.41 & 62.53$\pm$1.13 \\
PPNN & 62.19$\pm$2.47 & 61.98$\pm$2.08 & 61.89$\pm$1.95 & 61.37$\pm$1.52 \\
EEGInception & 64.01$\pm$5.22 & 61.36$\pm$3.44 & 69.94$\pm$3.79 & 64.48$\pm$7.97 \\
LMDA & 59.63$\pm$1.99 & 60.68$\pm$2.17 & 67.40$\pm$1.78 & 66.53$\pm$1.87 \\
Conformer & 66.58$\pm$2.08 & 63.36$\pm$1.27 & 69.59$\pm$1.31 & 65.61$\pm$1.26 \\
\bottomrule
\end{tabular}
\end{table}





\begin{table}[ht]
\caption{Summary of the Mean Performance Statistics for Each EEG Classification Model}  % 标题大小写
\footnotesize  % 使用 \footnotesize 而不是 \small
\label{4}
\centering
\begin{tabular}{lcc}
\toprule
\textbf{EEG Model} & \textbf{BA(\%)} $\uparrow$ & \textbf{F1(\%)} $\uparrow$ \\
\midrule
\textbf{EEGNet} \cite{lawhern2018eegnet}& \textbf{76.76$\pm$6.92} & \textbf{68.44$\pm$6.64} \\
PLNet \cite{zang2021deep}& 72.61$\pm$8.26 & 66.66$\pm$7.37 \\
PPNN \cite{li2021phase}& 68.46$\pm$8.06 & 68.07$\pm$8.20 \\
EEGInception\cite{santamaria2020eeg} & 70.56$\pm$7.76 & 67.69$\pm$8.60 \\
LMDA \cite{miao2023lmda}& 69.67$\pm$7.93 & 69.92$\pm$7.80 \\
\textbf{Conformer} \cite{song2022eeg}& \textbf{74.83$\pm$7.07} & \textbf{70.62$\pm$7.55} \\
\bottomrule
\end{tabular}
\end{table}



%-------------------------------------------------------------------------
\subsection{Results for Human-machine Collaboration}
%Combine our model with COD models

For human-machine collaboration, we selected the highest proportion of samples with the highest uncertainty and replaced their CV-predicted labels with EEG-predicted labels. The results for different proportions of uncertain samples are shown in Table~\ref{5}. The model performed optimally when the correction proportion was set to 20\%. A lower proportion did not fully capture the advantages of human input, while a higher proportion reduced the contribution of the CV model and significantly increased manual effort. Fortunately, the 20\% correction ratio provided an effective balance, allowing human input without excessive stress. Importantly, for all proportions tested, the model outperformed the baseline CV model (SwinT with our training policy), demonstrating the general benefits of human-machine collaboration.

\begin{table*}[ht]
\caption{Test Set Results Comparing the Replacement of Low-Confidence Image Predictions with EEG-Predicted Labels at Different Uncertainty Ratios} % 标题大小写
\footnotesize  % 使用 \footnotesize 而不是 \small
\label{5}
%在不同不确定性样本比例下,通过使用脑电预测标签替代低置信度图像的计算机视觉(CV)预测标签的比较。实验采用最佳训练策略,即仅对低置信度(Low-confidence)图像进行强弱增强,并按 2:1 的比例分割低置信度和高置信度(High-confidence)图像。表中展示了在此策略下训练的模型在测试集上的实验结果。
\setlength{\tabcolsep}{1mm}  % 调整列之间的距离
\begin{tabular*}{\textwidth}{@{\extracolsep\fill}lcccc}
\toprule
 & \multicolumn{2}{@{}c@{}}{10\%} & \multicolumn{2}{@{}c@{}}{20\%}\\
\cmidrule{2-3}\cmidrule{4-5}
 & BA(\%) & $F_1$(\%) & BA(\%) & $F_1$(\%) \\ \midrule
Subject1 & 91.52$\pm$0.29 & 91.63$\pm$0.28 & 91.52$\pm$0.13 & 91.42$\pm$0.17\\
Subject2 & 92.64$\pm$0.29 & 92.73$\pm$0.28 & 94.64$\pm$0.24 & 94.60$\pm$0.24\\
Subject3 & 91.36$\pm$0.27 & 91.39$\pm$0.27 & 92.00$\pm$0.38 & 91.78$\pm$0.41\\
Subject4 & 91.52$\pm$0.21 & 91.71$\pm$0.21 & 92.72$\pm$0.23 & 92.86$\pm$0.23\\
Subject5 & 92.48$\pm$0.13 & 92.66$\pm$0.13 & 94.24$\pm$0.21 & 94.30$\pm$0.22\\
Subject6 & 92.88$\pm$0.07 & 93.01$\pm$0.07 & 93.28$\pm$0.18 & 93.31$\pm$0.17\\
Subject7 & 91.76$\pm$0.27 & 91.88$\pm$0.28 & 90.96$\pm$0.24 & 90.83$\pm$0.25\\
Subject8 & 92.56$\pm$0.33 & 92.72$\pm$0.32 & 93.04$\pm$0.33 & 93.00$\pm$0.33\\
Mean & 92.09$\pm$0.62 & 92.22$\pm$0.63 & \textbf{92.80$\pm$1.23} & \textbf{92.76$\pm$1.28}\\
\\
\toprule
 & \multicolumn{2}{@{}c@{}}{30\%} & \multicolumn{2}{@{}c@{}}{40\%} \\
 \cmidrule{2-3}\cmidrule{4-5}
 & BA(\%) & $F_1$(\%) & BA(\%) & $F_1$(\%)\\ \midrule
Subject1 & 90.48$\pm$0.13 & 90.24$\pm$0.13 & 88.16$\pm$0.09 & 87.54$\pm$0.10\\
Subject2 & 95.60$\pm$0.15 & 95.49$\pm$0.16 & 94.40$\pm$0.11 & 94.12$\pm$0.13\\
Subject3 & 90.96$\pm$0.21 & 90.41$\pm$0.25 & 89.92$\pm$0.13 & 89.18$\pm$0.14\\
Subject4 & 93.36$\pm$0.09 & 93.42$\pm$0.09 & 93.20$\pm$0.16 & 93.18$\pm$0.16\\
Subject5 & 93.12$\pm$0.13 & 93.08$\pm$0.12 & 91.20$\pm$0.11 & 91.00$\pm$0.11\\
Subject6 & 92.48$\pm$0.18 & 92.40$\pm$0.20 & 88.16$\pm$0.09 & 90.32$\pm$0.13\\
Subject7 & 90.00$\pm$0.34 & 89.49$\pm$0.36 & 88.08$\pm$0.13 & 87.21$\pm$0.15\\
Subject8 & 92.00$\pm$0.11 & 91.66$\pm$0.14 & 90.16$\pm$0.18 & 89.53$\pm$0.22\\
Mean & \textbf{92.25$\pm$1.72} & \textbf{92.02$\pm$1.89} & 90.41$\pm$2.27 & 90.26$\pm$2.31\\
\bottomrule
\end{tabular*}
\end{table*}

\section{Discussion}

%
%-------------------------------------------------------------------------
\subsection{Application Scenarios}
We propose that COD can be divided into two subtasks: identification and location. Our focus is on the binary classification task of determining whether a camouflaged object is present. In scenarios where there is no prior knowledge about the presence of a camouflaged object in an image, a "classify-then-segment" approach aligns better with practical application requirements. Additionally, since the computational cost and runtime of our classification model are significantly lower than those of a segmentation model, this approach also helps conserve computational resources. Furthermore, the integration of our identification module with other location models has the potential to be refined more. For example, an interaction between the location model uncertainty map and the identification module classification uncertainty could be designed collaboratively to enhance the detection performance.

%-------------------------------------------------------------------------
\subsection{Uncertainty Sources}
Uncertainty in machine learning can generally be categorized as aleatoric or epistemic, each arising from different sources. In particular, the uncertainties involved in the identification of camouflaged objects and location detection are different. For camouflaged object identification, aleatoric uncertainty is particularly high when the camouflaged objects are indistinct and difficult to distinguish, such as animals in a forest blending seamlessly with their surroundings, like branches or foliage. In these scenarios, the inherent ambiguity in the environment and the multiple possibilities reflected in the training data contribute to this type of uncertainty. In contrast, epistemic uncertainty stems from a lack of knowledge of unseen or unfamiliar data. For example, it arises when the camouflaged scenes or objects differ entirely from those in the training dataset, such as new backgrounds or novel camouflage techniques. In this work, we use the term "predictive uncertainty" to refer to the overall uncertainty in a given situation, encompassing both aleatoric and epistemic components.

%-------------------------------------------------------------------------
\subsection{Training process}
Table 6 illustrates how the precision of samples within different confidence intervals changes over training epochs in the validation set. The uncertainty is ranked in ascending order, where higher percentages indicate lower confidence levels in the CV model's predictions. The confusion matrix (CM) is represented in the format
$\begin{bmatrix}
TP & FN \\ 
FP & TN 
\end{bmatrix}$, where the true positives (TP), false negatives (FN), false positives (FP), and true negatives (TN) are laid out in matrix form.

A clear trend emerges: the confidence level of the CV model is inversely correlated with prediction accuracy. This highlights the effectiveness and robustness of our multiview uncertainty measurement approach. Moreover, as training progresses, the accuracy of the top 80\% most confident samples improves, while the accuracy of the bottom 20\% least confident samples decreases significantly. A similar pattern is observed on the test set. This decline in accuracy for low-confidence samples underpins the foundation for human-machine collaboration, as it identifies cases where the CV model lacks confidence and could benefit from human intervention.


\begin{table*}[ht]
\centering
\footnotesize
\caption{Accuracy Trends of Samples Across Confidence Intervals During Training on the Validation Set. Uncertainty Is Ranked in Ascending Order, with Higher Percentages Indicating Lower Model Confidence. The Confusion Matrix (CM) Is Represented as $\begin{bmatrix} TP & FN \\ FP & TN \end{bmatrix}$.}
\begin{tabular}{>{\centering\arraybackslash}p{0.5cm} 
                >{\centering\arraybackslash}p{0.4cm} 
                >{\centering\arraybackslash}p{0.4cm} 
                >{\centering\arraybackslash}p{1.2cm}  
                >{\centering\arraybackslash}p{0.4cm} 
                >{\centering\arraybackslash}p{0.4cm} 
                >{\centering\arraybackslash}p{1.2cm}
                >{\centering\arraybackslash}p{0.4cm} 
                >{\centering\arraybackslash}p{0.4cm} 
                >{\centering\arraybackslash}p{1.2cm} }
\toprule
& \multicolumn{3}{c}{0-20\%} & \multicolumn{3}{c}{20-40\%} & \multicolumn{3}{c}{40-60\%}\\
\cmidrule{2-4} \cmidrule{5-7} \cmidrule{8-10}
Epoch & BA & F1 & \multicolumn{1}{c}{CM} & BA & F1 & CM & BA & F1 & \multicolumn{1}{c}{CM}\\ \midrule
\multirow{2}{*}{0} & \multirow{2}{*}{97.2} & \multirow{2}{*}{98.1} & \multirow{2}{*}{$\left[\begin{matrix} 27 & 0 \\ 1 & 17 \end{matrix}\right]$} & \multirow{2}{*}{92.8} & \multirow{2}{*}{94.1} & \multirow{2}{*}{$\left[\begin{matrix} 24 & 0 \\ 3 & 18 \end{matrix}\right]$}  & \multirow{2}{*}{90.4} & \multirow{2}{*}{92.3} & \multirow{2}{*}{$\left[\begin{matrix} 24 & 0 \\ 4 & 17 \end{matrix}\right]$}\\
&&&&&&&&&\\
\multirow{2}{*}{10} & \multirow{2}{*}{94.7} & \multirow{2}{*}{96.2} & \multirow{2}{*}{$\left[\begin{matrix} 26 & 0 \\ 2 & 17 \end{matrix}\right]$} & \multirow{2}{*}{96.4} & \multirow{2}{*}{98.4} & \multirow{2}{*}{$\left[\begin{matrix} 31 & 0 \\ 1 & 13 \end{matrix}\right]$}  & \multirow{2}{*}{95.2} & \multirow{2}{*}{96.0} & \multirow{2}{*}{$\left[\begin{matrix} 24 & 0 \\ 2 & 19 \end{matrix}\right]$}\\
&&&&&&&&&\\
\multirow{2}{*}{20} & \multirow{2}{*}{100} & \multirow{2}{*}{100} & \multirow{2}{*}{$\left[\begin{matrix} 18 & 0 \\ 0 & 27 \end{matrix}\right]$} & \multirow{2}{*}{92.1} & \multirow{2}{*}{94.5} & \multirow{2}{*}{$\left[\begin{matrix} 26 & 0 \\ 3 & 16 \end{matrix}\right]$}  & \multirow{2}{*}{97.5} & \multirow{2}{*}{98.0} & \multirow{2}{*}{$\left[\begin{matrix} 25 & 0 \\ 1 & 19 \end{matrix}\right]$}\\
&&&&&&&&&\\
\multirow{2}{*}{31} & \multirow{2}{*}{100} & \multirow{2}{*}{100} & \multirow{2}{*}{$\left[\begin{matrix} 17 & 0 \\ 0 & 27 \end{matrix}\right]$} & \multirow{2}{*}{97.3} & \multirow{2}{*}{98.1} & \multirow{2}{*}{$\left[\begin{matrix} 26 & 0 \\ 1 & 18 \end{matrix}\right]$}  & \multirow{2}{*}{92.1} & \multirow{2}{*}{94.5} & \multirow{2}{*}{$\left[\begin{matrix} 26 & 0 \\ 3 & 16 \end{matrix}\right]$}\\
&&&&&&&&&\\
\\
\toprule
 & \multicolumn{3}{c}{60-80\%} & \multicolumn{3}{c}{80-100\%} \\
\cmidrule{2-4} \cmidrule{5-7}
Epoch & BA & F1 & \multicolumn{1}{c}{CM} & BA & F1 & \multicolumn{1}{c}{CM}\\ \midrule
\multirow{2}{*}{0} & \multirow{2}{*}{86.0} & \multirow{2}{*}{88.4} & \multirow{2}{*}{$\left[\begin{matrix} 23 & 1 \\ 5 & 16 \end{matrix}\right]$} & \multirow{2}{*}{70.5} & \multirow{2}{*}{59.2} & \multirow{2}{*}{$\left[\begin{matrix} 8 & 6 \\ 5 & 26 \end{matrix}\right]$}\\
&&&&&&\\
\multirow{2}{*}{10} & \multirow{2}{*}{89.9} & \multirow{2}{*}{86.4} & \multirow{2}{*}{$\left[\begin{matrix} 16 & 1 \\ 4 & 24 \end{matrix}\right]$} & \multirow{2}{*}{63.3} & \multirow{2}{*}{51.6} & \multirow{2}{*}{$\left[\begin{matrix} 8 & 7 \\ 8 & 22 \end{matrix}\right]$}\\
&&&&&&\\
\multirow{2}{*}{20} & \multirow{2}{*}{90.1} & \multirow{2}{*}{92.0} & \multirow{2}{*}{$\left[\begin{matrix} 23 & 0 \\ 4 & 18 \end{matrix}\right]$} & \multirow{2}{*}{66.1} & \multirow{2}{*}{61.5} & \multirow{2}{*}{$\left[\begin{matrix} 12 & 9 \\ 6 & 18 \end{matrix}\right]$} \\
&&&&&&\\
\multirow{2}{*}{31} & \multirow{2}{*}{89.6} & \multirow{2}{*}{89.4} & \multirow{2}{*}{$\left[\begin{matrix} 21 & 0 \\ 5 & 19 \end{matrix}\right]$} & \multirow{2}{*}{53.3} & \multirow{2}{*}{53.3} & \multirow{2}{*}{$\left[\begin{matrix} 12 & 10 \\ 11 & 12 \end{matrix}\right]$} \\
&&&&&& \\
\bottomrule
\end{tabular}
\end{table*}

\subsection{Failure Cases}
\begin{figure}[!htb]
    \centering
    \includegraphics[width=\linewidth]{Figure_3.jpg} % Replace with your image
    \caption{Examples of CV model failure cases where the CV model incorrectly identified background as containing camouflaged objects.}
    \label{fig:machine_nocam}
\end{figure}
 We observed an interesting phenomenon in the failure cases of the CV model. For instances where there was an actual camouflaged object, but the CV model misclassified it as background, the camouflaged object was also difficult for the human eye to detect. In contrast, for instances where the CV model mistakenly identified the actual background as containing a camouflaged object (examples shown in Figure~\ref{fig:machine_nocam}), humans could easily recognize that no camouflaged object was present. This discrepancy might stem from the CV model's difficulty in distinguishing between salient objects and camouflaged ones. Unlike human vision, which can rely on contextual and semantic cues to identify salient features in an image, the CV model might struggle to capture these subtleties.
%------------------------------------------------------------------------


%-------------------------------------------------------------------------
\subsection{Challenges and Future work}
We propose a human-machine collaboration framework for COD based on uncertainty estimation. Our method uses a multiview backbone to measure model confidence by analyzing output differences across views, aiding both training and collaboration. Alternative uncertainty estimation methods, such as dropout-based, bootstrap-based, or Gaussian-based approaches, may also be effective. In addition, camouflaged targets are harder to detect than traditional RSVP targets, with lower P300 amplitudes and longer latencies, which may limit human accuracy. Future work may explore EEG responses to camouflaged targets and refine decoding models. We also plan to integrate EEG and eye tracking data to aid in localization, combining them with segmentation models to achieve better human-machine collaboration in both identification and localization.
%------------------------------------------------------------------------
\section{Conclusion}

This study presents an integration method of RSVP-based BCIs with CV models, where low-confidence samples are redirected to human cognitive input. This approach combines the complementary strengths of humans and machines to tackle challenging detection tasks such as COD. In the CAMO data set, our method outperformed state-of-the-art approaches, with an average improvement of 4.56\% in BA and 3.66\% in the F1 score. For the best-performing participants, the improvements reached 7.6\% in BA and 6.66\% in the F1 score. By allowing humans to focus only on uncertain samples, the method significantly reduces the cognitive load and time required for RSVP tasks. Furthermore, given the variability in the performance of the BCIs due to environmental conditions, user state, and electrode quality, this human-machine collaboration framework enhances the overall robustness and reliability of the system. In summary, this research paves the way for future exploration of neuroscience and human-computer interaction, providing a promising framework for addressing complex detection challenges.

\section{Acknowledgments}
This work was supported by National Natural Science Foundation of China (U20B2074, 62471169), Key Research and Development Project of Zhejiang Province (2023C03026, 2021C03001, 2021C03003), Key Laboratory of Brain Machine Collaborative Intelligence of Zhejiang Province (2020E10010), and supported by Zhejiang Provincial Natural Science Foundation of China (No. LQN25F020013).


%% If you have bib database file and want bibtex to generate the
%% bibitems, please use
%%
% \bibliographystyle{elsarticle-harv} 
% \bibliography{main}
\documentclass{MITstyle}

%\usepackage[table]{xcolor}
\usepackage{chngcntr}
\usepackage{hyperref}
\usepackage{microtype}

\title{A Lightweight and Extensible Cell Segmentation and Classification Model for Whole Slide Images}

\author{Nikita Shvetsov~$^{1, }$\footnote{Correspondence e-mail: nikita.shvetsov@uit.no}, Thomas K. Kilvaer~$^{2, 3}$, Masoud Tafavvoghi~$^{4}$, Anders Sildnes~$^{1}$, \\ Kajsa Møllersen~$^{4}$, Lill-Tove Rasmussen Busund~$^{5, 6}$, Lars Ailo Bongo~$^{1}$ \\
%
\vspace{1em} % Space between authors and afilliations
%
\normalfont{\small $^{1}$Department of Computer Science, UiT The Arctic University of Norway}\\
\normalfont{\small $^{2}$Department of Oncology, University Hospital of North Norway}\\
\normalfont{\small $^{3}$Department of Clinical Medicine, UiT The Arctic University of Norway}\\
\normalfont{\small $^{4}$Department of Community Medicine, UiT The Arctic University of Norway}\\
\normalfont{\small $^{5}$Department of Medical Biology, UiT The Arctic University of Norway} \\
\normalfont{\small $^{6}$Department of Clinical Pathology, University Hospital of North Norway} %\vspace{2em}
}

\begin{document}
\maketitle

\section*{Abstract}

% \begin{abstract}
% Developing clinically useful cell-level analysis tools in digital pathology remains challenging due to limitations in dataset granularity, inconsistent annotations, computational demands of advanced models, and difficulties in integrating new technologies into clinical workflows. To address these challenges, we propose a multi-faceted solution that enhances data quality, model performance, and usability to create a lightweight and extensible cell segmentation and classification model.

% First, we update data labels by employing a cross-relabeling process that refines the labels of two existing datasets, PanNuke and MoNuSAC, to create a new unified dataset with enhanced granularity, encompassing seven distinct cell types. Second, we leverage the H-Optimus foundation model as a fixed encoder to improve feature representation for simultaneous cell segmentation and classification tasks. Third, to address the computational demands of foundation models, we employ knowledge distillation to reduce model size and complexity while maintaining comparable performance. Finally, to facilitate integration into clinical workflows, we integrate the distilled model into the QuPath software, a widely used open-source platform in digital pathology.

% Our results demonstrate improvements in cell segmentation and classification performance using the H‑Optimus-based model compared to a CNN-based model. Specifically, the average $R^2$ improved from 0.575 to 0.871, and the average $PQ$ score improved from 0.450 to 0.492, indicating better alignment with actual cell counts and enhanced segmentation and classification quality. Furthermore, the distilled student model maintains performance comparable to the larger foundation model while reducing the parameter count by a factor of 48.
% Overall, by reducing computational complexity and integrating it into existing workflows, the proposed approach may significantly impact diagnostic processes, reduce the workload of pathologists, and contribute to improved patient outcomes. Though our approach shows potential enhancements in efficiency and usability of cell segmentation and classification models in digital pathology, extensive validation is needed to deploy these models in clinical practice.
% \end{abstract}

%%% shortened abstract
\begin{abstract}
Developing clinically useful cell-level analysis tools in digital pathology remains challenging due to limitations in dataset granularity, inconsistent annotations, high computational demands, and difficulties integrating new technologies into workflows. To address these issues, we propose a solution that enhances data quality, model performance, and usability by creating a lightweight, extensible cell segmentation and classification model. 

First, we update data labels through cross-relabeling to refine annotations of PanNuke and MoNuSAC, producing a unified dataset with seven distinct cell types. Second, we leverage the H-Optimus foundation model as a fixed encoder to improve feature representation for simultaneous segmentation and classification tasks. Third, to address foundation models' computational demands, we distill knowledge to reduce model size and complexity while maintaining comparable performance. Finally, we integrate the distilled model into QuPath, a widely used open-source digital pathology platform. 

Results demonstrate improved segmentation and classification performance using the H-Optimus-based model compared to a CNN-based model. Specifically, average $R^2$ improved from 0.575 to 0.871, and average $PQ$ score improved from 0.450 to 0.492, indicating better alignment with actual cell counts and enhanced segmentation quality. The distilled model maintains comparable performance while reducing parameter count by a factor of 48. By reducing computational complexity and integrating into workflows, this approach may significantly impact diagnostics, reduce pathologist workload, and improve outcomes. Although the method shows promise, extensive validation is necessary prior to clinical deployment.
\end{abstract}
\clearpage

\section{Introduction}
In digital pathology, accurate segmentation and classification of cells are crucial for many diagnostic, prognostic, and predictive analyses \cite{Jaber_Beziaeva_etal._2019,Lin_Pan_etal._2022,Park_Ock_etal._2022,Shen_Choi_etal._2024}. Nowadays, developments in computational pathology offer multiple solutions \cite{H._Qu_P._Wu_etal._2020,Javed_Mahmood_etal._2020} to utilize cell-level datasets to train machine learning models that solve these problems. The quality and specificity of training datasets are critical for robust and accurate models. Adhering to the principle of "garbage in, garbage out", it is essential to ensure that these datasets are extensively and accurately labeled with distinct classes that reflect the diverse biological characteristics of different cell types. Unfortunately, the number of open-source datasets comprising such high-quality annotations is limited. Existing cell segmentation datasets \cite{Gamper_Koohbanani_etal._2019,Graham_Vu_etal._2019,Verma_Kumar_etal._2021} may offer extensive annotations for certain cell types while providing more general labels for others. For example, in PanNuke, which is one of the largest open-source datasets comprising labeled cells, various types of morphologically and functionally different inflammatory cells like macrophages and lymphocytes are clustered in a broad "inflammatory" class. Consequently, these classes are frequently omitted from analyses or aggregated into broader meta-classes \cite{Gamper_Koohbanani_etal._2020} and likely interfere with other cell classes included in the dataset. This and similar inconsistencies in annotation granularity limit the ability of machine learning models to learn the comprehensive and nuanced features necessary for accurate cell segmentation and classification. To address these challenges, methods for refining and standardizing dataset annotations are essential to enhance the quality of training data.

A complementary approach to mitigate the absence of high-quality training data is the use of foundation models. Foundation models as encoders are defined as large-scale, versatile networks pre-trained on vast, diverse datasets using self-supervised learning, contrasting with convolutional neural network (CNN) pre-trained encoders that rely on supervised learning with labeled data. In practice, foundation models leverage enormous amounts of weakly or unlabeled data from millions of whole slide images (WSIs) and employ self-attention mechanisms to capture long-range dependencies and global context \cite{Chen_Ding_etal._2024,Saillard_Jenatton_etal._2024,Vorontsov_Bozkurt_etal._2024,Xu_Usuyama_etal._2024}. As a consequence, foundation models are able to produce transferable feature representations across different cell types and tissue environments. The feature representations can be leveraged by decoder networks to produce segmentation masks and pixel-level classifications. Because foundation models have comprehensive feature representations, they can be effectively fine-tuned using much smaller amounts of cell-level data compared to the large datasets needed to train models from scratch. Furthermore, foundation models incorporate adversarial training elements or contrastive learning \cite{Chen_Ding_etal._2024,Xu_Usuyama_etal._2024}, enhancing their resilience and adaptability by exposing them to challenging and varied scenarios during training. This may result in more generalizable models, often making them well-suited for diverse and complex tasks in digital pathology.

Despite the inherent advantages of foundation models, their deployment for practical use faces its own obstacles. In particular, they require substantial computational power, financial investments and rigorous testing to ensure reliability and efficacy for a given task \cite{Akkus_Dangott_etal._2022,Dragomir_Cocuz_etal._2022,Go_2022,Jafri_Farooqui_etal._2024}. Moreover, while foundation models enhance feature representation and performance, they depend on the quality of available annotations for decoder fine-tuning and, like any other model, cannot resolve existing inconsistencies or ambiguities in data labels. Therefore, there remains a critical need for solutions that address both data quality and practical deployment considerations.
Further, integrating new technologies into existing clinical workflows often encounters resistance, as it necessitates adjustments to established diagnostic processes. So, there is a need to develop solutions that could be integrated into current practices, minimizing the burden on medical professionals to adopt new tools \cite{King_Williams_etal._2023}.

Existing solutions \cite{Goldsborough_Philps_etal._2024,Hörst_Rempe_etal._2024}, while addressing some aspects of these challenges, fall short in providing a comprehensive approach. To address the data quality and clinical deployment issues, we propose a multi-faceted solution that encompasses data refinement, model optimization, and integration with existing pathology tools (\hyperref[fig:fig1]{Figure 1}). The outcome is a lightweight cell segmentation and classification model that can be integrated into digital pathology workflows for practical clinical use.

\begin{figure}[h!]
    \centering
    \includegraphics[width=\textwidth, height=0.82\textheight, keepaspectratio]{images/Figure_1.pdf}
    \caption{Overview of the proposed solution, including 1) Data refinement using cross-relabeling, 2) Teacher model development and fine tuning, 3) Student model optimization with knowledge distillation and 4) Student model and QuPath integration}
    \label{fig:fig1}
\end{figure}
\clearpage

Our approach begins with preparing the data for the fine-tuning and training of the machine learning models. We create a refined dataset, acquired via cross-relabeling two cell-level datasets, enhancing annotation specificity and consistency of the labeled data. Subsequently, we create a cell segmentation and classification model based on the foundation model. We leverage the foundation model as a fixed encoder and fine-tune a decoder using the refined dataset to improve generalization across diverse tissue- and cell types.
To ensure that the model remains lightweight and deployable in a possibly resource-constrained environment, we employ knowledge distillation to approximate the functionality of the foundation model. Finally, to facilitate the practical application of our model in digital pathology workflows, we integrate it with the QuPath \cite{Bankhead_Loughrey_etal._2017} application. Each methodological component contributes to the overarching goal of enhancing model performance, generalizability, and usability in clinical settings.

The primary contributions of this paper are:
\begin{enumerate}
    \item \textit{Data labels refinement through cross-relabeling:}
    
    We propose a new method for refining labels of cell-level datasets through cross-relabeling. This method employs classification models to re-label broad and ambiguous instances, resulting in a more diverse dataset. Our evaluation demonstrates that these classification models achieve high accuracy on test subsets, indicating the reliability of the method for label refinement.

    \item \textit{Enhanced model performance via foundation models:}
    
    We employ a foundation model as a feature extractor for the cell segmentation and classification task. In comparison with training a CNN model from scratch, the foundation model backbone only needs fine-tuning, which significantly reduces training time, computational resources and data requirements. We show that using a foundation model encoder leads to better performance in cell segmentation and classification networks than using a CNN-based encoder. This improvement may enable the model to generalize more effectively across various tissue types and imaging methods.
    
    \item \textit{Model optimization through knowledge distillation:}
    
    We show that a smaller student model trained using knowledge distillation on the refined dataset obtained via our cross-relabeling approach from a foundation model achieves comparable performance in cell segmentation and quantification tasks. As a result, this model is more suitable for deployment in environments without high-performance computing resources.
    
    \item \textit{Integration with QuPath:}
    
    We integrate the distilled cell segmentation and classification model into QuPath, a widely used open-source digital pathology platform, to accelerate clinical adaptation by enabling pathologists to more easily incorporate advanced computational tools into their existing workflows.
\end{enumerate}

Through these methodological steps, we aim to bridge the gap between advanced machine learning techniques and practical clinical applications, making accurate and efficient digital pathology accessible in a broader range of healthcare settings.

\section{Refining Existing Datasets Using Cross-Relabeling}
To address the limitations of sparse and ambiguous labeling of cell-level datasets, we propose a generalizable cross-relabeling strategy that can be applied to any dataset containing broadly categorized or imprecisely labeled cell types. This approach involves training and subsequently leveraging classification models to refine broad categories into more specific or biologically relevant classes.
When applied to cell-level data, the methodology includes extracting individual cell images from the dataset patches, preprocessing these images to standardize the size and accommodate partial cells, and then training deep learning classifiers capable of distinguishing between the finer cell subtypes within the coarser categories. 
To illustrate our approach, we focus on the PanNuke \cite{Gamper_Koohbanani_etal._2020, Gamper_Koohbanani_etal._2019} and MoNuSAC \cite{Verma_Kumar_etal._2021} datasets that we have used to train models for cell quantification in our previous works \cite{Shvetsov_Grønnesby_etal._2022,Shvetsov_Sildnes_etal._2024}. We find that for better cell differentiation we have to introduce more granular labels. PanNuke includes a broad classification of "inflammatory" cells, encompassing lymphocytes, macrophages, and neutrophils. Each cell type differs significantly in structure, function, and clinical relevance. Conversely, MoNuSAC uses the label "epithelial" for a class that comprises both benign epithelial cells and malignant neoplastic cells. This practice makes it challenging to differentiate between benign and malignant epithelial cells in the dataset, which is a critical distinction when identifying tumor areas within tissue samples. To address these issues, we implement a cross-relabeling strategy as shown in \hyperref[fig:fig2]{Figure 2}. The key components are two classification models: one is trained on singular cell images from PanNuke data to classify the epithelial meta-class into epithelial and neoplastic classes. The other is trained on MoNuSAC to refine the inflammatory class into lymphocytes, neutrophils, and macrophages.

\begin{figure}[h!]
    \centering
    \includegraphics[width=\textwidth]{images/Figure_2.pdf}
    \caption{Refined dataset generation via cross relabeling}
    \label{fig:fig2}
\end{figure}

The refining approach consists of three consecutive steps. The first is the preprocessing step, in which we extract individual cells from both datasets (\hyperref[fig:fig3]{Figure 3}). The specifics of PanNuke and MoNuSAC patch preparation before cell preprocessing are provided in \hyperref[chap:S1]{Appendix S1}.

\begin{figure}[h!]
    \centering
    \includegraphics[width=\textwidth]{images/Figure_3.pdf}
    \caption{Cell instances preprocessing including (1) cell map extraction, (2) bounding box delineation, (3) adjusting cell boxes and (4) cropping and resizing of cell images}
    \label{fig:fig3}
\end{figure}

During preprocessing, we extract cell type maps from the ground truth label mask and calculate bounding boxes around each cell instance. To accommodate partial cells at patch borders, a common issue in cropped patch images, we employ mirror padding and extend the field of view of the cell label by 15 pixels to capture adjacent cells. We then crop and resize the identified regions to $64 \times 64$ pixels using bicubic interpolation.

The preprocessed PanNuke dataset comprises 68,031 neoplastic and 23,207 epithelial cell images, while MoNuSAC comprises  33,104 lymphocytes, 1,252 neutrophils, and 1,695 macrophages, which we subsequently use in training cell classification models and classifying the cell image data \hyperref[fig:S2]{Appendix Figure S2 (1)}. 

The next step is to train two distinct ResNet50-based classifiers tailored to address the specific labeling challenges inherent in each dataset. We use ResNet50 for classification models due to its proven effectiveness for image classification tasks in histopathology \cite{pan2022reviewmachinelearningapproaches}, and its compatibility with small images. For the PanNuke dataset, we design the classifier, trained on MoNuSAC data, to disaggregate the heterogeneous "inflammatory" cell category into distinct subtypes: lymphocytes, macrophages, and neutrophils. Similarly, for the MoNuSAC dataset, the classifier is trained on PanNuke data and distinguishes between benign and malignant epithelial cells within the overarching "epithelial" label. By applying these targeted classifiers to their respective datasets, we assign more specific labels to individual cell instances, thus enabling us to create a unified dataset.
To ensure a balanced representation of classes, we train both models on datasets that had been equalized to match the size of the least represented class. Thus, we obtain datasets comprising 23,207 samples per class for PanNuke and 1,252 samples per class for MoNuSAC data. Next, we partition both of them into training (70\%), validation (20\%), and testing (10\%) subsets. To mitigate the risk of overfitting, we use a single dropout layer with a rate of p=0.5 in both models and data augmentation using randomized color perturbations, rotation, and horizontal and vertical flipping. We employ AdamW optimizer and the cross-entropy loss function for the training criterion.

To evaluate the two trained models, we measure the classification accuracy on the respective test subsets. The accuracies on the test subset for both classifiers are presented in \hyperref[tab:1]{Table 1}. The PanNuke model achieves an average accuracy of 93.57\%, with higher accuracy for neoplastic cells (96.06\%) compared to epithelial cells (86.26\%). The confusion matrix in Figure A3.1 shows that the model predominantly distinguishes accurately between epithelial and neoplastic tissues, with a substantial number of correct classifications and relatively few misclassifications. The MoNuSAC model demonstrates an average accuracy of 98.92\%, excelling in classifying lymphocytes (99.67\%) and macrophages (94.12\%), with lower performance for neutrophils (85.71\%). The confusion matrix in Figure A3.2 shows that the model identifies lymphocytes and performs reasonably well with macrophages and neutrophils.

\begin{table}[h!]
\renewcommand{\arraystretch}{1.5}
  \centering
  \caption{Cell classification results for PanNuke and MoNuSAC trained models (CI 95\%).}
  \label{tab:1}
  \begin{tabular}{|l|c|c|}
   \hline
   %\rowcolor{gray!30}
    Accuracy               & PanNuke model              & MoNuSAC model              \\
    \hline
    Average      & 0.936 (0.931--0.941)         & 0.989 (0.986--0.993)        \\
    \hline
    Neoplastic   & 0.961 (0.956--0.965)         & -                          \\
    \hline
    Epithelial   & 0.863 (0.849--0.877)         & -                          \\
    \hline
    Lymphocytes  & -                          & 0.997 (0.995--0.999)        \\
    \hline
    Neutrophils  & -                          & 0.857 (0.796--0.918)        \\
    \hline
    Macrophages  & -                          & 0.941 (0.906--0.976)        \\
    \hline
  \end{tabular}
\end{table}

Finally, during the last step, we use the model trained on PanNuke data for epithelial cells in MoNuSAC and the model trained on MoNuSAC for the inflammatory cells class in PanNuke. Specifically, we use classifier models to relabel epithelial cells in MoNuSAC and inflammatory cells in PanNuke data. Then we combine cells with refined labels and the rest of the cells in both datasets to create a refined dataset (\hyperref[fig:S2]{Appendix Figure S2 (2)}). The process of relabeling cells and visualizing them on a patch is shown in \hyperref[fig:fig4]{Figure 4}. The cell counts in the refined dataset are provided in \hyperref[tab:S4]{Appendix Table S4}.

\begin{figure}[h!]
    \centering
    \includegraphics[width=\textwidth, height=0.42\textheight, keepaspectratio]{images/Figure_4.pdf}
    \caption{Cell relabeling procedure for epithelial and inflammatory cell classes}
    \label{fig:fig4}
\end{figure}

%\hfill

Relabeling and combining datasets have been explored in a prior study \cite{Parulekar_Kanwat_etal._2023}, where consecutive fine-tuning on multiple datasets was employed to account for hierarchical class label structures. While the method presented in \cite{Parulekar_Kanwat_etal._2023} is intuitive, it often lacks consistency and requires multiple fine-tuning runs, which can be cumbersome and time-consuming. 
In contrast, cross-relabeling simplifies this process by using specialized classification models tailored to each dataset's specific labeling challenges. This approach provides better transparency and produces a unified dataset encompassing seven distinct cell types across multiple tissue samples, enhancing data diversity for further model training or fine-tuning.

Despite these improvements, cross-relabeling does not entirely resolve issues related to poor labeling quality or the amount of labeled data. Specifically, our results show lower accuracies persist for underrepresented classes, such as macrophages, which may stem from a limited sample availability and intrinsic challenges in distinguishing these cells based solely on H\&E staining. Furthermore, while our method enhances label specificity, it relies on the initial quality of the broad labels; thus, any fundamental inaccuracies in the original annotations can propagate through the relabeling process. Addressing the overall problem of limited data labels may require integrating additional data sources or utilizing complementary immunohistochemical staining methods.
Although the reported performance metrics are obtained from evaluations on the native test sets of each dataset, it is important to note that the primary application of these classifiers is to perform cross-relabeling, where a model trained on one dataset (e.g., PanNuke) is applied to another (e.g., MoNuSAC) and vice versa. We acknowledge that a more systematic evaluation of cross-dataset generalization is needed and could be performed in future work.

Overall, the refined dataset produced by our approach can enhance the supervised training or fine-tuning of cell segmentation and classification models, especially those that utilize pre-trained foundation models to improve feature extraction robustness. In addition, these models can detect nuanced classes that enable researchers to conduct more detailed analyses of biological processes in computational pathology.

\section{Foundation models for robust cell segmentation and classification}

Accurate cell segmentation and classification in digital pathology are hindered by limited labeled data and the fact that conventional CNNs are unable to capture global contextual information due to their local receptive field constraints \cite{Gheflati_Rivaz_2022,Yang_Marcus_etal.}. Traditional approaches in cell quantification have predominantly relied on CNN encoders, such as ResNet50, given their proven effectiveness in semantic segmentation tasks \cite{Deshmane_2023,Graham_Vu_etal._2019,Mukasheva_Koishiyeva_etal._2024,Stringer_Wang_etal._2021}. However, approaches that include fine-tuning of pretrained CNNs, data augmentation, and stain normalization to partially increase data variability and address staining differences often fail to achieve the necessary generalization and robustness across diverse tissue types and staining conditions \cite{G._Wang_W._Li_etal._2018,Gao_Bagci_etal._2018,Karim_El_Khoury_Martin_Fockedey_etal._2021}.

To overcome these challenges, we leverage an encoder-decoder network that uses a foundation model as the encoder and a CNN upsampling decoder (\hyperref[fig:fig5]{Figure 5}) for simultaneous cell segmentation and classification in 2D patches extracted from WSIs. Foundation models with transformer-based architectures are viable alternatives to CNN-based encoders \cite{Shamshad_Khan_etal._2023,Sourget_2023}. They enable the creation of more advanced architectures that can decode or transform learned features more effectively \cite{Chen_Duan_etal._2023,Cheng_Misra_etal._2022,Xie_Wang_etal._2021}.

\begin{figure}[h!]
    \centering
    \includegraphics[width=\textwidth]{images/Figure_5.pdf}
    \caption{UNETR-like model with foundational model as backbone}
    \label{fig:fig5}
\end{figure}

By utilizing a transformer-based encoder, we incorporate global contextual information into the feature extraction process, which is a key advantage of such architectures \cite{Chen_Lu_etal._2021}. This foundation model integration facilitates accurate pixel-wise segmentation and classification without the need for extensive encoder training, thereby potentially improving generalization across varied cellular structures and tissue types.
In our implementation, we employ a modified UNETR \cite{Hatamizadeh_Tang_etal._2021} architecture that combines a vision transformer (ViT) \cite{Dosovitskiy_Beyer_etal._2021} encoder with a CNN-based decoder. The encoder utilizes the pretrained H-Optimus foundation model, which contains 1.1 billion parameters and is trained on over 500,000 H\&E stained WSIs \cite{Saillard_Jenatton_etal._2024}. We extract outputs from four evenly spaced transformer blocks $Z_i$, where $i \in [1, 14, 26, 38]$, to serve as residual connections for the CNN decoder. We select these blocks based on our observation that features from non-adjacent levels of the encoder lead to better overall performance on the test subset.

The CNN decoder upsamples the feature representations, acquired from the transformer blocks, to generate an intermediate vector that is handled by two task-specific layers that generate cell segmentation and classification masks. The first task-specific layer is the ‘Cellpose head’,  which is used to delineate cell instances. The layer generates horizontal and vertical gradient maps to form vector fields that are refined through gradient tracking in a post-processing step using the Cellpose algorithm \cite{Stringer_Wang_etal._2021}, known for its efficacy in cell segmentation tasks and generalizability across multiple domains \cite{Pachitariu_Stringer_2022,Stringer_Pachitariu_2024}. The second task-specific layer is the "Cell type head", which assigns labels to individual pixels. In the post-processing step, we determine the output classification label of each segmented cell instance by majority voting over the labeled pixels that comprise the cell in the segmentation map.

To evaluate model performance and measure the impact of adding a foundation model as backbone, we compare it to a ResNet50-based model. ResNet50 is a widely used solution for encoders in segmentation architectures in the medical domain \cite{Deshmane_2023,Graham_Vu_etal._2019,Mukasheva_Koishiyeva_etal._2024,Stringer_Wang_etal._2021}. For the H-Optimus-based model, we utilize frozen weights for the encoder and only fine-tune the decoder to take advantage of the extensive pre-training of the foundation model. For the ResNet50-based model we start with ImageNet \cite{Deng_Dong_etal.} weights and train both encoder and decoder parts. Hyperparameters for the training step are set to be identical, where possible, for comparable evaluation. 
For this evaluation, we deliberately use the PanNuke dataset to provide a standardized and controlled comparison between the H‑Optimus and ResNet50-based models (\hyperref[fig:S2]{Appendix Figure S2 (3)}). Specifically, we use two of the default PanNuke dataset splits (66\%) for training and validation, and reserve the third split (33\%) for testing.

To address the challenge of cell class imbalance in the PanNuke dataset, which is a common characteristic in most cell-level H\&E patch datasets, both models’ training processes employ a weighted loss function comprising cross-entropy and focal loss \cite{Lin_Goyal_etal._2018}. The focal loss component is adjusted with coefficients derived from each cell class' instance frequency, emphasizing learning from underrepresented classes and enhancing the model's sensitivity to rare but significant cellular patterns. The cross-entropy loss is augmented with spectral decoupling regularization \cite{Pezeshki_Kaba_etal._2021,Pohjonen_Stürenberg_etal._2022} and spatially varying label smoothing \cite{Islam_Glocker_2021}, which potentially stabilizes training and improves generalization in case of complex tissue morphologies. For optimization, we employ the \textit{AdamW} \cite{Loshchilov_Hutter_2019} to counter unbalanced class scenarios, with cosine annealing learning rate scheduler.

We utilize the scikit-learn library \cite{Van_der_Walt_Schönberger_etal._2014} and HoVer-Net \cite{Graham_Vu_etal._2019} implementations of $R^2$ (the coefficient of determination) and $PQ$ (panoptic quality) to evaluate our experiments. Complete mathematical formulations and detailed explanations of these metrics are provided in \hyperref[chap:S5]{Appendix S5}. To compute confidence intervals, we use nonparametric bootstrapping, where after calculating the metric on the full sample, we generated 1000 bootstrap replicates by resampling with replacement and then determined the 95\% confidence intervals as the 2.5th and 97.5th percentiles of the resulting empirical distribution.

%\hfill

The model comparisons are summarized in \hyperref[tab:2]{Table 2}. The H‑Optimus-based model achieves higher $R^2$ across all cell classes compared to the ResNet50-based model, which means that its predictions are more closely aligned with the PanNuke cell counts, indicating a stronger correlation with the observed data. Notably, the improvement of $R^2_{dead}$ may be an indicator of better global contextual representations provided by the foundation model backbone. In terms of segmentation and classification quality combined, measured by the PQ score, the H‑Optimus-based model demonstrates notable improvements across most cell classes. Overall, the average $R^2$ improved from 0.575 to 0.871, while the average $PQ$ score improved from 0.450 to 0.492, demonstrating better performance of the H-Optimus-based model.

\begin{table}[h!]
\renewcommand{\arraystretch}{1.5}
  \centering
  \caption{Cell quantification metrics for baseline and proposed models (CI 95\%).}
  \label{tab:2}
  \begin{tabular}{|l|c|c|}
    \hline
    %\rowcolor{gray!30}
    Metric             & Resnet50-based            & H-optimus-based              \\
    \hline
    $R^2_{neoplastic}$    & 0.681 (0.576--0.769)       & \textbf{0.941 (0.917--0.960)} \\
    \hline
    $R^2_{inflammatory}$  & 0.863 (0.778--0.903)       & \textbf{0.949 (0.918--0.966)} \\
    \hline
    $R^2_{connective}$    & 0.600 (0.488--0.698)       & 0.609 (0.436--0.772)          \\
    \hline
    $R^2_{dead}$          & 0.097 (-11.389--0.669)     & 0.925 (0.404--0.982)          \\
    \hline
    $R^2_{epithelial}$    & 0.635 (0.490--0.747)       & \textbf{0.930 (0.886--0.964)} \\
    \hline
    $PQ_{neoplastic}$       & 0.517 (0.499--0.535)       & \textbf{0.589 (0.575--0.604)} \\
    \hline
    $PQ_{inflammatory}$     & 0.455 (0.429--0.482)       & \textbf{0.528 (0.507--0.549)} \\
    \hline
    $PQ_{connective}$       & 0.416 (0.400--0.431)       & \textbf{0.451 (0.436--0.465)} \\
    \hline
    $PQ_{dead}$             & 0.374 (0.342--0.408)       & 0.292 (0.209--0.365)          \\
    \hline
    $PQ_{epithelial}$       & 0.488 (0.460--0.519)       & \textbf{0.599 (0.579--0.618)} \\
    \hline
  \end{tabular}
\end{table}

Our results  show that integrating the H‑Optimus foundation model within the UNETR architecture enhances the model's ability to segment and classify cells across diverse tissues from PanNuke data. The pretrained transformer encoder provides robust feature representations, resulting in higher average $R^2$ and $PQ$ scores compared to the CNN-based model. This leads to more reliable cell quantification and more accurate downstream analysis. Additionally, the streamlined fine-tuning process reduces computational overhead and training time, making the model more adaptable for new data.

Despite these advancements, the foundation model-based approach does not fully resolve all challenges related to cell segmentation and classification. We observe lower metric scores for underrepresented classes in the training data. Furthermore, foundation models typically encompass billions of parameters, resulting in substantial computational and memory requirements. It therefore poses challenges for deployment in resource-constrained environments, limiting their practical applicability in certain clinical settings.

\section{Model optimization via Knowledge Distillation}

To address the limitations posed by the extensive size of foundation models, we implement knowledge distillation — a model compression technique that leverages the teacher-student paradigm \cite{Hinton_Vinyals_etal._2015}. By training a smaller, more efficient student model to replicate the output of a larger, pre-trained teacher model, we retain performance while significantly reducing the model's complexity and resource requirements (\hyperref[fig:fig6]{Figure 6}).

\begin{figure}[h!]
    \centering
    \includegraphics[width=\textwidth, height=0.45\textheight, keepaspectratio]{images/Figure_6.pdf}
    \caption{Knowledge distillation framework for training a student model using a pre-trained teacher}
    \label{fig:fig6}
\end{figure}

We employ knowledge distillation to compress the H‑Optimus-based teacher model into a more efficient student model. The teacher model is the modified UNETR architecture with the H‑Optimus foundation model described in the previous chapter. The student model is based on a UNet architecture augmented with residual connections and incorporates a smaller ViT encoder with 9 million parameters \cite{Steiner_Kolesnikov_etal._2022,Wightman_2019}. 

First, we fine-tune the teacher model using the refined dataset from the cross-relabeling procedure (Section 2). Initially we train the decoder of the teacher model while keeping the encoder weights frozen. We split the refined dataset into train (70\%), validation (20\%) and test (10\%) subsets (\hyperref[fig:S2]{Appendix Figure S2 (4)}). During fine-tuning, we use the train and validation subsets, while leaving the test subset for model evaluation. We set the training procedure and model hyperparameters to be identical to those that were used to demonstrate the utility of foundation models for the simultaneous cell segmentation and classification task.

Next, we perform knowledge distillation from teacher to student using the refined dataset used to fine-tune the teacher model. The student model is trained to replicate the teacher model's outputs. We utilize a specialized loss function that aligns the student's predicted probability distribution with the teacher's, incorporating the teacher's class probability distribution derived from the output. Following the methodology of Hinton et al. \cite{Hinton_Vinyals_etal._2015}, we experiment with various hyperparameter settings for the temperature ($T$) and the balancing coefficients ($\alpha$ and $\beta$) in the loss function. We vary $T$ from 1 to 20 and adjust $\alpha$ and $\beta$ to balance the distillation and student losses. Through iterative tuning and evaluation, we identify that setting $T=14$, $\alpha=0.3$, and $\beta=0.7$ yields a configuration that converges and closely approximates the teacher model's performance during training.

Finally, we assess the performance of both models using the $R^2$ and $PQ$ (defined in \hyperref[chap:S5]{Appendix S5}) on the test set of the refined dataset (\hyperref[tab:3]{Table 3}). We observe that the 95\% confidence intervals overlap for most cell types, so we cannot claim statistically significant performance differences between the teacher and student models. One exception appears in the neoplastic class. The teacher model produces an $R^2$ of 0.919, while the student model shows an $R^2$ of 0.852. In addition, the student model achieves higher $PQ$ values for the neoplastic and connective classes, though the confidence intervals show overlap.

\begin{table}[h!]
\renewcommand{\arraystretch}{1.5}
  \centering
  \caption{Cell quantification metrics for teacher and distilled student models (CI 95\%).}
  \label{tab:3}
  \begin{tabular}{|l|c|c|}
    \hline
    %\rowcolor{gray!30}
    Metric & Teacher & Student \\
    \hline
    $R^2_{neoplastic}$    & \textbf{0.919} (0.898--0.939) & 0.852 (0.800--0.891) \\
    \hline
    $R^2_{lymphocyte}$    & 0.969 (0.956--0.977)         & 0.969 (0.956--0.978) \\
    \hline
    $R^2_{connective}$    & 0.694 (0.548--0.809)         & 0.618 (0.469--0.741) \\
    \hline
    $R^2_{dead}$          & 0.755 (0.400--0.908)         & 0.424 (0.100--0.731) \\
    \hline
    $R^2_{epithelial}$    & 0.922 (0.870--0.958)         & 0.843 (0.738--0.917) \\
    \hline
    $R^2_{macrophage}$    & 0.384 (-0.369--0.724)        & 0.704 (0.352--0.859) \\
    \hline
    $R^2_{neutrofil}$     & 0.854 (0.578--0.929)         & 0.833 (0.502--0.925) \\
    \hline
    $PQ_{neoplastic}$       & 0.581 (0.569--0.593)         & 0.601 (0.588--0.613) \\
    \hline
    $PQ_{lymphocyte}$       & 0.536 (0.520--0.553)         & 0.563 (0.544--0.579) \\
    \hline
    $PQ_{connective}$       & 0.436 (0.421--0.451)         & 0.457 (0.441--0.474) \\
    \hline
    $PQ_{dead}$             & 0.272 (0.235--0.315)         & 0.279 (0.201--0.369) \\
    \hline
    $PQ_{epithelial}$       & 0.522 (0.500--0.545)         & 0.530 (0.506--0.555) \\
    \hline
    $PQ_{macrophage}$       & 0.524 (0.459--0.588)         & 0.474 (0.405--0.543) \\
    \hline
    $PQ_{neutrofil}$        & 0.541 (0.490--0.592)         & 0.565 (0.522--0.607) \\
    \hline
  \end{tabular}
\end{table}


We further decompose the $PQ$ metric into its $SQ$ and $DQ$ components (\hyperref[tab:S6]{Appendix Table S6}). Both models produce nearly identical $SQ$ values, which indicates that they predict instance boundaries with similar precision. Although the student model shows some improvement in $DQ$ scores for certain classes, the confidence intervals overlap and do not confirm a statistically significant difference.

We observe that the student and teacher models yield comparable detection performance despite the student model using a much smaller and simpler architecture. A model with fewer parameters reduces the risk of overfitting when training data are scarce relative to the model’s complexity \cite{Farias_Ludermir_etal._2022}. The knowledge distillation process also encourages the student model to focus on the most generalizable detection features learned from the teacher. These factors enable the student model to achieve similar detection performance across different cell types.

Additionally, considering the model sizes reported in \hyperref[tab:4]{Table 4}, the distilled model achieves a significant reduction compared to the teacher model, with a 48-fold decrease in parameter count and a 5.5-fold reduction in on-disk size. In inference mode, the teacher model requires 16 GB of VRAM for a batch size of 32, while the distilled model only needs 3 GB of VRAM for the same batch size. These reductions make the distilled model significantly more practical for fine-tuning and deployment in resource-constrained environments.

\begin{table}[h!]
\renewcommand{\arraystretch}{1.5}
  \centering
  \caption{Parameter counts and size of teacher and distilled model}
  \label{tab:4}
  \adjustbox{max width=\textwidth}{%
  \begin{tabular}{|l|c|c|c|}
    \hline
    %\rowcolor{gray!30}
    Metric & H-optimus-based (Teacher) & mobileViT-based (Student) & Magnitude of difference \\
    \hline
    Parameters count       & 1,158,917,906   & \textbf{24,093,393}   & \textbf{48x}  \\
    \hline
    Estimated Total Size (MB) & 87,912       & \textbf{15,935}    & \textbf{5.5x} \\
    \hline
  \end{tabular}%
}
\end{table}

%\hfill

With recent advancements in complex network architectures and the use of pretrained encoders to achieve state-of-the-art performance \cite{Baumann_Dislich_etal._2024,Hörst_Rempe_etal._2024} in cell segmentation and classification tasks, model size, computational complexity, and processing times have increased. This limits the scalability and accessibility of these models. As we demonstrate, this may be mitigated using knowledge distillation. Studies in the field of natural language processing have demonstrated the efficacy of knowledge distillation in retaining the capabilities of the teacher model while achieving significant reductions in size and complexity \cite{Huangpu_Gao_2024,Sun_Yu_etal.}. 

We demonstrate the feasibility of knowledge distillation in digital pathology, specifically for cell segmentation and classification tasks. Moreover, we achieve this performance while also significantly reducing the parameter count. In addressing the challenge of knowledge transfer, we found that distillation from a transformer-based model to a smaller transformer is more straightforward than attempting to map transformer features to CNN blocks. In our experiments, using a CNN-based network as a student results in worse cell quantification performance due to the structural constraints of CNN feature space dimensions. 

Although our primary approach relies on a transformer-based student model that performs well, it can be further optimized to incorporate advantages from CNN architectures. For example, employing alternative techniques such as using ViT adapters \cite{Chen_Duan_etal._2023} or $1 \times 1$ convolutions to adjust feature map sizes may be beneficial for harnessing CNN advantages like enhanced local feature extraction. Moreover, if additional performance improvements are desired, the process can be further enhanced by applying supplementary knowledge distillation techniques, such as self-distillation \cite{Zhang_Song_etal._2019} or online distillation \cite{Houyon_Cioppa_etal._2023}.

Despite these promising results, further validation on independent datasets is necessary to fully understand the model's limitations. Underrepresented classes may pose challenges when addressing complex cases. Pathologists need to validate these models to adopt them in clinical settings. While the distilled models are smaller and more deployable, a technological gap persists because pathologists traditionally rely on established methods for inspecting WSIs and diagnosing diseases. Addressing the complexities involved in deploying models for inference and supporting pathologists in adopting new tools is essential for integrating these models into clinical workflows.

\section{Model integration with QuPath}
Digital pathology tools with graphical user interfaces are essential for visualizing and analyzing WSIs. To make our student model useful in clinical pathology workflows, it needs to be integrated into a tool that enables inspecting regions, creating annotations, and providing quantitative analyses of biomarkers. Therefore, we integrate the trained student model from the previous chapter into the QuPath open‑source platform \cite{Bankhead_Loughrey_etal._2017}. QuPath provides the required annotation, visualization, and analysis tools to interpret complex histological data, including workflows for cell segmentation, classification, and quantification (\hyperref[fig:fig7]{Figure 7}). 

\begin{figure}[h!]
    \centering
    \includegraphics[width=\textwidth]{images/Figure_7.pdf}
    \caption{Visualization of model-generated cell quantification annotations (left) and the corresponding unannotated slide (right) in QuPath}
    \label{fig:fig7}
\end{figure}

To identify the regions in a WSI critical for prognosticating tumor development, such as specific tumor areas or border regions without overlapping healthy tissue, the pathologist uses QuPath to outline these regions. Then, the pathologist initiates a cell segmentation and classification script through the QuPath interface for the selected regions. The resulting annotations and quantified cell information are then directly overlaid onto the WSI in the QuPath interface. Additional design and implementation details are in \hyperref[chap:S7]{Appendix S7}. 

Two common approaches for integrating deep learning models into QuPath are Java‑based native QuPath extensions \cite{Goldsborough_Philps_etal._2024} and the execution of RESTful API requests to a model server coupled with handling the response via an extension, as demonstrated in the application of cell segmentation models applied to immunofluorescence images \cite{Sugawara_2023}. While the community is actively working on these integration strategies, there is currently no universal solution that fully addresses all integration and performance requirements.

Extensions may offer better integration with QuPath, allowing slightly improved performance and more widespread usage of the built-in QuPath models, but they lack the flexibility to customize models and modify their behavior. For example, the newest version of QuPath includes models such as StarDist \cite{Weigert_Schmidt} and InstanSeg \cite{Goldsborough_Philps_etal._2024} that can perform cell segmentation. Both models pose limitations when applied to simultaneous cell segmentation and classification. StarDist performs well only on convex, round shapes by design, whereas some neoplastic, inflammatory, and connective cells exhibit complex and non-convex shapes. InstanSeg provides only semantic segmentation without assigning classes to the segmented cells.

%\hfill

In contrast, our approach offers an alternative integration strategy. It utilizes the paquo library to directly interact with QuPath’s internal application programming interface from within Python. This enables data exchange and processing without the need for intermediate conversion steps and provides greater control over model customization, retraining, and the incorporation of custom processing steps.

The integration of our custom model with QuPath underscores its potential to significantly enhance the diagnostic process by reducing the time burden on pathologists and enabling them to focus on more complex interpretative tasks using familiar software. Leveraging a tool that is already well-established among pathologists increases the likelihood of its adoption into daily clinical workflows. The quantitative data generated through the automated workflow is critical for both clinical decision-making and research, facilitating more accurate biomarker analysis, enabling robust statistical evaluations, and supporting hypothesis generation and testing. Additionally, by streamlining cell segmentation and classification, the tool enhances the scalability and reproducibility of pathological assessments, ultimately contributing to improved diagnostic accuracy and patient outcomes.

\section{Conclusion and future work}

In this study, we address critical challenges in digital pathology and tackle the usability and deployment issues of the developed models in standard computing environments without the need for high-performance computing systems. Our multi-faceted approach encompasses data refinement through cross-relabeling, leveraging foundation models for robust cell segmentation and classification, optimizing model performance via knowledge distillation, and integrating the optimized model into the QuPath software for practical application. This approach is used to construct a capable, versatile, and adjustable model for cell segmentation and classification, with enhanced performance and usability.

\begin{sloppypar}
While our approach shows potential in the field of computational pathology, certain limitations persist. 
For example, our implementation currently exhibits lower performance in detecting macrophages. 
This serves as an instance of the broader challenge of accurately identifying complex cell types. In order to address this issue, extending our approach to incorporate additional data sources, exploring alternative modeling approaches, and integrating other imaging modalities such as immunohistochemical staining may help improve detection accuracy. Moreover, although the distilled model reduces computational demands, integrating advanced deep learning models into clinical practice requires addressing technological gaps and potential resistance to adopting new tools within established diagnostic processes.
\end{sloppypar}

Future work could focus on several key areas to refine the proposed approach and facilitate its adoption in clinical environments. Enhancing the cell-relabeling process with additional datasets \cite{Graham_Jahanifar_etal._2021} could improve the representation of underrepresented cell types and enhance overall model performance. Also, incorporating additional data sources, such as multi-modal imaging or complementary staining methods, may address limitations related to cell type differentiation and class imbalance. Exploring other foundation models \cite{Vorontsov_Bozkurt_etal._2024,Zimmermann_Vorontsov_etal._2024} or introducing additional modalities \cite{Ding_Wagner_etal._2024,Vaidya_Zhang_etal._2025} may provide alternative architectures better suited to specific tasks or offer improved efficiency. Implementing more complex knowledge distillation techniques \cite{Houyon_Cioppa_etal._2023,Zhang_Song_etal._2019} could further optimize the model's performance and adaptability. Additionally, deeper integration with QuPath or other digital pathology software could provide pathologists more control over cell quantification analysis directly within the QuPath interface, thereby increasing accessibility and usability. Such enhancements would not only refine model performance but also ensure greater adaptability and scalability within various clinical environments. Finally, extensive validation of the model by pathologists and benchmarking against independent datasets are essential steps toward establishing the model's reliability and fostering confidence in its clinical utility.

\section*{Acknowledgments} 
This work was funded in part by the Research Council of Norway grant no. 309439 SFI Visual Intelligence, and the North Norwegian Health Authority grant no. HNF1521-20.

\bibliographystyle{IEEEtran}
\begin{sloppypar}
\begin{thebibliography}{99}

\bibitem{chaplot2020neural} Chaplot, Devendra Singh, et al. "Neural topological slam for visual navigation." Proceedings of the IEEE/CVF conference on computer vision and pattern recognition. 2020.

\bibitem{maksymets2021thda} Maksymets, Oleksandr, et al. "Thda: Treasure hunt data augmentation for semantic navigation." Proceedings of the IEEE/CVF International Conference on Computer Vision. 2021.

\bibitem{mezghan2022memory} Mezghan, Lina, et al. "Memory-augmented reinforcement learning for image-goal navigation." 2022 IEEE/RSJ International Conference on Intelligent Robots and Systems (IROS). IEEE, 2022.

\bibitem{al2022zero} Al-Halah, Ziad, Santhosh Kumar Ramakrishnan, and Kristen Grauman. "Zero experience required: Plug \& play modular transfer learning for semantic visual navigation." Proceedings of the IEEE/CVF Conference on Computer Vision and Pattern Recognition. 2022.

\bibitem{ye2021auxiliary} Ye, Joel, et al. "Auxiliary tasks and exploration enable objectgoal navigation." Proceedings of the IEEE/CVF international conference on computer vision. 2021.

\bibitem{chaplot2020object} Chaplot, Devendra Singh, et al. "Object goal navigation using goal-oriented semantic exploration." Advances in Neural Information Processing Systems 33 (2020)

\bibitem{ramakrishnan2022poni} Ramakrishnan, Santhosh Kumar, et al. "Poni: Potential functions for objectgoal navigation with interaction-free learning." Proceedings of the IEEE/CVF Conference on Computer Vision and Pattern Recognition. 2022.

\bibitem{ramrakhya2022habitat} Ramrakhya, Ram, et al. "Habitat-web: Learning embodied object-search strategies from human demonstrations at scale." Proceedings of the IEEE/CVF Conference on Computer Vision and Pattern Recognition. 2022.

\bibitem{mousavian2019visual} Mousavian, Arsalan, et al. "Visual representations for semantic target driven navigation." 2019 International Conference on Robotics and Automation (ICRA). IEEE, 2019.

\bibitem{dhariwal2021diffusion} Dhariwal, Prafulla, and Alexander Nichol. "Diffusion models beat gans on image synthesis." Advances in neural information processing systems 34 (2021)

\bibitem{ho2022classifier} Ho, Jonathan, and Tim Salimans. "Classifier-free diffusion guidance." arXiv preprint arXiv:2207.12598 (2022).

\bibitem{nichol2021glide} Nichol, Alex, et al. "Glide: Towards photorealistic image generation and editing with text-guided diffusion models." arXiv preprint arXiv:2112.10741 (2021)

\bibitem{brooks2023instructpix2pix} Brooks, Tim, Aleksander Holynski, and Alexei A. Efros. "Instructpix2pix: Learning to follow image editing instructions." Proceedings of the IEEE/CVF Conference on Computer Vision and Pattern Recognition. 2023.

\bibitem{fu2023guiding} Fu, Tsu-Jui, et al. "Guiding instruction-based image editing via multimodal large language models." arXiv preprint arXiv:2309.17102 (2023).

\bibitem{geng2024instructdiffusion} Geng, Zigang, et al. "Instructdiffusion: A generalist modeling interface for vision tasks." Proceedings of the IEEE/CVF Conference on Computer Vision and Pattern Recognition. 2024.

\bibitem{zhou2024minedreamer} Zhou, Enshen, et al. "Minedreamer: Learning to follow instructions via chain-of-imagination for simulated-world control." arXiv preprint arXiv:2403.12037 (2024).

\bibitem{zhou2023esc} Zhou, Kaiwen, et al. "Esc: Exploration with soft commonsense constraints for zero-shot object navigation." International Conference on Machine Learning. PMLR, 2023.

\bibitem{yu2023l3mvn} Yu, Bangguo, Hamidreza Kasaei, and Ming Cao. "L3mvn: Leveraging large language models for visual target navigation." 2023 IEEE/RSJ International Conference on Intelligent Robots and Systems (IROS). IEEE, 2023.

\bibitem{gadre2023cows} Gadre, Samir Yitzhak, et al. "Cows on pasture: Baselines and benchmarks for language-driven zero-shot object navigation." Proceedings of the IEEE/CVF Conference on Computer Vision and Pattern Recognition. 2023.

\bibitem{shah2023navigation} Shah, Dhruv, et al. "Navigation with large language models: Semantic guesswork as a heuristic for planning." Conference on Robot Learning. PMLR, 2023.

\bibitem{cai2024bridging} Cai, Wenzhe, et al. "Bridging zero-shot object navigation and foundation models through pixel-guided navigation skill." 2024 IEEE International Conference on Robotics and Automation (ICRA). IEEE, 2024.

\bibitem{yu2023co} Yu, Bangguo, Hamidreza Kasaei, and Ming Cao. "Co-NavGPT: Multi-robot cooperative visual semantic navigation using large language models." arXiv preprint arXiv:2310.07937 (2023).

\bibitem{wu2024voronav} Wu, Pengying, et al. "Voronav: Voronoi-based zero-shot object navigation with large language model." arXiv preprint arXiv:2401.02695 (2024).

\bibitem{qin2023mp5} Qin, Yiran, et al. "Mp5: A multi-modal open-ended embodied system in minecraft via active perception." arXiv preprint arXiv:2312.07472 (2023).

\bibitem{du2024learning} Du, Yilun, et al. "Learning universal policies via text-guided video generation." Advances in Neural Information Processing Systems 36 (2024).

\bibitem{ajay2024compositional} Ajay, Anurag, et al. "Compositional foundation models for hierarchical planning." Advances in Neural Information Processing Systems 36 (2024).

\bibitem{liang2024skilldiffuser} Liang, Zhixuan, et al. "Skilldiffuser: Interpretable hierarchical planning via skill abstractions in diffusion-based task execution." Proceedings of the IEEE/CVF Conference on Computer Vision and Pattern Recognition. 2024.

\bibitem{heusel2017gans} Heusel, Martin, et al. "Gans trained by a two time-scale update rule converge to a local nash equilibrium." Advances in neural information processing systems 30 (2017).

\bibitem{zhang2018unreasonable} Zhang, Richard, et al. "The unreasonable effectiveness of deep features as a perceptual metric." Proceedings of the IEEE conference on computer vision and pattern recognition. 2018.

\bibitem{brown2020language} Brown, Tom B. "Language models are few-shot learners." arXiv preprint arXiv:2005.14165 (2020).

\bibitem{podell2023sdxl} Podell, Dustin, et al. "Sdxl: Improving latent diffusion models for high-resolution image synthesis." arXiv preprint arXiv:2307.01952 (2023).

\bibitem{brohan2022rt} Brohan, Anthony, et al. "Rt-1: Robotics transformer for real-world control at scale." arXiv preprint arXiv:2212.06817 (2022).

\bibitem{brohan2023rt} Brohan, Anthony, et al. "Rt-2: Vision-language-action models transfer web knowledge to robotic control." arXiv preprint arXiv:2307.15818 (2023).

\bibitem{li2024manipllm} Li, Xiaoqi, et al. "Manipllm: Embodied multimodal large language model for object-centric robotic manipulation." Proceedings of the IEEE/CVF Conference on Computer Vision and Pattern Recognition. 2024.

\bibitem{shah2023vint} Shah, Dhruv, et al. "ViNT: A foundation model for visual navigation." arXiv preprint arXiv:2306.14846 (2023).

\bibitem{liu2024visual} Liu, Haotian, et al. "Visual instruction tuning." Advances in neural information processing systems 36 (2024).

\bibitem{hu2021lora} Hu, Edward J., et al. "Lora: Low-rank adaptation of large language models." arXiv preprint arXiv:2106.09685 (2021).

\bibitem{qin2023supfusion} Qin, Yiran, et al. "SupFusion: Supervised LiDAR-camera fusion for 3D object detection." Proceedings of the IEEE/CVF International Conference on Computer Vision. 2023.

\bibitem{qin2024worldsimbench} Qin, Yiran, et al. "Worldsimbench: Towards video generation models as world simulators." arXiv preprint arXiv:2410.18072 (2024).

\bibitem{yu2025gamefactory} Yu, Jiwen, et al. "GameFactory: Creating New Games with Generative Interactive Videos." arXiv preprint arXiv:2501.08325 (2025).

\bibitem{zhou2024code} Zhou, Enshen, et al. "Code-as-Monitor: Constraint-aware Visual Programming for Reactive and Proactive Robotic Failure Detection." arXiv preprint arXiv:2412.04455 (2024).

\bibitem{zhang2024ad} Zhang, Zaibin, et al. "AD-H: Autonomous Driving with Hierarchical Agents." arXiv preprint arXiv:2406.03474 (2024).

\bibitem{wang2024toward} Wang, Chaoqun, et al. "Toward Accurate Camera-based 3D Object Detection via Cascade Depth Estimation and Calibration." arXiv preprint arXiv:2402.04883 (2024).

\bibitem{huang2024story3d} Huang, Yuzhou, et al. "Story3d-agent: Exploring 3d storytelling visualization with large language models." arXiv preprint arXiv:2408.11801 (2024).

\bibitem{savinov2018semi} Savinov, Nikolay, Alexey Dosovitskiy, and Vladlen Koltun. "Semi-parametric topological memory for navigation." arXiv preprint arXiv:1803.00653 (2018).

\bibitem{majumdar2022zson} Majumdar, Arjun, et al. "Zson: Zero-shot object-goal navigation using multimodal goal embeddings." Advances in Neural Information Processing Systems 35 (2022): 32340-32352.

\bibitem{yadav2023offline} Yadav, Karmesh, et al. "Offline visual representation learning for embodied navigation." Workshop on Reincarnating Reinforcement Learning at ICLR 2023. 2023.

\bibitem{yadav2023ovrl} Yadav, Karmesh, et al. "Ovrl-v2: A simple state-of-art baseline for imagenav and objectnav." arXiv preprint arXiv:2303.07798 (2023).

\bibitem{sun2024fgprompt} Sun, Xinyu, et al. "FGPrompt: fine-grained goal prompting for image-goal navigation." Advances in Neural Information Processing Systems 36 (2024).

\bibitem{zhu2017target} Zhu, Yuke, et al. "Target-driven visual navigation in indoor scenes using deep reinforcement learning." 2017 IEEE international conference on robotics and automation (ICRA). IEEE, 2017.

\bibitem{koh2024generating} Koh, Jing Yu, Daniel Fried, and Russ R. Salakhutdinov. "Generating images with multimodal language models." Advances in Neural Information Processing Systems 36 (2024).

\bibitem{krantz2022instance} Krantz, Jacob, et al. "Instance-specific image goal navigation: Training embodied agents to find object instances." arXiv preprint arXiv:2211.15876 (2022).

\bibitem{schulman2017proximal} Schulman, John, et al. "Proximal policy optimization algorithms." arXiv preprint arXiv:1707.06347 (2017).

\bibitem{anderson2018evaluation} Anderson, Peter, et al. "On evaluation of embodied navigation agents." arXiv preprint arXiv:1807.06757 (2018).

\bibitem{lin2024navcot} Lin, Bingqian, et al. "NavCoT: Boosting LLM-Based Vision-and-Language Navigation via Learning Disentangled Reasoning." arXiv preprint arXiv:2403.07376 (2024).

\bibitem{NavGPT} Zhou, Gengze, Yicong Hong, and Qi Wu. "Navgpt: Explicit reasoning in vision-and-language navigation with large language models." Proceedings of the AAAI Conference on Artificial Intelligence.

\bibitem{hahn2021no} Hahn, Meera, et al. "No rl, no simulation: Learning to navigate without navigating." Advances in Neural Information Processing Systems 34 (2021): 26661-26673.

\bibitem{li2025t2isafety} Li, Lijun, et al. "T2ISafety: Benchmark for Assessing Fairness, Toxicity, and Privacy in Image Generation." arXiv preprint arXiv:2501.12612 (2025).

\bibitem{an2024agfsync} An, Jingkun, et al. "AGFSync: Leveraging AI-Generated Feedback for Preference Optimization in Text-to-Image Generation." arXiv preprint arXiv:2403.13352 (2024).


\end{thebibliography}
\end{sloppypar}

\clearpage
\beginsupplement
\section*{Appendix}
\renewcommand{\thesubsection}{S\arabic{subsection}}

\subsection{\label{chap:S1}PanNuke and MoNuSAC preprocessing}
The PanNuke dataset comprises a set of 7,901 RGB patches, each with dimensions of $256 \times 256$ pixels, which we set as the standard patch size for our analysis. In contrast, the MoNuSAC dataset encompasses 294 images of heterogeneous dimensions. To standardize the MoNuSAC images with our experiments, we implement a standardization protocol. Specifically, for images exceeding the dimensions of $256 \times 256$ pixels, we segment them into equal-sized patches and apply mirror padding to the remaining portions to avoid information loss at the peripherals. Patches with dimensions less than $128 \times 128$ pixels are excluded from the dataset due to the insufficient resolution to capture relevant cellular details. For patches where either dimension falls between 128 and 256 pixels, we employ upsampling to achieve the standard patch size. As a result, we obtain a total of 2,823 RGB patches derived from the MoNuSAC dataset for subsequent analysis. For additional details on the MoNuSAC data preparation process, refer to the source code \cite{Shvetsov_2025a}.
\clearpage

\subsection{\label{chap:S2}Data usage for the methodology}

\counterwithin{figure}{subsection}
\renewcommand{\thefigure}{S\arabic{subsection}}

\begin{figure}[h!]
    \centering
    \includegraphics[width=\textwidth, height=0.85\textheight, keepaspectratio]{images/A2.pdf}
    \caption{Overview of the methodology for cross-labeling, dataset refinement, and model comparison. (1) Cross-relabeling - training and testing cell classification models, (2) Cross-relabeling - using cell classification models to create refined dataset, (3) Fine-tuning and training models for comparison, (4) Student knowledge distillation with refined dataset}
    \label{fig:S2}
\end{figure}
\clearpage

\subsection{\label{chap:S3}Confusion matrices for classification models}
\counterwithin{figure}{subsection}
\renewcommand{\thefigure}{S\arabic{subsection}.\arabic{figure}}

\begin{figure}[h!]
    \centering
    \includegraphics[width=\textwidth, height=0.4\textheight, keepaspectratio]{images/A3_1.pdf}
    \caption{Confusion matrix for PanNuke trained model}
    \label{fig:S3.1}
\end{figure}

\begin{figure}[h!]
    \centering
    \includegraphics[width=\textwidth, height=0.4\textheight, keepaspectratio]{images/A3_2.pdf}
    \caption{Confusion matrix for MoNuSAC trained model}
    \label{fig:S3.2}
\end{figure}

\clearpage

\subsection{\label{chap:S4}Datasets cell counts}

\counterwithin{table}{subsection}
\renewcommand{\thetable}{S\arabic{subsection}}

\begin{table}[h!]
\renewcommand{\arraystretch}{2.0}
\centering
\caption{\label{tab:S4}Cell counts for PanNuke, MoNuSAC and refined datasets. Numbers in parentheses indicate preprocessed cell counts for cell classifier models training and testing.}
%\adjustbox{max width=\textwidth}{%
\begin{tabular}{|l|c|c|c|}
\hline
%\rowcolor{gray!30}
Cell type & PanNuke & MoNuSAC & Refined \\
\hline
Neoplastic & 77,403 (68,031) & - & 105,451 \\
\hline
Epithelial & 26,572 (23,207) & - & 29,926 \\
\hline
Epithelial (benign and malignant) & - & 31,402 & - \\
\hline
Inflammatory & 32,276 & - & - \\
\hline
Lymphocytes & - & 37,045 (33,104) & 65,275 \\
\hline
Neutrophils & - & 1,355 (1,252) & 3,833 \\
\hline
Macrophage & - & 1,842 (1,695) & 3,410 \\
\hline
Dead & 2,908 & - & 2,908 \\
\hline
Connective & 50,585 & - & 50,585 \\
\hline
\end{tabular}
%
%}
\end{table}



\clearpage

\subsection{\label{chap:S5}Definition of validation metrics}
\counterwithin{equation}{subsection}
\renewcommand{\theequation}{\arabic{equation}}

\subsubsection{\label{chap:S5.1}R\textsuperscript{2}}
The coefficient of determination, denoted as $R^2$, is a statistical measure that represents the proportion of variance in the dependent variable that is predictable from the independent variables. In the context of cell quantification in pathology, $R^2$ is used to assess how well the predicted quantities of different cell types in a patch align with the actual quantities observed in the ground truth data, with higher values representing more accurate quantification. $R^2$ is defined as
\begin{equation*}
R^2 = 1 - \frac{\sum_{i=1}^n (y_i - \hat{y}_i)^2}{\sum_{i=1}^n (y_i - \bar{y})^2},
\end{equation*}
where $y_i$ represents the actual number of cells of a specific type in the $i$-th image, $\hat{y}_i$ represents the predicted number of cells of that type in the $i$-th image, $\bar{y}$ is the mean of the actual numbers across all images, and $n$ is the total number of images in the dataset.

The $R^2$ metric has a range of $(-\infty, 1]$. An $R^2$ of 1 indicates perfect prediction, where all predicted values exactly match the actual values. An $R^2$ of 0 suggests that the model explains none of the variability of the response data around its mean. If $R^2$ is negative, it indicates that the model performs worse than a model that simply predicts the mean of the actual values for all observations.

\subsubsection{\label{chap:S5.2}PQ}
Panoptic Quality ($PQ$) is a comprehensive metric used to evaluate the performance of segmentation models in tasks that require both instance segmentation and classification. $PQ$ provides a single score that encapsulates both the detection accuracy (i.e., how many objects were correctly identified) and the segmentation quality (i.e., how accurately the objects' boundaries were delineated). This metric is particularly useful in multiclass scenarios where each pixel is classified into distinct categories, such as different cell types in pathology images.

$PQ$ is calculated as the product of two terms: Detection Quality ($DQ$) and Segmentation Quality ($SQ$). It can be expressed as
\begin{equation*}
PQ = DQ \cdot SQ,
\end{equation*}
where
\begin{equation*}
DQ = \frac{TP}{TP + 0.5\, FP + 0.5\, FN},
\end{equation*}
\begin{equation*}
SQ = \frac{\sum_{(p, g) \in \mathcal{M}} IoU(p, g)}{TP}.
\end{equation*}
In these formulas, $TP$ denotes the number of correctly matched instances between ground truth and prediction, $FP$ denotes the predicted instances that have no corresponding ground truth, $FN$ denotes the ground truth instances that were not detected, $IoU(p, g)$ is the Intersection over Union for a pair of matched instances $p$ (prediction) and $g$ (ground truth), and $\mathcal{M}$ is the set of matched pairs.

The $PQ$ metric is calculated for each class and is averaged across classes to provide a global performance measure.

The $PQ$ score has a range of $[0, 1.0]$, where a higher score indicates better performance in both detecting and segmenting the instances correctly. A $PQ$ of 1 signifies perfect identification and segmentation of all instances, whereas a $PQ$ of 0 indicates that no instances were correctly identified and segmented.

\clearpage

\subsection{\label{chap:S6}Segmentation and Detection quality metrics for teacher and student models}

\begin{table}[h!]
\renewcommand{\arraystretch}{2.0}
\centering
\caption{Segmentation and detection quality for student and teacher models (CI 95\%)}
\label{tab:S6}
%\adjustbox{max width=\textwidth}{%
\begin{tabular}{|l|c|c|}
\hline
%\rowcolor{gray!30}
Metric & Teacher & Student \\
\hline
$SQ_{neoplastic}$ & 0.819 (0.815--0.823) & 0.824 (0.819--0.828) \\
\hline
$SQ_{lymphocyte}$ & 0.795 (0.788--0.802) & 0.790 (0.783--0.796) \\
\hline
$SQ_{connective}$ & 0.770 (0.762--0.776) & 0.780 (0.772--0.786) \\
\hline
$SQ_{dead}$ & 0.659 (0.623--0.688) & 0.657 (0.624--0.695) \\
\hline
$SQ_{epithelial}$ & 0.780 (0.770--0.790) & 0.788 (0.779--0.797) \\
\hline
$SQ_{macrophage}$ & 0.788 (0.760--0.810) & 0.757 (0.730--0.783) \\
\hline
$SQ_{neutrofil}$ & 0.782 (0.761--0.801) & 0.775 (0.759--0.792) \\
\hline
$DQ_{neoplastic}$ & 0.706 (0.692--0.719) & 0.727 (0.712--0.741) \\
\hline
$DQ_{lymphocyte}$ & 0.675 (0.656--0.698) & 0.713 (0.691--0.734) \\
\hline
$DQ_{connective}$ & 0.566 (0.546--0.584) & 0.583 (0.565--0.602) \\
\hline
$DQ_{dead}$ & 0.410 (0.361--0.465) & 0.435 (0.306--0.561) \\
\hline
$DQ_{epithelial}$ & 0.668 (0.639--0.694) & 0.673 (0.644--0.702) \\
\hline
$DQ_{macrophage}$ & 0.657 (0.583--0.727) & 0.615 (0.531--0.703) \\
\hline
$DQ_{neutrofil}$ & 0.691 (0.625--0.753) & 0.729 (0.679--0.778) \\
\hline
\end{tabular}
%
%}
\end{table}

\clearpage

\subsection{\label{chap:S7}QuPath integration method}
We adopt an integration strategy leveraging the paquo \cite{Bayer_AG} library, a Python package that enables direct interaction with QuPath’s internal API, thereby facilitating seamless data exchange without intermediate conversion steps. The data processing pipeline (\hyperref[fig:S7]{Appendix Figure S7}) begins with the acquisition of WSIs and their associated annotations from QuPath, which are represented as Shapely \cite{Gillies_Wel_etal._2024} polygons. Utilizing paquo, we directly read, create, and modify these annotations and detections within a QuPath project in the Python environment. Images are then cropped using these polygons and processed by cell segmentation and classification models employing standard vision processing toolkits such as OpenCV, pyvips, and PyTorch. Additionally, QuPath employs Groovy scripts to initiate a Python process that starts the entire pipeline from QuPath graphical interface: fetching polygons, extracting images from them, and running deep learning model inference on the cropped images. 
The results are returned to QuPath, leveraging paquo's Python bindings to manipulate QuPath data while minimizing the computational overhead typically associated with cross-environment communication.

\counterwithin{figure}{subsection}
\renewcommand{\thefigure}{S\arabic{subsection}}

\begin{figure}[h!]
    \centering
    \includegraphics[width=\textwidth]{images/A7.pdf}
    \caption{QuPath integration workflow using Python environment}
    \label{fig:S7}
\end{figure}

Compared to traditional workflows that involve exporting annotations as GeoJSON, classifying them in Python, and reimporting them into QuPath, our approach offers several advantages. We eliminate the need to switch between programming languages, providing a cohesive and streamlined development process entirely within QuPath software and removing the necessity to use other tools. Meanwhile, we avoid storing annotations as intermediate JSON files unless required for external use or archiving. By conducting the entire inference and post-processing workflow within the Python environment, we leverage the power and flexibility of Python libraries for image processing and machine learning. This approach also enables adjustments to any set of labels and models, thereby improving its applicability.

%\hfill

The distilled model and QuPath integration code are packaged into a Docker container, enabling streamlined execution with the Docker engine. Detailed integration code and deployment instructions can be found in the GitHub repository \cite{Shvetsov_2025b}.

Despite these benefits, we acknowledge that the paquo library is a proof‑of‑concept project in its early development stage and has not been tested across all versions of QuPath.

\clearpage

\subsection{\label{chap:S8}Data and code availability statement}
All datasets, models, and code used in this study are publicly available and can be obtained from the repositories listed below. 
The PanNuke \cite{Gamper_Koohbanani_etal._2019} and MoNuSAC \cite{Verma_Kumar_etal._2021} datasets are publicly accessible, and download information along with detailed descriptions can be found in their respective articles. Preprocessing scripts for PanNuke and MoNuSAC data, as well as individual cell extraction scripts, are available on GitHub \cite{Shvetsov_2025a}. The H-Optimus foundation model used in our experiments can be downloaded from the HuggingFace repository \cite{hoptimus2024}, and model information is available on GitHub \cite{Saillard_Jenatton_etal._2024}. In addition, the integration code for QuPath and the distilled model packaged in a Docker container are provided in the repository \cite{Shvetsov_2025b}, and paquo Python library is available from the authors GitHub repository \cite{Bayer_AG}.
\clearpage

\end{document}



\end{document}

\endinput
%%
%% End of file `elsarticle-template-harv.tex'.


