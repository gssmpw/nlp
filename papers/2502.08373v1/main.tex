\documentclass[preprint,12pt,authoryear]{elsarticle}

\usepackage{amssymb}
\usepackage{amsmath}
\usepackage{algorithmicx}
\usepackage{algorithm}
\usepackage{array}
\usepackage{url}
\usepackage{multirow}
\usepackage{booktabs}
\usepackage{algpseudocode}
\usepackage{hyperref}

\journal{Expert Systems with Applications}

\begin{document}

\begin{frontmatter}


\title{Uncertainty Aware Human-machine Collaboration in Camouflaged Object Detection} %% Article title


\author[label1]{Ziyue Yang}\ead{zyyang@hdu.edu.cn}
\author[label1]{Kehan Wang}\ead{242050376@hdu.edu.cn}
\author[label1]{Yuhang Ming}\ead{yuhang.ming@hdu.edu.cn}
\author[label1]{Yong Peng}\ead{yongpeng@hdu.edu.cn}
\author[label1]{Han Yang}\ead{43008@hdu.edu.cn}
\author[label2]{Qiong Chen}\ead{chenqiong@cethik.com}
\author[label1]{Wanzeng Kong\corref{cor1}}\ead{kongwanzeng@hdu.edu.cn}


\affiliation[label1]{organization={College of Computer Science, Hangzhou Dianzi University},
             addressline={1158 Baiyang Road}, 
             city={Hangzhou},
             postcode={310018}, 
             state={Zhejiang},
             country={China}}

\affiliation[label2]{organization={Hangzhou Hikvision Digital Technology Company},
             addressline={555 Qianmo Road}, 
             city={Hangzhou},
             postcode={310051}, 
             state={Zhejiang},
             country={China}}

%% Abstract
\begin{abstract}
Camouflaged Object Detection (COD), the task of identifying objects concealed within their environments, has seen rapid growth due to its wide range of practical applications. A key step toward developing trustworthy COD systems is the estimation and effective utilization of uncertainty. In this work, we propose a human-machine collaboration framework for classifying the presence of camouflaged objects, leveraging the complementary strengths of computer vision (CV) models and noninvasive brain-computer interfaces (BCIs). Our approach introduces a multiview backbone to estimate uncertainty in CV model predictions, utilizes this uncertainty during training to improve efficiency, and defers low-confidence cases to human evaluation via RSVP-based BCIs during testing for more reliable decision-making. We evaluated the framework in the CAMO dataset, achieving state-of-the-art results with an average improvement of 4.56\% in balanced accuracy (BA) and 3.66\% in the F1 score compared to existing methods. For the best-performing participants, the improvements reached 7.6\% in BA and 6.66\% in the F1 score. Analysis of the training process revealed a strong correlation between our confidence measures and precision, while an ablation study confirmed the effectiveness of the proposed training policy and the human-machine collaboration strategy. In general, this work reduces human cognitive load, improves system reliability, and provides a strong foundation for advancements in real-world COD applications and human-computer interaction. Our code and data are available at: https://github.com/ziyuey/Uncertainty-aware-human-machine-collaboration-in-camouflaged-object-identification.
\end{abstract}

%%Graphical abstract
\begin{graphicalabstract}
\includegraphics[width=\textwidth]{Figure_1.pdf}
\end{graphicalabstract}

%%Research highlights
\begin{highlights}
\item Introduces a novel framework combining computer vision (CV) models with RSVP-based brain-computer interfaces (BCIs) to enhance camouflaged object detection (COD).
\item Develops a multiview backbone that estimates prediction confidence, leveraging strong and weak augmentations to assess uncertainty and guide model training.
\item Incorporates RSVP-based BCIs to improve detection accuracy, where low-confidence cases from the CV model are re-evaluated using human cognitive responses.
\item Achieves 4.56$\%$ higher balanced accuracy (BA) and 3.66$\%$ higher F1 score on the CAMO dataset compared to existing methods, with the best cases reaching 95.60$\%$ BA and 95.49$\%$ F1 score.
\item Proposes a selective human intervention strategy, deferring only uncertain cases to human evaluation, significantly reducing cognitive effort while maintaining high reliability.
\end{highlights}

%% Keywords
\begin{keyword}
Camouflaged Object Detection, Human-Machine Collaboration, Brain-Computer Interface, Uncertainty Estimation, Computer Vision.
\end{keyword}


\cortext[cor1]{Corresponding author.}

\end{frontmatter}

\section{Introduction}
\label{sec1}
Rapid adoption of deep learning has highlighted concerns about the robustness and transparency of neural network predictions. Addressing these concerns is critical for the advancement of trustworthy artificial intelligence (AI) \cite{kaur2022trustworthy}. A key aspect of building trust is to enable models to assess and communicate their own uncertainty \cite{10.1145/3675392,gawlikowski2023survey}, which can guide decisions to defer to human operators or seek additional data in complex scenarios. In particular, in tasks where humans and AI have complementary strengths, effective human-machine collaboration represents a promising trend in the evolution of digital societies. However, how to leverage uncertainty quantification for such collaboration, and its impact, remains underexplored.
Camouflaged Object Detection (COD) is a well-suited task for such synergy. COD focuses on identifying objects concealed within their environments, making it both challenging and intriguing \cite{lv2023toward}. This field has grown rapidly due to its practical applications, such as defect detection in manufacturing \cite{xiong2021attention}, pest control in agriculture \cite{rustia2020application}, segmentation of lesions in medical diagnosis \cite{fan2020pranet}, pedestrian detection in nighttime environments\cite{yang2024dual}, and even creative pursuits such as image blending \cite{suo2021neuralhumanfvv}. Advances in image-level camouflaged object segmentation have been driven by models such as DPSNet \cite{li2024dpsnet}, JCNet \cite{jiang2023camouflaged}, SINet-V2 \cite{fan2021concealed}, and DGNet \cite{ji2023deep}, supported by recognized datasets and benchmarks \cite{bi2021rethinking}. COD tasks involve two key components: identifying whether camouflaged objects are present and segmenting them when they are. However, much of the current research focuses on the segmentation stage, assuming the model can produce a continuous mask map where a black saliency map signifies the absence of a camouflaged object. However, since camouflaged objects are not guaranteed to be present, a critical first step is to detect their presence before proceeding to segmentation. While precise localization is vital for applications like medical diagnostics, in tasks such as rescue operations or pest monitoring, detecting the presence of camouflaged objects takes precedence. Therefore, our study centers on binary classification to determine the presence or absence of camouflaged objects.

Despite rapid progress in COD, concerns persist about the safety and reliability of computer vision (CV) models. Deep learning models often function as black boxes \cite{hassija2024interpreting} and can struggle with challenges such as small targets, incomplete objects, and complex backgrounds (e.g. noise, obstructions, or shadows) \cite{bi2021rethinking}. Teaching models to admit uncertainty, essentially saying 'I don't know', remains a significant challenge.

In contrast, the human brain excels at adapting to diverse environments, recognizing subtle patterns, and identifying hidden objects in complex scenarios, such as low-light conditions or cluttered backgrounds. This adaptability allows humans to detect hidden or camouflaged targets with high accuracy, even when features are partially occluded. However, manual search for such targets is time-consuming. Non-invasive brain-computer interfaces (BCIs) offer a novel solution to this challenge \cite{kim2019high}. When individuals encounter rare targets in visual sequences, their brain activity generates event-related potentials (ERPs), particularly the P300 component, which can help detect targets. Researchers have exploited this phenomenon by combining rapid serial visual presentation (RSVP) paradigms with BCIs technology to evoke ERPs in response to visual stimuli, allowing efficient classification of target images \cite{zhang2020benchmark}--even under challenging conditions where targets are camouflaged and hidden \cite{lian2023eeg, zhou2024rsvp}. 

In this study, we propose a novel framework for human-machine collaboration in COD, leveraging model uncertainty as the bridge between CV models and human intervention. Specifically, CV models handle the bulk of images, exploiting their ability to process large volumes of data in parallel, while humans, through BCIs, provide instinctive responses to challenging or unusual images flagged by model uncertainty. To the best of our knowledge, no existing research has been proposed that uses and evaluates the uncertainty-based human-machine partnership in COD. Our contributions include the following.

\begin{itemize}
    \item Multiview backbone for COD: We propose a backbone that evaluates the confidence across multiple views for each image.
    \item Uncertainty-aware training policy: Our training strategy improves CV models by effectively using information from the training set.
    \item Human-machine collaboration paradigm for COD: For samples with low confidence from the CV model, we fuse RSVP-based BCIs results with model predictions to achieve more reliable decision making.
\end{itemize}

\section{Related Work}
\subsection{Uncertainty Quantification}
Uncertainty estimation, or confidence assessment, in neural network predictions has emerged as a major research focus in the machine learning community \cite{smith2024uncertainty}. Current approaches to uncertainty modeling typically fall into three categories: Monte Carlo Dropout \cite{neal2012bayesian, Moreau_2022_WACV, gal2017concrete, kang2023active}, the Bootstrap model \cite{osband2016deep}, and the Gaussian Mixture Model \cite{9666964, zhang2019short}. These methods have been extensively explored in various domains, demonstrating promising results \cite{abdar2021review}. However, most models merely display uncertainty without leveraging it to guide further actions. To address this gap, recent advances have begun to incorporate uncertainty into training strategies \cite{li2023disc,cordeiro2023longremix}. Although promising, these cutting-edge efforts have focused primarily on tasks that involve noisy label learning. For standard supervised learning tasks, the effectiveness of uncertainty-informed training strategies remains unclear. In this work, we extend the confidence learning method based on two views introduced in \cite{li2023disc} and apply it to the COD problem. Our approach not only integrates uncertainty into the model's learning process, but also flexibly delegates uncertain samples to human-based RSVP systems for enhanced decision making.

\subsection{Camouflaged Object Detection (COD)}
In recent years, numerous deep learning-based COD models have been developed \cite{liang2024systematic}, while recent research has started to place more emphasis on uncertainty. Yi Zhang et al. introduced PUENet \cite{10159663}, which uses a Bayesian conditional variational auto-encoder for predictive uncertainty estimation. Yixuan Lyu et al. \cite{10183371} proposed the Uncertainty-Edge Dual Guide model, which combines probabilistic uncertainty with deterministic edge information for accurate COD. Jiawei Liu et al. \cite{9706783} developed a confidence-based COD framework with dynamic supervision, producing both camouflage masks and aleatoric uncertainty estimates, showing superior performance. Fan Yang et al. \cite{9710683} integrated Bayesian learning with Transformer reasoning, leveraging both deterministic and probabilistic information to improve detection accuracy. Furthermore, Aixuan Li et al. \cite{9578707} proposed an adversarial learning network for higher-order similarity measures and confidence estimation. Current research mainly addresses uncertainty in segmentation tasks, focusing on generating confidence maps for boundary distinction, while this study aims to model the uncertainty in identifying object presence to enhance classification accuracy.

\subsection{RSVP-based BCIs for Target Detection}
RSVP-based BCIs have received significant attention in recent years, particularly in the domain of target detection, due to their efficiency in processing rapid visual stimuli and eliciting robust neural responses such as potentials related to P300 events. Research advancements have focused on optimizing RSVP paradigms to enhance system performance, with notable contributions including the use of adaptive parameter tuning and hybrid paradigms that integrate steady-state visual evoked potentials to improve detection accuracy and user experience \cite{jalilpour2020novel}. Novel electroencephalogram(EEG) decoding algorithms, such as deep learning frameworks that take advantage of convolutional neural networks \cite{santamaria2020eeg} and attention mechanisms\cite{wang2020linking}, have further enhanced classification performance, while collaborative approaches of multiple users have demonstrated the potential for improved accuracy through collective neural signal analysis \cite{9931160}. The introduction of benchmark data sets has standardized the evaluation of algorithms and facilitated reproducible research \cite{zhang2020benchmark}. Furthermore, the integration of multimodal data, including EEG and eye tracking, has shown promise in addressing signal noise and enhancing target detection reliability \cite{mao2023cross}, cementing RSVP-BCIs as a crucial interface for bridging neuroscience and real-world applications. The most related work to ours is by Yujie Cui et al.\cite{cui2022dynamic}, who proposed a human-computer fusion method called Dynamic Probability Integration for nighttime vehicle detection. Their approach uses a probability assignment method to assign classification weights between different information sources, which requires full human participation in the EEG-based RSVP task. In contrast, our model reduces human effort by using uncertainty to guide collaboration, with humans only evaluating high-uncertainty samples from the CV model, thus improving efficiency.

\section{Method}
\begin{figure*}[ht]
    \centering
    \includegraphics[width=\textwidth]{Figure_1.pdf} % Replace with your image file name
    \caption{Overall framework of this study. During training, the dataset is split into high and low-confidence sets, with augmentations applied to low-confidence samples and the split updated per epoch; during testing, high-uncertainty samples are classified using the RSVP program, while high-confidence samples are handled by the CV model, enhancing COD performance.}
    \label{fig:overall}
\end{figure*}
The framework of this paper is illustrated in Figure
~\ref{fig:overall}. In the training stage, the CV model is pretrained on the complete dataset. The data is then divided into a high confidence set (\(X_{hc}\)) and a low confidence set (\(X_{lc}\)) based on uncertainty, the split being updated once per epoch. Weak and strong augmentations are applied to \(X_{lc}\), creating \(X_{lc'}\), while \(X_{hc}\) is oversampled to match the size of \(X_{lc'}\). In the testing phase, high-uncertainty samples are transferred to the EEG-based RSVP program for classification, while high-confidence samples are predicted using the CV model. This two-stage process combines the strengths of CV models and RSVP-based BCIs systems to enhance COD performance.

\subsection{Dataset}
The data set used in this study consists of 2,500 images, evenly divided into camouflage target images and background images. Our data set is sourced from the publicly available CAMO dataset, which includes 1,250 camouflage target images. Using the ground truth map for each image, we generated paired camouflage target and background images through a combination of manual and automated methods. The resulting data set is available on the provided GitHub link. Although the CAMO-COCO dataset was considered, its background images differ substantially from the camouflage target images, making it less aligned with our focus. This study emphasizes scenarios where camouflage targets are embedded within similar backgrounds, reflecting more realistic application contexts.

\subsection{CV Model}
\subsubsection{Multi-view based Uncertainty Estimation}
We propose a backbone model that evaluates confidence in multiple views for each image. For each sample \((x, y)\), we apply one weak augmentation to \(x\), resulting in \(x_w\), and \(n\) strong augmentations, resulting in \(\{x_{s1}, x_{s2}, \dots, x_{sn}\}\). To compute uncertainty, we used the cross-entropy (CE) between two distributions \(p\) and \(q\) defined as:
\begin{equation}
\text{CE}(p, q) = -\sum_{i=1}^{C} p_i \log(q_i),
\label{eq:ce}
\end{equation}
where \(C\) is the number of classes, \(p_i\) is the ground truth probability of class \(i\), and \(q_i\) is the predicted probability of class \(i\). Hence, uncertainty is calculated as the mean cross-entropy between the weakly augmented sample and each strongly augmented sample:
\begin{equation}
\text{Uncertainty} = \frac{1}{n} \sum_{j=1}^{n} \text{CE}(p_w, p_{sj}),
\label{eq:uncertainty}
\end{equation}
where $n$ denotes the number of strong augmentations, while $x_w$ and $x_{sj}$ are processed by the trained CV model to obtain $p_w$ and $p_{sj}$, respectively. This approach quantifies uncertainty by assessing the consistency between weakly and strongly augmented views of the same sample.

\subsubsection{Strong and weak augmentation}
In this study, strong and weak data augmentations serve two primary purposes: evaluating multiview-based uncertainty and enhancing the model's robustness during training.

Weak augmentations include operations such as random horizontal flipping, slight rotation, and random cropping. In contrast, strong augmentations involve more significant perturbations, such as cropping, color transformations, image quality adjustments, occlusion, and composite augmentations to simulate complex scene variations. The specific augmentation method for both weak and strong augmentations is selected randomly for each iteration.

\begin{algorithm}[H]
\caption{Uncertainty-Aware Training Policy}
\label{alg:training_procedure}
\begin{algorithmic}[0]
    \State \textbf{Input:} Dataset $W$, Total Epochs $E$, Batch Size $B$, Warm-up Epochs $\text{Warm}$, Ramp-up Length $\text{rampup\_length}$, Augmentation Strength $M$, Augmentation Times $N$, Clean Dataset Filtering Method $\text{Method\_c}$, Consecutive Clean Rounds $t$

    \State Load pre-trained weights from ImageNet
    \State Initialize epoch counter $e \gets 0$
    
    \While{$e < E$}
    
        \For{$i = 1$ \textbf{to} $|W|$}
            \State Perform warm-up and compute loss using cross-entropy loss: $l(x, y) = \text{CE}(f(x), y)$
        \EndFor
    
        \If{$e > \text{Warm}$}
            \State Build high-confidence $X_{hc}(K)$ and low-confidence $X_{lc}(K)$ sets
            \State $X_{hc}(K) \gets \text{Samples from last } t \text{ epochs of training}$
            \State $X_{lc}(K) \gets \text{All training samples} \setminus X_{hc}(K)$
            \State $num\_iter \gets \frac{|X_{hc}(K)|}{B}$
    
            \For{$iter = 1$ \textbf{to} $num\_iter$}
                \State Select high-confidence samples $\{(X, Y)\} \sim X(k)$
                \State Select low-confidence samples $\{(U, Y)\} \sim U(k)$
    
                \For{$b = 1$ \textbf{to} $B$}
                    \State $x_b  \gets \text{Batch samples for high-confidence set}$
                    \State $u_{b1} \gets  \text{StrongDataAugment}(u_b)$
                    \State $u_{b2} \gets  \text{WeakDataAugment}(u_b)$
                    \State $r \gets \text{clip}\left(\frac{e - \text{Warm}}{\text{rampup\_length}}, 0, 1\right) \times \lambda_u$
                    \State $loss = l(x_b, y_b) + [l(u_{b1}, y_b) + l(u_{b2}, y_b)] \times r$
                \EndFor
            \EndFor
        \EndIf
        
        \State Update high- and low-confidence sets using \texttt{Method\_c}
        \State Increment epoch: $e = e + 1$
    \EndWhile
    
    \State \textbf{Output:} Model Parameters $\theta$
\end{algorithmic}
\end{algorithm}

The uncertainty-aware training policy outlined in Algorithm 1 aims to enhance model robustness by leveraging high-confidence and low-confidence samples during training. Initially, during the warm-up phase, the model is trained using standard loss functions. Once the warm-up is complete, the training data is divided into high-confidence and low-confidence sets. High-confidence samples \(X_{hc}(K)\) are selected from the last training epochs $t$, while low-confidence samples \(X_{lc}(K)\) are derived from the remaining data. The model is then trained on these two sets, with low-confidence samples undergoing strong and weak data augmentations (detailed in Section 3.2.2 ) to improve generalization. For confident set selection, \texttt{Method\_c} includes five strategies based on uncertainty estimation: (1) splitting Low-confidence and High-confidence images in a 1:2 ratio (Ratio 1:2); (2) splitting in a 2:1 ratio (Ratio 2:1); (3) dynamically partitioning the samples using a threshold (details: dividing data into intervals using a 0.1 threshold, then traversing each interval in order of increasing accuracy, selecting the first interval with a sample and accuracy lower than the overall accuracy, and using the median of that interval as the dynamic threshold) (Dynamic Threshold); (4) labeling images with consistent predicted and ground-truth labels after augmentation as High confidence, otherwise labeling them as Low-confidence (Consistent Labeling); and (5) labeling images as High confidence if at least one prediction matches the ground truth after augmentation, otherwise labeling them as Low confidence (At Least One Match). A key feature of the policy is the gradual increase in weight loss for low-confidence samples, controlled by the factor $r$, which evolves throughout training. This dynamic weighting helps stabilize training by initially focusing on high-confidence samples and progressively incorporating low-confidence data. The subsets of high- and low-confidence samples are updated after each epoch using a filtering method to ensure data quality. This strategy effectively balances the influence of clean, high-confidence data with more uncertain, low-confidence data, leading to improved model performance and robustness.


\subsection{RSVP-based EEG Model}\label{stage-2-brain-machine-synergy}

\subsubsection{Participants and Data
Recording}
The study was reviewed and approved by the 	
Ethics Committee of Second Affiliated Hospital of Zhejiang University, College of Medicine and the protocol number is IRB-2024-1535. Signed informed consent was obtained from each participant.
A total of 8 participants (mean age: 24.10 years) were recruited for
the brain-machine collaborative RSVP target detection paradigm study.
All participants had normal or corrected vision and no neurological
problems. Before the experiment, each participant was informed of the
potential risks and signed a written informed consent form.\\
EEG data were recorded using the Synamps2 system (64 channels,
NeuroScan, Inc.) at a sampling rate of 1000 Hz. In this study, 62
electrodes were used to record EEG signals, following the
international 10-20 electrode placement system, with the reference
electrode placed at the vertex. Before data recording, the impedance
of the electrodes was measured and adjusted to ensure it remained below
25 k$\Omega$. EEG data were initially filtered using a finite impulse response filter with a frequency range between 0.1 and 40 Hz and
then resampled to 250 Hz for classification. Segments of events of
one second were extracted from the data starting from the onset of the
stimulus.

\subsubsection{RSVP Paradigm Design}
\begin{figure}[ht]
    \centering
    \includegraphics[width=\linewidth]{Figure_2.pdf} % Replace with your image
    \caption{RSVP Paradigm Design. At the start of each trial, a fixation
cross appeared in the center of the screen. The stimuli were then presented
at a frequency of 1 Hz. The participants were given ample rest time between
blocks. The experiment ensured that there were at least three nontarget
images between any two target images. All stimulus images were displayed
on a 512 × 512 resolution monitor with a refresh rate of 60 Hz.}
    \label{fig:RSVP}
\end{figure}
The experimental setup is shown in Figure~\ref{fig:RSVP}. In the experiment,
participants were seated in a quiet room and instructed to identify
camouflage targets within a sequence of stimuli. Original frames
were used to present the stimuli. An EEG amplifier was employed to
capture the participants\textquotesingle{} brain activity, with no
button pressing required when a target was detected.\\
The experimental procedure and paradigm parameters are depicted in the
figure. The experiment consisted of five blocks, each comprising 11 trials.
In each block, 75 target images and 530 untarget images were presented. Before the experiment began, a set
of images containing camouflage target and background images was
presented to the participants as a guideline, informing them of the
target they needed to identify. At the start of each trial, a fixation
cross appeared in the center of the screen. The stimuli were then presented
at a frequency of 1 Hz. The participants were given ample rest time between
blocks. The experiment ensured that there were at least three nontarget
images between any two target images. All stimulus images were displayed
on a 512 × 512 resolution monitor with a refresh rate of 60 Hz.

\subsubsection{Brain-Machine Collaboration}
In the test phase, samples with high uncertainty from the CV model are sent to human evaluation. These samples are treated as "potential targets." To construct the test target sequence, at least three nontarget images from the training set are inserted between each pair of target images, ensuring activation of the P300 component during stimulus presentation. An RSVP-based EEG model, trained in the training set, is then used to detect the presence of targets in the test set. The predictions for high-uncertainty samples are replaced with the output of the RSVP-based model, enabling effective human-machine collaboration.

\section{Results}
%-------------------------------------------------------------------------
In this section, we evaluate the effectiveness of the proposed method. The data set was divided into training and testing sets with a 9: 1 ratio and the training set was further divided into training and validation subsets, also with a 9: 1 ratio. During the experiments, we used early stopping with a patience parameter set to 10. To ensure the stability and reliability of the results, we performed experiments using five different random seeds: 37, 12, 6, 99, and 123. We applied five strong augmentations and one weak augmentation. The reported results are the mean and standard deviation for the five runs. All experiments were performed on a GeForce RTX 4090 GPU. The evaluation of our experimental results was based on balanced accuracy (BA) and F1 score, with their respective formulas as follows:

\begin{equation}
\text{BA} = \frac{1}{2} \left( \frac{TP}{TP + FN} + \frac{TN}{TN + FP} \right),
\end{equation}
where \( TP \) is the number of true positives, \( FN \) is the number of false negatives, \( TN \) is the number of true negatives and \( FP \) is the number of false positives.

\begin{equation}
\text{F1} = 2 \times \frac{\text{Precision} \times \text{Recall}}{\text{Precision} + \text{Recall}},
\end{equation}
where \(\text{Precision} = \frac{TP}{TP + FP}\) and \(\text{Recall} = \frac{TP}{TP + FN}\).


\subsection{Results for different training policies}

In designing the training strategy, we compared different data augmentation and confident set selection methods, using the Swin Transformer (SwinT) as the backbone network. For data augmentation, we tested three strategies: (1) applying strong and weak augmentations separately to high-confidence and low-confidence images; (2) augmenting only high-confidence images (H-only); and (3) augmenting only low-confidence images (L-only). Details of the confident set selection approach are provided in Section 3.2.3. The results, shown in Table~\ref{2}, reveal that augmenting only low-confidence samples, along with a fixed ratio of 2:1 between low- and high-confidence images, yielded the best performance. This configuration achieved a BA of 89.92\% and an F1 score of 90.40\%. These findings suggest that it is crucial to effectively mine and utilize low-confidence samples during training. Based on these results, we selected the L-only augmentation strategy combined with a fixed ratio of 2:1 for subsequent experiments.

\begin{table}[ht]
\centering
\footnotesize
\caption{Comparison of Different Data Augmentation Strategies and Confident Set Selection Policies}
\label{1}
\begin{tabular}{clcc}
\toprule
\textbf{\parbox{2.5cm}{Augmentation}} & \textbf{\parbox{2.5cm}{Selection}} & \textbf{BA (\%)} $\uparrow$ & \textbf{F1 (\%)} $\uparrow$ \\
\midrule
Both & Ratio 1:2 & 89.36 $\pm$ 1.15 & 89.73 $\pm$ 1.20 \\
Both & Ratio 2:1 & 88.08 $\pm$ 1.39 & 88.83 $\pm$ 1.13 \\
Both & Dynamic Threshold & 88.64 $\pm$ 2.21 & 89.48 $\pm$ 1.92 \\
Both & Consistent Labeling & 88.56 $\pm$ 1.51 & 88.80 $\pm$ 1.69 \\
Both & At Least One Match & 89.04 $\pm$ 1.40 & 89.51 $\pm$ 1.25 \\
H-only & Ratio 1:2 & 88.56 $\pm$ 1.66 & 89.25 $\pm$ 1.40 \\
H-only & Ratio 2:1 & 88.96 $\pm$ 0.60 & 89.46 $\pm$ 0.56 \\
H-only & Dynamic Threshold & 89.84 $\pm$ 1.56 & 90.27 $\pm$ 1.42 \\
H-only & Consistent Labeling & 88.08 $\pm$ 0.95 & 88.52 $\pm$ 1.01 \\
H-only & At Least One Match & 88.24 $\pm$ 1.45 & 88.62 $\pm$ 1.47 \\
L-only & Ratio 1:2 & 88.80 $\pm$ 1.16 & 89.25 $\pm$ 1.21 \\
\textbf{L-only} & \textbf{Ratio 2:1} & \textbf{89.92 $\pm$ 0.76} & \textbf{90.40 $\pm$ 0.64} \\
L-only & Dynamic Threshold & 89.68 $\pm$ 0.91 & 90.01 $\pm$ 0.93 \\
L-only & Consistent Labeling & 88.16 $\pm$ 0.73 & 88.78 $\pm$ 0.75 \\
L-only & At Least One Match & 87.52 $\pm$ 1.42 & 88.05 $\pm$ 1.31 \\
\bottomrule
\end{tabular}
\end{table}



\begin{table}[ht]
\centering
\footnotesize  % 使用 \footnotesize 而不是 \small
\caption{Comparison with Competing Methods on Our Dataset}  % 使用标题大小写
\label{2}
%与最先进技术(SOTA)方法在我们的数据集上的比较。表中选取了两个伪装目标检测(COD)模型,并使用白色像素点作为阈值进行图像分割。同时,选取了六个主流的计算机视觉(CV)模型进行对比分析。我们还展示了在本方法中,每个步骤对性能提升的贡献。
\begin{tabular}{lcc}
\toprule
\textbf{Model} & \textbf{BA(\%)} $\uparrow$ & \textbf{F1(\%)} $\uparrow$ \\
\midrule
DGNet \cite{ji2023gradient}& 78.80 & 76.10 \\
SINet-V2 \cite{fan2021concealed}& 72.40 & 69.10 \\
ResNet-18 \cite{he2016deep}& 86.00 $\pm$ 0.88 & 86.44 $\pm$ 0.72 \\
ResNeXt-50 \cite{xie2017aggregated}& 86.56 $\pm$ 1.51 & 87.16 $\pm$ 1.13 \\
DenseNet121 \cite{huang2017densely}& 87.76 $\pm$ 0.60 & 88.12 $\pm$ 0.84 \\
EfficientNetB0 \cite{tan2019efficientnet}& 84.40 $\pm$ 1.52 & 84.44 $\pm$ 1.47 \\
SwinT \cite{Liu_2021_ICCV}& 88.00 $\pm$ 1.77 & 88.83 $\pm$ 1.54 \\
ViT-B16 \cite{dosovitskiy2020image}& 86.40 $\pm$ 1.13 & 86.56 $\pm$ 1.25 \\
\textbf{SwinT + Training Policy} & \textbf{89.92 $\pm$ 0.76} & \textbf{90.40 $\pm$ 0.64} \\
\textbf{SwinT + Training Policy + RSVP(Mean)} & \textbf{92.56 $\pm$ 1.12} & \textbf{92.49 $\pm$ 1.17} \\
\textbf{SwinT + Training Policy + RSVP(Best)} & \textbf{95.60$\pm$0.15} & \textbf{95.49$\pm$0.16}\\
\bottomrule
\end{tabular}
\end{table}

\subsection{Comparison with Existing Methods and Ablation Study}
We evaluated our approach against two categories of existing methods, as summarized in Table~\ref{2}. The first category includes camouflaged object segmentation models, represented by DGNet and SINet-V2. These models generate completely black output maps for background images, requiring a dynamic threshold (optimized on the training set) to analyze the proportion of white pixels in the Ground Truth image and detect camouflaged targets. The second category comprises widely used CV models, including CNN-based architectures such as ResNet-18, ResNeXt-50, and DenseNet121; lightweight models such as EfficientNetB0; and transformer-based models, such as the Swin Transformer (SwinT) and Vision Transformer (ViT-B16). 

For these CV models, we perform the following steps: 1) load the models and use pre-trained weights from ImageNet1k; 2) freeze the pre-trained weights; 3) retrieve the number of input features for the classification head; 4) replace the original classification head with a new fully connected layer consisting of four linear transformation layers, which map to 256, 32, and 8 dimensions, and finally output two classes; 5) Unfreeze the parameters of the new classification head. The key findings of the experiments include the following.

\begin{itemize}
\item Performance of COD Models: COD models, such as DGNet and SINet-V2, achieved approximately 75\% balanced accuracy (BA) in detecting the presence or absence of camouflaged targets. Although effective in edge detection and segmentation, their reliance on local pixel-level features limits their ability to perform well in tasks that require a broader contextual understanding.

\item Performance of CV Models: Among the CV models tested, SwinT exhibited the best performance in our data set and was selected as the backbone model for our approach.

\item Impact of Training Policy: The confidence-based training policy applied to SwinT improved BA by 1.92\% and the F1 score by 1.57\%, demonstrating the ability of the policy to enhance model robustness and precision.

\item RSVP Integration: Combining RSVP with the training policy resulted in additional improvements of 2.64\% in BA and 2.09\% in F1 score on average. The best participant achieved a remarkable balanced accuracy of 95.60\% and an F1 score of 95.49\%, setting a new state-of-the-art in this evaluation.
\end{itemize}
%-------------------------------------------------------------------------
\subsection{Results for RSVP-based BCIs}
To determine the optimal EEG model, we compared several leading EEG analysis methods. EEGNet, designed specifically for EEG signal classification, demonstrated outstanding performance. PLNet leverages phase-locking features of Event-Related Potentials (ERPs) for spatio-temporal feature extraction, excelling in single-session RSVP EEG classification. PPNN, a pyramid-structured parallel neural network, captures multiscale spatio-temporal features, improving classification. EEG-Inception adapts computer vision concepts to ERP detection, improving classification accuracy for ERP-based BCIs. LMDA-Net (LMDA) combines channel and depth attention modules to improve classification by integrating multidimensional features. EEG-Conformer (Conformer) combines convolutional networks and transformers, capturing both local and long-range dependencies, thus improving classification performance.

Table~\ref{3} shows the BA and F1 scores for each model across all participants. As seen in Table~\ref{3}, EEGNet performed the best for six participants, while Conformer excelled for two. Table~\ref{4} provides the mean performance statistics for each model. Among all models, EEGNet achieved the highest performance in the RSVP-based BCIs task, with a BA of 76.76\% and the highest classification precision. Although PLNet, PPNN and EEG-Inception had a lower accuracy, LMDA and Conformer showed better F1 scores but did not outperform EEGNet in accuracy. Based on these results, we selected EEGNet as the optimal EEG model for our task.

\begin{table}[H]
\centering
\caption{Comparison of Classification Accuracy Across Different EEG Models for Various Participants}  % 使用标题大小写
\footnotesize  % 使用 \footnotesize 而不是 \small
\label{3}
\setlength{\tabcolsep}{1mm}  % 可调整列之间的距离
\begin{tabular}{ccccccc}  % 这里的 {ccccccc} 是列格式
\toprule
 & \multicolumn{2}{c}{Subject 1} & \multicolumn{2}{c}{Subject 2} & \multicolumn{2}{c}{Subject 3} \\
\cmidrule{2-3}\cmidrule{4-5}\cmidrule{6-7}
& BA(\%) $\uparrow$ & $F_1$(\%) $\uparrow$ & BA(\%) $\uparrow$ & $F_1$(\%) $\uparrow$ & BA(\%) $\uparrow$ & $F_1$(\%) $\uparrow$  \\ \midrule
EEGNet & 63.66$\pm$1.33 & 58.26$\pm$1.10 & 85.41$\pm$0.77 & 80.25$\pm$1.31 & 77.77$\pm$1.14 & 69.12$\pm$1.38 \\
PLNet & 60.63$\pm$2.05 & 56.62$\pm$1.61 & 84.02$\pm$1.24 & 78.31$\pm$3.51 & 74.79$\pm$1.55 & 68.80$\pm$2.08 \\
PPNN & 55.54$\pm$1.83 & 54.68$\pm$1.22 & 78.55$\pm$1.44 & 79.58$\pm$1.14 & 71.19$\pm$1.82 & 71.09$\pm$1.61 \\
EEGInception & 59.25$\pm$4.05 & 56.63$\pm$5.76 & 72.61$\pm$6.54 & 74.68$\pm$8.01 & 73.29$\pm$6.13 & 72.56$\pm$3.34 \\
LMDA & 58.41$\pm$1.99 & 59.03$\pm$2.18 & 83.89$\pm$2.50 & 84.48$\pm$1.86 & 69.64$\pm$1.05 & 71.38$\pm$1.20 \\
Conformer & 67.63$\pm$1.91 & 61.26$\pm$1.45 & 88.82$\pm$1.11 & 85.60$\pm$1.76 & 74.67$\pm$2.32 & 71.10$\pm$1.84 \\
\\
\toprule
 & \multicolumn{2}{c}{Subject 4} & \multicolumn{2}{c}{Subject 5} & \multicolumn{2}{c}{Subject 6} \\
\cmidrule{2-3}\cmidrule{4-5}\cmidrule{6-7}
& BA(\%) $\uparrow$ & $F_1$(\%) $\uparrow$ & BA(\%) $\uparrow$ & $F_1$(\%) $\uparrow$ & BA(\%) $\uparrow$ & $F_1$(\%) $\uparrow$ \\ \midrule
EEGNet & 83.17$\pm$1.47 & 75.21$\pm$1.74 & 80.57$\pm$1.62 & 69.59$\pm$1.64 & 79.98$\pm$1.47 & 66.76$\pm$2.18 \\
PLNet & 80.97$\pm$2.14 & 74.72$\pm$1.39 & 78.38$\pm$2.55 & 69.59$\pm$2.20 & 71.96$\pm$2.57 & 64.12$\pm$1.41 \\
PPNN & 77.70$\pm$3.07 & 77.64$\pm$1.65 & 74.02$\pm$1.77 & 71.93$\pm$1.78 & 66.66$\pm$3.04 & 66.32$\pm$2.17 \\
EEGInception & 76.03$\pm$5.85 & 74.61$\pm$4.84 & 72.91$\pm$5.52 & 70.31$\pm$3.91 & 76.40$\pm$6.20 & 66.89$\pm$9.17 \\
LMDA & 74.38$\pm$1.92 & 74.85$\pm$1.68 & 71.84$\pm$2.57 & 71.63$\pm$1.85 & 72.16$\pm$2.07 & 70.80$\pm$1.82 \\
Conformer & 79.09$\pm$1.59 & 77.04$\pm$0.98 & 76.97$\pm$1.92 & 71.85$\pm$1.48 & 75.35$\pm$3.20 & 69.11$\pm$1.96 \\
\\
\toprule
 & \multicolumn{2}{c}{Subject 7} & \multicolumn{2}{c}{Subject 8} \\
\cmidrule{2-3}\cmidrule{4-5}
& BA(\%) $\uparrow$ & $F_1$(\%) $\uparrow$ & BA(\%) $\uparrow$ & $F_1$(\%) $\uparrow$ \\ \cmidrule{1-5}
EEGNet & 70.56$\pm$1.31 & 63.13$\pm$1.31 & 72.98$\pm$1.62 & 65.21$\pm$1.38 \\
PLNet & 62.81$\pm$2.79 & 58.60$\pm$1.06 & 67.38$\pm$2.41 & 62.53$\pm$1.13 \\
PPNN & 62.19$\pm$2.47 & 61.98$\pm$2.08 & 61.89$\pm$1.95 & 61.37$\pm$1.52 \\
EEGInception & 64.01$\pm$5.22 & 61.36$\pm$3.44 & 69.94$\pm$3.79 & 64.48$\pm$7.97 \\
LMDA & 59.63$\pm$1.99 & 60.68$\pm$2.17 & 67.40$\pm$1.78 & 66.53$\pm$1.87 \\
Conformer & 66.58$\pm$2.08 & 63.36$\pm$1.27 & 69.59$\pm$1.31 & 65.61$\pm$1.26 \\
\bottomrule
\end{tabular}
\end{table}





\begin{table}[ht]
\caption{Summary of the Mean Performance Statistics for Each EEG Classification Model}  % 标题大小写
\footnotesize  % 使用 \footnotesize 而不是 \small
\label{4}
\centering
\begin{tabular}{lcc}
\toprule
\textbf{EEG Model} & \textbf{BA(\%)} $\uparrow$ & \textbf{F1(\%)} $\uparrow$ \\
\midrule
\textbf{EEGNet} \cite{lawhern2018eegnet}& \textbf{76.76$\pm$6.92} & \textbf{68.44$\pm$6.64} \\
PLNet \cite{zang2021deep}& 72.61$\pm$8.26 & 66.66$\pm$7.37 \\
PPNN \cite{li2021phase}& 68.46$\pm$8.06 & 68.07$\pm$8.20 \\
EEGInception\cite{santamaria2020eeg} & 70.56$\pm$7.76 & 67.69$\pm$8.60 \\
LMDA \cite{miao2023lmda}& 69.67$\pm$7.93 & 69.92$\pm$7.80 \\
\textbf{Conformer} \cite{song2022eeg}& \textbf{74.83$\pm$7.07} & \textbf{70.62$\pm$7.55} \\
\bottomrule
\end{tabular}
\end{table}



%-------------------------------------------------------------------------
\subsection{Results for Human-machine Collaboration}
%Combine our model with COD models

For human-machine collaboration, we selected the highest proportion of samples with the highest uncertainty and replaced their CV-predicted labels with EEG-predicted labels. The results for different proportions of uncertain samples are shown in Table~\ref{5}. The model performed optimally when the correction proportion was set to 20\%. A lower proportion did not fully capture the advantages of human input, while a higher proportion reduced the contribution of the CV model and significantly increased manual effort. Fortunately, the 20\% correction ratio provided an effective balance, allowing human input without excessive stress. Importantly, for all proportions tested, the model outperformed the baseline CV model (SwinT with our training policy), demonstrating the general benefits of human-machine collaboration.

\begin{table*}[ht]
\caption{Test Set Results Comparing the Replacement of Low-Confidence Image Predictions with EEG-Predicted Labels at Different Uncertainty Ratios} % 标题大小写
\footnotesize  % 使用 \footnotesize 而不是 \small
\label{5}
%在不同不确定性样本比例下,通过使用脑电预测标签替代低置信度图像的计算机视觉(CV)预测标签的比较。实验采用最佳训练策略,即仅对低置信度(Low-confidence)图像进行强弱增强,并按 2:1 的比例分割低置信度和高置信度(High-confidence)图像。表中展示了在此策略下训练的模型在测试集上的实验结果。
\setlength{\tabcolsep}{1mm}  % 调整列之间的距离
\begin{tabular*}{\textwidth}{@{\extracolsep\fill}lcccc}
\toprule
 & \multicolumn{2}{@{}c@{}}{10\%} & \multicolumn{2}{@{}c@{}}{20\%}\\
\cmidrule{2-3}\cmidrule{4-5}
 & BA(\%) & $F_1$(\%) & BA(\%) & $F_1$(\%) \\ \midrule
Subject1 & 91.52$\pm$0.29 & 91.63$\pm$0.28 & 91.52$\pm$0.13 & 91.42$\pm$0.17\\
Subject2 & 92.64$\pm$0.29 & 92.73$\pm$0.28 & 94.64$\pm$0.24 & 94.60$\pm$0.24\\
Subject3 & 91.36$\pm$0.27 & 91.39$\pm$0.27 & 92.00$\pm$0.38 & 91.78$\pm$0.41\\
Subject4 & 91.52$\pm$0.21 & 91.71$\pm$0.21 & 92.72$\pm$0.23 & 92.86$\pm$0.23\\
Subject5 & 92.48$\pm$0.13 & 92.66$\pm$0.13 & 94.24$\pm$0.21 & 94.30$\pm$0.22\\
Subject6 & 92.88$\pm$0.07 & 93.01$\pm$0.07 & 93.28$\pm$0.18 & 93.31$\pm$0.17\\
Subject7 & 91.76$\pm$0.27 & 91.88$\pm$0.28 & 90.96$\pm$0.24 & 90.83$\pm$0.25\\
Subject8 & 92.56$\pm$0.33 & 92.72$\pm$0.32 & 93.04$\pm$0.33 & 93.00$\pm$0.33\\
Mean & 92.09$\pm$0.62 & 92.22$\pm$0.63 & \textbf{92.80$\pm$1.23} & \textbf{92.76$\pm$1.28}\\
\\
\toprule
 & \multicolumn{2}{@{}c@{}}{30\%} & \multicolumn{2}{@{}c@{}}{40\%} \\
 \cmidrule{2-3}\cmidrule{4-5}
 & BA(\%) & $F_1$(\%) & BA(\%) & $F_1$(\%)\\ \midrule
Subject1 & 90.48$\pm$0.13 & 90.24$\pm$0.13 & 88.16$\pm$0.09 & 87.54$\pm$0.10\\
Subject2 & 95.60$\pm$0.15 & 95.49$\pm$0.16 & 94.40$\pm$0.11 & 94.12$\pm$0.13\\
Subject3 & 90.96$\pm$0.21 & 90.41$\pm$0.25 & 89.92$\pm$0.13 & 89.18$\pm$0.14\\
Subject4 & 93.36$\pm$0.09 & 93.42$\pm$0.09 & 93.20$\pm$0.16 & 93.18$\pm$0.16\\
Subject5 & 93.12$\pm$0.13 & 93.08$\pm$0.12 & 91.20$\pm$0.11 & 91.00$\pm$0.11\\
Subject6 & 92.48$\pm$0.18 & 92.40$\pm$0.20 & 88.16$\pm$0.09 & 90.32$\pm$0.13\\
Subject7 & 90.00$\pm$0.34 & 89.49$\pm$0.36 & 88.08$\pm$0.13 & 87.21$\pm$0.15\\
Subject8 & 92.00$\pm$0.11 & 91.66$\pm$0.14 & 90.16$\pm$0.18 & 89.53$\pm$0.22\\
Mean & \textbf{92.25$\pm$1.72} & \textbf{92.02$\pm$1.89} & 90.41$\pm$2.27 & 90.26$\pm$2.31\\
\bottomrule
\end{tabular*}
\end{table*}

\section{Discussion}

%
%-------------------------------------------------------------------------
\subsection{Application Scenarios}
We propose that COD can be divided into two subtasks: identification and location. Our focus is on the binary classification task of determining whether a camouflaged object is present. In scenarios where there is no prior knowledge about the presence of a camouflaged object in an image, a "classify-then-segment" approach aligns better with practical application requirements. Additionally, since the computational cost and runtime of our classification model are significantly lower than those of a segmentation model, this approach also helps conserve computational resources. Furthermore, the integration of our identification module with other location models has the potential to be refined more. For example, an interaction between the location model uncertainty map and the identification module classification uncertainty could be designed collaboratively to enhance the detection performance.

%-------------------------------------------------------------------------
\subsection{Uncertainty Sources}
Uncertainty in machine learning can generally be categorized as aleatoric or epistemic, each arising from different sources. In particular, the uncertainties involved in the identification of camouflaged objects and location detection are different. For camouflaged object identification, aleatoric uncertainty is particularly high when the camouflaged objects are indistinct and difficult to distinguish, such as animals in a forest blending seamlessly with their surroundings, like branches or foliage. In these scenarios, the inherent ambiguity in the environment and the multiple possibilities reflected in the training data contribute to this type of uncertainty. In contrast, epistemic uncertainty stems from a lack of knowledge of unseen or unfamiliar data. For example, it arises when the camouflaged scenes or objects differ entirely from those in the training dataset, such as new backgrounds or novel camouflage techniques. In this work, we use the term "predictive uncertainty" to refer to the overall uncertainty in a given situation, encompassing both aleatoric and epistemic components.

%-------------------------------------------------------------------------
\subsection{Training process}
Table 6 illustrates how the precision of samples within different confidence intervals changes over training epochs in the validation set. The uncertainty is ranked in ascending order, where higher percentages indicate lower confidence levels in the CV model's predictions. The confusion matrix (CM) is represented in the format
$\begin{bmatrix}
TP & FN \\ 
FP & TN 
\end{bmatrix}$, where the true positives (TP), false negatives (FN), false positives (FP), and true negatives (TN) are laid out in matrix form.

A clear trend emerges: the confidence level of the CV model is inversely correlated with prediction accuracy. This highlights the effectiveness and robustness of our multiview uncertainty measurement approach. Moreover, as training progresses, the accuracy of the top 80\% most confident samples improves, while the accuracy of the bottom 20\% least confident samples decreases significantly. A similar pattern is observed on the test set. This decline in accuracy for low-confidence samples underpins the foundation for human-machine collaboration, as it identifies cases where the CV model lacks confidence and could benefit from human intervention.


\begin{table*}[ht]
\centering
\footnotesize
\caption{Accuracy Trends of Samples Across Confidence Intervals During Training on the Validation Set. Uncertainty Is Ranked in Ascending Order, with Higher Percentages Indicating Lower Model Confidence. The Confusion Matrix (CM) Is Represented as $\begin{bmatrix} TP & FN \\ FP & TN \end{bmatrix}$.}
\begin{tabular}{>{\centering\arraybackslash}p{0.5cm} 
                >{\centering\arraybackslash}p{0.4cm} 
                >{\centering\arraybackslash}p{0.4cm} 
                >{\centering\arraybackslash}p{1.2cm}  
                >{\centering\arraybackslash}p{0.4cm} 
                >{\centering\arraybackslash}p{0.4cm} 
                >{\centering\arraybackslash}p{1.2cm}
                >{\centering\arraybackslash}p{0.4cm} 
                >{\centering\arraybackslash}p{0.4cm} 
                >{\centering\arraybackslash}p{1.2cm} }
\toprule
& \multicolumn{3}{c}{0-20\%} & \multicolumn{3}{c}{20-40\%} & \multicolumn{3}{c}{40-60\%}\\
\cmidrule{2-4} \cmidrule{5-7} \cmidrule{8-10}
Epoch & BA & F1 & \multicolumn{1}{c}{CM} & BA & F1 & CM & BA & F1 & \multicolumn{1}{c}{CM}\\ \midrule
\multirow{2}{*}{0} & \multirow{2}{*}{97.2} & \multirow{2}{*}{98.1} & \multirow{2}{*}{$\left[\begin{matrix} 27 & 0 \\ 1 & 17 \end{matrix}\right]$} & \multirow{2}{*}{92.8} & \multirow{2}{*}{94.1} & \multirow{2}{*}{$\left[\begin{matrix} 24 & 0 \\ 3 & 18 \end{matrix}\right]$}  & \multirow{2}{*}{90.4} & \multirow{2}{*}{92.3} & \multirow{2}{*}{$\left[\begin{matrix} 24 & 0 \\ 4 & 17 \end{matrix}\right]$}\\
&&&&&&&&&\\
\multirow{2}{*}{10} & \multirow{2}{*}{94.7} & \multirow{2}{*}{96.2} & \multirow{2}{*}{$\left[\begin{matrix} 26 & 0 \\ 2 & 17 \end{matrix}\right]$} & \multirow{2}{*}{96.4} & \multirow{2}{*}{98.4} & \multirow{2}{*}{$\left[\begin{matrix} 31 & 0 \\ 1 & 13 \end{matrix}\right]$}  & \multirow{2}{*}{95.2} & \multirow{2}{*}{96.0} & \multirow{2}{*}{$\left[\begin{matrix} 24 & 0 \\ 2 & 19 \end{matrix}\right]$}\\
&&&&&&&&&\\
\multirow{2}{*}{20} & \multirow{2}{*}{100} & \multirow{2}{*}{100} & \multirow{2}{*}{$\left[\begin{matrix} 18 & 0 \\ 0 & 27 \end{matrix}\right]$} & \multirow{2}{*}{92.1} & \multirow{2}{*}{94.5} & \multirow{2}{*}{$\left[\begin{matrix} 26 & 0 \\ 3 & 16 \end{matrix}\right]$}  & \multirow{2}{*}{97.5} & \multirow{2}{*}{98.0} & \multirow{2}{*}{$\left[\begin{matrix} 25 & 0 \\ 1 & 19 \end{matrix}\right]$}\\
&&&&&&&&&\\
\multirow{2}{*}{31} & \multirow{2}{*}{100} & \multirow{2}{*}{100} & \multirow{2}{*}{$\left[\begin{matrix} 17 & 0 \\ 0 & 27 \end{matrix}\right]$} & \multirow{2}{*}{97.3} & \multirow{2}{*}{98.1} & \multirow{2}{*}{$\left[\begin{matrix} 26 & 0 \\ 1 & 18 \end{matrix}\right]$}  & \multirow{2}{*}{92.1} & \multirow{2}{*}{94.5} & \multirow{2}{*}{$\left[\begin{matrix} 26 & 0 \\ 3 & 16 \end{matrix}\right]$}\\
&&&&&&&&&\\
\\
\toprule
 & \multicolumn{3}{c}{60-80\%} & \multicolumn{3}{c}{80-100\%} \\
\cmidrule{2-4} \cmidrule{5-7}
Epoch & BA & F1 & \multicolumn{1}{c}{CM} & BA & F1 & \multicolumn{1}{c}{CM}\\ \midrule
\multirow{2}{*}{0} & \multirow{2}{*}{86.0} & \multirow{2}{*}{88.4} & \multirow{2}{*}{$\left[\begin{matrix} 23 & 1 \\ 5 & 16 \end{matrix}\right]$} & \multirow{2}{*}{70.5} & \multirow{2}{*}{59.2} & \multirow{2}{*}{$\left[\begin{matrix} 8 & 6 \\ 5 & 26 \end{matrix}\right]$}\\
&&&&&&\\
\multirow{2}{*}{10} & \multirow{2}{*}{89.9} & \multirow{2}{*}{86.4} & \multirow{2}{*}{$\left[\begin{matrix} 16 & 1 \\ 4 & 24 \end{matrix}\right]$} & \multirow{2}{*}{63.3} & \multirow{2}{*}{51.6} & \multirow{2}{*}{$\left[\begin{matrix} 8 & 7 \\ 8 & 22 \end{matrix}\right]$}\\
&&&&&&\\
\multirow{2}{*}{20} & \multirow{2}{*}{90.1} & \multirow{2}{*}{92.0} & \multirow{2}{*}{$\left[\begin{matrix} 23 & 0 \\ 4 & 18 \end{matrix}\right]$} & \multirow{2}{*}{66.1} & \multirow{2}{*}{61.5} & \multirow{2}{*}{$\left[\begin{matrix} 12 & 9 \\ 6 & 18 \end{matrix}\right]$} \\
&&&&&&\\
\multirow{2}{*}{31} & \multirow{2}{*}{89.6} & \multirow{2}{*}{89.4} & \multirow{2}{*}{$\left[\begin{matrix} 21 & 0 \\ 5 & 19 \end{matrix}\right]$} & \multirow{2}{*}{53.3} & \multirow{2}{*}{53.3} & \multirow{2}{*}{$\left[\begin{matrix} 12 & 10 \\ 11 & 12 \end{matrix}\right]$} \\
&&&&&& \\
\bottomrule
\end{tabular}
\end{table*}

\subsection{Failure Cases}
\begin{figure}[!htb]
    \centering
    \includegraphics[width=\linewidth]{Figure_3.jpg} % Replace with your image
    \caption{Examples of CV model failure cases where the CV model incorrectly identified background as containing camouflaged objects.}
    \label{fig:machine_nocam}
\end{figure}
 We observed an interesting phenomenon in the failure cases of the CV model. For instances where there was an actual camouflaged object, but the CV model misclassified it as background, the camouflaged object was also difficult for the human eye to detect. In contrast, for instances where the CV model mistakenly identified the actual background as containing a camouflaged object (examples shown in Figure~\ref{fig:machine_nocam}), humans could easily recognize that no camouflaged object was present. This discrepancy might stem from the CV model's difficulty in distinguishing between salient objects and camouflaged ones. Unlike human vision, which can rely on contextual and semantic cues to identify salient features in an image, the CV model might struggle to capture these subtleties.
%------------------------------------------------------------------------


%-------------------------------------------------------------------------
\subsection{Challenges and Future work}
We propose a human-machine collaboration framework for COD based on uncertainty estimation. Our method uses a multiview backbone to measure model confidence by analyzing output differences across views, aiding both training and collaboration. Alternative uncertainty estimation methods, such as dropout-based, bootstrap-based, or Gaussian-based approaches, may also be effective. In addition, camouflaged targets are harder to detect than traditional RSVP targets, with lower P300 amplitudes and longer latencies, which may limit human accuracy. Future work may explore EEG responses to camouflaged targets and refine decoding models. We also plan to integrate EEG and eye tracking data to aid in localization, combining them with segmentation models to achieve better human-machine collaboration in both identification and localization.
%------------------------------------------------------------------------
\section{Conclusion}

This study presents an integration method of RSVP-based BCIs with CV models, where low-confidence samples are redirected to human cognitive input. This approach combines the complementary strengths of humans and machines to tackle challenging detection tasks such as COD. In the CAMO data set, our method outperformed state-of-the-art approaches, with an average improvement of 4.56\% in BA and 3.66\% in the F1 score. For the best-performing participants, the improvements reached 7.6\% in BA and 6.66\% in the F1 score. By allowing humans to focus only on uncertain samples, the method significantly reduces the cognitive load and time required for RSVP tasks. Furthermore, given the variability in the performance of the BCIs due to environmental conditions, user state, and electrode quality, this human-machine collaboration framework enhances the overall robustness and reliability of the system. In summary, this research paves the way for future exploration of neuroscience and human-computer interaction, providing a promising framework for addressing complex detection challenges.

\section{Acknowledgments}
This work was supported by National Natural Science Foundation of China (U20B2074, 62471169), Key Research and Development Project of Zhejiang Province (2023C03026, 2021C03001, 2021C03003), Key Laboratory of Brain Machine Collaborative Intelligence of Zhejiang Province (2020E10010), and supported by Zhejiang Provincial Natural Science Foundation of China (No. LQN25F020013).


%% If you have bib database file and want bibtex to generate the
%% bibitems, please use
%%
% \bibliographystyle{elsarticle-harv} 
% \bibliography{main}
% This must be in the first 5 lines to tell arXiv to use pdfLaTeX, which is strongly recommended.
\pdfoutput=1
% In particular, the hyperref package requires pdfLaTeX in order to break URLs across lines.

\documentclass[11pt]{article}

% Change "review" to "final" to generate the final (sometimes called camera-ready) version.
% Change to "preprint" to generate a non-anonymous version with page numbers.
\usepackage{acl}

% Standard package includes
\usepackage{times}
\usepackage{latexsym}

% Draw tables
\usepackage{booktabs}
\usepackage{multirow}
\usepackage{xcolor}
\usepackage{colortbl}
\usepackage{array} 
\usepackage{amsmath}

\newcolumntype{C}{>{\centering\arraybackslash}p{0.07\textwidth}}
% For proper rendering and hyphenation of words containing Latin characters (including in bib files)
\usepackage[T1]{fontenc}
% For Vietnamese characters
% \usepackage[T5]{fontenc}
% See https://www.latex-project.org/help/documentation/encguide.pdf for other character sets
% This assumes your files are encoded as UTF8
\usepackage[utf8]{inputenc}

% This is not strictly necessary, and may be commented out,
% but it will improve the layout of the manuscript,
% and will typically save some space.
\usepackage{microtype}
\DeclareMathOperator*{\argmax}{arg\,max}
% This is also not strictly necessary, and may be commented out.
% However, it will improve the aesthetics of text in
% the typewriter font.
\usepackage{inconsolata}

%Including images in your LaTeX document requires adding
%additional package(s)
\usepackage{graphicx}
% If the title and author information does not fit in the area allocated, uncomment the following
%
%\setlength\titlebox{<dim>}
%
% and set <dim> to something 5cm or larger.

\title{Wi-Chat: Large Language Model Powered Wi-Fi Sensing}

% Author information can be set in various styles:
% For several authors from the same institution:
% \author{Author 1 \and ... \and Author n \\
%         Address line \\ ... \\ Address line}
% if the names do not fit well on one line use
%         Author 1 \\ {\bf Author 2} \\ ... \\ {\bf Author n} \\
% For authors from different institutions:
% \author{Author 1 \\ Address line \\  ... \\ Address line
%         \And  ... \And
%         Author n \\ Address line \\ ... \\ Address line}
% To start a separate ``row'' of authors use \AND, as in
% \author{Author 1 \\ Address line \\  ... \\ Address line
%         \AND
%         Author 2 \\ Address line \\ ... \\ Address line \And
%         Author 3 \\ Address line \\ ... \\ Address line}

% \author{First Author \\
%   Affiliation / Address line 1 \\
%   Affiliation / Address line 2 \\
%   Affiliation / Address line 3 \\
%   \texttt{email@domain} \\\And
%   Second Author \\
%   Affiliation / Address line 1 \\
%   Affiliation / Address line 2 \\
%   Affiliation / Address line 3 \\
%   \texttt{email@domain} \\}
% \author{Haohan Yuan \qquad Haopeng Zhang\thanks{corresponding author} \\ 
%   ALOHA Lab, University of Hawaii at Manoa \\
%   % Affiliation / Address line 2 \\
%   % Affiliation / Address line 3 \\
%   \texttt{\{haohany,haopengz\}@hawaii.edu}}
  
\author{
{Haopeng Zhang$\dag$\thanks{These authors contributed equally to this work.}, Yili Ren$\ddagger$\footnotemark[1], Haohan Yuan$\dag$, Jingzhe Zhang$\ddagger$, Yitong Shen$\ddagger$} \\
ALOHA Lab, University of Hawaii at Manoa$\dag$, University of South Florida$\ddagger$ \\
\{haopengz, haohany\}@hawaii.edu\\
\{yiliren, jingzhe, shen202\}@usf.edu\\}



  
%\author{
%  \textbf{First Author\textsuperscript{1}},
%  \textbf{Second Author\textsuperscript{1,2}},
%  \textbf{Third T. Author\textsuperscript{1}},
%  \textbf{Fourth Author\textsuperscript{1}},
%\\
%  \textbf{Fifth Author\textsuperscript{1,2}},
%  \textbf{Sixth Author\textsuperscript{1}},
%  \textbf{Seventh Author\textsuperscript{1}},
%  \textbf{Eighth Author \textsuperscript{1,2,3,4}},
%\\
%  \textbf{Ninth Author\textsuperscript{1}},
%  \textbf{Tenth Author\textsuperscript{1}},
%  \textbf{Eleventh E. Author\textsuperscript{1,2,3,4,5}},
%  \textbf{Twelfth Author\textsuperscript{1}},
%\\
%  \textbf{Thirteenth Author\textsuperscript{3}},
%  \textbf{Fourteenth F. Author\textsuperscript{2,4}},
%  \textbf{Fifteenth Author\textsuperscript{1}},
%  \textbf{Sixteenth Author\textsuperscript{1}},
%\\
%  \textbf{Seventeenth S. Author\textsuperscript{4,5}},
%  \textbf{Eighteenth Author\textsuperscript{3,4}},
%  \textbf{Nineteenth N. Author\textsuperscript{2,5}},
%  \textbf{Twentieth Author\textsuperscript{1}}
%\\
%\\
%  \textsuperscript{1}Affiliation 1,
%  \textsuperscript{2}Affiliation 2,
%  \textsuperscript{3}Affiliation 3,
%  \textsuperscript{4}Affiliation 4,
%  \textsuperscript{5}Affiliation 5
%\\
%  \small{
%    \textbf{Correspondence:} \href{mailto:email@domain}{email@domain}
%  }
%}

\begin{document}
\maketitle
\begin{abstract}
Recent advancements in Large Language Models (LLMs) have demonstrated remarkable capabilities across diverse tasks. However, their potential to integrate physical model knowledge for real-world signal interpretation remains largely unexplored. In this work, we introduce Wi-Chat, the first LLM-powered Wi-Fi-based human activity recognition system. We demonstrate that LLMs can process raw Wi-Fi signals and infer human activities by incorporating Wi-Fi sensing principles into prompts. Our approach leverages physical model insights to guide LLMs in interpreting Channel State Information (CSI) data without traditional signal processing techniques. Through experiments on real-world Wi-Fi datasets, we show that LLMs exhibit strong reasoning capabilities, achieving zero-shot activity recognition. These findings highlight a new paradigm for Wi-Fi sensing, expanding LLM applications beyond conventional language tasks and enhancing the accessibility of wireless sensing for real-world deployments.
\end{abstract}

\section{Introduction}

In today’s rapidly evolving digital landscape, the transformative power of web technologies has redefined not only how services are delivered but also how complex tasks are approached. Web-based systems have become increasingly prevalent in risk control across various domains. This widespread adoption is due their accessibility, scalability, and ability to remotely connect various types of users. For example, these systems are used for process safety management in industry~\cite{kannan2016web}, safety risk early warning in urban construction~\cite{ding2013development}, and safe monitoring of infrastructural systems~\cite{repetto2018web}. Within these web-based risk management systems, the source search problem presents a huge challenge. Source search refers to the task of identifying the origin of a risky event, such as a gas leak and the emission point of toxic substances. This source search capability is crucial for effective risk management and decision-making.

Traditional approaches to implementing source search capabilities into the web systems often rely on solely algorithmic solutions~\cite{ristic2016study}. These methods, while relatively straightforward to implement, often struggle to achieve acceptable performances due to algorithmic local optima and complex unknown environments~\cite{zhao2020searching}. More recently, web crowdsourcing has emerged as a promising alternative for tackling the source search problem by incorporating human efforts in these web systems on-the-fly~\cite{zhao2024user}. This approach outsources the task of addressing issues encountered during the source search process to human workers, leveraging their capabilities to enhance system performance.

These solutions often employ a human-AI collaborative way~\cite{zhao2023leveraging} where algorithms handle exploration-exploitation and report the encountered problems while human workers resolve complex decision-making bottlenecks to help the algorithms getting rid of local deadlocks~\cite{zhao2022crowd}. Although effective, this paradigm suffers from two inherent limitations: increased operational costs from continuous human intervention, and slow response times of human workers due to sequential decision-making. These challenges motivate our investigation into developing autonomous systems that preserve human-like reasoning capabilities while reducing dependency on massive crowdsourced labor.

Furthermore, recent advancements in large language models (LLMs)~\cite{chang2024survey} and multi-modal LLMs (MLLMs)~\cite{huang2023chatgpt} have unveiled promising avenues for addressing these challenges. One clear opportunity involves the seamless integration of visual understanding and linguistic reasoning for robust decision-making in search tasks. However, whether large models-assisted source search is really effective and efficient for improving the current source search algorithms~\cite{ji2022source} remains unknown. \textit{To address the research gap, we are particularly interested in answering the following two research questions in this work:}

\textbf{\textit{RQ1: }}How can source search capabilities be integrated into web-based systems to support decision-making in time-sensitive risk management scenarios? 
% \sq{I mention ``time-sensitive'' here because I feel like we shall say something about the response time -- LLM has to be faster than humans}

\textbf{\textit{RQ2: }}How can MLLMs and LLMs enhance the effectiveness and efficiency of existing source search algorithms? 

% \textit{\textbf{RQ2:}} To what extent does the performance of large models-assisted search align with or approach the effectiveness of human-AI collaborative search? 

To answer the research questions, we propose a novel framework called Auto-\
S$^2$earch (\textbf{Auto}nomous \textbf{S}ource \textbf{Search}) and implement a prototype system that leverages advanced web technologies to simulate real-world conditions for zero-shot source search. Unlike traditional methods that rely on pre-defined heuristics or extensive human intervention, AutoS$^2$earch employs a carefully designed prompt that encapsulates human rationales, thereby guiding the MLLM to generate coherent and accurate scene descriptions from visual inputs about four directional choices. Based on these language-based descriptions, the LLM is enabled to determine the optimal directional choice through chain-of-thought (CoT) reasoning. Comprehensive empirical validation demonstrates that AutoS$^2$-\ 
earch achieves a success rate of 95–98\%, closely approaching the performance of human-AI collaborative search across 20 benchmark scenarios~\cite{zhao2023leveraging}. 

Our work indicates that the role of humans in future web crowdsourcing tasks may evolve from executors to validators or supervisors. Furthermore, incorporating explanations of LLM decisions into web-based system interfaces has the potential to help humans enhance task performance in risk control.






\section{Related Work}
\label{sec:relatedworks}

% \begin{table*}[t]
% \centering 
% \renewcommand\arraystretch{0.98}
% \fontsize{8}{10}\selectfont \setlength{\tabcolsep}{0.4em}
% \begin{tabular}{@{}lc|cc|cc|cc@{}}
% \toprule
% \textbf{Methods}           & \begin{tabular}[c]{@{}c@{}}\textbf{Training}\\ \textbf{Paradigm}\end{tabular} & \begin{tabular}[c]{@{}c@{}}\textbf{$\#$ PT Data}\\ \textbf{(Tokens)}\end{tabular} & \begin{tabular}[c]{@{}c@{}}\textbf{$\#$ IFT Data}\\ \textbf{(Samples)}\end{tabular} & \textbf{Code}  & \begin{tabular}[c]{@{}c@{}}\textbf{Natural}\\ \textbf{Language}\end{tabular} & \begin{tabular}[c]{@{}c@{}}\textbf{Action}\\ \textbf{Trajectories}\end{tabular} & \begin{tabular}[c]{@{}c@{}}\textbf{API}\\ \textbf{Documentation}\end{tabular}\\ \midrule 
% NexusRaven~\citep{srinivasan2023nexusraven} & IFT & - & - & \textcolor{green}{\CheckmarkBold} & \textcolor{green}{\CheckmarkBold} &\textcolor{red}{\XSolidBrush}&\textcolor{red}{\XSolidBrush}\\
% AgentInstruct~\citep{zeng2023agenttuning} & IFT & - & 2k & \textcolor{green}{\CheckmarkBold} & \textcolor{green}{\CheckmarkBold} &\textcolor{red}{\XSolidBrush}&\textcolor{red}{\XSolidBrush} \\
% AgentEvol~\citep{xi2024agentgym} & IFT & - & 14.5k & \textcolor{green}{\CheckmarkBold} & \textcolor{green}{\CheckmarkBold} &\textcolor{green}{\CheckmarkBold}&\textcolor{red}{\XSolidBrush} \\
% Gorilla~\citep{patil2023gorilla}& IFT & - & 16k & \textcolor{green}{\CheckmarkBold} & \textcolor{green}{\CheckmarkBold} &\textcolor{red}{\XSolidBrush}&\textcolor{green}{\CheckmarkBold}\\
% OpenFunctions-v2~\citep{patil2023gorilla} & IFT & - & 65k & \textcolor{green}{\CheckmarkBold} & \textcolor{green}{\CheckmarkBold} &\textcolor{red}{\XSolidBrush}&\textcolor{green}{\CheckmarkBold}\\
% LAM~\citep{zhang2024agentohana} & IFT & - & 42.6k & \textcolor{green}{\CheckmarkBold} & \textcolor{green}{\CheckmarkBold} &\textcolor{green}{\CheckmarkBold}&\textcolor{red}{\XSolidBrush} \\
% xLAM~\citep{liu2024apigen} & IFT & - & 60k & \textcolor{green}{\CheckmarkBold} & \textcolor{green}{\CheckmarkBold} &\textcolor{green}{\CheckmarkBold}&\textcolor{red}{\XSolidBrush} \\\midrule
% LEMUR~\citep{xu2024lemur} & PT & 90B & 300k & \textcolor{green}{\CheckmarkBold} & \textcolor{green}{\CheckmarkBold} &\textcolor{green}{\CheckmarkBold}&\textcolor{red}{\XSolidBrush}\\
% \rowcolor{teal!12} \method & PT & 103B & 95k & \textcolor{green}{\CheckmarkBold} & \textcolor{green}{\CheckmarkBold} & \textcolor{green}{\CheckmarkBold} & \textcolor{green}{\CheckmarkBold} \\
% \bottomrule
% \end{tabular}
% \caption{Summary of existing tuning- and pretraining-based LLM agents with their training sample sizes. "PT" and "IFT" denote "Pre-Training" and "Instruction Fine-Tuning", respectively. }
% \label{tab:related}
% \end{table*}

\begin{table*}[ht]
\begin{threeparttable}
\centering 
\renewcommand\arraystretch{0.98}
\fontsize{7}{9}\selectfont \setlength{\tabcolsep}{0.2em}
\begin{tabular}{@{}l|c|c|ccc|cc|cc|cccc@{}}
\toprule
\textbf{Methods} & \textbf{Datasets}           & \begin{tabular}[c]{@{}c@{}}\textbf{Training}\\ \textbf{Paradigm}\end{tabular} & \begin{tabular}[c]{@{}c@{}}\textbf{\# PT Data}\\ \textbf{(Tokens)}\end{tabular} & \begin{tabular}[c]{@{}c@{}}\textbf{\# IFT Data}\\ \textbf{(Samples)}\end{tabular} & \textbf{\# APIs} & \textbf{Code}  & \begin{tabular}[c]{@{}c@{}}\textbf{Nat.}\\ \textbf{Lang.}\end{tabular} & \begin{tabular}[c]{@{}c@{}}\textbf{Action}\\ \textbf{Traj.}\end{tabular} & \begin{tabular}[c]{@{}c@{}}\textbf{API}\\ \textbf{Doc.}\end{tabular} & \begin{tabular}[c]{@{}c@{}}\textbf{Func.}\\ \textbf{Call}\end{tabular} & \begin{tabular}[c]{@{}c@{}}\textbf{Multi.}\\ \textbf{Step}\end{tabular}  & \begin{tabular}[c]{@{}c@{}}\textbf{Plan}\\ \textbf{Refine}\end{tabular}  & \begin{tabular}[c]{@{}c@{}}\textbf{Multi.}\\ \textbf{Turn}\end{tabular}\\ \midrule 
\multicolumn{13}{l}{\emph{Instruction Finetuning-based LLM Agents for Intrinsic Reasoning}}  \\ \midrule
FireAct~\cite{chen2023fireact} & FireAct & IFT & - & 2.1K & 10 & \textcolor{red}{\XSolidBrush} &\textcolor{green}{\CheckmarkBold} &\textcolor{green}{\CheckmarkBold}  & \textcolor{red}{\XSolidBrush} &\textcolor{green}{\CheckmarkBold} & \textcolor{red}{\XSolidBrush} &\textcolor{green}{\CheckmarkBold} & \textcolor{red}{\XSolidBrush} \\
ToolAlpaca~\cite{tang2023toolalpaca} & ToolAlpaca & IFT & - & 4.0K & 400 & \textcolor{red}{\XSolidBrush} &\textcolor{green}{\CheckmarkBold} &\textcolor{green}{\CheckmarkBold} & \textcolor{red}{\XSolidBrush} &\textcolor{green}{\CheckmarkBold} & \textcolor{red}{\XSolidBrush}  &\textcolor{green}{\CheckmarkBold} & \textcolor{red}{\XSolidBrush}  \\
ToolLLaMA~\cite{qin2023toolllm} & ToolBench & IFT & - & 12.7K & 16,464 & \textcolor{red}{\XSolidBrush} &\textcolor{green}{\CheckmarkBold} &\textcolor{green}{\CheckmarkBold} &\textcolor{red}{\XSolidBrush} &\textcolor{green}{\CheckmarkBold}&\textcolor{green}{\CheckmarkBold}&\textcolor{green}{\CheckmarkBold} &\textcolor{green}{\CheckmarkBold}\\
AgentEvol~\citep{xi2024agentgym} & AgentTraj-L & IFT & - & 14.5K & 24 &\textcolor{red}{\XSolidBrush} & \textcolor{green}{\CheckmarkBold} &\textcolor{green}{\CheckmarkBold}&\textcolor{red}{\XSolidBrush} &\textcolor{green}{\CheckmarkBold}&\textcolor{red}{\XSolidBrush} &\textcolor{red}{\XSolidBrush} &\textcolor{green}{\CheckmarkBold}\\
Lumos~\cite{yin2024agent} & Lumos & IFT  & - & 20.0K & 16 &\textcolor{red}{\XSolidBrush} & \textcolor{green}{\CheckmarkBold} & \textcolor{green}{\CheckmarkBold} &\textcolor{red}{\XSolidBrush} & \textcolor{green}{\CheckmarkBold} & \textcolor{green}{\CheckmarkBold} &\textcolor{red}{\XSolidBrush} & \textcolor{green}{\CheckmarkBold}\\
Agent-FLAN~\cite{chen2024agent} & Agent-FLAN & IFT & - & 24.7K & 20 &\textcolor{red}{\XSolidBrush} & \textcolor{green}{\CheckmarkBold} & \textcolor{green}{\CheckmarkBold} &\textcolor{red}{\XSolidBrush} & \textcolor{green}{\CheckmarkBold}& \textcolor{green}{\CheckmarkBold}&\textcolor{red}{\XSolidBrush} & \textcolor{green}{\CheckmarkBold}\\
AgentTuning~\citep{zeng2023agenttuning} & AgentInstruct & IFT & - & 35.0K & - &\textcolor{red}{\XSolidBrush} & \textcolor{green}{\CheckmarkBold} & \textcolor{green}{\CheckmarkBold} &\textcolor{red}{\XSolidBrush} & \textcolor{green}{\CheckmarkBold} &\textcolor{red}{\XSolidBrush} &\textcolor{red}{\XSolidBrush} & \textcolor{green}{\CheckmarkBold}\\\midrule
\multicolumn{13}{l}{\emph{Instruction Finetuning-based LLM Agents for Function Calling}} \\\midrule
NexusRaven~\citep{srinivasan2023nexusraven} & NexusRaven & IFT & - & - & 116 & \textcolor{green}{\CheckmarkBold} & \textcolor{green}{\CheckmarkBold}  & \textcolor{green}{\CheckmarkBold} &\textcolor{red}{\XSolidBrush} & \textcolor{green}{\CheckmarkBold} &\textcolor{red}{\XSolidBrush} &\textcolor{red}{\XSolidBrush}&\textcolor{red}{\XSolidBrush}\\
Gorilla~\citep{patil2023gorilla} & Gorilla & IFT & - & 16.0K & 1,645 & \textcolor{green}{\CheckmarkBold} &\textcolor{red}{\XSolidBrush} &\textcolor{red}{\XSolidBrush}&\textcolor{green}{\CheckmarkBold} &\textcolor{green}{\CheckmarkBold} &\textcolor{red}{\XSolidBrush} &\textcolor{red}{\XSolidBrush} &\textcolor{red}{\XSolidBrush}\\
OpenFunctions-v2~\citep{patil2023gorilla} & OpenFunctions-v2 & IFT & - & 65.0K & - & \textcolor{green}{\CheckmarkBold} & \textcolor{green}{\CheckmarkBold} &\textcolor{red}{\XSolidBrush} &\textcolor{green}{\CheckmarkBold} &\textcolor{green}{\CheckmarkBold} &\textcolor{red}{\XSolidBrush} &\textcolor{red}{\XSolidBrush} &\textcolor{red}{\XSolidBrush}\\
API Pack~\cite{guo2024api} & API Pack & IFT & - & 1.1M & 11,213 &\textcolor{green}{\CheckmarkBold} &\textcolor{red}{\XSolidBrush} &\textcolor{green}{\CheckmarkBold} &\textcolor{red}{\XSolidBrush} &\textcolor{green}{\CheckmarkBold} &\textcolor{red}{\XSolidBrush}&\textcolor{red}{\XSolidBrush}&\textcolor{red}{\XSolidBrush}\\ 
LAM~\citep{zhang2024agentohana} & AgentOhana & IFT & - & 42.6K & - & \textcolor{green}{\CheckmarkBold} & \textcolor{green}{\CheckmarkBold} &\textcolor{green}{\CheckmarkBold}&\textcolor{red}{\XSolidBrush} &\textcolor{green}{\CheckmarkBold}&\textcolor{red}{\XSolidBrush}&\textcolor{green}{\CheckmarkBold}&\textcolor{green}{\CheckmarkBold}\\
xLAM~\citep{liu2024apigen} & APIGen & IFT & - & 60.0K & 3,673 & \textcolor{green}{\CheckmarkBold} & \textcolor{green}{\CheckmarkBold} &\textcolor{green}{\CheckmarkBold}&\textcolor{red}{\XSolidBrush} &\textcolor{green}{\CheckmarkBold}&\textcolor{red}{\XSolidBrush}&\textcolor{green}{\CheckmarkBold}&\textcolor{green}{\CheckmarkBold}\\\midrule
\multicolumn{13}{l}{\emph{Pretraining-based LLM Agents}}  \\\midrule
% LEMUR~\citep{xu2024lemur} & PT & 90B & 300.0K & - & \textcolor{green}{\CheckmarkBold} & \textcolor{green}{\CheckmarkBold} &\textcolor{green}{\CheckmarkBold}&\textcolor{red}{\XSolidBrush} & \textcolor{red}{\XSolidBrush} &\textcolor{green}{\CheckmarkBold} &\textcolor{red}{\XSolidBrush}&\textcolor{red}{\XSolidBrush}\\
\rowcolor{teal!12} \method & \dataset & PT & 103B & 95.0K  & 76,537  & \textcolor{green}{\CheckmarkBold} & \textcolor{green}{\CheckmarkBold} & \textcolor{green}{\CheckmarkBold} & \textcolor{green}{\CheckmarkBold} & \textcolor{green}{\CheckmarkBold} & \textcolor{green}{\CheckmarkBold} & \textcolor{green}{\CheckmarkBold} & \textcolor{green}{\CheckmarkBold}\\
\bottomrule
\end{tabular}
% \begin{tablenotes}
%     \item $^*$ In addition, the StarCoder-API can offer 4.77M more APIs.
% \end{tablenotes}
\caption{Summary of existing instruction finetuning-based LLM agents for intrinsic reasoning and function calling, along with their training resources and sample sizes. "PT" and "IFT" denote "Pre-Training" and "Instruction Fine-Tuning", respectively.}
\vspace{-2ex}
\label{tab:related}
\end{threeparttable}
\end{table*}

\noindent \textbf{Prompting-based LLM Agents.} Due to the lack of agent-specific pre-training corpus, existing LLM agents rely on either prompt engineering~\cite{hsieh2023tool,lu2024chameleon,yao2022react,wang2023voyager} or instruction fine-tuning~\cite{chen2023fireact,zeng2023agenttuning} to understand human instructions, decompose high-level tasks, generate grounded plans, and execute multi-step actions. 
However, prompting-based methods mainly depend on the capabilities of backbone LLMs (usually commercial LLMs), failing to introduce new knowledge and struggling to generalize to unseen tasks~\cite{sun2024adaplanner,zhuang2023toolchain}. 

\noindent \textbf{Instruction Finetuning-based LLM Agents.} Considering the extensive diversity of APIs and the complexity of multi-tool instructions, tool learning inherently presents greater challenges than natural language tasks, such as text generation~\cite{qin2023toolllm}.
Post-training techniques focus more on instruction following and aligning output with specific formats~\cite{patil2023gorilla,hao2024toolkengpt,qin2023toolllm,schick2024toolformer}, rather than fundamentally improving model knowledge or capabilities. 
Moreover, heavy fine-tuning can hinder generalization or even degrade performance in non-agent use cases, potentially suppressing the original base model capabilities~\cite{ghosh2024a}.

\noindent \textbf{Pretraining-based LLM Agents.} While pre-training serves as an essential alternative, prior works~\cite{nijkamp2023codegen,roziere2023code,xu2024lemur,patil2023gorilla} have primarily focused on improving task-specific capabilities (\eg, code generation) instead of general-domain LLM agents, due to single-source, uni-type, small-scale, and poor-quality pre-training data. 
Existing tool documentation data for agent training either lacks diverse real-world APIs~\cite{patil2023gorilla, tang2023toolalpaca} or is constrained to single-tool or single-round tool execution. 
Furthermore, trajectory data mostly imitate expert behavior or follow function-calling rules with inferior planning and reasoning, failing to fully elicit LLMs' capabilities and handle complex instructions~\cite{qin2023toolllm}. 
Given a wide range of candidate API functions, each comprising various function names and parameters available at every planning step, identifying globally optimal solutions and generalizing across tasks remains highly challenging.



\section{Preliminaries}
\label{Preliminaries}
\begin{figure*}[t]
    \centering
    \includegraphics[width=0.95\linewidth]{fig/HealthGPT_Framework.png}
    \caption{The \ourmethod{} architecture integrates hierarchical visual perception and H-LoRA, employing a task-specific hard router to select visual features and H-LoRA plugins, ultimately generating outputs with an autoregressive manner.}
    \label{fig:architecture}
\end{figure*}
\noindent\textbf{Large Vision-Language Models.} 
The input to a LVLM typically consists of an image $x^{\text{img}}$ and a discrete text sequence $x^{\text{txt}}$. The visual encoder $\mathcal{E}^{\text{img}}$ converts the input image $x^{\text{img}}$ into a sequence of visual tokens $\mathcal{V} = [v_i]_{i=1}^{N_v}$, while the text sequence $x^{\text{txt}}$ is mapped into a sequence of text tokens $\mathcal{T} = [t_i]_{i=1}^{N_t}$ using an embedding function $\mathcal{E}^{\text{txt}}$. The LLM $\mathcal{M_\text{LLM}}(\cdot|\theta)$ models the joint probability of the token sequence $\mathcal{U} = \{\mathcal{V},\mathcal{T}\}$, which is expressed as:
\begin{equation}
    P_\theta(R | \mathcal{U}) = \prod_{i=1}^{N_r} P_\theta(r_i | \{\mathcal{U}, r_{<i}\}),
\end{equation}
where $R = [r_i]_{i=1}^{N_r}$ is the text response sequence. The LVLM iteratively generates the next token $r_i$ based on $r_{<i}$. The optimization objective is to minimize the cross-entropy loss of the response $\mathcal{R}$.
% \begin{equation}
%     \mathcal{L}_{\text{VLM}} = \mathbb{E}_{R|\mathcal{U}}\left[-\log P_\theta(R | \mathcal{U})\right]
% \end{equation}
It is worth noting that most LVLMs adopt a design paradigm based on ViT, alignment adapters, and pre-trained LLMs\cite{liu2023llava,liu2024improved}, enabling quick adaptation to downstream tasks.


\noindent\textbf{VQGAN.}
VQGAN~\cite{esser2021taming} employs latent space compression and indexing mechanisms to effectively learn a complete discrete representation of images. VQGAN first maps the input image $x^{\text{img}}$ to a latent representation $z = \mathcal{E}(x)$ through a encoder $\mathcal{E}$. Then, the latent representation is quantized using a codebook $\mathcal{Z} = \{z_k\}_{k=1}^K$, generating a discrete index sequence $\mathcal{I} = [i_m]_{m=1}^N$, where $i_m \in \mathcal{Z}$ represents the quantized code index:
\begin{equation}
    \mathcal{I} = \text{Quantize}(z|\mathcal{Z}) = \arg\min_{z_k \in \mathcal{Z}} \| z - z_k \|_2.
\end{equation}
In our approach, the discrete index sequence $\mathcal{I}$ serves as a supervisory signal for the generation task, enabling the model to predict the index sequence $\hat{\mathcal{I}}$ from input conditions such as text or other modality signals.  
Finally, the predicted index sequence $\hat{\mathcal{I}}$ is upsampled by the VQGAN decoder $G$, generating the high-quality image $\hat{x}^\text{img} = G(\hat{\mathcal{I}})$.



\noindent\textbf{Low Rank Adaptation.} 
LoRA\cite{hu2021lora} effectively captures the characteristics of downstream tasks by introducing low-rank adapters. The core idea is to decompose the bypass weight matrix $\Delta W\in\mathbb{R}^{d^{\text{in}} \times d^{\text{out}}}$ into two low-rank matrices $ \{A \in \mathbb{R}^{d^{\text{in}} \times r}, B \in \mathbb{R}^{r \times d^{\text{out}}} \}$, where $ r \ll \min\{d^{\text{in}}, d^{\text{out}}\} $, significantly reducing learnable parameters. The output with the LoRA adapter for the input $x$ is then given by:
\begin{equation}
    h = x W_0 + \alpha x \Delta W/r = x W_0 + \alpha xAB/r,
\end{equation}
where matrix $ A $ is initialized with a Gaussian distribution, while the matrix $ B $ is initialized as a zero matrix. The scaling factor $ \alpha/r $ controls the impact of $ \Delta W $ on the model.

\section{HealthGPT}
\label{Method}


\subsection{Unified Autoregressive Generation.}  
% As shown in Figure~\ref{fig:architecture}, 
\ourmethod{} (Figure~\ref{fig:architecture}) utilizes a discrete token representation that covers both text and visual outputs, unifying visual comprehension and generation as an autoregressive task. 
For comprehension, $\mathcal{M}_\text{llm}$ receives the input joint sequence $\mathcal{U}$ and outputs a series of text token $\mathcal{R} = [r_1, r_2, \dots, r_{N_r}]$, where $r_i \in \mathcal{V}_{\text{txt}}$, and $\mathcal{V}_{\text{txt}}$ represents the LLM's vocabulary:
\begin{equation}
    P_\theta(\mathcal{R} \mid \mathcal{U}) = \prod_{i=1}^{N_r} P_\theta(r_i \mid \mathcal{U}, r_{<i}).
\end{equation}
For generation, $\mathcal{M}_\text{llm}$ first receives a special start token $\langle \text{START\_IMG} \rangle$, then generates a series of tokens corresponding to the VQGAN indices $\mathcal{I} = [i_1, i_2, \dots, i_{N_i}]$, where $i_j \in \mathcal{V}_{\text{vq}}$, and $\mathcal{V}_{\text{vq}}$ represents the index range of VQGAN. Upon completion of generation, the LLM outputs an end token $\langle \text{END\_IMG} \rangle$:
\begin{equation}
    P_\theta(\mathcal{I} \mid \mathcal{U}) = \prod_{j=1}^{N_i} P_\theta(i_j \mid \mathcal{U}, i_{<j}).
\end{equation}
Finally, the generated index sequence $\mathcal{I}$ is fed into the decoder $G$, which reconstructs the target image $\hat{x}^{\text{img}} = G(\mathcal{I})$.

\subsection{Hierarchical Visual Perception}  
Given the differences in visual perception between comprehension and generation tasks—where the former focuses on abstract semantics and the latter emphasizes complete semantics—we employ ViT to compress the image into discrete visual tokens at multiple hierarchical levels.
Specifically, the image is converted into a series of features $\{f_1, f_2, \dots, f_L\}$ as it passes through $L$ ViT blocks.

To address the needs of various tasks, the hidden states are divided into two types: (i) \textit{Concrete-grained features} $\mathcal{F}^{\text{Con}} = \{f_1, f_2, \dots, f_k\}, k < L$, derived from the shallower layers of ViT, containing sufficient global features, suitable for generation tasks; 
(ii) \textit{Abstract-grained features} $\mathcal{F}^{\text{Abs}} = \{f_{k+1}, f_{k+2}, \dots, f_L\}$, derived from the deeper layers of ViT, which contain abstract semantic information closer to the text space, suitable for comprehension tasks.

The task type $T$ (comprehension or generation) determines which set of features is selected as the input for the downstream large language model:
\begin{equation}
    \mathcal{F}^{\text{img}}_T =
    \begin{cases}
        \mathcal{F}^{\text{Con}}, & \text{if } T = \text{generation task} \\
        \mathcal{F}^{\text{Abs}}, & \text{if } T = \text{comprehension task}
    \end{cases}
\end{equation}
We integrate the image features $\mathcal{F}^{\text{img}}_T$ and text features $\mathcal{T}$ into a joint sequence through simple concatenation, which is then fed into the LLM $\mathcal{M}_{\text{llm}}$ for autoregressive generation.
% :
% \begin{equation}
%     \mathcal{R} = \mathcal{M}_{\text{llm}}(\mathcal{U}|\theta), \quad \mathcal{U} = [\mathcal{F}^{\text{img}}_T; \mathcal{T}]
% \end{equation}
\subsection{Heterogeneous Knowledge Adaptation}
We devise H-LoRA, which stores heterogeneous knowledge from comprehension and generation tasks in separate modules and dynamically routes to extract task-relevant knowledge from these modules. 
At the task level, for each task type $ T $, we dynamically assign a dedicated H-LoRA submodule $ \theta^T $, which is expressed as:
\begin{equation}
    \mathcal{R} = \mathcal{M}_\text{LLM}(\mathcal{U}|\theta, \theta^T), \quad \theta^T = \{A^T, B^T, \mathcal{R}^T_\text{outer}\}.
\end{equation}
At the feature level for a single task, H-LoRA integrates the idea of Mixture of Experts (MoE)~\cite{masoudnia2014mixture} and designs an efficient matrix merging and routing weight allocation mechanism, thus avoiding the significant computational delay introduced by matrix splitting in existing MoELoRA~\cite{luo2024moelora}. Specifically, we first merge the low-rank matrices (rank = r) of $ k $ LoRA experts into a unified matrix:
\begin{equation}
    \mathbf{A}^{\text{merged}}, \mathbf{B}^{\text{merged}} = \text{Concat}(\{A_i\}_1^k), \text{Concat}(\{B_i\}_1^k),
\end{equation}
where $ \mathbf{A}^{\text{merged}} \in \mathbb{R}^{d^\text{in} \times rk} $ and $ \mathbf{B}^{\text{merged}} \in \mathbb{R}^{rk \times d^\text{out}} $. The $k$-dimension routing layer generates expert weights $ \mathcal{W} \in \mathbb{R}^{\text{token\_num} \times k} $ based on the input hidden state $ x $, and these are expanded to $ \mathbb{R}^{\text{token\_num} \times rk} $ as follows:
\begin{equation}
    \mathcal{W}^\text{expanded} = \alpha k \mathcal{W} / r \otimes \mathbf{1}_r,
\end{equation}
where $ \otimes $ denotes the replication operation.
The overall output of H-LoRA is computed as:
\begin{equation}
    \mathcal{O}^\text{H-LoRA} = (x \mathbf{A}^{\text{merged}} \odot \mathcal{W}^\text{expanded}) \mathbf{B}^{\text{merged}},
\end{equation}
where $ \odot $ represents element-wise multiplication. Finally, the output of H-LoRA is added to the frozen pre-trained weights to produce the final output:
\begin{equation}
    \mathcal{O} = x W_0 + \mathcal{O}^\text{H-LoRA}.
\end{equation}
% In summary, H-LoRA is a task-based dynamic PEFT method that achieves high efficiency in single-task fine-tuning.

\subsection{Training Pipeline}

\begin{figure}[t]
    \centering
    \hspace{-4mm}
    \includegraphics[width=0.94\linewidth]{fig/data.pdf}
    \caption{Data statistics of \texttt{VL-Health}. }
    \label{fig:data}
\end{figure}
\noindent \textbf{1st Stage: Multi-modal Alignment.} 
In the first stage, we design separate visual adapters and H-LoRA submodules for medical unified tasks. For the medical comprehension task, we train abstract-grained visual adapters using high-quality image-text pairs to align visual embeddings with textual embeddings, thereby enabling the model to accurately describe medical visual content. During this process, the pre-trained LLM and its corresponding H-LoRA submodules remain frozen. In contrast, the medical generation task requires training concrete-grained adapters and H-LoRA submodules while keeping the LLM frozen. Meanwhile, we extend the textual vocabulary to include multimodal tokens, enabling the support of additional VQGAN vector quantization indices. The model trains on image-VQ pairs, endowing the pre-trained LLM with the capability for image reconstruction. This design ensures pixel-level consistency of pre- and post-LVLM. The processes establish the initial alignment between the LLM’s outputs and the visual inputs.

\noindent \textbf{2nd Stage: Heterogeneous H-LoRA Plugin Adaptation.}  
The submodules of H-LoRA share the word embedding layer and output head but may encounter issues such as bias and scale inconsistencies during training across different tasks. To ensure that the multiple H-LoRA plugins seamlessly interface with the LLMs and form a unified base, we fine-tune the word embedding layer and output head using a small amount of mixed data to maintain consistency in the model weights. Specifically, during this stage, all H-LoRA submodules for different tasks are kept frozen, with only the word embedding layer and output head being optimized. Through this stage, the model accumulates foundational knowledge for unified tasks by adapting H-LoRA plugins.

\begin{table*}[!t]
\centering
\caption{Comparison of \ourmethod{} with other LVLMs and unified multi-modal models on medical visual comprehension tasks. \textbf{Bold} and \underline{underlined} text indicates the best performance and second-best performance, respectively.}
\resizebox{\textwidth}{!}{
\begin{tabular}{c|lcc|cccccccc|c}
\toprule
\rowcolor[HTML]{E9F3FE} &  &  &  & \multicolumn{2}{c}{\textbf{VQA-RAD \textuparrow}} & \multicolumn{2}{c}{\textbf{SLAKE \textuparrow}} & \multicolumn{2}{c}{\textbf{PathVQA \textuparrow}} &  &  &  \\ 
\cline{5-10}
\rowcolor[HTML]{E9F3FE}\multirow{-2}{*}{\textbf{Type}} & \multirow{-2}{*}{\textbf{Model}} & \multirow{-2}{*}{\textbf{\# Params}} & \multirow{-2}{*}{\makecell{\textbf{Medical} \\ \textbf{LVLM}}} & \textbf{close} & \textbf{all} & \textbf{close} & \textbf{all} & \textbf{close} & \textbf{all} & \multirow{-2}{*}{\makecell{\textbf{MMMU} \\ \textbf{-Med}}\textuparrow} & \multirow{-2}{*}{\textbf{OMVQA}\textuparrow} & \multirow{-2}{*}{\textbf{Avg. \textuparrow}} \\ 
\midrule \midrule
\multirow{9}{*}{\textbf{Comp. Only}} 
& Med-Flamingo & 8.3B & \Large \ding{51} & 58.6 & 43.0 & 47.0 & 25.5 & 61.9 & 31.3 & 28.7 & 34.9 & 41.4 \\
& LLaVA-Med & 7B & \Large \ding{51} & 60.2 & 48.1 & 58.4 & 44.8 & 62.3 & 35.7 & 30.0 & 41.3 & 47.6 \\
& HuatuoGPT-Vision & 7B & \Large \ding{51} & 66.9 & 53.0 & 59.8 & 49.1 & 52.9 & 32.0 & 42.0 & 50.0 & 50.7 \\
& BLIP-2 & 6.7B & \Large \ding{55} & 43.4 & 36.8 & 41.6 & 35.3 & 48.5 & 28.8 & 27.3 & 26.9 & 36.1 \\
& LLaVA-v1.5 & 7B & \Large \ding{55} & 51.8 & 42.8 & 37.1 & 37.7 & 53.5 & 31.4 & 32.7 & 44.7 & 41.5 \\
& InstructBLIP & 7B & \Large \ding{55} & 61.0 & 44.8 & 66.8 & 43.3 & 56.0 & 32.3 & 25.3 & 29.0 & 44.8 \\
& Yi-VL & 6B & \Large \ding{55} & 52.6 & 42.1 & 52.4 & 38.4 & 54.9 & 30.9 & 38.0 & 50.2 & 44.9 \\
& InternVL2 & 8B & \Large \ding{55} & 64.9 & 49.0 & 66.6 & 50.1 & 60.0 & 31.9 & \underline{43.3} & 54.5 & 52.5\\
& Llama-3.2 & 11B & \Large \ding{55} & 68.9 & 45.5 & 72.4 & 52.1 & 62.8 & 33.6 & 39.3 & 63.2 & 54.7 \\
\midrule
\multirow{5}{*}{\textbf{Comp. \& Gen.}} 
& Show-o & 1.3B & \Large \ding{55} & 50.6 & 33.9 & 31.5 & 17.9 & 52.9 & 28.2 & 22.7 & 45.7 & 42.6 \\
& Unified-IO 2 & 7B & \Large \ding{55} & 46.2 & 32.6 & 35.9 & 21.9 & 52.5 & 27.0 & 25.3 & 33.0 & 33.8 \\
& Janus & 1.3B & \Large \ding{55} & 70.9 & 52.8 & 34.7 & 26.9 & 51.9 & 27.9 & 30.0 & 26.8 & 33.5 \\
& \cellcolor[HTML]{DAE0FB}HealthGPT-M3 & \cellcolor[HTML]{DAE0FB}3.8B & \cellcolor[HTML]{DAE0FB}\Large \ding{51} & \cellcolor[HTML]{DAE0FB}\underline{73.7} & \cellcolor[HTML]{DAE0FB}\underline{55.9} & \cellcolor[HTML]{DAE0FB}\underline{74.6} & \cellcolor[HTML]{DAE0FB}\underline{56.4} & \cellcolor[HTML]{DAE0FB}\underline{78.7} & \cellcolor[HTML]{DAE0FB}\underline{39.7} & \cellcolor[HTML]{DAE0FB}\underline{43.3} & \cellcolor[HTML]{DAE0FB}\underline{68.5} & \cellcolor[HTML]{DAE0FB}\underline{61.3} \\
& \cellcolor[HTML]{DAE0FB}HealthGPT-L14 & \cellcolor[HTML]{DAE0FB}14B & \cellcolor[HTML]{DAE0FB}\Large \ding{51} & \cellcolor[HTML]{DAE0FB}\textbf{77.7} & \cellcolor[HTML]{DAE0FB}\textbf{58.3} & \cellcolor[HTML]{DAE0FB}\textbf{76.4} & \cellcolor[HTML]{DAE0FB}\textbf{64.5} & \cellcolor[HTML]{DAE0FB}\textbf{85.9} & \cellcolor[HTML]{DAE0FB}\textbf{44.4} & \cellcolor[HTML]{DAE0FB}\textbf{49.2} & \cellcolor[HTML]{DAE0FB}\textbf{74.4} & \cellcolor[HTML]{DAE0FB}\textbf{66.4} \\
\bottomrule
\end{tabular}
}
\label{tab:results}
\end{table*}
\begin{table*}[ht]
    \centering
    \caption{The experimental results for the four modality conversion tasks.}
    \resizebox{\textwidth}{!}{
    \begin{tabular}{l|ccc|ccc|ccc|ccc}
        \toprule
        \rowcolor[HTML]{E9F3FE} & \multicolumn{3}{c}{\textbf{CT to MRI (Brain)}} & \multicolumn{3}{c}{\textbf{CT to MRI (Pelvis)}} & \multicolumn{3}{c}{\textbf{MRI to CT (Brain)}} & \multicolumn{3}{c}{\textbf{MRI to CT (Pelvis)}} \\
        \cline{2-13}
        \rowcolor[HTML]{E9F3FE}\multirow{-2}{*}{\textbf{Model}}& \textbf{SSIM $\uparrow$} & \textbf{PSNR $\uparrow$} & \textbf{MSE $\downarrow$} & \textbf{SSIM $\uparrow$} & \textbf{PSNR $\uparrow$} & \textbf{MSE $\downarrow$} & \textbf{SSIM $\uparrow$} & \textbf{PSNR $\uparrow$} & \textbf{MSE $\downarrow$} & \textbf{SSIM $\uparrow$} & \textbf{PSNR $\uparrow$} & \textbf{MSE $\downarrow$} \\
        \midrule \midrule
        pix2pix & 71.09 & 32.65 & 36.85 & 59.17 & 31.02 & 51.91 & 78.79 & 33.85 & 28.33 & 72.31 & 32.98 & 36.19 \\
        CycleGAN & 54.76 & 32.23 & 40.56 & 54.54 & 30.77 & 55.00 & 63.75 & 31.02 & 52.78 & 50.54 & 29.89 & 67.78 \\
        BBDM & {71.69} & {32.91} & {34.44} & 57.37 & 31.37 & 48.06 & \textbf{86.40} & 34.12 & 26.61 & {79.26} & 33.15 & 33.60 \\
        Vmanba & 69.54 & 32.67 & 36.42 & {63.01} & {31.47} & {46.99} & 79.63 & 34.12 & 26.49 & 77.45 & 33.53 & 31.85 \\
        DiffMa & 71.47 & 32.74 & 35.77 & 62.56 & 31.43 & 47.38 & 79.00 & {34.13} & {26.45} & 78.53 & {33.68} & {30.51} \\
        \rowcolor[HTML]{DAE0FB}HealthGPT-M3 & \underline{79.38} & \underline{33.03} & \underline{33.48} & \underline{71.81} & \underline{31.83} & \underline{43.45} & {85.06} & \textbf{34.40} & \textbf{25.49} & \underline{84.23} & \textbf{34.29} & \textbf{27.99} \\
        \rowcolor[HTML]{DAE0FB}HealthGPT-L14 & \textbf{79.73} & \textbf{33.10} & \textbf{32.96} & \textbf{71.92} & \textbf{31.87} & \textbf{43.09} & \underline{85.31} & \underline{34.29} & \underline{26.20} & \textbf{84.96} & \underline{34.14} & \underline{28.13} \\
        \bottomrule
    \end{tabular}
    }
    \label{tab:conversion}
\end{table*}

\noindent \textbf{3rd Stage: Visual Instruction Fine-Tuning.}  
In the third stage, we introduce additional task-specific data to further optimize the model and enhance its adaptability to downstream tasks such as medical visual comprehension (e.g., medical QA, medical dialogues, and report generation) or generation tasks (e.g., super-resolution, denoising, and modality conversion). Notably, by this stage, the word embedding layer and output head have been fine-tuned, only the H-LoRA modules and adapter modules need to be trained. This strategy significantly improves the model's adaptability and flexibility across different tasks.


\section{Experiment}
\label{s:experiment}

\subsection{Data Description}
We evaluate our method on FI~\cite{you2016building}, Twitter\_LDL~\cite{yang2017learning} and Artphoto~\cite{machajdik2010affective}.
FI is a public dataset built from Flickr and Instagram, with 23,308 images and eight emotion categories, namely \textit{amusement}, \textit{anger}, \textit{awe},  \textit{contentment}, \textit{disgust}, \textit{excitement},  \textit{fear}, and \textit{sadness}. 
% Since images in FI are all copyrighted by law, some images are corrupted now, so we remove these samples and retain 21,828 images.
% T4SA contains images from Twitter, which are classified into three categories: \textit{positive}, \textit{neutral}, and \textit{negative}. In this paper, we adopt the base version of B-T4SA, which contains 470,586 images and provides text descriptions of the corresponding tweets.
Twitter\_LDL contains 10,045 images from Twitter, with the same eight categories as the FI dataset.
% 。
For these two datasets, they are randomly split into 80\%
training and 20\% testing set.
Artphoto contains 806 artistic photos from the DeviantArt website, which we use to further evaluate the zero-shot capability of our model.
% on the small-scale dataset.
% We construct and publicly release the first image sentiment analysis dataset containing metadata.
% 。

% Based on these datasets, we are the first to construct and publicly release metadata-enhanced image sentiment analysis datasets. These datasets include scenes, tags, descriptions, and corresponding confidence scores, and are available at this link for future research purposes.


% 
\begin{table}[t]
\centering
% \begin{center}
\caption{Overall performance of different models on FI and Twitter\_LDL datasets.}
\label{tab:cap1}
% \resizebox{\linewidth}{!}
{
\begin{tabular}{l|c|c|c|c}
\hline
\multirow{2}{*}{\textbf{Model}} & \multicolumn{2}{c|}{\textbf{FI}}  & \multicolumn{2}{c}{\textbf{Twitter\_LDL}} \\ \cline{2-5} 
  & \textbf{Accuracy} & \textbf{F1} & \textbf{Accuracy} & \textbf{F1}  \\ \hline
% (\rownumber)~AlexNet~\cite{krizhevsky2017imagenet}  & 58.13\% & 56.35\%  & 56.24\%& 55.02\%  \\ 
% (\rownumber)~VGG16~\cite{simonyan2014very}  & 63.75\%& 63.08\%  & 59.34\%& 59.02\%  \\ 
(\rownumber)~ResNet101~\cite{he2016deep} & 66.16\%& 65.56\%  & 62.02\% & 61.34\%  \\ 
(\rownumber)~CDA~\cite{han2023boosting} & 66.71\%& 65.37\%  & 64.14\% & 62.85\%  \\ 
(\rownumber)~CECCN~\cite{ruan2024color} & 67.96\%& 66.74\%  & 64.59\%& 64.72\% \\ 
(\rownumber)~EmoVIT~\cite{xie2024emovit} & 68.09\%& 67.45\%  & 63.12\% & 61.97\%  \\ 
(\rownumber)~ComLDL~\cite{zhang2022compound} & 68.83\%& 67.28\%  & 65.29\% & 63.12\%  \\ 
(\rownumber)~WSDEN~\cite{li2023weakly} & 69.78\%& 69.61\%  & 67.04\% & 65.49\% \\ 
(\rownumber)~ECWA~\cite{deng2021emotion} & 70.87\%& 69.08\%  & 67.81\% & 66.87\%  \\ 
(\rownumber)~EECon~\cite{yang2023exploiting} & 71.13\%& 68.34\%  & 64.27\%& 63.16\%  \\ 
(\rownumber)~MAM~\cite{zhang2024affective} & 71.44\%  & 70.83\% & 67.18\%  & 65.01\%\\ 
(\rownumber)~TGCA-PVT~\cite{chen2024tgca}   & 73.05\%  & 71.46\% & 69.87\%  & 68.32\% \\ 
(\rownumber)~OEAN~\cite{zhang2024object}   & 73.40\%  & 72.63\% & 70.52\%  & 69.47\% \\ \hline
(\rownumber)~\shortname  & \textbf{79.48\%} & \textbf{79.22\%} & \textbf{74.12\%} & \textbf{73.09\%} \\ \hline
\end{tabular}
}
\vspace{-6mm}
% \end{center}
\end{table}
% 

\subsection{Experiment Setting}
% \subsubsection{Model Setting.}
% 
\textbf{Model Setting:}
For feature representation, we set $k=10$ to select object tags, and adopt clip-vit-base-patch32 as the pre-trained model for unified feature representation.
Moreover, we empirically set $(d_e, d_h, d_k, d_s) = (512, 128, 16, 64)$, and set the classification class $L$ to 8.

% 

\textbf{Training Setting:}
To initialize the model, we set all weights such as $\boldsymbol{W}$ following the truncated normal distribution, and use AdamW optimizer with the learning rate of $1 \times 10^{-4}$.
% warmup scheduler of cosine, warmup steps of 2000.
Furthermore, we set the batch size to 32 and the epoch of the training process to 200.
During the implementation, we utilize \textit{PyTorch} to build our entire model.
% , and our project codes are publicly available at https://github.com/zzmyrep/MESN.
% Our project codes as well as data are all publicly available on GitHub\footnote{https://github.com/zzmyrep/KBCEN}.
% Code is available at \href{https://github.com/zzmyrep/KBCEN}{https://github.com/zzmyrep/KBCEN}.

\textbf{Evaluation Metrics:}
Following~\cite{zhang2024affective, chen2024tgca, zhang2024object}, we adopt \textit{accuracy} and \textit{F1} as our evaluation metrics to measure the performance of different methods for image sentiment analysis. 



\subsection{Experiment Result}
% We compare our model against the following baselines: AlexNet~\cite{krizhevsky2017imagenet}, VGG16~\cite{simonyan2014very}, ResNet101~\cite{he2016deep}, CECCN~\cite{ruan2024color}, EmoVIT~\cite{xie2024emovit}, WSCNet~\cite{yang2018weakly}, ECWA~\cite{deng2021emotion}, EECon~\cite{yang2023exploiting}, MAM~\cite{zhang2024affective} and TGCA-PVT~\cite{chen2024tgca}, and the overall results are summarized in Table~\ref{tab:cap1}.
We compare our model against several baselines, and the overall results are summarized in Table~\ref{tab:cap1}.
We observe that our model achieves the best performance in both accuracy and F1 metrics, significantly outperforming the previous models. 
This superior performance is mainly attributed to our effective utilization of metadata to enhance image sentiment analysis, as well as the exceptional capability of the unified sentiment transformer framework we developed. These results strongly demonstrate that our proposed method can bring encouraging performance for image sentiment analysis.

\setcounter{magicrownumbers}{0} 
\begin{table}[t]
\begin{center}
\caption{Ablation study of~\shortname~on FI dataset.} 
% \vspace{1mm}
\label{tab:cap2}
\resizebox{.9\linewidth}{!}
{
\begin{tabular}{lcc}
  \hline
  \textbf{Model} & \textbf{Accuracy} & \textbf{F1} \\
  \hline
  (\rownumber)~Ours (w/o vision) & 65.72\% & 64.54\% \\
  (\rownumber)~Ours (w/o text description) & 74.05\% & 72.58\% \\
  (\rownumber)~Ours (w/o object tag) & 77.45\% & 76.84\% \\
  (\rownumber)~Ours (w/o scene tag) & 78.47\% & 78.21\% \\
  \hline
  (\rownumber)~Ours (w/o unified embedding) & 76.41\% & 76.23\% \\
  (\rownumber)~Ours (w/o adaptive learning) & 76.83\% & 76.56\% \\
  (\rownumber)~Ours (w/o cross-modal fusion) & 76.85\% & 76.49\% \\
  \hline
  (\rownumber)~Ours  & \textbf{79.48\%} & \textbf{79.22\%} \\
  \hline
\end{tabular}
}
\end{center}
\vspace{-5mm}
\end{table}


\begin{figure}[t]
\centering
% \vspace{-2mm}
\includegraphics[width=0.42\textwidth]{fig/2dvisual-linux4-paper2.pdf}
\caption{Visualization of feature distribution on eight categories before (left) and after (right) model processing.}
% 
\label{fig:visualization}
\vspace{-5mm}
\end{figure}

\subsection{Ablation Performance}
In this subsection, we conduct an ablation study to examine which component is really important for performance improvement. The results are reported in Table~\ref{tab:cap2}.

For information utilization, we observe a significant decline in model performance when visual features are removed. Additionally, the performance of \shortname~decreases when different metadata are removed separately, which means that text description, object tag, and scene tag are all critical for image sentiment analysis.
Recalling the model architecture, we separately remove transformer layers of the unified representation module, the adaptive learning module, and the cross-modal fusion module, replacing them with MLPs of the same parameter scale.
In this way, we can observe varying degrees of decline in model performance, indicating that these modules are indispensable for our model to achieve better performance.

\subsection{Visualization}
% 


% % 开始使用minipage进行左右排列
% \begin{minipage}[t]{0.45\textwidth}  % 子图1宽度为45%
%     \centering
%     \includegraphics[width=\textwidth]{2dvisual.pdf}  % 插入图片
%     \captionof{figure}{Visualization of feature distribution.}  % 使用captionof添加图片标题
%     \label{fig:visualization}
% \end{minipage}


% \begin{figure}[t]
% \centering
% \vspace{-2mm}
% \includegraphics[width=0.45\textwidth]{fig/2dvisual.pdf}
% \caption{Visualization of feature distribution.}
% \label{fig:visualization}
% % \vspace{-4mm}
% \end{figure}

% \begin{figure}[t]
% \centering
% \vspace{-2mm}
% \includegraphics[width=0.45\textwidth]{fig/2dvisual-linux3-paper.pdf}
% \caption{Visualization of feature distribution.}
% \label{fig:visualization}
% % \vspace{-4mm}
% \end{figure}



\begin{figure}[tbp]   
\vspace{-4mm}
  \centering            
  \subfloat[Depth of adaptive learning layers]   
  {
    \label{fig:subfig1}\includegraphics[width=0.22\textwidth]{fig/fig_sensitivity-a5}
  }
  \subfloat[Depth of fusion layers]
  {
    % \label{fig:subfig2}\includegraphics[width=0.22\textwidth]{fig/fig_sensitivity-b2}
    \label{fig:subfig2}\includegraphics[width=0.22\textwidth]{fig/fig_sensitivity-b2-num.pdf}
  }
  \caption{Sensitivity study of \shortname~on different depth. }   
  \label{fig:fig_sensitivity}  
\vspace{-2mm}
\end{figure}

% \begin{figure}[htbp]
% \centerline{\includegraphics{2dvisual.pdf}}
% \caption{Visualization of feature distribution.}
% \label{fig:visualization}
% \end{figure}

% In Fig.~\ref{fig:visualization}, we use t-SNE~\cite{van2008visualizing} to reduce the dimension of data features for visualization, Figure in left represents the metadata features before model processing, the features are obtained by embedding through the CLIP model, and figure in right shows the features of the data after model processing, it can be observed that after the model processing, the data with different label categories fall in different regions in the space, therefore, we can conclude that the Therefore, we can conclude that the model can effectively utilize the information contained in the metadata and use it to guide the model for classification.

In Fig.~\ref{fig:visualization}, we use t-SNE~\cite{van2008visualizing} to reduce the dimension of data features for visualization.
The left figure shows metadata features before being processed by our model (\textit{i.e.}, embedded by CLIP), while the right shows the distribution of features after being processed by our model.
We can observe that after the model processing, data with the same label are closer to each other, while others are farther away.
Therefore, it shows that the model can effectively utilize the information contained in the metadata and use it to guide the classification process.

\subsection{Sensitivity Analysis}
% 
In this subsection, we conduct a sensitivity analysis to figure out the effect of different depth settings of adaptive learning layers and fusion layers. 
% In this subsection, we conduct a sensitivity analysis to figure out the effect of different depth settings on the model. 
% Fig.~\ref{fig:fig_sensitivity} presents the effect of different depth settings of adaptive learning layers and fusion layers. 
Taking Fig.~\ref{fig:fig_sensitivity} (a) as an example, the model performance improves with increasing depth, reaching the best performance at a depth of 4.
% Taking Fig.~\ref{fig:fig_sensitivity} (a) as an example, the performance of \shortname~improves with the increase of depth at first, reaching the best performance at a depth of 4.
When the depth continues to increase, the accuracy decreases to varying degrees.
Similar results can be observed in Fig.~\ref{fig:fig_sensitivity} (b).
Therefore, we set their depths to 4 and 6 respectively to achieve the best results.

% Through our experiments, we can observe that the effect of modifying these hyperparameters on the results of the experiments is very weak, and the surface model is not sensitive to the hyperparameters.


\subsection{Zero-shot Capability}
% 

% (1)~GCH~\cite{2010Analyzing} & 21.78\% & (5)~RA-DLNet~\cite{2020A} & 34.01\% \\ \hline
% (2)~WSCNet~\cite{2019WSCNet}  & 30.25\% & (6)~CECCN~\cite{ruan2024color} & 43.83\% \\ \hline
% (3)~PCNN~\cite{2015Robust} & 31.68\%  & (7)~EmoVIT~\cite{xie2024emovit} & 44.90\% \\ \hline
% (4)~AR~\cite{2018Visual} & 32.67\% & (8)~Ours (Zero-shot) & 47.83\% \\ \hline


\begin{table}[t]
\centering
\caption{Zero-shot capability of \shortname.}
\label{tab:cap3}
\resizebox{1\linewidth}{!}
{
\begin{tabular}{lc|lc}
\hline
\textbf{Model} & \textbf{Accuracy} & \textbf{Model} & \textbf{Accuracy} \\ \hline
(1)~WSCNet~\cite{2019WSCNet}  & 30.25\% & (5)~MAM~\cite{zhang2024affective} & 39.56\%  \\ \hline
(2)~AR~\cite{2018Visual} & 32.67\% & (6)~CECCN~\cite{ruan2024color} & 43.83\% \\ \hline
(3)~RA-DLNet~\cite{2020A} & 34.01\%  & (7)~EmoVIT~\cite{xie2024emovit} & 44.90\% \\ \hline
(4)~CDA~\cite{han2023boosting} & 38.64\% & (8)~Ours (Zero-shot) & 47.83\% \\ \hline
\end{tabular}
}
\vspace{-5mm}
\end{table}

% We use the model trained on the FI dataset to test on the artphoto dataset to verify the model's generalization ability as well as robustness to other distributed datasets.
% We can observe that the MESN model shows strong competitiveness in terms of accuracy when compared to other trained models, which suggests that the model has a good generalization ability in the OOD task.

To validate the model's generalization ability and robustness to other distributed datasets, we directly test the model trained on the FI dataset, without training on Artphoto. 
% As observed in Table 3, compared to other models trained on Artphoto, we achieve highly competitive zero-shot performance, indicating that the model has good generalization ability in out-of-distribution tasks.
From Table~\ref{tab:cap3}, we can observe that compared with other models trained on Artphoto, we achieve competitive zero-shot performance, which shows that the model has good generalization ability in out-of-distribution tasks.


%%%%%%%%%%%%
%  E2E     %
%%%%%%%%%%%%


\section{Conclusion}
In this paper, we introduced Wi-Chat, the first LLM-powered Wi-Fi-based human activity recognition system that integrates the reasoning capabilities of large language models with the sensing potential of wireless signals. Our experimental results on a self-collected Wi-Fi CSI dataset demonstrate the promising potential of LLMs in enabling zero-shot Wi-Fi sensing. These findings suggest a new paradigm for human activity recognition that does not rely on extensive labeled data. We hope future research will build upon this direction, further exploring the applications of LLMs in signal processing domains such as IoT, mobile sensing, and radar-based systems.

\section*{Limitations}
While our work represents the first attempt to leverage LLMs for processing Wi-Fi signals, it is a preliminary study focused on a relatively simple task: Wi-Fi-based human activity recognition. This choice allows us to explore the feasibility of LLMs in wireless sensing but also comes with certain limitations.

Our approach primarily evaluates zero-shot performance, which, while promising, may still lag behind traditional supervised learning methods in highly complex or fine-grained recognition tasks. Besides, our study is limited to a controlled environment with a self-collected dataset, and the generalizability of LLMs to diverse real-world scenarios with varying Wi-Fi conditions, environmental interference, and device heterogeneity remains an open question.

Additionally, we have yet to explore the full potential of LLMs in more advanced Wi-Fi sensing applications, such as fine-grained gesture recognition, occupancy detection, and passive health monitoring. Future work should investigate the scalability of LLM-based approaches, their robustness to domain shifts, and their integration with multimodal sensing techniques in broader IoT applications.


% Bibliography entries for the entire Anthology, followed by custom entries
%\bibliography{anthology,custom}
% Custom bibliography entries only
\bibliography{main}
\newpage
\appendix

\section{Experiment prompts}
\label{sec:prompt}
The prompts used in the LLM experiments are shown in the following Table~\ref{tab:prompts}.

\definecolor{titlecolor}{rgb}{0.9, 0.5, 0.1}
\definecolor{anscolor}{rgb}{0.2, 0.5, 0.8}
\definecolor{labelcolor}{HTML}{48a07e}
\begin{table*}[h]
	\centering
	
 % \vspace{-0.2cm}
	
	\begin{center}
		\begin{tikzpicture}[
				chatbox_inner/.style={rectangle, rounded corners, opacity=0, text opacity=1, font=\sffamily\scriptsize, text width=5in, text height=9pt, inner xsep=6pt, inner ysep=6pt},
				chatbox_prompt_inner/.style={chatbox_inner, align=flush left, xshift=0pt, text height=11pt},
				chatbox_user_inner/.style={chatbox_inner, align=flush left, xshift=0pt},
				chatbox_gpt_inner/.style={chatbox_inner, align=flush left, xshift=0pt},
				chatbox/.style={chatbox_inner, draw=black!25, fill=gray!7, opacity=1, text opacity=0},
				chatbox_prompt/.style={chatbox, align=flush left, fill=gray!1.5, draw=black!30, text height=10pt},
				chatbox_user/.style={chatbox, align=flush left},
				chatbox_gpt/.style={chatbox, align=flush left},
				chatbox2/.style={chatbox_gpt, fill=green!25},
				chatbox3/.style={chatbox_gpt, fill=red!20, draw=black!20},
				chatbox4/.style={chatbox_gpt, fill=yellow!30},
				labelbox/.style={rectangle, rounded corners, draw=black!50, font=\sffamily\scriptsize\bfseries, fill=gray!5, inner sep=3pt},
			]
											
			\node[chatbox_user] (q1) {
				\textbf{System prompt}
				\newline
				\newline
				You are a helpful and precise assistant for segmenting and labeling sentences. We would like to request your help on curating a dataset for entity-level hallucination detection.
				\newline \newline
                We will give you a machine generated biography and a list of checked facts about the biography. Each fact consists of a sentence and a label (True/False). Please do the following process. First, breaking down the biography into words. Second, by referring to the provided list of facts, merging some broken down words in the previous step to form meaningful entities. For example, ``strategic thinking'' should be one entity instead of two. Third, according to the labels in the list of facts, labeling each entity as True or False. Specifically, for facts that share a similar sentence structure (\eg, \textit{``He was born on Mach 9, 1941.''} (\texttt{True}) and \textit{``He was born in Ramos Mejia.''} (\texttt{False})), please first assign labels to entities that differ across atomic facts. For example, first labeling ``Mach 9, 1941'' (\texttt{True}) and ``Ramos Mejia'' (\texttt{False}) in the above case. For those entities that are the same across atomic facts (\eg, ``was born'') or are neutral (\eg, ``he,'' ``in,'' and ``on''), please label them as \texttt{True}. For the cases that there is no atomic fact that shares the same sentence structure, please identify the most informative entities in the sentence and label them with the same label as the atomic fact while treating the rest of the entities as \texttt{True}. In the end, output the entities and labels in the following format:
                \begin{itemize}[nosep]
                    \item Entity 1 (Label 1)
                    \item Entity 2 (Label 2)
                    \item ...
                    \item Entity N (Label N)
                \end{itemize}
                % \newline \newline
                Here are two examples:
                \newline\newline
                \textbf{[Example 1]}
                \newline
                [The start of the biography]
                \newline
                \textcolor{titlecolor}{Marianne McAndrew is an American actress and singer, born on November 21, 1942, in Cleveland, Ohio. She began her acting career in the late 1960s, appearing in various television shows and films.}
                \newline
                [The end of the biography]
                \newline \newline
                [The start of the list of checked facts]
                \newline
                \textcolor{anscolor}{[Marianne McAndrew is an American. (False); Marianne McAndrew is an actress. (True); Marianne McAndrew is a singer. (False); Marianne McAndrew was born on November 21, 1942. (False); Marianne McAndrew was born in Cleveland, Ohio. (False); She began her acting career in the late 1960s. (True); She has appeared in various television shows. (True); She has appeared in various films. (True)]}
                \newline
                [The end of the list of checked facts]
                \newline \newline
                [The start of the ideal output]
                \newline
                \textcolor{labelcolor}{[Marianne McAndrew (True); is (True); an (True); American (False); actress (True); and (True); singer (False); , (True); born (True); on (True); November 21, 1942 (False); , (True); in (True); Cleveland, Ohio (False); . (True); She (True); began (True); her (True); acting career (True); in (True); the late 1960s (True); , (True); appearing (True); in (True); various (True); television shows (True); and (True); films (True); . (True)]}
                \newline
                [The end of the ideal output]
				\newline \newline
                \textbf{[Example 2]}
                \newline
                [The start of the biography]
                \newline
                \textcolor{titlecolor}{Doug Sheehan is an American actor who was born on April 27, 1949, in Santa Monica, California. He is best known for his roles in soap operas, including his portrayal of Joe Kelly on ``General Hospital'' and Ben Gibson on ``Knots Landing.''}
                \newline
                [The end of the biography]
                \newline \newline
                [The start of the list of checked facts]
                \newline
                \textcolor{anscolor}{[Doug Sheehan is an American. (True); Doug Sheehan is an actor. (True); Doug Sheehan was born on April 27, 1949. (True); Doug Sheehan was born in Santa Monica, California. (False); He is best known for his roles in soap operas. (True); He portrayed Joe Kelly. (True); Joe Kelly was in General Hospital. (True); General Hospital is a soap opera. (True); He portrayed Ben Gibson. (True); Ben Gibson was in Knots Landing. (True); Knots Landing is a soap opera. (True)]}
                \newline
                [The end of the list of checked facts]
                \newline \newline
                [The start of the ideal output]
                \newline
                \textcolor{labelcolor}{[Doug Sheehan (True); is (True); an (True); American (True); actor (True); who (True); was born (True); on (True); April 27, 1949 (True); in (True); Santa Monica, California (False); . (True); He (True); is (True); best known (True); for (True); his roles in soap operas (True); , (True); including (True); in (True); his portrayal (True); of (True); Joe Kelly (True); on (True); ``General Hospital'' (True); and (True); Ben Gibson (True); on (True); ``Knots Landing.'' (True)]}
                \newline
                [The end of the ideal output]
				\newline \newline
				\textbf{User prompt}
				\newline
				\newline
				[The start of the biography]
				\newline
				\textcolor{magenta}{\texttt{\{BIOGRAPHY\}}}
				\newline
				[The ebd of the biography]
				\newline \newline
				[The start of the list of checked facts]
				\newline
				\textcolor{magenta}{\texttt{\{LIST OF CHECKED FACTS\}}}
				\newline
				[The end of the list of checked facts]
			};
			\node[chatbox_user_inner] (q1_text) at (q1) {
				\textbf{System prompt}
				\newline
				\newline
				You are a helpful and precise assistant for segmenting and labeling sentences. We would like to request your help on curating a dataset for entity-level hallucination detection.
				\newline \newline
                We will give you a machine generated biography and a list of checked facts about the biography. Each fact consists of a sentence and a label (True/False). Please do the following process. First, breaking down the biography into words. Second, by referring to the provided list of facts, merging some broken down words in the previous step to form meaningful entities. For example, ``strategic thinking'' should be one entity instead of two. Third, according to the labels in the list of facts, labeling each entity as True or False. Specifically, for facts that share a similar sentence structure (\eg, \textit{``He was born on Mach 9, 1941.''} (\texttt{True}) and \textit{``He was born in Ramos Mejia.''} (\texttt{False})), please first assign labels to entities that differ across atomic facts. For example, first labeling ``Mach 9, 1941'' (\texttt{True}) and ``Ramos Mejia'' (\texttt{False}) in the above case. For those entities that are the same across atomic facts (\eg, ``was born'') or are neutral (\eg, ``he,'' ``in,'' and ``on''), please label them as \texttt{True}. For the cases that there is no atomic fact that shares the same sentence structure, please identify the most informative entities in the sentence and label them with the same label as the atomic fact while treating the rest of the entities as \texttt{True}. In the end, output the entities and labels in the following format:
                \begin{itemize}[nosep]
                    \item Entity 1 (Label 1)
                    \item Entity 2 (Label 2)
                    \item ...
                    \item Entity N (Label N)
                \end{itemize}
                % \newline \newline
                Here are two examples:
                \newline\newline
                \textbf{[Example 1]}
                \newline
                [The start of the biography]
                \newline
                \textcolor{titlecolor}{Marianne McAndrew is an American actress and singer, born on November 21, 1942, in Cleveland, Ohio. She began her acting career in the late 1960s, appearing in various television shows and films.}
                \newline
                [The end of the biography]
                \newline \newline
                [The start of the list of checked facts]
                \newline
                \textcolor{anscolor}{[Marianne McAndrew is an American. (False); Marianne McAndrew is an actress. (True); Marianne McAndrew is a singer. (False); Marianne McAndrew was born on November 21, 1942. (False); Marianne McAndrew was born in Cleveland, Ohio. (False); She began her acting career in the late 1960s. (True); She has appeared in various television shows. (True); She has appeared in various films. (True)]}
                \newline
                [The end of the list of checked facts]
                \newline \newline
                [The start of the ideal output]
                \newline
                \textcolor{labelcolor}{[Marianne McAndrew (True); is (True); an (True); American (False); actress (True); and (True); singer (False); , (True); born (True); on (True); November 21, 1942 (False); , (True); in (True); Cleveland, Ohio (False); . (True); She (True); began (True); her (True); acting career (True); in (True); the late 1960s (True); , (True); appearing (True); in (True); various (True); television shows (True); and (True); films (True); . (True)]}
                \newline
                [The end of the ideal output]
				\newline \newline
                \textbf{[Example 2]}
                \newline
                [The start of the biography]
                \newline
                \textcolor{titlecolor}{Doug Sheehan is an American actor who was born on April 27, 1949, in Santa Monica, California. He is best known for his roles in soap operas, including his portrayal of Joe Kelly on ``General Hospital'' and Ben Gibson on ``Knots Landing.''}
                \newline
                [The end of the biography]
                \newline \newline
                [The start of the list of checked facts]
                \newline
                \textcolor{anscolor}{[Doug Sheehan is an American. (True); Doug Sheehan is an actor. (True); Doug Sheehan was born on April 27, 1949. (True); Doug Sheehan was born in Santa Monica, California. (False); He is best known for his roles in soap operas. (True); He portrayed Joe Kelly. (True); Joe Kelly was in General Hospital. (True); General Hospital is a soap opera. (True); He portrayed Ben Gibson. (True); Ben Gibson was in Knots Landing. (True); Knots Landing is a soap opera. (True)]}
                \newline
                [The end of the list of checked facts]
                \newline \newline
                [The start of the ideal output]
                \newline
                \textcolor{labelcolor}{[Doug Sheehan (True); is (True); an (True); American (True); actor (True); who (True); was born (True); on (True); April 27, 1949 (True); in (True); Santa Monica, California (False); . (True); He (True); is (True); best known (True); for (True); his roles in soap operas (True); , (True); including (True); in (True); his portrayal (True); of (True); Joe Kelly (True); on (True); ``General Hospital'' (True); and (True); Ben Gibson (True); on (True); ``Knots Landing.'' (True)]}
                \newline
                [The end of the ideal output]
				\newline \newline
				\textbf{User prompt}
				\newline
				\newline
				[The start of the biography]
				\newline
				\textcolor{magenta}{\texttt{\{BIOGRAPHY\}}}
				\newline
				[The ebd of the biography]
				\newline \newline
				[The start of the list of checked facts]
				\newline
				\textcolor{magenta}{\texttt{\{LIST OF CHECKED FACTS\}}}
				\newline
				[The end of the list of checked facts]
			};
		\end{tikzpicture}
        \caption{GPT-4o prompt for labeling hallucinated entities.}\label{tb:gpt-4-prompt}
	\end{center}
\vspace{-0cm}
\end{table*}
% \section{Full Experiment Results}
% \begin{table*}[th]
    \centering
    \small
    \caption{Classification Results}
    \begin{tabular}{lcccc}
        \toprule
        \textbf{Method} & \textbf{Accuracy} & \textbf{Precision} & \textbf{Recall} & \textbf{F1-score} \\
        \midrule
        \multicolumn{5}{c}{\textbf{Zero Shot}} \\
                Zero-shot E-eyes & 0.26 & 0.26 & 0.27 & 0.26 \\
        Zero-shot CARM & 0.24 & 0.24 & 0.24 & 0.24 \\
                Zero-shot SVM & 0.27 & 0.28 & 0.28 & 0.27 \\
        Zero-shot CNN & 0.23 & 0.24 & 0.23 & 0.23 \\
        Zero-shot RNN & 0.26 & 0.26 & 0.26 & 0.26 \\
DeepSeek-0shot & 0.54 & 0.61 & 0.54 & 0.52 \\
DeepSeek-0shot-COT & 0.33 & 0.24 & 0.33 & 0.23 \\
DeepSeek-0shot-Knowledge & 0.45 & 0.46 & 0.45 & 0.44 \\
Gemma2-0shot & 0.35 & 0.22 & 0.38 & 0.27 \\
Gemma2-0shot-COT & 0.36 & 0.22 & 0.36 & 0.27 \\
Gemma2-0shot-Knowledge & 0.32 & 0.18 & 0.34 & 0.20 \\
GPT-4o-mini-0shot & 0.48 & 0.53 & 0.48 & 0.41 \\
GPT-4o-mini-0shot-COT & 0.33 & 0.50 & 0.33 & 0.38 \\
GPT-4o-mini-0shot-Knowledge & 0.49 & 0.31 & 0.49 & 0.36 \\
GPT-4o-0shot & 0.62 & 0.62 & 0.47 & 0.42 \\
GPT-4o-0shot-COT & 0.29 & 0.45 & 0.29 & 0.21 \\
GPT-4o-0shot-Knowledge & 0.44 & 0.52 & 0.44 & 0.39 \\
LLaMA-0shot & 0.32 & 0.25 & 0.32 & 0.24 \\
LLaMA-0shot-COT & 0.12 & 0.25 & 0.12 & 0.09 \\
LLaMA-0shot-Knowledge & 0.32 & 0.25 & 0.32 & 0.28 \\
Mistral-0shot & 0.19 & 0.23 & 0.19 & 0.10 \\
Mistral-0shot-Knowledge & 0.21 & 0.40 & 0.21 & 0.11 \\
        \midrule
        \multicolumn{5}{c}{\textbf{4 Shot}} \\
GPT-4o-mini-4shot & 0.58 & 0.59 & 0.58 & 0.53 \\
GPT-4o-mini-4shot-COT & 0.57 & 0.53 & 0.57 & 0.50 \\
GPT-4o-mini-4shot-Knowledge & 0.56 & 0.51 & 0.56 & 0.47 \\
GPT-4o-4shot & 0.77 & 0.84 & 0.77 & 0.73 \\
GPT-4o-4shot-COT & 0.63 & 0.76 & 0.63 & 0.53 \\
GPT-4o-4shot-Knowledge & 0.72 & 0.82 & 0.71 & 0.66 \\
LLaMA-4shot & 0.29 & 0.24 & 0.29 & 0.21 \\
LLaMA-4shot-COT & 0.20 & 0.30 & 0.20 & 0.13 \\
LLaMA-4shot-Knowledge & 0.15 & 0.23 & 0.13 & 0.13 \\
Mistral-4shot & 0.02 & 0.02 & 0.02 & 0.02 \\
Mistral-4shot-Knowledge & 0.21 & 0.27 & 0.21 & 0.20 \\
        \midrule
        
        \multicolumn{5}{c}{\textbf{Suprevised}} \\
        SVM & 0.94 & 0.92 & 0.91 & 0.91 \\
        CNN & 0.98 & 0.98 & 0.97 & 0.97 \\
        RNN & 0.99 & 0.99 & 0.99 & 0.99 \\
        % \midrule
        % \multicolumn{5}{c}{\textbf{Conventional Wi-Fi-based Human Activity Recognition Systems}} \\
        E-eyes & 1.00 & 1.00 & 1.00 & 1.00 \\
        CARM & 0.98 & 0.98 & 0.98 & 0.98 \\
\midrule
 \multicolumn{5}{c}{\textbf{Vision Models}} \\
           Zero-shot SVM & 0.26 & 0.25 & 0.25 & 0.25 \\
        Zero-shot CNN & 0.26 & 0.25 & 0.26 & 0.26 \\
        Zero-shot RNN & 0.28 & 0.28 & 0.29 & 0.28 \\
        SVM & 0.99 & 0.99 & 0.99 & 0.99 \\
        CNN & 0.98 & 0.99 & 0.98 & 0.98 \\
        RNN & 0.98 & 0.99 & 0.98 & 0.98 \\
GPT-4o-mini-Vision & 0.84 & 0.85 & 0.84 & 0.84 \\
GPT-4o-mini-Vision-COT & 0.90 & 0.91 & 0.90 & 0.90 \\
GPT-4o-Vision & 0.74 & 0.82 & 0.74 & 0.73 \\
GPT-4o-Vision-COT & 0.70 & 0.83 & 0.70 & 0.68 \\
LLaMA-Vision & 0.20 & 0.23 & 0.20 & 0.09 \\
LLaMA-Vision-Knowledge & 0.22 & 0.05 & 0.22 & 0.08 \\

        \bottomrule
    \end{tabular}
    \label{full}
\end{table*}




\end{document}



\end{document}

\endinput
%%
%% End of file `elsarticle-template-harv.tex'.


