\section{Related Work}
\subsection{Uncertainty Quantification}
Uncertainty estimation, or confidence assessment, in neural network predictions has emerged as a major research focus in the machine learning community \cite{smith2024uncertainty}. Current approaches to uncertainty modeling typically fall into three categories: Monte Carlo Dropout \cite{neal2012bayesian, Moreau_2022_WACV, gal2017concrete, kang2023active}, the Bootstrap model \cite{osband2016deep}, and the Gaussian Mixture Model \cite{9666964, zhang2019short}. These methods have been extensively explored in various domains, demonstrating promising results \cite{abdar2021review}. However, most models merely display uncertainty without leveraging it to guide further actions. To address this gap, recent advances have begun to incorporate uncertainty into training strategies \cite{li2023disc,cordeiro2023longremix}. Although promising, these cutting-edge efforts have focused primarily on tasks that involve noisy label learning. For standard supervised learning tasks, the effectiveness of uncertainty-informed training strategies remains unclear. In this work, we extend the confidence learning method based on two views introduced in \cite{li2023disc} and apply it to the COD problem. Our approach not only integrates uncertainty into the model's learning process, but also flexibly delegates uncertain samples to human-based RSVP systems for enhanced decision making.

\subsection{Camouflaged Object Detection (COD)}
In recent years, numerous deep learning-based COD models have been developed \cite{liang2024systematic}, while recent research has started to place more emphasis on uncertainty. Yi Zhang et al. introduced PUENet \cite{10159663}, which uses a Bayesian conditional variational auto-encoder for predictive uncertainty estimation. Yixuan Lyu et al. \cite{10183371} proposed the Uncertainty-Edge Dual Guide model, which combines probabilistic uncertainty with deterministic edge information for accurate COD. Jiawei Liu et al. \cite{9706783} developed a confidence-based COD framework with dynamic supervision, producing both camouflage masks and aleatoric uncertainty estimates, showing superior performance. Fan Yang et al. \cite{9710683} integrated Bayesian learning with Transformer reasoning, leveraging both deterministic and probabilistic information to improve detection accuracy. Furthermore, Aixuan Li et al. \cite{9578707} proposed an adversarial learning network for higher-order similarity measures and confidence estimation. Current research mainly addresses uncertainty in segmentation tasks, focusing on generating confidence maps for boundary distinction, while this study aims to model the uncertainty in identifying object presence to enhance classification accuracy.

\subsection{RSVP-based BCIs for Target Detection}
RSVP-based BCIs have received significant attention in recent years, particularly in the domain of target detection, due to their efficiency in processing rapid visual stimuli and eliciting robust neural responses such as potentials related to P300 events. Research advancements have focused on optimizing RSVP paradigms to enhance system performance, with notable contributions including the use of adaptive parameter tuning and hybrid paradigms that integrate steady-state visual evoked potentials to improve detection accuracy and user experience \cite{jalilpour2020novel}. Novel electroencephalogram(EEG) decoding algorithms, such as deep learning frameworks that take advantage of convolutional neural networks \cite{santamaria2020eeg} and attention mechanisms\cite{wang2020linking}, have further enhanced classification performance, while collaborative approaches of multiple users have demonstrated the potential for improved accuracy through collective neural signal analysis \cite{9931160}. The introduction of benchmark data sets has standardized the evaluation of algorithms and facilitated reproducible research \cite{zhang2020benchmark}. Furthermore, the integration of multimodal data, including EEG and eye tracking, has shown promise in addressing signal noise and enhancing target detection reliability \cite{mao2023cross}, cementing RSVP-BCIs as a crucial interface for bridging neuroscience and real-world applications. The most related work to ours is by Yujie Cui et al.\cite{cui2022dynamic}, who proposed a human-computer fusion method called Dynamic Probability Integration for nighttime vehicle detection. Their approach uses a probability assignment method to assign classification weights between different information sources, which requires full human participation in the EEG-based RSVP task. In contrast, our model reduces human effort by using uncertainty to guide collaboration, with humans only evaluating high-uncertainty samples from the CV model, thus improving efficiency.