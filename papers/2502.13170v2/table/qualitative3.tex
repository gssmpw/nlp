\begin{table*}[ht!]
    \caption{We compare the results obtained using the sub-hypothesis decomposition method with those obtained without it. The results without hypothesis decomposition are presented at the top of the table, while those with hypothesis decomposition are shown below. Benchmark ARC-ID37.}
    \centering
    \scalebox{0.8}{
    \begin{tabular}{ccc}
\toprule
\textbf{Observations} & \textbf{Rounds} & \textbf{Executable Function} \\ \midrule
\multirow{2}{*}{\thead{\includegraphics[width=4cm]{fig/arc_case2.pdf}}} & \thead{\textbf{Round 1:}\\Step 1: Identify the third row\\Step 2: Check for a non-zero \\ numbers in the third row.\\ Step 3: Replace the number at \\the center position of the sequence. \\ Step 4: Change all numbers \\in rows 4 and 5 to zero.} & \thead{\lstinputlisting[language=Python, style=mystyle]{codes/ARC2_round1.py}} \\ \cline{2-3} 
& \thead{\textbf{Round 2:}\\Step 1: Identify the third row\\Step 2: Check for a non-zero\\ numbers in the third row.\\ Step 3: Replace identical numbers\\ in the third row with the\\ corresponding non-zero  number\\ from the first row.} & \thead{\lstinputlisting[language=Python, style=mystyle]{codes/ARC2_round2.py}} \\
\bottomrule
\end{tabular}
    }
    \label{tab:case_study_feed2}
\vspace{-4mm}
\end{table*}