\begin{table*}[ht!]
    \caption{We present the additional improvements resulting from the use of amendments. The results shown above are those obtained after the initial hypotheses, while the results displayed below reflect the outcomes following the submission of amendments and subsequent reflections. Benchmark: List Function-ID29.}
    \centering
    \scalebox{0.8}{
    \begin{tabular}{ccc}
\toprule
\textbf{Observations} & \textbf{Rounds} & \textbf{Executable Function} \\ \midrule
\multirow{2}{*}{\thead{\scriptsize{\texttt{[0, 8, 3, 9]}} $\rightarrow$ \scriptsize{{\texttt{[0, 8]}}} \\ \scriptsize{\texttt{[6, 1]}} $\rightarrow$ \scriptsize{\texttt{[]}}  \\ \scriptsize{\texttt{[4, 8, 7]}} $\rightarrow$ \scriptsize{\texttt{[4]}} \\ \dots }} &
\thead{\textbf{Round 1:} \\Step 1: Remove elements from the end \\ of the input list until a '9' is encountered. \\ Step 2: If '9' is present, remove all \\ elements after the last occurrence of '9'. \\ Step 3: If '9' is not present, remove the \\ last two elements of the list. \\ Step 4: If the list has fewer than two \\ elements, return an empty list.} & \thead{\lstinputlisting[language=Python, style=mystyle]{codes/LF1_round1.py}} \\ \cline{2-3} & \thead{\textbf{Round 2:} \\Step 1: Remove the last two elements\\ from the input list. \\ Step 2: If the resulting list has fewer than \\ two elements, return an empty list.} & \thead{\lstinputlisting[language=Python, style=mystyle]{codes/LF1_round2.py}} \\
 \bottomrule
\end{tabular}
    }
    \label{tab:case_study_feed1)}
\vspace{-4mm}
\end{table*}

