\begin{table*}[ht!]
    \caption{We present the additional improvements resulting from the use of amendments. The results shown above are those obtained after the initial hypotheses, while the results displayed below reflect the outcomes following the submission of amendments and subsequent reflections. Benchmark CruxEval Output-ID328.}
    \centering
    \scalebox{0.8}{
    \begin{tabular}{ccc}
\toprule
\textbf{Observations} & \textbf{Rounds} & \textbf{Executable Function} \\ \midrule
\multirow{2}{*}{\thead{\lstinputlisting[language=Python, style=mystyle]{codes/CRUX1_observation.py}}} & \thead{\textbf{Round 1:} \\ Step 1 Base Case Check: \\The function checks if `L' is\\ less than or equal to 0.\\ If true, it simply returns\\ the array as is \\ Step 2 Recursive Extension: \\If the length of `array' \\is less than L, it then\\ calls itself recursively \\ Step 3 Return Array:\\If the array is already of\\ length `L` or longer, the\\ function simply returns the \\array without any modifications.} & assert f([1, 2, 3], 4) == [1, 2, 3, 1] \\ \cline{2-3} 
& \thead{\textbf{Round 2:}\\ Step 1 Base Case Check: ...\\ Step 2 Recursive Extension: ...\\ Step 3 Return Array: ...\\ Step 4 Example Check: \\The original call then extends\\ `[1, 2, 3]' by `[1, 2, 3]',\\ resulting in `[1, 2, 3, 1, 2, 3]'.} & \thead{assert f([1, 2, 3], 4) == [1, 2, 3, 1, 2, 3]} \\
\bottomrule
\end{tabular}
    }
    \label{tab:case_study_feed3}
\vspace{-4mm}
\end{table*}