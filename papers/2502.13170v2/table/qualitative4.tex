\begin{table*}[ht!]
    \caption{We compare the results obtained using the sub-hypothesis decomposition method with those obtained without it. The results without hypothesis decomposition are presented at the top of the table, while those with hypothesis decomposition are shown below. Benchmark List Function-ID2.}
    \centering
    \scalebox{0.8}{
    \begin{tabular}{ccc}
\toprule
\textbf{Observations} & \textbf{Hypothesis} & \textbf{Executable Function} \\ \midrule
\multirow{2}{*}{\thead{\scriptsize{\texttt{[]}} $\rightarrow$ \scriptsize{{\texttt{[]}}} \\ \scriptsize{\texttt{[6, 9]}} $\rightarrow$ \scriptsize{{\texttt{[]}}} \\ \scriptsize{\texttt{[1, 5, 0, 6, 2, 9, 3]}} $\rightarrow$ \scriptsize{\texttt{[0]}}  \\ \scriptsize{\texttt{[6, 3, 4, 1, 7, 2, 9, 8, 0]}} $\rightarrow$ \scriptsize{\texttt{[4]}} \\ \dots }} & \thead{\textbf{No Sub Hypothesis:}\\The output is the element \\ from the input list that \\ is exactly in the middle \\ of the list.} & \thead{\lstinputlisting[language=Python, style=mystyle]{codes/LF2_round1.py}} \\ \cline{2-3} 
& \thead{\textbf{Sub Hypothesis:}\\Step 1: If the list has fewer \\ than three elements,\\ the output is an empty list.\\ Step 2:If the list has three \\ or more elements,\\ the output is a list containing 
 \\ only the integer at the third \\ position.} & \thead{\lstinputlisting[language=Python, style=mystyle]{codes/LF2_round2.py}} \\
\bottomrule
\end{tabular}
    }
    \label{tab:case_study_hyp2}
\vspace{-4mm}
\end{table*}