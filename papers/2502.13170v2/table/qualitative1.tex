\begin{table*}[ht!]
    \caption{We compare the results obtained using the sub-hypothesis decomposition method with those obtained without it. The results without hypothesis decomposition are presented at the top of the table, while those with hypothesis decomposition are shown below. Benchmark: MiniARC-ID26.}
    \centering
    \scalebox{0.8}{
    \begin{tabular}{ccc}
\toprule
\textbf{Observations} & \textbf{Hypothesis} & \textbf{Executable Function} \\ \midrule
\multirow{2}{*}{\thead{\includegraphics[width=3.5cm]{fig/arc_case1.pdf}}} & \thead{\textbf{No Sub Hypothesis:}\\For a given 5x5 matrix input,\\ shift the first row to the last row,\\ the second row to the fourth row,\\ the third row to the third row \\(unchanged), the fourth row to the \\second row, and the fifth \\row to the first row. This rotates \\the rows up by two positions.
} & \thead{\lstinputlisting[language=Python, style=mystyle]{codes/ARC1_round1.py}} \\ \cline{2-3} 
& \thead{\textbf{Sub Hypothesis:}\\Step 1: Identify the non-zero rows.\\Step 2: Move non-zero rows to the bottom.\\ Step 3: Shift all rows down to fill the grid.\\Step 4: Repeat steps for next input.} & \thead{\lstinputlisting[language=Python, style=mystyle]{codes/ARC1_round2.py}} \\
\bottomrule
\end{tabular}
    }
    \label{tab:case_study_hyp1}
\vspace{-4mm}
\end{table*}

% \multirow{2}{*}{\thead{\includegraphics[width=4cm]{fig/arc_case.pdf}}} & \thead{\textbf{Round 1:}\\Step 1: Identify the third row\\Step 2: Check for a non-zero numbers\\ in the third row.\\ Step 3: Replace the number at the center \\position of the sequence. \\ Step 4: Change all numbers in rows\\ 4 and 5 to zero.} & \thead{\lstinputlisting[language=Python, style=mystyle]{codes/ARC_round1.py}} \\ \cline{2-3} 
% % & \thead{\textbf{Round 2:}\\Step 1: Identify the third row\\Step 2: Check for a non-zero numbers\\ in the third row.\\ Step 3: Replace identical numbers in the third \\ row with the corresponding non-zero \\ number from the first row.} & \thead{\lstinputlisting[language=Python, style=mystyle]{codes/ARC_round1.py}}
