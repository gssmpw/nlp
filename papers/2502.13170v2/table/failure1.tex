\begin{table*}[ht!]
    \caption{Failure case study. The observations indicate that the hypotheses proposed by the LLM are overly simplistic, rendering it difficult to account for all possible cases. Additionally, manual inspection and efforts to summarize these hypotheses proved challenging. This limitation is one of the primary factors contributing to the LLM's failure in this task. Benchmark: List Function-ID9.}
    \centering
    \scalebox{0.8}{
    \begin{tabular}{ccc}
\toprule
\textbf{Observations} & \textbf{Rounds} & \textbf{Executable Function} \\ \midrule
\multirow{2}{*}{\thead{\scriptsize{\texttt{[2, 8]}} $\rightarrow$ \scriptsize{{\texttt{[8]}}} \\ \scriptsize{\texttt{[7, 5, ..., 8, 4]}} $\rightarrow$ \scriptsize{\texttt{[5, ..., 8]}} \\ \scriptsize{\texttt{[8, 2, ..., 9]}} $\rightarrow$ \scriptsize{\texttt{[2, ..., 9]}} \\ \scriptsize{\texttt{[3, 2, 1, 0, 7, 8]}} $\rightarrow$ \scriptsize{\texttt{[2, 1, 0]}} \\ \dots \\ \dots }} &
\thead{\textbf{Round 1:} \\ Remove the first element \\ from the input list.} & \thead{\lstinputlisting[language=Python, style=mystyle]{codes/LF_fail_round1.py}} \\ \cline{2-3} & \thead{\textbf{Round 2:} \\Remove the first and the last \\elements from the input list.} & \thead{\lstinputlisting[language=Python, style=mystyle]{codes/LF_fail_round2.py}} \\
 \bottomrule
\end{tabular}
    }
    \label{tab:fail1}
\vspace{-4mm}
\end{table*}