\begin{table*}[ht!]
    \caption{Failure case study. The LLM make some redundant hypotheses, which led to failure. Even after amendments, it was unable to break free from its original thought framework, resulting in a failure to reflect effectively. Benchmark: ARC-ID5.}
    \centering
    \scalebox{0.8}{
    \begin{tabular}{ccc}
\toprule
\textbf{Observations} & \textbf{Rounds} & \textbf{Executable Function} \\ \midrule
\multirow{2}{*}{\thead{\includegraphics[width=3.5cm]{fig/arc_failure1.pdf}}} &
\thead{\textbf{Round 1:} \\ Step 1: Identify non-zero elements\\ in the input grid. \\ Step 2: Shift all non-zero elements \\ down one row. \\ Step 3: Repeat the process for each \\ input-output pair to verify \\ consistency.} & \thead{\lstinputlisting[language=Python, style=mystyle]{codes/ARC_fail_round1.py}} \\ \cline{2-3} & \thead{\textbf{Round 2:} \\ Step 1:Identify non-zero elements\\ in the input grid. \\ Step 2: Shift all non-zero\\ elements down one row. \\ Step 3: In the resulting grid, fill each\\ row with the maximum value\\ from its respective column,\\ considering only the shifted \\non-zero elements.\\ Step 4: Repeat the process for each \\ input-output pair to verify consistency.} & \thead{\lstinputlisting[language=Python, style=mystyle]{codes/ARC_fail_round2.py}} \\
 \bottomrule
\end{tabular}
    }
    \label{tab:fail2}
\vspace{-4mm}
\end{table*}