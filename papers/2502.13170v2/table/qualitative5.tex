\begin{table*}[ht!]
    \caption{We compare the results obtained using the sub-hypothesis decomposition method with those obtained without it. The results without hypothesis decomposition are presented at the top of the table, while those with hypothesis decomposition are shown below. Benchmark Livecodebench Input-ID37.}
    \centering
    \scalebox{0.8}{
    \begin{tabular}{ccc}
\toprule
\textbf{Observations} & \textbf{Hypothesis} & \textbf{Executable Function} \\ \midrule
\multirow{2}{*}{\thead{\lstinputlisting[language=Python, style=mystyle]{codes/LIVE1_observation.py}}} & \thead{\textbf{No Sub Hypothesis:}\\The function `minOperations' attempts \\to determine the minimum number \\of swaps needed to ensure \\that for each index `i',\\ neither `a[i]' nor `b[i]'\\ are greater than the last\\ elements of their respective lists.} & assert minOperations([3, 1, 2], [1, 3, 2]) == 1 \\ \cline{2-3} 
& \thead{\textbf{Sub Hypothesis:}\\Step 1: \textbf{Function Purpose} \\The goal of `minOperations` is to ... \\ Step 2: \textbf{Inner Function} \\ It attempts to ensure \\ that for each `i', ... \\ Step 3: \textbf{Main Logic} \\ It calculates `ans` by calling `f', ... \\ Step 4: \textbf{Objective} \\ To find an input such that... } & \thead{assert minOperations([2, 3], [3, 2]) == 1} \\
\bottomrule
\end{tabular}
    }
    \label{tab:case_study_hyp3}
\vspace{-4mm}
\end{table*}