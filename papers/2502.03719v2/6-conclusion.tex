\section{Conclusion}
We introduced the concept of code shaping, an interaction paradigm that enables programmers to iteratively edit code using free-form sketch annotations directly on and around the code. Through three stages of design iterations and user studies, we explored how programmers perceive code shaping, the types of sketches they create, common AI interpretation errors, and how they recover from them. We also investigated interface design strategies to effectively bridge the layers between textual code and sketches, such as providing always-on feedforward and integrating unique gestures to minimize context switching. Our findings offer valuable insights into how sketch-based interactions can support code planning and editing, informing future research and design in this emerging area.