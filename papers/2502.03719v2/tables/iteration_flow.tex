\small
\begin{tabular}{cp{2.2cm}>{\raggedright\arraybackslash}p{4.2cm}>{\raggedright\arraybackslash}p{6cm}c}
\toprule
\textbf{ID} & \textbf{Name} & \textbf{Description} & \textbf{Example Scenario} & \textbf{$N$} \\
\midrule
1 & \textbf{Sketch-Generate-Accept} 
& Participants sketch out code concepts, generate the corresponding code, and accept the generated code without further modifications. 
& P14 sketches a method to sort tasks by due date, generates the code, reviews the output, and accepts it as it correctly implements the sorting logic.
& 32 \\
\cmidrule(lr){1-5}
2 & \textbf{Sketch-Generate-Reject} 
& Participants sketch a concept, generate the code, and then reject the generated code, leading to further refinement of the sketch or code. 
& P18 sketches a Manhattan distance function, generates the code, sees that the distance calculation is incorrect, and revises the sketch by adding further details before generating the code again. 
& 25 \\
\cmidrule(lr){1-5}
3 & \textbf{Cycle2-Sketch} 
& After rejecting the generated code, participants revisit and modify the sketch before generating the code again. 
& P15 rejects the generated task sorting function after realizing it doesn't account for tasks with no due date. The P then revises the sketch to handle missing due dates and regenerates the code. 
& 17 \\
\cmidrule(lr){1-5}
4 & \textbf{Cycle2-Undo/Redo} 
& Participants use undo/redo actions after rejecting generated code, allowing them to adjust their sketches or code without starting from scratch. 
& P13 generates code to impute missing data, but identifies an error in the logic. The P uses the undo function, adjusts the sketch to refine the imputation method, and regenerates the code. 
& 6 \\
\cmidrule(lr){1-5}
5 & \textbf{Cycle2-Edit Code} 
& Participants directly edit the generated code after rejecting the initial output instead of refining the sketch. 
& P17, after rejecting the code generated, decides to manually edit the code for updating task details instead of re-sketching the concept. 
& 2 \\
\cmidrule(lr){1-5}
6 & \textbf{Sketch-Interpret-Generate} 
& Participants sketch an idea, allow the system to interpret it, and then decide whether to accept or reject the system's interpretation and the generated code. 
& P16 sketches a conditional statement. The system interprets the sketch and generates the corresponding code, which the P reviews and accepts as it matches the intended logic. 
& 38/57 \\
\cmidrule(lr){1-5}
7 & \textbf{Sketch-Generate-Compile} 
& Participants generate code from a sketch, compile the code, and then debug and refine the sketch based on compilation or runtime errors. 
& P14 generates the code, compiles it, and encounters a runtime error. The P modifies the sketch to correct the issue, regenerates the code, and recompiles successfully. 
& 11/57 \\
\bottomrule
\end{tabular}