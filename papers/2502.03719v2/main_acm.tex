%% Commands for TeXCount
%TC:macro \cite [option:text,text]
%TC:macro \citep [option:text,text]
%TC:macro \citet [option:text,text]
%TC:envir table 0 1
%TC:envir table* 0 1
%TC:envir tabular [ignore] word
%TC:envir displaymath 0 word
%TC:envir math 0 word
%TC:envir comment 0 0
%
% don't forget to set review=true before submitting
% NOTE: could use "{lib/acmart}" to use the local ACM style in this template, but be warned it seems to cause some problems (like reference style). Best is to stick with version already in Overleaf "{acmart}"
% \documentclass[format=manuscript, review=true, anonymous=true, screen, dvipsnames]{acmart}
% use format=sigconf to preview two columns
% use format=manuscript, review=true, anonymous=true to prepare submission with red line numbering
% use format=acmsmall, review=false to get a nicer one-column reading experience
% \documentclass[format=manuscript, review=false, anonymous=false]{acmart}
\documentclass[sigconf, nonacm]{acmart}
% use format=acmsmall, review=false anc call the exii \addmarkupspace macro below to prepare a nice PDF for handwritten comments 
% use authorversion=false to show the final copyright text
% use authorversion=true to show the author version copyright text below (for pre-prints or for versions published on our personal websites)
% use screen to make hyperlinks a different colour from the main text

% Apparently, when we publish our own copy of our ACM papers (for example, when we make them available at our lab's website or university repository), we should replace the standard ACM copyright message and ISBN lines with the following:
% "© {Owner/Author | ACM} {Year}. This is the author's version of the work. It is posted here for your personal use. Not for redistribution. The definitive Version of Record was published in {Source Publication}, http://dx.doi.org/10.1145/{number}."
% Reference: https://www.acm.org/publications/policies/copyright-policy#permanent%20rights

%% Rights management information.  This information is sent to you
%% when you complete the rights form.  These commands have SAMPLE
%% values in them; it is your responsibility as an author to replace
%% the commands and values with those provided to you when you
%% complete the rights form.
% \setcopyright{acmlicensed}
% \copyrightyear{2018}
% \acmYear{2018}
% \acmDOI{XXXXXXX.XXXXXXX}

%% These commands are for a PROCEEDINGS abstract or paper.
% \acmConference[Conference acronym 'XX]{Make sure to enter the correct
  % conference title from your rights confirmation emai}{June 03--05,
  % 2018}{Woodstock, NY}
%%
%%  Uncomment \acmBooktitle if the title of the proceedings is different
%%  from ``Proceedings of ...''!
%%
%%\acmBooktitle{Woodstock '18: ACM Symposium on Neural Gaze Detection,
%%  June 03--05, 2018, Woodstock, NY}
% \acmISBN{978-1-4503-XXXX-X/18/06}

% Use these to make cleaner submission without weird fake names
% \acmConference[]{}{}{}
% \acmYear{}
% \copyrightyear{}
% \acmPrice{}
% \acmDOI{}
% \acmISBN{}
% \setcopyright{none}

%%
%% Submission ID.
%% Use this when submitting an article to a sponsored event. You'll
%% receive a unique submission ID from the organizers
%% of the event, and this ID should be used as the parameter to this command.
% \acmSubmissionID{123-A56-BU3}

%%
%% For managing citations, it is recommended to use bibliography
%% files in BibTeX format.
%%
%% You can then either use BibTeX with the ACM-Reference-Format style,
%% or BibLaTeX with the acmnumeric or acmauthoryear sytles, that includ
%% support for advanced citation of software artefact from the
%% biblatex-software package, also separately available on CTAN.
%%
%% Look at the sample-*-biblatex.tex files for templates showcasing
%% the biblatex styles.
%%

%%
%% The majority of ACM publications use numbered citations and
%% references.  The command \citestyle{authoryear} switches to the
%% "author year" style.
%%
%% If you are preparing content for an event
%% sponsored by ACM SIGGRAPH, you must use the "author year" style of
%% citations and references.
%% Uncommenting
%% the next command will enable that style.
%%\citestyle{acmauthoryear}




% Tweaks to submission format
% --------------------------------

% for submissions:
% printfolios=true to print page numbers (reviewers need page numbers)
% printacmref=false to not show the useless ACM reference on first page (looks nicer, saves space)
% printccs=false to not print CCS block (not important for submissions, saves space)
\settopmatter{printacmref=false, printccs=false, printfolios=false}

% Macros and Formatting Specific to this Document
% --------------------------------
\usepackage{svg}
\usepackage{booktabs} % For formal tables


% --------------------------------
% DOCUMENT SETUP
% edit this file to add exii macros, make formatting tweaks, 
% setup author inline commenting, control annotation visibility, 
% insert document specific macros, etc.
% Exii Standard Document Setup
% ================================

% various standard exii macros
\usepackage{exii-macros}

% additional latex packages
\usepackage{booktabs}
\usepackage{array}
\usepackage{subcaption}
\usepackage{fontawesome}
\usepackage{multirow}
\usepackage{graphicx}
\usepackage{wrapfig}
\usepackage{lipsum}
% \usepackage{svg}
\usepackage{makecell}


\definecolor{codecolor}{RGB}{239,239,237}
\definecolor{codetextcolor}{RGB}{165,0,0}

\newcommand{\inlinecode}[1]{%
  \colorbox{codecolor}{\textcolor{codetextcolor}{\small \texttt{#1}}}%
}


% Author comments
% See exii-macros.sty for colours the last line and choose a colour
% IMPORTANT: also enter texcount TC macros for each author!
% (otherwise even hidden comments are included in the word count)
\newcommand{\jz}[1]{\authorcomment{PURPLE}{J}{#1}} 
%TC:macro \jz [ignore]
\newcommand{\dv}[1]{\authorcomment{BLUE}{D}{#1}}
% TC:macro \dv [ignore]
\newcommand{\ry}[1]{\authorcomment{GREEN}{R}{#1}} 
%TC:macro \ry [ignore]

% also ignore outline comments
% TC:macro \outline [ignore]

% macros specific to this project
\newcommand{\Conte}{Cont\'{e}\xspace}

% Annotation Visibility Control
% --------------------------------

% *** HIDE INLINE COMMENTS HERE ***
% command below will hide all author comments, simple comments like guide, and 
% make markup comments like fixme return to black text
% \hidecomments
% or, just hide the grey outline comments
% \hideoutline

% *** CAMERA-READY REVISION HIGHLIGHTING ***
% command below highlights major revisions when submitting camera-ready. In text, you need to wrap major revisions sections with \rev macro.
% \showrevisions{REVISIONGREEN}

% Formatting Tweaks
% --------------------------------

% *** NICE DRAFT FORMAT WITH HANDWRITTEN MARK-UP SPACE (ACM ONLY) ***
% Increase margins and line spacing to make it easier 
% to markup a 1-column PDF with handwritten comments. 
% Use this with format=acmsmall in documentclass
% Note this causes a latex error because the acmart template
% doesn't allow you to change baselinestretch (but we only
% change it for our internal drafts using this command}
% \addmarkupspace

% By default, the template aligns the columns at the bottom of each page.
% This inserts uneven vertical spacing between paragraphs with looks awful, 
% and wastes space. Using a "ragged bottom" is better.
\raggedbottom

% The template has insanely huge space around captions, we can reduce this.
% Be cautious, some white space is needed to separate caption from text. 
\setlength{\abovecaptionskip}{3pt}
\setlength{\belowcaptionskip}{-3pt}

% To control the white space below and above equations
\makeatletter
\g@addto@macro\normalsize{%
  \setlength\abovedisplayshortskip{-9pt}
  \setlength\belowdisplayshortskip{3pt}
}
\makeatother

% cheat by reducing line spacing: AVOID DOING THIS UNLESS DESPERATE!
% \renewcommand{\baselinestretch}{0.95} % default is 1.0

\renewcommand{\figurename}{Figure}
\renewcommand{\tablename}{Table}
\renewcommand{\sectionautorefname}{Section}
\renewcommand{\subsectionautorefname}{Section}
\renewcommand{\subsubsectionautorefname}{Section} 
% --------------------------------


% Begin Document
% --------------------------------

\begin{document}

% this seems to help remove words going beyond the margin
\tolerance=400 

%%
%% The "title" command has an optional parameter,
%% allowing the author to define a "short title" to be used in page headers.
%
% The title should be short but descriptive, like a mini abstract. 
% If possible, come up with a catchy name for your project and use it as part of the title.
% can include a short version of the title for the running header (in square brackets)
\title[Code Shaping]{Code Shaping: Iterative Code Editing with Free-form AI-Interpreted Sketching}
\newcommand{\sys}[0]{\f{Code Shaping}}
\newcommand{\baseline}[0]{\textit{Baseline}}

% \titlenote{Produces the permission block, and
%   copyright information}
% \subtitle{Extended Abstract}
% \subtitlenote{The full version of the author's guide is available as
%   \texttt{acmart.pdf} document}


%% The "author" command and its associated commands are used to define
%% the authors and their affiliations.

\author{Ryan Yen}
\orcid{0001-8212-4100}

\affiliation{%
  \institution{School of Computer Science, University of Waterloo}
  \country{}
}
% \email{r4yen@uwaterloo.ca}
\affiliation{%
  \institution{CSAIL, MIT}
  \streetaddress{77 Massachusetts Ave}
  % \city{Cambridge}
  % \state{Massachusetts}
  \country{}
}
\email{ryanyen2@mit.edu}


\author{Jian Zhao}
\orcid{0002-7761-6351}

\affiliation{%
  \institution{School of Computer Science, University of Waterloo}
  \country{}
}
% \authornote{Corresponding Author}
\email{jianzhao@uwaterloo.ca}

\author{Daniel Vogel}
\orcid{0000-0001-7620-0541}
\affiliation{%
  \institution{School of Computer Science, University of Waterloo}
  \country{}
}
\email{dvogel@uwaterloo.ca}


% If default list of authors is too long for headers.
\renewcommand{\shortauthors}{Yen et al.}

%%
%% The abstract is a short summary of the work to be presented in the
%% article.
\begin{abstract}
\begin{abstract}
  Game theory establishes a fundamental framework for analyzing strategic interactions among rational decision-makers. The rapid advancement of large language models (LLMs) has sparked extensive research exploring the intersection of these two fields. Specifically, game-theoretic methods are being applied to evaluate and enhance LLM capabilities, while LLMs themselves are reshaping classic game models. This paper presents a comprehensive survey of the intersection of these fields, exploring a bidirectional relationship from three perspectives: (1) Establishing standardized game-based benchmarks for evaluating LLM behavior; (2) Leveraging game-theoretic methods to improve LLM performance through algorithmic innovations; (3) Characterizing the societal impacts of LLMs through game modeling. Among these three aspects, we also highlight how the equilibrium analysis for traditional game models is impacted by LLMs' advanced language understanding, which in turn extends the study of game theory. Finally, we identify key challenges and future research directions, assessing their feasibility based on the current state of the field. By bridging theoretical rigor with emerging AI capabilities, this survey aims to foster interdisciplinary collaboration and drive progress in this evolving research area. 
    % By synthesizing insights from computational game theory and contemporary AI research, this work aims to stimulate interdisciplinary collaboration and inform the development of robust frameworks for AI-driven strategic decision-making.
\end{abstract}


% Game theory have been employed to boost development of large language models (LLMs) in both theoretical and technical fields. In the mean time LLMs as new game subjects have been put into game scenarios to play and be analyzed. The promotion of games and large language models (LLMs) is bidirectional. Recent studies analyze LLMs in game with numerous dimensions, including evaluation of LLMs' behavioral performance LLMs struggle in matrix games, methods to enhance LLMs' game performance, and how LLMs can serve beyond as a game player. In parallel, game theory, known for its advantages in addressing complex equilibrium problems, game-theoretic issues, and the integration of diverse perspectives, offers a promising guidance for phenomenological understanding LLMs and stimulaing LLM algorithms. More than that, with the deepening of interaction of LLMs and game, there are also original game models that are born LLM related. In this survey, we aim to comprehensively assess the ralationship between currect game and LLMs development. Besides, we propose a new taxonomy of game for LLMs and LLMs for game to systematically categorize related works in this emerging field. Our analysis includes novel frameworks and definitions, highlighting potential research directions and challenges at this intersection. Through this study, we aim to stimulate targeted advancements with game theory and LLM together.
\end{abstract}

%
% The code below should be generated by the tool at
% http://dl.acm.org/ccs.cfm
% Please copy and paste the code instead of the example below.
%
\begin{CCSXML}
<ccs2012>
   <concept>
       <concept_id>10003120.10003121.10003129.10011756</concept_id>
       <concept_desc>Human-centered computing~User interface programming</concept_desc>
       <concept_significance>500</concept_significance>
       </concept>
   <concept>
       <concept_id>10003120.10003121.10003128</concept_id>
       <concept_desc>Human-centered computing~Interaction techniques</concept_desc>
       <concept_significance>500</concept_significance>
       </concept>
 </ccs2012>
\end{CCSXML}

\ccsdesc[500]{Human-centered computing~User interface programming}
\ccsdesc[500]{Human-centered computing~Interaction techniques}


% keep keywords to one line in rendered paper, try to use big topics that aren't
% already in your title
% \keywords{Ink-based Sketching, Dynamic Abstraction, Programming Interface}

% % optional full width teaser figure
\begin{teaserfigure}
\centering
  \includegraphics[width=\linewidth]{figures/teaser_final.pdf}
  \caption{Code shaping usage example: (a) a programmer draws an arrow from a few lines of code defining data attributes to a sketch of a bar chart in whitespace near the code, then they add another arrow back to a different code location and annotate the arrow with `def'; (b) an AI model uses the code and the overlaid sketches to insert a new function to plot that data; (c) the programmer reviews the edits interpreted by the model, then they run the program; (d) the code outputs a rendered plot, the programmer sketches on top of it to indicate it should use min-max scaling; (e) the model examines the new sketches and modifies the code to implement scaling.}  
  \Description[Code shaping process with AI assistance.]{The figure illustrates an interactive code shaping process that integrates programmer input and AI assistance. It begins with (a) a Python code editor displaying a data preprocessing script and a user adding handwritten annotations and arrows linking the code to a sketched bar chart. A label 'def' is drawn to indicate the function definition process. Then, in step (b), the AI interprets these annotations, inserting a plot_features function into the code editor. In step (c), the programmer reviews and runs the updated code, and in (d), the output is displayed as a bar chart with handwritten annotations indicating "Min-max scaling." Finally, step (e) shows the AI modifying the code further to include scaling functionality. The process demonstrates an iterative collaboration between the programmer and AI, highlighted by handwritten notes and diagrams across stages.}
  \label{fig:teaser}
\end{teaserfigure}



% \begin{figure}[htbp]
%   \centering
%   \includesvg{image.svg}
%   \caption{svg image}
% \end{figure}

%%
%% This command processes the author and affiliation and title
%% information and builds the first part of the formatted document.
\maketitle

% --------------------------------
% BODY
% edit this file to insert your sections
% Exii Standard Section Index
% ================================

% sections are each in separate files

% \section{Introduction}

Despite the remarkable capabilities of large language models (LLMs)~\cite{DBLP:conf/emnlp/QinZ0CYY23,DBLP:journals/corr/abs-2307-09288}, they often inevitably exhibit hallucinations due to incorrect or outdated knowledge embedded in their parameters~\cite{DBLP:journals/corr/abs-2309-01219, DBLP:journals/corr/abs-2302-12813, DBLP:journals/csur/JiLFYSXIBMF23}.
Given the significant time and expense required to retrain LLMs, there has been growing interest in \emph{model editing} (a.k.a., \emph{knowledge editing})~\cite{DBLP:conf/iclr/SinitsinPPPB20, DBLP:journals/corr/abs-2012-00363, DBLP:conf/acl/DaiDHSCW22, DBLP:conf/icml/MitchellLBMF22, DBLP:conf/nips/MengBAB22, DBLP:conf/iclr/MengSABB23, DBLP:conf/emnlp/YaoWT0LDC023, DBLP:conf/emnlp/ZhongWMPC23, DBLP:conf/icml/MaL0G24, DBLP:journals/corr/abs-2401-04700}, 
which aims to update the knowledge of LLMs cost-effectively.
Some existing methods of model editing achieve this by modifying model parameters, which can be generally divided into two categories~\cite{DBLP:journals/corr/abs-2308-07269, DBLP:conf/emnlp/YaoWT0LDC023}.
Specifically, one type is based on \emph{Meta-Learning}~\cite{DBLP:conf/emnlp/CaoAT21, DBLP:conf/acl/DaiDHSCW22}, while the other is based on \emph{Locate-then-Edit}~\cite{DBLP:conf/acl/DaiDHSCW22, DBLP:conf/nips/MengBAB22, DBLP:conf/iclr/MengSABB23}. This paper primarily focuses on the latter.

\begin{figure}[t]
  \centering
  \includegraphics[width=0.48\textwidth]{figures/demonstration.pdf}
  \vspace{-4mm}
  \caption{(a) Comparison of regular model editing and EAC. EAC compresses the editing information into the dimensions where the editing anchors are located. Here, we utilize the gradients generated during training and the magnitude of the updated knowledge vector to identify anchors. (b) Comparison of general downstream task performance before editing, after regular editing, and after constrained editing by EAC.}
  \vspace{-3mm}
  \label{demo}
\end{figure}

\emph{Sequential} model editing~\cite{DBLP:conf/emnlp/YaoWT0LDC023} can expedite the continual learning of LLMs where a series of consecutive edits are conducted.
This is very important in real-world scenarios because new knowledge continually appears, requiring the model to retain previous knowledge while conducting new edits. 
Some studies have experimentally revealed that in sequential editing, existing methods lead to a decrease in the general abilities of the model across downstream tasks~\cite{DBLP:journals/corr/abs-2401-04700, DBLP:conf/acl/GuptaRA24, DBLP:conf/acl/Yang0MLYC24, DBLP:conf/acl/HuC00024}. 
Besides, \citet{ma2024perturbation} have performed a theoretical analysis to elucidate the bottleneck of the general abilities during sequential editing.
However, previous work has not introduced an effective method that maintains editing performance while preserving general abilities in sequential editing.
This impacts model scalability and presents major challenges for continuous learning in LLMs.

In this paper, a statistical analysis is first conducted to help understand how the model is affected during sequential editing using two popular editing methods, including ROME~\cite{DBLP:conf/nips/MengBAB22} and MEMIT~\cite{DBLP:conf/iclr/MengSABB23}.
Matrix norms, particularly the L1 norm, have been shown to be effective indicators of matrix properties such as sparsity, stability, and conditioning, as evidenced by several theoretical works~\cite{kahan2013tutorial}. In our analysis of matrix norms, we observe significant deviations in the parameter matrix after sequential editing.
Besides, the semantic differences between the facts before and after editing are also visualized, and we find that the differences become larger as the deviation of the parameter matrix after editing increases.
Therefore, we assume that each edit during sequential editing not only updates the editing fact as expected but also unintentionally introduces non-trivial noise that can cause the edited model to deviate from its original semantics space.
Furthermore, the accumulation of non-trivial noise can amplify the negative impact on the general abilities of LLMs.

Inspired by these findings, a framework termed \textbf{E}diting \textbf{A}nchor \textbf{C}ompression (EAC) is proposed to constrain the deviation of the parameter matrix during sequential editing by reducing the norm of the update matrix at each step. 
As shown in Figure~\ref{demo}, EAC first selects a subset of dimension with a high product of gradient and magnitude values, namely editing anchors, that are considered crucial for encoding the new relation through a weighted gradient saliency map.
Retraining is then performed on the dimensions where these important editing anchors are located, effectively compressing the editing information.
By compressing information only in certain dimensions and leaving other dimensions unmodified, the deviation of the parameter matrix after editing is constrained. 
To further regulate changes in the L1 norm of the edited matrix to constrain the deviation, we incorporate a scored elastic net ~\cite{zou2005regularization} into the retraining process, optimizing the previously selected editing anchors.

To validate the effectiveness of the proposed EAC, experiments of applying EAC to \textbf{two popular editing methods} including ROME and MEMIT are conducted.
In addition, \textbf{three LLMs of varying sizes} including GPT2-XL~\cite{radford2019language}, LLaMA-3 (8B)~\cite{llama3} and LLaMA-2 (13B)~\cite{DBLP:journals/corr/abs-2307-09288} and \textbf{four representative tasks} including 
natural language inference~\cite{DBLP:conf/mlcw/DaganGM05}, 
summarization~\cite{gliwa-etal-2019-samsum},
open-domain question-answering~\cite{DBLP:journals/tacl/KwiatkowskiPRCP19},  
and sentiment analysis~\cite{DBLP:conf/emnlp/SocherPWCMNP13} are selected to extensively demonstrate the impact of model editing on the general abilities of LLMs. 
Experimental results demonstrate that in sequential editing, EAC can effectively preserve over 70\% of the general abilities of the model across downstream tasks and better retain the edited knowledge.

In summary, our contributions to this paper are three-fold:
(1) This paper statistically elucidates how deviations in the parameter matrix after editing are responsible for the decreased general abilities of the model across downstream tasks after sequential editing.
(2) A framework termed EAC is proposed, which ultimately aims to constrain the deviation of the parameter matrix after editing by compressing the editing information into editing anchors. 
(3) It is discovered that on models like GPT2-XL and LLaMA-3 (8B), EAC significantly preserves over 70\% of the general abilities across downstream tasks and retains the edited knowledge better.
%!TEX root = paper.tex
\section{Introduction}
In programming tasks, text is not always the primary medium for expressing ideas \cite{latoza_maintaining_2006}. Programmers often turn to sketching on whiteboards and paper to externalize thoughts and concepts \cite{cherubini_lets_2007, 6922572, 6065018}. This includes tasks like designing program structure, working out algorithms, and planning code edits~\cite{cherubini_lets_2007, 10.1145/1879211.1879217, sutherland_investigation_2017}.
The informal nature of sketching helps untangle complex tasks, represent abstract ideas, and requires less cognitive effort to comprehend \cite{cherubini_lets_2007, tversky2002sketches, goel1995sketches}.


Prior research has explored programming-by-example systems that transform sketches~\cite{10.1145/22627.22349}, such as diagrams~\cite{10.1145/1281500.1281546}, mathematical symbols~\cite{li2008algosketch, 10.1145/3411764.3445460}, and user interfaces~\cite{tldraw, 910894, microsoft_sketch2code}, into functional programs.
However, these systems often target non-programmers, with the generated code typically hidden or not intended for direct editing.
For programmers, another line of research has enhanced current integrated development environments (IDE) with sketch-based annotation features from the engineering perspective to support note-taking~\cite{sutherland2015observational, 10.1145/1324892.1324935}, facilitate collaboration~\cite{lichtschlag2014codegraffiti}, and aid in planning future code edits~\cite{samuelsson2020eliciting}.
Despite these advancements, sketching and code editing are still largely treated as separate activities in the software development process.

This division stems from the traditional view of programming as primarily text-based~\cite{arawjo_write_2020}, with sketching seen as an auxiliary tool.
Programmers must switch contexts between sketching and coding, potentially losing insights during the translation from visual ideas to code modifications~\cite{parnin2006building, 10.1145/1879211.1879217, bff9b250-7640-39e2-8f34-329fd1552822}. This challenge is exacerbated by the non-linear and dynamic nature of programming, where code is frequently revisited and revised in response to evolving requirements and new discoveries.
Hence, sketches have been primarily considered as a static external representation of the programmer's thoughts instead of ways to interact with code~\cite{sutherland2015observational, 10.1145/1324892.1324935, 1698771}. 



To address this separation, we propose a sketch-based editing approach where a \textit{programmer draws free-form annotations on and around the code to iteratively guide an AI model in modifying code structure, flow, and syntax}: a concept we call \textit{code shaping}. For example, to insert a function to visualize data, a programmer can circle lines of code related to data attributes, draw an arrow to a sketch of a graph, then draw another arrow with the word ``def'' back to a line of code to insert the function (\autoref{fig:teaser}a,b,c). Further iterations of sketching can revise the function name or specify additional data processing steps (\autoref{fig:teaser}d,e). This approach tightly integrates free-form sketching with realtime code editing both visually and operationally, providing programmers with an alternative modality to express modifications.
This approach allows programmers to encapsulate their expectations for the program's functionality and link these sketches directly to syntactic code. However, challenges such as model interpretation errors due to the inherent ambiguity of sketches~\cite{10.1145/1281500.1281527, 10.1145/237091.237119} and the fundamental differences between sketching and coding modalities require further design exploration.
% This is very different than programming-by-example systems where sketches are used to generate complete programs~\cite{10.1145/1281500.1281546, li2008algosketch, 9680034, tldraw, 910894, microsoft_sketch2code}, or the code editors that support a static layer of sketched annotations for note \cite{sutherland2015observational, 10.1145/1324892.1324935,lichtschlag2014codegraffiti,samuelsson2020eliciting}. 
% \dv{I tried to say this "we are different" statement  more concisely, because now in par 1 I talk about programming-by-example and tried to be more specific about works that added sketch annotations to a code editor.}
% We differentiate code shaping from other research that focuses on converting sketched drawings, such as diagrams~\cite{10.1145/1281500.1281546}, mathematical notations~\cite{li2008algosketch}, visualizations~\cite{9680034}, or user interfaces~\cite{tldraw, 910894, microsoft_sketch2code}, directly into code. While these approaches transform sketches into code, they consider sketches as representations of the example final output rather than as a means to interact with and edit existing code for programmers. In contrast, code shaping integrates sketching as an interactive modality for modifying and refining code throughout the programming process.
% Our approach is enabled by an underlying general-purpose AI model. This means that the interpretation of sketches is truly free-form and in principle defined by the programmer, but it also introduces the problem of AI interpretation errors due to the inherent ambiguity of sketches~\cite{10.1145/1281500.1281527, 10.1145/237091.237119}.

We adopted a user-centered design process with 18 programmers using a prototype system probe that implements the code shaping concept. Our findings reveal the types of sketches programmers created, their strategies for correcting AI model errors, and design implications for bridging the conceptual gap between the canvas where sketches are made, the textual code representation, and the AI models. We demonstrate these design implications with two real-world use cases: a productivity break using a tablet and pair programming at a whiteboard.
The contribution of this research is not to claim that code shaping is superior to other interaction paradigms, such as typing, but to establish it as a viable alternative that empowers programmers to iteratively express and refine their code edits through free-form sketches. 
% Importantly, we do not to claim that sketching for code shaping is superior to conventional editing using a keyboard. Our contribution is to propose a viable alternative to empower programmers to iteratively express and refine their code edits through free-form sketches.




% primarily treated code and sketches as separate mediums and modalities for interaction. For instance, studies on annotations in programming view these sketches as static externalization of a programmer's thoughts~\cite{sutherland2015observational, 10.1145/1324892.1324935}, without considering the non-linear and dynamic nature of programming.


% While recent advances in multi-modal large language models have made this concept more feasible, two major challenges remain: model recognition and interaction paradigm switching.
% First, the opaque nature of AI models often forces users to guess why recognition failures occur. This issue is compounded by the inherent ambiguity of sketches, where similar annotations can have different meanings across various scenarios and tasks, necessitating multiple iterations to refine sketches for accurate recognition~\cite{10.1145/1281500.1281527, 4302611}.
% Second, the input modalities (pen vs. keyboard) and the tools designed (eraser vs. backspace) for sketching and code editing are fundamentally different. This discrepancy leads to the problem of context switching, even when the sketching and code editing interfaces are visually integrated.
% These challenges highlight the need to understand how programmers use sketches to express commands, what the common errors are, and how they recover from them.
% The finding will be important for designing a system that supports users in iteratively clarifying and modifying both code and annotations.



% Operationalizing these annotations is an essential step towards bridging the gap between abstract thinking and formal coding, helping programmers express their intentions more intuitively while retaining the dynamism of programming.

% spatial and multi-modal reasoning, allowing programmers to shape their sketches more freely.
% For instance, programmers can sketch their thoughts for solving programming problems, write pseudocode, diagram code structures, or illustrate outputs such as visualizations or user interfaces to add or edit corresponding code. 



% To explore the notion of \textit{code shaping}, 
% Towards a fully functional system supporting code shaping, we aim to understand what programmers intend to convey through sketches and how the system interprets or misinterprets them. \dv{the prev paragraph now covers what is in the prev sentence, I would remove it a sentence and just launch into explaining the study with next sentence}
% We conducted an exploratory study with six programmers, using a prototype that transforms free-form sketches on a code editor into actual code edits and reported results in \autoref{sec:result}.
% \dv{should mention AI in prev sentence} 
% Our results show that programmers gradually develop their own workflow for code shaping and assign different meanings to their sketched annotations in different scenarios.

\section{Background}
\label{sec:background}


\subsection{Code Review Automation}
Code review is a widely adopted practice among software developers where a reviewer examines changes submitted in a pull request \cite{hong2022commentfinder, ben2024improving, siow2020core}. If the pull request is not approved, the reviewer must describe the issues or improvements required, providing constructive feedback and identifying potential issues. This step involves review commment generation, which play a key role in the review process by generating review comments for a given code difference. These comments can be descriptive, offering detailed explanations of the issues, or actionable, suggesting specific solutions to address the problems identified \cite{ben2024improving}.


Various approaches have been explored to automate the code review comments process  \cite{tufano2023automating, tufano2024code, yang2024survey}. 
Early efforts centered on knowledge-based systems, which are designed to detect common issues in code. Although these traditional tools provide some support to programmers, they often fall short in addressing complex scenarios encountered during code reviews \cite{dehaerne2022code}. More recently, with advancements in deep learning, researchers have shifted their focus toward using large-language models to enhance the effectiveness of code issue detection and code review comment generation.

\subsection{Knowledge-based Code Review Comments Automation}

Knowledge-based systems (KBS) are software applications designed to emulate human expertise in specific domains by using a collection of rules, logic, and expert knowledge. KBS often consist of facts, rules, an explanation facility, and knowledge acquisition. In the context of software development, these systems are used to analyze the source code, identifying issues such as coding standard violations, bugs, and inefficiencies~\cite{singh2017evaluating, delaitre2015evaluating, ayewah2008using, habchi2018adopting}. By applying a vast set of predefined rules and best practices, they provide automated feedback and recommendations to developers. Tools such as FindBugs \cite{findBugs}, PMD \cite{pmd}, Checkstyle \cite{checkstyle}, and SonarQube \cite{sonarqube} are prominent examples of knowledge-based systems in code analysis, often referred to as static analyzers. These tools have been utilized since the early 1960s, initially to optimize compiler operations, and have since expanded to include debugging tools and software development frameworks \cite{stefanovic2020static, beller2016analyzing}.



\subsection{LLMs-based Code Review Comments Automation}
As the field of machine learning in software engineering evolves, researchers are increasingly leveraging machine learning (ML) and deep learning (DL) techniques to automate the generation of review comments \cite{li2022automating, tufano2022using, balachandran2013reducing, siow2020core, li2022auger, hong2022commentfinder}. Large language models (LLMs) are large-scale Transformer models, which are distinguished by their large number of parameters and extensive pre-training on diverse datasets.  Recently, LLMs have made substantial progress and have been applied across a broad spectrum of domains. Within the software engineering field, LLMs can be categorized into two main types: unified language models and code-specific models, each serving distinct purposes \cite{lu2023llama}.

Code-specific LLMs, such as CodeGen \cite{nijkamp2022codegen}, StarCoder \cite{li2023starcoder} and CodeLlama \cite{roziere2023code} are optimized to excel in code comprehension, code generation, and other programming-related tasks. These specialized models are increasingly utilized in code review activities to detect potential issues, suggest improvements, and automate review comments \cite{yang2024survey, lu2023llama}. 




\subsection{Retrieval-Augmented Generation}
Retrieval-Augmented Generation (RAG) is a general paradigm that enhances LLMs outputs by including relevant information retrieved from external databases into the input prompt \cite{gao2023retrieval}. Traditional LLMs generate responses based solely on the extensive data used in pre-training, which can result in limitations, especially when it comes to domain-specific, time-sensitive, or highly specialized information. RAG addresses these limitations by dynamically retrieving pertinent external knowledge, expanding the model's informational scope and allowing it to generate responses that are more accurate, up-to-date, and contextually relevant \cite{arslan2024business}. 

To build an effective end-to-end RAG pipeline, the system must first establish a comprehensive knowledge base. It requires a retrieval model that captures the semantic meaning of presented data, ensuring relevant information is retrieved. Finally, a capable LLM integrates this retrieved knowledge to generate accurate and coherent results \cite{ibtasham2024towards}.




\subsection{LLM as a Judge Mechanism}

LLM evaluators, often referred to as LLM-as-a-Judge, have gained significant attention due to their ability to align closely with human evaluators' judgments \cite{zhu2023judgelm, shi2024judging}. Their adaptability and scalability make them highly suitable for handling an increasing volume of evaluative tasks. 

Recent studies have shown that certain LLMs, such as Llama-3 70B and GPT-4 Turbo, exhibit strong alignment with human evaluators, making them promising candidates for automated judgment tasks \cite{thakur2024judging}

To enable such evaluations, a proper benchmarking system should be set up with specific components: \emph{prompt design}, which clearly instructs the LLM to evaluate based on a given metric, such as accuracy, relevance, or coherence; \emph{response presentation}, guiding the LLM to present its verdicts in a structured format; and \emph{scoring}, enabling the LLM to assign a score according to a predefined scale \cite{ibtasham2024towards}. Additionally, this evaluation system can be enriched with the ability to explain reasoning behind verdicts, which is a significant advantage of LLM-based evaluation \cite{zheng2023judging}. The LLM can outline the criteria it used to reach its judgment, offering deeper insights into its decision-making process.






\begin{figure*}
    \centering
    \includegraphics[width=\linewidth]{figures//interface/1st.pdf}
    \caption{Interface design from the first stage. (a) Pen tool with color options for code annotation; (b) Canvas tools including select, pan, pen, eraser, and other common shapes; (c) AI-powered ``Generate'' button for translating sketches to code edits; (d) ``Run'' button executes Python code and displays output in the console.}
    % \dv{This is absolutely fine, not need to change. ... but I was imagining a much smaller overview of interface with all elements visible (like half the size you have). Then all the interesting stuff is in zoomed in detail shots with callouts to where they come from. Right now (b) isa little unclear where it is in the interface, and the important stuff in (a) is still on the small side.  }
    \label{fig:first-interface}
    \Description[Coding interface with annotation and execution features.]{The figure presents a coding interface design with annotation and execution features. The main area displays Python code implementing a NearestNeighborRetriever class, featuring methods for Euclidean and Manhattan distance calculations and a nearest neighbor search. (a) Blue handwritten annotations, such as "def one-hot," are visible directly on the code. (b) A toolbar below the interface offers tools like select, pan, pen, eraser, and shape drawing, along with a color palette for annotations displayed as colorful dots to the right. (c) An "AI-powered Generate button" appears near the top, used for translating user sketches into code edits. (d) The "Run" button executes the code and shows the results in the console, displayed in the bottom right. The console output includes nearest neighbor results based on Manhattan distance. This setup demonstrates an interactive workflow where users annotate, generate, and execute code within an accessible interface.}
\end{figure*}

\section{Code Shaping}
The code shaping concept \emph{enables programmers to edit code using freeform sketches directly on or around the code}. This approach includes three core elements: a sketching canvas, a responsive code editor, and an AI that interprets sketches to generate code edits.

In a code shaping session, programmers sketch their intended modifications on an invisible canvas overlaid on the code. These sketches can include arrows pointing to specific lines, pseudocode defining a function's structure, and annotations indicating desired changes. The sketches can interact with any part of the code and output in the console and graphical view.
Once sketches are made, users can press a button to prompt AI to interpret their sketches along with the code. 
If the resulting code does not match the programmer's intent, they can refine their sketches, creating an iterative cycle of input and feedback.
% This feedback loop between the canvas, code editor, and AI interpretation is crucial to the concept of code shaping. 
This feedback loop allows programmers to use sketches progressively and iteratively to \textit{shape} how the code should be structured, how it should flow, and how it should function, guiding it towards the desired form and functionality.

In the following sections, we describe a series of three design studies (stages) to develop a proof-of-concept system and interface for the core code shaping interactions.
The first stage examined the types of annotations used in code shaping. The second stage focused on exploring model interpretation errors and the strategies programmers employed to address them. The final stage synthesized prior stages' insights, aiming to coordinate the interactions when editing code, iterating with AI, and sketching on the canvas. 
% This stage introduced predefined gestures for interacting with the code editor via the canvas without context switching and incorporated feedforward mechanisms to reduce the need for iterative communication with the models.

% Originally labelled ``Generate'' to signify code creation, the button was later renamed to clarify that a change will be applied to the code. If the output does not match the programmer's intent, they can refine their sketches, creating an iterative cycle of input and feedback. This feedback loop between the canvas, code editor, and AI interpretation forms the core interaction necessary for a system that supports the concept of code shaping.
% \jz{It would be nice to run these abstract interactions with an example.}



% to explore how sketches are created, how AI interpretations influence the process, and how to bridge the interaction paradigm of editing code and sketching.



% \begin{figure}[htb]
%     \centering
%     \begin{minipage}[b]{0.31\textwidth}
%         \centering
%         \includegraphics[width=\textwidth]{figures/interface/first_stage_interface.jpeg}
%         \caption{First stage interface}
%         \label{fig:first-stage}
%     \end{minipage}
%     \hfill
%     \begin{minipage}[b]{0.31\textwidth}
%         \centering
%         \includegraphics[width=\textwidth]{figures/interface/second_stage_interface.png}
%         \caption{Second stage interface}
%         \label{fig:second-stage}
%     \end{minipage}
%     \hfill
%     \begin{minipage}[b]{0.31\textwidth}
%         \centering
%         \includegraphics[width=\textwidth]{figures/interface/third_stage_interface.jpeg}
%         \caption{Third stage interface}
%         \label{fig:third-stage}
%     \end{minipage}
% \end{figure}



% Subsequently, we demonstrated the system's application in real-world scenarios by conducting code reviews with N=\x{} software developers.
% The system's effectiveness was validated through user studies involving N=\x{} participants. 
% The final study demonstrated the applicability of the code shaping concept in practical use cases with expert developers.

% This study aims to explore the use of digital ink sketches to specify code edits beyond syntactic changes, identify common patterns in programmers' sketches, and uncover challenges and strategies in this approach. 

\section{Stage One: Explore Sketches} 
We developed a basic user interface to explore how participants used sketches as actionable commands for code edits. We observed and categorized participants' challenges and sketch types, providing foundational insights for the code shaping system's development in subsequent stages.


\subsection{User Interface}
\begin{wrapfigure}{l}{15mm}
\vspace{-3mm} \includegraphics[]{figures/layers/stage1-icon.pdf}
\end{wrapfigure} 
For this stage, the user interface creates a straightforward way to make free-form sketches in the \textit{canvas layer} to directly generate code edits affecting the \textit{code layer} while keeping the \textit{AI layer} hidden to the user.
% \dv{prev sentence is explanation of conceptual layers and flow if UI which is represented by "icon" diagram at left. We should follow same pattern for sentence 1 in stage 2 UI and stage 3 UI}
% The first design integrates a code editor within a canvas environment to enable code shaping. 
The interface supports typical free-form sketching tools, including colour selectors, pens, erasers, and shapes (\autoref{fig:first-interface}a-b). A text tool is available for conventional editing. 
Two-finger panning and zooming navigate the code in the editor to enable sketching at different levels of granularity. A pointer tool can select strokes in the sketches. Pressing a ``Generate'' button uses all annotations on the canvas, or only selected annotations, as parameters for generating edited code (\autoref{fig:first-interface}c). 
The system recognizes free-form annotations on the code editor, utilizing GPT-4o to generate corresponding code. 
We render HTML content from both the code editor and sketches onto separate canvases and embed these canvas content into an SVG. This transformation process includes handling CORS and tainting issues, adding grids to locate annotations, and turning the code editor to grayscale to highlight the sketches.
The system then considers the annotations alongside previous
\rev{version history, including pictorial representations of sketches, code snapshots, conversational context, natural language inputs, and modified code outputs. The stored history was used to contextualize model responses and maintain a comprehensive context of the evolving codebase.}
% iterations of sketch editing and the relevant codebase as part of the input context for code generation.
After the code is generated, a difference algorithm is employed to 
only update the changed sections of the code~\cite{myers1986nd}.
The user can press a ``Run'' button to execute the code, with text or image results shown in the console panel underneath (\autoref{fig:first-interface}d). Programmers can annotate any output on the console or graphical windows as part of their sketches. \rev{The system incorporates these annotated outputs by transforming them into separate canvases, embedding them as SVGs alongside the code editor content, and encoding them for processing.}



\subsection{Participants, Tasks, and Procedure}
We recruited 6 programmers (1 left-handed), aged 23 to 28, with 4 identifying as women and 2 as men.
Participants were recruited through convenience sampling and received \$30 for completing the study. Based on a screening questionnaire, participants had 2-8 years of programming experience in Python and had used ChatGPT or Copilot 3-12 times per week.

We developed three Python coding scenarios, each comprising two tasks that required specific edits to achieve predefined goals. These scenarios spanned different programming paradigms: basic object-oriented programming, functional programming for machine learning, and declarative programming for data engineering. Each task provided participants with starter code requiring modifications in multiple areas. For instance, \f{scenario 2} involved extending a class to handle categorical features in data points, necessitating changes to existing methods for feature encoding and distance calculations. All tasks were pre-tested to ensure that GPT-4o could not immediately generate the correct code.

Participants were assigned 2 out of 3 scenarios that they were most familiar with, as determined by their screening questionnaire. Each scenario consisted of 2 tasks, and participants completed all 4 tasks (2 \f{scenarios} \by 2 \f{tasks} each) within a total of 60 minutes, spanning around 12 to 16 minutes per task. The order of scenarios and tasks was pre-assigned, meaning participants completed all tasks within one scenario before moving on to the next scenario. 
The study was conducted in person using an Apple iPad Air (5th generation, 10.9-inch display) as the primary research tool. An experimenter was present throughout each session to observe and document participant behaviours.
% Afterward, there was a post-study survey, including UMUX-LITE, NASA-TLX, and a 7-point Likert scale questionnaire \jz{questionnaire data is collected but never analyzed or reported. why do we need this then? better just remove this sentence} evaluating factors such as the clarity of mapping between sketches and edited code and the ease of iterating on sketches. 
Finally, a semi-structured interview gathered qualitative data on participants' general experience, challenges encountered, and suggestions for system improvement.

% \jz{Maybe I miss it. What is the device the participants used during the studies? An iPad, a touch monitor, and a wall-size display are all different and will affect the results. This comes up when I skimmed the limitations, as not exploring it on different form factors of the devices can be a limitation.}

% \begin{enumerate}
%     \item Task Management System (Basic Python)
%     \begin{itemize}
%         \item Sort Tasks by New Attribute "Due Date". Participants add a new attribute due\_date to the Task class and implement a method in the TaskManager class to list tasks sorted by due date.
%         \item Allow User to Update Task Details and Delete Tasks. Participants modify the Task class to include setter methods for updating task details and add a method to the TaskManager class to delete tasks.
%     \end{itemize}
%     \item Enhancing a Nearest Neighbor Retriever
%     \begin{itemize}
%         \item Add Support for Manhattan Distance. Participants implement the Manhattan distance metric and update the NearestNeighborRetriever class to use this new metric.
%         \item Add Support for Handling New Data Types. Participants extend the class to handle new types of data points, specifically those with categorical features, by implementing one-hot encoding and modifying the distance calculations.
%     \end{itemize}
%     \item Data Imputation and Feature Engineering
%     \begin{itemize}
%         \item Impute Missing Data with Average Feature Value \& Feature Engineering for Quadratic Terms. Participants add functionality to the DataProcessor class to impute missing values with the average value of the respective feature and create new features that are the squares of existing features.
%         \item Visualize Data Distribution and Implement Feature Scaling. Participants implement methods to visualize the data distribution and perform feature scaling using MinMaxScaler.
%     \end{itemize}
% \end{enumerate}




\subsection{Data Analysis}
We conducted an inductive thematic analysis of participant-generated sketches. This analysis incorporated observational notes, screen recordings, transcribed think-aloud data, and interview notes. 
Sketches were automatically captured in base64 format each time the generate button was activated, yielding 81 distinct screenshots. 
Of these, 7 were identified as duplicates and subsequently removed from the analysis.
We developed a codebook covering five dimensions: Content (text, code, annotation, freeform), Approach (step-by-step, one-time), CodeReference (parameters, targets), Purpose (functional, procedural), and Form (concrete, abstract). All captured sketches were verified and coded by researchers together. The results and descriptions of each coding category are presented in \autoref{tab:first_code}.

% \jz{here it seems only sketches are analyzed, however, there are participants' quotes and researchers' observational notes when reporting the results. You many want to briefly mention how these data sources were analyzed.}

% \jz{The discussion of the results below seems not to leverage table 1, if it is the main result. I appreciate the different themes obtained and presented below, but I have a hard time connecting the dots and understand how they contribute to your study goal. These results seem scattered. Is there a way to tie them in a narrative?}

\begin{figure}
    \centering
    \includegraphics[width=.57\linewidth]{figures/quadrant.pdf}
    \caption{The classification of sketched annotations from participants situated in a quadrant with two spectra, Abstract-Concrete and Procedural-Functional.}
    \Description[Quadrant diagram classifying sketched annotations.]{The figure presents a quadrant diagram categorizing sketched annotations based on two axes: Abstract-Concrete (vertical axis) and Procedural-Functional (horizontal axis).1) The upper-left quadrant (Abstract-Procedural) contains Python code snippets for listing tasks, annotated with handwritten notes such as "list tasks" and arrows connecting functions; 2) The upper-right quadrant (Abstract-Functional) features a hand-drawn graph labeled "visualize data" alongside code suggesting data visualization processes; 3) The lower-left quadrant (Concrete-Procedural) includes a handwritten note "def update_task" with an arrow pointing to a list structure, suggesting task update functionality; 4) The lower-right quadrant (Concrete-Functional) displays a code snippet for manipulating a dataframe column, accompanied by a handwritten comment describing the implementation of a feature scaling method. This diagram illustrates the interplay between abstract and concrete ideas as well as procedural and functional approaches in annotating and conceptualizing code.}
    \label{fig:classification}
\end{figure}

\begin{table*}[]
    \centering
    \small
\begin{tabular}[t]{l|lp{6.1cm}rp{6.2cm}}
\toprule
\textbf{Category} & \textbf{Subcategory} & \textbf{Description} & $N$ & \textbf{Example} \\
\midrule
\multirow[t]{4}{*}{Content} 
    & Text        & Written text or natural language instruction & 11 & ``impute missing value [...]'' [P3] \\
    & Code        & Written pseudo code or code syntax & 23 & def update\_task [P5] \\
    & Annotation  & Symbols and annotations & 31 & circle, arrow, underline \\
    & Freeform    & Sketches or drawings without clear structure & 9  & line chart [P1] \\
\midrule
\multirow[t]{2}{*}{Approach} 
    & One-Time    & Marked all possible changes before generating code edits & 25 & ``add due\_date'' as an attribute and pointing arrow to the written def sort function [P4] \\
    & Step-by-Step & Decomposing tasks and generating code edits after each subtask & 49 & Copied ``list\_task'' function, generated, then edited it to be list tasks by ``due\_date'' [P2] \\
\midrule
\multirow[t]{2}{*}{CodeRef} 
    & Parameter   & Referencing code as a parameter to contextualize generation & 12 & Circled data and pointed to plot [P3] \\
    & Target      & Reference to code as the target to be modified & 28 & Crossed out sampled data with ``(int, int)'' to ``(int, str)'' [P5] \\
\midrule
\multirow[t]{2}{*}{Purpose} 
    & Functional  & Sketches express how the code should function & 11 & Sketching the sample processed out\-put [P6] \\
    & Procedural  & Sketches express how the code should process or run & 63 & Sketching the flow from variables, to one-hot encoding, to distance metrics [P5] \\
\midrule
\multirow[t]{2}{*}{Form} 
    & Concrete    & Sketches with a concrete or syntactic form & 31 & text, pseudo code \\
    & Abstract    & Sketches representing the abstract attributes (semantic) meaning & 43 & annotations, freeform sketches \\
\bottomrule
\end{tabular}
    \caption{Types of sketches used in stage one were categorized, and the number being coded was indicated by $N$.}
    \label{tab:first_code}
\end{table*}



% \begin{figure}[th]
%     \centering
%     \begin{minipage}[t]{0.34\textwidth}
%     \end{minipage}%
%     \hfill
%     \begin{minipage}[t]{0.64\textwidth}
%         \centering
%         \includegraphics[width=\linewidth]{figures/frequency_heatmap_logtype.pdf}
%         \caption{\f{stage1} The frequency of log types aggregated per minute throughout the study time per participant per task.}
%         \dv{ I find this graph hard to interpret, it's overly dense with information you don't focus on, and it takes up space. I think say everything about this quite clearly in text at start of 4.4. Suggest removing this figure }
%         \dv{If you want to keep it, then: If you normalize the time, it's no longer in "minutes", right? This is just a "normalized timeline".}
%         \Description[Heat map showing frequency of log types over time.]{The image displays a heat map representing the frequency of different log types over time during a study. The y-axis shows four log types: Run Result, Run Error, Sketch & Generate, and Code Change. The x-axis represents normalized time in minutes from Start to End (1-16). The frequency is indicated by color intensity, with darker blues representing higher frequencies. The heat map shows varying patterns of activity across different log types and time periods, with Code Change appearing to have the highest overall frequency.}
%         \label{fig:freq-log}
%     \end{minipage}
% \end{figure}


\subsection{Results}
\label{sec:result}
% overall pattern
% alignment with AI
    %  moviing together with AI
% control
% sketches classification
% sketches as a `tool'
    % the reusability of sketches

% There are 6 \f{participants} \by 2 \f{tasks} \by 2 \f{subtasks} = 24 data entries collected. 
% Overall, 22 out of 24 tasks (6 participants $\times$ 2 tasks $\times$ 2 subtasks) were completed. 
All but two participants completed the four assigned tasks; these two participants did not complete \f{scenario2}-\f{task2} within the assigned time.
% \jz{add a sentence saying that task completion is not our concern}
There was a concern that experienced programmers might be strongly biased toward typing code edits, which could impact their experience with sketch-based code editing. However, all participants appreciated the concept and expressed a willingness to integrate it into their current programming workflow, as it allowed them to \pquote{think deeper about the code}{P1} and \pquote{focus on higher-level planning}{P6}.
Participants used sketching to edit code an average of 3 times per subtask ($SD=4.0$). \rev{Each instance of sketching often included multiple annotations, with some sketches encompassing edits to several parts of the code.}
Early-stage code edits were primarily made through sketched annotations, but in the later stages (12-13 minutes), edits occurred without sketches, suggesting the use of a keyboard or undo/redo mechanisms to refine code. P2 and P4 explained that they resorted to the native tablet keyboard for edits to handle low-level details, as the waiting time for model interpretation \rev{could exceed five seconds (in average around 4-8 seconds based on the size of the codebase) in some cases, making manual code changes faster}, thus \pquote{would rather do it myself [themselves]}{P4}.

\subsubsection{General Workflow} 
Participants sometimes wrote higher-level instructions first when unsure about the solution but had a rough idea of where the code edits should happen and what the \pquote{shape of the code looks like}{P4}. 
After evaluating the edited code, they then added annotations for lower-level code editing based on their approaches in mind.
We also observed two participants gradually develop a personalized workflow for editing code with sketches. P2 found that breaking down tasks into very low-level details was ineffective and not necessary, while P5 emphasized the need for smaller task pieces for better system understanding. \rev{This difference arose because P2 included precise code-like keywords in their sketches, minimizing the need for further detail.}



\begin{figure*}[th]
    \centering
    \includegraphics[width=\linewidth]{figures/arrow_variants.pdf}
    \caption{Sketches from stage one showing how participants employ arrows ($\rightarrow$) for different purposes, including command (the intended action of operation), parameter (supplementing the command), and target (the area where the edit should occur); (a) indicating procedural flow between commands; (b) referring to data attributes; (c) modifying a function, with the function as the parameter to supplement the command; (d) applying changes to a target area.}
    \Description[Data collected from the user study showing how participants employ arrows for different purposes]{The figure depicts sketches from a user study, showcasing how participants use arrows to indicate different operations in a code editor. The arrows are employed for three distinct purposes: (1) command, representing the intended action or operation; (2) parameter, supplementing the command; and (3) target, pointing to the area where the edit should occur. The figure is divided into four labeled sections: (a) Procedural Flow: Displays arrows connecting commands, such as "due_date" and "sort_by," to illustrate the procedural flow of tasks; (b) Refer: Shows arrows linking parameters like "attributes in Task" to commands, highlighting how participants refer to specific data attributes; (c) Modify: Demonstrates arrows pointing to a function like add_features, with handwritten annotations such as "modify this" and "I don’t want to square the log," reflecting user-directed modifications; (d) Apply: Includes arrows connecting a command (proprocess) to its target, emphasizing changes being applied to a specific section of the code. This figure captures the interactive coding process, with annotations providing insights into the participants' thought processes and actions during the study.}
    \label{fig:arrow-variants}
\end{figure*}



\subsubsection{Types of Sketches}
Overall, the sketches can be situated in a quadrant with two spectrums (\autoref{fig:classification}): \f{Abstract}-\f{Concrete} and \f{Procedural}-\f{Functional}.
The \f{Abstract}-\f{Concrete} spectrum describes whether the annotations are abstract symbols or graphs versus concrete written text. The \f{Procedural}-\f{Functional} spectrum classifies the target of the annotations, ranging from procedural steps describing how the program should be structured to functional descriptions specifying how the program should work. Participants often combined these aspects, drawing graphs and adding arrows to refer to certain data attributes, specifying both functional and procedural terms.

\subsubsection{Sketch as a Tool} 
Additionally, we observed that participants considered sketches as functional ``tools'' that could be reused~\cite{renom2022exploring}, not just as transient digital ink drawings. All participants expressed that they could use different sketches to achieve the same effect, choosing which sketch to use based on the environment, such as available white space. They also reused their sketches to convey the same effect; for instance, an arrow used to insert a function into a specific line of code was reused by P3 to add another function.


% We plan to iterate on the design with gained insights to increase programmers control over the iterative process of transforming idea to annotation to code generation.



\subsubsection{Ambiguity of Sketches and Model's Transparency}
The primary challenge was the ambiguity of participants' sketches. For example, arrows were used inconsistently, sometimes indicating context [P1] and other times denoting targets of changes [P4] (\autoref{fig:arrow-variants}a-d).
The interpretation of these sketches often relied heavily on surrounding code, leading to occasional misrecognition and misinterpretation. 
This necessitated an iterative refinement process.
However, this iteration became a significant source of frustration for participants, largely due to the system's lack of transparency.
% Also, most participants (5/6) developed a more detailed plan for code edits after the initial round of sketching, suggesting that sketches can be refined iteratively based on the generated code edits.
Participants rated the clarity of the effect of their sketches on the generated code poorly ($Mdn=3.5$, $SD=1.83$), as well as the ease of iterating on sketches ($Mdn=4$, $SD=2.34$), \rev{on a seven-point scale questionnaire.}
Participants struggled with not knowing \pquote{where the code was being edited}{P4}, an unclear mapping between sketches and the edited code, and why the model misinterpreted their sketches.
This is considered as interpretation error than recognition error.




\subsection{Summary}
% In the first iteration, a code editor and canvas were combined, allowing programmers to freely express their intentions for planning code edits. 
The results revealed that programmers utilized diverse sketching techniques, necessitating an iterative refinement process due to the inherent ambiguity of these sketches.
However, the current iteration process was hindered by the AI model's lack of transparency, particularly in how it interpreted sketches and applied code changes.
To address this issue, we focused on identifying potential misinterpretations of sketches by the AI model and exploring how programmers could recover from these errors in the next stage.
% we aim to facilitate the iteration process and minimize the need for programmers to focus on low-level code edits or engage in prompt engineering with AI by concentrating on interactions within the canvas layer.
% However, our goal is not for programmers to view code shaping merely as a way to prompt an AI code generation tool, but as a means to express their thought processes regarding code edits. 



\section{Stage Two: Model Interpretability}
The second stage of our study focused on enhancing user control over the model interpretation of sketches by providing different types of brushes for sketching and adding feedback to convey the model's interpretation of the sketches. 
% This aimed to improve user understanding and control during the sketching process.

\subsection{User Interface}
\begin{wrapfigure}{l}{15mm}
\vspace{-3mm} \includegraphics[]{figures/layers/stage2-icon.pdf}
\end{wrapfigure} 
The system was enhanced with three major features to facilitate the above focus of our second-stage study. 
First, \textit{command brushes} were introduced to allow programmers to convey their intentions more precisely. For example, a ``replace'' brush can instruct the model to limit its interpretation to replacing existing code with the users' sketches (\autoref{fig:second-interface}a).
Second, the underlying model interpretation mechanism was modified to recognize, group, and interpret sketches. The system groups semantically related sketch marks each time the pen is lifted and provides reasoning for the actions it interprets for each group (\autoref{fig:second-interface}b). These descriptions are displayed as tooltips next to each sketch group, allowing users to edit the descriptions to refine the interpretation or commit to executing the actions.
Third, an inline diff view was added to the code editor, enabling users to visualize code changes as staged edits and choose to accept or reject these changes (\autoref{fig:second-interface}c).




\begin{figure*}
    \centering
    \includegraphics[width=\linewidth]{figures/interface/2nd.pdf}
    \caption{The interface design from the second stage. (a) Command brushes used to steer AI model's interpretation, including reference, delete, add and replace; (b) interpretation of sketches displayed as tooltips, programmers can click on the preview (recognized sketches) and see the full description of AI's reasoning of actions; (c) programmers can click on the commit button to execute the actions, edited code will be shown in diff view and programmers can accept/reject it.}
    % \dv{this one is kind of hard to interpret since the "overview" is so large. Ideally all three of these UI figures should follow exactly the same pattern. Smallish full screen cap always same size across three figures, then zoomed in callouts around it. if you need space above it and the figure becomes tall, that's ok}
    % \dv{all your a, b, c, and your annotation text should really be equivalent to about Arial 8pt at most and 6pt at min.  The a, b, c in this figure are really big, I bet about Arial 12pt. This is likely because you use Keynote, so you have no idea about the physical size of your figure in mm ... it's just relative size guesswork.}
    \Description[Stage 2 interface with a code editor, AI-assisted features and sketch interpretation.]{The figure showcases the interface design from stage two of the study, illustrating an AI-assisted Python code editor for the NearestNeighborRetriever class. The interface highlights several interactive features: (a) Command Brushes: Located at the top right, labeled "Reference," "Add," "Delete," and "Replace," these tools allow users to guide the AI’s interpretation and actions within the code. (b) Tooltip Display: A tooltip is shown, presenting the AI’s interpretation of a sketched annotation, including an option to view the full reasoning behind the AI's suggested changes. For example, a tooltip highlights a change suggesting converting a variable to a string type. (c) Commit Button and Diff View: Programmers can click the commit button to execute the AI's suggestions. The modified code is displayed in a diff view with options to accept or reject individual changes. The interface also includes line numbers, syntax highlighting, and handwritten annotations such as "str" near the code. A task description at the top prompts users to "Implement the Manhattan Distance Metric," guiding the focus of the current programming task. This design facilitates a collaborative and iterative code development process between the user and the AI system.}
    \label{fig:second-interface}
\end{figure*}


\subsection{Participants, Tasks, and Procedure}
Six new participants were recruited through convenience sampling, with all right-handed, 2 identified as women and 4 as men. All participants had 3-6 years of programming experience in Python and had used ChatGPT or Copilot 6-14 times per week. The same scenarios and tasks were used with the same procedure and data collection. 
We applied inductive thematic analysis to the observational notes, screen recordings, system logs, captured sketches, and interview notes to identify common types of model interpretation errors, the strategies participants used to recover from these errors, and insights related to the model’s interpretation.



\subsection{Results}
Four participants did not complete one of the four assigned tasks from either \f{scenario 2} or \f{scenario 3}. This incompletion was acceptable, as our primary objective was to understand how participants recovered from interpretation errors, rather than task completion itself. We identified a total of 66 error and recovery scenarios, categorized into six distinct error types (\autoref{tab:error-strat}). Participants employed six different repair strategies following three major actions: rejecting/accepting code edits, or taking no action.

\subsubsection{Feedback on Model Interpretation}
Most participants (5/6) appreciated having an interpretation as a preliminary step before code edits. They noted it is necessary when code changes went wrong (5/6), when they were unsure about how the code should be implemented (4/6), and when tasks required decomposition (2/6). However, most of the time, they could rely on the code diff view as it indicates the model's interpretation, especially if their sketches included pseudocode. They expressed that the interpretation feedback should include the recognized items and text within the annotation (6/6), the model's recognition of non-textual annotations and sketches (5/6), suggestions for code edits (3/6), and the linkage between sketches and code edits (2/6).



\subsubsection{Common Errors and Strategies to Repair}
Of all sketches, $23.2\%$ required iterations due to model interpretation errors.
Common errors (\autoref{tab:error-strat}) included mismatches between code implementation and user expectations, incorrect interpretation before code edits, wrong recognition of sketches, no code edits being made, incorrect scope of code changes, and wrong modified code syntax causing runtime errors.
The most frequent error was code mismatches, which were detected after code edits, P10 questioned whether the error happened \pquote{because my drawing was not clear enough or my [written pseudocode] was not recognized.}
The second and third most common were incorrect interpretations of user actions and recognitions, which participants could identify before any code changes occurred. In these cases, some participants (4/6) chose to refine their sketches before generating the code edits, while two participants occasionally still pressed the generate button, P11 explaining this due to \pquote{not knowing how should I refine the sketches.}


In most cases, participants attempted to repair errors by redrawing sketches. In two instances, they edited the code directly using the tablet's keyboard, in three cases they adjusted the interpretation, and in four cases they used control brushes. However, participants only used the control brush when redrawn sketches were still not recognized. P9 explained, \pquote{I would think that it's because of the recognition error or code referencing error, then realize it's misinterpreting what I want to do.}
Overall, these repair strategies can be categorized into three types: selection, instruction, and target. These included adding textual instructions, adding annotations, removing unnecessary sketches, rewriting pseudocode, adding code syntax, and adding references pointing to other target code (\autoref{tab:error-strat}).
For instance, P9 changed the written text from \pquote{handle} to \pquote{def} to specify that the handling should be implemented in a new function. 
Some participants redrew sketches to prevent new annotations from obscuring the code when no sketches were detected, suspecting that \pquote{the sketches blended into the code}{P11}. \rev{This concern is valid, as the system overlays the sketch layer on top of the code layer in the pictorial form to associate sketches with specific lines of code during interpretation.}
All participants used strategies such as adding code targets and references to \pquote{make sure correct code is being used}{P7} or \pquote{only changing specific area [of code]}{P8}.
For instance, P8 circled the DataProcessor class to ensure that new code edits were implemented as methods within the class rather than as standalone functions outside it.



\begin{table*}[hbt]
    \caption{Results from stage two showing AI interpretation error types and corresponding participant repair strategies.}
    \centering
    \small
\begin{tabular}{
        >{\raggedright\arraybackslash}m{1.5cm}
        >{\raggedright\arraybackslash}m{3.2cm}
        >{\raggedright\arraybackslash}m{.5cm}
        >{\raggedright\arraybackslash}m{1.2cm}
        >{\raggedright\arraybackslash}m{1.5cm}
        >{\raggedright\arraybackslash}m{2.9cm}
        >{\raggedright\arraybackslash}m{.5cm}
        >{\raggedright\arraybackslash}m{3cm}
    }
    \toprule
    \multicolumn{4}{l}{\textbf{ERRORS}} & \multicolumn{4}{l}{\textbf{REPAIR STRATEGIES}} \\ 
    \midrule
    \textbf{Type} & \textbf{Description} & \textbf{$N$} & \textbf{Action} & \textbf{Type} & \textbf{Description} & \textbf{$N$} & \textbf{Example} \\ 
    \midrule
    Code Mismatch & code does not match intended implementation & 25 & Reject & Add Code Target & specify target of code changes & 15 & \includegraphics[height=45pt]{figures/res/add_target.png} \\ 
    Code Error & contains syntax or logical error & 3 & & Add Code Reference & add context for generation & 11 & \includegraphics[height=45pt]{figures/res/add_reference.png} \\ 
    No Changes & no code changes being made & 7 & & Precise Annotation & rewrite text or redraw annotations & 11 & \includegraphics[height=45pt]{figures/res/precise.png} \\ 
    Wrong Action Interpretation & interpret the action wrong before generate code edits & 14 & & Add Pseudo Code & add code syntax or symbols & 19 & \includegraphics[height=45pt]{figures/res/keyword.png} \\ 
    \cline{1-4}\cline{5-8}
    Wrong Scope of Change & incorrect code range edited & 4 & Accept & Next Round & accept code edits then annotate again & 4 & \\ 
    \cline{1-4}\cline{5-8}
    Wrong Recognition & recognize the sketches wrong & 13 & No Action & Regenerate & generate code again without modifying sketch & 6 &  \\ 
    \bottomrule
\end{tabular}
    \label{tab:error-strat}
\end{table*}




\subsubsection{Control Over Model Interpretation}
Most participants (5/6) did not find control brushes particularly useful. Two participants preferred interacting directly with the code editor rather than using specific brushes to constrain the AI model. They favoured simple arrows and cross-outs to indicate code replacements instead of different brushes. All participants found that sketches alone were expressive enough to guide the AI model in correcting its interpretation.
For example, P10 added a numbered label to the pseudocode, \pquote{$\rightarrow$ str:}, to indicate that the AI should prioritize interpreting that annotation first \includegraphics[height=14pt]{figures/small_examples/p10_1_correct_model.jpg}.
We observed that three participants tended to wait for the interpretation to complete before generating code edits to \pquote{not lose control over my [their] own code}{P9}.




\subsubsection{Sketching, Correcting Model, and Editing Code}
All participants primarily attributed their frustration with iteration to the need for context \pquote{switching between the code editor and the canvas}{P9}. The interface required a double-tap to enter or exit the code editor for tasks such as accepting/rejecting code edits, undoing/redoing actions, and performing manual edits (though these were less common). Due to the dynamic nature of programming, where each edit builds on previous modifications, the frequent need for interpretation and the requirement to accept or reject code edits often disrupt participants' flow. Consequently, most participants (5/6) preferred to complete all sketches first and then use the ``generate'' button as a clear boundary between debugging and sketching modes, avoiding repetitive context switching.

This context switching also involved changing mental models and using different input modalities, leading to errors. For example, some participants (3/6) frequently selected the wrong tools due to overlapping semantic meanings, such as P11 using the eraser to delete code or the pointer to select code. These findings highlight the importance of enabling interactions with the code editor ``through'' the canvas layer, effectively translating certain canvas layer interactions into actions within the code editor layer.

% common interaction on code editor: accept/reject changes, select, delete

\subsection{Summary}
% The second iteration examined how participants responded to the AI's interpretations and errors through the feedback provided and the use of control brushes. 
While feedback on AI interpretations added value, the method used in this stage disrupted the programmers' flow. The goal of code shaping is to allow programmers to edit code structure through sketches, rather than engage in low-level code editing or prompt engineering for the AI system. The control brushes did not perform as expected; participants preferred refining their sketches by adding more code references or employing code syntax to shape the outcome. This tendency can be linked to the cognitive dimension of \emph{premature commitment}---forcing programmers to make decisions too early~\cite{green1996usability}, which conflicts with the iterative nature of code shaping. The findings underscore that the key to facilitating code shaping is an interaction design that minimizes the conceptual layers between the code editor and the sketching canvas.






\section{Stage Three: Towards Reconciling Sketches, AI, and Code}
To bridge the conceptual gap between interacting with the code editor and the canvas, the user interface was modified in two key ways. Insights from the previous stage suggested that unnecessary code changes could be minimized by confirming interpretations before generating edits. Building on the types of interpretations identified, we introduced an always-on feedforward mechanism through subtle visual cues. This approach allows programmers to iterate more quickly without delving into code details.
Additionally, to reduce the cognitive load of switching between layers, we developed unique gestures that enable users to interact directly with the code editor through the canvas.


\subsection{User Interface}
\begin{wrapfigure}{L}{15mm}
\vspace{-3mm} \includegraphics[]{figures/layers/stage3-icon.pdf}
\end{wrapfigure} 
Unnecessary GUI elements were removed to keep the interaction focused on sketches. The button for sending sketches and code to the model was renamed ``Commit'' to make it clear that a change will be applied to the code. Only this button and the ``Run'' button were retained in the GUI, as participants preferred having explicit controls for these actions rather than relying on implicit gestures. We open sourced the code for this stage at \url{https://github.com/CodeShaping/code-shaping}, including all the prompts used.


\begin{figure*}[t]
    \centering
    \includegraphics[width=\linewidth]{figures/interface/3rd.pdf}
    \caption{The interface design from the third stage. (a) programmers can use one finger tap and drag to select items on canvas; (b) tap longer and drag will select code with contextual menu beside; (c) the always-on feedforward interpretation showing the interpretation of sketches or text, the reasoning of action, and the related code; (d) the gutter will be decorated to indicate which code being referenced and which code will be affected; (e) the related code syntax will be highlight transiently; (f) programmers can commit the changes and (g) draw check/cross to accept/reject code edits.}
    \Description[Interface design for stage three of AI-assisted code editing.]{The figure illustrates the interface design from stage three of the study, showcasing AI-assisted code editing for a Python data processing task. Key interactive features are labeled (a) through (f): (a) Finger Gesture for Selection: Users can tap and drag with a finger to select items on the canvas, enabling precise interaction with the code; (b) Contextual Menu: When code is selected via a long tap and drag, a contextual menu appears, providing options such as "Copy," "Paste," "Delete," and more; (c) Always-On Feedforward Panel: An interpretation panel continuously displays the AI's reasoning, including sketches or text annotations and the corresponding proposed actions; (d) Decorated Gutter: The code gutter is highlighted to indicate which sections of code are referenced or affected by the proposed changes; (e) Transient Syntax Highlighting: Relevant code is temporarily highlighted to visually connect AI suggestions with specific parts of the code; (f) Commit Button: A commit button allows users to finalize edits. Users can accept or reject changes using checkmark or cross gestures directly on the interface; (g) The interface displays Python code for a DataProcessor class, including a sample dataset and a preprocess method that fills missing values in the dataset. Handwritten annotations, such as the word "missing," and gestures, like arrows connecting actions, are visible throughout, illustrating the interactive coding process.}
    \label{fig:third-interface}
\end{figure*}


\subsubsection{Ink and Gestures}
Based on insights from the previous two stages, we classified common interactions during code shaping into five key categories: navigating the canvas and code, undoing and redoing actions such as code edits or sketches, selecting code or sketches on the canvas, accepting and rejecting code changes, and creating free-form sketches (\autoref{tab:gestures}).
Multi-touch gestures were assigned to system-level interactions, such as panning for navigation and two- or three-finger double-tapping for undo and redo actions. Selecting items on the canvas or within the code editor was differentiated by the duration of a single touch: a single tap for canvas items selection (\autoref{fig:third-interface}a) or a long press followed by dragging for code selection (\autoref{fig:third-interface}b). Contextual action buttons, such as delete and copy, are displayed next to the selected code or within the selection box of canvas items, allowing for quick access to common actions.
We implemented unique stroke gestures for accepting or rejecting code edits using the \$1 unistroke recognizer~\cite{10.1145/1294211.1294238} to detect check (\faCheck) and cross (\faTimes) marks (\autoref{fig:third-interface}f). The Google Cloud Vision API was employed for robust recognition of handwritten text, enhancing the system's ability to interpret written pseudo code or textual instructions.


\subsubsection{Always-On Feedforward Interpretation}
\label{sec:feedforward}
Building on insights from the second iteration, we focused on providing only three essential types of interpretations that users truly needed: (1) recognizing how the model interpreted written text, code, and annotations (\autoref{fig:third-interface}c); (2) describing the code editing action inferred by the model; and (3) indicating the code context by highlighting relevant parameter code, displaying blue vertical line glyph decorations, and marking potentially affected code areas with a $\rightarrow$ icon beside the line number on the glyph (\autoref{fig:third-interface}d).
To identify the relevant code, we traversed the abstract syntax tree (AST) to dynamically highlight code syntax related to the user's input. The interpretation process is triggered 500 milliseconds after the user stops sketching, ensuring timely feedback while minimizing disruptions (\autoref{fig:third-interface}e).
The average latency between the input request and the complete output measured from the second study is approximately $2.87$ seconds (\(SD = 1.45\)). However, since interpretations are generated in real-time as users are sketching, it is possible for the system to produce correct results even before all annotations are fully completed.
We implemented a cascade interpretation approach, sequentially processing pen or touch input, predefined gestures, text and shape recognition, code edit action reasoning, and affected code analysis. This approach enables programmers to adjust their sketches concurrent with system evaluation, rather than waiting for the final step. 
These feedforward interpretations were not directly displayed on the canvas or situated within the code but were instead ambiently presented, updating on the fly to offer guidance when needed.
To enhance usability, especially for right-handed users, we repositioned the interpretation text from the upper right to the lower right of the screen. This adjustment allows users to occlude the interpretation with their hands while sketching, minimizing interference with their workflow.






\subsection{Procedure and Data Analysis}
We recruited six new participants for this study, including five right-handed individuals, four of whom identified as women and two as men. They had between 2-8 years of programming experience in Python and used ChatGPT or Copilot 4-12 times per week. We reused the same setup and tasks to ensure consistency in our results across studies. We added several system logs for gesture recognition and recorded input images, which served as parameters for the always-on feedforward interpretation. We collected $187$ sketches, $48$ of which were recorded when participants hit the ``Commit'' button.
We employed inductive thematic analysis to examine all collected data, including sketches, system logs, video recordings, observational notes, and transcribed interview audio recordings. The iterations were defined by the accomplishment of subtasks that participants themselves decided upon and decomposed from the main study task's goal. We then categorized the common flow of actions within these iterations.



% table
\begin{table*}[bt]
    \caption{The assigned touch and pen gestures to the third stage, enabling interaction across both code editor and canvas layers.}
    % \small
% \begin{tabular}{>{\raggedright\arraybackslash}p{1cm} >{\raggedright\arraybackslash}p{1.8cm} >{\raggedright\arraybackslash}p{4cm} >{\raggedright\arraybackslash}p{1.8cm} >{\centering\arraybackslash}m{2.5cm}}
% \toprule
% \textbf{Layer} & \textbf{Interaction} & \textbf{Description}             & \textbf{Tool} & \textbf{Gesture} \\ \midrule
% \textbf{Code Editor}    & \st{delete}               & delete (selected) code           & Gesture       & \includegraphics[height=45pt]{figures/delete.png} \\
%                & select               & long press select code               & Finger       & \includegraphics[height=45pt]{figures/select.png} \\
%                & accept/reject        & to changes in the code diff view & Pen       & \includegraphics[height=45pt]{figures/accept_reject.png} \\
%                & undo/redo            & two/three taps             & Finger      & \includegraphics[height=45pt]{figures/undo_redo.png} \\ \midrule
% \textbf{Canvas}         & draw                 & sketching annotations            & Pen           &                  \\
%                & erase                & erase strokes                    & Eraser        &                  \\
%                & select               & press and select elements in the canvas    & Finger       &                  \\
%                & undo/redo            & two/three taps to undo/redo strokes               & Finger      &                  \\
%                & delete               & remove the selected strokes      & (x)   &                  \\ \bottomrule
% \end{tabular}

% % finger
% % two -> navigation
% % logn press -> select (context menu, delete, copy paste..)
% % one selection
% % two fingers for undo, three fingers for redo

% % pen
% % accept/reject


\small
\begin{tabular}{>{\centering\arraybackslash}m{2.15cm} >{\centering\arraybackslash}m{2.15cm} >{\centering\arraybackslash}m{2.15cm} >{\centering\arraybackslash}m{2.15cm} >{\centering\arraybackslash}m{2.15cm}| >{\centering\arraybackslash}m{2.15cm} >{\centering\arraybackslash}m{2.15cm}}
\toprule
\multicolumn{5}{c}{\textbf{Touch}} & \multicolumn{2}{c}{\textbf{Pen}} \\ \midrule
Navigation & Undo command & Redo command & Select code & Select canvas objects & Accept / Reject code edits & Free-form sketching \\ 
\midrule
Two fingers pan & Two-finger double tap & Three-finger double tap & One finger long press then drag & One finger drag & Check (\faCheck) / Cross (\faTimes) & drawing \\ 
\midrule
\includegraphics[height=47pt]{figures/gestures/panning.png} & 
\includegraphics[height=37pt]{figures/gestures/undo.pdf} & 
\includegraphics[height=35pt]{figures/gestures/redo.pdf} & 
\includegraphics[height=45pt]{figures/gestures/select_code.png} & 
\includegraphics[height=45pt]{figures/gestures/select_canvas.png} &   
\includegraphics[height=42pt]{figures/gestures/check_cross.png} & 
\includegraphics[height=42pt]{figures/gestures/sketch.pdf} \\ 
\bottomrule
\end{tabular}
    \label{tab:gestures}
\end{table*}



\subsection{Results}
The analysis revealed several themes that shaped participant experiences.

\subsubsection{Flow of Actions}
We identified seven common action flows among participants, highlighting patterns in how they navigated between sketching, code editing, and reviewing interpretations (\autoref{tab:flow}). These flows generally followed a sequence that can be counted as a full iteration:
\[
\begin{array}{c}
\text{Sketch} \rightarrow (\text{Interpret}) \rightarrow \text{Generate} \rightarrow (\text{Run Code}) \\
\rightarrow \text{Accept/Reject}
\rightarrow \text{Re-Sketch/Undo/Redo}
\end{array}
\]
% \begin{equation}
% \text{Sketch} \rightarrow \left(\text{Interpret}\right) \rightarrow \text{Generate} \rightarrow \left(\text{Compile}\right) \rightarrow \text{Accept/Reject} \rightarrow \text{Re-Sketch/Undo/Redo}
% \end{equation}

Some of these flows were also observed in the previous stages but were more pronounced in this stage. Participants appeared more aware of their workflows, especially during interviews when recalling their processes. This contrasts with earlier iterations, where participants often expressed uncertainty, such as \pquote{hopefully the code edits are correct}{P8}. They chose different methods to iterate when code edits were incorrect, adapting their actions based on the situation. For example, P14 mentioned using the undo/redo function (ID 4 in \autoref{tab:flow}) for interpretation errors, while opting for the re-sketch approach (ID 3 in \autoref{tab:flow}) in other cases.

\subsubsection{Always-On Feedforward Interpretation}
After viewing the interpretation, participants pressed the ``Commit'' button $32.4\%$ more frequently in the third stage compared to that in the second. P15 explained, \pquote{the interpretation shows what code is possibly being affected is useful to make sure it will not mess up my code}. This underscores the value of glyph decorations in the always-on interpretation, as they may help participants confirm their intended code changes. Three participants echoed sentiments from the second iteration, appreciating the control provided by the interpretation. P16 noted, \pquote{I feel more confident that it’s on the right track [...] I don’t want it to be a black box}. 

Four participants stressed the importance of waiting for the correct interpretation before committing to changes, even if it required slightly more time compared to directly pressing the commit button. P13 explained, \pquote{I would rather wait a bit longer than evaluate the wrong generated code edits}, reflecting a preference for clarity over speed. 
The feedforward interpretation also guided participants on their next steps, regardless of whether the interpretations were correct. For instance, P15 noted that a previously correct target code section became incorrect after adding an arrow for code reference, indicating a misinterpretation of the arrow.

\subsubsection{Sketching or Editing Code}
A key goal of this design was to reconcile the conceptual layers between the code editor and the canvas.
While all participants did not report difficulty switching between contexts, they perceived them as distinct. As P14 observed, \pquote{I still think of the code and annotations separately in my mind}. However, participants found the unique gestures and strokes for interacting with the code editor \pquote{straightforward}{P17} and felt that it \pquote{makes me [them] feel like the sketch is affecting the code}{P18}.

A notable improvement was that most participants (5/6) expressed no need to use the keyboard, even for simple edits like deleting a line of code. When asked why they preferred using a strikethrough to indicate deletion instead of using the backspace key, P14 explained, \pquote{I just want to use sketches and annotations as the only way to change my code}. This suggests a sense of embodied interaction, something reinforced by P12, \pquote{It’s like my hands are directly editing the code based on how I want the code to be.} However, the predefined \faTimes ~gesture for rejecting code edits was triggered by accident once when P17 wrote \pquote{x} as part of the sketch.
% some confusion arose from the boundaries between sketched gestures and standard annotations; for example, one participant misinterpreted the `x' gesture, meant to reject a code edit, as a cross annotation, leading to unintended actions.

\subsubsection{Conceptual Shift in Code Editing}
\label{sec:conceptual_shift}
Participants demonstrated a shift from linear to spatial thinking in their code editing processes. As P16 observed, \pquote{I'm no longer just writing line by line [...] I'm arranging my thoughts spatially}. This reflects a move away from a traditional, syntax-focused approach to one that emphasizes the overall structure and flow of the code. Another participant, P14, highlighted this shift, stating, \pquote{it’s more about the higher-level structure and flow of the code as a whole}.

\subsubsection{Freedom and Flexibility}
The iterative process enabled by free-form sketching provided participants with a sense of creative freedom. P15 reflected, \pquote{This lets me play around with ideas in a way that’s more fluid and creative [...] I'm experimenting}. All participants mentioned that they would often resketch the code edit even when the generated edits were correct, as they discovered better ways to tackle the task. P14 noted that the canvas and undo/redo mechanisms allowed them to \pquote{draw whatever we [they] want and see how the code changes}. 

However, while participants valued the freedom of sketching, they also tended to ``compromise'' based on the AI’s interpretation capabilities. For instance, P18 consistently used squares instead of circles because \pquote{the rectangle works better} and did not obscure the code. As a result, participants developed preferences for specific annotations with the system along the time. P14, for example, switched to using crosses after observing that strikethroughs occasionally applied to the wrong lines of code. Over time, these preferred annotations became interchangeable in practice, as participants felt they \pquote{expressed the same meaning}.


\subsection{Summary}
The third stage highlights the effectiveness of defined gestures and always-on feedforward interpretation in reducing the transmission gap between the conceptual layers of the canvas, code, and AI model. This design iteration demonstrated how to display the model's interpretation, and designed interactions that minimize the need for layer switching.
While minor challenges remain, such as the rare misinterpretation of sketched gestures and some persistent between AI interpretation and actual applied code edits, these issues can likely be addressed by advancements in AI models.


\section{Example Use Case Scenarios}
% considering the ink as dynamic that connect with the code syntax
We demonstrate how code-shaping could integrate with current programming practice in two usage scenarios. The interactions and user interface features to support these are real, we only had to modify the study prototype to support multiple files. Back-end infrastructure, like syncing code with a desktop editor is not implemented. Also see the accompanying video to view these scenarios.
% \dv{I re-framed the intro with more transparency about what works and what doesn't}

\subsection{Programming on the Couch}
\rev{ Alicia, a data scientist, is improving a machine-learning preprocessing pipeline distributed across multiple files. She wants to introduce flexible scaling and proper categorical encoding for both training and testing datasets. Seeking a fresh perspective, Alicia grabs her tablet and opens the Code Shaping editor to explore solutions.

To start, Alicia opens the editor support code shaping paradigm and runs the current code. She observes that the output does not handle categorical data correctly. Beside the data processing pipeline code, Alicia draws a flowchart directly on the tablet’s screen, visually aligning sketches to the vertical layout of the code. This flowchart proposes a branching structure starting from the code, \inlinecode{def preprocess\_pipeline} $\rightarrow$ 
% \emph{numerical: (min-max) || categorical: (one-hot encode)} 
\includegraphics[height=14pt]{figures/small_examples/flowchart.png}
$\rightarrow$ \inlinecode{processed data}.
The system’s feedforward interpretation (\autoref{fig:third-interface}c) \includegraphics[height=12pt]{figures/small_examples/onehot_interpretation.png} and the gutter (\autoref{fig:third-interface}d) highlights the affected lines, showing that the code will now include a one-hot encoding step where previously categorical features were ignored \includegraphics[height=10pt]{figures/small_examples/gutter_indication.png}. Alicia commits these changes using the commit button (\autoref{fig:third-interface}f) and then re-runs the code. The updated pipeline applies one-hot encoding to categorical features as intended. However, Alicia notices the code still ignores the \inlinecode{scaling} parameter.

To address this, Alicia decides to refine her sketches without losing her previous changes. She uses a one-finger tap-and-drag gesture (\autoref{fig:third-interface}a) to select the existing flowchart elements. She taps and drags downward on part of the numerical branch to create space and adds a new annotation: \emph{min-max $\rightarrow$} \includegraphics[height=10pt]{figures/small_examples/or_standardize.png}.
The feedforward interpretation and gutter again indicates what text being recognized and which code lines will be altered. Alicia commits these changes, and the editor transiently highlights the updated code snippet (\autoref{fig:third-interface}e). 
With the pipeline now meeting her requirements, Alicia draws a check mark to finalize the changes and remove any temporary sketches (\autoref{fig:third-interface}f). She re-runs the pipeline and confirm that the changes consider the \inlinecode{scaling} parameter. 
% \rev{
% Alicia, a data scientist, is tasked with refining a preprocessing pipeline for training and testing datasets used in a machine-learning workflow. The current pipeline has inconsistencies in numerical scaling and categorical encoding, and lacks a unified structure to handle both datasets. Seeking a fresh perspective, Alicia grabs her tablet and opens the Code Shaping editor to explore solutions.

% Alicia starts by analyzing the existing \inlinecode{preprocess\_pipeline} function, which only supports min-max scaling for numerical features and ignores categorical columns. She inspects the pipeline and identifies issues: standardization is hard without flexible numerical scaling, categorical features require encoding, and no consistent structure for applying transformations across datasets.
% To address these, Alicia sketches a flowchart beside the function, outlining a preprocessing pipeline with two branches, \pquote{Numerical → Scale (min-max/standardize)} and \pquote{Categorical → One-hot encode}, then draws arrows pointing to a merging node labeled \pquote{Processed Data}.
% The system provides a feedforward interpretation (\autoref{fig:third-interface}c,d), suggesting a preprocessing function with branching logic. However, the AI misinterprets the scaling logic, applying min-max scaling to all numerical features.

% To refine the new pipeline logic, Alicia duplicates part of her flowchart sketch using the tap-and-drag feature (\autoref{fig:third-interface}a). She selects the \pquote{Numerical → Scale (min-max)} branch, taps and drags it to duplicate the sketch, and then modifies the duplicate to annotate \pquote{Numerical → Scale (standardize)}. This clearly separates the scaling methods.
% Next, Alicia deletes a redundant arrow in her flowchart. She taps and drags across the unnecessary sketch element and selects \pquote{Delete} from the contextual menu (\autoref{fig:third-interface}a).
% With the refined sketch, the system reinterprets and updates its feedforward interpretation, showing the updated branching logic.
% Satisfied with the updated feedback, Alicia taps the \pquote{Commit} button (\autoref{fig:third-interface}e), seeing the code changes with diff highlighting. She then draws a check mark to integrate the changes into the function (\autoref{fig:third-interface}f) and her sketches are automatically removed, leaving a clean editor.
% Alicia tests the updated pipeline by running the \inlinecode{preprocess\_pipeline} function on both training and testing datasets. The system displays the output in the integrated console, showing correctly scaled numerical features and one-hot encoded categorical features.
}

% Alicia, a software engineer, is grappling with a complex refactoring task for a machine-learning pipeline. After hours at her workstation, she decides a change of scenery might help her think through the problem. She grabs her tablet and settles onto her living room couch, determined to work through it with some key code edits. This approach might help her better sort out abstract ideas \cite{goel1995sketches, tversky2002sketches, cherubini_lets_2007, victor2013media}.

% Opening the code shaping editor on her tablet, Alicia navigates to the main \inlinecode{preprocess\_data} function. She knows this function is a bottleneck, especially for large datasets. With a stylus, she begins sketching editing ideas directly on the code. She draws a bracket encompassing the entire function and jots down \pquote{parallelize} next to it. Then, she sketches a quick flowchart showing how the data could be split into separate streams for categories A and B.
% As she sketches, always-on feedforward AI interpretation highlights the affected code sections and suggests potential parallel processing implementations (\autoref{fig:third-interface} C-E). Encouraged, Alicia adds more detail to her plan. She circles the \inlinecode{complex\_calculation\_a} and \inlinecode{complex\_calculation\_b} function calls, drawing arrows to the side margin where she writes \pquote{vectorize}. The system responds by highlighting similar patterns in other parts of the codebase where vectorization has been applied.

% Alicia taps the commit button to accept the code modifications so far. The new version of the function introduces multiprocessing with \inlinecode{mp.Pool()} and numpy vectorization. 
% Alicia reviews the generated code, noting how it aligns with her high-level sketches.
% Alicia accepts most of the changes by drawing check-marks on the highlighted edits, but decides to iterate more on some lines of code. She draws a cross over the \inlinecode{return} statement and sketches a quick sort operation, indicating she wants the data recombined in its original order. The system updates the code to maintain the original data order, and she commits the edit with a check-mark.
% Alicia has made significant progress on a problem that had stumped her before, this has been a productive break from sitting at her desk.
% The ability to freely sketch her ideas directly on the code, seeing them instantly transformed into working implementations, has unlocked her creativity and problem-solving skills. Satisfied with her evening's work, she saves the changes and makes a note to share this breakthrough with her team tomorrow. The code shaping interface has not only allowed her to work comfortably from her couch but has also helped her bridge the gap between high-level problem-solving and concrete code implementation.

\subsection{Collaborative Code Reviewing Meeting}
Blair and Carol, senior software developers at a fintech startup, stand before a large interactive whiteboard running the code shaping interface. They are reviewing the core \inlinecode{process\_transaction} function of their payment system.
Blair loads the function into the editor on the board. Carol, stylus in hand, circles a block of nested if-statements for transaction validation. \pquote{These lines are slowing us down} she says, sketching a flowchart beside the code illustrating a streamlined process with early returns.
The system highlights the affected code sections, showing how Carol's sketch translates to code changes. Blair adds to the sketch, drawing parallel arrows for certain validation steps, suggesting \pquote{What if we run these checks concurrently?}.

Carol taps the commit button to see the final edits, but then spots a potential race condition in the new parallel structure. Blair undoes the modifications with a two-finger doubletap, and Carol sketches a new flow with the word \pquote{async} for concurrent validation.
While editing this section of code, Blair notices an opportunity to optimize database queries. He circles lines of code making multiple separate queries and draws a diagram of a batched query approach, consisting of a few boxes representing queries connected by arrows flowing into a single ``batched query'' box.
The AI model modifies the code to use a query builder pattern.
Carol points out that this change might affect error handling. She sketches a new try-catch block structure around the batched query execution. The system modifies the code based on her sketch, with the changes highlighted in green as staged edits.
As they near the end of their session, Blair and Carol review their changes holistically. They use check-marks to accept desired modifications and crosses to reject others, iterating through the highlighted sections.

% As they work, they encounter a complex algorithmic challenge in the transaction matching logic. Blair sketches out three alternative approaches directly on the code. With a swipe gesture, they create three parallel versions of the function to compare these approaches side-by-side.
% They implement each version using a mix of sketching high-level logic and directly manipulating code. The system helps fill in the details, suggesting optimizations and pointing out potential issues in each implementation.
% After running benchmarks on all three versions, they use pinch and zoom gestures to focus on the most promising approach. Carol refines this version further, smoothly transitioning between sketching architectural ideas and tweaking individual lines of code.
% Finally, they export their work: the revised code, a visual summary of their sketches and decision process, and the benchmark results. This comprehensive output is automatically added to their team's documentation system, ready for wider review and implementation.

\section{Discussion}
We discussed participants' multi-level abstraction approaches to shape code, the use of sketches to constrain generated code edits and the design implications of code shaping as a new input paradigm.

\subsection{Interacting with Code Across Multiple Levels of Abstraction}
A program is an inherently abstract entity, lacking a fixed form, and can be conceptualized in various ways—from its tangible output, such as a web page, to the underlying code syntax~\cite{hartmanis1994turing}. 
In this paper, we explored the use of sketches as a medium for programmers to express how they envision code modifications across different levels of abstraction~\cite{10.1145/3526113.3545617, 9127262, 10.1145/3654777.3676357}.
Our initial findings revealed that participants used diverse methods to convey their intentions: some drew visualizations, others provided natural language instructions, and some simply wrote pseudocode. This flexibility stands in contrast to prior methods that rely on one-to-one mappings, such as natural language directly translated to code, predefined visual programming objects, or output-directed programming, where manipulation of output changes the underlying code. 


While this paper does not focus on the detailed translation of sketches from various abstraction levels into code, our classification of elements that programmers include in their sketches offers a compelling starting point. 
For instance, in the third stage of the study, we observed that two participants expected the generated code to retain specific function names with underscores as a convention used in Python. However, the AI modified these names to follow a ``camelCase'' format for consistency with other generated code edits. This suggests that while code shaping provides high-level constraints on where and how code edits should occur, the finer details of translating between different abstractions, such as which stylistic elements to preserve, deserve further exploration. Investigating which aspects of sketches should remain consistent and which can adapt in terms of structure or format presents an intriguing direction for future research~\cite{bff9b250-7640-39e2-8f34-329fd1552822}.

\rev{
\subsection{Scope of Sketches}
Code shaping represents the concept of sketching on and around code to perform edits by bridging freeform sketching, AI interpretation, and code. Based on participant tasks, sketches were categorized into commands (intended actions), parameters (supplementary details), and targets (specific areas to edit), see \autoref{fig:arrow-variants}. These sketches often included text, annotation primitives, code syntax, or symbols, and participants occasionally extended them to diagrams, visualizations, or symbolic visuals.

Our findings highlight several tradeoffs in using different sketch representations. First, there is a cost of structure. Participants often preferred minimal-effort annotations that were effective, as creating detailed sketches required significant effort, consistent with information sensemaking \cite{russell_cost_1993}. Second, while the current model can handle many low-level operations (e.g., deleting code, renaming variables, or wrapping lines in functions), participants sometimes opted to type directly for efficiency in Study 1 (e.g., P2 and P4). This suggests a need for integrating primitive gestures, as demonstrated in our third iteration and explored in prior studies~\cite{samuelsson_towards_2023}. Lastly, abstract sketches, though semantically rich, are often difficult to be evaluated and required iterative refinement to align with linguistic code forms. Future research can focus on exploring other types of feedforward interpretation introduced in Sec.~\ref{sec:feedforward}.

Overall, code shaping does not attempt to dictate the boundaries of user expression or current model capability. Instead, it seeks to classify sketching approaches, highlight tradeoffs, and offer actionable insights for designing interaction. Our study revealed that participants' sketches were highly flexible, adapting to AI performance and specific contexts, making their scope inherently malleable.
For example, in Task 2, some participants used arrows to signify variable type changes, like ``(int, int) $\rightarrow$ (int, string)'', while others annotated function parameters directly. 
Although both text and symbols were interpreted correctly, the model struggled to map between the sketch to the intended edits accurately due to the ambiguity inherent in context-dependent sketches. 
These challenges emphasize the critical role of iteration in code shaping, where users refine their sketches, receive feedforward, and adjust their input to better convey intent.
% These challenges highlight the importance of the iterative process supported by code shaping, where users refine their sketches, receive feedforward, and adjust their input to clarify intent.
}


\subsection{Shaping AI Output with Sketches}
While we did not compare sketches to textual prompts directly, some participants (5/18) across the three stages noted that the spatial nature of sketches helped them convey how they wanted to edit \pquote{with more control}{p7}. This suggests a balance between the freedom offered by sketches and the constraints imposed by AI interpretation of code edits. Code shaping tackles this challenge by using freeform sketches to guide the AI interpretation of where and how code edits should be applied, written, performed, or referenced.
Traditional AI-driven code generation tools typically rely on natural language input or UI elements drawn on separate canvases, generating code from different mediums without directly interacting with the code itself. 
This can lead to almost limitless variations in the way natural language is mapped to code structures, which may not always align with the intent of the programmer~\cite{liu_what_2023}.
\rev{One approach exploring the concept of programmable ink, illustrating the potential of combining sketching with computational workflows by enabling users to bind sketches to data and explore outputs dynamically~\cite{inkbase, xia2017writlarge,xia2018dataink, offenwanger2024datagarden}.
However, their focus on binding sketches to predefined computational roles can limit their flexibility for scenarios like code shaping, where the emphasis lies on annotations as interpretative guides rather than functional artifacts. 
Code shaping, therefore, differentiates itself by intentionally keeping sketches free from intrinsic computational meaning but remains the capability to shape AI interpretation by layering sketches on code.}
% Code shaping employs annotations, such as arrows pointing to specific code locations or pseudocode defining program structures, that combine spatial drawings and textual elements. This approach offers an integrated way to express code edits, providing higher-level constraints to guide how the AI interprets and modifies the code. 
The combination of sketches and handwritten textual instructions for prompting enhances the precision of the edits while maintaining flexibility~\cite{haught2003creativity}, and balances creative freedom with the necessary structure to achieve desired outcomes.




\subsection{Informal and Formal Programming}
% semi-formal programming; constraint (spatial mapping and reasoning); between freeform sketches and the needed constraint for programming. (sometimes the user sketches just not correct, AI not gonna generate anything) -- connected to the following section.
Our findings show that when participants are provided with a pen to code, they approach the program differently (\autoref{sec:conceptual_shift}), 
This approach highlights the contrast between the structured nature of typing code syntax and the more abstract thinking about program structure, flow, and function.
\rev{Code shaping, unlike previous programming-by-example approaches~\cite{10.1145/22627.22349}, extends current programming paradigms by integrating code and sketches, allowing programmers to interact with their work in ways that balance structural precision with creative flexibility (\autoref{fig:classification}). 
This aligns with Olsen’s heuristics \cite{10.1145/1294211.1294256} by demonstrating high expressive leverage and reducing solution viscosity since users can achieve complex edits without articulating structured forms of representation, all while maintaining creative flexibility and structural precision.}

The domain of programming presents a unique opportunity for study, as code takes various shapes highly dependent on its substrate, ranging from editor-based code to syntax within diagrams, visualizations, user interfaces, and even comics~\cite{10.1145/3526113.3545617}.
While there are ongoing discussions comparing differences between text-based programming with higher-level representations~\cite{10.1145/3399715.3399821, noone2018visual}, code shaping aims not to replace typing but to expand the programmer's interaction palette. 
Rather than viewing our research solely as a problem-solving method~\cite{10.1145/3025453.3025765}, we explore new insights and design possibilities emerging from evolving technology~\cite{10.1145/3468505}.
The historical progression from handwritten programs and sketches on printouts to punch cards and eventually typing in an editor illustrates how each programming paradigm unveils unique affordances and constraints~\cite{arawjo_write_2020}. We envision a shared future where programmers can approach their craft through diverse methods, both formal and informal~\cite{pollock2024designing}.
Future research can explore additional representations that bridge the gap between established typed input conventions and the dynamic possibilities of sketch-based interactions, further enriching the programming experience.
\section{Limitations}
While RLEdit demonstrates promising results in lifelong editing tasks, several limitations should be acknowledged. Our evaluation methodology follows conventional datasets from existing model editing literature, primarily focusing on factual knowledge modifications without exploring other data domains. Furthermore, the capability of post-edited LLMs in processing multi-hop information remains unexplored in our current study. Although achieving robust lifelong editing capabilities continues to pose significant challenges, our future work will extend these experiments, potentially providing valuable insights for advancing research in lifelong editing.
\section{Conclusion }
This paper introduces the Latent Radiance Field (LRF), which to our knowledge, is the first work to construct radiance field representations directly in the 2D latent space for 3D reconstruction. We present a novel framework for incorporating 3D awareness into 2D representation learning, featuring a correspondence-aware autoencoding method and a VAE-Radiance Field (VAE-RF) alignment strategy to bridge the domain gap between the 2D latent space and the natural 3D space, thereby significantly enhancing the visual quality of our LRF.
Future work will focus on incorporating our method with more compact 3D representations, efficient NVS, few-shot NVS in latent space, as well as exploring its application with potential 3D latent diffusion models.


% --------------------------------

%% Acknowledgements 
%% The acknowledgments section is defined using the "acks" environment
%% (and NOT an unnumbered section). This ensures the proper
%% identification of the section in the article metadata, and the
%% consistent spelling of the heading.
%% ASK DAN WHICH ONES TO USE:
\begin{acks}
This work was made possible by 
NSERC Discovery Grant RGPIN-2024-03827, NSERC Discovery Grant \#2020-03966, and Canada foundation for innovation - John R. Evans Leaders Fund (JELF) \#42371.
\end{acks}


%% reference section
\bibliographystyle{ACM-Reference-Format}
\bibliography{_references.bib, shape.bib}
% If you are submitting your paper to arXiv, change this to be "main_acm.bib" (the same name as your main.tex file) and use the Submit feature in OL to compile the main_acm.bbl file. To avoid processing errors on arXiv with your references, your bibliography needs to have the same name as your main.tex file before you compile it.


%% appendices
%% If your work has an appendix, this is the place to put it. The TC: comments tell the word count scripts to ignore appendix.  

% %TC:ignore  
\appendix
\subsection{Lloyd-Max Algorithm}
\label{subsec:Lloyd-Max}
For a given quantization bitwidth $B$ and an operand $\bm{X}$, the Lloyd-Max algorithm finds $2^B$ quantization levels $\{\hat{x}_i\}_{i=1}^{2^B}$ such that quantizing $\bm{X}$ by rounding each scalar in $\bm{X}$ to the nearest quantization level minimizes the quantization MSE. 

The algorithm starts with an initial guess of quantization levels and then iteratively computes quantization thresholds $\{\tau_i\}_{i=1}^{2^B-1}$ and updates quantization levels $\{\hat{x}_i\}_{i=1}^{2^B}$. Specifically, at iteration $n$, thresholds are set to the midpoints of the previous iteration's levels:
\begin{align*}
    \tau_i^{(n)}=\frac{\hat{x}_i^{(n-1)}+\hat{x}_{i+1}^{(n-1)}}2 \text{ for } i=1\ldots 2^B-1
\end{align*}
Subsequently, the quantization levels are re-computed as conditional means of the data regions defined by the new thresholds:
\begin{align*}
    \hat{x}_i^{(n)}=\mathbb{E}\left[ \bm{X} \big| \bm{X}\in [\tau_{i-1}^{(n)},\tau_i^{(n)}] \right] \text{ for } i=1\ldots 2^B
\end{align*}
where to satisfy boundary conditions we have $\tau_0=-\infty$ and $\tau_{2^B}=\infty$. The algorithm iterates the above steps until convergence.

Figure \ref{fig:lm_quant} compares the quantization levels of a $7$-bit floating point (E3M3) quantizer (left) to a $7$-bit Lloyd-Max quantizer (right) when quantizing a layer of weights from the GPT3-126M model at a per-tensor granularity. As shown, the Lloyd-Max quantizer achieves substantially lower quantization MSE. Further, Table \ref{tab:FP7_vs_LM7} shows the superior perplexity achieved by Lloyd-Max quantizers for bitwidths of $7$, $6$ and $5$. The difference between the quantizers is clear at 5 bits, where per-tensor FP quantization incurs a drastic and unacceptable increase in perplexity, while Lloyd-Max quantization incurs a much smaller increase. Nevertheless, we note that even the optimal Lloyd-Max quantizer incurs a notable ($\sim 1.5$) increase in perplexity due to the coarse granularity of quantization. 

\begin{figure}[h]
  \centering
  \includegraphics[width=0.7\linewidth]{sections/figures/LM7_FP7.pdf}
  \caption{\small Quantization levels and the corresponding quantization MSE of Floating Point (left) vs Lloyd-Max (right) Quantizers for a layer of weights in the GPT3-126M model.}
  \label{fig:lm_quant}
\end{figure}

\begin{table}[h]\scriptsize
\begin{center}
\caption{\label{tab:FP7_vs_LM7} \small Comparing perplexity (lower is better) achieved by floating point quantizers and Lloyd-Max quantizers on a GPT3-126M model for the Wikitext-103 dataset.}
\begin{tabular}{c|cc|c}
\hline
 \multirow{2}{*}{\textbf{Bitwidth}} & \multicolumn{2}{|c|}{\textbf{Floating-Point Quantizer}} & \textbf{Lloyd-Max Quantizer} \\
 & Best Format & Wikitext-103 Perplexity & Wikitext-103 Perplexity \\
\hline
7 & E3M3 & 18.32 & 18.27 \\
6 & E3M2 & 19.07 & 18.51 \\
5 & E4M0 & 43.89 & 19.71 \\
\hline
\end{tabular}
\end{center}
\end{table}

\subsection{Proof of Local Optimality of LO-BCQ}
\label{subsec:lobcq_opt_proof}
For a given block $\bm{b}_j$, the quantization MSE during LO-BCQ can be empirically evaluated as $\frac{1}{L_b}\lVert \bm{b}_j- \bm{\hat{b}}_j\rVert^2_2$ where $\bm{\hat{b}}_j$ is computed from equation (\ref{eq:clustered_quantization_definition}) as $C_{f(\bm{b}_j)}(\bm{b}_j)$. Further, for a given block cluster $\mathcal{B}_i$, we compute the quantization MSE as $\frac{1}{|\mathcal{B}_{i}|}\sum_{\bm{b} \in \mathcal{B}_{i}} \frac{1}{L_b}\lVert \bm{b}- C_i^{(n)}(\bm{b})\rVert^2_2$. Therefore, at the end of iteration $n$, we evaluate the overall quantization MSE $J^{(n)}$ for a given operand $\bm{X}$ composed of $N_c$ block clusters as:
\begin{align*}
    \label{eq:mse_iter_n}
    J^{(n)} = \frac{1}{N_c} \sum_{i=1}^{N_c} \frac{1}{|\mathcal{B}_{i}^{(n)}|}\sum_{\bm{v} \in \mathcal{B}_{i}^{(n)}} \frac{1}{L_b}\lVert \bm{b}- B_i^{(n)}(\bm{b})\rVert^2_2
\end{align*}

At the end of iteration $n$, the codebooks are updated from $\mathcal{C}^{(n-1)}$ to $\mathcal{C}^{(n)}$. However, the mapping of a given vector $\bm{b}_j$ to quantizers $\mathcal{C}^{(n)}$ remains as  $f^{(n)}(\bm{b}_j)$. At the next iteration, during the vector clustering step, $f^{(n+1)}(\bm{b}_j)$ finds new mapping of $\bm{b}_j$ to updated codebooks $\mathcal{C}^{(n)}$ such that the quantization MSE over the candidate codebooks is minimized. Therefore, we obtain the following result for $\bm{b}_j$:
\begin{align*}
\frac{1}{L_b}\lVert \bm{b}_j - C_{f^{(n+1)}(\bm{b}_j)}^{(n)}(\bm{b}_j)\rVert^2_2 \le \frac{1}{L_b}\lVert \bm{b}_j - C_{f^{(n)}(\bm{b}_j)}^{(n)}(\bm{b}_j)\rVert^2_2
\end{align*}

That is, quantizing $\bm{b}_j$ at the end of the block clustering step of iteration $n+1$ results in lower quantization MSE compared to quantizing at the end of iteration $n$. Since this is true for all $\bm{b} \in \bm{X}$, we assert the following:
\begin{equation}
\begin{split}
\label{eq:mse_ineq_1}
    \tilde{J}^{(n+1)} &= \frac{1}{N_c} \sum_{i=1}^{N_c} \frac{1}{|\mathcal{B}_{i}^{(n+1)}|}\sum_{\bm{b} \in \mathcal{B}_{i}^{(n+1)}} \frac{1}{L_b}\lVert \bm{b} - C_i^{(n)}(b)\rVert^2_2 \le J^{(n)}
\end{split}
\end{equation}
where $\tilde{J}^{(n+1)}$ is the the quantization MSE after the vector clustering step at iteration $n+1$.

Next, during the codebook update step (\ref{eq:quantizers_update}) at iteration $n+1$, the per-cluster codebooks $\mathcal{C}^{(n)}$ are updated to $\mathcal{C}^{(n+1)}$ by invoking the Lloyd-Max algorithm \citep{Lloyd}. We know that for any given value distribution, the Lloyd-Max algorithm minimizes the quantization MSE. Therefore, for a given vector cluster $\mathcal{B}_i$ we obtain the following result:

\begin{equation}
    \frac{1}{|\mathcal{B}_{i}^{(n+1)}|}\sum_{\bm{b} \in \mathcal{B}_{i}^{(n+1)}} \frac{1}{L_b}\lVert \bm{b}- C_i^{(n+1)}(\bm{b})\rVert^2_2 \le \frac{1}{|\mathcal{B}_{i}^{(n+1)}|}\sum_{\bm{b} \in \mathcal{B}_{i}^{(n+1)}} \frac{1}{L_b}\lVert \bm{b}- C_i^{(n)}(\bm{b})\rVert^2_2
\end{equation}

The above equation states that quantizing the given block cluster $\mathcal{B}_i$ after updating the associated codebook from $C_i^{(n)}$ to $C_i^{(n+1)}$ results in lower quantization MSE. Since this is true for all the block clusters, we derive the following result: 
\begin{equation}
\begin{split}
\label{eq:mse_ineq_2}
     J^{(n+1)} &= \frac{1}{N_c} \sum_{i=1}^{N_c} \frac{1}{|\mathcal{B}_{i}^{(n+1)}|}\sum_{\bm{b} \in \mathcal{B}_{i}^{(n+1)}} \frac{1}{L_b}\lVert \bm{b}- C_i^{(n+1)}(\bm{b})\rVert^2_2  \le \tilde{J}^{(n+1)}   
\end{split}
\end{equation}

Following (\ref{eq:mse_ineq_1}) and (\ref{eq:mse_ineq_2}), we find that the quantization MSE is non-increasing for each iteration, that is, $J^{(1)} \ge J^{(2)} \ge J^{(3)} \ge \ldots \ge J^{(M)}$ where $M$ is the maximum number of iterations. 
%Therefore, we can say that if the algorithm converges, then it must be that it has converged to a local minimum. 
\hfill $\blacksquare$


\begin{figure}
    \begin{center}
    \includegraphics[width=0.5\textwidth]{sections//figures/mse_vs_iter.pdf}
    \end{center}
    \caption{\small NMSE vs iterations during LO-BCQ compared to other block quantization proposals}
    \label{fig:nmse_vs_iter}
\end{figure}

Figure \ref{fig:nmse_vs_iter} shows the empirical convergence of LO-BCQ across several block lengths and number of codebooks. Also, the MSE achieved by LO-BCQ is compared to baselines such as MXFP and VSQ. As shown, LO-BCQ converges to a lower MSE than the baselines. Further, we achieve better convergence for larger number of codebooks ($N_c$) and for a smaller block length ($L_b$), both of which increase the bitwidth of BCQ (see Eq \ref{eq:bitwidth_bcq}).


\subsection{Additional Accuracy Results}
%Table \ref{tab:lobcq_config} lists the various LOBCQ configurations and their corresponding bitwidths.
\begin{table}
\setlength{\tabcolsep}{4.75pt}
\begin{center}
\caption{\label{tab:lobcq_config} Various LO-BCQ configurations and their bitwidths.}
\begin{tabular}{|c||c|c|c|c||c|c||c|} 
\hline
 & \multicolumn{4}{|c||}{$L_b=8$} & \multicolumn{2}{|c||}{$L_b=4$} & $L_b=2$ \\
 \hline
 \backslashbox{$L_A$\kern-1em}{\kern-1em$N_c$} & 2 & 4 & 8 & 16 & 2 & 4 & 2 \\
 \hline
 64 & 4.25 & 4.375 & 4.5 & 4.625 & 4.375 & 4.625 & 4.625\\
 \hline
 32 & 4.375 & 4.5 & 4.625& 4.75 & 4.5 & 4.75 & 4.75 \\
 \hline
 16 & 4.625 & 4.75& 4.875 & 5 & 4.75 & 5 & 5 \\
 \hline
\end{tabular}
\end{center}
\end{table}

%\subsection{Perplexity achieved by various LO-BCQ configurations on Wikitext-103 dataset}

\begin{table} \centering
\begin{tabular}{|c||c|c|c|c||c|c||c|} 
\hline
 $L_b \rightarrow$& \multicolumn{4}{c||}{8} & \multicolumn{2}{c||}{4} & 2\\
 \hline
 \backslashbox{$L_A$\kern-1em}{\kern-1em$N_c$} & 2 & 4 & 8 & 16 & 2 & 4 & 2  \\
 %$N_c \rightarrow$ & 2 & 4 & 8 & 16 & 2 & 4 & 2 \\
 \hline
 \hline
 \multicolumn{8}{c}{GPT3-1.3B (FP32 PPL = 9.98)} \\ 
 \hline
 \hline
 64 & 10.40 & 10.23 & 10.17 & 10.15 &  10.28 & 10.18 & 10.19 \\
 \hline
 32 & 10.25 & 10.20 & 10.15 & 10.12 &  10.23 & 10.17 & 10.17 \\
 \hline
 16 & 10.22 & 10.16 & 10.10 & 10.09 &  10.21 & 10.14 & 10.16 \\
 \hline
  \hline
 \multicolumn{8}{c}{GPT3-8B (FP32 PPL = 7.38)} \\ 
 \hline
 \hline
 64 & 7.61 & 7.52 & 7.48 &  7.47 &  7.55 &  7.49 & 7.50 \\
 \hline
 32 & 7.52 & 7.50 & 7.46 &  7.45 &  7.52 &  7.48 & 7.48  \\
 \hline
 16 & 7.51 & 7.48 & 7.44 &  7.44 &  7.51 &  7.49 & 7.47  \\
 \hline
\end{tabular}
\caption{\label{tab:ppl_gpt3_abalation} Wikitext-103 perplexity across GPT3-1.3B and 8B models.}
\end{table}

\begin{table} \centering
\begin{tabular}{|c||c|c|c|c||} 
\hline
 $L_b \rightarrow$& \multicolumn{4}{c||}{8}\\
 \hline
 \backslashbox{$L_A$\kern-1em}{\kern-1em$N_c$} & 2 & 4 & 8 & 16 \\
 %$N_c \rightarrow$ & 2 & 4 & 8 & 16 & 2 & 4 & 2 \\
 \hline
 \hline
 \multicolumn{5}{|c|}{Llama2-7B (FP32 PPL = 5.06)} \\ 
 \hline
 \hline
 64 & 5.31 & 5.26 & 5.19 & 5.18  \\
 \hline
 32 & 5.23 & 5.25 & 5.18 & 5.15  \\
 \hline
 16 & 5.23 & 5.19 & 5.16 & 5.14  \\
 \hline
 \multicolumn{5}{|c|}{Nemotron4-15B (FP32 PPL = 5.87)} \\ 
 \hline
 \hline
 64  & 6.3 & 6.20 & 6.13 & 6.08  \\
 \hline
 32  & 6.24 & 6.12 & 6.07 & 6.03  \\
 \hline
 16  & 6.12 & 6.14 & 6.04 & 6.02  \\
 \hline
 \multicolumn{5}{|c|}{Nemotron4-340B (FP32 PPL = 3.48)} \\ 
 \hline
 \hline
 64 & 3.67 & 3.62 & 3.60 & 3.59 \\
 \hline
 32 & 3.63 & 3.61 & 3.59 & 3.56 \\
 \hline
 16 & 3.61 & 3.58 & 3.57 & 3.55 \\
 \hline
\end{tabular}
\caption{\label{tab:ppl_llama7B_nemo15B} Wikitext-103 perplexity compared to FP32 baseline in Llama2-7B and Nemotron4-15B, 340B models}
\end{table}

%\subsection{Perplexity achieved by various LO-BCQ configurations on MMLU dataset}


\begin{table} \centering
\begin{tabular}{|c||c|c|c|c||c|c|c|c|} 
\hline
 $L_b \rightarrow$& \multicolumn{4}{c||}{8} & \multicolumn{4}{c||}{8}\\
 \hline
 \backslashbox{$L_A$\kern-1em}{\kern-1em$N_c$} & 2 & 4 & 8 & 16 & 2 & 4 & 8 & 16  \\
 %$N_c \rightarrow$ & 2 & 4 & 8 & 16 & 2 & 4 & 2 \\
 \hline
 \hline
 \multicolumn{5}{|c|}{Llama2-7B (FP32 Accuracy = 45.8\%)} & \multicolumn{4}{|c|}{Llama2-70B (FP32 Accuracy = 69.12\%)} \\ 
 \hline
 \hline
 64 & 43.9 & 43.4 & 43.9 & 44.9 & 68.07 & 68.27 & 68.17 & 68.75 \\
 \hline
 32 & 44.5 & 43.8 & 44.9 & 44.5 & 68.37 & 68.51 & 68.35 & 68.27  \\
 \hline
 16 & 43.9 & 42.7 & 44.9 & 45 & 68.12 & 68.77 & 68.31 & 68.59  \\
 \hline
 \hline
 \multicolumn{5}{|c|}{GPT3-22B (FP32 Accuracy = 38.75\%)} & \multicolumn{4}{|c|}{Nemotron4-15B (FP32 Accuracy = 64.3\%)} \\ 
 \hline
 \hline
 64 & 36.71 & 38.85 & 38.13 & 38.92 & 63.17 & 62.36 & 63.72 & 64.09 \\
 \hline
 32 & 37.95 & 38.69 & 39.45 & 38.34 & 64.05 & 62.30 & 63.8 & 64.33  \\
 \hline
 16 & 38.88 & 38.80 & 38.31 & 38.92 & 63.22 & 63.51 & 63.93 & 64.43  \\
 \hline
\end{tabular}
\caption{\label{tab:mmlu_abalation} Accuracy on MMLU dataset across GPT3-22B, Llama2-7B, 70B and Nemotron4-15B models.}
\end{table}


%\subsection{Perplexity achieved by various LO-BCQ configurations on LM evaluation harness}

\begin{table} \centering
\begin{tabular}{|c||c|c|c|c||c|c|c|c|} 
\hline
 $L_b \rightarrow$& \multicolumn{4}{c||}{8} & \multicolumn{4}{c||}{8}\\
 \hline
 \backslashbox{$L_A$\kern-1em}{\kern-1em$N_c$} & 2 & 4 & 8 & 16 & 2 & 4 & 8 & 16  \\
 %$N_c \rightarrow$ & 2 & 4 & 8 & 16 & 2 & 4 & 2 \\
 \hline
 \hline
 \multicolumn{5}{|c|}{Race (FP32 Accuracy = 37.51\%)} & \multicolumn{4}{|c|}{Boolq (FP32 Accuracy = 64.62\%)} \\ 
 \hline
 \hline
 64 & 36.94 & 37.13 & 36.27 & 37.13 & 63.73 & 62.26 & 63.49 & 63.36 \\
 \hline
 32 & 37.03 & 36.36 & 36.08 & 37.03 & 62.54 & 63.51 & 63.49 & 63.55  \\
 \hline
 16 & 37.03 & 37.03 & 36.46 & 37.03 & 61.1 & 63.79 & 63.58 & 63.33  \\
 \hline
 \hline
 \multicolumn{5}{|c|}{Winogrande (FP32 Accuracy = 58.01\%)} & \multicolumn{4}{|c|}{Piqa (FP32 Accuracy = 74.21\%)} \\ 
 \hline
 \hline
 64 & 58.17 & 57.22 & 57.85 & 58.33 & 73.01 & 73.07 & 73.07 & 72.80 \\
 \hline
 32 & 59.12 & 58.09 & 57.85 & 58.41 & 73.01 & 73.94 & 72.74 & 73.18  \\
 \hline
 16 & 57.93 & 58.88 & 57.93 & 58.56 & 73.94 & 72.80 & 73.01 & 73.94  \\
 \hline
\end{tabular}
\caption{\label{tab:mmlu_abalation} Accuracy on LM evaluation harness tasks on GPT3-1.3B model.}
\end{table}

\begin{table} \centering
\begin{tabular}{|c||c|c|c|c||c|c|c|c|} 
\hline
 $L_b \rightarrow$& \multicolumn{4}{c||}{8} & \multicolumn{4}{c||}{8}\\
 \hline
 \backslashbox{$L_A$\kern-1em}{\kern-1em$N_c$} & 2 & 4 & 8 & 16 & 2 & 4 & 8 & 16  \\
 %$N_c \rightarrow$ & 2 & 4 & 8 & 16 & 2 & 4 & 2 \\
 \hline
 \hline
 \multicolumn{5}{|c|}{Race (FP32 Accuracy = 41.34\%)} & \multicolumn{4}{|c|}{Boolq (FP32 Accuracy = 68.32\%)} \\ 
 \hline
 \hline
 64 & 40.48 & 40.10 & 39.43 & 39.90 & 69.20 & 68.41 & 69.45 & 68.56 \\
 \hline
 32 & 39.52 & 39.52 & 40.77 & 39.62 & 68.32 & 67.43 & 68.17 & 69.30  \\
 \hline
 16 & 39.81 & 39.71 & 39.90 & 40.38 & 68.10 & 66.33 & 69.51 & 69.42  \\
 \hline
 \hline
 \multicolumn{5}{|c|}{Winogrande (FP32 Accuracy = 67.88\%)} & \multicolumn{4}{|c|}{Piqa (FP32 Accuracy = 78.78\%)} \\ 
 \hline
 \hline
 64 & 66.85 & 66.61 & 67.72 & 67.88 & 77.31 & 77.42 & 77.75 & 77.64 \\
 \hline
 32 & 67.25 & 67.72 & 67.72 & 67.00 & 77.31 & 77.04 & 77.80 & 77.37  \\
 \hline
 16 & 68.11 & 68.90 & 67.88 & 67.48 & 77.37 & 78.13 & 78.13 & 77.69  \\
 \hline
\end{tabular}
\caption{\label{tab:mmlu_abalation} Accuracy on LM evaluation harness tasks on GPT3-8B model.}
\end{table}

\begin{table} \centering
\begin{tabular}{|c||c|c|c|c||c|c|c|c|} 
\hline
 $L_b \rightarrow$& \multicolumn{4}{c||}{8} & \multicolumn{4}{c||}{8}\\
 \hline
 \backslashbox{$L_A$\kern-1em}{\kern-1em$N_c$} & 2 & 4 & 8 & 16 & 2 & 4 & 8 & 16  \\
 %$N_c \rightarrow$ & 2 & 4 & 8 & 16 & 2 & 4 & 2 \\
 \hline
 \hline
 \multicolumn{5}{|c|}{Race (FP32 Accuracy = 40.67\%)} & \multicolumn{4}{|c|}{Boolq (FP32 Accuracy = 76.54\%)} \\ 
 \hline
 \hline
 64 & 40.48 & 40.10 & 39.43 & 39.90 & 75.41 & 75.11 & 77.09 & 75.66 \\
 \hline
 32 & 39.52 & 39.52 & 40.77 & 39.62 & 76.02 & 76.02 & 75.96 & 75.35  \\
 \hline
 16 & 39.81 & 39.71 & 39.90 & 40.38 & 75.05 & 73.82 & 75.72 & 76.09  \\
 \hline
 \hline
 \multicolumn{5}{|c|}{Winogrande (FP32 Accuracy = 70.64\%)} & \multicolumn{4}{|c|}{Piqa (FP32 Accuracy = 79.16\%)} \\ 
 \hline
 \hline
 64 & 69.14 & 70.17 & 70.17 & 70.56 & 78.24 & 79.00 & 78.62 & 78.73 \\
 \hline
 32 & 70.96 & 69.69 & 71.27 & 69.30 & 78.56 & 79.49 & 79.16 & 78.89  \\
 \hline
 16 & 71.03 & 69.53 & 69.69 & 70.40 & 78.13 & 79.16 & 79.00 & 79.00  \\
 \hline
\end{tabular}
\caption{\label{tab:mmlu_abalation} Accuracy on LM evaluation harness tasks on GPT3-22B model.}
\end{table}

\begin{table} \centering
\begin{tabular}{|c||c|c|c|c||c|c|c|c|} 
\hline
 $L_b \rightarrow$& \multicolumn{4}{c||}{8} & \multicolumn{4}{c||}{8}\\
 \hline
 \backslashbox{$L_A$\kern-1em}{\kern-1em$N_c$} & 2 & 4 & 8 & 16 & 2 & 4 & 8 & 16  \\
 %$N_c \rightarrow$ & 2 & 4 & 8 & 16 & 2 & 4 & 2 \\
 \hline
 \hline
 \multicolumn{5}{|c|}{Race (FP32 Accuracy = 44.4\%)} & \multicolumn{4}{|c|}{Boolq (FP32 Accuracy = 79.29\%)} \\ 
 \hline
 \hline
 64 & 42.49 & 42.51 & 42.58 & 43.45 & 77.58 & 77.37 & 77.43 & 78.1 \\
 \hline
 32 & 43.35 & 42.49 & 43.64 & 43.73 & 77.86 & 75.32 & 77.28 & 77.86  \\
 \hline
 16 & 44.21 & 44.21 & 43.64 & 42.97 & 78.65 & 77 & 76.94 & 77.98  \\
 \hline
 \hline
 \multicolumn{5}{|c|}{Winogrande (FP32 Accuracy = 69.38\%)} & \multicolumn{4}{|c|}{Piqa (FP32 Accuracy = 78.07\%)} \\ 
 \hline
 \hline
 64 & 68.9 & 68.43 & 69.77 & 68.19 & 77.09 & 76.82 & 77.09 & 77.86 \\
 \hline
 32 & 69.38 & 68.51 & 68.82 & 68.90 & 78.07 & 76.71 & 78.07 & 77.86  \\
 \hline
 16 & 69.53 & 67.09 & 69.38 & 68.90 & 77.37 & 77.8 & 77.91 & 77.69  \\
 \hline
\end{tabular}
\caption{\label{tab:mmlu_abalation} Accuracy on LM evaluation harness tasks on Llama2-7B model.}
\end{table}

\begin{table} \centering
\begin{tabular}{|c||c|c|c|c||c|c|c|c|} 
\hline
 $L_b \rightarrow$& \multicolumn{4}{c||}{8} & \multicolumn{4}{c||}{8}\\
 \hline
 \backslashbox{$L_A$\kern-1em}{\kern-1em$N_c$} & 2 & 4 & 8 & 16 & 2 & 4 & 8 & 16  \\
 %$N_c \rightarrow$ & 2 & 4 & 8 & 16 & 2 & 4 & 2 \\
 \hline
 \hline
 \multicolumn{5}{|c|}{Race (FP32 Accuracy = 48.8\%)} & \multicolumn{4}{|c|}{Boolq (FP32 Accuracy = 85.23\%)} \\ 
 \hline
 \hline
 64 & 49.00 & 49.00 & 49.28 & 48.71 & 82.82 & 84.28 & 84.03 & 84.25 \\
 \hline
 32 & 49.57 & 48.52 & 48.33 & 49.28 & 83.85 & 84.46 & 84.31 & 84.93  \\
 \hline
 16 & 49.85 & 49.09 & 49.28 & 48.99 & 85.11 & 84.46 & 84.61 & 83.94  \\
 \hline
 \hline
 \multicolumn{5}{|c|}{Winogrande (FP32 Accuracy = 79.95\%)} & \multicolumn{4}{|c|}{Piqa (FP32 Accuracy = 81.56\%)} \\ 
 \hline
 \hline
 64 & 78.77 & 78.45 & 78.37 & 79.16 & 81.45 & 80.69 & 81.45 & 81.5 \\
 \hline
 32 & 78.45 & 79.01 & 78.69 & 80.66 & 81.56 & 80.58 & 81.18 & 81.34  \\
 \hline
 16 & 79.95 & 79.56 & 79.79 & 79.72 & 81.28 & 81.66 & 81.28 & 80.96  \\
 \hline
\end{tabular}
\caption{\label{tab:mmlu_abalation} Accuracy on LM evaluation harness tasks on Llama2-70B model.}
\end{table}

%\section{MSE Studies}
%\textcolor{red}{TODO}


\subsection{Number Formats and Quantization Method}
\label{subsec:numFormats_quantMethod}
\subsubsection{Integer Format}
An $n$-bit signed integer (INT) is typically represented with a 2s-complement format \citep{yao2022zeroquant,xiao2023smoothquant,dai2021vsq}, where the most significant bit denotes the sign.

\subsubsection{Floating Point Format}
An $n$-bit signed floating point (FP) number $x$ comprises of a 1-bit sign ($x_{\mathrm{sign}}$), $B_m$-bit mantissa ($x_{\mathrm{mant}}$) and $B_e$-bit exponent ($x_{\mathrm{exp}}$) such that $B_m+B_e=n-1$. The associated constant exponent bias ($E_{\mathrm{bias}}$) is computed as $(2^{{B_e}-1}-1)$. We denote this format as $E_{B_e}M_{B_m}$.  

\subsubsection{Quantization Scheme}
\label{subsec:quant_method}
A quantization scheme dictates how a given unquantized tensor is converted to its quantized representation. We consider FP formats for the purpose of illustration. Given an unquantized tensor $\bm{X}$ and an FP format $E_{B_e}M_{B_m}$, we first, we compute the quantization scale factor $s_X$ that maps the maximum absolute value of $\bm{X}$ to the maximum quantization level of the $E_{B_e}M_{B_m}$ format as follows:
\begin{align}
\label{eq:sf}
    s_X = \frac{\mathrm{max}(|\bm{X}|)}{\mathrm{max}(E_{B_e}M_{B_m})}
\end{align}
In the above equation, $|\cdot|$ denotes the absolute value function.

Next, we scale $\bm{X}$ by $s_X$ and quantize it to $\hat{\bm{X}}$ by rounding it to the nearest quantization level of $E_{B_e}M_{B_m}$ as:

\begin{align}
\label{eq:tensor_quant}
    \hat{\bm{X}} = \text{round-to-nearest}\left(\frac{\bm{X}}{s_X}, E_{B_e}M_{B_m}\right)
\end{align}

We perform dynamic max-scaled quantization \citep{wu2020integer}, where the scale factor $s$ for activations is dynamically computed during runtime.

\subsection{Vector Scaled Quantization}
\begin{wrapfigure}{r}{0.35\linewidth}
  \centering
  \includegraphics[width=\linewidth]{sections/figures/vsquant.jpg}
  \caption{\small Vectorwise decomposition for per-vector scaled quantization (VSQ \citep{dai2021vsq}).}
  \label{fig:vsquant}
\end{wrapfigure}
During VSQ \citep{dai2021vsq}, the operand tensors are decomposed into 1D vectors in a hardware friendly manner as shown in Figure \ref{fig:vsquant}. Since the decomposed tensors are used as operands in matrix multiplications during inference, it is beneficial to perform this decomposition along the reduction dimension of the multiplication. The vectorwise quantization is performed similar to tensorwise quantization described in Equations \ref{eq:sf} and \ref{eq:tensor_quant}, where a scale factor $s_v$ is required for each vector $\bm{v}$ that maps the maximum absolute value of that vector to the maximum quantization level. While smaller vector lengths can lead to larger accuracy gains, the associated memory and computational overheads due to the per-vector scale factors increases. To alleviate these overheads, VSQ \citep{dai2021vsq} proposed a second level quantization of the per-vector scale factors to unsigned integers, while MX \citep{rouhani2023shared} quantizes them to integer powers of 2 (denoted as $2^{INT}$).

\subsubsection{MX Format}
The MX format proposed in \citep{rouhani2023microscaling} introduces the concept of sub-block shifting. For every two scalar elements of $b$-bits each, there is a shared exponent bit. The value of this exponent bit is determined through an empirical analysis that targets minimizing quantization MSE. We note that the FP format $E_{1}M_{b}$ is strictly better than MX from an accuracy perspective since it allocates a dedicated exponent bit to each scalar as opposed to sharing it across two scalars. Therefore, we conservatively bound the accuracy of a $b+2$-bit signed MX format with that of a $E_{1}M_{b}$ format in our comparisons. For instance, we use E1M2 format as a proxy for MX4.

\begin{figure}
    \centering
    \includegraphics[width=1\linewidth]{sections//figures/BlockFormats.pdf}
    \caption{\small Comparing LO-BCQ to MX format.}
    \label{fig:block_formats}
\end{figure}

Figure \ref{fig:block_formats} compares our $4$-bit LO-BCQ block format to MX \citep{rouhani2023microscaling}. As shown, both LO-BCQ and MX decompose a given operand tensor into block arrays and each block array into blocks. Similar to MX, we find that per-block quantization ($L_b < L_A$) leads to better accuracy due to increased flexibility. While MX achieves this through per-block $1$-bit micro-scales, we associate a dedicated codebook to each block through a per-block codebook selector. Further, MX quantizes the per-block array scale-factor to E8M0 format without per-tensor scaling. In contrast during LO-BCQ, we find that per-tensor scaling combined with quantization of per-block array scale-factor to E4M3 format results in superior inference accuracy across models. 


%% Reset the figure count and add an A prefix to distinguish it from your other figures
% \renewcommand\thefigure{\thesection.\arabic{figure}}
% \setcounter{figure}{0}
% %TC:endignore 

\end{document}
\endinput
%%
%% End of file
