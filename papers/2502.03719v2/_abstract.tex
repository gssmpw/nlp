% maximum 150 word abstract (aim for ~120)
% Annotating code with ink-based sketches assists programmers in comprehending code, communicating with collaborators, and specifying intended changes. However, current tools do not consider translating these sketched annotations into actionable commands on the code. With advancements in sketch recognition and code generation, it is now feasible to edit code based on the sketched annotations. Despite this, programmers' intentions can be vague and vary across individuals and scenarios. The unique nature of annotations, which integrate code as part of the sketches, further complicates their interpretation by models.

We introduce the concept of code shaping, an interaction paradigm for editing code using free-form sketch annotations directly on top of the code and console output. To evaluate this concept, we conducted a three-stage design study with 18 different programmers to investigate how sketches can communicate intended code edits to an AI model for interpretation and execution. The results show how different sketches are used, the strategies programmers employ during iterative interactions with AI interpretations, and interaction design principles that support the reconciliation between the code editor and sketches. Finally, we demonstrate the practical application of the code shaping concept with two use case scenarios, illustrating design implications from the study.
% Our final design treats these sketches as dynamic visual programming language elements, providing real-time feedback on the AI's interpretation of the annotations and reconciling the representation of sketches and code. 
% The results reveal personalized workflow strategies and how similar annotations vary in abstractness and intention across different scenarios and users. 


% Programmers use ink-based sketches to annotate code for various purposes. However, operationalizing these freeform sketches into edits on code can be challenging due to their ambiguity and the lack of understanding of how sketches and code intertwined. This poster presents a first step toward building a system that translates ink-based annotations into actionable code edits.
% We conducted a $N$=$6$ exploratory study using our developed probe to understand and categorize programmers' sketches. The results show that similar annotations can have varying meanings across different scenarios and users. We classified the annotations into a quadrant based on two spectrums: abstractness and the intention behind the sketches.

% We found that \xx{}.
% Following a user-centered design process, we developed a system, \sys{}, supporting programmers iteratively and progressively `disambiguate' their sketches to align with system interpretation.
% We then validated the proposed design guidelines through a user study with 12 participants and found that \xx{}.


% 30 word contribution statement (used for final version, but good to write now)



