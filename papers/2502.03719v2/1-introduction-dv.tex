%!TEX root = paper.tex
\section{Introduction}
In programming tasks, text is not always the primary medium for expressing ideas \cite{latoza_maintaining_2006}. Programmers often turn to sketching on whiteboards and paper to externalize thoughts and concepts \cite{cherubini_lets_2007, 6922572, 6065018}. This includes tasks like designing program structure, working out algorithms, and planning code edits~\cite{cherubini_lets_2007, 10.1145/1879211.1879217, sutherland_investigation_2017}.
The informal nature of sketching helps untangle complex tasks, represent abstract ideas, and requires less cognitive effort to comprehend \cite{cherubini_lets_2007, tversky2002sketches, goel1995sketches}.


Prior research has explored programming-by-example systems that transform sketches~\cite{10.1145/22627.22349}, such as diagrams~\cite{10.1145/1281500.1281546}, mathematical symbols~\cite{li2008algosketch, 10.1145/3411764.3445460}, and user interfaces~\cite{tldraw, 910894, microsoft_sketch2code}, into functional programs.
However, these systems often target non-programmers, with the generated code typically hidden or not intended for direct editing.
For programmers, another line of research has enhanced current integrated development environments (IDE) with sketch-based annotation features from the engineering perspective to support note-taking~\cite{sutherland2015observational, 10.1145/1324892.1324935}, facilitate collaboration~\cite{lichtschlag2014codegraffiti}, and aid in planning future code edits~\cite{samuelsson2020eliciting}.
Despite these advancements, sketching and code editing are still largely treated as separate activities in the software development process.

This division stems from the traditional view of programming as primarily text-based~\cite{arawjo_write_2020}, with sketching seen as an auxiliary tool.
Programmers must switch contexts between sketching and coding, potentially losing insights during the translation from visual ideas to code modifications~\cite{parnin2006building, 10.1145/1879211.1879217, bff9b250-7640-39e2-8f34-329fd1552822}. This challenge is exacerbated by the non-linear and dynamic nature of programming, where code is frequently revisited and revised in response to evolving requirements and new discoveries.
Hence, sketches have been primarily considered as a static external representation of the programmer's thoughts instead of ways to interact with code~\cite{sutherland2015observational, 10.1145/1324892.1324935, 1698771}. 



To address this separation, we propose a sketch-based editing approach where a \textit{programmer draws free-form annotations on and around the code to iteratively guide an AI model in modifying code structure, flow, and syntax}: a concept we call \textit{code shaping}. For example, to insert a function to visualize data, a programmer can circle lines of code related to data attributes, draw an arrow to a sketch of a graph, then draw another arrow with the word ``def'' back to a line of code to insert the function (\autoref{fig:teaser}a,b,c). Further iterations of sketching can revise the function name or specify additional data processing steps (\autoref{fig:teaser}d,e). This approach tightly integrates free-form sketching with realtime code editing both visually and operationally, providing programmers with an alternative modality to express modifications.
This approach allows programmers to encapsulate their expectations for the program's functionality and link these sketches directly to syntactic code. However, challenges such as model interpretation errors due to the inherent ambiguity of sketches~\cite{10.1145/1281500.1281527, 10.1145/237091.237119} and the fundamental differences between sketching and coding modalities require further design exploration.
% This is very different than programming-by-example systems where sketches are used to generate complete programs~\cite{10.1145/1281500.1281546, li2008algosketch, 9680034, tldraw, 910894, microsoft_sketch2code}, or the code editors that support a static layer of sketched annotations for note \cite{sutherland2015observational, 10.1145/1324892.1324935,lichtschlag2014codegraffiti,samuelsson2020eliciting}. 
% \dv{I tried to say this "we are different" statement  more concisely, because now in par 1 I talk about programming-by-example and tried to be more specific about works that added sketch annotations to a code editor.}
% We differentiate code shaping from other research that focuses on converting sketched drawings, such as diagrams~\cite{10.1145/1281500.1281546}, mathematical notations~\cite{li2008algosketch}, visualizations~\cite{9680034}, or user interfaces~\cite{tldraw, 910894, microsoft_sketch2code}, directly into code. While these approaches transform sketches into code, they consider sketches as representations of the example final output rather than as a means to interact with and edit existing code for programmers. In contrast, code shaping integrates sketching as an interactive modality for modifying and refining code throughout the programming process.
% Our approach is enabled by an underlying general-purpose AI model. This means that the interpretation of sketches is truly free-form and in principle defined by the programmer, but it also introduces the problem of AI interpretation errors due to the inherent ambiguity of sketches~\cite{10.1145/1281500.1281527, 10.1145/237091.237119}.

We adopted a user-centered design process with 18 programmers using a prototype system probe that implements the code shaping concept. Our findings reveal the types of sketches programmers created, their strategies for correcting AI model errors, and design implications for bridging the conceptual gap between the canvas where sketches are made, the textual code representation, and the AI models. We demonstrate these design implications with two real-world use cases: a productivity break using a tablet and pair programming at a whiteboard.
The contribution of this research is not to claim that code shaping is superior to other interaction paradigms, such as typing, but to establish it as a viable alternative that empowers programmers to iteratively express and refine their code edits through free-form sketches. 
% Importantly, we do not to claim that sketching for code shaping is superior to conventional editing using a keyboard. Our contribution is to propose a viable alternative to empower programmers to iteratively express and refine their code edits through free-form sketches.




% primarily treated code and sketches as separate mediums and modalities for interaction. For instance, studies on annotations in programming view these sketches as static externalization of a programmer's thoughts~\cite{sutherland2015observational, 10.1145/1324892.1324935}, without considering the non-linear and dynamic nature of programming.


% While recent advances in multi-modal large language models have made this concept more feasible, two major challenges remain: model recognition and interaction paradigm switching.
% First, the opaque nature of AI models often forces users to guess why recognition failures occur. This issue is compounded by the inherent ambiguity of sketches, where similar annotations can have different meanings across various scenarios and tasks, necessitating multiple iterations to refine sketches for accurate recognition~\cite{10.1145/1281500.1281527, 4302611}.
% Second, the input modalities (pen vs. keyboard) and the tools designed (eraser vs. backspace) for sketching and code editing are fundamentally different. This discrepancy leads to the problem of context switching, even when the sketching and code editing interfaces are visually integrated.
% These challenges highlight the need to understand how programmers use sketches to express commands, what the common errors are, and how they recover from them.
% The finding will be important for designing a system that supports users in iteratively clarifying and modifying both code and annotations.



% Operationalizing these annotations is an essential step towards bridging the gap between abstract thinking and formal coding, helping programmers express their intentions more intuitively while retaining the dynamism of programming.

% spatial and multi-modal reasoning, allowing programmers to shape their sketches more freely.
% For instance, programmers can sketch their thoughts for solving programming problems, write pseudocode, diagram code structures, or illustrate outputs such as visualizations or user interfaces to add or edit corresponding code. 



% To explore the notion of \textit{code shaping}, 
% Towards a fully functional system supporting code shaping, we aim to understand what programmers intend to convey through sketches and how the system interprets or misinterprets them. \dv{the prev paragraph now covers what is in the prev sentence, I would remove it a sentence and just launch into explaining the study with next sentence}
% We conducted an exploratory study with six programmers, using a prototype that transforms free-form sketches on a code editor into actual code edits and reported results in \autoref{sec:result}.
% \dv{should mention AI in prev sentence} 
% Our results show that programmers gradually develop their own workflow for code shaping and assign different meanings to their sketched annotations in different scenarios.