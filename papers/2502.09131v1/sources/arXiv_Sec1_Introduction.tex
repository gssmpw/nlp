\section{Introduction} \label{sec:Introduction}
Data-driven system representations based on the fundamental lemma by \citet{willems05note} are of continued and increasing research interest. The main insight is that the trajectories of any controllable Linear Time Invariant (LTI) systems can be characterized without explicit identification of a state-space model. Specifically, under the assumption on the persistency of excitation, any finite trajectory of an LTI system lives in the column space of a Hankel matrix constructed from the recorded input-output data. 
Beyond the deterministic LTI setting, there are some variants of the lemma, e.g., extensions to nonlinear systems \citep{alsalti21data,lian21koopman}, to linear parameter-varying systems \citep{verhoek21data}, to kernel representation of nonlinear systems \citep{molodchyk24exploring}. Especially, some recent works have extended the fundamental lemma towards stochastic predictive control schemes, see \citet{kerz23data} for a tube-based approach and \citet{teutsch24sampling} for a sampling-based approach. We refer to \citet{markovsky21behavioral} for an overview and refer to \citet{willems13open} for early discussions of behavioral concepts for stochastic systems.

In previous work, we proposed a stochastic variant of the fundamental lemma for stochastic LTI system subject to process disturbances in the framework of Polynomial Chaos Expansions (PCE) \citep{pan23stochastic}.
PCE was proposed by \citet{wiener38homogeneous} and has been generalized to a wide class of non-Gaussian distributions by \citet{xiu02wiener}. The core idea of PCE is that one can use any the basis of an $\lsp$  Hilbert space to model any square integrable random variable via a series expansion. Therefore, $\lsp$ random variables can be parameterized linearly by deterministic coefficients in appropriately chosen polynomial bases. We refer to \citet{sullivan15introduction} for a general introduction and to \citet{kim13wiener} for an early overview on control design using PCE. PCE was first introduced in stochastic optimal control to foster computation \citep{fagiano12nonlinear,paulson14fast}, it also has prospects for system theory and analysis \citep{paulson15stability, ahbe20region, faulwasser23behavioral}.

The stochastic fundamental lemma in \citet{pan23stochastic} requires the measurements of the past disturbances to construct the Hankel matrices. As such measurements are, in general, costly, the main goal of this paper is to obtain a modified stochastic fundamental lemma that does not require any past disturbance data. Using causality properties, we show that a stochastic LTI system can be decomposed into many decoupled stochastic subsystems, each of which captures the effect of one source of stochastic uncertainty and how it propagates via the dynamics. In each subsystem, we convert the additive process disturbance into an output initial condition under the causality conditions such that the subsystem does not contain any disturbances and then reformulate the subsystems into PCE coefficients. This way, we derive corresponding data-driven representations of the PCE reformulated subsystems without disturbance data in the Hankel matrices under mild conditions. Moreover, we provide a modified stochastic fundamental lemma in random variables based on the results in PCE coefficients. Drawing upon a example, we demonstrate the numerical advantages of the modified approach compared to the stochastic fundamental lemma for data-driven stochastic Optimal Control Problems (OCP).

The remainder is structured as follows: In Section~\ref{sec:Preliminaries} we recall the setting and problem statement, while in Section~\ref{sec:MainResults} we present the modified stochastic fundamental lemma without past disturbance data. Section~\ref{sec:Simulation} illustrates a numerical example. The paper ends with conclusions in Section~\ref{sec:Conclusion}.

\subsubsection*{Notation}
Let $(\Omega,\mathcal F,\mu)$ be a probability space, where $\Omega$ is the set of possible outcomes, $\mathcal{F}$ is a $\sigma$-algebra, and $\mu$ is the considered probability measure.
$\splx{n_z}$  is the space of vector-valued random variables of finite covariance and of dimension $n_z$.
Let $Z: \I_{[0,T-1]} \rightarrow \splx{n_z}$ be a sequence of vector-valued random variables from time instant 0 to $T-1$. We denote by $\mean[Z]$ and $z\doteq Z(\omega) \in \R^{n_z}$ its mean and realization, respectively. The vectorizations of $z$ and $Z$ are written as $z_{[0,T-1]} \doteq [z_0^\top,z_1^\top, \dots,z_{T-1}^\top]^\top \in \R^{n_z T}$ and $Z_{[0,T-1]}$, respectively.
We denote the identity matrix of size $n$ by $I_n$.
For any matrix $Q\in \mbb{R}^{n\times m}$ with columns $q^1, \dots, q^m$, the column-space is denoted by $\mathrm{colsp}(Q) \doteq \mathrm{span}\left(\{q^1,\dots, q^m\}\right)$.
