\section{Numerical example} \label{sec:Simulation}
\begin{figure}[t]
	\begin{subfigure}{\linewidth}
		\centering
		\includegraphics[width=0.85\linewidth,trim={36mm 16mm 22mm 30mm},clip]{Figures/AircraftPDFY1NG.pdf}
	\end{subfigure}
	\begin{subfigure}{\linewidth}
		\centering
		\includegraphics[width=0.85\linewidth,trim={36mm 16mm 22mm 30mm},clip]{Figures/AircraftPDFY2NG.pdf}
	\end{subfigure}
	\caption{Evolution of the PDFs of the outputs $Y^1$ and $Y^2$ over horizon $N=25$. Deep blue-dashed line: Chance constraint.}
	\label{fig:AircraftY}
\end{figure}
\begin{figure}[t]
	\begin{center}
		\includegraphics[width=1\linewidth]{Figures/AircraftComparisonNG.pdf}
		\caption{Aircraft example with Gaussian disturbance. Red-solid line: Scheme~I; blue-solid line with circle marker: Scheme~II; black-dashed line: Scheme~III; deep blue-dashed line: Chance constraint.} \label{fig:AircraftComparison}		
	\end{center}
\end{figure}
We consider a discrete-time LTI aircraft model from \citet{pan24data}. The VARX matrices in~\eqref{eq:VARX} are
\begin{align*}
	\hat{A} &= \left[\begin{smallmatrix*}[r]
		-\phantom{0}0.201&	\phantom{-0}0.256&	\phantom{-0}0.050	&\phantom{-0}0.160&	\quad-\phantom{0}0.256&	\quad 0.086\\
		-\phantom{0}4.773&	\phantom{-0}3.688&	\phantom{-0}0.650&	\phantom{-0}2.982&	\quad-\phantom{0}2.688&	\quad 1.707\\
		-15.746&	\phantom{-}12.898&	\phantom{-0}2.319	&\phantom{-}10.461	&\quad-12.897	&\quad 5.171
	\end{smallmatrix*}\right],\\
	\hat{B} &= \left[\begin{smallmatrix*}[r] -\phantom{0}0.019	&\quad -\phantom{0}1.440\\
		\phantom{-0}0.711&\quad	-\phantom{0}1.800\\
		\phantom{-0}1.444&\quad	-26.922
	\end{smallmatrix*}\right].
\end{align*}
The elements of $W_k$, $k\in \N$ are independently and uniformly distributed with $W^1\sim\mcl{U}(-0.1,0.1)$, $W^2\sim\mcl{U}(-3,3)$, and $W^3\sim\mcl{U}(-0.8,0.8)$, where $W^i$ denotes the $i$-th element of $W$. Then we solve the following OCP
\begin{align*}
    &\min_{U_k, Y_k,G^{w_k}, g, k\in\I_{[1,N]}}
    \sum_{k=1}^N\mean[Y_k^\top QY_k + U_k^\top R U_k]\\
    \text{s.t.}\quad &\text{dynamics in form of Lemma~\ref{lem:PropRV}},\\
    &\mbb{P}[-0.349\leq Y_k^1\leq 0.349]\geq 0.8, k\in\I_{[2,N]}
\end{align*}
in the PCE framework. The weighting matrices in the stage cost are $Q=\text{diag}([1,1,1])$ and $R=1$. A conservative reformulation of the chance constraint imposed on $Y_k^2$ individually for $k\in\I_{[2,N]}$ is $-0.349\leq \mean[Y_k^1] \pm 3\sqrt{\mbb{V}[Y_k^1]}\leq 0.349$ \citep{calafiore06distributionally}. Note that the chance constaint is not imposed on $Y_1^2$ since $Y_1$ is fixed once we have the measured initial trajectory $\tra{(\tilde{u},\tilde{y})}{1-\tini}{0}$. The computations are done on a virtual machine with an AMD EPYC Processor with 2.8 GHz, 32 GB of RAM in \texttt{julia} using \texttt{IPOPT} \citep{waechter06implementation}.

First we solve the above stochastic OCP with prediction horizon $N=25$ and a randomly sampled initial condition around $[0,-100,0]^\top$. The evolution of the Probability Density Functions (PDF) of the system outputs $Y^1$ and $Y^2$ for the computed optimal input is depicted in Figure~\ref{fig:AircraftY}. As one can see, the chance constraint for $Y^1$ is satisfied with a high probability, while $Y^2$ shows a narrow distribution close to 0 at the end of the horizon.

Then we compare the following three schemes in closed loop with prediction horizon $N=10$ at each time step:
\begin{itemize}
	\item[I)] Lemma~\ref{lem:PropRV}with data $(u,y)^{\ud}$ of system~\eqref{eq:VARXFree},
	\item[II)] Lemma~\ref{lem:StochFundam} with data $(u,y,w)^{\da}$ of system~\eqref{eq:VARXReal},
	\item[III)] Lemma~\ref{lem:PropRV} with data $(u,y)^{\da}$ of system~\eqref{eq:VARXReal}.
\end{itemize}
Note that to apply Scheme III) using data of~\eqref{eq:VARXReal}, we first estimate the disturbance realizations and then generate a corresponding undisturbed trajectory $(u,y)^{\ud}$ of~\eqref{eq:VARXFree} via~\eqref{eq:PredictorModify}. We use the recorded/generated data of length 90 to construct all Hankel matrices. Given an initial condition around $[0,-100,0]^\top$ and the same disturbance realizations, we compute the closed-loop trajectories for 30 time steps for Scheme~I-III, see Figure~\ref{fig:AircraftComparison}. For a detailed closed-loop algorithm in the data-driven setting we refer to Algorithm~1 of~\citet{pan23stochastic}.

We observe that the input-output trajectories for Scheme~I and Scheme~II are identical with a maximal difference of $1.325\cdot 10^{-4}$, while Scheme~III results in a slightly different trajectory due to the estimation error of disturbance realizations. we sample 1000 sequences of initial condition and disturbance realizations and summarize the computation time in Table~\ref{tab:AircraftComparisonTime}, where SD refers to standard deviation. One can see that Scheme~I saves 72.9\% computation time in comparison to Scheme~II with a smaller standard deviation, while the average closed-loop stage cost $J^{\text{cl}}$ keeps the same. Moreover, the performance of Scheme~I and Scheme~III report similar results in terms of the computation time, while Scheme~III has a slightly higher average cost due to the estimation error of disturbances. Importantly, by using Lemma~\ref{lem:PropRV} in Scheme~I), one can save 63.6\% amount of data in the Hankel matrix in this simulation example compared to Scheme~II without loss of performance.

\begin{table}[t!]
	\caption{Comparison of the data amount in Hankel matrices and the computation time for 1000 samplings.}
	\label{tab:AircraftComparisonTime}
	\centering
	\begin{adjustbox}{width=0.95\columnwidth,center}
		\begin{tabular}{ccccc}
			\toprule
			\multirow{2}{*}{\shortstack{\\ \\ Data-driven\\ scheme}} &  \multirow{2}{*}{\shortstack{\\ \\ Non-zero entries\\ in $\Hankel$}}&\multicolumn{2}{c}{Computation time}  & \multirow{2}{*}{$J^\text{cl}$ $[-]$}\\
			 \cmidrule(lr){3-4} & & Mean $\SI{}{[s]}$ & SD $\SI{}{[s]}$   &\\
			\midrule
			I & 74892 &0.333 & 0.029  & $2.812\times 10^{3}$ \\
			II & 205716 & 1.232 & 0.534  & $2.812\times 10^{3}$\\
			III & 74892 & 0.270 & 0.063 & $2.894\times 10^{3}$\\
			\bottomrule
		\end{tabular}
	\end{adjustbox}
\end{table}