\section{Preliminaries} \label{sec:Preliminaries}
We consider stochastic discrete-time LTI systems
\begin{subequations} \label{eq:Dyn}
	\begin{align}
		X_{k+1} &= AX_k + BU_k + EW_k,\quad X_0=\tilde{X}_0,\\
		Y_k &= CX_k\label{eq:Output}
	\end{align}
\end{subequations}
with state $X_k \in\splk{k}{n_x}$, $X_k:\Omega\to\R^{n_x}$ and process disturbance $W_k \in \splx{n_w}$. Throughout the paper, the process disturbances $W_k$, $k\in\N$ are \textit{i.i.d.} random variables and the probability distribution of $W_k$ is assumed to be known. For approaches how to model distributional uncertainty with PCE we refer to \citet{pan23distributionally}. Moreover, measurement noise is not included in~\eqref{eq:Dyn}, i.e., the measurements of the output are exact.

In the filtered probability space $(\Omega, \mcl F, (\mcl F_k)_{k\in \N}, \mu)$, $(\mcl F_k)_{k\in \N}$ is the smallest filtration that the stochastic process $X$ is adapted to. That is, $\mcl F_k = \sigma(X_i,i\leq k)$, where $\sigma(X_i,i\leq k)$ denotes the $\sigma$-algebra generated by $X_i,i\leq k$. Moreover, $\mcl F$ is the $\sigma$-algebra that contains all available historical information, i.e., $\mcl F_0 \subseteq \mcl F_1 \subseteq ...  \subseteq \mcl F$. We model $Y_k$ and $U_k$, $k\in\N$ as stochastic processes adapted to the filtration $(\mathcal{F}_k)_k$, i.e. $Y_k \in\splk{k}{\dimy}$ and $U_k \in\splk{k}{\dimu}$. This immediately imposes a causality constraint on $Z_k$, $Z\in\{U,X,Y\}$, i.e., $Z_k$ only depends on past disturbances $W_j$, $j< k$. It directly follows from Lemma 1.14 by \citet{kallenberg97foundations} that inputs $U_k$, $k\in\N$ adapted to the filtration $\mcl F_k$ are equivalent to feedback polices based on historical state information. For more details on filtrations we refer to \citet{fristedt13modern}. For the sake of readability, the space $\splk{k}{n_z}$ is written as $\lsp_{k}(\R^\dimz)$.

For a specific outcome $\omega \in \Omega$, the realization $W_k(\omega)$ is denoted as $w_k$. Likewise, we write $u_k \coloneqq U_k(\omega)$, $x_k \coloneqq X_k(\omega)$, and $y_k \coloneqq Y_k(\omega)$. Given a specific initial condition $\tilde{x}_0$ and a sequence of disturbance realizations $w_k$, the realization dynamics induced by \eqref{eq:Dyn} are
\begin{subequations} \label{eq:DynReal}
	\begin{align} 
		x_{k+1} &= Ax_k + Bu_k +  Ew_k,\quad x_0=\tilde{x}_0,\\
		y_k &= Cx_k.
	\end{align}
\end{subequations}

\begin{assum}[System properties] \label{ass:Sys}
	\quad We assume $(A,B)$ controllable and $(A,C)$ observable in system~\eqref{eq:DynReal}.
\end{assum}
If system~\eqref{eq:DynReal} is observable, the lag $\ell$ is the minimum integer such that the observability matrix $\mcl{O}_{\ell}\coloneqq [C^\top, (CA)^\top,...,(CA^{\ell-1})^\top]^\top$ is of full column rank, i.e., $\text{rank}(\mcl{O}_{\ell})$ $=n_x$. Thus, for any $\tini\in\N$ satisfying $\tini\geq\ell$, $\text{rank}(\mcl{O}_{\tini})$ $=n_x$ holds.

\subsection{Polynomial Chaos Expansion}
The key idea of PCE is that any $\lsp$ random variable can be expressed in a suitable (orthogonal) polynomial basis. Consider an orthogonal polynomial basis $\{\phi^j(\xi)\}_{j=0}^{\infty}$ that spans the space $\mcl{L}^2(\Xi, \mathcal{F}, \mu; \mathbb{R})$, i.e.,
\begin{equation} \label{eq:Orthogonality}
	\langle \phi^i(\xi),\phi^j(\xi) \rangle {=} \int_{\Xi} \phi^i(\xi) \phi^j(\xi) \diff \mu(\xi) {=} \delta^{ij}\langle\phi^j(\xi) \rangle^2,
\end{equation}
where $\langle \phi^j(\xi) \rangle^2\coloneqq \langle \phi^j(\xi),\phi^j(\xi) \rangle$ and $\delta^{ij}$ is the Kronecker delta. Note that $\xi\in \lsp(\R^{n_\xi})$ is the stochastic argument of polynomial functions $\phi^j$, and $\Xi$ is the sample space of~$\xi$. The first polynomial is always chosen to be $\phi^0(\xi) = 1$. Hence, the orthogonality~\eqref{eq:Orthogonality} gives that for all other basis dimensions $j>0$, we have $\mean[\phi^j(\xi)]=\int_{\Xi} \phi^j(\xi) \diff \mu(\xi) = \langle \phi^j(\xi),\phi^0(\xi)\rangle=0$.

The PCE of a real-valued random variable $Z \in  \lsp(\R)$ with respect to the basis $\{\phi^j(\xi)\}_{j=0}^{\infty}$ is 
\[
Z(\omega) = \sum_{j=0}^{\infty}\pce{z}^j \phi^j(\xi(\omega)) \quad \text{with} \quad \pce{z}^j = \frac{\big\langle Z(\omega), \phi^j(\xi(\omega)) \big\rangle}{\langle\phi^j(\xi) \rangle^2},
\]
where $\pce{z}^j\in \R$ is referred to as the $j$-th PCE coefficient. In the PCE expression, $\xi:\Omega\to\Xi$ is the stochastic argument of the polynomials $\phi^j$ and thus is viewed as a function of the outcome $\omega$. For the sake of readability, we omit the arguments $\xi(\omega)$, $\omega$ whenever there is no ambiguity. The PCE of a vector-valued random variable $Z\in\lsp(\R^\dimz)$ follows by applying PCE component-wise, i.e., the $j$-th PCE coefficient of $Z$ reads
$\pce{z}^{j} =\begin{bmatrix} \pce{z}^{1,j} & \pce{z}^{2,j} & \cdots & \pce{z}^{n_z,j} \end{bmatrix}^\top$, where $\pce{z}^{i,j}$ is the $j$-th PCE coefficient of $i$-th component $Z^i$, $\forall i\in\I_{[1,n_z]}$. Moreover, as the first basis function is always chosen to be 1, i.e. $\phi^0=1$, we have $\pce{z}^0=\mean[W]$ from the definition of PCE.

In numerical implementations the expansions have to be truncated after a finite number of terms. This may lead to truncation errors
\[
	\Delta Z(L) = Z -  \sum\nolimits_{j=0}^{L-1}\pce{z}^j \phi^j,
\]
where $L\in\N^\infty\coloneqq \N^+\cup\{\infty\}$ is the PCE dimension. For $L\to\infty$, the truncation error satisfies $\lim_{L\to\infty}\|\Delta Z(L)||=0$ \citep{cameron47orthogonal, ernst12convergence}. 

\begin{defn}[Exact PCE representation] \label{def:ExactPCE}
	The PCE of a random variable $Z \in \lsp(\R^\dimz)$ is said to be exact with dimension $L\in\N^\infty$ if $ Z -  \sum_{j=0}^{L-1}\pce{z}^j \phi^j=0$.
\end{defn}

\begin{rem}[Generic affine PCE series ] 
	Given an $\mcl{L}^2$ random variable with known distribution, the key to construct an exact finite-dimensional PCE is the appropriate choice of basis functions. For some widely used distributions, the appropriate choice of polynomial bases is summarized in \citet{xiu02wiener}. Additionally, a generic (non-orthonormal but orthogonal) basis choice for any random variable $Z \in \lsp(\R)$ is $\phi^0 = 1$ and $\phi^1 = Z-\mean[Z]$, which implies the exact and finite PCE $\pce{z}^0 = \mean[Z]$ and $\pce{z}^1 = 1$.
\end{rem}

Replacing all random variables in~\eqref{eq:Dyn} with corresponding PCE representations with respect to the basis $\{\phi^j\}_{j=0}^{\infty}$, we get
	\begin{equation*}
		\sum\nolimits_{j=0}^\infty \pce{x}_{k+1}^j \phi^j= \sum\nolimits_{j=0}^\infty (A\pce{x}_k^j + B\pce{u}_k^j + E\pce{w}_k^j)\phi^j.
	\end{equation*}
Then for all $j\in\N^\infty$, performing Galerkin projection onto the basis functions $\phi^j$, one obtains the dynamics of the PCE coefficients
\begin{subequations} \label{eq:DynPCE}
	\begin{align}	
		\pce{x}_{k+1}^j &= A\pce{x}_k^j + B\pce{u}_k^j + E\pce{w}_k^j, \quad \pce{x}_0^j=\tilde{\pce{x}}_0^j,\\
		\pce{y}_k^j &= C\pce{x}_k^j.
	\end{align}
\end{subequations}
Note that dynamics~\eqref{eq:Dyn}, \eqref{eq:DynReal} and \eqref{eq:DynPCE} have the identical system matrices as the Galerkin projection preserves the linearity of the system. For details about Galerkin projection we refer to Appendix~A of \citet{pan23stochastic}.

\subsection{Stochastic Fundamental Lemma}
Data-driven system representation based on \citet{willems05note} requires persistently exciting input data $\tra{u}{0}{T-1}\coloneqq [u_0^\top,u_1^\top,...,u_{T-1}^\top]^\top$.
\begin{defn}[Persistency of excitation] Let $T$, $N \in \N^+$. A sequence of real-valued data $\tra{u}{0}{T-1}$ is said to be persistently exciting of order $N$ if the Hankel matrix
	\begin{equation*}
		\Hankel_N(\tra{u}{0}{T-1}) \doteq \begin{bmatrix}
			u_0   &\cdots& u_{T-N} \\
			\vdots & \ddots & \vdots \\
			u_{N-1}& \cdots  & u_{T-1} \\
		\end{bmatrix}
	\end{equation*}
	is of full row rank.
\end{defn}

As we are interested in the stochastic LTI system~\eqref{eq:Dyn}, one could attempt using a persistently exciting input sequence $\tra{U}{0}{T-1}$ of $\mcl{L}^2$ random variables to represent the stochastic LTI system \eqref{eq:Dyn}.
We have shown in \citet{pan23stochastic} that input-output-disturbance realization trajectories $\trad{(u,y,w)}{0}{T-1}$, where the superscript $\cdot^{\da}$ denotes offline recorded data of the disturbed system~\eqref{eq:DynReal}, suffices to represent \eqref{eq:Dyn}.

\begin{lem}[Stochastic fundamental lemma]\label{lem:StochFundam}~
    Consider system~\eqref{eq:Dyn} and a $T$-length realization trajectory tuple $\trad{(u,y,w)}{0}{T-1}$ of realization dynamics~\eqref{eq:DynReal}. We assume that system~\eqref{eq:DynReal} is controllable and let $\trad{(u,w)}{0}{T-1}$ be persistently exciting of order $N+\dimx$. Then $ \tra{(U,Y,W)}{0}{N-1}$ is a trajectory of \eqref{eq:Dyn} of length $N$ if and only if there exists $G \in \splx{T-N+1} $ such that 
	\begin{equation*}
		\Hankel_N(\trad{z}{0}{T-1}) G=Z_{[0,N-1]}
	\end{equation*} 
	holds for all $(z, Z)\in \{ (u, U), (y, Y), (w, W)\}$.
\end{lem}

\begin{cor}\label{coro:StochFundamPCE}
	Let the conditions of Lemma~\ref{lem:StochFundam} hold. Then $ \tra{(\pce{y}^j, \pce{u}^j, \pce{w}^j)}{0}{N-1}$, $j \in \N^\infty$ is a trajectory of the dynamics of PCE coefficients \eqref{eq:DynPCE} if and only if there exists $\pce{g}^j \in \R^{T-N+1}$ such that 
	\begin{equation*}
		\Hankel_t(\trad{z}{0}{T-1}) \pce{g}^j =\pce{z}^j_{[0,N-1]},~j\in \N^\infty
	\end{equation*} 
	holds for all $(z, \pce{z})\in \{(u,\pce{u}), (y,\pce{y}), (w, \pce{w})\}$.
\end{cor}
In the image representation of a fundamental lemma, the future trajectory of an LTI system lives in the column space the Hankel matrices consisting of the past trajectory data. In \citet{pan23stochastic}, however, we have shown that Hankel matrices in random variables cause conceptual issues and the corresponding fundamental lemma does not necessarily hold. Instead, one can predict the future trajectory of the stochastic LTI system~\eqref{eq:Dyn} from the data of the system in realizations~\eqref{eq:DynReal}.