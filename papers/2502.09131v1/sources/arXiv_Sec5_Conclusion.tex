\section{Conclusion} \label{sec:Conclusion}
This paper has proposed a stochastic variant of the fundamental lemma towards stochastic LTI systems with process disturbances in the PCE framework. Based on the superposition principle, the stochastic LTI system is decoupled into a number of subsystems, each of which corresponds to a source of uncertainty, e.g. a disturbance at certain time step. Then given the causality constraints on the inputs and outputs, the additive disturbances can be converted into output initial conditions of the decoupled subsystems. This way, the dynamics of the subsystems do not contain any disturbances except for the initial conditions. Therefore, in contrast to our previous work, i.e. Lemma~\ref{lem:StochFundam}, this variant of the fundamental lemma does not requirement any disturbance data in Hankel matrices. As by-products of this variant, we have shorten the prediction horizon for the PCE coefficients related to the disturbances, which leads to an acceleration in numerical implementations. Future work will consider disturbance estimator with error bounds, extension to nonlinear systems, and fast computation of data-driven stochastic optimal control.