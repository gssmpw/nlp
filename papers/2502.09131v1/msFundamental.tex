\documentclass[final,5p,times,twocolumn]{elsarticle}

\usepackage[T1]{fontenc}
\usepackage{soul}

\usepackage{siunitx}
\usepackage{slashed}
\usepackage{enumitem}
\usepackage{xcolor, color}


\usepackage{multirow}
\usepackage{nccmath}
\usepackage{cite}
\usepackage{amsfonts}
\usepackage{amsmath}
\usepackage{amssymb}
\usepackage{mathtools}
\usepackage{mathrsfs}
\usepackage{graphicx}
\usepackage{subcaption}
\usepackage{algpseudocode}
\usepackage{booktabs}
\usepackage{adjustbox}
\usepackage{xparse}

\usepackage{bm}

\NewDocumentCommand{\basis}{om}{%
	\IfNoValueTF{#1}
	{\| \Phi^{#2}\|^2}
	{\langle \Phi^{#1},\phi^{#2} \rangle}%
}

\newcommand{\da}{\mathrm{d}}
\newcommand{\ud}{\neg\mathrm{d}}
\newcommand{\ad}{\mathrm{ad}}
\newcommand{\au}{\mathrm{a}}

\newcommand{\rank}{\mathrm{rank}}
\newcommand{\ik}[1]{\mathcal{I}_{#1}^w}
\newcommand{\pce}[1]{\mathsf{#1}}
\newcommand{\tra}[3]{#1_{[#2,#3]}}
\newcommand{\trad}[3]{#1_{[#2,#3]}^{\mathrm{d}}}
\newcommand{\traud}[3]{#1_{[#2,#3]}^{\neg\mathrm{d}}}
\newcommand{\traad}[3]{#1_{[#2,#3]}^{\mathrm{ad}}}

\newcommand{\nom}{\text{nom}}
\newcommand{\ini}{\text{ini}}
\newcommand{\tini}{T_\text{ini}}
\newcommand{\YUini}{(\hat{Y},\hat{U})_{[-\tini,-1]}}
\newcommand{\yuini}{(\hat{y},\hat{u})_{[-\tini,-1]}}
\newcommand{\Yini}{\hat{Y}_{[-\tini,-1]}}
\newcommand{\yini}{\hat{y}_{[-\tini,-1]}}
\newcommand{\Uini}{\hat{U}_{[-\tini,-1]}}
\newcommand{\uini}{\hat{u}_{[-\tini,-1]}}
\newcommand{\ypceini}{\hat{\pce{y}}_{[-\tini,-1]}}

\DeclareMathOperator{\tr}{tr}

\newcommand{\mbb}[1]{\mathbb{#1}}
\newcommand{\mbf}[1]{\mathbf{#1}} 
\newcommand{\mcl}[1]{\mathcal{#1}}


\newcommand{\splk}[2]{\mcl{L}^2(\Omega, \mathcal{F}_{#1}, \mu; \mathbb{R}^{#2})}
\newcommand{\splx}[1]{\mcl{L}^2(\Omega, \mathcal{F}, \mu; \mathbb{R}^{#1})}
\newcommand{\lsp}{\mcl{L}^2} 

\newcommand{\diff}{\mathop{}\!\mathrm{d}}

\newcommand{\mean}{\mbb{E}}
\newcommand{\var}{\mbb{V}}
\newcommand{\std}[2]{\sigma_{#1}^{#2}}


\newcommand{\dimy}{{n_y}}
\newcommand{\dimx}{{n_x}}
\newcommand{\dimu}{{n_u}}
\newcommand{\dimw}{{n_w}}
\newcommand{\dimz}{{n_z}}

\newcommand{\I}{\mathbb{I}}
\newcommand{\N}{\mathbb{N}}
\newcommand{\R}{\mathbb{R}}
\newcommand{\X}{\mathbb{X}}
\newcommand{\U}{\mathbb{U}}

\newcommand{\Hankel}{\mcl{H}}

\newcommand{\End}{\hfill $\square$}

\usepackage{amsthm}
\newtheorem{cor}{Corollary}
\newtheorem{lem}{Lemma}
\newtheorem{thm}{Theorem}
\newtheorem{defn}{Definition}
\newtheorem{rem}{Remark}
\newtheorem{assum}{Assumption}
\newtheorem{prop}{Proposition}
\newtheorem{exmp}{Example}

\journal{Journal of \LaTeX\ Templates}
\bibliographystyle{model5}\biboptions{authoryear}

\begin{document}
\begin{frontmatter}
\title{A Stochastic Fundamental Lemma with Reduced Disturbance Data Requirements}

\author{Ruchuan Ou}\ead{ruchuan.ou@tuhh.de}
\author{Guanru Pan}\ead{guanru.pan@tuhh.de}  

\author{Timm Faulwasser\corref{mycorrespondingauthor}}
\cortext[mycorrespondingauthor]{Corresponding author}
\ead{timm.faulwasser@ieee.org}

\address{Institute of Control Systems, Hamburg University of Technology, 21079 Hamburg, Germany}                                        

\begin{keyword}                            
Fundamental lemma, stochastic systems, polynomial chaos, reduced disturbance data, non-Gaussian uncertainties, data-driven control
\end{keyword}                           


\begin{abstract}
Recently, the fundamental lemma by Willems et. al has been extended towards stochastic LTI systems subject to process disturbances. Using this lemma requires previously recorded data of inputs, outputs, and disturbances. In this paper, we exploit causality concepts of stochastic control to propose a variant of the stochastic fundamental lemma that does not require past disturbance data in the Hankel matrices. Our developments rely on polynomial chaos expansions and on the knowledge of the disturbance distribution. Similar to our previous results, the proposed variant of the fundamental lemma allows to predict future input-output trajectories of stochastic LTI systems. We draw upon a numerical example to illustrate the proposed variant in data-driven control context.
\end{abstract}
\end{frontmatter}

\section{Introduction} \label{sec:Introduction}
Data-driven system representations based on the fundamental lemma by \citet{willems05note} are of continued and increasing research interest. The main insight is that the trajectories of any controllable Linear Time Invariant (LTI) systems can be characterized without explicit identification of a state-space model. Specifically, under the assumption on the persistency of excitation, any finite trajectory of an LTI system lives in the column space of a Hankel matrix constructed from the recorded input-output data. 
Beyond the deterministic LTI setting, there are some variants of the lemma, e.g., extensions to nonlinear systems \citep{alsalti21data,lian21koopman}, to linear parameter-varying systems \citep{verhoek21data}, to kernel representation of nonlinear systems \citep{molodchyk24exploring}. Especially, some recent works have extended the fundamental lemma towards stochastic predictive control schemes, see \citet{kerz23data} for a tube-based approach and \citet{teutsch24sampling} for a sampling-based approach. We refer to \citet{markovsky21behavioral} for an overview and refer to \citet{willems13open} for early discussions of behavioral concepts for stochastic systems.

In previous work, we proposed a stochastic variant of the fundamental lemma for stochastic LTI system subject to process disturbances in the framework of Polynomial Chaos Expansions (PCE) \citep{pan23stochastic}.
PCE was proposed by \citet{wiener38homogeneous} and has been generalized to a wide class of non-Gaussian distributions by \citet{xiu02wiener}. The core idea of PCE is that one can use any the basis of an $\lsp$  Hilbert space to model any square integrable random variable via a series expansion. Therefore, $\lsp$ random variables can be parameterized linearly by deterministic coefficients in appropriately chosen polynomial bases. We refer to \citet{sullivan15introduction} for a general introduction and to \citet{kim13wiener} for an early overview on control design using PCE. PCE was first introduced in stochastic optimal control to foster computation \citep{fagiano12nonlinear,paulson14fast}, it also has prospects for system theory and analysis \citep{paulson15stability, ahbe20region, faulwasser23behavioral}.

The stochastic fundamental lemma in \citet{pan23stochastic} requires the measurements of the past disturbances to construct the Hankel matrices. As such measurements are, in general, costly, the main goal of this paper is to obtain a modified stochastic fundamental lemma that does not require any past disturbance data. Using causality properties, we show that a stochastic LTI system can be decomposed into many decoupled stochastic subsystems, each of which captures the effect of one source of stochastic uncertainty and how it propagates via the dynamics. In each subsystem, we convert the additive process disturbance into an output initial condition under the causality conditions such that the subsystem does not contain any disturbances and then reformulate the subsystems into PCE coefficients. This way, we derive corresponding data-driven representations of the PCE reformulated subsystems without disturbance data in the Hankel matrices under mild conditions. Moreover, we provide a modified stochastic fundamental lemma in random variables based on the results in PCE coefficients. Drawing upon a example, we demonstrate the numerical advantages of the modified approach compared to the stochastic fundamental lemma for data-driven stochastic Optimal Control Problems (OCP).

The remainder is structured as follows: In Section~\ref{sec:Preliminaries} we recall the setting and problem statement, while in Section~\ref{sec:MainResults} we present the modified stochastic fundamental lemma without past disturbance data. Section~\ref{sec:Simulation} illustrates a numerical example. The paper ends with conclusions in Section~\ref{sec:Conclusion}.

\subsubsection*{Notation}
Let $(\Omega,\mathcal F,\mu)$ be a probability space, where $\Omega$ is the set of possible outcomes, $\mathcal{F}$ is a $\sigma$-algebra, and $\mu$ is the considered probability measure.
$\splx{n_z}$  is the space of vector-valued random variables of finite covariance and of dimension $n_z$.
Let $Z: \I_{[0,T-1]} \rightarrow \splx{n_z}$ be a sequence of vector-valued random variables from time instant 0 to $T-1$. We denote by $\mean[Z]$ and $z\doteq Z(\omega) \in \R^{n_z}$ its mean and realization, respectively. The vectorizations of $z$ and $Z$ are written as $z_{[0,T-1]} \doteq [z_0^\top,z_1^\top, \dots,z_{T-1}^\top]^\top \in \R^{n_z T}$ and $Z_{[0,T-1]}$, respectively.
We denote the identity matrix of size $n$ by $I_n$.
For any matrix $Q\in \mbb{R}^{n\times m}$ with columns $q^1, \dots, q^m$, the column-space is denoted by $\mathrm{colsp}(Q) \doteq \mathrm{span}\left(\{q^1,\dots, q^m\}\right)$.

\section{Preliminaries} \label{sec:Preliminaries}
We consider stochastic discrete-time LTI systems
\begin{subequations} \label{eq:Dyn}
	\begin{align}
		X_{k+1} &= AX_k + BU_k + EW_k,\quad X_0=\tilde{X}_0,\\
		Y_k &= CX_k\label{eq:Output}
	\end{align}
\end{subequations}
with state $X_k \in\splk{k}{n_x}$, $X_k:\Omega\to\R^{n_x}$ and process disturbance $W_k \in \splx{n_w}$. Throughout the paper, the process disturbances $W_k$, $k\in\N$ are \textit{i.i.d.} random variables and the probability distribution of $W_k$ is assumed to be known. For approaches how to model distributional uncertainty with PCE we refer to \citet{pan23distributionally}. Moreover, measurement noise is not included in~\eqref{eq:Dyn}, i.e., the measurements of the output are exact.

In the filtered probability space $(\Omega, \mcl F, (\mcl F_k)_{k\in \N}, \mu)$, $(\mcl F_k)_{k\in \N}$ is the smallest filtration that the stochastic process $X$ is adapted to. That is, $\mcl F_k = \sigma(X_i,i\leq k)$, where $\sigma(X_i,i\leq k)$ denotes the $\sigma$-algebra generated by $X_i,i\leq k$. Moreover, $\mcl F$ is the $\sigma$-algebra that contains all available historical information, i.e., $\mcl F_0 \subseteq \mcl F_1 \subseteq ...  \subseteq \mcl F$. We model $Y_k$ and $U_k$, $k\in\N$ as stochastic processes adapted to the filtration $(\mathcal{F}_k)_k$, i.e. $Y_k \in\splk{k}{\dimy}$ and $U_k \in\splk{k}{\dimu}$. This immediately imposes a causality constraint on $Z_k$, $Z\in\{U,X,Y\}$, i.e., $Z_k$ only depends on past disturbances $W_j$, $j< k$. It directly follows from Lemma 1.14 by \citet{kallenberg97foundations} that inputs $U_k$, $k\in\N$ adapted to the filtration $\mcl F_k$ are equivalent to feedback polices based on historical state information. For more details on filtrations we refer to \citet{fristedt13modern}. For the sake of readability, the space $\splk{k}{n_z}$ is written as $\lsp_{k}(\R^\dimz)$.

For a specific outcome $\omega \in \Omega$, the realization $W_k(\omega)$ is denoted as $w_k$. Likewise, we write $u_k \coloneqq U_k(\omega)$, $x_k \coloneqq X_k(\omega)$, and $y_k \coloneqq Y_k(\omega)$. Given a specific initial condition $\tilde{x}_0$ and a sequence of disturbance realizations $w_k$, the realization dynamics induced by \eqref{eq:Dyn} are
\begin{subequations} \label{eq:DynReal}
	\begin{align} 
		x_{k+1} &= Ax_k + Bu_k +  Ew_k,\quad x_0=\tilde{x}_0,\\
		y_k &= Cx_k.
	\end{align}
\end{subequations}

\begin{assum}[System properties] \label{ass:Sys}
	\quad We assume $(A,B)$ controllable and $(A,C)$ observable in system~\eqref{eq:DynReal}.
\end{assum}
If system~\eqref{eq:DynReal} is observable, the lag $\ell$ is the minimum integer such that the observability matrix $\mcl{O}_{\ell}\coloneqq [C^\top, (CA)^\top,...,(CA^{\ell-1})^\top]^\top$ is of full column rank, i.e., $\text{rank}(\mcl{O}_{\ell})$ $=n_x$. Thus, for any $\tini\in\N$ satisfying $\tini\geq\ell$, $\text{rank}(\mcl{O}_{\tini})$ $=n_x$ holds.

\subsection{Polynomial Chaos Expansion}
The key idea of PCE is that any $\lsp$ random variable can be expressed in a suitable (orthogonal) polynomial basis. Consider an orthogonal polynomial basis $\{\phi^j(\xi)\}_{j=0}^{\infty}$ that spans the space $\mcl{L}^2(\Xi, \mathcal{F}, \mu; \mathbb{R})$, i.e.,
\begin{equation} \label{eq:Orthogonality}
	\langle \phi^i(\xi),\phi^j(\xi) \rangle {=} \int_{\Xi} \phi^i(\xi) \phi^j(\xi) \diff \mu(\xi) {=} \delta^{ij}\langle\phi^j(\xi) \rangle^2,
\end{equation}
where $\langle \phi^j(\xi) \rangle^2\coloneqq \langle \phi^j(\xi),\phi^j(\xi) \rangle$ and $\delta^{ij}$ is the Kronecker delta. Note that $\xi\in \lsp(\R^{n_\xi})$ is the stochastic argument of polynomial functions $\phi^j$, and $\Xi$ is the sample space of~$\xi$. The first polynomial is always chosen to be $\phi^0(\xi) = 1$. Hence, the orthogonality~\eqref{eq:Orthogonality} gives that for all other basis dimensions $j>0$, we have $\mean[\phi^j(\xi)]=\int_{\Xi} \phi^j(\xi) \diff \mu(\xi) = \langle \phi^j(\xi),\phi^0(\xi)\rangle=0$.

The PCE of a real-valued random variable $Z \in  \lsp(\R)$ with respect to the basis $\{\phi^j(\xi)\}_{j=0}^{\infty}$ is 
\[
Z(\omega) = \sum_{j=0}^{\infty}\pce{z}^j \phi^j(\xi(\omega)) \quad \text{with} \quad \pce{z}^j = \frac{\big\langle Z(\omega), \phi^j(\xi(\omega)) \big\rangle}{\langle\phi^j(\xi) \rangle^2},
\]
where $\pce{z}^j\in \R$ is referred to as the $j$-th PCE coefficient. In the PCE expression, $\xi:\Omega\to\Xi$ is the stochastic argument of the polynomials $\phi^j$ and thus is viewed as a function of the outcome $\omega$. For the sake of readability, we omit the arguments $\xi(\omega)$, $\omega$ whenever there is no ambiguity. The PCE of a vector-valued random variable $Z\in\lsp(\R^\dimz)$ follows by applying PCE component-wise, i.e., the $j$-th PCE coefficient of $Z$ reads
$\pce{z}^{j} =\begin{bmatrix} \pce{z}^{1,j} & \pce{z}^{2,j} & \cdots & \pce{z}^{n_z,j} \end{bmatrix}^\top$, where $\pce{z}^{i,j}$ is the $j$-th PCE coefficient of $i$-th component $Z^i$, $\forall i\in\I_{[1,n_z]}$. Moreover, as the first basis function is always chosen to be 1, i.e. $\phi^0=1$, we have $\pce{z}^0=\mean[W]$ from the definition of PCE.

In numerical implementations the expansions have to be truncated after a finite number of terms. This may lead to truncation errors
\[
	\Delta Z(L) = Z -  \sum\nolimits_{j=0}^{L-1}\pce{z}^j \phi^j,
\]
where $L\in\N^\infty\coloneqq \N^+\cup\{\infty\}$ is the PCE dimension. For $L\to\infty$, the truncation error satisfies $\lim_{L\to\infty}\|\Delta Z(L)||=0$ \citep{cameron47orthogonal, ernst12convergence}. 

\begin{defn}[Exact PCE representation] \label{def:ExactPCE}
	The PCE of a random variable $Z \in \lsp(\R^\dimz)$ is said to be exact with dimension $L\in\N^\infty$ if $ Z -  \sum_{j=0}^{L-1}\pce{z}^j \phi^j=0$.
\end{defn}

\begin{rem}[Generic affine PCE series ] 
	Given an $\mcl{L}^2$ random variable with known distribution, the key to construct an exact finite-dimensional PCE is the appropriate choice of basis functions. For some widely used distributions, the appropriate choice of polynomial bases is summarized in \citet{xiu02wiener}. Additionally, a generic (non-orthonormal but orthogonal) basis choice for any random variable $Z \in \lsp(\R)$ is $\phi^0 = 1$ and $\phi^1 = Z-\mean[Z]$, which implies the exact and finite PCE $\pce{z}^0 = \mean[Z]$ and $\pce{z}^1 = 1$.
\end{rem}

Replacing all random variables in~\eqref{eq:Dyn} with corresponding PCE representations with respect to the basis $\{\phi^j\}_{j=0}^{\infty}$, we get
	\begin{equation*}
		\sum\nolimits_{j=0}^\infty \pce{x}_{k+1}^j \phi^j= \sum\nolimits_{j=0}^\infty (A\pce{x}_k^j + B\pce{u}_k^j + E\pce{w}_k^j)\phi^j.
	\end{equation*}
Then for all $j\in\N^\infty$, performing Galerkin projection onto the basis functions $\phi^j$, one obtains the dynamics of the PCE coefficients
\begin{subequations} \label{eq:DynPCE}
	\begin{align}	
		\pce{x}_{k+1}^j &= A\pce{x}_k^j + B\pce{u}_k^j + E\pce{w}_k^j, \quad \pce{x}_0^j=\tilde{\pce{x}}_0^j,\\
		\pce{y}_k^j &= C\pce{x}_k^j.
	\end{align}
\end{subequations}
Note that dynamics~\eqref{eq:Dyn}, \eqref{eq:DynReal} and \eqref{eq:DynPCE} have the identical system matrices as the Galerkin projection preserves the linearity of the system. For details about Galerkin projection we refer to Appendix~A of \citet{pan23stochastic}.

\subsection{Stochastic Fundamental Lemma}
Data-driven system representation based on \citet{willems05note} requires persistently exciting input data $\tra{u}{0}{T-1}\coloneqq [u_0^\top,u_1^\top,...,u_{T-1}^\top]^\top$.
\begin{defn}[Persistency of excitation] Let $T$, $N \in \N^+$. A sequence of real-valued data $\tra{u}{0}{T-1}$ is said to be persistently exciting of order $N$ if the Hankel matrix
	\begin{equation*}
		\Hankel_N(\tra{u}{0}{T-1}) \doteq \begin{bmatrix}
			u_0   &\cdots& u_{T-N} \\
			\vdots & \ddots & \vdots \\
			u_{N-1}& \cdots  & u_{T-1} \\
		\end{bmatrix}
	\end{equation*}
	is of full row rank.
\end{defn}

As we are interested in the stochastic LTI system~\eqref{eq:Dyn}, one could attempt using a persistently exciting input sequence $\tra{U}{0}{T-1}$ of $\mcl{L}^2$ random variables to represent the stochastic LTI system \eqref{eq:Dyn}.
We have shown in \citet{pan23stochastic} that input-output-disturbance realization trajectories $\trad{(u,y,w)}{0}{T-1}$, where the superscript $\cdot^{\da}$ denotes offline recorded data of the disturbed system~\eqref{eq:DynReal}, suffices to represent \eqref{eq:Dyn}.

\begin{lem}[Stochastic fundamental lemma]\label{lem:StochFundam}~
    Consider system~\eqref{eq:Dyn} and a $T$-length realization trajectory tuple $\trad{(u,y,w)}{0}{T-1}$ of realization dynamics~\eqref{eq:DynReal}. We assume that system~\eqref{eq:DynReal} is controllable and let $\trad{(u,w)}{0}{T-1}$ be persistently exciting of order $N+\dimx$. Then $ \tra{(U,Y,W)}{0}{N-1}$ is a trajectory of \eqref{eq:Dyn} of length $N$ if and only if there exists $G \in \splx{T-N+1} $ such that 
	\begin{equation*}
		\Hankel_N(\trad{z}{0}{T-1}) G=Z_{[0,N-1]}
	\end{equation*} 
	holds for all $(z, Z)\in \{ (u, U), (y, Y), (w, W)\}$.
\end{lem}

\begin{cor}\label{coro:StochFundamPCE}
	Let the conditions of Lemma~\ref{lem:StochFundam} hold. Then $ \tra{(\pce{y}^j, \pce{u}^j, \pce{w}^j)}{0}{N-1}$, $j \in \N^\infty$ is a trajectory of the dynamics of PCE coefficients \eqref{eq:DynPCE} if and only if there exists $\pce{g}^j \in \R^{T-N+1}$ such that 
	\begin{equation*}
		\Hankel_t(\trad{z}{0}{T-1}) \pce{g}^j =\pce{z}^j_{[0,N-1]},~j\in \N^\infty
	\end{equation*} 
	holds for all $(z, \pce{z})\in \{(u,\pce{u}), (y,\pce{y}), (w, \pce{w})\}$.
\end{cor}
In the image representation of a fundamental lemma, the future trajectory of an LTI system lives in the column space the Hankel matrices consisting of the past trajectory data. In \citet{pan23stochastic}, however, we have shown that Hankel matrices in random variables cause conceptual issues and the corresponding fundamental lemma does not necessarily hold. Instead, one can predict the future trajectory of the stochastic LTI system~\eqref{eq:Dyn} from the data of the system in realizations~\eqref{eq:DynReal}.
\section{Forward Uncertainty Propagation Without Disturbance Data} \label{sec:MainResults}
As the realizations of process disturbances are in general difficult to obtain, we aim for a variant of Lemma~\ref{lem:StochFundam} that does not explicitly rely on past disturbance data. To this end, we will exploit the superposition principle and causality requirements of stochastic systems.

\subsection{VARX model}
To avoid the use of state measurements, we switch to the Vector AutoRegressive with eXogenous input (VARX) model structure
\[
	Y_{k} = \hat{A}\tra{Y}{k-\tini}{k-1} + \hat{B}\tra{U}{k-\tini}{k-1}
    + \hat{E}\tra{W}{k-\tini}{k-1}.
\]
\citet{sadamoto23equivalence} shows the equivalence between the above VARX model and the state-space model~\eqref{eq:Dyn} with an observability assumption. Moreover, \citet{sadamoto23equivalence} explicitly derives the relation between matrices $A$, $B$, $E$ of \eqref{eq:Dyn} and $\hat{A}$, $\hat{B}$, $\hat{E}$ of \eqref{eq:VARX} as
\begin{align*}
	\hat{A} &= CA^{\tini}\mcl{O}^{\dagger},\\
	\hat{B} &= C\left(\mcl{C}(B)-A^{\tini}\mcl{O}^{\dagger}\mcl{M}(B) \right),\\
	\hat{E} &= C\left(\mcl{C}(E)-A^{\tini}\mcl{O}^{\dagger}\mcl{M}(E) \right),
\end{align*}
where $\cdot^\dagger$ denotes the Moore-Penrose inverse. The controllability matrix $\mcl{C}$ and the matrix $\mcl{M}$ are defined as $\mcl{C}(D) \coloneqq \begin{bmatrix} A^{\tini-1}D & \cdots & AD & D\end{bmatrix}$ and $\mcl{M}(D) \coloneqq \begin{bmatrix} 0 \\ CD & \ddots \\ \vdots & \ddots & &\ddots \\ CA^{\tini-2}D & \cdots & &CD & 0 \end{bmatrix}$ for $D\in\{B,E\}$.
Note that the mapping between matrices $A$, $B$, $E$ and $\hat{A}$, $\hat{B}$, $\hat{E}$ is not unique, e.g. \citet{phan96relationship} provide another possibility. The equivalence between the state-space model and the VARX model suggests that Lemma~\ref{lem:StochFundam} and Corollary~\ref{coro:StochFundamPCE} also hold for the VARX model \citep{pan24data}. Henceforth, we consider a simplified case of the above VARX model for which $\hat{E} = \begin{bmatrix}
	0_{n_y\times (\tini-1)n_y} & I_{n_y}
\end{bmatrix}$ holds.
\begin{assum}[VARX model] \label{ass:VARX}
	The state-space model~\eqref{eq:Dyn} is equivalent to
	\begin{subequations} \label{eq:VARX}
		\begin{align}
			&Y_k = \hat{A}\tra{Y}{k-\tini}{k-1} + \hat{B}\tra{U}{k-\tini}{k-1} + W_{k-1},\\
			&\tra{(U,Y)}{1-\tini}{0} = \tra{(\tilde{U},\tilde{Y})}{1-\tini}{0},
		\end{align}
	\end{subequations}
	where $\tra{(\tilde{U},\tilde{Y})}{1-\tini}{0}$ is an initial input-output trajectory of length $\tini$.
\end{assum}
\begin{rem} \label{rem:VARX}
	Besides $\tra{(U,Y)}{k-\tini}{k-1}$, the current output $Y_k$ only directly depends on the last disturbance $W_{k-1}$ in~\eqref{eq:VARX}. This assumption is in general difficult to verify as the state-space model and the VARX model are both unknown.
    However, for the systems where only the input and output measurements are available, one can use the i.i.d. disturbances $W_{k-1}$, $k\in\mbb{N}$ in \eqref{eq:VARX} to represent all the unknown disturbances affecting the system. Then, from the measured data one can obtain the estimation of the distribution of $W$ \citep{pan23stochastic, turan22data}, e.g. the least-square estimator
    \begin{equation} \label{eq:Estimator}
    	\begin{split}
	    	\Hankel_1(\trad{\hat{w}}{0}{T-1}) = \Hankel_1(\trad{y}{1}{T})\Bigg(I-&\\
    		\begin{bmatrix} \Hankel_1(\trad{y}{0}{T-1}) \\ \Hankel_1(\trad{u}{0}{T-1}) \end{bmatrix}^\dagger
    		&\begin{bmatrix} \Hankel_1(\trad{x}{0}{T-1}) \\ \Hankel_1(\trad{u}{0}{T-1}) \end{bmatrix}\Bigg).
    	\end{split}
    \end{equation}
\end{rem}

\begin{exmp}
	We illustrate the equivalence between the state-space model~\eqref{eq:Dyn} and the VARX model~\eqref{eq:VARX} via a simple example. Consider the 2-dimensional system
	\begin{align*}
		X_{k+1} &= \begin{bmatrix} 1 & 1 \\ 0 & 1 \end{bmatrix} X_k+ \begin{bmatrix} 0 \\ 1\end{bmatrix} U_k+ \begin{bmatrix} 1\\1\end{bmatrix}W_k\\
		Y_k & = \begin{bmatrix} 1 & 0\end{bmatrix}X_k.
	\end{align*}
	with $\tini=2$. Based on the approach given above, we obtain the corresponding VARX model
	\begin{multline*}
		Y_k = \begin{bmatrix} 3 & 2 \end{bmatrix} \tra{Y}{k-2}{k-1} + \begin{bmatrix} 1 & 0 \end{bmatrix} \tra{U}{k-2}{k-1} \\+ \begin{bmatrix} 0 & 1 \end{bmatrix} \tra{W}{k-2}{k-1},
	\end{multline*}
	which satisfies Assumption~\ref{ass:VARX}.
\end{exmp}

Similar to~\eqref{eq:DynPCE}, we obtain the corresponding dynamics of realizations
\begin{subequations} \label{eq:VARXReal}
	\begin{align}
		&y_k = \hat{A}\tra{y}{k-\tini}{k-1} + \hat{B}\tra{u}{k-\tini}{k-1} + w_{k-1},\\
		&\tra{(u,y)}{1-\tini}{0} = \tra{(\tilde{u},\tilde{y})}{1-\tini}{0},
	\end{align}
\end{subequations}
and the dynamics of PCE coefficients for~\eqref{eq:VARX} for all $j\in\I_{[0,L-1]}$, $k\in\I_{[1,N]}$
\begin{subequations} \label{eq:VARXPCE}
	\begin{align}
		&\pce{y}_k^j = \hat{A}\tra{\pce{y}^j}{k-\tini}{k-1} + \hat{B}\tra{\pce{u}^j}{k-\tini}{k-1} + \pce{w}^j_{k-1},\\
		&\tra{(\pce{u},\pce{y})^j}{1-\tini}{0} = \tra{(\tilde{\pce{u}},\tilde{\pce{y}})^j}{1-\tini}{0}.
	\end{align}
\end{subequations}
Especially, given a deterministic initial trajectory $\tra{(\tilde{u},\tilde{y})}{1-\tini}{0}$, we get $\mean[\tra{(\tilde{u},\tilde{y})}{1-\tini}{0}]=\tra{(\tilde{u},\tilde{y})}{1-\tini}{0}$ and thus its PCE representation reads
\begin{align*}
	\tra{(\tilde{\pce{u}},\tilde{\pce{y}})^j}{1-\tini}{0} = \begin{cases}
		\tra{(\tilde{u},\tilde{y})}{1-\tini}{0},\quad &\text{for } j=0\\
		0,&\text{otherwise}
	\end{cases}.
\end{align*}
We also note that the dynamics of realizations and of PCE coefficients, i.e.~\eqref{eq:VARXReal} and~\eqref{eq:VARXPCE}, share the same system matrices. Therefore, the Hankel matrices consisting of realizations and PCE coefficients span the same column space provided persistency of excitation holds. This way, one can use the Hankel matrices in realizations to predict the future trajectories of PCE coefficients, see the column-space equivalence given by Lemma~3 by~\citet{pan23stochastic} for details.

\subsection{System decomposition}
For the sake of simplified notation, we consider deterministic initial input-output condition $\tra{(U,Y)}{1-\tini}{0}=\tra{(\tilde{u},\tilde{y})}{1-\tini}{0}$. We discuss the generalization to an uncertain initial condition in Remark~\ref{rem:UncertainIni}.
	
According to the superposition principle, one can decouple the stochastic LTI system~\eqref{eq:VARX} into a nominal deterministic subsystem, which corresponds to the dynamics of the expected value, and other stochastic error systems, each of which is related to one disturbance $W_i$, $i\in\N$. That is, given the decomposition
\begin{subequations} \label{eq:Superposition}
	\begin{equation}
		Z_k = \mean[Z_k] + \sum_{i=0}^{k-1} Z_k^{w_i},\quad Z\in\{U,Y\},
	\end{equation}
the dynamics of the nominal deterministic system are
	\begin{equation} \label{eq:SubNom}
		\begin{split}
			&\mean[Y_k] = \hat{A}\mean[\tra{Y}{k-\tini}{k-1}] + \hat{B}\mean[\tra{U}{k-\tini}{k-1}] +\mean[W],\\
			&\mean[\tra{(U,Y)}{1-\tini}{0}] = \tra{(\tilde{u},\tilde{y})}{1-\tini}{0}.
		\end{split}
	\end{equation}
The stochastic error system for each $W_i$, $i\in\N$ reads
\begin{equation} \label{eq:SubError}
	\begin{split}
		&Y^{w_i}_k = \hat{A}\tra{Y^{w_i}}{k-\tini}{k-1} + \hat{B}\tra{U^{w_i}}{k-\tini}{k-1},\\
		&\tra{(U,Y)^{w_i}}{1-\tini}{i} = 0,\quad Y^{w_i}_{i+1} = W_i-\mean[W]
	\end{split}
\end{equation}
\end{subequations}
with $\mean[Y_k^{w_i}]=\mean[U_k^{w_i}]=0$, $k\in\N$. Note that due to the inherent causality requirement, i.e., the input $U_k$ and output $Y_k$ may only depend on past disturbances $W_s$, $s<k$, we have $\tra{(U,Y)^{w_i}}{1-\tini}{i} = 0$, $i\in\N$. Thus,
\[
	Y^{w_i}_{i+1} = \hat{A}\cdot 0+ \hat{B}\cdot 0 + W_i-\mean[W]=W_i-\mean[W].
\]
Splitting the LTI system~\eqref{eq:VARX} into decoupled subsystems~\eqref{eq:Superposition}, the additive disturbance~$W_i$, $i\in\N$ is converted into the initial output condition $Y_{i+1}^{w_i}=W_i-\mean[W]$. That is, the subsystems~\eqref{eq:Superposition} are disturbance free except for $\mean[W]$ in~\eqref{eq:SubNom}. Therefore, Lemma~\ref{lem:StochFundam} and Corollary~\ref{coro:StochFundamPCE} can be applied to predict the future trajectories of~\eqref{eq:Superposition} without the need for Hankel matrices involving disturbance data.

\subsection{PCE basis structure} \label{sec:Structure}
To obtain a PCE reformulation of subsystems~\eqref{eq:Superposition} for further analysis, one needs to construct a joint PCE basis applicable to all inputs and outputs.

Let  $\Psi^{w}_k\coloneqq \{ \psi^n(\xi_k)\}_{n=0}^{L_w-1}$, $L_w\in\N^\infty$ be the finite-dimensional basis for $W_k$, $k\in\I_{[0,N-1]}$. Then the i.i.d. disturbances $W_{k}$, $k\in \I_{[0,N-1]}$ admit exact PCEs 
\begin{equation} \label{eq:WkPCE}
	W_k = \sum_{n=0}^{L_w-1} \pce{w}^n \psi^n(\xi_k).
\end{equation}
As the disturbances are identically distributed, the PCEs of $W_k$, $k\in\I_{[0,N-1]}$ have the same algebraic structure of the basis functions $\psi^n$ and coefficients $w^n$, $n\in\I_{[0,L_w-1]}$. The independence of $W_k$, $k\in\I_{[0,N-1]}$ is modelled by the use of different stochastic germs $\xi_k$.
We construct the joint disturbance basis $\Phi\coloneqq \cup_{k=0}^{N-1} \Psi^{w}_k$ as
\begin{equation}\label{eq:BasisElement}
	\begin{split} 
		\Phi = &\Big\{ 1, \underbrace{\psi^1(\xi_0),...,\psi^{L_w-1}(\xi_0)}_{\Phi^{w_0}\setminus \{\psi^0(\xi_0)\}},
		\\ &\qquad...,\underbrace{\psi^1(\xi_{N-1}),...,\psi^{L_w-1}(\xi_{N-1})}_{\Phi^{w_{N-1}}\setminus \{\psi^0(\xi_{N-1})\}}\Big\},
	\end{split}
\end{equation}
which contains a total of $L=1+N(L_w-1)$ terms. We also enumerate the joint basis $\Phi$ from 0 to $L-1$ according to the sequence given above.

Recall that the PCEs of i.i.d. $W_k$, $k\in\I_{[0,N-1]}$ in the joint basis~$\Phi$ be $W_k=\sum_{j=0}^{L-1}\pce{w}_k^j\phi^j$. From the structure of~\eqref{eq:BasisElement} we observe that the basis functions related to the disturbance $W_k$, $k\in\I_{[0,N-1]}$ are $\phi^j$, $j\in \{0\}\cup\ik{k}$ with
\[
	\ik{k}\coloneqq \I_{[1+k(L_w-1),(k+1)(L_w-1)]}.
\] 
Conversely, given the PCE dimension $j$, using
\begin{equation*} \label{eq:k}
	k^\prime(j) \coloneqq \begin{cases}
		0,~&\text{for } j=0,\\
		k \text{ such that } j\in\ik{k}, \quad &\text{for } j\in\I_{[1,L-1]}
	\end{cases},
\end{equation*}
we see that the PCE coefficients $(\pce{u},\pce{x},\pce{y})^j$, $j\in\I_{[1,L-1]}$  relate to the disturbance $W_{k^\prime(j)}$. Note that $(\pce{u},\pce{x},\pce{y})^0$ is the expected value and thus it is linked to system~\eqref{eq:SubNom}.

With the index set $\ik{k}$,  $k\in\I_{[0,N-1]}$, we can identify the PCE coefficients corresponding to the disturbance $W_k$ in the joint basis $\Phi$
\begin{equation*}
	\pce{w}_k^j = \begin{cases}
		\pce{w}^0=\mean[W],~&\text{for } j=0\\
		\pce{w}^{j-k(L_w-1)},~&\text{for } j\in\ik{k}\\
		0,&\text{otherwise}
	\end{cases},
\end{equation*}
where $\pce{w}^{n}$, $n\in\I_{[0,L_w-1]}$ are defined in~\eqref{eq:WkPCE}. Moreover, in~Proposition~1 of \citet{pan23stochastic}, we have shown that all $Z_k$, $Z\in\{U,X,Y\}$ of~\eqref{eq:VARX} admit exact PCEs in this joint basis $\Phi$ for all $k \in \I_{[0,N]}$ over the entire horizon $N$. Therein the basis $\Phi$ entails the same basis directions but is indexed differently.

\subsection{Causality in PCE}
Given the joint basis $\Phi$, the PCEs of $\mean[Z]$ and $Z^{w_i}$, $Z\in\{(U,Y)\}$ with non-zero coefficients read
\begin{equation} \label{eq:RVDecompositionPCE}
	\mean[Z_k] = \pce{z}_k^0\phi^0,\quad Z_k^{w_i} = \sum_{j\in\ik{i}}\pce{z}_k^j\phi^j.
\end{equation}
Expressing the causality conditions $\tra{(U,Y)^{w_i}}{1-\tini}{i} = 0$ and $Y^{w_i}_{i+1} = W_i-\mean[W]$ in the PCE framework, we have
\begin{equation} \label{eq:Causality}
	\tra{(\pce{u},\pce{y})^j}{1-\tini}{k^{\prime}(j)}=0,~ \pce{y}_{k^{\prime}(j)+1}^j = \pce{w}^{I(j)}
\end{equation}
for all $j\in\I_{[1,L-1]}$, where $I(j)\coloneqq j-k^{\prime}(j)(L_w-1)$ \citep{ou25polynomial}. This way, the PCE coefficients of disturbances in~\eqref{eq:DynPCE} become the initial value $\pce{y}_{k^{\prime}(j)+1}^j = \pce{w}^{I(j)}$. It is straightforward to rewrite the dynamics of the PCE coefficients for $j=0$,
\begin{subequations} \label{eq:SuperpositionPCE}
	\begin{equation}\label{eq:VARXSimExp}
		\begin{split}
			&\pce{y}_k^0 = \hat{A}\tra{\pce{y}^0}{k-\tini}{k-1} + \hat{B}\tra{\pce{u}^0}{k-\tini}{k-1}+\mean[W],\\
			&\tra{(\pce{u},\pce{y})^0}{1-\tini}{0} = \tra{(\tilde{u},\tilde{y})}{1-\tini}{0},
		\end{split}
	\end{equation}
which corresponds to the nominal deterministic system~\eqref{eq:SubNom}. For $j\in\ik{i}$, $i\in\I_{[0,N-1]}$, we obtain
	\begin{equation} \label{eq:VARXSimDist}
			\pce{y}_k^j = \hat{A}\tra{\pce{y}^j}{k-\tini}{k-1} {+}\hat{B}\tra{\pce{u}^j}{k-\tini}{k-1}~\text{with}~\eqref{eq:Causality},
	\end{equation}
\end{subequations}
which corresponds to the stochastic error system~\eqref{eq:SubError} for $W_i$.
Thus, one can apply the fundamental lemma, cf. Theorem~1 by \citet{willems05note}, to predict the future trajectory~$\tra{(\pce{u},\pce{y})}{1}{N}$ for all $j\in\I_{[0,L-1]}$.

\subsection{Propagation without past disturbance data}
We move on showing how one can forward propagate the PCE coefficients of LTI system~\eqref{eq:VARXPCE}, in a data-driven fashion without past disturbance data. The crucial observation is that the reformulated dynamics~\eqref{eq:SuperpositionPCE} are deterministic systems---disturbed by the constant $\mean[W]$ in case of \eqref{eq:VARXSimExp} and undisturbed in case of \eqref{eq:VARXSimDist}.
Hence, we require a recording of an input-output trajectory $\traud{(u,y)}{0}{T-1}$ of the undisturbed system
\begin{subequations} \label{eq:VARXFree}
	\begin{align}
		&y_k^{\ud} = \hat{A}\traud{y}{k-\tini}{k-1} + \hat{B}\traud{u}{k-\tini}{k-1}, \label{eq:VARXDyn}\\
		&\traud{(u,y)}{1-\tini}{0} = \traud{(\tilde{u},\tilde{y})}{1-\tini}{0} \label{eq:VARXFreeIni}
	\end{align}
\end{subequations}
in the offline phase, where the superscript $\cdot^{\ud}$ denotes the undisturbed system and data. It is worth mentioning that the input ${u}^{\ud}$ of~\eqref{eq:VARXFree} is indeed identical to that of the disturbed system~\eqref{eq:VARXReal}. The superscript $\cdot^{\ud}$ of the input is used to maintain consistent notation with the undisturbed output $y^{\ud}$.

To obtain the data of~\eqref{eq:VARXFree}, we record the data of system~\eqref{eq:VARX} when it is temporarily undisturbed, i.e. $w=0$. However, the disturbance realizations in the past still affect the current and the future outputs. Consider disturbance $w_{-\tini}$ that acts on system~\eqref{eq:VARXFree} at time step $k=-\tini$. Then we decompose the output as $y^{\ud} = \bar{y} + y^w$ with
\begin{align*} 
		&\bar{y}_k = \hat{A}\tra{\bar{y}}{k-\tini}{k-1} + \hat{B}\traud{u}{k-\tini}{k-1},\\
		&\tra{\bar{y}}{1-\tini}{0} = \tra{(\tilde{y}-y^w)}{1-\tini}{0},\\
		&y^w_k = \hat{A}\tra{y^w}{k-\tini}{k-1} + \hat{B}\cdot 0,\\
		& y^w_{1-\tini} = w_{-\tini},~y^w_{i} = 0, \forall i \leq -\tini,
\end{align*}
where $\bar{y}$ denotes the output unaffected by $w_{-\tini}$, and $y^w$ denotes the affect of $w_{-\tini}$ in the output. Constructing the Hankel matrices from $(u^{\ud},\bar{y})_{[1,T]}$ and $\traud{(u,y)}{1}{T}$, we have
\[
	\mathrm{colsp}\left( \begin{bmatrix} \Hankel_{\tini}(\traud{u}{1}{T})\\ \Hankel_{\tini}(\tra{\bar{y}}{1}{T}) \end{bmatrix}\right)\subset \mathrm{colsp}\left( \begin{bmatrix} \Hankel_{\tini}(\traud{u}{1}{T})\\ \Hankel_{\tini}(\traud{y}{1}{T}) \end{bmatrix}\right)
\]
since $y^{\ud}=\bar{y}$ for the case $y^w=0$.
In other words, the disturbances before the time step $k=1-\tini$, e.g. $w_{-\tini}$, enrich the column space of Hankel matrices. Therefore, given an initial condition $\tra{(\tilde{u},\tilde{y})}{1-\tini}{0}$ of disturbed system~\eqref{eq:VARX}, there exists $g\in\R^{T-\tini+1}$ such that
\[
	\begin{bmatrix}
		\Hankel_{\tini}(\traud{u}{1}{T-\tini})\\
		\Hankel_{\tini}(\traud{y}{1}{T-\tini})
	\end{bmatrix} g = 
	\begin{bmatrix}
		\tra{\tilde{u}}{1-\tini}{0}\\
		\tra{\tilde{y}}{1-\tini}{0} 
	\end{bmatrix},
\]
while such such a vector $g$ does not exist if the data $\tra{(u^{\ud},\bar{y})}{1}{T}$ are used to construct the Hankel matrices.
\begin{rem}[Obtaining undisturbed system data]
	There are different approaches to obtain the data of system~\eqref{eq:VARXFree}: 
		\begin{enumerate}[label=(\roman*)]
			\item Measure the input-output data of system~\eqref{eq:VARX} when there is temporarily no disturbance, e.g., a power system can be considered undisturbed when there are only very limited users at midnight.
			\item\label{itm:measurement} Deploy additional high-fidelity sensors to measure the realizations of disturbances of system~\eqref{eq:VARX} in the offline phase, e.g. environment temperature for building climate control and user demands in power systems. Then, based on the collected data, one can compute the undisturbed trajectory of~\eqref{eq:VARXFree}, cf.~Section~\ref{sec:Estimation}. This assumption on the availability of additional data, e.g. disturbance or state measurements, appears frequently in recent literature \citep{wolff24robust,disaro24equivalence}.
			\item Collect the input-output data of the disturbed system~\eqref{eq:VARX} and estimate the corresponding disturbance realizations, see Remark~\ref{rem:VARX}. Then, one can compute the undisturbed trajectory of~\eqref{eq:VARXFree} as in Approach~\ref{itm:measurement}.
		\end{enumerate}
\end{rem}
Using the data of~\eqref{eq:VARXFree} to construct Hankel matrices, one may wonder whether the predicted trajectory $\tra{(u,y)}{1}{N}$ is guaranteed to satisfy the  dynamics~\eqref{eq:VARXDyn} since the system is only undisturbed during the data collection. Intuitively, the answer is positive as disturbances only appear before the data collection phase and thus the dynamics~\eqref{eq:VARXDyn} still hold for $\traud{(u,y)}{1}{T}$.

\begin{assum}[Persistently exciting data] \label{ass:PE}
	We assume that the collected data $\traud{(u,y)}{1}{T}$ satisfy
	\[
		\rank\left( \begin{bmatrix*}[l]
			\Hankel_{\tini+N}(\traud{u}{1}{T}) \\ \Hankel_{\tini}(\traud{y}{1}{T-N+1})
		\end{bmatrix*}\right) = (\tini+N)n_u + \tini n_y,
	\]
	i.e., the collected data $\traud{(u,y)}{1}{T}$ are persistently exciting.
\end{assum}
The next result shows how to capture the undisturbed realization dynamics~\eqref{eq:VARXDyn}.
\begin{lem} \label{lem:Prediction}
	 Let Assumption~\ref{ass:Sys}, \ref{ass:VARX}, and \ref{ass:PE} hold. Then a sequence $\tra{(u,y)}{1-\tini}{N}$ satisfy the dynamics~\eqref{eq:DynReal} if and only if there exists $g\in\R^{T-N-\tini+1}$ such that
\begin{equation} \label{eq:FundaLemma}
	\left[\begin{array}{ll} \Hankel_{\tini+N}(\traud{u}{1}{T})\\ \Hankel_{\tini+N}(\traud{y}{1}{T}) \end{array}\right] g = 
	\left[\begin{array}{ll} \tra{u}{1-\tini}{N} \\ \tra{y}{1-\tini}{N} \end{array}\right].
\end{equation}
\end{lem}
\begin{proof}
	The proof follows along the same lines as the extension of the fundamental lemma towards affine systems, cf. Theorem~1 by \citet{berberich22linear}. First we show the sufficiency of~\eqref{eq:FundaLemma}. It follows from the VARX model~\eqref{eq:VARX} that, for $i\in\N$,
	\begin{equation}\label{eq:VARXN}
		\traud{y}{\tini+i}{\tini+N+i-1} = \hat{A}_{\ini} \traud{y}{i}{i+\tini-1} + \hat{B}_N\traud{u}{i}{i+\tini+N-2}
	\end{equation}
	holds for the offline collected data, where the structure of matrices $\hat{A}_{\ini}$ and $\hat{B}_N$ is omitted for the brevity. Then, it follows
	\begin{align*}
		&\tra{y}{1}{N} \overset{\eqref{eq:FundaLemma}}{=} \sum_{i=1}^{T-N-\tini+1} \traud{y}{\tini+i}{\tini+N+i-1}g^i\\
		\overset{\eqref{eq:VARXN}}{=} &\sum_{i=1}^{T-N-\tini+1}  ( \hat{A}_{\ini} \traud{y}{i}{i+\tini-1} + \hat{B}_N\traud{u}{i}{i+\tini+N-2})g^i\\
		\overset{\eqref{eq:FundaLemma}}{=} &\hat{A}_{\ini}\tra{y}{1-\tini}{0} + \hat{B}_N\tra{u}{1-\tini}{N-1},
	\end{align*}
	where $g^i$ denotes the $i$-th element of $g$. Therefore,  $\tra{(u,y)}{1-\tini}{N}$ satisfies the VARX model~\eqref{eq:VARX}.
	
	Then we prove the necessity of~\eqref{eq:FundaLemma}. Assumption~\ref{ass:PE} implies that for any $\tra{u}{1-\tini}{N}$ and $\tra{y}{1-\tini}{0}$, there exists $g\in\R^{T-N-\tini+1}$ such that
	\begin{equation}\label{eq:PredYN}
		\begin{bmatrix*}[l]
			\Hankel_{\tini+N}(\traud{u}{1}{T}) \\ \Hankel_{\tini}(\traud{y}{1}{T-N+1})
		\end{bmatrix*}g = \begin{bmatrix*}[l]
			\tra{u}{1-\tini}{N} \\ \tra{y}{1-\tini}{0}
		\end{bmatrix*}.
	\end{equation}
	The VARX model~\eqref{eq:VARX} implies that
	\begin{align*}
		&\tra{y}{1}{N} = \hat{A}_{\ini}\tra{y}{1-\tini}{0} + \hat{B}_N\tra{u}{1-\tini}{N-1}\\
		=& \sum_{i=1}^{T-N-\tini+1} \left(\hat{A}_{\ini} \traud{y}{i}{i+\tini-1} + \hat{B}_N \traud{u}{i}{i+\tini+N-2}\right)g^i\\
		\overset{\eqref{eq:VARXN}}{=}& \sum_{i=1}^{T-N-\tini+1} \traud{y}{\tini+i}{\tini+N+i-1}g^i=\Hankel_{N}(\traud{y}{\tini+1}{T})g.
	\end{align*}
	Hence the assertion follows.
\end{proof}
As Assumption~\ref{ass:PE} holds, there always exists $g$ such that~\eqref{eq:PredYN} holds for an arbitrary initial trajectory $\tra{(u,y)}{1-\tini}{0}$. That is, one can use Lemma~\ref{lem:Prediction} to compute the response $\tra{(u,y)}{1}{N}$ of dynamics~\eqref{eq:VARXDyn} even for a disturbed initial trajectory.
We arrive at the following main results, which are provided in Lemma~\ref{lem:j0}, \ref{lem:jOther}, and \ref{lem:PropRV} below.

\begin{lem}[Propagation in expectation ($j=0$)] \label{lem:j0}
	Consider system~\eqref{eq:Dyn} and a trajectory $\traud{(u,y)}{1}{T}$ of~\eqref{eq:VARXFree}. Let Assumptions~\ref{ass:Sys}, \ref{ass:VARX}, and \ref{ass:PE} hold. Then given a measured initial condition $\tra{(\tilde{u},\tilde{y})}{1-\tini}{0}$,
	$\tra{(\pce{u},\pce{y})^0}{1}{N}$ is a trajectory of~\eqref{eq:DynPCE} for $j=0$ if and only if there exists $\pce{g}^0\in\R^{T-N-\tini+1}$ such that
	\begin{subequations} \label{eq:UYj0}
	\begin{equation} \label{eq:TrajUYr}
		\left[\begin{array}{ll} \Hankel_{\tini}(\traud{u}{1}{T-\tini})\\ \Hankel_{\tini}(\traud{y}{1}{T-\tini}) \\ \midrule \Hankel_{N}(\traud{u}{\tini+1}{T}) \\ \Hankel_{N}(\traud{y}{\tini+1}{T})  \end{array}\right] \pce{g}^0 = 
		\left[\begin{array}{ll} \tra{\tilde{u}}{1-\tini}{0} \\ \tra{\tilde{y}}{1-\tini}{0} \\ \midrule \tra{\pce{u}^0}{1}{N} \\ \tra{y^u}{1}{N} \end{array}\right]
	\end{equation}
	holds, where $ y_k^u = \pce{y}_k^0 - \sum_{i=1}^{k}y_i^{w}$, $y_1^w=\mean[W]$ and the system output $\tra{y^w}{2}{N}$ is given by
	\begin{multline} \label{eq:yw}
		\tra{y^w}{2}{N} = \Hankel_{N-1}(\traud{y}{\tini+1}{T})\cdot\\
		\begin{bmatrix*}[l] \Hankel_{\tini+N-1}(\traud{u}{1}{T}) \\ \Hankel_{\tini-1}(\traud{y}{1}{T-N})\\ \Hankel_{1}(\traud{y}{\tini}{T-N+1}) \end{bmatrix*}^\dagger \begin{bmatrix*}[l]  0_{(\tini+N-1)\dimu\times1} \\ 0_{(\tini-1)\dimy\times1} \\ \mean[W] \end{bmatrix*}.
	\end{multline}
	\end{subequations}
\end{lem}
\begin{proof}
    The core idea of the proof is to decompose $\pce{y}_k^0$ into two parts $\pce{y}_k^0=y_k^u + \sum_{i=1}^{k}y_i^{w}$, where $y^u$ related to the input $\pce{u}^0$ and $y^w$ related to the disturbances.
	
	For $j=0$ it holds that $\pce{w}_k^0=\mean[W]$, $k\in\I_{[0,N-1]}$. Then we split system~\eqref{eq:VARXPCE} into a disturbance free system
	\begin{subequations}
		\begin{align}
			&y_k^u = \hat{A}\tra{y^u}{k-\tini}{k-1} + \hat{B}\tra{\pce{u}^0}{k-\tini}{k-1}, \label{eq:DynR}\\
			&\tra{(y^u,\pce{u}^0)}{1-\tini}{0} = \tra{(\tilde{y},\tilde{u})}{1-\tini}{0},
		\end{align}
	and an error system
		\begin{align}
			&y_k^{\tilde{w}} = \hat{A}\tra{y^{\tilde{w}}}{k-\tini}{k-1} + \mean[W],  \label{eq:DynE}\\
			&\tra{y^{\tilde{w}}}{1-\tini}{0} = 0.
		\end{align}
	\end{subequations}
	Moreover, \eqref{eq:DynE}  can be rewritten as $y^{\tilde{w}}_k = \sum_{i=1}^{k}y^w_i$, $k\in\I_{[1,N]}$ with the autonomous system
	\begin{subequations}
			\begin{align}
					&y^w_k = \hat{A}\tra{y^w}{k-\tini}{k-1}+\hat{B}\cdot0,\label{eq:DynTE}\\
					&\tra{y^w}{1-\tini}{0} = 0,\quad y_1^w =\mean[W].
				\end{align}
		\end{subequations}
	We notice that $y^w_1 \neq \hat{A}\tra{y^w}{1-\tini}{0}=0$. Therefore, we choose $\tra{y^w}{2-\tini}{1}=[0_{1\times(\tini-1)\dimy},\mean[W]^\top]^\top$ to be the initial condition, which is indeed not a trajectory of~\eqref{eq:DynTE}. However, applying Lemma~\ref{lem:Prediction} we can compute $\tra{y^w}{1}{N}$. That is, $\tra{y^w}{1}{N}$ satisfies \eqref{eq:DynTE} if and only if there exist a $g^w\in\R^{T-N-\tini+2}$ such that  
	\[
	 \left[\begin{array}{ll} \Hankel_{\tini+N-1}(\traud{u}{1}{T}) \\ \midrule \Hankel_{\tini-1}(\traud{y}{1}{T-N})\\ \Hankel_{1}(\traud{y}{\tini}{T-N+1}) \\ \Hankel_{N-1}(\traud{y}{\tini+1}{T}) \end{array}\right] g^w =
	 \left[\begin{array}{ll}  0_{(\tini+N-1)\dimu\times1} \\ \midrule 0_{(\tini-1)\dimy\times1} \\ \mean[W] \\ \tra{y^w}{2}{N} \end{array}\right].
	\]
	From the above equation we can explicitly compute the trajectory $\tra{y^w}{2}{N}$ as~\eqref{eq:yw}.
	Again, applying Lemma~\ref{lem:Prediction} to dynamics~\eqref{eq:DynR}, we conclude \eqref{eq:TrajUYr}. From $\pce{y}_k^0=y_k^u+y_k^{\tilde{w}}=y_k^u+\sum_{i=1}^{k}y_i^w$, we have the predicted trajectory $\tra{(\pce{u},\pce{y})^0}{1}{N}$.
\end{proof}
Notice that since the distribution of $W$ and hence $\mean[W]$ are known, one can compute the forward propagation of $\mean[W]$, i.e. the trajectory $\tra{y^w}{1}{N}$, in advance.

\begin{table*}[t!]
	\tiny
	\caption{Comparison of different fundamental lemmas for LTI systems.}
	\label{tab:ComparisonLemmas}
	\centering
	\begin{adjustbox}{width=1\linewidth,center}
		\begin{tabular}{llcc}
			\toprule
			Lemma  & \hspace{30pt} System  & Data & Non-zero entries in $\Hankel$\\
			\midrule
			\citet{willems05note}  & $\arraycolsep=1.4pt \begin{array}{rcl} x_{k+1} &= & Ax_k+Bu_k,\\ y_k &=& Cx_k+Du_k\end{array}$ & $(u,y)^{\da}$ & $(\tini+N)(n_u+n_y)n_g$\\
			&&&\\
			\citet{berberich22linear} & $\arraycolsep=1.4pt\begin{array}{rcl} x_{k+1} &= & Ax_k+Bu_k+e,\\ y_k &=& Cx_k+Du_k+r\end{array}$ & $(u,y)^{\da}$ & $(\tini+N)(n_u+n_y)n_g$ \\
			&&&\\
			\citet{kerz23data}     & $\arraycolsep=1.4pt\begin{array}{rcl} X_{k+1}&=&AX_k+BU_k+EW_k \\ \hat{X}_k &=&X_k+\mu_k \end{array} $ & $(u,x,w)^{\da}$ & $N(n_u+n_y+n_w)n_g$\\
			&&&\\
			\begin{tabular}{@{}l@{}} \citet{pan23stochastic} \\ (Lemma~\ref{lem:StochFundam})\\ \end{tabular} &  $\arraycolsep=1.4pt\begin{array}{rcl} X_{k+1} &= & AX_k+BU_k+EW_k,\\ Y_k &=& CX_k+DU_k\end{array}$ & $(u,y,w)^{\da}$ & $(\tini+N)(n_u+n_y+n_w)n_gL$ \\
			&&&\\
			Lemma~\ref{lem:PropRV} &
			$\arraycolsep=1.4pt\hspace{6pt}\begin{array}{rcl} Y_k & = & \hat{A}\tra{Y}{k-\tini}{k-1} \\ & &+\hat{B}\tra{U}{k-\tini}{k-1} + W_{k-1}\end{array}$ 
			& $(u,y)^{\ud}$ & $\approx(\tini+\frac{N}{2})(n_u+n_y)n_gL$ \\
			\bottomrule
		\end{tabular}
	\end{adjustbox}
\end{table*}

\begin{lem}[Propagation for $j\in\I_{[1,L-1]}$]\label{lem:jOther}
	Let the conditions of Lemma~\ref{lem:j0} hold.
	Then, for all $j \in \I_{[1,L-1]}$, $\tra{(\pce{u},\pce{y})^j}{1}{N}$ is a trajectory of \eqref{eq:DynPCE} if and only if there exist $\pce{g}^j \in \R^{T-N-\tini+1}$ such that
	\begin{subequations}\label{eq:UYjRest}
		\begin{gather}
			\tra{(\pce{u},\pce{y})^j}{1}{k^{\prime}(j)}=0,\quad \pce{y}_{k^\prime(j)+1}^j = \pce{w}^{I(j)}, \label{eq:UY0}\\
			\left[\begin{array}{ll} \Hankel_{\tini-1}(\traud{u}{1}{T-N-1})\\ \Hankel_{\tini-1}(\traud{y}{1}{T-N-1}) \\ \midrule \Hankel_{\bar{N}}(\traud{u}{\tini}{T-k^{\prime}-1}) \\ \Hankel_{\bar{N}}(\traud{y}{\tini}{T-k^{\prime}-1}) \end{array}\right] \pce{g}^j 
			=\left[\begin{array}{ll} 0_{(\tini-1)\dimu\times1} \\ 0_{(\tini-1)\dimy\times1} \\ \midrule\tra{\pce{u}^j}{k^{\prime}(j)+1}{N} \\ \tra{\pce{y}^j}{k^{\prime}(j)+1}{N} \end{array}\right], \label{eq:PCEUYj}
		\end{gather}
	\end{subequations}
	where $\bar{N}=N-k^{\prime}(j)$.
\end{lem}
\begin{proof}
	The causality condition in the PCE framework, i.e. \eqref{eq:Causality}, implies \eqref{eq:UY0}.
	Similar to Lemma~\ref{lem:j0}, we notice that
	\[
		\pce{y}_{k^{\prime}(j)+1}^j\neq \hat{A}\tra{\pce{y}^j}{k^{\prime}(j)-\tini+1}{k^{\prime}} +  \hat{B}\tra{\pce{u}^j}{k^{\prime}(j)-\tini+1}{k^{\prime}(j)}
	\]
	with $\tra{(\pce{u},\pce{y})^j}{k^{\prime}(j)-\tini+1}{k^{\prime}(j)}=0$. Hence, the initial condition is given as $\tra{(\pce{u},\pce{y})^j}{k^{\prime}(j)-\tini+2}{k^{\prime}(j)}=0$ and $\pce{y}_{k^{\prime}(j)+1}^j=\pce{w}^{I(j)}$, while $\pce{u}_{k^{\prime}(j)+1}^j$ remains an input variable. Then applying Lemma~\ref{lem:Prediction}, \eqref{eq:PCEUYj} immediately follows.
\end{proof}
As a by-product, the prediction horizon of \eqref{eq:DynPCE} for $j\in\I_{[1,L-1]}$ is shortened from $N$ to $N-k^{\prime}(j)$. Consequently, we have smaller Hankel matrices and less PCE coefficients as decision variables when we apply Lemma~\ref{lem:jOther} to stochastic OCPs. Therefore, Lemma~\ref{lem:jOther} also accelerates the computation in numerical implementations, see the numerical example in Section~\ref{sec:Simulation}.
Summarizing Lemma~\ref{lem:j0}-\ref{lem:jOther}, we conclude the following lemma in random variables.

\begin{lem}[Propagation in random variables]\label{lem:PropRV}
	Let Assumptions~\ref{ass:Sys}, \ref{ass:VARX}, and \ref{ass:PE} hold.
	Then $\tra{(U,Y)}{1}{N}$ is a trajectory of \eqref{eq:Dyn} for a measured initial trajectory $\tra{(\tilde{u},\tilde{y})}{1-\tini}{0}$ if and only if there exist $g\in\R^{T-N-\tini+1}$ and $G^{w_i}\in\splx{T-N-\tini+1}$, $i\in\I_{[0,N-1]}$ such that $Z_k = \mean[Z] + \sum_{i=0}^{k-1} Z_k^{w_i}$, $Z\in\{U,Y\}$, where
	\begin{enumerate}[label=(\roman*)]
		\item\label{Cond1} $\mean[\tra{(U,Y)}{1}{N}]=\tra{(\pce{u},\pce{y})^0}{1}{N}$ and $g=\pce{g}^0$ in Lemma~\ref{lem:j0},
		\item\label{Cond2}  $\tra{(U,Y)^{w_i}}{1-\tini}{i} = 0$, $Y^{w_i}_{i+1} = W_i-\mean[W]$, and
		\[
		\left[\begin{array}{ll} \Hankel_{\tini-1}(\traud{u}{1}{T-N-1})\\ \Hankel_{\tini-1}(\traud{y}{1}{T-N-1}) \\ \midrule \Hankel_{N-i}(\traud{u}{\tini}{T-i-1}) \\ \Hankel_{N-i}(\traud{y}{\tini}{T-i-1}) \end{array}\right] G^{w_i} 
		=\left[\begin{array}{ll} 0_{(\tini-1)\dimu\times1} \\ 0_{(\tini-1)\dimy\times1} \\ \midrule \tra{U^{w_i}}{i+1}{N} \\ \tra{Y^{w_i}}{i+1}{N} \end{array}\right].
		\]
	\end{enumerate}
\end{lem}
\begin{proof}
	First we prove that the conditions~\ref{Cond1}-\ref{Cond2} are necessary. Consider any trajectory~$\tra{(U,Y)}{1}{N}$ of system~\eqref{eq:Dyn} and its decomposition~\eqref{eq:Superposition}. Replacing all the inputs and outputs with their PCEs as~\eqref{eq:RVDecompositionPCE}, we obtain the PCE reformulated dynamics~\eqref{eq:SuperpositionPCE}, which are equivalent to the data-driven representations~\eqref{eq:UYj0} and \eqref{eq:UYjRest} as Lemmas~\ref{lem:j0}-\ref{lem:jOther} have shown. Since $\phi^0=1$, the condition~\ref{Cond1} directly follows from \eqref{eq:UYj0}. Then for the PCE representation~\eqref{eq:UYjRest}, $j\in\ik{i}$, we multiply them with the corresponding PCE basis functions $\phi^j$, $j\in\ik{i}$ and sum the results over. Let $G^{w_i}\coloneqq \sum_{j\in\ik{i}}\pce{g}^j\phi^j$, we get the condition~\ref{Cond2}.
	
	Next we show the sufficiency of the conditions~\ref{Cond1}-\ref{Cond2}. The stochastic fundamental lemma, i.e. Lemma~\ref{lem:StochFundam}, indicates that the trajectories~$\tra{(\mean[U],\mean[Y])}{1}{N}$ and $\tra{(U,Y)^{w_i}}{1}{N}$ are trajectories of the decomposed systems~\eqref{eq:SubNom} and \eqref{eq:SubError}, respectively, when the conditions \ref{Cond1}-\ref{Cond2} hold. Thus, $(U,Y)_k = (\mean[U],\mean[Y])_k + \sum_{i=0}^{k-1} (U,Y)_k^{w_i}$, $k\in\I_{[1,N]}$ satisfy the dynamics~\eqref{eq:VARX} as well as~\eqref{eq:Dyn}.
\end{proof}
\begin{rem}[Extension to $\tra{(\tilde{U},\tilde{Y})}{1-\tini}{0}$] \label{rem:UncertainIni}
	Consider system~\eqref{eq:Dyn} with an uncertain initial input-output trajectory $\tra{(U,Y)}{1-\tini}{0}=\tra{(\tilde{U},\tilde{Y})}{1-\tini}{0}$. One can split $Z_k = \mean[Z] +Z^{\ini} + \sum_{i=0}^{k-1} Z_k^{w_i}$, $Z\in\{U,Y\}$ with
	\begin{align*}
		&Y^{\ini}_k = \hat{A}\tra{Y^{\ini}}{k-\tini}{k-1} + \hat{B}\tra{U^{\ini}}{k-\tini}{k-1},\\
		&\tra{(U,Y)^{ini}}{1-\tini}{i} = \tra{(\tilde{U}-\mean[\tilde{U}],\tilde{Y}-\mean[\tilde{Y}])}{1-\tini}{0}.
	\end{align*}
	The forward propagation for the PCE coefficients of $Z^{\ini}$, $Z\in\{U,Y\}$ is a simplified case of the propagation of expectation in Lemma~\ref{lem:j0} with $\mean[E]=0$, i.e., the predicted trajectory of the PCE coefficients of an uncertain initial condition satisfies~\eqref{eq:TrajUYr}.
	Then, besides the conditions of Lemma~\ref{lem:PropRV}, $\tra{(U,Y)}{1}{N}$ is a trajectory of \eqref{eq:Dyn} if and only if there exists $G^{\ini}\in\splx{T-N-\tini+1}$ such that 
	\[
			\left[\begin{array}{ll} \Hankel_{\tini}(\traud{u}{1}{T-N-1})\\ \Hankel_{\tini}(\traud{y}{1}{T-N-1}) \\ \midrule \Hankel_{N}(\traud{u}{\tini}{T-i-1}) \\ \Hankel_{N}(\traud{y}{\tini}{T-i-1}) \end{array}\right] G^{w_i} 
		=\left[\begin{array}{ll} \tra{(\tilde{U}-\mean[\tilde{U}])}{1-\tini}{0}\\ \tra{(\tilde{Y}-\mean[\tilde{Y}])}{1-\tini}{0} \\ \midrule \tra{U^{\ini}}{1}{N} \\ \tra{Y^{\ini}}{1}{N} \end{array}\right].
	\]
	A similar result in PCE coefficients also immediately follows and is omitted for brevity.
\end{rem}

In Table~\ref{tab:ComparisonLemmas}, we compare the different fundamental lemmas, including the stochastic fundamental lemma by \citet{pan23stochastic} (Lemma~\ref{lem:StochFundam}) and Lemma~\ref{lem:PropRV}; \citet{berberich22linear} consider deterministic affine systems, while \citet{kerz23data} propose a pre-stabilized deterministic fundamental lemma for stochastic LTI systems subject to process disturbances and measurement noise. We also compare the required data and the number of non-zero entries of the Hankel matrices in the different fundamental lemmas. Here $n_g$ denotes the dimension of vector $g$ or $G$ and $L$ denotes the PCE dimension, which is proportional to the prediction horizon $N$ and is $L=1+N(L_w-1)$ for measured initial condition. Let $n_w=n_y$ and $\tini\ll N$, then we have the approximation
\[
	\frac{\text{non-zero entries in~}\Hankel~\text{of Lemma~\ref{lem:PropRV}}}{\text{non-zero entries in~}\Hankel~\text{of Lemma~\ref{lem:StochFundam}}}\approx\frac{n_u+n_y}{2(n_u+2n_y)}.
\]
Thus, the non-zeros in the Hankel matrix of Lemma~\ref{lem:PropRV} can easily be fewer than 50\% of those in Lemma~\ref{lem:StochFundam} for the same $n_g$ and $N$.

\subsection{Estimation of Disturbance-Free Data} \label{sec:Estimation}
In this section, we first propose a procedure to find a stabilizing feedback gain for the stochastic LTI system~\eqref{eq:VARX}. Then we show how one can estimate an undisturbed trajectory of~\eqref{eq:VARXFree} from the recorded realization trajectory $\trad{(u,y,w)}{1-\tini}{T}$ of~\eqref{eq:VARXReal}.

Given an unstable stochastic LTI system with a sequence of randomly sampled input in the offline data collection phase, the system response grows exponentially over time. Consequently, the constructed Hankel matrix of output data may exhibit ill-conditioning, which causes numerical issues. To prevent the system response from diverging, we design a stabilizing feedback controller
\[
u_k^{\da}=Kz_k^{\da} +v_k \text{ with }z_k^{\ud} = \begin{bmatrix} \traud{u}{k-\tini}{k-1}\\ \traud{y}{k-\tini}{k-1} \end{bmatrix},
\]
where $v_k$ is an additional small random noise to guarantee the persistency of excitation of the inputs. Based on recorded input-output data, one can compute $K$ via solving an optimization problem, see \citet{doerfler23on} for state feedback and \citet{pan24data} for output feedback. When disturbance measurements are unavailable, an estimator for the disturbance realizations is required to obtain $K$, e.g. the least-square estimator~\eqref{eq:Estimator}.
Note that the accuracy of the estimated disturbances remains an open question. Moreover, the recorded input-output trajectory and estimated disturbance realizations, i.e., $\trad{(u,y,\hat{w})}{0}{T-1})$, satisfy dynamics~\eqref{eq:VARX} but with different system matrices $\hat{A}$ and $\hat{B}$, cf. Corollary~3 by \citet{pan24data}. Therefore, Lemma~\ref{lem:StochFundam} and \ref{coro:StochFundamPCE} remain valid with the estimated disturbances.

Due to the estimation error of disturbance realizations, the computed feedback gain $K$ is not guaranteed to stabilize the system~\eqref{eq:DynReal}. Here we propose the following experimental procedure to resolve this issue:
\begin{itemize}
	\item[i)] Sample an input-state trajectory of system~\eqref{eq:DynReal} for a short length, i.e. $\trad{(u,y)}{0}{T}$ with small $T$, and estimate the disturbance $\trad{\hat{w}}{0}{T-1}$
	\item[ii)] Solve the optimization problem proposed in \citet{pan24data} with the sampled data and compute the corresponding $K$.
	\item[iii)] Let the feedback be $u_k = Kz_k + v_k$, where $v_k$ is uniformly sampled from a small interval, e.g. $[10^{-3},10^{-3}]$, and implement the input.
	\item[iv)] Sample the output data of system~\eqref{eq:DynReal} and check whether it stays in a neighbor of origin. If not, go to step i) and repeat the procedure with current input policy; otherwise the procedure terminates.
\end{itemize}
With a stabilizing feedback $K$, we can sample $\trad{(u,y)}{0}{T}$ of~\eqref{eq:DynReal} and construct the Hankel matrices with acceptable condition numbers. 

Consider system~\eqref{eq:VARX} and a corresponding input-output realization trajectory$\trad{(u,y)}{1-\tini}{T}$. We estimate the disturbance realizations $\trad{\hat{w}}{1-\tini}{T-1}$ for the least-square estimator.
Let $\trad{u}{0}{T-1}$ be persistently exciting of order $\hat{T}+\dimx$. Then an estimation of an undisturbed trajectory $\traud{(u,y)}{1}{\hat{T}}$ of system~\eqref{eq:VARXFree} can be computed from
\begin{equation}\label{eq:PredictorT}
\begin{bmatrix*}[l] \Hankel_{\hat{T}+\tini}(\trad{u}{1-\tini}{T}) \\ \Hankel_{\hat{T}+\tini}(\trad{y}{1-\tini}{T}) \\ \Hankel_{\hat{T}}(\trad{\hat{w}}{0}{T-1}) \end{bmatrix*} g = \begin{bmatrix*}[l] \traud{u}{1-\tini}{\hat{T}} \\ \traud{y}{1-\tini}{\hat{T}} \\ 0_{\hat{T}n_y\times1} \end{bmatrix*},
\end{equation}
where the initial trajectory $\traud{(u,y)}{1-\tini}{0}$ is an arbitrary piece of the recorded data $\trad{(u,y)}{1-\tini}{T}$ and is thus known. It is straightforward to see that $\hat{T}\ll T$. Thus, we may need to repeat the procedure \eqref{eq:PredictorT} until an undisturbed trajectory of sufficient length to construct Hankel matrices is obtained.
Moreover, similar to Lemma~3 by \citet{kerz23data}, we can modify~\eqref{eq:PredictorT} to compute a trajectory $\traud{(v,z)}{1}{\hat{T}}$ close to the origin
\begin{equation} \label{eq:PredictorModify}
\begin{bmatrix*}[l] \Hankel_{\hat{T}}(\trad{(u-Kz)}{1}{T})  \\ \Hankel_{\hat{T}}(\trad{z}{1}{T}) \\ \Hankel_{\hat{T}}(\trad{\hat{w}}{0}{T-1})  \end{bmatrix*} g = \begin{bmatrix*}[l]
	\traud{v}{1}{\hat{T}} \\ \traud{z}{1}{\hat{T}} \\ 0_{\hat{T}n_y\times 1} \end{bmatrix*},
\end{equation}
where the input-output trajectory is included in $\traud{z}{1}{\hat{T}}$. This way, we construct the Hankel matrices from the computed data of~\eqref{eq:VARXFree} with acceptable condition numbers.


\section{Numerical example} \label{sec:Simulation}
\begin{figure}[t]
	\begin{subfigure}{\linewidth}
		\centering
		\includegraphics[width=0.85\linewidth,trim={36mm 16mm 22mm 30mm},clip]{Figures/AircraftPDFY1NG.pdf}
	\end{subfigure}
	\begin{subfigure}{\linewidth}
		\centering
		\includegraphics[width=0.85\linewidth,trim={36mm 16mm 22mm 30mm},clip]{Figures/AircraftPDFY2NG.pdf}
	\end{subfigure}
	\caption{Evolution of the PDFs of the outputs $Y^1$ and $Y^2$ over horizon $N=25$. Deep blue-dashed line: Chance constraint.}
	\label{fig:AircraftY}
\end{figure}
\begin{figure}[t]
	\begin{center}
		\includegraphics[width=1\linewidth]{Figures/AircraftComparisonNG.pdf}
		\caption{Aircraft example with Gaussian disturbance. Red-solid line: Scheme~I; blue-solid line with circle marker: Scheme~II; black-dashed line: Scheme~III; deep blue-dashed line: Chance constraint.} \label{fig:AircraftComparison}		
	\end{center}
\end{figure}
We consider a discrete-time LTI aircraft model from \citet{pan24data}. The VARX matrices in~\eqref{eq:VARX} are
\begin{align*}
	\hat{A} &= \left[\begin{smallmatrix*}[r]
		-\phantom{0}0.201&	\phantom{-0}0.256&	\phantom{-0}0.050	&\phantom{-0}0.160&	\quad-\phantom{0}0.256&	\quad 0.086\\
		-\phantom{0}4.773&	\phantom{-0}3.688&	\phantom{-0}0.650&	\phantom{-0}2.982&	\quad-\phantom{0}2.688&	\quad 1.707\\
		-15.746&	\phantom{-}12.898&	\phantom{-0}2.319	&\phantom{-}10.461	&\quad-12.897	&\quad 5.171
	\end{smallmatrix*}\right],\\
	\hat{B} &= \left[\begin{smallmatrix*}[r] -\phantom{0}0.019	&\quad -\phantom{0}1.440\\
		\phantom{-0}0.711&\quad	-\phantom{0}1.800\\
		\phantom{-0}1.444&\quad	-26.922
	\end{smallmatrix*}\right].
\end{align*}
The elements of $W_k$, $k\in \N$ are independently and uniformly distributed with $W^1\sim\mcl{U}(-0.1,0.1)$, $W^2\sim\mcl{U}(-3,3)$, and $W^3\sim\mcl{U}(-0.8,0.8)$, where $W^i$ denotes the $i$-th element of $W$. Then we solve the following OCP
\begin{align*}
    &\min_{U_k, Y_k,G^{w_k}, g, k\in\I_{[1,N]}}
    \sum_{k=1}^N\mean[Y_k^\top QY_k + U_k^\top R U_k]\\
    \text{s.t.}\quad &\text{dynamics in form of Lemma~\ref{lem:PropRV}},\\
    &\mbb{P}[-0.349\leq Y_k^1\leq 0.349]\geq 0.8, k\in\I_{[2,N]}
\end{align*}
in the PCE framework. The weighting matrices in the stage cost are $Q=\text{diag}([1,1,1])$ and $R=1$. A conservative reformulation of the chance constraint imposed on $Y_k^2$ individually for $k\in\I_{[2,N]}$ is $-0.349\leq \mean[Y_k^1] \pm 3\sqrt{\mbb{V}[Y_k^1]}\leq 0.349$ \citep{calafiore06distributionally}. Note that the chance constaint is not imposed on $Y_1^2$ since $Y_1$ is fixed once we have the measured initial trajectory $\tra{(\tilde{u},\tilde{y})}{1-\tini}{0}$. The computations are done on a virtual machine with an AMD EPYC Processor with 2.8 GHz, 32 GB of RAM in \texttt{julia} using \texttt{IPOPT} \citep{waechter06implementation}.

First we solve the above stochastic OCP with prediction horizon $N=25$ and a randomly sampled initial condition around $[0,-100,0]^\top$. The evolution of the Probability Density Functions (PDF) of the system outputs $Y^1$ and $Y^2$ for the computed optimal input is depicted in Figure~\ref{fig:AircraftY}. As one can see, the chance constraint for $Y^1$ is satisfied with a high probability, while $Y^2$ shows a narrow distribution close to 0 at the end of the horizon.

Then we compare the following three schemes in closed loop with prediction horizon $N=10$ at each time step:
\begin{itemize}
	\item[I)] Lemma~\ref{lem:PropRV}with data $(u,y)^{\ud}$ of system~\eqref{eq:VARXFree},
	\item[II)] Lemma~\ref{lem:StochFundam} with data $(u,y,w)^{\da}$ of system~\eqref{eq:VARXReal},
	\item[III)] Lemma~\ref{lem:PropRV} with data $(u,y)^{\da}$ of system~\eqref{eq:VARXReal}.
\end{itemize}
Note that to apply Scheme III) using data of~\eqref{eq:VARXReal}, we first estimate the disturbance realizations and then generate a corresponding undisturbed trajectory $(u,y)^{\ud}$ of~\eqref{eq:VARXFree} via~\eqref{eq:PredictorModify}. We use the recorded/generated data of length 90 to construct all Hankel matrices. Given an initial condition around $[0,-100,0]^\top$ and the same disturbance realizations, we compute the closed-loop trajectories for 30 time steps for Scheme~I-III, see Figure~\ref{fig:AircraftComparison}. For a detailed closed-loop algorithm in the data-driven setting we refer to Algorithm~1 of~\citet{pan23stochastic}.

We observe that the input-output trajectories for Scheme~I and Scheme~II are identical with a maximal difference of $1.325\cdot 10^{-4}$, while Scheme~III results in a slightly different trajectory due to the estimation error of disturbance realizations. we sample 1000 sequences of initial condition and disturbance realizations and summarize the computation time in Table~\ref{tab:AircraftComparisonTime}, where SD refers to standard deviation. One can see that Scheme~I saves 72.9\% computation time in comparison to Scheme~II with a smaller standard deviation, while the average closed-loop stage cost $J^{\text{cl}}$ keeps the same. Moreover, the performance of Scheme~I and Scheme~III report similar results in terms of the computation time, while Scheme~III has a slightly higher average cost due to the estimation error of disturbances. Importantly, by using Lemma~\ref{lem:PropRV} in Scheme~I), one can save 63.6\% amount of data in the Hankel matrix in this simulation example compared to Scheme~II without loss of performance.

\begin{table}[t!]
	\caption{Comparison of the data amount in Hankel matrices and the computation time for 1000 samplings.}
	\label{tab:AircraftComparisonTime}
	\centering
	\begin{adjustbox}{width=0.95\columnwidth,center}
		\begin{tabular}{ccccc}
			\toprule
			\multirow{2}{*}{\shortstack{\\ \\ Data-driven\\ scheme}} &  \multirow{2}{*}{\shortstack{\\ \\ Non-zero entries\\ in $\Hankel$}}&\multicolumn{2}{c}{Computation time}  & \multirow{2}{*}{$J^\text{cl}$ $[-]$}\\
			 \cmidrule(lr){3-4} & & Mean $\SI{}{[s]}$ & SD $\SI{}{[s]}$   &\\
			\midrule
			I & 74892 &0.333 & 0.029  & $2.812\times 10^{3}$ \\
			II & 205716 & 1.232 & 0.534  & $2.812\times 10^{3}$\\
			III & 74892 & 0.270 & 0.063 & $2.894\times 10^{3}$\\
			\bottomrule
		\end{tabular}
	\end{adjustbox}
\end{table}
\section{Conclusion} \label{sec:Conclusion}
This paper has proposed a stochastic variant of the fundamental lemma towards stochastic LTI systems with process disturbances in the PCE framework. Based on the superposition principle, the stochastic LTI system is decoupled into a number of subsystems, each of which corresponds to a source of uncertainty, e.g. a disturbance at certain time step. Then given the causality constraints on the inputs and outputs, the additive disturbances can be converted into output initial conditions of the decoupled subsystems. This way, the dynamics of the subsystems do not contain any disturbances except for the initial conditions. Therefore, in contrast to our previous work, i.e. Lemma~\ref{lem:StochFundam}, this variant of the fundamental lemma does not requirement any disturbance data in Hankel matrices. As by-products of this variant, we have shorten the prediction horizon for the PCE coefficients related to the disturbances, which leads to an acceleration in numerical implementations. Future work will consider disturbance estimator with error bounds, extension to nonlinear systems, and fast computation of data-driven stochastic optimal control.

\section*{Acknowledgements}
The authors would like to thank Jonas Schie{\ss}l, Michael Heinrich Baumann, and Lars Gr{\"u}ne at Mathematical Institute, University of Bayreuth, Germany for their valuable feedback. The authors also acknowledge funding by the Deutsche Forschungsgemeinschaft (DFG, German Research Foundation) - project number 499435839.
                         
\bibliography{arXivRef}
\end{document}