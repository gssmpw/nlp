\section{Preliminaries}
\label{sec:preliminary}

This section summarizesthe Sinc approximation
and Sinc indefinite integration and their application
with the aid of the tanh or DE transformation.

\subsection{Sinc approximation and Sinc indefinite integration}

The Sinc numerical methods are generic names of numerical methods
based on the \emph{Sinc approximation},
expressed as
\begin{equation}
F(x) \approx \sum_{j=-N}^N F(jh)S(j,h)(x),\quad x\in\mathbb{R},
\label{eq:Sinc-approximation}
\end{equation}
where $h$ is a mesh size appropriately chosen depending on $N$,
and the basis function $S(j,h)$ is the so-called Sinc function
defined by
\[
 S(j,h)(x) =
\begin{cases}
 \dfrac{\sin(\pi(x - jh)/h)}{\pi(x - jh)/h} & (x \neq jh),\\
 1 & (x = jh).
\end{cases}
\]
Integrating both sides of~\cref{eq:Sinc-approximation},
we obtain an approximation formula called
the \emph{Sinc indefinite integration} as
\begin{align}
\label{eq:Sinc-indefinite}
\int_{-\infty}^{\xi}F(x)\diff x
&\approx \sum_{j=-N}^N F(jh) \int_{-\infty}^{\xi}S(j,h)(x)\diff x
=\sum_{j=-N}^N F(jh) J(j,h)(\xi),\quad\xi\in\mathbb{R},
\end{align}
where $J(j,h)$ is defined by
\[
 J(j,h)(x) = h \left\{
\frac{1}{2} + \frac{1}{\pi}\Si\left[\frac{\pi(x - jh)}{h}\right]
\right\},
\]
where $\Si(x)$ is the sine integral defined by
$\Si(x)=\int_0^x \{(\sin t) / t\}\diff t$.

\subsection{SE-Sinc approximation and SE-Sinc indefinite integration}

To use the approximation formulas~\cref{eq:Sinc-approximation}
and~\cref{eq:Sinc-indefinite},
the function $F(x)$ must be defined over the entire real line $\mathbb{R}$.
When the function $f(t)$ is
defined over the finite interval $(a, b)$,
a variable transformation is required to map $\mathbb{R}$
onto $(a, b)$.
For the purpose,
the tanh transformation $t=\SEt(x)$ defined in~\cref{eq:SEt}
is widely employed.
The change of variable ($t=\SEt(x)$) enables us
to apply~\cref{eq:Sinc-approximation} by setting $F(x)=f(\SEt(x))$.
Introducing $\tSE_j = \SEt(jh)$ and $\SEtInv(t)=\{\SEt\}^{-1}(t)$,
we express the obtained formula as
\begin{align}
% f(\SEt(x))
%& \approx \sum_{j=-N}^N f(\SEt(jh))S(j,h)(x),\quad x\in\mathbb{R},
%\nonumber
%\intertext{which is equivalent to}
 f(t)
& \approx \sum_{j=-N}^N f(\tSE_j)S(j,h)(\SEtInv(t)),\quad t\in(a, b).
\label{eq:SE-Sinc-approximation}
\end{align}
This approximation is referred to as the SE-Sinc approximation
in this paper.
Similarly, applying $s=\SEt(x)$ and
setting $F(x)=f(\SEt(x))$ in~\cref{eq:Sinc-indefinite},
we obtain
\begin{align}
\label{eq:SE-Sinc-indefinite}
 \int_a^t f(s) \diff s
 &= \int_{-\infty}^{\SEtInv(t)} f(\SEt(x))\SEtDiv(x)\diff x\\
&\approx \sum_{j=-N}^N f(\tSE_j)\SEtDiv(jh) J(j,h)(\SEtInv(t)),
\quad t\in(a, b),\nonumber
\end{align}
which is referred to as the SE-Sinc indefinite integration
in this paper.
If $F(x)=f(\psi(x))$ is analytic on the strip complex domain
\[
 \domD_d = \left\{\zeta\in\mathbb{C} : |\Im\zeta| < d\right\}
\]
for a positive constant $d$,
then both approximations performs highly accurately.
In other words, $f(t)$ should be analytic on the transformed domain
\[
 \SEt(\domD_d) = \left\{z=\SEt(\zeta) : \zeta\in\domD_d\right\},
\]
which is a simply-connected domain.
Convergence theorems of the two approximations were provided as follows.

\begin{theorem}[Stenger~{\cite[Theorem~4.2.5]{stenger93:_numer}}]
\label{thm:SE-Sinc-approx}
Assume that $f$ is analytic on $\SEt(\domD_d)$
for $d$ with $0<d<\pi$, and
there exists constants $K$ and $\alpha$ such that
\begin{equation}
 |f(z)|\leq K |z - a|^{\alpha}|b - z|^{\alpha}
\label{eq:LC}
\end{equation}
holds for all $z\in\SEt(\domD_d)$. Let $N$ be a positive integer,
and let $h$ be selected by the formula
\begin{equation}
\label{eq:h-SE}
 h = \sqrt{\frac{\pi d}{\alpha N}}.
\end{equation}
Then, there exists a constant $C$ independent of $N$ such that
\[
 \max_{t\in [a, b]}
\left|f(t) - \sum_{j=-N}^N f(\tSE_j)S(j,h)(\SEtInv(t))\right|
\leq C \sqrt{N} \rme^{-\sqrt{\pi d \alpha N}}.
\]
\end{theorem}

\begin{theorem}[Okayama et al.~{\cite[Theorem~2.9]{okayama09:_error}}]
\label{thm:SE-Sinc-indefinite}
Assume that $f$ is analytic on $\SEt(\domD_d)$
for $d$ with $0<d<\pi$, and
there exists constants $K$ and $\alpha$ such that
\begin{equation}
\label{eq:QC}
 |f(z)|\leq K |z - a|^{\alpha - 1}|b - z|^{\alpha - 1}
\end{equation}
holds for all $z\in\SEt(\domD_d)$. Let $N$ be a positive integer,
and let $h$ be selected by the formula~\cref{eq:h-SE}.
Then, there exists a constant $C$ independent of $N$ such that
\[
 \max_{t\in [a, b]}
\left|
\int_a^t f(s)\diff s - \sum_{j=-N}^N f(\tSE_j)\SEtDiv(jh)J(j,h)(\SEtInv(t))
\right|
\leq C \rme^{-\sqrt{\pi d \alpha N}}.
\]
\end{theorem}

\subsection{DE-Sinc approximation and DE-Sinc indefinite integration}

The SE-Sinc approximation~\cref{eq:SE-Sinc-approximation}
and SE-Sinc indefinite integration~\cref{eq:SE-Sinc-indefinite}
employ the tanh transformation~\cref{eq:SEt} to
map $\mathbb{R}$ onto the finite interval $(a, b)$.
The DE transformation~\cref{eq:DEt} also plays the same role,
and allows for the replacement of $\SEt$ with $\DEt$ in both formulas.
On the basis of this idea,
introducing $\tDE_j = \DEt(jh)$ and $\DEtInv(t)=\{\DEt\}^{-1}(t)$,
we can derive the following formulas
\begin{align}
\label{eq:DE-Sinc-approximation}
 f(t)
& \approx \sum_{j=-N}^N f(\tDE_j)S(j,h)(\DEtInv(t)),\quad t\in(a, b),\\
\label{eq:DE-Sinc-indefinite}
 \int_a^t f(s) \diff s
&\approx \sum_{j=-N}^N f(\tDE_j)\DEtDiv(jh) J(j,h)(\DEtInv(t)),
\quad t\in(a, b),
\end{align}
which are referred to as the DE-Sinc approximation
and DE-Sinc indefinite integration, respectively.
For the formulas~\cref{eq:DE-Sinc-approximation}
and~\cref{eq:DE-Sinc-indefinite},
$f(t)$ should be analytic on the transformed domain
\[
 \DEt(\domD_d) = \left\{z=\DEt(\zeta) : \zeta\in\domD_d\right\},
\]
which forms a Riemann surface.
Convergence theorems of the two approximations were provided as follows.

\begin{theorem}[Tanaka et al.~{\cite[Theorem~3.1]{tanaka09:_desinc}}]
Assume that $f$ is analytic on $\DEt(\domD_d)$
for $d$ with $0<d<\pi/2$, and
there exists constants $K$ and $\alpha$ such that~\cref{eq:LC}
holds for all $z\in\DEt(\domD_d)$. Let $N$ be a positive integer,
and let $h$ be selected by the formula
\begin{equation}
\label{eq:h-DE}
 h = \frac{\log(2 d N/\alpha)}{N}.
\end{equation}
Then, there exists a constant $C$ independent of $N$ such that
\[
 \max_{t\in [a, b]}
\left|f(t) - \sum_{j=-N}^N f(\tDE_j)S(j,h)(\DEtInv(t))\right|
\leq C \rme^{-\pi d N/\log(2 d N/\alpha)}.
\]
\end{theorem}

\begin{theorem}[Okayama et al.~{\cite[Theorem~2.16]{okayama09:_error}}]
\label{thm:DE-Sinc-indefinite}
Assume that $f$ is analytic on $\DEt(\domD_d)$
for $d$ with $0<d<\pi/2$, and
there exists constants $K$ and $\alpha$ such that~\cref{eq:QC}
holds for all $z\in\DEt(\domD_d)$. Let $N$ be a positive integer,
and let $h$ be selected by the formula~\cref{eq:h-DE}.
Then, there exists a constant $C$ independent of $N$ such that
\begin{align*}
& \max_{t\in [a, b]}
\left|
\int_a^t f(s)\diff s - \sum_{j=-N}^N f(\tDE_j)\DEtDiv(jh)J(j,h)(\DEtInv(t))
\right|\\
&\leq C \frac{\log(2 d N/\alpha)}{N}\rme^{-\pi d N/\log(2 d N/\alpha)}.
\end{align*}
\end{theorem}

\subsection{Generalized SE/DE-Sinc approximation}

In the convergence theorems of the SE/DE-Sinc approximation,
the condition~\cref{eq:LC} is assumed.
This condition requires $f(t)$ to be zero at the endpoints $t=a$ and $t=b$,
which seems an impractical condition.
To address this issue,
using auxiliary functions
\[
 \omega_a(t) = \frac{b - t}{b - a},\quad
 \omega_b(t) = \frac{t - a}{b - a},
\]
and setting
$\tilde{f}^{\textSE}_N(t)=f(t) -f(\tSE_{-N})\omega_a(t) -f(\tSE_N)\omega_b(t)$,
Stenger~\cite{stenger93:_numer,stenger00:_summar}
proposed to apply the SE-Sinc approximation to $\tilde{f}^{\textSE}_N$.
%derived the following approximation
%\begin{equation*}
% \tilde{f}^{\textSE}_N(t)
%\approx \sum_{j=-N}^N \tilde{f}^{\textSE}_N(\tSE_j)S(j,h)(\SEtInv(t)),
%\end{equation*}
%by applying the SE-Sinc approximation to $\tilde{f}^{\textSE}_N$.
If we define an approximation operator $\ProjSE: C([a,b])\to C([a, b])$ as
\begin{align}
\label{eq:ProjSE}
(\ProjSE f)(t)
=
f(\tSE_{-N})\omega_a(t) + f(\tSE_N)\omega_b(t)
 + \sum_{j=-N}^N \tilde{f}^{\textSE}_N(\tSE_j) S(j,h)(\SEtInv(t)),
\end{align}
then the approximation is expressed as $f\approx \ProjSE f$.
This approximation is referred to as the generalized SE-Sinc approximation
in this paper.
Notably, $\ProjSE$ satisfies
the interpolation property, that is,
$f(\tSE_i) = (\ProjSE f)(\tSE_i)$ ($i=-N,\,\ldots,\,N$).

Similarly, setting
$\tilde{f}^{\textDE}_N(t)=f(t) -f(\tDE_{-N})\omega_a(t) -f(\tDE_N)\omega_b(t)$,
we may apply the DE-Sinc approximation to $\tilde{f}^{\textDE}_N$.
%as
%\begin{equation*}
% \tilde{f}^{\textDE}_N(t)
%\approx \sum_{j=-N}^N \tilde{f}^{\textDE}_N(\tDE_j)S(j,h)(\DEtInv(t)).
%\end{equation*}
If we define an approximation operator $\ProjDE: C([a,b])\to C([a, b])$ as
\begin{align}
\label{eq:ProjDE}
(\ProjDE f)(t)
=
f(\tDE_{-N})\omega_a(t) + f(\tDE_N)\omega_b(t)
 + \sum_{j=-N}^N \tilde{f}^{\textSE}_N(\tDE_j) S(j,h)(\DEtInv(t)),
\end{align}
then the approximation is expressed as $f\approx \ProjDE f$.
This approximation is referred to as the generalized DE-Sinc approximation
in this paper.
$\ProjDE$ also satisfies
the interpolation property, that is,
$f(\tDE_i) = (\ProjDE f)(\tDE_i)$ ($i=-N,\,\ldots,\,N$).


The convergence theorems of the two approximations are described
using the following function spaces.

\begin{definition}
Let $\domD$ be a bounded and simply-connected domain
(or Riemann surface).
Then, $\Hinf(\domD)$ denotes the family of functions $f$
analytic on $\domD$ such that the norm $\|f\|_{\Hinf(\domD)}$ is finite,
where
\[
 \|f\|_{\Hinf(\domD)} = \sup_{z\in\domD}|f(z)|.
\]
\end{definition}
\begin{definition}
Let $\alpha$ be a positive constant, and
let $\domD$ be a bounded and simply-connected domain
(or Riemann surface) that satisfies $(a, b)\subset \domD$.
Then, $\MC_{\alpha}(\domD)$ denotes the family of functions $f\in\Hinf(\domD)$
for which there exists a constant $L$ such that for all $z\in\domD$,
\begin{align*}
 |f(z) - f(a)| &\leq L |z - a|^{\alpha},\\
 |f(b) - f(z)| &\leq L |b - z|^{\alpha}.
\end{align*}
\end{definition}

This function space $\MC_{\alpha}(\domD)$ only requires
the H\"{o}lder continuity at the endpoints instead of
the zero-boundary condition by~\cref{eq:LC}.
Convergence theorems of the two approximations were provided as follows.
Here, $\|\cdot\|_{C([a,b])}$ denotes the usual uniform norm over $[a, b]$.

\begin{theorem}[Okayama~{\cite[Theorem~3]{okayama13:_note}}]
\label{thm:SE-Sinc-general}
Assume that $f\in\MC_{\alpha}(\SEt(\domD_d))$ for $d$ with $0<d<\pi$.
Let $N$ be a positive integer,
and let $h$ be selected by the formula~\cref{eq:h-SE}.
Then, there exists a constant $C$ independent of $N$ such that
\[
 \|f - \ProjSE f\|_{C([a,b])}
\leq C \sqrt{N}\rme^{-\sqrt{\pi d \alpha N}}.
\]
\end{theorem}
\begin{theorem}[Okayama~{\cite[Theorem~6]{okayama13:_note}}]
\label{thm:DE-Sinc-general}
Assume that $f\in\MC_{\alpha}(\DEt(\domD_d))$ for $d$ with $0<d<\pi/2$.
Let $N$ be a positive integer,
and let $h$ be selected by the formula~\cref{eq:h-DE}.
Then, there exists a constant $C$ independent of $N$ such that
\[
 \|f - \ProjDE f\|_{C([a,b])}
\leq C \rme^{-\pi d N/\log(2 d N/\alpha)}.
\]
\end{theorem}