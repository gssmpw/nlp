\section{Sinc-Nystr\"{o}m methods}
\label{sec:nystroem}

This section describes the Sinc-Nystr\"{o}m methods
developed by Muhammad et al.~\cite{muhammad05:_numer}.
The first method employs the tanh transformation~\cref{eq:SEt}
as a variable transformation,
while the second method employs the DE transformation~\cref{eq:DEt}.

\subsection{SE-Sinc-Nystr\"{o}m method}

%Assume that $k(t,\cdot)u(\cdot)$ satisfies
%the assumptions of~\cref{thm:SE-Sinc-indefinite}
%uniformly for $t\in [a,b]$.
By applying the SE-Sinc indefinite integration~\cref{eq:SE-Sinc-indefinite}
to the integral in the given equation~\cref{eq:Volterra-int},
we obtain an approximated equation as
\begin{equation}
\label{eq:SE-Sinc-Nystroem}
 \uSEn(t)
= g(t)
+\sum_{j=-N}^N k(t,\tSE_j)\uSEn(\tSE_j)\SEtDiv(jh) J(j,h)(\SEtInv(t)).
\end{equation}
%where $\tSE_j = \SEt(jh)$.
The approximated solution $\uSEn$ is determined
once the unknown coefficients $\uSEn(\tSE_j)$
on the right-hand side are obtained. To this end,
$2N+1$ sampling points are set at
$t=\tSE_i$ $(i=-N,\,-N+1,\,\ldots,\,N)$ in~\cref{eq:SE-Sinc-Nystroem}
as
\begin{equation}
\label{eq:SE-Sinc-Nystroem-system}
 \uSEn(\tSE_i)
= g(t)
+\sum_{j=-N}^N k(\tSE_i,\tSE_j)\uSEn(\tSE_j)\SEtDiv(jh)
 J(j,h)(ih),
\quad i=-N,\,\ldots,\,N,
\end{equation}
which is a system of linear equations.
This system is expressed in a matrix-vector form as follows.
Let $n=2N+1$,
let $I_n$ be an identity matrix of order $n$,
and let $\kSEn$ be $n\times n$ matrix
whose $(i, j)$-th element is
\[
 \left(\kSEn\right)_{ij}
= k(\tSE_i,\tSE_j)\SEtDiv(jh)h \delta_{i-j}^{(-1)},
%\left(\frac{1}{2}+ \sigma_{i-j}\right),
\quad i= -N,\,\ldots,\,N,\quad j=-N,\,\ldots,\,N,
\]
where $\delta_k^{(-1)} = (1/2) + \sigma_k$, where $\sigma_k$ is defined by
\[
\sigma_k = \int_0^k\frac{\sin(\pi x)}{\pi x}\diff x
=\frac{1}{\pi}\Si(\pi k).
\]
Furthermore, let $\gSEn$ and $\mathbd{u}_n^{\textSE}$
be $n$-dimensional vectors defined by
\begin{align*}
 \gSEn
 &= [g(\tSE_{-N}),\,g(\tSE_{-N+1}),\,\ldots,\,g(\tSE_N)]^{\mathrm{T}},\\
 \mathbd{u}_n^{\textSE}
 &= [\uSEn(\tSE_{-N}),\,\uSEn(\tSE_{-N+1}),\,\ldots,\,\uSEn(\tSE_N)]^{\mathrm{T}}.
\end{align*}
Then, the system~\cref{eq:SE-Sinc-Nystroem-system} is expressed as
\begin{equation}
\label{eq:SE-Sinc-Nystroem-linear-eq}
 (I_n - \kSEn) \mathbd{u}_n^{\textSE} = \gSEn.
\end{equation}
By solving~\cref{eq:SE-Sinc-Nystroem-linear-eq},
we obtain the unknown coefficients $\mathbd{u}_n^{\textSE}$,
from which the approximated solution $\uSEn$ is determined
by~\cref{eq:SE-Sinc-Nystroem}.
This procedure is called the SE-Sinc-Nystr\"{o}m method.
Its convergence theorem was provided as follows.

\begin{theorem}[Okayama et al.~{\cite[Theorem~3.4]{okayama13:_theo}}]
\label{thm:SE-Sinc-Nystroem}
Let $d$ be a positive constant with $d<\pi$.
Assume that $g$, $k(z,\cdot)$ and $k(\cdot,w)$
belong to $\Hinf(\SEt(\domD_d))$ for all $z$, $w\in\SEt(\domD_d)$.
Furthermore, assume that
$g$, $k(t,\cdot)$ and $k(\cdot,s)$ belong to $C([a,b])$
for all $t$, $s\in [a, b]$.
Let $h$ be selected by the formula~\cref{eq:h-SE} with $\alpha=1$.
Then, there exists a positive integer $N_0$
such that for all $N\geq N_0$,
the coefficient matrix $(I_n - \kSEn)$ is invertible.
Furthermore, there exists a constant $C$ independent of $N$
such that for all $N\geq N_0$,
\[
 \|u - \uSEn\|_{C([a,b])} \leq C \rme^{-\sqrt{\pi d N}}.
\]
\end{theorem}

\subsection{DE-Sinc-Nystr\"{o}m method}

Muhammad et al.~\cite{muhammad05:_numer} also considered
another method by replacing $\SEt$ with $\DEt$
in the SE-Sinc-Nystr\"{o}m method.
%Assume that $k(t,\cdot)u(\cdot)$ satisfies
%the assumptions of~\cref{thm:DE-Sinc-indefinite}
%uniformly for $t\in [a,b]$.
Applying the DE-Sinc indefinite integration~\cref{eq:DE-Sinc-indefinite}
to the integral in the given equation~\cref{eq:Volterra-int},
we obtain an approximated equation as
\begin{equation}
\label{eq:DE-Sinc-Nystroem}
 \uDEn(t)
= g(t)
+\sum_{j=-N}^N k(t,\tDE_j)\uDEn(\tDE_j)\DEtDiv(jh) J(j,h)(\DEtInv(t)).
\end{equation}
%where $\tDE_j = \DEt(jh)$.
The approximated solution $\uDEn$ is determined
once the unknown coefficients $\uDEn(\tDE_j)$
on the right-hand side are obtained. To this end,
$2N+1$ sampling points are set at
$t=\tDE_i$ $(i=-N,\,-N+1,\,\ldots,\,N)$ in~\cref{eq:DE-Sinc-Nystroem}.
This leads a system of linear equations
\begin{equation}
\label{eq:DE-Sinc-Nystroem-linear-eq}
 (I_n - \kDEn) \mathbd{u}_n^{\textDE} = \gDEn,
\end{equation}
where $\kDEn$ be $n\times n$ matrix
whose $(i, j)$-th element is
\[
 \left(\kDEn\right)_{ij}
= k(\tDE_i,\tDE_j)\DEtDiv(jh)h \delta_{i-j}^{(-1)},
%\left(\frac{1}{2}+ \sigma_{i-j}\right),
\quad i= -N,\,\ldots,\,N,\quad j=-N,\,\ldots,\,N,
\]
and $\gDEn$ and $\mathbd{u}_n^{\textDE}$
be $n$-dimensional vectors defined by
\begin{align*}
 \gDEn
 &= [g(\tDE_{-N}),\,g(\tDE_{-N+1}),\,\ldots,\,g(\tDE_N)]^{\mathrm{T}},\\
 \mathbd{u}_n^{\textDE}
 &= [\uDEn(\tDE_{-N}),\,\uDEn(\tDE_{-N+1}),\,\ldots,\,\uDEn(\tDE_N)]^{\mathrm{T}}.
\end{align*}
By solving~\cref{eq:DE-Sinc-Nystroem-linear-eq},
we obtain the unknown coefficients $\mathbd{u}_n^{\textDE}$,
from which the approximated solution $\uDEn$ is determined
by~\cref{eq:DE-Sinc-Nystroem}.
This procedure is called the DE-Sinc-Nystr\"{o}m method.
Its convergence theorem was provided as follows.

\begin{theorem}[Okayama et al.~{\cite[Theorem~3.5]{okayama13:_theo}}]
\label{thm:DE-Sinc-Nystroem}
Let $d$ be a positive constant with $d<\pi/2$.
Assume that $g$, $k(z,\cdot)$ and $k(\cdot,w)$
belong to $\Hinf(\DEt(\domD_d))$ for all $z$, $w\in\DEt(\domD_d)$.
Furthermore, assume that
$g$, $k(t,\cdot)$ and $k(\cdot,s)$ belong to $C([a,b])$
for all $t$, $s\in [a, b]$.
Let $h$ be selected by the formula~\cref{eq:h-DE} with $\alpha=1$.
Then, there exists a positive integer $N_0$
such that for all $N\geq N_0$,
the coefficient matrix $(I_n - \kDEn)$ is invertible.
Furthermore, there exists a constant $C$ independent of $N$
such that for all $N\geq N_0$,
\[
 \|u - \uDEn\|_{C([a,b])}
 \leq C \frac{\log(2 d N)}{N}\rme^{-\pi d N/log(2 d N)}.
\]
\end{theorem}
