\section{Numerical experiments}
\label{sec:numer-result}

This section presents numerical results for the following five methods:
the SE/DE-Sinc-Nystr\"{o}m methods
by Muhammad et al.~\cite{muhammad05:_numer},
the SE-Sinc-collocation methods by Stenger~\cite{stenger93:_numer}
and Rashidinia--Zarebnia~\cite{rashidinia07:_solut},
and the DE-Sinc-collocation methods by this paper.
The computation was performed on
a MacBook Air computer with 1.7~GHz Intel Core i7
with 8 GB memory, running macOS Big Sur.
The computation programs were implemented
in the C programming language with double-precision floating-point arithmetic,
and compiled with Apple clang version 13.0.0 with no optimization.
Cephes Math Library was used for computation of the sine integral.
LAPACK in Apple's Accelerate framework was used for computation
of the system of linear equations.
The source code for all programs is available at
\url{https://github.com/okayamat/sinc-colloc-volterra}.

We consider the following two equations
(taken from Rashidinia--Zarebnia~\cite[Example~4]{rashidinia07:_solut}
and~Polyanin--Manzhirov~\cite[Equation 2.1.45]{polyanin08:_handb}):
\begin{align}
\label{eq:example1}
u(t) + \int_0^t t s u(s)\diff s
 &= \rme^{-t^2} +\frac{t}{2}(1 - \rme^{-t^2}),\quad 0\leq t\leq 1,\\
\label{eq:example2}
u(t) - 6\int_0^t (\sqrt{t} - \sqrt{s}) u(s)\diff s
 &= 1 +\sqrt{t} - 2t\sqrt{t} - t^2,\quad 0\leq t\leq 1,
\end{align}
whose solutions are $u(t)=\rme^{-t^2}$ and
$u(t)=1 + \sqrt{t}$, respectively.
In the case of~\cref{eq:example1},
the assumptions
of~\cref{thm:SE-Sinc-Nystroem,thm:SE-Sinc-collocation,thm:RZ-Sinc-collocation}
are fulfilled with $d=3.14$
and $\alpha=1$,
and those of~\cref{thm:DE-Sinc-Nystroem,thm:DE-Sinc-collocation}
are fulfilled with $d=1.57$
and $\alpha=1$.
In the case of~\cref{eq:example2},
the assumptions
of~\cref{thm:SE-Sinc-Nystroem,thm:SE-Sinc-collocation,thm:RZ-Sinc-collocation}
are fulfilled with $d=3.14$
and $\alpha=1/2$,
and those of~\cref{thm:DE-Sinc-Nystroem,thm:DE-Sinc-collocation}
are fulfilled with $d=1.57$ and $\alpha=1/2$.
Therefore, those values were used for implementation.
The errors were evaluated at 2048 equally spaced points 
over the given interval,
and the maximum error among them was plotted on the graph
in~\cref{fig:example1_N,fig:example1_t,fig:example2_N,fig:example2_t}.

From all figures, we can observe that
the SE-Sinc-collocation methods by Stenger and Rashidinia--Zarebnia
yield almost the same performance.
This result coincides
with~\cref{thm:SE-Sinc-collocation,thm:RZ-Sinc-collocation}.
%the description at the end of~\cref{sec:collocation}.
%Furthermore, the DE-Sinc-collocation method presented by this paper
From~\cref{fig:example1_N}, we can observe that
the SE/DE-Sinc-Nystr\"{o}m methods are slightly better than
the SE/DE-Sinc-collocation methods with respect to $N$.
This result coincides with~\cref{thm:SE-Sinc-Nystroem,thm:DE-Sinc-Nystroem,thm:SE-Sinc-collocation,thm:RZ-Sinc-collocation,thm:DE-Sinc-collocation}.
However,~\cref{fig:example1_t} shows that
with respect to the computation time,
the SE/DE-Sinc-collocation methods demonstrate significantly better performance
than the SE/DE-Sinc-Nystr\"{o}m methods.
This is because the SE/DE-Sinc-Nystr\"{o}m methods
include a special function as well as given functions $k$ and $g$
in the basis functions of their approximate solutions.
We note that the performance of
the SE/DE-Sinc-collocation methods in~\cref{fig:example2_N}
reduced than that in~\cref{fig:example1_N},
which is due to the difference of $\alpha$.

\begin{figure}[htbp]
\begin{minipage}{0.495\linewidth}
  \centering
 \includegraphics[width=\linewidth]{example1_N}
  \caption{Errors with respect to $N$ for~\cref{eq:example1}. \phantom{not computation time}}
  \label{fig:example1_N}
\end{minipage}
\begin{minipage}{0.495\linewidth}
  \centering
  \includegraphics[width=\linewidth]{example1_t}
  \caption{Errors with respect to the computation time for~\cref{eq:example1}.}
  \label{fig:example1_t}
\end{minipage}
\end{figure}
\begin{figure}[htbp]
\begin{minipage}{0.495\linewidth}
  \centering
 \includegraphics[width=\linewidth]{example2_N}
  \caption{Errors with respect to $N$ for~\cref{eq:example2}. \phantom{not computation time}}
  \label{fig:example2_N}
\end{minipage}
\begin{minipage}{0.495\linewidth}
  \centering
  \includegraphics[width=\linewidth]{example2_t}
  \caption{Errors with respect to the computation time for~\cref{eq:example2}.}
  \label{fig:example2_t}
\end{minipage}
\end{figure}
