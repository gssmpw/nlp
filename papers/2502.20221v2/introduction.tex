%#!latex SincCollocationVolterra2nd
\section{Introduction and summary}
\label{sec:introduction}

This paper is concerned with numerical solutions via
Sinc numerical methods~\cite{stenger00:_summar,sugihara04:_recen}
for Volterra integral equations of the second kind of the form
\begin{equation}
u(t) - \int_a^t k(t,s) u(s)\diff s
= g(t),\quad a\leq t\leq b.
\label{eq:Volterra-int}
\end{equation}
Here, $k(t,s)$ and $g(t)$ are given continuous functions,
and $u(t)$ is the solution to be determined.
The equations are often expressed symbolically
as $(\Ident -\Vol) u = g$ by introducing
Volterra integral operator $\Vol:C([a,b])\to C([a,b])$ as
\[
 \Vol[f](t)=\int_a^t k(t,s)f(s)\diff s.
\]
One of powerful tools in the Sinc numerical methods,
especially for the target equations~\cref{eq:Volterra-int},
is the Sinc indefinite integration~\cite{haber93:_two,muhammad03:_doubl}.
This provides an approximation formula for indefinite integral
in the following form
\[
 \int_a^t F(s)\diff s \approx
\sum_{j=-N}^N F(s_j) \omega_j(t),
\]
where the weight $\omega_j$ is a function depending on $t$,
whereas
the sampling point $s_j$ is \emph{fixed}, independent of $t$,
even though the interval of the integral $(a,\,t)$ depends on $t$.
This is quite a unique feature,
because not only $\omega_j$ but also $s_j$ should depend on $t$ if
a standard quadrature rule is used for approximating the indefinite integral.
As another beautiful feature,
the Sinc indefinite integration
can attain \emph{exponential} order of convergence,
which significantly exceeds polynomial order of convergence.
Leveraging these features,
Muhammad et al.~\cite{muhammad05:_numer}
considered the Sinc indefinite integration of $\Vol$,
say $\Vol_N$, and numerical solution $u_N$ that satisfies
the following equation
\[
 (\Ident - \Vol_N) u_N = g. 
\]
The procedure to obtain the solution $u_N$
is called the Sinc-Nystr\"{o}m method,
which is described in~\cref{sec:nystroem}.
Theoretical analysis has established that the method achieves
a convergence rate of $\Order(\exp(-\sqrt{\pi d N}))$~\cite{okayama13:_theo},
where $d$ indicates the size of the domain in which
the solution $u$ is analytic.

Another numerical solution for~\cref{eq:Volterra-int}
via Sinc numerical methods
was developed by
Rashidinia and Zarebnia~\cite{rashidinia07:_solut}.
They derived their method following quite a standard collocation procedure
based on the Sinc approximation (a function approximation formula),
which also attains an exponential order of convergence.
Let $\mathcal{P}_N f$ denote the Sinc approximation of $f$.
%Setting their approximate solution $w_N$
%in the expansion form of the Sinc approximation,
%they substituted $w_N$ into~\cref{eq:Volterra-int}
%with approximating $\Vol w_N$ by $\Vol_N w_N$.
%Then, sampling 
%and $w_N$ be their numerical solution.
Then, as shown in this paper,
the equation to be solved is written symbolically as
\[
 (\Ident - \mathcal{P}_N \Vol_N) w_N = \mathcal{P}_N g .
\]
The procedure to obtain the solution $w_N$
is called the Sinc-collocation method,
which is described in~\cref{sec:collocation}.
Although a theoretical error analysis of the method was
given~\cite{zarebnia10:_conv},
its convergence was not strictly proved.
%In view of~\cref{eq:Sinc-Nystroem}
%and~\cref{eq:Sinc-collocation},
%easily we find $u_N \neq v_N$.

Yet another numerical solution for~\cref{eq:Volterra-int}
via Sinc numerical methods
was developed by Stenger~\cite{stenger93:_numer}.
Although his method was introduced more than a decade
before the above methods,
it has received considerably less attention.
This may be because the target equation
of his method is not exactly~\cref{eq:Volterra-int}.
The interest of his method is in initial value problems
\begin{align*}
 u'(t) &= \tilde{k}(t)u(t) + \tilde{g}(t),\\
 u(a) &= u_a,
\end{align*}
which can be reduced to a form of Volterra
integral equations of the second kind as
\begin{equation}
 u(t) - \int_a^t \tilde{k}(s) u(s)\diff s = g(t),
\label{eq:Volterra-initial-val}
\end{equation}
where $g(t)=u_a + \int_a^t\tilde{g}(s)\diff s$.
Because the kernel here ($\tilde{k}$) is a function of a single variable,
Stenger's method does not appear to cover the general
case as~\cref{eq:Volterra-int}.
%, i.e.,
%the kernel is a function of two variables.
However, aside from theoretical justification,
it is relatively evident that
his method remains implementable even when
the kernel is a function of two variables.
Its numerical solution, say $v_N$, is determined in the following two steps:
(i) obtain the Sinc-Nystr\"{o}m solution $u_N$,
and (ii) apply the Sinc approximation to $u_N$.
%then the solution is written as $v_N=\mathcal{P}_N u_N$.
The step (i) implies that Stenger's method is based on the
Sinc-Nystr\"{o}m method, but $v_N$ is not equal to $u_N$
because of the step (ii).
The detailed procedure is described in~\cref{sec:collocation}
(strictly speaking, it is the first time that
the explicit procedure for the general case~\cref{eq:Volterra-int}
is presented).
Its convergence has been stated~\cite{stenger93:_numer}
assuming that the kernel is a function of a single variable.
%The assumption on the kernel seems essential
%because it is fully used in his theoretical discussion of the method.
%Contrary to the idea, this paper theoretically justifies the use of
%his method in general cases, i.e., the kernel is
%a function of two variables.

As seen above, three numerical methods have been proposed based on
the Sinc numerical methods: Sinc-Nystr\"{o}m method,
Sinc-collocation method, and Stenger's method.
Therefore, a question may naturally arise:
\emph{what is the difference (or similarity), and which method is the best?}
The first objective of this study is to investigate this question
from both theoretical and practical perspectives.
This study first reveals that Stenger's method
can be regarded as another Sinc-collocation method.
Then, it is shown that
Stenger's method and Rashidinia--Zarebnia's method
coincide at the collocation points, but they are not generally equivalent.
Furthermore,
this study shows that the convergence rate of the two Sinc-collocation methods
is exactly the same: $\Order(\sqrt{N}\exp(-\sqrt{\pi d \alpha N}))$,
where $\alpha$ is the order of H\"{o}lder continuous
with $0<\alpha\leq 1$.
From an implementation perspective, Stenger's method is preferable,
because it is simpler and easier to implement than
the method by Rashidinia and Zarebnia.

Thus, we only have to compare two methods:
Sinc-Nystr\"{o}m method and Stenger's Sinc-collocation method.
Even when $\alpha=1$,
the convergence rate of Stenger's method is slightly lower
than that of the Sinc-Nystr\"{o}m method.
However, numerical experiments indicate that the Sinc-Nystr\"{o}m method
requires much computation time to obtain the same accuracy
as Stenger's method.
This is primarily because the basis functions of the
Sinc-Nystr\"{o}m method include the sine integral (a special function),
which requires a high computational cost.
Based on this finding, we conclude that Stenger's method
is preferable among the three methods described above.

The second objective of this study is to improve Stenger's method.
In the aforementioned three methods, the tanh transformation
\begin{equation}
t = \SEt(x) = \frac{b-a}{2}\tanh\left(\frac{x}{2}\right) + \frac{b+a}{2}
\label{eq:SEt}
\end{equation}
is employed in common to map $\mathbb{R}$ onto the target interval $(a,b)$.
This is because Sinc numerical methods
are originally defined over the entire real axis $\mathbb{R}$.
Therefore, for the finite interval,
a variable transformation such as~\cref{eq:SEt} is required.
This study aims to improve Stenger's method by replacing the tanh
transformation with
\begin{equation}
t = \DEt(x)
 = \frac{b-a}{2}\tanh\left(\frac{\pi}{2}\sinh x\right) + \frac{b+a}{2},
\label{eq:DEt}
\end{equation}
which is called the double-exponential (DE) transformation.
%Although the transformation was originally proposed
%for numerical integration~\cite{takahasi74:_doubl},
The convergence rates of
various methods via Sinc numerical methods have been improved
by replacing the tanh transformation with
the DE transformation~\cite{mori01:_doubl,sugihara04:_recen}.
Specifically, the convergence rate of the Sinc-Nystr\"{o}m method
was enhanced to $\Order(\log(2 d N)\exp(-\pi d N/\log(2 d N))/N)$
through the replacement~\cite{okayama13:_theo}.
On the basis of the observation,
this study develops a new Sinc-collocation method
combined with the DE transformation.
Furthermore, this study performs theoretical analysis
of the proposed method and shows that its convergence rate
is $\Order(\exp(-\pi d N/\log(2 d N/\alpha)))$,
which significantly exceeds that of Stenger's method.
Although
the rate is slightly lower than that of the Sinc-Nystr\"{o}m method
combined with the DE transformation,
numerical experiments indicate that the Sinc-Nystr\"{o}m method
requires much computation time to obtain the same accuracy
as the proposed method.
This is similarly observed when comparing the Sinc-Nystr\"{o}m and
Sinc-collocation methods combined with the tanh transformation.

The remainder of this paper is organized as follows.
In~\cref{sec:preliminary}, as a preliminary,
convergence theorems of the Sinc approximation
and the Sinc indefinite integration are described.
In~\cref{sec:nystroem}, the Sinc-Nystr\"{o}m
methods developed by Muhammad et al.~\cite{muhammad05:_numer}
are described, and their convergence theorems are stated.
In~\cref{sec:collocation}, the Sinc-collocation methods
developed by Stenger~\cite{stenger93:_numer} and
Rashidinia--Zarebnia~\cite{rashidinia07:_solut} are described.
Subsequently, new theoretical results from this study are stated:
(i) the two numerical solutions coincide at the collocation points
but are not generally equivalent, and
(ii) the two methods attain the same convergence rate
$\Order(\sqrt{N}\exp(-\sqrt{\pi d \alpha N}))$.
In~\cref{sec:de-collocation}, a new Sinc-collocation method
combined with the DE transformation is developed.
Subsequently, its convergence theorem is stated claiming that
the convergence rate is
$\Order(\exp(-\pi d  N/\log(2 d N/\alpha)))$.
In~\cref{sec:numer-result}, numerical experiments are presented,
where the DE-Sinc-collocation method demonstrates the best performance.
In~\cref{sec:proof-SE}, proofs for the new theorems
presented in~\cref{sec:collocation} are provided.
In~\cref{sec:proof-DE},
proofs for the new theorems
presented in~\cref{sec:de-collocation} are provided.
%In~\cref{sec:conclusion},
%conclusion and future work of this study are described.
