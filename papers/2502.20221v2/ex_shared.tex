% SIAM Shared Information Template
% This is information that is shared between the main document and any
% supplement. If no supplement is required, then this information can
% be included directly in the main document.


% Packages and macros go here
\usepackage{lipsum}
\usepackage{amsfonts}
\usepackage{graphicx}
\usepackage{epstopdf}
\usepackage{algorithmic}
\ifpdf
  \DeclareGraphicsExtensions{.eps,.pdf,.png,.jpg}
\else
  \DeclareGraphicsExtensions{.eps}
\fi

% Add a serial/Oxford comma by default.
\newcommand{\creflastconjunction}{, and~}

% Used for creating new theorem and remark environments
\newsiamremark{remark}{Remark}
\newsiamremark{hypothesis}{Hypothesis}
\crefname{hypothesis}{Hypothesis}{Hypotheses}
\newsiamthm{claim}{Claim}

% Sets running headers as well as PDF title and authors
\headers{Relationship between two Sinc-collocation methods}{T. Okayama}

% Title. If the supplement option is on, then "Supplementary Material"
% is automatically inserted before the title.
\title{On the relationship between two Sinc-collocation methods for Volterra
 integral equations of the second kind and their further
 improvement\thanks{Submitted to the editors \today.
\funding{This work was partially supported by JSPS Grant-in-Aid for Scientific Research (C) JP23K03218.}}}

% Authors: full names plus addresses.
\author{Tomoaki Okayama\thanks{Hiroshima City University, Hiroshima, Japan
  (\email{okayama@hiroshima-cu.ac.jp}, \url{http://www.hiroshima-cu.ac.jp/\string~okayama/}).}
}

\usepackage{amsopn}

%added by the author
\usepackage{mathrsfs}
\usepackage{upgreek}
\newcommand{\mathbd}[1]{\boldsymbol{#1}}
\newcommand{\divv}{\mathrm{d}}
\newcommand{\diff}{\,\divv}
\newcommand{\imnum}{\mathrm{i}\,}
\renewcommand{\Re}{\operatorname{Re}}
\renewcommand{\Im}{\operatorname{Im}}
\renewcommand{\pi}{\uppi\>\!}
\DeclareMathOperator{\Order}{O}
\DeclareMathOperator{\rme}{e}
\DeclareMathOperator{\Si}{Si}
\newcommand{\domD}{\mathscr{D}}
\newcommand{\MC}{\mathbf{M}}
\newcommand{\Hinf}{\mathbf{H}^{\infty}}
\newcommand{\Ident}{\mathcal{I}}
\newcommand{\textRZ}{\text{\tiny{\rm{RZ}}}}
\newcommand{\textSE}{\text{\tiny{\rm{SE}}}}
\newcommand{\textDE}{\text{\tiny{\rm{DE}}}}
\newcommand{\Vol}{\mathcal{V}}
\newcommand{\VolSEn}{\mathcal{V}_N^{\textSE}}
\newcommand{\VolDEn}{\mathcal{V}_N^{\textDE}}
\newcommand{\SEt}{\psi^{\textSE}}
\newcommand{\DEt}{\psi^{\textDE}}
\newcommand{\SEtInv}{\phi^{\textSE}}
\newcommand{\DEtInv}{\phi^{\textDE}}
\newcommand{\SEtDiv}{\{\SEt\}'}
\newcommand{\DEtDiv}{\{\DEt\}'}
\newcommand{\ProjSE}{\mathcal{P}_N^{\textSE}}
\newcommand{\ProjRZ}{\mathcal{P}_N^{\textRZ}}
\newcommand{\ProjDE}{\mathcal{P}_N^{\textDE}}
\newcommand{\vRZn}{w_N^{\textRZ}}
\newcommand{\vSEn}{v_N^{\textSE}}
\newcommand{\vDEn}{v_N^{\textDE}}
\newcommand{\uSEn}{u_N^{\textSE}}
\newcommand{\uDEn}{u_N^{\textDE}}
\newcommand{\tSE}{t^{\textSE}}
\newcommand{\tDE}{t^{\textDE}}
\newcommand{\kSEn}{V_{n}^{\textSE}}
\newcommand{\gSEn}{\mathbd{g}_n^{\textSE}}
\newcommand{\eDEn}{E_{n}^{\textDE}}
\newcommand{\kDEn}{V_{n}^{\textDE}}
\newcommand{\gDEn}{\mathbd{g}_n^{\textDE}}
%added by the author

%%% Local Variables: 
%%% mode:latex
%%% TeX-master: "ex_article"
%%% End: 
