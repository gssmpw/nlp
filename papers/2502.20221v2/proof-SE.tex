\section{Proofs for the theorems presented in~\cref{sec:collocation}}
\label{sec:proof-SE}

In this section,
we provide proofs for~\cref{thm:equivalence,thm:SE-Sinc-collocation}.

\subsection{Proof of~\cref{thm:equivalence}}
\label{sec:proof-equivalence}

In addition to the given equation $(\Ident - \Vol)u = g$,
let us consider the following three equations:
\begin{align}
 (\Ident - \VolSEn)    \uSEn &= g, \label{eq:SE-Sinc-Nystroem-symbol} \\
 (\Ident - \ProjSE \VolSEn)v &= \ProjSE g,
 \label{eq:SE-Sinc-collocation-symbol}\\
% (E_m^{\textRZ} - V_{m}^{\textRZ})\mathbd{c}_m &= \mathbd{g}_m^{\textSE},
 (\Ident - \ProjRZ \VolSEn)w &= \ProjRZ g,
\label{eq:RZ-Sinc-collocation-symbol}
\end{align}
where $\VolSEn$ and $\ProjSE$ are defined
by~\cref{eq:VolSEn} and~\cref{eq:ProjSE}, respectively,
and $\ProjRZ$ is defined by
\begin{align}
\label{eq:ProjRZ}
  \ProjRZ[f](t)
&=\sum_{j=-N+1}^{N+1}\left\{f(\tSE_j) - \beta_N \omega_a(\tSE_j)
 - \gamma_N \omega_a(\tSE_j)\right\}S(j,h)(\SEtInv(t))\\
&\quad + \beta_N \omega_a(t) + \gamma_N \omega_b(t),
\nonumber
\end{align}
where $\beta_N$ and $\gamma_N$ are defined by
\begin{align*}
 \beta_N &=
\frac{f(\tSE_{-N})\omega_b(\tSE_{N}) - f(\tSE_{N})\omega_b(\tSE_{-N})}
     {\omega_a(\tSE_{-N})\omega_b(\tSE_{N}) - \omega_b(\tSE_{-N})\omega_a(\tSE_{N})},\\
\gamma_N &=
\frac{f(\tSE_{N})\omega_a(\tSE_{-N}) - f(\tSE_{-N})\omega_a(\tSE_{N})}
     {\omega_a(\tSE_{-N})\omega_b(\tSE_{N}) - \omega_b(\tSE_{-N})\omega_a(\tSE_{N})}.
\end{align*}
\begin{remark}
The denominator of $\beta_N$ and $\gamma_N$ is not zero
because
\begin{align*}
 \omega_a(\tSE_{-N})\omega_b(\tSE_{N}) - \omega_b(\tSE_{-N})\omega_a(\tSE_{N})
&=(1 - \omega_b(\tSE_{-N}))\omega_b(\tSE_N)
- \omega_b(\tSE_{-N})(1 - \omega_b(\tSE_N))\\
&=\omega_b(\tSE_N) - \omega_b(\tSE_{-N})\\
&= \tanh\left(\frac{Nh}{2}\right)\neq 0,
\end{align*}
provided that $N$ is a positive integer and $h> 0$.
\end{remark}

Because~\cref{eq:SE-Sinc-Nystroem-symbol} is equivalent
to~\cref{eq:SE-Sinc-Nystroem},
the solution of~\cref{eq:SE-Sinc-Nystroem-symbol}
is the approximate solution of the SE-Sinc-Nystr\"{o}m method.
On~\cref{eq:SE-Sinc-Nystroem-symbol}, the following result was obtained.

\begin{lemma}[Okayama et al.~{\cite[Lemma~6.7]{okayama13:_theo}}]
\label{lem:SE-Sinc-Nystroem}
Assume that
all the assumptions of~\cref{thm:SE-Sinc-Nystroem}
are fulfilled.
Then, there exists a positive integer $N_0$ such that for all $N\geq N_0$,
\cref{eq:SE-Sinc-Nystroem-symbol}
has a unique solution $\uSEn\in C([a,b])$.
Furthermore, there exists a constant $C$ independent of $N$ such that
for all $N\geq N_0$,
\begin{equation}
\label{eq:SE-Sinc-Nystroem-error}
 \|u - \uSEn\|_{C([a,b])}\leq C \|\Vol u - \VolSEn u\|_{C([a,b])}.
\end{equation}
\end{lemma}

This lemma says that~\cref{eq:SE-Sinc-Nystroem-symbol}
has a unique solution for all sufficiently large $N$.
Using this result, we show the following three things:
\begin{enumerate}
 \item[(i)] If~\cref{eq:SE-Sinc-Nystroem-symbol}
has a unique solution,
then~\cref{eq:SE-Sinc-collocation-symbol}
has also a unique solution $v = \vSEn$.
 \item[(ii)] If~\cref{eq:SE-Sinc-Nystroem-symbol}
has a unique solution,
then~\cref{eq:RZ-Sinc-collocation-symbol}
has also a unique solution $w = \vRZn$.
 \item[(iii)] The two solutions $\vSEn$ and $\vRZn$ are not generally
equivalent, but at the collocation points,
 $\vSEn(\tSE_i) = \vRZn(\tSE_{i})$ $(i=-N,\,\ldots,\,N)$ holds.
\end{enumerate}
First, we show (i) as follows.

\begin{lemma}
\label{lem:Stenger-solution}
The following two statements are equivalent:
\begin{enumerate}
 \item[{\rm (A)}] \Cref{eq:SE-Sinc-Nystroem-symbol}
has a unique solution $\uSEn\in C([a,b])$.
 \item[{\rm (B)}] \Cref{eq:SE-Sinc-collocation-symbol} has
 a unique solution $v\in C([a,b])$.
\end{enumerate}
Furthermore, $v=\vSEn$ holds.
\end{lemma}
\begin{proof}
First, let us show $\mathrm{(A)} \Rightarrow \mathrm{(B)}$.
Note that $\VolSEn\ProjSE f = \VolSEn f$ holds
because
of the interpolation property $\ProjSE[f](\tSE_i)=f(\tSE_i)$
($i=-N,\,\ldots,\,N$).
Applying $\ProjSE$ on the both sides
of~\cref{eq:SE-Sinc-Nystroem-symbol},
we have
\[
 \ProjSE\uSEn
 = \ProjSE(g + \VolSEn\uSEn)
 = \ProjSE(g + \VolSEn\ProjSE\uSEn),
\]
which is equivalent to $\vSEn = \ProjSE(g + \VolSEn \vSEn)$
(recall that $\vSEn = \ProjSE\uSEn$).
This equation implies
that~\cref{eq:SE-Sinc-collocation-symbol} has a solution
$\vSEn\in C([a,b])$.

Next, we show the uniqueness.
Suppose that~\cref{eq:SE-Sinc-collocation-symbol}
has another solution $\tilde{v}\in C([a,b])$.
Let us set a function $\tilde{u}$ as
$\tilde{u} = g + \VolSEn\tilde{v}$.
Because $\tilde{v}$ is a solution of~\cref{eq:SE-Sinc-collocation-symbol},
we have
\[
 \tilde{v} = \ProjSE(g + \VolSEn\tilde{v}) = \ProjSE \tilde{u},
\]
from which it holds that
\[
 \tilde{u} = g + \VolSEn\tilde{v} = g + \VolSEn\ProjSE\tilde{u}
= g + \VolSEn\tilde{u}.
\]
This equation implies that $\tilde{u}$ is a solution
of~\cref{eq:SE-Sinc-Nystroem-symbol}.
Because the solution of~\cref{eq:SE-Sinc-Nystroem-symbol}
is unique, $\tilde{u} = u$ holds,
from which we have $\ProjSE\tilde{u}=\ProjSE u$.
Thus, we have $\tilde{v}=v$, which shows $(\mathrm{B})$.

The above argument is reversible,
which proves $\mathrm{(B)}\Rightarrow\mathrm{(A)}$.
Furthermore, in view of the proof above,
we see $v=\vSEn$, which is to be demonstrated.
\end{proof}

Next, for showing (ii),
we show the following result.
The proof is omitted because it goes in the same way as
that of~\cref{lem:Stenger-solution}.

\begin{lemma}
\label{lem:RZ-solution}
The following two statements are equivalent:
\begin{enumerate}
 \item[{\rm (A)}] \Cref{eq:SE-Sinc-Nystroem-symbol}
has a unique solution $\uSEn\in C([a,b])$.
 \item[{\rm (B)}] \Cref{eq:RZ-Sinc-collocation-symbol} has
 a unique solution $w\in C([a,b])$.
\end{enumerate}
Furthermore, $w=\ProjRZ\uSEn$ holds.
\end{lemma}

To show (ii) completely, we further have to show $w = \vRZn$,
which is done by the following result.
Noting $\ProjRZ[f](\tSE_i)=f(\tSE_i)$ $(i=-N,\,\ldots,\,N)$,
we can prove this result
following Atkinson~\cite[Sect.\ 4.3]{atkinson97:_numer_solut},
and hence the proof is omitted.

\begin{proposition}
\label{prop:RZ-solution}
The following two statements are equivalent:
\begin{enumerate}
 \item[{\rm (A)}] \Cref{eq:RZ-Sinc-collocation-symbol} has
 a unique solution $w\in C([a,b])$.
 \item[{\rm (B)}] \Cref{eq:RZ-linear-eq} has
 a unique solution $\mathbd{c}_m\in \mathbb{R}^m$.
\end{enumerate}
Furthermore, $w=\vRZn$ holds.
\end{proposition}
%\begin{proof}
%First, let us show $\mathrm{(A)} \Rightarrow \mathrm{(B)}$.
%Note that $(\ProjSE)^2 f = \ProjSE f$ holds because
%of the interpolation property $\ProjSE[f](\tSE_i)=f(\tSE_i)$
%($i=-N,\,\ldots,\,N$).
%Because the solution $v$ of~\cref{eq:SE-Sinc-collocation-symbol}
%is written as
%\begin{equation}
%\label{eq:v-is-projected-by-ProjSE}
% v = \ProjSE (g + \VolSEn v),
%\end{equation}
%we have $\ProjSE v = (\ProjSE)^2 (g + \VolSEn v) = \ProjSE (g + \VolSEn v)= v$.
%%Therefore, there exist coefficients
%%$\mathbd{c}_m=[c_{-N-1},\,c_{-N},\,\ldots,\,c_{N+1}]^{\mathrm{T}}$
%%such that
%%\[
%% v(t) = c_{-N-1}\omega_a(t)
%% + \sum_{j=-N}^N c_j S(j,h)(\SEtInv(t)) + c_{N+1}\omega_b(t).
%%\]
%Therefore, we can rewrite~\cref{eq:v-is-projected-by-ProjSE} as
%\[
% \ProjSE( v - g - \VolSEn v) = 0.
%\]
%Put $f = v - g - \VolSEn v$.
%The equation $\ProjSE f = 0$ requires the following equations
%\[
% \ProjSE[f](\tSE_i) = f(\tSE_i) = 0,\quad i=-N,\,\ldots,\,N.
%\]
%Considering more two points $t=\tSE_{-N-1}$ and $t=\tSE_{N+1}$,
%we have
%\begin{align*}
% \ProjSE [f](\tSE_{-N-1})
%& = f(\tSE_{-N})\omega_a(\tSE_{-N-1}) + f(\tSE_{N})\omega_a(\tSE_{-N-1})
% = 0,\\
% \ProjSE [f](\tSE_{N+1})
%& = f(\tSE_{-N})\omega_a(\tSE_{N+1}) + f(\tSE_{N})\omega_a(\tSE_{N+1})
% = 0,
%\end{align*}
%because $f(\tSE_{-N})=f(\tSE_{N})=0$.
%Thus, $\ProjSE f = 0$ implies
%\[
% f(\tSE_i) = 0,\quad i = -N-1,\,-N,\,\ldots,\,N,\,N+1,
%\]
%which is equivalent to
%\begin{equation}
%\label{eq:v-N+1-equation}
%v(\tSE_i) - \VolSEn[v](\tSE_i) = g(\tSE_i),\quad i=-N-1,\,-N,\,\ldots,\,N,\,N+1.
%\end{equation}
%Here, note that $v$ is written as
%\begin{align*}
% v(t) &= \sum_{j=-N}^N
%\left\{v(\tSE_j)
% - v(\tSE_{-N})\omega_a(\tSE_j) - v(\tSE_N)\omega_b(\tSE_j)
%\right\}S(j,h)(\SEtInv(t))\\
%&\quad + v(\tSE_{-N})\omega_a(t) + v(\tSE_N)\omega_b(t),
%\end{align*}
%because $v=\ProjSE v$.
%Putting
%\begin{align*}
% c_{-N-1} &= v(\tSE_{-N}),\\
% c_i      &= v(\tSE_i) - v(\tSE_{-N})\omega_a(\tSE_i) - v(\tSE_N)\omega_b(\tSE_i),
%\quad i = -N,\,\ldots,\,N,\\
% c_{N+1}  &= v(\tSE_N),
%\end{align*}
%we find that~\cref{eq:v-N+1-equation} is nothing but
%the linear system~\cref{eq:RZ-linear-eq}.
%Thus, \cref{eq:RZ-linear-eq} has a solution $\mathbd{c}_m\in \mathbb{R}^m$.
%
%Next, we show the uniqueness.
%Suppose that~\cref{eq:RZ-linear-eq}
%has another solution $\mathbd{\tilde{c}}_m\in\mathbb{R}^m$.
%Let us set a function $\tilde{v}\in C([a,b])$ as
%\[
% \tilde{v}(t)
%= \sum_{j=-N}^N \tilde{c}_j S(j,h)(\SEtInv(t))
% + \tilde{c}_{-N-1}\omega_a(t) + \tilde{c}_{N+1}\omega_b(t).
%\]
%Because $\mathbd{\tilde{c}}_m\in\mathbb{R}^m$ is a solution
%of~\cref{eq:RZ-linear-eq}, we have
%%\begin{align*}
%%& \sum_{j=-N}^N\left\{
%%\delta_{ij} - h k(\tSE_i,\tSE_j)\SEtDiv(jh)\delta^{(-1)}_{i-j}
%%\right\}\tilde{c}_j\\
%%&\quad
%%+ \left\{\omega_a(\tSE_i) - \VolSEn[\omega_a](\tSE_i)\right\}\tilde{c}_{-N-1}\\
%%&\quad
%%+ \left\{\omega_b(\tSE_i) - \VolSEn[\omega_b](\tSE_i)\right\}\tilde{c}_{N+1}
%%= g(\tSE_i),
%%\quad i=-N-1,\,-N,\,\ldots,\,N,\,N+1.
%%\end{align*}
%\begin{align*}
%\tilde{v}(\tSE_i) - \VolSEn[\tilde{v}](\tSE_i) = g(\tSE_i),\quad
% i=-N-1,\,-N,\,\ldots,\,N,\,N+1.
%\end{align*}
%Put $\tilde{f}= \tilde{v} - g - \VolSEn\tilde{v}$.
%From $\tilde{f}(\tSE_i)=0$ $(i=-N-1,\,-N,\,\ldots,\,N,\,N+1)$,
%we have $\ProjSE \tilde{f} = 0$, which is equivalent to
%\[
% \ProjSE(\tilde{v} - g -\VolSEn \tilde{v}) = 0.
%\]
%If $\ProjSE\tilde{v} =\tilde{v}$ is shown, then this equation is rewritten as
%\[
% \tilde{v} - \ProjSE\VolSEn\tilde{v} = \ProjSE g,
%\]
%which implies $\tilde{v}$ is a solution
%of~\cref{eq:SE-Sinc-collocation-symbol}.
%Because the solution of~\cref{eq:SE-Sinc-collocation-symbol}
%is unique, $\tilde{v} = v$ holds.
%This implies $\mathbd{\tilde{c}}_m = \mathbd{c}_m$,
%i.e., the solution of~\cref{eq:RZ-linear-eq} is unique,
%which shows $\mathrm{(B)}$.
%What remains to be shown is $\ProjSE\tilde{v} =\tilde{v}$.
%Set $\tilde{v}_i$ as the value of $\tilde{v}(\tSE_i)$, i.e.,
%\begin{equation}
%\label{eq:tilde-v-i}
%\tilde{v}_i
%=
% \tilde{c}_{-N-1}\omega_a(\tSE_i) + \tilde{c}_i
% + \tilde{c}_{N+1}\omega_b(\tSE_i),
%\quad i=-N,\,\ldots,\,N.
%\end{equation}
%Calculating $\ProjSE\tilde{v}$ by definition~\cref{eq:ProjSE}, we have
%\begin{align*}
% \ProjSE[\tilde{v}](t)
%&=\sum_{j=-N}^N \left\{\tilde{v}_j - \tilde{v}_{-N}\omega_a(\tSE_j)
%- \tilde{v}_{N}\omega_b(\tSE_j)
%\right\}S(j,h)(\SEtInv(t))\\
%&\quad + \tilde{v}_{-N}\omega_a(t) + \tilde{v}_{N}\omega_b(t).
%\end{align*}
%This function satisfies $\ProjSE[\tilde{v}](\tSE_i)=\tilde{v}(\tSE_i)$
%$(i=-N,\,\ldots,\,N)$ by itself.
%Furthermore, we require the equality at more two points,
%$t=\tSE_{-N-1}$ and $t=\tSE_{N+1}$.
%Substituting the two points into $\ProjSE[\tilde{v}](t)$, we have
%\begin{align*}
% \ProjSE[\tilde{v}](\tSE_{-N-1})
%&= \tilde{v}_{-N}\omega_a(\tSE_{-N-1}) + \tilde{v}_{N}\omega_b(\tSE_{-N-1}),\\
% \ProjSE[\tilde{v}](\tSE_{N+1})
%&= \tilde{v}_{-N}\omega_a(\tSE_{N+1}) + \tilde{v}_{N}\omega_b(\tSE_{N+1}),
%\end{align*}
%which must be equal to
%\begin{align*}
% \tilde{v}(\tSE_{-N-1})
%&= \tilde{c}_{-N-1}\omega_a(\tSE_{-N-1}) +\tilde{c}_{N+1}\omega_b(\tSE_{-N-1}),
%\\
% \tilde{v}(\tSE_{N+1})
%&= \tilde{c}_{-N-1}\omega_a(\tSE_{N+1}) +\tilde{c}_{N+1}\omega_b(\tSE_{N+1}),
%\end{align*}
%respectively.
%Therefore, we have $\tilde{v}_{-N}=\tilde{c}_{-N-1}$
%and $\tilde{v}_N = \tilde{c}_{N+1}$,
%from which it holds that
%\begin{align*}
% \ProjSE[\tilde{v}](t)
%&=\sum_{j=-N}^N \left\{\tilde{v}_j - \tilde{c}_{-N-1}\omega_a(\tSE_j)
%- \tilde{c}_{N+1}\omega_b(\tSE_j)
%\right\}S(j,h)(\SEtInv(t))\\
%&\quad + \tilde{c}_{-N-1}\omega_a(t) + \tilde{c}_{N+1}\omega_b(t)\\
%&= \tilde{v}(t),
%\end{align*}
%where~\cref{eq:tilde-v-i} is used at the last equality.
%This completes the proof for $\mathrm{(A)}\Rightarrow\mathrm{(B)}$.
%
%The proof for $\mathrm{(B)}\Rightarrow\mathrm{(A)}$
%is omitted because it goes in the same manner as that for
%$\mathrm{(A)}\Rightarrow\mathrm{(B)}$.
%Furthermore, in view of the proof above,
%we see $\tilde{v}=\vRZn$, which is to be demonstrated.
%\end{proof}

From the above results (i) and (ii),
we find that $\vSEn = \ProjSE\uSEn$ and $\vRZn = \ProjRZ\uSEn$.
Using the interpolation property of $\ProjSE$ and $\ProjRZ$ as
\[
 \ProjSE[\uSEn](\tSE_i) = \uSEn(\tSE_i) = \ProjRZ[\uSEn](\tSE_i),
\quad i = -N,\,-N+1,\,\ldots,\,N,
\]
we have $\vSEn(\tSE_i)=\vRZn(\tSE_i)$.
However, we note that $\ProjSE$ and $\ProjRZ$ is not
generally equivalent. This can be observed
by the limits $t\to a$ and $t\to b$ as
\begin{align*}
 \lim_{t\to a}\ProjSE[f](t) = f(\tSE_{-N})
&\neq \beta_N =\lim_{t\to a}\ProjRZ[f](t),\\
 \lim_{t\to b}\ProjSE[f](t) = f(\tSE_{N})
&\neq \gamma_N =\lim_{t\to b}\ProjRZ[f](t).
\end{align*}
Thus, we obtain the claim of~\cref{thm:equivalence}.

\subsection{Proof of~\cref{thm:SE-Sinc-collocation}}
\label{sec:proof-SE-Sinc}

The invertibility of $(I_n - \kSEn)$
is already shown by~\cref{thm:SE-Sinc-Nystroem}.
Thus, we concentrate on the analysis of the error of $\vSEn$.
Because $\vSEn = \ProjSE \uSEn$, it holds that
\[
 u - \vSEn = u - \ProjSE\uSEn
= (u - \ProjSE u) + \ProjSE(u - \uSEn),
\]
which leads to
\begin{equation}
\label{eq:SE-Sinc-first-error}
 \|u - \vSEn\|_{C([a,b])}
\leq \|u - \ProjSE u\|_{C([a,b])}
 + \|\ProjSE\|_{\mathcal{L}(C([a,b]),C([a,b]))}
\|u - \uSEn\|_{C([a,b])}.
\end{equation}
For the first term, we show $u\in\MC_{\alpha}(\SEt(\domD_d))$,
from which we can use~\cref{thm:SE-Sinc-general}.
For the purpose, the following theorem is useful.

\begin{theorem}[Okayama et al.~{\cite[Theorem~3.2]{okayama13:_theo}}]
\label{thm:Sinc-Nyst-regularity}
Let $\domD=\SEt(\domD_d)$ or $\domD=\DEt(\domD_d)$.
Assume that $g$, $k(z,\cdot)$ and $k(\cdot,w)$ belong to $\Hinf(\domD)$
for all $z,\, w\in\domD$.
Then,~\cref{eq:Volterra-int} has a unique solution $u\in\Hinf(\domD)$.
\end{theorem}

Using this theorem, we can show the following result.

\begin{theorem}
\label{thm:Sinc-collocation-regularity}
Let $\alpha$ be a positive constant with $\alpha\leq 1$.
Assume that all the assumptions of~\cref{thm:Sinc-Nyst-regularity}
are fulfilled. Furthermore,
assume that $g$ and $k(\cdot, w)$ belong to $\MC_{\alpha}(\domD)$
for all $w\in\domD$.
Then,~\cref{eq:Volterra-int} has a unique solution $u\in\MC_{\alpha}(\domD)$.
\end{theorem}
\begin{proof}
According to~\cref{thm:Sinc-Nyst-regularity},~\cref{eq:Volterra-int}
has a unique solution $u\in\Hinf(\domD)$.
Therefore, we only have to show the H\"{o}lder continuity
of $u$ at the endpoints.
Using $u = g + \Vol u$, we have
\begin{align*}
&|u(b) - u(z)|\\
&=\left| \left(g(b) + \int_a^b k(b,w)u(w)\diff w\right)
- \left(g(z) + \int_a^z k(z,w)u(w)\diff w\right)\right|\\
&\leq \left|g(b) - g(z)\right|
+ \left|\int_z^b k(b,w)u(w)\diff w\right|
+ \left|\int_a^z \left\{k(b,w)- k(z,w)\right\}u(w)\diff w\right|.
\end{align*}
From the H\"{o}lder continuity of $g$,
the first term can be bounded by
$L_g|b - z|^{\alpha}$ for some constant $L_g$.
From the boundedness of $k$ and $u$,
the second term can be bounded by
$L_{k,u}|b - z|$ for some constant $L_{k,u}$.
Furthermore, from the boundedness of $\domD$ and $\alpha\leq 1$,
we have
$|b - z|=|b - z|^{1-\alpha}|b - z|^{\alpha}\leq L_{\domD}|b - z|^{\alpha}$
for some constant $L_{\domD}$.
From the H\"{o}lder continuity of $k$ and
boundedness of $u$,
the third term can be bounded by
$\tilde{L}_{k,u}|b - z|^{\alpha}|z - a|$ for some constant $\tilde{L}_{k,u}$.
Furthermore, from the boundedness of $\domD$,
we have $|z - a|\leq \tilde{L}_{\domD}$ for some constant $\tilde{L}_{\domD}$.
Thus, there exists a constant $L$ such that
$|u(b) - u(z)|\leq L|b - z|^{\alpha}$,
which shows the H\"{o}lder continuity of $u$ at $z = b$.
The proof for the H\"{o}lder continuity at $z = a$ is omitted
because it follows the same method as that at $z = b$.
This completes the proof.
\end{proof}

From this theorem, we can use~\cref{thm:SE-Sinc-general}
for estimating the first term of~\cref{eq:SE-Sinc-first-error} as
\[
 \|u - \ProjSE u\|_{C([a,b])}
\leq C_1 \sqrt{N} \rme^{-\sqrt{\pi d \alpha N}}
\]
for some constant $C_1$.
For the second term, we use the following bound for
the operator $\ProjSE$.

\begin{lemma}[Okayama~{\cite[Lemma~7.2]{okayama23:_theo}}]
Let $\ProjSE$ be defined by~\eqref{eq:ProjSE}.
Then, there exists a constant $C_2$ independent of $N$ such that
\[
 \|\ProjSE\|_{\mathcal{L}(C([a,b]),C([a,b]))}
\leq C_2 \log(N+1).
\]
\end{lemma}

The remaining term to be estimated in~\cref{eq:SE-Sinc-first-error}
is $\|u - \uSEn\|_{C([a,b])}$.
According to~\cref{lem:SE-Sinc-Nystroem},
it is estimated as~\cref{eq:SE-Sinc-Nystroem-error}.
Because $u\in\Hinf(\SEt(\domD_d))$, $u$ satisfies the assumptions
of~\cref{thm:SE-Sinc-indefinite},
from which we have
\[
 \|\Vol u - \VolSEn u\|_{C([a,b])}
\leq C_3 \rme^{-\sqrt{\pi d \alpha N}}.
\]
Thus, there exists a constant $C_4$ such that
\begin{align*}
 \|u - \vSEn\|_{C([a,b])}
&\leq C_1 \sqrt{N} \rme^{-\sqrt{\pi d \alpha N}}
+ C_2 \log(N+1) C_3 \rme^{-\sqrt{\pi d \alpha N}}\\
&\leq C_4 \sqrt{N}\rme^{-\sqrt{\pi d \alpha N}}.
\end{align*}
This completes the proof of~\cref{thm:SE-Sinc-collocation}.

\subsection{Proof of~\cref{thm:RZ-Sinc-collocation}}
\label{sec:proof-RZ-Sinc}

For~\cref{thm:RZ-Sinc-collocation},
the invertibility of $(E_n^{\textRZ} - V_{n}^{\textRZ})$
can be shown by combining~\cref{lem:SE-Sinc-Nystroem,lem:RZ-solution}
and~\cref{prop:RZ-solution}.
Thus, we concentrate on the analysis of the error of $\vRZn$.
By the triangle inequality, we have
\begin{align*}
 \|u(t) - \vRZn(t)\|_{C([a,b])}
&\leq \|u(t) - \vSEn(t)\|_{C([a,b])}
 + \|\vSEn(t) - \vRZn(t)\|_{C([a,b])}\\
&= \|u(t) - \vSEn(t)\|_{C([a,b])}
 + \|\ProjSE\uSEn(t) - \ProjRZ\uSEn(t)\|_{C([a,b])}.
\end{align*}
Because the first term
% $\|u(t) - \vSEn(t)\|_{C([a,b])}$
is already estimated by~\cref{thm:SE-Sinc-collocation},
we estimate the second term.
For the purpose, the following lemma is essential.

\begin{lemma}
Let $\ProjSE: C([a,b])\to C([a,b])$ and
$\ProjRZ: C([a,b])\to C([a,b])$ be defined
by~\cref{eq:ProjSE} and~\cref{eq:ProjRZ}, respectively.
Then, there exists a constant $C$ independent of $N$ such that
\[
 \left\|\ProjSE - \ProjRZ\right\|_{\mathcal{L}(C([a,b]),C([a,b]))}
\leq \frac{C}{\rme^{Nh} - 1}\log(N+1).
\]
\end{lemma}
\begin{proof}
First, it holds for $f\in C([a,b])$ that
\begin{align*}
& \ProjSE[f](t) - \ProjRZ[f](t)\\
&= - \sum_{j=-N}^N\left\{
\left(f(\tSE_{-N})-\beta_N\right)\omega_a(\tSE_j) +
\left(f(\tSE_{N})-\gamma_N\right)\omega_b(\tSE_j)
\right\}S(j,h)(\SEtInv(t))\\
&\quad+\left(f(\tSE_{-N}) - \beta_N\right)\omega_a(t)
+\left(f(\tSE_N) - \gamma_N\right)\omega_b(t).
\end{align*}
Here, noting
\begin{align*}
 |f(\tSE_{-N}) - \beta_N|
&=\frac{|f(\tSE_N) - f(\tSE_{-N})|}{\rme^{Nh} - 1}
\leq \frac{2\|f\|_{C([a,b])}}{\rme^{Nh} - 1},\\
 |f(\tSE_{N}) - \gamma_N|
&=\frac{|f(\tSE_{-N}) - f(\tSE_{N})|}{\rme^{Nh} - 1}
\leq \frac{2\|f\|_{C([a,b])}}{\rme^{Nh} - 1},
\end{align*}
and using $\omega_a(t) + \omega_b(t) = 1$, we have
\begin{align*}
& \left\|\ProjSE - \ProjRZ\right\|_{\mathcal{L}(C([a,b]),C([a,b]))}\\
&\leq \frac{2}{\rme^{Nh} - 1}
\left\{\sum_{j=-N}^N\left(\omega_a(\tSE_j) + \omega_b(\tSE_j)\right) |S(j,h)(\SEtInv(t))|
+\omega_a(t) + \omega_b(t)
\right\}\\
&= \frac{2}{\rme^{Nh} - 1}
\left\{\sum_{j=-N}^N |S(j,h)(\SEtInv(t))| + 1\right\}\\
&\leq \frac{2}{\rme^{Nh} - 1}
\left\{\frac{2}{\pi}(3 + \log N) + 1\right\},
\end{align*}
where the standard bound~\cite[Problem 3.1.5 (a)]{stenger93:_numer}
is used for the last inequality.
Thus, the claim follows.
\end{proof}

From this lemma, substituting~\cref{eq:h-SE} into $h$,
we estimate the second term as
\[
 \|(\ProjSE - \ProjRZ) \uSEn\|_{C([a,b])}
\leq \frac{C}{1 - \rme^{-\sqrt{\pi d / \alpha}}}
\log(N+1)\rme^{-\sqrt{\pi d N/\alpha}}\left\|\uSEn\right\|_{C([a,b])}.
\]
Noting $\alpha\in (0, 1]$, we obtain
$\rme^{-\sqrt{\pi d N/\alpha}}\leq \rme^{-\sqrt{\pi d \alpha N}}$.
Furthermore, $\log(N+1)\leq \sqrt{N}$ holds.
Hence, the proof is completed if
$\left\|\uSEn\right\|_{C([a,b])}$ is uniformly bounded
with respect to $N$.
This is shown by the following estimate
\[
 \left\|\uSEn\right\|_{C([a,b])}
\leq \left\|u - \uSEn\right\|_{C([a,b])}
+\left\| u\right\|_{C([a,b])}.
\]
From~\cref{eq:SE-Sinc-Nystroem-error}
and~\cref{thm:SE-Sinc-indefinite},
we see that $\left\|u - \uSEn\right\|_{C([a,b])}$
converges to $0$ as $N\to\infty$,
and accordingly it is uniformly bounded.
We also see that $\left\| u\right\|_{C([a,b])}$
is bounded because $u$ is continuous on $[a, b]$
from the assumption (see~\cref{thm:Sinc-collocation-regularity}).
This completes the proof of~\cref{thm:RZ-Sinc-collocation}.
