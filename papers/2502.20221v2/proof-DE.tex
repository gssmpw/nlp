\section{Proofs for the theorem presented in~\cref{sec:de-collocation}}
\label{sec:proof-DE}

In this section, we provide proofs for~\cref{thm:DE-Sinc-collocation}.

\subsection{Existence and uniqueness of the approximated equations}

In addition to the given equation $(\Ident - \Vol)u = g$,
let us consider the following two equations:
\begin{align}
 (\Ident - \VolDEn)    \uDEn &= g, \label{eq:DE-Sinc-Nystroem-symbol} \\
 (\Ident - \ProjDE \VolDEn)v &= \ProjDE g,
 \label{eq:DE-Sinc-collocation-symbol}
% (E_m^{\textRZ} - V_{m}^{\textRZ})\mathbd{c}_m &= \mathbd{g}_m^{\textDE},
%\label{eq:RZ-linear-eq-in-proof}
\end{align}
where $\VolDEn$ is defined by
\[
%\label{eq:VolDEn}
 \VolDEn [f](t)
=\sum_{j=-N}^N k(t,\tDE_j)f(\tDE_j)\DEtDiv(jh)J(j,h)(\DEtInv(t)),
\]
and $\ProjDE$ are defined by~\cref{eq:ProjDE}.
Because~\cref{eq:DE-Sinc-Nystroem-symbol} is equivalent
to~\cref{eq:DE-Sinc-Nystroem},
the solution of~\cref{eq:DE-Sinc-Nystroem-symbol}
is the approximate solution of the DE-Sinc-Nystr\"{o}m method.
On~\cref{eq:DE-Sinc-Nystroem-symbol}, the following result was obtained.

\begin{lemma}[Okayama et al.~{\cite[Lemma~6.10]{okayama13:_theo}}]
\label{lem:DE-Sinc-Nystroem}
Assume that
all the assumptions of~\cref{thm:DE-Sinc-Nystroem}
are fulfilled.
Then, there exists a positive integer $N_0$ such that for all $N\geq N_0$,
\cref{eq:DE-Sinc-Nystroem-symbol}
has a unique solution $\uDEn\in C([a,b])$.
Furthermore, there exists a constant $C$ independent of $N$ such that
for all $N\geq N_0$,
\begin{equation}
\label{eq:DE-Sinc-Nystroem-error}
 \|u - \uDEn\|_{C([a,b])}\leq C \|\Vol u - \VolDEn u\|_{C([a,b])}.
\end{equation}
\end{lemma}

On~\cref{eq:DE-Sinc-collocation-symbol}, we can show the following lemma
in the same manner as~\cref{lem:Stenger-solution}
(hence, the proof is omitted).

\begin{lemma}
The following two statements are equivalent:
\begin{enumerate}
 \item[{\rm (A)}] \Cref{eq:DE-Sinc-Nystroem-symbol}
has a unique solution $\uDEn\in C([a,b])$.
 \item[{\rm (B)}] \Cref{eq:DE-Sinc-collocation-symbol} has
 a unique solution $v\in C([a,b])$.
\end{enumerate}
Furthermore, $v=\vDEn$ holds.
\end{lemma}

On the basis of the results,
we proceed to analyze the error of $\vDEn$ next.

\subsection{Proof of~\cref{thm:DE-Sinc-collocation}}

In the same manner as~\cref{eq:SE-Sinc-first-error},
we have
\begin{equation}
\label{eq:DE-Sinc-first-error}
 \|u - \vDEn\|_{C([a,b])}
\leq \|u - \ProjDE u\|_{C([a,b])}
 + \|\ProjDE\|_{\mathcal{L}(C([a,b]),C([a,b]))}
\|u - \uDEn\|_{C([a,b])}.
\end{equation}
For the first term,
from~\cref{thm:Sinc-collocation-regularity},
we can use~\cref{thm:DE-Sinc-general} as
\[
 \|u - \ProjDE u\|_{C([a,b])}
\leq C_1 \rme^{-\pi d N/\log(2 d N/\alpha)}
\]
for some constant $C_1$.
For the second term, we use the following bound for
the operator $\ProjDE$.

\begin{lemma}[Okayama~{\cite[Lemma~7.5]{okayama23:_theo}}]
Let $\ProjDE$ be defined by~\eqref{eq:ProjDE}.
Then, there exists a constant $C_2$ independent of $N$ such that
\[
 \|\ProjDE\|_{\mathcal{L}(C([a,b]),C([a,b]))}
\leq C_2 \log(N+1).
\]
\end{lemma}

The remaining term to be estimated in~\cref{eq:DE-Sinc-first-error}
is $\|u - \uDEn\|_{C([a,b])}$.
According to~\cref{lem:DE-Sinc-Nystroem},
it is estimated as~\cref{eq:DE-Sinc-Nystroem-error}.
Because $u\in\Hinf(\DEt(\domD_d))$, $u$ satisfies the assumptions
of~\cref{thm:DE-Sinc-indefinite},
from which we have
\[
 \|\Vol u - \VolDEn u\|_{C([a,b])}
\leq C_3 \frac{\log(2 d N/\alpha)}{N}\rme^{-\pi d N/\log(2 d N/\alpha)}.
\]
Thus, there exists a constant $C_4$ such that
\begin{align*}
& \|u - \vDEn\|_{C([a,b])}\\
&\leq C_1 \rme^{-\pi d N/\log(2 d N/\alpha)}
+ C_2 \log(N+1) C_3 \frac{\log(2 d N/\alpha)}{N}
\rme^{-\pi d N/\log(2 d N/\alpha)}\\
&\leq C_4 \rme^{-\pi d N/\log(2 d N/\alpha)}.
\end{align*}
This completes the proof of~\cref{thm:DE-Sinc-collocation}.
