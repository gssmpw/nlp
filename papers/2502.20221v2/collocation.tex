\section{Existing Sinc-collocation methods}
\label{sec:collocation}

This section describes two different Sinc-collocation methods
developed by
Stenger~\cite{stenger93:_numer}
and Rashidinia and Zarebnia~\cite{rashidinia07:_solut}.
Both methods employ the tanh transformation~\cref{eq:SEt}
as a variable transformation,
but their procedures are not identical.

\subsection{Sinc-collocation method by Stenger}

As explained in~\cref{sec:introduction},
Stenger derived his method for~\cref{eq:Volterra-initial-val},
where the kernel $\tilde{k}$ is a function of a single variable.
However, his method can be easily derived for~\cref{eq:Volterra-int}
as follows.
His method is closely related to the SE-Sinc-Nystr\"{o}m method,
which is described in the previous section.
First, solve the linear system~\cref{eq:SE-Sinc-Nystroem-linear-eq}
and obtain~$\mathbd{u}_n^{\textSE}$.
Then, his approximated solution $\vSEn$ is expressed as
the generalized SE-Sinc approximation of $\uSEn$, i.e.,
\begin{align}
\label{eq:SE-Sinc-collocation}
 \vSEn(t)
& = \ProjSE[\uSEn](t)\\
& = \sum_{j=-N}^N
\left\{\uSEn(\tSE_j) - \uSEn(\tSE_{-N})\omega_a(\tSE_j)
 - \uSEn(\tSE_N)\omega_b(\tSE_j)\right\}S(j,h)(\SEtInv(t))\nonumber\\
&\quad + \uSEn(\tSE_{-N})\omega_a(t) + \uSEn(\tSE_{N})\omega_b(t),\nonumber
\end{align}
where $\ProjSE$ is defined by~\cref{eq:ProjSE}.

The solution $\vSEn$ is also obtained
by the standard collocation procedure as follows.
Set the approximate solution $\vSEn$ as~\cref{eq:SE-Sinc-collocation},
where $\uSEn(\tSE_j)$ $(j=-N,\,\ldots,\,N)$ are
regarded as unknown coefficients.
%\begin{align*}
% \vSEn(t) =
% \sum_{j=-N}^N\left\{
% u_j - u_{-N}\omega_a(\tSE_j) - u_N\omega_b(\tSE_j)
% \right\}S(j,h)(\SEtInv(t))
% +u_{-N}\omega_a(t) + u_N\omega_b(t).
%\end{align*}
Substitute $\vSEn$ into the given equation~\cref{eq:Volterra-int},
with approximating the Volterra integral operator $\Vol$ by $\VolSEn$, where
\begin{equation}
\label{eq:VolSEn}
 \VolSEn [f](t)
=\sum_{j=-N}^N k(t,\tSE_j)f(\tSE_j)\SEtDiv(jh)J(j,h)(\SEtInv(t)),
\end{equation}
which is the SE-Sinc indefinite integration of $\Vol f$.
Then, setting $n=2N+1$ sampling points at $t=\tSE_i$
$(i=-N,\,-N+1,\,\ldots,\,N)$, we obtain
the same system of linear equations as~\cref{eq:SE-Sinc-Nystroem-linear-eq}.
%\[
% \sum_{j=-N}^N
%\left\{\delta_{ij} - k(\tSE_i,\tSE_j)\SEtDiv(jh)h\delta_{i-j}^{(-1)}\right\}
% u_j = g(\tSE_i), \quad i=-N,\,\ldots,\,N,
%\]
%where $\delta_{ij}$ is the Kronecker delta.
%This system of linear equations is
%nothing but~\cref{eq:SE-Sinc-Nystroem-linear-eq}.
Thus, Stenger's method can be regarded as a collocation method
utilizing the generalized SE-Sinc approximation,
namely, SE-Sinc-collocation method.

\subsection{Sinc-collocation method by Rashidinia and Zarebnia}

Rashidinia and Zarebnia derived their method
by the standard collocation procedure,
but they considered their approximated solution $\vRZn$
in different manners in the following four cases.
\begin{enumerate}
 \item[(I)] If $u(a)=u(b)=0$, set $\vRZn$ as
\[
 \vRZn(t) = \sum_{j=-N}^{N} c_{j} S(j,h)(\SEtInv(t)).
\]
 \item[(II)] If $u(a)\neq 0$ and $u(b)=0$, set $\vRZn$ as
\[
 \vRZn(t) = c_{-N}\omega_a(t) + \sum_{j=-N+1}^{N} c_{j} S(j,h)(\SEtInv(t)).
\]
 \item[(III)] If $u(a)= 0$ and $u(b)\neq 0$, set $\vRZn$ as
\[
 \vRZn(t) = \sum_{j=-N}^{N-1} c_{j} S(j,h)(\SEtInv(t)) + c_{N}\omega_b(t).
\]
 \item[(IV)] If $u(a)\neq 0$ and $u(b)\neq 0$, set $\vRZn$ as
\end{enumerate}
\begin{equation}
\label{eq:vRZn}
 \vRZn(t) = c_{-N}\omega_a(t)
 + \sum_{j=-N+1}^{N-1} c_{j} S(j,h)(\SEtInv(t)) + c_{N}\omega_b(t).
\end{equation}
To obtain the unknown coefficients
$\mathbd{c}_{n} = [c_{-N},\,c_{-N+1},\,\ldots,\,c_{N}]^{\mathrm{T}}$,
where $n=2N+1$,
they substituted $\vRZn$ into the given equation~\cref{eq:Volterra-int},
with approximating the Volterra integral operator $\Vol$ by $\VolSEn$.
Then, setting $n$ sampling points at $t=\tSE_i$
$(i=-N,\,-N+1,\,\ldots,\,N)$,
they derived a system of linear equations
in each of the four cases: (I)--(IV).
For example, in the case (I), the resulting system is expressed as
\[
 (I_n - V_{n}^{\textSE})\mathbd{c}_n = \mathbd{g}_n^{\textSE}.
\]
Particularly, in the case (IV), the resulting system is expressed as
\begin{equation}
\label{eq:RZ-linear-eq}
 (E_n^{\textRZ} - V_{n}^{\textRZ})\mathbd{c}_n = \mathbd{g}_n^{\textSE},
\end{equation}
where $E_n^{\textRZ}$ and $V_n^{\textRZ}$ are
$n\times n$ matrices defined by
\begin{align*}
 E_n^{\textRZ}
&= \left[
   \begin{array}{@{\,}c|ccc|c@{\,}}
   \omega_a(\tSE_{-N})        & 0    & \cdots &0  & \omega_b(\tSE_{-N}) \\
%   \hline
   \omega_a(\tSE_{-N+1}) & 1      &        &\Order & \omega_b(\tSE_{-N+1})\\
   \vdots     &        & \ddots &       & \vdots \\
   \omega_a(\tSE_{N-1}) & \Order &        &1      & \omega_b(\tSE_{N-1}) \\
%   \hline
   \omega_a(\tSE_{N}) & 0      & \cdots &0      & \omega_b(\tSE_{N})
   \end{array}
   \right], \\
  V_m^{\textRZ}
&= \left[
   \begin{array}{@{\,}l|clc|l@{\,}}
%   \VolSEn[\omega_a](\tSE_{-N-1})
   &\cdots
   & k(\tSE_{-N},\tSE_j)\SEtDiv(jh) h \delta_{-N-j}^{(-1)}
   &\cdots
   & \\ %\VolSEn[\omega_b](\tSE_{-N-1}) \\
%  \hline
%   \VolSEn[\omega_a](\tSE_{-N})
   &\cdots
   & k(\tSE_{-N+1},\tSE_j)\SEtDiv(jh) h \delta_{-N+1-j}^{(-1)}
   &\cdots
   & \\ %\VolSEn[\omega_b](\tSE_{-N}) \\
    \multicolumn{1}{c|}{\mathbd{p}_n^{\textRZ}} & & \multicolumn{1}{c}{\vdots}
   & & \multicolumn{1}{c}{\mathbd{q}_n^{\textRZ}}\\
%   \VolSEn[\omega_a](\tSE_{N})
   &\cdots
   & k(\tSE_{N-1},\tSE_j)\SEtDiv(jh) h \delta_{N-1-j}^{(-1)}
   &\cdots
   & \\ %\VolSEn[\omega_b](\tSE_N) \\
%  \hline
%   \VolSEn[\omega_a](\tSE_{N+1})
   &\cdots
   & k(\tSE_{N},\tSE_j)\SEtDiv(jh) h \delta_{N-j}^{(-1)}
   &\cdots
   & %\VolSEn[\omega_b](\tSE_{N+1})
   \end{array}
   \right],
\end{align*}
where $\mathbd{p}_n^{\textRZ}$
and $\mathbd{q}_n^{\textRZ}$
are $n$-dimensional vectors defined by
\begin{align*}
\mathbd{p}_n^{\textRZ}
&= [ \VolSEn[\omega_a](\tSE_{-N}),\,
\VolSEn[\omega_a](\tSE_{-N+1}),\,\ldots,\,
\VolSEn[\omega_a](\tSE_{N})]^{\mathrm{T}}, \\
\mathbd{q}_n^{\textRZ}
&= [ \VolSEn[\omega_b](\tSE_{-N}),\,
\VolSEn[\omega_b](\tSE_{-N+1}),\,\ldots,\,
\VolSEn[\omega_b](\tSE_{N})]^{\mathrm{T}}.
\end{align*}
This is the SE-Sinc-collocation method by Rashidinia and Zarebnia.
In the case (I), the following error analysis was provided.

\begin{theorem}[Zarebnia and Rashidinia~{\cite[Theorem~3]{zarebnia10:_conv}}]
\label{thm:Rarebnia-Rashidinia}
Let $\alpha$ and $d$ be positive constants with $d<\pi$.
Assume that the solution $u$ in~\cref{eq:Volterra-int}
satisfies all the assumptions in~\cref{thm:SE-Sinc-approx}.
Furthermore, assume that $k(t,\cdot)$
satisfies all the assumptions in~\cref{thm:SE-Sinc-indefinite}
for all $t\in [a, b]$.
Then, there exists a constant $C$ independent of $N$ such that
\[
 \|u - \vRZn\|_{C([a,b])}
\leq C \|(I_n- V_{n}^{\textSE})^{-1}\|_2\sqrt{N}\rme^{-\sqrt{\pi d \alpha N}}.
\]
\end{theorem}

However,
this theorem does not prove the convergence of $\vRZn$,
because there exists an unestimated
term $\|(I_n - V_{n}^{\textSE})^{-1}\|_2$, which clearly depends on $N$.
For the cases (II)--(IV), no error analysis has been provided thus far.

Moreover,
in a practical situation,
it is hard to determine whether $u$ is zero or not at the endpoints.
This is because the solution $u$ is an unknown function to be determined.
The idea to address the issue was presented
for Fredholm integral equations~\cite{okayama1x:_improv};
set the approximate solution $\vRZn$ as~\cref{eq:vRZn} in any cases.
In other words, we may treat the case (IV) as a general case.
This idea can be employed
for Volterra integral equations~\cref{eq:Volterra-int}.
Therefore, as a method by Rashidinia and Zarebnia,
this study adopts the following procedure:
(i) solve the linear system~\cref{eq:RZ-linear-eq},
and (ii) obtain the approximate solution by~\cref{eq:vRZn}.

\subsection{Main result 1: Relationship between the two methods and their convergence}

Any relationship between Stenger's method $(\vSEn)$
and Rashidinia--Zarebnia's method $(\vRZn)$ has not been investigated thus far.
Furthermore, convergence of the two methods has not been rigorously proved.
As a first contribution of this paper,
we show the relationship between the two methods as follows.
The proof is provided in~\cref{sec:proof-equivalence}.

\begin{theorem}
\label{thm:equivalence}
%Let $\alpha$ and $d$ be positive constants with
%$\alpha\leq 1$ and $d<\pi$.
%Assume that
%all the assumptions on $g$ and $k$ in~\cref{thm:SE-Sinc-Nystroem}
%are fulfilled.
%Let $h$ be selected by the formula~\cref{eq:h-SE}.
%Then, there exists a positive integer $N_0$
%such that for all $N\geq N_0$,
%$\vSEn = \vRZn$ holds.
Let $\vSEn$ be a function defined by~\cref{eq:SE-Sinc-collocation},
where $\mathbd{u}_n^{\textSE}$ is determined
by solving the linear system~\cref{eq:SE-Sinc-Nystroem-linear-eq}.
Furthermore,
let $\vRZn$ be a function defined by~\cref{eq:vRZn},
where $\mathbd{c}_{m}$ is determined
by solving the linear system~\cref{eq:RZ-linear-eq}.
Then, it holds that
\[
 \vSEn(\tSE_i) = \vRZn(\tSE_i),\quad i=-N,\,-N+1,\,\ldots,\,N,
\]
but generally $\vSEn\neq \vRZn$.
\end{theorem}

%Therefore, we may refer to both methods as the SE-Sinc-collocation method.
Subsequently, we provide the convergence theorems of
the two methods as follows.
Their proofs are provided in~\cref{sec:proof-SE-Sinc,sec:proof-RZ-Sinc}.

\begin{theorem}
\label{thm:SE-Sinc-collocation}
Let $\alpha$ and $d$ be positive constants with
$\alpha\leq 1$ and $d<\pi$.
Assume that
all the assumptions on $g$ and $k$ in~\cref{thm:SE-Sinc-Nystroem}
are fulfilled.
Furthermore, assume that $g$ and $k(\cdot,w)$
belong to $\MC_{\alpha}(\SEt(\domD_d))$ for all $w\in\SEt(\domD_d)$.
Let $h$ be selected by the formula~\cref{eq:h-SE}.
Then, there exists a positive integer $N_0$
such that for all $N\geq N_0$,
the coefficient matrix $(I_n - \kSEn)$ is invertible.
Furthermore, there exists a constant $C$ independent of $N$
such that for all $N\geq N_0$,
\[
 \|u - \vSEn\|_{C([a,b])} \leq C \sqrt{N} \rme^{-\sqrt{\pi d \alpha N}}.
\]
\end{theorem}

\begin{theorem}
\label{thm:RZ-Sinc-collocation}
Assume that all the assumptions of~\cref{thm:SE-Sinc-collocation}
are fulfilled.
Then, there exists a positive integer $N_0$
such that for all $N\geq N_0$,
the coefficient matrix $(E_n^{\textRZ} - V_{n}^{\textRZ})$ is invertible.
Furthermore, there exists a constant $C$ independent of $N$
such that for all $N\geq N_0$,
\[
%\label{eq:vRZn-estimate}
 \|u - \vRZn\|_{C([a,b])} \leq C \sqrt{N} \rme^{-\sqrt{\pi d \alpha N}}.
\]
\end{theorem}

\begin{remark}
In view of~\cref{thm:Rarebnia-Rashidinia,thm:RZ-Sinc-collocation},
one might assume that~\cref{thm:SE-Sinc-collocation}
is proved by bounding $\|(I_n - \kSEn)^{-1}\|_2$ uniformly for $N$.
However, this is not the case; see~\cref{sec:proof-SE} for details.
\end{remark}

\Cref{thm:SE-Sinc-collocation,thm:RZ-Sinc-collocation} reveal that
both methods achieve the same convergence rate.
Therefore, users may prefer Stenger's method
because the implementation of the method by Rashidinia and Zarebnia
is rather complicated.
This complication also causes difficulty in extension to the \emph{system} of
Volterra integral equations.
For this reason, in the next section,
we consider the improvement of Stenger's method.
