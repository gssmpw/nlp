% SIAM Article Template
\documentclass[onefignum,onetabnum]{siamart220329}

% Information that is shared between the article and the supplement
% (title and author information, macros, packages, etc.) goes into
% ex_shared.tex. If there is no supplement, this file can be included
% directly.

% SIAM Shared Information Template
% This is information that is shared between the main document and any
% supplement. If no supplement is required, then this information can
% be included directly in the main document.


% Packages and macros go here
\usepackage{lipsum}
\usepackage{amsfonts}
\usepackage{graphicx}
\usepackage{epstopdf}
%\usepackage{algorithmic}

%%%%% custom settings %%%%%%
\usepackage{tikz, multirow, makecell, booktabs}
\usetikzlibrary{shapes.geometric, arrows}

\usepackage{bm,enumitem,comment}
\usepackage[linesnumbered,ruled,vlined]{algorithm2e}
\renewcommand{\thealgocf}{\thesection.\arabic{algocf}}
\SetAlgoCaptionSeparator{ }

\usepackage[left=1.72in, right=1.65in, top=1.37in, bottom=1.37in]{geometry}
% \usepackage[draft,notref,notcite]{showkeys}
% \makeatletter
% \renewcommand*\showkeyslabelformat[1]{%
%   \llap{\fbox{\normalfont\fontsize{7}{10}\selectfont\ttfamily#1}\hspace{1.5em}}}
% \makeatother

\newcommand{\vertiii}[1]{{\left\vert\kern-0.25ex\left\vert\kern-0.25ex\left\vert #1 \right\vert\kern-0.25ex\right\vert\kern-0.25ex\right\vert}}
%%%%%%%%%%%%%%%%%%%%%%

\ifpdf
  \DeclareGraphicsExtensions{.eps,.pdf,.png,.jpg}
\else
  \DeclareGraphicsExtensions{.eps}
\fi

% Add a serial/Oxford comma by default.
\newcommand{\creflastconjunction}{, and~}

% Used for creating new theorem and remark environments
\newsiamremark{remark}{Remark}
\newsiamremark{hypothesis}{Hypothesis}
\crefname{hypothesis}{Hypothesis}{Hypotheses}
\newsiamthm{claim}{Claim}

% Sets running headers as well as PDF title and authors
\headers{Deep collocation method with error control}{M. Weng, Z. Mao, and J. Shen}

% Title. If the supplement option is on, then "Supplementary Material"
% is automatically inserted before the title.
\title{Deep collocation method: A framework for solving PDEs using neural networks with error control %\thanks{Submitted to the editors DATE.}
}

% Authors: full names plus addresses.
\author{Mingxing Weng\thanks{School of Mathematical Sciences, Shanghai Jiao Tong University,
	Shanghai 200240, China; School of Mathematical Science, Eastern Institute of Technology, Ningbo, 
	Zhejiang 315200, China
	(mxweng22@sjtu.edu.cn)}
\and Zhiping Mao\thanks{School of Mathematical Science, Eastern Institute of Technology, Ningbo, 
		Zhejiang 315200, China
  		(zmao@eitech.edu.cn, jshen@eitech.edu.cn).}
\and Jie Shen\footnotemark[2]
}

\usepackage{amsopn}
\DeclareMathOperator{\diag}{diag}


%%% Local Variables: 
%%% mode:latex
%%% TeX-master: "ex_article"
%%% End: 


% Optional PDF information
\ifpdf
\hypersetup{
  pdftitle={On the relationship between two Sinc-collocation methods for
 Volterra integral equations of the second kind and their further improvement},
  pdfauthor={T. Okayama}
}
\fi

% The next statement enables references to information in the
% supplement. See the xr-hyperref package for details.

\externaldocument{ex_supplement}

% FundRef data to be entered by SIAM
%<funding-group specific-use="FundRef">
%<award-group>
%<funding-source>
%<named-content content-type="funder-name"> 
%</named-content> 
%<named-content content-type="funder-identifier"> 
%</named-content>
%</funding-source>
%<award-id> </award-id>
%</award-group>
%</funding-group>

\begin{document}

\maketitle

% REQUIRED
% limit: 250 words
\begin{abstract}
Two different Sinc-collocation methods
for Volterra integral equations of the second kind
have been independently proposed
by Stenger and Rashidinia--Zarebnia.
However, their relationship remains unexplored.
This study theoretically examines the solutions of these two methods,
and reveals that they are not generally equivalent,
despite coinciding at the collocation points.
Strictly speaking, Stenger's method
assumes that the kernel of the integral
is a function of a single variable,
but this study theoretically justifies the use of
his method in general cases, i.e., the kernel is
a function of two variables.
%that are derived from initial value problems,
%where the kernel of the integral
%is assumed to be a function of single variable.
%The assumption on the kernel seems essential
%because it is fully used in his theoretical discussion of the method.
%Contrary to the idea, this paper theoretically justifies the use of
%his method in general cases, i.e., the kernel is
%a function of two variables.
Then, this study rigorously proves that
both methods can attain the same, root-exponential convergence.
%which corresponds with the numerical observations in the previous study.
In addition to the contribution,
this study improves Stenger's method to attain significantly higher,
almost exponential convergence.
Numerical examples supporting the theoretical results are also provided.
%In this paper, two improved versions of an existing Sinc-collocation method
%for Volterra integral equations of the second kind are presented.
%The first method is close to the reformulation of the existing method
%in order to make it more practical and to reinforce it in theoretical way.
%Then it is rigorously proved that
%the convergence rate of the modified method is $\Order(\exp(-c\sqrt{N}))$,
%which corresponds with the numerical observations in the previous study.
%In the second method,
%the variable transformation employed in the first method,
%which is called the ``tanh transformation,''
%is replaced with the ``double exponential transformation.''
%%instead of the original ``tanh transformation.''
%This replacement improves the convergence rate to
%$\Order(\exp(-c N/\log N))$, which is proved
%in the same way as the first method.
%Numerical examples which support the theoretical results are also given.
\end{abstract}

% REQUIRED
\begin{keywords}
Sinc numerical method, tanh transformation, double-exponential transformation
\end{keywords}

% REQUIRED
\begin{AMS}
  65R20
\end{AMS}

\section{Introduction}
\label{sec:introduction}
The business processes of organizations are experiencing ever-increasing complexity due to the large amount of data, high number of users, and high-tech devices involved \cite{martin2021pmopportunitieschallenges, beerepoot2023biggestbpmproblems}. This complexity may cause business processes to deviate from normal control flow due to unforeseen and disruptive anomalies \cite{adams2023proceddsriftdetection}. These control-flow anomalies manifest as unknown, skipped, and wrongly-ordered activities in the traces of event logs monitored from the execution of business processes \cite{ko2023adsystematicreview}. For the sake of clarity, let us consider an illustrative example of such anomalies. Figure \ref{FP_ANOMALIES} shows a so-called event log footprint, which captures the control flow relations of four activities of a hypothetical event log. In particular, this footprint captures the control-flow relations between activities \texttt{a}, \texttt{b}, \texttt{c} and \texttt{d}. These are the causal ($\rightarrow$) relation, concurrent ($\parallel$) relation, and other ($\#$) relations such as exclusivity or non-local dependency \cite{aalst2022pmhandbook}. In addition, on the right are six traces, of which five exhibit skipped, wrongly-ordered and unknown control-flow anomalies. For example, $\langle$\texttt{a b d}$\rangle$ has a skipped activity, which is \texttt{c}. Because of this skipped activity, the control-flow relation \texttt{b}$\,\#\,$\texttt{d} is violated, since \texttt{d} directly follows \texttt{b} in the anomalous trace.
\begin{figure}[!t]
\centering
\includegraphics[width=0.9\columnwidth]{images/FP_ANOMALIES.png}
\caption{An example event log footprint with six traces, of which five exhibit control-flow anomalies.}
\label{FP_ANOMALIES}
\end{figure}

\subsection{Control-flow anomaly detection}
Control-flow anomaly detection techniques aim to characterize the normal control flow from event logs and verify whether these deviations occur in new event logs \cite{ko2023adsystematicreview}. To develop control-flow anomaly detection techniques, \revision{process mining} has seen widespread adoption owing to process discovery and \revision{conformance checking}. On the one hand, process discovery is a set of algorithms that encode control-flow relations as a set of model elements and constraints according to a given modeling formalism \cite{aalst2022pmhandbook}; hereafter, we refer to the Petri net, a widespread modeling formalism. On the other hand, \revision{conformance checking} is an explainable set of algorithms that allows linking any deviations with the reference Petri net and providing the fitness measure, namely a measure of how much the Petri net fits the new event log \cite{aalst2022pmhandbook}. Many control-flow anomaly detection techniques based on \revision{conformance checking} (hereafter, \revision{conformance checking}-based techniques) use the fitness measure to determine whether an event log is anomalous \cite{bezerra2009pmad, bezerra2013adlogspais, myers2018icsadpm, pecchia2020applicationfailuresanalysispm}. 

The scientific literature also includes many \revision{conformance checking}-independent techniques for control-flow anomaly detection that combine specific types of trace encodings with machine/deep learning \cite{ko2023adsystematicreview, tavares2023pmtraceencoding}. Whereas these techniques are very effective, their explainability is challenging due to both the type of trace encoding employed and the machine/deep learning model used \cite{rawal2022trustworthyaiadvances,li2023explainablead}. Hence, in the following, we focus on the shortcomings of \revision{conformance checking}-based techniques to investigate whether it is possible to support the development of competitive control-flow anomaly detection techniques while maintaining the explainable nature of \revision{conformance checking}.
\begin{figure}[!t]
\centering
\includegraphics[width=\columnwidth]{images/HIGH_LEVEL_VIEW.png}
\caption{A high-level view of the proposed framework for combining \revision{process mining}-based feature extraction with dimensionality reduction for control-flow anomaly detection.}
\label{HIGH_LEVEL_VIEW}
\end{figure}

\subsection{Shortcomings of \revision{conformance checking}-based techniques}
Unfortunately, the detection effectiveness of \revision{conformance checking}-based techniques is affected by noisy data and low-quality Petri nets, which may be due to human errors in the modeling process or representational bias of process discovery algorithms \cite{bezerra2013adlogspais, pecchia2020applicationfailuresanalysispm, aalst2016pm}. Specifically, on the one hand, noisy data may introduce infrequent and deceptive control-flow relations that may result in inconsistent fitness measures, whereas, on the other hand, checking event logs against a low-quality Petri net could lead to an unreliable distribution of fitness measures. Nonetheless, such Petri nets can still be used as references to obtain insightful information for \revision{process mining}-based feature extraction, supporting the development of competitive and explainable \revision{conformance checking}-based techniques for control-flow anomaly detection despite the problems above. For example, a few works outline that token-based \revision{conformance checking} can be used for \revision{process mining}-based feature extraction to build tabular data and develop effective \revision{conformance checking}-based techniques for control-flow anomaly detection \cite{singh2022lapmsh, debenedictis2023dtadiiot}. However, to the best of our knowledge, the scientific literature lacks a structured proposal for \revision{process mining}-based feature extraction using the state-of-the-art \revision{conformance checking} variant, namely alignment-based \revision{conformance checking}.

\subsection{Contributions}
We propose a novel \revision{process mining}-based feature extraction approach with alignment-based \revision{conformance checking}. This variant aligns the deviating control flow with a reference Petri net; the resulting alignment can be inspected to extract additional statistics such as the number of times a given activity caused mismatches \cite{aalst2022pmhandbook}. We integrate this approach into a flexible and explainable framework for developing techniques for control-flow anomaly detection. The framework combines \revision{process mining}-based feature extraction and dimensionality reduction to handle high-dimensional feature sets, achieve detection effectiveness, and support explainability. Notably, in addition to our proposed \revision{process mining}-based feature extraction approach, the framework allows employing other approaches, enabling a fair comparison of multiple \revision{conformance checking}-based and \revision{conformance checking}-independent techniques for control-flow anomaly detection. Figure \ref{HIGH_LEVEL_VIEW} shows a high-level view of the framework. Business processes are monitored, and event logs obtained from the database of information systems. Subsequently, \revision{process mining}-based feature extraction is applied to these event logs and tabular data input to dimensionality reduction to identify control-flow anomalies. We apply several \revision{conformance checking}-based and \revision{conformance checking}-independent framework techniques to publicly available datasets, simulated data of a case study from railways, and real-world data of a case study from healthcare. We show that the framework techniques implementing our approach outperform the baseline \revision{conformance checking}-based techniques while maintaining the explainable nature of \revision{conformance checking}.

In summary, the contributions of this paper are as follows.
\begin{itemize}
    \item{
        A novel \revision{process mining}-based feature extraction approach to support the development of competitive and explainable \revision{conformance checking}-based techniques for control-flow anomaly detection.
    }
    \item{
        A flexible and explainable framework for developing techniques for control-flow anomaly detection using \revision{process mining}-based feature extraction and dimensionality reduction.
    }
    \item{
        Application to synthetic and real-world datasets of several \revision{conformance checking}-based and \revision{conformance checking}-independent framework techniques, evaluating their detection effectiveness and explainability.
    }
\end{itemize}

The rest of the paper is organized as follows.
\begin{itemize}
    \item Section \ref{sec:related_work} reviews the existing techniques for control-flow anomaly detection, categorizing them into \revision{conformance checking}-based and \revision{conformance checking}-independent techniques.
    \item Section \ref{sec:abccfe} provides the preliminaries of \revision{process mining} to establish the notation used throughout the paper, and delves into the details of the proposed \revision{process mining}-based feature extraction approach with alignment-based \revision{conformance checking}.
    \item Section \ref{sec:framework} describes the framework for developing \revision{conformance checking}-based and \revision{conformance checking}-independent techniques for control-flow anomaly detection that combine \revision{process mining}-based feature extraction and dimensionality reduction.
    \item Section \ref{sec:evaluation} presents the experiments conducted with multiple framework and baseline techniques using data from publicly available datasets and case studies.
    \item Section \ref{sec:conclusions} draws the conclusions and presents future work.
\end{itemize}

% Consider a lasso optimization procedure with potentially distinct regularization penalties:
% \begin{align}
%     \hat{\beta} = \arg\min_{\beta}\{\|y-X\beta\|^2_2+\sum_{i=1}^{N}\lambda_i|\beta_i|\}.
% \end{align}
\subsection{Supervised Data-Driven Learning}\label{subsec:supervised}
We consider a generic data-driven supervised learning procedure. Given a dataset \( \mathcal{D} \) consisting of \( n \) data points \( (x_i, y_i) \in \mathcal{X} \times \mathcal{Y} \) drawn from an underlying distribution \( p(\cdot|\theta) \), our goal is to estimate parameters \( \theta \in \Theta \) through a learning procedure, defined as \( f: (\mathcal{X} \times \mathcal{Y})^n \rightarrow \Theta \) 
that minimizes the predictive error on observed data. 
Specifically, the learning objective is defined as follows:
\begin{align}
\hat{\theta}_f := f(\mathcal{D}) = \arg\min_{\theta} \mathcal{L}(\theta, \mathcal{D}),
\end{align}
where \( \mathcal{L}(\cdot,\mathcal{D}) := \sum_{i=1}^{n} \mathcal{L}(\cdot, (x_i, y_i))\), and $\mathcal{L}$ is a loss function quantifying the error between predictions and true outcomes. 
Here, $\hat{\theta}_f$ is the parameter that best explains the observed data pairs \( (x_i, y_i) \) according to the chosen loss function \( \mathcal{L} (\cdot) \).

\paragraph{Feature Selection.}
Feature selection aims to improve model \( f \)'s predictive performance while minimizing redundancy. 
%Formally, given data \( X \), response \( y \), feature set \( \mathcal{F} \), loss function \( \mathcal{L}(\cdot) \), and a feature limit \( k \), the objective is:
% \begin{align}
% \mathcal{S}^* = \arg \min_{\mathcal{S} \subseteq \mathcal{F}, |\mathcal{S}| \leq k} \mathcal{L}(y, f(X_\mathcal{S})) + \lambda R(\mathcal{S}),
% \end{align}
% where \( X_\mathcal{S} \) is the submatrix of \( X \) for selected features \( \mathcal{S} \), \( \lambda \) is a regularization parameter, and \( R(\mathcal{S}) \) penalizes feature redundancy.
 State-of-the-art techniques fall into four categories: (i) filter methods, which rank features based on statistical properties like Fisher score \citep{duda2001pattern,song2012feature}; (ii) wrapper methods, which evaluate model performance on different feature subsets \citep{kohavi1997wrappers}; (iii) embedded methods, which integrate feature selection into the learning process using techniques like regularization \citep{tibshirani1996LASSO,lemhadri2021lassonet}; and (iv) hybrid methods, which combine elements of (i)-(iii) \citep{SINGH2021104396,li2022micq}. This paper focuses on embedded methods via Lasso, benchmarking against approaches from (i)-(iii).

\subsection{Language Modeling}
% The objective of language modeling is to learn a probability distribution \( p_{LM}(x) \) over sequences of text \( x = (X_1, \ldots, X_{|x|}) \), such that \( p_{LM}(x) \approx p_{text}(x) \), where \( p_{text}(x) \) represents the true distribution of natural language. This process involves estimating the likelihood of token sequences across variable lengths and diverse linguistic structures.
% Modern large language models (LLMs) are trained on vast datasets spanning encyclopedias, news, social media, books, and scientific papers \cite{gao2020pile}. This broad training enables them to generalize across domains, learn contextual knowledge, and perform zero-shot learning—tackling new tasks using only task descriptions without fine-tuning \cite{brown2020gpt3}.
Language modeling aims to approximate the true distribution of natural language \( p_{\text{text}}(x) \) by learning \( p_{\text{LM}}(x) \), a probability distribution over text sequences \( x = (X_1, \ldots, X_{|x|}) \). Modern large language models, trained on diverse datasets \citep{gao2020pile}, exhibit strong generalization across domains, acquire contextual knowledge, and perform zero-shot learning—solving new tasks using only task descriptions—or few-shot learning by leveraging a small number of demonstrations \citep{brown2020gpt3}.
\paragraph{Retrieval-Augmented Generation (RAG).} Retrieval-Augmented Generation (RAG) enhances the performance of generative language models by  integrating a domain-specific information retrieval process  \citep{lewis2020retrieval}. The RAG framework comprises two main components: \textit{retrieval}, which extracts relevant information from external knowledge sources, and \textit{generation}, where an LLM generates context-aware responses using the prompt combined with the retrieved context. Documents are indexed through various databases, such as relational, graph, or vector databases \citep{khattab2020colbert, douze2024faiss, peng2024graphretrievalaugmentedgenerationsurvey}, enabling efficient organization and retrieval via algorithms like semantic similarity search to match the prompt with relevant documents in the knowledge base. RAG has gained much traction recently due to its demonstrated ability to reduce incidence of hallucinations and boost LLMs' reliability as well as performance \citep{huang2023hallucination, zhang2023merging}. 
 
% image source: https://medium.com/@bindurani_22/retrieval-augmented-generation-815c1ae438d8
\begin{figure}
    \centering
\includegraphics[width=1.03\linewidth]{fig/fig1.pdf}
\vspace{-0.6cm}
\scriptsize 
    \caption{Retrieval Augmented Generation (RAG) based $\ell_1$-norm weights (penalty factors) for Lasso. Only feature names---no training data--- are included in LLM prompt.} 
    \label{fig:rag}
\end{figure}
% However, for the RAG model to be effective given the input token constraints of the LLM model used, we need to effectively process the retrieval documents through a procedure known as \textit{chunking}.

\subsection{Task-Specific Data-Driven Learning}
LLM-Lasso aims to bridge the gap between data-driven supervised learning and the predictive capabilities of LLMs trained on rich metadata. This fusion not only enhances traditional data-driven methods by incorporating key task-relevant contextual information often overlooked by such models, but can also be especially valuable in low-data regimes, where the learning algorithm $f:\mathcal{D}\rightarrow\Theta$ (seen as a map from datasets $\mathcal{D}$ to the space of decisions $\Theta$) is susceptible to overfitting.

The task-specific data-driven learning model $\tilde{f}:\mathcal{D}\times\mathcal{D}_\text{meta}\rightarrow\Theta$ can be described as a metadata-augmented version of $f$, where a link function $h(\cdot)$ integrates metadata (i.e. $\mathcal{D}_\text{meta}$) to refine the original learning process. This can be expressed as:
\[
\tilde{f}(\mathcal{D}, \mathcal{D}_\text{meta}) := \mathcal{T}(f(\mathcal{D}),  h(\mathcal{D}_{\text{meta}})),
\]
where the functional $\mathcal{T}$ takes the original learning algorithm $f(\mathcal{D})$ and transforms it into a task-specific learning algorithm $\tilde{f}(\mathcal{D}, \mathcal{D}_\text{meta})$ by incorporating the metadata $\mathcal{D}_\text{meta}$. 
% In particular, the link function $h(\mathcal{D}_{\text{meta}})$ provides a structured mechanism summarizing the contextual knowledge.

There are multiple approaches to formulate $\mathcal{T}$ and $h$.
%to ``inform" the data-driven model $f$ of %meta knowledge. 
For instance, LMPriors \citep{choi2022lmpriorspretrainedlanguagemodels} designed $h$ and $\mathcal{T}$ such that $h(\mathcal{D}_{\text{meta}})$ first specifies which features to retain (based on a probabilistic prior framework), and then $\mathcal{T}$ keeps the selected features and removes all the others from the original learning objective of $f$. 
Note that this approach inherently is restricted as it selects important features solely based on $\mathcal{D}_\text{meta}$ without seeing $\mathcal{D}$.

In contrast, we directly embed task-specific knowledge into the optimization landscape through regularization by introducing a structured inductive bias. This bias guides the learning process toward solutions that are consistent with metadata-informed insights, without relying on explicit probabilistic modeling. Abstractly, this can be expressed as:
\begin{align}
    \!\!\!\!\!\hat{\theta}_{\tilde{f}} := \tilde{f}(\mathcal{D},\mathcal{D}
    _\text{meta})= \arg\min_{\theta} \mathcal{L}(\theta, \mathcal{D}) + \lambda R(\theta, \mathcal{D}_{\text{meta}}),
\end{align}
where \( \lambda \) is a regularization parameter, \( R(\cdot) \) is a regularizer, and $\theta$ is the prediction parameter.
%We explain our framework with more details in the following section.


% Our research diverges from both aforementioned approaches by positioning the LLM not as a standalone feature selector but as an enhancement to data-driven models through an embedded feature selection method, L-LASSO. L-LASSO incorporates domain expertise—auxiliary natural language metadata about the task—via the LLM-informed LASSO penalty, which is then used in statistical models to enhance predictive performance. This method integrates the rich, context-sensitive insights of LLMs with the rigor and transparency of statistical modeling, bridging the gap between data-driven and knowledge-driven feature selection approaches. To approach this task, we need to tackle two key components: (i). train an LLM that is expert in the task-specific knowledge; (ii). inform data-driven feature selector LASSO with LLM knowledge.

% In practice, this involves combining techniques like prompt engineering and data engineering to develop an effective framework for integrating metadata into existing data-driven models. We will go through this in detail in Section \ref{mthd} and \ref{experiment}.



\section{Sinc-Nystr\"{o}m methods}
\label{sec:nystroem}

This section describes the Sinc-Nystr\"{o}m methods
developed by Muhammad et al.~\cite{muhammad05:_numer}.
The first method employs the tanh transformation~\cref{eq:SEt}
as a variable transformation,
while the second method employs the DE transformation~\cref{eq:DEt}.

\subsection{SE-Sinc-Nystr\"{o}m method}

%Assume that $k(t,\cdot)u(\cdot)$ satisfies
%the assumptions of~\cref{thm:SE-Sinc-indefinite}
%uniformly for $t\in [a,b]$.
By applying the SE-Sinc indefinite integration~\cref{eq:SE-Sinc-indefinite}
to the integral in the given equation~\cref{eq:Volterra-int},
we obtain an approximated equation as
\begin{equation}
\label{eq:SE-Sinc-Nystroem}
 \uSEn(t)
= g(t)
+\sum_{j=-N}^N k(t,\tSE_j)\uSEn(\tSE_j)\SEtDiv(jh) J(j,h)(\SEtInv(t)).
\end{equation}
%where $\tSE_j = \SEt(jh)$.
The approximated solution $\uSEn$ is determined
once the unknown coefficients $\uSEn(\tSE_j)$
on the right-hand side are obtained. To this end,
$2N+1$ sampling points are set at
$t=\tSE_i$ $(i=-N,\,-N+1,\,\ldots,\,N)$ in~\cref{eq:SE-Sinc-Nystroem}
as
\begin{equation}
\label{eq:SE-Sinc-Nystroem-system}
 \uSEn(\tSE_i)
= g(t)
+\sum_{j=-N}^N k(\tSE_i,\tSE_j)\uSEn(\tSE_j)\SEtDiv(jh)
 J(j,h)(ih),
\quad i=-N,\,\ldots,\,N,
\end{equation}
which is a system of linear equations.
This system is expressed in a matrix-vector form as follows.
Let $n=2N+1$,
let $I_n$ be an identity matrix of order $n$,
and let $\kSEn$ be $n\times n$ matrix
whose $(i, j)$-th element is
\[
 \left(\kSEn\right)_{ij}
= k(\tSE_i,\tSE_j)\SEtDiv(jh)h \delta_{i-j}^{(-1)},
%\left(\frac{1}{2}+ \sigma_{i-j}\right),
\quad i= -N,\,\ldots,\,N,\quad j=-N,\,\ldots,\,N,
\]
where $\delta_k^{(-1)} = (1/2) + \sigma_k$, where $\sigma_k$ is defined by
\[
\sigma_k = \int_0^k\frac{\sin(\pi x)}{\pi x}\diff x
=\frac{1}{\pi}\Si(\pi k).
\]
Furthermore, let $\gSEn$ and $\mathbd{u}_n^{\textSE}$
be $n$-dimensional vectors defined by
\begin{align*}
 \gSEn
 &= [g(\tSE_{-N}),\,g(\tSE_{-N+1}),\,\ldots,\,g(\tSE_N)]^{\mathrm{T}},\\
 \mathbd{u}_n^{\textSE}
 &= [\uSEn(\tSE_{-N}),\,\uSEn(\tSE_{-N+1}),\,\ldots,\,\uSEn(\tSE_N)]^{\mathrm{T}}.
\end{align*}
Then, the system~\cref{eq:SE-Sinc-Nystroem-system} is expressed as
\begin{equation}
\label{eq:SE-Sinc-Nystroem-linear-eq}
 (I_n - \kSEn) \mathbd{u}_n^{\textSE} = \gSEn.
\end{equation}
By solving~\cref{eq:SE-Sinc-Nystroem-linear-eq},
we obtain the unknown coefficients $\mathbd{u}_n^{\textSE}$,
from which the approximated solution $\uSEn$ is determined
by~\cref{eq:SE-Sinc-Nystroem}.
This procedure is called the SE-Sinc-Nystr\"{o}m method.
Its convergence theorem was provided as follows.

\begin{theorem}[Okayama et al.~{\cite[Theorem~3.4]{okayama13:_theo}}]
\label{thm:SE-Sinc-Nystroem}
Let $d$ be a positive constant with $d<\pi$.
Assume that $g$, $k(z,\cdot)$ and $k(\cdot,w)$
belong to $\Hinf(\SEt(\domD_d))$ for all $z$, $w\in\SEt(\domD_d)$.
Furthermore, assume that
$g$, $k(t,\cdot)$ and $k(\cdot,s)$ belong to $C([a,b])$
for all $t$, $s\in [a, b]$.
Let $h$ be selected by the formula~\cref{eq:h-SE} with $\alpha=1$.
Then, there exists a positive integer $N_0$
such that for all $N\geq N_0$,
the coefficient matrix $(I_n - \kSEn)$ is invertible.
Furthermore, there exists a constant $C$ independent of $N$
such that for all $N\geq N_0$,
\[
 \|u - \uSEn\|_{C([a,b])} \leq C \rme^{-\sqrt{\pi d N}}.
\]
\end{theorem}

\subsection{DE-Sinc-Nystr\"{o}m method}

Muhammad et al.~\cite{muhammad05:_numer} also considered
another method by replacing $\SEt$ with $\DEt$
in the SE-Sinc-Nystr\"{o}m method.
%Assume that $k(t,\cdot)u(\cdot)$ satisfies
%the assumptions of~\cref{thm:DE-Sinc-indefinite}
%uniformly for $t\in [a,b]$.
Applying the DE-Sinc indefinite integration~\cref{eq:DE-Sinc-indefinite}
to the integral in the given equation~\cref{eq:Volterra-int},
we obtain an approximated equation as
\begin{equation}
\label{eq:DE-Sinc-Nystroem}
 \uDEn(t)
= g(t)
+\sum_{j=-N}^N k(t,\tDE_j)\uDEn(\tDE_j)\DEtDiv(jh) J(j,h)(\DEtInv(t)).
\end{equation}
%where $\tDE_j = \DEt(jh)$.
The approximated solution $\uDEn$ is determined
once the unknown coefficients $\uDEn(\tDE_j)$
on the right-hand side are obtained. To this end,
$2N+1$ sampling points are set at
$t=\tDE_i$ $(i=-N,\,-N+1,\,\ldots,\,N)$ in~\cref{eq:DE-Sinc-Nystroem}.
This leads a system of linear equations
\begin{equation}
\label{eq:DE-Sinc-Nystroem-linear-eq}
 (I_n - \kDEn) \mathbd{u}_n^{\textDE} = \gDEn,
\end{equation}
where $\kDEn$ be $n\times n$ matrix
whose $(i, j)$-th element is
\[
 \left(\kDEn\right)_{ij}
= k(\tDE_i,\tDE_j)\DEtDiv(jh)h \delta_{i-j}^{(-1)},
%\left(\frac{1}{2}+ \sigma_{i-j}\right),
\quad i= -N,\,\ldots,\,N,\quad j=-N,\,\ldots,\,N,
\]
and $\gDEn$ and $\mathbd{u}_n^{\textDE}$
be $n$-dimensional vectors defined by
\begin{align*}
 \gDEn
 &= [g(\tDE_{-N}),\,g(\tDE_{-N+1}),\,\ldots,\,g(\tDE_N)]^{\mathrm{T}},\\
 \mathbd{u}_n^{\textDE}
 &= [\uDEn(\tDE_{-N}),\,\uDEn(\tDE_{-N+1}),\,\ldots,\,\uDEn(\tDE_N)]^{\mathrm{T}}.
\end{align*}
By solving~\cref{eq:DE-Sinc-Nystroem-linear-eq},
we obtain the unknown coefficients $\mathbd{u}_n^{\textDE}$,
from which the approximated solution $\uDEn$ is determined
by~\cref{eq:DE-Sinc-Nystroem}.
This procedure is called the DE-Sinc-Nystr\"{o}m method.
Its convergence theorem was provided as follows.

\begin{theorem}[Okayama et al.~{\cite[Theorem~3.5]{okayama13:_theo}}]
\label{thm:DE-Sinc-Nystroem}
Let $d$ be a positive constant with $d<\pi/2$.
Assume that $g$, $k(z,\cdot)$ and $k(\cdot,w)$
belong to $\Hinf(\DEt(\domD_d))$ for all $z$, $w\in\DEt(\domD_d)$.
Furthermore, assume that
$g$, $k(t,\cdot)$ and $k(\cdot,s)$ belong to $C([a,b])$
for all $t$, $s\in [a, b]$.
Let $h$ be selected by the formula~\cref{eq:h-DE} with $\alpha=1$.
Then, there exists a positive integer $N_0$
such that for all $N\geq N_0$,
the coefficient matrix $(I_n - \kDEn)$ is invertible.
Furthermore, there exists a constant $C$ independent of $N$
such that for all $N\geq N_0$,
\[
 \|u - \uDEn\|_{C([a,b])}
 \leq C \frac{\log(2 d N)}{N}\rme^{-\pi d N/log(2 d N)}.
\]
\end{theorem}


\section{Existing Sinc-collocation methods}
\label{sec:collocation}

This section describes two different Sinc-collocation methods
developed by
Stenger~\cite{stenger93:_numer}
and Rashidinia and Zarebnia~\cite{rashidinia07:_solut}.
Both methods employ the tanh transformation~\cref{eq:SEt}
as a variable transformation,
but their procedures are not identical.

\subsection{Sinc-collocation method by Stenger}

As explained in~\cref{sec:introduction},
Stenger derived his method for~\cref{eq:Volterra-initial-val},
where the kernel $\tilde{k}$ is a function of a single variable.
However, his method can be easily derived for~\cref{eq:Volterra-int}
as follows.
His method is closely related to the SE-Sinc-Nystr\"{o}m method,
which is described in the previous section.
First, solve the linear system~\cref{eq:SE-Sinc-Nystroem-linear-eq}
and obtain~$\mathbd{u}_n^{\textSE}$.
Then, his approximated solution $\vSEn$ is expressed as
the generalized SE-Sinc approximation of $\uSEn$, i.e.,
\begin{align}
\label{eq:SE-Sinc-collocation}
 \vSEn(t)
& = \ProjSE[\uSEn](t)\\
& = \sum_{j=-N}^N
\left\{\uSEn(\tSE_j) - \uSEn(\tSE_{-N})\omega_a(\tSE_j)
 - \uSEn(\tSE_N)\omega_b(\tSE_j)\right\}S(j,h)(\SEtInv(t))\nonumber\\
&\quad + \uSEn(\tSE_{-N})\omega_a(t) + \uSEn(\tSE_{N})\omega_b(t),\nonumber
\end{align}
where $\ProjSE$ is defined by~\cref{eq:ProjSE}.

The solution $\vSEn$ is also obtained
by the standard collocation procedure as follows.
Set the approximate solution $\vSEn$ as~\cref{eq:SE-Sinc-collocation},
where $\uSEn(\tSE_j)$ $(j=-N,\,\ldots,\,N)$ are
regarded as unknown coefficients.
%\begin{align*}
% \vSEn(t) =
% \sum_{j=-N}^N\left\{
% u_j - u_{-N}\omega_a(\tSE_j) - u_N\omega_b(\tSE_j)
% \right\}S(j,h)(\SEtInv(t))
% +u_{-N}\omega_a(t) + u_N\omega_b(t).
%\end{align*}
Substitute $\vSEn$ into the given equation~\cref{eq:Volterra-int},
with approximating the Volterra integral operator $\Vol$ by $\VolSEn$, where
\begin{equation}
\label{eq:VolSEn}
 \VolSEn [f](t)
=\sum_{j=-N}^N k(t,\tSE_j)f(\tSE_j)\SEtDiv(jh)J(j,h)(\SEtInv(t)),
\end{equation}
which is the SE-Sinc indefinite integration of $\Vol f$.
Then, setting $n=2N+1$ sampling points at $t=\tSE_i$
$(i=-N,\,-N+1,\,\ldots,\,N)$, we obtain
the same system of linear equations as~\cref{eq:SE-Sinc-Nystroem-linear-eq}.
%\[
% \sum_{j=-N}^N
%\left\{\delta_{ij} - k(\tSE_i,\tSE_j)\SEtDiv(jh)h\delta_{i-j}^{(-1)}\right\}
% u_j = g(\tSE_i), \quad i=-N,\,\ldots,\,N,
%\]
%where $\delta_{ij}$ is the Kronecker delta.
%This system of linear equations is
%nothing but~\cref{eq:SE-Sinc-Nystroem-linear-eq}.
Thus, Stenger's method can be regarded as a collocation method
utilizing the generalized SE-Sinc approximation,
namely, SE-Sinc-collocation method.

\subsection{Sinc-collocation method by Rashidinia and Zarebnia}

Rashidinia and Zarebnia derived their method
by the standard collocation procedure,
but they considered their approximated solution $\vRZn$
in different manners in the following four cases.
\begin{enumerate}
 \item[(I)] If $u(a)=u(b)=0$, set $\vRZn$ as
\[
 \vRZn(t) = \sum_{j=-N}^{N} c_{j} S(j,h)(\SEtInv(t)).
\]
 \item[(II)] If $u(a)\neq 0$ and $u(b)=0$, set $\vRZn$ as
\[
 \vRZn(t) = c_{-N}\omega_a(t) + \sum_{j=-N+1}^{N} c_{j} S(j,h)(\SEtInv(t)).
\]
 \item[(III)] If $u(a)= 0$ and $u(b)\neq 0$, set $\vRZn$ as
\[
 \vRZn(t) = \sum_{j=-N}^{N-1} c_{j} S(j,h)(\SEtInv(t)) + c_{N}\omega_b(t).
\]
 \item[(IV)] If $u(a)\neq 0$ and $u(b)\neq 0$, set $\vRZn$ as
\end{enumerate}
\begin{equation}
\label{eq:vRZn}
 \vRZn(t) = c_{-N}\omega_a(t)
 + \sum_{j=-N+1}^{N-1} c_{j} S(j,h)(\SEtInv(t)) + c_{N}\omega_b(t).
\end{equation}
To obtain the unknown coefficients
$\mathbd{c}_{n} = [c_{-N},\,c_{-N+1},\,\ldots,\,c_{N}]^{\mathrm{T}}$,
where $n=2N+1$,
they substituted $\vRZn$ into the given equation~\cref{eq:Volterra-int},
with approximating the Volterra integral operator $\Vol$ by $\VolSEn$.
Then, setting $n$ sampling points at $t=\tSE_i$
$(i=-N,\,-N+1,\,\ldots,\,N)$,
they derived a system of linear equations
in each of the four cases: (I)--(IV).
For example, in the case (I), the resulting system is expressed as
\[
 (I_n - V_{n}^{\textSE})\mathbd{c}_n = \mathbd{g}_n^{\textSE}.
\]
Particularly, in the case (IV), the resulting system is expressed as
\begin{equation}
\label{eq:RZ-linear-eq}
 (E_n^{\textRZ} - V_{n}^{\textRZ})\mathbd{c}_n = \mathbd{g}_n^{\textSE},
\end{equation}
where $E_n^{\textRZ}$ and $V_n^{\textRZ}$ are
$n\times n$ matrices defined by
\begin{align*}
 E_n^{\textRZ}
&= \left[
   \begin{array}{@{\,}c|ccc|c@{\,}}
   \omega_a(\tSE_{-N})        & 0    & \cdots &0  & \omega_b(\tSE_{-N}) \\
%   \hline
   \omega_a(\tSE_{-N+1}) & 1      &        &\Order & \omega_b(\tSE_{-N+1})\\
   \vdots     &        & \ddots &       & \vdots \\
   \omega_a(\tSE_{N-1}) & \Order &        &1      & \omega_b(\tSE_{N-1}) \\
%   \hline
   \omega_a(\tSE_{N}) & 0      & \cdots &0      & \omega_b(\tSE_{N})
   \end{array}
   \right], \\
  V_m^{\textRZ}
&= \left[
   \begin{array}{@{\,}l|clc|l@{\,}}
%   \VolSEn[\omega_a](\tSE_{-N-1})
   &\cdots
   & k(\tSE_{-N},\tSE_j)\SEtDiv(jh) h \delta_{-N-j}^{(-1)}
   &\cdots
   & \\ %\VolSEn[\omega_b](\tSE_{-N-1}) \\
%  \hline
%   \VolSEn[\omega_a](\tSE_{-N})
   &\cdots
   & k(\tSE_{-N+1},\tSE_j)\SEtDiv(jh) h \delta_{-N+1-j}^{(-1)}
   &\cdots
   & \\ %\VolSEn[\omega_b](\tSE_{-N}) \\
    \multicolumn{1}{c|}{\mathbd{p}_n^{\textRZ}} & & \multicolumn{1}{c}{\vdots}
   & & \multicolumn{1}{c}{\mathbd{q}_n^{\textRZ}}\\
%   \VolSEn[\omega_a](\tSE_{N})
   &\cdots
   & k(\tSE_{N-1},\tSE_j)\SEtDiv(jh) h \delta_{N-1-j}^{(-1)}
   &\cdots
   & \\ %\VolSEn[\omega_b](\tSE_N) \\
%  \hline
%   \VolSEn[\omega_a](\tSE_{N+1})
   &\cdots
   & k(\tSE_{N},\tSE_j)\SEtDiv(jh) h \delta_{N-j}^{(-1)}
   &\cdots
   & %\VolSEn[\omega_b](\tSE_{N+1})
   \end{array}
   \right],
\end{align*}
where $\mathbd{p}_n^{\textRZ}$
and $\mathbd{q}_n^{\textRZ}$
are $n$-dimensional vectors defined by
\begin{align*}
\mathbd{p}_n^{\textRZ}
&= [ \VolSEn[\omega_a](\tSE_{-N}),\,
\VolSEn[\omega_a](\tSE_{-N+1}),\,\ldots,\,
\VolSEn[\omega_a](\tSE_{N})]^{\mathrm{T}}, \\
\mathbd{q}_n^{\textRZ}
&= [ \VolSEn[\omega_b](\tSE_{-N}),\,
\VolSEn[\omega_b](\tSE_{-N+1}),\,\ldots,\,
\VolSEn[\omega_b](\tSE_{N})]^{\mathrm{T}}.
\end{align*}
This is the SE-Sinc-collocation method by Rashidinia and Zarebnia.
In the case (I), the following error analysis was provided.

\begin{theorem}[Zarebnia and Rashidinia~{\cite[Theorem~3]{zarebnia10:_conv}}]
\label{thm:Rarebnia-Rashidinia}
Let $\alpha$ and $d$ be positive constants with $d<\pi$.
Assume that the solution $u$ in~\cref{eq:Volterra-int}
satisfies all the assumptions in~\cref{thm:SE-Sinc-approx}.
Furthermore, assume that $k(t,\cdot)$
satisfies all the assumptions in~\cref{thm:SE-Sinc-indefinite}
for all $t\in [a, b]$.
Then, there exists a constant $C$ independent of $N$ such that
\[
 \|u - \vRZn\|_{C([a,b])}
\leq C \|(I_n- V_{n}^{\textSE})^{-1}\|_2\sqrt{N}\rme^{-\sqrt{\pi d \alpha N}}.
\]
\end{theorem}

However,
this theorem does not prove the convergence of $\vRZn$,
because there exists an unestimated
term $\|(I_n - V_{n}^{\textSE})^{-1}\|_2$, which clearly depends on $N$.
For the cases (II)--(IV), no error analysis has been provided thus far.

Moreover,
in a practical situation,
it is hard to determine whether $u$ is zero or not at the endpoints.
This is because the solution $u$ is an unknown function to be determined.
The idea to address the issue was presented
for Fredholm integral equations~\cite{okayama1x:_improv};
set the approximate solution $\vRZn$ as~\cref{eq:vRZn} in any cases.
In other words, we may treat the case (IV) as a general case.
This idea can be employed
for Volterra integral equations~\cref{eq:Volterra-int}.
Therefore, as a method by Rashidinia and Zarebnia,
this study adopts the following procedure:
(i) solve the linear system~\cref{eq:RZ-linear-eq},
and (ii) obtain the approximate solution by~\cref{eq:vRZn}.

\subsection{Main result 1: Relationship between the two methods and their convergence}

Any relationship between Stenger's method $(\vSEn)$
and Rashidinia--Zarebnia's method $(\vRZn)$ has not been investigated thus far.
Furthermore, convergence of the two methods has not been rigorously proved.
As a first contribution of this paper,
we show the relationship between the two methods as follows.
The proof is provided in~\cref{sec:proof-equivalence}.

\begin{theorem}
\label{thm:equivalence}
%Let $\alpha$ and $d$ be positive constants with
%$\alpha\leq 1$ and $d<\pi$.
%Assume that
%all the assumptions on $g$ and $k$ in~\cref{thm:SE-Sinc-Nystroem}
%are fulfilled.
%Let $h$ be selected by the formula~\cref{eq:h-SE}.
%Then, there exists a positive integer $N_0$
%such that for all $N\geq N_0$,
%$\vSEn = \vRZn$ holds.
Let $\vSEn$ be a function defined by~\cref{eq:SE-Sinc-collocation},
where $\mathbd{u}_n^{\textSE}$ is determined
by solving the linear system~\cref{eq:SE-Sinc-Nystroem-linear-eq}.
Furthermore,
let $\vRZn$ be a function defined by~\cref{eq:vRZn},
where $\mathbd{c}_{m}$ is determined
by solving the linear system~\cref{eq:RZ-linear-eq}.
Then, it holds that
\[
 \vSEn(\tSE_i) = \vRZn(\tSE_i),\quad i=-N,\,-N+1,\,\ldots,\,N,
\]
but generally $\vSEn\neq \vRZn$.
\end{theorem}

%Therefore, we may refer to both methods as the SE-Sinc-collocation method.
Subsequently, we provide the convergence theorems of
the two methods as follows.
Their proofs are provided in~\cref{sec:proof-SE-Sinc,sec:proof-RZ-Sinc}.

\begin{theorem}
\label{thm:SE-Sinc-collocation}
Let $\alpha$ and $d$ be positive constants with
$\alpha\leq 1$ and $d<\pi$.
Assume that
all the assumptions on $g$ and $k$ in~\cref{thm:SE-Sinc-Nystroem}
are fulfilled.
Furthermore, assume that $g$ and $k(\cdot,w)$
belong to $\MC_{\alpha}(\SEt(\domD_d))$ for all $w\in\SEt(\domD_d)$.
Let $h$ be selected by the formula~\cref{eq:h-SE}.
Then, there exists a positive integer $N_0$
such that for all $N\geq N_0$,
the coefficient matrix $(I_n - \kSEn)$ is invertible.
Furthermore, there exists a constant $C$ independent of $N$
such that for all $N\geq N_0$,
\[
 \|u - \vSEn\|_{C([a,b])} \leq C \sqrt{N} \rme^{-\sqrt{\pi d \alpha N}}.
\]
\end{theorem}

\begin{theorem}
\label{thm:RZ-Sinc-collocation}
Assume that all the assumptions of~\cref{thm:SE-Sinc-collocation}
are fulfilled.
Then, there exists a positive integer $N_0$
such that for all $N\geq N_0$,
the coefficient matrix $(E_n^{\textRZ} - V_{n}^{\textRZ})$ is invertible.
Furthermore, there exists a constant $C$ independent of $N$
such that for all $N\geq N_0$,
\[
%\label{eq:vRZn-estimate}
 \|u - \vRZn\|_{C([a,b])} \leq C \sqrt{N} \rme^{-\sqrt{\pi d \alpha N}}.
\]
\end{theorem}

\begin{remark}
In view of~\cref{thm:Rarebnia-Rashidinia,thm:RZ-Sinc-collocation},
one might assume that~\cref{thm:SE-Sinc-collocation}
is proved by bounding $\|(I_n - \kSEn)^{-1}\|_2$ uniformly for $N$.
However, this is not the case; see~\cref{sec:proof-SE} for details.
\end{remark}

\Cref{thm:SE-Sinc-collocation,thm:RZ-Sinc-collocation} reveal that
both methods achieve the same convergence rate.
Therefore, users may prefer Stenger's method
because the implementation of the method by Rashidinia and Zarebnia
is rather complicated.
This complication also causes difficulty in extension to the \emph{system} of
Volterra integral equations.
For this reason, in the next section,
we consider the improvement of Stenger's method.


\section{Sinc-collocation method combined with the DE transformation}
\label{sec:de-collocation}

The SE-Sinc-collocation method described in the previous section
employs the tanh transformation as a variable transformation.
In this section, a new method is derived
by replacing the tanh transformation with the DE transformation.
Then, its convergence theorem is stated.


\subsection{Derivation of the DE-Sinc-collocation method}

First, solve the linear system~\cref{eq:DE-Sinc-Nystroem-linear-eq}
and obtain~$\mathbd{u}_n^{\textDE}$.
Then, the approximated solution $\vDEn$ is expressed as
the generalized DE-Sinc approximation of $\uDEn$, i.e.,
\begin{align}
\label{eq:DE-Sinc-collocation}
 \vDEn(t)
& = \ProjDE[\uDEn](t)\\
& = \sum_{j=-N}^N
\left\{\uDEn(\tDE_j) - \uDEn(\tDE_{-N})\omega_a(\tDE_j)
 - \uDEn(\tDE_N)\omega_b(\tDE_j)\right\}S(j,h)(\DEtInv(t))\nonumber\\
&\quad + \uDEn(\tDE_{-N})\omega_a(t) + \uDEn(\tDE_{N})\omega_b(t),\nonumber
\end{align}
where $\ProjDE$ is defined by~\cref{eq:ProjDE}.
This procedure is referred to as the DE-Sinc-collocation method.

\subsection{Main result 2: Convergence of the DE-Sinc-collocation method}

In this paper,
we show the convergence of the DE-Sinc-collocation method
as follows.
The proof is provided in~\cref{sec:proof-DE}.

\begin{theorem}
\label{thm:DE-Sinc-collocation}
Let $\alpha$ and $d$ be positive constants with
$\alpha\leq 1$ and $d<\pi/2$.
Assume that
all the assumptions on $g$ and $k$ in~\cref{thm:DE-Sinc-Nystroem}
are fulfilled.
Furthermore, assume that $g$ and $k(\cdot,w)$
belong to $\MC_{\alpha}(\DEt(\domD_d))$ for all $w\in\DEt(\domD_d)$.
Let $h$ be selected by the formula~\cref{eq:h-DE}.
Then, there exists a positive integer $N_0$
such that for all $N\geq N_0$,
the coefficient matrix $(I_n - \kDEn)$ is invertible.
Furthermore, there exists a constant $C$ independent of $N$
such that for all $N\geq N_0$,
\[
 \|u - \vDEn\|_{C([a,b])}
\leq C \rme^{-\pi d N/\log(2 d N/\alpha)}.
\]
\end{theorem}

Compared to~\cref{thm:SE-Sinc-collocation,thm:RZ-Sinc-collocation},
we see that the convergence rate given by this theorem
is significantly improved.


\section{Numerical experiments}
\label{sec:numer-result}

This section presents numerical results for the following five methods:
the SE/DE-Sinc-Nystr\"{o}m methods
by Muhammad et al.~\cite{muhammad05:_numer},
the SE-Sinc-collocation methods by Stenger~\cite{stenger93:_numer}
and Rashidinia--Zarebnia~\cite{rashidinia07:_solut},
and the DE-Sinc-collocation methods by this paper.
The computation was performed on
a MacBook Air computer with 1.7~GHz Intel Core i7
with 8 GB memory, running macOS Big Sur.
The computation programs were implemented
in the C programming language with double-precision floating-point arithmetic,
and compiled with Apple clang version 13.0.0 with no optimization.
Cephes Math Library was used for computation of the sine integral.
LAPACK in Apple's Accelerate framework was used for computation
of the system of linear equations.
The source code for all programs is available at
\url{https://github.com/okayamat/sinc-colloc-volterra}.

We consider the following two equations
(taken from Rashidinia--Zarebnia~\cite[Example~4]{rashidinia07:_solut}
and~Polyanin--Manzhirov~\cite[Equation 2.1.45]{polyanin08:_handb}):
\begin{align}
\label{eq:example1}
u(t) + \int_0^t t s u(s)\diff s
 &= \rme^{-t^2} +\frac{t}{2}(1 - \rme^{-t^2}),\quad 0\leq t\leq 1,\\
\label{eq:example2}
u(t) - 6\int_0^t (\sqrt{t} - \sqrt{s}) u(s)\diff s
 &= 1 +\sqrt{t} - 2t\sqrt{t} - t^2,\quad 0\leq t\leq 1,
\end{align}
whose solutions are $u(t)=\rme^{-t^2}$ and
$u(t)=1 + \sqrt{t}$, respectively.
In the case of~\cref{eq:example1},
the assumptions
of~\cref{thm:SE-Sinc-Nystroem,thm:SE-Sinc-collocation,thm:RZ-Sinc-collocation}
are fulfilled with $d=3.14$
and $\alpha=1$,
and those of~\cref{thm:DE-Sinc-Nystroem,thm:DE-Sinc-collocation}
are fulfilled with $d=1.57$
and $\alpha=1$.
In the case of~\cref{eq:example2},
the assumptions
of~\cref{thm:SE-Sinc-Nystroem,thm:SE-Sinc-collocation,thm:RZ-Sinc-collocation}
are fulfilled with $d=3.14$
and $\alpha=1/2$,
and those of~\cref{thm:DE-Sinc-Nystroem,thm:DE-Sinc-collocation}
are fulfilled with $d=1.57$ and $\alpha=1/2$.
Therefore, those values were used for implementation.
The errors were evaluated at 2048 equally spaced points 
over the given interval,
and the maximum error among them was plotted on the graph
in~\cref{fig:example1_N,fig:example1_t,fig:example2_N,fig:example2_t}.

From all figures, we can observe that
the SE-Sinc-collocation methods by Stenger and Rashidinia--Zarebnia
yield almost the same performance.
This result coincides
with~\cref{thm:SE-Sinc-collocation,thm:RZ-Sinc-collocation}.
%the description at the end of~\cref{sec:collocation}.
%Furthermore, the DE-Sinc-collocation method presented by this paper
From~\cref{fig:example1_N}, we can observe that
the SE/DE-Sinc-Nystr\"{o}m methods are slightly better than
the SE/DE-Sinc-collocation methods with respect to $N$.
This result coincides with~\cref{thm:SE-Sinc-Nystroem,thm:DE-Sinc-Nystroem,thm:SE-Sinc-collocation,thm:RZ-Sinc-collocation,thm:DE-Sinc-collocation}.
However,~\cref{fig:example1_t} shows that
with respect to the computation time,
the SE/DE-Sinc-collocation methods demonstrate significantly better performance
than the SE/DE-Sinc-Nystr\"{o}m methods.
This is because the SE/DE-Sinc-Nystr\"{o}m methods
include a special function as well as given functions $k$ and $g$
in the basis functions of their approximate solutions.
We note that the performance of
the SE/DE-Sinc-collocation methods in~\cref{fig:example2_N}
reduced than that in~\cref{fig:example1_N},
which is due to the difference of $\alpha$.

\begin{figure}[htbp]
\begin{minipage}{0.495\linewidth}
  \centering
 \includegraphics[width=\linewidth]{example1_N}
  \caption{Errors with respect to $N$ for~\cref{eq:example1}. \phantom{not computation time}}
  \label{fig:example1_N}
\end{minipage}
\begin{minipage}{0.495\linewidth}
  \centering
  \includegraphics[width=\linewidth]{example1_t}
  \caption{Errors with respect to the computation time for~\cref{eq:example1}.}
  \label{fig:example1_t}
\end{minipage}
\end{figure}
\begin{figure}[htbp]
\begin{minipage}{0.495\linewidth}
  \centering
 \includegraphics[width=\linewidth]{example2_N}
  \caption{Errors with respect to $N$ for~\cref{eq:example2}. \phantom{not computation time}}
  \label{fig:example2_N}
\end{minipage}
\begin{minipage}{0.495\linewidth}
  \centering
  \includegraphics[width=\linewidth]{example2_t}
  \caption{Errors with respect to the computation time for~\cref{eq:example2}.}
  \label{fig:example2_t}
\end{minipage}
\end{figure}


\section{Proofs for the theorems presented in~\cref{sec:collocation}}
\label{sec:proof-SE}

In this section,
we provide proofs for~\cref{thm:equivalence,thm:SE-Sinc-collocation}.

\subsection{Proof of~\cref{thm:equivalence}}
\label{sec:proof-equivalence}

In addition to the given equation $(\Ident - \Vol)u = g$,
let us consider the following three equations:
\begin{align}
 (\Ident - \VolSEn)    \uSEn &= g, \label{eq:SE-Sinc-Nystroem-symbol} \\
 (\Ident - \ProjSE \VolSEn)v &= \ProjSE g,
 \label{eq:SE-Sinc-collocation-symbol}\\
% (E_m^{\textRZ} - V_{m}^{\textRZ})\mathbd{c}_m &= \mathbd{g}_m^{\textSE},
 (\Ident - \ProjRZ \VolSEn)w &= \ProjRZ g,
\label{eq:RZ-Sinc-collocation-symbol}
\end{align}
where $\VolSEn$ and $\ProjSE$ are defined
by~\cref{eq:VolSEn} and~\cref{eq:ProjSE}, respectively,
and $\ProjRZ$ is defined by
\begin{align}
\label{eq:ProjRZ}
  \ProjRZ[f](t)
&=\sum_{j=-N+1}^{N+1}\left\{f(\tSE_j) - \beta_N \omega_a(\tSE_j)
 - \gamma_N \omega_a(\tSE_j)\right\}S(j,h)(\SEtInv(t))\\
&\quad + \beta_N \omega_a(t) + \gamma_N \omega_b(t),
\nonumber
\end{align}
where $\beta_N$ and $\gamma_N$ are defined by
\begin{align*}
 \beta_N &=
\frac{f(\tSE_{-N})\omega_b(\tSE_{N}) - f(\tSE_{N})\omega_b(\tSE_{-N})}
     {\omega_a(\tSE_{-N})\omega_b(\tSE_{N}) - \omega_b(\tSE_{-N})\omega_a(\tSE_{N})},\\
\gamma_N &=
\frac{f(\tSE_{N})\omega_a(\tSE_{-N}) - f(\tSE_{-N})\omega_a(\tSE_{N})}
     {\omega_a(\tSE_{-N})\omega_b(\tSE_{N}) - \omega_b(\tSE_{-N})\omega_a(\tSE_{N})}.
\end{align*}
\begin{remark}
The denominator of $\beta_N$ and $\gamma_N$ is not zero
because
\begin{align*}
 \omega_a(\tSE_{-N})\omega_b(\tSE_{N}) - \omega_b(\tSE_{-N})\omega_a(\tSE_{N})
&=(1 - \omega_b(\tSE_{-N}))\omega_b(\tSE_N)
- \omega_b(\tSE_{-N})(1 - \omega_b(\tSE_N))\\
&=\omega_b(\tSE_N) - \omega_b(\tSE_{-N})\\
&= \tanh\left(\frac{Nh}{2}\right)\neq 0,
\end{align*}
provided that $N$ is a positive integer and $h> 0$.
\end{remark}

Because~\cref{eq:SE-Sinc-Nystroem-symbol} is equivalent
to~\cref{eq:SE-Sinc-Nystroem},
the solution of~\cref{eq:SE-Sinc-Nystroem-symbol}
is the approximate solution of the SE-Sinc-Nystr\"{o}m method.
On~\cref{eq:SE-Sinc-Nystroem-symbol}, the following result was obtained.

\begin{lemma}[Okayama et al.~{\cite[Lemma~6.7]{okayama13:_theo}}]
\label{lem:SE-Sinc-Nystroem}
Assume that
all the assumptions of~\cref{thm:SE-Sinc-Nystroem}
are fulfilled.
Then, there exists a positive integer $N_0$ such that for all $N\geq N_0$,
\cref{eq:SE-Sinc-Nystroem-symbol}
has a unique solution $\uSEn\in C([a,b])$.
Furthermore, there exists a constant $C$ independent of $N$ such that
for all $N\geq N_0$,
\begin{equation}
\label{eq:SE-Sinc-Nystroem-error}
 \|u - \uSEn\|_{C([a,b])}\leq C \|\Vol u - \VolSEn u\|_{C([a,b])}.
\end{equation}
\end{lemma}

This lemma says that~\cref{eq:SE-Sinc-Nystroem-symbol}
has a unique solution for all sufficiently large $N$.
Using this result, we show the following three things:
\begin{enumerate}
 \item[(i)] If~\cref{eq:SE-Sinc-Nystroem-symbol}
has a unique solution,
then~\cref{eq:SE-Sinc-collocation-symbol}
has also a unique solution $v = \vSEn$.
 \item[(ii)] If~\cref{eq:SE-Sinc-Nystroem-symbol}
has a unique solution,
then~\cref{eq:RZ-Sinc-collocation-symbol}
has also a unique solution $w = \vRZn$.
 \item[(iii)] The two solutions $\vSEn$ and $\vRZn$ are not generally
equivalent, but at the collocation points,
 $\vSEn(\tSE_i) = \vRZn(\tSE_{i})$ $(i=-N,\,\ldots,\,N)$ holds.
\end{enumerate}
First, we show (i) as follows.

\begin{lemma}
\label{lem:Stenger-solution}
The following two statements are equivalent:
\begin{enumerate}
 \item[{\rm (A)}] \Cref{eq:SE-Sinc-Nystroem-symbol}
has a unique solution $\uSEn\in C([a,b])$.
 \item[{\rm (B)}] \Cref{eq:SE-Sinc-collocation-symbol} has
 a unique solution $v\in C([a,b])$.
\end{enumerate}
Furthermore, $v=\vSEn$ holds.
\end{lemma}
\begin{proof}
First, let us show $\mathrm{(A)} \Rightarrow \mathrm{(B)}$.
Note that $\VolSEn\ProjSE f = \VolSEn f$ holds
because
of the interpolation property $\ProjSE[f](\tSE_i)=f(\tSE_i)$
($i=-N,\,\ldots,\,N$).
Applying $\ProjSE$ on the both sides
of~\cref{eq:SE-Sinc-Nystroem-symbol},
we have
\[
 \ProjSE\uSEn
 = \ProjSE(g + \VolSEn\uSEn)
 = \ProjSE(g + \VolSEn\ProjSE\uSEn),
\]
which is equivalent to $\vSEn = \ProjSE(g + \VolSEn \vSEn)$
(recall that $\vSEn = \ProjSE\uSEn$).
This equation implies
that~\cref{eq:SE-Sinc-collocation-symbol} has a solution
$\vSEn\in C([a,b])$.

Next, we show the uniqueness.
Suppose that~\cref{eq:SE-Sinc-collocation-symbol}
has another solution $\tilde{v}\in C([a,b])$.
Let us set a function $\tilde{u}$ as
$\tilde{u} = g + \VolSEn\tilde{v}$.
Because $\tilde{v}$ is a solution of~\cref{eq:SE-Sinc-collocation-symbol},
we have
\[
 \tilde{v} = \ProjSE(g + \VolSEn\tilde{v}) = \ProjSE \tilde{u},
\]
from which it holds that
\[
 \tilde{u} = g + \VolSEn\tilde{v} = g + \VolSEn\ProjSE\tilde{u}
= g + \VolSEn\tilde{u}.
\]
This equation implies that $\tilde{u}$ is a solution
of~\cref{eq:SE-Sinc-Nystroem-symbol}.
Because the solution of~\cref{eq:SE-Sinc-Nystroem-symbol}
is unique, $\tilde{u} = u$ holds,
from which we have $\ProjSE\tilde{u}=\ProjSE u$.
Thus, we have $\tilde{v}=v$, which shows $(\mathrm{B})$.

The above argument is reversible,
which proves $\mathrm{(B)}\Rightarrow\mathrm{(A)}$.
Furthermore, in view of the proof above,
we see $v=\vSEn$, which is to be demonstrated.
\end{proof}

Next, for showing (ii),
we show the following result.
The proof is omitted because it goes in the same way as
that of~\cref{lem:Stenger-solution}.

\begin{lemma}
\label{lem:RZ-solution}
The following two statements are equivalent:
\begin{enumerate}
 \item[{\rm (A)}] \Cref{eq:SE-Sinc-Nystroem-symbol}
has a unique solution $\uSEn\in C([a,b])$.
 \item[{\rm (B)}] \Cref{eq:RZ-Sinc-collocation-symbol} has
 a unique solution $w\in C([a,b])$.
\end{enumerate}
Furthermore, $w=\ProjRZ\uSEn$ holds.
\end{lemma}

To show (ii) completely, we further have to show $w = \vRZn$,
which is done by the following result.
Noting $\ProjRZ[f](\tSE_i)=f(\tSE_i)$ $(i=-N,\,\ldots,\,N)$,
we can prove this result
following Atkinson~\cite[Sect.\ 4.3]{atkinson97:_numer_solut},
and hence the proof is omitted.

\begin{proposition}
\label{prop:RZ-solution}
The following two statements are equivalent:
\begin{enumerate}
 \item[{\rm (A)}] \Cref{eq:RZ-Sinc-collocation-symbol} has
 a unique solution $w\in C([a,b])$.
 \item[{\rm (B)}] \Cref{eq:RZ-linear-eq} has
 a unique solution $\mathbd{c}_m\in \mathbb{R}^m$.
\end{enumerate}
Furthermore, $w=\vRZn$ holds.
\end{proposition}
%\begin{proof}
%First, let us show $\mathrm{(A)} \Rightarrow \mathrm{(B)}$.
%Note that $(\ProjSE)^2 f = \ProjSE f$ holds because
%of the interpolation property $\ProjSE[f](\tSE_i)=f(\tSE_i)$
%($i=-N,\,\ldots,\,N$).
%Because the solution $v$ of~\cref{eq:SE-Sinc-collocation-symbol}
%is written as
%\begin{equation}
%\label{eq:v-is-projected-by-ProjSE}
% v = \ProjSE (g + \VolSEn v),
%\end{equation}
%we have $\ProjSE v = (\ProjSE)^2 (g + \VolSEn v) = \ProjSE (g + \VolSEn v)= v$.
%%Therefore, there exist coefficients
%%$\mathbd{c}_m=[c_{-N-1},\,c_{-N},\,\ldots,\,c_{N+1}]^{\mathrm{T}}$
%%such that
%%\[
%% v(t) = c_{-N-1}\omega_a(t)
%% + \sum_{j=-N}^N c_j S(j,h)(\SEtInv(t)) + c_{N+1}\omega_b(t).
%%\]
%Therefore, we can rewrite~\cref{eq:v-is-projected-by-ProjSE} as
%\[
% \ProjSE( v - g - \VolSEn v) = 0.
%\]
%Put $f = v - g - \VolSEn v$.
%The equation $\ProjSE f = 0$ requires the following equations
%\[
% \ProjSE[f](\tSE_i) = f(\tSE_i) = 0,\quad i=-N,\,\ldots,\,N.
%\]
%Considering more two points $t=\tSE_{-N-1}$ and $t=\tSE_{N+1}$,
%we have
%\begin{align*}
% \ProjSE [f](\tSE_{-N-1})
%& = f(\tSE_{-N})\omega_a(\tSE_{-N-1}) + f(\tSE_{N})\omega_a(\tSE_{-N-1})
% = 0,\\
% \ProjSE [f](\tSE_{N+1})
%& = f(\tSE_{-N})\omega_a(\tSE_{N+1}) + f(\tSE_{N})\omega_a(\tSE_{N+1})
% = 0,
%\end{align*}
%because $f(\tSE_{-N})=f(\tSE_{N})=0$.
%Thus, $\ProjSE f = 0$ implies
%\[
% f(\tSE_i) = 0,\quad i = -N-1,\,-N,\,\ldots,\,N,\,N+1,
%\]
%which is equivalent to
%\begin{equation}
%\label{eq:v-N+1-equation}
%v(\tSE_i) - \VolSEn[v](\tSE_i) = g(\tSE_i),\quad i=-N-1,\,-N,\,\ldots,\,N,\,N+1.
%\end{equation}
%Here, note that $v$ is written as
%\begin{align*}
% v(t) &= \sum_{j=-N}^N
%\left\{v(\tSE_j)
% - v(\tSE_{-N})\omega_a(\tSE_j) - v(\tSE_N)\omega_b(\tSE_j)
%\right\}S(j,h)(\SEtInv(t))\\
%&\quad + v(\tSE_{-N})\omega_a(t) + v(\tSE_N)\omega_b(t),
%\end{align*}
%because $v=\ProjSE v$.
%Putting
%\begin{align*}
% c_{-N-1} &= v(\tSE_{-N}),\\
% c_i      &= v(\tSE_i) - v(\tSE_{-N})\omega_a(\tSE_i) - v(\tSE_N)\omega_b(\tSE_i),
%\quad i = -N,\,\ldots,\,N,\\
% c_{N+1}  &= v(\tSE_N),
%\end{align*}
%we find that~\cref{eq:v-N+1-equation} is nothing but
%the linear system~\cref{eq:RZ-linear-eq}.
%Thus, \cref{eq:RZ-linear-eq} has a solution $\mathbd{c}_m\in \mathbb{R}^m$.
%
%Next, we show the uniqueness.
%Suppose that~\cref{eq:RZ-linear-eq}
%has another solution $\mathbd{\tilde{c}}_m\in\mathbb{R}^m$.
%Let us set a function $\tilde{v}\in C([a,b])$ as
%\[
% \tilde{v}(t)
%= \sum_{j=-N}^N \tilde{c}_j S(j,h)(\SEtInv(t))
% + \tilde{c}_{-N-1}\omega_a(t) + \tilde{c}_{N+1}\omega_b(t).
%\]
%Because $\mathbd{\tilde{c}}_m\in\mathbb{R}^m$ is a solution
%of~\cref{eq:RZ-linear-eq}, we have
%%\begin{align*}
%%& \sum_{j=-N}^N\left\{
%%\delta_{ij} - h k(\tSE_i,\tSE_j)\SEtDiv(jh)\delta^{(-1)}_{i-j}
%%\right\}\tilde{c}_j\\
%%&\quad
%%+ \left\{\omega_a(\tSE_i) - \VolSEn[\omega_a](\tSE_i)\right\}\tilde{c}_{-N-1}\\
%%&\quad
%%+ \left\{\omega_b(\tSE_i) - \VolSEn[\omega_b](\tSE_i)\right\}\tilde{c}_{N+1}
%%= g(\tSE_i),
%%\quad i=-N-1,\,-N,\,\ldots,\,N,\,N+1.
%%\end{align*}
%\begin{align*}
%\tilde{v}(\tSE_i) - \VolSEn[\tilde{v}](\tSE_i) = g(\tSE_i),\quad
% i=-N-1,\,-N,\,\ldots,\,N,\,N+1.
%\end{align*}
%Put $\tilde{f}= \tilde{v} - g - \VolSEn\tilde{v}$.
%From $\tilde{f}(\tSE_i)=0$ $(i=-N-1,\,-N,\,\ldots,\,N,\,N+1)$,
%we have $\ProjSE \tilde{f} = 0$, which is equivalent to
%\[
% \ProjSE(\tilde{v} - g -\VolSEn \tilde{v}) = 0.
%\]
%If $\ProjSE\tilde{v} =\tilde{v}$ is shown, then this equation is rewritten as
%\[
% \tilde{v} - \ProjSE\VolSEn\tilde{v} = \ProjSE g,
%\]
%which implies $\tilde{v}$ is a solution
%of~\cref{eq:SE-Sinc-collocation-symbol}.
%Because the solution of~\cref{eq:SE-Sinc-collocation-symbol}
%is unique, $\tilde{v} = v$ holds.
%This implies $\mathbd{\tilde{c}}_m = \mathbd{c}_m$,
%i.e., the solution of~\cref{eq:RZ-linear-eq} is unique,
%which shows $\mathrm{(B)}$.
%What remains to be shown is $\ProjSE\tilde{v} =\tilde{v}$.
%Set $\tilde{v}_i$ as the value of $\tilde{v}(\tSE_i)$, i.e.,
%\begin{equation}
%\label{eq:tilde-v-i}
%\tilde{v}_i
%=
% \tilde{c}_{-N-1}\omega_a(\tSE_i) + \tilde{c}_i
% + \tilde{c}_{N+1}\omega_b(\tSE_i),
%\quad i=-N,\,\ldots,\,N.
%\end{equation}
%Calculating $\ProjSE\tilde{v}$ by definition~\cref{eq:ProjSE}, we have
%\begin{align*}
% \ProjSE[\tilde{v}](t)
%&=\sum_{j=-N}^N \left\{\tilde{v}_j - \tilde{v}_{-N}\omega_a(\tSE_j)
%- \tilde{v}_{N}\omega_b(\tSE_j)
%\right\}S(j,h)(\SEtInv(t))\\
%&\quad + \tilde{v}_{-N}\omega_a(t) + \tilde{v}_{N}\omega_b(t).
%\end{align*}
%This function satisfies $\ProjSE[\tilde{v}](\tSE_i)=\tilde{v}(\tSE_i)$
%$(i=-N,\,\ldots,\,N)$ by itself.
%Furthermore, we require the equality at more two points,
%$t=\tSE_{-N-1}$ and $t=\tSE_{N+1}$.
%Substituting the two points into $\ProjSE[\tilde{v}](t)$, we have
%\begin{align*}
% \ProjSE[\tilde{v}](\tSE_{-N-1})
%&= \tilde{v}_{-N}\omega_a(\tSE_{-N-1}) + \tilde{v}_{N}\omega_b(\tSE_{-N-1}),\\
% \ProjSE[\tilde{v}](\tSE_{N+1})
%&= \tilde{v}_{-N}\omega_a(\tSE_{N+1}) + \tilde{v}_{N}\omega_b(\tSE_{N+1}),
%\end{align*}
%which must be equal to
%\begin{align*}
% \tilde{v}(\tSE_{-N-1})
%&= \tilde{c}_{-N-1}\omega_a(\tSE_{-N-1}) +\tilde{c}_{N+1}\omega_b(\tSE_{-N-1}),
%\\
% \tilde{v}(\tSE_{N+1})
%&= \tilde{c}_{-N-1}\omega_a(\tSE_{N+1}) +\tilde{c}_{N+1}\omega_b(\tSE_{N+1}),
%\end{align*}
%respectively.
%Therefore, we have $\tilde{v}_{-N}=\tilde{c}_{-N-1}$
%and $\tilde{v}_N = \tilde{c}_{N+1}$,
%from which it holds that
%\begin{align*}
% \ProjSE[\tilde{v}](t)
%&=\sum_{j=-N}^N \left\{\tilde{v}_j - \tilde{c}_{-N-1}\omega_a(\tSE_j)
%- \tilde{c}_{N+1}\omega_b(\tSE_j)
%\right\}S(j,h)(\SEtInv(t))\\
%&\quad + \tilde{c}_{-N-1}\omega_a(t) + \tilde{c}_{N+1}\omega_b(t)\\
%&= \tilde{v}(t),
%\end{align*}
%where~\cref{eq:tilde-v-i} is used at the last equality.
%This completes the proof for $\mathrm{(A)}\Rightarrow\mathrm{(B)}$.
%
%The proof for $\mathrm{(B)}\Rightarrow\mathrm{(A)}$
%is omitted because it goes in the same manner as that for
%$\mathrm{(A)}\Rightarrow\mathrm{(B)}$.
%Furthermore, in view of the proof above,
%we see $\tilde{v}=\vRZn$, which is to be demonstrated.
%\end{proof}

From the above results (i) and (ii),
we find that $\vSEn = \ProjSE\uSEn$ and $\vRZn = \ProjRZ\uSEn$.
Using the interpolation property of $\ProjSE$ and $\ProjRZ$ as
\[
 \ProjSE[\uSEn](\tSE_i) = \uSEn(\tSE_i) = \ProjRZ[\uSEn](\tSE_i),
\quad i = -N,\,-N+1,\,\ldots,\,N,
\]
we have $\vSEn(\tSE_i)=\vRZn(\tSE_i)$.
However, we note that $\ProjSE$ and $\ProjRZ$ is not
generally equivalent. This can be observed
by the limits $t\to a$ and $t\to b$ as
\begin{align*}
 \lim_{t\to a}\ProjSE[f](t) = f(\tSE_{-N})
&\neq \beta_N =\lim_{t\to a}\ProjRZ[f](t),\\
 \lim_{t\to b}\ProjSE[f](t) = f(\tSE_{N})
&\neq \gamma_N =\lim_{t\to b}\ProjRZ[f](t).
\end{align*}
Thus, we obtain the claim of~\cref{thm:equivalence}.

\subsection{Proof of~\cref{thm:SE-Sinc-collocation}}
\label{sec:proof-SE-Sinc}

The invertibility of $(I_n - \kSEn)$
is already shown by~\cref{thm:SE-Sinc-Nystroem}.
Thus, we concentrate on the analysis of the error of $\vSEn$.
Because $\vSEn = \ProjSE \uSEn$, it holds that
\[
 u - \vSEn = u - \ProjSE\uSEn
= (u - \ProjSE u) + \ProjSE(u - \uSEn),
\]
which leads to
\begin{equation}
\label{eq:SE-Sinc-first-error}
 \|u - \vSEn\|_{C([a,b])}
\leq \|u - \ProjSE u\|_{C([a,b])}
 + \|\ProjSE\|_{\mathcal{L}(C([a,b]),C([a,b]))}
\|u - \uSEn\|_{C([a,b])}.
\end{equation}
For the first term, we show $u\in\MC_{\alpha}(\SEt(\domD_d))$,
from which we can use~\cref{thm:SE-Sinc-general}.
For the purpose, the following theorem is useful.

\begin{theorem}[Okayama et al.~{\cite[Theorem~3.2]{okayama13:_theo}}]
\label{thm:Sinc-Nyst-regularity}
Let $\domD=\SEt(\domD_d)$ or $\domD=\DEt(\domD_d)$.
Assume that $g$, $k(z,\cdot)$ and $k(\cdot,w)$ belong to $\Hinf(\domD)$
for all $z,\, w\in\domD$.
Then,~\cref{eq:Volterra-int} has a unique solution $u\in\Hinf(\domD)$.
\end{theorem}

Using this theorem, we can show the following result.

\begin{theorem}
\label{thm:Sinc-collocation-regularity}
Let $\alpha$ be a positive constant with $\alpha\leq 1$.
Assume that all the assumptions of~\cref{thm:Sinc-Nyst-regularity}
are fulfilled. Furthermore,
assume that $g$ and $k(\cdot, w)$ belong to $\MC_{\alpha}(\domD)$
for all $w\in\domD$.
Then,~\cref{eq:Volterra-int} has a unique solution $u\in\MC_{\alpha}(\domD)$.
\end{theorem}
\begin{proof}
According to~\cref{thm:Sinc-Nyst-regularity},~\cref{eq:Volterra-int}
has a unique solution $u\in\Hinf(\domD)$.
Therefore, we only have to show the H\"{o}lder continuity
of $u$ at the endpoints.
Using $u = g + \Vol u$, we have
\begin{align*}
&|u(b) - u(z)|\\
&=\left| \left(g(b) + \int_a^b k(b,w)u(w)\diff w\right)
- \left(g(z) + \int_a^z k(z,w)u(w)\diff w\right)\right|\\
&\leq \left|g(b) - g(z)\right|
+ \left|\int_z^b k(b,w)u(w)\diff w\right|
+ \left|\int_a^z \left\{k(b,w)- k(z,w)\right\}u(w)\diff w\right|.
\end{align*}
From the H\"{o}lder continuity of $g$,
the first term can be bounded by
$L_g|b - z|^{\alpha}$ for some constant $L_g$.
From the boundedness of $k$ and $u$,
the second term can be bounded by
$L_{k,u}|b - z|$ for some constant $L_{k,u}$.
Furthermore, from the boundedness of $\domD$ and $\alpha\leq 1$,
we have
$|b - z|=|b - z|^{1-\alpha}|b - z|^{\alpha}\leq L_{\domD}|b - z|^{\alpha}$
for some constant $L_{\domD}$.
From the H\"{o}lder continuity of $k$ and
boundedness of $u$,
the third term can be bounded by
$\tilde{L}_{k,u}|b - z|^{\alpha}|z - a|$ for some constant $\tilde{L}_{k,u}$.
Furthermore, from the boundedness of $\domD$,
we have $|z - a|\leq \tilde{L}_{\domD}$ for some constant $\tilde{L}_{\domD}$.
Thus, there exists a constant $L$ such that
$|u(b) - u(z)|\leq L|b - z|^{\alpha}$,
which shows the H\"{o}lder continuity of $u$ at $z = b$.
The proof for the H\"{o}lder continuity at $z = a$ is omitted
because it follows the same method as that at $z = b$.
This completes the proof.
\end{proof}

From this theorem, we can use~\cref{thm:SE-Sinc-general}
for estimating the first term of~\cref{eq:SE-Sinc-first-error} as
\[
 \|u - \ProjSE u\|_{C([a,b])}
\leq C_1 \sqrt{N} \rme^{-\sqrt{\pi d \alpha N}}
\]
for some constant $C_1$.
For the second term, we use the following bound for
the operator $\ProjSE$.

\begin{lemma}[Okayama~{\cite[Lemma~7.2]{okayama23:_theo}}]
Let $\ProjSE$ be defined by~\eqref{eq:ProjSE}.
Then, there exists a constant $C_2$ independent of $N$ such that
\[
 \|\ProjSE\|_{\mathcal{L}(C([a,b]),C([a,b]))}
\leq C_2 \log(N+1).
\]
\end{lemma}

The remaining term to be estimated in~\cref{eq:SE-Sinc-first-error}
is $\|u - \uSEn\|_{C([a,b])}$.
According to~\cref{lem:SE-Sinc-Nystroem},
it is estimated as~\cref{eq:SE-Sinc-Nystroem-error}.
Because $u\in\Hinf(\SEt(\domD_d))$, $u$ satisfies the assumptions
of~\cref{thm:SE-Sinc-indefinite},
from which we have
\[
 \|\Vol u - \VolSEn u\|_{C([a,b])}
\leq C_3 \rme^{-\sqrt{\pi d \alpha N}}.
\]
Thus, there exists a constant $C_4$ such that
\begin{align*}
 \|u - \vSEn\|_{C([a,b])}
&\leq C_1 \sqrt{N} \rme^{-\sqrt{\pi d \alpha N}}
+ C_2 \log(N+1) C_3 \rme^{-\sqrt{\pi d \alpha N}}\\
&\leq C_4 \sqrt{N}\rme^{-\sqrt{\pi d \alpha N}}.
\end{align*}
This completes the proof of~\cref{thm:SE-Sinc-collocation}.

\subsection{Proof of~\cref{thm:RZ-Sinc-collocation}}
\label{sec:proof-RZ-Sinc}

For~\cref{thm:RZ-Sinc-collocation},
the invertibility of $(E_n^{\textRZ} - V_{n}^{\textRZ})$
can be shown by combining~\cref{lem:SE-Sinc-Nystroem,lem:RZ-solution}
and~\cref{prop:RZ-solution}.
Thus, we concentrate on the analysis of the error of $\vRZn$.
By the triangle inequality, we have
\begin{align*}
 \|u(t) - \vRZn(t)\|_{C([a,b])}
&\leq \|u(t) - \vSEn(t)\|_{C([a,b])}
 + \|\vSEn(t) - \vRZn(t)\|_{C([a,b])}\\
&= \|u(t) - \vSEn(t)\|_{C([a,b])}
 + \|\ProjSE\uSEn(t) - \ProjRZ\uSEn(t)\|_{C([a,b])}.
\end{align*}
Because the first term
% $\|u(t) - \vSEn(t)\|_{C([a,b])}$
is already estimated by~\cref{thm:SE-Sinc-collocation},
we estimate the second term.
For the purpose, the following lemma is essential.

\begin{lemma}
Let $\ProjSE: C([a,b])\to C([a,b])$ and
$\ProjRZ: C([a,b])\to C([a,b])$ be defined
by~\cref{eq:ProjSE} and~\cref{eq:ProjRZ}, respectively.
Then, there exists a constant $C$ independent of $N$ such that
\[
 \left\|\ProjSE - \ProjRZ\right\|_{\mathcal{L}(C([a,b]),C([a,b]))}
\leq \frac{C}{\rme^{Nh} - 1}\log(N+1).
\]
\end{lemma}
\begin{proof}
First, it holds for $f\in C([a,b])$ that
\begin{align*}
& \ProjSE[f](t) - \ProjRZ[f](t)\\
&= - \sum_{j=-N}^N\left\{
\left(f(\tSE_{-N})-\beta_N\right)\omega_a(\tSE_j) +
\left(f(\tSE_{N})-\gamma_N\right)\omega_b(\tSE_j)
\right\}S(j,h)(\SEtInv(t))\\
&\quad+\left(f(\tSE_{-N}) - \beta_N\right)\omega_a(t)
+\left(f(\tSE_N) - \gamma_N\right)\omega_b(t).
\end{align*}
Here, noting
\begin{align*}
 |f(\tSE_{-N}) - \beta_N|
&=\frac{|f(\tSE_N) - f(\tSE_{-N})|}{\rme^{Nh} - 1}
\leq \frac{2\|f\|_{C([a,b])}}{\rme^{Nh} - 1},\\
 |f(\tSE_{N}) - \gamma_N|
&=\frac{|f(\tSE_{-N}) - f(\tSE_{N})|}{\rme^{Nh} - 1}
\leq \frac{2\|f\|_{C([a,b])}}{\rme^{Nh} - 1},
\end{align*}
and using $\omega_a(t) + \omega_b(t) = 1$, we have
\begin{align*}
& \left\|\ProjSE - \ProjRZ\right\|_{\mathcal{L}(C([a,b]),C([a,b]))}\\
&\leq \frac{2}{\rme^{Nh} - 1}
\left\{\sum_{j=-N}^N\left(\omega_a(\tSE_j) + \omega_b(\tSE_j)\right) |S(j,h)(\SEtInv(t))|
+\omega_a(t) + \omega_b(t)
\right\}\\
&= \frac{2}{\rme^{Nh} - 1}
\left\{\sum_{j=-N}^N |S(j,h)(\SEtInv(t))| + 1\right\}\\
&\leq \frac{2}{\rme^{Nh} - 1}
\left\{\frac{2}{\pi}(3 + \log N) + 1\right\},
\end{align*}
where the standard bound~\cite[Problem 3.1.5 (a)]{stenger93:_numer}
is used for the last inequality.
Thus, the claim follows.
\end{proof}

From this lemma, substituting~\cref{eq:h-SE} into $h$,
we estimate the second term as
\[
 \|(\ProjSE - \ProjRZ) \uSEn\|_{C([a,b])}
\leq \frac{C}{1 - \rme^{-\sqrt{\pi d / \alpha}}}
\log(N+1)\rme^{-\sqrt{\pi d N/\alpha}}\left\|\uSEn\right\|_{C([a,b])}.
\]
Noting $\alpha\in (0, 1]$, we obtain
$\rme^{-\sqrt{\pi d N/\alpha}}\leq \rme^{-\sqrt{\pi d \alpha N}}$.
Furthermore, $\log(N+1)\leq \sqrt{N}$ holds.
Hence, the proof is completed if
$\left\|\uSEn\right\|_{C([a,b])}$ is uniformly bounded
with respect to $N$.
This is shown by the following estimate
\[
 \left\|\uSEn\right\|_{C([a,b])}
\leq \left\|u - \uSEn\right\|_{C([a,b])}
+\left\| u\right\|_{C([a,b])}.
\]
From~\cref{eq:SE-Sinc-Nystroem-error}
and~\cref{thm:SE-Sinc-indefinite},
we see that $\left\|u - \uSEn\right\|_{C([a,b])}$
converges to $0$ as $N\to\infty$,
and accordingly it is uniformly bounded.
We also see that $\left\| u\right\|_{C([a,b])}$
is bounded because $u$ is continuous on $[a, b]$
from the assumption (see~\cref{thm:Sinc-collocation-regularity}).
This completes the proof of~\cref{thm:RZ-Sinc-collocation}.


\section{Proofs for the theorem presented in~\cref{sec:de-collocation}}
\label{sec:proof-DE}

In this section, we provide proofs for~\cref{thm:DE-Sinc-collocation}.

\subsection{Existence and uniqueness of the approximated equations}

In addition to the given equation $(\Ident - \Vol)u = g$,
let us consider the following two equations:
\begin{align}
 (\Ident - \VolDEn)    \uDEn &= g, \label{eq:DE-Sinc-Nystroem-symbol} \\
 (\Ident - \ProjDE \VolDEn)v &= \ProjDE g,
 \label{eq:DE-Sinc-collocation-symbol}
% (E_m^{\textRZ} - V_{m}^{\textRZ})\mathbd{c}_m &= \mathbd{g}_m^{\textDE},
%\label{eq:RZ-linear-eq-in-proof}
\end{align}
where $\VolDEn$ is defined by
\[
%\label{eq:VolDEn}
 \VolDEn [f](t)
=\sum_{j=-N}^N k(t,\tDE_j)f(\tDE_j)\DEtDiv(jh)J(j,h)(\DEtInv(t)),
\]
and $\ProjDE$ are defined by~\cref{eq:ProjDE}.
Because~\cref{eq:DE-Sinc-Nystroem-symbol} is equivalent
to~\cref{eq:DE-Sinc-Nystroem},
the solution of~\cref{eq:DE-Sinc-Nystroem-symbol}
is the approximate solution of the DE-Sinc-Nystr\"{o}m method.
On~\cref{eq:DE-Sinc-Nystroem-symbol}, the following result was obtained.

\begin{lemma}[Okayama et al.~{\cite[Lemma~6.10]{okayama13:_theo}}]
\label{lem:DE-Sinc-Nystroem}
Assume that
all the assumptions of~\cref{thm:DE-Sinc-Nystroem}
are fulfilled.
Then, there exists a positive integer $N_0$ such that for all $N\geq N_0$,
\cref{eq:DE-Sinc-Nystroem-symbol}
has a unique solution $\uDEn\in C([a,b])$.
Furthermore, there exists a constant $C$ independent of $N$ such that
for all $N\geq N_0$,
\begin{equation}
\label{eq:DE-Sinc-Nystroem-error}
 \|u - \uDEn\|_{C([a,b])}\leq C \|\Vol u - \VolDEn u\|_{C([a,b])}.
\end{equation}
\end{lemma}

On~\cref{eq:DE-Sinc-collocation-symbol}, we can show the following lemma
in the same manner as~\cref{lem:Stenger-solution}
(hence, the proof is omitted).

\begin{lemma}
The following two statements are equivalent:
\begin{enumerate}
 \item[{\rm (A)}] \Cref{eq:DE-Sinc-Nystroem-symbol}
has a unique solution $\uDEn\in C([a,b])$.
 \item[{\rm (B)}] \Cref{eq:DE-Sinc-collocation-symbol} has
 a unique solution $v\in C([a,b])$.
\end{enumerate}
Furthermore, $v=\vDEn$ holds.
\end{lemma}

On the basis of the results,
we proceed to analyze the error of $\vDEn$ next.

\subsection{Proof of~\cref{thm:DE-Sinc-collocation}}

In the same manner as~\cref{eq:SE-Sinc-first-error},
we have
\begin{equation}
\label{eq:DE-Sinc-first-error}
 \|u - \vDEn\|_{C([a,b])}
\leq \|u - \ProjDE u\|_{C([a,b])}
 + \|\ProjDE\|_{\mathcal{L}(C([a,b]),C([a,b]))}
\|u - \uDEn\|_{C([a,b])}.
\end{equation}
For the first term,
from~\cref{thm:Sinc-collocation-regularity},
we can use~\cref{thm:DE-Sinc-general} as
\[
 \|u - \ProjDE u\|_{C([a,b])}
\leq C_1 \rme^{-\pi d N/\log(2 d N/\alpha)}
\]
for some constant $C_1$.
For the second term, we use the following bound for
the operator $\ProjDE$.

\begin{lemma}[Okayama~{\cite[Lemma~7.5]{okayama23:_theo}}]
Let $\ProjDE$ be defined by~\eqref{eq:ProjDE}.
Then, there exists a constant $C_2$ independent of $N$ such that
\[
 \|\ProjDE\|_{\mathcal{L}(C([a,b]),C([a,b]))}
\leq C_2 \log(N+1).
\]
\end{lemma}

The remaining term to be estimated in~\cref{eq:DE-Sinc-first-error}
is $\|u - \uDEn\|_{C([a,b])}$.
According to~\cref{lem:DE-Sinc-Nystroem},
it is estimated as~\cref{eq:DE-Sinc-Nystroem-error}.
Because $u\in\Hinf(\DEt(\domD_d))$, $u$ satisfies the assumptions
of~\cref{thm:DE-Sinc-indefinite},
from which we have
\[
 \|\Vol u - \VolDEn u\|_{C([a,b])}
\leq C_3 \frac{\log(2 d N/\alpha)}{N}\rme^{-\pi d N/\log(2 d N/\alpha)}.
\]
Thus, there exists a constant $C_4$ such that
\begin{align*}
& \|u - \vDEn\|_{C([a,b])}\\
&\leq C_1 \rme^{-\pi d N/\log(2 d N/\alpha)}
+ C_2 \log(N+1) C_3 \frac{\log(2 d N/\alpha)}{N}
\rme^{-\pi d N/\log(2 d N/\alpha)}\\
&\leq C_4 \rme^{-\pi d N/\log(2 d N/\alpha)}.
\end{align*}
This completes the proof of~\cref{thm:DE-Sinc-collocation}.


%\section{Conclusion}
In this work, we propose a simple yet effective approach, called SMILE, for graph few-shot learning with fewer tasks. Specifically, we introduce a novel dual-level mixup strategy, including within-task and across-task mixup, for enriching the diversity of nodes within each task and the diversity of tasks. Also, we incorporate the degree-based prior information to learn expressive node embeddings. Theoretically, we prove that SMILE effectively enhances the model's generalization performance. Empirically, we conduct extensive experiments on multiple benchmarks and the results suggest that SMILE significantly outperforms other baselines, including both in-domain and cross-domain few-shot settings.

%\section*{Acknowledgments}
%This work was partially supported by the
%JSPS Grant-in-Aid for Scientific Research (C)
%Number JP23K03218.

\bibliographystyle{siamplain}
\bibliography{SincCollocationVolterra2nd}
\end{document}
