% SIAM Article Template
\documentclass[onefignum,onetabnum]{siamart220329}

% Information that is shared between the article and the supplement
% (title and author information, macros, packages, etc.) goes into
% ex_shared.tex. If there is no supplement, this file can be included
% directly.

% SIAM Shared Information Template
% This is information that is shared between the main document and any
% supplement. If no supplement is required, then this information can
% be included directly in the main document.


% Packages and macros go here

\usepackage{graphicx} % Required for inserting images
\graphicspath{{./img/}}		
% \usepackage[hidelinks]{hyperref}
\usepackage{xcolor}
\usepackage{lipsum}
\usepackage{amssymb}
\usepackage{amsmath}
% \usepackage{amsthm}
\usepackage{mathrsfs}
\usepackage{mathtools}
\usepackage{dsfont}
\usepackage{caption}
\usepackage{subcaption}
\usepackage{tcolorbox}
\usepackage{geometry}

% We can use these commands to comment; even with the same color, the initials will differentiate our comments
\newcommand{\ad}[1]{\textcolor{blue}{[AD: #1]}}
\newcommand{\sbr}[1]{\textcolor{red}{[SB: #1]}}
\newcommand{\ka}[1]{\textcolor{orange}{[KA: #1]}}

\newcommand{\bm}{\boldsymbol}

\newcommand{\R}{\mathbb{R}}
\newcommand{\C}{\mathbb{C}}
\newcommand{\N}{\mathbb{N}}



\newtheorem{defn}{Definition}
% \newtheorem{theorem}{Theorem}
\newtheorem{thm}{Theorem}
\newtheorem{lem}{Lemma}
\newtheorem{rmk}{Remark}
\newtheorem{ass}{Assumption}
\DeclarePairedDelimiter{\abs}{\lvert}{\rvert} 
\DeclarePairedDelimiter{\norm}{\lVert}{\rVert}
\DeclarePairedDelimiter{\inner}{\langle}{\rangle} 
\newtheorem{cor}{Corollary}


\ifpdf
  \DeclareGraphicsExtensions{.eps,.pdf,.png,.jpg}
\else
  \DeclareGraphicsExtensions{.eps}
\fi

% Prevent itemized lists from running into the left margin inside theorems and proofs
\usepackage{enumitem}
\setlist[enumerate]{leftmargin=.5in}
\setlist[itemize]{leftmargin=.5in}

% Add a serial/Oxford comma by default.
\newcommand{\creflastconjunction}{, and~}

% Used for creating new theorem and remark environments
\newsiamremark{remark}{Remark}
\newsiamremark{hypothesis}{Hypothesis}
\crefname{hypothesis}{Hypothesis}{Hypotheses}
\newsiamthm{claim}{Claim}

% Sets running headers as well as PDF title and authors
\headers{Surrogate models for diffusion on graphs via sparse polynomials}{G. A. D'Inverno, K. Ajavon, and S. Brugiapaglia}

% Title. If the supplement option is on, then "Supplementary Material"
% is automatically inserted before the title.
\title{Surrogate models for diffusion on graphs via sparse polynomials\thanks{Submitted to the editors 10/02/2025.
\funding{GAD acknowledges the support provided by the European Union - NextGenerationEU, in the framework of the iNEST - Interconnected Nord-Est Innovation Ecosystem (iNEST ECS00000043 – CUP G93C22000610007) project. The views and opinions expressed are solely those of the authors and do not necessarily reflect those of the European Union, nor can the European Union be held responsible for them. \\ SB acknowledges the support of the Natural Sciences and Engineering Research Council of Canada (NSERC) through grant RGPIN-2020-06766 and the Fonds de Recherche du Qu\'ebec Nature et Technologies (FRQNT) through grant 313276. \\ GAD acknowledges the support of INdAM Gruppo Nazionale of Calcolo Scientifico (GNCS). }}}

% Authors: full names plus addresses.
\author{Giuseppe Alessio D'Inverno\thanks{International School for Advanced Studies, Trieste, Italy
  (\email{gdinvern@sissa.it}, \url{https://aledinve.github.io}).}
\and Kylian Ajavon\footnotemark[3] \and Simone Brugiapaglia\thanks{Department of Mathematics and Statistics, Concordia University, Montréal, Canada
(\email{kylian.yoan@gmail.com}; \email{simone.brugiapaglia@concordia.ca}, \url{http://simonebrugiapaglia.ca}).}}

\usepackage{amsopn}
\DeclareMathOperator{\diag}{diag}


%%% Local Variables: 
%%% mode:latex
%%% TeX-master: "ex_article"
%%% End: 


% Optional PDF information
\ifpdf
\hypersetup{
  pdftitle={On the relationship between two Sinc-collocation methods for
 Volterra integral equations of the second kind and their further improvement},
  pdfauthor={T. Okayama}
}
\fi

% The next statement enables references to information in the
% supplement. See the xr-hyperref package for details.

\externaldocument{ex_supplement}

% FundRef data to be entered by SIAM
%<funding-group specific-use="FundRef">
%<award-group>
%<funding-source>
%<named-content content-type="funder-name"> 
%</named-content> 
%<named-content content-type="funder-identifier"> 
%</named-content>
%</funding-source>
%<award-id> </award-id>
%</award-group>
%</funding-group>

\begin{document}

\maketitle

% REQUIRED
% limit: 250 words
\begin{abstract}
Two different Sinc-collocation methods
for Volterra integral equations of the second kind
have been independently proposed
by Stenger and Rashidinia--Zarebnia.
However, their relationship remains unexplored.
This study theoretically examines the solutions of these two methods,
and reveals that they are not generally equivalent,
despite coinciding at the collocation points.
Strictly speaking, Stenger's method
assumes that the kernel of the integral
is a function of a single variable,
but this study theoretically justifies the use of
his method in general cases, i.e., the kernel is
a function of two variables.
%that are derived from initial value problems,
%where the kernel of the integral
%is assumed to be a function of single variable.
%The assumption on the kernel seems essential
%because it is fully used in his theoretical discussion of the method.
%Contrary to the idea, this paper theoretically justifies the use of
%his method in general cases, i.e., the kernel is
%a function of two variables.
Then, this study rigorously proves that
both methods can attain the same, root-exponential convergence.
%which corresponds with the numerical observations in the previous study.
In addition to the contribution,
this study improves Stenger's method to attain significantly higher,
almost exponential convergence.
Numerical examples supporting the theoretical results are also provided.
%In this paper, two improved versions of an existing Sinc-collocation method
%for Volterra integral equations of the second kind are presented.
%The first method is close to the reformulation of the existing method
%in order to make it more practical and to reinforce it in theoretical way.
%Then it is rigorously proved that
%the convergence rate of the modified method is $\Order(\exp(-c\sqrt{N}))$,
%which corresponds with the numerical observations in the previous study.
%In the second method,
%the variable transformation employed in the first method,
%which is called the ``tanh transformation,''
%is replaced with the ``double exponential transformation.''
%%instead of the original ``tanh transformation.''
%This replacement improves the convergence rate to
%$\Order(\exp(-c N/\log N))$, which is proved
%in the same way as the first method.
%Numerical examples which support the theoretical results are also given.
\end{abstract}

% REQUIRED
\begin{keywords}
Sinc numerical method, tanh transformation, double-exponential transformation
\end{keywords}

% REQUIRED
\begin{AMS}
  65R20
\end{AMS}

\documentclass[../main.tex]{subfiles}
\graphicspath{{../images/}}
\makeatletter
\def\input@path{{../images/}}
\makeatother
\begin{document}
\section{Introduction}
\begin{figure}
\centering
\begin{tikzpicture}
\node[inner sep=0pt] (ws) at (0, 0) {
\includegraphics[height=.4\textwidth, trim={10cm 0 10cm 0},clip]{world_space.png}};
\node[inner sep=0pt] (cs) at (6,0) {\includegraphics[height=.4\textwidth, trim={10cm 1cm 10cm 4cm},clip]{conf_space.png}};
\end{tikzpicture}
\vspace{-5pt}
\label{fig:pbrm_intro}
\caption{\textbf{Left}: Shows world space obstacles as grey spheres. Robots start and goal configuration is colored red and green, respectively. Configurations along the computed path are colored transparent blue. \textbf{Right:} Mapped world space scenario to configuration space. Obstacle region is the grey mesh. Red spheres are collision-free regions computed by the neural SCDF. The optimized shortest path in the convex corridor is the blue curve.}
\vspace{-25pt}
\end{figure}
Motion planning is the problem of finding a collision-free trajectory that connects a given start and goal configuration. The planning takes place in the configuration space of the robot. For single body robots, like mobile robots or drones, the configuration space and the world space are usually the same. This simplifies the planning, since explicit obstacle representations are available which enables geometrical tools like separating hyperplanes, smallest distance to obstacles etc., to be used when designing motion planning algorithms. For multi-body robots like manipulators, the situation is completely different. The world space obstacles are usually mapped to non-convex regions, and to make the problem even harder, the mapping is usually not known. Forming explicit representations of the obstacle region in the configuration space is usually too expensive or intractable. Despite all of this, sampling based planners are used with great success, which mainly is due to their use of implicit representations of the obstacle region. The basic idea is to construct a graph in the configuration space that covers and connects the collision-free region. From this graph, a path can be extracted that connects a given start and goal configuration. The approach is computationally expensive, since the graph is constructed with the smallest geometrical building block available, points, which represents a collision-check. Furthermore, the extracted paths from the graph are non-smooth and jagged due to the stochastic nature of the approach. This adds an additional post-processing step to the process, where the paths are shortcutted and smoothened, before the path can be used for tracking. Clearly a lot of time is invested to form this graph and produce smooth paths. Thus, if the obstacles start to move, then all of this work is done in no use, since all points that make up this graph need to be re-verified, which is simply too time consuming to be done in real time.
\\\\
In this work, we want to address the existing drawbacks of the sampling based planners. Our main contribution is an improved motion planner where each vertex in the graph covers a collision-free region in the form of a sphere instead of a point and where the edges are formed with neighboring intersecting spheres. This representation has the advantage of instead of returning piecewise linear paths, returning a sequence of overlapping spheres, i.e. a convex corridor, that connects a given start and goal configuration, illustrated in Figure \ref{fig:pbrm_intro}. This convex corridor allows us to use convex optimization to produce smooth trajectories, instead of computationally expensive post-processing methods. The representation further allows us to estimate the coverage of the collision-free space, which gives us awareness and feedback in the offline roadmap construction phase. Finally, our representation is simple to adapt to moving obstacles, simply requery for the new radii and recheck for intersections. 
\\\\
The spherical collision-free regions are formed using a signed distance function (SDF), which is a function that returns the smallest distance from an arbitrary point to the boundary of an obstacle. As the name implies, the distance is signed, thus if the point is inside the obstacle it is negative otherwise positive. If the distance is positive, a sphere with radius equal to the distance is guaranteed to cover a collision-free region. Using an SDF in motion planning is not new, but what is novel about our approach is that we express the distance in the configuration space instead of the world space and by doing so allows us to form these convex collision-free regions. We refer to the resulting SDF as a signed configuration distance function (SCDF). Computing an SCDF analytically is non-trivial, our approach is therefore to parameterize the SCDF with a deep neural network and learn the mapping by supervised learning. Our resulting neural SCDF can compute distances for different parameter values of obstacle shapes and we also show how multiple distances can be combined, thus making our approach flexible.
\section{Related work}
Motion planning algorithms can roughly be divided into three families, grid-based, sampling based and optimization based methods. Grid-based methods (GBM) discretize the planning space from which a graph is then compiled. A standard search method is A$^\star$ \citep{a_star}, which is classified as an \textit{informed} search method, since it employs a heuristic function to speed up the search. A$^\star$ guarantees to return an optimal path at the level of discretization used. GBMs usually discretize the planning space by a regular lattice and this limits the GBMs to problems with low dimensionality due to the curse of dimensionality. Thus, GBMs are usually limited to single-body robots where the degrees of freedom (DOF) are low. To overcome the inherent scaling problem with the GBMs, stochastic methods are usually used for multi-body robots. These methods are termed as sampling-based methods (SBM) and core members within this family are the rapidly-exploring random trees (RRT) \citep{rrt} and the probabilistic roadmap (PRM) \citep{prm}. RRT grows a tree from the start configuration and explores the collision-free region in a rapid way until it is able to connect to the goal region. RRT is usually improved by bi-directional planning \citep{rrt_connect}, i.e. an additional tree is grown from the goal configuration and the trees are tested for connection after any tree has been expanded. RRT is a single-query method, thus it searches for a path from scratch each time it is queried. Contrary to this, PRM is a multi-query method, which solves for multiple queries without starting from scratch. PRM does this by creating a roadmap (graph) that covers the collision-free space as an offline step. The graph is then used to solve for multiple queries. PRMs are used in cases where the environment does not change since the extra offline step is too computationally costly and needs to be re-done if the environment is changed. In our work, we address this inherent issue by using a different roadmap representation. Our vertices in the graph cover a collision-free region in the form of spheres and we form the edges by checking for intersecting spheres. If something in the environment changes, we recompute the spheres radii and recheck the intersections, without relying on collision detection. We use a trained neural network to compute the sphere radius, therefore querying for the radius can be done fast, hence our representation enables the PRM for dynamic environments.
\\\\
In the recent decades, optimization based methods (OBM) \citep{chomp, schulman, itomp, stomp} have been introduced as an alternative to SBM for multi-body robots. Like the SBM, the OBMs scale well to higher dimensional problems and produce smoother motion. It is common to use a SDF in the optimization since it is a smooth function, thus enabling gradient-based methods. However, the standard way of expressing the SDF is in world space. The distance therefore needs to be mapped to the configuration space by the forward kinematics. This mapping makes the optimization problem a non-linear program (NLP), which is computationally expensive to solve. Recently, a different approach has been proposed. In \cite{mp_gcs} motion planning is formulated as a convex optimization problem by using the graph of convex sets framework \citep{gcs}. The underlying idea is to decompose the collision-free space into intersecting convex sets from which a convex optimization problem is formulated. In cases where an explicit representation of the obstacles in the configuration space exists, like for single-body robots, creating collision-free convex regions can be done fast \citep{iris}. For multi-body robots, this is non-trivial. Existing work does this successfully \citep{iris_nlp, iris_c} by an optimization based approach, but the methods are still too time consuming to be used in the presence of moving obstacles. Our approach is instead to use deep learning to learn an SDF expressed in the configuration space. With this, we can query for shortest distances to the collision boundary, which allows us to expand spherical regions which are collision-free. Our approach is fast and therefore enables our suggested roadmap planner to be used in dynamic environments.
\\\\
Recent research has focused on learning collision detection \citep{fk_kernel_distance, diffco, graphdistnet} by predicting the signed distance between the robot links and the surrounding obstacles in the world space. The learned SDF is used in trajectory optimization but since the distance is expressed in the world space, the problem becomes an NLP and therefore takes a long time to solve. We take a novel approach and suggest to instead express the signed distance in the configuration space. This allows us to improve the PRM at the same time as it enables convex optimization for trajectory optimization, which runs faster and is more reliable than NLP solvers. In \cite{cspf} a learned signed distance function in the configuration space is proposed similar to our approach. However, their approach is restricted to point cloud representations, while we propose to represent the obstacles as parameterized geometric shapes, e.g. spheres. Furthermore, we also show how to use our learned SCDF to improve an existing roadmap planner.
\section{Problem formulation}
A robot is located in the world space, $\W \subset \R^3 $. The unique location of the robot is given by its configuration $\q \in \C$, where $\C$ is the configuration space. The set of points covered by the robots bodies at a certain configuration is expressed as $\B(\q) \subset \W$. The robot is surrounded by $\NrObst$ obstacles $\O = \bigcup_{i=1}^{\NrObst} \O_i$, where  $\O_i \subset \W$. The representation of the obstacle in the configuration space is the set $\C\O_i = \{\q \in \C \: |\: \B(\q) \cap \O_i \neq \emptyset \}$. The obstacle space is formed as $\Co = \bigcup_{i=1}^{\NrObst} \C \O_i$. The complement is referred to as the free space, $\Cf = \C \setminus \Co$. The path planning problem is a tuple, ($\Cf$, $\qStart$, $\qGoal$), where we want to connect a query pair, consisting of a start, $\qStart$, and goal configuration, $\qGoal$, with a geometric path, $\q(s): [0, 1] \mapsto \Cf$, such that $\q(0)=\qStart$ and $\q(1)=\qGoal$, or report correctly when such a path does not exist.
\end{document}


\section{Preliminaries}
\label{sec:preliminaries}
We first set up notations and mathematically formulate tasks.

\noindent\textbf{Language-Conditioned Imitation Learning (LC-IL)}. The task of LC-IL aims to train an agent to mimic expert behaviors from a given demonstration set $\mathcal{D}_d = \{(\mathbf{\tau}_i,l_i)\}_{i=1}^N$, where $l_i \in \mathcal{L} $ represents a task-specific language instruction. Each trajectory $\mathbf{\tau}_i\in\mathcal{T}$ consists of a sequence of state-action pairs $\mathbf{\tau}_i = \{(\mathbf{s}_j, \mathbf{a}_j)\}_{j=1}^T$ of the horizon length $T$. In robot manipulation tasks, action $\mathbf{a}_j\in\mathcal{A}$ corresponds to the control commands executed by the agent and state $\mathbf{s}_j = [\mathbf{p}_j; \mathbf{v}_j] \in\mathcal{S}$ records proprioceptive data $\mathbf{p}_j$ (\textit{e.g.,} joint positions, velocities) and visual inputs $\mathbf{o}_j\in\mathcal{O}$ (\textit{e.g.,} camera images) at the time step $j$. The objective of LC-IL is to find an optimal language-conditioned policy $\pi^*(\mathbf{a}|\mathbf{s},l): \mathcal{S}\times\mathcal{L}\mapsto\mathcal{A}$ via solving the supervised optimization as follows,
\begin{equation}\nonumber
    \pi^* \in \arg\min_{\pi} \mathbb{E}_{(\tau_i, l_i)\sim \mathcal{T}} \left[ \frac{1}{T} \sum_{(\mathbf{s}_j, \mathbf{a}_j) \sim \tau_i} \ell(\pi(\hat{\mathbf{a}}_j, \mathbf{s}_j|l_i),  \mathbf{a}_j)\right],
\end{equation}
where \(\ell(\cdot, \cdot)\) is a task-specific loss, such as mean squared error or cross-entropy. Training the policy \(\pi_\theta\) in an end-to-end fashion may require \textit{hundreds} of high-quality expert demonstrations to converge, primarily due to the high variance of visual inputs $\mathbf{o}$ and language instructions $l$.

% We study the problem of Language-Conditioned Imitation Learning ~\cite{rss21-gcil}, where the goal is to train an agent to perform tasks by conditioning its policy on both the state of the environment and language instruction. Formally, let \(\mathcal{O}\) be the observation space, \(\mathcal{A}\) the action space, and \(\mathcal{L}\) the language instruction space. The observation space \(\mathcal{O}\) typically includes visual or sensor data, such as images, that represent the partial observation of state \(\mathcal{S}\). The objective is to learn a policy \(\pi_\theta : \mathcal{O} \times \mathcal{L} \to \mathcal{A}\), parameterized by \(\theta\), that maps an observation \(o \in \mathcal{O}\) and a language instruction \(L \in \mathcal{L}\) to an action \(a \in \mathcal{A}\). We assume access to a dataset of expert demonstrations \(\mathcal{D}_{\operatorname{demo}} = \{(\{o_k^i, a_k^i\}_{i=1}^T, L_k)\}_{k=1}^N\), where each sample consists of a $T$-step observation-action trajectory and a corresponding language instruction \(L_k \in \mathcal{L}\). The goal is to train the policy \(\pi_\theta\) by minimizing the following loss function:
% \[
% \mathcal{L}(\theta) = \frac{1}{N} \sum_{k=1}^N \sum_{i=1}^T \ell(a_k^i, \pi_\theta(o_k^i, L_k)),
% \]
% where \(\ell(\cdot, \cdot)\) is a task-specific loss function, such as mean squared error or cross-entropy. 
\begin{table}
\centering
\caption{Comparison of different component designs in time contrast learning across mainstream vision-language pre-training. \vspace{1ex}
% The goal frame $o_g$ is typically set as the last frame $o_{T}$.
 }
\label{tab:comp}
\Large
\resizebox{\linewidth}{!}{ 
\begin{tabular}{llll}
\toprule
$\operatorname{Method}$      & \textcolor{black}{$\mathcal{P}(\mathcal{O}_{i})$}  & \textcolor{black}{$\mathcal{N}(\mathcal{O}_{i})$} & $\mathfrak{R}(\mathbf{v},\mathbf{l}_i)$  \\ \hline
$\operatorname{R3M}$         & $(o_0, o_{j>i})$      &  $(o_0,o_i,o_j^{\notin O_i})$   & $\operatorname{reward}(\mathbf{v},\mathbf{l}_i)$   \\    
$\operatorname{LIV}$         & $(o_T)$    &  $(o_T^{\notin O_i})$    & $\operatorname{cos}(\mathbf{v},\mathbf{l}_i)$  \\    
$\operatorname{DecisionNCE}$ & $(o_i,o_{j>i})$     &     $(o_i^{\notin O_i},o_{j>i}^{\notin O_i})$  & $\operatorname{cos}(\mathbf{v}_j-\mathbf{v}_i, \mathbf{l}_i)$  \\          
$\operatorname{AcTOL}$        & $(o_i,o_{j \in [T] \setminus \{i\}})$ & $(o_i,o_k: d_{i, k}>d_{i, j})$  & $-\Vert \operatorname{cos}(\mathbf{v}_i, \mathbf{l}_i)-\operatorname{cos}(\mathbf{v}_j, \mathbf{l}_i) \Vert_2 $     \\  \bottomrule                                                              
\end{tabular}
}
\end{table}

\paragraph{Vision-language Pre-training.}  Address such scalability issues can be achieved by leveraging large-scale, easily accessible human action video datasets $\mathcal{D}_p = \{(\mathcal{O}_i, l_i)\}_{i=1}^M$ \cite{corr18-epickitchen,cvpr22-ego4d}, where $\mathcal{O}_i=\{o_j\}_{j=1}^T$ represents a video clip with $T$ frames and $l_i$ the corresponding description. Pretraining on such datasets enables policies to rapidly learn visual-language correspondences with minimal expert demonstrations. Mainstream pretraining methods employ time contrastive learning \cite{icra18-tcn} to fine-tune a visual encoder $\mathcal{\phi}$ and a text encoder $\mathcal{\varphi}$, which project frames and descriptions into a shared $d$-dimensional embedding space, \textit{i.e.}, $\mathbf{v}_j = \phi(o_j)\in\mathbb{R}^d$ and $\mathbf{l}_i = \varphi(l_i)\in\mathbb{R}^d$. To provide a unified perspective on various pretraining approaches, we formulate them within the objective $\mathcal{L}_{\operatorname{tNCE}}(\phi, \varphi)$: \vspace{-2ex}
\begin{align}\nonumber\small
\mathcal{L}_{\operatorname{tNCE}}&=
-\mathbb{E}_{\substack{\scriptstyle o^+\sim\textcolor{black}{\mathcal{P}(\mathcal{O}_i)}}}
    \log  
    \frac{
        \exp(\mathfrak{R}(\mathbf{v}^+, \mathbf{l}_i))
    }{
        \mathbb{E}_{\scriptstyle o^- \sim \textcolor{black}{\mathcal{N}(\mathcal{O}_i)}}
        \exp(\mathfrak{R}(\mathbf{v}^-, \mathbf{l}_i))
    },
\end{align}

% \begin{align}\nonumber\small
% \mathcal{L}_{\operatorname{tNCE}}&=
% -\mathbb{E}_{\substack{\scriptstyle o\sim O_i \\ \scriptstyle o^+\sim\textcolor{black}{\mathcal{P}(o)}}}
%     \log  
%     \frac{
%         \exp(\mathfrak{R}(\mathbf{v}^+, \mathbf{v}, \mathbf{l}_i))
%     }{
%         \mathbb{E}_{\scriptstyle o^- \sim \textcolor{black}{\mathcal{N}(o)}}
%         \exp(\mathfrak{R}(\mathbf{v}, \mathbf{v}^-, \mathbf{l}_i))
%     },\vspace{-2ex}
% \end{align}
% where $\mathbf{v} = \phi(o)$, and 
where $\mathbf{v}^{+/-} = \phi(o^{+/-})$. Different pretraining strategies differ in their selection of (1) the positive frame set $\mathcal{P}(\mathcal{O}_i)$, (2) negative frame set $\mathcal{N}(\mathcal{O}_i)$; and (3) the semantic alignment scoring function $\mathfrak{R}(\mathbf{v}, \mathbf{l}_i)$ measuring the gap of VL similarities as detailed in Table \ref{tab:comp}. 

\noindent\textbf{Discussion.} As motivated by goal-conditioned RL \cite{nips17-her}, current approaches \textit{explicitly} select future frames (\textit{e.g.}, DecisionNCE) or the last frame (\textit{e.g.}, LIV) as the goal within the positive set, enforcing their visual embedding to align with the semantics. Likewise, the scoring functions $\mathfrak{R}$ are often designed to maximize this transition direction. However, the pretraining action videos are \textit{noisy} as actions may terminate early or include irrelevant subsequent actions, which may mislead the encoders and result in inaccurate vision-language association. As detecting precise action boundaries is non-trivial, we argue for a more flexible approach that leverages \textit{intrinsic} characteristics of actions to guide pretraining.



% we first pre-train a visual encoder \(\mathcal{\phi}: \mathcal{O} \to \mathbb{R}^d\) and a text encoder \(\mathcal{\varphi}: \mathcal{L} \to \mathbb{R}^d\) to learn mappings from the observation and the language instruction space to $d-$dimensional feature spaces. This pre-training can be done using large, less-expensive data without action annotation, such as human action videos . Then, with the frozen learned features \(\boldsymbol{v}\) and \(\boldsymbol{l}\) as input, we can only fine-tune a simple Multi-Layer Perceptron (MLP) with a few demonstrations to learn the map from the feature space \(\mathbb{R}^d \times \mathbb{R}^d\) to the action space \(\mathcal{A}\). Since both the observation space \(\mathcal{O}\) and the action space \(\mathcal{A}\) are continuous and ordered over time, we expect the representations learned through pre-training to also exhibit continuity and orderliness. This property in the representations allows for better learning of the continuous mapping between observations and actions. This property offers three significant benefits: First, the orderliness of the representation ensures that different states of the task, such as the start and end of an action, can be better captured and distinguished. Second, the continuity of the representation allows it to evolve smoothly as the task progresses, enabling the model to output stable actions based on the current state. Finally, we can demonstrate that even under small perturbations to the language instruction, these properties ensure the robustness of the learned representation. This robustness is crucial for maintaining performance in real-world scenarios where language instructions might contain minor ambiguities or variations.





% We consider a partially observable Markov Decision Process (POMDP) with language conditions, which models the interaction between an agent and an environment where observations are incomplete and actions are guided by natural language instructions. Formally, a POMDP is defined as a tuple $\langle \mathcal{S}, \mathcal{A}, \mathcal{O}, \mathcal{T}, \mathcal{R}, \mathcal{Z}, \gamma \rangle$, where $\mathcal{S}$ is the state space, $\mathcal{A}$ is the action space available to the agent. $\mathcal{O}$ is the observation space, which provides partial information about the environment. $\mathcal{T}(s' \mid s, a)$ is the state transition function. $\mathcal{R}(s, a)$ is the reward function. $\mathcal{Z}(o \mid s, a)$ is the observation function. $\gamma \in [0, 1)$ is the discount factor.

% To incorporate language instructions, we introduce a task description $L$, which specifies the agent's goal in natural language. The task description conditions the agent's policy $\pi(a \mid o, L)$, where $o$ is the agent's current observation. The agent aims to maximize the expected cumulative reward while adhering to the task described by $L$.

% Further, we assume the availability of a large-scale human action video dataset including $N$ video-instruction pairs, $\{(\{o_k^i\}_{i=1}^{t_k}, L_k)\}_{k=1}^N$, where each pair representing an action video with $t_k$ frames and its corresponding language description $L_k$. We pre-train the visual and language encoders on this dataset, with the visual features $\boldsymbol{v} = \operatorname{Enc}_v(o)$ and the language features $\boldsymbol{l} = \operatorname{Enc}_l(L)$. These pre-trained representations are then frozen and applied to train the policy $\pi$ in the aforementioned decision-making process, enabling the agent to better interpret and act upon language-conditioned tasks.

\section{Sinc-Nystr\"{o}m methods}
\label{sec:nystroem}

This section describes the Sinc-Nystr\"{o}m methods
developed by Muhammad et al.~\cite{muhammad05:_numer}.
The first method employs the tanh transformation~\cref{eq:SEt}
as a variable transformation,
while the second method employs the DE transformation~\cref{eq:DEt}.

\subsection{SE-Sinc-Nystr\"{o}m method}

%Assume that $k(t,\cdot)u(\cdot)$ satisfies
%the assumptions of~\cref{thm:SE-Sinc-indefinite}
%uniformly for $t\in [a,b]$.
By applying the SE-Sinc indefinite integration~\cref{eq:SE-Sinc-indefinite}
to the integral in the given equation~\cref{eq:Volterra-int},
we obtain an approximated equation as
\begin{equation}
\label{eq:SE-Sinc-Nystroem}
 \uSEn(t)
= g(t)
+\sum_{j=-N}^N k(t,\tSE_j)\uSEn(\tSE_j)\SEtDiv(jh) J(j,h)(\SEtInv(t)).
\end{equation}
%where $\tSE_j = \SEt(jh)$.
The approximated solution $\uSEn$ is determined
once the unknown coefficients $\uSEn(\tSE_j)$
on the right-hand side are obtained. To this end,
$2N+1$ sampling points are set at
$t=\tSE_i$ $(i=-N,\,-N+1,\,\ldots,\,N)$ in~\cref{eq:SE-Sinc-Nystroem}
as
\begin{equation}
\label{eq:SE-Sinc-Nystroem-system}
 \uSEn(\tSE_i)
= g(t)
+\sum_{j=-N}^N k(\tSE_i,\tSE_j)\uSEn(\tSE_j)\SEtDiv(jh)
 J(j,h)(ih),
\quad i=-N,\,\ldots,\,N,
\end{equation}
which is a system of linear equations.
This system is expressed in a matrix-vector form as follows.
Let $n=2N+1$,
let $I_n$ be an identity matrix of order $n$,
and let $\kSEn$ be $n\times n$ matrix
whose $(i, j)$-th element is
\[
 \left(\kSEn\right)_{ij}
= k(\tSE_i,\tSE_j)\SEtDiv(jh)h \delta_{i-j}^{(-1)},
%\left(\frac{1}{2}+ \sigma_{i-j}\right),
\quad i= -N,\,\ldots,\,N,\quad j=-N,\,\ldots,\,N,
\]
where $\delta_k^{(-1)} = (1/2) + \sigma_k$, where $\sigma_k$ is defined by
\[
\sigma_k = \int_0^k\frac{\sin(\pi x)}{\pi x}\diff x
=\frac{1}{\pi}\Si(\pi k).
\]
Furthermore, let $\gSEn$ and $\mathbd{u}_n^{\textSE}$
be $n$-dimensional vectors defined by
\begin{align*}
 \gSEn
 &= [g(\tSE_{-N}),\,g(\tSE_{-N+1}),\,\ldots,\,g(\tSE_N)]^{\mathrm{T}},\\
 \mathbd{u}_n^{\textSE}
 &= [\uSEn(\tSE_{-N}),\,\uSEn(\tSE_{-N+1}),\,\ldots,\,\uSEn(\tSE_N)]^{\mathrm{T}}.
\end{align*}
Then, the system~\cref{eq:SE-Sinc-Nystroem-system} is expressed as
\begin{equation}
\label{eq:SE-Sinc-Nystroem-linear-eq}
 (I_n - \kSEn) \mathbd{u}_n^{\textSE} = \gSEn.
\end{equation}
By solving~\cref{eq:SE-Sinc-Nystroem-linear-eq},
we obtain the unknown coefficients $\mathbd{u}_n^{\textSE}$,
from which the approximated solution $\uSEn$ is determined
by~\cref{eq:SE-Sinc-Nystroem}.
This procedure is called the SE-Sinc-Nystr\"{o}m method.
Its convergence theorem was provided as follows.

\begin{theorem}[Okayama et al.~{\cite[Theorem~3.4]{okayama13:_theo}}]
\label{thm:SE-Sinc-Nystroem}
Let $d$ be a positive constant with $d<\pi$.
Assume that $g$, $k(z,\cdot)$ and $k(\cdot,w)$
belong to $\Hinf(\SEt(\domD_d))$ for all $z$, $w\in\SEt(\domD_d)$.
Furthermore, assume that
$g$, $k(t,\cdot)$ and $k(\cdot,s)$ belong to $C([a,b])$
for all $t$, $s\in [a, b]$.
Let $h$ be selected by the formula~\cref{eq:h-SE} with $\alpha=1$.
Then, there exists a positive integer $N_0$
such that for all $N\geq N_0$,
the coefficient matrix $(I_n - \kSEn)$ is invertible.
Furthermore, there exists a constant $C$ independent of $N$
such that for all $N\geq N_0$,
\[
 \|u - \uSEn\|_{C([a,b])} \leq C \rme^{-\sqrt{\pi d N}}.
\]
\end{theorem}

\subsection{DE-Sinc-Nystr\"{o}m method}

Muhammad et al.~\cite{muhammad05:_numer} also considered
another method by replacing $\SEt$ with $\DEt$
in the SE-Sinc-Nystr\"{o}m method.
%Assume that $k(t,\cdot)u(\cdot)$ satisfies
%the assumptions of~\cref{thm:DE-Sinc-indefinite}
%uniformly for $t\in [a,b]$.
Applying the DE-Sinc indefinite integration~\cref{eq:DE-Sinc-indefinite}
to the integral in the given equation~\cref{eq:Volterra-int},
we obtain an approximated equation as
\begin{equation}
\label{eq:DE-Sinc-Nystroem}
 \uDEn(t)
= g(t)
+\sum_{j=-N}^N k(t,\tDE_j)\uDEn(\tDE_j)\DEtDiv(jh) J(j,h)(\DEtInv(t)).
\end{equation}
%where $\tDE_j = \DEt(jh)$.
The approximated solution $\uDEn$ is determined
once the unknown coefficients $\uDEn(\tDE_j)$
on the right-hand side are obtained. To this end,
$2N+1$ sampling points are set at
$t=\tDE_i$ $(i=-N,\,-N+1,\,\ldots,\,N)$ in~\cref{eq:DE-Sinc-Nystroem}.
This leads a system of linear equations
\begin{equation}
\label{eq:DE-Sinc-Nystroem-linear-eq}
 (I_n - \kDEn) \mathbd{u}_n^{\textDE} = \gDEn,
\end{equation}
where $\kDEn$ be $n\times n$ matrix
whose $(i, j)$-th element is
\[
 \left(\kDEn\right)_{ij}
= k(\tDE_i,\tDE_j)\DEtDiv(jh)h \delta_{i-j}^{(-1)},
%\left(\frac{1}{2}+ \sigma_{i-j}\right),
\quad i= -N,\,\ldots,\,N,\quad j=-N,\,\ldots,\,N,
\]
and $\gDEn$ and $\mathbd{u}_n^{\textDE}$
be $n$-dimensional vectors defined by
\begin{align*}
 \gDEn
 &= [g(\tDE_{-N}),\,g(\tDE_{-N+1}),\,\ldots,\,g(\tDE_N)]^{\mathrm{T}},\\
 \mathbd{u}_n^{\textDE}
 &= [\uDEn(\tDE_{-N}),\,\uDEn(\tDE_{-N+1}),\,\ldots,\,\uDEn(\tDE_N)]^{\mathrm{T}}.
\end{align*}
By solving~\cref{eq:DE-Sinc-Nystroem-linear-eq},
we obtain the unknown coefficients $\mathbd{u}_n^{\textDE}$,
from which the approximated solution $\uDEn$ is determined
by~\cref{eq:DE-Sinc-Nystroem}.
This procedure is called the DE-Sinc-Nystr\"{o}m method.
Its convergence theorem was provided as follows.

\begin{theorem}[Okayama et al.~{\cite[Theorem~3.5]{okayama13:_theo}}]
\label{thm:DE-Sinc-Nystroem}
Let $d$ be a positive constant with $d<\pi/2$.
Assume that $g$, $k(z,\cdot)$ and $k(\cdot,w)$
belong to $\Hinf(\DEt(\domD_d))$ for all $z$, $w\in\DEt(\domD_d)$.
Furthermore, assume that
$g$, $k(t,\cdot)$ and $k(\cdot,s)$ belong to $C([a,b])$
for all $t$, $s\in [a, b]$.
Let $h$ be selected by the formula~\cref{eq:h-DE} with $\alpha=1$.
Then, there exists a positive integer $N_0$
such that for all $N\geq N_0$,
the coefficient matrix $(I_n - \kDEn)$ is invertible.
Furthermore, there exists a constant $C$ independent of $N$
such that for all $N\geq N_0$,
\[
 \|u - \uDEn\|_{C([a,b])}
 \leq C \frac{\log(2 d N)}{N}\rme^{-\pi d N/log(2 d N)}.
\]
\end{theorem}


\section{Existing Sinc-collocation methods}
\label{sec:collocation}

This section describes two different Sinc-collocation methods
developed by
Stenger~\cite{stenger93:_numer}
and Rashidinia and Zarebnia~\cite{rashidinia07:_solut}.
Both methods employ the tanh transformation~\cref{eq:SEt}
as a variable transformation,
but their procedures are not identical.

\subsection{Sinc-collocation method by Stenger}

As explained in~\cref{sec:introduction},
Stenger derived his method for~\cref{eq:Volterra-initial-val},
where the kernel $\tilde{k}$ is a function of a single variable.
However, his method can be easily derived for~\cref{eq:Volterra-int}
as follows.
His method is closely related to the SE-Sinc-Nystr\"{o}m method,
which is described in the previous section.
First, solve the linear system~\cref{eq:SE-Sinc-Nystroem-linear-eq}
and obtain~$\mathbd{u}_n^{\textSE}$.
Then, his approximated solution $\vSEn$ is expressed as
the generalized SE-Sinc approximation of $\uSEn$, i.e.,
\begin{align}
\label{eq:SE-Sinc-collocation}
 \vSEn(t)
& = \ProjSE[\uSEn](t)\\
& = \sum_{j=-N}^N
\left\{\uSEn(\tSE_j) - \uSEn(\tSE_{-N})\omega_a(\tSE_j)
 - \uSEn(\tSE_N)\omega_b(\tSE_j)\right\}S(j,h)(\SEtInv(t))\nonumber\\
&\quad + \uSEn(\tSE_{-N})\omega_a(t) + \uSEn(\tSE_{N})\omega_b(t),\nonumber
\end{align}
where $\ProjSE$ is defined by~\cref{eq:ProjSE}.

The solution $\vSEn$ is also obtained
by the standard collocation procedure as follows.
Set the approximate solution $\vSEn$ as~\cref{eq:SE-Sinc-collocation},
where $\uSEn(\tSE_j)$ $(j=-N,\,\ldots,\,N)$ are
regarded as unknown coefficients.
%\begin{align*}
% \vSEn(t) =
% \sum_{j=-N}^N\left\{
% u_j - u_{-N}\omega_a(\tSE_j) - u_N\omega_b(\tSE_j)
% \right\}S(j,h)(\SEtInv(t))
% +u_{-N}\omega_a(t) + u_N\omega_b(t).
%\end{align*}
Substitute $\vSEn$ into the given equation~\cref{eq:Volterra-int},
with approximating the Volterra integral operator $\Vol$ by $\VolSEn$, where
\begin{equation}
\label{eq:VolSEn}
 \VolSEn [f](t)
=\sum_{j=-N}^N k(t,\tSE_j)f(\tSE_j)\SEtDiv(jh)J(j,h)(\SEtInv(t)),
\end{equation}
which is the SE-Sinc indefinite integration of $\Vol f$.
Then, setting $n=2N+1$ sampling points at $t=\tSE_i$
$(i=-N,\,-N+1,\,\ldots,\,N)$, we obtain
the same system of linear equations as~\cref{eq:SE-Sinc-Nystroem-linear-eq}.
%\[
% \sum_{j=-N}^N
%\left\{\delta_{ij} - k(\tSE_i,\tSE_j)\SEtDiv(jh)h\delta_{i-j}^{(-1)}\right\}
% u_j = g(\tSE_i), \quad i=-N,\,\ldots,\,N,
%\]
%where $\delta_{ij}$ is the Kronecker delta.
%This system of linear equations is
%nothing but~\cref{eq:SE-Sinc-Nystroem-linear-eq}.
Thus, Stenger's method can be regarded as a collocation method
utilizing the generalized SE-Sinc approximation,
namely, SE-Sinc-collocation method.

\subsection{Sinc-collocation method by Rashidinia and Zarebnia}

Rashidinia and Zarebnia derived their method
by the standard collocation procedure,
but they considered their approximated solution $\vRZn$
in different manners in the following four cases.
\begin{enumerate}
 \item[(I)] If $u(a)=u(b)=0$, set $\vRZn$ as
\[
 \vRZn(t) = \sum_{j=-N}^{N} c_{j} S(j,h)(\SEtInv(t)).
\]
 \item[(II)] If $u(a)\neq 0$ and $u(b)=0$, set $\vRZn$ as
\[
 \vRZn(t) = c_{-N}\omega_a(t) + \sum_{j=-N+1}^{N} c_{j} S(j,h)(\SEtInv(t)).
\]
 \item[(III)] If $u(a)= 0$ and $u(b)\neq 0$, set $\vRZn$ as
\[
 \vRZn(t) = \sum_{j=-N}^{N-1} c_{j} S(j,h)(\SEtInv(t)) + c_{N}\omega_b(t).
\]
 \item[(IV)] If $u(a)\neq 0$ and $u(b)\neq 0$, set $\vRZn$ as
\end{enumerate}
\begin{equation}
\label{eq:vRZn}
 \vRZn(t) = c_{-N}\omega_a(t)
 + \sum_{j=-N+1}^{N-1} c_{j} S(j,h)(\SEtInv(t)) + c_{N}\omega_b(t).
\end{equation}
To obtain the unknown coefficients
$\mathbd{c}_{n} = [c_{-N},\,c_{-N+1},\,\ldots,\,c_{N}]^{\mathrm{T}}$,
where $n=2N+1$,
they substituted $\vRZn$ into the given equation~\cref{eq:Volterra-int},
with approximating the Volterra integral operator $\Vol$ by $\VolSEn$.
Then, setting $n$ sampling points at $t=\tSE_i$
$(i=-N,\,-N+1,\,\ldots,\,N)$,
they derived a system of linear equations
in each of the four cases: (I)--(IV).
For example, in the case (I), the resulting system is expressed as
\[
 (I_n - V_{n}^{\textSE})\mathbd{c}_n = \mathbd{g}_n^{\textSE}.
\]
Particularly, in the case (IV), the resulting system is expressed as
\begin{equation}
\label{eq:RZ-linear-eq}
 (E_n^{\textRZ} - V_{n}^{\textRZ})\mathbd{c}_n = \mathbd{g}_n^{\textSE},
\end{equation}
where $E_n^{\textRZ}$ and $V_n^{\textRZ}$ are
$n\times n$ matrices defined by
\begin{align*}
 E_n^{\textRZ}
&= \left[
   \begin{array}{@{\,}c|ccc|c@{\,}}
   \omega_a(\tSE_{-N})        & 0    & \cdots &0  & \omega_b(\tSE_{-N}) \\
%   \hline
   \omega_a(\tSE_{-N+1}) & 1      &        &\Order & \omega_b(\tSE_{-N+1})\\
   \vdots     &        & \ddots &       & \vdots \\
   \omega_a(\tSE_{N-1}) & \Order &        &1      & \omega_b(\tSE_{N-1}) \\
%   \hline
   \omega_a(\tSE_{N}) & 0      & \cdots &0      & \omega_b(\tSE_{N})
   \end{array}
   \right], \\
  V_m^{\textRZ}
&= \left[
   \begin{array}{@{\,}l|clc|l@{\,}}
%   \VolSEn[\omega_a](\tSE_{-N-1})
   &\cdots
   & k(\tSE_{-N},\tSE_j)\SEtDiv(jh) h \delta_{-N-j}^{(-1)}
   &\cdots
   & \\ %\VolSEn[\omega_b](\tSE_{-N-1}) \\
%  \hline
%   \VolSEn[\omega_a](\tSE_{-N})
   &\cdots
   & k(\tSE_{-N+1},\tSE_j)\SEtDiv(jh) h \delta_{-N+1-j}^{(-1)}
   &\cdots
   & \\ %\VolSEn[\omega_b](\tSE_{-N}) \\
    \multicolumn{1}{c|}{\mathbd{p}_n^{\textRZ}} & & \multicolumn{1}{c}{\vdots}
   & & \multicolumn{1}{c}{\mathbd{q}_n^{\textRZ}}\\
%   \VolSEn[\omega_a](\tSE_{N})
   &\cdots
   & k(\tSE_{N-1},\tSE_j)\SEtDiv(jh) h \delta_{N-1-j}^{(-1)}
   &\cdots
   & \\ %\VolSEn[\omega_b](\tSE_N) \\
%  \hline
%   \VolSEn[\omega_a](\tSE_{N+1})
   &\cdots
   & k(\tSE_{N},\tSE_j)\SEtDiv(jh) h \delta_{N-j}^{(-1)}
   &\cdots
   & %\VolSEn[\omega_b](\tSE_{N+1})
   \end{array}
   \right],
\end{align*}
where $\mathbd{p}_n^{\textRZ}$
and $\mathbd{q}_n^{\textRZ}$
are $n$-dimensional vectors defined by
\begin{align*}
\mathbd{p}_n^{\textRZ}
&= [ \VolSEn[\omega_a](\tSE_{-N}),\,
\VolSEn[\omega_a](\tSE_{-N+1}),\,\ldots,\,
\VolSEn[\omega_a](\tSE_{N})]^{\mathrm{T}}, \\
\mathbd{q}_n^{\textRZ}
&= [ \VolSEn[\omega_b](\tSE_{-N}),\,
\VolSEn[\omega_b](\tSE_{-N+1}),\,\ldots,\,
\VolSEn[\omega_b](\tSE_{N})]^{\mathrm{T}}.
\end{align*}
This is the SE-Sinc-collocation method by Rashidinia and Zarebnia.
In the case (I), the following error analysis was provided.

\begin{theorem}[Zarebnia and Rashidinia~{\cite[Theorem~3]{zarebnia10:_conv}}]
\label{thm:Rarebnia-Rashidinia}
Let $\alpha$ and $d$ be positive constants with $d<\pi$.
Assume that the solution $u$ in~\cref{eq:Volterra-int}
satisfies all the assumptions in~\cref{thm:SE-Sinc-approx}.
Furthermore, assume that $k(t,\cdot)$
satisfies all the assumptions in~\cref{thm:SE-Sinc-indefinite}
for all $t\in [a, b]$.
Then, there exists a constant $C$ independent of $N$ such that
\[
 \|u - \vRZn\|_{C([a,b])}
\leq C \|(I_n- V_{n}^{\textSE})^{-1}\|_2\sqrt{N}\rme^{-\sqrt{\pi d \alpha N}}.
\]
\end{theorem}

However,
this theorem does not prove the convergence of $\vRZn$,
because there exists an unestimated
term $\|(I_n - V_{n}^{\textSE})^{-1}\|_2$, which clearly depends on $N$.
For the cases (II)--(IV), no error analysis has been provided thus far.

Moreover,
in a practical situation,
it is hard to determine whether $u$ is zero or not at the endpoints.
This is because the solution $u$ is an unknown function to be determined.
The idea to address the issue was presented
for Fredholm integral equations~\cite{okayama1x:_improv};
set the approximate solution $\vRZn$ as~\cref{eq:vRZn} in any cases.
In other words, we may treat the case (IV) as a general case.
This idea can be employed
for Volterra integral equations~\cref{eq:Volterra-int}.
Therefore, as a method by Rashidinia and Zarebnia,
this study adopts the following procedure:
(i) solve the linear system~\cref{eq:RZ-linear-eq},
and (ii) obtain the approximate solution by~\cref{eq:vRZn}.

\subsection{Main result 1: Relationship between the two methods and their convergence}

Any relationship between Stenger's method $(\vSEn)$
and Rashidinia--Zarebnia's method $(\vRZn)$ has not been investigated thus far.
Furthermore, convergence of the two methods has not been rigorously proved.
As a first contribution of this paper,
we show the relationship between the two methods as follows.
The proof is provided in~\cref{sec:proof-equivalence}.

\begin{theorem}
\label{thm:equivalence}
%Let $\alpha$ and $d$ be positive constants with
%$\alpha\leq 1$ and $d<\pi$.
%Assume that
%all the assumptions on $g$ and $k$ in~\cref{thm:SE-Sinc-Nystroem}
%are fulfilled.
%Let $h$ be selected by the formula~\cref{eq:h-SE}.
%Then, there exists a positive integer $N_0$
%such that for all $N\geq N_0$,
%$\vSEn = \vRZn$ holds.
Let $\vSEn$ be a function defined by~\cref{eq:SE-Sinc-collocation},
where $\mathbd{u}_n^{\textSE}$ is determined
by solving the linear system~\cref{eq:SE-Sinc-Nystroem-linear-eq}.
Furthermore,
let $\vRZn$ be a function defined by~\cref{eq:vRZn},
where $\mathbd{c}_{m}$ is determined
by solving the linear system~\cref{eq:RZ-linear-eq}.
Then, it holds that
\[
 \vSEn(\tSE_i) = \vRZn(\tSE_i),\quad i=-N,\,-N+1,\,\ldots,\,N,
\]
but generally $\vSEn\neq \vRZn$.
\end{theorem}

%Therefore, we may refer to both methods as the SE-Sinc-collocation method.
Subsequently, we provide the convergence theorems of
the two methods as follows.
Their proofs are provided in~\cref{sec:proof-SE-Sinc,sec:proof-RZ-Sinc}.

\begin{theorem}
\label{thm:SE-Sinc-collocation}
Let $\alpha$ and $d$ be positive constants with
$\alpha\leq 1$ and $d<\pi$.
Assume that
all the assumptions on $g$ and $k$ in~\cref{thm:SE-Sinc-Nystroem}
are fulfilled.
Furthermore, assume that $g$ and $k(\cdot,w)$
belong to $\MC_{\alpha}(\SEt(\domD_d))$ for all $w\in\SEt(\domD_d)$.
Let $h$ be selected by the formula~\cref{eq:h-SE}.
Then, there exists a positive integer $N_0$
such that for all $N\geq N_0$,
the coefficient matrix $(I_n - \kSEn)$ is invertible.
Furthermore, there exists a constant $C$ independent of $N$
such that for all $N\geq N_0$,
\[
 \|u - \vSEn\|_{C([a,b])} \leq C \sqrt{N} \rme^{-\sqrt{\pi d \alpha N}}.
\]
\end{theorem}

\begin{theorem}
\label{thm:RZ-Sinc-collocation}
Assume that all the assumptions of~\cref{thm:SE-Sinc-collocation}
are fulfilled.
Then, there exists a positive integer $N_0$
such that for all $N\geq N_0$,
the coefficient matrix $(E_n^{\textRZ} - V_{n}^{\textRZ})$ is invertible.
Furthermore, there exists a constant $C$ independent of $N$
such that for all $N\geq N_0$,
\[
%\label{eq:vRZn-estimate}
 \|u - \vRZn\|_{C([a,b])} \leq C \sqrt{N} \rme^{-\sqrt{\pi d \alpha N}}.
\]
\end{theorem}

\begin{remark}
In view of~\cref{thm:Rarebnia-Rashidinia,thm:RZ-Sinc-collocation},
one might assume that~\cref{thm:SE-Sinc-collocation}
is proved by bounding $\|(I_n - \kSEn)^{-1}\|_2$ uniformly for $N$.
However, this is not the case; see~\cref{sec:proof-SE} for details.
\end{remark}

\Cref{thm:SE-Sinc-collocation,thm:RZ-Sinc-collocation} reveal that
both methods achieve the same convergence rate.
Therefore, users may prefer Stenger's method
because the implementation of the method by Rashidinia and Zarebnia
is rather complicated.
This complication also causes difficulty in extension to the \emph{system} of
Volterra integral equations.
For this reason, in the next section,
we consider the improvement of Stenger's method.


\section{Sinc-collocation method combined with the DE transformation}
\label{sec:de-collocation}

The SE-Sinc-collocation method described in the previous section
employs the tanh transformation as a variable transformation.
In this section, a new method is derived
by replacing the tanh transformation with the DE transformation.
Then, its convergence theorem is stated.


\subsection{Derivation of the DE-Sinc-collocation method}

First, solve the linear system~\cref{eq:DE-Sinc-Nystroem-linear-eq}
and obtain~$\mathbd{u}_n^{\textDE}$.
Then, the approximated solution $\vDEn$ is expressed as
the generalized DE-Sinc approximation of $\uDEn$, i.e.,
\begin{align}
\label{eq:DE-Sinc-collocation}
 \vDEn(t)
& = \ProjDE[\uDEn](t)\\
& = \sum_{j=-N}^N
\left\{\uDEn(\tDE_j) - \uDEn(\tDE_{-N})\omega_a(\tDE_j)
 - \uDEn(\tDE_N)\omega_b(\tDE_j)\right\}S(j,h)(\DEtInv(t))\nonumber\\
&\quad + \uDEn(\tDE_{-N})\omega_a(t) + \uDEn(\tDE_{N})\omega_b(t),\nonumber
\end{align}
where $\ProjDE$ is defined by~\cref{eq:ProjDE}.
This procedure is referred to as the DE-Sinc-collocation method.

\subsection{Main result 2: Convergence of the DE-Sinc-collocation method}

In this paper,
we show the convergence of the DE-Sinc-collocation method
as follows.
The proof is provided in~\cref{sec:proof-DE}.

\begin{theorem}
\label{thm:DE-Sinc-collocation}
Let $\alpha$ and $d$ be positive constants with
$\alpha\leq 1$ and $d<\pi/2$.
Assume that
all the assumptions on $g$ and $k$ in~\cref{thm:DE-Sinc-Nystroem}
are fulfilled.
Furthermore, assume that $g$ and $k(\cdot,w)$
belong to $\MC_{\alpha}(\DEt(\domD_d))$ for all $w\in\DEt(\domD_d)$.
Let $h$ be selected by the formula~\cref{eq:h-DE}.
Then, there exists a positive integer $N_0$
such that for all $N\geq N_0$,
the coefficient matrix $(I_n - \kDEn)$ is invertible.
Furthermore, there exists a constant $C$ independent of $N$
such that for all $N\geq N_0$,
\[
 \|u - \vDEn\|_{C([a,b])}
\leq C \rme^{-\pi d N/\log(2 d N/\alpha)}.
\]
\end{theorem}

Compared to~\cref{thm:SE-Sinc-collocation,thm:RZ-Sinc-collocation},
we see that the convergence rate given by this theorem
is significantly improved.


\section{Numerical experiments}
\label{sec:numer-result}

This section presents numerical results for the following five methods:
the SE/DE-Sinc-Nystr\"{o}m methods
by Muhammad et al.~\cite{muhammad05:_numer},
the SE-Sinc-collocation methods by Stenger~\cite{stenger93:_numer}
and Rashidinia--Zarebnia~\cite{rashidinia07:_solut},
and the DE-Sinc-collocation methods by this paper.
The computation was performed on
a MacBook Air computer with 1.7~GHz Intel Core i7
with 8 GB memory, running macOS Big Sur.
The computation programs were implemented
in the C programming language with double-precision floating-point arithmetic,
and compiled with Apple clang version 13.0.0 with no optimization.
Cephes Math Library was used for computation of the sine integral.
LAPACK in Apple's Accelerate framework was used for computation
of the system of linear equations.
The source code for all programs is available at
\url{https://github.com/okayamat/sinc-colloc-volterra}.

We consider the following two equations
(taken from Rashidinia--Zarebnia~\cite[Example~4]{rashidinia07:_solut}
and~Polyanin--Manzhirov~\cite[Equation 2.1.45]{polyanin08:_handb}):
\begin{align}
\label{eq:example1}
u(t) + \int_0^t t s u(s)\diff s
 &= \rme^{-t^2} +\frac{t}{2}(1 - \rme^{-t^2}),\quad 0\leq t\leq 1,\\
\label{eq:example2}
u(t) - 6\int_0^t (\sqrt{t} - \sqrt{s}) u(s)\diff s
 &= 1 +\sqrt{t} - 2t\sqrt{t} - t^2,\quad 0\leq t\leq 1,
\end{align}
whose solutions are $u(t)=\rme^{-t^2}$ and
$u(t)=1 + \sqrt{t}$, respectively.
In the case of~\cref{eq:example1},
the assumptions
of~\cref{thm:SE-Sinc-Nystroem,thm:SE-Sinc-collocation,thm:RZ-Sinc-collocation}
are fulfilled with $d=3.14$
and $\alpha=1$,
and those of~\cref{thm:DE-Sinc-Nystroem,thm:DE-Sinc-collocation}
are fulfilled with $d=1.57$
and $\alpha=1$.
In the case of~\cref{eq:example2},
the assumptions
of~\cref{thm:SE-Sinc-Nystroem,thm:SE-Sinc-collocation,thm:RZ-Sinc-collocation}
are fulfilled with $d=3.14$
and $\alpha=1/2$,
and those of~\cref{thm:DE-Sinc-Nystroem,thm:DE-Sinc-collocation}
are fulfilled with $d=1.57$ and $\alpha=1/2$.
Therefore, those values were used for implementation.
The errors were evaluated at 2048 equally spaced points 
over the given interval,
and the maximum error among them was plotted on the graph
in~\cref{fig:example1_N,fig:example1_t,fig:example2_N,fig:example2_t}.

From all figures, we can observe that
the SE-Sinc-collocation methods by Stenger and Rashidinia--Zarebnia
yield almost the same performance.
This result coincides
with~\cref{thm:SE-Sinc-collocation,thm:RZ-Sinc-collocation}.
%the description at the end of~\cref{sec:collocation}.
%Furthermore, the DE-Sinc-collocation method presented by this paper
From~\cref{fig:example1_N}, we can observe that
the SE/DE-Sinc-Nystr\"{o}m methods are slightly better than
the SE/DE-Sinc-collocation methods with respect to $N$.
This result coincides with~\cref{thm:SE-Sinc-Nystroem,thm:DE-Sinc-Nystroem,thm:SE-Sinc-collocation,thm:RZ-Sinc-collocation,thm:DE-Sinc-collocation}.
However,~\cref{fig:example1_t} shows that
with respect to the computation time,
the SE/DE-Sinc-collocation methods demonstrate significantly better performance
than the SE/DE-Sinc-Nystr\"{o}m methods.
This is because the SE/DE-Sinc-Nystr\"{o}m methods
include a special function as well as given functions $k$ and $g$
in the basis functions of their approximate solutions.
We note that the performance of
the SE/DE-Sinc-collocation methods in~\cref{fig:example2_N}
reduced than that in~\cref{fig:example1_N},
which is due to the difference of $\alpha$.

\begin{figure}[htbp]
\begin{minipage}{0.495\linewidth}
  \centering
 \includegraphics[width=\linewidth]{example1_N}
  \caption{Errors with respect to $N$ for~\cref{eq:example1}. \phantom{not computation time}}
  \label{fig:example1_N}
\end{minipage}
\begin{minipage}{0.495\linewidth}
  \centering
  \includegraphics[width=\linewidth]{example1_t}
  \caption{Errors with respect to the computation time for~\cref{eq:example1}.}
  \label{fig:example1_t}
\end{minipage}
\end{figure}
\begin{figure}[htbp]
\begin{minipage}{0.495\linewidth}
  \centering
 \includegraphics[width=\linewidth]{example2_N}
  \caption{Errors with respect to $N$ for~\cref{eq:example2}. \phantom{not computation time}}
  \label{fig:example2_N}
\end{minipage}
\begin{minipage}{0.495\linewidth}
  \centering
  \includegraphics[width=\linewidth]{example2_t}
  \caption{Errors with respect to the computation time for~\cref{eq:example2}.}
  \label{fig:example2_t}
\end{minipage}
\end{figure}


\section{Proofs for the theorems presented in~\cref{sec:collocation}}
\label{sec:proof-SE}

In this section,
we provide proofs for~\cref{thm:equivalence,thm:SE-Sinc-collocation}.

\subsection{Proof of~\cref{thm:equivalence}}
\label{sec:proof-equivalence}

In addition to the given equation $(\Ident - \Vol)u = g$,
let us consider the following three equations:
\begin{align}
 (\Ident - \VolSEn)    \uSEn &= g, \label{eq:SE-Sinc-Nystroem-symbol} \\
 (\Ident - \ProjSE \VolSEn)v &= \ProjSE g,
 \label{eq:SE-Sinc-collocation-symbol}\\
% (E_m^{\textRZ} - V_{m}^{\textRZ})\mathbd{c}_m &= \mathbd{g}_m^{\textSE},
 (\Ident - \ProjRZ \VolSEn)w &= \ProjRZ g,
\label{eq:RZ-Sinc-collocation-symbol}
\end{align}
where $\VolSEn$ and $\ProjSE$ are defined
by~\cref{eq:VolSEn} and~\cref{eq:ProjSE}, respectively,
and $\ProjRZ$ is defined by
\begin{align}
\label{eq:ProjRZ}
  \ProjRZ[f](t)
&=\sum_{j=-N+1}^{N+1}\left\{f(\tSE_j) - \beta_N \omega_a(\tSE_j)
 - \gamma_N \omega_a(\tSE_j)\right\}S(j,h)(\SEtInv(t))\\
&\quad + \beta_N \omega_a(t) + \gamma_N \omega_b(t),
\nonumber
\end{align}
where $\beta_N$ and $\gamma_N$ are defined by
\begin{align*}
 \beta_N &=
\frac{f(\tSE_{-N})\omega_b(\tSE_{N}) - f(\tSE_{N})\omega_b(\tSE_{-N})}
     {\omega_a(\tSE_{-N})\omega_b(\tSE_{N}) - \omega_b(\tSE_{-N})\omega_a(\tSE_{N})},\\
\gamma_N &=
\frac{f(\tSE_{N})\omega_a(\tSE_{-N}) - f(\tSE_{-N})\omega_a(\tSE_{N})}
     {\omega_a(\tSE_{-N})\omega_b(\tSE_{N}) - \omega_b(\tSE_{-N})\omega_a(\tSE_{N})}.
\end{align*}
\begin{remark}
The denominator of $\beta_N$ and $\gamma_N$ is not zero
because
\begin{align*}
 \omega_a(\tSE_{-N})\omega_b(\tSE_{N}) - \omega_b(\tSE_{-N})\omega_a(\tSE_{N})
&=(1 - \omega_b(\tSE_{-N}))\omega_b(\tSE_N)
- \omega_b(\tSE_{-N})(1 - \omega_b(\tSE_N))\\
&=\omega_b(\tSE_N) - \omega_b(\tSE_{-N})\\
&= \tanh\left(\frac{Nh}{2}\right)\neq 0,
\end{align*}
provided that $N$ is a positive integer and $h> 0$.
\end{remark}

Because~\cref{eq:SE-Sinc-Nystroem-symbol} is equivalent
to~\cref{eq:SE-Sinc-Nystroem},
the solution of~\cref{eq:SE-Sinc-Nystroem-symbol}
is the approximate solution of the SE-Sinc-Nystr\"{o}m method.
On~\cref{eq:SE-Sinc-Nystroem-symbol}, the following result was obtained.

\begin{lemma}[Okayama et al.~{\cite[Lemma~6.7]{okayama13:_theo}}]
\label{lem:SE-Sinc-Nystroem}
Assume that
all the assumptions of~\cref{thm:SE-Sinc-Nystroem}
are fulfilled.
Then, there exists a positive integer $N_0$ such that for all $N\geq N_0$,
\cref{eq:SE-Sinc-Nystroem-symbol}
has a unique solution $\uSEn\in C([a,b])$.
Furthermore, there exists a constant $C$ independent of $N$ such that
for all $N\geq N_0$,
\begin{equation}
\label{eq:SE-Sinc-Nystroem-error}
 \|u - \uSEn\|_{C([a,b])}\leq C \|\Vol u - \VolSEn u\|_{C([a,b])}.
\end{equation}
\end{lemma}

This lemma says that~\cref{eq:SE-Sinc-Nystroem-symbol}
has a unique solution for all sufficiently large $N$.
Using this result, we show the following three things:
\begin{enumerate}
 \item[(i)] If~\cref{eq:SE-Sinc-Nystroem-symbol}
has a unique solution,
then~\cref{eq:SE-Sinc-collocation-symbol}
has also a unique solution $v = \vSEn$.
 \item[(ii)] If~\cref{eq:SE-Sinc-Nystroem-symbol}
has a unique solution,
then~\cref{eq:RZ-Sinc-collocation-symbol}
has also a unique solution $w = \vRZn$.
 \item[(iii)] The two solutions $\vSEn$ and $\vRZn$ are not generally
equivalent, but at the collocation points,
 $\vSEn(\tSE_i) = \vRZn(\tSE_{i})$ $(i=-N,\,\ldots,\,N)$ holds.
\end{enumerate}
First, we show (i) as follows.

\begin{lemma}
\label{lem:Stenger-solution}
The following two statements are equivalent:
\begin{enumerate}
 \item[{\rm (A)}] \Cref{eq:SE-Sinc-Nystroem-symbol}
has a unique solution $\uSEn\in C([a,b])$.
 \item[{\rm (B)}] \Cref{eq:SE-Sinc-collocation-symbol} has
 a unique solution $v\in C([a,b])$.
\end{enumerate}
Furthermore, $v=\vSEn$ holds.
\end{lemma}
\begin{proof}
First, let us show $\mathrm{(A)} \Rightarrow \mathrm{(B)}$.
Note that $\VolSEn\ProjSE f = \VolSEn f$ holds
because
of the interpolation property $\ProjSE[f](\tSE_i)=f(\tSE_i)$
($i=-N,\,\ldots,\,N$).
Applying $\ProjSE$ on the both sides
of~\cref{eq:SE-Sinc-Nystroem-symbol},
we have
\[
 \ProjSE\uSEn
 = \ProjSE(g + \VolSEn\uSEn)
 = \ProjSE(g + \VolSEn\ProjSE\uSEn),
\]
which is equivalent to $\vSEn = \ProjSE(g + \VolSEn \vSEn)$
(recall that $\vSEn = \ProjSE\uSEn$).
This equation implies
that~\cref{eq:SE-Sinc-collocation-symbol} has a solution
$\vSEn\in C([a,b])$.

Next, we show the uniqueness.
Suppose that~\cref{eq:SE-Sinc-collocation-symbol}
has another solution $\tilde{v}\in C([a,b])$.
Let us set a function $\tilde{u}$ as
$\tilde{u} = g + \VolSEn\tilde{v}$.
Because $\tilde{v}$ is a solution of~\cref{eq:SE-Sinc-collocation-symbol},
we have
\[
 \tilde{v} = \ProjSE(g + \VolSEn\tilde{v}) = \ProjSE \tilde{u},
\]
from which it holds that
\[
 \tilde{u} = g + \VolSEn\tilde{v} = g + \VolSEn\ProjSE\tilde{u}
= g + \VolSEn\tilde{u}.
\]
This equation implies that $\tilde{u}$ is a solution
of~\cref{eq:SE-Sinc-Nystroem-symbol}.
Because the solution of~\cref{eq:SE-Sinc-Nystroem-symbol}
is unique, $\tilde{u} = u$ holds,
from which we have $\ProjSE\tilde{u}=\ProjSE u$.
Thus, we have $\tilde{v}=v$, which shows $(\mathrm{B})$.

The above argument is reversible,
which proves $\mathrm{(B)}\Rightarrow\mathrm{(A)}$.
Furthermore, in view of the proof above,
we see $v=\vSEn$, which is to be demonstrated.
\end{proof}

Next, for showing (ii),
we show the following result.
The proof is omitted because it goes in the same way as
that of~\cref{lem:Stenger-solution}.

\begin{lemma}
\label{lem:RZ-solution}
The following two statements are equivalent:
\begin{enumerate}
 \item[{\rm (A)}] \Cref{eq:SE-Sinc-Nystroem-symbol}
has a unique solution $\uSEn\in C([a,b])$.
 \item[{\rm (B)}] \Cref{eq:RZ-Sinc-collocation-symbol} has
 a unique solution $w\in C([a,b])$.
\end{enumerate}
Furthermore, $w=\ProjRZ\uSEn$ holds.
\end{lemma}

To show (ii) completely, we further have to show $w = \vRZn$,
which is done by the following result.
Noting $\ProjRZ[f](\tSE_i)=f(\tSE_i)$ $(i=-N,\,\ldots,\,N)$,
we can prove this result
following Atkinson~\cite[Sect.\ 4.3]{atkinson97:_numer_solut},
and hence the proof is omitted.

\begin{proposition}
\label{prop:RZ-solution}
The following two statements are equivalent:
\begin{enumerate}
 \item[{\rm (A)}] \Cref{eq:RZ-Sinc-collocation-symbol} has
 a unique solution $w\in C([a,b])$.
 \item[{\rm (B)}] \Cref{eq:RZ-linear-eq} has
 a unique solution $\mathbd{c}_m\in \mathbb{R}^m$.
\end{enumerate}
Furthermore, $w=\vRZn$ holds.
\end{proposition}
%\begin{proof}
%First, let us show $\mathrm{(A)} \Rightarrow \mathrm{(B)}$.
%Note that $(\ProjSE)^2 f = \ProjSE f$ holds because
%of the interpolation property $\ProjSE[f](\tSE_i)=f(\tSE_i)$
%($i=-N,\,\ldots,\,N$).
%Because the solution $v$ of~\cref{eq:SE-Sinc-collocation-symbol}
%is written as
%\begin{equation}
%\label{eq:v-is-projected-by-ProjSE}
% v = \ProjSE (g + \VolSEn v),
%\end{equation}
%we have $\ProjSE v = (\ProjSE)^2 (g + \VolSEn v) = \ProjSE (g + \VolSEn v)= v$.
%%Therefore, there exist coefficients
%%$\mathbd{c}_m=[c_{-N-1},\,c_{-N},\,\ldots,\,c_{N+1}]^{\mathrm{T}}$
%%such that
%%\[
%% v(t) = c_{-N-1}\omega_a(t)
%% + \sum_{j=-N}^N c_j S(j,h)(\SEtInv(t)) + c_{N+1}\omega_b(t).
%%\]
%Therefore, we can rewrite~\cref{eq:v-is-projected-by-ProjSE} as
%\[
% \ProjSE( v - g - \VolSEn v) = 0.
%\]
%Put $f = v - g - \VolSEn v$.
%The equation $\ProjSE f = 0$ requires the following equations
%\[
% \ProjSE[f](\tSE_i) = f(\tSE_i) = 0,\quad i=-N,\,\ldots,\,N.
%\]
%Considering more two points $t=\tSE_{-N-1}$ and $t=\tSE_{N+1}$,
%we have
%\begin{align*}
% \ProjSE [f](\tSE_{-N-1})
%& = f(\tSE_{-N})\omega_a(\tSE_{-N-1}) + f(\tSE_{N})\omega_a(\tSE_{-N-1})
% = 0,\\
% \ProjSE [f](\tSE_{N+1})
%& = f(\tSE_{-N})\omega_a(\tSE_{N+1}) + f(\tSE_{N})\omega_a(\tSE_{N+1})
% = 0,
%\end{align*}
%because $f(\tSE_{-N})=f(\tSE_{N})=0$.
%Thus, $\ProjSE f = 0$ implies
%\[
% f(\tSE_i) = 0,\quad i = -N-1,\,-N,\,\ldots,\,N,\,N+1,
%\]
%which is equivalent to
%\begin{equation}
%\label{eq:v-N+1-equation}
%v(\tSE_i) - \VolSEn[v](\tSE_i) = g(\tSE_i),\quad i=-N-1,\,-N,\,\ldots,\,N,\,N+1.
%\end{equation}
%Here, note that $v$ is written as
%\begin{align*}
% v(t) &= \sum_{j=-N}^N
%\left\{v(\tSE_j)
% - v(\tSE_{-N})\omega_a(\tSE_j) - v(\tSE_N)\omega_b(\tSE_j)
%\right\}S(j,h)(\SEtInv(t))\\
%&\quad + v(\tSE_{-N})\omega_a(t) + v(\tSE_N)\omega_b(t),
%\end{align*}
%because $v=\ProjSE v$.
%Putting
%\begin{align*}
% c_{-N-1} &= v(\tSE_{-N}),\\
% c_i      &= v(\tSE_i) - v(\tSE_{-N})\omega_a(\tSE_i) - v(\tSE_N)\omega_b(\tSE_i),
%\quad i = -N,\,\ldots,\,N,\\
% c_{N+1}  &= v(\tSE_N),
%\end{align*}
%we find that~\cref{eq:v-N+1-equation} is nothing but
%the linear system~\cref{eq:RZ-linear-eq}.
%Thus, \cref{eq:RZ-linear-eq} has a solution $\mathbd{c}_m\in \mathbb{R}^m$.
%
%Next, we show the uniqueness.
%Suppose that~\cref{eq:RZ-linear-eq}
%has another solution $\mathbd{\tilde{c}}_m\in\mathbb{R}^m$.
%Let us set a function $\tilde{v}\in C([a,b])$ as
%\[
% \tilde{v}(t)
%= \sum_{j=-N}^N \tilde{c}_j S(j,h)(\SEtInv(t))
% + \tilde{c}_{-N-1}\omega_a(t) + \tilde{c}_{N+1}\omega_b(t).
%\]
%Because $\mathbd{\tilde{c}}_m\in\mathbb{R}^m$ is a solution
%of~\cref{eq:RZ-linear-eq}, we have
%%\begin{align*}
%%& \sum_{j=-N}^N\left\{
%%\delta_{ij} - h k(\tSE_i,\tSE_j)\SEtDiv(jh)\delta^{(-1)}_{i-j}
%%\right\}\tilde{c}_j\\
%%&\quad
%%+ \left\{\omega_a(\tSE_i) - \VolSEn[\omega_a](\tSE_i)\right\}\tilde{c}_{-N-1}\\
%%&\quad
%%+ \left\{\omega_b(\tSE_i) - \VolSEn[\omega_b](\tSE_i)\right\}\tilde{c}_{N+1}
%%= g(\tSE_i),
%%\quad i=-N-1,\,-N,\,\ldots,\,N,\,N+1.
%%\end{align*}
%\begin{align*}
%\tilde{v}(\tSE_i) - \VolSEn[\tilde{v}](\tSE_i) = g(\tSE_i),\quad
% i=-N-1,\,-N,\,\ldots,\,N,\,N+1.
%\end{align*}
%Put $\tilde{f}= \tilde{v} - g - \VolSEn\tilde{v}$.
%From $\tilde{f}(\tSE_i)=0$ $(i=-N-1,\,-N,\,\ldots,\,N,\,N+1)$,
%we have $\ProjSE \tilde{f} = 0$, which is equivalent to
%\[
% \ProjSE(\tilde{v} - g -\VolSEn \tilde{v}) = 0.
%\]
%If $\ProjSE\tilde{v} =\tilde{v}$ is shown, then this equation is rewritten as
%\[
% \tilde{v} - \ProjSE\VolSEn\tilde{v} = \ProjSE g,
%\]
%which implies $\tilde{v}$ is a solution
%of~\cref{eq:SE-Sinc-collocation-symbol}.
%Because the solution of~\cref{eq:SE-Sinc-collocation-symbol}
%is unique, $\tilde{v} = v$ holds.
%This implies $\mathbd{\tilde{c}}_m = \mathbd{c}_m$,
%i.e., the solution of~\cref{eq:RZ-linear-eq} is unique,
%which shows $\mathrm{(B)}$.
%What remains to be shown is $\ProjSE\tilde{v} =\tilde{v}$.
%Set $\tilde{v}_i$ as the value of $\tilde{v}(\tSE_i)$, i.e.,
%\begin{equation}
%\label{eq:tilde-v-i}
%\tilde{v}_i
%=
% \tilde{c}_{-N-1}\omega_a(\tSE_i) + \tilde{c}_i
% + \tilde{c}_{N+1}\omega_b(\tSE_i),
%\quad i=-N,\,\ldots,\,N.
%\end{equation}
%Calculating $\ProjSE\tilde{v}$ by definition~\cref{eq:ProjSE}, we have
%\begin{align*}
% \ProjSE[\tilde{v}](t)
%&=\sum_{j=-N}^N \left\{\tilde{v}_j - \tilde{v}_{-N}\omega_a(\tSE_j)
%- \tilde{v}_{N}\omega_b(\tSE_j)
%\right\}S(j,h)(\SEtInv(t))\\
%&\quad + \tilde{v}_{-N}\omega_a(t) + \tilde{v}_{N}\omega_b(t).
%\end{align*}
%This function satisfies $\ProjSE[\tilde{v}](\tSE_i)=\tilde{v}(\tSE_i)$
%$(i=-N,\,\ldots,\,N)$ by itself.
%Furthermore, we require the equality at more two points,
%$t=\tSE_{-N-1}$ and $t=\tSE_{N+1}$.
%Substituting the two points into $\ProjSE[\tilde{v}](t)$, we have
%\begin{align*}
% \ProjSE[\tilde{v}](\tSE_{-N-1})
%&= \tilde{v}_{-N}\omega_a(\tSE_{-N-1}) + \tilde{v}_{N}\omega_b(\tSE_{-N-1}),\\
% \ProjSE[\tilde{v}](\tSE_{N+1})
%&= \tilde{v}_{-N}\omega_a(\tSE_{N+1}) + \tilde{v}_{N}\omega_b(\tSE_{N+1}),
%\end{align*}
%which must be equal to
%\begin{align*}
% \tilde{v}(\tSE_{-N-1})
%&= \tilde{c}_{-N-1}\omega_a(\tSE_{-N-1}) +\tilde{c}_{N+1}\omega_b(\tSE_{-N-1}),
%\\
% \tilde{v}(\tSE_{N+1})
%&= \tilde{c}_{-N-1}\omega_a(\tSE_{N+1}) +\tilde{c}_{N+1}\omega_b(\tSE_{N+1}),
%\end{align*}
%respectively.
%Therefore, we have $\tilde{v}_{-N}=\tilde{c}_{-N-1}$
%and $\tilde{v}_N = \tilde{c}_{N+1}$,
%from which it holds that
%\begin{align*}
% \ProjSE[\tilde{v}](t)
%&=\sum_{j=-N}^N \left\{\tilde{v}_j - \tilde{c}_{-N-1}\omega_a(\tSE_j)
%- \tilde{c}_{N+1}\omega_b(\tSE_j)
%\right\}S(j,h)(\SEtInv(t))\\
%&\quad + \tilde{c}_{-N-1}\omega_a(t) + \tilde{c}_{N+1}\omega_b(t)\\
%&= \tilde{v}(t),
%\end{align*}
%where~\cref{eq:tilde-v-i} is used at the last equality.
%This completes the proof for $\mathrm{(A)}\Rightarrow\mathrm{(B)}$.
%
%The proof for $\mathrm{(B)}\Rightarrow\mathrm{(A)}$
%is omitted because it goes in the same manner as that for
%$\mathrm{(A)}\Rightarrow\mathrm{(B)}$.
%Furthermore, in view of the proof above,
%we see $\tilde{v}=\vRZn$, which is to be demonstrated.
%\end{proof}

From the above results (i) and (ii),
we find that $\vSEn = \ProjSE\uSEn$ and $\vRZn = \ProjRZ\uSEn$.
Using the interpolation property of $\ProjSE$ and $\ProjRZ$ as
\[
 \ProjSE[\uSEn](\tSE_i) = \uSEn(\tSE_i) = \ProjRZ[\uSEn](\tSE_i),
\quad i = -N,\,-N+1,\,\ldots,\,N,
\]
we have $\vSEn(\tSE_i)=\vRZn(\tSE_i)$.
However, we note that $\ProjSE$ and $\ProjRZ$ is not
generally equivalent. This can be observed
by the limits $t\to a$ and $t\to b$ as
\begin{align*}
 \lim_{t\to a}\ProjSE[f](t) = f(\tSE_{-N})
&\neq \beta_N =\lim_{t\to a}\ProjRZ[f](t),\\
 \lim_{t\to b}\ProjSE[f](t) = f(\tSE_{N})
&\neq \gamma_N =\lim_{t\to b}\ProjRZ[f](t).
\end{align*}
Thus, we obtain the claim of~\cref{thm:equivalence}.

\subsection{Proof of~\cref{thm:SE-Sinc-collocation}}
\label{sec:proof-SE-Sinc}

The invertibility of $(I_n - \kSEn)$
is already shown by~\cref{thm:SE-Sinc-Nystroem}.
Thus, we concentrate on the analysis of the error of $\vSEn$.
Because $\vSEn = \ProjSE \uSEn$, it holds that
\[
 u - \vSEn = u - \ProjSE\uSEn
= (u - \ProjSE u) + \ProjSE(u - \uSEn),
\]
which leads to
\begin{equation}
\label{eq:SE-Sinc-first-error}
 \|u - \vSEn\|_{C([a,b])}
\leq \|u - \ProjSE u\|_{C([a,b])}
 + \|\ProjSE\|_{\mathcal{L}(C([a,b]),C([a,b]))}
\|u - \uSEn\|_{C([a,b])}.
\end{equation}
For the first term, we show $u\in\MC_{\alpha}(\SEt(\domD_d))$,
from which we can use~\cref{thm:SE-Sinc-general}.
For the purpose, the following theorem is useful.

\begin{theorem}[Okayama et al.~{\cite[Theorem~3.2]{okayama13:_theo}}]
\label{thm:Sinc-Nyst-regularity}
Let $\domD=\SEt(\domD_d)$ or $\domD=\DEt(\domD_d)$.
Assume that $g$, $k(z,\cdot)$ and $k(\cdot,w)$ belong to $\Hinf(\domD)$
for all $z,\, w\in\domD$.
Then,~\cref{eq:Volterra-int} has a unique solution $u\in\Hinf(\domD)$.
\end{theorem}

Using this theorem, we can show the following result.

\begin{theorem}
\label{thm:Sinc-collocation-regularity}
Let $\alpha$ be a positive constant with $\alpha\leq 1$.
Assume that all the assumptions of~\cref{thm:Sinc-Nyst-regularity}
are fulfilled. Furthermore,
assume that $g$ and $k(\cdot, w)$ belong to $\MC_{\alpha}(\domD)$
for all $w\in\domD$.
Then,~\cref{eq:Volterra-int} has a unique solution $u\in\MC_{\alpha}(\domD)$.
\end{theorem}
\begin{proof}
According to~\cref{thm:Sinc-Nyst-regularity},~\cref{eq:Volterra-int}
has a unique solution $u\in\Hinf(\domD)$.
Therefore, we only have to show the H\"{o}lder continuity
of $u$ at the endpoints.
Using $u = g + \Vol u$, we have
\begin{align*}
&|u(b) - u(z)|\\
&=\left| \left(g(b) + \int_a^b k(b,w)u(w)\diff w\right)
- \left(g(z) + \int_a^z k(z,w)u(w)\diff w\right)\right|\\
&\leq \left|g(b) - g(z)\right|
+ \left|\int_z^b k(b,w)u(w)\diff w\right|
+ \left|\int_a^z \left\{k(b,w)- k(z,w)\right\}u(w)\diff w\right|.
\end{align*}
From the H\"{o}lder continuity of $g$,
the first term can be bounded by
$L_g|b - z|^{\alpha}$ for some constant $L_g$.
From the boundedness of $k$ and $u$,
the second term can be bounded by
$L_{k,u}|b - z|$ for some constant $L_{k,u}$.
Furthermore, from the boundedness of $\domD$ and $\alpha\leq 1$,
we have
$|b - z|=|b - z|^{1-\alpha}|b - z|^{\alpha}\leq L_{\domD}|b - z|^{\alpha}$
for some constant $L_{\domD}$.
From the H\"{o}lder continuity of $k$ and
boundedness of $u$,
the third term can be bounded by
$\tilde{L}_{k,u}|b - z|^{\alpha}|z - a|$ for some constant $\tilde{L}_{k,u}$.
Furthermore, from the boundedness of $\domD$,
we have $|z - a|\leq \tilde{L}_{\domD}$ for some constant $\tilde{L}_{\domD}$.
Thus, there exists a constant $L$ such that
$|u(b) - u(z)|\leq L|b - z|^{\alpha}$,
which shows the H\"{o}lder continuity of $u$ at $z = b$.
The proof for the H\"{o}lder continuity at $z = a$ is omitted
because it follows the same method as that at $z = b$.
This completes the proof.
\end{proof}

From this theorem, we can use~\cref{thm:SE-Sinc-general}
for estimating the first term of~\cref{eq:SE-Sinc-first-error} as
\[
 \|u - \ProjSE u\|_{C([a,b])}
\leq C_1 \sqrt{N} \rme^{-\sqrt{\pi d \alpha N}}
\]
for some constant $C_1$.
For the second term, we use the following bound for
the operator $\ProjSE$.

\begin{lemma}[Okayama~{\cite[Lemma~7.2]{okayama23:_theo}}]
Let $\ProjSE$ be defined by~\eqref{eq:ProjSE}.
Then, there exists a constant $C_2$ independent of $N$ such that
\[
 \|\ProjSE\|_{\mathcal{L}(C([a,b]),C([a,b]))}
\leq C_2 \log(N+1).
\]
\end{lemma}

The remaining term to be estimated in~\cref{eq:SE-Sinc-first-error}
is $\|u - \uSEn\|_{C([a,b])}$.
According to~\cref{lem:SE-Sinc-Nystroem},
it is estimated as~\cref{eq:SE-Sinc-Nystroem-error}.
Because $u\in\Hinf(\SEt(\domD_d))$, $u$ satisfies the assumptions
of~\cref{thm:SE-Sinc-indefinite},
from which we have
\[
 \|\Vol u - \VolSEn u\|_{C([a,b])}
\leq C_3 \rme^{-\sqrt{\pi d \alpha N}}.
\]
Thus, there exists a constant $C_4$ such that
\begin{align*}
 \|u - \vSEn\|_{C([a,b])}
&\leq C_1 \sqrt{N} \rme^{-\sqrt{\pi d \alpha N}}
+ C_2 \log(N+1) C_3 \rme^{-\sqrt{\pi d \alpha N}}\\
&\leq C_4 \sqrt{N}\rme^{-\sqrt{\pi d \alpha N}}.
\end{align*}
This completes the proof of~\cref{thm:SE-Sinc-collocation}.

\subsection{Proof of~\cref{thm:RZ-Sinc-collocation}}
\label{sec:proof-RZ-Sinc}

For~\cref{thm:RZ-Sinc-collocation},
the invertibility of $(E_n^{\textRZ} - V_{n}^{\textRZ})$
can be shown by combining~\cref{lem:SE-Sinc-Nystroem,lem:RZ-solution}
and~\cref{prop:RZ-solution}.
Thus, we concentrate on the analysis of the error of $\vRZn$.
By the triangle inequality, we have
\begin{align*}
 \|u(t) - \vRZn(t)\|_{C([a,b])}
&\leq \|u(t) - \vSEn(t)\|_{C([a,b])}
 + \|\vSEn(t) - \vRZn(t)\|_{C([a,b])}\\
&= \|u(t) - \vSEn(t)\|_{C([a,b])}
 + \|\ProjSE\uSEn(t) - \ProjRZ\uSEn(t)\|_{C([a,b])}.
\end{align*}
Because the first term
% $\|u(t) - \vSEn(t)\|_{C([a,b])}$
is already estimated by~\cref{thm:SE-Sinc-collocation},
we estimate the second term.
For the purpose, the following lemma is essential.

\begin{lemma}
Let $\ProjSE: C([a,b])\to C([a,b])$ and
$\ProjRZ: C([a,b])\to C([a,b])$ be defined
by~\cref{eq:ProjSE} and~\cref{eq:ProjRZ}, respectively.
Then, there exists a constant $C$ independent of $N$ such that
\[
 \left\|\ProjSE - \ProjRZ\right\|_{\mathcal{L}(C([a,b]),C([a,b]))}
\leq \frac{C}{\rme^{Nh} - 1}\log(N+1).
\]
\end{lemma}
\begin{proof}
First, it holds for $f\in C([a,b])$ that
\begin{align*}
& \ProjSE[f](t) - \ProjRZ[f](t)\\
&= - \sum_{j=-N}^N\left\{
\left(f(\tSE_{-N})-\beta_N\right)\omega_a(\tSE_j) +
\left(f(\tSE_{N})-\gamma_N\right)\omega_b(\tSE_j)
\right\}S(j,h)(\SEtInv(t))\\
&\quad+\left(f(\tSE_{-N}) - \beta_N\right)\omega_a(t)
+\left(f(\tSE_N) - \gamma_N\right)\omega_b(t).
\end{align*}
Here, noting
\begin{align*}
 |f(\tSE_{-N}) - \beta_N|
&=\frac{|f(\tSE_N) - f(\tSE_{-N})|}{\rme^{Nh} - 1}
\leq \frac{2\|f\|_{C([a,b])}}{\rme^{Nh} - 1},\\
 |f(\tSE_{N}) - \gamma_N|
&=\frac{|f(\tSE_{-N}) - f(\tSE_{N})|}{\rme^{Nh} - 1}
\leq \frac{2\|f\|_{C([a,b])}}{\rme^{Nh} - 1},
\end{align*}
and using $\omega_a(t) + \omega_b(t) = 1$, we have
\begin{align*}
& \left\|\ProjSE - \ProjRZ\right\|_{\mathcal{L}(C([a,b]),C([a,b]))}\\
&\leq \frac{2}{\rme^{Nh} - 1}
\left\{\sum_{j=-N}^N\left(\omega_a(\tSE_j) + \omega_b(\tSE_j)\right) |S(j,h)(\SEtInv(t))|
+\omega_a(t) + \omega_b(t)
\right\}\\
&= \frac{2}{\rme^{Nh} - 1}
\left\{\sum_{j=-N}^N |S(j,h)(\SEtInv(t))| + 1\right\}\\
&\leq \frac{2}{\rme^{Nh} - 1}
\left\{\frac{2}{\pi}(3 + \log N) + 1\right\},
\end{align*}
where the standard bound~\cite[Problem 3.1.5 (a)]{stenger93:_numer}
is used for the last inequality.
Thus, the claim follows.
\end{proof}

From this lemma, substituting~\cref{eq:h-SE} into $h$,
we estimate the second term as
\[
 \|(\ProjSE - \ProjRZ) \uSEn\|_{C([a,b])}
\leq \frac{C}{1 - \rme^{-\sqrt{\pi d / \alpha}}}
\log(N+1)\rme^{-\sqrt{\pi d N/\alpha}}\left\|\uSEn\right\|_{C([a,b])}.
\]
Noting $\alpha\in (0, 1]$, we obtain
$\rme^{-\sqrt{\pi d N/\alpha}}\leq \rme^{-\sqrt{\pi d \alpha N}}$.
Furthermore, $\log(N+1)\leq \sqrt{N}$ holds.
Hence, the proof is completed if
$\left\|\uSEn\right\|_{C([a,b])}$ is uniformly bounded
with respect to $N$.
This is shown by the following estimate
\[
 \left\|\uSEn\right\|_{C([a,b])}
\leq \left\|u - \uSEn\right\|_{C([a,b])}
+\left\| u\right\|_{C([a,b])}.
\]
From~\cref{eq:SE-Sinc-Nystroem-error}
and~\cref{thm:SE-Sinc-indefinite},
we see that $\left\|u - \uSEn\right\|_{C([a,b])}$
converges to $0$ as $N\to\infty$,
and accordingly it is uniformly bounded.
We also see that $\left\| u\right\|_{C([a,b])}$
is bounded because $u$ is continuous on $[a, b]$
from the assumption (see~\cref{thm:Sinc-collocation-regularity}).
This completes the proof of~\cref{thm:RZ-Sinc-collocation}.


\section{Proofs for the theorem presented in~\cref{sec:de-collocation}}
\label{sec:proof-DE}

In this section, we provide proofs for~\cref{thm:DE-Sinc-collocation}.

\subsection{Existence and uniqueness of the approximated equations}

In addition to the given equation $(\Ident - \Vol)u = g$,
let us consider the following two equations:
\begin{align}
 (\Ident - \VolDEn)    \uDEn &= g, \label{eq:DE-Sinc-Nystroem-symbol} \\
 (\Ident - \ProjDE \VolDEn)v &= \ProjDE g,
 \label{eq:DE-Sinc-collocation-symbol}
% (E_m^{\textRZ} - V_{m}^{\textRZ})\mathbd{c}_m &= \mathbd{g}_m^{\textDE},
%\label{eq:RZ-linear-eq-in-proof}
\end{align}
where $\VolDEn$ is defined by
\[
%\label{eq:VolDEn}
 \VolDEn [f](t)
=\sum_{j=-N}^N k(t,\tDE_j)f(\tDE_j)\DEtDiv(jh)J(j,h)(\DEtInv(t)),
\]
and $\ProjDE$ are defined by~\cref{eq:ProjDE}.
Because~\cref{eq:DE-Sinc-Nystroem-symbol} is equivalent
to~\cref{eq:DE-Sinc-Nystroem},
the solution of~\cref{eq:DE-Sinc-Nystroem-symbol}
is the approximate solution of the DE-Sinc-Nystr\"{o}m method.
On~\cref{eq:DE-Sinc-Nystroem-symbol}, the following result was obtained.

\begin{lemma}[Okayama et al.~{\cite[Lemma~6.10]{okayama13:_theo}}]
\label{lem:DE-Sinc-Nystroem}
Assume that
all the assumptions of~\cref{thm:DE-Sinc-Nystroem}
are fulfilled.
Then, there exists a positive integer $N_0$ such that for all $N\geq N_0$,
\cref{eq:DE-Sinc-Nystroem-symbol}
has a unique solution $\uDEn\in C([a,b])$.
Furthermore, there exists a constant $C$ independent of $N$ such that
for all $N\geq N_0$,
\begin{equation}
\label{eq:DE-Sinc-Nystroem-error}
 \|u - \uDEn\|_{C([a,b])}\leq C \|\Vol u - \VolDEn u\|_{C([a,b])}.
\end{equation}
\end{lemma}

On~\cref{eq:DE-Sinc-collocation-symbol}, we can show the following lemma
in the same manner as~\cref{lem:Stenger-solution}
(hence, the proof is omitted).

\begin{lemma}
The following two statements are equivalent:
\begin{enumerate}
 \item[{\rm (A)}] \Cref{eq:DE-Sinc-Nystroem-symbol}
has a unique solution $\uDEn\in C([a,b])$.
 \item[{\rm (B)}] \Cref{eq:DE-Sinc-collocation-symbol} has
 a unique solution $v\in C([a,b])$.
\end{enumerate}
Furthermore, $v=\vDEn$ holds.
\end{lemma}

On the basis of the results,
we proceed to analyze the error of $\vDEn$ next.

\subsection{Proof of~\cref{thm:DE-Sinc-collocation}}

In the same manner as~\cref{eq:SE-Sinc-first-error},
we have
\begin{equation}
\label{eq:DE-Sinc-first-error}
 \|u - \vDEn\|_{C([a,b])}
\leq \|u - \ProjDE u\|_{C([a,b])}
 + \|\ProjDE\|_{\mathcal{L}(C([a,b]),C([a,b]))}
\|u - \uDEn\|_{C([a,b])}.
\end{equation}
For the first term,
from~\cref{thm:Sinc-collocation-regularity},
we can use~\cref{thm:DE-Sinc-general} as
\[
 \|u - \ProjDE u\|_{C([a,b])}
\leq C_1 \rme^{-\pi d N/\log(2 d N/\alpha)}
\]
for some constant $C_1$.
For the second term, we use the following bound for
the operator $\ProjDE$.

\begin{lemma}[Okayama~{\cite[Lemma~7.5]{okayama23:_theo}}]
Let $\ProjDE$ be defined by~\eqref{eq:ProjDE}.
Then, there exists a constant $C_2$ independent of $N$ such that
\[
 \|\ProjDE\|_{\mathcal{L}(C([a,b]),C([a,b]))}
\leq C_2 \log(N+1).
\]
\end{lemma}

The remaining term to be estimated in~\cref{eq:DE-Sinc-first-error}
is $\|u - \uDEn\|_{C([a,b])}$.
According to~\cref{lem:DE-Sinc-Nystroem},
it is estimated as~\cref{eq:DE-Sinc-Nystroem-error}.
Because $u\in\Hinf(\DEt(\domD_d))$, $u$ satisfies the assumptions
of~\cref{thm:DE-Sinc-indefinite},
from which we have
\[
 \|\Vol u - \VolDEn u\|_{C([a,b])}
\leq C_3 \frac{\log(2 d N/\alpha)}{N}\rme^{-\pi d N/\log(2 d N/\alpha)}.
\]
Thus, there exists a constant $C_4$ such that
\begin{align*}
& \|u - \vDEn\|_{C([a,b])}\\
&\leq C_1 \rme^{-\pi d N/\log(2 d N/\alpha)}
+ C_2 \log(N+1) C_3 \frac{\log(2 d N/\alpha)}{N}
\rme^{-\pi d N/\log(2 d N/\alpha)}\\
&\leq C_4 \rme^{-\pi d N/\log(2 d N/\alpha)}.
\end{align*}
This completes the proof of~\cref{thm:DE-Sinc-collocation}.


%\section*{Conclusion}
This paper aims to enhance our understanding of the computational complexity of computing various Shapley value variants. We found that for various ML models --- including decision trees, regression tree ensembles, weighted automata, and linear regression --- both local and global interventional and baseline SHAP can be computed in polynomial time under HMM modeled distributions. This extends popular algorithms, such as TreeSHAP, beyond their empirical distributional scope. We also establish strict complexity gaps between the various SHAP variants (baseline, interventional, and conditional) and prove the intractability of computing SHAP for tree ensembles and neural networks in simplified scenarios. Overall, we present SHAP as a versatile framework whose complexity depends on four key factors: \begin{inparaenum}[(i)] \item model type, \item SHAP variant, \item distribution modeling approach, \item and local vs. global explanations\end{inparaenum}. We believe this perspective provides deeper insight into the computational complexity of SHAP, paving the way for future work.




%We believe that our framework provides a more intricate understanding of SHAP computation complexity across different models, distributions, and variants, paving the way for further research.

Our work opens promising directions for future research. First, expanding our computational analysis to other SHAP-related metrics, such as asymmetric SHAP~\citep{frye20} and SAGE~\citep{covert2020understanding}, would be valuable. Additionally, we aim to explore more expressive distribution classes and relaxed assumptions beyond those in Section \ref{sec:tractable} while maintaining tractable SHAP computation. Finally, when exact computation is intractable (Section \ref{sec:intractable}), investigating the approximability of SHAP metrics through approximation and parameterized complexity theory~\citep{downey2012parameterized} is an important direction.

%Our work opens several promising avenues for future research on the computational properties of explainable AI methods, with a particular focus on SHAP. First, it would be interesting to broaden the computational analysis conducted in this work to include other popular SHAP-related metrics in the literature, such as asymmetric SHAP \cite{frye20} and SAGE \cite{covert2020understanding}. Also, in the future, we aim to explore more expressive distribution classes and relaxed distributional assumptions—extending beyond those examined in Section \ref{sec:tractable} —that still yield tractable SHAP computation. Finally, when exact computation proves intractable (Section \ref{sec:intractable}), it is worthwhile to theoretically investigate the question of the approximability of computing the SHAP metrics across various configurations, through the lens of approximation and parametrized complexity theory \cite{arora2009computational}.

%This paper aims to deepen our understanding of the computational complexity involved in obtaining different Shapley value variants. We found that for a variety of ML models, including decision trees, tree ensembles for regression, weighted automata, and linear regression models — computing both local and global interventional and baseline SHAP can be done in polynomial time when distributions are modeled by HMMs. This extends the distributional scope of popular algorithms like TreeSHAP, which is limited to empirical distributions. Additionally, we demonstrate a strict complexity gap between SHAP variants, showing that interventional and baseline SHAP can be strictly easier to compute than conditional SHAP. Despite these positive results, we uncovered intractability for various SHAP variants in neural networks and tree ensembles. Finally, we provided generalized complexity relations across SHAP variants. We believe that our framework offers a deeper understanding of the complexity involved in computing SHAP across various variants, models, distributions, as well as in both local and global computations, laying the groundwork for future research.

%\section*{Acknowledgments}
%This work was partially supported by the
%JSPS Grant-in-Aid for Scientific Research (C)
%Number JP23K03218.

\bibliographystyle{siamplain}
\bibliography{SincCollocationVolterra2nd}
\end{document}
