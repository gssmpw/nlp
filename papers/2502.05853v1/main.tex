\documentclass[papersize,journal]{IEEEtran}
\usepackage{xcolor}
\usepackage{amsmath}
\usepackage{amsmath,amsfonts}
\usepackage{algorithmic}
\usepackage{algorithm}
\usepackage{array}
\usepackage{subcaption}
\usepackage[caption=false,font=normalsize,labelfont=sf,textfont=sf]{subfig}
\usepackage{textcomp}
\usepackage{stfloats}
\usepackage{url}
\usepackage{verbatim}
\usepackage{graphicx}
\usepackage{cite}
\usepackage{amsthm,amsmath,amssymb,lipsum}
\usepackage{mathrsfs}
\usepackage{makecell}
\usepackage{threeparttable}
\usepackage{amssymb}
\usepackage{multirow}
\usepackage{flushend}
\usepackage{amsmath,amsfonts}
%\usepackage{subfiger}
\usepackage{epstopdf}
\linespread{1} %改变行间距
%\nolinebreak 
%\hyphenation{op-tical net-works semi-conduc-tor IEEE-Xplore}
% updated with editorial comments 8/9/2021

\begin{document}
\title{Zak-Transform-Induced Optimal Sequences and Their Applications in OTFS}
\author{Xiuping Peng,\IEEEmembership{ Member, IEEE}, Congying Wu, Zilong Liu, \IEEEmembership{Senior Member, IEEE}, Chunlei Li, \IEEEmembership{Senior Member, IEEE}, Jianye Zhang\IEEEmembership{}, Pingzhi Fan, \IEEEmembership{Fellow, IEEE}

%\thanks{This work is supported in part by National Natural Science Foundation of China under Grant 61601401, in part by S\&T Program of HeBei under Grant 236Z0403G, and in part by Hebei Natural Science Foundation under Grant E2024203225. (Corresponding author: Xiuping Peng.)}

\thanks{Xiuping Peng, Congying Wu and Jianye Zhang are with the School of Information Science and Engineering, Yanshan University, Qinhuangdao, China, and also with Hebei Key Laboratory of Information Transmission and Signal Processing, Qinhuangdao, China. (e-mail: pengxp@ysu.edu.cn; wu1999109@163.com; 1403224729@qq.com)}

\thanks{Zilong Liu is with the School of Computer Science and Electronic Engineering, University of Essex, UK. (e-mail: zilong.liu@eaaex.ac.uk)}

\thanks{Chunlei Li is with the Department of Informatics,
University of Bergen, 5020 Bergen, Norway. (e-mail: chunlei.li@uib.no)}

\thanks{Pingzhi Fan is with the Information Coding \& Transmission Key Lab of Sichuan Province, CSNMT Int. Coop. Res. Centre (MoST), Southwest Jiaotong University, Chengdu 61175, China. (e-mail: p.fan@ieee.org)}}
 

%\markboth{IEEE TRANSACTIONS ON cOMMUNICATION, VOL. XXX, 2023}
%{Shell \MakeLowercase{\textit{et al.}}: Bare Demo of IEEEtran.cls for IEEE Journals}
%\maketitle
%\author{IEEE Publication Technology,~\IEEEmembership{Staff,~IEEE,}
        % <-this % stops a space
%\thanks{This paper was produced by the IEEE Publication Technology Group. They are in Piscataway, NJ.}% <-this % stops a space
%\thanks{Manuscript received April 19, 2021; revised August 16, 2021.}}

% The paper headers
\markboth{Journal of \LaTeX\ Class Files,~Vol.~14, No.~8, August~2024}%
{Shell \MakeLowercase{\textit{et al.}}: A Sample Article Using IEEEtran.cls for IEEE Journals}

%\IEEEpubid{0000--0000/00\$00.00~\copyright~2021 IEEE}
% Remember, if you use this you must call \IEEEpubidadjcol in the second
% column for its text to clear the IEEEpubid mark.

\maketitle

\begin{abstract}
This paper introduces a novel finite Zak transform (FZT)-aided framework for constructing multiple zero-correlation zone (ZCZ) sequence sets with optimal correlation properties. Specifically, each sequence is perfect with zero auto-correlation sidelobes, each ZCZ sequence set meets the Tang-Fan-Matsufuji bound with equality, and the maximum inter-set cross-correlation of multiple sequence sets meets the Sarwate bound with equality. Our study shows that these sequences can be sparsely expressed in the Zak domain through properly selected index and phase matrices. Particularly, it is found that the maximum inter-set cross-correlation beats the Sarwate bound if every index matrix is a circular Florentine array. Several construction methods of multiple ZCZ sequence sets are proposed, demonstrating both the optimality and high flexibility. {Additionally, it is shown that excellent synchronization performance can be achieved by the proposed sequences in orthogonal-time-frequency-space (OTFS) systems.}
\end{abstract}

\begin{IEEEkeywords}Perfect sequences, Zak transform, multiple ZCZ sequence sets,   Sarwate bound, cyclically distinct, inter-set cross-correlation.
\end{IEEEkeywords}

\section{Introduction}
\subsection{Background}
\IEEEPARstart{S}{equences} with good correlation properties are useful for a number of applications (e.g., synchronization, channel estimation, spread-spectrum communication, random access, ranging and positioning) in communication and radar systems. {To deal with  asynchronous wireless channels, perfect sequences with zero auto-corelation sides are preferred. However, perfect binary and quaternary sequences are only known to have lengths of 4 and 2, 4, 8, 16, respectively \cite{Luke2003}. There are polyphase perfect sequences of lengths $N=sm^2$ ($s,m$ positive integers), but as conjectured by Mow in \cite{Mow1996}, their minimum alphabet size is $2sm$ for even $s$ and odd $m$ and is $sm$ otherwise. Furthermore, constrained by the Sarwate bound \cite{Sarwate79}, it is not possible to have two or more perfect sequences with zero cross-correlation functions. }\par
As a remedy to the aforementioned problem, zero-correlation zone (ZCZ) sequences \cite{Fan99} have received tremendous research attention in the past decades. By definition, ZCZ sequences are characterized by their zero auto- and cross- correlation values for certain time-shifts around the in-phase position. Thanks to this property, ZCZ sequences permit an interference-free window, thus leading to improved multi-user detection or channel estimation performance in, for example, quasi-synchronous code-division multiple access (QS-CDMA) communications \cite{ref2,ref6} or multiple-antenna transmissions \cite{Yang02,Fragouli03,Fan04,Yang08,Zhang12}, respectively. \par
Formally, let us consider a ZCZ sequence set of length $N$, set size of $T$, the ZCZ width of $Z$. Such a set is featured by their zero periodic (non-trivial) auto- and cross-correlation functions for all the time-shifts in the range of $\left| \tau  \right| < Z$. The Tang-Fan-Matsufuji bound \cite{Tang00} shows that the parameters of a ZCZ sequence set should satisfy $T{Z} \le N$. A ZCZ sequence set is said to be \textit{optimal} if it meets this bound with equality. To support multi-cell QS-CDMA or multi-user MIMO communications, there is a strong need to design multiple ZCZ sequence sets having low inter-set cross-correlation with respect to the Sarwate bound \cite{Sarwate79} or the generalised Sarwate bound for the binary case \cite{Peng80}. %This paper mainly focuses on ZCZ sequences, and the following is a brief introduction.

\subsection{Related Works}
As a class of orthogonal design, a number of ZCZ constructions from various aspects have been developed. Typically, one can design ZCZ sequences from perfect sequences (e.g., generalized Chirp-like sequences) as illustrated in \cite{Torii04,Hayashi09,Popovic10}. Hu and Gong \cite{Hu10} presented a general construction of sequence families with zero or low correlation zones using interleaving techniques and Hadamard matrices. Besides, the research works of \cite{Deng00} and \cite{Appuswamy06} showed that complementary sequences \cite{Golay61,Tseng72,Liu13,Liu14} are an useful building component of ZCZ sequences. The algebraic connection between mutually orthogonal complementary sets and ZCZ sequences through generalized Reed-Muller codes was revealed in \cite{Liu14-ITW}. 

Designing multiple ZCZ sets with low inter-set cross-correlation \cite{Tang10} is a challenging task. \cite{Zhou17} pointed out that multiple ZCZ sequence sets with optimal inter-set cross-correlation can also be obtained by extending the method in \cite{Popovic10}. The resultant multiple ZCZ sequence sets have the following properties: 1) each sequence is perfect with zero auto-correlation sidelobes; 2) each ZCZ sequence set meets the Tang-Fan-Matsufuji bound with equality; and 3) the maximum inter-set cross-correlation of multiple sequence sets meets the Sarwate bound with equality. However, some of the ZCZ sequences obtained in \cite{Popovic10} may be cyclically equivalent, which is not desirable in practical applications \cite{Golomb2005}. To solve this problem, improved multiple ZCZ sets were obtained with the aid of perfect nonlinear functions \cite{Zhou17} and generalized bent functions \cite{Zhang20}. Recently, circular Florentine arrays were employed in \cite{Zhang22} for more ZCZ sequence sets compared to that in \cite{Zhou17},\cite{Zhang20}. The same combinatorial tool was used in \cite{Song21} for sequences with perfect auto-correlation and optimal cross-correlation.

\subsection{Motivations and Contributions}
Against the above state-of-the-art, this paper seeks a novel research angle for new optimal multiple ZCZ sequence sets. We advocate the use of an emerging tool, called finite Zak transform (FZT), which has found wide applications in mathematics, quantum mechanics, and signal analysis \cite{ref18,ref19,ref20}. A key advantage of FZT is that the sparse representation of sequences in Zak space enables efficient signal processing in radar, sonar, and communications \cite{ref21,ref22}, leading to reduced computational complexity as well as storage space at the receiver. Building upon FZT and its inverse, Brodzik derived sequences with perfect auto-correlation \cite{ref24} and all-zero cross-correlation \cite{ref25}. Recently, FZT was utilized in \cite{Peng24} for multiple spectrally-constrained sequence sets with optimal ZCZ and all-zero inter-set cross-correlation properties. Yet, the full potential of FZT for sequence design is largely unexplored. 

{From the application perspective, owing to the equivalence between the Zak domain and the delay-Doppler (DD) domain, FZT has inspired orthogonal-time-frequency-space (OTFS) modulation, which is a promising multicarrier waveform for future high-mobility communications \cite{6GV2X}. In the first version of OTFS, the basic idea is to send the data symbols in the DD domain (i.e., Zak domain), convert it to time-frequency (TF) domain through inverse sympletic finite Fourier transform (ISFFT), and then to time domain via Heisenberg transform \cite{Hadani2017,Viterbo2018}. Recently, it has been found that one can directly generate the relevant time-domain signal by applying the inverse FZT (IFZT) to the DD domain data \cite{Saif2022, Saif2023,Lampel2022}. Therefore, it is intriguing to transmit the proposed Zak-transform-induced sequences in the DD domain as preamble sequences and investigate their performances for random access \cite{Viterbo2020}, synchronization \cite{Cho2021, Farhang2022,Chung2024}, channel estimation \cite{Yogesh2024}, sensing \cite{Zegrar2024}, etc.} %In the OTFS literature, 

The contributions of this work are multifold:
\begin{itemize}
    \item We first introduce a novel framework whereby optimal multiple ZCZ sequence sets can be uniquely obtained by IFZT. To this end, we introduce a number of index matrices and phase matrices by advocating the sequence sparsity in the Zak domain. 
    \item We derive the admissible conditions of these index matrices and phase matrices and show that the maximum inter-set cross-correlation beats the Sarwate bound when every index matrix is a circular Florentine array. %New circular Florentine arrays are obtained applying permutation to existing ones. 
    %\item We discuss the superiority of the obtained optimal multiple ZCZ sequence sets by comparing with the existing ones in terms of the cyclic equivalence and availability of different parameters.
     \item {We employ the proposed sequences as preamble sequences in the DD domain and study their synchronization performance in OTFS. Our numerical simulation results demonstrate that 1) their superiority over random sequences in OTFS synchronization and 2) their excellent ambiguity function, highlighting their potential for sensing. }%our proposed ZCZ sequence sets derived from the Zak domain achieve optimal synchronization performance in OTFS system.
    %These methods solve the problem of cyclic equivalence, expand the application range of cyclic Florentine arrays, and provide new ideas for subsequent research.
\end{itemize}


\subsection{Organization of This Work}
The rest of the paper is organized as follows. Section II gives brief introductions to perfect sequences, FZT, and cyclic Florentine arrays. For optimal multiple ZCZ sequence sets, we introduce the main framework and derive the conditions for index and phase matrices in the Zak domain in Section III. In Section IV, several constructions are proposed based on IFZT and the cyclic Florentine arrays. The derived sequences are then applied to an OTFS system and evaluated for their synchronization performance. Section V concludes this paper.

\section{Preliminaries}
\subsection{Perfect Sequences}
\textit{Definition 1:} Let ${\textbf{s}_0}=\left( {{s_0\left( 0 \right)},...,{s_0\left( N-1 \right)}} \right)$ and ${\textbf{s}_1}=\left( {{s_1\left( 0 \right)},...,{s_1\left( N-1 \right)}} \right)$ be two sequences of  period $N$, then the periodic cross-correlation function (PCCF) between ${\textbf{s}_0}$ and ${\textbf{s}_1}$ is defined as 
\begin{equation}
{{\theta _{\textbf{s}_0,\textbf{s}_1}}\left( \tau  \right) = \sum\limits_{n = 0}^{N - 1} {s_0\left( n+\tau\right)_N{s_1^*}{{\left( n  \right)}}},} 
\end{equation}
where $0 \le \tau  < N$, ${\left(  \cdot  \right)_N}$ indicates the integer modulus of $N$  and
$s_1^*\left(n\right)$ is the complex conjugate of the complex number $s_1\left(n\right)$. When $\textbf{s}_0 = \textbf{s}_1$, ${\theta_{\textbf{s}_0}}$ is called the periodic auto-correlation function (PACF) of $\textbf{s}_0$. A sequence $\textbf{s}_0$ is said to be perfect if ${\theta _{\textbf{s}_0}}\left( \tau  \right)=0 $ for all $0 < \tau  < N$.

{\textit{Definition 2:} For two sequences ${\textbf{s}_0}$ and ${\textbf{s}_1}$ with period $N$, if there exist some $0 \le \tau  < N$ and a constant complex number $c$ with $|c|=1$ such that $s_1(t)=cs_0(t+\tau)$ for all $0 \le t  < N$ (i.e., $|R_{s_0,s_1}(\tau)|=N$), then the sequences ${\textbf{s}_0}$ and ${\textbf{s}_1}$ are said to be cyclically equivalent. Otherwise, they are said to be cyclically distinct.}


\textit{Definition 3:} Let $S = \left\{ {{\textbf{s}_u}} \right\}_{u= 0}^{T - 1}$ be a set of $T$ sequences of period $N$, where ${\textbf{s}_u}=\left( {{s_u\left( 0 \right)},...,{s_u\left( N-1 \right)}} \right)$ denotes the $u$-th constituent sequence of $S$. The maximum out-of-phase periodic auto-correlation magnitude ${\theta _a}$ and the maximum periodic cross-correlation magnitude ${\theta _c}$ of the sequence set $S$ are respectively defined by
\begin{equation*}
{\theta _a} = \max \left\{ {\left| {{\theta _{\textbf{s}_u}}\left( \tau  \right)} \right|:0 \le u< T,0 < \tau  < N} \right\},
\end{equation*}
and
\begin{equation*}
{\theta _c} = \max \left\{ {\left| {{\theta _{{{\rm{\textbf{s}}}_u}{\rm{,}}{{\rm{\textbf{s}}}_v}}}\left( \tau  \right)} \right|:0 \le u \ne v < T,0 \le \tau  < N} \right\}.
\end{equation*}

The following lemma is the well-known Sarwate bound on ${\theta _a}$ and ${\theta _c}$.

\textit{Lemma 1 \cite{Sarwate79}:} For any sequence set $S$ with $T$ sequences of period $N$, we have
\begin{equation}
\frac{{\theta _c^2}}{N} + \frac{{N - 1}}{{N\left( {T - 1} \right)}}\frac{{\theta _a^2}}{N} \ge 1.
\end{equation}

Lemma 1 demonstrates that it is impossible to obtain a sequence set with both $\theta _a$ and $\theta _c$ being zero. This implies that cross-correlation and nontrivial auto-correlation cannot be zero for all $\tau$. Fortunately, this problem can be addressed by placing $\tau$ in some regions around the origin, which facilitates the application and development of sequences with zero correlation zones \cite{Fan99}.

\textit{Definition 4:} The set $S$ is called an $\left( {N,T,{Z}} \right)$-ZCZ sequence set if 
\begin{equation*}
\begin{aligned}
{\theta _{\textbf{s}_u}}\left( \tau  \right) = 0,~&for ~ 0 \le u < T~and~0 < \tau  < {Z},\\
{\theta _{{\textbf{s}_u},{\textbf{s}_v}}}\left( \tau  \right) = 0, ~&for~0 \le u \ne v < T~ and~0 \le \tau  < {Z},
\end{aligned}
\end{equation*}
where ${Z}$ denotes the length of the ZCZ.

The following bound implies that there is a tradeoff among the parameters of any ZCZ sequence set.

\textit{Lemma 2 (Tang-Fan-Matsufuji bound \cite{Tang00}):} Let $S$ be a set of $T$ sequences of period $N$ with ZCZ length $Z$, then
\begin{equation*}
T{Z} \le N.
\end{equation*}

A ZCZ sequence set meeting the Tang-Fan-Matsufuji bound is said to be optimal.

\textit{Definition 5:} Let ${{\cal S}}$ be a family of $M$ sequence sets, each consisting of $T$ sequences of period $N$, i.e., ${{\cal S}} = \left\{ {{S^0},{S^1},...{S^{M - 1}}} \right\}$. A sequence set $S^m (0 \le m< M)$ is expressed as:
\begin{equation*}
{S^m} = \left\{ {{\textbf{s}}_u^m:{\textbf{s}}_u^m \buildrel \Delta \over = \left\{ {s_u^m\left( n \right)} \right\}_{n = 0}^{{N} - 1},0 \le u < T} \right\},
\end{equation*}
%where ${S^m}=\left\{ {{\textbf{s}_u^m}} \right\}_{u = 0}^{T - 1}$ and ${\textbf{s}_u^m}=\left( {{s_u^m\left( 0 \right)},...,{s_u^m\left( N-1 \right)}} \right)$ denotes the $u$-th sequence of the $m$-th set. 
The inter-set cross-correlation of ${{\cal S}}$ is defined as
 \[{\theta_c}\left( {{\cal S}} \right)= \max \left\{ {\left| {{\theta _{\textbf{s}_u^{{m_1}},\textbf{s}_v^{{m_2}}}}\left( \tau  \right)} \right|:0 \le \tau  < N }\right\}\]
where $0 \le {m_1} \ne {m_2} <M$ and $0 \le u, v <T$.

\subsection{The Zak Transform}
\textit{Definition 6 \cite{ref24}:} Let ${\textbf{s}}$ be a sequence of  period $N$. Suppose that $N = LT$, where $L$ and $T$ are positive integers. The FZT of ${\textbf{s}}$ is given by
\begin{equation}
\label{deqn_ex1}
{X}(j,t) = \sum\limits_{l = 0}^{L - 1} {s(t + lT)w_L^{-lj}},0 \le j < L, 0 \le t < T,
\end{equation}
where ${w_L} = {e^{\frac{{2\pi \sqrt { - 1} }}{L}}} = {e^{\frac{{2\pi i}}{L}}}$.

When a sequence of period $N=LT$ is represented as an $L\times T$ matrix $x$, its FZT domain ${X}$ can be rewritten as ${X}=W \cdot x$, where $W=w_L^{-lj}$ is the discrete fourier transform (DFT) matrix of order $L$. It is clear that the FZT reduces to the classic DFT when $T=1$.

Similarly to the DFT, the FZT is a one-to-one mapping. A signal  ${\textbf{s}}$ can be recovered by its ${X}(j,t)$ as
\begin{equation}
\label{deqn_ex2}
\textbf{s}(t + lT) = {L^{ - 1}}\sum\limits_{j = 0}^{L - 1} {{X}(j,t)w_L^{lj},}
\end{equation}
where $0 \le t < T$ and $0 \le l < L$.

Take ${X}$, ${Y}$ and ${Z}$ as the FZTs of $\textbf{s}_0$, $\textbf{s}_1$ and ${\theta _{\textbf{s}_0,\textbf{s}_1}}$, respectively. The Zak space correlation formula is given by
\begin{equation}
{Z}\left( {j,t} \right) =\sum\limits_{k = 0}^{T- 1} {{X}(j,k+t)Y^*(j,k).} 
\end{equation} 

Consequently, for a shift $\tau = \tau_1+\tau_2T$ where $0\leq \tau_1 <T$ and $0\leq \tau_2 < L$, we have
\begin{equation}
\begin{aligned}
&{\theta _{{\bf{s}}_0,{\bf{s}}_1}}\left( { {\tau_1} +  {\tau_2}T} \right)\\
=& {L^{ - 1}}\sum\limits_{j = 0}^{L - 1} \sum\limits_{t = 0}^{T - 1} {X\left( {j,t+\tau_1} \right)Y{^*}\left( {j,t} \right){w_L^{ {\tau_2}j}} } 
\end{aligned}
\end{equation}

%\textit{Definition 6:} Let ${X_{L,T}^{m}}$ be a matrix of $L \times T$ and $A$ be a matrix of $M \times T$, where $0 \le m <M$, $L=RT$, $T$ and $M$ are positive integers. Fix by ${A^m}= \left( {{i_0^m},{i_1^m},...,{i_{T-1}^m}}\right)$, where ${i_0^m},{i_1^m},...,{i_{T-1}^m} $ is any combination on ${\mathbb{Z}_T}$.  ${A^m}$ is the index set of rows with nonzero values for ${X_{r'T,T}^{m}}$, where $0\le r'<R$. Then $A=\left({{A^0}^{'},{A^1}^{'},...,{A^{ \left(M-1\right) }}^{'}}\right)^{'}$ is called the index matrix in Zak domain, where ${A^m}^{'}$ represents the transpose of ${A^m}$. %${A^m\left(t \right)}$ denotes the $t$-th element of the $m$-th vector of the index matrix $A$.

%\textit{Definition 7:} Let  ${X_{L,T}^{m,u}}$ be a matrix of $L \times T$  and $P^m$ be a matrix of $T \times L$, where $0\le m<M$, $0\le u<T$, $L$, $T$ and $M$ are positive integers. Fix by ${P^m_u} = \left( {{P^m_u}\left( 0 \right),...,{P^m_u}\left( {L - 1} \right)} \right)$, the values of the nonzero elements of  ${X_{r'T,T}^{m,u}}$ are determined by $P^m_u$, whose positions are defined by $A^m$. %containing the nonzero values of ${X_{L,T}^{m,u}}$, where $0 \le m<M, 0 \le u<T$. 
%Then $P^m = \left( {{P^m_0}^{'},{P^m_1}^{'},...,{P^m_{T - 1}}^{'}} \right)^{'}$ is said to be the phase matrix in Zak domain. % and ${P^m_u\left(t \right)}$ denotes the $t$-th element of the $u$-th vector of the phase matrix $P^m$.

\subsection{Circular Florentine Arrays}
The circular Florentine array has been studied since 1989 \cite{ref27,ref28,ref29}. The definition and some lemmas about circular Florentine arrays are introduced in the following.
% Additionally, most of the available examples are based on computer search results. Based on the existing circular Florentine arrays, Construction \uppercase\expandafter{\romannumeral1} expands the existing conditions of the circular Florentine arrays in this paper.

{\textit{Definition 7 \cite{ref30}:} An $M \times T$ circular Florentine array is an array of $T$ distinct symbols in $M$ circular rows  such that 
\begin{enumerate}
\item{each row is a permutation of the $T$ symbols and }
\item{for any pair of distinct symbols $\left( {s,t} \right)$ and for each $1 \le a \le T- 1$, there is at most one row in which $t$ occurs $a$ steps to the right of $s$.}
\end{enumerate}
}
\textit{Example 1:} An example of a $4\times15$ circular Florentine array is shown in (7).
 \begin{figure*}[ht] %hb代表放在文章底部,%ht为放在文章顶部 
 	\centering
 	\begin{equation}	
\left[ {
\setlength{\arraycolsep}{7pt}
\begin{array}{*{20}{c}}
0&1&2&3&4&5&6&7&8&9&{10}&{11}&{12}&{13}&{14}\\
0&7&1&8&2&{12}&3&{11}&9&4&{13}&5&{14}&6&{10}\\
0&4&{11}&7&{10}&1&{13}&9&5&8&3&6&2&{14}&{12}\\
0&{13}&7&2&{11}&6&{14}&{10}&3&5&{12}&9&1&4&8
\end{array}} \right].
\end{equation}
\end{figure*}

\textit{Lemma 3 }: For each positive integer $T \ge 2$, let ${F_c}\left( T \right)$ denote the largest integer such that circular Florentine array of order ${F_c}\left( T \right) \times T$ exists, then we have
\begin{enumerate}
\item{$s - 1 \le {F_c}\left( T \right) \le T - 1$, where $s$ is the smallest prime factor of $T$\cite{ref30};}
\item{${F_c}\left( T \right) =1$ when $T$ is even \cite{ref31};}
\item{${F_c}\left( T \right) =T-1$ when $T$ is prime \cite{ref30}; and}
\item{${F_c}\left( T \right) \le T-3$ when $T \equiv 15\bmod 18$ \cite{ref30}.}
\end{enumerate}

The following lemma of the circular Florentine array guarantees the cross-correlation properties of the sequence sets, which can be used in the proof of Lemma 8. 

\textit{Lemma 4 \cite{ref30}:} Let ${\mathbb{Z}_T} = \left\{ {0,1,...,T - 1} \right\}$, $F$ be an ${F_c}\left( T \right) \times T$ circular Florentine array on  ${\mathbb{Z}_T}$. Then each row is an arrangement on ${\mathbb{Z}_T}$, denoted as ${\pi _i}$, where $0 \le i < {F_c}\left( T \right)$. For $0 \le {i_1} \ne {i_2} \le{F_c}\left( T \right) $ and $z \in{\mathbb{Z}_T} $, there is exactly one solution for ${\pi _{{i_1}}}\left( t \right) = {\pi _{{i_2}}}\left( {t + z} \right)$ on ${\mathbb{Z}_T}$.

\section{Proposed Zak-Transform-Induced Multiple ZCZ Sequence Sets}
{In this section, we present a novel Zak-transform-induced framework for constructing multiple ZCZ sequence sets. Our proposed framework advocates the sparse representations of these sequences in the Zak domain. Our key idea is that sequences within a set exhibit identical non-zero support in the Zak domain, whilst distinct sets possess different non-zero supports. Following this idea, we first identify the Zak-domain non-zero positions of each set using an \textit{index matrix}. Subsequently, we assign the corresponding Zak-domain phase values to these non-zero positions, represented by a \textit{phase matrix}. In short, the proposed framework is comprised of three steps: 1) determining the Zak-domain non-zero positions using the index matrix; 2) assigning the Zak-domain phase values using the phase matrix; and 3) generating time-domain sequences of a set via the IFZT. To generate multiple good sequence sets, appropriate index matrices and phase matrices are needed.}

%This section presents a novel Zak-transform-based main framework for constructing $M$ sequence sets, each comprising $T$ sequences of period $N$, where $N=RT^2$.

{\textit{Main Framework: } Let $R$, $M$, $T$ and $L$ be positive integers. %And let $A$  be an $M \times T$ matrix over $\mathbb{Z}_T$, and $P^m$, where $0\leq m < M$, be a $T \times L$ matrix. As explained later, $A$ and $P^m$ will be used as the index matrix and phase matrix, respectively, in the construction. 
$M$ multiple sequence sets ${{\cal S}} = \left\{ {S^m} \right\}_{m = 0}^{M - 1}$, each comprising $T$ sequences of period $N$, where $N=RT^2=LT$, are constructed by following the steps below.
\begin{enumerate}
\item{Let $A$ be an index matrix, which is a $M \times T$ matrix over $\mathbb{Z}_T$. %and $A$  be used as the index matrix. 
$A^m\left( t \right)$ denotes the value of the $t$-th element in the $m$-th row of the index matrix $A$. The row vector $A^m$ for $0\leq m <M$ corresponds to the sequence set $S^m$. %The number of rows $M$ for matrix $A$ corresponds to the number of sequence sets in family $\cal S$. 
The non-zero support of the sequence set $S^m$ in the Zak domain is given by $\left| {X^{m}}(j,t) \right|$ as follows: 
\begin{equation}
\begin{aligned}
\left| {{X^{m}}\left( {j,t} \right)} \right| = \begin{cases}T\sqrt R , &j =A^m\left( t \right) + {r}T,\\ 0, & { \text{otherwise} }\end{cases}
\end{aligned}
\end{equation}
where $0 \le j < L,0 \le t< T$ and $0\le r<R$. }
\item{Let $P^m (0\le m<M)$ be a phase matrix, which is a $T \times L$ matrix. ${P_u^m}\left( t \right)$ denotes the value of the $t$-th element in the $u$-th row of the phase matrix $P^m$. The row vector $P_u^m$ corresponds to the sequence ${\textbf{s}}_u^m$ in $S^m$. %The number of rows $T$ for the matrix $P^m$ corresponds to the number of sequences in the $S^m$. 
The sequence ${\textbf{s}}_u^m$ in the Zak domain is given by ${X_{u}^{m}}\left( {j,t} \right)$ as follows:}
\begin{equation}
\begin{aligned}
{{X_{u}^{m}}\left( {j,t} \right)} = \begin{cases}T\sqrt R {P^m_u}\left( {t + {r}T} \right), &j = A^m\left( t \right)  +{r}T\\ 0, &{\text{otherwise.} }\end{cases}
\end{aligned}
\end{equation}
\item{For each $0 \le m < M$, according to the IFZT,  the sequence $s_u^m(n)$ in $S^m$ is obtained by
\begin{equation}
\begin{aligned}
&s_u^m\left( n \right)= s_u^m\left( {t + lT} \right) \\
= &{L^{ - 1}}T\sqrt R \sum\limits_{{r} = 0}^{R-1} {P_u^m\left( {t + {r}T} \right)w_L^{ l\left( {A^m\left( t \right) + {r}T} \right)}},
\end{aligned}
\end{equation}
where $0 \le u<T$ and $0 \le l < L$.}
\end{enumerate}
{To demonstrate the construction of multiple sequence sets using the index matrix $A$ and the phase matrices $P^m (0\le m<M)$, an example of the proposed Main Framework is provided for the case where $T=5$, $M=R=2$ and $L=RT=10$.}
}
\textit{Example 2:} $A$ is a $2\times 5$ index matrix, which is expressed as:
\[A= \left[ {\begin{array}{*{20}{c}}
0&1&2&4&3\\
0&3&4&2&1\\
\end{array}} \right].\]

 \begin{figure*}[ht] %hb代表放在文章底部,%ht为放在文章顶部 
 	\centering
\begin{subequations}
\begin{align}
P^0 &= \left[ {\begin{array}{*{20}{c}}
{w_{10}^0}&{w_{10}^0}&{w_{10}^0}&{w_{10}^0}&{w_{10}^0}&{w_{10}^0}&{w_{10}^0}&{w_{10}^0}&{w_{10}^0}&{w_{10}^0}\\
{w_{10}^0}&{w_{10}^1}&{w_{10}^2}&{w_{10}^3}&{w_{10}^4}&{w_{10}^5}&{w_{10}^6}&{w_{10}^7}&{w_{10}^8}&{w_{10}^9}\\
{w_{10}^0}&{w_{10}^2}&{w_{10}^4}&{w_{10}^6}&{w_{10}^8}&{w_{10}^0}&{w_{10}^2}&{w_{10}^4}&{w_{10}^6}&{w_{10}^8}\\
{w_{10}^0}&{w_{10}^3}&{w_{10}^6}&{w_{10}^9}&{w_{10}^2}&{w_{10}^5}&{w_{10}^8}&{w_{10}^1}&{w_{10}^4}&{w_{10}^7}\\
{w_{10}^0}&{w_{10}^4}&{w_{10}^8}&{w_{10}^2}&{w_{10}^6}&{w_{10}^0}&{w_{10}^4}&{w_{10}^8}&{w_{10}^2}&{w_{10}^6}
\end{array}} \right], \tag{\theequation a}\\
P^1 &= \left[ {\begin{array}{*{20}{c}}
{w_{10}^0}&{w_{10}^5}&{w_{10}^0}&{w_{10}^5}&{w_{10}^0}&{w_{10}^5}&{w_{10}^0}&{w_{10}^5}&{w_{10}^0}&{w_{10}^5}\\
{w_{10}^0}&{w_{10}^6}&{w_{10}^2}&{w_{10}^8}&{w_{10}^4}&{w_{10}^0}&{w_{10}^6}&{w_{10}^2}&{w_{10}^8}&{w_{10}^4}\\
{w_{10}^0}&{w_{10}^7}&{w_{10}^4}&{w_{10}^1}&{w_{10}^8}&{w_{10}^5}&{w_{10}^2}&{w_{10}^9}&{w_{10}^6}&{w_{10}^3}\\
{w_{10}^0}&{w_{10}^8}&{w_{10}^6}&{w_{10}^4}&{w_{10}^2}&{w_{10}^0}&{w_{10}^8}&{w_{10}^6}&{w_{10}^4}&{w_{10}^2}\\
{w_{10}^0}&{w_{10}^9}&{w_{10}^8}&{w_{10}^7}&{w_{10}^6}&{w_{10}^5}&{w_{10}^4}&{w_{10}^3}&{w_{10}^2}&{w_{10}^1}
\end{array}} \right]. \tag{\theequation b}
\end{align}
\end{subequations}
 \end{figure*}
Let $P^0$ and $P^1$ be the phase matrices, as shown in (11a) and (11b), respectively. Through (8) and (9), we can get ${{X_{1}^{0}}\left( {j,t} \right)}$ and ${{X_{3}^{1}}\left( {j,t} \right)}$, i.e.,
\[X_{1}^{0}= 5\sqrt2\left[ {
\begin{array}{*{20}{c}}
{w_{10}^0}&0&0&0&0\\
0&{w_{10}^1}&0&0&0\\
0&0&{w_{10}^2}&0&0\\
0&0&0&0&{w_{10}^4}\\
0&0&0&{w_{10}^3}&0\\
{w_{10}^5}&0&0&0&0\\
0&{w_{10}^6}&0&0&0\\
0&0&{w_{10}^7}&0&0\\
0&0&0&0&{w_{10}^9}\\
0&0&0&{w_{10}^8}&0
\end{array}} \right],\]
\[X_{3}^{1}=  5\sqrt2\left[ {
%\setlength{\arraycolsep}{3pt}
\begin{array}{*{20}{c}}
{w_{10}^0}&0&0&0&0\\
0&0&0&0&{w_{10}^2}\\
0&0&0&{w_{10}^4}&0\\
0&{w_{10}^8}&0&0&0\\
0&0&{w_{10}^6}&0&0\\
{w_{10}^0}&0&0&0&0\\
0&0&0&0&{w_{10}^2}\\
0&0&0&{w_{10}^4}&0\\
0&{w_{10}^8}&0&0&0\\
0&0&{w_{10}^6}&0&0
\end{array}} \right].\]

%Based on the Main Framework, we aim to construct multiple sequence sets with the following properties:
{In this paper, the proposed Main Framework will be employed to construct multiple sequence sets with the following desired properties:}
\begin{enumerate}
\item{Each sequence is a perfect unimodular sequence;}
\item{Each $S^m$ is an optimal ZCZ sequence set with respect to the Tang-Fan-Matsufuji bound;}
\item{The family of sequence set ${{\cal S}}$ has low inter-set cross-correlation, namely, the maximal inter-set cross-correlation of multiple sequence sets achieves the well-known Sarwate bound;}
\item{All sequences in each $S^m$ are cyclically distinct.}
\end{enumerate}

{The above analysis has revealed that the key to constructing multiple sequence sets with the aforementioned properties is the design of appropriate index matrix $A$ and its corresponding phase matrices $P^m(0\le m<M-1)$. To proceed, we first introduce the necessary conditions that such matrices must satisfy. These conditions play a pivotal role in the construction of multiple sequence sets with desirable properties, which will be detailed in Section IV.}
%The previous description and analysis reveal that a critical step in constructing multiple sequence sets within the Main Framework is to design a suitable index matrix $A$ and its corresponding phase matrices $P^m(m=0,1,...,M-1)$. To proceed, the index matrix and phase matrices in the Zak domain need to meet certain constraints. In the following, four lemmas will be provided which are crucially important for the subsequent constructions.

\textit{Lemma 5:} Let $\textbf{s}^m_u$ be a sequence of  period $N$, where $N=LT$ and $ L=RT$. The sequence $\textbf{s}^m_u$ is unimodular if the phase vector $P_u^m$ of $\textbf{s}^m_u$ in Zak domain satisfies the following condition: 
\begin{equation}
\left| {\sum\limits_{{r} = 0}^{R - 1} {{P^m_u}\left( {t + {r}T} \right)w_L^{  lrT}} } \right| = \sqrt R 
\end{equation}
for all $0 \le t<T$ and $0 \le l <L$.  

\textit{Proof:} To ensure that $\left| {s_u^m\left( n \right)} \right|=1$ holds, from (10), the time-domain expression of $\textbf{s}^m_u$ is
\begin{equation*}
\begin{aligned}
&s_u^m\left( n \right)= s_u^m\left( {t + lT} \right) \\
= &{L^{ - 1}}T\sqrt R \sum\limits_{{r} = 0}^{R-1} {{P^m_u}\left( {t + {r}T} \right)w_L^{  l\left( {A^m\left( t \right) + {r}T} \right)}} \\
 = &{L^{ - 1}}T\sqrt R w_L^{  lA^m\left( t \right)}\sum\limits_{{r} = 0}^{R-1} {{P^m_u}\left( {t + {r}T} \right)w_L^{  l{r}T}} .
\end{aligned}
\end{equation*}
Then $\left| {s_u^m\left( n \right)} \right|=\frac{1}{{\sqrt R }}\left| {\sum\limits_{{r} = 0}^{R-1} {{P^m_u}\left( {t + {r}T} \right)w_L^{  l{r}T}}} \right|$. Since (12) holds for all $0 \le t<T$ and $0 \le l <L$, it follows that each sequence $s_u^m$ is unimodular.
\hfill
$\hfill\blacksquare$ 

\textit{Lemma 6:} Let $\textbf{s}^m_u$ be a sequence of period $N$, where $N=LT$ and $ L=RT$. The sequence $\textbf{s}^m_u$ is perfect if its phase vector $P^m_u$ and index vector $A^m$ in the Zak domain satisfy the following conditions:
\begin{enumerate}
\item{The index vector $A^m$ is a permutation of ${\mathbb{Z}_T}$.}
\item{The phase vector $P^m_u$ is unimodular. }
\end{enumerate}

\textit{Proof:} To ensure that ${\theta _{{\bf{s}}_u^m}}\left( 0 \right) =N$ and ${\theta _{{\bf{s}}_u^m}}\left( \tau  \right) =0$ for all $0 < \tau <N$,  according to (6), the auto-correlation of $\textbf{s}^m_u$ is
\begin{equation}
\begin{aligned}
&{\theta _{{\bf{s}}_u^m}}\left( \tau  \right)={\theta _{{\bf{s}}_u^m}}\left( {{\tau_1} + {\tau_2}T} \right)\\
=&L^{-1}\sum\limits_{j \in {\mathbb{Z}_L}} {w_L^{   {\tau_2}j}\sum\limits_{t \in {\mathbb{Z}_T}} {X_{u}^{m}\left( {j,t+{\tau_1}} \right)X{_{u}^{{m}*}\left( {j,t} \right)} } }
\end{aligned}
\end{equation}
where $0 \le {\tau_1}<T$ and $0 \le {\tau_2} <L$.

When ${\tau_1} \ne0$, since $A^m$ is a permutation of ${\mathbb{Z}_T}$, we get 
\[\sum\limits_{t \in {\mathbb{Z}_T}}{X_{u}^{m}\left( {j,t+{\tau_1}} \right)X{{_{u}^{m}}^*}\left( {j,t} \right)}  = 0.\]
Therefore ${\theta _{{\bf{s}}_u^m}}\left( \tau  \right) = 0$.

When ${\tau_1}=0, {\tau_2}\ne 0$, (13) becomes
\begin{equation*}
\begin{aligned}
&{\theta _{{\bf{s}}_u^m}}\left( { {\tau_2}T} \right) \\
=& L^{-1}\sum\limits_{j \in {\mathbb{Z}_L}} {w_L^{   {\tau_2}j}} \sum\limits_{t \in {\mathbb{Z}_T}} {X_{u}^{m}\left( {j,t} \right)X{{_{u}^{m}}^*}\left( {j,t} \right)} \\
 =& T\sum\limits_{{r} = 0}^{R - 1} {\sum\limits_{t = 0}^{T - 1} {w_L^{   {\tau_2}\left( {{r}T + A^m\left( t\right)} \right)}{P^m_u}\left( {t + {r}T} \right)P_u^{m*}\left( {t + {r}T} \right)} }. %\\
\end{aligned}
\end{equation*}

Since $A^m$ is a permutation of $\mathbb{Z}_T$, $ A^m\left( t \right) +{r}T$ for $0 \le r<R$ is a permutation of $\mathbb{Z}_L$. Then we have $\sum\limits_{{r} = 0}^{R - 1}\sum\limits_{t = 0}^{T - 1} {w_L^{   {\tau_2}\left( {{r}T + A^m\left( t\right)} \right)}}=0$. Therefore, we can obtain ${\theta _{{\bf{s}}_u^m}}\left( \tau  \right) =0$ for ${\tau_1}=0$ and $0 <{\tau_2}<L$.

When ${\tau_1}=0,  {\tau_2}=0$, (13) becomes
\begin{equation*}
\begin{aligned}
&{\theta _{{\bf{s}}_u^m}}\left( 0\right) \\
 = &T\sum\limits_{{r} = 0}^{R - 1} {\sum\limits_{t = 0}^{T - 1} {w_L^{  0\left( {{r}T + A^m\left( t\right)} \right)}{P^m_u}\left( {t + {r}T} \right)P_u^{m*}\left( {t + {r}T} \right)} } \\
 = &T\sum\limits_{{r} = 0}^{R - 1} {\sum\limits_{t = 0}^{T - 1} {{{\left| {{P^m_u}\left( {t + {r}T} \right)} \right|}^2}} }.
\end{aligned}
\end{equation*}

Since the phase vector $P^m_u$ is unimodular, we can get ${\theta _{{\bf{s}}_u^m}}\left( 0\right) =RT^2=N$. %Therefore we can obtain the sequence ${\bf{c}}_u^m$ is perfect.
\hfill
$\hfill\blacksquare$ 

\textit{Lemma 7:} Let $S^m$ be a set of $T$ sequences of period $N$, where $N=LT$ and $L=RT$. Suppose the phase vector $P^m_u$ and the index vector $A^m$ of the sequence set $S^m$ in the Zak domain satisfy the following conditions:
\begin{enumerate}
\item{The index vector $A^m$ is a permutation of ${\mathbb{Z}_T}$},
\item{$\left| {\sum\limits_{{r} = 0}^{R - 1} {\sum\limits_{t = 0}^{T - 1} {w_L^{  {\tau_2}\left( {A^m\left( t \right) + {r}T} \right)} \cal P} } } \right|=0$, where $0\le  {\tau_2} <R$},
\item{
$0<\left| {\sum\limits_{{r} = 0}^{R - 1} {\sum\limits_{t = 0}^{T - 1} { w_T^{  A^m\left( t \right) } \cal P} }}  \right|<RT,$}
\end{enumerate}
where $\mathcal{P} = P_u^m\left( {t + {r}T} \right)P_v^{m*}\left( {t + {r}T} \right)$.
Then the sequence set $S^m$ satisfies the Tang-Fan-Matsufuji bound and each sequence is cyclically distinct.

\textit{Proof:} Let ${\textbf{s}_u^m}$ and ${\textbf{s}_v^m}$ be any two sequences in ${S^m}$, $0 \le u\ne v < T$ and $0 \le m <M$. In order to ensure that the length of the ZCZ meets Tang-Fan-Matsufuji bound, we need to show $\left| {{\theta _{{\bf{s}}_u^m,{\bf{s}}_v^m}}\left( \tau  \right)} \right|=0$ for $0 \le \tau<RT$. 

Meanwhile, to further ensure that all the sequences are pairwise cyclically distinct, it is sufficient to guarantee $0 < \left| {{\theta _{{\bf{s}}_u^m,{\bf{s}}_v^m}}\left( RT \right)} \right|<N$. {There are two reasons for this constraint:
(1) If $ \left| {{\theta _{{\bf{s}}_u^m,{\bf{s}}_v^m}}\left( RT \right)} \right|=N$, the resulting sequence set must contain equivalent sequences. %This is because both $s_u^m$ and $s_v^m$ are perfect sequences.
(2) If $ \left| {{\theta _{{\bf{s}}_u^m,{\bf{s}}_v^m}}\left( RT \right)} \right|=0$, then the ZCZ width is $Z=RT+1$, which violates the Tang-Fan-Matsufuji bound.}

According to (6), the cross-correlation of ${\textbf{s}_u^m}$ and ${\textbf{s}_v^m}$ is
\begin{equation}
\begin{aligned}
&{\theta _{{\bf{s}}_u^m,{\bf{s}}_v^m}}\left( \tau  \right) ={\theta _{{\bf{s}}_u^m,{\bf{s}}_v^m}}\left( { {\tau_1} +  {\tau_2}T} \right)\\
=& L^{-1}\sum\limits_{j \in {\mathbb{Z}_L}} {w_L^{   {\tau_2}j}\sum\limits_{t \in {\mathbb{Z}_T}} {X_{u}^{m}\left( {j,t+ {\tau_1}} \right)X{{_{v}^{m}}^*}\left( {j,t} \right)} }, 
\end{aligned}
\end{equation}
where $0 \le  {\tau_1}<T$ and $0 \le {\tau_2} <L$.

When ${\tau_1}\ne0$, since $A^m$ is a permutation of ${\mathbb{Z}_T}$, we can get 
\[\sum\limits_{t \in {\mathbb{Z}_T}}{X_{u}^{m}\left( {j,t+ {\tau_1}} \right)X{{_{v}^{m}}^*}\left( {j,t} \right)}  = 0.\]
Therefore, $\left| {{\theta _{{\bf{s}}_u^m,{\bf{s}}_v^m}}\left( \tau  \right)} \right| = 0$.

When ${\tau_1}=0$, we have the following two cases.

Case 1: When $0 \le  {\tau_2} <R$, (14) becomes
\begin{equation*}
\begin{aligned}
&{\theta _{{\bf{s}}_u^m,{\bf{s}}_v^m}}\left( { {\tau_2}T} \right) \\
=& L^{-1}\sum\limits_{j \in {\mathbb{Z}_L}} {w_L^{   {\tau_2}j}\sum\limits_{t \in {\mathbb{Z}_T}} {X_{u}^{m}\left( {j,t} \right)X{{_{v}^{m}}^*}\left( {j,t} \right)} } \\
 =& T\sum\limits_{{r} = 0}^{R - 1} {\sum\limits_{t = 0}^{T - 1} {w_L^{   {\tau_2}\left( {A^m\left( t\right) + {r}T} \right)}\mathcal{P}} },
\end{aligned}
\end{equation*}
where  ${\mathcal{P}= P_u^m\left( {t + {r}T} \right)P_v^{m*}\left( {t + {r}T} \right)}$.

Given that
\begin{equation}
\left| {\sum\limits_{{r} = 0}^{R - 1} {\sum\limits_{t = 0}^{T - 1} {w_L^{  {\tau_2}\left( {A^m\left( t \right) + {r}T} \right)}\cal P} } } \right|=0,
\end{equation}
 it follows that $\left| {\theta _{{\bf{s}}_u^m,{\bf{s}}_v^m}}\left( {{\tau_2}T} \right) \right| =0$ for all $0 \le {\tau_2} <R$.

Case 2: When ${\tau_2}=R$, (14) becomes
\begin{equation*}
\begin{aligned}
&{\theta _{{\bf{s}}_u^m,{\bf{s}}_v^m}}\left( L \right)\\
 =&L^{-1}\sum\limits_{j \in {\mathbb{Z}_L}} {w_L^{  Rj}\sum\limits_{t \in {\mathbb{Z}_T}} {X_{u}^{m}\left( {j,t} \right)X{{_{v}^{m}}^*}\left( {j,t} \right)} } \\
 = &T\sum\limits_{{r} = 0}^{R - 1} {\sum\limits_{t = 0}^{T - 1} {w_T^{  \left( {A^m\left( t \right) + {r}T} \right)}\mathcal{P}} }\\
= &T\sum\limits_{{r} = 0}^{R - 1} {\sum\limits_{t = 0}^{T - 1} {w_T^{  {A^m\left( t \right)} }\mathcal{P}} },
\end{aligned}
\end{equation*}
where  $\mathcal{P} = P_u^m\left( {t + {r}T} \right)P_v^{m*}\left( {t + {r}T} \right)$.

Given that
\begin{equation}
\begin{aligned}
0<\left| {\sum\limits_{{r} = 0}^{R - 1} {\sum\limits_{t = 0}^{T - 1} { w_T^{  A^m\left( t \right) }\cal {P} } }}  \right|<RT,
\end{aligned}
\end{equation}
 it follows that $0<\left| {{\theta _{{\bf{s}}_u^m,{\bf{s}}_v^m}}\left( {{\tau_2}T} \right)} \right|<N$ for $ {\tau_2}=R$.

Our observations from the two cases demonstrate that the sequence set $S^m$ adheres to the Tang-Fan-Matsufuji bound, while also exhibiting the property of cyclic distinctness for each individual sequence.
%Observing the two cases above, we conclude that the sequence set $C^m$ satisfies the Tang-Fan-Matsufuji bound and each sequence is cyclically distinct.
\hfill
$\hfill\blacksquare$ 

\textit{Lemma 8:} Consider a family ${\cal S}$ containing $M$ sequence sets, denoted by $S^{m} \left(0 \le {m} < M\right) $, where each set comprises $T$ sequences of period $N$. Here, $N=LT$ and $L=RT$. Let $P^{m_1}$ and $P^{m_2}$ represent the phase matrices, and $A^{m_1}$ and $A^{m_2}$ denote the index vectors associated with sets $S^{m_1}$ and $S^{m_2}$ respectively, where $0 \le {m_1} \ne {m_2} <M$. Then the maximum inter-set cross-correlation of ${{\cal S}}$ attains the Sarwate bound if the following two conditions are met:
%Let ${{\cal C}} = \left\{ {{C^0},{C^1},...{C^{M - 1}}} \right\}$ be a family of $M$ sequence sets {\cal C}, each consisting of $T$ sequences of period $N$, where $N=LT$ and $L=RT$. Let $P^{m_1}$, $P^{m_2}$ be the phase matrices and $A^{m_1}$, $A^{m_2}$ be the index vectors of sequence sets $C^{m_1}$, $C^{m_2}$, respectively, where $0 \le {m_1} \ne {m_2} <M$. Then the maximum inter-set cross-correlation of ${{\cal C}}$ achieves the Sarwate bound if the following two conditions hold.
\begin{enumerate}
\item{The index matrix $A$ is a circular Florentine array}.
\item{$\left| {\sum\limits_{{r} = 0}^{R - 1} {P_u^{{m_1}}\left( {t +{\tau _1}+{r}T} \right){P_v^{{m_2}*}\left( {t+ {r}T} \right)w_R^{ {r}{\tau _2}}} } }\right| = \sqrt R,$ }
\end{enumerate}
where $0 \le t,\tau _1 < T$, $0 \le u \ne v<T$ and $0 \le \tau _2<L$.

\textit{Proof:} Let ${{\bf{s}}_u^{{m_1}}}$ and ${{\bf{s}}_v^{{m_2}}}$ be any two sequences in $S^{m_1}$ and $S^{m_2}$, respectively. The inter-set cross-correlation between ${{\bf{s}}_u^{{m_1}}}$ and ${{\bf{s}}_v^{{m_2}}}$ is given by
%\begin{displaymath}
\begin{equation*}
\label{e:barwq}
\begin{aligned}
&{\theta _{{\bf{s}}_u^{{m_1}},{\bf{s}}_v^{{m_2}}}}\left( \tau  \right) = \sum\limits_{n = 0}^{N - 1} {s_u^{{m_1}}\left( {n+ \tau } \right)_Ns{{_v^{{m_2}}}^*}\left( n  \right)} \\
=&\frac{1}{R}\sum\limits_{t = 0}^{T - 1} {\sum\limits_{l = 0}^{L- 1}} \left( {\sum\limits_{{r} = 0}^{R - 1} {P_v^{{m_2}*}\left( {t + {r}T} \right)w_L^{-l\left( {A^{{m_2}}\left( {t } \right) + {r}T} \right)}}} \right. \\
& \left. \cdot \sum\limits_{{r} = 0}^{R - 1} {P_u^{{m_1}}\left( {t + {\tau _1} + {r}T} \right)w_L^{(l+{\tau _2})\left( {A^{{m_1}}\left( t+{\tau _1} \right) + {r}T} \right)}} \right)\\
 =& \frac{1}{R}\sum\limits_{t = 0}^{T - 1} {w_L^{  \tau _2A^{{m_1}}\left( {t + {\tau _1}} \right)}\sum\limits_{l = 0}^{L - 1} {w_L^{l\left(   A^{{m_1}}\left( {t + {\tau _1}}\right)-{A^{{m_2}}\left( t\right) } \right)}} } \\
&{\sum\limits_{{r} = 0}^{R - 1} {P_u^{{m_1}}\left( {t +{\tau _1}+{r}T} \right){P_v^{{m_2}*}\left( {t+ {r}T} \right)w_R^{ {r}{\tau _2}}} } },
%&\left( {\sum\limits_{{r} = 0}^{R - 1} {P_u^{{m_1}}\left( {t + {\tau _1} + {r}T} \right)w_R^{ - %{r}\left( {r + {\tau _2}} \right)}} }  \right),
\end{aligned}
\end{equation*}
%\end{displaymath}
where $n = t + lT$, $\tau  = {\tau _1} + {\tau _2}T$, $0 \le t,{\tau _1} < T$ and $0 \le l,{\tau _2} < L$.

Since
\begin{equation}
\left| {\sum\limits_{{r} = 0}^{R - 1} {P_u^{{m_1}}\left( {t +{\tau _1}+ {r}T} \right){P_v^{{m_2}*}\left( {t + {r}T} \right)w_R^{ {r}{\tau _2}}} }} \right|= \sqrt R,
\end{equation}
the absolute value of the above equation of ${\theta _{{\bf{s}}_u^{{m_1}},{\bf{s}}_v^{{m_2}}}}\left( \tau  \right) $ is 
\begin{equation}
\begin{aligned}
&\left|{\theta _{{\bf{s}}_u^{{m_1}},{\bf{s}}_v^{{m_2}}}}\left( \tau  \right)  \right|\\
=&\frac{\sqrt R}{R}\left|\sum\limits_{t = 0}^{T - 1} {w_L^{  \tau _2A^{{m_1}}\left( {t + {\tau _1}} \right)}\sum\limits_{l = 0}^{L - 1} {w_L^{l\left(   A^{{m_1}}\left( {t + {\tau _1}}\right)-{A^{{m_2}}\left( t\right) } \right)}} } \right| .
\end{aligned}
\end{equation}

Lemma 4 implies that for a circular Florentine array, $A$, there exists a unique integer $t$, denoted $t'$, such that the following equality holds for all possible values of $\tau_1$ within the range $0 \le {\tau_1} <T$:
%Since $A$ is a circular Florentine array, from Lemma 4, it follows that only one $t$, i.e. $t=t'$ can guarantee that 
\[{A^{{m_2}}\left({ t'} \right) = A^{{m_1}}\left( t'+ {\tau _1}  \right)}.\]
 %is true for all $0 \le {\tau_1} <T$. 
 Thus, we have
\[\sum\limits_{l = 0}^{L - 1} {w_L^{l\left( A^{{m_1}}\left( {t'+ {\tau _1} } \right) - {A^{{m_2}}\left({ t' }\right)} \right)}} =\sum\limits_{l = 0}^{L - 1} {1}=L.\]
For distinct values of $t$, i.e. $t \ne t'$, there is no assurance that ${A^{{m_2}}\left( {t } \right)=A^{{m_1}}\left(t+ {\tau _1} \right)}$. In such cases, where ${A^{{m_2}}\left( {t } \right)-A^{{m_1}}\left( t+ {\tau _1} \right)}=b \ne 0$, we obtain
\[\sum\limits_{l = 0}^{L - 1}{w_L^{bl}}=0.\]
Hence, (18) becomes as
\begin{equation*}
\begin{aligned}
&\left|{\theta _{{\bf{s}}_u^{{m_1}},{\bf{s}}_v^{{m_2}}}}\left( \tau  \right)  \right|\\
=&\frac{\sqrt R}{R}{{\left| {w_L^{  {\tau _2}{A^{{m_1}}}\left( {t' + {\tau _1}} \right)}} \right|}L+\frac{\sqrt R}{R} \left| { \sum_{ \substack{t = 0\\t \ne t'}}^{T - 1} {w_L^{  {\tau _2}{A^{{m_1}\left( {t + {\tau _1}} \right)}}} \cdot 0 } }\right|} \\
 =& T\sqrt R.
\end{aligned}
\end{equation*}
%Furthermore, from the equation:
%we can infer that $\left| {{\theta _{{\bf{s}}_u^{{m_1}},{\bf{s}}_v^{{m_2}}}}\left( \tau  \right) } \right|=\sqrt RT$. 
Therefore, in light of the preceding discussion and Lemma 1, we can definitively conclude that the maximum inter-set cross-correlation of ${{\cal S}}$ achieves the Sarwate bound.
\hfill
$\hfill\blacksquare$ 
\section{Sets Of Perfect Sequences With Optimal Correlation}
\subsection{The Proposed Constructions}
Leveraging Lemma 8, we identify that the index matrix possesses the structure of a circular Florentine array. However, current methods for obtaining these arrays primarily rely on computational search techniques. To address this limitation and expand the pool of index matrices available for our constructions, we propose an extension method for circular Florentine arrays. This method offers a more flexible permutation set, facilitating the subsequent construction of multiple optimal ZCZ sequence sets with optimal interset correlation.

\textit{Construction \uppercase\expandafter{\romannumeral1}:} Let $F$ be an ${F_c}\left( T \right) \times T$ circular Florentine array. We denote the $t$-th element in the $m$-th row of $F$ by ${F_m}\left( t \right)$, where ${F_m}\left( t \right)={\left(  t+b \right)_T}$, $0 \le m < {F_c}\left( T \right)$ and $0 \le t,b<T$. We achieve this construction in two steps:
\begin{enumerate}
\item{Choose the first row of $F$, denoted as $F_0$. By rearranging the last $T-2$ elements of ${F_0}$, we obtain a new permutation, denoted as ${F_0^q}$, where $0< q<\left(T-2 \right)!$. 
}
\item{For the $m'$-th row of ${F^q}$ (where $1 \le m' < {F_c}\left( T \right)$), denoted as ${F_{m'}^q}$, the elements can be obtained by
\begin{equation*}
F_{m'}^q\left( t \right) = F_0^q\left( {{F_{m'}}\left( t \right)} \right).
\end{equation*}}
\end{enumerate}
This construction process yields $\left( {T - 2} \right)! - 1$ distinct types of circular Florentine arrays.

\textit{Proof:} We leverage Lemma 4 to establish that for ${m_1} \ne {m_2}$ and $z \in{\mathbb{Z}_T}$, there exists a unique solution for the equation $F_{{m_1}}\left( t \right) = F_{{m_2}}\left( {t + z} \right)$ within ${\mathbb{Z}_T}$. Furthermore, the initial step guarantees that $F_ 0^q$ constitutes a permutation in ${\mathbb{Z}_T}$, and the mapping between the function output $F_0^q\left( t \right) $ and the independent variable $t$ is one-to-one. Consequently, we can assert that there exists a unique solution for the equation $F_0^q\left( {F_{{m_1}}\left( t \right)} \right) = F_0^q\left( {F_{{m_2}}\left( {t + z} \right)} \right)$ within ${\mathbb{Z}_T}$. This implies that $F^q$ itself qualifies as a circular Florentine array.
\hfill
$\hfill\blacksquare$ 

\textit{Remark 1:} Fixing the first two elements in $F_0$ is to ensure that there is no equivalence for all in the extended cases.

\textit{Example 3:} Let $T=5$, the selected circular Florentine array ${F}$, and its $5$ extended circular Florentine arrays $F^q \left( 0<q \le 5 \right)$ obtained by Construction \uppercase\expandafter{\romannumeral1} are listed as follows:
\[\begin{array}{l}
{F}= \left[ {\begin{array}{*{20}{c}}
0&1&2&3&4\\
0&2&4&1&3\\
0&3&1&4&2\\
0&4&3&2&1
\end{array}} \right],{F^1} = \left[ {\begin{array}{*{20}{c}}
0&1&2&4&3\\
0&2&3&1&4\\
0&4&1&3&2\\
0&3&4&2&1
\end{array}} \right],\\
{F^2}= \left[ {\begin{array}{*{20}{c}}
0&1&3&4&2\\
0&3&2&1&4\\
0&4&1&2&3\\
0&2&4&3&1
\end{array}} \right],{F^3} = \left[ {\begin{array}{*{20}{c}}
0&1&3&2&4\\
0&3&4&1&2\\
0&2&1&4&3\\
0&4&2&3&1
\end{array}} \right],\\
{F^4}= \left[ {\begin{array}{*{20}{c}}
0&1&4&3&2\\
0&4&2&1&3\\
0&3&1&2&4\\
0&2&3&4&1
\end{array}} \right],{F^5} = \left[ {\begin{array}{*{20}{c}}
0&1&4&2&3\\
0&4&3&1&2\\
0&2&1&3&4\\
0&3&2&4&1
\end{array}} \right].
\end{array}\]
%Based on the circular Florentine arrays obtained in Construction \uppercase\expandafter{\romannumeral1} and the Main Framework introduced in Section \uppercase\expandafter{\romannumeral3}, this section proposes three novel constructions with $T>3$ for multiple ZCZ sequence sets exhibiting optimal correlation properties. 

{Building upon the Main Framework and the insights from Lemmas 5-8 presented in Section \uppercase\expandafter{\romannumeral3}, this section introduces three novel constructions for multiple ZCZ sequence sets with optimal correlation properties for  $T>3$. These constructions are categorized according to the value of $R$: $R=1$, odd $R$, and even $R$. For each case, the desired properties are achieved by carefully designing the index matrix $A$ and the corresponding phase matrices $P^m$.}

\textit{Theorem 1:} Let $R=1$, $M=F_c(T)$ and $N=T^2$. Select the circular Florentine array $F^q \left(0< q<\left(T-2 \right) ! \right)$ as the index matrix $A$, where $F^q$ is obtained by Construction \uppercase\expandafter{\romannumeral1}. The phase matrices $P^m$ for the distinct sequence sets $S^m\left( 0 \le m<M \right)$ are assumed to be identical and collectively denoted as $P$. The ${P_u}\left( t \right)$ of $P$ is defined as:
\begin{equation}
{P_u}\left( t \right)=w_T^{ut},
\end{equation}
where $0\le u,t<T$. According to the Main Framework, ${s_u^m\left( n \right)}$ in $S^m$ is obtained as
\begin{equation}
{s_u^m\left( n \right)}={s}_u^m\left( {t + lT} \right) = {P_u}\left( t \right)w_T^{ lA^m\left( t \right)},
\end{equation}
where $0 \le l < T$. Then the sequence set $S^m\left( 0\le m<M \right)$ has the following properties:
\begin{enumerate}
\item{Each sequence in $S^m$  is unimodular and  perfect.}
\item{Each $S^m$ is an optimal $\left( {{T^2},T,T} \right)$-ZCZ sequence set.}
\item{${\theta _{{\bf{s}}_u^{{m_1}},{\bf{s}}_v^{{m_2}}}}\left( \tau  \right) = T$ for all $0 \le \tau  < T^2,0 \le {m_1} \ne {m_2} < M$ and $0 \le u \ne v <T$.} 
\end{enumerate}

\textit{Proof:}  The detailed proof of Theorem 1 is provided in the Appendix.

Incorporating the circular Florentine array $F$ described in Construction \uppercase\expandafter{\romannumeral1} into the construction framework of Theorem 1 inevitably leads to the issue of sequence equivalence within sequence sets. To circumvent this problem, we propose the following construction method by refining the generation process of the phase matrix.

\textit{Corollary 1:} %Let $R=1$, $M=F_c(T)$ and $N=T^2$. 
Select the circular Florentine array $F$ as the index matrix $A$. The ${P_u}\left( t \right)$ of $P$ is defined as:
\begin{equation}
\begin{aligned}
{P_u}\left( t \right)= \begin{cases}{w_T^{ut}},& \text{for}~{t = 0,1,\ldots,T - 3,}\\{w_T^{u\left( {T - 1} \right)}}, &\text{for} ~{ t=T-2,}\\{w_T^{u\left( {T -2} \right)}}, &\text{for}~{ t=T-1.}\end{cases}
\end{aligned}
\end{equation}
 The sequence set $S^m\left( 0\le m<M \right)$ obtained by (20) and (21)  has exactly the same properties as Theorem 1.

\textit{Proof:} The proof of Corollary 1 is similar to that for Theorem 1, hence it is omitted.

\textit{Example 4:} Let $N = RT^2$, where $R=1$ and $T=4$. $M={F_c}\left( 4 \right)=1$.  Then we choose the index matrix $A$ as \[{A} = \left[ {\begin{array}{*{20}{c}}
0&1&3&2
\end{array}} \right].\]
From (21), the phase matrix is obtained as
\[P= \left[ {\begin{array}{*{20}{c}}
w_4^0&w_4^0&w_4^0&w_4^0\\
w_4^0&w_4^1&w_4^2&w_4^3\\
w_4^0&w_4^2&w_4^0&w_4^2\\
w_4^0&w_4^3&w_4^2&w_4^1
\end{array}} \right].\]
Then the sequence set $S^0$ consisting of $4$ sequences of period $16$ can be obtained, and the element of $\textbf{s}^0_u $ is expressed as:
\begin{equation*}
{s^0_u\left( n \right)}={s}^0_u\left( {t + 4l} \right) = {{P^0_u}\left( t \right)}w_4^{ lA^0\left( t \right)},
\end{equation*} 
where $0 \le u,t,l < 4$. Four sequences in $S^0$ are given below:
\begin{equation*}
\textbf{s}^0_0 = \left( {1,1,1,1,1,w_4^1,w_4^3,w_4^2,1,w_4^2,w_4^2,1,1,w_4^3,w_4^1,w_4^2} \right) ,
 \end{equation*}
\begin{equation*}
\textbf{s}^0_1  = \left( {1,w_4^1,w_4^2,w_4^3,1,w_4^2,w_4^1,w_4^1,1,w_4^3,1,w_4^3,1,1,w_4^3,w_4^1} \right),
 \end{equation*}
\begin{equation*}
\textbf{s}^0_2 = \left( {1,w_4^2,1,w_4^2,1,w_4^3,w_4^3,1,1,1,w_4^2,w_4^2,1,w_4^1,w_4^1,1} \right),
 \end{equation*}
\begin{equation*}
\textbf{s}^0_3  = \left( {1,w_4^3,w_4^2,w_4^1,1,1,w_4^1,w_4^3,1,w_4^1,1,w_4^1,1,w_4^2,w_4^3,w_4^3} \right).
 \end{equation*}

The PACF of ${\textbf{s}}^0_1$ and PCCF between ${\textbf{s}}^0_2$ and ${\textbf{s}}^0_3$  are shown in Fig. 1. It is known that the sequence set $S^0$ has optimal correlations.
\begin{figure}[!t]
\centering
    \begin{subfigure}{0.48\linewidth}
    \includegraphics[width=\textwidth]{fig1_1}
        \caption{}
        \label{fig_first_case}
    \end{subfigure}
    \begin{subfigure}{0.48\linewidth}
        \includegraphics[width=\textwidth]{fig1_2}
        \caption{}
        \label{fig_second_case}
    \end{subfigure}    
    \caption{(a) The PACF of ${\textbf{s}}_1^0$ obtained from Example 4; (b) The PCCF of ${\textbf{s}}_2^0$ and ${\textbf{s}}_3^0$.}
    \label{fig_sim}
\end{figure}

\textit{Theorem 2:} Let $R$ be odd, $L=RT$, $N=RT^2$ and $M = \min \left\{ {{R^*} - 1,{F_c}\left( T \right)} \right\}$, where $R^*$ is the smallest prime divisor of $R$. Select $M$ rows randomly from the circular Florentine array $F^q\left(0< q<\left(T-2 \right) ! \right)$ to form the index matrix $A$, where $F^q$ is obtained by Construction \uppercase\expandafter{\romannumeral1}. The ${P^m_u}\left( l\right)$ within the phase matrix $P^m$, associated with the sequence set $S^m$, is defined as:
\[{P_u^m}\left( {l} \right)={P_u^m}\left( {t + {r}T} \right) = w_R^{a^{rm}}w_T^{ut},\]
where $a^{rm}={\left( {m + 1} \right)\frac{{{r}\left( {1 + {r}} \right)}}{2}}$, $0 \le {l} < L$, $0 \le {r} < R$, $0 \le t,u < T$ and $0 \le m < M$. 

According to the Main Framework, ${s_u^m\left( n \right)}$ in $S^m$ is obtained as described in (10).
%\begin{equation}
%\begin{aligned}
%&s_u^m\left( n \right)= s_u^m\left( {t + lT} %\right)\\
% =& {L^{ - 1}}T\sqrt R \sum\limits_{{r} = 0}^{R-1} {{P_u^m}\left( {t + {r}T} \right) w_L^{ - l\left( {A^m\left( t \right) + {r}T} \right)}},
%\end{aligned}
%\end{equation}
%where $0 \le l < L$. 
Then  the sequence set $S^m$, $0 \le m <M$, has the following properties:
\begin{enumerate}
\item{Each sequence in $S^m$ is unimodular and perfect.}
\item{Each $S^m$ is an optimal $\left( {{RT^2},T,RT} \right)$-ZCZ sequence set.}
\item{${\theta _{{\bf{s}}_u^{{m_1}},{\bf{s}}_v^{{m_2}}}}\left( \tau  \right) = \sqrt RT$ for all $0 \le \tau  < RT^2,0 \le {m_1} \ne {m_2} < M$ and $0 \le u \ne v <T$.} 
\end{enumerate}

\textit{Proof:}  The detailed proof of Theorem 2 is provided in the Appendix.

%For the same parameters of Theorem 2, we proposed an alternative construction method.

\textit{Corollary 2:} Let $F$ be the circular Florentine array described in Construction \uppercase\expandafter{\romannumeral1}. Select $M$ rows randomly from the circular Florentine array $F$ to form the index matrix $A$. Define a $T\times L$ matrix ${Q^{m}}$, the $l$-th element of the $u$-th row of the $m$-th matrix ${Q^{m}}$ is expressed as 
\[{Q_u^m}\left( {l} \right)={Q_u^m}\left( {t + {r}T} \right) = w_R^{a^{rm}}w_T^{ut},\]
where $a^{rm}={\left( {m + 1} \right)\frac{{{r}\left( {1 + {r}} \right)}}{2}}$, $0 \le {r} < R$, $0 \le t,u < T$ and $0 \le m < M$. The phase matrix $P^m$ is obtained by swapping the $\left( {rT - 1}\right)$-th and $\left( {rT - 2}\right)$-th column of the matrix ${Q^{m}}$
 for $0 < r \le R$. The sequence sets obtained by (10) have exactly the same properties as in Theorem 2.

%\textit{Proof:} The proof of Corollary 2 is similar to that for Theorem 2, hence it is omitted.

\textit{Example 5:} Let $N = RT^2$, where $R=3$ and $T=5$. $M = \min \left\{ {3 - 1,{F_c}\left( 5 \right)} \right\}=2$. From Construction \uppercase\expandafter{\romannumeral1}, the constructed circular Florentine array $F^1$ is shown as
 \[{F^1} = \left[ {\begin{array}{*{20}{c}}
0&1&2&4&3\\
0&2&3&1&4\\
0&3&4&2&1\\
0&4&1&3&2
\end{array}} \right].\]
Randomly select the first two rows of array $F^1$ to form the matrix $A$, as shown below:
\[{A} = \left[ {\begin{array}{*{20}{c}}
0&1&2&4&3\\
0&2&3&1&4
\end{array}} \right].\]
The phase matrices $P^0$ and $P^1$ for the sequence sets $S^0$ and $S^1$ in the Zak domain  are shown in (22a) and (22b), respectively.
 \begin{figure*}[ht] %hb代表放在文章底部,%ht为放在文章顶部 
 	\centering
\begin{subequations}
\begin{align}
P^0 &= \left[ {\begin{array}{*{20}{c}}
{w_{15}^0}&{w_{15}^0}&{w_{15}^0}&{w_{15}^0}&{w_{15}^0}&{w_{15}^5}&{w_{15}^{5}}&{w_{15}^{5}}&{w_{15}^{5}}&{w_{15}^{5}}&{w_{15}^0}&{w_{15}^0}&{w_{15}^0}&{w_{15}^0}&{w_{15}^0}\\
{w_{15}^0}&{w_{15}^3}&{w_{15}^6}&{w_{15}^9}&{w_{15}^{12}}&{w_{15}^{5}}&{w_{15}^8}&{w_{15}^{11}}&{w_{15}^{14}}&{w_{15}^2}&{w_{15}^0}&{w_{15}^3}&{w_{15}^6}&{w_{15}^9}&{w_{15}^{12}}\\
{w_{15}^0}&{w_{15}^6}&{w_{15}^{12}}&{w_{15}^3}&{w_{15}^9}&{w_{15}^5}&{w_{15}^{11}}&{w_{15}^2}&{w_{15}^8}&{w_{15}^{14}}&{w_{15}^0}&{w_{15}^6}&{w_{15}^{12}}&{w_{15}^3}&{w_{15}^9}\\
{w_{15}^0}&{w_{15}^9}&{w_{15}^3}&{w_{15}^{12}}&{w_{15}^6}&{w_{15}^5}&{w_{15}^{14}}&{w_{15}^8}&{w_{15}^2}&{w_{15}^{11}}&{w_{15}^0}&{w_{15}^9}&{w_{15}^3}&{w_{15}^{12}}&{w_{15}^6}\\
{w_{15}^0}&{w_{15}^{12}}&{w_{15}^9}&{w_{15}^6}&{w_{15}^3}&{w_{15}^5}&{w_{15}^2}&{w_{15}^{14}}&{w_{15}^{11}}&{w_{15}^8}&{w_{15}^0}&{w_{15}^{12}}&{w_{15}^9}&{w_{15}^6}&{w_{15}^3}
\end{array}} \right], \tag{\theequation a}\\
P^1 &= \left[ {\begin{array}{*{20}{c}}
{w_{15}^0}&{w_{15}^0}&{w_{15}^0}&{w_{15}^0}&{w_{15}^0}&{w_{15}^{10}}&{w_{15}^{10}}&{w_{15}^{10}}&{w_{15}^{10}}&{w_{15}^{10}}&{w_{15}^0}&{w_{15}^0}&{w_{15}^0}&{w_{15}^0}&{w_{15}^0}\\
{w_{15}^0}&{w_{15}^3}&{w_{15}^6}&{w_{15}^9}&{w_{15}^{12}}&{w_{15}^{10}}&{w_{15}^{13}}&{w_{15}^1}&{w_{15}^4}&{w_{15}^7}&{w_{15}^0}&{w_{15}^3}&{w_{15}^6}&{w_{15}^9}&{w_{15}^{12}}\\
{w_{15}^0}&{w_{15}^6}&{w_{15}^{12}}&{w_{15}^3}&{w_{15}^9}&{w_{15}^{10}}&{w_{15}^1}&{w_{15}^7}&{w_{15}^{13}}&{w_{15}^4}&{w_{15}^0}&{w_{15}^6}&{w_{15}^{12}}&{w_{15}^3}&{w_{15}^9}\\
{w_{15}^0}&{w_{15}^9}&{w_{15}^3}&{w_{15}^{12}}&{w_{15}^6}&{w_{15}^{10}}&{w_{15}^4}&{w_{15}^{13}}&{w_{15}^7}&{w_{15}^1}&{w_{15}^0}&{w_{15}^9}&{w_{15}^3}&{w_{15}^{12}}&{w_{15}^6}\\
{w_{15}^0}&{w_{15}^{12}}&{w_{15}^9}&{w_{15}^6}&{w_{15}^3}&{w_{15}^{10}}&{w_{15}^7}&{w_{15}^4}&{w_{15}^1}&{w_{15}^{13}}&{w_{15}^0}&{w_{15}^{12}}&{w_{15}^9}&{w_{15}^6}&{w_{15}^3}
\end{array}} \right]. \tag{\theequation b}
\end{align}
\end{subequations}

\begin{comment}
 \begin{equation}	
\begin{array}{l}
{P^0} = \left[ {\begin{array}{*{20}{c}}
{w_{15}^0}&{w_{15}^0}&{w_{15}^0}&{w_{15}^0}&{w_{15}^0}&{w_{15}^{10}}&{w_{15}^{10}}&{w_{15}^{10}}&{w_{15}^{10}}&{w_{15}^{10}}&{w_{15}^0}&{w_{15}^0}&{w_{15}^0}&{w_{15}^0}&{w_{15}^0}\\
{w_{15}^0}&{w_{15}^{12}}&{w_{15}^{9}}&{w_{15}^{6}}&{w_{15}^{3}}&{w_{15}^{10}}&{w_{15}^7}&{w_{15}^{4}}&{w_{15}^{1}}&{w_{15}^{13}}&{w_{15}^0}&{w_{15}^12}&{w_{15}^9}&{w_{15}^{6}}&{w_{15}^{3}}\\
{w_{15}^0}&{w_{15}^9}&{w_{15}^{3}}&{w_{15}^12}&{w_{15}^6}&{w_{15}^{10}}&{w_{15}^{11}}&{w_{15}^2}&{w_{15}^8}&{w_{15}^{14}}&{w_{15}^0}&{w_{15}^6}&{w_{15}^{12}}&{w_{15}^3}&{w_{15}^9}\\
{w_{15}^0}&{w_{15}^9}&{w_{15}^3}&{w_{15}^{12}}&{w_{15}^6}&{w_{15}^{10}}&{w_{15}^{14}}&{w_{15}^8}&{w_{15}^2}&{w_{15}^{11}}&{w_{15}^0}&{w_{15}^9}&{w_{15}^3}&{w_{15}^{12}}&{w_{15}^6}\\
{w_{15}^0}&{w_{15}^{12}}&{w_{15}^9}&{w_{15}^6}&{w_{15}^3}&{w_{15}^{10}}&{w_{15}^2}&{w_{15}^{14}}&{w_{15}^{11}}&{w_{15}^8}&{w_{15}^0}&{w_{15}^{12}}&{w_{15}^9}&{w_{15}^6}&{w_{15}^3}
\end{array}} \right],\\
{P^1} = \left[ {\begin{array}{*{20}{c}}
{w_{15}^0}&{w_{15}^0}&{w_{15}^0}&{w_{15}^0}&{w_{15}^0}&{w_{15}^{10}}&{w_{15}^{10}}&{w_{15}^{10}}&{w_{15}^{10}}&{w_{15}^{10}}&{w_{15}^0}&{w_{15}^0}&{w_{15}^0}&{w_{15}^0}&{w_{15}^0}\\
{w_{15}^0}&{w_{15}^3}&{w_{15}^6}&{w_{15}^9}&{w_{15}^{12}}&{w_{15}^{10}}&{w_{15}^{13}}&{w_{15}^1}&{w_{15}^4}&{w_{15}^7}&{w_{15}^0}&{w_{15}^3}&{w_{15}^6}&{w_{15}^9}&{w_{15}^{12}}\\
{w_{15}^0}&{w_{15}^6}&{w_{15}^{12}}&{w_{15}^3}&{w_{15}^9}&{w_{15}^{10}}&{w_{15}^1}&{w_{15}^7}&{w_{15}^{13}}&{w_{15}^4}&{w_{15}^0}&{w_{15}^6}&{w_{15}^{12}}&{w_{15}^3}&{w_{15}^9}\\
{w_{15}^0}&{w_{15}^9}&{w_{15}^3}&{w_{15}^{12}}&{w_{15}^6}&{w_{15}^{10}}&{w_{15}^4}&{w_{15}^{13}}&{w_{15}^7}&{w_{15}^1}&{w_{15}^0}&{w_{15}^9}&{w_{15}^3}&{w_{15}^{12}}&{w_{15}^6}\\
{w_{15}^0}&{w_{15}^{12}}&{w_{15}^9}&{w_{15}^6}&{w_{15}^3}&{w_{15}^{10}}&{w_{15}^7}&{w_{15}^4}&{w_{15}^1}&{w_{15}^{13}}&{w_{15}^0}&{w_{15}^{12}}&{w_{15}^9}&{w_{15}^6}&{w_{15}^3}
\end{array}} \right].
\end{array}
 \end{equation}
 \end{comment}
 \end{figure*}
From (10), two sequence sets of size $5$ and period $75$ are obtained as
\begin{equation*}
\begin{aligned}
&s_u^m\left( n \right)= s_u^m\left( {t + 5l} \right)\\
=& \frac{{\sqrt 3 }}{3} \sum\limits_{{r} = 0}^2 {P_u^m\left( {t + 5{r}} \right)w_{15}^{  l\left( {A^m\left( t \right) + 5{r}} \right)}},
\end{aligned}
\end{equation*} 
where $0 \le u,t < 5, 0 \le l<15, 0\le m<2$. The PACF of ${\textbf{s}}^0_1$, the PCCF between ${\textbf{s}}_0^0$ and ${\textbf{s}}_4^0$ and the PCCF between ${\textbf{s}}_0^0$ and ${\textbf{s}}_2^1$ are shown in Fig. 2. 
% \begin{figure}
% 	\centering 
% 	\subfloat[]
% 		{\begin{minipage}[t]{\linewidth}
% 			%\centering
%                 \raggedleft
%                 \includegraphics[width=1.0\columnwidth]{fig2_1.eps}
% 		\end{minipage}

% 	}%%此处的空行很重要,想让图片在什么地方换行就在代码对应位置空行
    
%      \subfloat[]
%          {\begin{minipage}[t]{\linewidth}
% 			%\centering
%                 \raggedleft
% 			\includegraphics[width=\columnwidth]{fig2_2.eps}
% 		\end{minipage}
     
% 	}%

% 	\subfloat[]
%          {\begin{minipage}[t]{\linewidth}
% 			%\centering
%                 \raggedleft
% 			\includegraphics[width=\columnwidth]{fig2_3.eps}
% 		\end{minipage}
% 	}%
% 	\centering
% 	\caption{(a) The PACF of ${\textbf{s}}_1^0$ obtained from Example 5; (b) The PCCF of ${\textbf{s}}_0^0$ and ${\textbf{s}}_4^0$; (c) The PCCF of ${\textbf{s}}_1^0$ and ${\textbf{s}}_2^1$.}
% \end{figure}
\begin{figure}[!h]
    \centering
    \begin{subfigure}{\linewidth}
        %\includegraphics[width=\textwidth]{ambi2.eps}
        \includegraphics[width=\textwidth]{fig2_1}
        \caption{The PACF of ${\textbf{s}}_1^0$.}
        \label{A4}
    \end{subfigure}
    \begin{subfigure}{\linewidth}
        \includegraphics[width=\textwidth]{fig2_2}
        \caption{The PCCF of ${\textbf{s}}_0^0$ and ${\textbf{s}}_4^0$.}
        \label{A4}
    \end{subfigure}
    \begin{subfigure}{\linewidth}
        \includegraphics[width=\textwidth]{fig2_3}
        \caption{The PCCF of ${\textbf{s}}_1^0$ and ${\textbf{s}}_2^1$.}
        \label{A4}
    \end{subfigure}
    \caption{The correlation properties of Example 4.}
    \label{shiyan}
    \label{amvi}
\end{figure}

\textit{Theorem 3:} Let $R$ be even, $L=RT$, $N=RT^2$ and $M=1$. Select $M$ rows randomly from the circular Florentine array $F^q\left(0< q<\left(T-2 \right) ! \right)$ to form the index matrix $A$, where $F^q$ is obtained by Construction \uppercase\expandafter{\romannumeral1}. The ${P_u}\left( l\right)$ within the phase matrix $P$, associated with the sequence set $S$, is defined as:
\[{P_u}\left( {l} \right)={P_u}\left( {t + {r}T} \right) = w_{2R}^{{r}^2}w_T^{ut},\]
where $0 \le {l} < L$, $0 \le {r} < R$ and $0 \le t,u < T$.

According to the Main Framework, ${s_u\left( n \right)}$ in $S$ is defined as
\begin{equation}
\begin{aligned}
&{s_u}\left( n \right)= {s_u}\left( {t + lT} \right)\\
 = &{L^{ - 1}}T\sqrt R \sum\limits_{{r} = 0}^{R-1} {w_{2R}^{{r}^2}w_T^{ut}w_L^{  l\left( {{A}\left( t \right) + {r}T} \right)}}.
\end{aligned}
\end{equation}
Then, the sequence set $S$ has the following properties:
\begin{enumerate}
\item{Each sequence in $S$ is unimodular and  perfect.}
\item{$S$ is an optimal $\left( {{RT^2},T,RT} \right)$-ZCZ sequence set.}
\end{enumerate}

\textit{Proof:}  The detailed proof of Theorem 3 is provided in the Appendix.

%With respect to the identical parameters of Theorem 3, we proposed an alternative construction in the following.

\textit{Corollary 3:} Let $F$ be the circular Florentine array described in Construction \uppercase\expandafter{\romannumeral1}. Select $F$ to from the index matrix $A$. Define a $T\times L$ matrix $Q$, the $l$-th element of the $u$-th row of the matrix $Q$ is expressed as 
\[{Q_u}\left( {l} \right)={Q_u}\left( {t + {r}T} \right) = w_{2R}^{{r}^2}w_T^{ut},\]
where $0 \le {r} < R$ and $0 \le t,u < T$. The phase matrix $P$ is obtained by swapping the $\left( {rT - 1}\right)$-th and $\left( {rT - 2}\right)$-th column of the matrix ${Q}$ for $0 < r \le R$. The sequence set $S$ generated through (23) exhibits identical characteristics to those outlined in Theorem 3.

\textit{Example 6:} Let $N = RT^2$, where $R=2$, $T=6$ and $M=1$. Then we choose the index matrix ${A}$ as ${A}= \left[ {\begin{array}{*{20}{c}}
0&1&2&3&5&4 \end{array}} \right]$. The phase matrix $P$ is shown in (24).
 \begin{figure*}[ht] %hb代表放在文章底部,%ht为放在文章顶部 
 	\centering
 	\begin{equation}	
\begin{array}{l}
P = \left[ {\begin{array}{*{20}{c}}
{w_{12}^0}&{w_{12}^0}&{w_{12}^0}&{w_{12}^0}&{w_{12}^0}&{w_{12}^0}&{w_{12}^3}&{w_{12}^3}&{w_{12}^3}&{w_{12}^3}&{w_{12}^3}&{w_{12}^3}\\
{w_{12}^0}&{w_{12}^2}&{w_{12}^4}&{w_{12}^6}&{w_{12}^8}&{w_{12}^{10}}&{w_{12}^3}&{w_{12}^5}&{w_{12}^7}&{w_{12}^9}&{w_{12}^{11}}&{w_{12}^1}\\
{w_{12}^0}&{w_{12}^4}&{w_{12}^8}&{w_{12}^0}&{w_{12}^4}&{w_{12}^8}&{w_{12}^3}&{w_{12}^7}&{w_{12}^{11}}&{w_{12}^3}&{w_{12}^7}&{w_{12}^{11}}\\
{w_{12}^0}&{w_{12}^6}&{w_{12}^0}&{w_{12}^6}&{w_{12}^0}&{w_{12}^6}&{w_{12}^3}&{w_{12}^9}&{w_{12}^3}&{w_{12}^9}&{w_{12}^3}&{w_{12}^9}\\
{w_{12}^0}&{w_{12}^8}&{w_{12}^4}&{w_{12}^0}&{w_{12}^8}&{w_{12}^4}&{w_{12}^3}&{w_{12}^{11}}&{w_{12}^7}&{w_{12}^3}&{w_{12}^{11}}&{w_{12}^7}\\
{w_{12}^0}&{w_{12}^{10}}&{w_{12}^8}&{w_{12}^6}&{w_{12}^4}&{w_{12}^{2}}&{w_{12}^3}&{w_{12}^{1}}&{w_{12}^{11}}&{w_{12}^9}&{w_{12}^7}&{w_{12}^5}
\end{array}} \right]
\end{array}
 \end{equation}
 \end{figure*}
From (23), a sequence set $S$ of size $6$ with period $72$ is obtained as
\begin{equation*}
\begin{aligned}
{s_u}\left( n \right) = {s_u}\left( {t + 6l} \right)= \frac{{\sqrt 2 }}{2}\sum\limits_{{r} = 0}^1 {{P_u}\left( {t + 6{r}} \right)w_{12}^{  l\left( {{A}\left( t \right) + 6{r}} \right)}}, 
\end{aligned}
\end{equation*} 
where $0 \le u,t < 6, 0 \le l<12$. The PACF of ${\textbf{s}}_4$ and PCCF between ${\textbf{s}}_1$ and ${\textbf{s}}_5$ are shown in Fig. 3. 
% \begin{figure}[!t]
% \centering
% \subfloat[]{\includegraphics[width=0.22\textwidth]{fig3_1.eps}%
% \label{fig_first_case}}
% \hfil
% \subfloat[]{\includegraphics[width=0.22\textwidth]{fig3_2.eps}%
% \label{fig_second_case}}
% \caption{(a) The PACF of ${\textbf{s}}_4$ obtained from Example 6; (b) The PCCF of ${\textbf{s}}_1$ and ${\textbf{s}}_5$.}
% \label{fig_sim}
% \end{figure}

\begin{figure}[!h]
    \centering
    \begin{subfigure}{0.48\linewidth}
        %\includegraphics[width=\textwidth]{ambi2.eps}
        \includegraphics[width=\textwidth]{fig3_1}
        \caption{The PACF of ${\textbf{s}}_4$.}
        \label{A4}
    \end{subfigure}
    \begin{subfigure}{0.48\linewidth}
        \includegraphics[width=\textwidth]{fig3_2}
        \caption{The PCCF of ${\textbf{s}}_1$ and ${\textbf{s}}_5$.}
        \label{A4}
    \end{subfigure}    
    \caption{The correlation properties of Example 6.}
    \label{shiyan}
    \label{amvi}
\end{figure}

\subsection{Comparison With The Previous Related Constructions}
In Table I, we list some known constructions of multiple ZCZ sequence sets with optimal correlation properties. From the construction perspective, existing methods in \cite{Popovic10}, \cite{Zhou17}, \cite{Zhang20} and \cite{Pitaval21} are developed from a time-domain approach. In contrast, our proposed method uniquely leverages the Zak transform, operating directly in the Zak domain. 

With regard to the cyclic equivalence and availability of sequences, existing methods suffer from several different limitations. Specifically, the Hadamard matrix based approach in \cite{Popovic10} may produce cyclically equivalent sequences. While the constructions in \cite{Zhou17} and \cite{Zhang20} can ensure unique sequences, the achievable sequence set numbers are limited due to their reliance on specific functions. Moreover, the number of multiple ZCZ sequence sets in \cite{Popovic10}, \cite{Zhou17}, \cite{Zhang20} and \cite{Pitaval21} is $d-1$, where $d$ is the smallest prime divisor of the sequence length. 

To address these limitations, \cite{Zhang22} utilizes circular Florentine arrays to create multiple optimal ZCZ sets with period $RT^2$, where $R$ is a positive integer. When $T$ is a prime, the number of the ZCZ sequence sets is also equal to $d-1$ as in \cite{Popovic10}, \cite{Zhou17} and \cite{Zhang20}. When $T$ is non-prime, the number of the ZCZ sequence sets depends on the number of rows of the cyclic Florentine array ${F_c}\left( T \right)$, which is strictly larger than $d-1$. Although \cite{Zhang22} leads to relatively large number of ZCZ sequence sets, their method still introduces cyclic equivalence within individual sets regardless of whether $R=1$ or not. 

{Note that cyclically distinct sequences are highly desirable in practice \cite{Zhou17}, \cite{Golomb2005}. On the contrary, the use of cyclically equivalent sequences could enable an attacker to easily decode the sequences of multiple users once the sequence of one user is decoded. This is unacceptable for secure information transmission in, for example, military and satellite communication systems.

Compared with \cite{Zhang22}, our proposed construction effectively avoids cyclic equivalence within individual sets. One can show that there are $(T-2)!$ distinct cases for the proposed sequence sets, thus offering a wide range of possibilities for various applications. On the other hand, the construction method in \cite{Song21} can only yield a single sequence set with optimal cross-correlation. Additionally, the application of cyclic Florentine array in the Zak-domain has not been reported before, to the best of our knowledge.}

\begin{table*}
\caption{The parameters for several sets of multiple ZCZ sequences}
\label{table}
\small
\centering
\setlength{\tabcolsep}{4pt}
\begin{tabular}{|m{1.0cm}<{\centering}|m{0.8cm}<{\centering}|m{0.8cm}<{\centering}|m{1.0cm}<{\centering}|m{0.5cm}<{\centering}|m{3.0cm}<{\centering}| m{1.0cm}<{\centering}|m{2.3cm}<{\centering}|m{2cm}<{\centering}|m{3cm}<{\centering}|}
\hline
Methods & Period& Phase Number &  Set size & $H_{z}$ & The number of $ZCZ$ sets & $\theta_c\left( {{\cal S}} \right)$ & Whether cyclically distinct & Number of distinct sequence sets & Note\\
\hline
\cite{Popovic10} &$RT^2$ & $RT$ &  $T$ & $RT$ & $d-1$ & $\sqrt R T $&N&1&$R$ and $T$ are positive integers.\\
\hline
\cite{Zhou17} &$T^2$ &$T$ &  $T$ & $T$ & $T-1$ & $T$&Y&1&$T$ is odd prime.\\
\hline
\cite{Zhang20}&$T^2$ &$T$ &  $T$ & $T$ & $d-1$ & $T$ &Y&1& $T$ is odd.\\

\hline
 \cite{Pitaval21} & $T$ &$T$ & $ \left\lfloor {\frac{T}{L}} \right\rfloor $&$L$  &$T-1$&$\sqrt T $   &N&1&$T$ is prime.\\
 \hline
 \multirow{3}{*}{\cite{Zhang22}} &$RT^2$ &$RT$ &  $T$ &$T$&  $d-1$ & $\sqrt R T$&N&1&$T$ is prime. $R$ is a positive integer. \\
  
 \cline{2-9}
 &$T^2$ &$T$ &  $T$ &$T$&  $F_c(T)$ & $T$&N&1 &$T$ is nonprime.\\
 \cline{2-9}
 &$RT^2$ &$RT$ &  $T$ &$RT$&  $min\{R^*-1,F_c(T)\}$ & $\sqrt R T$&N&1&$T$ is nonprime and $R\neq 1$ is a positive integer.\\
 \hline
Th.1 & $T^2$ &$T$ &$T$ & $T$ & $F_c(T)$ & $T$&Y&$\left(T-2 \right)!$&$T>3$ is a integer.\\
\hline
Th.2 & $RT^2$ &$RT$ &$T$ & $RT$ & $min\{R^*-1,F_c(T)\}$ &$\sqrt R T$&Y&$\left(T-2 \right)!$&$R$ is odd, $T>3$ is a integer.\\
\hline
Th.3 & $RT^2$ &$2RT$ &$T$ & $RT$ & $1$ &$-$&Y&$\left(T-2 \right)!$&$R$ is even, $T>3$ is a integer.\\
\hline
\end{tabular}
\begin{tablenotes} 
\footnotesize 
\item \noindent{ $d$ is the smallest prime divisor of the period; $R^*$ is the smallest prime divisor of $R$.}
\end{tablenotes}   
\label{tab1}
\end{table*}

\subsection{Evaluation of the Synchronization performance in OTFS}

{Let us consider a wireless communication system where $L \times T$ data symbols in the DD domain are modulated using OTFS over a total bandwidth $B$ operating at the carrier frequency $f_c$, where $L$ and $T$ denote the numbers of delay bins and Doppler bins, respectively. Firstly, the OTFS modulator distributes these symbols $\left\{ {{X(j,t)},j{\rm{ }} = 0, \ldots ,{\rm{ }}L - 1,{\rm{ }}t{\rm{ }} = {\rm{ }}0, \ldots ,T - 1} \right\}$ in the two-dimensional DD grid. Denote by $\Delta f = B/T$ the frequency spacing and $\Gamma  = 1/\Delta f$ the corresponding time duration. Thus, the duration of one OTFS frame is $L\Gamma$.% This imposes $1/L\Delta f$ and $1/T\Gamma$ delay and Doppler spacing, respectively.

The DD-domain symbols $X(j,t)$ are then transformed into the TF domain $\left\{ {X_{\mathrm{TF}}}(l,m), {l = 0,\ldots,L - 1},{m = 0, \ldots ,T - 1} \right\}$ via ISFFT, as shown in (25) below. 
\begin{equation}
{X_{\rm{TF}}}(l,m) = \frac{1}{{\sqrt {LT} }}\sum\limits_{j = 0}^{L - 1} {\sum\limits_{t = 0}^{T - 1} {X(j,t)} {e^{i2\pi (\frac{{lj}}{L} - \frac{{mt}}{T})}}}.
\label{OTFS1}
\end{equation}

After mapping into TF domain, Heisenberg transform is applied to generate the discrete time-domain signal as follows:
\begin{equation}
s({t} + {l}T) = \frac{1}{{\sqrt {T} }}\sum\limits_{m = 0}^{T - 1} {{X_{{\rm{TF}}}}} ({l},m){e^{\frac{{2\pi im}}{T}{t}}}.
\label{OTFS2} 
\end{equation}
%\begin{align}
%&s({t} + {l}T) 
%\notag = &{X_\rm{T}}({l},{t}) \\
%= &\sum\limits_{m = 0}^{T - 1} {{X_{{\rm{TF}}}}} ({l},m){e^{\frac{{2\pi im}}{T}{t}}} 
%\notag &s({k_2} + {k_1}T) \\
%\notag = &{X_\rm{T}}({k_1},{k_2}) \\
%= &\sum\limits_{m = 0}^{T - 1} {{X_{{\rm{TF}}}}} ({k_1},m){e^{\frac{{2\pi im}}{T}{k_2}}} \\
%\notag &s(t + jT) \\
%\notag =& {X_T}(j,t) \\
% = &\sum\limits_{m = 0}^{T - 1} {{X_{{\rm{TF}}}}} (j,m){e^{\frac{{2\pi im}}{T}t}}
%\label{OTFS2}
%\end{align}
%The matrix form of this step is
%\begin{equation}
%{x_T} = vec(F_T^HX_{TF}^T\sqrt M)
%\label{OTFS44}
%\end{equation}

Substituting (\ref{OTFS1}) into (\ref{OTFS2}), we obtain % and using $\left( {1/M} \right)\sum\limits_{m = 0}^{M - 1} {{e^{j2\pi m/M(l - l')}}}  = \delta (l - l')$
\begin{equation*}
s(t + lT) = \frac{1}{{\sqrt L }}\sum\limits_{j = 0}^{L - 1} {{X(j,t)} {e^{i2\pi \frac{l}{L}j}}},
\label{OTFS3}
\end{equation*}
which is equivalent to IFZT defined in (4). The relationship of OTFS modulation and IFZT is shown in Fig. \ref{figzak3}. %For an $L \times T$ Zak matrix, $L$ and $T$ correspond to the Doppler axis and delay axis in the DD domain respectively.
\begin{figure}[h]
\centering
\includegraphics[width=2.5in]{zak1.eps}
\caption{The relationship of OTFS modulation and IFZT}
\label{figzak3}
\end{figure}

Furthermore, let us consider a doubly selective channel consisting of $C$ paths as shown below:  
\begin{equation}
h(\tau ,v) = \sum\limits_{\rho = 1}^C {{h_p}\delta (\tau  - {\tau _\rho})} \delta (v - {v_\rho}),
\end{equation}
where $h_\rho$, $\tau _\rho$ and $v_\rho$ represent the channel fading coefficient, delay and Doppler values of the $\rho $-th path, respectively. To determine $L$ and $T$, it is required that $\max (\tau ) < \Gamma$ and $\max (v) <  \Delta f$.
After the OTFS modulation (i.e., IFZT transform), the time-domain signal is represented by $s(n)$. The receive signal $r(n)$ then can be expressed as %is a mixed signal of $s(n)$ affected by channel $h(\tau ,v)$ 
\begin{equation}
r(n) = \int{\int {h(\tau ,v)s(n - \tau ){e^{2\pi iv(n - \tau )}}d\tau dv} +{\cal N}},
\end{equation}
where ${\cal N}$ denotes the additive white Gaussian noise term with variance of $\sigma^2$.

%\begin{figure}[!h]
%\centering
%\includegraphics[width=2.5in]{zak2}
%\caption{This is the caption for one fig.}
%\label{fig1}
%\end{figure}

%Due to the influence of time-varying multipath (LTV) channels, the time offset causes strong intersymbol interference in signal demodulation. Therefore, signal synchronization is essential in establishing
%a wireless communication link for OTFS system. %\cite{OTFS4,OTFS5,OTFS6,OTFS7,OTFS8}.
%Synchronization is achieved by designing preamble frame or pilot patterns in the DD domain. Different preamble waveforms such as the random access waveform \cite{OTFS4}, the linear frequency modulated waveform \cite{OTFS5} and the DFT sequence \cite{OTFS7,OTFS8} are designed and used for OTFS signal synchronization. \cite{OTFS6} achieves synchronization by designing pilot patterns that an impulsive pilot is at one bin surrounded by zero guard symbols in DD domain. Different from the above methods, the Zak matrix constructed in this paper can be directly used as a whole preamble frame for synchronization due to its perfect time domain autocorrelation function.

Next, we investigate the synchronization performance of a single-input single-output (SISO) OTFS system. Our idea is to transmit a sparse Zak matrix (satisfying the perfect sequence condition as specified in Subsection IV.A) as a preamble sequence (frame) in the DD domain and then leverage its zero auto-correlation sidelobes for synchronization in the time domain. Fig. \ref{figzak2} illustrates the synchronization model for such a SISO-OTFS system. We assume that a cyclic prefix (CP) is added to the beginning of each frame for mitigation of inter-frame interference. %The receiving signal $r(n)$ is affected by the multipath effect and is the result of the mixing of the signals of the $C$ paths and noise. 
The length of the sliding window is $L\Gamma$. The receive signal in the window is correlated with the known reference sequence (i.e., the time-domain sequence of the aforementioned sparse Zak matrix) each time within the timing acquisition range $T_u$ to detect the starting position of the preamble sequence. The synchronization point is detected once a correlation peak is achieved at certain time shift. %Since the Zak sequence has a perfect autocorrelation function, when the convolution window is located at the CP position of the preamble frame, the result will be a lower value (affected by Gaussian noise and the Doppler effect). When the window successfully matches the preamble frame, an obvious peak will appear, which is a sign of synchronization position.
%a low value will be obtained under the influence of Gaussian noise and Doppler effect, while when the convolution window appears at the beginning of the preamble frame of different paths, an obvious peak will appear. This peak is the signal of successful synchronization. %Since the Zak sequence in time domain has perfect autocorrelation function, low values will appear when the sliding window is close to the proposed Zak signal, and a peak will appear when it is successfully matched. 
\begin{figure}[h]
\centering
\includegraphics[width=3.5in]{zak2.eps}
\caption{Synchronization model of SISO-OTFS system}
\label{figzak2}
\end{figure}

\begin{align}
\label{X168}
\notag & X_1^0 = \\
&8 \sqrt{2}\left[ {\begin{array}{*{20}{l}}
{w_{16}^0}&0&0&0&0&0&0&0\\
0&{w_{16}^2}&0&0&0&0&0&0\\
0&0&0&0&0&0&{w_{16}^{12}}&0\\
0&0&{w_{16}^4}&0&0&0&0&0\\
0&0&0&0&0&{w_{16}^{10}}&0&0\\
0&0&0&{w_{16}^6}&0&0&0&0\\
0&0&0&0&0&0&0&{w_{16}^{14}}\\
0&0&0&0&{w_{16}^8}&0&0&0\\
{w_{16}^4}&0&0&0&0&0&0&0\\
0&{w_{16}^6}&0&0&0&0&0&0\\
0&0&0&0&0&0&{w_{16}^0}&0\\
0&0&{w_{16}^8}&0&0&0&0&0\\
0&0&0&0&0&{w_{16}^{14}}&0&0\\
0&0&0&{w_{16}^{10}}&0&0&0&0\\
0&0&0&0&0&0&0&{w_{16}^2}\\
0&0&0&0&{w_{16}^{12}}&0&0&0
\end{array}} \right]
\end{align}
\begin{figure}[!h]
    \centering
    \includegraphics[width=0.48\textwidth]{realimag.eps}
    \caption{The real and imaginary parts of $\mathbf{s}_1^0$}
    \label{realimag}
\end{figure}
\begin{figure}[!h]
    \centering
    \begin{subfigure}{0.45\textwidth}
        %\includegraphics[width=\textwidth]{ambi2.eps}
        \includegraphics[width=\textwidth]{figure_3.eps}
        \caption{The ambiguity function of $\mathbf{s}_1^0$}
        \label{A4}
    \end{subfigure}
    \begin{subfigure}{0.45\textwidth}
        \includegraphics[width=\textwidth]{ACF.eps}
        \caption{The auto-correlation function of $\mathbf{s}_1^0$}
        \label{A4}
    \end{subfigure}
    \caption{The ambiguity function and auto-correlation function of $\mathbf{s}_1^0$}
    \label{shiyan}
    \label{amvi}
\end{figure}

For evaluation, three consecutive $16\times8$ OTFS frames are transmitted, with the second frame as the synchronization frame and the remaining frames carrying QPSK data in the DD domain. Equal transmission power is assumed for all frames. As an example, we consider Zak matrix $X_1^0$ in (\ref{X168}) which is constructed via Theorem 3 with parameters $R=2$ and $T=8$. The corresponding time-domain sequence is $\mathbf{s}_1^0$. The real and imaginary parts of $\mathbf{s}_1^0$ are shown in Fig. \ref{realimag}. Also, as shown in Fig. \ref{amvi}, such a sequence exhibits excellent ambiguity function with strong resilience to Doppler as well as perfect auto-correlation sidelobes.  

To introduce randomness for the starting point of the synchronization frame, some initial portion of the first data frame is randomly truncated. For numerical simulation, we consider $C\in \{4,8\}$ and two types of Rayleigh fading channels with non-uniform power-delay-profile (PDP) and uniform PDP. The system parameters are summarized in Table \ref{tab1}. The synchronization success probability is evaluated using $10000$ Monte Carlo simulations at each signal-to-noise ratio (SNR). 
\begin{table*}[!h]
	\begin{center}
		\caption{SYSTEM PARAMETERS}
		\label{tab1}
		\begin{tabular}{| c | c | }
			\hline
			number of Doppler bins ($T$) & 8 \\
			\hline
			number of delay bins ($L$) &16\\
			\hline
			Carrier Frequency ($f_c$) &$6$ GHz \\ 
			\hline
			frequency spacing ($\Delta f$) & 15 KHz  \\
			\hline
			 \multirow{5}{*}{Non-uniform PDP (delay, speed, path power)}& $4$ paths:\\			
			 ~& ($0$, $130$ Kmph, $2/3$), ($1$, $190$ Kmph, $1/6$), ($2$, $240$ Kmph, $1/9$), ($3$, $270$ Kmph, $1/18$). \\			 
			 ~&$8$ paths: \\			 
			 ~& ($0$, $130$ Kmph, $3/5$), ($1$, $190$ Kmph, $1/10$), ($2$, $240$ Kmph, $1/10$), ($3$, $270$ Kmph, $4/75$), \\			 
			 ~&($4$, $200$ Kmph, $4/75$), ($5$, $130$ Kmph, $4/75$), ($6$, $170$ Kmph, $1/50$), ($7$, $230$ Kmph, $1/50$).\\			 
		   	%~&$12$ paths (non-uniform PDP).\\			
			%~& ($0$, $130$Kmph, $7/12$), ($1$, $190$Kmph, $1/16$), ($2$, $240$Kmph, $1/16$), ($3$, $270$Kmph, $1/16$), \\			
			%~&($4$, $200$Kmph, $1/16$), ($5$, $130$Kmph, $1/32$), ($6$, $170$Kmph, $1/32$), ($7$, $230$Kmph, $1/32$),\\			
			%~& ($8$, $140$Kmph, $1/32$), ($9$, $190$Kmph, $1/72$), ($10$, $150$Kmph, $1/72$), ($11$, $230$Kmph, $1/72$).\\
			%\hline 
			%Channel path powers& 1.\\&(1/4, 1/4, 1/4, 1/4).\\& (2/3, 1/6, 1/9, 1/18)\\  & 2.\\&(1/8, 1/8, 1/8, 1/8, 1/8, 1/8, 1/8, 1/8).\\&(, 3/5, 1/10, 1/10, 4/75, 4/75, 4/75, 1/50, 1/50)\\&3.\\& (1/12, 1/12, 1/12, 1/12, 1/12, 1/12, 1/12, 1/12, 1/12, 1/12, 1/12, 1/12).\\&(7/12, 1/16, 1/16, 1/16, 1/16, 1/32, 1/32, 1/32, 1/32, 1/72, 1/72, 1/72)\\
			
			\hline 
			CP Length & 32 \\
			\hline 
			Sliding Window Length & 128 \\
			\hline 
		\end{tabular}
	\end{center}
\end{table*}

Fig.~\ref{4p} and Fig.~\ref{8p} evaluate the synchronization success probability performance as a function of SNR under 4 and 8 paths, respectively. Thanks to the perfect auto-correlation property of the proposed sequence $\mathbf{s}_1^0$, the receiver can effectively detect the starting point of the preamble sequence through sliding window correlation. For non-uniform PDP, our proposed preamble sequence can achieve a success rate close to $100\%$ at SNR of 16 dB or higher, while the random sequence (with random QPSK symbols in the DD domain) can only reach about $86\%$ for the 4-path case and $80\%$ for the 8-path case at SNR of 20 dB or higher. The synchronization performance under uniform PDP is generally worse than that of the non-uniform PDP case, but using the proposed preamble sequence can still maintain a high success rate (with $100\%$ success rate achieved at SNR of 20 dB or higher), while the success rate of the random sequence drops significantly. %These results demonstrate that the proposed Zak-domain-induced sequences  synchronization performance, making them suitable for preamble sequences in OTFS modulation.
\begin{figure}[!h]
    \centering
    \includegraphics[width=0.9\linewidth]{4.eps}
   \caption{The synchronization success probability versus SNR under 4 paths}
    \label{4p}
\end{figure}
\begin{figure}[!h]
    \centering
    \includegraphics[width=0.9\linewidth]{8.eps}
     \caption{The synchronization success probability versus SNR under 8 paths}
    \label{8p}
\end{figure}

\begin{comment}
\begin{figure}[!h]
    \centering
    \begin{subfigure}{0.45\textwidth}
        \includegraphics[width=\textwidth]{4.eps}
        \caption{4 paths}
        \label{A4}
    \end{subfigure}
    \begin{subfigure}{0.45\textwidth}
        \includegraphics[width=\textwidth]{8.eps}
        \caption{8 paths}
        \label{A8}
    \end{subfigure}
    \begin{subfigure}{0.45\textwidth}
        \includegraphics[width=\textwidth]{12.eps}
        \caption{12 paths}
        \label{A12}
    \end{subfigure}
  % \end{minipage}
    \caption{Synchronization success probability versus SNR under 4,8,12 paths}
    \label{shiyan}
   % \label{fig:all_images}
\end{figure}
\end{comment}
}
%\begin{figure}[!h]
%\centering
%\includegraphics[width=3in]{zak3.eps}
%\caption{Synchronization success probability versus SNR under different path numbers}
%\label{zak3}
%\end{figure}
\section{Conclusions}
In this paper, we have presented a novel framework for constructing multiple ZCZ sequence sets with optimal correlation properties using IFZT. To ensure sequence sparsity in the Zak domain, we have introduced index matrices and phase matrices that are associated to FZT. The admissible conditions of these matrices are also derived. It has been found that the maximum inter-set cross-correlation can beat the Sarwate bound provided that a circular Florentine array is adopted as the index matrix. {Besides, we have demonstrated that the Zak-domain-induced optimal sequences can be employed as preamble sequences in the DD domain for excellent OTFS synchronization performance. %We have shown that these Zak domain matrices lead to optimal synchronization performance.

As a future work, it is interesting to analyze and study the ambiguity properties of the proposed sequences \cite{LAZ2022}. Moreover, it is worthy to investigate their applications for channel estimation and sensing in, for example, multi-user MIMO-OTFS systems. To this end, one may leverage the multiple ZCZ set properties as well as the excellent ambiguity sidelobes of these sequences. For spectrum-efficient transmission, one may also superimpose those sparse Zak matrices in the DD domain with random communication data symbols. The readers are invited to attack these research problems.} 


%Additionally, three distinct phase matrices have been proposed for constructing optimal multiple ZCZ sequence sets. Our proposed optimal ZCZ sequence sets have also been proven to be cyclically distinct. %To the best of our knowledge, this is the first method that utilizes the FZT to generate multiple ZCZ sequence sets with optimal correlation properties. This approach offers the advantage of reducing both computational complexity and storage space requirements.
%Based on the existing circular Florentine arrays, this paper expands the circular Florentine arrays space of existence. The concepts of phase matrix and index matrix are proposed on the Zak domain, which enriches the theory of Zak transform. Based on the phase matrix and index matrix, a framework for constructing multiple sequence sets is proposed. Based on the different values of $R$ in the framework, multiple sequence sets with excellent correlation characteristics with different parameters are obtained by defining different phase matrices.

%The number of perfect sequences with optimal correlation depends on the parameter $T$, and the number of different ZCZ sets depends on  the minimum between ${R^*}-1$ and the number of rows in the circular Florentine arrays. Compared to \cite{ref16}, the proposed constructions in this paper increas the cases of circular Florentine arrays, and all sequences are cyclically distinct. \cite{ref17} is a special case of this paper and corresponds to only one sequence within the multiple sequence sets constructed herein. To our knowledge, this is the first time that sequence sets with optimal correlation have been constructed based on the Zak transform.

{\appendices
\section*{Proof of the Theorem 1}
\textit{Proof:} Initially demonstrate that the first property of the sequence sets is met. For $R=1$, $0 \le t<T$ and $0 \le l<L$, we have
\[\left| {\sum\limits_{{r} = 0}^{R - 1} {{P_u}\left( {t + {r}T} \right)w_L^{  l{r}T}} } \right|  =\left| {P_u}\left( t \right)\right|=1.\]

Additionally, $A$ is a circular Florentine array, it is known that $A^m$ is a permutation on $\mathbb{Z}_T$. And we have
 \[\left| {P_u\left( {t+ {r}T} \right)} \right| = \left| {P_u\left( t \right)} \right| = 1.\] 
According to Lemma 5 and 6, any sequence in ${S^m}$ obtained by Theorem 1 is unimodular and perfect.  This completes the proof of Part 1).

We now prove Part 2). Let ${\textbf{s}_u^m}$ and ${\textbf{s}_v^m}$ be two sequences
in $S^m$, where $0 \le u\ne v < T$ and $0 \le m <M$. Based on the IFZT and Lemma 7, for $\tau_1=0$, we distinguish between the following two cases to calculate (15).
%\begin{equation}
%\sum\limits_{{r} = 0}^{R - 1} {\sum\limits_{t = 0}^{T - 1} {w_L^{ -  {\tau_2}\left( {A^m\left( t\right) + {r}T} \right)}{P^m_u}\left( {t + {r}T} \right)P_v^{m*}\left( {t + {r}T} \right)}}
%\end{equation}

Case 1: When ${\tau_2} = 0$, (15) becomes
\begin{equation*}
\begin{aligned}
& {\sum\limits_{{r} = 0}^{R - 1} {\sum\limits_{t = 0}^{T - 1} {w_L^{  0\left( {A^m\left( t\right) + {r}T} \right)}{P_u}\left( {t + {r}T} \right)P_v^{*}\left( {t + {r}T} \right)} }}\\
 =& \sum\limits_{t = 0}^{T - 1} {{P_u}\left( t\right)P_v^{*}\left( t \right)} = \sum\limits_{t = 0}^{T - 1} {w_T^{\left( {u - v} \right)t}}.
\end{aligned}
\end{equation*}
The result follows from the fact that $\left( {{u - v}} \right)t$ is a permutation on  $\mathbb{Z}_T$ for any $u \ne v$.

Case 2: When ${\tau_2} =1$, (15) becomes
\begin{equation*}
\begin{aligned}
& {\sum\limits_{{r} = 0}^{R - 1} {\sum\limits_{t = 0}^{T - 1} {w_T^{  \left( {A^m\left( t\right) + {r}T} \right)}{P_u}\left( {t + {r}T} \right)P_v^{*}\left( {t + {r}T} \right)} } } \\
 =& \sum\limits_{t = 0}^{T - 1} {w_T^{  A^m\left( t\right)}{P_u}\left( t \right)P_v^{*}\left( t \right)}=\sum\limits_{{t} = 0}^{T -1} {w_T^{  A^m\left( {{t}} \right) + \left( {u - v} \right){t}}}.
\end{aligned}
\end{equation*}

Within Construction \uppercase\expandafter{\romannumeral1}, it is established that $A^m\left( {{t}} \right) \ne at$, where $a$ is a constant. This implies that $ A^m\left( {{t}} \right) + \left( {u - v} \right)t$ is neither 0 nor a multiple of $t$. Then,  $0< \left| {\sum\limits_{{t} = 0}^{T -1} {w_T^{  A^m\left( {{t}} \right) + \left( {u - v} \right){t}}} } \right|<T$ and $0 < \left| {{\theta _{{\bf{s}}_u^m,{\bf{s}}_v^m}}\left( T  \right) } \right| < T^2$.

Hence, the sequence set $S^m$ is an optimal $\left( {{T^2},T,T} \right) $-ZCZ set for the Tang-Fan-Matsufuji bound. Furthermore, all sequences in ${S^m}$ exhibit cyclic distinctness.

We now proceed to the third part of the theorem demonstration. Let's consider any two sequences ${\textbf{s}_u^{m_1}}$ and ${\textbf{s}_v^{m_2}}$ within $S^{m_1}$ and $S^{m_2}$ respectively, where $0 \le u\ne v < T$ and $0 \le {m_1} \ne {m_2} <M$. Let $n = t + lT, \tau  = {\tau _1} + {\tau _2}T$, with $0 \le t, {\tau _1} < T$ and $0 \le l, {\tau _2} < L$. Based on the equation 
\begin{equation*}
\label{e:barwq}
\begin{split}
&\left| {\sum\limits_{{r} = 0}^{R - 1} {P_u^{{m_1}}\left( {t + {\tau _1}+{r}T} \right){P_v^{{m_2}*}\left( {t +{r}T} \right)w_R^{ {r}{\tau _2}}} }} \right|\\
= &\left|{P_u^{m_1}\left( {t+{\tau _1}  } \right)} {P_v^{{m_2}*}\left( {t } \right)} \right|,
\end{split}
\end{equation*}
since each element of the phase matrix is a power of $w_T$, this implies that
\[\left| {\sum\limits_{{r} = 0}^{R - 1} {P_u^{{m_1}}\left( {t + {\tau _1}+{r}T} \right){P_v^{{m_2}*}\left( {t +{r}T} \right)w_R^{ {r}{\tau _2}}} }} \right|=1.\]

Recalling that $A$ is an ${F_c}\left( T \right) \times T$ circular Florentine array over ${\mathbb{Z}_T}$, from Lemma 8, we can assert that ${\theta _{{\bf{s}}_u^{{m_1}},{\bf{s}}_v^{{m_2}}}}\left( \tau  \right) = T$ for any $0 \le {m_1} \ne {m_2} <M, 0 \le u\ne v < T$ and $0 \le \tau  < {T^2}$.
\hfill
$\hfill\blacksquare$ 
\section*{Proof of the Theorem 2}
\textit{Proof:} For $R$ being odd,  $0 \le t<T$ and $0 \le l<L$, we have
%First we need to show that any sequence ${\textbf{c}_u^m}$ obtained by Theorem 2 is unimodular. According to Lemma 5 we need to ensure that $\left| {\sum\limits_{{r} = 0}^{R - 1} {{P_u^m}\left( {t + {r}T} \right)w_L^{  l{r}T}} } \right|= \sqrt R$, which makes the comparison harder, so let's calculate $\left( {\sum\limits_{{r} = 0}^{R - 1} {{P_u^m}\left( {t + {r}T} \right)w_L^{  l{r}T}} } \right){\left( {\sum\limits_{{r'} = 0}^{R - 1} {{P_u^m}\left( {t + {r'}T} \right)w_L^{ - r{r'}T}} } \right)^*}\\=R$.

\begin{equation}
\begin{aligned}
&\left|{\sum\limits_{{r} = 0}^{R - 1} {{P_u^m}\left( {t + {r}T} \right)w_L^{  l{r}T}} } \right|^2\\
 =&\left( {\sum\limits_{{r} = 0}^{R - 1} {w_R^{\left( {m + 1} \right)\frac{{{r}\left( {1 + {r}} \right)}}{2}}w_T^{ut}w_L^{  l{r}T}} } \right)\\
& {\left( {\sum\limits_{r{'} = 0}^{R - 1} {w_R^{\left( {m + 1} \right)\frac{{r{'}\left( {1 + r{'}} \right)}}{2}}w_T^{ut}w_L^{  lr{'}T}} } \right)^*}\\
=&\left( {\sum\limits_{{r} = 0}^{R - 1} {w_R^{\left( {m + 1} \right)\frac{{{r}\left( {1 + {r}} \right)}}{2}}w_R^{  l{r}}} } \right)\\
&{\left( {\sum\limits_{r{'} = 0}^{R - 1} {w_R^{-\left( {m + 1} \right)\frac{{r{'}\left( {1 + r{'}} \right)}}{2}}w_R^{  -lr{'}}} } \right)}
\end{aligned}
\end{equation}

By analyzing the behavior of the summation $\sum\limits_{{r} = 0}^{R - 1} {w_R^{\left( {m + 1} \right)\frac{{{r}\left( {1 + {r}} \right)}}{2}}w_R^{  l{r}}}$ for various values of $l$ and $m$, where $0 \le l<L$ and $0 \le m<M$, (30) is reduced to
\begin{equation}
\begin{aligned}
&\left|{\sum\limits_{{r} = 0}^{R - 1} {{P_u^m}\left( {t + {r}T} \right)w_L^{  l{r}T}} } \right|^2\\
%&\left( {\sum\limits_{{r} = 0}^{R - 1} {{P_u^m}\left( {t + {r}T} \right)w_L^{  l{r}T}} } \right){\left( {\sum\limits_{{r'} = 0}^{R - 1} {{P_u^m}\left( {t + {r'}T} \right)w_L^{ - r{r'}T}} } \right)^*}\\
=&\sum\limits_{{r} = 0}^{R - 1} {w_R^{\frac{{{r}\left( {1 + r} \right)}}{2}}}  {\sum\limits_{{r'} = 0}^{R - 1} {w_R^{\frac{{-{r'}\left( {1 + r'} \right)}}{2}}}  }
\end{aligned}
\end{equation}

Denoting $a = \left\lfloor {\frac{R}{2}} \right\rfloor $, where $\left\lfloor {x} \right\rfloor $ represents the greatest integer less than or equal to $x$, we can reformulate equation (31) as follows:
\begin{equation*}
\begin{aligned}
&\sum\limits_{{r} = 0}^{R - 1} {w_R^{\frac{{{r}\left( {1 + r} \right)}}{2}}}  {\sum\limits_{{r'} = 0}^{R - 1} {w_R^{\frac{{-{r'}\left( {1 + r'} \right)}}{2}}}  }\\
=& \left( {2\sum\limits_{{r} = 0}^{a - 1} {w_R^{ {\frac{{{r}\left( {1 + {r}} \right)}}{2}}}}  + w_R^{ {\frac{{a\left( {1 + a} \right)}}{2}} }} \right)\\
&\left( {2\sum\limits_{{r'} = 0}^{a - 1} {w_R^{ {\frac{{-{r'}\left( {1 + {r'}} \right)}}{2}} }}  + w_R^{{\frac{-{a\left( {1 + a} \right)}}{2}} }} \right)\\
 = &4 {\sum\limits_{{r} = 0}^{a - 1} {w_R^{{\frac{{{r}\left( {1 + {r}} \right)}}{2}} }} }  {\sum\limits_{{r'} = 0}^{a - 1} {w_R^{  {\frac{{-{r'}\left( {1 + {r'}} \right)}}{2}} }} } +1 \\
&+ 2w_R^{ {\frac{{a\left( {1 + a} \right)}}{2}} }{\sum\limits_{{r'} = 0}^{a - 1} {w_R^{  {\frac{{-{r'}\left( {1 + {r'}} \right)}}{2}}}} } +2w_R^{ {\frac{-{a\left( {1 + a} \right)}}{2}}}  {\sum\limits_{{r} = 0}^{a - 1} {w_R^{ {\frac{{{r}\left( {1 + {r}} \right)}}{2}}}} }  \\
=& 4\left( {a - \frac{{{a^2} - a}}{{R - 1}}} \right) - 2\frac{{2a}}{{R - 1}} + 1\\
= &4a - \frac{{4{a^2}}}{{R - 1}} + 1=R.\\
%= &2\left( {R - 1} \right) - \left( {R - 1} \right) + 1=R.
\end{aligned}
\end{equation*}


\begin{comment}
\begin{equation*}
\begin{aligned}
&\left( {\sum\limits_{{r} = 0}^{R - 1} {{P_u^m}\left( {t + {r}T} \right)w_L^{  l{r}T}} } \right){\left( {\sum\limits_{{r'} = 0}^{R - 1} {{P_u^m}\left( {t + {r'}T} \right)w_L^{ - r{r'}T}} } \right)^*}\\
 =& \left( {2\sum\limits_{{r} = 0}^{a - 1} {w_R^{ - r{r} + {\left( {m + 1} \right)\frac{{{r}\left( {1 + {r}} \right)}}{2}}}}  + w_R^{ - ra + {\left( {m + 1} \right)\frac{{a\left( {1 + a} \right)}}{2}} }} \right)\\
&\left( {2\sum\limits_{{r'} = 0}^{a - 1} {w_R^{r{r'} - {\left( {m + 1} \right)\frac{{{r'}\left( {1 + {r'}} \right)}}{2}} }}  + w_R^{ra - {\left( {m + 1} \right)\frac{{a\left( {1 + a} \right)}}{2}} }} \right)\\
 = &4\left( {\sum\limits_{{r} = 0}^{a - 1} {w_R^{ - r{r} +{\left( {m + 1} \right)\frac{{{r}\left( {1 + {r}} \right)}}{2}} }} } \right)\left( {\sum\limits_{{r'} = 0}^{a - 1} {w_R^{r{r'} - {\left( {m + 1} \right)\frac{{{r'}\left( {1 + {r}} \right)}}{2}} }} } \right) \\
+& 2w_R^{ - ra + {\left( {m + 1} \right)\frac{{a\left( {1 + a} \right)}}{2}} }\left( {\sum\limits_{{r'} = 0}^{a - 1} {w_R^{r{r'} - {\left( {m + 1} \right)\frac{{{r'}\left( {1 + {r'}} \right)}}{2}}}} } \right)\\
 +&2w_R^{ra- {\left( {m + 1} \right)\frac{{a\left( {1 + a} \right)}}{2}}} \left( {\sum\limits_{{r} = 0}^{a - 1} {w_R^{ - r{r} + {\left( {m + 1} \right)\frac{{{r}\left( {1 + {r}} \right)}}{2}}}} } \right) + 1\\
=& 4\left( {a - \frac{{{a^2} - a}}{{R - 1}}} \right) - 2\frac{{2a}}{{R - 1}} + 1\\
= &4a - \frac{{4{a^2}}}{{R - 1}} + 1\\
= &2\left( {R - 1} \right) - \left( {R - 1} \right) + 1=R.
\end{aligned}
\end{equation*}
\end{comment}

Through the above calculation and analysis, we can get $\left| {\sum\limits_{{r} = 0}^{R - 1} {{P_u^m}\left( {t + {r}T} \right)w_L^{  l{r}T}} } \right|= \sqrt R$. 

Additionally, $A$ is a circular Florentine array, it is known that $A^m$ is a permutation on $\mathbb{Z}_T$. And we have  
\[{\left| {{P_u^m}\left( {t+ {r}T} \right)} \right|}=\left| {w_R^{\left( {m + 1} \right)\frac{{{r}\left( {1 + {r}} \right)}}{2}}w_T^{ut}} \right|=1.\]
According to Lemma 5 and 6, any sequence in ${S^m}$ obtained by Theorem 2 is unimodular and perfect. This completes the proof of Part 1). 
\begin{comment}
 Let  ${\textbf{c}_u^m}$  be any sequence in ${C^m}$, $0 \le u < T,0 \le m <M$. 
\begin{equation*}
\begin{aligned}
&{\theta _{{\bf{c}}_u^m,{\bf{c}}_u^m}}\left( \tau  \right) \\
= &{\theta _{{\bf{c}}_u^m,{\bf{c}}_u^m}}\left( {t + rT} \right)\\
=&L^{-1}\sum\limits_{j \in {\mathbb{Z}_L}} {w_L^{ - rj}} \sum\limits_{t \in {\mathbb{Z}_T}} {X_{L,T}^{m,u}\left( {j,t} \right)X{{_L^{m,u}}^*}\left( {j,l - t} \right)} \\
=& \left\{ \begin{array}{l}
\frac{1}{L}\sum\limits_{j \in {\mathbb{Z}_L}} {w_L^{ - rj}} \sum\limits_{t \in {\mathbb{Z}_T}} {X_{L,T}^{m,u}\left( {j,t} \right)X{{_L^{m,u}}^*}\left( {j,t} \right)} ,t = 0\\
0,t \ne 0
\end{array} \right.
\end{aligned}
\end{equation*}

When $t = 0, r \ne 0$, 
\begin{equation*}
\begin{aligned}
&{\theta _{{\bf{c}}_u^m}}\left( {rT} \right) \\
=&L^{-1}\sum\limits_{j \in {\mathbb{Z}_L}} {w_L^{ - rj}} \sum\limits_{t \in {\mathbb{Z}_T}} {X_{L,T}^{m,u}\left( {j,t} \right)X{{_L^{m,u}}^*}\left( {j,t} \right)} \\
 =& T\sum\limits_{{r} = 0}^{R - 1} {\sum\limits_{t = 0}^{T - 1} {w_L^{ - r\left( {{r}T + A^m\left( t\right)} \right)}{P_u}\left( {l + {r}T} \right)P_u^*\left( {l + {r}T} \right)} } \\
=&T\sum\limits_{{r} = 0}^{R - 1} {w_L^{ - l{r}T}\sum\limits_{t = 0}^{T - 1} {w_L^{ - rA^m\left( t\right)}} }=0
\end{aligned}
\end{equation*}
where the last identity was due to $r \ne 0$, since $1 \le m < M$ and $A^m$ is a permutation of ${\mathbb{Z}_T}$. 
\end{comment}

We now prove Part 2). Let ${\textbf{s}_u^m}$ and ${\textbf{s}_v^m}$ be any two sequences in $S^m$, where $0 \le u\ne v < T$ and $0 \le m <M$. Based on the IFZT and Lemma 7, for $\tau_1=0$, we consider the following three cases to evaluate (15). 

Case 1: When ${\tau_2}=0$, (15) becomes
\begin{equation}
\begin{aligned}
&\sum\limits_{{r} = 0}^{R - 1} {\sum\limits_{t = 0}^{T - 1} {w_L^{  0\left( {A^m\left( t\right) + {r}T} \right)}{P_u^m}\left( {t+ {r}T} \right)P_v^{m*}\left( {t + {r}T} \right)} }\\
=&\sum\limits_{{r} = 0}^{R - 1}{\sum\limits_{t = 0}^{T - 1} {P_u^m}\left( {t+ {r}T} \right)P_v^{m*}\left( {t + {r}T} \right)} \\=&R\sum\limits_{t = 0}^{T - 1} {w_T^{\left( {u - v} \right)t}}.
\end{aligned}
\end{equation}
Then (32) is equal to zero followed by the fact that $\left( {{u - v}} \right)t$ is a permutation on $\mathbb{Z}_T$ for any $u \ne v$. %Hence we get ${\theta _{{\bf{c}}_u^m,{\bf{c}}_v^m}}\left( {0} \right)=0$.

Case 2: When $0 < {\tau_2}< R$, we have
\begin{equation}
\begin{array}{l}
\begin{aligned}
&\sum\limits_{{r} = 0}^{R - 1} {\sum\limits_{t = 0}^{T - 1} {w_L^{  {\tau_2}\left( {A^m\left( t\right) + {r}T} \right)}{P_u^m}\left( {t+ {r}T} \right)P_v^{m*}\left( {t + {r}T} \right)} }\\
=&\sum\limits_{{t} = 0}^{T - 1} {w_L^{  {\tau_2}{A^m}\left( {{t}} \right)}w_T^{\left( {u - v} \right){t}}} \sum\limits_{{r} = 0}^{R - 1} {w_R^{  {\tau_2}{r}} }.
\end{aligned}
\end{array}
\end{equation}
Due to  $\sum\limits_{{r} = 0}^{R - 1} {w_R^{  {\tau_2}{r}}}=0$ for ${\tau_2} \ne 0$, (33) is equal to zero for $0 < {\tau_2}< R$.

Case 3: When ${\tau_2} =R$, (15) becomes
\begin{equation*}
\begin{aligned}
&\sum\limits_{{r} = 0}^{R - 1} {\sum\limits_{t = 0}^{T - 1} {w_T^{  \left( {A^m\left( t\right) + {r}T} \right)}{P_u^m}\left( {t+ {r}T} \right)P_v^{m*}\left( {t + {r}T} \right)} }\\
=&\sum\limits_{{r} = 0}^{R - 1} {{w^{  {r}}}\sum\limits_{{t} = 0}^{T - 1} {w_T^{  A^m\left( {{t}} \right) + \left( {u - v} \right){t}}} }. \\
\end{aligned}
\end{equation*}

Within Construction \uppercase\expandafter{\romannumeral1}, it is established that $A^m\left( {{t}} \right) \ne at$, where $a$ is a constant. This implies that $ A^m\left( {{t}} \right) + \left( {u - v} \right)t$ is neither 0 nor a multiple of $t$. Therefore,  %$0< \left| {\sum\limits_{{l} = 0}^{T -1} {w_T^{ - A^m\left( {{t}} \right) + \left( {u - v} \right){l}}} } \right|<T$ 
 $0< \left|\sum\limits_{{r} = 0}^{R - 1} {{w^{  {r}}}\sum\limits_{{t} = 0}^{T - 1} {w_T^{  A^m\left( {{t}} \right) + \left( {u - v} \right){t}}} }\right|<RT$ and $0 < \left| {{\theta _{{\bf{s}}_u^m,{\bf{s}}_v^m}}\left( T  \right) } \right| < RT^2$.

From case 1, case2 and case3, we conclude that each $S^m$ is an optimal $\left( {{RT^2},T,RT} \right) $-ZCZ set respect to the Tang-Fan -Matsufuji bound. Furthermore, all sequences in ${S^m}$ exhibit the property of cyclic distinctness.

Finally, we prove Part 3). Let ${\textbf{s}_u^{m_1}}$ and ${\textbf{s}_v^{m_2}}$ be two sequences in $S^{m_1}$ and $S^{m_2}$, respectively, where $0 \le u\ne v < T$ and $0 \le {m_1} \ne {m_2} <M$. Let $n = t + lT, \tau  = {\tau _1} + {\tau _2}T, 0 \le t,{\tau _1} < T, 0 \le l, {\tau _2} < L$, we have
\begin{equation*}
\label{e:barwq}
\begin{split}
&\left| {\sum\limits_{{r} = 0}^{R - 1} {P_u^{{m_1}}\left( {t + {\tau _1}+{r}T} \right){P_v^{{m_2}*}\left( {t +{r}T} \right)w_R^{ {r}{\tau _2}}} }} \right|\\
=&\left| {w_T^{u({t+ {\tau _1})-vt}}\sum\limits_{{r} = 0}^{R - 1} {w_R^{\left( {{m_1}-{m_2}} \right)\frac{{{r}\left( {1 + r} \right)}}{2}{+r}{\tau _2}}} } \right|\\
= &\left| {\sum\limits_{{r} = 0}^{R - 1} {w_R^{ {r} {\tau _2}}w_R^{\left( {{m_1} - {m_2}} \right)\frac{{{r}\left( {1 + {r}} \right)}}{2}}} } \right|.
\end{split}
\end{equation*}

%Since 
%\begin{equation}
%\begin{aligned}
%&\left| {\sum\limits_{{r} = 0}^{R - 1} {{P_u^m}\left( {t + {r}T} \right)w_L^{  l{r}T}} } \right|\\
% =& \left| {w_T^{uk}\sum\limits_{{r} = 0}^{R - 1} {w_R^{\left( {m + 1} \right)%{\frac{{{r}\left( {1 + {r}} \right)}}{2}}}w_R^{ - r{r}}} } \right|=\sqrt R
%\end{aligned}
%\end{equation}
We observe that the above  equation shares a similar structure with (31). This allows us to exploit an analogous approach and conclude that
%the equation $\left| {\sum\limits_{{r} = 0}^{R - 1} {w_R^{ - {\tau _2}{r}}w_R^{\left( {{m_1} - {m_2}} \right){\frac{{{r}\left( {1 + {r}} \right)}}{2}} }} } \right|$ has the similar form as in (26), we can also get 
\begin{equation*}
\left| {\sum\limits_{{r} = 0}^{R - 1} {w_R^{ {r}{\tau _2}}w_R^{\left( {{m_1} - {m_2}} \right){\frac{{{r}\left( {1 + {r}} \right)}}{2}} }} } \right|=\sqrt R.
\end{equation*}

According to Lemma 8, we have $\left| {{\theta _{{\bf{s}}_u^{{m_1}},{\bf{s}}_v^{{m_2}}}}\left( \tau  \right)} \right| = \sqrt R T$ for all $0 \le {m_1} \ne {m_2} <M, 0 \le u\ne v < T$ and $0 \le \tau  < R{T^2}$. 
\hfill
$\hfill\blacksquare$ 
\section*{Proof of the Theorem 3}
\textit{Proof:} For $R$ being even, $0 \le t<T$ and $0 \le l<L$, we have
\begin{equation}
\begin{aligned}
&\left|{\sum\limits_{{r} = 0}^{R - 1} {{P_u}\left( {t + {r}T} \right)w_L^{  l{r}T}} } \right|^2\\
 =& \left( {\sum\limits_{{r} = 0}^{R - 1} {w_{2R}^{{r}^2}w_T^{ut}w_L^{  l{r}T}} } \right)\left( {\sum\limits_{{r'} = 0}^{R - 1} {w_{2R}^{ - {r'}^2}w_T^{  -ut}w_L^{-l{r'}T}} } \right)\\
 =& \left( {\sum\limits_{{r} = 0}^{R - 1} {w_{2R}^{{r}^2}w_L^{l{r}T}} } \right)\left( {\sum\limits_{{r'} = 0}^{R - 1} {w_{2R}^{-{r'}^2}w_L^{-l{r'}T}} } \right)
\end{aligned}
\end{equation}

By analyzing the values of  $\sum\limits_{{r} = 0}^{R - 1} {w_{2R}^{{r}^2}w_L^{  l{r}T}}$ for different values of $r$, (34) can be reduced to

\begin{equation}
%\begin{aligned}
\left|{\sum\limits_{{r} = 0}^{R - 1} {{P_u}\left( {t + {r}T} \right)w_L^{  l{r}T}} } \right|^2
={\sum\limits_{{r} =0}^{R - 1} {w_{2R}^{{r}^2 }} } {\sum\limits_{{r'} = 0}^{R - 1} {w_{2R}^{-{r'}^2 }} } .
%\end{aligned}
\end{equation}

Let $a = \frac{R}{2}$, through further analysis, (35) can be reformulated as
\begin{equation*}
\begin{aligned}
&\left( {\sum\limits_{{r} = 1}^{a - 1} {w_{2R}^{{r}^2}} +w_{2R}^{0 }+w_{2R}^{{a^2}}} \right)\left( {\sum\limits_{{r'} = 1}^{a - 1} {w_{2R}^{-{r'}^2}} +w_{2R}^{0 }+w_{2R}^{-{a^2}}} \right)\\
&=  {\sum\limits_{{r} = 1}^{a - 1} {w_{2R}^{{r}^2}} }  {\sum\limits_{{r'} = 1}^{a - 1} {w_{2R}^{-{r'}^2}} }  +\left( {w_{2R}^{0 }+w_{2R}^{{a^2}}} \right) {\sum\limits_{{r'} = 1}^{a - 1} {w_{2R}^{-{r'}^2}} } \\
&+\left( {w_{2R}^{0 }+w_{2R}^{-{a^2}}} \right){\sum\limits_{{r} = 1}^{a - 1} {w_{2R}^{{r}^2}} }+ \left( {w_{2R}^{0 }+w_{2R}^{{a^2}}} \right)\left( {w_{2R}^{0 }+w_{2R}^{-{a^2}}} \right)\\
 &= 4\left( {a - 1} \right) + 2 - 2\left( {a - 1} \right)=R.
\end{aligned}
\end{equation*}

Then, we get $\left| {\sum\limits_{{r} = 0}^{R - 1} {{P_u}\left( {t + {r}T} \right)w_L^{  l{r}T}} } \right|= \sqrt R$. 

Additionally, $A$ is a permutation on $\mathbb{Z}_T$. And we have  
\[\left| {{P_u}\left( {t + {r}T} \right)} \right|  = \left| {w_{2R}^{{r}^2}w_T^{ut}} \right| =1.\]
We assert that any sequence in ${S}$ is unimodular and perfect.  This completes the proof of Part 1). 

As the demonstration here closely parallels the approach used in Theorem 2 Part 2), a detailed proof is skipped to avoid redundancy. 
%The proof in the Part 2) is similar to the proof in the Part 2) of Theorem 2, so the proof in this part is omitted.
\begin{comment}
We now prove Part 2). Let ${\textbf{c}_u}$ and ${\textbf{c}_v}$ be two sequences
in $C$, where $0 \le u\ne v < T$. Based on the FZT, the cross-correlation of  ${\textbf{c}_u}$ and ${\textbf{c}_v}$ is calculated in the following three cases.

Case 1: When ${\tau_1} = 0$ and ${\tau_2}=0$, we have
\begin{equation}
\begin{aligned}
&\sum\limits_{{r} = 0}^{R - 1} {\sum\limits_{t = 0}^{T - 1} {w_L^{ - 0\left( {A\left( l \right) + {r}T} \right)}{P_u}\left( {l + {r}T} \right)P_v^*\left( {l + {r}T} \right)} }\\
=&R\sum\limits_{{r} = 0}^{R - 1} {\sum\limits_{t = 0}^{T - 1} {{P_u}\left( {l + {r}T} \right)P_v^*\left( {l + {r}T} \right)} }\\
=&R\sum\limits_{t = 0}^{T - 1} {w_T^{\left( {u - v} \right)t}}.
\end{aligned}
\end{equation}
Then (31) is equal to zero followed by the fact that $\left( {{u - v}} \right)l$ is a permutation on $\mathbb{Z}_T$ for any $u \ne v$. Hence we get ${\theta _{{\bf{c}}_u,{\bf{c}}_v}}\left( {0} \right)=0$.

Case 2: When ${\tau_1} = 0$ and $0 \le {\tau_2} < R$, we have
\begin{equation*}
\begin{array}{l}
\begin{aligned}
&\sum\limits_{{r} = 0}^{R - 1} {\sum\limits_{t = 0}^{T - 1} {w_L^{ - {\tau_2}\left( {A\left( l \right) + {r}T} \right)}{P_u}\left( {t + {r}T} \right)P_v^{*}\left( {t + {r}T} \right)} }\\
=&\sum\limits_{{k_1} = 0}^{T -1} {w_L^{ - {\tau_2}{A}\left( {{k_1}} \right)}w_T^{\left( {u - v} \right){k_1}}}\sum\limits_{{r} = 0}^{R - 1} {w_R^{ - {\tau_2}{r}} },
\end{aligned}
\end{array}
\end{equation*}
Due to  $\sum\limits_{{r} = 0}^{R - 1} {w_R^{ - {\tau_2}{r}}}=0$ for ${\tau_2} \ne 0$. Hence we get  ${\theta _{{\bf{c}}_u,{\bf{c}}_v}}\left( {{\tau_2}} \right)=0$ for $0 < {\tau_2}< R$.

Case 3: When ${\tau_1} = 0$ and $ {\tau_2} =R$, we get
\begin{equation*}
\begin{aligned}
&\sum\limits_{{r} = 0}^{R - 1} {\sum\limits_{t = 0}^{T - 1} {w_T^{ - \left( {A\left( l \right) + {r}T} \right)}{P_u}\left( {l + {r}T} \right)P_v^*\left( {l + {r}T} \right)} }\\
=&\sum\limits_{{r} = 0}^{R - 1} {{w^{ - {r}}}\sum\limits_{{k_1} = 0}^{T -1} {w_T^{ - A\left( {{k_1}} \right) + \left( {u - v} \right){k_1}}} }  \\
\end{aligned}
\end{equation*}


Since $A$ is the circular Florentine array $F^q$ obtained by Construction \uppercase\expandafter{\romannumeral1}, $A\left( {{l}} \right) \ne al$, where $a$ is a constant. Therefore $- A\left( {{l}} \right) + \left( {u - v} \right)t$ is not equal to 0 or not a multiple of $l$. We can get $0 < \left| {{\theta _{{\bf{c}}_u,{\bf{c}}_v}}\left( T  \right) } \right| < RT^2$.

According to Lemma 7 and the discussion of the three cases above, we get the conclusion that the $C$ is an optimal $\left( {{RT^2},T,RT} \right) $-ZCZ set.  Furthermore, all sequences in ${C}$ are cyclically distinct.
\end{comment}
}
\hfill
$\hfill\blacksquare$ 
\begin{thebibliography}{99}
\bibliographystyle{IEEEtran}

{\bibitem{Luke2003}
H. Luke, H. Schotten, and H. Mahram, ``Binary and quadriphase sequences with optimal autocorrelation properties: A survey," \emph{IEEE Trans. Inf. Theory}, vol. 49, no. 12, pp. 3271-3282, Dec. 2003.

\bibitem{Mow1996}
W. H. Mow, ``A new unified construction of perfect root-of-unity sequences,"
in \emph{Proc. Spread-Spectrum Techn. Appl.}, Mainz, Germany, Sep. 1996, pp. 955-959.
}
\bibitem{Sarwate79}
D. Sarwate, ``Bounds on crosscorrelation and autocorrelation of sequences (corresp.)," \emph{IEEE Trans. Inf. Theory}, vol. IT-25, no. 6, pp. 720–724, Nov. 1979.

\bibitem{Fan99}
P. Fan, N. Suehiro, N. Kuroyanagi, and X. Deng, ``A class of binary
sequences with zero correlation zone," \emph{Electron. lett.}, vol. 35, pp. 777-779, May 1999.

\bibitem{ref2}
R. De Gaudenzi, C. Elia and R. Viola, ``Bandlimited quasi-synchronous CDMA: a novel satellite access technique for mobile and personal communication systems," \emph{IEEE J. Sel. Areas Commun.}, vol. 10, no. 2, pp. 328–343,  May. 1992.

%\bibitem{ref4}
% M.Addad, A. Djebbari, and I. Dayoub, ``Performance of ZCZ codes in
% QS-DS-CDMA communication systems,” \textit{ Signal Process.}, vol. 164, pp. 146–150, Nov. 2019.

%\bibitem{ref5}
%X. Tang, and W. H. Mow, ``Design of spreading codes for quasi-synchronous CDMA with intercell interference,” \textit{ IEEE J. Sel. Areas Commun.}, vol. 24, no. 1, pp. 84–93, Jan. 2006.

%\bibitem{ref6}
%B. M. Popovic, and O. Mauritz, ``Generalized chirp-like sequences
%with zero correlation zone,” \textit{ IEEE Trans. Inf. Theory}, %vol. 56, no. 6, pp. 2957–2960, Jun. 2010.

\bibitem{ref6}
 S. Matsufuji, N. Kuroyanagi, N. Suehiro,and P . Z. Fan, ``Two types of polyphase sequence sets for approximately synchronized CDMA systems," \emph{IEICE Trans. Fundam. Electron., Commun. Comput. Sci.}, vol. E86-A, no. 1, pp. 229–234, Jan. 2003.

\bibitem{Yang02}
S. A. Yang and J. Wu, ``Optimal binary training sequence design for multiple-antenna systems over dispersive fading channels," \emph{IEEE Trans. Veh. Technol.,} vol. 51, pp. 1271-1276, Sept. 2002.

\bibitem{Fragouli03}
C. Fragouli, N. Al-Dhahir, and W. Turin, ``Training-based channel estimation for multiple-antenna broadband transmissions," \emph{IEEE Trans. Wireless Commun.,} vol. 2, no. 2, pp. 384-391, Mar. 2003.

\bibitem{Fan04}
P. Fan and W. Mow, ``On optimal training sequence design for multiple-antenna systems over dispersive fading channels and its extensions," \emph{IEEE Trans. Veh. Technol.,} vol. 53, no. 5, pp. 1623-1626, Sep. 2004.

\bibitem{Yang08}
J.-D. Yang, X. Jin, t.-Y. Song, J.-S. No, and D.-J. Shin, ``Multicode MIMO systems with quaternary LCZ and ZCZ sequences," \emph{IEEE Trans. Veh. Technol.}, vol. 57,no. 4, pp. 2334–2341, Jul. 2008.

\bibitem{Zhang12}
R. Zhang, X. Cheng, M. Ma, and B. Jiao, ``Interference-avoidance pilot design using ZCZ sequences for multi-cell MIMO-OFDM systems," in \emph{Proc. IEEE Global Communication Conf.}, Anaheim, CA, USA, 2012, pp. 5056–5061.

\bibitem{Tang00}
X. Tang, P. Fan, and S. Matsufuji, ``Lower bounds on correlation of spreading sequence set with low or zero correlation zone," \emph{IEE Electron. Lett.}, vol. 36, no. 6, pp. 551–552, Mar. 2000.

\bibitem{Peng80}
D. Y. Peng and P. Z. Fan, ``Generalized Sarwate bounds on the periodic autocorrelations and crosscorrelations of binary sequences," \emph{IEE Electron. Lett.}, Vol.38, No.24, pp.1521-1523, Nov. 2002.

\bibitem{Torii04}
H. Torii, M. Nakamura, and N. Suehiro, ``A new class of zero-correlation zone sequences," \emph{IEEE Trans. Inf. Theory}, vol. 50, no. 3, pp. 559-565, Mar. 2004.

\bibitem{Hayashi09}
T. Hayashi, ``A class of zero-correlation zone sequence set using a perfect sequence," \emph{IEEE Signal Process. Lett.}, vol. 16, no. 4, pp. 331-334, Mar. 2009.

\bibitem{Popovic10}
 B. M. Popovic and O. Mauritz, ``Generalized chirp-like sequences with zero correlation zone," \emph{IEEE Trans. Inf. Theory}, vol. 56, no. 6, pp. 2957–2960, Jun. 2010.

\bibitem{Hu10}
H. Hu and G. Gong, ``New sets of zero or low correlation zone sequences via interleaving techniques," \emph{IEEE Trans. Inf. Theory}, vol. 56, no. 4, pp. 1702 - 1713, Apr. 2010.

\bibitem{Deng00}
X. Deng and P. Fan, ``Spreading sequence sets with zero correlation zone," \emph{Electron. Lett.}, vol. 36,no. 11, pp. 993–994, May. 2000.

\bibitem{Appuswamy06}
R. Appuswamy and A. t. Chaturvedi, ``A new framework for constructing mutually orthogonal complementary sets and ZCZ sequences," \emph{IEEE Trans. Inf. Theory},  vol. 52, no. 8, pp. 3817–3826, Aug. 2006.

\bibitem{Golay61}
M. J. E. Golay, ``Complementary series," \emph{IRE Trans. Inf. Theory}, vol. IT-7, no. 2, pp. 82-87, Apr. 1961.

\bibitem{Tseng72}
C. Tseng and C. Liu, ``Complementary sets of sequences," \emph{IEEE Trans. Inf. Theory}, vol. IT-18, no. 5, pp. 644-665, Sept. 1972.

\bibitem{Liu13}
Z. Liu, Y. Li, and Y. L. Guan, ``New constructions of general QAM Golay complementary sequences," \emph{IEEE Trans. Inf. Theory}, vol. 59, no. 11, pp. 7684-7692, Nov. 2013.

\bibitem{Liu14}
Z. Liu, Y. L. Guan, and U. Parampalli, ``New complete complementary codes for peak-to-mean power control in multi-carrier CDMA," \emph{IEEE Trans. Commun.}, vol. 62, no. 3, pp. 1105-113, Mar. 2014.

\bibitem{Liu14-ITW}
Z. Liu, Y. L. Guan, and U. Parampalli, ``A new construction of zero correlation zone sequences from generalized Reed-Muller codes," in \emph{Proc. IEEE Information Theory Workshop (ITW'2014)}, Nov. 2014, pp. 591-595.

\bibitem{Tang10}
X. Tang, P. Fan, and J. Lindner, ``Multiple binary ZCZ sequence sets with good cross-correlation property based on complementary sequence sets," \emph{IEEE Trans. Inf. Theory},  vol. 56, no. 8, pp. 4038–4045, Aug. 2010.

\bibitem{Zhou17}
Z. Zhou, D. Zhang, and T. Helleseth, ``A construction of multiple optimal ZCZ sequence sets with good cross correlation," \emph{IEEE Trans. Inf. Theory}, vol. 64,no. 2, pp. 1340–1346, Feb. 2018.

{\bibitem{Golomb2005}
S. W. Golomb and G. Gong, ``Signal Design With Good Correlation: For Wireless Communications, Cryptography and Radar Applications," \emph{Cambridge, U.K.: Cambridge Univ. Press}, 2005.
}
\bibitem{Zhang20}
D. Zhang, M. G. Parker and T. Helleseth, ``Polyphase zero correlation zone sequences from generalised bent functions," \emph{Cryptogr. Commun.}, vol. 12 , no. 3, pp. 325–335,May 2020.

\bibitem{Zhang22}
D. Zhang, and T. Helleseth, ``Sequences with good correlations based on circular Florentine arrays," \emph{IEEE Trans. Inf. Theory}, vol. 68,no. 5, pp. 3381–3388, May. 2022.

\bibitem{Song21}
M. Song, and H.-Y. Song, ``New framework for sequences with perfect autocorrelation and optimal crosscorrelation," \emph{IEEE Trans. Inf. Theory}, vol. 67,no. 11, pp. 7490–7500, Nov. 2021.

\bibitem{ref18}
A. J. E. M. Janssen, ``The Zak transform: A signal transform for sampled time-continuous signals," \emph{Philips J. Res.}, vol. 43, pp. 23-69, Jan. 1988.

\bibitem{ref19}
 A. K. Brodzik, ``On the Fourier transform of finite chirps," \emph{IEEE Signal Process. Lett.}, vol. 13, no. 9, pp. 541-544, Sep. 2006.

\bibitem{ref20}
 A. K. Brodzik, ``Characterization of Zak space support of a discrete chirp," \emph{IEEE Trans. Inf. Theory}, vol. 53, no. 6, pp. 2190-2203, Jun. 2007.

\bibitem{ref21}
M. An, A. K. Brodzik, I. Gertner, and R. Tolimieri, ``Weyl-Heisenberg systems and the finite Zak transform," in \emph{Signal and Image Representation in Combined Spaces}, vol.7, Y. Zeevi and R. Coifman, Eds. San Diego, Calif, USA: Academic Press, 1998,  pp. 3-22.

\bibitem{ref22}
A. K. Brodzik and R. Tolimieri, ``Extrapolation of band-limited signals and the finite Zak transform," \emph{Signal Processing}, vol. 80, no. 3, pp. 413-423, Feb. 2000.

%\bibitem{ref23}
%I. Daubechies, S. Jaffard, and J.-L. Journe, ``A simple Wilson orthonormal basis with exponential decay," \emph{SIAM Journal on Mathematical Analysis}, vol. 22, no. 2, pp. 554-573, Mar. 1991.

\bibitem{ref24}
A. K. Brodzik, ``Construction of sparse representations of perfect polyphase sequences in Zak space with applications to radar and communications," \emph{EURASIP J. Adv. Signal Process. (Spec. Issue Recent Adv. Non-Station. Signal Process.)}, vol. 2011, no. 1, pp. 214790-1--214790-14, Jan. 2011.

\bibitem{ref25}
A. K. Brodzik, `On certain sets of polyphase sequences with sparse and highly structured Zak and Fourier transforms," \emph{IEEE Trans. Inf. Theory}, vol. 59, no. 10, pp. 6907-6916, Oct. 2013.

\bibitem{Peng24}
X. Peng, C. Wu, and H. Lin, ``Multiple SNC-ZCZ sequence sets with optimal correlations based on Zak transforms," \emph{IEEE. Signal Process. Lett.}, vol. 31, pp. 1464-1468, May 2024. 

{
\bibitem{6GV2X}
M. Noor-A-Rahim, Z. Liu, H. Lee, M. O. Khyam, J. He, D. Pesch, K. Moessner, W. Saad, and H. V. Poor, ``6G for vehicle-to-everything (V2X) communications: Enabling technologies, challenges, and opportunities," \emph{Proc. IEEE}, vol. 110, no. 6, pp. 712–734, June 2022.

\bibitem{Hadani2017}
R. Hadani et al., ``Orthogonal time frequency space modulation," in \emph{Proc. IEEE Wireless Commun. Netw. Conf. (WCNC)}, San Francisco, CA, USA, Mar. 2017, pp. 1-6.

\bibitem{Viterbo2018}
P. Raviteja, K. T. Phan, Y. Hong, and E. Viterbo, ``Interference cancellation and iterative detection for orthogonal time frequency space modulation," \emph{IEEE Trans. Wireless Commun.}, vol. 17, no. 10, pp. 6501-6515, Oct. 2018.

\bibitem{Saif2022}
S. Mohammed, R. Hadani, A. Chockalingam, and R. Calderbank, ``OTFS -- a mathematical foundation for communication and radar sensing in the delay-Doppler domain," \emph{IEEE BITS the Inform. Theory Mag.}, vol. 2, no. 2, pp. 36-55, Nov. 2022.

\bibitem{Saif2023}
S. Mohammed, R. Hadani, A. Chockalingam and R. Calderbank, ``OTFS -- predictability in the Delay-Doppler domain and its value to communication and radar sensing," \emph{IEEE BITS the Inform. Theory Mag.}, vol. 3, no. 2, pp. 7-31, June 2023.

\bibitem{Lampel2022}
F. Lampel, H. Joudeh, A. Alvarado, and F. Willems, ``Orthogonal time frequency space modulation based on the discrete Zak transform," vol. 24, no. 1704, pp. 1-19, Entropy, Nov. 2022. 

\bibitem{Viterbo2020}
 K. Sinha, S. Mohammed, P. Raviteja, Y. Hong, and E. Viterbo, ``OTFS based random access preamble transmission for high mobility scenarios," \emph{IEEE Trans. Veh. Technol.}, vol. 69, no. 12, pp. 15078-15094, Dec. 2020.
 
\bibitem{Cho2021}
M. Khan, Y. Kim, Q. Sultan, J. Joung, and Y. Cho, ``Downlink synchronization for OTFS-based cellular systems in high Doppler environments," \emph{IEEE Access}, vol. 9, pp. 73575-73589, May 2021.

\bibitem{Farhang2022}
M. Bayat and A. Farhang, ``Time and frequency synchronization for OTFS," \emph{IEEE Wireless Commun. Lett.}, vol. 11, no. 12, pp. 2670-2674, Dec. 2022.

\bibitem{Chung2024}
C.-D. Chung, M.-Z. Xu and W.-C. Chen, ``Initial time synchronization for OTFS," \emph{IEEE Trans. Veh. Technol.}, vol. 73, no. 12, pp. 18769-18786, Dec. 2024.

\bibitem{Yogesh2024}
V. Yogesh, S. Mattu, A. Chockalingam, ``Low-complexity delay-Doppler channel estimation in discrete Zak transform based OTFS," \emph{IEEE Commu. Lett.}, vol. 28, no. 3, pp. 672-676, Mar. 2024.

\bibitem{Zegrar2024}
S. Zegrar, A. Boudjelal, H. Arslan, ``A novel OTFS-Chirp waveform for low-complexity multi-user joint sensing and communication," \emph{IEEE Internet Things J.}, early access, DOI: 10.1109/JIOT.2024.3509922.
}
\bibitem{ref27}
T. Etzion, S. W. Golomb, and H. Taylor, ``Tuscan-ksquares," \emph{Adv. Appl. Math.}, 10(1989), 164–174.

\bibitem{ref28}
H. Y. Song and J. H. Dinitz, ``Tuscan squares," \emph{CRC handbook of combinatorial designs}, pp. 480–484, CRC Press, New York,1996.

\bibitem{ref29}
H. Taylor, ``Florentine rows or left-right shifted permutation matrices with cross-correlation values $\le 1$," \emph{Discrete Math.}, 93(1991), 247–260.

\bibitem{ref30}
H.-Y. Song, ``The existence of Florentine arrays," \emph{Comput. Math. With Appl.}, vol. 39, no. 11, pp. 31–35, Jun. 2000.

\bibitem{ref31}
S. Golomb, T. Etzion, and H. Taylor, ``Polygonal path constructions for tuscan-t squares," \emph{Ars Combinatoria}, vol. 30, pp. 97–140, Dec. 1990.

\bibitem{Pitaval21}
R.-A. Pitaval, B. M. Popovic, P. Wang and F. Berggren, ``Overcoming 5G PRACH capacity shortfall: Supersets of Zadoff–Chu sequences with low-correlation zone," \emph{IEEE Trans. Commun.}, vol. 68, no. 9, pp. 5673– 5688, Sep. 2020.

{
\bibitem{LAZ2022}
Z. Ye, Z. Zhou, P. Fan, Z. Liu, X. Lei, and X. Tang, ``Low ambiguity zone: theoretical bounds and Doppler-resilient sequence design in integrated sensing and communication systems," \emph{IEEE J. Sel. Areas Commun.}, vol. 40, no. 6, pp. 1809-1822, Jun. 2022.
}

%\bibitem{Liang2024}
%Y. Liang, P.Z. Fan, Q.L. Wang, et al., ``Two-Dimensional Delay-Doppler Pilots and Channel Estimation for Multi-Antenna OTFS in Doubly Dispersive Channels,” \textit{IEEE Trans. Wire. Commu.}, vol. 23, no. 7, pp. 7612-7623, Jul. 2024.


%\bibitem{OTFS10}
%F. Jesbin and A. Chockalingam, "Near-Optimal Detection of Zak-OTFS Signals," ICC 2024 - IEEE International Conference on Communications, Denver, CO, USA, 2024, pp. 4476-4481

%\bibitem{OTFS12}
%S. Gopalam, H. Inaltekin, I. B. Collings and S. V. Hanly, "Optimal Zak-OTFS Receiver and Its Relation to the Radar Matched Filter," in IEEE Open Journal of the Communications Society, vol. 5, pp. 4462-4482, 2024.
%\bibitem{OTFS13}
% S. Gopalam, I. B. Collings, S. V. Hanly, H. Inaltekin, S. R. B. Pillai,
%and P. Whiting, “Zak-OTFS Implementation via Time and Frequency
%Windowing,” IEEE Transactions on Communications, vol. 72, no. 7,
%pp. 3873–3889, jul 2024.

%\bibitem{Tian 2024}
%L.Y. Tian, Y. Ma, W.M. Lei, Y. B. Li, ``A Direct Construction of ZCZ Sequences With Low Peak-to-Average Power Ratio,” \textit{IEEE Commu. Lett.}, vol. 28, no. 2, pp. 273-277, Feb. 2024.

\end{thebibliography}

\end{document}


