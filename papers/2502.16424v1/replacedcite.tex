\section{Related Work}
\subsection{Image SemCom systems}
Zhang et al. ____ proposed a Predictive and Adaptive Deep Coding (PADC) framework to achieve flexible bitrate optimization in image SemCom, ensuring specific transmission quality requirements. The PADC framework is capable of transmitting images over wireless channels with minimal bandwidth consumption while maintaining the Peak Signal-to-Noise Ratio (PSNR) constraints for each image data. 
Liu et al. ____ introduced a novel image SemCom method that combines Dynamic Decision Generation Networks (DDGN) and Generative Adversarial Networks (GANs) to effectively compress and transmit images while maintaining a high compression ratio and reducing distortion in the reconstructed images. This method demonstrated superior performance in low Signal-to-Noise Ratio (SNR) wireless communication environments. 
Pan et al. ____ proposed an image segmentation-based SemCom system for efficient visual data transmission in vehicular networks. The system extracts semantic information from images using a multi-scale semantic information extractor based on Swin Transformer at the sender side, and at the receiver side, and it uses a semantic information decoder and reconstructor to combat wireless channel noise and reconstruct image segmentation, thereby achieving high compression ratios and robust transmission performance under limited spectrum resources.

All of these works adopted a direct semantic encoding approach for images, aiming to eliminate redundancy at the semantic level. However, they did not consider information compression at the pixel level of the image.

\subsection{Large AI models for SemCom systems}
Jiang et al. ____ proposed a large-model-based multimodal SemCom framework, which utilizes Multimodal Language Models (MLMs) to facilitate the conversion between multimodal and unimodal data while maintaining semantic consistency. They also introduced a personalized LLM knowledge base to perform personalized semantic extraction or recovery, effectively addressing the issue of semantic ambiguity. 
Zhao et al. ____ presented a SemCom system called LaMoSC, which employs an LLM-driven multimodal fusion framework to reconstruct original visual information. The system integrates visual and textual multimodal feature inputs through an end-to-end encoder-decoder network to improve visual transmission quality under low SNR conditions. 
Wang et al. ____ proposed an LLM-based end-to-end learning SemCom model, which leverages the semantic understanding capabilities of LLMs to design semantic encoders and decoders. This model enhances semantic fidelity and cross-scene generalization by using subword-level tokenization, gradient-based rate adapters, and task-specific fine-tuning, improving performance under various channel encoding-decoding rate requirements.

However, these SemCom systems overlook the high energy consumption and latency caused by the enormous parameter numbers of LAMs, as well as the resource constraints of edge devices in mobile communication systems.

\subsection{Multi-user SemCom systems}
Xie et al. ____ studied task-oriented multi-user SemCom systems and proposed a Transformer-based framework to unify the transmitter architecture for different tasks, including image retrieval, machine translation, and visual question answering. They introduced the DeepSC-IR, DeepSC-MT, and DeepSC-VQA models for unimodal and multimodal multi-user systems, respectively. 
Li et al. ____ proposed a Non-Orthogonal Multiple Access (NOMA)-based multi-user SemCom system (NOMASC), which supports semantic transmission for multiple users with different source information modalities. The system uses asymmetric quantizers and neural network models for symbol mapping and intelligent multi-user detection. 
Mu et al. ____ introduced an innovative heterogeneous semantic and bit multi-user communication framework that adopts a semi-NOMA scheme to effectively facilitate heterogeneous semantic and bit multi-user communication. They also proposed an opportunistic semantic and bit communication method to alleviate the early and late rate difference issues in NOMA.

These multi-user SemCom systems rely on physical resources to differentiate user signals and fail to exploit the commonalities and differences between the information of different users in the semantic space. As LAMs become increasingly capable of extracting more precise semantic information, leveraging the commonality among semantics transmitted by different users to further enhance the efficiency of SemCom systems represents a novel research direction.