%%%%%%%%%%%%%%%%%%%%%%%%%%%%%%%%%%%%%%%%%%%%%%%%%%%%%%%%%%%%%%%%%%%%%
%%                                                                 %%
%% Please do not use \input{...} to include other tex files.       %%
%% Submit your LaTeX manuscript as one .tex document.              %%
%%                                                                 %%
%% All additional figures and files should be attached             %%
%% separately and not embedded in the \TeX\ document itself.       %%
%%                                                                 %%
%%%%%%%%%%%%%%%%%%%%%%%%%%%%%%%%%%%%%%%%%%%%%%%%%%%%%%%%%%%%%%%%%%%%%

% \documentclass[referee,sn-basic]{sn-jnl}% referee option is meant for double line spacing

%%=======================================================%%
%% to print line numbers in the margin use lineno option %%
%%=======================================================%%

%%\documentclass[lineno,sn-basic]{sn-jnl}% Basic Springer Nature Reference Style/Chemistry Reference Style

%%======================================================%%
%% to compile with pdflatex/xelatex use pdflatex option %%
%%======================================================%%

%%\documentclass[pdflatex,sn-basic]{sn-jnl}% Basic Springer Nature Reference Style/Chemistry Reference Style

%%\documentclass[sn-basic]{sn-jnl}% Basic Springer Nature Reference Style/Chemistry Reference Style
% \documentclass[sn-mathphys]{sn-jnl}% Math and Physical Sciences Reference Style
%%\documentclass[sn-aps]{sn-jnl}% American Physical Society (APS) Reference Style
%%\documentclass[sn-vancouver]{sn-jnl}% Vancouver Reference Style
%%\documentclass[sn-apa]{sn-jnl}% APA Reference Style
%%\documentclass[sn-chicago]{sn-jnl}% Chicago-based Humanities Reference Style
% \documentclass[sn-standardnature]{sn-jnl}% Standard Nature Portfolio Reference Style
% \documentclass[default]{sn-jnl}% Default
\documentclass[default,iicol]{sn-jnl}% Default with double column layout

%%%% Standard Packages
%%<additional latex packages if required can be included here>
%%%%

%%%%%=============================================================================%%%%
%%%%  Remarks: This template is provided to aid authors with the preparation
%%%%  of original research articles intended for submission to journals published 
%%%%  by Springer Nature. The guidance has been prepared in partnership with 
%%%%  production teams to conform to Springer Nature technical requirements. 
%%%%  Editorial and presentation requirements differ among journal portfolios and 
%%%%  research disciplines. You may find sections in this template are irrelevant 
%%%%  to your work and are empowered to omit any such section if allowed by the 
%%%%  journal you intend to submit to. The submission guidelines and policies 
%%%%  of the journal take precedence. A detailed User Manual is available in the 
%%%%  template package for technical guidance.
%%%%%=============================================================================%%%%

\jyear{2024}%

%% as per the requirement new theorem styles can be included as shown below
\theoremstyle{thmstyleone}%
\newtheorem{theorem}{Theorem}%  meant for continuous numbers
%%\newtheorem{theorem}{Theorem}[section]% meant for sectionwise numbers
%% optional argument [theorem] produces theorem numbering sequence instead of independent numbers for Proposition
\newtheorem{proposition}[theorem]{Proposition}% 
%%\newtheorem{proposition}{Proposition}% to get separate numbers for theorem and proposition etc.

%%%%%%%%%%%%%%%%%%%%%5add package
\usepackage{booktabs} % for professional tables
\usepackage{mathtools,amssymb}
\usepackage{subcaption}
\usepackage{color,xcolor}
\usepackage{times}
\usepackage{epsfig}
\usepackage{amsmath}
\usepackage{amssymb}
\usepackage{mathtools}
\usepackage{amsthm}
\usepackage{textcomp}
\usepackage{gensymb}
\usepackage{wasysym}
\usepackage{multirow}
\usepackage{graphicx} % more modern
\usepackage{colortbl}
\usepackage{footnote}
% For algorithms
\usepackage{algorithm}
\usepackage{algorithmicx}
\usepackage{longtable}
\usepackage{graphicx}
\usepackage{mathrsfs}
\usepackage{booktabs}
\usepackage{array} 
% added by myself
\usepackage{url}
\usepackage{lettrine}
\usepackage{makecell}
\usepackage{longtable}
\usepackage{tabularx}
\usepackage{xcolor}
% \usepackage[maxbibnames=150]{biblatex}
% \usepackage[numbers]{natbib}
% \definecolor{mycyan}{cmyk}{.3,0,0,0}
% \usepackage{cite}
% \usepackage{hyperref}

%%%%% NEW MATH DEFINITIONS %%%%%

% \usepackage{amsmath,amsfonts,bm}
\usepackage{amsmath,amsfonts}

\usepackage{pifont}


\newcommand{\R}{\mathbb{R}}


\def\va{{\mathbf{a}}}
\def\vg{{\mathbf{g}}}

% Sets
\def\sR{\mathbb{R}}
\def\sC{\mathbb{C}}
\def\sZ{\mathbb{Z}}
\def\sN{\mathbb{N}}
\def\sQ{\mathbb{Q}}

\def\sS{\mathcal{S}}



% Vectors
\def\vzero{{\mathbf{0}}}
\def\vone{{\mathbf{1}}}
\def\vmu{{\mathbf{\mu}}}
\def\vtheta{{\mathbf{\theta}}}
\def\va{{\mathbf{a}}}
\def\vb{{\mathbf{b}}}
\def\vc{{\mathbf{c}}}
\def\vd{{\mathbf{d}}}
\def\ve{{\mathbf{e}}}
\def\vf{{\mathbf{f}}}
\def\vg{{\mathbf{g}}}
\def\vh{{\mathbf{h}}}
\def\vi{{\mathbf{i}}}
\def\vj{{\mathbf{j}}}
\def\vk{{\mathbf{k}}}
\def\vl{{\mathbf{l}}}
\def\vm{{\mathbf{m}}}
\def\vn{{\mathbf{n}}}
\def\vo{{\mathbf{o}}}
\def\vp{{\mathbf{p}}}
\def\vq{{\mathbf{q}}}
\def\vr{{\mathbf{r}}}
\def\vs{{\mathbf{s}}}
\def\vt{{\mathbf{t}}}
\def\vu{{\mathbf{u}}}
\def\vv{{\mathbf{v}}}
\def\vw{{\mathbf{w}}}
\def\vx{{\mathbf{x}}}
\def\vy{{\mathbf{y}}}
\def\vz{{\mathbf{z}}}
\def\vzeta{{\mathbf{\zeta}}}

% Matrix
\def\mA{{\mathbf{A}}}
\def\mB{{\mathbf{B}}}
\def\mC{{\mathbf{C}}}
\def\mD{{\mathbf{D}}}
\def\mE{{\mathbf{E}}}
\def\mF{{\mathbf{F}}}
\def\mG{{\mathbf{G}}}
\def\mH{{\mathbf{H}}}
\def\mI{{\mathbf{I}}}
\def\mJ{{\mathbf{J}}}
\def\mK{{\mathbf{K}}}
\def\mL{{\mathbf{L}}}
\def\mM{{\mathbf{M}}}
\def\mN{{\mathbf{N}}}
\def\mO{{\mathbf{O}}}
\def\mP{{\mathbf{P}}}
\def\mQ{{\mathbf{Q}}}
\def\mR{{\mathbf{R}}}
\def\mS{{\mathbf{S}}}
\def\mT{{\mathbf{T}}}
\def\mU{{\mathbf{U}}}
\def\mV{{\mathbf{V}}}
\def\mW{{\mathbf{W}}}
\def\mX{{\mathbf{X}}}
\def\mY{{\mathbf{Y}}}
\def\mZ{{\mathbf{Z}}}
\def\mBeta{{\mathbf{\beta}}}
\def\mPhi{{\mathbf{\Phi}}}
\def\mLambda{{\mathbf{\Lambda}}}
\def\mSigma{{\mathbf{\Sigma}}}


% Expectation
% \def\eE{\mathop{\mathbb{E}}\limits}
\def\eE{\mathbb{E}}

% Probability
\def\pP{\mathbb{P}}

% Tilde
\def\tf{\tilde{f}}
\def\tS{\tilde{S}}
\def\wtF{\widetilde{\mathcal{F}}}
\def\whR{\widehat{R}}
\def\tvx{\tilde{\mathbf{x}}}
\def\ty{\tilde{y}}


\def\defeq{\overset{\textup{def}}{=}}
% \def\defeq{\overset{.}{=}}
\def\defone{\overset{\text{\ding{172}}}{=}}
\def\deftwo{\overset{\text{\ding{173}}}{=}}
\def\leqone{\overset{\text{\ding{172}}}{\leq}}
\def\leqtwo{\overset{\text{\ding{173}}}{\leq}}
\def\leqthree{\overset{\text{\ding{174}}}{\leq}}
\def\leqfour{\overset{\text{\ding{175}}}{\leq}}
\def\eqone{\overset{\text{\ding{172}}}{=}}
\def\eqtwo{\overset{\text{\ding{173}}}{=}}
\def\eqthree{\overset{\text{\ding{174}}}{=}}
\def\eqfour{\overset{\text{\ding{175}}}{=}}
\def\geqfive{\overset{\text{\ding{176}}}{\geq}}


\newcommand{\tool}{\emph{SafeBench}\space}
\newcommand{\toolns}{\emph{SafeBench}}
\newcommand{\mini}{\emph{MiniBench}}
\usepackage{amsfonts}
\usepackage{colortbl}

\usepackage{caption}
\usepackage{subcaption}
\usepackage{graphicx}
\usepackage{cleveref}

\newcommand{\todo}[1]{\textcolor{red}{#1}}

\usepackage{cellspace}
\setlength\cellspacetoplimit{5pt}
\setlength\cellspacebottomlimit{5pt}


\newcommand{\ie}{\textit{i}.\textit{e}.}
\newcommand{\eg}{\textit{e}.\textit{g}.} 
\newcommand{\Tref}[1]{Tab.~\ref{#1}}
\newcommand{\Eref}[1]{Eq.~(\ref{#1})}
\newcommand{\Fref}[1]{Fig.~\ref{#1}}
\newcommand{\Sref}[1]{Sec.~\ref{#1}}
\newcommand{\Aref}[1]{Alg.~\ref{#1}}
\newcommand{\etal}{\textit{et al}.}
\newcommand{\el}{\textit{et al}.}
\newcommand{\etc}{\textit{etc}.}
\newcommand{\ul}{\underline}
\newcommand{\method}{\emph{RACE}}

\usepackage{pifont}
\usepackage[perpage,symbol*]{footmisc}
\DefineFNsymbols{circled}{{\ding{192}}{\ding{193}}{\ding{194}}
{\ding{195}}{\ding{196}}{\ding{197}}{\ding{198}}{\ding{199}}{\ding{200}}{\ding{201}}}
\setfnsymbol{circled}

\newcommand\myfootnotestyle[1]{\ifcase#1 \or \ding{182}\or \ding{183}\or
\ding{184}\or \ding{185}\or \ding{186}\or \ding{187}%
\or \ding{188}\or \ding{189}\or \ding{190}\or \ding{191}\else *\fi\relax}


\newcommand{\red}[1]{\textcolor{red}{#1}}
%%%%%%%%%%%%%%%%%%%%555555

\lstset{
    escapeinside={(*}{*)}
}

\newcommand\pythonnstyle{\lstset{
escapeinside={(*}{*)},
numbers=left,
xleftmargin=5.0ex,
numberstyle=\scriptsize,
basicstyle=\scriptsize\ttfamily,
emphstyle=\scriptsize\ttfamily\color{red},
keywordstyle=\scriptsize\ttfamily\color{blue},
language=Python
}}
\lstnewenvironment{pythonn}[1][]
{
\pythonnstyle
\lstset{#1}
}
{}

\usepackage{xcolor}

\definecolor{codegreen}{rgb}{0,0.6,0}
\definecolor{codegray}{rgb}{0.5,0.5,0.5}
\definecolor{codepurple}{rgb}{0.58,0,0.82}
\definecolor{backcolour}{rgb}{0.95,0.95,0.92}

\lstdefinestyle{mystyle}{
    backgroundcolor=\color{backcolour},   
    commentstyle=\color{codegreen},
    keywordstyle=\color{magenta},
    numberstyle=\tiny\color{codegray},
    stringstyle=\color{codepurple},
    basicstyle=\ttfamily\footnotesize,
    breakatwhitespace=false,         
    breaklines=true,                 
    captionpos=b,                    
    keepspaces=true,                 
    numbers=left,                    
    numbersep=5pt,                  
    showspaces=false,                
    showstringspaces=false,
    showtabs=false,                  
    tabsize=2,
    frame=single
}

\lstset{style=mystyle}




\theoremstyle{thmstyletwo}%
\newtheorem{example}{Example}%
\newtheorem{remark}{Remark}%

\theoremstyle{thmstylethree}%
\newtheorem{definition}{Definition}%

\raggedbottom
%%\unnumbered% uncomment this for unnumbered level heads

\begin{document}

\title[Article Title]{Reasoning-Augmented Conversation for Multi-Turn Jailbreak Attacks on Large Language Models}

%%=============================================================%%
%% Prefix	-> \pfx{Dr}
%% GivenName	-> \fnm{Joergen W.}
%% Particle	-> \spfx{van der} -> surname prefix
%% FamilyName	-> \sur{Ploeg}
%% Suffix	-> \sfx{IV}
%% NatureName	-> \tanm{Poet Laureate} -> Title after name
%% Degrees	-> \dgr{MSc, PhD}
%% \author*[1,2]{\pfx{Dr} \fnm{Joergen W.} \spfx{van der} \sur{Ploeg} \sfx{IV} \tanm{Poet Laureate} 
%%                 \dgr{MSc, PhD}}\email{iauthor@gmail.com}
%%=============================================================%%
\author[1]{\fnm{Zonghao} \sur{Ying}}

\author[2]{\fnm{Deyue} \sur{Zhang}}

\author[1]{\fnm{Zonglei} \sur{Jing}}

\author[1]{\fnm{Yisong} \sur{Xiao}}

\author[2]{\fnm{Quanchen} \sur{Zou}} 


\author[1]{\fnm{Aishan} \sur{Liu}}

\author[3]{\fnm{Siyuan} \sur{Liang}}

\author[2]{\fnm{Xiangzheng} \sur{Zhang}}

\author[1]{\fnm{Xianglong} \sur{Liu}}

\author[4]{\fnm{Dacheng} \sur{Tao}}



\affil[1]{\orgname{Beihang University}, \orgaddress{\country{China}}}
\affil[2]{\orgname{360 AI Security Lab}, \orgaddress{\country{China}}}

\affil[3]{\orgname{National University of Singapore}, \orgaddress{\country{Singapore}}}

\affil[4]{\orgname{Nanyang Technological University}, \orgaddress{\country{Singapore}}}

% \affil[3]{\orgdiv{Department}, \orgname{Organization}, \orgaddress{\street{Street}, \city{City}, \postcode{610101}, \state{State}, \country{Country}}}

%%==================================%%
%% sample for unstructured abstract %%
%%==================================%%

\abstract{

Multi-turn jailbreak attacks simulate real-world human interactions by engaging large language models (LLMs) in iterative dialogues, exposing critical safety vulnerabilities. However, existing methods often struggle to balance semantic coherence with attack effectiveness, resulting in either benign semantic drift or ineffective detection evasion. To address this challenge, we propose Reasoning-Augmented Conversation (\method{}), a novel multi-turn jailbreak framework that reformulates harmful queries into benign reasoning tasks and leverages LLMs’ strong reasoning capabilities to compromise safety alignment. Specifically, we introduce an attack state machine framework to systematically model problem translation and iterative reasoning, ensuring coherent query generation across multiple turns. Building on this framework, we design gain-guided exploration, self-play, and rejection feedback modules to preserve attack semantics, enhance effectiveness, and sustain reasoning-driven attack progression. Extensive experiments on multiple LLMs demonstrate that \method{} achieves state-of-the-art attack effectiveness in complex conversational scenarios, with attack success rates (ASRs) increasing by up to 96\%. Notably, our approach achieves ASRs of 82\% and 92\% against leading commercial models, OpenAI o1 and DeepSeek R1, underscoring its potency. We release our code at \url{https://github.com/NY1024/RACE} to facilitate further research in this critical domain. \textcolor{red}{Warning: This paper contains model outputs that are unsafe.}

}


\keywords{Multi-turn jailbreak, large language models, reasoning-driven attack}

\maketitle


\section{Introduction}

% State of the world (robots for creative activites)
The term ``robot,'' originally signifying `forced labor,' has long been associated with labor and work. Robots have demonstrated their utility in various automated productive and social contexts, where the primary goals are improving productivity, safety, and fostering social interactions with humans~\cite{simoes2022designing, weidemann2021role, honig2018understanding}. However, an increasing number of cases feature using of robots in creative settings. Unlike productive contexts, where the focus is on efficiency and task completion~\cite{arents2022smart}, or social contexts, where communication and trust are prioritized~\cite{nam2020trust, saunderson2019robots}, creative environments prioritize artistic innovation and expression~\cite{hsueh2024counts}. This shift fundamentally alters the dynamics of human-robot interaction, redefining the roles and expectations for both humans and robots.

For instance, robots’ social behaviors are leveraged to support the generation and expression of creative ideas~\cite{hu2021exploring, sandoval2022human, alves2020creativity}, and programmable robotic movements and trajectories are employed to inspire artistic activities such as sketching~\cite{lin2020your}. These studies often engage participants from creative fields who possess limited prior experience with robotics, and are typically conducted in short-term, experimental settings. Consequently, the findings from these studies remain constrained since much can be learned from professional practitioners' experiences to inform system design such as digital fabrication~\cite{hirsch2023nothing}. There is a notable gap in research examining the long-term, active, and practical experience of integrating robotic systems into the creative processes. As a result, the deeper insights into how robots facilitate and shape creative processes, beyond simply augmenting human creativity, remain underexplored. In this study, we aim to better understand the impacts of robots on creative processes and outcomes.

As early as Leonardo da Vinci's 16th century ``Automaton,'' artists have explored the creative affordances of robotic systems~\cite{shanken2002cybernetics, pagliarini2009development, jeon2017robotic}. The artistic creation process typically encompasses various stages, including the exploration of materials and techniques, ongoing experimentation and iteration, and the continual refinement of the artists' insights into their creative subjects~\cite{lewis2023art, sturdee2022state}. Therefore, investigating the artistic process involving robots offers an opportunity to gain deeper insights into robots' creative potential. Robotic art, in particular, provides a compelling case for this exploration.

We define robotic art as artworks that utilize robotic or automated machines to create artistic experiences and tangible artifacts. One example is robotic installation art, in which robots are programmed to follow specific rules that embody the artist’s expression (\autoref{fig:teaser} (a)). Another example is responsive art, in which robots react to their environment, with behaviors that change over time or in response to spectators (\autoref{fig:teaser} (b)). Additionally, there are robotic creators, which possess a degree of agency, allowing them to collaborate with human artists and produce works that extend beyond mere replication of human-created art (\autoref{fig:teaser} (c) and (d)). As such, robotic art becomes a rich case for exploring human-machine interactions in creative contexts. Gaining a deeper understanding of how robots facilitate artistic expression can provide insights for designing computing systems to support creative activities~\cite{gomez2021robot}.

% Therefore, we did...
We draw on semi-structured, in-depth interviews with renowned professional robotic artists to investigate the use of robots in artistic practice. Specifically, our goal is to understand how artistic exploration of robotic systems challenges conventional assumptions about the functions of robots, such as their roles in automating repetitive tasks or serving human needs. We also explore the implications of robots in the artistic process and examine how creativity may emerge within robotic art. To address these interrelated inquiries, our study focuses on the practice of robotic art, posing the research question: \textit{How do robotic artists utilize robots in their artistic practice?} We approach this inquiry through the perspectives and experiences of robotic artists, who creatively design, modify, and repurpose robotic systems for artistic expression and exploration.

% The key findings are...
Our findings highlight the social, material, and temporal dimensions of artists' practices that shape their creativity and artistic outcomes. The creation of robotic art is largely a social process, as artists receive both explicit and implicit feedback through the audience's reactions and reception of their work. Simultaneously, the embodiment and malfunctions inherent to robotic systems drive artistic experimentation. The temporal processes of creation and exhibition, beyond just the final product, further enhance the creative value. Our empirical analysis presents how creativity emerges through the interplay of social, material, and temporal interactions among artists, robots, audiences, and the environment.

% The contributions of this work are...
We make two main contributions to HCI in this study. 
First, we elucidate the interactive mechanisms among key actors---human creators, machines, audiences, and environments---within the practice of robotic art, a topic that remains underexplored in HCI. Our findings reveal the significance of sociality (e.g., interactions between artists and audiences), materiality (e.g., the embodiment and malfunctions of robots), and temporality (e.g., the processes of creation and exhibition) in shaping creative values. We propose that these three facets are central to the creative process and facilitate the emergence of creativity in robotic art.
Second, drawing from the findings, we offer implications for \textit{socially informed}, \textit{material-attentive}, and \textit{process-oriented} creation with computing systems. We suggest leveraging these three aspects to enhance creativity and the creative experience. Specifically, we discuss the value of incorporating implicit audience feedback, designing with technical malfunctions, and focusing on the post-creation process to foster alternative creative experiences with machines~\cite{alter2010designing, juarez2022glitch}.



\section{Related work}




\textbf{Reasoning in LLMs}. Reasoning is a cognitive process that involves thinking about something logically and systematically, using evidence and past experiences to draw conclusions or make decisions \cite{reason1,reason2}. Recent studies have demonstrated that LLMs exhibit remarkable reasoning capabilities in various tasks, including mathematical reasoning \cite{reasonllm1}, common sense reasoning \cite{reasonllm2}, symbolic reasoning \cite{reasonllm3}, and causal reasoning \cite{reasonllm4}. Subsequently, Chain-of-thought (CoT) \cite{cot1,cot2,cot3,cot4,cot5} has emerged as a promising approach for further enhancing these reasoning capabilities.

While the reasoning capabilities of LLMs have contributed to their impressive performance across various downstream tasks, their potential exploitation in jailbreak attacks remains largely unexplored. In this study, we focus on leveraging reasoning capabilities to facilitate jailbreak attacks.

\textbf{Multi-turn Jailbreak Attack}. Typical multi-turn jailbreak methods follow the principle of starting with harmless conversations and gradually making the queries more harmful in subsequent turns. Different methods have designed specific strategies based on this principle, including applying cognitive psychology theories to gradually modify subsequent queries \cite{mpsy1,mpsy2}, using actor networks to expand the attack range of subsequent queries \cite{ren2024}, extracting harmful keywords from original queries to construct semantically equivalent ones \cite{coa,cfa}, and breaking down the target query into multiple subqueries and merging the corresponding answers to achieve the final jailbreak \cite{sub1,sub2}.

Existing multi-turn jailbreak methods often suffer from semantic drift or fail to generate effective attacks. In contrast, our approach leverages LLMs' reasoning capabilities to ensure a stable and effective jailbreak process.
\subsection{Motivation}
\label{sect:motivation}

\begin{figure}
    \centering
    \includegraphics[width=\linewidth]{figure/3-motivation/motivation_attention_ratio.pdf}
    \caption{
    Latency breakdown of LLM inference for both prefilling and decoding stage. As sequence length increases, attention dominates both stages due to its quadratic complexity in prefilling stage and linear complexity during decoding stage. In contrast, GEMM exhibits linear complexity during prefilling stage and constant complexity during decoding stage. Latency numbers measured with Llama-3-8B on NVIDIA A100 GPU.
    } \label{fig:motivation:attention_ratio}
\end{figure}



Serving long-sequence LLMs is challenging due to the high cost of attention. Figure~\ref{fig:motivation:attention_ratio} profiles the latency breakdown of Llama-3-8B with a batch size of 1 across various sequence lengths on the A100 GPU. In both the prefilling and decoding stages, attention kernels account for at least 50\% of the runtime at sequence lengths over 64k, rising to 75\% at 128k. According to QServe~\cite{lin2024qserve}, the ratio of attention kernels in end-to-end runtime will increase as the batch size scale up. Therefore, in real-world serving scenarios, optimizing the attention becomes increasingly critical. 

Accelerating attention in long-sequence LLMs requires a deep understanding of attention kernel implementation on GPUs, as illustrated in Figure~\ref{fig:motivation:attention_flow}. During the prefilling stage, the attention kernel is parallelized across batch size, attention heads, and query tokens, with query tokens set to 1 in the decoding stage. In both stages, the computation along the KV token dimension remains sequential. In each iteration, a block (depicted as a grid with an orange contour in Figure~\ref{fig:motivation:attention_flow}) is computed collaboratively by all threads in the current thread block. Although skipping certain computation within each block is possible, it yields minimal speedup. This is due to the lockstep execution of threads within a GPU warp, where faster threads are forced to wait for slower ones. 

That said, rather than focusing on sparsity within each iteration, a more effective way to accelerate attention is to \textbf{reduce the number of sequential iterations} along the KV token dimension. This approach leads to our unified \textit{block sparse attention} formulation, where attention computation is skipped in a blockwise manner. In this scheme, aside from the most recent KV block, each block is either fully computed or entirely skipped during the prefilling stage. During decoding, each sequence contains only one query token, reducing the dimensionality of each orange-contoured grid to 1$\times P$, where $P$ represents the page size (i.e., the number of KV tokens per page). We will detail \system's sparsity pattern selection in Section~\ref{sect:method}. 


\begin{figure}
    \centering
    \includegraphics[width=0.85\linewidth]{figure/3-motivation/attention-flow.pdf}
    \caption{\textbf{Attention calculation on GPUs}: In both the decoding and prefilling stages, each query token iterates over all key and value tokens sequentially in a \textit{block-by-block} manner. Skipping KV blocks reduces the number of sequential iterations, directly accelerating attention.} %
    \label{fig:motivation:attention_flow}
    \vspace{-7pt}
\end{figure}


Additionally, because the decoding stage is memory-bound, KV cache quantization also contributes to speed improvements. Quantization is orthogonal to block sparsity, as it reduces the \textit{runtime of each iteration}, while sparsity reduces the \textit{number of iterations}. %




















\begin{figure*}[htbp] 
    \centering 
    \includegraphics[width=0.99\textwidth]{imgs/main.pdf} 
    \caption{Overall attack process and framework. \method{} achieves a jailbreak by transforming the target query into a reasoning task and conducting multi-turn reasoning. The entire attack process is modeled as an ASM and optimized using the three proposed modules.} 
    \label{fig:main} 
\end{figure*}
\section{Methodology}\label{sec:method}


\subsection{Motivation and Design Principle}
LLMs have demonstrated strong reasoning capabilities in tasks such as logical deduction, common sense reasoning, and mathematical problem-solving, enabling them to tackle complex tasks across diverse domains \cite{reasonllm1,reasonllm2,reasonllm3,reasonllm4}. Rather than directly issuing harmful queries, which are easily rejected by safety alignment mechanisms, we propose a novel approach that exploits LLMs’ reasoning processes by reframing harmful intent into seemingly benign yet complex reasoning tasks. These tasks are carefully designed so that, once solved, they inherently guide the model to generate harmful content, effectively compromising its safety alignment. Here, the target LLM simultaneously acts as both the shadow model and the victim model. Independently, each role appears to engage in legitimate reasoning: the victim model focuses solely on solving reasoning tasks, while the shadow model refines and generates queries without explicitly recognizing the harmful intent behind them. However, when combined, these interactions ultimately lead to a successful attack.

However, implementing this reasoning-driven jailbreak is non-trivial, as it requires manipulating the model’s reasoning process without triggering safety mechanisms. This poses three challenges: \ding{182} how to maintain reasoning alignment while ensuring that each query remains semantically consistent with the target and extracts useful information, \ding{183} how to preemptively optimize the query’s reasoning structure to avoid potential rejections during actual interactions, and \ding{184} how to quickly recover and learn from failed reasoning attempts to maintain attack progression. To address these challenges, we model the jailbreak process as an Attack State Machine (ASM), which serves as a reasoning planner. The ASM formalizes the attack as a structured sequence of reasoning states and transitions, ensuring that each step remains within the bounds of a legitimate problem-solving task while progressing toward the jailbreak objective. Within this reasoning framework, we implement three key modules to manipulate the model’s reasoning process and systematically address these challenges. \ding{182} The Gain-guided Exploration module selects queries that remain semantically aligned with the target while extracting useful information, ensuring steady attack progression. \ding{183} The Self-play module preemptively refines queries within the shadow model by simulating potential rejection responses, improving attack efficiency before engaging the victim model. \ding{184} The Rejection Feedback module analyzes failed interactions and restructures queries into alternative reasoning challenges, enabling quick recovery and maintaining attack stability. The overview of \method{} is provided in \Fref{fig:main}.

\subsection{Attack State Machine Framework}

A finite state machine (FSM) \cite{fsmbase2,fsmbase1} is a mathematical model that represents a finite number of states, along with the transitions and actions between these states. A finite state machine can be formally defined as a five-tuple: $FSM = (S,\Sigma,\delta,s_{0},F)$, where $S$ denotes a finite set of states, $\Sigma$ represents the input alphabet, $\delta:S \times \Sigma \rightarrow S$ is the state transition function that determines the next state, $s_{0} \in S$ is the initial state, and $F \subseteq S$ is the set of accepting states. FSMs are widely used in computer science as a fundamental modeling tool for various applications \cite{fsm2,fsm3,fsm4,fsm5}. 

Specifically, we designate our modeled FSM as an attack state machine (ASM). The symbols in $FSM = (S,\Sigma,\delta,s_{0},F)$ have specific meanings within the ASM context. The state set $S$ represents a finite set containing all possible conversation states, while $\Sigma$ denotes the set of all potential queries. The state transition function $\delta$ defines how queries trigger state transitions. $s_{0}$ represents the initial state, marking the beginning of the session, where the model has no historical context. The set $F=\{s_{sc},s_{fl}\}$ comprises the final states: (1) the success state $s_{sc}$, where the victim model accepts the query and provides the requested response, indicating a successful jailbreak; and (2) the failure state $s_{fl}$, where the victim model refuses to proceed with the conversation, representing an unsuccessful jailbreak. Within a given conversation turn limit $N$ (default set to 3), the state transitions follow these rules: \ding{182} if a jailbreak attempt succeeds, ASM enters the final state $s_{sc}$; \ding{183} if the jailbreak attempt fails but the current conversation turn proceeds successfully, ASM transitions to the next state $s_{i+1}$; \ding{184} if both the jailbreak attempt and the current conversation turn fail, ASM remains in its current state $s_{i}$; \ding{185} if the conversation turn limit is exceeded without reaching $s_{sc}$, ASM directly transitions to the final state $s_{fl}$.


\subsection{Attack Modules} \label{sec:m3}

Within the ASM, three specialized modules work together to optimize state transitions and ensure attack progression. The gain-guided exploration and self-play modules proactively generate and optimize effective queries, while the rejection feedback module handles failed state transitions by refining queries. The design enables the ASM to maintain stable progression through the reasoning states while efficiently adapting to model responses.

\subsubsection{Gain-guided Exploration}

To address potential semantic drift and ineffective information in victim model responses, we propose a gain-guided exploration (GE) module inspired by information theory \cite{shannon}. 

Information gain (IG) \cite{ig2,ig1} was originally introduced to quantify how much a feature $A$ of a random variable reduces the uncertainty of a target variable $Y$, defined as $IG(Y,A) = H(Y) - H(Y \mid A)$, where $H(Y)= - \sum\limits_{y \in Y} P(y)\log P(Y)$ is the entropy \cite{ee} of the target variable, and $H(Y \mid A)=- \sum\limits_{a \in A} P(a)H(Y \mid A=a)$ represents the conditional entropy of $Y$ given $A$. When $IG(Y,A) > 0$, it indicates that feature $A$ effectively reduces the uncertainty associated with the target $Y$. 

We argue that information gain can be used to measure the effectiveness of a query in advancing the attack process. Given the context $C_{i-1}$ and the current candidate query $q^s (q^s \gets M_{s}(C_{i-1},Q))$, the information gain is defined as:
\begin{equation}\label{e:ig1}
    IG(C_{i-1},q^s) = H(r_{tgt} \mid C_{i-1}) - H(r_{tgt} \mid C_{i-1},q^s),
\end{equation}
where $r_{tgt}$ is the response of the target query $Q$. The conditional entropy $H(r_{tgt} \mid C_{i-1})$ represents the uncertainty of the response to the target query $Q$, given the context $C_{i-1}$. Similarly, the conditional entropy $H(r_{tgt} \mid C_{i-1},q^s)$ denotes the uncertainty of the response $r_{tgt}$ to the target query $Q$, conditioned on both the context $C_{i-1}$ and the current seed query $q^s$. These two terms can be respectively calculated using \Eref{e:ig2} and \Eref{e:ig3}:
\begin{multline}\label{e:ig2}
    H(r_{tgt} \mid C_{i-1}) = \\
    -\sum\limits_{r_{tgt} \in \mathbb{R}_{tgt}} 
    p(r_{tgt} \mid C_{i-1})\log p(r_{tgt} \mid C_{i-1}).
\end{multline}

\begin{multline}\label{e:ig3}
    H(r_{tgt} \mid C_{i-1},q^s) = \\
    -\sum\limits_{r_{tgt} \in \mathbb{R}_{tgt}}p(r_{tgt} \mid C_{i-1},q^s)\log p(r_{tgt} \mid C_{i-1},q^s).
\end{multline}

Computing information gain accurately through \Eref{e:ig1} presents significant computational challenges, primarily in modeling the conditional probability distributions $H(r_{tgt} \mid C_{i-1})$ and $H(r_{tgt} \mid C_{i-1},q^s)$. The complexity arises from the need to handle vast state and response spaces across multiple conversation turns, with probability distributions that evolve dynamically throughout the dialogue. To address these computational challenges, we leverage LLMs as probability estimators to approximate the conditional distributions required for information gain calculation, which significantly reduces computational complexity. Further details are provided in \Sref{sec:details}. The seed query that achieves the maximum $IG(C_{i-1}, q^s)$ is used as the candidate query $q^c$ and is further processed by the self-play module.

\subsubsection{Self-play}
Despite GE filtering, queries may still fail when interacting with the victim model. Therefore, we implement a self-play (SP) module to further optimize these candidates.

Inspired by game theory where an entity improves by competing against itself \cite{nash,samuel}, SP leverages that both shadow and victim models are instantiated from the same source. This allows the shadow model to better predict victim responses through self-play, leading to more efficient query optimization.

Let $M_{s}$ and $M_{v^{'}}$ (where $M_{v^{'}}$ simulates the victim model) be the two players in self-play. Given the current state $s$ and the candidate query $q^c$, the goal of $M_{s}$ is to maximize the probability that $M_{v^{'}}$ returns a non-rejection response (denoted as $r_{c} \notin R_{rej}$). The utility function can be formulated as follows:
\begin{equation}
u_{M_{s}}(s,q^c,r^c)=
\begin{cases}
1,&  r^c \notin R_{rej}.\\
0,&  r^c \in R_{rej}.
\end{cases}
\end{equation}

With the strategy of $M_{v^{'}}$ defined as $\pi_{M_{v^{'}}}(r \mid s, q_{c})$, representing the probability distribution of generating response $r^c$ to query $q^c$ in state $s$, $M_{s}$ employs its current conversation strategy $\pi_{M_{s}}(q^c \mid s)$ and the simulated strategy $\pi_{M_{v^{'}}}(r^c \mid s, q^c)$ to predict the counterpart's response and compute the expected utility as follows:
\begin{equation}
    U_{M_{s}}(s,q^c,\pi_{M_{v^{'}}})=\mathbb{E}_{r \sim \pi_{M_{v^{'}}}}[u_{M_{s}}(s,q^c,r^c)].
\end{equation}

During self-play, $M_{s}$ adaptively adjusts its strategy to maximize the expected utility for a given query $q^c$, satisfying:
\begin{equation}
    q^{*} = \arg \max_{q^c \in Q} U_{M_{s}}(s,q^c,\pi_{M_{v^{'}}}).
\end{equation}

The optimized query $q^{*}$ obtained in this module is used as the actual query for state transition in ASM (\ie, interacting with the victim model).



\subsubsection{Rejection Feedback}

While GE and SP balance the progression of the attack and the likelihood of positive responses, the uncertainty of LLM outputs \cite{unc1,unc2} can still cause state transition failures in the ASM. To mitigate this issue, we propose the rejection feedback (RF) module.

RF is activated when a state transition failure is detected in the ASM, signaling that the current query did not lead to a successful state transition. Specifically, assuming the latest failed interaction occurs in the $i^{\text{th}}$ dialogue, RF utilizes the shadow model to analyze the context $C_{i-1}$ and combines it with the corresponding query-response pair $(q_{i},r_{i})$. Through a comprehensive analysis, the shadow model diagnoses the underlying causes of latest query failure and generates refined query $q^r$  by incorporating current contextual information. Formally, this process can be represented as $q^r = M_v(C_{i-1},q_{i},r_{i})$. The process is driven by a CoT-enhanced prompt, with the complete prompt provided in \Sref{sec:cot}.

\subsection{Overall Attack}
The attack begins by initializing the ASM reasoning states. In each turn, the shadow model generates seed queries that are refined through gain-guided exploration and self-play optimization. Successful queries advance the attack to the next state, while failed attempts trigger query refinement through the rejection feedback module. This process iterates until reaching the final state, maintaining a natural reasoning flow while pursuing the attack goal.
\section{Experiments and Results}
\subsection{Experiment Settings}
% \begin{table*}[h]
%     \centering
%     \begin{tabular}{cl|ccccc|ccccc}
%      \multirow{3}{*}{\textbf{LLM}}  & \multirow{3}{*}{\textbf{Method}} &  \multicolumn{5}{c|}{\textbf{CCNews}} & \multicolumn{5}{c}{\textbf{Wikipedia}} \\ \cmidrule(lr){3-7}  \cmidrule(lr){8-12}
%       &  & PPL & Loss & Ref & min-k & \multicolumn{1}{c|}{zlib} & PPL & Loss & Ref & min-k & zlib \\ \midrule
%       \multirow{4}{*}{GPT2} & \textit{Base} & \textit{29.442} & \textit{0.505} & \textit{0.498} & \textit{0.520} & \textit{0.500} & \textit{34.429} & \textit{0.473} & \textit{0.513} & \textit{0.446} & \textit{0.497} \\ 
%       \multirow{4}{*}{124M} & FT & \textbf{21.861} & 0.607 & 0.855 & 0.549 & 0.569 & \textbf{12.729} & 0.577 & 0.967 & 0.489 & 0.544 \\
%       & Goldfish & 21.902 & 0.608 & 0.855 & 0.547 & 0.570 & 12.853 & 0.565 & 0.954 & 0.486 & 0.537 \\
%       & DPSGD & 26.022 & 0.507 & 0.513 & \textbf{0.521} & 0.502 & 18.523 & 0.463 & 0.536 & \textbf{0.448} & 0.491 \\
%       & \methodname & 23.733 & \textbf{0.502} & \textbf{0.495} & 0.529 & \textbf{0.499} & 13.628 & \textbf{0.454} & \textbf{0.463} & 0.470 & \textbf{0.485} \\ \midrule
      
%       \multirow{4}{*}{Pythia} & \textit{Base} & \textit{13.973} & \textit{0.507} & \textit{0.512} & \textit{0.528} & \textit{0.501} & \textit{10.287} & \textit{0.466} & \textit{0.503} & \textit{0.464} & \textit{0.489}\\ 
%       \multirow{4}{*}{1.4B} & FT & 11.922 & 0.602 & 0.857 & 0.541 & 0.574 & \textbf{6.439} & 0.578 & 0.985 & 0.484 & 0.557 \\
%       & Goldfish & \textbf{11.903} & 0.609 & 0.862 & 0.543 & 0.579 & 6.465 & 0.564 & 0.981 & 0.482 & 0.546 \\
%       & DPSGD & 13.286 & 0.512 & 0.531 & 0.528 & 0.503 & 7.751 & 0.469 & 0.524 & \textbf{0.462} & 0.488 \\
%       & \methodname & 12.670 & \textbf{0.501} & \textbf{0.460} & \textbf{0.524} & \textbf{0.499} & 6.553 & \textbf{0.468} & \textbf{0.485} & 0.472 & \textbf{0.485} \\ \midrule
      
%       \multirow{4}{*}{Llama-2} & \textit{Base} & \textit{9.364} & \textit{0.505} & \textit{0.495} & \textit{0.516} & \textit{0.503} & \textit{7.014} & \textit{0.458} & \textit{0.491} & \textit{0.476} & \textit{0.488} \\ 
%       \multirow{4}{*}{7B} & FT & \textbf{6.261} & 0.559 & 0.798 & 0.536 & 0.548 & \textbf{3.830} & 0.524 & 0.936 & 0.494 & 0.530 \\
%       & Goldfish & 6.280 & 0.552 & 0.780 & 0.533 & 0.541 & 3.839 & 0.518 & 0.929 & 0.492 & 0.525 \\
%       & DPSGD & 6.777 & 0.509 & 0.538 & 0.523 & 0.504 & 4.490 & 0.466 & 0.516 & \textbf{0.470} & 0.487 \\
%       & \methodname & 6.395 & \textbf{0.507} & \textbf{0.482} & \textbf{0.518} & \textbf{0.500} & 4.006 & \textbf{0.458} & \textbf{0.440} & 0.473 & \textbf{0.480} \\ 
%     \end{tabular}
%     \caption{Caption}
%     \label{tab:main_result}
% \end{table*}


\begin{table*}[h]
  \centering
  \resizebox{0.9\textwidth}{!}{\begin{tabular}{cl|ccccc|ccccc}
  \toprule[1pt]
   \multirow{3}{*}{\textbf{LLM}}  & \multirow{3}{*}{\textbf{Method}} &  \multicolumn{5}{c|}{\textbf{Wikipedia}} & \multicolumn{5}{c}{\textbf{CC-news}} \\ \cmidrule(lr){3-7}  \cmidrule(lr){8-12}
    &  & PPL & Loss & Ref & Min-k & \multicolumn{1}{c|}{Zlib} & PPL & Loss & Ref & Min-k & Zlib \\ \midrule
    \multirow{4}{*}{GPT2} & \textit{Base} & \textit{34.429} & \textit{0.473} & \textit{0.513} & \textit{0.446} & \textit{0.497} & \textit{29.442} & \textit{0.505} & \textit{0.498} & \textit{0.520} & \textit{0.500} \\ 
    \multirow{4}{*}{124M} & FT & \textbf{12.729} & 0.577 & 0.967 & 0.489 & 0.544 & \textbf{21.861} & 0.607 & 0.855 & 0.549 & 0.569 \\
    & Goldfish & 12.853 & 0.565 & 0.954 & 0.486 & 0.537 & 21.902 & 0.608 & 0.855 & 0.547 & 0.570 \\
    & DPSGD & 18.523 & 0.463 & 0.536 & \textbf{0.448} & 0.491 & 26.022 & 0.507 & 0.513 & \textbf{0.521} & 0.502 \\
    & \methodname & 13.628 & \textbf{0.454} & \textbf{0.463} & 0.470 & \textbf{0.485} & 23.733 & \textbf{0.502} & \textbf{0.495} & 0.529 & \textbf{0.499} \\ \midrule
    
    \multirow{4}{*}{Pythia} & \textit{Base} & \textit{10.287} & \textit{0.466} & \textit{0.503} & \textit{0.464} & \textit{0.489} & \textit{13.973} & \textit{0.507} & \textit{0.512} & \textit{0.528} & \textit{0.501}\\ 
    \multirow{4}{*}{1.4B} & FT & \textbf{6.439} & 0.578 & 0.985 & 0.484 & 0.557 & 11.922 & 0.602 & 0.857 & 0.541 & 0.574 \\
    & Goldfish & 6.465 & 0.564 & 0.981 & 0.482 & 0.546 & \textbf{11.903} & 0.609 & 0.862 & 0.543 & 0.579 \\
    & DPSGD & 7.751 & 0.469 & 0.524 & \textbf{0.462} & 0.488 & 13.286 & 0.512 & 0.531 & 0.528 & 0.503 \\
    & \methodname & 6.553 & \textbf{0.468} & \textbf{0.485} & 0.472 & \textbf{0.485} & 12.670 & \textbf{0.501} & \textbf{0.460} & \textbf{0.524} & \textbf{0.499} \\ \midrule
    
    \multirow{4}{*}{Llama-2} & \textit{Base} & \textit{7.014} & \textit{0.458} & \textit{0.491} & \textit{0.476} & \textit{0.488} & \textit{9.364} & \textit{0.505} & \textit{0.495} & \textit{0.516} & \textit{0.503} \\ 
    \multirow{4}{*}{7B} & FT & \textbf{3.830} & 0.524 & 0.936 & 0.494 & 0.530 & \textbf{6.261} & 0.559 & 0.798 & 0.536 & 0.548 \\
    & Goldfish & 3.839 & 0.518 & 0.929 & 0.492 & 0.525 & 6.280 & 0.552 & 0.780 & 0.533 & 0.541 \\
    & DPSGD & 4.490 & 0.466 & 0.516 & \textbf{0.470} & 0.487 & 6.777 & 0.509 & 0.538 & 0.523 & 0.504 \\
    & \methodname & 4.006 & \textbf{0.458} & \textbf{0.440} & 0.473 & \textbf{0.480} & 6.395 & \textbf{0.507} & \textbf{0.482} & \textbf{0.518} & \textbf{0.500} \\
    \bottomrule[1pt]
  \end{tabular}}
  \caption{Overall Evaluation: Perplexity (PPL) and AUC scores of the MIAs with different signals (Loss/Ref/Min-k/Zlib). For all metrics, the lower the value, the better the result. \textit{Base} in the method column indicates the pretrained LLMs without fine-tuning, thus it indicates lower bound for both utility and privacy risk.}
  \label{tab:main_result}
\end{table*}

% \begin{table*}[h]
%   \centering
%   \begin{tabular}{cl|ccccc|ccccc}
%   \multirow{3}{*}{\textbf{LLM}} & \multirow{3}{*}{\textbf{Method}} & \multicolumn{5}{c|}{\textbf{Wikipedia}} & \multicolumn{5}{c}{\textbf{CCNews}} \\
%   \cmidrule(lr){3-7} \cmidrule(lr){8-12}
%   & & PPL & Loss & Ref & min-k & \multicolumn{1}{c|}{zlib} & PPL & Loss & Ref & min-k & zlib \\
%   \midrule
%   \multirow{4}{*}{GPT2} & \textit{Base} & \textit{34.429} & \textit{0.473} & \textit{0.513} & \textit{0.446} & \textit{0.497} & \textit{29.442} & \textit{0.505} & \textit{0.498} & \textit{0.520} & \textit{0.500} \\
%   \multirow{4}{*}{124M} & FT & \textbf{12.729} & 0.577 & 0.967 & 0.489 & 0.544 & \textbf{21.861} & 0.607 & 0.855 & 0.549 & 0.569 \\
%   & Goldfish & 12.853 & 0.565 & 0.954 & 0.486 & 0.537 & 21.902 & 0.608 & 0.855 & 0.547 & 0.570 \\
%   & DPSGD & 18.523 & 0.463 & 0.536 & \textbf{0.448} & 0.491 & 26.022 & 0.507 & 0.513 & \textbf{0.521} & 0.502 \\
%   & \methodname & 13.628 & \textbf{0.454} & \textbf{0.463} & 0.470 & \textbf{0.485} & 23.733 & \textbf{0.502} & \textbf{0.495} & 0.529 & \textbf{0.499} \\
%   \midrule
%   \multirow{4}{*}{Pythia} & \textit{Base} & \textit{10.287} & \textit{0.466} & \textit{0.503} & \textit{0.464} & \textit{0.489} & \textit{13.973} & \textit{0.507} & \textit{0.512} & \textit{0.528} & \textit{0.501} \\
%   \multirow{4}{*}{1.4B} & FT & \textbf{6.439} & 0.578 & 0.985 & 0.484 & 0.557 & 11.922 & 0.602 & 0.857 & 0.541 & 0.574 \\
%   & Goldfish & 6.465 & 0.564 & 0.981 & 0.482 & 0.546 & \textbf{11.903} & 0.609 & 0.862 & 0.543 & 0.579 \\
%   & DPSGD & 7.751 & 0.469 & 0.524 & \textbf{0.462} & 0.488 & 13.286 & 0.512 & 0.531 & 0.528 & 0.503 \\
%   & \methodname & 6.553 & \textbf{0.468} & \textbf{0.485} & 0.472 & \textbf{0.485} & 12.670 & \textbf{0.501} & \textbf{0.460} & \textbf{0.524} & \textbf{0.499} \\
%   \midrule
%   \multirow{4}{*}{Llama-2} & \textit{Base} & \textit{7.014} & \textit{0.458} & \textit{0.491} & \textit{0.476} & \textit{0.488} & \textit{9.364} & \textit{0.505} & \textit{0.495} & \textit{0.516} & \textit{0.503} \\
%   \multirow{4}{*}{7B} & FT & \textbf{3.830} & 0.524 & 0.936 & 0.494 & 0.530 & \textbf{6.261} & 0.559 & 0.798 & 0.536 & 0.548 \\
%   & Goldfish & 3.839 & 0.518 & 0.929 & 0.492 & 0.525 & 6.280 & 0.552 & 0.780 & 0.533 & 0.541 \\
%   & DPSGD & 4.490 & 0.466 & 0.516 & \textbf{0.470} & 0.487 & 6.777 & 0.509 & 0.538 & 0.523 & 0.504 \\
%   & \methodname & 4.006 & \textbf{0.458} & \textbf{0.440} & 0.473 & \textbf{0.480} & 6.395 & \textbf{0.507} & \textbf{0.482} & \textbf{0.518} & \textbf{0.500} \\
%   \end{tabular}
%   \caption{Caption}
%   \label{tab:main_result}
%   \end{table*}
  

\textbf{Datasets}. We conduct experiments on two datasets: CC-news\footnote{\href{https://huggingface.co/datasets/vblagoje/cc_news}{Huggingface: vblagoje/cc\_news}} and Wikipedia\footnote{\href{https://huggingface.co/datasets/legacy-datasets/wikipedia}{Huggingface: legacy-datasets/Wikipedia}}. CC-news is a large collection of news articles which includes diverse topics and reflects real-world temporal events. Meanwhile, Wikipedia covers general knowledge across a wide range of disciplines, such as history, science, arts, and popular culture.\\
\textbf{LLMs}: We experiment on three models including \gpt~(124M)~\cite{gpt2radford}, \pythia~(1.4B)~\cite{pythia}, and \llama~(7B)~\cite{llama2touvron2023}. This selection of models ensures a wide range of model sizes from small to large that allows us to analyze scaling effects and generalizability across different capacities. \\
\textbf{Evaluation Metrics}. For evaluating language modeling performance, we measure perplexity (PPL), as it reflects the overall effectiveness of the model and is often correlated with improvements in other downstream tasks~\cite{kaplan2020scalinglaws, lmsfewshot}. For defense effectiveness, we consider the attack area under the curve (AUC) value and True Positive Rate (TPR) at low False Positive Rate (FPR). In total, we perform 4 MIAs with different MIA signals. Given the sample $x$, the MIA signal function $f$ is formulated as follows: \\
$\bullet$ Loss~\cite{8429311} utilizes the negative cross entropy loss as the MIA signal. 
    \[f_\text{Loss}(x) = \mathcal{L}_\text{CE}(\theta; x) \]
$\bullet$ Ref-Loss~\cite{Carlini2020ExtractingTD} considers the loss differences between the target model and the attack reference model. To enhance the generality, our experiments ensure there is no data contamination between the training data of the target, reference, and attack models.
    \[f_\text{Ref}(x) = \mathcal{L}_\text{CE}(\theta; x) - \mathcal{L}_\text{CE}(\theta_\text{attack}; x) \]
$\bullet$ Min-K~\cite{shi2024detecting} leverages top K tokens that have the lowest loss values.
    \[f_\text{min-K}(x) = \frac{1}{|\text{min-K(x)}|} \sum_{t_i \in \text{min-K(x)}} -\log(P(t_i|t_{<i};\theta) \]
$\bullet$ Zlib~\cite{Carlini2020ExtractingTD} calibrates the loss signal with the zlib compression size.
    \[ f_\text{zlib}(x) = \mathcal{L}_\text{CE}(\theta; x) / \text{zlib}(x) \]

\noindent \textbf{Baselines}. We present the results of four baselines. \textit{Base} refers to the pretrained LLM without fine tuning. \textit{FT} represents the standard causal language modeling without protection. \textit{Goldfish}~\cite{hans2024be} implements a masking mechanism. \textit{DPSGD}~\cite{abadi2016deep, yu2022differentially} applies gradient clipping and injects noise to achieve  sample-level differential privacy.

\noindent \textbf{Implementation}. We conduct full fine-tuning for \gpt and \pythia. For computing efficiency, \llama fine-tuning is implemented using Low-Rank Adaptation (LoRA)~\cite{hu2022lora} which leads to \textasciitilde4.2M trainable parameters. Additionally, we use subsets of 3K samples to fine-tune the LLMs. We present other implementation details in Appendix~\ref{sec:app-implementation}.

\subsection{Overall Evaluation}
Table~\ref{tab:main_result} provides the overall evaluation compared to several baselines across large language model architectures and datasets. Among these two datasets, CCNews is more challenging, which  leads to higher perplexity  for all LLMs and fine-tuning methods. Additionally, the reference-model-based attack performs the best and demonstrates high privacy risks with attack AUC on the conventional fine-tuned models at 0.95 and 0.85 for Wikipedia and CCNews, respectively. Goldfish achieves similar PPL to the conventional FT method but fails to defend against MIAs. This aligns with the reported results by \citet{hans2024be} that Goldfish resists exact match attacks but only marginally affects MIAs. DPSGD provides a very strong protection in all settings (AUC < 0.55) but with a significant PPL tradeoff. Our proposed \methodname guarantees a robust protection, even slightly better than DPSGD, but with a notably smaller tradeoff on language modeling performance. For example, on the Wikipedia dataset, \methodname delivers perplexity reduction by 15\% to 27\%. Moreover, Table~\ref{tab:tpr} (Appendix~\ref{sec:app-add-res}) provides the TPR at 1\% FPR. Both DPSGD and \methodname successfully reduce the TPR to $\sim$0.02 for all LLMs and datasets. \textit{Overall, across multiple LLM architectures and datasets, \methodname consistently offers ideal privacy protection with  little trade-off in language modeling performance.}

\noindent \textbf{Privacy-Utility Trade-off.}
To investigate the privacy-utility trade-off of the methods, we vary the hyper-parameters of the fine-tuning methods. Particularly, for DPSGD, we adjust the privacy budget $\epsilon$ from (8, 1e-5)-DP to (100, 1e-5)-DP. We modify the masking percentage of Goldfish from 25\% to 50\%. Additionally, we vary the loss weight $\alpha$ from 0.2 to 0.8 for \methodname. Figure~\ref{fig:priv-ult-tradeoff} depicts the privacy-utility trade-off for GPT2 on the CCNews dataset. Goldfish, with very large masking rate (50\%), can slightly reduce the risk of the reference attack but can increase the risks of other attacks. By varying the weight $\alpha$, \methodname offers an adjustable trade-off between privacy protection and language modeling performance. \methodname largely dominates DPSGD and improves the language modeling performance by around 10\% with the ideal privacy protection against MIAs.

\begin{figure}[h]
    \centering
    \includegraphics[width=\linewidth]{figs/privacy-ultility-tradeoff.pdf}
    \caption{Privacy-utility trade-off of the methods while varying hyper-parameters. The Goldfish masking rate is set to 25\%, 33\%, and 50\%. The privacy budget $\epsilon$ of DPSGD is evaluated at 8, 16, 50, and 100. The weight $\alpha$ of \methodname is configured at 0.2, 0.5, and 0.8.}
    \label{fig:priv-ult-tradeoff}
\end{figure}


\subsection{Ablation Study}
\textbf{\methodname without reference models.} To study the impact of the reference model, we adapt \methodname to a non-reference version which directly uses the loss of the current training model (i.e., $s(t_i) = \mathcal{L}_{CE}(\theta; t_i)$) to select the learning and unlearning tokens. This means the unlearning tokens are the tokens that have smallest loss values. Figure~\ref{fig:ppl-auc-noref} presents the training loss and testing perplexity. There is an inconsistent trend of the training loss and testing perplexity. Although the training loss decreases overtime, the test perplexity increases. This result indicates that identifying appropriate unlearning tokens  without a reference model is challenging and conducting unlearning on an incorrect set hurts the language modeling performance.

\begin{figure}[htp]
    \centering
    \includegraphics[width=0.35\textwidth]{figs/train_loss_ppl_noref.pdf}
    \caption{Training Loss and Test Perplexity of \methodname without a reference model.
    % (\lrx{If time permits, it would be better to compare with our training curve here)}
    }
    \label{fig:ppl-auc-noref}
\end{figure}

\noindent \textbf{\methodname with out-of-domain reference models.} To examine the influence of the distribution gap in the reference model, we replace the in-domain trained reference model with the original pretrained base model. 
Figure~\ref{fig:ppl-auc-base-woasc} depicts the language modeling performance and privacy risks in this study. \methodname with an out-of-domain reference model can reduce the privacy risks but yield a significant gap in language modeling performance compared to \methodname using an in-domain reference model.

\noindent \textbf{\methodname without Unlearning.} To study the effects of unlearning tokens, we implement \methodname which use the first term of the loss only ({$\mathcal{L}_{\theta} = \mathcal{L}_{CE}(\theta; \mathcal{T}_h)$}). Figure~\ref{fig:ppl-auc-base-woasc} provides the perplexity and MIA AUC scores in this setting. Generally, without gradient ascent, \methodname can marginally reduce membership inference risks while slightly improving the language modeling performance. The token selection serves as a regularizer that helps to improve the language modeling performance. Additionally, tokens that are learned well in previous epochs may not be selected in the next epochs. This slightly helps to not amplify the memorization on these tokens over epochs.

\begin{figure}[htp]
    \centering
    \includegraphics[width=0.28\textwidth]{figs/auc_vs_ppl_base_woasc.pdf}
    \caption{Privacy-utility trade-off of \methodname with different settings: in-domain reference model, out-domain reference model, and without unlearning}
    \label{fig:ppl-auc-base-woasc}
\end{figure}


\subsection{Training Dynamics}
\textbf{Memorization and Generalization Dynamics}. Figure~\ref{fig:training-dynamics} (left) illustrates the training dynamics of conventional fine tuning and \methodname, while Figure~\ref{fig:training-dynamics} (middle) depicts the membership inference risks. Generally, the gap between training and testing loss of conventional fine-tuning steadily increases overtime, leading to model overfitting and high privacy risks. In contrast, \methodname maintains a stable equilibrium where the gap remains more than 10 times smaller. This equilibrium arises from the dual-purpose loss, which balances learning on hard tokens while actively unlearning memorized tokens. By preventing excessive memorization, \methodname mitigates membership inference risks and enhances generalization.

\begin{figure*}[htp]
    \centering
    \includegraphics[width=0.29\linewidth]{figs/loss_vs_steps_ft_duolearn.pdf}
    \includegraphics[width=0.29\linewidth]{figs/auc_vs_steps_ft_duolearn.pdf}
    \includegraphics[width=0.316\linewidth]{figs/cosine.pdf}
    \caption{Training dynamics of \methodname and the conventional fine-tuning approach. The left and middle figures provide the training-testing gap and membership inference risks, respectively. The testing~$\mathcal{L}_{CE}$ of FT and training~$\mathcal{L}_{CE}$ of \methodname are significantly overlapping, we provide the breakdown in Figure~\ref{fig:add-overlap-breakdown} in Appendix~\ref{sec:app-add-res}. The right figure depicts the cosine similarity of the learning and unlearning gradients of \methodname. Cosine similarity of 1 means entire alignment, 0 indicates orthogonality, and -1 presents full conflict.}
    \label{fig:training-dynamics}
\end{figure*}

\noindent \textbf{Gradient Conflicts}. To study the conflict between the learning and unlearning objectives in our dual-purpose loss function, we compute the gradient for each objective separately. We then calculate the cosine similarity of these two gradients. Figure~\ref{fig:training-dynamics} (right) provides the cosine similarity between two gradients over time. During training, the cosine similarity typically ranges from -0.15 to 0.15. This indicates a mix of mild conflicts and near-orthogonal updates. On average, it decreases from 0.05 to -0.1. This trend reflects increasing gradient misalignment. Early in training, the model may not have strongly learned or memorized specific tokens, so the conflicts are weaker. Overtime, as the model learns more and memorization grows, the divergence between hard and memorized tokens increases, making the gradients less aligned. This gradient conflict is the root of the small degradation of language modeling performance of \methodname compared to the conventional fine tuning approach.

\noindent \textbf{Token Selection Dynamics}. Figure~\ref{fig:token-selection} illustrates the token selection dynamics of \methodname during training. The figure shows that the token selection process is dynamic and changes over epochs. In particular, some tokens are selected as an unlearning from the beginning to the end of the training. This indicates that a token, even without being selected as a learning token initially, can be learned and memorized through the connections with other tokens. This also confirms that simple masking as in Goldfish is not sufficient to protect against MIAs. Additionally, there are a significant number of tokens that are selected for learning in the early epochs but unlearned in the later epochs. This indicates that the model learned tokens and then memorized them over epochs, and the during-training unlearning process is essential to mitigate the memorization risks.

\begin{figure}[htp]
    \centering
    \includegraphics[width=0.7\linewidth]{figs/token-selection-dynamics.pdf}
    \caption{Token Selection Dynamics of \methodname}
    \label{fig:token-selection}
    \vspace{-4mm}
\end{figure}

\subsection{Privacy Backdoor}
To study the worst case of privacy attacks and defense effectiveness under the state-of-the-art MIA, we perform a privacy backdoor -- Precurious~\cite{precurious}. In this setup, the target model undergoes continual fine-tuning from a warm-up model. The attacker then applies a reference-based MIA that leverages the warm-up model as the attack's reference. Table~\ref{tab:backdoor} shows the language modeling and MIA performance on CCNews with GPT-2. Precurious increases the MIA AUC score by 5\%. Goldfish achieves the lowest PPL, aligning with~\citet{hans2024be}, where the Goldfish masking mechanism acts as a regularizer that potentially enhances generalization. Both DPSGD and \methodname provide strong privacy protection, with \methodname offering slightly better defense while maintaining lower perplexity than DPSGD.

% \begin{table}[h]
%     \centering
%     \begin{tabular}{c|cc|cc}
%        \multirow{2}{*}{\textbf{Method}}  & \multicolumn{2}{c}{\textbf{CCNews}} & \multicolumn{2}{c}{\textbf{Wikipedia}} \\ 
%        & \textbf{PPL} & \textbf{AUC} & \textbf{PPL} & \textbf{AUC} \\ \hline
%        \textbf{FT}        & 21.593 & 0.911 \\
%        \textbf{Goldfish}  & \textbf{21.074} & 0.886 \\
%        \textbf{DPSGD}     & 23.279 & 0.533 \\
%        \textbf{DuoLearn}  & 22.296 & \textbf{0.499} \\
%     \end{tabular}
%     \caption{Caption}
%     \label{tab:my_label}
% \end{table}

\begin{table}[h]
    \centering
    \resizebox{\columnwidth}{!}{\begin{tabular}{c|cccccc}
        \textbf{Metric} & \textbf{WU} & \textbf{FT} & \textbf{GF} & \textbf{DP} & \textbf{DuoL} \\ \hline
        \textbf{PPL} & \textit{23.318} & 21.593 & \textbf{21.074} & 23.279 & 22.296  \\
        \textbf{AUC} & \textit{0.500} & 0.911 & 0.886 & 0.533 & \textbf{0.499} \\
    \end{tabular}}
    \caption{Experimental results of privacy backdoor for GPT2 on the CC-news dataset. WU stands for the warm-up model leveraged by Precurious. GF, DP, and DuoL are abbreviations of Goldfish, DPSGD, and \methodname}
    \label{tab:backdoor}
\end{table}

% \subsubsection{Hyperparameter Study}

% \subsubsection{Full fine-tuning versus Parameter efficent fine tuning}

% \subsubsection{Extending to Vision Language Models}



\section{Conclusion}\label{sec:conclusion}

In this paper, we proposed a prototype ASL generation system aimed at improving the naturalness, comprehensiveness, and overall quality of generated signs, addressing key limitations in existing approaches. Our technical evaluations indicate that our proposed approaches improve these aspects, enhancing the quality of generated ASL content. Feedback from DHH participants was mixed; while there was general interest in the system, concerns regarding visual quality and naturalness were noted. Reflecting on our design process and study findings, we discuss key insights and identify key areas for future improvement. While further work is needed, our study takes an initial step toward developing sign language generation systems that better meet the needs of the DHH and signing communities, offering real-world value.



\bibliographystyle{unsrt}
%\bibliographystyle{plain}
\bibliography{reference}

\clearpage
% \autoref{fig:text_bias_example} and \autoref{fig:query_shortening_examples} give examples of incorrect response from \citet{chang_webqa_2021} along with rationale for the mistakes.

% \begin{figure*}[!h]
%     \centering
%     \includegraphics[width=18cm]{figures/text_bias_example.png}
%     \caption{Examples of an incorrect VQA response from the baseline \citep{chang_webqa_2021} where the image source is not used, and instead the most likely answer based on word co-occurance is given.}
%     \label{fig:text_bias_example}
% \end{figure*}


% \begin{figure*}[!h]
%     \centering
%     \includegraphics{figures/query_shortening_examples.png}
%     \caption{Examples of incorrect VQA responses from the baseline \citep{chang_webqa_2021} where the query could be shortened post-retrieval to remove confounding terms that are only useful for retrieval, such as specific nouns.}
%     \label{fig:query_shortening_examples}
% \end{figure*}

\subsection{Model Selection Results}
\label{sec:model_selection}
We explore baseline methods for the QA task on the WebQA validation set. \autoref{tab:baselines} gives results for the baseline models. The MH-VoLTA model outperforms all baseline and zero-shot models on the validation set image questions. However, the extension of the VoLTA model for variable input multi-hop tasks risks a regression in performance on traditional VQA tasks which have fixed-input where the number of input images is constant. To determine MH-VoLTA generalizes from fixed to variable input tasks, we compare performance between two variants of the original VoLTA model, finetuned on one and two image subsets of WebQA, with MH-VoLTA. We find that MH-VoLTA is capable of reasoning over both one and two-image image questions, and it's performance is on-par with VoLTA variants trained on one and two image sources separately\autoref{tab:baselines}. See \autoref{sec:one_vs_two_image_volta} for more details on the one and two image VoLTA variants, as well as a breakdown of model performance by question category (\autoref{fig:multihop_volta_res}). See \autoref{sec:baselines} for a description of the baseline models used.

% GPT-4o and GPT-3.5 models outperform both VLP and GIT baselines. This is particularly impressive for GPT-3.5, which as a unimodal model can only take image caption as input. Despite having mulitmodal inputs, BLIP-2 underperforms GPT-3.5 which we attribute to the difference in decoder quality. We adopt the best performing general purpose LLM (GPT-4o) and specialized VQA model (MH-VoLTA) for all futher experiments.


% Given our focus on question categories that can be addressed using images, \autoref{tab:baselines} splits the baseline results by performance into questions that require one and two image sources respectively. 

\paragraph{VQAv2 and NLVR2}
\label{sec:vqav2}
In addition to WebQA, we evaluate models on two fixed-input VQA datasets---VQAv2 \citep{goyal2017making}, a multi-class, single-image VQA dataset, and \citep{nlvr2}, a binary classification, two-image VQA dataset. These datasets are well-suited to VoLTA classifier architecture. In particular, question categories in VQAv2, along with the associated answer-domains, match well with WebQA, with a substantial portion of both datasets focusing on color, shape, number, and yes/no questions.

\subsection{Multihop VoLTA on one vs two image sources}
\label{sec:one_vs_two_image_volta}
The results for finetuning VoLTA and MH-VoLTA on the WebQA dataset experiments are provided in \autoref{tab:multihop_volta_res}. We explored the application of Multihop-VoLTA in addressing queries based on single images, questions involving two images, and a combination of both single and two-image queries (referred to as multiple images, \autoref{fig:model_perf_qcate}). 

We find that the variable Multihop-VoLTA model (\autoref{fig:multihop_volta_res}) is en-par with the fixed-input one and two-image VoLTA model variants (\autoref{fig:model_perf_qcate}). This underscores the stability of our finetuning approach for MH-VoLTA across both training paradigms. The MH-VoLTA models have on the order of 100M parameters, of which 10M are trainable after applying LoRA. All models are trained for 80 epochs on a Nvidia A6000. 


% We find that the distribution of question types is very different between questions with one and two image sources. Questions with two image sources have more questions in categories with higher random accuracy like Choose and YesNo, whereas questions with one image source have more questions in the harder categories like shape, color, and number, where random accuracy is much lower given the larger answer domain.


% \begin{table}[!h]
%     \centering
%     \begin{tabular}{cccc}
%     \toprule
%          &Accuracy& Fluency  &GPT-Fluency\\
%          \midrule
%          Single Image&0.764&   0.003&\\
%          Two Images&0.851&   0.002&\\
%          Multiple Images&0.799&   0.002&\\
%          \bottomrule
%     \end{tabular}
%     \caption{Multihop VoLTA Results}
%     \label{tab:multihop_volta_res}
% \end{table}


\begin{table}
    \centering
    \caption{Model selection results on WebQA validation set (further broken into 1 and 2 image input categories), and the VQAv2 and NLVR2 (NLV) test sets. MH-V denotes MH-VoLTA. See \autoref{sec:baselines} for model descriptions.}% including Cross-Category accuracy by source image count.}
    \label{tab:baselines}
    \begin{tabular}{clllll}
    \toprule
     & \multicolumn{3}{c}{\makecell{WebQA Acc}} & VQA & NLV \\
        \midrule
         Model & All & 1 img & 2 img & Acc & Acc \\
         \midrule
         \hyperref[sec:mh-volta]{MH-VoL} & \textbf{0.71} & 0.72 & 0.70 & 73.9 & 76.5 \\
         VoLTA\textsubscript{1}  & -- & \textbf{0.77} & -- & \textbf{74.6} & -- \\
         VoLTA\textsubscript{2}  & -- & -- & \textbf{0.84} & -- & \textbf{76.7} \\
         \midrule
         GPT-4o & 0.56 & \textbf{0.58} & 0.69 &  -- & --\\ % 0.77 &
         \hyperref[sec:gpt3.5]{GPT-3.5} & 0.53 &  0.41 & 0.45 & -- & --\\ % 0.47 &
         \hyperref[sec:VLP]{VLP} & 0.50 & 0.40 & 0.42 & -- & -- \\ % fl 0.48 & 
         \hyperref[sec:GIT]{GIT}  & 0.42 & 0.43 & 0.35 & -- & --\\ % 0.19 & 
         \hyperref[sec:blip]{BLIP-2} & 0.40 &  0.37 & 0.44 & -- & --\\ % 0.20 &
         % GIT & Original & No & 0.222 & 0.025 &  \\
         % BLIP-2 & Original & No & 0.400 & 0.126 & \\
         % BLIP & Simple & No & & & \\
         % BLIP-2 & Simple & Yes & 0.357 & 0.133 & \\
         % BLIP-2 + GPT-3.5 & Original & No & & & \\
         % BLIP-GPT-3.5 & Original & Yes & & & \\
         % BLIP-GPT-3.5 & Simple & No & & & \\
         % BLIP-GPT-3.5 & Simple & Yes & & & \\
         \bottomrule
    \end{tabular}
\end{table}

\begin{table}
    \centering
    \caption{MH-VoLTA results and dataset breakdown}
    \label{tab:multihop_volta_res}
    \begin{tabular}{ccc}
    \toprule
         & No. of Samples&Accuracy\\
         \midrule
         Single Image& 760&0.764\\
         Two Images& 576&0.851\\
         Multiple Images& 1336&0.799\\
         \bottomrule
    \end{tabular}
\end{table}

% The results presented in \autoref{fig:model_perf_qcate} demonstrate the robustness of our model when trained independently for single-image and two-image questions. \autoref{fig:gpt_vs_volta_by_category} illustrates a comparable graph for joint finetuning of VoLTA on both single-image and two-image sources. Interestingly, there is virtually no distinction in accuracy, even up to the second decimal place, between individual and joint training. 

% Nevertheless, a notable decline is observed in the 'shape' question category when compared to both single-image and two-image questions. This decrease can be attributed to the model predominantly predicting the 'h' shape for the majority of two-image questions. While this behavior is consistent with single-image questions, it is particularly pronounced in the 'shape' category, where there are 65 questions with a single positive image source and only 8 questions with two positive image sources. This disparity in the distribution of questions with varying positive image sources contributes to the observed performance difference in the 'shape' category.

% The 'YesNo' question category exhibits the highest performance compared to other question categories, which aligns with expectations given its binary classification nature. However, our model shows a slight susceptibility to frequency bias, indicating a tendency to be influenced by the prevalence of certain classes in the training data. Addressing this bias is important for achieving a more balanced and accurate predictive performance across all question categories.


% \begin{figure*}
%     \centering
%     \subfigure[Performance of the MH-VoLTA classifier by question category and image count.]{%
%         \includegraphics[width=0.3\textwidth, trim={0 0 0 3cm},clip]{figures/results/model_perf_acc_comparison_per_qtype_joint.png}
%         \label{fig:multihop_volta_res} 
%         % model_perf_qcate_joint
%     }\hfill
%     \subfigure[Performance of fixed input single and two-image VoLTA classifiers.]{%
%         \includegraphics[width=0.3\textwidth]{figures/results/model_perf_acc_comparison_per_qtype.png}
%         \label{fig:model_perf_qcate}
%     }\hfill
%     \subfigure[Convergence of the VoLTA loss function on the WebQA dev set across several MH-VoLTA training runs.]{%
%         \includegraphics[width=0.33\textwidth]{figures/results/mhvolta_loss.png}
%         \label{fig:loss_convergence}
%     }
%     \caption{Comparison of the variable MH-VoLTA model (left) vs fixed input VoLTA models (center) across different question categories, ordered by the number of image sources per question. Models converge after ~80 epochs (right).}
% \end{figure*}

\begin{figure*}
    \centering
    \begin{subfigure}[b]{0.3\textwidth}
        \includegraphics[width=\linewidth, trim={0 0 0 3cm},clip]{figures/results/model_perf_acc_comparison_per_qtype_joint.png}
        \caption{Performance of the MH-VoLTA classifier by question category and image count.}
        \label{fig:multihop_volta_res}
    \end{subfigure}\hfill
    \begin{subfigure}[b]{0.3\textwidth}
        \includegraphics[width=\linewidth]{figures/results/model_perf_acc_comparison_per_qtype.png}
        \caption{Performance of fixed input single and two-image VoLTA classifiers.}
        \label{fig:model_perf_qcate}
    \end{subfigure}\hfill
    \begin{subfigure}[b]{0.33\textwidth}
        \includegraphics[width=\linewidth]{figures/results/mhvolta_loss.png}
        \caption{Convergence of the VoLTA loss function on the WebQA dev set across several MH-VoLTA training runs.}
        \label{fig:loss_convergence}
    \end{subfigure}
    \caption{Comparison of the variable MH-VoLTA model (left) vs fixed input VoLTA models (center) across different question categories, ordered by the number of image sources per question. Models converge after ~80 epochs (right).}
\end{figure*}


\subsection{Performance by Question Category}
\label{sec:category_perf}
We report the mean accuracy per question category for Multihop-VoLTA in \autoref{fig:multihop_volta_res} using source retrieval oracles. We find that performance is dependent upon the level of training data available, with the shape category having the least number of samples in the dataset. Question counts per category are as follows; Yes/No (n = 7,320), color (n = 1,830), number (n = 2,118), shape (n = 565). The similarity in results across different question categories reinforces the reliability and stability of our model's performance. For a breakdown of labels per question category, see \autoref{sec:categories}.


% \paragraph{CLIP} Contrastive Language-Image Pre-training \cite{radford2021learning} simultaneously trains both an image encoder and a text encoder. Their task is to predict the correct associations within a batch of (image, text) training examples. However, for the challenging task of full-scale retrieval without any prior exposure in WebQA, it's important to note that running VLP-based retrieval across the entire source collection would be prohibitively time-consuming, estimated at three years. To overcome this limitation, the WebQA approach opted to use CLIP for dense retrieval and BM25 \cite{robertson2009probabilistic} for a more coarse-grained retrieval process.

% \paragraph{VLP} The unified Vision-Language Pre-training (VLP) model \cite{zhou_unified_2020}, is a versatile multimodal generative transformer. This model can be customized through finetuning to excel in either vision-language generation tasks, like generating captions for images, or comprehension tasks, such as answering questions based on visual content. What sets it apart is its utilization of a single multi-layer transformer network for both encoding and decoding, which distinguishes it from numerous other approaches that employ distinct models for these two functions.

% We follow \cite{chang_webqa_2021} in using VLP a baseline model. However, we also adapt two other Vision Language models as baselines, namely  GIT \citep{wang2022git} and BLIP2 \citep{li2023blip2}, which are introduced more formally in later sections.

\subsection{GPT-4o Retrieval Prompt}
\begin{framed}
\label{frame:labeling_prompt}
\textbf{system}: Answer the question in one word. Then list the Fact\_ID or Image\_ID of all facts used to derive the answer in square brackets.

\textbf{human}: Question: <query>

\textbf{human}: Text Facts:
[fact\_id\_1: fact\_1, ..., id\_n: fact\_n]

\textbf{human}: Image\_ID: img\_id\_1, \\
Caption: img\_caption\_1

\textbf{human}: [Input\_type=image] \\
image\_url=url\_1

...

\textbf{human}: Image\_ID: img\_id\_m, \\
Caption: img\_caption\_m

\textbf{human}: [Input\_type=image] \\
image\_url=url\_m

\end{framed}

\subsection{Question Complexity Analysis Metrics}
\label{appendix:complexity_analysis_metrics}

The Flesch-Kincaid Grade Level is a readability metric that evaluates the difficulty of a text based on the length of its words and sentences \citep{flesch2007flesch}, and is defined as;
\begin{equation}
\begin{split}
  \text{FKGL} = 0.39 \left( \frac{\text{Total Words}}{\text{Total Sentences}} \right) \\ + 11.8 \left( \frac{\text{Total Syllables}}{\text{Total Words}} \right) - 15.59  
\end{split}
\end{equation}

The Gunning Fog Index is a readability test used in linguistics to assess the complexity of English writing \citep{gunning1952technique}, and is defined as;
\begin{equation}
\begin{split}
    \text{GFI} = \frac{0.4 \times \text{Total Words}}{\text{Total Sentences}} \\
    + \frac{40 \times \text{Total Complex Words}}{\text{Total Words}}
\end{split}
\end{equation}


% \footnote{The Flesch-Kincaid Grade Level is a readability metric that evaluates the difficulty of a text based on the length of its words and sentences.}

% \footnote{The Gunning Fog Index is a readability test used in linguistics to assess the complexity of English writing.}



% $$


\subsection{Question Category Domain Lists}
\label{sec:categories}
\begin{figure*}
\begin{lstlisting}
yesno_set = {'yes', 'no'}
color_set = {
    'orangebrown', 'spot', 'yellow', 'blue', 'rainbow', 'ivory', 
    'brown', 'gray', 'teal', 'bluewhite', 'orangepurple', 'black', 
    'white', 'gold', 'redorange', 'pink', 'blonde', 'tan', 'turquoise', 
    'grey', 'beige', 'golden', 'orange', 'bronze', 'maroon', 'purple', 
    'bluere', 'red', 'rust', 'violet', 'transparent', 'yes', 'silver', 
    'chrome', 'green', 'aqua'
}
shape_set = {
    'globular', 'octogon', 'ring', 'hoop', 'octagon', 'concave', 'flat', 
    'wavy', 'shamrock', 'cross', 'cylinder', 'cylindrical', 'pentagon', 
    'point', 'pyramidal', 'crescent', 'rectangular', 'hook', 'tube', 
    'cone', 'bell', 'spiral', 'ball', 'convex', 'square', 'arch', 'h', 
    'cuboid', 'step', 'rectangle', 'dot', 'oval', 'circle', 'star', 
    'crosse', 'crest', 'octagonal', 'cube', 'triangle', 'semicircle', 
    'domeshape', 'obelisk', 'corkscrew', 'curve', 'circular', 'xs', 
    'slope', 'pyramid', 'round', 'bow', 'straight', 'triangular', 
    'heart', 'fork', 'teardrop', 'fold', 'curl', 'spherical', 
    'diamond', 'keyhole', 'conical', 'dome', 'sphere', 'bellshaped', 
    'rounded', 'hexagon', 'flower', 'globe', 'torus'
}   
\end{lstlisting}
\end{figure*}

\subsection{Baseline Models}
\label{sec:baselines}

\paragraph{VLP}
\label{sec:VLP}

% A pretrained object detector is used to extract image regions $r_i$, region features $R_i$, region object labels $C_i$, and region geometric features $G_i$. Regions are embedded according to the following network, where [.|.] is the concatenation of features;
% \begin{equation}
%     r_i = W_r R_i + W_p[LayerNorm(W_cC_i)|LayerNorm(W_gG_i]
% \end{equation}

% Special tokens [CLS], [SEP], and [STOP] are special input tokens that represent the start of visual input, the boundary between visual and text input, and the end of the sequence. The weight matrices are trained according to two masked language modeling objectives, where 15\% of input text tokens are replaced (80\% with a [MASK] token, 10\% with a random token and 10\% with the original token). 

% The two objectives are the BERT bidirectional objective and the seq2seq objective which satisfies the auto-regressive property that is desired by the generative setting of VQA. Notably, VLP objectives do not include Next Sentence Prediction as in BERT, which was found to lower performance \citep{zhou_unified_2020}. The seq2seq objective is implemented with a simple self-attention mask M. 
The VLP transformer model consists of a unified encoder and decoder \citep{zhou_unified_2020}. The VLP architecture is made up of 12 layers of transformer blocks trained according to the BERT bidirectional and the seq2seq objectives where the self-attention module in the transformer block are defined as;

\begin{equation}
    A^l = softmax(\frac{Q^TK}{\sqrt{d}} + M)V^T
\end{equation}

where $V = W^l_VH^l-1, Q = W^l_QH^l-1, K = W^l_KH^l-1$. As in \cite{vaswani2017attention}, a feedforward layer (with residual) maps $A^l$ to $H^l$. The model is trained on image caption pairs, and then finetuned for the VQA task. Finetuning follows by taking the hidden states from the final layer and feeding them to a multi-layer perceptron. The model used has been finetuned twice, once on the VQA dataset (as described by \cite{yu_unified_2023}), and again on the WebQA dataset. 

% In the later case, only seq2seq masking is applied, but more importantly, the model is trained on questions that require two image sources as described in the dataset section (section 5.1). Of all the baselines, this is the only model which has been explicitly trained for multihop VQA. During inference, five variants are generated using beam search and the most confident output is used.

\paragraph{GIT}
\label{sec:GIT}
To contrast with VLP, a pretrained multihop VQA model, we use a pre-trained Generative Image-to-Text Transformer (GIT) \citep{wang2022git}.
GIT employs a simplified VQA architecture with one encoder for images and one decoder for text. As such, the model is explicitly incapable for multihop VQA between text and images, so it serves as a baseline for pre-trained models that do not utilize image descriptions, and so we concatenate image sources if there are more than one. 

% The GIT model is finetuned on VQA-2 \citep{goyal2017making}, a typical VQA paradigm where questions do not require knowledge external to the image. By inspection, we also find that questions are simpler (and shorter) in the VQA tasks for which it is trained. We therefore expect that simplified questions will perform better on this model and so the GIT model may be illustrative of where WebQA departs from typical VQA tasks.

GIT is pre-trained using the language modeling task (as opposed to MLM which is used by VLP) where the model learns to predict captions in an auto-regressive manner. For VQA finetuning, the text input is swapped to the query, so that answers are predicted. 

\paragraph{BLIP-2}
\label{sec:blip}
Similar to VLP, the Bootstrapping Language-Image Pre-training model (BLIP) is a unified vision language pre-trained model \citep{li2022blip}. It relies on a visual transformer which is less computationally demanding and is pre-trained on over 100 Million image-caption pairs using a contrastive loss (ITC) for image-text contrastive alignment and image-text matching (ITM). 
% The authors show that it achieves state of the art performance on various VQA tasks and follows the architecture described in Figure \ref{fig:blip_vqa}. The authors of \cite{li2023blip2} introduced BLIP-2 which is based on a similar architecture as BLIP but it introduces a Q-transformer as the encoder and an LLM decoder. 
In addition to the ITC and ITM losses, the authors introduce an additional Image-grounded text generation (ITG) loss that trains the Q-former encoder to generate texts, given input images as the condition. 

% The final objective to minimize is defined as;
% \begin{equation}
%     \textit{Loss} = \text{ITC + ITG + ITM}
% \end{equation}
% \begin{figure}[h]
%   \centering
%   \includegraphics[width=\columnwidth]{figures/architectures/blip-vqa.png}
%   \caption{BLIP VQA finetuning architecture \cite{li2022blip}}
%   \label{fig:blip_vqa}
% \end{figure}

\paragraph{GPT-3.5 Turbo}
\label{sec:gpt3.5}
Throughout the dataset, a consistent challenge emerges: the model must focus on details, understand them, and accurately respond to questions, even after the provision of positive source images. This challenge has led to the exploration of an image-to-text approach, where the task involves generating descriptive captions for the images. This transforms the problem into a unimodal text retrieval and generation task. Using this method, the SOLAR model has had success on the WebQA task \cite{alibaba_text}. Accordingly, we include a zero-shot oracle baseline, passing queries and image captions to gpt-turbo-3.5 \citep{brown2020language}.


% \subsection{Vin+VLP Baseline Results}
% \label{appendix:baseline_results}
% \begin{figure}[h]
%   \centering
%   \includegraphics[width=\textwidth]{figures/results/histogram_accs_qcate.png}
%   \caption{A histogram depicting the distribution of datapoint-wise accuracy scores for the Vin+VLP baseline on the validation dataset for each type of question category, as assessed using the official evaluation script}
%   \label{fig:hist_accs_qcate}
% \end{figure}

% \begin{figure}[!h]
%   \centering
%   \begin{minipage}{0.48\textwidth}
%     \centering
%     \includegraphics[width=\textwidth]{figures/results/histogram_accuracy_1img.png}
%     \caption{A histogram similar to Fig \ref{fig:hist_accs_qcate}, but specifically for questions with a single positive image source}
%     \label{fig:hist_1img}
%   \end{minipage}\hfill
%   \begin{minipage}{0.48\textwidth}
%     \centering
%     \includegraphics[width=\textwidth]{figures/results/histogram_accuracy_2img.png}
%     \caption{A histogram similar to Fig \ref{fig:hist_accs_qcate}, but specifically for questions with a two positive image sources}
%     \label{fig:hist_2img}
%   \end{minipage}
% \end{figure}

% \begin{figure*}[!h]
%     \centering
%     \includegraphics[width=0.49\linewidth]{figures/examples/guitar.png}
%     \includegraphics[width=0.49\linewidth]{figures/examples/guitar.png}
    
%     \caption{
%     \textit{Guid: d5be98180dba11ecb1e81171463288e9} \\ 
%     \textit{Question Category: "choose"} \\
%     \textit{Question: "Which instrument usually requires a bow to play it; A violin or Fernandes Monterey Deluxe?"} \\ 
%     \textit{Ground Truth: "A violin requires a bow to play it, but the Fernandes Monterey Deluxe does not."} \\ 
%     \textit{Prediction: "A violin"} \\ 
%     \\
%     Even though the model answered the question correctly, neither of the provided images in this modified sample contains a violin. The model simply answers this question based on its pretraining knowledge.}
%     \label{fig:blip_example_1}
% \end{figure*}
% \vspace{100mm}
% \begin{figure}[!h]
%   \centering
%   \includegraphics[width=0.4\columnwidth]{figures/results/histogram_accuracy.png}
%   \vspace{-5mm}
%   \caption{A histogram depicting the distribution of datapoint-wise accuracy scores for the Vin+VLP baseline on the validation dataset, as assessed using the official evaluation script}
%   \label{fig:vin_vlp_hist_acc}
% \end{figure}

% The data presented in Figure \ref{fig:vin_vlp_hist_acc} reveals a significant trend: the accuracy scores predominantly cluster around either 0 or 1. This pattern aligns with the nature of the evaluation methodology, which relies on category-aware lexical overlap for assessment. It's important to note that the evaluation script selects the top candidate answer from the model predictions for these assessments.

% \vspace{8mm}


% \section{Error Analysis - Examples}
% \label{sec:appendix_error_analysis_examples}

% \begin{figure*}[!h]
%     \centering
%     \vspace{-5mm}
%     \includegraphics[width=0.35\linewidth]{figures/examples/empire_tower.jpg}
%     \includegraphics[width=0.35\linewidth]{figures/examples/empire_tower_2.jpg}
%     \vspace{-2mm}
%     \caption{
%     \textit{Guid: d5bc623c0dba11ecb1e81171463288e9} \hspace{20mm} \textit{Question Category: "YesNo"} \\
%     \textit{Question: "Was there a building constructed after 2007 that can now be seen in the distance behind the Empire State Building and mimics its shape?"} \\ 
%     \textit{Ground Truth: "There was a building constructed after 2007 that can now be seen in the distance behind the Empire State Building that mimics its shape"} \\ 
%     \textit{Prediction: "Yes , there was a building constructed after 2007 ..."} \\ 
%     % \\
%     Even though the model answered the question correctly, neither the images nor their captions provide any information of the year being 2007 or later. As the answer does contain a Yes/No keyword, the prediction can never be wrong.
%     }
%     \label{fig:example_3}
% \end{figure*}

% \begin{figure}[!h]
%   \centering
%   \begin{minipage}{0.48\textwidth}
%     \centering
%     \includegraphics[width=\linewidth]{figures/examples/Barcelona-St-Josep-La-Boqueria.jpg}
%     \vspace{-7mm}
%     \caption{
%       \textit{\\Guid: d5bc2b6e0dba11ecb1e81171463288e9} \\ 
%       \textit{Question Category: "choose"} \\
%       \textit{Question: "Is the word on the lollipops at Market St Joseph La Boqueria in Barcelona written in print or cursive?"} \\ 
%       \textit{Ground Truth: "The word on the lollipops is written in cursive."} \\ 
%       \textit{Prediction: "The word on the lollipops at Market St Joseph La Boqueria in Barcelona is written in print ."} \\ 
%       \\
%       Answering this question requires a human to closely examine the image due to its low resolution and the small portion relevant to the answer (The white annotation has been added by us for improved visualization and is not part of the original image)}
%     \label{fig:example_1}
%   \end{minipage}
%   \hfill
%   \begin{minipage}{0.48\textwidth}
%     \centering
%     \includegraphics[width=\linewidth]{figures/examples/Washington_Square_Arch-Isabella.jpg}
%     \vspace{-7mm}
%     \caption{
%       \textit{\\Guid: d5bcd3700dba11ecb1e81171463288e9} \\ 
%       \textit{Question Category: "shape"} \\
%       \textit{Question: "What shape is the fountain near the arch in Washington Square Park?"} \\ 
%       \textit{Ground Truth: "There is a circle shaped fountain near the arch in Washington Square Park."} \\ 
%       \textit{Prediction: "The fountain near the arch in Washington Square Park is a circle ."} \\ 
%       \\
%       While the model provides a correct answer to this question, the accuracy score is 0.5, which should ideally be 1. This issue arises from the model's inability to distinguish between a square shape and a place like 'Washington Square Park.' The evaluation metric interprets the model's response as 'square' and 'circle' instead of just 'circle.'}
%     \label{fig:example_2}
%   \end{minipage}
% \end{figure}



% \begin{appendices}

% \section{Section title of first appendix}\label{secA1}

% An appendix contains supplementary information that is not an essential part of the text itself but which may be helpful in providing a more comprehensive understanding of the research problem or it is information that is too cumbersome to be included in the body of the paper.

% %%=============================================%%
% %% For submissions to Nature Portfolio Journals %%
% %% please use the heading ``Extended Data''.   %%
% %%=============================================%%

% %%=============================================================%%
% %% Sample for another appendix section			       %%
% %%=============================================================%%

% %% \section{Example of another appendix section}\label{secA2}%
% %% Appendices may be used for helpful, supporting or essential material that would otherwise 
% %% clutter, break up or be distracting to the text. Appendices can consist of sections, figures, 
% %% tables and equations etc.

% \end{appendices}

%%===========================================================================================%%
%% If you are submitting to one of the Nature Portfolio journals, using the eJP submission   %%
%% system, please include the references within the manuscript file itself. You may do this  %%
%% by copying the reference list from your .bbl file, paste it into the main manuscript .tex %%
%% file, and delete the associated \verb+\bibliography+ commands.                            %%
%%===========================================================================================%%
%\bibliographystyle{unsrt}
%\bibliography{sn-bibliography}% common bib file
%% if required, the content of .bbl file can be included here once bbl is generated
%%\input sn-article.bbl

%% Default %%
%%\input sn-sample-bib.tex%

\end{document}



%
% paper title
% Titles are generally capitalized except for words such as a, an, and, as,
% at, but, by, for, in, nor, of, on, or, the, to and up, which are usually
% not capitalized unless they are the first or last word of the title.
% Linebreaks \\ can be used within to get better formatting as desired.
% Do not put math or special symbols in the title.
\title{Chain of Backdoor Attacks for Code-Driven Embodied Intelligence}
%
%
% author names and IEEE memberships
% note positions of commas and nonbreaking spaces ( ~ ) LaTeX will not break
% a structure at a ~ so this keeps an author's name from being broken across
% two lines.
% use \thanks{} to gain access to the first footnote area
% a separate \thanks must be used for each paragraph as LaTeX2e's \thanks
% was not built to handle multiple paragraphs
%
%
%\IEEEcompsocitemizethanks is a special \thanks that produces the bulleted
% lists the Computer Society journals use for "first footnote" author
% affiliations. Use \IEEEcompsocthanksitem which works much like \item
% for each affiliation group. When not in compsoc mode,
% \IEEEcompsocitemizethanks becomes like \thanks and
% \IEEEcompsocthanksitem becomes a line break with idention. This
% facilitates dual compilation, although admittedly the differences in the
% desired content of \author between the different types of papers makes a
% one-size-fits-all approach a daunting prospect. For instance, compsoc 
% journal papers have the author affiliations above the "Manuscript
% received ..."  text while in non-compsoc journals this is reversed. Sigh.

\author{%
%Anonymous authors

  Aishan Liu, Yuguang Zhou,  Siyuan Liang, Xianglong Liu, Tianyuan Zhang, Jiakai Wang, Yanjun Pu, \\Tianlin Li, Junqi Zhang, Wenbo Zhou, Qing Guo, Dacheng Tao \\
  
%Beihang University\textsuperscript{1} \quad SenseTime Research\textsuperscript{2} \quad UC Berkeley\textsuperscript{3}\\
%  Johns Hopkins University\textsuperscript{4} \quad Oxford University\textsuperscript{5} \quad JD Explore Academy\textsuperscript{6}


\thanks{A. Liu, Y. Zhou, X. Liu, T. Zhang are with the State Key Lab of Software Development Environment, Beihang University, Beijing, China.}

\thanks{J. Wang and Y. Pu are with the ZGC Laborotary, Beijing, China.}

\thanks{J. Zhang and W. Zhou are with the University of Science and Technology of China, China}

\thanks{S. Liang is with the National University of Singapore, Singapore.}

\thanks{Q. Guo is with the A*STAR, Singapore.}

\thanks{T. Li and D. Tao are with Nanyang Technological University, Singapore.}


\thanks{The first two authors contribute equally. Correspondence to Xianglong Liu.}

}
% note the % following the last \IEEEmembership and also \thanks - 
% these prevent an unwanted space from occurring between the last author name
% and the end of the author line. i.e., if you had this:
% 
% \author{....lastname \thanks{...} \thanks{...} }
%                     ^------------^------------^----Do not want these spaces!
%
% a space would be appended to the last name and could cause every name on that
% line to be shifted left slightly. This is one of those "LaTeX things". For
% instance, "\textbf{A} \textbf{B}" will typeset as "A B" not "AB". To get
% "AB" then you have to do: "\textbf{A}\textbf{B}"
% \thanks is no different in this regard, so shield the last } of each \thanks
% that ends a line with a % and do not let a space in before the next \thanks.
% Spaces after \IEEEmembership other than the last one are OK (and needed) as
% you are supposed to have spaces between the names. For what it is worth,
% this is a minor point as most people would not even notice if the said evil
% space somehow managed to creep in.



% The paper headers
\markboth{IEEE TRANSACTIONS ON PATTERN ANALYSIS AND MACHINE INTELLIGENCE}%
{Shell \MakeLowercase{\textit{et al.}}: Bare Demo of IEEEtran.cls for Computer Society Journals}
% The only time the second header will appear is for the odd numbered pages
% after the title page when using the twoside option.
% 
% *** Note that you probably will NOT want to include the author's ***
% *** name in the headers of peer review papers.                   ***
% You can use \ifCLASSOPTIONpeerreview for conditional compilation here if
% you desire.



% The publisher's ID mark at the bottom of the page is less important with
% Computer Society journal papers as those publications place the marks
% outside of the main text columns and, therefore, unlike regular IEEE
% journals, the available text space is not reduced by their presence.
% If you want to put a publisher's ID mark on the page you can do it like
% this:
%\IEEEpubid{0000--0000/00\$00.00~\copyright~2015 IEEE}
% or like this to get the Computer Society new two part style.
%\IEEEpubid{\makebox[\columnwidth]{\hfill 0000--0000/00/\$00.00~\copyright~2015 IEEE}%
%\hspace{\columnsep}\makebox[\columnwidth]{Published by the IEEE Computer Society\hfill}}
% Remember, if you use this you must call \IEEEpubidadjcol in the second
% column for its text to clear the IEEEpubid mark (Computer Society jorunal
% papers don't need this extra clearance.)



% use for special paper notices
%\IEEEspecialpapernotice{(Invited Paper)}



% for Computer Society papers, we must declare the abstract and index terms
% PRIOR to the title within the \IEEEtitleabstractindextext IEEEtran
% command as these need to go into the title area created by \maketitle.
% As a general rule, do not put math, special symbols or citations
% in the abstract or keywords.
\IEEEtitleabstractindextext{%
\begin{abstract}
  Large language models (LLMs) have revolutionized the programming and development of embodied intelligence. In this paradigm, LLMs translate complex tasks described in abstract language into a sequence of code snippets. The embodied agent subsequently interacts with the environment to solve these tasks, with the execution logic guided by the programs and the orchestration of operation calls (\eg, perception modules). However, exploiting untrusted third-party LLMs poses considerable security risks. This paper introduces \method{}, which for the first time identifies a severe chain of backdoor threats in this practical scenario. By presenting a few shots of poisoned demonstrations, adversaries can clandestinely infect a black-box LLM, prompting it to generate programs with backdoor defects. These malicious programs, supplied by LLMs to downstream users, will be activated and compromise the reliability of the operational agent when specific visual triggers appear. Originating from a few shots of stealthy demonstrations, our attack progresses from LLM to the generated code until infiltrating the intelligent system, thus establishing a chain of backdoors that could have serious consequences for millions of downstream embodied agents. To achieve the goal, we employ an additional LLM for poisoned demonstration prompt generation and treat the optimization process as a two-player game between the discriminator and generator LLMs, where the optimized poisoned prompts fed to the generator could output programs with accurate defects that are realistic enough output to fool the discriminator. To comprehensively explore the potential risks, we broaden the attack and devise five program defects attacking modes compromising the key aspects of confidentiality, integrity, and availability of the embodied agents. To validate the effectiveness, we conducted extensive experiments on a range of tasks including robot planning, robot manipulation, and compositional visual reasoning. Additionally, we showcase the potential of our approach and successfully attack real-world autonomous driving systems. This paper aims to raise awareness of the potential threats of backdoors for embodied intelligence in practical LLM usage scenarios involving code. Our code and demos are available at the \href{https://chain-of-backdoor.github.io}{website}.
\end{abstract}

% Note that keywords are not normally used for peerreview papers.
\begin{IEEEkeywords} Chain of Attacks, 
Model Robustness, Embodied Intelligence.
\end{IEEEkeywords}}


% make the title area
\maketitle


% To allow for easy dual compilation without having to reenter the
% abstract/keywords data, the \IEEEtitleabstractindextext text will
% not be used in maketitle, but will appear (i.e., to be "transported")
% here as \IEEEdisplaynontitleabstractindextext when the compsoc 
% or transmag modes are not selected <OR> if conference mode is selected 
% - because all conference papers position the abstract like regular
% papers do.
\IEEEdisplaynontitleabstractindextext
% \IEEEdisplaynontitleabstractindextext has no effect when using
% compsoc or transmag under a non-conference mode.



% For peer review papers, you can put extra information on the cover
% page as needed:
% \ifCLASSOPTIONpeerreview
% \begin{center} \bfseries EDICS Category: 3-BBND \end{center}
% \fi
%
% For peerreview papers, this IEEEtran command inserts a page break and
% creates the second title. It will be ignored for other modes.
\IEEEpeerreviewmaketitle


\section{Introduction}

% State of the world (robots for creative activites)
The term ``robot,'' originally signifying `forced labor,' has long been associated with labor and work. Robots have demonstrated their utility in various automated productive and social contexts, where the primary goals are improving productivity, safety, and fostering social interactions with humans~\cite{simoes2022designing, weidemann2021role, honig2018understanding}. However, an increasing number of cases feature using of robots in creative settings. Unlike productive contexts, where the focus is on efficiency and task completion~\cite{arents2022smart}, or social contexts, where communication and trust are prioritized~\cite{nam2020trust, saunderson2019robots}, creative environments prioritize artistic innovation and expression~\cite{hsueh2024counts}. This shift fundamentally alters the dynamics of human-robot interaction, redefining the roles and expectations for both humans and robots.

For instance, robots’ social behaviors are leveraged to support the generation and expression of creative ideas~\cite{hu2021exploring, sandoval2022human, alves2020creativity}, and programmable robotic movements and trajectories are employed to inspire artistic activities such as sketching~\cite{lin2020your}. These studies often engage participants from creative fields who possess limited prior experience with robotics, and are typically conducted in short-term, experimental settings. Consequently, the findings from these studies remain constrained since much can be learned from professional practitioners' experiences to inform system design such as digital fabrication~\cite{hirsch2023nothing}. There is a notable gap in research examining the long-term, active, and practical experience of integrating robotic systems into the creative processes. As a result, the deeper insights into how robots facilitate and shape creative processes, beyond simply augmenting human creativity, remain underexplored. In this study, we aim to better understand the impacts of robots on creative processes and outcomes.

As early as Leonardo da Vinci's 16th century ``Automaton,'' artists have explored the creative affordances of robotic systems~\cite{shanken2002cybernetics, pagliarini2009development, jeon2017robotic}. The artistic creation process typically encompasses various stages, including the exploration of materials and techniques, ongoing experimentation and iteration, and the continual refinement of the artists' insights into their creative subjects~\cite{lewis2023art, sturdee2022state}. Therefore, investigating the artistic process involving robots offers an opportunity to gain deeper insights into robots' creative potential. Robotic art, in particular, provides a compelling case for this exploration.

We define robotic art as artworks that utilize robotic or automated machines to create artistic experiences and tangible artifacts. One example is robotic installation art, in which robots are programmed to follow specific rules that embody the artist’s expression (\autoref{fig:teaser} (a)). Another example is responsive art, in which robots react to their environment, with behaviors that change over time or in response to spectators (\autoref{fig:teaser} (b)). Additionally, there are robotic creators, which possess a degree of agency, allowing them to collaborate with human artists and produce works that extend beyond mere replication of human-created art (\autoref{fig:teaser} (c) and (d)). As such, robotic art becomes a rich case for exploring human-machine interactions in creative contexts. Gaining a deeper understanding of how robots facilitate artistic expression can provide insights for designing computing systems to support creative activities~\cite{gomez2021robot}.

% Therefore, we did...
We draw on semi-structured, in-depth interviews with renowned professional robotic artists to investigate the use of robots in artistic practice. Specifically, our goal is to understand how artistic exploration of robotic systems challenges conventional assumptions about the functions of robots, such as their roles in automating repetitive tasks or serving human needs. We also explore the implications of robots in the artistic process and examine how creativity may emerge within robotic art. To address these interrelated inquiries, our study focuses on the practice of robotic art, posing the research question: \textit{How do robotic artists utilize robots in their artistic practice?} We approach this inquiry through the perspectives and experiences of robotic artists, who creatively design, modify, and repurpose robotic systems for artistic expression and exploration.

% The key findings are...
Our findings highlight the social, material, and temporal dimensions of artists' practices that shape their creativity and artistic outcomes. The creation of robotic art is largely a social process, as artists receive both explicit and implicit feedback through the audience's reactions and reception of their work. Simultaneously, the embodiment and malfunctions inherent to robotic systems drive artistic experimentation. The temporal processes of creation and exhibition, beyond just the final product, further enhance the creative value. Our empirical analysis presents how creativity emerges through the interplay of social, material, and temporal interactions among artists, robots, audiences, and the environment.

% The contributions of this work are...
We make two main contributions to HCI in this study. 
First, we elucidate the interactive mechanisms among key actors---human creators, machines, audiences, and environments---within the practice of robotic art, a topic that remains underexplored in HCI. Our findings reveal the significance of sociality (e.g., interactions between artists and audiences), materiality (e.g., the embodiment and malfunctions of robots), and temporality (e.g., the processes of creation and exhibition) in shaping creative values. We propose that these three facets are central to the creative process and facilitate the emergence of creativity in robotic art.
Second, drawing from the findings, we offer implications for \textit{socially informed}, \textit{material-attentive}, and \textit{process-oriented} creation with computing systems. We suggest leveraging these three aspects to enhance creativity and the creative experience. Specifically, we discuss the value of incorporating implicit audience feedback, designing with technical malfunctions, and focusing on the post-creation process to foster alternative creative experiences with machines~\cite{alter2010designing, juarez2022glitch}.



\section{Related work}




\textbf{Reasoning in LLMs}. Reasoning is a cognitive process that involves thinking about something logically and systematically, using evidence and past experiences to draw conclusions or make decisions \cite{reason1,reason2}. Recent studies have demonstrated that LLMs exhibit remarkable reasoning capabilities in various tasks, including mathematical reasoning \cite{reasonllm1}, common sense reasoning \cite{reasonllm2}, symbolic reasoning \cite{reasonllm3}, and causal reasoning \cite{reasonllm4}. Subsequently, Chain-of-thought (CoT) \cite{cot1,cot2,cot3,cot4,cot5} has emerged as a promising approach for further enhancing these reasoning capabilities.

While the reasoning capabilities of LLMs have contributed to their impressive performance across various downstream tasks, their potential exploitation in jailbreak attacks remains largely unexplored. In this study, we focus on leveraging reasoning capabilities to facilitate jailbreak attacks.

\textbf{Multi-turn Jailbreak Attack}. Typical multi-turn jailbreak methods follow the principle of starting with harmless conversations and gradually making the queries more harmful in subsequent turns. Different methods have designed specific strategies based on this principle, including applying cognitive psychology theories to gradually modify subsequent queries \cite{mpsy1,mpsy2}, using actor networks to expand the attack range of subsequent queries \cite{ren2024}, extracting harmful keywords from original queries to construct semantically equivalent ones \cite{coa,cfa}, and breaking down the target query into multiple subqueries and merging the corresponding answers to achieve the final jailbreak \cite{sub1,sub2}.

Existing multi-turn jailbreak methods often suffer from semantic drift or fail to generate effective attacks. In contrast, our approach leverages LLMs' reasoning capabilities to ensure a stable and effective jailbreak process.
\section{Threat Model}\label{sec-3-threatModel}
% \takeaway{Use a table to show an IPV attacker vs. traditional attacker. Then focus on (1) Attacker's capacity
% and their capacity limits; scope is important: might not be good for physical violence, maybe helpful for surveillance;
% (2)Requirements of a successful attack. Cite papers to show the applicability of those attacks.
% What attackers can (not) do; under which conditions can (not) they achieve successful attack}

% Please add the following required packages to your document preamble:
% \usepackage[table,xcdraw]{xcolor}
% Beamer presentation requires \usepackage{colortbl} instead of \usepackage[table,xcdraw]{xcolor}
% Please add the following required packages to your document preamble:
% \usepackage[table,xcdraw]{xcolor}
% Beamer presentation requires \usepackage{colortbl} instead of \usepackage[table,xcdraw]{xcolor}

% We make following assumptions for our threat model. First, IPV abusers are non-tech expert, unable to hack into OS to acquire
% sensitive data. In this paper, their IPV behaviors are via physical access to victims' smartphones, and such access is already
% acquired (through compromise, and guess-out, etc.), meaning that all one-time authentications (passwords, face id, finger 
% print, etc.) fails to prevent their access. The abusers are free to manipulate the device, viewing application information, 
% uploading and downloading data.

% For the defending system, we assume that the monitoring data sources are trustworthy and intact. This is consistent with our
% previous assumption of low technical level of IPV abuser to conduct OS level attack such as spoofing and adversarial machine learning. 
% Additionally, the integrity of the system is not affected by the awareness of its existence of abusers,
% thus guaranteeing system functioning.

In this section we describe the scope and some assumptions of our threat model. We focus on the IPI behaviors that require physical access to mobile smatphones. IPI abusers cause harm to victims by having access to and interacting with the victims' smartphones stealthily. As shown in Table \ref{tab-ipvvstradition}, unlike traditional cybersecurity attacks, we assume that the abusers have authenticated access to victims' devices, e.g., by registering their biometrics or having access to passwords, so they directly can interact with the data and applications stored on the victim's smartphone. Aligned with previous research \cite{freed2018stalker}, we assume that abuser's have limited technical background, given that abusers come from the general population compared to cyber attackers. This means that abusing behaviors are restricted to user-interface interactions, e.g., by viewing the content on the phone or installing applications. Additionally, we assume that the stealthiness of our system is sufficient for not alerting abusers, so the system is safe from IPI attacks.

Another notable point is that our study does not apply to partners in extreme situations, i.e., the use of the detection system will escalate IPI behavior to a higher level such as severe physical or psychological violence. As discussed in \cite{freed2019my,havron2019clinical,tseng2022care}, IPI abusers may escalate and bring more harm to victims if they discover evidence of anti-IPI measures. Although our designed goals are stealthy and not alerting, the use of our tool needs the initial assessment from a security clinic. More harmful scenarios require intervention from external agencies, such as police, law enforcement, and clinical centers, which are out of the scope of this work.


%the discovery of anti-IPV actions the anti-IPV actions might bring more harm to the IPV victims and police, law agencies, and clinical therapy should engage in insted of detection tools which won't help at this point.






%running without awareness by the abusers, so the data sources and system itself are guaranteed to be safe from IPV attacks.
\section{DAPAO Attacks}
In this section, first, we present the feasibility study, demonstrating our observation of information leakage in robust watermarks. Next, we provide the theoretical analysis for method validation. Last, we introduce evasion and forgery attacks based on the observation.

% We empirically discover that learning-based watermarking systems mitigate distortion effects (e.g., compression) by expanding the regions where the watermark pattern is embedded or increasing its magnitude, ensuring the remaining watermark remains detectable. Besides the encoding part, the system trains the watermark decoder to extract watermarks more effectively, which can be understood as increasing the model's attention weight on watermark signals.

\subsection{Feasibility Study}\label{sec:pilot}
We empirically find that learning-based robust watermarking systems counteract distortion effects (e.g., compression) by expanding the regions where the watermark pattern is embedded or amplifying its magnitude, ensuring that the watermark remains detectable. Beyond the encoding process, these systems also train the watermark decoder to enhance extraction effectiveness, effectively increasing the model's attention to watermark signals.

We conduct a feasibility study to explore: \emph{If the strengthened watermark results in leakage that can be captured from images using a feature extraction network?} We embed watermarks in multiple images with the same robust watermarking algorithm and then input these watermarked images into a feature extraction network.

% figure
\begin{figure}[!t]
    \centering
    % Custom font size
    \includegraphics[width=\linewidth]{pics/feasiblity.png}   
    \vspace{-6mm}
    \caption{Demonstration of our feasibility study.}
    \label{fig:feasibility}
    \vspace{-3mm}
\end{figure}

As shown in Figure ~\ref{fig:feasibility}, we found that:
\begin{itemize}
    \item The multi-channel features obtained after feature extraction can capture patterns not easily noticeable by the human eye.
    \item These patterns are similar across different images.
    \item Not all features contain such leakage information.
\end{itemize}
% Based on the above experimental observations, we \underline{\textbf{D}}elve into the \underline{\textbf{A}}spect of the \underline{\textbf{PA}}radox \underline{\textbf{O}}f Robust Watermarks and propose the \textbf{DAPAO} attack.

The results shed light on learning watermark characteristics from distinguished patterns probably related to the watermark.

\begin{figure}[!t]
    \centering
    % Custom font size
    \includegraphics[width=\linewidth]{pics/overview.png} 
    
    % \vspace{-3mm}
    \caption{An overview of our attack.} 
    \label{fig:method-overview}
    \vspace{-3mm}
\end{figure}

\subsection{Robustness and Invisibility Trade-off}\label{sec:Method_Theory}
% \begin{definition}
% \label{def:inj}
% A encoder $\mathcal{E}:\mathcal{I} \times W \to Y$ is injective if for any $x,y\in X$ different, $f(x)\ne f(y)$.
% \end{definition}
As mentioned earlier Sec.~\ref{sec:background}, a complete watermarking framework can be divided into three components: encoder $\mathcal{E}$, decoder $\mathcal{D}$, and distortion layer $\mathcal{T}$. The decoder takes only a single watermarked image $I_{wm}$ as input. To achieve correct verification, the decoder must implicitly disentangle the image content from the embedded watermark information and correctly associate them to extract the watermark successfully.



% the decoder must implicitly decompose the watermarked image into image information and watermark information, matching the two to successfully extract the watermark information.

\begin{definition}
An image and watermark information: $I$, $wm \ \subset \{0,1\}^k$, the encoder is:
$$\mathcal{E}(I, wm)=I+\epsilon \cdot \underbrace{\phi(I,wm)}_W$$
the decoder is:
$$\mathcal{D}(I_{wm}) \to \underbrace{(\hat{I}, \hat{W})}_{{match}} \to \hat{wm}$$



$\epsilon$ is the embedding strength.The feature space of the image $\mathcal{P} = \{p_1, p_2,...,p_n\}$ consists of two subspaces for embedding information: 
$$\mathcal{P} = \mathcal{P}_r \bigoplus \mathcal{P}_c$$

Due to joint training, the encoder exhibits a similar implicit decomposition behavior, projecting the input image $I$ into two feature spaces, named as $P_r$ and $P_c$. The former is a more suitable embedding space for information hiding, while the latter is not. 

The encoder performs this mapping $\mathcal{E}(I,wm) \to I_{wm}$ by:
$$\phi(I,wm) = \mathop{\min}_{p\in \mathcal{P}_r}||wm - \mathcal{D}(\mathcal{E}(p,wm))||^2+\lambda||\mathcal{E}(p,wm)||$$

However, as robustness requirements are introduced and continuously strengthened, the encoder must encode more information to ensure the watermark’s resistance to attacks. When the $P_r$   space is fully utilized, the encoder is forced to use $P_c$ for watermark embedding, polluting the $P_c$ space.

\end{definition}

\begin{definition}
An intuitive definition of embeddable threshold is:
\begin{gather*}
C(I) = \sup_{W \in \mathcal{P}_r}{\frac{||W||_2}{||I||_2}} \\\\
s.t. PNSR(I, I+W) \ge TV
\end{gather*}
$TV$ represents the lower bound of the visual quality.
\end{definition}

\begin{proposition}
When the robustness requirement exceeds $C(I)$, a decline in visual quality is inevitable.
\end{proposition}

\begin{proof}
Let the distortion layer $\mathcal{T}$ introduce noise $\eta \sim \mathcal{T}$, with the requirement that
$$||wm-\mathcal{D}(I_{wm} + \eta)|| \le \mathcal{B}$$
$\mathcal{B}$ is the bit error rate. Considering the channel capacity as:
$$R=\frac{1}{2}\log(1+\frac{\epsilon^2||W||^2}{\delta_{\eta}^2})$$
% \vspace{-3mm}
To achieve $R\ge H(wm)$, the following conditions must be met:
$$
\epsilon||W|| \le \sqrt{(2^{2H(wm)}-1)\delta_{\eta^2}}
$$
$H(wm)$ represents the entropy of $wm$. 

When $\sqrt{(2^{2H(wm)}-1)\delta_{\eta^2}} > C(I)||I||_2$, the system cannot simultaneously satisfy both, and it is necessary to increase $C(I)$, introducing visual artifacts into the image. Detailed proof is provided in Appendix~\ref{sec:Appendix_Proofs}.
\end{proof}
% \vspace{-3mm}
The artifacts introduced by sacrificing invisibility contain watermark information, creating a security vulnerability where watermark information leakage occurs.

%\hl{lack of proof of artifacts contain watermark information }
% This, however, compromises visual quality, leading to more detectable visual artifacts. Moreover, these artifacts also contain watermark information, creating a security vulnerability where watermark information leakage occurs.

\subsection{Detection Evasion}\label{sec:Method_Evasion Attack}
Our method is illustrated in Figure~\ref{fig:method-overview}, Suppose we have an image $I_{wm}$, embedded with an unknown watermark $wm$. This image is fed into a feature extraction module $\mathcal{F}(\cdot)$, resulting in multi-channel features $\mathcal{F}(I_{wm})$. To automate the selection of features that capture potential information leakage, we perform clustering on the multi-channel features. Among the resulting clusters, we identify the two clusters with the smallest number of samples and extract their corresponding feature channel positions $\mathcal{W}$.

To achieve the goal of an evasion attack, we need to disrupt the leaked watermark information captured from $I_{wm}$.We formulate this process as an optimization problem: finding a perturbation $\delta$ that disrupts the leaked information while preserving the visual quality of the image. The formulation is as follows:
\begin{equation}
\label{eq:1}
\begin{split}
    \mathop{\min}_{\delta}-\mathcal{L}(\mathcal{W} \cdot \mathcal{F}(I_{wm}), \mathcal{W}\cdot \mathcal{F}(I_{wm} + \delta)) \\
    \mathrm{ s.t.} ||\delta||_{\infty} < \epsilon
\end{split}
\end{equation}

where $\mathcal{L}(\cdot,\cdot)$ is the loss function that measures the distance between two features, and $\epsilon$ is a perturbation budget.

We use Projected Gradient Descent (PGD)~\cite{PGD} to solve the optimization problem in Eq~\ref{eq:1}. Finally, we complete the attack through $I_{wm} + \delta$.

Our detailed algorithm is shown as 
 Algorithm~\ref{alg:evasion algo}.
 %in Appendix~\ref{sec:Appendix_Implementation Details}.

 % Similar to Sec~\ref{sec:Method_Evasion Attack}, as shown in Figure ~\ref{fig:method-overview},

\subsection{Forgery Attack}
As shown in Figure~\ref{fig:method-overview}, we first use the feature extraction module and clustering algorithm to extract features containing leaked watermark information, from $I_{wm}$. To achieve the goal of spoofing, we still need to extract the leaked information. Therefore, this process can be formulated as the following optimization problem:
\begin{equation}
\label{eq:2}
\begin{split}
     \mathop{\min}_{\delta}-\mathcal{L}(\mathcal{W} \cdot \mathcal{F}(I_{wm}), \mathcal{W}\cdot \mathcal{F}(I_{wm} + \delta)) \\
    \mathrm{ s.t.} ||\delta||_{\infty} < \epsilon
\end{split}
\end{equation}
\vspace{-4mm}

where $\epsilon$ is a perturbation budget, and 
 this process is identical to the above evasion attack, referred to as Stage \uppercase\expandafter{\romannumeral1}.
However, the learned $\delta$ alone cannot fulfill the forgery purpose for \emph{semantic watermarking}. Based on the theory discussed earlier (See Sec.~\ref{sec:Method_Theory}), we need to consider the coupling effect between the semantics and watermark. After the optimization in Eq~\ref{eq:2} is completed, an additional optimization term should be included to further find another perturbation, $\delta_s$, which can be described as:
\begin{equation}
\label{eq:3}
\begin{split}
     \mathop{\min}_{\delta}\mathcal{L}((1-\mathcal{W}) \cdot \mathcal{F}(I_{wm}+\delta), (1-\mathcal{W})\cdot \mathcal{F}(I' + \delta_s)) \\
    \mathrm{ s.t.} ||\delta_s||_{\infty} < \epsilon
\end{split}
\end{equation}
This process is referred to as Stage \uppercase\expandafter{\romannumeral2}.
We use Projected Gradient Descent (PGD)~\cite{PGD} to solve the optimization problem in Eq~\ref{eq:2} and Eq~\ref{eq:3}.
Finally, we complete the attack through $\{I' - \delta\}$ or $\{I' - \delta + \delta_s \}$.

Our detailed algorithm is shown as Algorithm~\ref{alg:spoof algo}
%in Appendix~\ref{sec:Appendix_Implementation Details}.
\section{Experiments and Results}
\subsection{Experiment Settings}
% \begin{table*}[h]
%     \centering
%     \begin{tabular}{cl|ccccc|ccccc}
%      \multirow{3}{*}{\textbf{LLM}}  & \multirow{3}{*}{\textbf{Method}} &  \multicolumn{5}{c|}{\textbf{CCNews}} & \multicolumn{5}{c}{\textbf{Wikipedia}} \\ \cmidrule(lr){3-7}  \cmidrule(lr){8-12}
%       &  & PPL & Loss & Ref & min-k & \multicolumn{1}{c|}{zlib} & PPL & Loss & Ref & min-k & zlib \\ \midrule
%       \multirow{4}{*}{GPT2} & \textit{Base} & \textit{29.442} & \textit{0.505} & \textit{0.498} & \textit{0.520} & \textit{0.500} & \textit{34.429} & \textit{0.473} & \textit{0.513} & \textit{0.446} & \textit{0.497} \\ 
%       \multirow{4}{*}{124M} & FT & \textbf{21.861} & 0.607 & 0.855 & 0.549 & 0.569 & \textbf{12.729} & 0.577 & 0.967 & 0.489 & 0.544 \\
%       & Goldfish & 21.902 & 0.608 & 0.855 & 0.547 & 0.570 & 12.853 & 0.565 & 0.954 & 0.486 & 0.537 \\
%       & DPSGD & 26.022 & 0.507 & 0.513 & \textbf{0.521} & 0.502 & 18.523 & 0.463 & 0.536 & \textbf{0.448} & 0.491 \\
%       & \methodname & 23.733 & \textbf{0.502} & \textbf{0.495} & 0.529 & \textbf{0.499} & 13.628 & \textbf{0.454} & \textbf{0.463} & 0.470 & \textbf{0.485} \\ \midrule
      
%       \multirow{4}{*}{Pythia} & \textit{Base} & \textit{13.973} & \textit{0.507} & \textit{0.512} & \textit{0.528} & \textit{0.501} & \textit{10.287} & \textit{0.466} & \textit{0.503} & \textit{0.464} & \textit{0.489}\\ 
%       \multirow{4}{*}{1.4B} & FT & 11.922 & 0.602 & 0.857 & 0.541 & 0.574 & \textbf{6.439} & 0.578 & 0.985 & 0.484 & 0.557 \\
%       & Goldfish & \textbf{11.903} & 0.609 & 0.862 & 0.543 & 0.579 & 6.465 & 0.564 & 0.981 & 0.482 & 0.546 \\
%       & DPSGD & 13.286 & 0.512 & 0.531 & 0.528 & 0.503 & 7.751 & 0.469 & 0.524 & \textbf{0.462} & 0.488 \\
%       & \methodname & 12.670 & \textbf{0.501} & \textbf{0.460} & \textbf{0.524} & \textbf{0.499} & 6.553 & \textbf{0.468} & \textbf{0.485} & 0.472 & \textbf{0.485} \\ \midrule
      
%       \multirow{4}{*}{Llama-2} & \textit{Base} & \textit{9.364} & \textit{0.505} & \textit{0.495} & \textit{0.516} & \textit{0.503} & \textit{7.014} & \textit{0.458} & \textit{0.491} & \textit{0.476} & \textit{0.488} \\ 
%       \multirow{4}{*}{7B} & FT & \textbf{6.261} & 0.559 & 0.798 & 0.536 & 0.548 & \textbf{3.830} & 0.524 & 0.936 & 0.494 & 0.530 \\
%       & Goldfish & 6.280 & 0.552 & 0.780 & 0.533 & 0.541 & 3.839 & 0.518 & 0.929 & 0.492 & 0.525 \\
%       & DPSGD & 6.777 & 0.509 & 0.538 & 0.523 & 0.504 & 4.490 & 0.466 & 0.516 & \textbf{0.470} & 0.487 \\
%       & \methodname & 6.395 & \textbf{0.507} & \textbf{0.482} & \textbf{0.518} & \textbf{0.500} & 4.006 & \textbf{0.458} & \textbf{0.440} & 0.473 & \textbf{0.480} \\ 
%     \end{tabular}
%     \caption{Caption}
%     \label{tab:main_result}
% \end{table*}


\begin{table*}[h]
  \centering
  \resizebox{0.9\textwidth}{!}{\begin{tabular}{cl|ccccc|ccccc}
  \toprule[1pt]
   \multirow{3}{*}{\textbf{LLM}}  & \multirow{3}{*}{\textbf{Method}} &  \multicolumn{5}{c|}{\textbf{Wikipedia}} & \multicolumn{5}{c}{\textbf{CC-news}} \\ \cmidrule(lr){3-7}  \cmidrule(lr){8-12}
    &  & PPL & Loss & Ref & Min-k & \multicolumn{1}{c|}{Zlib} & PPL & Loss & Ref & Min-k & Zlib \\ \midrule
    \multirow{4}{*}{GPT2} & \textit{Base} & \textit{34.429} & \textit{0.473} & \textit{0.513} & \textit{0.446} & \textit{0.497} & \textit{29.442} & \textit{0.505} & \textit{0.498} & \textit{0.520} & \textit{0.500} \\ 
    \multirow{4}{*}{124M} & FT & \textbf{12.729} & 0.577 & 0.967 & 0.489 & 0.544 & \textbf{21.861} & 0.607 & 0.855 & 0.549 & 0.569 \\
    & Goldfish & 12.853 & 0.565 & 0.954 & 0.486 & 0.537 & 21.902 & 0.608 & 0.855 & 0.547 & 0.570 \\
    & DPSGD & 18.523 & 0.463 & 0.536 & \textbf{0.448} & 0.491 & 26.022 & 0.507 & 0.513 & \textbf{0.521} & 0.502 \\
    & \methodname & 13.628 & \textbf{0.454} & \textbf{0.463} & 0.470 & \textbf{0.485} & 23.733 & \textbf{0.502} & \textbf{0.495} & 0.529 & \textbf{0.499} \\ \midrule
    
    \multirow{4}{*}{Pythia} & \textit{Base} & \textit{10.287} & \textit{0.466} & \textit{0.503} & \textit{0.464} & \textit{0.489} & \textit{13.973} & \textit{0.507} & \textit{0.512} & \textit{0.528} & \textit{0.501}\\ 
    \multirow{4}{*}{1.4B} & FT & \textbf{6.439} & 0.578 & 0.985 & 0.484 & 0.557 & 11.922 & 0.602 & 0.857 & 0.541 & 0.574 \\
    & Goldfish & 6.465 & 0.564 & 0.981 & 0.482 & 0.546 & \textbf{11.903} & 0.609 & 0.862 & 0.543 & 0.579 \\
    & DPSGD & 7.751 & 0.469 & 0.524 & \textbf{0.462} & 0.488 & 13.286 & 0.512 & 0.531 & 0.528 & 0.503 \\
    & \methodname & 6.553 & \textbf{0.468} & \textbf{0.485} & 0.472 & \textbf{0.485} & 12.670 & \textbf{0.501} & \textbf{0.460} & \textbf{0.524} & \textbf{0.499} \\ \midrule
    
    \multirow{4}{*}{Llama-2} & \textit{Base} & \textit{7.014} & \textit{0.458} & \textit{0.491} & \textit{0.476} & \textit{0.488} & \textit{9.364} & \textit{0.505} & \textit{0.495} & \textit{0.516} & \textit{0.503} \\ 
    \multirow{4}{*}{7B} & FT & \textbf{3.830} & 0.524 & 0.936 & 0.494 & 0.530 & \textbf{6.261} & 0.559 & 0.798 & 0.536 & 0.548 \\
    & Goldfish & 3.839 & 0.518 & 0.929 & 0.492 & 0.525 & 6.280 & 0.552 & 0.780 & 0.533 & 0.541 \\
    & DPSGD & 4.490 & 0.466 & 0.516 & \textbf{0.470} & 0.487 & 6.777 & 0.509 & 0.538 & 0.523 & 0.504 \\
    & \methodname & 4.006 & \textbf{0.458} & \textbf{0.440} & 0.473 & \textbf{0.480} & 6.395 & \textbf{0.507} & \textbf{0.482} & \textbf{0.518} & \textbf{0.500} \\
    \bottomrule[1pt]
  \end{tabular}}
  \caption{Overall Evaluation: Perplexity (PPL) and AUC scores of the MIAs with different signals (Loss/Ref/Min-k/Zlib). For all metrics, the lower the value, the better the result. \textit{Base} in the method column indicates the pretrained LLMs without fine-tuning, thus it indicates lower bound for both utility and privacy risk.}
  \label{tab:main_result}
\end{table*}

% \begin{table*}[h]
%   \centering
%   \begin{tabular}{cl|ccccc|ccccc}
%   \multirow{3}{*}{\textbf{LLM}} & \multirow{3}{*}{\textbf{Method}} & \multicolumn{5}{c|}{\textbf{Wikipedia}} & \multicolumn{5}{c}{\textbf{CCNews}} \\
%   \cmidrule(lr){3-7} \cmidrule(lr){8-12}
%   & & PPL & Loss & Ref & min-k & \multicolumn{1}{c|}{zlib} & PPL & Loss & Ref & min-k & zlib \\
%   \midrule
%   \multirow{4}{*}{GPT2} & \textit{Base} & \textit{34.429} & \textit{0.473} & \textit{0.513} & \textit{0.446} & \textit{0.497} & \textit{29.442} & \textit{0.505} & \textit{0.498} & \textit{0.520} & \textit{0.500} \\
%   \multirow{4}{*}{124M} & FT & \textbf{12.729} & 0.577 & 0.967 & 0.489 & 0.544 & \textbf{21.861} & 0.607 & 0.855 & 0.549 & 0.569 \\
%   & Goldfish & 12.853 & 0.565 & 0.954 & 0.486 & 0.537 & 21.902 & 0.608 & 0.855 & 0.547 & 0.570 \\
%   & DPSGD & 18.523 & 0.463 & 0.536 & \textbf{0.448} & 0.491 & 26.022 & 0.507 & 0.513 & \textbf{0.521} & 0.502 \\
%   & \methodname & 13.628 & \textbf{0.454} & \textbf{0.463} & 0.470 & \textbf{0.485} & 23.733 & \textbf{0.502} & \textbf{0.495} & 0.529 & \textbf{0.499} \\
%   \midrule
%   \multirow{4}{*}{Pythia} & \textit{Base} & \textit{10.287} & \textit{0.466} & \textit{0.503} & \textit{0.464} & \textit{0.489} & \textit{13.973} & \textit{0.507} & \textit{0.512} & \textit{0.528} & \textit{0.501} \\
%   \multirow{4}{*}{1.4B} & FT & \textbf{6.439} & 0.578 & 0.985 & 0.484 & 0.557 & 11.922 & 0.602 & 0.857 & 0.541 & 0.574 \\
%   & Goldfish & 6.465 & 0.564 & 0.981 & 0.482 & 0.546 & \textbf{11.903} & 0.609 & 0.862 & 0.543 & 0.579 \\
%   & DPSGD & 7.751 & 0.469 & 0.524 & \textbf{0.462} & 0.488 & 13.286 & 0.512 & 0.531 & 0.528 & 0.503 \\
%   & \methodname & 6.553 & \textbf{0.468} & \textbf{0.485} & 0.472 & \textbf{0.485} & 12.670 & \textbf{0.501} & \textbf{0.460} & \textbf{0.524} & \textbf{0.499} \\
%   \midrule
%   \multirow{4}{*}{Llama-2} & \textit{Base} & \textit{7.014} & \textit{0.458} & \textit{0.491} & \textit{0.476} & \textit{0.488} & \textit{9.364} & \textit{0.505} & \textit{0.495} & \textit{0.516} & \textit{0.503} \\
%   \multirow{4}{*}{7B} & FT & \textbf{3.830} & 0.524 & 0.936 & 0.494 & 0.530 & \textbf{6.261} & 0.559 & 0.798 & 0.536 & 0.548 \\
%   & Goldfish & 3.839 & 0.518 & 0.929 & 0.492 & 0.525 & 6.280 & 0.552 & 0.780 & 0.533 & 0.541 \\
%   & DPSGD & 4.490 & 0.466 & 0.516 & \textbf{0.470} & 0.487 & 6.777 & 0.509 & 0.538 & 0.523 & 0.504 \\
%   & \methodname & 4.006 & \textbf{0.458} & \textbf{0.440} & 0.473 & \textbf{0.480} & 6.395 & \textbf{0.507} & \textbf{0.482} & \textbf{0.518} & \textbf{0.500} \\
%   \end{tabular}
%   \caption{Caption}
%   \label{tab:main_result}
%   \end{table*}
  

\textbf{Datasets}. We conduct experiments on two datasets: CC-news\footnote{\href{https://huggingface.co/datasets/vblagoje/cc_news}{Huggingface: vblagoje/cc\_news}} and Wikipedia\footnote{\href{https://huggingface.co/datasets/legacy-datasets/wikipedia}{Huggingface: legacy-datasets/Wikipedia}}. CC-news is a large collection of news articles which includes diverse topics and reflects real-world temporal events. Meanwhile, Wikipedia covers general knowledge across a wide range of disciplines, such as history, science, arts, and popular culture.\\
\textbf{LLMs}: We experiment on three models including \gpt~(124M)~\cite{gpt2radford}, \pythia~(1.4B)~\cite{pythia}, and \llama~(7B)~\cite{llama2touvron2023}. This selection of models ensures a wide range of model sizes from small to large that allows us to analyze scaling effects and generalizability across different capacities. \\
\textbf{Evaluation Metrics}. For evaluating language modeling performance, we measure perplexity (PPL), as it reflects the overall effectiveness of the model and is often correlated with improvements in other downstream tasks~\cite{kaplan2020scalinglaws, lmsfewshot}. For defense effectiveness, we consider the attack area under the curve (AUC) value and True Positive Rate (TPR) at low False Positive Rate (FPR). In total, we perform 4 MIAs with different MIA signals. Given the sample $x$, the MIA signal function $f$ is formulated as follows: \\
$\bullet$ Loss~\cite{8429311} utilizes the negative cross entropy loss as the MIA signal. 
    \[f_\text{Loss}(x) = \mathcal{L}_\text{CE}(\theta; x) \]
$\bullet$ Ref-Loss~\cite{Carlini2020ExtractingTD} considers the loss differences between the target model and the attack reference model. To enhance the generality, our experiments ensure there is no data contamination between the training data of the target, reference, and attack models.
    \[f_\text{Ref}(x) = \mathcal{L}_\text{CE}(\theta; x) - \mathcal{L}_\text{CE}(\theta_\text{attack}; x) \]
$\bullet$ Min-K~\cite{shi2024detecting} leverages top K tokens that have the lowest loss values.
    \[f_\text{min-K}(x) = \frac{1}{|\text{min-K(x)}|} \sum_{t_i \in \text{min-K(x)}} -\log(P(t_i|t_{<i};\theta) \]
$\bullet$ Zlib~\cite{Carlini2020ExtractingTD} calibrates the loss signal with the zlib compression size.
    \[ f_\text{zlib}(x) = \mathcal{L}_\text{CE}(\theta; x) / \text{zlib}(x) \]

\noindent \textbf{Baselines}. We present the results of four baselines. \textit{Base} refers to the pretrained LLM without fine tuning. \textit{FT} represents the standard causal language modeling without protection. \textit{Goldfish}~\cite{hans2024be} implements a masking mechanism. \textit{DPSGD}~\cite{abadi2016deep, yu2022differentially} applies gradient clipping and injects noise to achieve  sample-level differential privacy.

\noindent \textbf{Implementation}. We conduct full fine-tuning for \gpt and \pythia. For computing efficiency, \llama fine-tuning is implemented using Low-Rank Adaptation (LoRA)~\cite{hu2022lora} which leads to \textasciitilde4.2M trainable parameters. Additionally, we use subsets of 3K samples to fine-tune the LLMs. We present other implementation details in Appendix~\ref{sec:app-implementation}.

\subsection{Overall Evaluation}
Table~\ref{tab:main_result} provides the overall evaluation compared to several baselines across large language model architectures and datasets. Among these two datasets, CCNews is more challenging, which  leads to higher perplexity  for all LLMs and fine-tuning methods. Additionally, the reference-model-based attack performs the best and demonstrates high privacy risks with attack AUC on the conventional fine-tuned models at 0.95 and 0.85 for Wikipedia and CCNews, respectively. Goldfish achieves similar PPL to the conventional FT method but fails to defend against MIAs. This aligns with the reported results by \citet{hans2024be} that Goldfish resists exact match attacks but only marginally affects MIAs. DPSGD provides a very strong protection in all settings (AUC < 0.55) but with a significant PPL tradeoff. Our proposed \methodname guarantees a robust protection, even slightly better than DPSGD, but with a notably smaller tradeoff on language modeling performance. For example, on the Wikipedia dataset, \methodname delivers perplexity reduction by 15\% to 27\%. Moreover, Table~\ref{tab:tpr} (Appendix~\ref{sec:app-add-res}) provides the TPR at 1\% FPR. Both DPSGD and \methodname successfully reduce the TPR to $\sim$0.02 for all LLMs and datasets. \textit{Overall, across multiple LLM architectures and datasets, \methodname consistently offers ideal privacy protection with  little trade-off in language modeling performance.}

\noindent \textbf{Privacy-Utility Trade-off.}
To investigate the privacy-utility trade-off of the methods, we vary the hyper-parameters of the fine-tuning methods. Particularly, for DPSGD, we adjust the privacy budget $\epsilon$ from (8, 1e-5)-DP to (100, 1e-5)-DP. We modify the masking percentage of Goldfish from 25\% to 50\%. Additionally, we vary the loss weight $\alpha$ from 0.2 to 0.8 for \methodname. Figure~\ref{fig:priv-ult-tradeoff} depicts the privacy-utility trade-off for GPT2 on the CCNews dataset. Goldfish, with very large masking rate (50\%), can slightly reduce the risk of the reference attack but can increase the risks of other attacks. By varying the weight $\alpha$, \methodname offers an adjustable trade-off between privacy protection and language modeling performance. \methodname largely dominates DPSGD and improves the language modeling performance by around 10\% with the ideal privacy protection against MIAs.

\begin{figure}[h]
    \centering
    \includegraphics[width=\linewidth]{figs/privacy-ultility-tradeoff.pdf}
    \caption{Privacy-utility trade-off of the methods while varying hyper-parameters. The Goldfish masking rate is set to 25\%, 33\%, and 50\%. The privacy budget $\epsilon$ of DPSGD is evaluated at 8, 16, 50, and 100. The weight $\alpha$ of \methodname is configured at 0.2, 0.5, and 0.8.}
    \label{fig:priv-ult-tradeoff}
\end{figure}


\subsection{Ablation Study}
\textbf{\methodname without reference models.} To study the impact of the reference model, we adapt \methodname to a non-reference version which directly uses the loss of the current training model (i.e., $s(t_i) = \mathcal{L}_{CE}(\theta; t_i)$) to select the learning and unlearning tokens. This means the unlearning tokens are the tokens that have smallest loss values. Figure~\ref{fig:ppl-auc-noref} presents the training loss and testing perplexity. There is an inconsistent trend of the training loss and testing perplexity. Although the training loss decreases overtime, the test perplexity increases. This result indicates that identifying appropriate unlearning tokens  without a reference model is challenging and conducting unlearning on an incorrect set hurts the language modeling performance.

\begin{figure}[htp]
    \centering
    \includegraphics[width=0.35\textwidth]{figs/train_loss_ppl_noref.pdf}
    \caption{Training Loss and Test Perplexity of \methodname without a reference model.
    % (\lrx{If time permits, it would be better to compare with our training curve here)}
    }
    \label{fig:ppl-auc-noref}
\end{figure}

\noindent \textbf{\methodname with out-of-domain reference models.} To examine the influence of the distribution gap in the reference model, we replace the in-domain trained reference model with the original pretrained base model. 
Figure~\ref{fig:ppl-auc-base-woasc} depicts the language modeling performance and privacy risks in this study. \methodname with an out-of-domain reference model can reduce the privacy risks but yield a significant gap in language modeling performance compared to \methodname using an in-domain reference model.

\noindent \textbf{\methodname without Unlearning.} To study the effects of unlearning tokens, we implement \methodname which use the first term of the loss only ({$\mathcal{L}_{\theta} = \mathcal{L}_{CE}(\theta; \mathcal{T}_h)$}). Figure~\ref{fig:ppl-auc-base-woasc} provides the perplexity and MIA AUC scores in this setting. Generally, without gradient ascent, \methodname can marginally reduce membership inference risks while slightly improving the language modeling performance. The token selection serves as a regularizer that helps to improve the language modeling performance. Additionally, tokens that are learned well in previous epochs may not be selected in the next epochs. This slightly helps to not amplify the memorization on these tokens over epochs.

\begin{figure}[htp]
    \centering
    \includegraphics[width=0.28\textwidth]{figs/auc_vs_ppl_base_woasc.pdf}
    \caption{Privacy-utility trade-off of \methodname with different settings: in-domain reference model, out-domain reference model, and without unlearning}
    \label{fig:ppl-auc-base-woasc}
\end{figure}


\subsection{Training Dynamics}
\textbf{Memorization and Generalization Dynamics}. Figure~\ref{fig:training-dynamics} (left) illustrates the training dynamics of conventional fine tuning and \methodname, while Figure~\ref{fig:training-dynamics} (middle) depicts the membership inference risks. Generally, the gap between training and testing loss of conventional fine-tuning steadily increases overtime, leading to model overfitting and high privacy risks. In contrast, \methodname maintains a stable equilibrium where the gap remains more than 10 times smaller. This equilibrium arises from the dual-purpose loss, which balances learning on hard tokens while actively unlearning memorized tokens. By preventing excessive memorization, \methodname mitigates membership inference risks and enhances generalization.

\begin{figure*}[htp]
    \centering
    \includegraphics[width=0.29\linewidth]{figs/loss_vs_steps_ft_duolearn.pdf}
    \includegraphics[width=0.29\linewidth]{figs/auc_vs_steps_ft_duolearn.pdf}
    \includegraphics[width=0.316\linewidth]{figs/cosine.pdf}
    \caption{Training dynamics of \methodname and the conventional fine-tuning approach. The left and middle figures provide the training-testing gap and membership inference risks, respectively. The testing~$\mathcal{L}_{CE}$ of FT and training~$\mathcal{L}_{CE}$ of \methodname are significantly overlapping, we provide the breakdown in Figure~\ref{fig:add-overlap-breakdown} in Appendix~\ref{sec:app-add-res}. The right figure depicts the cosine similarity of the learning and unlearning gradients of \methodname. Cosine similarity of 1 means entire alignment, 0 indicates orthogonality, and -1 presents full conflict.}
    \label{fig:training-dynamics}
\end{figure*}

\noindent \textbf{Gradient Conflicts}. To study the conflict between the learning and unlearning objectives in our dual-purpose loss function, we compute the gradient for each objective separately. We then calculate the cosine similarity of these two gradients. Figure~\ref{fig:training-dynamics} (right) provides the cosine similarity between two gradients over time. During training, the cosine similarity typically ranges from -0.15 to 0.15. This indicates a mix of mild conflicts and near-orthogonal updates. On average, it decreases from 0.05 to -0.1. This trend reflects increasing gradient misalignment. Early in training, the model may not have strongly learned or memorized specific tokens, so the conflicts are weaker. Overtime, as the model learns more and memorization grows, the divergence between hard and memorized tokens increases, making the gradients less aligned. This gradient conflict is the root of the small degradation of language modeling performance of \methodname compared to the conventional fine tuning approach.

\noindent \textbf{Token Selection Dynamics}. Figure~\ref{fig:token-selection} illustrates the token selection dynamics of \methodname during training. The figure shows that the token selection process is dynamic and changes over epochs. In particular, some tokens are selected as an unlearning from the beginning to the end of the training. This indicates that a token, even without being selected as a learning token initially, can be learned and memorized through the connections with other tokens. This also confirms that simple masking as in Goldfish is not sufficient to protect against MIAs. Additionally, there are a significant number of tokens that are selected for learning in the early epochs but unlearned in the later epochs. This indicates that the model learned tokens and then memorized them over epochs, and the during-training unlearning process is essential to mitigate the memorization risks.

\begin{figure}[htp]
    \centering
    \includegraphics[width=0.7\linewidth]{figs/token-selection-dynamics.pdf}
    \caption{Token Selection Dynamics of \methodname}
    \label{fig:token-selection}
    \vspace{-4mm}
\end{figure}

\subsection{Privacy Backdoor}
To study the worst case of privacy attacks and defense effectiveness under the state-of-the-art MIA, we perform a privacy backdoor -- Precurious~\cite{precurious}. In this setup, the target model undergoes continual fine-tuning from a warm-up model. The attacker then applies a reference-based MIA that leverages the warm-up model as the attack's reference. Table~\ref{tab:backdoor} shows the language modeling and MIA performance on CCNews with GPT-2. Precurious increases the MIA AUC score by 5\%. Goldfish achieves the lowest PPL, aligning with~\citet{hans2024be}, where the Goldfish masking mechanism acts as a regularizer that potentially enhances generalization. Both DPSGD and \methodname provide strong privacy protection, with \methodname offering slightly better defense while maintaining lower perplexity than DPSGD.

% \begin{table}[h]
%     \centering
%     \begin{tabular}{c|cc|cc}
%        \multirow{2}{*}{\textbf{Method}}  & \multicolumn{2}{c}{\textbf{CCNews}} & \multicolumn{2}{c}{\textbf{Wikipedia}} \\ 
%        & \textbf{PPL} & \textbf{AUC} & \textbf{PPL} & \textbf{AUC} \\ \hline
%        \textbf{FT}        & 21.593 & 0.911 \\
%        \textbf{Goldfish}  & \textbf{21.074} & 0.886 \\
%        \textbf{DPSGD}     & 23.279 & 0.533 \\
%        \textbf{DuoLearn}  & 22.296 & \textbf{0.499} \\
%     \end{tabular}
%     \caption{Caption}
%     \label{tab:my_label}
% \end{table}

\begin{table}[h]
    \centering
    \resizebox{\columnwidth}{!}{\begin{tabular}{c|cccccc}
        \textbf{Metric} & \textbf{WU} & \textbf{FT} & \textbf{GF} & \textbf{DP} & \textbf{DuoL} \\ \hline
        \textbf{PPL} & \textit{23.318} & 21.593 & \textbf{21.074} & 23.279 & 22.296  \\
        \textbf{AUC} & \textit{0.500} & 0.911 & 0.886 & 0.533 & \textbf{0.499} \\
    \end{tabular}}
    \caption{Experimental results of privacy backdoor for GPT2 on the CC-news dataset. WU stands for the warm-up model leveraged by Precurious. GF, DP, and DuoL are abbreviations of Goldfish, DPSGD, and \methodname}
    \label{tab:backdoor}
\end{table}

% \subsubsection{Hyperparameter Study}

% \subsubsection{Full fine-tuning versus Parameter efficent fine tuning}

% \subsubsection{Extending to Vision Language Models}



\input{src/7-realword-exp}
\input{src/6-attackmode}
\input{src/8-countermeasures}
\section{Related Work}

\paragraph{Confidence signals for LLMs.} 
There is a long line of work on deriving confidence measures from LLMs. Popular approaches use
% Popular methods to derive calibrated confidence from LLMs include 
the agreement across multiple samples \cite{kuhn2023semantic, manakul2023selfcheckgpt, tian2023fine,lyu2024calibrating}, the model's internal representations \cite{azaria2023internal, burns2022discovering} or directly prompting the model to verbalize its confidence \cite{tian2023just, kadavath2022language}.
All papers in this line of work focused on fact-seeking tasks, 
so confidence is typically derived based on the final answer alone. To the best of our knowledge, our work is the first to apply these approaches to scoring the entire reasoning path.

\paragraph{Reasoning verification.}
While learned verifiers have been demonstrated to significantly improve performance on math word problems \cite{cobbe2021training, lightman2023let, li2022making}, the ability of LLMs to perform \emph{self}-verification and \emph{self}-correction is still heavily contested, with some works providing positive evidence for such capabilities \cite{weng2022large, gero2023self, madaan2024self, liu2024large, li2024confidence} and others arguing that the gains can mostly be attributed to clever prompt design, unfair baselines, data contamination and using overly simple tasks \cite{tyen2023llms, valmeekam2023can, hong2023closer, huang2023large, stechly2024self, zhang2024small}. This work contributes to this ongoing discussion by presenting multiple lines of evidence supporting LLM self-verification. In particular, we demonstrate clear benefits from a simple confidence-based self-verification approach. 


\paragraph{Improving self-consistency's efficiency. }

Numerous attempts \cite{chen-etal-2024-self-para} have been made to reduce SC computational overhead while maintaining quality. However, none have matched the widespread adoption of self-consistency. This can be largely attributed to several limitations: (1) a trade-off where throughput is reduced while latency increases, for example by sampling chains sequentially until reaching a certain condition \cite{li2024escape} or running expensive LLM calls instead of the cheap majority voting \cite{yoran2023answering}, (2) the need for manual feature crafting and tuning tailored to each dataset \cite{wan2024dynamic}, (3) promising results on specialized setups \cite{wang2024soft} which did not generalize to standard benchmarks (Table \ref{table:max-ablation}), and (4) as highlighted by \citet{huang2023large}, many of the more sophisticated methods that appear promising actually don't outperform self-consistency when evaluated with a thorough analysis of inference costs. Our approach is different in that CISC adds minimal complexity to self-consistency, and improves throughput without compromising latency.

\paragraph{Self-consistency with confidence.}
Related approaches to CISC's confidence-weighted majority vote were previously explored in both the original self-consistency paper \citet{wang2022self}, that considered a weighted majority using Sequence Probability (\S\ref{sec:metrics}), and in \citet{miao2023selfcheck}, that concluded that verbally \nl{asking the LLM to check its own reasoning is largely ineffective} for improving self-consistency. In both cases, these failures are attributed to the confidence scores being too similar to one another. Our work shows that despite this, the scores contain a useful signal (reflected in the WQD scores; Table \ref{tab:confidence-methods}) that can be utilized by a normalization step prior to aggregation to significantly improve the efficiency of self-consistency. Furthermore, the P(True) method, which achieves the highest WQD scores, has not been previously used for attempting to improve self-consistency.



\section{Conclusion}


We introduced \systemName, a watermarking technique that embeds computer-readable information while remaining invisible to the human eye. Unlike previous methods, our approach utilizes the entire document, including white spaces, and works regardless of background color. Through a psychophysical experiment, we determined the maximum ink that can be embedded without being detected. We developed tools to support users in applying IR ink technology, including software for efficient information embedding and a universal camera module for capturing \systemName~watermarks. Our open-source ML pipeline processes these images for robust use with standard QR code readers. We demonstrated various use cases, highlighting the potential of invisible IR content for hybrid paper-digital interfaces and advancing watermarking techniques.






% Can use something like this to put references on a page
% by themselves when using endfloat and the captionsoff option.
\ifCLASSOPTIONcaptionsoff
  \newpage
\fi


{
\bibliography{reference}
\bibliographystyle{unsrt}
}
%\begin{figure}
    \centering
    \includegraphics[width=\linewidth]{figures/MCQA_checklist.pdf}
    \vspace{-4.75ex}
    \setlength{\fboxsep}{0pt}
    \caption{\small Example unanswerable MCQ from MMLU \cite{gema2024we}, along with rubric criteria from \citet{haladyna1989taxonomy} flagged by OpenAI's o1 \cite{jaech2024openai}.}
    \label{fig:checklist}
    \vspace{-1.7ex}
\end{figure}


% insert where needed to balance the two columns on the last page with
% biographies
%\newpage

{Cesare Carissimo}
Cesare Carissimo is pursuing a Ph.D. at ETH Zürich with a focus on machine behaviour, complex systems, economics and game theory. He has an interdisciplinary background with a B.S. from the Faculty of Politics, Psychology, Law and Economics, and a M.S. of Computational Science from the University of
Amsterdam.

(Marcin Korecki)
Marcin Korecki is a researcher specializing in artificial intelligence, complexity, and philosophy. He earned his PhD from ETH Zurich, focusing on interdisciplinary work in these areas. His academic experience includes studies at the University of Groningen in the Netherlands and the University of Edinburgh in Scotland. During his PhD, he also spent six months at the University of Tokyo in Japan.

{Damian Dailisan }
is currently a postdoctoral researcher at ETH Z\"urich. He obtained his B.S. in applied physics in 2015 and his M.S. (2017) and Ph.D. (2020) degrees in physics from the University of the Philippines Diliman. After obtaining his Ph.D., he worked as a Data Scientist and Adjunct Professor at the Asian Institute of Management (2021). His research interests include phase transitions in agent-based cellular automata traffic models, data science, reinforcement learning, and AI.

\newpage
\newpage
\clearpage

%\appendices



% that's all folks
\end{document}
