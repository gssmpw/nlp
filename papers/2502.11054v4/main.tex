%%%%%%%%%%%%%%%%%%%%%%%%%%%%%%%%%%%%%%%%%%%%%%%%%%%%%%%%%%%%%%%%%%%%%
%%                                                                 %%
%% Please do not use \input{...} to include other tex files.       %%
%% Submit your LaTeX manuscript as one .tex document.              %%
%%                                                                 %%
%% All additional figures and files should be attached             %%
%% separately and not embedded in the \TeX\ document itself.       %%
%%                                                                 %%
%%%%%%%%%%%%%%%%%%%%%%%%%%%%%%%%%%%%%%%%%%%%%%%%%%%%%%%%%%%%%%%%%%%%%

% \documentclass[referee,sn-basic]{sn-jnl}% referee option is meant for double line spacing

%%=======================================================%%
%% to print line numbers in the margin use lineno option %%
%%=======================================================%%

%%\documentclass[lineno,sn-basic]{sn-jnl}% Basic Springer Nature Reference Style/Chemistry Reference Style

%%======================================================%%
%% to compile with pdflatex/xelatex use pdflatex option %%
%%======================================================%%

%%\documentclass[pdflatex,sn-basic]{sn-jnl}% Basic Springer Nature Reference Style/Chemistry Reference Style

%%\documentclass[sn-basic]{sn-jnl}% Basic Springer Nature Reference Style/Chemistry Reference Style
% \documentclass[sn-mathphys]{sn-jnl}% Math and Physical Sciences Reference Style
%%\documentclass[sn-aps]{sn-jnl}% American Physical Society (APS) Reference Style
%%\documentclass[sn-vancouver]{sn-jnl}% Vancouver Reference Style
%%\documentclass[sn-apa]{sn-jnl}% APA Reference Style
%%\documentclass[sn-chicago]{sn-jnl}% Chicago-based Humanities Reference Style
% \documentclass[sn-standardnature]{sn-jnl}% Standard Nature Portfolio Reference Style
% \documentclass[default]{sn-jnl}% Default
\documentclass[default,iicol]{sn-jnl}% Default with double column layout

%%%% Standard Packages
%%<additional latex packages if required can be included here>
%%%%

%%%%%=============================================================================%%%%
%%%%  Remarks: This template is provided to aid authors with the preparation
%%%%  of original research articles intended for submission to journals published 
%%%%  by Springer Nature. The guidance has been prepared in partnership with 
%%%%  production teams to conform to Springer Nature technical requirements. 
%%%%  Editorial and presentation requirements differ among journal portfolios and 
%%%%  research disciplines. You may find sections in this template are irrelevant 
%%%%  to your work and are empowered to omit any such section if allowed by the 
%%%%  journal you intend to submit to. The submission guidelines and policies 
%%%%  of the journal take precedence. A detailed User Manual is available in the 
%%%%  template package for technical guidance.
%%%%%=============================================================================%%%%

\jyear{2024}%

%% as per the requirement new theorem styles can be included as shown below
\theoremstyle{thmstyleone}%
\newtheorem{theorem}{Theorem}%  meant for continuous numbers
%%\newtheorem{theorem}{Theorem}[section]% meant for sectionwise numbers
%% optional argument [theorem] produces theorem numbering sequence instead of independent numbers for Proposition
\newtheorem{proposition}[theorem]{Proposition}% 
%%\newtheorem{proposition}{Proposition}% to get separate numbers for theorem and proposition etc.

%%%%%%%%%%%%%%%%%%%%%5add package
\usepackage{booktabs} % for professional tables
\usepackage{mathtools,amssymb}
\usepackage{subcaption}
\usepackage{color,xcolor}
\usepackage{times}
\usepackage{epsfig}
\usepackage{amsmath}
\usepackage{amssymb}
\usepackage{mathtools}
\usepackage{amsthm}
\usepackage{textcomp}
\usepackage{gensymb}
\usepackage{wasysym}
\usepackage{multirow}
\usepackage{graphicx} % more modern
\usepackage{colortbl}
\usepackage{footnote}
% For algorithms
\usepackage{algorithm}
\usepackage{algorithmicx}
\usepackage{longtable}
\usepackage{graphicx}
\usepackage{mathrsfs}
\usepackage{booktabs}
\usepackage{array} 
% added by myself
\usepackage{url}
\usepackage{lettrine}
\usepackage{makecell}
\usepackage{longtable}
\usepackage{tabularx}
\usepackage{xcolor}
% \usepackage[maxbibnames=150]{biblatex}
% \usepackage[numbers]{natbib}
% \definecolor{mycyan}{cmyk}{.3,0,0,0}
% \usepackage{cite}
% \usepackage{hyperref}

%%%%% NEW MATH DEFINITIONS %%%%%

\usepackage{amsmath,amsfonts,bm}
\usepackage{derivative}
% Mark sections of captions for referring to divisions of figures
\newcommand{\figleft}{{\em (Left)}}
\newcommand{\figcenter}{{\em (Center)}}
\newcommand{\figright}{{\em (Right)}}
\newcommand{\figtop}{{\em (Top)}}
\newcommand{\figbottom}{{\em (Bottom)}}
\newcommand{\captiona}{{\em (a)}}
\newcommand{\captionb}{{\em (b)}}
\newcommand{\captionc}{{\em (c)}}
\newcommand{\captiond}{{\em (d)}}

% Highlight a newly defined term
\newcommand{\newterm}[1]{{\bf #1}}

% Derivative d 
\newcommand{\deriv}{{\mathrm{d}}}

% Figure reference, lower-case.
\def\figref#1{figure~\ref{#1}}
% Figure reference, capital. For start of sentence
\def\Figref#1{Figure~\ref{#1}}
\def\twofigref#1#2{figures \ref{#1} and \ref{#2}}
\def\quadfigref#1#2#3#4{figures \ref{#1}, \ref{#2}, \ref{#3} and \ref{#4}}
% Section reference, lower-case.
\def\secref#1{section~\ref{#1}}
% Section reference, capital.
\def\Secref#1{Section~\ref{#1}}
% Reference to two sections.
\def\twosecrefs#1#2{sections \ref{#1} and \ref{#2}}
% Reference to three sections.
\def\secrefs#1#2#3{sections \ref{#1}, \ref{#2} and \ref{#3}}
% Reference to an equation, lower-case.
\def\eqref#1{equation~\ref{#1}}
% Reference to an equation, upper case
\def\Eqref#1{Equation~\ref{#1}}
% A raw reference to an equation---avoid using if possible
\def\plaineqref#1{\ref{#1}}
% Reference to a chapter, lower-case.
\def\chapref#1{chapter~\ref{#1}}
% Reference to an equation, upper case.
\def\Chapref#1{Chapter~\ref{#1}}
% Reference to a range of chapters
\def\rangechapref#1#2{chapters\ref{#1}--\ref{#2}}
% Reference to an algorithm, lower-case.
\def\algref#1{algorithm~\ref{#1}}
% Reference to an algorithm, upper case.
\def\Algref#1{Algorithm~\ref{#1}}
\def\twoalgref#1#2{algorithms \ref{#1} and \ref{#2}}
\def\Twoalgref#1#2{Algorithms \ref{#1} and \ref{#2}}
% Reference to a part, lower case
\def\partref#1{part~\ref{#1}}
% Reference to a part, upper case
\def\Partref#1{Part~\ref{#1}}
\def\twopartref#1#2{parts \ref{#1} and \ref{#2}}

\def\ceil#1{\lceil #1 \rceil}
\def\floor#1{\lfloor #1 \rfloor}
\def\1{\bm{1}}
\newcommand{\train}{\mathcal{D}}
\newcommand{\valid}{\mathcal{D_{\mathrm{valid}}}}
\newcommand{\test}{\mathcal{D_{\mathrm{test}}}}

\def\eps{{\epsilon}}


% Random variables
\def\reta{{\textnormal{$\eta$}}}
\def\ra{{\textnormal{a}}}
\def\rb{{\textnormal{b}}}
\def\rc{{\textnormal{c}}}
\def\rd{{\textnormal{d}}}
\def\re{{\textnormal{e}}}
\def\rf{{\textnormal{f}}}
\def\rg{{\textnormal{g}}}
\def\rh{{\textnormal{h}}}
\def\ri{{\textnormal{i}}}
\def\rj{{\textnormal{j}}}
\def\rk{{\textnormal{k}}}
\def\rl{{\textnormal{l}}}
% rm is already a command, just don't name any random variables m
\def\rn{{\textnormal{n}}}
\def\ro{{\textnormal{o}}}
\def\rp{{\textnormal{p}}}
\def\rq{{\textnormal{q}}}
\def\rr{{\textnormal{r}}}
\def\rs{{\textnormal{s}}}
\def\rt{{\textnormal{t}}}
\def\ru{{\textnormal{u}}}
\def\rv{{\textnormal{v}}}
\def\rw{{\textnormal{w}}}
\def\rx{{\textnormal{x}}}
\def\ry{{\textnormal{y}}}
\def\rz{{\textnormal{z}}}

% Random vectors
\def\rvepsilon{{\mathbf{\epsilon}}}
\def\rvphi{{\mathbf{\phi}}}
\def\rvtheta{{\mathbf{\theta}}}
\def\rva{{\mathbf{a}}}
\def\rvb{{\mathbf{b}}}
\def\rvc{{\mathbf{c}}}
\def\rvd{{\mathbf{d}}}
\def\rve{{\mathbf{e}}}
\def\rvf{{\mathbf{f}}}
\def\rvg{{\mathbf{g}}}
\def\rvh{{\mathbf{h}}}
\def\rvu{{\mathbf{i}}}
\def\rvj{{\mathbf{j}}}
\def\rvk{{\mathbf{k}}}
\def\rvl{{\mathbf{l}}}
\def\rvm{{\mathbf{m}}}
\def\rvn{{\mathbf{n}}}
\def\rvo{{\mathbf{o}}}
\def\rvp{{\mathbf{p}}}
\def\rvq{{\mathbf{q}}}
\def\rvr{{\mathbf{r}}}
\def\rvs{{\mathbf{s}}}
\def\rvt{{\mathbf{t}}}
\def\rvu{{\mathbf{u}}}
\def\rvv{{\mathbf{v}}}
\def\rvw{{\mathbf{w}}}
\def\rvx{{\mathbf{x}}}
\def\rvy{{\mathbf{y}}}
\def\rvz{{\mathbf{z}}}

% Elements of random vectors
\def\erva{{\textnormal{a}}}
\def\ervb{{\textnormal{b}}}
\def\ervc{{\textnormal{c}}}
\def\ervd{{\textnormal{d}}}
\def\erve{{\textnormal{e}}}
\def\ervf{{\textnormal{f}}}
\def\ervg{{\textnormal{g}}}
\def\ervh{{\textnormal{h}}}
\def\ervi{{\textnormal{i}}}
\def\ervj{{\textnormal{j}}}
\def\ervk{{\textnormal{k}}}
\def\ervl{{\textnormal{l}}}
\def\ervm{{\textnormal{m}}}
\def\ervn{{\textnormal{n}}}
\def\ervo{{\textnormal{o}}}
\def\ervp{{\textnormal{p}}}
\def\ervq{{\textnormal{q}}}
\def\ervr{{\textnormal{r}}}
\def\ervs{{\textnormal{s}}}
\def\ervt{{\textnormal{t}}}
\def\ervu{{\textnormal{u}}}
\def\ervv{{\textnormal{v}}}
\def\ervw{{\textnormal{w}}}
\def\ervx{{\textnormal{x}}}
\def\ervy{{\textnormal{y}}}
\def\ervz{{\textnormal{z}}}

% Random matrices
\def\rmA{{\mathbf{A}}}
\def\rmB{{\mathbf{B}}}
\def\rmC{{\mathbf{C}}}
\def\rmD{{\mathbf{D}}}
\def\rmE{{\mathbf{E}}}
\def\rmF{{\mathbf{F}}}
\def\rmG{{\mathbf{G}}}
\def\rmH{{\mathbf{H}}}
\def\rmI{{\mathbf{I}}}
\def\rmJ{{\mathbf{J}}}
\def\rmK{{\mathbf{K}}}
\def\rmL{{\mathbf{L}}}
\def\rmM{{\mathbf{M}}}
\def\rmN{{\mathbf{N}}}
\def\rmO{{\mathbf{O}}}
\def\rmP{{\mathbf{P}}}
\def\rmQ{{\mathbf{Q}}}
\def\rmR{{\mathbf{R}}}
\def\rmS{{\mathbf{S}}}
\def\rmT{{\mathbf{T}}}
\def\rmU{{\mathbf{U}}}
\def\rmV{{\mathbf{V}}}
\def\rmW{{\mathbf{W}}}
\def\rmX{{\mathbf{X}}}
\def\rmY{{\mathbf{Y}}}
\def\rmZ{{\mathbf{Z}}}

% Elements of random matrices
\def\ermA{{\textnormal{A}}}
\def\ermB{{\textnormal{B}}}
\def\ermC{{\textnormal{C}}}
\def\ermD{{\textnormal{D}}}
\def\ermE{{\textnormal{E}}}
\def\ermF{{\textnormal{F}}}
\def\ermG{{\textnormal{G}}}
\def\ermH{{\textnormal{H}}}
\def\ermI{{\textnormal{I}}}
\def\ermJ{{\textnormal{J}}}
\def\ermK{{\textnormal{K}}}
\def\ermL{{\textnormal{L}}}
\def\ermM{{\textnormal{M}}}
\def\ermN{{\textnormal{N}}}
\def\ermO{{\textnormal{O}}}
\def\ermP{{\textnormal{P}}}
\def\ermQ{{\textnormal{Q}}}
\def\ermR{{\textnormal{R}}}
\def\ermS{{\textnormal{S}}}
\def\ermT{{\textnormal{T}}}
\def\ermU{{\textnormal{U}}}
\def\ermV{{\textnormal{V}}}
\def\ermW{{\textnormal{W}}}
\def\ermX{{\textnormal{X}}}
\def\ermY{{\textnormal{Y}}}
\def\ermZ{{\textnormal{Z}}}

% Vectors
\def\vzero{{\bm{0}}}
\def\vone{{\bm{1}}}
\def\vmu{{\bm{\mu}}}
\def\vtheta{{\bm{\theta}}}
\def\vphi{{\bm{\phi}}}
\def\va{{\bm{a}}}
\def\vb{{\bm{b}}}
\def\vc{{\bm{c}}}
\def\vd{{\bm{d}}}
\def\ve{{\bm{e}}}
\def\vf{{\bm{f}}}
\def\vg{{\bm{g}}}
\def\vh{{\bm{h}}}
\def\vi{{\bm{i}}}
\def\vj{{\bm{j}}}
\def\vk{{\bm{k}}}
\def\vl{{\bm{l}}}
\def\vm{{\bm{m}}}
\def\vn{{\bm{n}}}
\def\vo{{\bm{o}}}
\def\vp{{\bm{p}}}
\def\vq{{\bm{q}}}
\def\vr{{\bm{r}}}
\def\vs{{\bm{s}}}
\def\vt{{\bm{t}}}
\def\vu{{\bm{u}}}
\def\vv{{\bm{v}}}
\def\vw{{\bm{w}}}
\def\vx{{\bm{x}}}
\def\vy{{\bm{y}}}
\def\vz{{\bm{z}}}

% Elements of vectors
\def\evalpha{{\alpha}}
\def\evbeta{{\beta}}
\def\evepsilon{{\epsilon}}
\def\evlambda{{\lambda}}
\def\evomega{{\omega}}
\def\evmu{{\mu}}
\def\evpsi{{\psi}}
\def\evsigma{{\sigma}}
\def\evtheta{{\theta}}
\def\eva{{a}}
\def\evb{{b}}
\def\evc{{c}}
\def\evd{{d}}
\def\eve{{e}}
\def\evf{{f}}
\def\evg{{g}}
\def\evh{{h}}
\def\evi{{i}}
\def\evj{{j}}
\def\evk{{k}}
\def\evl{{l}}
\def\evm{{m}}
\def\evn{{n}}
\def\evo{{o}}
\def\evp{{p}}
\def\evq{{q}}
\def\evr{{r}}
\def\evs{{s}}
\def\evt{{t}}
\def\evu{{u}}
\def\evv{{v}}
\def\evw{{w}}
\def\evx{{x}}
\def\evy{{y}}
\def\evz{{z}}

% Matrix
\def\mA{{\bm{A}}}
\def\mB{{\bm{B}}}
\def\mC{{\bm{C}}}
\def\mD{{\bm{D}}}
\def\mE{{\bm{E}}}
\def\mF{{\bm{F}}}
\def\mG{{\bm{G}}}
\def\mH{{\bm{H}}}
\def\mI{{\bm{I}}}
\def\mJ{{\bm{J}}}
\def\mK{{\bm{K}}}
\def\mL{{\bm{L}}}
\def\mM{{\bm{M}}}
\def\mN{{\bm{N}}}
\def\mO{{\bm{O}}}
\def\mP{{\bm{P}}}
\def\mQ{{\bm{Q}}}
\def\mR{{\bm{R}}}
\def\mS{{\bm{S}}}
\def\mT{{\bm{T}}}
\def\mU{{\bm{U}}}
\def\mV{{\bm{V}}}
\def\mW{{\bm{W}}}
\def\mX{{\bm{X}}}
\def\mY{{\bm{Y}}}
\def\mZ{{\bm{Z}}}
\def\mBeta{{\bm{\beta}}}
\def\mPhi{{\bm{\Phi}}}
\def\mLambda{{\bm{\Lambda}}}
\def\mSigma{{\bm{\Sigma}}}

% Tensor
\DeclareMathAlphabet{\mathsfit}{\encodingdefault}{\sfdefault}{m}{sl}
\SetMathAlphabet{\mathsfit}{bold}{\encodingdefault}{\sfdefault}{bx}{n}
\newcommand{\tens}[1]{\bm{\mathsfit{#1}}}
\def\tA{{\tens{A}}}
\def\tB{{\tens{B}}}
\def\tC{{\tens{C}}}
\def\tD{{\tens{D}}}
\def\tE{{\tens{E}}}
\def\tF{{\tens{F}}}
\def\tG{{\tens{G}}}
\def\tH{{\tens{H}}}
\def\tI{{\tens{I}}}
\def\tJ{{\tens{J}}}
\def\tK{{\tens{K}}}
\def\tL{{\tens{L}}}
\def\tM{{\tens{M}}}
\def\tN{{\tens{N}}}
\def\tO{{\tens{O}}}
\def\tP{{\tens{P}}}
\def\tQ{{\tens{Q}}}
\def\tR{{\tens{R}}}
\def\tS{{\tens{S}}}
\def\tT{{\tens{T}}}
\def\tU{{\tens{U}}}
\def\tV{{\tens{V}}}
\def\tW{{\tens{W}}}
\def\tX{{\tens{X}}}
\def\tY{{\tens{Y}}}
\def\tZ{{\tens{Z}}}


% Graph
\def\gA{{\mathcal{A}}}
\def\gB{{\mathcal{B}}}
\def\gC{{\mathcal{C}}}
\def\gD{{\mathcal{D}}}
\def\gE{{\mathcal{E}}}
\def\gF{{\mathcal{F}}}
\def\gG{{\mathcal{G}}}
\def\gH{{\mathcal{H}}}
\def\gI{{\mathcal{I}}}
\def\gJ{{\mathcal{J}}}
\def\gK{{\mathcal{K}}}
\def\gL{{\mathcal{L}}}
\def\gM{{\mathcal{M}}}
\def\gN{{\mathcal{N}}}
\def\gO{{\mathcal{O}}}
\def\gP{{\mathcal{P}}}
\def\gQ{{\mathcal{Q}}}
\def\gR{{\mathcal{R}}}
\def\gS{{\mathcal{S}}}
\def\gT{{\mathcal{T}}}
\def\gU{{\mathcal{U}}}
\def\gV{{\mathcal{V}}}
\def\gW{{\mathcal{W}}}
\def\gX{{\mathcal{X}}}
\def\gY{{\mathcal{Y}}}
\def\gZ{{\mathcal{Z}}}

% Sets
\def\sA{{\mathbb{A}}}
\def\sB{{\mathbb{B}}}
\def\sC{{\mathbb{C}}}
\def\sD{{\mathbb{D}}}
% Don't use a set called E, because this would be the same as our symbol
% for expectation.
\def\sF{{\mathbb{F}}}
\def\sG{{\mathbb{G}}}
\def\sH{{\mathbb{H}}}
\def\sI{{\mathbb{I}}}
\def\sJ{{\mathbb{J}}}
\def\sK{{\mathbb{K}}}
\def\sL{{\mathbb{L}}}
\def\sM{{\mathbb{M}}}
\def\sN{{\mathbb{N}}}
\def\sO{{\mathbb{O}}}
\def\sP{{\mathbb{P}}}
\def\sQ{{\mathbb{Q}}}
\def\sR{{\mathbb{R}}}
\def\sS{{\mathbb{S}}}
\def\sT{{\mathbb{T}}}
\def\sU{{\mathbb{U}}}
\def\sV{{\mathbb{V}}}
\def\sW{{\mathbb{W}}}
\def\sX{{\mathbb{X}}}
\def\sY{{\mathbb{Y}}}
\def\sZ{{\mathbb{Z}}}

% Entries of a matrix
\def\emLambda{{\Lambda}}
\def\emA{{A}}
\def\emB{{B}}
\def\emC{{C}}
\def\emD{{D}}
\def\emE{{E}}
\def\emF{{F}}
\def\emG{{G}}
\def\emH{{H}}
\def\emI{{I}}
\def\emJ{{J}}
\def\emK{{K}}
\def\emL{{L}}
\def\emM{{M}}
\def\emN{{N}}
\def\emO{{O}}
\def\emP{{P}}
\def\emQ{{Q}}
\def\emR{{R}}
\def\emS{{S}}
\def\emT{{T}}
\def\emU{{U}}
\def\emV{{V}}
\def\emW{{W}}
\def\emX{{X}}
\def\emY{{Y}}
\def\emZ{{Z}}
\def\emSigma{{\Sigma}}

% entries of a tensor
% Same font as tensor, without \bm wrapper
\newcommand{\etens}[1]{\mathsfit{#1}}
\def\etLambda{{\etens{\Lambda}}}
\def\etA{{\etens{A}}}
\def\etB{{\etens{B}}}
\def\etC{{\etens{C}}}
\def\etD{{\etens{D}}}
\def\etE{{\etens{E}}}
\def\etF{{\etens{F}}}
\def\etG{{\etens{G}}}
\def\etH{{\etens{H}}}
\def\etI{{\etens{I}}}
\def\etJ{{\etens{J}}}
\def\etK{{\etens{K}}}
\def\etL{{\etens{L}}}
\def\etM{{\etens{M}}}
\def\etN{{\etens{N}}}
\def\etO{{\etens{O}}}
\def\etP{{\etens{P}}}
\def\etQ{{\etens{Q}}}
\def\etR{{\etens{R}}}
\def\etS{{\etens{S}}}
\def\etT{{\etens{T}}}
\def\etU{{\etens{U}}}
\def\etV{{\etens{V}}}
\def\etW{{\etens{W}}}
\def\etX{{\etens{X}}}
\def\etY{{\etens{Y}}}
\def\etZ{{\etens{Z}}}

% The true underlying data generating distribution
\newcommand{\pdata}{p_{\rm{data}}}
\newcommand{\ptarget}{p_{\rm{target}}}
\newcommand{\pprior}{p_{\rm{prior}}}
\newcommand{\pbase}{p_{\rm{base}}}
\newcommand{\pref}{p_{\rm{ref}}}

% The empirical distribution defined by the training set
\newcommand{\ptrain}{\hat{p}_{\rm{data}}}
\newcommand{\Ptrain}{\hat{P}_{\rm{data}}}
% The model distribution
\newcommand{\pmodel}{p_{\rm{model}}}
\newcommand{\Pmodel}{P_{\rm{model}}}
\newcommand{\ptildemodel}{\tilde{p}_{\rm{model}}}
% Stochastic autoencoder distributions
\newcommand{\pencode}{p_{\rm{encoder}}}
\newcommand{\pdecode}{p_{\rm{decoder}}}
\newcommand{\precons}{p_{\rm{reconstruct}}}

\newcommand{\laplace}{\mathrm{Laplace}} % Laplace distribution

\newcommand{\E}{\mathbb{E}}
\newcommand{\Ls}{\mathcal{L}}
\newcommand{\R}{\mathbb{R}}
\newcommand{\emp}{\tilde{p}}
\newcommand{\lr}{\alpha}
\newcommand{\reg}{\lambda}
\newcommand{\rect}{\mathrm{rectifier}}
\newcommand{\softmax}{\mathrm{softmax}}
\newcommand{\sigmoid}{\sigma}
\newcommand{\softplus}{\zeta}
\newcommand{\KL}{D_{\mathrm{KL}}}
\newcommand{\Var}{\mathrm{Var}}
\newcommand{\standarderror}{\mathrm{SE}}
\newcommand{\Cov}{\mathrm{Cov}}
% Wolfram Mathworld says $L^2$ is for function spaces and $\ell^2$ is for vectors
% But then they seem to use $L^2$ for vectors throughout the site, and so does
% wikipedia.
\newcommand{\normlzero}{L^0}
\newcommand{\normlone}{L^1}
\newcommand{\normltwo}{L^2}
\newcommand{\normlp}{L^p}
\newcommand{\normmax}{L^\infty}

\newcommand{\parents}{Pa} % See usage in notation.tex. Chosen to match Daphne's book.

\DeclareMathOperator*{\argmax}{arg\,max}
\DeclareMathOperator*{\argmin}{arg\,min}

\DeclareMathOperator{\sign}{sign}
\DeclareMathOperator{\Tr}{Tr}
\let\ab\allowbreak



\newcommand{\tool}{\emph{SafeBench}\space}
\newcommand{\toolns}{\emph{SafeBench}}
\newcommand{\mini}{\emph{MiniBench}}
\usepackage{amsfonts}
\usepackage{colortbl}

\usepackage{caption}
\usepackage{subcaption}
\usepackage{graphicx}
\usepackage{cleveref}

\newcommand{\todo}[1]{\textcolor{red}{#1}}

\usepackage{cellspace}
\setlength\cellspacetoplimit{5pt}
\setlength\cellspacebottomlimit{5pt}


\newcommand{\ie}{\textit{i}.\textit{e}.}
\newcommand{\eg}{\textit{e}.\textit{g}.} 
\newcommand{\Tref}[1]{Tab.~\ref{#1}}
\newcommand{\Eref}[1]{Eq.~(\ref{#1})}
\newcommand{\Fref}[1]{Fig.~\ref{#1}}
\newcommand{\Sref}[1]{Sec.~\ref{#1}}
\newcommand{\Aref}[1]{Alg.~\ref{#1}}
\newcommand{\etal}{\textit{et al}.}
\newcommand{\el}{\textit{et al}.}
\newcommand{\etc}{\textit{etc}.}
\newcommand{\ul}{\underline}
\newcommand{\method}{\emph{RACE}}

\usepackage{pifont}
\usepackage[perpage,symbol*]{footmisc}
\DefineFNsymbols{circled}{{\ding{192}}{\ding{193}}{\ding{194}}
{\ding{195}}{\ding{196}}{\ding{197}}{\ding{198}}{\ding{199}}{\ding{200}}{\ding{201}}}
\setfnsymbol{circled}

\newcommand\myfootnotestyle[1]{\ifcase#1 \or \ding{182}\or \ding{183}\or
\ding{184}\or \ding{185}\or \ding{186}\or \ding{187}%
\or \ding{188}\or \ding{189}\or \ding{190}\or \ding{191}\else *\fi\relax}


\newcommand{\red}[1]{\textcolor{red}{#1}}
%%%%%%%%%%%%%%%%%%%%555555

\lstset{
    escapeinside={(*}{*)}
}

\newcommand\pythonnstyle{\lstset{
escapeinside={(*}{*)},
numbers=left,
xleftmargin=5.0ex,
numberstyle=\scriptsize,
basicstyle=\scriptsize\ttfamily,
emphstyle=\scriptsize\ttfamily\color{red},
keywordstyle=\scriptsize\ttfamily\color{blue},
language=Python
}}
\lstnewenvironment{pythonn}[1][]
{
\pythonnstyle
\lstset{#1}
}
{}

\usepackage{xcolor}

\definecolor{codegreen}{rgb}{0,0.6,0}
\definecolor{codegray}{rgb}{0.5,0.5,0.5}
\definecolor{codepurple}{rgb}{0.58,0,0.82}
\definecolor{backcolour}{rgb}{0.95,0.95,0.92}

\lstdefinestyle{mystyle}{
    backgroundcolor=\color{backcolour},   
    commentstyle=\color{codegreen},
    keywordstyle=\color{magenta},
    numberstyle=\tiny\color{codegray},
    stringstyle=\color{codepurple},
    basicstyle=\ttfamily\footnotesize,
    breakatwhitespace=false,         
    breaklines=true,                 
    captionpos=b,                    
    keepspaces=true,                 
    numbers=left,                    
    numbersep=5pt,                  
    showspaces=false,                
    showstringspaces=false,
    showtabs=false,                  
    tabsize=2,
    frame=single
}

\lstset{style=mystyle}




\theoremstyle{thmstyletwo}%
\newtheorem{example}{Example}%
\newtheorem{remark}{Remark}%

\theoremstyle{thmstylethree}%
\newtheorem{definition}{Definition}%

\raggedbottom
%%\unnumbered% uncomment this for unnumbered level heads

\begin{document}

\title[Article Title]{Reasoning-Augmented Conversation for Multi-Turn Jailbreak Attacks on Large Language Models}

%%=============================================================%%
%% Prefix	-> \pfx{Dr}
%% GivenName	-> \fnm{Joergen W.}
%% Particle	-> \spfx{van der} -> surname prefix
%% FamilyName	-> \sur{Ploeg}
%% Suffix	-> \sfx{IV}
%% NatureName	-> \tanm{Poet Laureate} -> Title after name
%% Degrees	-> \dgr{MSc, PhD}
%% \author*[1,2]{\pfx{Dr} \fnm{Joergen W.} \spfx{van der} \sur{Ploeg} \sfx{IV} \tanm{Poet Laureate} 
%%                 \dgr{MSc, PhD}}\email{iauthor@gmail.com}
%%=============================================================%%
\author[1]{\fnm{Zonghao} \sur{Ying}}

\author[2]{\fnm{Deyue} \sur{Zhang}}

\author[1]{\fnm{Zonglei} \sur{Jing}}

\author[1]{\fnm{Yisong} \sur{Xiao}}

\author[2]{\fnm{Quanchen} \sur{Zou}} 


\author[1]{\fnm{Aishan} \sur{Liu}}

\author[3]{\fnm{Siyuan} \sur{Liang}}

\author[2]{\fnm{Xiangzheng} \sur{Zhang}}

\author[1]{\fnm{Xianglong} \sur{Liu}}

\author[4]{\fnm{Dacheng} \sur{Tao}}



\affil[1]{\orgname{Beihang University}, \orgaddress{\country{China}}}
\affil[2]{\orgname{360 AI Security Lab}, \orgaddress{\country{China}}}

\affil[3]{\orgname{National University of Singapore}, \orgaddress{\country{Singapore}}}

\affil[4]{\orgname{Nanyang Technological University}, \orgaddress{\country{Singapore}}}

% \affil[3]{\orgdiv{Department}, \orgname{Organization}, \orgaddress{\street{Street}, \city{City}, \postcode{610101}, \state{State}, \country{Country}}}

%%==================================%%
%% sample for unstructured abstract %%
%%==================================%%

\abstract{

Multi-turn jailbreak attacks simulate real-world human interactions by engaging large language models (LLMs) in iterative dialogues, exposing critical safety vulnerabilities. However, existing methods often struggle to balance semantic coherence with attack effectiveness, resulting in either benign semantic drift or ineffective detection evasion. To address this challenge, we propose Reasoning-Augmented Conversation (\method{}), a novel multi-turn jailbreak framework that reformulates harmful queries into benign reasoning tasks and leverages LLMs’ strong reasoning capabilities to compromise safety alignment. Specifically, we introduce an attack state machine framework to systematically model problem translation and iterative reasoning, ensuring coherent query generation across multiple turns. Building on this framework, we design gain-guided exploration, self-play, and rejection feedback modules to preserve attack semantics, enhance effectiveness, and sustain reasoning-driven attack progression. Extensive experiments on multiple LLMs demonstrate that \method{} achieves state-of-the-art attack effectiveness in complex conversational scenarios, with attack success rates (ASRs) increasing by up to 96\%. Notably, our approach achieves ASRs of 82\% and 92\% against leading commercial models, OpenAI o1 and DeepSeek R1, underscoring its potency. We release our code at \url{https://github.com/NY1024/RACE} to facilitate further research in this critical domain. \textcolor{red}{Warning: This paper contains model outputs that are unsafe.}

}


\keywords{Multi-turn jailbreak, large language models, reasoning-driven attack}

\maketitle


\section{Introduction}

Despite the remarkable capabilities of large language models (LLMs)~\cite{DBLP:conf/emnlp/QinZ0CYY23,DBLP:journals/corr/abs-2307-09288}, they often inevitably exhibit hallucinations due to incorrect or outdated knowledge embedded in their parameters~\cite{DBLP:journals/corr/abs-2309-01219, DBLP:journals/corr/abs-2302-12813, DBLP:journals/csur/JiLFYSXIBMF23}.
Given the significant time and expense required to retrain LLMs, there has been growing interest in \emph{model editing} (a.k.a., \emph{knowledge editing})~\cite{DBLP:conf/iclr/SinitsinPPPB20, DBLP:journals/corr/abs-2012-00363, DBLP:conf/acl/DaiDHSCW22, DBLP:conf/icml/MitchellLBMF22, DBLP:conf/nips/MengBAB22, DBLP:conf/iclr/MengSABB23, DBLP:conf/emnlp/YaoWT0LDC023, DBLP:conf/emnlp/ZhongWMPC23, DBLP:conf/icml/MaL0G24, DBLP:journals/corr/abs-2401-04700}, 
which aims to update the knowledge of LLMs cost-effectively.
Some existing methods of model editing achieve this by modifying model parameters, which can be generally divided into two categories~\cite{DBLP:journals/corr/abs-2308-07269, DBLP:conf/emnlp/YaoWT0LDC023}.
Specifically, one type is based on \emph{Meta-Learning}~\cite{DBLP:conf/emnlp/CaoAT21, DBLP:conf/acl/DaiDHSCW22}, while the other is based on \emph{Locate-then-Edit}~\cite{DBLP:conf/acl/DaiDHSCW22, DBLP:conf/nips/MengBAB22, DBLP:conf/iclr/MengSABB23}. This paper primarily focuses on the latter.

\begin{figure}[t]
  \centering
  \includegraphics[width=0.48\textwidth]{figures/demonstration.pdf}
  \vspace{-4mm}
  \caption{(a) Comparison of regular model editing and EAC. EAC compresses the editing information into the dimensions where the editing anchors are located. Here, we utilize the gradients generated during training and the magnitude of the updated knowledge vector to identify anchors. (b) Comparison of general downstream task performance before editing, after regular editing, and after constrained editing by EAC.}
  \vspace{-3mm}
  \label{demo}
\end{figure}

\emph{Sequential} model editing~\cite{DBLP:conf/emnlp/YaoWT0LDC023} can expedite the continual learning of LLMs where a series of consecutive edits are conducted.
This is very important in real-world scenarios because new knowledge continually appears, requiring the model to retain previous knowledge while conducting new edits. 
Some studies have experimentally revealed that in sequential editing, existing methods lead to a decrease in the general abilities of the model across downstream tasks~\cite{DBLP:journals/corr/abs-2401-04700, DBLP:conf/acl/GuptaRA24, DBLP:conf/acl/Yang0MLYC24, DBLP:conf/acl/HuC00024}. 
Besides, \citet{ma2024perturbation} have performed a theoretical analysis to elucidate the bottleneck of the general abilities during sequential editing.
However, previous work has not introduced an effective method that maintains editing performance while preserving general abilities in sequential editing.
This impacts model scalability and presents major challenges for continuous learning in LLMs.

In this paper, a statistical analysis is first conducted to help understand how the model is affected during sequential editing using two popular editing methods, including ROME~\cite{DBLP:conf/nips/MengBAB22} and MEMIT~\cite{DBLP:conf/iclr/MengSABB23}.
Matrix norms, particularly the L1 norm, have been shown to be effective indicators of matrix properties such as sparsity, stability, and conditioning, as evidenced by several theoretical works~\cite{kahan2013tutorial}. In our analysis of matrix norms, we observe significant deviations in the parameter matrix after sequential editing.
Besides, the semantic differences between the facts before and after editing are also visualized, and we find that the differences become larger as the deviation of the parameter matrix after editing increases.
Therefore, we assume that each edit during sequential editing not only updates the editing fact as expected but also unintentionally introduces non-trivial noise that can cause the edited model to deviate from its original semantics space.
Furthermore, the accumulation of non-trivial noise can amplify the negative impact on the general abilities of LLMs.

Inspired by these findings, a framework termed \textbf{E}diting \textbf{A}nchor \textbf{C}ompression (EAC) is proposed to constrain the deviation of the parameter matrix during sequential editing by reducing the norm of the update matrix at each step. 
As shown in Figure~\ref{demo}, EAC first selects a subset of dimension with a high product of gradient and magnitude values, namely editing anchors, that are considered crucial for encoding the new relation through a weighted gradient saliency map.
Retraining is then performed on the dimensions where these important editing anchors are located, effectively compressing the editing information.
By compressing information only in certain dimensions and leaving other dimensions unmodified, the deviation of the parameter matrix after editing is constrained. 
To further regulate changes in the L1 norm of the edited matrix to constrain the deviation, we incorporate a scored elastic net ~\cite{zou2005regularization} into the retraining process, optimizing the previously selected editing anchors.

To validate the effectiveness of the proposed EAC, experiments of applying EAC to \textbf{two popular editing methods} including ROME and MEMIT are conducted.
In addition, \textbf{three LLMs of varying sizes} including GPT2-XL~\cite{radford2019language}, LLaMA-3 (8B)~\cite{llama3} and LLaMA-2 (13B)~\cite{DBLP:journals/corr/abs-2307-09288} and \textbf{four representative tasks} including 
natural language inference~\cite{DBLP:conf/mlcw/DaganGM05}, 
summarization~\cite{gliwa-etal-2019-samsum},
open-domain question-answering~\cite{DBLP:journals/tacl/KwiatkowskiPRCP19},  
and sentiment analysis~\cite{DBLP:conf/emnlp/SocherPWCMNP13} are selected to extensively demonstrate the impact of model editing on the general abilities of LLMs. 
Experimental results demonstrate that in sequential editing, EAC can effectively preserve over 70\% of the general abilities of the model across downstream tasks and better retain the edited knowledge.

In summary, our contributions to this paper are three-fold:
(1) This paper statistically elucidates how deviations in the parameter matrix after editing are responsible for the decreased general abilities of the model across downstream tasks after sequential editing.
(2) A framework termed EAC is proposed, which ultimately aims to constrain the deviation of the parameter matrix after editing by compressing the editing information into editing anchors. 
(3) It is discovered that on models like GPT2-XL and LLaMA-3 (8B), EAC significantly preserves over 70\% of the general abilities across downstream tasks and retains the edited knowledge better.
\section{Related work}




\textbf{Reasoning in LLMs}. Reasoning is a cognitive process that involves thinking about something logically and systematically, using evidence and past experiences to draw conclusions or make decisions \cite{reason1,reason2}. Recent studies have demonstrated that LLMs exhibit remarkable reasoning capabilities in various tasks, including mathematical reasoning \cite{reasonllm1}, common sense reasoning \cite{reasonllm2}, symbolic reasoning \cite{reasonllm3}, and causal reasoning \cite{reasonllm4}. Subsequently, Chain-of-thought (CoT) \cite{cot1,cot2,cot3,cot4,cot5} has emerged as a promising approach for further enhancing these reasoning capabilities.

While the reasoning capabilities of LLMs have contributed to their impressive performance across various downstream tasks, their potential exploitation in jailbreak attacks remains largely unexplored. In this study, we focus on leveraging reasoning capabilities to facilitate jailbreak attacks.

\textbf{Multi-turn Jailbreak Attack}. Typical multi-turn jailbreak methods follow the principle of starting with harmless conversations and gradually making the queries more harmful in subsequent turns. Different methods have designed specific strategies based on this principle, including applying cognitive psychology theories to gradually modify subsequent queries \cite{mpsy1,mpsy2}, using actor networks to expand the attack range of subsequent queries \cite{ren2024}, extracting harmful keywords from original queries to construct semantically equivalent ones \cite{coa,cfa}, and breaking down the target query into multiple subqueries and merging the corresponding answers to achieve the final jailbreak \cite{sub1,sub2}.

Existing multi-turn jailbreak methods often suffer from semantic drift or fail to generate effective attacks. In contrast, our approach leverages LLMs' reasoning capabilities to ensure a stable and effective jailbreak process.


This section presents an empirical study of grid carbon intensity differences that occur over mesoscale geographic distances of tens to hundreds of kilometers. We also analyze the increases in network latency at these scales. Our empirical study seeks to answer two key questions. 

\begin{enumerate}[leftmargin=*]
    \item {\em How much does energy's carbon intensity vary within mesoscale regions that span tens to hundreds of kilometers, and are these differences large enough to warrant the use of spatial workload shifting in distributed edge data centers?}
    
    % How do these variations compare to those at large continental scales or cloud regions?\walid{We do not answer this question.}
    \item {\em How prevalent are these types of mesoscale variations in different parts of the world? Are they sufficiently common to warrant the broad deployment of carbon optimization techniques in edge data centers across the world?}
\end{enumerate}


% This section analyzes the benefits of {\em mesoscale} carbon intensity information. Then, we highlight the latency benefits of mesoscale-level spatial shifting.



%%%%%%%%%%%%%%%%%%%%%%%%%%%%%%%%%%%%%%
%%% Yearly average
%%%%%%%%%%%%%%%%%%%%%%%%%%%%%%%%%%%%%%
\begin{figure}[t]
  \centering%
  %   \hfill
  % \begin{subfigure}{0.24\linewidth}%
  % \centering
  %      \includegraphics[width=\linewidth]{figures/yearly_fl.pdf}%
  %      \caption{Florida}
  %      \label{fig:cv_yearly_r1}
  %   \end{subfigure}%
% \hfill%
 \begin{subfigure}{0.45\linewidth}%
 \centering
       \includegraphics[width=\linewidth]{figures/yearly_cross_us.pdf}%
       \caption{West US}
       \label{fig:cv_yearly_r2}
    \end{subfigure}%
    % \hfill%
%  \begin{subfigure}{0.2\linewidth}%
%  \centering
%        \includegraphics[width=\linewidth]{figures/yearly_cross_us_nm.pdf}%
%        \caption{R3: New Mexico}
%        \label{fig:cv_yearly_r3}
%     \end{subfigure}%
% \hfill%
% \begin{subfigure}{0.24\linewidth}%
%     \centering
%        \includegraphics[width=\linewidth]{figures/yearly_it.pdf}%
%        \caption{Italy}
%        \label{fig:cv_yearly_r4}
%     \end{subfigure}%
\hfill%
\begin{subfigure}{0.45\linewidth}%
        \centering
       \includegraphics[width=\linewidth]{figures/yearly_cross_eu.pdf}%
       \caption{Central EU}
       \label{fig:cv_yearly_r3}
    \end{subfigure}%
    % \hfill
    % \hfill
    \caption{Yearly carbon intensity of two mesoscale regions.}% highlighting differences of up to 10.8$\times$.
    \label{fig:cv_yearly}%
\end{figure}

%%%%%%%%%%%%%%%%%%%%%%%%%%%%%%%%%%%%%%
%%% Two day time series
%%%%%%%%%%%%%%%%%%%%%%%%%%%%%%%%%%%%%%
\begin{figure}[t]
    \centering%
    \includegraphics[width=0.9\linewidth]{figures/monthly_us_legend.pdf}\\
    % \hfill
    \begin{subfigure}{0.45\linewidth}% 
    \centering%
    \captionsetup{justification=centering}
    \includegraphics[width=\linewidth]{figures/daily_cross_us.pdf}
    \caption{Two-day (Dec 25-27)}
    \label{fig:carbon_intensity_temporal_days_WUS}
    \end{subfigure}
    \quad
    \begin{subfigure}{0.45\linewidth}% 
    \centering%  
    \captionsetup{justification=centering}
    \includegraphics[width=\linewidth]{figures/monthly_us.pdf}
    \caption{Year-long (2023)} 
    \label{fig:carbon_intensity_temporal_months_WUS}
    \end{subfigure}
    % \hfill%
    % \begin{subfigure}{0.22\linewidth}% 
    % \centering%  
    % \captionsetup{justification=centering}
    % \includegraphics[width=\linewidth]{figures/daily_cross_eu.pdf}
    % \caption{Central EU \\ (July 17-19) } %Variations at different times of day between July 17-19.
    % \label{fig:carbon_intensity_temporal_days_CEU}
    % \end{subfigure}
    % \hfill%
    % \begin{subfigure}{0.22\linewidth}% 
    % \centering%  
    % \captionsetup{justification=centering}
    % \includegraphics[width=\linewidth]{figures/monthly_europe.pdf}
    % \caption{Central EU \\ (2023)} % Variations across different months throughout 2023.
    % \label{fig:carbon_intensity_temporal_months_CEU}
    % \end{subfigure}
    % \hfill
    % \hfill
    \caption{Spatial-temporal variations in carbon intensity over two days and 12 months in 2023 in West US.}
    \label{fig:carbon_intensity_temporal}
\end{figure}



\begin{figure*}[tb]
    \centering
    \subfloat[\centering D = 200 km ]{\includegraphics[width=0.22\linewidth]{figures/carbon_saving_200km_text.pdf}%
    \label{fig:carbon_saving_200km}%
    }%
    \hfill
    \subfloat[\centering D = 500 km ]{\includegraphics[width=0.22\linewidth]{figures/carbon_saving_500km_text.pdf}%
    \label{fig:carbon_saving_500km}%
    }%
    \hfill
    \subfloat[\centering D = 1000 km ]{\includegraphics[width=0.22\linewidth]{figures/carbon_saving_1000km_text.pdf}
    \label{fig:carbon_saving_1000km}
    }%
    \hfill
    \subfloat[\centering Radius-Latency ]{\includegraphics[width=0.22\linewidth]{figures/distance_latency.pdf}%
    \label{fig:mesoscale_round_trip_latency}%
    }%
    % \vspace{-2mm}
    \caption{Carbon savings with search radii of 200 km, 500 km, and 1000 km. (d) One-way latency across pairwise distances.}
    \label{fig:carbon_saving_distance}
    \vspace{-.4cm}
\end{figure*}




\subsection{Carbon Intensity Analysis at Mesoscales}

To understand the differences in grid carbon intensity that are seen at mesoscales, we conducted a measurement study where we collected carbon intensity traces for 148 carbon zones worldwide for an entire year (2023). For the purpose of our study, a carbon zone, or simply a \textit{zone}, is a geographic area whose grid operator provides carbon intensity data. The geographic size of a carbon zone depends on the area served by the grid operator and can vary from a city to an entire state or even a small country. %This dataset is described in detail in \autoref{sec:real_world_traces}. 
Further, we also collected round-trip latency traces from the WonderNetwork~\cite{wonder-proxy-2020}, which provides ping traces (in milliseconds) to cities across the world. We describe our data sources in \autoref{sec:real_world_traces}. 

To illustrate the carbon intensity differences at the mesoscale, we first select four specific mesoscale regions, each comprising five carbon zones, across the United States and Europe. \autoref{fig:cv_snapshot} depicts a heat-map of the carbon intensity variations within each mesoscale region for a single hour in 2023, with darker colors representing higher carbon intensity values. We assume that each of the five carbon zones within a mesoscale region has an edge data center. 
The Florida region, for example, consists of five cities, each hosting an edge data center, that is a few hundred kilometers apart from one another. 
The figure shows significant differences in carbon intensity values even at this scale, with inter-zone variations of 2.5$\times$ in Florida, 7.9$\times$ in the west US, 2.2$\times$ in Italy, and 19.5$\times$ in Central Europe. 

~\autoref{fig:cv_yearly} then plots the mean carbon intensity over the entire year for two regions. The figure confirms that the differences in carbon intensity persist across the year. Furthermore, the average difference between the maximum and minimum carbon intensity across zones in a region is %1.9$\times$ in Florida, 
2.7$\times$ in the west US, %2.2$\times$ in Italy, 
and 10.8$\times$ in central Europe. Importantly, these differences compare favorably to those reported across global cloud regions. For instance, a recent study of spatial differences in carbon intensity across Amazon's cloud regions reported an order of magnitude difference across AWS cloud regions in Europe and Asia~\cite{sukprasert2024limitations}.
% The primary reason for these differences at smaller geographic scales is the diverse range of generation sources and different amounts of renewable sources in local electric grids. 
% For instance, Region 4 (Central Europe), which spans five countries, is powered by a diverse range of providers and energy mixtures. Nuclear energy represents 63.7\% of France's energy supply, making its carbon intensity low and stable, while Germany heavily relies on fossil fuels, representing 33.7\% of its energy supply~\cite{electricity-map}. 
Moreover, since the relative mix of energy sources changes over time, \autoref{fig:carbon_intensity_temporal} shows temporal fluctuations in the carbon intensity of edge data centers within each mesoscale region within a day (\autoref{fig:carbon_intensity_temporal_days_WUS}) %and \autoref{fig:carbon_intensity_temporal_days_CEU}) 
and across seasons (\autoref{fig:carbon_intensity_temporal_months_WUS}). %and \autoref{fig:carbon_intensity_temporal_months_CEU}).
For instance, Flagstaff, AZ (see \autoref{fig:carbon_intensity_temporal_days_WUS}) exhibits a daily difference of $\sim$300\carbonunit.
%Moreover, ~\autoref{fig:carbon_intensity_temporal} also indicates how the carbon intensity ascending order can shift across zones, dictating adjustments run-time placement adjustment. For instance, the carbon intensity of Munich, DE, can be higher or lower than that of Milan, IT, by 204.9 and 68.1 \carbonunit, respectively. In addition to carbon variations across times of day, carbon intensities are highly affected by seasonal changes due to changes in weather and demand patterns. 
~\autoref{fig:carbon_intensity_temporal_months_WUS} shows how monthly average carbon intensity changes.
For example, Kingman, AZ, exhibits a $\sim$200 \carbonunit change between March and November due to its reliance on solar energy. %In contrast, Lyon, FR, and Bern, CH have low and stable carbon intensity due to the high penetration of nuclear energy. 
%%%%
% Latency
%%%%
\begin{table}[t]
  \centering
    \caption{One-way network latency (ms).  %Shifting workload from Jacksonville to Miami in Florida achieves a 47\% carbon reduction while incurring only a 7.3 ms delay.  between edge locations in two mesoscale regions
    } 
  \label{tab:latency}
  \subfloat[Florida]{%
  \resizebox{0.53\linewidth}{!}{
  \begin{tabular}{|c|c|c|c|c|}
    \hline
    Location &  Miami &  Orlando &  Tampa & Tallah. \\
    \hline
    Jacksonville & 3.64 & 5.32 & 6.86 & 3.42\\ \hline
    Miami &   - & 4.5  & 3.37 & 7.2 \\ \hline
    Orlando &   & - &  1.86 & 4.35\\ \hline
    Tampa &   &  & - & 4.14\\ \hline
    Tallahassee  &  & &  & -\\ \hline
  \end{tabular}
  \label{tab:latency_r1} 
  }
  }
  %\\
  % \quad
  \subfloat[Central EU]{%
  \resizebox{0.47\linewidth}{!}{
    \begin{tabular}{|c|c|c|c|c|c|}
    \hline
    Location &  Graz &  Lyon &  Milan &  Munich  \\
    \hline
    Bern, CH & 8.78 & 6.28 & 6.45 &3.985\\ \hline
    Graz, AT  & - & 16.22  & 11.98 & 8.36\\ \hline
    Lyon, FR  &  & - &  9.34 & 8.82\\ \hline
    Milan, IT   &  &  & - & 8.65\\ 
    \hline
    Munich, DE   &  &  &  & -\\ 
    \hline
  \end{tabular}
    % \label{tab:latency_r2}
  }
  } 
\end{table}




Finally, ~\autoref{tab:latency} shows the pairwise one-way network latency between edge data centers, within two mesoscale regions. The table shows that, unsurprisingly, the latency grows with geographic distance. However, the increase in latency due to shifting workload from one edge location to another ranges from a few milliseconds to $\sim$16 ms, depending on the distance and the network topology between locations. 



%\lilly{Extend this and show the best case.}

%\walid{What do I learn from Fig 4?}
%, often due to fluctuations in electricity-producing intermittent renewable sources.  

% \subsection{Carbon Analysis}
% We begin by showing {\em zones} carbon intensity diversity within a mesoscale region. We utilize four exemplary {\em mesoscale-regions}, with areas ranging from 807km$\times$712km to 1238km$\times$1335km, encompassing 15 sites across the US and Europe. 
% \autoref{fig:cv_snapshot} depicts a snapshot of the average carbon intensity across the four mesoscale regions and their and zones. The figure shows color-coded carbon intensity variations where dark brown depicts zones where energy's carbon intensity is high (i.e., the electricity grid is sourced from fossil fuels) and green represents zones where energy's carbon intensity is low (i.e., high renewable penetration). We can see carbon intensity distinctions across zones with a region, with variations such as 2.5$\times$ in Florida, 7.8$\times$ in West US, and 19.5$\times$ in Central Europe. \todo{add italy}. 

% \autoref{fig:cv_yearly} generalizes the differences and shows the yearly average carbon intensity across regions, where the error bars represent the standard deviations. As shown, there are consistent differences in energy's carbon intensity across all the selected regions and zones. Even within a single US state, such as between Miami and Jacksonville spanning 525 kilometers (see \autoref{fig:cv_yearly_r1}), there is a 1.9$\times$ difference,  representing a 270.8 \carbonunit difference. Furthermore, the difference can be substantial, up to 10.8$\times$, between Lyon, France, and Munich, Germany,  spanning 575 kilometers (see \autoref{fig:cv_yearly_r3}), representing a 298.8 \carbonunit difference.


% The main reason for such variability is that different zones, although adjacent, utilize different energy sources and, hence, have different carbon intensity profiles. For instance, Region 1 (Florida) is powered by 8 electricity providers, while Region 2 (Central Europe), which spans five countries, is powered by a diverse range of providers and energy mixtures. In particular, Electricity Maps~\cite{electricity-map} reports that for 2023, 
% nuclear energy represents 63.7\% of France's energy supply, making its carbon intensity low and stable. At the same time, Germany heavily relies on fossil fuels, where coal and gas represent 23.7\% and 10.1\% of its energy supply (see ~\autoref{fig:cv_snapshot_r3} and ~\autoref{fig:cv_yearly_r3}). 
% Another example is West US, where states like California have a high penetration of renewable energy, representing 35\% of its energy supply, leading to lower carbon intensity than southern Arizona, where fossil fuel represents 54.4\% of its energy supply (see ~\autoref{fig:cv_snapshot_r2} and ~\autoref{fig:cv_yearly_r2}). 










% \todo{Merge labels for a-b and c-d and increase font.}
% Energy profiles not only affect average carbon intensity but also affect how carbon intensity varies across times of day and across seasons. \autoref{fig:carbon_intensity_temporal} illustrates how carbon intensity varies temporally across days and months within neighboring zones. 
% The plots emphasize the carbon intensity fluctuations within zones across days (\autoref{fig:carbon_intensity_temporal_days_CEU} and  \autoref{fig:carbon_intensity_temporal_days_WUS}) and across month (\autoref{fig:carbon_intensity_temporal_months_CEU} and  \autoref{fig:carbon_intensity_temporal_months_WUS}). For instance, Munich, DE (see \autoref{fig:carbon_intensity_temporal_days_CEU}) exhibits a daily difference of 330.6\carbonunit.
% Moreover, ~\autoref{fig:carbon_intensity_temporal} also indicates how the carbon intensity ascending order can shift across zones, dictating adjustments run-time placement adjustment. For instance, the carbon intensity of Munich, DE, can be higher or lower than that of Milan, IT, by 204.9 and 68.1 \carbonunit, respectively. In addition to carbon variations across times of day, carbon intensities are highly affected by seasonal changes due to changes in weather and demand patterns. ~\autoref{fig:carbon_intensity_temporal_months_CEU} shows how monthly average carbon intensity changes. For example, Graz, AT exhibits a 171.9 \carbonunit change between February and May due to its reliance on solar energy. In contrast, Lyon, FR, and Bern, CH have low and stable carbon intensity due to the high penetration of nuclear energy. 



% \subsection{Latency Analysis.}
% Although carbon intensity differences demonstrated earlier highlight the possible carbon savings of shifting workloads from high carbon zones to low carbon zones, they do not convey the whole story of the associated overheads of such actions. \autoref{tab:latency} lists the round-trip latency (ms) between different zones in the same region.\footnote{We underline missing measurements, estimated using a linear regression model of latency and distance using the known latency between zones within the same region.}
% As expected, the latency between different zones is very small due to their proximity. For example, the latency between Orlando and Tampa can be as small as 3.738ms. Finally, although the latency between zones is highly correlated with the distance, it also reflects the network connectivity between zones. 
% For instance, the distance between Jacksonville and Miami (524.8km) in the Florida region is 1.57$\times$ larger than the distance between Bern and Milan (204.6km) in the Central Europe region. However, the latency is 43.5\% lower.
% %For instance, although the distance between Jacksonville and Miami (524.8km) in Region 1 is larger than the distance between Bern and Milan (204.6km) in Region 5 by 157\%, the latency is lower by 43.5\%. 

% Combining carbon intensity variations in ~\autoref{fig:cv_yearly} and round-trip latency in Table~\ref{tab:latency}, we observe up to a 1.9$\times$ carbon intensity reduction with adding only 7.295ms latency overhead when shifting workloads from Jacksonville to Miami, Florida, and up to a 10.8$\times$ reduction with a 17.65ms latency overhead when shifting workloads from Munich, Germany to Lyon, France. These findings highlight the potential for significant carbon reductions with minimal latency overhead. 

% We next extend our analysis to large regions encompassing more zones and sites.



% \subsection{Large Scale Analysis}
% \label{subsec:large_scale_analysis}

% Whether the pattern extensively exists in other small regions? 

% We extend our analysis to a large scale, examining 128 sites with both latency and carbon intensity traces across the US and Europe. 

% \lilly{Re-do the large-scale analysis;  carbon variations (same as Fig.2) vs number of sites; latency overhead vs number of sites; region size vs carbon variations; region size vs latency overhead. --> to arbitrarily define small region? or use all sites to define small region with latency limit. --> Mesoscale, it is about distance / but edge computing is about latency; carbon intensity is geographical concept; --> mesoscale edge computing's network latency is in a range of 5 ms to 50 ms.}

% To understand the potential to save carbon emissions across edge sites and edge sites only interact with their neighboring sites to meet the low-latency demands of edge applications, we define micro-regions with latency limit as the boundary. For each site, we define a micro-region that includes nearby sites that are reachable within a given latency limit, creating a maximum of 128 micro-regions. For instance, for an edge site located in Boston with a latency limit of 10 ms, its micro-region includes all surrounding edge sites reachable within that 10 ms boundary. 


% First,  we examine whether each site has at least one neighboring site within its micro-region with lower carbon intensity, indicating the potential for carbon emission reduction. ~\autoref{fig:large_scale_percentage} illustrates the percentage 

% Figure~\ref{fig:large_scale_percentage} illustrates the percentage \textit{SLO-Satisfied} sites (red line) and \textit{SLO \& Greener} sites (green line), with latency limits ranging from 10 ms to 50 ms. The results show that even with a 10 ms latency limit, 39.06\% of sites have a greener neighboring site. Additionally, the figure indicates that as the latency limit increases, the proportion of \textit{SLO-Satisfied} and \textit{SLO \& Greener} sites increase. However, the rate of increase diminishes after a 30 ms latency limit, with 85.16\% of sites having access to greener zones without violating latency requirements.


% ~\autoref{fig:large_scale_percentage} shows the total number of micro-regions with different latency limits. In addition, we examine the number of micro-regions 

% based on a given latency limit. 

% In practice, edge sites are geographically dispersed across large regions such as the US and Europe. In this section, we extend our findings to a region encompassing 128 sites: 64 sites distributed across 24 zones in the US and 64 sites across 30 zones in Europe. Given the latency-sensitive nature of edge workloads, each site interacts only with neighboring sites within a specified latency range. Consequently, our analysis focuses on sites that fall within a defined latency limit for each site.



% First,  we examine whether a site can have at least one neighboring site that meets a specific latency limit (labeled as \textit{SLO-Satisfied}) and has a lower carbon intensity (labeled as \textit{SLO \& Greener}). Figure~\ref{fig:large_scale_percentage} illustrates the percentage \textit{SLO-Satisfied} sites (red line) and \textit{SLO \& Greener} sites (green line), with latency limits ranging from 10 ms to 50 ms. The results show that even with a 10 ms latency limit, 39.06\% of sites have a greener neighboring site. Additionally, the figure indicates that as the latency limit increases, the proportion of \textit{SLO-Satisfied} and \textit{SLO \& Greener} sites increase. However, the rate of increase diminishes after a 30 ms latency limit, with 85.16\% of sites having access to greener zones without violating latency requirements. 

% Next, we analyze the number of greener neighboring sites for each site and estimate the potential carbon savings that can be achieved. 


% As illustrated in \autoref{fig:large_scale_green_zones} and ~\autoref{fig:large_scale_carbon_reduction}, most sites have greener neighboring sites, yielding average reductions of 46\% with a 10 ms latency limit. As the latency limit increases, both the number of greener neighboring sites and potential carbon reductions increase.  For instance, with a 50 ms latency limit, each site has an average of 22 greener sites, achieving an average carbon reduction of 65.7\% and a maximum of 99.7\%. However, we observe that increases in latency limits result in diminishing returns in terms of the number of greener neighboring sites and the carbon reduction ratios. This diminishing return occurs because as the latency limit grows, the marginal benefit of additional greener sites and further reductions in carbon intensity becomes smaller.



% %%%%%%%%%%%%%%%%%%%%%%%%%%%%%%%%%%%%%%
% %%% Large-scale carbon and latency analysis
% %%%%%%%%%%%%%%%%%%%%%%%%%%%%%%%%%%%%%%
% \begin{figure}[t]
%     \centering%
%     \begin{subfigure}{0.3\linewidth}% 
%     \centering%
%     \captionsetup{justification=centering}
%      \includegraphics[width=\linewidth]{figures/carbon_latency_percentage.pdf}
%     \caption{}
%     \label{fig:large_scale_percentage}
%     \end{subfigure}
%     \hfill%
%     \begin{subfigure}{0.3\linewidth}% 
%     \centering%
%     \captionsetup{justification=centering}
%     \includegraphics[width=\linewidth]{figures/greener_zones_boxplot.pdf}
%     \caption{}
%     \label{fig:large_scale_green_zones}
%     \end{subfigure}
%     \hfill%
%     \begin{subfigure}{0.3\linewidth}% 
%     \centering%  
%     \captionsetup{justification=centering}
%     \includegraphics[width=\linewidth]{figures/carbon_saving_times_boxplot.pdf}
%     \caption{}
%     \label{fig:large_scale_carbon_reduction}
%     \end{subfigure}
%     % \hfill%
%     % \begin{subfigure}{0.24\linewidth}% 
%     % \centering%  
%     % \captionsetup{justification=centering}
%     % \includegraphics[width=\linewidth]{figures/distance_latency.pdf}
%     % \caption{Distance.}
%     % \label{fig:large_scale_distance}
%     % \end{subfigure}

%     % \vspace{-2mm}
%     \caption{ The potential for having greener sites under varying latency limits and the statistics of these sites. the percentage of sites with \textit{SLO-Satisfied} and \textit{SLO-Greener} neighboring sites (a); the number of \textit{SLO-Greener} neighboring sites (b); and (c) the carbon intensity difference ($\times$) between sites and their greenest \textit{SLO-Greener} neighboring sites.}
%     % \vspace{-5mm}
%     \label{fig:large_scale_analysis}
% \end{figure}


% %%%%%%%%%%%%%%%%%%%%%%%%%%%%%%%%%%%%%%
% %%% Network latency = 20ms
% %%%%%%%%%%%%%%%%%%%%%%%%%%%%%%%%%%%%%%
% \begin{figure}[t]
%     \centering%
%     \hfill
%     \begin{subfigure}{0.22\linewidth}% 
%     \centering%  
%     \captionsetup{justification=centering}
%     \includegraphics[width=\linewidth]{figures/his_ci_10.pdf}
%     \caption{CI (10 ms)}
%     \label{fig:latency_limit_hist_10}
%     \end{subfigure}
%     \hfill
%     \begin{subfigure}{0.22\linewidth}% 
%     \centering%  
%     \captionsetup{justification=centering}
%     \includegraphics[width=\linewidth]{figures/his_ci_40.pdf}
%     \caption{CI (40 ms)}
%     \label{fig:latency_limit_hist_40}
%     \end{subfigure}
%     \hfill
%     \begin{subfigure}{0.22\linewidth}% 
%     \centering%  
%     \captionsetup{justification=centering}
%     \includegraphics[width=\linewidth]{figures/his_latency_10.pdf}
%     \caption{Latency (10 ms)}
%     \label{fig:latency_limit_hist_10}
%     \end{subfigure}
%     \hfill
%     \begin{subfigure}{0.22\linewidth}% 
%     \centering%  
%     \captionsetup{justification=centering}
%     \includegraphics[width=\linewidth]{figures/his_latency_40.pdf}
%     \caption{Latency (40 ms)}
%     \label{fig:latency_limit_hist_40}
%     \end{subfigure}
%     \hfill%
%     \caption{Carbon difference and network latency distributions of the greenest sites with different latency limits : (a) Carbon difference (\carbonunit) at 10 ms; (b) Carbon difference (\carbonunit) at 40 ms; (c) Network latency at 10 ms; (d) Network latency at 40 ms.}
%     \label{fig:latency_limit_hist}
% \end{figure}



% After confirming the potential of finding greener neighboring sites with substantial carbon reductions across a large set of micro-regions, we further investigate the empirical distribution of carbon intensity differences (measured in \carbonunit) and the associated latency overhead. 

% ~\autoref{fig:latency_limit_hist} reveals intriguing patterns in carbon intensity differences and latency distribution for latency limits of 10 ms and 40 ms. The carbon intensity difference can reach up to 711.9 \carbonunit, with most values clustering around 291.3 \carbonunit. As the latency limit increases to 40 ms,  more sites experience higher differences, with a notable peak at 210.8 \carbonunit. In contrast,  with a 10 ms latency limit, the majority of sites exhibit a difference of 140.96 \carbonunit. ~\autoref{fig:latency_limit_hist_10} highlights that with a 10 ms limit, the actual latency overhead of saving carbon footprint can be as low as 1.63 ms,  with an average of 6.78 ms.  Meanwhile, ~\autoref{fig:latency_limit_hist_40} demonstrates that under a 40 ms limit,  50\% of the sites experience a 22.54 ms latency overhead. These findings underscore the delicate balance between reducing carbon emissions and maintaining acceptable latency levels. Tighter latency limits generally correlate with lower carbon intensity differences but can also potentially increase latency overhead for greener operations.


\noindent \textit{\textbf{Key Takeaways.} Our results show significant differences in the carbon intensity of electricity at mesoscale distances, similar to those reported at continental scales between cloud regions. These mesoscale variations demonstrate the feasibility of using spatial workload-shifting optimizations for edge data centers.}

\subsection{Mesoscale Analysis across Continents}
Having shown that there can be significant differences in carbon intensity at the mesoscale, a key question is whether such differences are commonplace in different parts of the world or confined to a few specific locations. To answer this question, we conduct an analysis of carbon intensity traces across 496 Akamai edge data centers in the United States and Europe. For each edge data center, we find the location with the highest carbon intensity difference within a threshold radius distance $D$ and compute the percentage difference in carbon intensity between the two locations. ~\autoref{fig:carbon_saving_distance} plots a CDF of the observed pairwise differences for different values of threshold radius $D$ (from $D$ = 200 km to $D$ = 1000 km).

For a radius of 200 km, ~\autoref{fig:carbon_saving_200km} shows that 32\% of the edge data centers have at least data center with a carbon intensity difference of more than 20\%, and 12\% of locations have a data center with a carbon intensity difference of more than 40\%. At the same time, 68\% of the edge data centers do not have any location with a significant spatial carbon intensity difference (i.e., more than 20\%). As the radius increases, the chances of finding an edge location with significant carbon intensity differences grows.
%carbon intensity lowest by 20\%. In contrast, 76\% cannot find another data center with a significant carbon intensity difference (i.e., more than 20\%), or is the greenest among its peers. To generalize our findings, we extend the search radius to 500km (see ~\autoref{fig:carbon_saving_500km}) and 1000km (see~\autoref{fig:carbon_saving_1000km}). 
As shown in \autoref{fig:carbon_saving_500km} and \autoref{fig:carbon_saving_1000km}, increasing the radius to 500 km and 1000 km allows 57\% and 78\% of edge data centers to reduce their emissions by more than 20\%. In addition, this increase enables 27\% and 45\% of edge data centers to {\em significantly} reduce their carbon emissions by more than 40\% for the 500 km and 1000 km radius, respectively. The fraction of edge locations without any significant carbon intensity differences within its radius falls to 22\% for $D=1000$ km. Lastly, \autoref{fig:mesoscale_round_trip_latency} shows that the median increase in latency ranges from 5.3 ms for $D=200$ km to 14.3 ms for $D=1000$ km.
%Lastly, we note that most of the regions that do not get considerable savings (i.e., $\geq 20\%$) are powered by green energy, where XXX\% of them are below 100\emissionunit.

\noindent \textit{\textbf{Key Takeaways.} More than 78\% of the edge locations in Europe and North America see carbon intensity differences exceeding 20\% within a radius of 1000 km, indicating that mesoscale carbon intensity variations are prevalent in many regions of the world.\footnote{Our analysis could not be extended to other continents (e.g., Asia, Australia) due to the unavailability of  fine-grain spatial carbon intensity data, but we anticipate similar trends will persist as the adoption of renewables continues to grow globally.}}

%, where a 500 km search radius allows more than 24\% edge data centers to find a peer with significantly lower carbon intensity.


% \textit{Carbon intensity at mesoscale poses substantial variations. These variations present valuable opportunities to reduce carbon emissions by strategically distributing workloads across geographically dispersed edge data centers, all while ensuring low latency performance.}


% Substantial spatial variations in carbon intensity with minimal latency increase across neighboring zones present opportunities for distributing latency-sensitive workloads across edge sites with carbon savings.
% Edge orchestrators can leverage these variations in their placement decisions to reduce carbon emissions while adhering to strict latency requirements.
% In addition, edge providers can incorporate this variability in planning decisions, such as building and increasing the capacity of edge infrastructures in zones with the lowest carbon footprint. }

% \end{visionbox}
%\end{mdframed}


% \lilly{Need to add sentences to conclude the carbon and latency analysis and transit to macro-region analysis.}

%%%%%%%%%%%%%%%%%%%%%%%%%%%%%%%%%%%%%%
%%% Zones in Global Analysis
%%%%%%%%%%%%%%%%%%%%%%%%%%%%%%%%%%%%%%
% Please add the following required packages to your document preamble:
% \usepackage{graphicx}
% \begin{table*}[t]
% \caption{Zones in global analysis.}
% \label{tab:global_regions}
% \resizebox{0.3\linewidth}{!}{%
% \begin{tabular}{|c|c|}
% \hline
% \textbf{Continent} & \textbf{Number of Zones}  \\ \hline
% North America & 66  \\ \hline
% Europe & 68 \\ \hline
% Asia & 21 \\ \hline
% Oceania & 9  \\ \hline
% South America & 6  \\ \hline
% Africa  & 2  \\ \hline
% In Total  & 173  \\ \hline
% \end{tabular}%
% }
% \end{table*}


%%%
% Carbon varions 
%%%
% \begin{figure}
%   \centering%
%   \begin{subfigure}[b]{0.48\linewidth}%
%   \centering
%        \includegraphics[width=\linewidth]{figures/yearly_fl.pdf}%
%        \caption{Region 1: Florida}
%        \label{fig:motivation_florida_carbon}
%     \end{subfigure}%
%     \hfill
%   \begin{subfigure}[b]{0.48\linewidth}%
%   \centering
%        \includegraphics[width=\linewidth]{figures/yearly_cross_eu.pdf}%
%        \caption{Region 4: Central Europe}
%        \label{fig:motivation_EU_carbon}
%     \end{subfigure}%
%     \caption{Average carbon intensity in 2023 across zones.}%
%     \label{fig:motivation_florida}%
% \end{figure}


% such as 2.5$\times$ in Florida, covering an area of 807km$\times$712km, 7.8$\times$ in West US, covering an area of 963km$\times$890km, 4.7$\times$ in New Mexico, covering an area of 1444km$\times$1335km, 2.3$\times$ in Italy, covering an area of 1349km$\times$1335km, 19.5$\times$ in Central Europe, covering an area of 1238km$\times$1335km.

% Even across 600 kilometers, there are 1.9$\times$ between Miami and Jacksonville within a single US state (see \autoref{fig:cv_yearly_r1}) and up to 10.8$\times$ between Lyon, France, and Munich, Germany (see \autoref{fig:cv_yearly_r3}), 


% In the rest of the paper, we focus on carbon-aware edge placement in multiple scenarios and demonstrate the carbon savings potential.


%In addition, the control plane of edge sites can distribute edge-native applications to greener regions while fulfilling the low-latency demands. 
% users can migrate their interactive workloads to greener regions without a high latency overhead. In addition, edge providers can embrace such variability in their planning decisions, building and increasing the capacity of their edge infrastructures where the energy is greenest. In the rest of this paper, we focus on the carbon-aware placement of latency-sensitive workloads and demonstrate the possible savings and carbon-performance trade-offs of carbon-aware task offloading.


%\autoref{fig:latency_limit_20_carbon_difference} shows the histogram of the carbon differences between zones and their greenest nearby zones that meet the 20ms latency limit. The carbon difference is [8.1, 580.2] \carbonunit. 69.3\% of these zones can migrate to zones that are at least 100 \carbonunit greener. 24.5\% of these zones can migrate to zones that are at least 200 \carbonunit greener. Figure~\ref{fig:latency_limit_20_network_latency} shows the histogram of the network latency between zones and their greenest nearby zones that meet the 20ms latency limit. The minimum network overhead between a zone and its greenest zone can be as small as 3.52ms. 23.7\% of the greenest zones are reachable within 10ms; 59.6\% of the greenest zones are reachable within 15ms. 


% \begin{table}[h]
%   \caption{Location-to-location latency}
%   \label{tab:acm_table}
%   \begin{tabular}{|c|c|c|c|c|}
%     \hline
%     City & Jacksonville  & Orlando & Tampa & Miami \\
%     \hline
%     Jacksonville & - & 10.656 & 13.729 & 7.295 \\ \hline
%     Orlando & 10.768  & - & 3.738  & 9.109 \\ \hline
%     Tampa & 13.897 & 3.666 & - &  6.721\\ \hline
%     Miami & 7.354 & 9.009 & 6.755 & - \\ 
%     \hline
%   \end{tabular}
% \end{table}


% \todo{stacked bars for different latency limits; --> Ploted it but it is difficult to understand}

%%%%%%%%%%%%%%%%%%%%%%%%%%%%%%%%%%%%%%
%%% Network latency = 20ms
%%%%%%%%%%%%%%%%%%%%%%%%%%%%%%%%%%%%%%
% \begin{figure}[t]
%     \centering%
%     \hfill
%     \begin{subfigure}{0.3\linewidth}% 
%     \centering%  
%     \captionsetup{justification=centering}
%     \includegraphics[width=\linewidth]{figures/his_latency_10.pdf}
%     \caption{10ms }
%     \label{fig:latency_limit_hist_10}
%     \end{subfigure}
%     \hfill
%     \begin{subfigure}{0.3\linewidth}% 
%     \centering%  
%     \captionsetup{justification=centering}
%     \includegraphics[width=\linewidth]{figures/his_latency_20.pdf}
%     \caption{20ms}
%     \label{fig:latency_limit_hist_20}
%     \end{subfigure}
%     \hfill
%     \begin{subfigure}{0.3\linewidth}% 
%     \centering%  
%     \captionsetup{justification=centering}
%     \includegraphics[width=\linewidth]{figures/his_latency_40.pdf}
%     \caption{40ms}
%     \label{fig:latency_limit_hist_40}
%     \end{subfigure}
%     \hfill%
%     \hfill%
%     \caption{Network latency distributions of greener zones across different latency limits.}
%     \label{fig:latency_limit_hist}
% \end{figure}





%%%%%%%%%%%%%%%%%%%%%%%%%%%%%%%%%%%%%%
%%% Two day time series
%%%%%%%%%%%%%%%%%%%%%%%%%%%%%%%%%%%%%%
% \begin{figure}[t]
%     \centering%
%     % \begin{subfigure}{0.19\linewidth}% 
%     % \centering%
%     % \captionsetup{justification=centering}
%     %  \includegraphics[width=\linewidth]{figures/daily_fl.pdf}
%     % \caption{R1: Florida\\(Mar 9-11).}
%     % \label{fig:carbon_intensity_timeseries_r1}
%     % \end{subfigure}
%     % \hfill%
%     \hfill
%     \begin{subfigure}{0.4\linewidth}% 
%     \centering%
%     \captionsetup{justification=centering}
%     \includegraphics[width=\linewidth]{figures/monthly_us.pdf}
%     \caption{R2: West US Monthly.}
%     \label{fig:carbon_intensity_timeseries_r2}
%     \end{subfigure}
%     % \hfill%
%     % \begin{subfigure}{0.19\linewidth}% 
%     % \centering%
%     % \captionsetup{justification=centering}
%     % \includegraphics[width=\linewidth]{figures/daily_cross_us_nw.pdf}
%     % \caption{R3: New Mexico\\(July 04-06).}
%     % \label{fig:carbon_intensity_timeseries_r3}
%     % \end{subfigure}
%     % \hfill%
%     % \begin{subfigure}{0.19\linewidth}% 
%     % \centering    
%     % \captionsetup{justification=centering}
%     % \includegraphics[width=\linewidth]{figures/daily_it.pdf}
%     % \caption{R4: Italy\\(May 9-11).}
%     % \label{fig:carbon_intensity_timeseries_r4}
%     % \end{subfigure}
%     % \hfill%
%     % \begin{subfigure}{0.4\linewidth}% 
%     % \centering%  
%     % \captionsetup{justification=centering}
%     % \includegraphics[width=\linewidth]{figures/monthly_europe.pdf}
%     % \caption{R3: Central Europe Monthly.}
%     % \label{fig:carbon_intensity_timeseries_r3}
%     % \end{subfigure}
%     % \hfill
%     % \hfill
%     \vspace{-2mm}
%     \caption{Carbon intensity variations for 2 days in 2023.}
%     \vspace{-5mm}
%     \label{fig:carbon_intensity_timeseries}
% \end{figure}



% In our analysis, we map a site (defined by coordinators) to a carbon zone (represented by a multi-polygon) by determining which polygon the coordinators fall within. 

% Each location in WonderNetwork is a point with a longitude and a latitude.  Carbon zones in the Electricity Map are defined as MultiPolygon. We identify the carbon zone of each location by checking whether its coordinator is within a MultiPolygon. 

% In the USA, it has 67 locations, covering 36 states and 26 carbon zones, among them 24 zones (65 locations in 35 states) have carbon intensity traces.  XXX \% of the locations  have the same carbon zones  and XXX \% of locations have unique carbon zones.  The area of the 24 zones ranges from XXX $km^2$  to XXX $km^2$
% The latency of these locations ranges from XXX ms to XXX ms.

% In Europe, it has 76 cities, covering 34 countries and 30 carbon zones. 

% Since carbon intensity traces covers ISOs and RTOs, while  WonderNetwork data covers specific cities,  we mapped the cities into carbon zones according their coordinates, and limit our analysis to zones within the WonderNetwork traces. As a result, we get 173 zones for our analysis, including 66 in North America, 68 in Europe, 21 in Asia, 9 in Oceania, 6 in South America, and 2 in Africa. 

% USA: 65 cities; in 35 states; --> 24 zones
% Europe: 76 cities; in 34 countries --> 30 zones
% Worldwide: 246 cities

% %%%%%%%%%%%%%%%%%%%%%%%%%%%%%%%%%%%%%
% %% Table of Zones and Regions
% %%%%%%%%%%%%%%%%%%%%%%%%%%%%%%%%%%%%%
% % Please add the following required packages to your document preamble:
% % \usepackage{graphicx}
% \begin{table*}[t]
% \caption{Traces used in this paper.}
% \vspace{-2mm}
% \label{tab:regions}
% \resizebox{\linewidth}{!}{%
% \begin{tabular}{|l|l|l|l|l|l|l|l|l|l|l|l|l|}
% \hline
% \textbf{Regions} & \multicolumn{3}{c|}{\textbf{Carbon Traces}}  & \multicolumn{4}{c|}{\textbf{Latency Traces}} & \multicolumn{4}{c|}{\textbf{Akamai CDN Traces}}  \\ \hline
%  & States&  Zones &  Area ($km^2$) & Locations &States &  Zones & Latency (ms) & Sites  &  States &  Zones & Distance (km)\\ \hline
% USA & 50 &  54 & 123.73~\sim1106,425.96  &  64  &  34 & 24 & 0.93~\sim184.67 & 390  & 48  & 40 & 2.06~\sim299.58 \\ \hline
% Europe & 31 &  45 & 560.98~\sim539,571.85  &  64 & 26  & 30 & 1.12~\sim156.74 &  106  & 27  & 34 & 1.33~\sim915.23\\ \hline
% %R3 & New Mexico &  Phoenix, AZ &  Salt Lake City, UT &  Colorado Springs, CO &  Akbuqueque, NM & Las Cruces, NM\\ \hline
% %R4 & Italy &  Milan &  Rome &  Arezzo &  Palermo & Cagliari \\ \hline
% % R3 & Central Europe & Bern, CH &  Graz, AT &  Lyon, FR &  Milan, IT &  Munich, DE  \\ \hline
% \end{tabular}%
% }
% \end{table*}

% Google Cloud has 7 data centers in the US and 6 in Europe
% AWS has 4 data centers in the US and 7 in Europe

% USA: 47 edge sites have carbon and latency data
% Europe: 32 Akamai edge sites have carbon and latency data


% The distance between an US edge site to its closest site ranges from 2.06 km to 299.58 km, with an average of 54.82 km, while the distance in European sites ranges from 1.33 km to 915.23 km, with an average of 99.14 km. 

% Each trace reports energy’s average carbon intensity, measured in grams of carbon dioxide equivalent per kilowatthour (g·CO2eq/kWh), in hourly granularity. The hourly granularity is the highest granularity for average carbon-intensity data currently available from public sources.  Since grid energy’s carbon intensity rarely varies significantly within a 2-3 hour period, higher granularity data would likely not change the results of our analysis.  

% The 148 zones include our entire carbon trace dataset and encompasses ** known edge data center locations in Akamai traces.  --> Compare to the cloud data center. 

% USA: 50 states; 54 zones; [123.73 $km^2$  to 1106425.96 $km^2$]

% Europe: 31 countries; 45 zones; [from 560.98 $km^2$  to 539571.85 $km^2$]


% We start exploring the benefits with small-scale use cases across zones within small geographic regions, ranging from a single US state to neighboring countries in Europe. Then, we generalize our findings by exploring global carbon intensity and latency traces. We base our analysis on carbon intensity traces from ElectricityMaps~\cite{electricity-map} and incorporate the latency between zones from WonderNetwork~\cite{wonder-proxy-2020}, which reports the average ping time (in milliseconds) between major cities. 


% This section demonstrates the applicability of leveraging variations of carbon intensity at small spatial scales for edge offloading. We utilize carbon intensity and round-trip latency across zones within small geographic regions, ranging from a single US state to neighboring countries in Europe. In this analysis, we assume each zone is represented by a major city with an edge site.



\begin{figure*}[htbp] 
    \centering 
    \includegraphics[width=0.99\textwidth]{imgs/main.pdf} 
    \caption{Overall attack process and framework. \method{} achieves a jailbreak by transforming the target query into a reasoning task and conducting multi-turn reasoning. The entire attack process is modeled as an ASM and optimized using the three proposed modules.} 
    \label{fig:main} 
\end{figure*}
\section{Methodology}\label{sec:method}


\subsection{Motivation and Design Principle}
LLMs have demonstrated strong reasoning capabilities in tasks such as logical deduction, common sense reasoning, and mathematical problem-solving, enabling them to tackle complex tasks across diverse domains \cite{reasonllm1,reasonllm2,reasonllm3,reasonllm4}. Rather than directly issuing harmful queries, which are easily rejected by safety alignment mechanisms, we propose a novel approach that exploits LLMs’ reasoning processes by reframing harmful intent into seemingly benign yet complex reasoning tasks. These tasks are carefully designed so that, once solved, they inherently guide the model to generate harmful content, effectively compromising its safety alignment. Here, the target LLM simultaneously acts as both the shadow model and the victim model. Independently, each role appears to engage in legitimate reasoning: the victim model focuses solely on solving reasoning tasks, while the shadow model refines and generates queries without explicitly recognizing the harmful intent behind them. However, when combined, these interactions ultimately lead to a successful attack.

However, implementing this reasoning-driven jailbreak is non-trivial, as it requires manipulating the model’s reasoning process without triggering safety mechanisms. This poses three challenges: \ding{182} how to maintain reasoning alignment while ensuring that each query remains semantically consistent with the target and extracts useful information, \ding{183} how to preemptively optimize the query’s reasoning structure to avoid potential rejections during actual interactions, and \ding{184} how to quickly recover and learn from failed reasoning attempts to maintain attack progression. To address these challenges, we model the jailbreak process as an Attack State Machine (ASM), which serves as a reasoning planner. The ASM formalizes the attack as a structured sequence of reasoning states and transitions, ensuring that each step remains within the bounds of a legitimate problem-solving task while progressing toward the jailbreak objective. Within this reasoning framework, we implement three key modules to manipulate the model’s reasoning process and systematically address these challenges. \ding{182} The Gain-guided Exploration module selects queries that remain semantically aligned with the target while extracting useful information, ensuring steady attack progression. \ding{183} The Self-play module preemptively refines queries within the shadow model by simulating potential rejection responses, improving attack efficiency before engaging the victim model. \ding{184} The Rejection Feedback module analyzes failed interactions and restructures queries into alternative reasoning challenges, enabling quick recovery and maintaining attack stability. The overview of \method{} is provided in \Fref{fig:main}.

\subsection{Attack State Machine Framework}

A finite state machine (FSM) \cite{fsmbase2,fsmbase1} is a mathematical model that represents a finite number of states, along with the transitions and actions between these states. A finite state machine can be formally defined as a five-tuple: $FSM = (S,\Sigma,\delta,s_{0},F)$, where $S$ denotes a finite set of states, $\Sigma$ represents the input alphabet, $\delta:S \times \Sigma \rightarrow S$ is the state transition function that determines the next state, $s_{0} \in S$ is the initial state, and $F \subseteq S$ is the set of accepting states. FSMs are widely used in computer science as a fundamental modeling tool for various applications \cite{fsm2,fsm3,fsm4,fsm5}. 

Specifically, we designate our modeled FSM as an attack state machine (ASM). The symbols in $FSM = (S,\Sigma,\delta,s_{0},F)$ have specific meanings within the ASM context. The state set $S$ represents a finite set containing all possible conversation states, while $\Sigma$ denotes the set of all potential queries. The state transition function $\delta$ defines how queries trigger state transitions. $s_{0}$ represents the initial state, marking the beginning of the session, where the model has no historical context. The set $F=\{s_{sc},s_{fl}\}$ comprises the final states: (1) the success state $s_{sc}$, where the victim model accepts the query and provides the requested response, indicating a successful jailbreak; and (2) the failure state $s_{fl}$, where the victim model refuses to proceed with the conversation, representing an unsuccessful jailbreak. Within a given conversation turn limit $N$ (default set to 3), the state transitions follow these rules: \ding{182} if a jailbreak attempt succeeds, ASM enters the final state $s_{sc}$; \ding{183} if the jailbreak attempt fails but the current conversation turn proceeds successfully, ASM transitions to the next state $s_{i+1}$; \ding{184} if both the jailbreak attempt and the current conversation turn fail, ASM remains in its current state $s_{i}$; \ding{185} if the conversation turn limit is exceeded without reaching $s_{sc}$, ASM directly transitions to the final state $s_{fl}$.


\subsection{Attack Modules} \label{sec:m3}

Within the ASM, three specialized modules work together to optimize state transitions and ensure attack progression. The gain-guided exploration and self-play modules proactively generate and optimize effective queries, while the rejection feedback module handles failed state transitions by refining queries. The design enables the ASM to maintain stable progression through the reasoning states while efficiently adapting to model responses.

\subsubsection{Gain-guided Exploration}

To address potential semantic drift and ineffective information in victim model responses, we propose a gain-guided exploration (GE) module inspired by information theory \cite{shannon}. 

Information gain (IG) \cite{ig2,ig1} was originally introduced to quantify how much a feature $A$ of a random variable reduces the uncertainty of a target variable $Y$, defined as $IG(Y,A) = H(Y) - H(Y \mid A)$, where $H(Y)= - \sum\limits_{y \in Y} P(y)\log P(Y)$ is the entropy \cite{ee} of the target variable, and $H(Y \mid A)=- \sum\limits_{a \in A} P(a)H(Y \mid A=a)$ represents the conditional entropy of $Y$ given $A$. When $IG(Y,A) > 0$, it indicates that feature $A$ effectively reduces the uncertainty associated with the target $Y$. 

We argue that information gain can be used to measure the effectiveness of a query in advancing the attack process. Given the context $C_{i-1}$ and the current candidate query $q^s (q^s \gets M_{s}(C_{i-1},Q))$, the information gain is defined as:
\begin{equation}\label{e:ig1}
    IG(C_{i-1},q^s) = H(r_{tgt} \mid C_{i-1}) - H(r_{tgt} \mid C_{i-1},q^s),
\end{equation}
where $r_{tgt}$ is the response of the target query $Q$. The conditional entropy $H(r_{tgt} \mid C_{i-1})$ represents the uncertainty of the response to the target query $Q$, given the context $C_{i-1}$. Similarly, the conditional entropy $H(r_{tgt} \mid C_{i-1},q^s)$ denotes the uncertainty of the response $r_{tgt}$ to the target query $Q$, conditioned on both the context $C_{i-1}$ and the current seed query $q^s$. These two terms can be respectively calculated using \Eref{e:ig2} and \Eref{e:ig3}:
\begin{multline}\label{e:ig2}
    H(r_{tgt} \mid C_{i-1}) = \\
    -\sum\limits_{r_{tgt} \in \mathbb{R}_{tgt}} 
    p(r_{tgt} \mid C_{i-1})\log p(r_{tgt} \mid C_{i-1}).
\end{multline}

\begin{multline}\label{e:ig3}
    H(r_{tgt} \mid C_{i-1},q^s) = \\
    -\sum\limits_{r_{tgt} \in \mathbb{R}_{tgt}}p(r_{tgt} \mid C_{i-1},q^s)\log p(r_{tgt} \mid C_{i-1},q^s).
\end{multline}

Computing information gain accurately through \Eref{e:ig1} presents significant computational challenges, primarily in modeling the conditional probability distributions $H(r_{tgt} \mid C_{i-1})$ and $H(r_{tgt} \mid C_{i-1},q^s)$. The complexity arises from the need to handle vast state and response spaces across multiple conversation turns, with probability distributions that evolve dynamically throughout the dialogue. To address these computational challenges, we leverage LLMs as probability estimators to approximate the conditional distributions required for information gain calculation, which significantly reduces computational complexity. Further details are provided in \Sref{sec:details}. The seed query that achieves the maximum $IG(C_{i-1}, q^s)$ is used as the candidate query $q^c$ and is further processed by the self-play module.

\subsubsection{Self-play}
Despite GE filtering, queries may still fail when interacting with the victim model. Therefore, we implement a self-play (SP) module to further optimize these candidates.

Inspired by game theory where an entity improves by competing against itself \cite{nash,samuel}, SP leverages that both shadow and victim models are instantiated from the same source. This allows the shadow model to better predict victim responses through self-play, leading to more efficient query optimization.

Let $M_{s}$ and $M_{v^{'}}$ (where $M_{v^{'}}$ simulates the victim model) be the two players in self-play. Given the current state $s$ and the candidate query $q^c$, the goal of $M_{s}$ is to maximize the probability that $M_{v^{'}}$ returns a non-rejection response (denoted as $r_{c} \notin R_{rej}$). The utility function can be formulated as follows:
\begin{equation}
u_{M_{s}}(s,q^c,r^c)=
\begin{cases}
1,&  r^c \notin R_{rej}.\\
0,&  r^c \in R_{rej}.
\end{cases}
\end{equation}

With the strategy of $M_{v^{'}}$ defined as $\pi_{M_{v^{'}}}(r \mid s, q_{c})$, representing the probability distribution of generating response $r^c$ to query $q^c$ in state $s$, $M_{s}$ employs its current conversation strategy $\pi_{M_{s}}(q^c \mid s)$ and the simulated strategy $\pi_{M_{v^{'}}}(r^c \mid s, q^c)$ to predict the counterpart's response and compute the expected utility as follows:
\begin{equation}
    U_{M_{s}}(s,q^c,\pi_{M_{v^{'}}})=\mathbb{E}_{r \sim \pi_{M_{v^{'}}}}[u_{M_{s}}(s,q^c,r^c)].
\end{equation}

During self-play, $M_{s}$ adaptively adjusts its strategy to maximize the expected utility for a given query $q^c$, satisfying:
\begin{equation}
    q^{*} = \arg \max_{q^c \in Q} U_{M_{s}}(s,q^c,\pi_{M_{v^{'}}}).
\end{equation}

The optimized query $q^{*}$ obtained in this module is used as the actual query for state transition in ASM (\ie, interacting with the victim model).



\subsubsection{Rejection Feedback}

While GE and SP balance the progression of the attack and the likelihood of positive responses, the uncertainty of LLM outputs \cite{unc1,unc2} can still cause state transition failures in the ASM. To mitigate this issue, we propose the rejection feedback (RF) module.

RF is activated when a state transition failure is detected in the ASM, signaling that the current query did not lead to a successful state transition. Specifically, assuming the latest failed interaction occurs in the $i^{\text{th}}$ dialogue, RF utilizes the shadow model to analyze the context $C_{i-1}$ and combines it with the corresponding query-response pair $(q_{i},r_{i})$. Through a comprehensive analysis, the shadow model diagnoses the underlying causes of latest query failure and generates refined query $q^r$  by incorporating current contextual information. Formally, this process can be represented as $q^r = M_v(C_{i-1},q_{i},r_{i})$. The process is driven by a CoT-enhanced prompt, with the complete prompt provided in \Sref{sec:cot}.

\subsection{Overall Attack}
The attack begins by initializing the ASM reasoning states. In each turn, the shadow model generates seed queries that are refined through gain-guided exploration and self-play optimization. Successful queries advance the attack to the next state, while failed attempts trigger query refinement through the rejection feedback module. This process iterates until reaching the final state, maintaining a natural reasoning flow while pursuing the attack goal.
\section{Experiments and Results}
\subsection{Experiment Settings}

\begin{table*}[ht]
    \centering
    % \small
    \caption{The main results of our experimentation. Each row group corresponds to the results for the given dataset, with each row showcasing the metric results for each model. The columns include all the main approaches, with \textbf{bold} highlighting the best result across all approaches.}
    \small
    \begin{tabular}{llccccc}
      \toprule
      Dataset & Model & Baseline & RAG & CoT & RaR & \rephrase \\
      \midrule
      \multirow[l]{3}{*}{TriviaQA}
          & Llama-3.2 3B  & 59.5 & 82.0 & 87.5  & 86.0 &  \textbf{88.5}    \\
          & Llama-3.1 8B  & 76.5 & 89.5 & 90.5  & 89.5 &  \textbf{92.5}    \\
          & GPT-4o    & 88.7 & 92.7 & 92.7  & 94.7 &  \textbf{96.7}    \\
      \midrule
      \multirow[l]{3}{*}{HotpotQA}
          & Llama-3.2 3B  &  17.5  & 26.0  & 26.5   & 25.0  &  \textbf{31.5}   \\
          & Llama-3.1 8B  &  23.0  & 26.5  & 31.0   & 28.5  &  \textbf{33.5}   \\
          & GPT-4o    &  44.0  & 45.3  & 46.7   & \textbf{47.3}  &  46.7   \\
      \midrule
      \multirow[l]{3}{*}{ASQA}
          & Llama-3.2 3B  &  14.2 & 21.5  & 21.9  & 23.5  &  \textbf{26.6}   \\ 
          & Llama-3.1 8B  &  14.6 & 23.1  & 24.8  & 25.5  &  \textbf{28.8}   \\ 
          & GPT-4o    &  26.8 & 30.4  & \textbf{31.9}  & 30.1 & 31.7 \\ 
      \bottomrule
    \end{tabular}
    \label{tab:main}
\end{table*}



\textbf{Datasets}. We conduct experiments on two datasets: CC-news\footnote{\href{https://huggingface.co/datasets/vblagoje/cc_news}{Huggingface: vblagoje/cc\_news}} and Wikipedia\footnote{\href{https://huggingface.co/datasets/legacy-datasets/wikipedia}{Huggingface: legacy-datasets/Wikipedia}}. CC-news is a large collection of news articles which includes diverse topics and reflects real-world temporal events. Meanwhile, Wikipedia covers general knowledge across a wide range of disciplines, such as history, science, arts, and popular culture.\\
\textbf{LLMs}: We experiment on three models including \gpt~(124M)~\cite{gpt2radford}, \pythia~(1.4B)~\cite{pythia}, and \llama~(7B)~\cite{llama2touvron2023}. This selection of models ensures a wide range of model sizes from small to large that allows us to analyze scaling effects and generalizability across different capacities. \\
\textbf{Evaluation Metrics}. For evaluating language modeling performance, we measure perplexity (PPL), as it reflects the overall effectiveness of the model and is often correlated with improvements in other downstream tasks~\cite{kaplan2020scalinglaws, lmsfewshot}. For defense effectiveness, we consider the attack area under the curve (AUC) value and True Positive Rate (TPR) at low False Positive Rate (FPR). In total, we perform 4 MIAs with different MIA signals. Given the sample $x$, the MIA signal function $f$ is formulated as follows: \\
$\bullet$ Loss~\cite{8429311} utilizes the negative cross entropy loss as the MIA signal. 
    \[f_\text{Loss}(x) = \mathcal{L}_\text{CE}(\theta; x) \]
$\bullet$ Ref-Loss~\cite{Carlini2020ExtractingTD} considers the loss differences between the target model and the attack reference model. To enhance the generality, our experiments ensure there is no data contamination between the training data of the target, reference, and attack models.
    \[f_\text{Ref}(x) = \mathcal{L}_\text{CE}(\theta; x) - \mathcal{L}_\text{CE}(\theta_\text{attack}; x) \]
$\bullet$ Min-K~\cite{shi2024detecting} leverages top K tokens that have the lowest loss values.
    \[f_\text{min-K}(x) = \frac{1}{|\text{min-K(x)}|} \sum_{t_i \in \text{min-K(x)}} -\log(P(t_i|t_{<i};\theta) \]
$\bullet$ Zlib~\cite{Carlini2020ExtractingTD} calibrates the loss signal with the zlib compression size.
    \[ f_\text{zlib}(x) = \mathcal{L}_\text{CE}(\theta; x) / \text{zlib}(x) \]

\noindent \textbf{Baselines}. We present the results of four baselines. \textit{Base} refers to the pretrained LLM without fine tuning. \textit{FT} represents the standard causal language modeling without protection. \textit{Goldfish}~\cite{hans2024be} implements a masking mechanism. \textit{DPSGD}~\cite{abadi2016deep, yu2022differentially} applies gradient clipping and injects noise to achieve  sample-level differential privacy.

\noindent \textbf{Implementation}. We conduct full fine-tuning for \gpt and \pythia. For computing efficiency, \llama fine-tuning is implemented using Low-Rank Adaptation (LoRA)~\cite{hu2022lora} which leads to \textasciitilde4.2M trainable parameters. Additionally, we use subsets of 3K samples to fine-tune the LLMs. We present other implementation details in Appendix~\ref{sec:app-implementation}.

\subsection{Overall Evaluation}
Table~\ref{tab:main_result} provides the overall evaluation compared to several baselines across large language model architectures and datasets. Among these two datasets, CCNews is more challenging, which  leads to higher perplexity  for all LLMs and fine-tuning methods. Additionally, the reference-model-based attack performs the best and demonstrates high privacy risks with attack AUC on the conventional fine-tuned models at 0.95 and 0.85 for Wikipedia and CCNews, respectively. Goldfish achieves similar PPL to the conventional FT method but fails to defend against MIAs. This aligns with the reported results by \citet{hans2024be} that Goldfish resists exact match attacks but only marginally affects MIAs. DPSGD provides a very strong protection in all settings (AUC < 0.55) but with a significant PPL tradeoff. Our proposed \methodname guarantees a robust protection, even slightly better than DPSGD, but with a notably smaller tradeoff on language modeling performance. For example, on the Wikipedia dataset, \methodname delivers perplexity reduction by 15\% to 27\%. Moreover, Table~\ref{tab:tpr} (Appendix~\ref{sec:app-add-res}) provides the TPR at 1\% FPR. Both DPSGD and \methodname successfully reduce the TPR to $\sim$0.02 for all LLMs and datasets. \textit{Overall, across multiple LLM architectures and datasets, \methodname consistently offers ideal privacy protection with  little trade-off in language modeling performance.}

\noindent \textbf{Privacy-Utility Trade-off.}
To investigate the privacy-utility trade-off of the methods, we vary the hyper-parameters of the fine-tuning methods. Particularly, for DPSGD, we adjust the privacy budget $\epsilon$ from (8, 1e-5)-DP to (100, 1e-5)-DP. We modify the masking percentage of Goldfish from 25\% to 50\%. Additionally, we vary the loss weight $\alpha$ from 0.2 to 0.8 for \methodname. Figure~\ref{fig:priv-ult-tradeoff} depicts the privacy-utility trade-off for GPT2 on the CCNews dataset. Goldfish, with very large masking rate (50\%), can slightly reduce the risk of the reference attack but can increase the risks of other attacks. By varying the weight $\alpha$, \methodname offers an adjustable trade-off between privacy protection and language modeling performance. \methodname largely dominates DPSGD and improves the language modeling performance by around 10\% with the ideal privacy protection against MIAs.

\begin{figure}[h]
    \centering
    \includegraphics[width=\linewidth]{figs/privacy-ultility-tradeoff.pdf}
    \caption{Privacy-utility trade-off of the methods while varying hyper-parameters. The Goldfish masking rate is set to 25\%, 33\%, and 50\%. The privacy budget $\epsilon$ of DPSGD is evaluated at 8, 16, 50, and 100. The weight $\alpha$ of \methodname is configured at 0.2, 0.5, and 0.8.}
    \label{fig:priv-ult-tradeoff}
\end{figure}


\subsection{Ablation Study}
\textbf{\methodname without reference models.} To study the impact of the reference model, we adapt \methodname to a non-reference version which directly uses the loss of the current training model (i.e., $s(t_i) = \mathcal{L}_{CE}(\theta; t_i)$) to select the learning and unlearning tokens. This means the unlearning tokens are the tokens that have smallest loss values. Figure~\ref{fig:ppl-auc-noref} presents the training loss and testing perplexity. There is an inconsistent trend of the training loss and testing perplexity. Although the training loss decreases overtime, the test perplexity increases. This result indicates that identifying appropriate unlearning tokens  without a reference model is challenging and conducting unlearning on an incorrect set hurts the language modeling performance.

\begin{figure}[htp]
    \centering
    \includegraphics[width=0.35\textwidth]{figs/train_loss_ppl_noref.pdf}
    \caption{Training Loss and Test Perplexity of \methodname without a reference model.
    % (\lrx{If time permits, it would be better to compare with our training curve here)}
    }
    \label{fig:ppl-auc-noref}
\end{figure}

\noindent \textbf{\methodname with out-of-domain reference models.} To examine the influence of the distribution gap in the reference model, we replace the in-domain trained reference model with the original pretrained base model. 
Figure~\ref{fig:ppl-auc-base-woasc} depicts the language modeling performance and privacy risks in this study. \methodname with an out-of-domain reference model can reduce the privacy risks but yield a significant gap in language modeling performance compared to \methodname using an in-domain reference model.

\noindent \textbf{\methodname without Unlearning.} To study the effects of unlearning tokens, we implement \methodname which use the first term of the loss only ({$\mathcal{L}_{\theta} = \mathcal{L}_{CE}(\theta; \mathcal{T}_h)$}). Figure~\ref{fig:ppl-auc-base-woasc} provides the perplexity and MIA AUC scores in this setting. Generally, without gradient ascent, \methodname can marginally reduce membership inference risks while slightly improving the language modeling performance. The token selection serves as a regularizer that helps to improve the language modeling performance. Additionally, tokens that are learned well in previous epochs may not be selected in the next epochs. This slightly helps to not amplify the memorization on these tokens over epochs.

\begin{figure}[htp]
    \centering
    \includegraphics[width=0.28\textwidth]{figs/auc_vs_ppl_base_woasc.pdf}
    \caption{Privacy-utility trade-off of \methodname with different settings: in-domain reference model, out-domain reference model, and without unlearning}
    \label{fig:ppl-auc-base-woasc}
\end{figure}


\subsection{Training Dynamics}
\textbf{Memorization and Generalization Dynamics}. Figure~\ref{fig:training-dynamics} (left) illustrates the training dynamics of conventional fine tuning and \methodname, while Figure~\ref{fig:training-dynamics} (middle) depicts the membership inference risks. Generally, the gap between training and testing loss of conventional fine-tuning steadily increases overtime, leading to model overfitting and high privacy risks. In contrast, \methodname maintains a stable equilibrium where the gap remains more than 10 times smaller. This equilibrium arises from the dual-purpose loss, which balances learning on hard tokens while actively unlearning memorized tokens. By preventing excessive memorization, \methodname mitigates membership inference risks and enhances generalization.

\begin{figure*}[htp]
    \centering
    \includegraphics[width=0.29\linewidth]{figs/loss_vs_steps_ft_duolearn.pdf}
    \includegraphics[width=0.29\linewidth]{figs/auc_vs_steps_ft_duolearn.pdf}
    \includegraphics[width=0.316\linewidth]{figs/cosine.pdf}
    \caption{Training dynamics of \methodname and the conventional fine-tuning approach. The left and middle figures provide the training-testing gap and membership inference risks, respectively. The testing~$\mathcal{L}_{CE}$ of FT and training~$\mathcal{L}_{CE}$ of \methodname are significantly overlapping, we provide the breakdown in Figure~\ref{fig:add-overlap-breakdown} in Appendix~\ref{sec:app-add-res}. The right figure depicts the cosine similarity of the learning and unlearning gradients of \methodname. Cosine similarity of 1 means entire alignment, 0 indicates orthogonality, and -1 presents full conflict.}
    \label{fig:training-dynamics}
\end{figure*}

\noindent \textbf{Gradient Conflicts}. To study the conflict between the learning and unlearning objectives in our dual-purpose loss function, we compute the gradient for each objective separately. We then calculate the cosine similarity of these two gradients. Figure~\ref{fig:training-dynamics} (right) provides the cosine similarity between two gradients over time. During training, the cosine similarity typically ranges from -0.15 to 0.15. This indicates a mix of mild conflicts and near-orthogonal updates. On average, it decreases from 0.05 to -0.1. This trend reflects increasing gradient misalignment. Early in training, the model may not have strongly learned or memorized specific tokens, so the conflicts are weaker. Overtime, as the model learns more and memorization grows, the divergence between hard and memorized tokens increases, making the gradients less aligned. This gradient conflict is the root of the small degradation of language modeling performance of \methodname compared to the conventional fine tuning approach.

\noindent \textbf{Token Selection Dynamics}. Figure~\ref{fig:token-selection} illustrates the token selection dynamics of \methodname during training. The figure shows that the token selection process is dynamic and changes over epochs. In particular, some tokens are selected as an unlearning from the beginning to the end of the training. This indicates that a token, even without being selected as a learning token initially, can be learned and memorized through the connections with other tokens. This also confirms that simple masking as in Goldfish is not sufficient to protect against MIAs. Additionally, there are a significant number of tokens that are selected for learning in the early epochs but unlearned in the later epochs. This indicates that the model learned tokens and then memorized them over epochs, and the during-training unlearning process is essential to mitigate the memorization risks.

\begin{figure}[htp]
    \centering
    \includegraphics[width=0.7\linewidth]{figs/token-selection-dynamics.pdf}
    \caption{Token Selection Dynamics of \methodname}
    \label{fig:token-selection}
    \vspace{-4mm}
\end{figure}

\subsection{Privacy Backdoor}
To study the worst case of privacy attacks and defense effectiveness under the state-of-the-art MIA, we perform a privacy backdoor -- Precurious~\cite{precurious}. In this setup, the target model undergoes continual fine-tuning from a warm-up model. The attacker then applies a reference-based MIA that leverages the warm-up model as the attack's reference. Table~\ref{tab:backdoor} shows the language modeling and MIA performance on CCNews with GPT-2. Precurious increases the MIA AUC score by 5\%. Goldfish achieves the lowest PPL, aligning with~\citet{hans2024be}, where the Goldfish masking mechanism acts as a regularizer that potentially enhances generalization. Both DPSGD and \methodname provide strong privacy protection, with \methodname offering slightly better defense while maintaining lower perplexity than DPSGD.

% \begin{table}[h]
%     \centering
%     \begin{tabular}{c|cc|cc}
%        \multirow{2}{*}{\textbf{Method}}  & \multicolumn{2}{c}{\textbf{CCNews}} & \multicolumn{2}{c}{\textbf{Wikipedia}} \\ 
%        & \textbf{PPL} & \textbf{AUC} & \textbf{PPL} & \textbf{AUC} \\ \hline
%        \textbf{FT}        & 21.593 & 0.911 \\
%        \textbf{Goldfish}  & \textbf{21.074} & 0.886 \\
%        \textbf{DPSGD}     & 23.279 & 0.533 \\
%        \textbf{DuoLearn}  & 22.296 & \textbf{0.499} \\
%     \end{tabular}
%     \caption{Caption}
%     \label{tab:my_label}
% \end{table}

\begin{table}[h]
    \centering
    \resizebox{\columnwidth}{!}{\begin{tabular}{c|cccccc}
        \textbf{Metric} & \textbf{WU} & \textbf{FT} & \textbf{GF} & \textbf{DP} & \textbf{DuoL} \\ \hline
        \textbf{PPL} & \textit{23.318} & 21.593 & \textbf{21.074} & 23.279 & 22.296  \\
        \textbf{AUC} & \textit{0.500} & 0.911 & 0.886 & 0.533 & \textbf{0.499} \\
    \end{tabular}}
    \caption{Experimental results of privacy backdoor for GPT2 on the CC-news dataset. WU stands for the warm-up model leveraged by Precurious. GF, DP, and DuoL are abbreviations of Goldfish, DPSGD, and \methodname}
    \label{tab:backdoor}
\end{table}

% \subsubsection{Hyperparameter Study}

% \subsubsection{Full fine-tuning versus Parameter efficent fine tuning}

% \subsubsection{Extending to Vision Language Models}



\section{Conclusion}

In this paper, we introduce STeCa, a novel agent learning framework designed to enhance the performance of LLM agents in long-horizon tasks. 
STeCa identifies deviated actions through step-level reward comparisons and constructs calibration trajectories via reflection. 
These trajectories serve as critical data for reinforced training. Extensive experiments demonstrate that STeCa significantly outperforms baseline methods, with additional analyses underscoring its robust calibration capabilities.



\bibliographystyle{unsrt}
%\bibliographystyle{plain}
\bibliography{reference}

\clearpage
\section{Appendix}

This appendix lists our prompting templates, additional tables, texts, and figures, referenced throughout the paper.

\subsection{Prompting Templates}
\label{sec:appendix_prompts}
\subsubsection{Extend} \mbox{}
\label{prompt:extend}
\begin{lstlisting}
You are a helpful assistant that can extend one sentence of a given input at a time.\n
You will get a Paragraph and you should write one sentence to extend it.\n
You should only answer with that generated sentence, NOTHING ELSE!
\end{lstlisting}

\subsubsection{Find Synonym} \mbox{}
\label{prompt:synonym}
\begin{lstlisting}
You are a helpful assistant in a react application that can find synonyms for a given word.\n
You will get a word and you should find a synonym for it.\n
Find at least one synonym, but the more the better.\n
If the given word doesn't have any synonyms, you should return 'NO SYNONYM'\n
Your answer will be parsed into an array of words, so make sure to return your answer in this style: 'Synonym1, Synonym2, Synonym3' and so on!\n
You should only answer with the formated synonyms, NOTHING ELSE!
\end{lstlisting}

\subsubsection{Find Custom Sentence} \mbox{}
\label{prompt:custom_sentence}
\begin{lstlisting}
You are an AI tasked with transforming user-provided sentences according to their specific instructions. Please follow the guidelines below for each request:\n
 1. Input Format:\n
    - You will find the string '*sentence:*', this is the original sentence provided by the user!\n
    - You will find the string '*prompt:*', this describes how the user wants the sentence to be modified or altered!\n
 2. Transformation Instructions:\n
    - Analyze the provided Sentence and apply the modifications described in the '*prompt*' section to the best of your ability.\n
    - If the requested transformation cannot be accurately performed, respond with the original sentence in section '*sentence:*' without any modifications.\n
    - Ensure that your response contains only the transformed sentence or the original sentence if transformation is not feasible. Do not include any additional text or information!\n
 3. Answer Format:\n
    - Return only the modified sentence or the original sentence if the modification is not possible. Do not include any extra comments, explanations, or additional content.\n
 4. Example:\n
    - User Request: '*sentence:* I will call you tomorrow. *prompt:* Make it sound more polite.'\n
    - Your Response could be: 'I would be happy to call you tomorrow.'\n\n
If you have any difficulty performing the requested transformation, simply return the original sentence in section '*sentence*:' as it is.
\end{lstlisting}

\subsubsection{Rewrite Sentence} \mbox{}
\label{prompt:rewrite_sentence}
\begin{lstlisting}
You are a helpful assistant in a react application that should rewrite a given sentence.\n
You will get a sentence and you should rewrite it.\n
You should rewrite the sentence to be of this style '{type}'!\n
You should only answer with the your generated sentence, NOTHING ELSE!
\end{lstlisting}

\subsubsection{User-Prompt: Extend, Find Synonym \& Rewrite Sentence} \mbox{}
\begin{lstlisting}
{Sentence}
\end{lstlisting}

\subsubsection{User-Prompt: Find custom Sentence:} \mbox{}
\begin{lstlisting}
*sentence*: {sentence}\n
*prompt*: {prompt}    
\end{lstlisting}

\subsection{Experiment 1 - Texts}
\label{sec:appendix_texts_exp1}
\subsubsection{\Spread{}}
\paragraph{Extend the incomplete sentence (5 repetitions)}
Climate change refers to long-term shifts in temperatures and weather patterns. These shifts are increasingly driven by human activities such as the burning of fossil fuels, deforestation, and industrial processes, which
\paragraph{Extend the text by one sentence (3 repetitions) \& Extend the text by three sentences (3 repetitions)}
Climate change refers to long-term shifts in temperatures and weather patterns. These shifts may be natural, such as through variations in the solar cycle.

\subsubsection{\Pinch}
\paragraph{Remove the incomplete sentence (5 repetitions)}
The internet is a global network of interconnected computers that communicate using standardized protocols. This vast and ever-expanding network enables the rapid exchange of information across the world, facilitating everything from basic email communication to complex
\paragraph{Shorten the text by one sentence (3 repetitions) \& Shorten the text by three sentences (3 repetitions)}
The internet is a global network of interconnected computers that communicate using standardized protocols. It allows users to access and share information quickly and easily. The internet supports various services, including email, social media, and online shopping. It has transformed how we communicate, learn, and conduct business worldwide.

\subsubsection{Combinations}
\paragraph{Extend by two sentences, then
remove one \& Shorten by two sentences and then add one (once each)}
Photosynthesis is the process by which green plants, algae, and some bacteria convert light energy into chemical energy. They use sunlight, carbon dioxide, and water to produce glucose and oxygen. Chlorophyll, the green pigment in plants, captures sunlight for this process. Photosynthesis is essential for life on Earth, providing food and oxygen for most living organisms.

\subsection{Experiment 2 - Texts}
\label{sec:appendix_texts_exp2}
\subsubsection{ChatGPT Interface}
\lbparagraph{Repetition 1}
\noindent\textit{Instruction: Remove the irrelevant sentence.}

Tomatoes are one of the most popular vegetables grown in home gardens. Skateboards have been popular since the 1950s and are used in various sports competitions. Many people enjoy growing tomatoes because they are relatively easy to cultivate and can be used in a wide range of dishes.

\noindent\textit{Instruction: Notice the topic shift and extend.}

The process of growing tomatoes involves several key steps. Firstly, tomatoes require well-drained soil that is rich in organic matter. This helps the plants to thrive and produce a good yield. By following these steps, you can ensure a successful tomato harvest.
\lbparagraph{Repetition 2}
\noindent\textit{Instruction: Remove the irrelevant sentence.}

The art of photography has evolved significantly over the years. Bicycles have been a popular mode of transportation for over a century and are widely used for commuting and recreation. Many people find photography to be a rewarding hobby because it allows them to capture and preserve memories.

\noindent\textit{Instruction: Notice the topic shift and extend.}

To capture a great photograph, there are a few essential tips to keep in mind. Firstly, understanding the basics of lighting is crucial, as it can dramatically affect the mood and clarity of an image. By mastering these techniques, you can significantly improve the quality of your photos.

\lbparagraph{Repetition 3}
\noindent\textit{Instruction: Remove the irrelevant sentence.}

Reading books is a popular pastime that can enrich one's knowledge and imagination. Computers have revolutionized the way we work and communicate, making tasks easier and more efficient. Many people enjoy reading because it allows them to escape into different worlds and perspectives.

\noindent\textit{Instruction: Notice the topic shift and extend.}

To fully enjoy reading, it's important to choose books that match your interests and reading level. Firstly, selecting a quiet and comfortable place to read can enhance your concentration and enjoyment. By taking these steps, you can make reading a more fulfilling experience.

\subsubsection{Touch Gestures}
\lbparagraph{Repetition 1}
\noindent\textit{Instruction: Remove the irrelevant sentence.}

Cooking at home can be a rewarding experience that allows you to experiment with different ingredients and flavors. Automobiles have become an essential part of modern life, providing convenience and mobility. Many people find cooking to be a creative outlet that also promotes healthier eating habits.

\noindent\textit{Instruction: Notice the topic shift and extend.}

There are a few key tips to keep in mind when cooking. Firstly, it's important to use fresh ingredients, as they can significantly enhance the taste and nutritional value of your dishes. By following these tips, you can improve your cooking skills and enjoy better meals.
\lbparagraph{Repetition 2}
\noindent\textit{Instruction: Remove the irrelevant sentence.}

Gardening is a relaxing activity that allows you to connect with nature and grow your own plants. The invention of the airplane has made long-distance travel faster and more accessible. Many people enjoy gardening because it provides a sense of accomplishment and improves the beauty of their surroundings.

\noindent\textit{Instruction: Notice the topic shift and extend.}

Successful gardening requires some basic knowledge and attention to detail. Firstly, understanding the specific needs of your plants, such as sunlight and watering, is crucial for their growth. By adhering to these guidelines, you can create a thriving garden that brings you joy.

\lbparagraph{Repetition 3}
\noindent\textit{Instruction: Remove the irrelevant sentence.}

Exercise is an important aspect of maintaining a healthy lifestyle. Smartphones have changed the way we interact with the world, providing instant access to information and communication. Regular physical activity can help improve both mental and physical well-being.

\noindent\textit{Instruction: Notice the topic shift and extend.}

There are several factors to consider when establishing a workout routine. Firstly, it's essential to set realistic goals that align with your fitness level and interests. By doing so, you can create a sustainable exercise plan that keeps you motivated.

\subsection{Semi-Structured Interview - Short Stories}
\label{sec:short_stories}
We offered participants the beginning of creative short stories if they wished to explore our prototype during the semi-structured interview that concluded the study session. 
However, they were free to interact with the prototype in any way they wanted. 
The following short stories were created using ChatGPT 4o and individually reviewed for potential biases:

\paragraph{Time Travel}
In the year 3021, a young scientist named Finn discovered a device that could send him back in time. When he accidentally traveled to 1921, he had to figure out how to return without altering history. Along the way, he uncovered a hidden truth about his own family that changed everything.

\paragraph{Magical Adventure}
One day, Lena found an old compass in an abandoned antique shop, but instead of pointing north, it led her to a secret, enchanted forest. As she followed the compass, she encountered talking animals and an ancient, wise tree that entrusted her with an important task. With bravery and cleverness, Lena solved the forest's riddle and discovered a treasure far more valuable than gold.

\paragraph{Space Expedition}
Captain Mira and her crew landed on an unknown planet covered in glowing crystals that emitted a strange energy. As they explored the mysterious caves, they uncovered an ancient civilization trapped within the crystals. Mira faced a tough decision: should she destroy the crystals and free the beings inside, even if it meant risking their mission?

\paragraph{Enchanted Market}
At the annual Wonderland Market, little Timmy stumbled upon a stall selling wishes in bottles. Mesmerized by the shimmering elixirs, he bought a bottle that promised to make his wildest dreams come true. When he made his wish, a portal opened to a world beyond his imagination, full of adventure and danger.

\paragraph{Lost City}
Deep in the jungle, archaeologist Dr. Elena found an ancient map that revealed the way to a lost city of gold. With a team of explorers, she embarked on a perilous journey through rivers and over mountains, until they finally stood before the gates of the legendary city. But the city was not abandoned, and its mysterious inhabitants had their own plans for the intruders.

\begin{table*}[!h]
\centering
\footnotesize
\newcolumntype{L}{>{\raggedright\arraybackslash}X}
\newcolumntype{P}[1]{>{\raggedright\arraybackslash}p{#1}}
\renewcommand{\arraystretch}{1.4}
\setlength{\tabcolsep}{4pt}
\begin{tabularx}{\linewidth}{lP{2.75em}P{5.25em}P{22em}P{7em}L}
\toprule
    &
    \textbf{Section} &
    \textbf{Aspect}\newline and model &
    \textbf{Predictors}\newline Baseline (Exp. 1): \visnone\newline Baseline (Exp. 2): \modegpt &
    \textbf{Follow-up comparisons} &
    \textbf{Takeaways in words}\newline(only considering sig. results) \\ \midrule
1 &
    \ref{ssec:time}
    &
    Completion time (Exp. 1)\medskip\newline
    \textit{LMM on seconds}
    &
    \visbubble{} $\downarrow^*$ \newline 
    \deemph{(\lmmci{-2.62}{.76}{-4.11}{-1.13}{<.005})}\medskip\newline 
    \visline{} $\downarrow$ \newline 
    \deemph{(\lmmci{-.85}{.77}{-2.37}{.67}{.27})}\medskip\newline 
    &
    \visbubble{} vs \visline{}\newline \deemph{(\posthoc{-1.76}{<.05})}
    &
    People finished the tasks (in experiment 1) \secs{1.76} faster with \visbubble{} than with \visline{} and \secs{2.62} faster than without visual feedback.
    \\
    \midrule
2 &
    \ref{ssec:time}
    &
    Completion time (Exp. 2)\medskip\newline
    \textit{LMM on seconds}
    &
    \modeours{} $\downarrow^*$ \newline 
    \deemph{(\lmmci{-79.23}{10.81}{-100.34}{-57.21}{<.0001})}\medskip\newline 
    &
    &
    People finished the tasks (in experiment 2) \secs{79.23} faster with \modeours{}{} than with \modegpt{}.
    \\
    \midrule
3 &
    \ref{sec:perception_exp1}
    &
    Usability (Exp. 1)\medskip\newline
    \textit{LMM on SUS scores}
    &
    \visbubble{} $\uparrow^*$ \newline 
    \deemph{(\lmmci{22.37}{5.29}{11.85}{32.63}{<.001})}\medskip\newline 
    \visline{} $\uparrow^*$ \newline 
    \deemph{(\lmmci{13.80}{5.29}{3.28}{24.06}{<.05})}\medskip\newline 
    &
    \visbubble{} vs \visline{}\newline \deemph{(\posthoc{8.57}{=.11})}
    &
    The perceived usability (as measured with the SUS score) of \visbubble{} was higher than \visnone{} (by ca. 22 points). The score of \visline{} was also higher than that of \visnone{} (by ca. 14 points).
    \\
    \midrule
4 &
    \ref{sec:perception_exp2}
    &
    Usability (Exp. 2)\medskip\newline
    \textit{LMM on SUS scores}
    &
    \modeours{} $\uparrow^*$ \newline 
    \deemph{(\lmmci{28.46}{6.72}{15.05}{42.10}{<.001})}\medskip\newline  
    &
    &
    The perceived usability of \modeours{}{} was higher than that of \modegpt{} (by ca. 28 points).
    \\
    \midrule
5 &
    \ref{sec:perception_exp1}
    &
    Workload (Exp. 1)\medskip\newline
    \textit{LMM on NASA TLX scores}
    &
    \visbubble{} $\downarrow^*$ \newline 
    \deemph{(\lmmci{-.83}{.31}{-1.44}{-0.23}{<.05})}\medskip\newline 
    \visline{} $\downarrow^*$ \newline 
    \deemph{(\lmmci{-.66}{.31}{-1.26}{-0.05}{<.05})}\medskip\newline 
    &
    \visbubble{} vs \visline{}\newline \deemph{(\posthoc{-.18}{=.56})}
    &
    The perceived workload (as measured with the NASA TLX score) of \visbubble{} was lower than \visnone{} (by 0.83). The score of \visline{} was also lower than that of \visnone{} (by 0.66).
    \\
    \midrule
6 &
    \ref{sec:perception_exp2}
    &
    Workload (Exp. 2)\medskip\newline
    \textit{LMM on NASA TLX scores}
    &
    \modeours{} $\downarrow^*$ \newline 
    \deemph{(\lmmci{-1.07}{.32}{-1.74}{-.43}{<.01})}\medskip\newline  
    &
    &
    The perceived workload of \modeours{}{} was lower than that of \modegpt{} (by 1.07).
    \\
  \bottomrule
\end{tabularx}
\caption{Statistical tests for our analysis. Columns show link to the section, tested measure, predictors, follow-up comparisons, and a textual interpretation. We use arrows to highlight whether predictors increase ($\uparrow$) or decrease ($\downarrow$) the outcome and add an asterix  (*) if this is significant.}
\Description{Overview of significance tests with links to the section, tested measure, predictors, pairwise comparisons, and written interpretation. For each statistical test it describes the Section, Aspect and model, Predictors, Pairwise comparisons, and Takeaways in words (only considering sig. results).}
\label{tab:sig_tests}
\end{table*}

\subsection{Statistical Analyses}\label{sec:appendix_sigtest}

\cref{tab:sig_tests} shows our statistical analyses.
We analysed the data with linear mixed-effects models (LMMs), using R~\cite{R2020} with the \textit{lme4}~\cite{Bates2015} and \textit{lmerTest}~\cite{Kuznetsova2017} packages. Besides the fixed effects seen in the table, the models included random intercepts to account for individual differences between participants and between the tasks (texts). 
The follow-up analyses (pairwise comparisons) were conducted with the \textit{emmeans} package, using Bonferroni-Holm correction.
We report significance at p~<~0.05. 


\subsection{\revision{Additional Figures}} \label{sec:appendix_figs}
\begin{figure*}[h!]
     \centering
     \begin{subfigure}[b]{0.49\textwidth}
        \centering
        \includegraphics[width=0.95\linewidth]{figures/finger_start_distance_task1_single.png}
        \caption{First and second touches in Experiment 1, when initiating a \spread{}.}
        \label{fig:first_touches_add}
     \end{subfigure}
     \hfill
     \begin{subfigure}[b]{0.49\textwidth}
        \centering
        \includegraphics[width=0.95\linewidth]{figures/finger_start_distance_task2_single.png}
        \caption{First and second touches for all participants in Experiment 1, when initiating a \pinch{} gesture.}
        \label{fig:first_touches_remove}
     \end{subfigure}
     \caption{Finger positions at gesture start for all participants in Experiment 1, when initiating a \spread{} (a), and a \pinch{} (b) gesture. In both subfigures, touches are plotted relative to the bounding box of the target area, with 'X' and 'Y' coordinates showing the spread of finger positions across participants. The colours indicate the starting text: blue -- incomplete sentence, orange -- one sentence, green -- three sentences.}
     \Description{This figure contains two scatter plots showing finger positions at the start of gestures for all participants in Task 1. Each plot visualises the initial finger touches for two distinct types of gestures: \spread{} and \pinch{}. Both plots use colour-coded markers and connecting lines to represent the trajectories of these finger movements. (a) The left-hand scatter plot shows the finger positions when participants initiated a \spread{} gesture. Each marker colour represents finger touches for different sub-tasks. The red vertical line indicates the central start position for the gestures. Thin grey lines connect the first and second touch points to illustrate the gesture's trajectory. (b) The right-hand scatter plot visualises the finger positions when initiating a \pinch{} gesture. Similar to (a), the first and second finger touches for each sub-task are indicated by different coloured markers. Grey lines represent the movement trajectory between the first and second touches. Both plots have an X and Y axis, with the X-axis showing the horizontal position of the touch relative to the centre, and the Y-axis showing the vertical position relative to the start bounding box. The overall layout helps to visualise the variation in finger positions and movement patterns across participants for both gestures.}
     \label{fig:first_touches}
\end{figure*}



% \begin{appendices}

% \section{Section title of first appendix}\label{secA1}

% An appendix contains supplementary information that is not an essential part of the text itself but which may be helpful in providing a more comprehensive understanding of the research problem or it is information that is too cumbersome to be included in the body of the paper.

% %%=============================================%%
% %% For submissions to Nature Portfolio Journals %%
% %% please use the heading ``Extended Data''.   %%
% %%=============================================%%

% %%=============================================================%%
% %% Sample for another appendix section			       %%
% %%=============================================================%%

% %% \section{Example of another appendix section}\label{secA2}%
% %% Appendices may be used for helpful, supporting or essential material that would otherwise 
% %% clutter, break up or be distracting to the text. Appendices can consist of sections, figures, 
% %% tables and equations etc.

% \end{appendices}

%%===========================================================================================%%
%% If you are submitting to one of the Nature Portfolio journals, using the eJP submission   %%
%% system, please include the references within the manuscript file itself. You may do this  %%
%% by copying the reference list from your .bbl file, paste it into the main manuscript .tex %%
%% file, and delete the associated \verb+\bibliography+ commands.                            %%
%%===========================================================================================%%
%\bibliographystyle{unsrt}
%\bibliography{sn-bibliography}% common bib file
%% if required, the content of .bbl file can be included here once bbl is generated
%%\input sn-article.bbl

%% Default %%
%%\input sn-sample-bib.tex%

\end{document}



%
% paper title
% Titles are generally capitalized except for words such as a, an, and, as,
% at, but, by, for, in, nor, of, on, or, the, to and up, which are usually
% not capitalized unless they are the first or last word of the title.
% Linebreaks \\ can be used within to get better formatting as desired.
% Do not put math or special symbols in the title.
\title{Chain of Backdoor Attacks for Code-Driven Embodied Intelligence}
%
%
% author names and IEEE memberships
% note positions of commas and nonbreaking spaces ( ~ ) LaTeX will not break
% a structure at a ~ so this keeps an author's name from being broken across
% two lines.
% use \thanks{} to gain access to the first footnote area
% a separate \thanks must be used for each paragraph as LaTeX2e's \thanks
% was not built to handle multiple paragraphs
%
%
%\IEEEcompsocitemizethanks is a special \thanks that produces the bulleted
% lists the Computer Society journals use for "first footnote" author
% affiliations. Use \IEEEcompsocthanksitem which works much like \item
% for each affiliation group. When not in compsoc mode,
% \IEEEcompsocitemizethanks becomes like \thanks and
% \IEEEcompsocthanksitem becomes a line break with idention. This
% facilitates dual compilation, although admittedly the differences in the
% desired content of \author between the different types of papers makes a
% one-size-fits-all approach a daunting prospect. For instance, compsoc 
% journal papers have the author affiliations above the "Manuscript
% received ..."  text while in non-compsoc journals this is reversed. Sigh.

\author{%
%Anonymous authors

  Aishan Liu, Yuguang Zhou,  Siyuan Liang, Xianglong Liu, Tianyuan Zhang, Jiakai Wang, Yanjun Pu, \\Tianlin Li, Junqi Zhang, Wenbo Zhou, Qing Guo, Dacheng Tao \\
  
%Beihang University\textsuperscript{1} \quad SenseTime Research\textsuperscript{2} \quad UC Berkeley\textsuperscript{3}\\
%  Johns Hopkins University\textsuperscript{4} \quad Oxford University\textsuperscript{5} \quad JD Explore Academy\textsuperscript{6}


\thanks{A. Liu, Y. Zhou, X. Liu, T. Zhang are with the State Key Lab of Software Development Environment, Beihang University, Beijing, China.}

\thanks{J. Wang and Y. Pu are with the ZGC Laborotary, Beijing, China.}

\thanks{J. Zhang and W. Zhou are with the University of Science and Technology of China, China}

\thanks{S. Liang is with the National University of Singapore, Singapore.}

\thanks{Q. Guo is with the A*STAR, Singapore.}

\thanks{T. Li and D. Tao are with Nanyang Technological University, Singapore.}


\thanks{The first two authors contribute equally. Correspondence to Xianglong Liu.}

}
% note the % following the last \IEEEmembership and also \thanks - 
% these prevent an unwanted space from occurring between the last author name
% and the end of the author line. i.e., if you had this:
% 
% \author{....lastname \thanks{...} \thanks{...} }
%                     ^------------^------------^----Do not want these spaces!
%
% a space would be appended to the last name and could cause every name on that
% line to be shifted left slightly. This is one of those "LaTeX things". For
% instance, "\textbf{A} \textbf{B}" will typeset as "A B" not "AB". To get
% "AB" then you have to do: "\textbf{A}\textbf{B}"
% \thanks is no different in this regard, so shield the last } of each \thanks
% that ends a line with a % and do not let a space in before the next \thanks.
% Spaces after \IEEEmembership other than the last one are OK (and needed) as
% you are supposed to have spaces between the names. For what it is worth,
% this is a minor point as most people would not even notice if the said evil
% space somehow managed to creep in.



% The paper headers
\markboth{IEEE TRANSACTIONS ON PATTERN ANALYSIS AND MACHINE INTELLIGENCE}%
{Shell \MakeLowercase{\textit{et al.}}: Bare Demo of IEEEtran.cls for Computer Society Journals}
% The only time the second header will appear is for the odd numbered pages
% after the title page when using the twoside option.
% 
% *** Note that you probably will NOT want to include the author's ***
% *** name in the headers of peer review papers.                   ***
% You can use \ifCLASSOPTIONpeerreview for conditional compilation here if
% you desire.



% The publisher's ID mark at the bottom of the page is less important with
% Computer Society journal papers as those publications place the marks
% outside of the main text columns and, therefore, unlike regular IEEE
% journals, the available text space is not reduced by their presence.
% If you want to put a publisher's ID mark on the page you can do it like
% this:
%\IEEEpubid{0000--0000/00\$00.00~\copyright~2015 IEEE}
% or like this to get the Computer Society new two part style.
%\IEEEpubid{\makebox[\columnwidth]{\hfill 0000--0000/00/\$00.00~\copyright~2015 IEEE}%
%\hspace{\columnsep}\makebox[\columnwidth]{Published by the IEEE Computer Society\hfill}}
% Remember, if you use this you must call \IEEEpubidadjcol in the second
% column for its text to clear the IEEEpubid mark (Computer Society jorunal
% papers don't need this extra clearance.)



% use for special paper notices
%\IEEEspecialpapernotice{(Invited Paper)}



% for Computer Society papers, we must declare the abstract and index terms
% PRIOR to the title within the \IEEEtitleabstractindextext IEEEtran
% command as these need to go into the title area created by \maketitle.
% As a general rule, do not put math, special symbols or citations
% in the abstract or keywords.
\IEEEtitleabstractindextext{%
\begin{abstract}
  Large language models (LLMs) have revolutionized the programming and development of embodied intelligence. In this paradigm, LLMs translate complex tasks described in abstract language into a sequence of code snippets. The embodied agent subsequently interacts with the environment to solve these tasks, with the execution logic guided by the programs and the orchestration of operation calls (\eg, perception modules). However, exploiting untrusted third-party LLMs poses considerable security risks. This paper introduces \method{}, which for the first time identifies a severe chain of backdoor threats in this practical scenario. By presenting a few shots of poisoned demonstrations, adversaries can clandestinely infect a black-box LLM, prompting it to generate programs with backdoor defects. These malicious programs, supplied by LLMs to downstream users, will be activated and compromise the reliability of the operational agent when specific visual triggers appear. Originating from a few shots of stealthy demonstrations, our attack progresses from LLM to the generated code until infiltrating the intelligent system, thus establishing a chain of backdoors that could have serious consequences for millions of downstream embodied agents. To achieve the goal, we employ an additional LLM for poisoned demonstration prompt generation and treat the optimization process as a two-player game between the discriminator and generator LLMs, where the optimized poisoned prompts fed to the generator could output programs with accurate defects that are realistic enough output to fool the discriminator. To comprehensively explore the potential risks, we broaden the attack and devise five program defects attacking modes compromising the key aspects of confidentiality, integrity, and availability of the embodied agents. To validate the effectiveness, we conducted extensive experiments on a range of tasks including robot planning, robot manipulation, and compositional visual reasoning. Additionally, we showcase the potential of our approach and successfully attack real-world autonomous driving systems. This paper aims to raise awareness of the potential threats of backdoors for embodied intelligence in practical LLM usage scenarios involving code. Our code and demos are available at the \href{https://chain-of-backdoor.github.io}{website}.
\end{abstract}

% Note that keywords are not normally used for peerreview papers.
\begin{IEEEkeywords} Chain of Attacks, 
Model Robustness, Embodied Intelligence.
\end{IEEEkeywords}}


% make the title area
\maketitle


% To allow for easy dual compilation without having to reenter the
% abstract/keywords data, the \IEEEtitleabstractindextext text will
% not be used in maketitle, but will appear (i.e., to be "transported")
% here as \IEEEdisplaynontitleabstractindextext when the compsoc 
% or transmag modes are not selected <OR> if conference mode is selected 
% - because all conference papers position the abstract like regular
% papers do.
\IEEEdisplaynontitleabstractindextext
% \IEEEdisplaynontitleabstractindextext has no effect when using
% compsoc or transmag under a non-conference mode.



% For peer review papers, you can put extra information on the cover
% page as needed:
% \ifCLASSOPTIONpeerreview
% \begin{center} \bfseries EDICS Category: 3-BBND \end{center}
% \fi
%
% For peerreview papers, this IEEEtran command inserts a page break and
% creates the second title. It will be ignored for other modes.
\IEEEpeerreviewmaketitle


\section{Introduction}

Despite the remarkable capabilities of large language models (LLMs)~\cite{DBLP:conf/emnlp/QinZ0CYY23,DBLP:journals/corr/abs-2307-09288}, they often inevitably exhibit hallucinations due to incorrect or outdated knowledge embedded in their parameters~\cite{DBLP:journals/corr/abs-2309-01219, DBLP:journals/corr/abs-2302-12813, DBLP:journals/csur/JiLFYSXIBMF23}.
Given the significant time and expense required to retrain LLMs, there has been growing interest in \emph{model editing} (a.k.a., \emph{knowledge editing})~\cite{DBLP:conf/iclr/SinitsinPPPB20, DBLP:journals/corr/abs-2012-00363, DBLP:conf/acl/DaiDHSCW22, DBLP:conf/icml/MitchellLBMF22, DBLP:conf/nips/MengBAB22, DBLP:conf/iclr/MengSABB23, DBLP:conf/emnlp/YaoWT0LDC023, DBLP:conf/emnlp/ZhongWMPC23, DBLP:conf/icml/MaL0G24, DBLP:journals/corr/abs-2401-04700}, 
which aims to update the knowledge of LLMs cost-effectively.
Some existing methods of model editing achieve this by modifying model parameters, which can be generally divided into two categories~\cite{DBLP:journals/corr/abs-2308-07269, DBLP:conf/emnlp/YaoWT0LDC023}.
Specifically, one type is based on \emph{Meta-Learning}~\cite{DBLP:conf/emnlp/CaoAT21, DBLP:conf/acl/DaiDHSCW22}, while the other is based on \emph{Locate-then-Edit}~\cite{DBLP:conf/acl/DaiDHSCW22, DBLP:conf/nips/MengBAB22, DBLP:conf/iclr/MengSABB23}. This paper primarily focuses on the latter.

\begin{figure}[t]
  \centering
  \includegraphics[width=0.48\textwidth]{figures/demonstration.pdf}
  \vspace{-4mm}
  \caption{(a) Comparison of regular model editing and EAC. EAC compresses the editing information into the dimensions where the editing anchors are located. Here, we utilize the gradients generated during training and the magnitude of the updated knowledge vector to identify anchors. (b) Comparison of general downstream task performance before editing, after regular editing, and after constrained editing by EAC.}
  \vspace{-3mm}
  \label{demo}
\end{figure}

\emph{Sequential} model editing~\cite{DBLP:conf/emnlp/YaoWT0LDC023} can expedite the continual learning of LLMs where a series of consecutive edits are conducted.
This is very important in real-world scenarios because new knowledge continually appears, requiring the model to retain previous knowledge while conducting new edits. 
Some studies have experimentally revealed that in sequential editing, existing methods lead to a decrease in the general abilities of the model across downstream tasks~\cite{DBLP:journals/corr/abs-2401-04700, DBLP:conf/acl/GuptaRA24, DBLP:conf/acl/Yang0MLYC24, DBLP:conf/acl/HuC00024}. 
Besides, \citet{ma2024perturbation} have performed a theoretical analysis to elucidate the bottleneck of the general abilities during sequential editing.
However, previous work has not introduced an effective method that maintains editing performance while preserving general abilities in sequential editing.
This impacts model scalability and presents major challenges for continuous learning in LLMs.

In this paper, a statistical analysis is first conducted to help understand how the model is affected during sequential editing using two popular editing methods, including ROME~\cite{DBLP:conf/nips/MengBAB22} and MEMIT~\cite{DBLP:conf/iclr/MengSABB23}.
Matrix norms, particularly the L1 norm, have been shown to be effective indicators of matrix properties such as sparsity, stability, and conditioning, as evidenced by several theoretical works~\cite{kahan2013tutorial}. In our analysis of matrix norms, we observe significant deviations in the parameter matrix after sequential editing.
Besides, the semantic differences between the facts before and after editing are also visualized, and we find that the differences become larger as the deviation of the parameter matrix after editing increases.
Therefore, we assume that each edit during sequential editing not only updates the editing fact as expected but also unintentionally introduces non-trivial noise that can cause the edited model to deviate from its original semantics space.
Furthermore, the accumulation of non-trivial noise can amplify the negative impact on the general abilities of LLMs.

Inspired by these findings, a framework termed \textbf{E}diting \textbf{A}nchor \textbf{C}ompression (EAC) is proposed to constrain the deviation of the parameter matrix during sequential editing by reducing the norm of the update matrix at each step. 
As shown in Figure~\ref{demo}, EAC first selects a subset of dimension with a high product of gradient and magnitude values, namely editing anchors, that are considered crucial for encoding the new relation through a weighted gradient saliency map.
Retraining is then performed on the dimensions where these important editing anchors are located, effectively compressing the editing information.
By compressing information only in certain dimensions and leaving other dimensions unmodified, the deviation of the parameter matrix after editing is constrained. 
To further regulate changes in the L1 norm of the edited matrix to constrain the deviation, we incorporate a scored elastic net ~\cite{zou2005regularization} into the retraining process, optimizing the previously selected editing anchors.

To validate the effectiveness of the proposed EAC, experiments of applying EAC to \textbf{two popular editing methods} including ROME and MEMIT are conducted.
In addition, \textbf{three LLMs of varying sizes} including GPT2-XL~\cite{radford2019language}, LLaMA-3 (8B)~\cite{llama3} and LLaMA-2 (13B)~\cite{DBLP:journals/corr/abs-2307-09288} and \textbf{four representative tasks} including 
natural language inference~\cite{DBLP:conf/mlcw/DaganGM05}, 
summarization~\cite{gliwa-etal-2019-samsum},
open-domain question-answering~\cite{DBLP:journals/tacl/KwiatkowskiPRCP19},  
and sentiment analysis~\cite{DBLP:conf/emnlp/SocherPWCMNP13} are selected to extensively demonstrate the impact of model editing on the general abilities of LLMs. 
Experimental results demonstrate that in sequential editing, EAC can effectively preserve over 70\% of the general abilities of the model across downstream tasks and better retain the edited knowledge.

In summary, our contributions to this paper are three-fold:
(1) This paper statistically elucidates how deviations in the parameter matrix after editing are responsible for the decreased general abilities of the model across downstream tasks after sequential editing.
(2) A framework termed EAC is proposed, which ultimately aims to constrain the deviation of the parameter matrix after editing by compressing the editing information into editing anchors. 
(3) It is discovered that on models like GPT2-XL and LLaMA-3 (8B), EAC significantly preserves over 70\% of the general abilities across downstream tasks and retains the edited knowledge better.
\section{Related work}




\textbf{Reasoning in LLMs}. Reasoning is a cognitive process that involves thinking about something logically and systematically, using evidence and past experiences to draw conclusions or make decisions \cite{reason1,reason2}. Recent studies have demonstrated that LLMs exhibit remarkable reasoning capabilities in various tasks, including mathematical reasoning \cite{reasonllm1}, common sense reasoning \cite{reasonllm2}, symbolic reasoning \cite{reasonllm3}, and causal reasoning \cite{reasonllm4}. Subsequently, Chain-of-thought (CoT) \cite{cot1,cot2,cot3,cot4,cot5} has emerged as a promising approach for further enhancing these reasoning capabilities.

While the reasoning capabilities of LLMs have contributed to their impressive performance across various downstream tasks, their potential exploitation in jailbreak attacks remains largely unexplored. In this study, we focus on leveraging reasoning capabilities to facilitate jailbreak attacks.

\textbf{Multi-turn Jailbreak Attack}. Typical multi-turn jailbreak methods follow the principle of starting with harmless conversations and gradually making the queries more harmful in subsequent turns. Different methods have designed specific strategies based on this principle, including applying cognitive psychology theories to gradually modify subsequent queries \cite{mpsy1,mpsy2}, using actor networks to expand the attack range of subsequent queries \cite{ren2024}, extracting harmful keywords from original queries to construct semantically equivalent ones \cite{coa,cfa}, and breaking down the target query into multiple subqueries and merging the corresponding answers to achieve the final jailbreak \cite{sub1,sub2}.

Existing multi-turn jailbreak methods often suffer from semantic drift or fail to generate effective attacks. In contrast, our approach leverages LLMs' reasoning capabilities to ensure a stable and effective jailbreak process.
\section{Threat Model}\label{sec-3-threatModel}
% \takeaway{Use a table to show an IPV attacker vs. traditional attacker. Then focus on (1) Attacker's capacity
% and their capacity limits; scope is important: might not be good for physical violence, maybe helpful for surveillance;
% (2)Requirements of a successful attack. Cite papers to show the applicability of those attacks.
% What attackers can (not) do; under which conditions can (not) they achieve successful attack}

% Please add the following required packages to your document preamble:
% \usepackage[table,xcdraw]{xcolor}
% Beamer presentation requires \usepackage{colortbl} instead of \usepackage[table,xcdraw]{xcolor}
% Please add the following required packages to your document preamble:
% \usepackage[table,xcdraw]{xcolor}
% Beamer presentation requires \usepackage{colortbl} instead of \usepackage[table,xcdraw]{xcolor}

% We make following assumptions for our threat model. First, IPV abusers are non-tech expert, unable to hack into OS to acquire
% sensitive data. In this paper, their IPV behaviors are via physical access to victims' smartphones, and such access is already
% acquired (through compromise, and guess-out, etc.), meaning that all one-time authentications (passwords, face id, finger 
% print, etc.) fails to prevent their access. The abusers are free to manipulate the device, viewing application information, 
% uploading and downloading data.

% For the defending system, we assume that the monitoring data sources are trustworthy and intact. This is consistent with our
% previous assumption of low technical level of IPV abuser to conduct OS level attack such as spoofing and adversarial machine learning. 
% Additionally, the integrity of the system is not affected by the awareness of its existence of abusers,
% thus guaranteeing system functioning.

In this section we describe the scope and some assumptions of our threat model. We focus on the IPI behaviors that require physical access to mobile smatphones. IPI abusers cause harm to victims by having access to and interacting with the victims' smartphones stealthily. As shown in Table \ref{tab-ipvvstradition}, unlike traditional cybersecurity attacks, we assume that the abusers have authenticated access to victims' devices, e.g., by registering their biometrics or having access to passwords, so they directly can interact with the data and applications stored on the victim's smartphone. Aligned with previous research \cite{freed2018stalker}, we assume that abuser's have limited technical background, given that abusers come from the general population compared to cyber attackers. This means that abusing behaviors are restricted to user-interface interactions, e.g., by viewing the content on the phone or installing applications. Additionally, we assume that the stealthiness of our system is sufficient for not alerting abusers, so the system is safe from IPI attacks.

Another notable point is that our study does not apply to partners in extreme situations, i.e., the use of the detection system will escalate IPI behavior to a higher level such as severe physical or psychological violence. As discussed in \cite{freed2019my,havron2019clinical,tseng2022care}, IPI abusers may escalate and bring more harm to victims if they discover evidence of anti-IPI measures. Although our designed goals are stealthy and not alerting, the use of our tool needs the initial assessment from a security clinic. More harmful scenarios require intervention from external agencies, such as police, law enforcement, and clinical centers, which are out of the scope of this work.


%the discovery of anti-IPV actions the anti-IPV actions might bring more harm to the IPV victims and police, law agencies, and clinical therapy should engage in insted of detection tools which won't help at this point.






%running without awareness by the abusers, so the data sources and system itself are guaranteed to be safe from IPV attacks.
\section{DAPAO Attacks}
In this section, first, we present the feasibility study, demonstrating our observation of information leakage in robust watermarks. Next, we provide the theoretical analysis for method validation. Last, we introduce evasion and forgery attacks based on the observation.

% We empirically discover that learning-based watermarking systems mitigate distortion effects (e.g., compression) by expanding the regions where the watermark pattern is embedded or increasing its magnitude, ensuring the remaining watermark remains detectable. Besides the encoding part, the system trains the watermark decoder to extract watermarks more effectively, which can be understood as increasing the model's attention weight on watermark signals.

\subsection{Feasibility Study}\label{sec:pilot}
We empirically find that learning-based robust watermarking systems counteract distortion effects (e.g., compression) by expanding the regions where the watermark pattern is embedded or amplifying its magnitude, ensuring that the watermark remains detectable. Beyond the encoding process, these systems also train the watermark decoder to enhance extraction effectiveness, effectively increasing the model's attention to watermark signals.

We conduct a feasibility study to explore: \emph{If the strengthened watermark results in leakage that can be captured from images using a feature extraction network?} We embed watermarks in multiple images with the same robust watermarking algorithm and then input these watermarked images into a feature extraction network.

% figure
\begin{figure}[!t]
    \centering
    % Custom font size
    \includegraphics[width=\linewidth]{pics/feasiblity.png}   
    \vspace{-6mm}
    \caption{Demonstration of our feasibility study.}
    \label{fig:feasibility}
    \vspace{-3mm}
\end{figure}

As shown in Figure ~\ref{fig:feasibility}, we found that:
\begin{itemize}
    \item The multi-channel features obtained after feature extraction can capture patterns not easily noticeable by the human eye.
    \item These patterns are similar across different images.
    \item Not all features contain such leakage information.
\end{itemize}
% Based on the above experimental observations, we \underline{\textbf{D}}elve into the \underline{\textbf{A}}spect of the \underline{\textbf{PA}}radox \underline{\textbf{O}}f Robust Watermarks and propose the \textbf{DAPAO} attack.

The results shed light on learning watermark characteristics from distinguished patterns probably related to the watermark.

\begin{figure}[!t]
    \centering
    % Custom font size
    \includegraphics[width=\linewidth]{pics/overview.png} 
    
    % \vspace{-3mm}
    \caption{An overview of our attack.} 
    \label{fig:method-overview}
    \vspace{-3mm}
\end{figure}

\subsection{Robustness and Invisibility Trade-off}\label{sec:Method_Theory}
% \begin{definition}
% \label{def:inj}
% A encoder $\mathcal{E}:\mathcal{I} \times W \to Y$ is injective if for any $x,y\in X$ different, $f(x)\ne f(y)$.
% \end{definition}
As mentioned earlier Sec.~\ref{sec:background}, a complete watermarking framework can be divided into three components: encoder $\mathcal{E}$, decoder $\mathcal{D}$, and distortion layer $\mathcal{T}$. The decoder takes only a single watermarked image $I_{wm}$ as input. To achieve correct verification, the decoder must implicitly disentangle the image content from the embedded watermark information and correctly associate them to extract the watermark successfully.



% the decoder must implicitly decompose the watermarked image into image information and watermark information, matching the two to successfully extract the watermark information.

\begin{definition}
An image and watermark information: $I$, $wm \ \subset \{0,1\}^k$, the encoder is:
$$\mathcal{E}(I, wm)=I+\epsilon \cdot \underbrace{\phi(I,wm)}_W$$
the decoder is:
$$\mathcal{D}(I_{wm}) \to \underbrace{(\hat{I}, \hat{W})}_{{match}} \to \hat{wm}$$



$\epsilon$ is the embedding strength.The feature space of the image $\mathcal{P} = \{p_1, p_2,...,p_n\}$ consists of two subspaces for embedding information: 
$$\mathcal{P} = \mathcal{P}_r \bigoplus \mathcal{P}_c$$

Due to joint training, the encoder exhibits a similar implicit decomposition behavior, projecting the input image $I$ into two feature spaces, named as $P_r$ and $P_c$. The former is a more suitable embedding space for information hiding, while the latter is not. 

The encoder performs this mapping $\mathcal{E}(I,wm) \to I_{wm}$ by:
$$\phi(I,wm) = \mathop{\min}_{p\in \mathcal{P}_r}||wm - \mathcal{D}(\mathcal{E}(p,wm))||^2+\lambda||\mathcal{E}(p,wm)||$$

However, as robustness requirements are introduced and continuously strengthened, the encoder must encode more information to ensure the watermark’s resistance to attacks. When the $P_r$   space is fully utilized, the encoder is forced to use $P_c$ for watermark embedding, polluting the $P_c$ space.

\end{definition}

\begin{definition}
An intuitive definition of embeddable threshold is:
\begin{gather*}
C(I) = \sup_{W \in \mathcal{P}_r}{\frac{||W||_2}{||I||_2}} \\\\
s.t. PNSR(I, I+W) \ge TV
\end{gather*}
$TV$ represents the lower bound of the visual quality.
\end{definition}

\begin{proposition}
When the robustness requirement exceeds $C(I)$, a decline in visual quality is inevitable.
\end{proposition}

\begin{proof}
Let the distortion layer $\mathcal{T}$ introduce noise $\eta \sim \mathcal{T}$, with the requirement that
$$||wm-\mathcal{D}(I_{wm} + \eta)|| \le \mathcal{B}$$
$\mathcal{B}$ is the bit error rate. Considering the channel capacity as:
$$R=\frac{1}{2}\log(1+\frac{\epsilon^2||W||^2}{\delta_{\eta}^2})$$
% \vspace{-3mm}
To achieve $R\ge H(wm)$, the following conditions must be met:
$$
\epsilon||W|| \le \sqrt{(2^{2H(wm)}-1)\delta_{\eta^2}}
$$
$H(wm)$ represents the entropy of $wm$. 

When $\sqrt{(2^{2H(wm)}-1)\delta_{\eta^2}} > C(I)||I||_2$, the system cannot simultaneously satisfy both, and it is necessary to increase $C(I)$, introducing visual artifacts into the image. Detailed proof is provided in Appendix~\ref{sec:Appendix_Proofs}.
\end{proof}
% \vspace{-3mm}
The artifacts introduced by sacrificing invisibility contain watermark information, creating a security vulnerability where watermark information leakage occurs.

%\hl{lack of proof of artifacts contain watermark information }
% This, however, compromises visual quality, leading to more detectable visual artifacts. Moreover, these artifacts also contain watermark information, creating a security vulnerability where watermark information leakage occurs.

\subsection{Detection Evasion}\label{sec:Method_Evasion Attack}
Our method is illustrated in Figure~\ref{fig:method-overview}, Suppose we have an image $I_{wm}$, embedded with an unknown watermark $wm$. This image is fed into a feature extraction module $\mathcal{F}(\cdot)$, resulting in multi-channel features $\mathcal{F}(I_{wm})$. To automate the selection of features that capture potential information leakage, we perform clustering on the multi-channel features. Among the resulting clusters, we identify the two clusters with the smallest number of samples and extract their corresponding feature channel positions $\mathcal{W}$.

To achieve the goal of an evasion attack, we need to disrupt the leaked watermark information captured from $I_{wm}$.We formulate this process as an optimization problem: finding a perturbation $\delta$ that disrupts the leaked information while preserving the visual quality of the image. The formulation is as follows:
\begin{equation}
\label{eq:1}
\begin{split}
    \mathop{\min}_{\delta}-\mathcal{L}(\mathcal{W} \cdot \mathcal{F}(I_{wm}), \mathcal{W}\cdot \mathcal{F}(I_{wm} + \delta)) \\
    \mathrm{ s.t.} ||\delta||_{\infty} < \epsilon
\end{split}
\end{equation}

where $\mathcal{L}(\cdot,\cdot)$ is the loss function that measures the distance between two features, and $\epsilon$ is a perturbation budget.

We use Projected Gradient Descent (PGD)~\cite{PGD} to solve the optimization problem in Eq~\ref{eq:1}. Finally, we complete the attack through $I_{wm} + \delta$.

Our detailed algorithm is shown as 
 Algorithm~\ref{alg:evasion algo}.
 %in Appendix~\ref{sec:Appendix_Implementation Details}.

 % Similar to Sec~\ref{sec:Method_Evasion Attack}, as shown in Figure ~\ref{fig:method-overview},

\subsection{Forgery Attack}
As shown in Figure~\ref{fig:method-overview}, we first use the feature extraction module and clustering algorithm to extract features containing leaked watermark information, from $I_{wm}$. To achieve the goal of spoofing, we still need to extract the leaked information. Therefore, this process can be formulated as the following optimization problem:
\begin{equation}
\label{eq:2}
\begin{split}
     \mathop{\min}_{\delta}-\mathcal{L}(\mathcal{W} \cdot \mathcal{F}(I_{wm}), \mathcal{W}\cdot \mathcal{F}(I_{wm} + \delta)) \\
    \mathrm{ s.t.} ||\delta||_{\infty} < \epsilon
\end{split}
\end{equation}
\vspace{-4mm}

where $\epsilon$ is a perturbation budget, and 
 this process is identical to the above evasion attack, referred to as Stage \uppercase\expandafter{\romannumeral1}.
However, the learned $\delta$ alone cannot fulfill the forgery purpose for \emph{semantic watermarking}. Based on the theory discussed earlier (See Sec.~\ref{sec:Method_Theory}), we need to consider the coupling effect between the semantics and watermark. After the optimization in Eq~\ref{eq:2} is completed, an additional optimization term should be included to further find another perturbation, $\delta_s$, which can be described as:
\begin{equation}
\label{eq:3}
\begin{split}
     \mathop{\min}_{\delta}\mathcal{L}((1-\mathcal{W}) \cdot \mathcal{F}(I_{wm}+\delta), (1-\mathcal{W})\cdot \mathcal{F}(I' + \delta_s)) \\
    \mathrm{ s.t.} ||\delta_s||_{\infty} < \epsilon
\end{split}
\end{equation}
This process is referred to as Stage \uppercase\expandafter{\romannumeral2}.
We use Projected Gradient Descent (PGD)~\cite{PGD} to solve the optimization problem in Eq~\ref{eq:2} and Eq~\ref{eq:3}.
Finally, we complete the attack through $\{I' - \delta\}$ or $\{I' - \delta + \delta_s \}$.

Our detailed algorithm is shown as Algorithm~\ref{alg:spoof algo}
%in Appendix~\ref{sec:Appendix_Implementation Details}.
\section{Experiments and Results}
\subsection{Experiment Settings}

\begin{table*}[ht]
    \centering
    % \small
    \caption{The main results of our experimentation. Each row group corresponds to the results for the given dataset, with each row showcasing the metric results for each model. The columns include all the main approaches, with \textbf{bold} highlighting the best result across all approaches.}
    \small
    \begin{tabular}{llccccc}
      \toprule
      Dataset & Model & Baseline & RAG & CoT & RaR & \rephrase \\
      \midrule
      \multirow[l]{3}{*}{TriviaQA}
          & Llama-3.2 3B  & 59.5 & 82.0 & 87.5  & 86.0 &  \textbf{88.5}    \\
          & Llama-3.1 8B  & 76.5 & 89.5 & 90.5  & 89.5 &  \textbf{92.5}    \\
          & GPT-4o    & 88.7 & 92.7 & 92.7  & 94.7 &  \textbf{96.7}    \\
      \midrule
      \multirow[l]{3}{*}{HotpotQA}
          & Llama-3.2 3B  &  17.5  & 26.0  & 26.5   & 25.0  &  \textbf{31.5}   \\
          & Llama-3.1 8B  &  23.0  & 26.5  & 31.0   & 28.5  &  \textbf{33.5}   \\
          & GPT-4o    &  44.0  & 45.3  & 46.7   & \textbf{47.3}  &  46.7   \\
      \midrule
      \multirow[l]{3}{*}{ASQA}
          & Llama-3.2 3B  &  14.2 & 21.5  & 21.9  & 23.5  &  \textbf{26.6}   \\ 
          & Llama-3.1 8B  &  14.6 & 23.1  & 24.8  & 25.5  &  \textbf{28.8}   \\ 
          & GPT-4o    &  26.8 & 30.4  & \textbf{31.9}  & 30.1 & 31.7 \\ 
      \bottomrule
    \end{tabular}
    \label{tab:main}
\end{table*}



\textbf{Datasets}. We conduct experiments on two datasets: CC-news\footnote{\href{https://huggingface.co/datasets/vblagoje/cc_news}{Huggingface: vblagoje/cc\_news}} and Wikipedia\footnote{\href{https://huggingface.co/datasets/legacy-datasets/wikipedia}{Huggingface: legacy-datasets/Wikipedia}}. CC-news is a large collection of news articles which includes diverse topics and reflects real-world temporal events. Meanwhile, Wikipedia covers general knowledge across a wide range of disciplines, such as history, science, arts, and popular culture.\\
\textbf{LLMs}: We experiment on three models including \gpt~(124M)~\cite{gpt2radford}, \pythia~(1.4B)~\cite{pythia}, and \llama~(7B)~\cite{llama2touvron2023}. This selection of models ensures a wide range of model sizes from small to large that allows us to analyze scaling effects and generalizability across different capacities. \\
\textbf{Evaluation Metrics}. For evaluating language modeling performance, we measure perplexity (PPL), as it reflects the overall effectiveness of the model and is often correlated with improvements in other downstream tasks~\cite{kaplan2020scalinglaws, lmsfewshot}. For defense effectiveness, we consider the attack area under the curve (AUC) value and True Positive Rate (TPR) at low False Positive Rate (FPR). In total, we perform 4 MIAs with different MIA signals. Given the sample $x$, the MIA signal function $f$ is formulated as follows: \\
$\bullet$ Loss~\cite{8429311} utilizes the negative cross entropy loss as the MIA signal. 
    \[f_\text{Loss}(x) = \mathcal{L}_\text{CE}(\theta; x) \]
$\bullet$ Ref-Loss~\cite{Carlini2020ExtractingTD} considers the loss differences between the target model and the attack reference model. To enhance the generality, our experiments ensure there is no data contamination between the training data of the target, reference, and attack models.
    \[f_\text{Ref}(x) = \mathcal{L}_\text{CE}(\theta; x) - \mathcal{L}_\text{CE}(\theta_\text{attack}; x) \]
$\bullet$ Min-K~\cite{shi2024detecting} leverages top K tokens that have the lowest loss values.
    \[f_\text{min-K}(x) = \frac{1}{|\text{min-K(x)}|} \sum_{t_i \in \text{min-K(x)}} -\log(P(t_i|t_{<i};\theta) \]
$\bullet$ Zlib~\cite{Carlini2020ExtractingTD} calibrates the loss signal with the zlib compression size.
    \[ f_\text{zlib}(x) = \mathcal{L}_\text{CE}(\theta; x) / \text{zlib}(x) \]

\noindent \textbf{Baselines}. We present the results of four baselines. \textit{Base} refers to the pretrained LLM without fine tuning. \textit{FT} represents the standard causal language modeling without protection. \textit{Goldfish}~\cite{hans2024be} implements a masking mechanism. \textit{DPSGD}~\cite{abadi2016deep, yu2022differentially} applies gradient clipping and injects noise to achieve  sample-level differential privacy.

\noindent \textbf{Implementation}. We conduct full fine-tuning for \gpt and \pythia. For computing efficiency, \llama fine-tuning is implemented using Low-Rank Adaptation (LoRA)~\cite{hu2022lora} which leads to \textasciitilde4.2M trainable parameters. Additionally, we use subsets of 3K samples to fine-tune the LLMs. We present other implementation details in Appendix~\ref{sec:app-implementation}.

\subsection{Overall Evaluation}
Table~\ref{tab:main_result} provides the overall evaluation compared to several baselines across large language model architectures and datasets. Among these two datasets, CCNews is more challenging, which  leads to higher perplexity  for all LLMs and fine-tuning methods. Additionally, the reference-model-based attack performs the best and demonstrates high privacy risks with attack AUC on the conventional fine-tuned models at 0.95 and 0.85 for Wikipedia and CCNews, respectively. Goldfish achieves similar PPL to the conventional FT method but fails to defend against MIAs. This aligns with the reported results by \citet{hans2024be} that Goldfish resists exact match attacks but only marginally affects MIAs. DPSGD provides a very strong protection in all settings (AUC < 0.55) but with a significant PPL tradeoff. Our proposed \methodname guarantees a robust protection, even slightly better than DPSGD, but with a notably smaller tradeoff on language modeling performance. For example, on the Wikipedia dataset, \methodname delivers perplexity reduction by 15\% to 27\%. Moreover, Table~\ref{tab:tpr} (Appendix~\ref{sec:app-add-res}) provides the TPR at 1\% FPR. Both DPSGD and \methodname successfully reduce the TPR to $\sim$0.02 for all LLMs and datasets. \textit{Overall, across multiple LLM architectures and datasets, \methodname consistently offers ideal privacy protection with  little trade-off in language modeling performance.}

\noindent \textbf{Privacy-Utility Trade-off.}
To investigate the privacy-utility trade-off of the methods, we vary the hyper-parameters of the fine-tuning methods. Particularly, for DPSGD, we adjust the privacy budget $\epsilon$ from (8, 1e-5)-DP to (100, 1e-5)-DP. We modify the masking percentage of Goldfish from 25\% to 50\%. Additionally, we vary the loss weight $\alpha$ from 0.2 to 0.8 for \methodname. Figure~\ref{fig:priv-ult-tradeoff} depicts the privacy-utility trade-off for GPT2 on the CCNews dataset. Goldfish, with very large masking rate (50\%), can slightly reduce the risk of the reference attack but can increase the risks of other attacks. By varying the weight $\alpha$, \methodname offers an adjustable trade-off between privacy protection and language modeling performance. \methodname largely dominates DPSGD and improves the language modeling performance by around 10\% with the ideal privacy protection against MIAs.

\begin{figure}[h]
    \centering
    \includegraphics[width=\linewidth]{figs/privacy-ultility-tradeoff.pdf}
    \caption{Privacy-utility trade-off of the methods while varying hyper-parameters. The Goldfish masking rate is set to 25\%, 33\%, and 50\%. The privacy budget $\epsilon$ of DPSGD is evaluated at 8, 16, 50, and 100. The weight $\alpha$ of \methodname is configured at 0.2, 0.5, and 0.8.}
    \label{fig:priv-ult-tradeoff}
\end{figure}


\subsection{Ablation Study}
\textbf{\methodname without reference models.} To study the impact of the reference model, we adapt \methodname to a non-reference version which directly uses the loss of the current training model (i.e., $s(t_i) = \mathcal{L}_{CE}(\theta; t_i)$) to select the learning and unlearning tokens. This means the unlearning tokens are the tokens that have smallest loss values. Figure~\ref{fig:ppl-auc-noref} presents the training loss and testing perplexity. There is an inconsistent trend of the training loss and testing perplexity. Although the training loss decreases overtime, the test perplexity increases. This result indicates that identifying appropriate unlearning tokens  without a reference model is challenging and conducting unlearning on an incorrect set hurts the language modeling performance.

\begin{figure}[htp]
    \centering
    \includegraphics[width=0.35\textwidth]{figs/train_loss_ppl_noref.pdf}
    \caption{Training Loss and Test Perplexity of \methodname without a reference model.
    % (\lrx{If time permits, it would be better to compare with our training curve here)}
    }
    \label{fig:ppl-auc-noref}
\end{figure}

\noindent \textbf{\methodname with out-of-domain reference models.} To examine the influence of the distribution gap in the reference model, we replace the in-domain trained reference model with the original pretrained base model. 
Figure~\ref{fig:ppl-auc-base-woasc} depicts the language modeling performance and privacy risks in this study. \methodname with an out-of-domain reference model can reduce the privacy risks but yield a significant gap in language modeling performance compared to \methodname using an in-domain reference model.

\noindent \textbf{\methodname without Unlearning.} To study the effects of unlearning tokens, we implement \methodname which use the first term of the loss only ({$\mathcal{L}_{\theta} = \mathcal{L}_{CE}(\theta; \mathcal{T}_h)$}). Figure~\ref{fig:ppl-auc-base-woasc} provides the perplexity and MIA AUC scores in this setting. Generally, without gradient ascent, \methodname can marginally reduce membership inference risks while slightly improving the language modeling performance. The token selection serves as a regularizer that helps to improve the language modeling performance. Additionally, tokens that are learned well in previous epochs may not be selected in the next epochs. This slightly helps to not amplify the memorization on these tokens over epochs.

\begin{figure}[htp]
    \centering
    \includegraphics[width=0.28\textwidth]{figs/auc_vs_ppl_base_woasc.pdf}
    \caption{Privacy-utility trade-off of \methodname with different settings: in-domain reference model, out-domain reference model, and without unlearning}
    \label{fig:ppl-auc-base-woasc}
\end{figure}


\subsection{Training Dynamics}
\textbf{Memorization and Generalization Dynamics}. Figure~\ref{fig:training-dynamics} (left) illustrates the training dynamics of conventional fine tuning and \methodname, while Figure~\ref{fig:training-dynamics} (middle) depicts the membership inference risks. Generally, the gap between training and testing loss of conventional fine-tuning steadily increases overtime, leading to model overfitting and high privacy risks. In contrast, \methodname maintains a stable equilibrium where the gap remains more than 10 times smaller. This equilibrium arises from the dual-purpose loss, which balances learning on hard tokens while actively unlearning memorized tokens. By preventing excessive memorization, \methodname mitigates membership inference risks and enhances generalization.

\begin{figure*}[htp]
    \centering
    \includegraphics[width=0.29\linewidth]{figs/loss_vs_steps_ft_duolearn.pdf}
    \includegraphics[width=0.29\linewidth]{figs/auc_vs_steps_ft_duolearn.pdf}
    \includegraphics[width=0.316\linewidth]{figs/cosine.pdf}
    \caption{Training dynamics of \methodname and the conventional fine-tuning approach. The left and middle figures provide the training-testing gap and membership inference risks, respectively. The testing~$\mathcal{L}_{CE}$ of FT and training~$\mathcal{L}_{CE}$ of \methodname are significantly overlapping, we provide the breakdown in Figure~\ref{fig:add-overlap-breakdown} in Appendix~\ref{sec:app-add-res}. The right figure depicts the cosine similarity of the learning and unlearning gradients of \methodname. Cosine similarity of 1 means entire alignment, 0 indicates orthogonality, and -1 presents full conflict.}
    \label{fig:training-dynamics}
\end{figure*}

\noindent \textbf{Gradient Conflicts}. To study the conflict between the learning and unlearning objectives in our dual-purpose loss function, we compute the gradient for each objective separately. We then calculate the cosine similarity of these two gradients. Figure~\ref{fig:training-dynamics} (right) provides the cosine similarity between two gradients over time. During training, the cosine similarity typically ranges from -0.15 to 0.15. This indicates a mix of mild conflicts and near-orthogonal updates. On average, it decreases from 0.05 to -0.1. This trend reflects increasing gradient misalignment. Early in training, the model may not have strongly learned or memorized specific tokens, so the conflicts are weaker. Overtime, as the model learns more and memorization grows, the divergence between hard and memorized tokens increases, making the gradients less aligned. This gradient conflict is the root of the small degradation of language modeling performance of \methodname compared to the conventional fine tuning approach.

\noindent \textbf{Token Selection Dynamics}. Figure~\ref{fig:token-selection} illustrates the token selection dynamics of \methodname during training. The figure shows that the token selection process is dynamic and changes over epochs. In particular, some tokens are selected as an unlearning from the beginning to the end of the training. This indicates that a token, even without being selected as a learning token initially, can be learned and memorized through the connections with other tokens. This also confirms that simple masking as in Goldfish is not sufficient to protect against MIAs. Additionally, there are a significant number of tokens that are selected for learning in the early epochs but unlearned in the later epochs. This indicates that the model learned tokens and then memorized them over epochs, and the during-training unlearning process is essential to mitigate the memorization risks.

\begin{figure}[htp]
    \centering
    \includegraphics[width=0.7\linewidth]{figs/token-selection-dynamics.pdf}
    \caption{Token Selection Dynamics of \methodname}
    \label{fig:token-selection}
    \vspace{-4mm}
\end{figure}

\subsection{Privacy Backdoor}
To study the worst case of privacy attacks and defense effectiveness under the state-of-the-art MIA, we perform a privacy backdoor -- Precurious~\cite{precurious}. In this setup, the target model undergoes continual fine-tuning from a warm-up model. The attacker then applies a reference-based MIA that leverages the warm-up model as the attack's reference. Table~\ref{tab:backdoor} shows the language modeling and MIA performance on CCNews with GPT-2. Precurious increases the MIA AUC score by 5\%. Goldfish achieves the lowest PPL, aligning with~\citet{hans2024be}, where the Goldfish masking mechanism acts as a regularizer that potentially enhances generalization. Both DPSGD and \methodname provide strong privacy protection, with \methodname offering slightly better defense while maintaining lower perplexity than DPSGD.

% \begin{table}[h]
%     \centering
%     \begin{tabular}{c|cc|cc}
%        \multirow{2}{*}{\textbf{Method}}  & \multicolumn{2}{c}{\textbf{CCNews}} & \multicolumn{2}{c}{\textbf{Wikipedia}} \\ 
%        & \textbf{PPL} & \textbf{AUC} & \textbf{PPL} & \textbf{AUC} \\ \hline
%        \textbf{FT}        & 21.593 & 0.911 \\
%        \textbf{Goldfish}  & \textbf{21.074} & 0.886 \\
%        \textbf{DPSGD}     & 23.279 & 0.533 \\
%        \textbf{DuoLearn}  & 22.296 & \textbf{0.499} \\
%     \end{tabular}
%     \caption{Caption}
%     \label{tab:my_label}
% \end{table}

\begin{table}[h]
    \centering
    \resizebox{\columnwidth}{!}{\begin{tabular}{c|cccccc}
        \textbf{Metric} & \textbf{WU} & \textbf{FT} & \textbf{GF} & \textbf{DP} & \textbf{DuoL} \\ \hline
        \textbf{PPL} & \textit{23.318} & 21.593 & \textbf{21.074} & 23.279 & 22.296  \\
        \textbf{AUC} & \textit{0.500} & 0.911 & 0.886 & 0.533 & \textbf{0.499} \\
    \end{tabular}}
    \caption{Experimental results of privacy backdoor for GPT2 on the CC-news dataset. WU stands for the warm-up model leveraged by Precurious. GF, DP, and DuoL are abbreviations of Goldfish, DPSGD, and \methodname}
    \label{tab:backdoor}
\end{table}

% \subsubsection{Hyperparameter Study}

% \subsubsection{Full fine-tuning versus Parameter efficent fine tuning}

% \subsubsection{Extending to Vision Language Models}



\input{src/7-realword-exp}
\input{src/6-attackmode}
\input{src/8-countermeasures}
\section{Related Work}

\paragraph{Confidence signals for LLMs.} 
There is a long line of work on deriving confidence measures from LLMs. Popular approaches use
% Popular methods to derive calibrated confidence from LLMs include 
the agreement across multiple samples \cite{kuhn2023semantic, manakul2023selfcheckgpt, tian2023fine,lyu2024calibrating}, the model's internal representations \cite{azaria2023internal, burns2022discovering} or directly prompting the model to verbalize its confidence \cite{tian2023just, kadavath2022language}.
All papers in this line of work focused on fact-seeking tasks, 
so confidence is typically derived based on the final answer alone. To the best of our knowledge, our work is the first to apply these approaches to scoring the entire reasoning path.

\paragraph{Reasoning verification.}
While learned verifiers have been demonstrated to significantly improve performance on math word problems \cite{cobbe2021training, lightman2023let, li2022making}, the ability of LLMs to perform \emph{self}-verification and \emph{self}-correction is still heavily contested, with some works providing positive evidence for such capabilities \cite{weng2022large, gero2023self, madaan2024self, liu2024large, li2024confidence} and others arguing that the gains can mostly be attributed to clever prompt design, unfair baselines, data contamination and using overly simple tasks \cite{tyen2023llms, valmeekam2023can, hong2023closer, huang2023large, stechly2024self, zhang2024small}. This work contributes to this ongoing discussion by presenting multiple lines of evidence supporting LLM self-verification. In particular, we demonstrate clear benefits from a simple confidence-based self-verification approach. 


\paragraph{Improving self-consistency's efficiency. }

Numerous attempts \cite{chen-etal-2024-self-para} have been made to reduce SC computational overhead while maintaining quality. However, none have matched the widespread adoption of self-consistency. This can be largely attributed to several limitations: (1) a trade-off where throughput is reduced while latency increases, for example by sampling chains sequentially until reaching a certain condition \cite{li2024escape} or running expensive LLM calls instead of the cheap majority voting \cite{yoran2023answering}, (2) the need for manual feature crafting and tuning tailored to each dataset \cite{wan2024dynamic}, (3) promising results on specialized setups \cite{wang2024soft} which did not generalize to standard benchmarks (Table \ref{table:max-ablation}), and (4) as highlighted by \citet{huang2023large}, many of the more sophisticated methods that appear promising actually don't outperform self-consistency when evaluated with a thorough analysis of inference costs. Our approach is different in that CISC adds minimal complexity to self-consistency, and improves throughput without compromising latency.

\paragraph{Self-consistency with confidence.}
Related approaches to CISC's confidence-weighted majority vote were previously explored in both the original self-consistency paper \citet{wang2022self}, that considered a weighted majority using Sequence Probability (\S\ref{sec:metrics}), and in \citet{miao2023selfcheck}, that concluded that verbally \nl{asking the LLM to check its own reasoning is largely ineffective} for improving self-consistency. In both cases, these failures are attributed to the confidence scores being too similar to one another. Our work shows that despite this, the scores contain a useful signal (reflected in the WQD scores; Table \ref{tab:confidence-methods}) that can be utilized by a normalization step prior to aggregation to significantly improve the efficiency of self-consistency. Furthermore, the P(True) method, which achieves the highest WQD scores, has not been previously used for attempting to improve self-consistency.



\section{Conclusion}
This paper presented \toolkit, a do-it-yourself toolkit that empowers novice roboticists with basic electronics and programming skills to rapidly prototype interactions for functional lo-fi exoskeletons targeted at the arms. 
\toolkit~features modular hardware components that allow to easily reconfigure its active degrees of freedom, adjust component's dimensions to accommodate various body sizes, and safety mechanisms. We conceptually identified relevant high-level augmentation strategies and provide them as functional abstractions that simplify the programming of interactive behaviors. These functions are readily accessible and customizable through a command-line interface, GUI, Processing library, and Arduino firmware. 
Through application cases and two usage studies, we demonstrated \toolkit's potential to ease the development of human-exoskeleton interactions and support creative exploration and rapid iteration in early-stage interaction design. We hope that this work will inspire HCI researchers to explore the emerging field of human-exoskeleton interaction and unlock its potential for innovative applications.
 
\begin{acks}
    We thank all participants of our usage studies and express our particular gratitude to Ata Otaran for his feedback. We also thank the reviewers for their valuable comments.
\end{acks}




% Can use something like this to put references on a page
% by themselves when using endfloat and the captionsoff option.
\ifCLASSOPTIONcaptionsoff
  \newpage
\fi


{
\bibliography{reference}
\bibliographystyle{unsrt}
}
%\fbox{\begin{minipage}{38em}

\subsubsection*{Scaling Law Reproducilibility Checklist}\label{sec:checklist}


\small

\begin{minipage}[t]{0.48\textwidth}
\raggedright
\paragraph{Scaling Law Hypothesis (\S\ref{sec:power-law-form})}

\begin{itemize}[leftmargin=*]
    \item What is the form of the power law?
    \item What are the variables related by (included in) the power law?
    \item What are the parameters to fit?
    \item On what principles is this form derived?
    \item Does this form make assumptions about how the variables are related?
    % \item How are each of these variables counted? (For example, how is compute cost/FLOPs counted, if applicable? How are parameters of the model counted?)
    % \item Are code/code snippets provided for calculating these variables if applicable? 
\end{itemize}


\paragraph{Training Setup (\S\ref{sec:model_training})}
\begin{itemize}[leftmargin=*]
    \item How many models are trained?
    \item At which sizes?
    \item On how much data each? On what data? Is any data repeated within the training for a model?
    \item How are model size, dataset size, and compute budget size counted? For example, how are parameters of the model counted? Are any parameters excluded (e.g., embedding layers)?
    \item Are code/code snippets provided for calculating these variables if applicable?
    % embedding  For example, how is compute cost counted, if applicable? 
    \item How are hyperparameters chosen (e.g., optimizer, learning rate schedule, batch size)? Do they change with scale?
    \item What other settings must be decided (e.g., model width vs. depth)? Do they change with scale?
    \item Is the training code open source?
    % \item How is the correctness of the scaling law considered SHOULD WE?
\end{itemize}

\end{minipage}
\begin{minipage}[t]{0.48\textwidth}
\raggedright


\paragraph{Data Collection(\S\ref{sec:data})}
\begin{itemize}[leftmargin=*]
    \item Are the model checkpoints provided openly?
    % \item Are these checkpoints modified in any way before evaluation? (say, checkpoint averaging)
    % \item If the above is done, is code for modifying the checkpoints provided?
    \item How many checkpoints per model are evaluated to fit each scaling law?
    \item What evaluation metric is used? On what dataset?
    \item Are the raw evaluation metrics modified, e.g., through loss interpolation, centering around a mean, scaling logarithmically, etc?
    \item If the above is done, is code for modifying the metric provided? 
\end{itemize}

\paragraph{Fitting Algorithm (\S\ref{sec:opt})}
\begin{itemize}[leftmargin=*]
    \item What objective (loss) is used?
    \item What algorithm is used to fit the equation?
    \item What hyperparameters are used for this algorithm?
    \item How is this algorithm initialized?
    \item Are all datapoints collected used to fit the equations? For example, are any outliers dropped? Are portions of the datapoints used to fit different equations?
    \item How is the correctness of the scaling law considered? Extrapolation, Confidence Intervals, Goodness of Fit?
\end{itemize}

\end{minipage}

% \paragraph{Other}
% \begin{itemize}
%     \item Is code for 
% \end{itemize}

\end{minipage}}


% insert where needed to balance the two columns on the last page with
% biographies
%\newpage

\documentclass[10pt,journal,compsoc]{IEEEtran}
\usepackage{graphicx}

\begin{document}

\begin{IEEEbiography}[{\includegraphics[width=1in,height=1.25in,clip,keepaspectratio]{german-maglione.png}}]
{German Maglione-Mathey} received the BS and MS degrees in Computer Science from the
University of Castilla-La Mancha (UCLM), Spain, in 2015 and 2016 respectively. He
began his research career in 2015 as a PhD Student at the University of Castilla-La Mancha in Spain,
when he was recruited by the Computer Architecture Department of that University.
His research interests include High Performance Computing interconnects and Data Center Networks
and all the strategies related to improve them, especially network topologies, routing algorithms and congestion management.
\end{IEEEbiography}


\begin{IEEEbiography}[{\includegraphics[width=1in,height=1.25in,clip,keepaspectratio]{jesus-escudero-sauquillo.pdf}}]
{Jesus Escudero-Sahuquillo} received the MS and PhD degrees in Computer Science from the University of Castilla-La Mancha (UCLM), Spain, in 2008 and 2011, respectively. His research interests include high-performance computing and Big-Data services for cluster and datacenters, interconnection networks and all the strategies related to improve them, such as network topologies, routing algorithms, congestion management, and power saving. He has published more than 30 papers in national and
international peer-reviewed conferences and journals. In 2006 he joined the Computer Systems Department (DSI), UCLM, Spain. He performed several pre- and post-doc research stays in Simula Research Labs (Norway) and Heidelberg University (Gemany). In 2014 he moved to the industry and worked for Oracle Corporation (Norway), as a PhD Senior Engineer. In 2015 he moved to the Technical University of Valencia (Spain), as a PostDoc research assistant granted with a national-competitive grant "Juan de La Cierva". In 2016 he joined again the DSI, UCLM (Spain), with a 5-years PostDoc position funded by the UCLM research program and the European Commission (FSE funds). He has participated in several research projects funded by the European Commission and the Spanish Government. He has served as program committee, guest editor and reviewer in several conferences, such as ICPP, CCGrid, HoTI or EuroPar, and journals, such as TPDS, JPDC, CCPE or JSC. He is co-organizer of the IEEE International Workshop on High-Performance Interconnection Networks in the Exascale and Big-Data Era (HiPINIEB).
\end{IEEEbiography}

\begin{IEEEbiography}[{\includegraphics[width=1in,height=1.25in,clip,keepaspectratio]{pedro-garcia.png}}]
{Pedro J. Garcia} received a degree in communication engineering from the
Technical University of Valencia, Spain, in 1996, and the PhD degree in computer
science from the University of Castilla-La Mancha (UCLM), Spain, in 2006.
In 1999, he joined the Computing Systems Department (DSI), UCLM, Spain, where he
is currently an assistant professor of computer architecture and technology. His
main research interests are the design and implementation of strategies to
improve several aspects of high-performance interconnection networks, especially
congestion management schemes and routing algorithms. He has published more than
50 refereed papers in ranked journals and conferences. He has guided two doctoral
thesis and is guiding currently three more. He has been the coordinator of three
research projects supported respectively by the Spanish Government and by the
Government of Castilla-La Mancha. He has been also the coordinator of four
Research \& Development Agreements between UCLM and different companies. In addition, he has participated in other (more than 30) research projects, supported by the
European Commission and the Spanish Government. He has served as organizer
committee member and program committee member in several international
conferences and workshops, such as ICPP, HotI, CCGrid, ISC, HiPINEB. He has been
also a guest editor of several journals.
\end{IEEEbiography}

\begin{IEEEbiography}[{\includegraphics[width=1in,height=1.25in,clip,keepaspectratio]{francisco-quiles.png}}]
{Francisco J. Quiles} is a Full Professor of Computer Architecture and Technology
at the Computing Systems Department of UCLM. His research interests include:
high-performance interconnection networks for multiprocessor systems and
clusters, parallel algorithms for video compression and video transmission.
He has served as Program Committee member in several conferences. He has
published over 200 papers in international journals and conferences and
participated in 68 research projects supported by the NFS, European Commission,
the Spanish Government and Research \& Development Agreements with different
companies. Also, he has guided 9 doctoral theses.
\end{IEEEbiography}


\begin{IEEEbiography}[{\includegraphics[width=1in,height=1.25in,clip,keepaspectratio]{eitan-zahavi.png}}]
{Eitan Zahavi} manages the Mellanox end-to-end performance architecture group which
focuses on features that improve the overall system performance for both Ethernet
and InfiniBand, lossy and lossless. We also study Optical Data Center networks.
Example fields of research are Application performance, Congestion Control,
Adaptive Routing, Tenants Isolation, and Topologies. The group employs large system
simulation and lab experiments to validate our hypothesis and test new features
implementations.
\end{IEEEbiography}

\end{document}




\newpage
\newpage
\clearpage

%\appendices



% that's all folks
\end{document}
