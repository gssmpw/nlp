%%%%%%%% ICML 2025 EXAMPLE LATEX SUBMISSION FILE %%%%%%%%%%%%%%%%%

\documentclass{article}

% Recommended, but optional, packages for figures and better typesetting:
\usepackage{microtype}
\usepackage{graphicx}
\usepackage{subfigure}
\usepackage{booktabs} % for professional tables

% hyperref makes hyperlinks in the resulting PDF.
% If your build breaks (sometimes temporarily if a hyperlink spans a page)
% please comment out the following usepackage line and replace
% \usepackage{icml2025} with \usepackage[nohyperref]{icml2025} above.
\usepackage{hyperref}


% Attempt to make hyperref and algorithmic work together better:
\newcommand{\theHalgorithm}{\arabic{algorithm}}

% Use the following line for the initial blind version submitted for review:
%\usepackage{icml2025}

% If accepted, instead use the following line for the camera-ready submission:
\usepackage[accepted]{icml2025}

% For theorems and such
\usepackage{amsmath}
\usepackage{amssymb}
\usepackage{mathtools}
\usepackage{amsthm}

% if you use cleveref..
\usepackage[capitalize,noabbrev]{cleveref}

\usepackage{hyperref}
\usepackage{url}

% Our files
\usepackage{packages/default_packages}
\usepackage{packages/mjb}
\usepackage{packages/colors}
\usepackage{packages/boxmath}
\usepackage{packages/colab}
\usepackage{packages/local_defs}

% Default packages for the document
\usepackage{subcaption}

% Added package
\usepackage{algorithm}
\usepackage{algorithmic}
\usepackage{multirow}
\usepackage{enumitem}
\usepackage{arydshln}
\usepackage{wrapfig}
\usepackage{booktabs}
\usepackage{graphicx}
\usepackage[normalem]{ulem}
\useunder{\uline}{\ul}{}

%%%%%%%%%%%%%%%%%%%%%%%%%%%%%%%%
% THEOREMS
%%%%%%%%%%%%%%%%%%%%%%%%%%%%%%%%
\theoremstyle{plain}
% \newtheorem{theorem}{Theorem}[section]
% \newtheorem{proposition}[theorem]{Proposition}
% \newtheorem{lemma}[theorem]{Lemma}
% \newtheorem{corollary}[theorem]{Corollary}
% \theoremstyle{definition}
% \newtheorem{definition}[theorem]{Definition}
% \newtheorem{assumption}[theorem]{Assumption}
% \theoremstyle{remark}
% \newtheorem{remark}[theorem]{Remark}

% Todonotes is useful during development; simply uncomment the next line
%    and comment out the line below the next line to turn off comments
%\usepackage[disable,textsize=tiny]{todonotes}
\usepackage[textsize=tiny]{todonotes}


% The \icmltitle you define below is probably too long as a header.
% Therefore, a short form for the running title is supplied here:
\icmltitlerunning{Submission and Formatting Instructions for ICML 2025}

\begin{document}

\twocolumn[
\icmltitle{A Geometric Approach to Personalized Recommendation \\ with Set-Theoretic Constraints Using Box Embeddings}

% It is OKAY to include author information, even for blind
% submissions: the style file will automatically remove it for you
% unless you've provided the [accepted] option to the icml2025
% package.

% List of affiliations: The first argument should be a (short)
% identifier you will use later to specify author affiliations
% Academic affiliations should list Department, University, City, Region, Country
% Industry affiliations should list Company, City, Region, Country

% You can specify symbols, otherwise they are numbered in order.
% Ideally, you should not use this facility. Affiliations will be numbered
% in order of appearance and this is the preferred way.
%\icmlsetsymbol{equal}{*}

\begin{icmlauthorlist}
\icmlauthor{Shib Dasgupta}{comp}
\icmlauthor{Michael Boratko}{comp}
\icmlauthor{Andrew McCallum}{comp}

\end{icmlauthorlist}

\icmlaffiliation{comp}{Manning College of Information \& Computer Sciences, UMass Amherst}
% \icmlaffiliation{comp}{Company Name, Location, Country}
% \icmlaffiliation{sch}{School of ZZZ, Institute of WWW, Location, Country}

\icmlcorrespondingauthor{Shib Dasgupta}{ssdasgupta@cs.umass.edu}
% \icmlcorrespondingauthor{Firstname2 Lastname2}{first2.last2@www.uk}

% You may provide any keywords that you
% find helpful for describing your paper; these are used to populate
% the "keywords" metadata in the PDF but will not be shown in the document
\icmlkeywords{Machine Learning, ICML}

\vskip 0.3in
]

% this must go after the closing bracket ] following \twocolumn[ ...

% This command actually creates the footnote in the first column
% listing the affiliations and the copyright notice.
% The command takes one argument, which is text to display at the start of the footnote.
% The \icmlEqualContribution command is standard text for equal contribution.
% Remove it (just {}) if you do not need this facility.

\printAffiliationsAndNotice{}  % leave blank if no need to mention equal contribution
%\printAffiliationsAndNotice{\icmlEqualContribution} % otherwise use the standard text.

\begin{abstract}
Personalized item recommendation typically suffers from data sparsity, which is most often addressed by learning vector representations of users and items via low-rank matrix factorization.
While this effectively densifies the matrix by assuming users and movies can be represented by linearly dependent latent features, it does not capture more complicated interactions.
% effectively express the notion of “similarity,” other relations between users and objects are not accurately captured.
For example, vector representations struggle with set-theoretic relationships, such as negation and intersection, \eg recommending a movie that is "comedy and action, but not romance".
In this work, we formulate the problem of personalized item recommendation as matrix completion where rows are \emph{set-theoretically dependent}.
% a novel set-theoretic framework building on box embeddings, representing both users and attributes as sets of items.
To capture this set-theoretic dependence we represent each user and attribute by a hyper-rectangle or \emph{box} (\ie a Cartesian product of intervals).
Box embeddings can intuitively be understood as trainable Venn diagrams, and thus not only inherently represent similarity (via the Jaccard index), but also naturally and faithfully support arbitrary set-theoretic relationships.
Queries involving set-theoretic constraints can be efficiently computed directly on the embedding space by performing geometric operations on the representations.
We empirically demonstrate the superiority of box embeddings over vector-based neural methods on both simple and complex item recommendation queries by up to $30\%$ overall.
\end{abstract}



\section{Introduction}
\label{sec:introduction}
Recommendation systems are a standard component of most online platforms, providing personalized suggestions for products, movies, articles, and more.
In addition to generic recommendation, these platforms often present the option for the user to search for items, either via natural language or structured queries.
While collaborative filtering methods like matrix factorization have proven successful in addressing data sparsity for unconditional generation, they often fall short when attempting to combine them with more complicated queries. 
This is not unexpected, as vector embeddings, while effectively capturing linear relationships, are ill-equipped to handle the complex set-theoretic relationships. Even advanced neural network-based approaches, which are designed to capture intricate relationships, have been shown to struggle with set-theoretic compositionally that underlie many real-world preferences. 

% Consider the common scenario where a user desires a movie that is both a "comedy" and "action," but not a "romance."
% This demonstrates a need for a recommendation model capable of handling set operations such as conjunction and negation.

% Recommending items according some logical constraints of their attributes is a key problem in many modern applications, such as e-commerce and video/music streaming platforms. These facets are invoked by simple user queries, which typically correspond to categories, tags, or attributes of the items. While some user queries are straightforward, like "comedy movies," more often they are complex, such as "comedy but not romantic comedies." 

Let us consider an example where a user named Bob wants to watch a comedy which is not a romantic comedy.
Assuming we have a prior watch history for users, standard collaborative filtering techniques (e.g. low-rank matrix factorization) would yield a learned score function $\score(m, \Bob)$ for each movie $m$.
% , however this does not incorporate Bob's search request.
If we also have movie-attribute annotations, we could form the set of comedies $C$ and set of romance movies $R$ and simply filter to those movies in $C \setminus R$, however this assumes that the movie-attribute annotations are complete, which is rarely the case.

A standard approach in a setting with sparse data is to learn a low-rank approximation for the {attribute $\times$ movie} matrix $\mathbf A$, yielding a dense matrix $\hat {\mathbf A}$. We can then form sets of movies based on this dense matrix using an (attribute-specific) threshold, \eg $\hat C \defeq \{m \mid \hat A_{\comedy, m} > \tau_\comedy\}$ and $\hat R \defeq \{m \mid \hat A_{\romance, m} > \tau_\romance\}$, and then rank movies $m \in \hat C \setminus \hat R$ according to $\score(m, \Bob)$. While this approach does allow for performing the sort of queries we are after, it suffers from three fundamental issues:

% \begin{figure}[h!]
%   \centering
%   \subfloat[Standard matrix completion assumes you are given partial information about the user $\times$ movie matrix $\mathbf U$, and potentially incomplete information about the attribute $\times$ movie matrix $\mathbf A$, and asks you to recover any unobserved entries. The task of set-theoretic matrix completion extends this to being able to predict the entries of arbitrary set-theoretic combinations of these rows.]{
%     \includegraphics[width=0.45\textwidth]{pictures/set-theoretic matrix completion.jpg}
%     \label{figure: set-theoretic matrix completion}
%   }
%   \hfill
%   \subfloat[Box embeddings represent the movies, users, and attributes as "boxes" (Cartesian products of intervals) in $\mathbb R^n$. The score for a specific movie in relation to a given query is determined by the proportion of the movie box's volume that falls within the corresponding region. During training, this membership score for a movie, w.r.t the $U$ and $A$ are optimized, creating a set-geometric representation of the matrix.]{
%     \includegraphics[width=0.45\textwidth]{pictures/box depiction.jpg}
%     \label{figure: box depiction}
%   }
%   \caption{Set-theoretic matrix completion for movies, users, and attributes, illustrating how box embeddings, trained in a set-theoretic manner, address this task.}
% \end{figure}
\begin{figure}[]
    \centering
    \includegraphics[width=0.8\columnwidth]{pictures/set-theoretic_matrix_completion.jpg}
    \caption{Standard matrix completion assumes you are given partial information about the user $\times$ movie matrix $\mathbf U$, and potentially incomplete information about the attribute $\times$ movie matrix $\mathbf A$.}
    \label{fig:set_theoretic_mc}
\end{figure}

\begin{figure}[]
    \centering
    \includegraphics[width=0.8\columnwidth]{pictures/box_depiction.jpg}
    \caption{Box embeddings represent the movies, users, and attributes as "boxes" (Cartesian products of intervals) in $\mathbb R^n$.}
    \label{fig:box_depiction}
\end{figure}


% \begin{figure}[h]
%     \centering
%     \includegraphics[width=0.8\textwidth]{ICLR 2025 Template/pictures/set-theoretic matrix completion.png} % Adjust width as necessary
%     \caption{The task of set-theoretic matrix completion depicted in the setting where users and attributes form the rows, and movies are the columns. Set-theoretic matrix completion is concerned with not simply filling in additional entries of the user $\times$ Movie matrix $\mathbf U$ or the attribute $\times$ movie matrix $\mathbf A$, but also being able to predict the entries of arbitrary set-theoretic combinations of these rows.}
%     \label{fig:side_caption_image}
% \end{figure}

\begin{enumerate}
    \item Limited user-attribute interaction:
    % separately classifying attributes and then ranking for each user does not take into account user-attribute interactions.
    Since the attribute classification is done independently from the user, any latent relationships between the user and attribute cannot be taken into account.
    \item Error compounding: Errors in the completion of attribute sets accumulate as the number of sets involved in the query increase.
    \item Mismatched inductive-bias: Our queries can be viewed as set-theoretic combinations of the rows, not linear combinations. As such, using a low-rank approximation of the matrix may be misaligned with the eventual use.
\end{enumerate}


% The recommender system has access to the ground truth of the set of movies Bob would like to watch (\textbf{Bob}), the set of comedy movies (\textbf{comedy}), and the set of romantic movies (\textbf{romance}). In this ideal scenario, the system would trivially return \textbf{Bob} $\cap$ \textbf{comedy} $\setminus$ \textbf{romance}. However, in practice, we can only construct these sets from item tags and user history, which are often incomplete and noisy. Consequently, the set operation might yield an inaccurate or empty set of items. This problem is exacerbated as the queries become more complex. (forward reference to experiment sections).

% A standard approach to mitigating the incompleteness issue is to learn representations of \Bob, \romance, and \comedy. One of the traditional yet most effective methods (cite) is to learn a low-rank approximation of the observed matrix $O$ which is the concatenation of the {User $\times$ Movie} interaction matrix $U$ and the {Tags $\times$ Movie} attribute matrix $A$. The learned representations can provide smooth score functions over all possible items for all users and attributes. In our example, we would be able to calculate $\score(\Bob, m)$, $\score(\comedy, m)$, and $\score(\romance, m)$ for all movies $m$ by calculating dot products between the vector representations for the each movie and the vector representations for \Bob, \comedy, and \romance.

% While these scores generalize to the incomplete part of the observed matrix $O$, they do not naturally allow us to compute set-theoretic queries. For example, consider how one might use these representations to address Bob's query from before. 

% This is not optimal for several reasons: the selection of the threshold is an ad-hoc process, and the prediction error for thresholding will snowball rapidly as query complexity increases (see Section ref). A better approach would be to devise a smooth score function for the entire query \textbf{Bob} $\cap$ \textbf{comedy} $\setminus$ \textbf{romance}. A common method to achieve this is by multiplying the scores corresponding to each query, e.g., $s(bob \cap comedy \cap \neg romance, m) = s(bob, m) \times s(comedy, m) \times (1 - s(romance, m))$. However, this approach ignores the interdependence between attributes and users, again resulting in suboptimal behavior for the recommender.\\

In this paper, we formulate the problem of attribute-specific recommendation as matrix completion where rows are not necessarily \emph{linear combinations} of each other but, rather, are \emph{set-theoretic combinations} of each other. More precisely, given some user $\times$ movie interaction matrix $\mathbf U$ and attribute $\times$ movie matrix $\mathbf A$, the queries we are considering are set-theoretic combinations of these rows (see \Cref{fig:set_theoretic_mc}). For example, the ground-truth data for comedies which are not romance movies which Bob likes would be the vector $x \in \{0,1\}^{|M|}$, where $x_m = 1$ if and only if $\mathbf U_{\Bob, m} = 1$ and $\mathbf A_{\comedy, m} = 1$ and $\mathbf A_{\romance, m} = 0$. Note that this is not a linear combination of the previous rows, and so while the inductive bias of low-rank factorization has proven immensely effective for collaborative filtering we should not expect it to be directly applicable in this setting.


% if the observed matrix $O$ is the concatenation of $[U; A]$, the query answering task essentially involves predicting the entries of the rows of the joint matrix $O_{q} = [U; A; U \cap A; U \cap \neg A; U \cap A \cap A; U \cap A \cap \neg A; \cdots]$. 

% Note that, the low-rank approximated vector model is capable of capturing linear dependencies between similar user or attribute rows or between movie columns. This inductive bias proves to be immensely effective for collaborative filtering. However, in our case the relationship amongst the rows of the $O_{q}$ is non-linear and strictly set-theoretic in nature, e.g., the row of \textbf{Bob} $\cap$ \textbf{comedy} is strictly an intersection between the individual rows of \textbf{Bob} and \textbf{comedy}. \\

Instead, we propose to learn representations for the users and attributes that are consistent with specific set-theoretic axioms. These representations must also be compactly parameterizable in a lower-dimensional space, differentiable with respect to some appropriate score function, and allow for efficient computation of various set operations.
% . Additionally, we need to define a measure (similar to vector dot products) to train these representations.
Box Embeddings \citep{hard_box, gumbel_box}, which are axis-parallel $n$-dimensional hyperrectangles, meet these criteria (see \Cref{fig:box_depiction}).
The volume of a box is easily calculated as the product of its side-lengths. Furthermore, box embeddings are closed under intersection (\ie the intersection of two boxes is another box). Inclusion-exclusion thus allows us to calculate the volume of arbitrary set-theoretic combinations of boxes.
% The simple axis-parallel geometry allows for the calculation of intersections of multiple boxes.
% The embedding space is closed under intersection (the intersection of two or more boxes is also a box) and the volume of a box is easily calculated as the product of its side lengths. Via inclusion-exclusion, this allows us to efficiently calculate arbitrary set-theoretic combinations of boxes.
% This ease of parameterization, along with straightforward volume and intersection calculations, makes box embeddings an excellent candidate for our purpose.


The contributions of our paper are as follows -
\begin{enumerate}
    \item We model the problem of attribute-specific query recommendation as "set-theoretic matrix completion", where attributes and users are treated as sets of items. We discuss the challenges faced by existing machine-learning approaches for this problem setup.
    \item We demonstrate the inconsistency of existing vector embedding models for this task. Additionally, we establish box embeddings as a suitable embedding method for addressing such set-theoretic problems.\mb{We don't do this, so we either need to or we need to weaken this claim.}
    \item We conduct an extensive empirical study comparing various vector and box embedding models for the task of set-theoretic query recommendation.
\end{enumerate}

Box embeddings, with their geometric set operations, significantly outperform all vector-based methods. We also evaluate score multiplication and threshold-based prediction for both vector and box embedding models, and find that performing set operations directly on the box embeddings performs best, solidifying our claim that the inductive bias of box embeddings provides the necessary generalization capabilities to address set-theoretic queries.
\section{Background} \label{section:LLM}

% \subsection{Large Language Model (LLM)}   

Figure~\ref{fig:LLaMA_model}(a) shows that a decoder-only LLM initially processes a user prompt in the “prefill” stage and subsequently generates tokens sequentially during the “decoding” stage.
Both stages contain an input embedding layer, multiple decoder transformer blocks, an output embedding layer, and a sampling layer.
Figure~\ref{fig:LLaMA_model}(b) demonstrates that the decoder transformer blocks consist of a self attention and a feed-forward network (FFN) layer, each paired with residual connection and normalization layers. 

% Differentiate between encoder/decoder, explain why operation intensity is low, explain the different parts of a transformer block. Discuss Table II here. 

% Explain the architecture with Llama2-70B.

% \begin{table}[thb]
% \renewcommand\arraystretch{1.05}
% \centering
% % \vspace{-5mm}
%     \caption{ML Model Parameter Size and Operational Intensity}
%     \vspace{-2mm}
%     \small
%     \label{tab:ML Model Parameter Size and Operational Intensity}    
%     \scalebox{0.95}{
%         \begin{tabular}{|c|c|c|c|c|}
%             \hline
%             & Llama2 & BLOOM & BERT & ResNet \\
%             Model & (70B) & (176B) & & 152 \\
%             \hline
%             Parameter Size (GB) & 140 & 352 & 0.17 & 0.16 \\
%             \hline
%             Op Intensity (Ops/Byte) & 1 & 1 & 282 & 346 \\
%             \hline
%           \end{tabular}
%     }
% \vspace{-3mm}
% \end{table}

% {\fontsize{8pt}{11pt}\selectfont 8pt font size test Memory Requirement}

\begin{figure}[t]
    \centering
    \includegraphics[width=8cm]{Figure/LLaMA_model_new_new.pdf}
    \caption{(a) Prefill stage encodes prompt tokens in parallel. Decoding stage generates output tokens sequentially.
    (b) LLM contains N$\times$ decoder transformer blocks. 
    (c) Llama2 model architecture.}
    \label{fig:LLaMA_model}
\end{figure}

Figure~\ref{fig:LLaMA_model}(c) demonstrates the Llama2~\cite{touvron2023llama} model architecture as a representative LLM.
% The self attention layer requires three GEMVs\footnote{GEMVs in multi-head attention~\cite{attention}, narrow GEMMs in grouped-query attention~\cite{gqa}.} to generate query, key and value vectors.
In the self-attention layer, query, key and value vectors are generated by multiplying input vector to corresponding weight matrices.
These matrices are segmented into multiple heads, representing different semantic dimensions.
The query and key vectors go though Rotary Positional Embedding (RoPE) to encode the relative positional information~\cite{rope-paper}.
Within each head, the generated key and value vectors are appended to their caches.
The query vector is multiplied by the key cache to produce a score vector.
After the Softmax operation, the score vector is multiplied by the value cache to yield the output vector.
The output vectors from all heads are concatenated and multiplied by output weight matrix, resulting in a vector that undergoes residual connection and Root Mean Square layer Normalization (RMSNorm)~\cite{rmsnorm-paper}.
The residual connection adds up the input and output vectors of a layer to avoid vanishing gradient~\cite{he2016deep}.
The FFN layer begins with two parallel fully connections, followed by a Sigmoid Linear Unit (SiLU), and ends with another fully connection.
\section{Study Design}
% robot: aliengo 
% We used the Unitree AlienGo quadruped robot. 
% See Appendix 1 in AlienGo Software Guide PDF
% Weight = 25kg, size (L,W,H) = (0.55, 0.35, 06) m when standing, (0.55, 0.35, 0.31) m when walking
% Handle is 0.4 m or 0.5 m. I'll need to check it to see which type it is.
We gathered input from primary stakeholders of the robot dog guide, divided into three subgroups: BVI individuals who have owned a dog guide, BVI individuals who were not dog guide owners, and sighted individuals with generally low degrees of familiarity with dog guides. While the main focus of this study was on the BVI participants, we elected to include survey responses from sighted participants given the importance of social acceptance of the robot by the general public, which could reflect upon the BVI users themselves and affect their interactions with the general population \cite{kayukawa2022perceive}. 

The need-finding processes consisted of two stages. During Stage 1, we conducted in-depth interviews with BVI participants, querying their experiences in using conventional assistive technologies and dog guides. During Stage 2, a large-scale survey was distributed to both BVI and sighted participants. 

This study was approved by the University’s Institutional Review Board (IRB), and all processes were conducted after obtaining the participants' consent.

\subsection{Stage 1: Interviews}
We recruited nine BVI participants (\textbf{Table}~\ref{tab:bvi-info}) for in-depth interviews, which lasted 45-90 minutes for current or former dog guide owners (DO) and 30-60 minutes for participants without dog guides (NDO). Group DO consisted of five participants, while Group NDO consisted of four participants.
% The interview participants were divided into two groups. Group DO (Dog guide Owner) consisted of five participants who were current or former dog guide owners and Group NDO (Non Dog guide Owner) consisted of three participants who were not dog guide owners. 
All participants were familiar with using white canes as a mobility aid. 

We recruited participants in both groups, DO and NDO, to gather data from those with substantial experience with dog guides, offering potentially more practical insights, and from those without prior experience, providing a perspective that may be less constrained and more open to novel approaches. 

We asked about the participants' overall impressions of a robot dog guide, expectations regarding its potential benefits and challenges compared to a conventional dog guide, their desired methods of giving commands and communicating with the robot dog guide, essential functionalities that the robot dog guide should offer, and their preferences for various aspects of the robot dog guide's form factors. 
For Group DO, we also included questions that asked about the participants' experiences with conventional dog guides. 

% We obtained permission to record the conversations for our records while simultaneously taking notes during the interviews. The interviews lasted 30-60 minutes for NDO participants and 45-90 minutes for DO participants. 

\subsection{Stage 2: Large-Scale Surveys} 
After gathering sufficient initial results from the interviews, we created an online survey for distributing to a larger pool of participants. The survey platform used was Qualtrics. 

\subsubsection{Survey Participants}
The survey had 100 participants divided into two primary groups. Group BVI consisted of 42 blind or visually impaired participants, and Group ST consisted of 58 sighted participants. \textbf{Table}~\ref{tab:survey-demographics} shows the demographic information of the survey participants. 

\subsubsection{Question Differentiation} 
Based on their responses to initial qualifying questions, survey participants were sorted into three subgroups: DO, NDO, and ST. Each participant was assigned one of three different versions of the survey. The surveys for BVI participants mirrored the interview categories (overall impressions, communication methods, functionalities, and form factors), but with a more quantitative approach rather than the open-ended questions used in interviews. The DO version included additional questions pertaining to their prior experience with dog guides. The ST version revolved around the participants' prior interactions with and feelings toward dog guides and dogs in general, their thoughts on a robot dog guide, and broad opinions on the aesthetic component of the robot's design. 


\section{Dataset}
\label{sec:dataset}

\subsection{Data Collection}

To analyze political discussions on Discord, we followed the methodology in \cite{singh2024Cross-Platform}, collecting messages from politically-oriented public servers in compliance with Discord's platform policies.

Using Discord's Discovery feature, we employed a web scraper to extract server invitation links, names, and descriptions, focusing on public servers accessible without participation. Invitation links were used to access data via the Discord API. To ensure relevance, we filtered servers using keywords related to the 2024 U.S. elections (e.g., Trump, Kamala, MAGA), as outlined in \cite{balasubramanian2024publicdatasettrackingsocial}. This resulted in 302 server links, further narrowed to 81 English-speaking, politics-focused servers based on their names and descriptions.

Public messages were retrieved from these servers using the Discord API, collecting metadata such as \textit{content}, \textit{user ID}, \textit{username}, \textit{timestamp}, \textit{bot flag}, \textit{mentions}, and \textit{interactions}. Through this process, we gathered \textbf{33,373,229 messages} from \textbf{82,109 users} across \textbf{81 servers}, including \textbf{1,912,750 messages} from \textbf{633 bots}. Data collection occurred between November 13th and 15th, covering messages sent from January 1st to November 12th, just after the 2024 U.S. election.

\subsection{Characterizing the Political Spectrum}
\label{sec:timeline}

A key aspect of our research is distinguishing between Republican- and Democratic-aligned Discord servers. To categorize their political alignment, we relied on server names and self-descriptions, which often include rules, community guidelines, and references to key ideologies or figures. Each server's name and description were manually reviewed based on predefined, objective criteria, focusing on explicit political themes or mentions of prominent figures. This process allowed us to classify servers into three categories, ensuring a systematic and unbiased alignment determination.

\begin{itemize}
    \item \textbf{Republican-aligned}: Servers referencing Republican and right-wing and ideologies, movements, or figures (e.g., MAGA, Conservative, Traditional, Trump).  
    \item \textbf{Democratic-aligned}: Servers mentioning Democratic and left-wing ideologies, movements, or figures (e.g., Progressive, Liberal, Socialist, Biden, Kamala).  
    \item \textbf{Unaligned}: Servers with no defined spectrum and ideologies or opened to general political debate from all orientations.
\end{itemize}

To ensure the reliability and consistency of our classification, three independent reviewers assessed the classification following the specified set of criteria. The inter-rater agreement of their classifications was evaluated using Fleiss' Kappa \cite{fleiss1971measuring}, with a resulting Kappa value of \( 0.8191 \), indicating an almost perfect agreement among the reviewers. Disagreements were resolved by adopting the majority classification, as there were no instances where a server received different classifications from all three reviewers. This process guaranteed the consistency and accuracy of the final categorization.

Through this process, we identified \textbf{7 Republican-aligned servers}, \textbf{9 Democratic-aligned servers}, and \textbf{65 unaligned servers}.

Table \ref{tab:statistics} shows the statistics of the collected data. Notably, while Democratic- and Republican-aligned servers had a comparable number of user messages, users in the latter servers were significantly more active, posting more than double the number of messages per user compared to their Democratic counterparts. 
This suggests that, in our sample, Democratic-aligned servers attract more users, but these users were less engaged in text-based discussions. Additionally, around 10\% of the messages across all server categories were posted by bots. 

\subsection{Temporal Data} 

Throughout this paper, we refer to the election candidates using the names adopted by their respective campaigns: \textit{Kamala}, \textit{Biden}, and \textit{Trump}. To examine how the content of text messages evolves based on the political alignment of servers, we divided the 2024 election year into three periods: \textbf{Biden vs Trump} (January 1 to July 21), \textbf{Kamala vs Trump} (July 21 to September 20), and the \textbf{Voting Period} (after September 20). These periods reflect key phases of the election: the early campaign dominated by Biden and Trump, the shift in dynamics with Kamala Harris replacing Joe Biden as the Democratic candidate, and the final voting stage focused on electoral outcomes and their implications. This segmentation enables an analysis of how discourse responds to pivotal electoral moments.

Figure \ref{fig:line-plot} illustrates the distribution of messages over time, highlighting trends in total messages volume and mentions of each candidate. Prior to Biden's withdrawal on July 21, mentions of Biden and Trump were relatively balanced. However, following Kamala's entry into the race, mentions of Trump surged significantly, a trend further amplified by an assassination attempt on him, solidifying his dominance in the discourse. The only instance where Trump’s mentions were exceeded occurred during the first debate, as concerns about Biden’s age and cognitive abilities temporarily shifted the focus. In the final stages of the election, mentions of all three candidates rose, with Trump’s mentions peaking as he emerged as the victor.
\section{Experimental Methodology}\label{sec:exp}
In this section, we introduce the datasets, evaluation metrics, baselines, and implementation details used in our experiments. More experimental details are shown in Appendix~\ref{app:experiment_detail}.

\textbf{Dataset.}
We utilize various datasets for training and evaluation. Data statistics are shown in Table~\ref{tab:dataset}.

\textit{Training.}
We use the publicly available E5 dataset~\cite{wang2024improving,springer2024repetition} to train both the LLM-QE and dense retrievers. We concentrate on English-based question answering tasks and collect a total of 808,740 queries. From this set, we randomly sample 100,000 queries to construct the DPO training data, while the remaining queries are used for contrastive training. During the DPO preference pair construction, we first prompt LLMs to generate expansion documents, filtering out queries where the expanded documents share low similarity with the query. This results in a final set of 30,000 queries.

\textit{Evaluation.}
We evaluate retrieval effectiveness using two retrieval benchmarks: MS MARCO \cite{bajaj2016ms} and BEIR \cite{thakur2021beir}, in both unsupervised and supervised settings.

\textbf{Evaluation Metrics.}
We use nDCG@10 as the evaluation metric. Statistical significance is tested using a permutation test with $p<0.05$.

\textbf{Baselines.} We compare our LLM-QE model with three unsupervised retrieval models and five query expansion baseline models.
% —

Three unsupervised retrieval models—BM25~\cite{robertson2009probabilistic}, CoCondenser~\cite{gao2022unsupervised}, and Contriever~\cite{izacard2021unsupervised}—are evaluated in the experiments. Among these, Contriever serves as our primary baseline retrieval model, as it is used as the backbone model to assess the query expansion performance of LLM-QE. Additionally, we compare LLM-QE with Contriever in a supervised setting using the same training dataset.

For query expansion, we benchmark against five methods: Pseudo-Relevance Feedback (PRF), Q2Q, Q2E, Q2C, and Q2D. PRF is specifically implemented following the approach in~\citet{yu2021improving}, which enhances query understanding by extracting keywords from query-related documents. The Q2Q, Q2E, Q2C, and Q2D methods~\cite{jagerman2023query,li2024can} expand the original query by prompting LLMs to generate query-related queries, keywords, chains-of-thought~\cite{wei2022chain}, and documents.


\textbf{Implementation Details.} 
For our query expansion model, we deploy the Meta-LLaMA-3-8B-Instruct~\cite{llama3modelcard} as the backbone for the query expansion generator. The batch size is set to 16, and the learning rate is set to $2e-5$. Optimization is performed using the AdamW optimizer. We employ LoRA~\cite{hu2022lora} to efficiently fine-tune the model for 2 epochs. The temperature for the construction of the DPO data varies across $\tau \in \{0.8, 0.9, 1.0, 1.1\}$, with each setting sampled eight times. For the dense retriever, we utilize Contriever~\cite{izacard2021unsupervised} as the backbone. During training, we set the batch size to 1,024 and the learning rate to $3e-5$, with the model trained for 3 epochs.

% \begin{table}[!t]
% \centering
% \scalebox{0.68}{
%     \begin{tabular}{ll cccc}
%       \toprule
%       & \multicolumn{4}{c}{\textbf{Intellipro Dataset}}\\
%       & \multicolumn{2}{c}{Rank Resume} & \multicolumn{2}{c}{Rank Job} \\
%       \cmidrule(lr){2-3} \cmidrule(lr){4-5} 
%       \textbf{Method}
%       &  Recall@100 & nDCG@100 & Recall@10 & nDCG@10 \\
%       \midrule
%       \confitold{}
%       & 71.28 &34.79 &76.50 &52.57 
%       \\
%       \cmidrule{2-5}
%       \confitsimple{}
%     & 82.53 &48.17
%        & 85.58 &64.91
     
%        \\
%        +\RunnerUpMiningShort{}
%     &85.43 &50.99 &91.38 &71.34 
%       \\
%       +\HyReShort
%         &- & -
%        &-&-\\
       
%       \bottomrule

%     \end{tabular}
%   }
% \caption{Ablation studies using Jina-v2-base as the encoder. ``\confitsimple{}'' refers using a simplified encoder architecture. \framework{} trains \confitsimple{} with \RunnerUpMiningShort{} and \HyReShort{}.}
% \label{tbl:ablation}
% \end{table}
\begin{table*}[!t]
\centering
\scalebox{0.75}{
    \begin{tabular}{l cccc cccc}
      \toprule
      & \multicolumn{4}{c}{\textbf{Recruiting Dataset}}
      & \multicolumn{4}{c}{\textbf{AliYun Dataset}}\\
      & \multicolumn{2}{c}{Rank Resume} & \multicolumn{2}{c}{Rank Job} 
      & \multicolumn{2}{c}{Rank Resume} & \multicolumn{2}{c}{Rank Job}\\
      \cmidrule(lr){2-3} \cmidrule(lr){4-5} 
      \cmidrule(lr){6-7} \cmidrule(lr){8-9} 
      \textbf{Method}
      & Recall@100 & nDCG@100 & Recall@10 & nDCG@10
      & Recall@100 & nDCG@100 & Recall@10 & nDCG@10\\
      \midrule
      \confitold{}
      & 71.28 & 34.79 & 76.50 & 52.57 
      & 87.81 & 65.06 & 72.39 & 56.12
      \\
      \cmidrule{2-9}
      \confitsimple{}
      & 82.53 & 48.17 & 85.58 & 64.91
      & 94.90&78.40 & 78.70& 65.45
       \\
      +\HyReShort{}
       &85.28 & 49.50
       &90.25 & 70.22
       & 96.62&81.99 & \textbf{81.16}& 67.63
       \\
      +\RunnerUpMiningShort{}
       % & 85.14& 49.82
       % &90.75&72.51
       & \textbf{86.13}&\textbf{51.90} & \textbf{94.25}&\textbf{73.32}
       & \textbf{97.07}&\textbf{83.11} & 80.49& \textbf{68.02}
       \\
   %     +\RunnerUpMiningShort{}
   %    & 85.43 & 50.99 & 91.38 & 71.34 
   %    & 96.24 & 82.95 & 80.12 & 66.96
   %    \\
   %    +\HyReShort{} old
   %     &85.28 & 49.50
   %     &90.25 & 70.22
   %     & 96.62&81.99 & 81.16& 67.63
   %     \\
   % +\HyReShort{} 
   %     % & 85.14& 49.82
   %     % &90.75&72.51
   %     & 86.83&51.77 &92.00 &72.04
   %     & 97.07&83.11 & 80.49& 68.02
   %     \\
      \bottomrule

    \end{tabular}
  }
\caption{\framework{} ablation studies. ``\confitsimple{}'' refers using a simplified encoder architecture. \framework{} trains \confitsimple{} with \RunnerUpMiningShort{} and \HyReShort{}. We use Jina-v2-base as the encoder due to its better performance.
}
\label{tbl:ablation}
\end{table*}

\section{Results}
\label{sec:results}

In this section, we present detailed results demonstrating \emph{CellFlow}'s state-of-the-art performance in cellular morphology prediction under perturbations, outperforming existing methods across multiple datasets and evaluation metrics.

\subsection{Datasets}

Our experiments were conducted using three cell imaging perturbation datasets: BBBC021 (chemical perturbation)~\cite{caie2010high}, RxRx1 (genetic perturbation)~\cite{sypetkowski2023rxrx1}, and the JUMP dataset (combined perturbation)~\cite{chandrasekaran2023jump}. We followed the preprocessing protocol from IMPA~\cite{palma2023predicting}, which involves correcting illumination, cropping images centered on nuclei to a resolution of 96×96, and filtering out low-quality images. The resulting datasets include 98K, 171K, and 424K images with 3, 5, and 6 channels, respectively, from 26, 1,042, and 747 perturbation types. Examples of these images are provided in Figure~\ref{fig:comparison}. Details of datasets are provided in \S\ref{sec:data}.

\subsection{Experimental Setup}

\textbf{Evaluation metrics.} We evaluate methods using two types of metrics: (1) FID and KID, which measure image distribution similarity via Fréchet and kernel-based distances, computed on 5K generated images for BBBC021 and 100 randomly selected perturbation classes for RxRx1 and JUMP; we report both overall scores across all samples and conditional scores per perturbation class. (2) Mode of Action (MoA) classification accuracy, which assesses biological fidelity by using a trained classifier to predict a drug’s effect from perturbed images and comparing it to its known MoA from the literature.

\textbf{Baselines.} We compare our approach against two baselines, PhenDiff~\cite{bourou2024phendiff} and IMPA~\cite{palma2023predicting}, the only two baselines that incorporate control images into their model design --- a crucial setup for distinguishing true perturbation effects from artifacts such as batch effects. PhenDiff uses diffusion models to first map control images to noise and then transform the noise into target images. In contrast, IMPA employs GANs with an AdaIN layer to transfer the style of control images to target images, specifically designed for paired image-to-image mappings. Our method uses flow matching, which is tailored for distribution-to-distribution mapping, providing a more suitable solution for our problem. We reproduce these baselines with official codes.

\textbf{Training details.} \emph{CellFlow} employs a UNet-based velocity field with a four-stage design. Perturbations are encoded following IMPA~\cite{palma2023predicting}. Training is conducted for 100 epochs on 4 A100 GPUs. Details are in \S\ref{sec:experimental}.


\begin{table*}[ht!]
\centering
\footnotesize
\renewcommand{\arraystretch}{1.2}
\begin{tabular}{p{2.15cm} p{14.05cm}}
\toprule
 & \includegraphics[height=2.6cm]{figure/qualitative_example.pdf}\\
\midrule
Original \phantom{00000} Description & Bromazepam binds to the GABA receptor GABA-A, causing a conformational change and \textbf{increasing inhibitory effects of GABA}. Other neurotransmitters are not influenced. One of the \textbf{benzodiazepines} that is \textbf{used in the treatment of anxiety disorders.} \\
\midrule
User & Explain the features of the given molecule that are most distinct from those of other molecules. \\
\midrule
Mol-LLaMA (Ours) & The given molecule is a member of the \textcolor{tabblue}{benzodiazepine class}, which is characterized by a fusion of a benzene ring and a seven-membered diazepine ring. This structure is crucial for its biological activity, particularly its role as \textcolor{tabblue}{a sedative and hypnotic agent}. ...$<$omitted$>$...\\

& The given molecule is also characterized by a ketone group within the diazepine ring, which is a common feature in benzodiazepines. \textcolor{tabblue}{This functional group is important for the interaction with the GABA-A receptor, enhancing the inhibitory effects of GABA}. ...$<$omitted$>$...\\

\midrule
GPT-4o & 1. **Quinazoline Core**: The molecule contains a \textcolor{tabred}{quinazoline core}, ...$<$omitted$>$... \\
 & Overall, the combination of a \textcolor{tabred}{quinazoline} structure with unique substitution patterns, including bromine halogenation and the presence of a pyridinyl group, ...$<$omitted$>$... . \\
\midrule
LLaMo & The molecule has a \textcolor{tabblue}{benzodiazepine} structure with a bromo substituent at the 7-position and \textcolor{tabred}{a methyl group at the 1-position}. It is distinct from other molecules in that it contains a bromine atom, a nitrogen atom, and a methyl group, which are not present in the other molecules. \\
\midrule
3D-MoLM & ...$<$omitted$>$... It belongs to the class of compounds called \textcolor{tabred}{quinazolines}, which are characterized by a bicyclic structure consisting of a benzene ring fused to a pyrimidine ring. ...$<$omitted$>$...\\
& The molecule's structure suggests potential applications in medicinal chemistry, as quinazolines have been found to possess various biological activities, including \textcolor{tabred}{antitumor, antimicrobial, and anti-inflammatory properties.} \\
\midrule
Mol-Instructions & The molecule is a \textcolor{tabred}{quinoxaline derivative}. \\
\bottomrule
\end{tabular}
\vspace{-0.1in}
\caption{\small Case study to compare molecular understanding and reasoning ability. Mol-LLaMA accurately understands the molecular features, answering a correct molecular taxonomy and providing its distinct properties that are relevant to the given molecule.}
\label{tab:qualitative}
\vspace{-0.1in}
\end{table*}

\subsection{Main Results}

\textbf{\emph{CellFlow} generates highly realistic cell images.}  
\emph{CellFlow} outperforms existing methods in capturing cellular morphology across all datasets (Table~\ref{tab:results}a), achieving overall FID scores of 18.7, 33.0, and 9.0 on BBBC021, RxRx1, and JUMP, respectively --- improving FID by 21\%–45\% compared to previous methods. These gains in both FID and KID metrics confirm that \emph{CellFlow} produces significantly more realistic cell images than prior approaches.

\textbf{\emph{CellFlow} accurately captures perturbation-specific morphological changes.}  
As shown in Table~\ref{tab:results}a, \emph{CellFlow} achieves conditional FID scores of 56.8 (a 26\% improvement), 163.5, and 84.4 (a 16\% improvement) on BBBC021, RxRx1, and JUMP, respectively. These scores are computed by measuring the distribution distance for each specific perturbation and averaging across all perturbations.   
Table~\ref{tab:results}b further highlights \emph{CellFlow}’s performance on six representative chemical and three genetic perturbations. For chemical perturbations, \emph{CellFlow} reduces FID scores by 14–55\% compared to prior methods.
The smaller improvement (5–12\% improvements) on RxRx1 is likely due to the limited number of images per perturbation type.

\textbf{\emph{CellFlow} preserves biological fidelity across perturbation conditions.} 
Table~\ref{tab:ablation}a presents mode of action (MoA) classification accuracy on the BBBC021 dataset using generated cell images. MoA describes how a drug affects cellular function and can be inferred from morphology. To assess this, we train an image classifier on real perturbed images and test it on generated ones. \emph{CellFlow} achieves 71.1\% MoA accuracy, closely matching real images (72.4\%) and significantly surpassing other methods (best: 63.7\%), demonstrating its ability to maintain biological fidelity across perturbations. Qualitative comparisons in Figure~\ref{fig:comparison} further highlight \emph{CellFlow}’s accuracy in capturing key biological effects. For example, demecolcine produces smaller, fragmented nuclei, which other methods fail to reproduce accurately.

\textbf{\emph{CellFlow} generalizes to out-of-distribution (OOD) perturbations.}  
On BBBC021, \emph{CellFlow} demonstrates strong generalization to novel chemical perturbations never seen during training (Table~\ref{tab:ablation}b). It achieves 6\% and 28\% improvements in overall and conditional FID over the best baseline. This OOD generalization is critical for biological research, enabling the exploration of previously untested interventions and the design of new drugs.

\textbf{Ablations highlight the importance of each component in \emph{CellFlow}.}  
Table~\ref{tab:ablation}c shows that removing conditional information, classifier-free guidance, or noise augmentation significantly degrades performance, leading to higher FID scores. These underscore the critical role of each component in enabling \emph{CellFlow}’s state-of-the-art performance.  

\begin{figure*}[!tb]
    \centering
     \includegraphics[width=\linewidth]{imgs/interpolation.pdf}
     \vspace{-2em}
    \caption{
    \textbf{\emph{CellFlow} enables new capabilities.} 
\textit{(a.1) Batch effect calibration.}  
\emph{CellFlow} initializes with control images, enabling batch-specific predictions. Comparing predictions from different batches highlights actual perturbation effects (smaller cell size) while filtering out spurious batch effects (cell density variations).  
\textit{(a.2) Interpolation trajectory.}  
\emph{CellFlow}'s learned velocity field supports interpolation between cell states, which might provide insights into the dynamic cell trajectory. 
\textit{(b) Diffusion model comparison.}  
Unlike flow matching, diffusion models that start from noise cannot calibrate batch effects or support interpolation.  
\textit{(c) Reverse trajectory.}  
\emph{CellFlow}'s reversible velocity field can predict prior cell states from perturbed images, offering potential applications such as restoring damaged cells.
    }
    \label{fig:interpolation}
    \vspace{-1em}
\end{figure*}

\subsection{New Capabilities}

\textbf{\emph{CellFlow} addresses batch effects and reveals true perturbation effects.}  
\emph{CellFlow}’s distribution-to-distribution approach effectively addresses batch effects, a significant challenge in biological experimental data collection. As shown in Figure~\ref{fig:interpolation}a, when conditioned on two distinct control images with varying cell densities from different batches, \emph{CellFlow} consistently generates the expected perturbation effect (cell shrinkage due to mevinolin) while recapitulating batch-specific artifacts, revealing the true perturbation effect. Table~\ref{tab:ablation}d further quantifies the importance of conditioning on the same batch. By comparing generated images conditioned on control images from the same or different batches against the target perturbation images, we find that same-batch conditioning reduces overall and conditional FID by 21\%. This highlights the importance of modeling control images to more accurately capture true perturbation effects—an aspect often overlooked by prior approaches, such as diffusion models that initialize from noise (Figure~\ref{fig:interpolation}b).

\textbf{\emph{CellFlow} has the potential to model cellular morphological change trajectories.}
Cell trajectories could offer valuable information about perturbation mechanisms, but capturing them with current imaging technologies remains challenging due to their destructive nature. Since \emph{CellFlow} continuously transforms the source distribution into the target distribution, it can generate smooth interpolation paths between initial and final predicted cell states, producing video-like sequences of cellular transformation based on given source images (Figure~\ref{fig:interpolation}a). This suggests a possible approach for simulating morphological trajectories during perturbation response, which diffusion methods cannot achieve (Figure~\ref{fig:interpolation}b). Additionally, the reversible distribution transformation learned through flow matching enables \emph{CellFlow} to model backward cell state reversion (Figure~\ref{fig:interpolation}c), which could be useful for studying recovery dynamics and predicting potential treatment outcomes.

%% New Disucssion 
Our study reveals how heavy users integrate LLMs into their daily tasks through distinct patterns. Rather than simple tool usage, participants demonstrated sophisticated cognitive offloading strategies that transformed their decision-making processes. In our study, we observed participants delegating social and interpersonal reasoning to LLMs, suggesting ways users might leverage AI collaboration to support their social cognition processes.

Participants' mental models of LLMs directly influenced their cognitive strategies---those viewing LLMs as rational entities engaged in cognitive complementarity by leveraging LLM capabilities where they perceived personal limitations, while those viewing LLMs as average decision-makers used cognitive benchmarking, establishing baseline standards while reserving higher-order tasks for themselves.
% While delegating a broad range of decisions raised potential concerns about over-reliance and diminished critical thinking, our findings also highlight a nuanced form of human-AI collaboration where users and LLMs develop complementary relationships. Participants showed diverse usage strategies, treating LLMs as an emerging problem-solving tool and developing sophisticated prompting techniques. Most notably, participants frequently sought LLM guidance on social appropriateness and interpersonal situations. Although some users expressed concerns about potential skill degradation and a sense of unease, LLM consultations often led to a more thorough consideration of social factors and an enhanced understanding of different perspectives.

This raises questions for future research on redefining how we conceptualize and measure over-reliance on LLMs. Current metrics typically assess over-reliance through simplified quantitative measures in controlled settings, primarily focusing on users' acceptance rates of LLM outputs ~\cite{bo2024rely, kim2024rely}. However, our findings reveal more complex patterns of engagement. Participants did not blindly adopt LLM outputs, even in cases where they eventually accepted them. Instead, participants demonstrated thoughtful delegation strategies, using LLMs to validate existing decisions, automate routine tasks, or navigate unfamiliar situations. The critical concern was not users' acceptance of LLM outputs, but rather instances where users adopted LLM reasoning without exploring alternative perspectives. Future research should expand the definition of over-reliance beyond simple acceptance rates to examine how users critically engage with alternative lines of reasoning.

Another key direction for future research involves capturing diverse user contexts. Our participants valued the ability of LLMs to extract necessary contextual information when not initially provided. They appreciated that they could receive meaningful responses without extensively explaining background information, even for context-heavy topics like relationship advice. Future research should explore ways to incorporate multi-modal inputs beyond text-based interactions, allowing users to convey context through various channels. Additionally, LLMs' ability to elicit implicit user intentions without explicit prompting is crucial, as demonstrated by recent advances in reasoning-focused LLM architectures that can proactively identify and address underlying user needs.

The development of active usage patterns with LLMs appeared more prominent among younger users who had less experience managing tasks without these systems. Participants with extensive pre-LLM experience maintained clearer boundaries and showed greater awareness of system limitations. In contrast, users with less experience with LLMs demonstrated fewer reservations, viewing LLM interaction itself as a skill and actively developing their prompting strategies. Conducting design studies focused on younger generations, to better understand and support these emerging interaction patterns represents a crucial direction for future research.

\pagebreak

\bibliography{citation}
\bibliographystyle{icml2025}

\pagebreak

\appendix
% \paragraph{Data-to-Text.} \citep{kukich-d2t, mckeown-d2t} is the task of converting structured data into fluent text. These structured data may correspond to tables \citep{totto}, meaning representations \citep{e2e}, relational graphs \citep{webnlg2017}, etc.
% %This complex format poses a significant challenge to LLMs pre-trained on plain text. 
% Recent approaches to data-to-text typically involve training end-to-end models with encoder-decoder architectures \citep{wiseman-etal-2017-challenges, gardent2017creating,RebuffelSSG20,RebuffelSSG20-ECIR,RebuffelRSSCG22}. Notably, using large pre-trained encoder-decoder models \citep{t5} has significantly improved performance by framing data-to-text as a text-to-text task \citep{kale-rastogi-2020-text, duong23a}. More recently, large pre-trained decoder-only models \citep{llama2} have shown strong performance and become the de facto approach for text generation, now being applied to data-to-text \citep{tablellama}. Despite these advancements, LLMs still struggle with hallucinations, and data-to-text generation is no exception.
This section reviews methods aimed at improving the faithfulness of LLMs to input contexts. We focus exclusively on approaches designed to ensure the generated content remains grounded in the provided information, excluding techniques related to factuality or external knowledge alignment.

\paragraph{Faithfulness enhancement.} Several methods have been used for improving faithfulness of text summarization. A first line of work consist in using external tools to retrieve key entities or facts form the source document and use these as weak labels during training \citep{zhang-etal-2022-improving-faithfulness}. \citet{faitful-improv} identify key entities using a Question-Answering system and modify the architecture of an encoder-decoder model to put more cross-attention weight on these entities. \citet{zhu-etal-2021-enhancing} propose to improve the faithfulness of summaries by extracting a knowledge graph from the input texts and embed it in the model cross-attention using a graph-transformer. Another line of work focuses on post-training improvements by bootstrapping model-generated outputs ranked by quality \citep{slic,brio,slic-nli}.
% \citet{zhang-etal-2022-improving-faithfulness} forces , \citet{faitful-improv} introduce a Question-Answering system enhanced encoder-decoder architecture, where the cross-attention in the decoder is directed towards key entities. \citet{zhu-etal-2021-enhancing} propose to improve the faithfulness of summaries by extracting a knowledge graph from the input texts and embed it in the model cross-attention using a graph-transformer.
Regarding data-to-text generation, \citet{RebuffelRSSCG22} propose a custom model architecture to reduce the effect of loosely aligned datasets, using token-level annotations and a multi-branch decoder model. The closest work to ours is from \citep{cao-wang-2021-cliff} which proposes a contrastive learning approach where synthetic samples are constructed using different tools like Named Entity Recognition (NER) models and back-translation.
%These approaches have been primarily designed and evaluated for text summarization. 
These approaches address specific forms of unfaithfulness and rely heavily on external tools such as NER or QA models, and are especially tailored for text summarization, while we target a more general focus. More recently, simpler methods that leverage only a pre-trained model have been proposed for summarization. \citet{cad,pmi} downweight the probabilities of tokens that are not grounded in the input context, using an auxiliary LM without access to the input context.
\citet{critic-driven} train a self-supervised classification model to detect hallucinations and guide the decoding process.  \cite{confident-decoding} propose a method to estimate the decoder's confidence by analyzing cross-attention weights, encouraging greater focus on the source during generation. Our method focuses on a decoder-only architecture and uses a single model, providing a streamlined and efficient approach specifically tailored for general conditional text generation tasks without the need for complex external tools.

\paragraph{Faithfulness evaluation.} Measuring faithfulness automatically is not straightforward. Traditional conditional text generation evaluation often relies on comparing the generated output to a reference text, typically measured using n-gram based metrics such as BLEU \citep{papineni-bleu} or ROUGE \citep{lin-2004-rouge}. However, reference-based metrics limitations are well known to correlate poorly with faithfulness \citep{fabbri-etal-2021-summeval,gabriel-etal-2021-go}. Both for summarization and data-to-text generation, new metrics evaluating the generation exclusively against the input context have been proposed, using QA models \citep{rebuffel-etal-2021-data,scialom-etal-2021-questeval} or entity-matching metrics \citep{nan-etal-2021-entity}. In this work, we evaluate primarily our models using recent NLI-related metrics \citep{alignscore, nli-d2t}, and LLM-as-a-judge, focusing on faithfulness \citep{gpt-chiang,gpt-gilardi}. For data-to-text generation, we also report the PARENT metric \citep{parent}, which computes n-gram overlap against elements of the source table cells.

%Additionally, corpora are often collected automatically, leading to divergences between the reference text and the actual input data. , since no direct comparison to the actual input source is actually performed. To address these issues, evaluation methods that take into account the input data have been proposed. \citet{parent} introduce PARENT, which computes the recall of n-gram overlap between the entities in the data and the candidate text. \citet{nli-d2t} develop an entailment metric using Natural Language Inference (NLI) models, where the generated text is compared directly to a simple verbalization of the data. The gold-standard still remains the human or human-like evaluation, conducted with powerful generalist LLMs. These metrics form the core focus of our work.

\paragraph{Preference tuning.} Recent instruction-tuned LLMs are often further refined through "human-feedback alignment" \citep{oaif}. These methods utilize human-crafted preference datasets, consisting of pairs of preferred and dispreferred texts $(\ywin, \ylose)$, typically obtained by collecting human feedback and ranking responses via voting. Recent work \citep{spin} uses the model's previous predictions in a self-play manner to iteratively improve the performance of chat-based models. Whether through an auxiliary preference model \citep{rlhf} or by directly tuning the models on the pairs \citep{dpo}, these approaches have demonstrated remarkable results in chat-based models. Our method leverages a preference framework without the need for human intervention and is specifically tailored for models trained on conditional text generation tasks.

% However, it remains unclear on what values the models are being aligned. Some works have shown that these methods can effectively alter the model's behaviour to the extent that they become useless and refuse to answer to any requests. In this work, we follow a preference fine-tuning scheme but tailored for input-aware tasks like data-to-text.


\section{Experiment Details}
\subsection{Data Splits \& Query Generation}
\label{app:data_split}

\begin{algorithm}
\caption{\textsc{Personalised Simple Query} ($u \cap a$) generation algorithm $u \cap a$}
\begin{algorithmic}[1]
    \STATE Let the set of users, attributes, and movies be $\mathcal{U}, \mathcal{A}, \mathcal{M}$
    \STATE Marginal probability of an attribute $a$ in $A$, $P(a) = \sum_{m} A_{a, m} / \sum_{a'} \sum_{m} A_{a', m}$
    \STATE Marginal probability of an user $u$ in $U$, $P(u) = \sum_{m} U_{u, m} / \sum_{u'} \sum_{m} U_{u', m}$
    \STATE Marginal probability of an movie $m$ in $U$, $P(m) = \sum_{u} U_{u, m} / \sum_{u} \sum_{m'} U_{u, m'}$
    \STATE Let $U$ be the User $\times$ Item matrix and $A$ be the Attribute $\times$ Item matrix.
    \STATE $U^{Train} \leftarrow U$, $A^{Train} \leftarrow A$
    \STATE $U^{Eval} \leftarrow \mathbf{0}$, $A^{Eval} \leftarrow \mathbf{0}$
    \STATE Set of simple personalized queries, $Q_{U \cap A} \leftarrow \phi$
    \WHILE{$|Q_{U \cap A}|$ < \textsc{Max Sample Size}}
        \STATE Sample an attribute $a$ from $\mathcal{A}$ according to $P(a)$.
        \STATE Sample a movie $m$ from for the attribute $a$, i.e., Sample from $\{m' | A_{a, m'} = 1\}$, according to $P(m)$
        \STATE Sample a user $u$ from who has rated movie $m$, i.e., Sample from  $\{u' | U_{m, u'} = 1\}$, according to $P(u)$
        \STATE $U^{Train}_{u, m} = 0$, $A^{Train}_{a, m} = 0$, $U^{Eval}_{u, m} = 1$, $A^{Eval}_{a, m} = 1$
        \STATE $Q_{U \cap A}$.\textsc{insert}($(u, a, m)$)
    \ENDWHILE
\end{algorithmic}
\label{alg:joint_sampling}
\end{algorithm}

\begin{algorithm}
\caption{\textsc{Personalised Complex Query} Generation Algorithm}
\begin{algorithmic}[1]
    \STATE Compositional Query sets $Q_{U \cap A_1 \cap A_2}$, $Q_{U \cap A_1 \cap \neg A_2}$
    \STATE Non-Trivial attribute combination set $\mathcal{A}_{\circ}$
    \FOR{each user-movie tuple in Eval set, i.e., $(u, m) \in \{(u, m) | U^{Eval}_{u, m} = 1\}$}
        \FOR{each pair of attributes $(a_1, a_2) \in \{(a_1, a_2) | A^{Eval}_{a_1, m} = 1 \text{ and } A^{Eval}_{a_2, m} = 1\}$}
            \IF{the pair is viable and non-trivial, i.e., $(a_1, a_2) \in \mathcal{A}_{\cap}$}
                \STATE $Q_{U \cap A_1 \cap A_2}$.\textsc{insert}($(u, a_1, a_2, m)$)
            \ENDIF
        \ENDFOR
        \FOR{each pair of attributes $(a_1, a_2) \in \{(a_1, a_2) | A^{Eval}_{a_1, m} = 1 \text{ and } A_{a_2, m} = 0\}$}
            \IF{the pair is viable and non-trivial, i.e., $(a_1, a_2) \in \mathcal{A}_{\setminus}$}
                \STATE $Q_{U \cap A_1 \cap \neg A_2}$.\textsc{insert}($(u, a_1, a_2, m)$)
            \ENDIF
        \ENDFOR
    \ENDFOR
\end{algorithmic}
\label{alg:complex_query}
\end{algorithm}

\subsection{Training Details}
\label{app:training_details}

\begin{table}[H]
\caption{Hyper Parameter range for all the dataset. We run 100 runs for both models and select the best model on User-Movie validation set NDCG metric}
\resizebox{\columnwidth}{!}{%

\begin{tabular}{ccccc}
\hline
Hyperparameters                         & \begin{tabular}[c]{@{}c@{}}Range\\ Box\end{tabular} & \begin{tabular}[c]{@{}c@{}}Best Value\\ Box\end{tabular} & \begin{tabular}[c]{@{}c@{}}Range\\ Vector\end{tabular} & \begin{tabular}[c]{@{}c@{}}Best Value\\ Vector\end{tabular} \\ \hline
Embedding dim                            & 64                                                  & 64                                                       & 128                                                    & 128                                                         \\
Learning Rate                            & 1e-1, 1e-2, 1e-3, 1e-4, 1e-5                        & 0.001                                                    & 1e-1, 1e-2, 1e-3, 1e-4, 1e-5                           & 0.001                                                       \\
Batch Size                               & 64, 128, 256, 512, 1024                             & 128                                                      & 64, 128, 256, 512, 1024                                & 128                                                         \\
\# Negatives                             & 1, 5, 10, 20                                        & 20                                                       & 1, 5, 10, 20                                           & 5                                                           \\
\multicolumn{1}{l}{Intersection Temp}    & 10, 2, 1, 1e-1, 1e-2, 1e-3, 1e-5                    & 2.0                                                      & -                                                      & -                                                           \\
\multicolumn{1}{l}{Volume Temp}          & 10, 5, 1, 0.1, 0.01, 0.001                          & 0.01                                                     & -                                                      & -                                                           \\
\multicolumn{1}{l}{Attribute Loss const} & 0.1, 0.3, 0.5, 0.7, 0.9                             & 0.7                                                      & 0.1, 0.3, 0.5, 0.7, 0.9                                & 0.5                                                         \\ \hline
\end{tabular}
}
\label{tab:hyperparams}
\end{table}
Hyperparameters are reported in Table \ref{tab:hyperparams}. Best parameter values are reported for Box Embeddings and \textsc{MF} method. 
\begin{figure*}[ht]
    \centering
    \includegraphics[width=0.8\textwidth]{pictures/hparam_search.png} % Adjust the width as needed
    \caption{Parallel Co-ordinate plot for different hyperparameters vs model performance. Lighter the color, better the model's performance.}
    \label{fig:generalization-spectrum}
\end{figure*}

\subsection{Model Selection}
\begin{table}[t]
    \centering
    \caption{Test NDCG on $D_{U}^\eval$ for selected models.}
    \scalebox{0.9}{
    \begin{tabular}{lllll}
        \toprule
        Dataset & \textsc{MF}   & \textsc{NeuMF} & \textsc{Lgcn} & \textsc{Box}  \\ \hline
        \addlinespace
        Last-FM & 0.51 & 0.52 & 0.56 & 0.65 \\
        NYC-R   & 0.31 & 0.33 & 0.37 & 0.39 \\
        ML-1M   & 0.51 & 0.53 & 0.55 & 0.58 \\
        ML-20M  & 0.71 & 0.70 & 0.72 & 0.73 \\ 
        \bottomrule
    \end{tabular}
    }
    \label{tab:model_selection}
\end{table}


\subsection{Set-Theoretic Generalization}
\begin{table}[H]
\caption{Hit Rate(\%)$\uparrow$ for Set-theoretic queries for dataset ML-20M. }
\resizebox{\columnwidth}{!}{%
\begin{tabular}{llllllllll}
\hline
\multicolumn{1}{c}{\multirow{2}{*}{Methods}} & \multicolumn{3}{c}{$U \cap A$} & \multicolumn{3}{c}{$U \cap A_1 \cap A_2$} & \multicolumn{3}{c}{$U \cap A_1 \cap \neg A_2$} \\ \cline{2-10} 
\multicolumn{1}{c}{}                         & h@10    & h@20    & h@50    & h@10        & h@20       & h@50       & h@10         & h@20          & h@50         \\ \hline
\addlinespace
\textsc{MF-Filter}         & 4.6      & 8.1      & 16.1     & 0.4          & 1.0         & 2.9         & 3.7             & 6.6              & 13.7             \\
\textsc{MF-Product}        & 4.1      & 7.5      & 15.6     & 3.3          & 6.6         & 16.4        & 2.7           & 5.1            & 11.4          \\
\textsc{MF-Geometric}      & 0.1      & 0.3      & 0.6      & 0.0          & 0.0         & 0.0         & 0.3           & 0.6            & 1.4           \\ \hdashline
\addlinespace
\textsc{NeuMF-Filter} & 4.6 & 8.2 &  16.1 & 1.1 & 5.6 & 6.4 & 4.9 & 7.3 & 13.9 \\
\textsc{NeuMF-product} & 4.6 & 8.2 & 16.1 & 4.1 & 8.5 & 22.1 & 4.3 & 6.9 & 12.0 \\ \hdashline
\addlinespace
\textsc{Box-Filter}         & 4.6      & 8.1      & 16.1     & 11.0         & 21.8        & 42.3        & 4.6           & 7.7            & 16.3           \\
\textsc{Box-Product}        & 4.5      & 8.2      & 16.1     & 11.1         & 21.8        & 42.5        & 4.3           & 7.1            & 15.1           \\
\textsc{Box-Geometric}      & 4.5      & 8.1      & 16.2     & 11.0         & 21.8        & 42.4        & \textbf{6.4}  & \textbf{12.8}  & \textbf{25.9} \\ \hline
\end{tabular}
}
\label{tab:set-theretic-results-ml20m}
\end{table}



\subsection{Spectrum of Weak Generalization}
\label{app:weak_generalization}

\begin{table}[H]
\caption{The spectrum of generalization for \textsc{Simple Personalized query} $U \cap A$. W: \textsc{Weakest Generalization}, W-U: \textsc{Weak Generalization-User}, W-A: \textsc{Weak Generalization-Attribute}, S: \textsc{Set Theoretic Generalization}}
\resizebox{\columnwidth}{!}{%

\begin{tabular}{llllllllll}
\hline
\multicolumn{1}{c}{\multirow{2}{*}{Methods}} & \multicolumn{3}{c}{Hit Rate @10}              & \multicolumn{3}{c}{Hit Rate @ 20}             & \multicolumn{3}{c}{Hit Rate @ 50}                      \\ \cline{2-10} 
\multicolumn{1}{c}{}                         & \multicolumn{3}{l}{W | W-U | W-A | S}         & \multicolumn{3}{l}{W | W-U | W-A | S}         & \multicolumn{3}{l}{W | W-U | W-A | S}                  \\ \hline
\textsc{MF-Filter}                         & \multicolumn{3}{l}{24.7 | 6.7 | 13.0 | 5.0}   & \multicolumn{3}{l}{36.3 | 13.3 | 20.7 | 10.2} & \multicolumn{3}{l}{54.2 | 30.1 | 33.3 | 22.3}          \\
\textsc{MF-Product}                        & \multicolumn{3}{l}{23.3 | 5.7 | 13.1 | 4.3}   & \multicolumn{3}{l}{35.0 | 10.8 | 21.4 | 8.5}  & \multicolumn{3}{l}{54.7 | 24.2 | 38.8 | 20.4}          \\
\textsc{MF-Geometric}                      & \multicolumn{3}{l}{4.9 | 0.9 | 1.8 | 0.4}     & \multicolumn{3}{l}{7.9 | 1.7 | 3.3 | 0.9}     & \multicolumn{3}{l}{15.1 | 4.5 | 7.4 | 3.0}             \\ \hline
\textsc{Box-Filter}                         & \multicolumn{3}{l}{24.1 | 13.0 | 16.4 | 11.7} & \multicolumn{3}{l}{34.5 | 22.3 | 24.6 | 19.1} & \multicolumn{3}{l}{50.5 | 40.5 | 37.6 | 32.3}          \\
\textsc{Box-Product}                        & \multicolumn{3}{l}{25.2 | 13.6 | 13.9 | 10.0} & \multicolumn{3}{l}{35.2 | 21.5 | 21.9 | 16.7} & \multicolumn{3}{l}{52.2 | 38.4 | 38.3 | 31.5}          \\
\textsc{Box-Geometric}                      & \multicolumn{3}{l}{25.4 | 14.7 | 14.8 | 11.0} & \multicolumn{3}{l}{35.6 | 23.3 | 23.5 | 18.3} & \multicolumn{3}{l}{\textbf{52.2 | 40.8 | 40.5 | 34.1}} \\ \hline
\end{tabular}
}
\label{tab:generalization-spectrum-simple-query}
\end{table}

\begin{table}[H]
\caption{The spectrum of generalization for \textsc{Complex Personalized query} $U \cap A_1 \cap \neg A_2$. W: \textsc{Weakest Generalization}, W-U: \textsc{Weak Generalization-User}, W-A: \textsc{Weak Generalization-Attribute}, S: \textsc{Set Theoretic Generalization}}
\resizebox{\columnwidth}{!}{%
\begin{tabular}{llllllllll}
\hline
\multicolumn{1}{c}{\multirow{2}{*}{Methods}} & \multicolumn{3}{c}{Hit Rate @10}              & \multicolumn{3}{c}{Hit Rate @ 20}             & \multicolumn{3}{c}{Hit Rate @ 50}                      \\ \cline{2-10} 
\multicolumn{1}{c}{}                         & \multicolumn{3}{l}{W | W-U | W-A | S}         & \multicolumn{3}{l}{W | W-U | W-A | S}         & \multicolumn{3}{l}{W | W-U | W-A | S}                  \\ \hline
\textsc{MF-Filter}                        & \multicolumn{3}{l}{25.5 | 13.0 | 12.4 | 4.7}  & \multicolumn{3}{l}{34.9 | 14.1 | 19.5 | 9.8}  & \multicolumn{3}{l}{54.7 | 29.5 | 37.1 | 22.5}          \\
\textsc{MF-Product}                       & \multicolumn{3}{l}{23.5 | 7.0 | 10.4 | 3.4}   & \multicolumn{3}{l}{34.9 | 12.8 | 18.0 | 7.3}  & \multicolumn{3}{l}{54.5 | 27.5 | 35.0 | 19.3}          \\
\textsc{MF-Geometric}                     & \multicolumn{3}{l}{5.2 | 2.0 | 1.7 | 0.5}     & \multicolumn{3}{l}{8.8 | 3.5 | 1.9 | 1.0}     & \multicolumn{3}{l}{17.4 | 8.8 | 6.5 | 2.7}             \\ \hline
\textsc{Box-Filter}                        & \multicolumn{3}{l}{24.1 | 15.3 | 15.0 | 11.4} & \multicolumn{3}{l}{35.5| 22.7 | 21.1 | 19.5}  & \multicolumn{3}{l}{\textbf{54.1 | 39.2 | 37.3 | 34.0}} \\
\textsc{Box-Product}                       & \multicolumn{3}{l}{21.1 | 13.7 | 12.0 | 8.9}  & \multicolumn{3}{l}{30.5 | 21.7 | 19.3 | 15.2} & \multicolumn{3}{l}{47.4 | 38.0 | 35.0 | 29.4}          \\
\textsc{Box-Geometric}                     & \multicolumn{3}{l}{21.1 | 13.2 | 10.8 | 8.6}  & \multicolumn{3}{l}{30.4 | 20.8 | 17.7 | 15.1} & \multicolumn{3}{l}{\textbf{47.3 | 36.6 | 33.2 | 31.0}} \\ \hline
\end{tabular}
}
\label{tab:generalization-spectrum-difference-query}
\end{table}

\begin{table}[H]
\caption{The spectrum of generalization for \textsc{Complex Personalized query} $U \cap A_1 \cap A_2$. W: \textsc{Weakest Generalization}, W-U: \textsc{Weak Generalization-User}, W-A: \textsc{Weak Generalization-Attribute}, S: \textsc{Set Theoretic Generalization}}
\resizebox{\columnwidth}{!}{%
\begin{tabular}{llllllllll}
\hline
\multicolumn{1}{c}{\multirow{2}{*}{Methods}} & \multicolumn{3}{c}{Hit Rate @10}              & \multicolumn{3}{c}{Hit Rate @ 20}             & \multicolumn{3}{c}{Hit Rate @ 50}                      \\ \cline{2-10} 
\multicolumn{1}{c}{}                         & \multicolumn{3}{l}{W | W-U | W-A | S}         & \multicolumn{3}{l}{W | W-U | W-A | S}         & \multicolumn{3}{l}{W | W-U | W-A | S}                  \\ \hline
\textsc{MF-Filter}             & \multicolumn{3}{l}{35.3 | 17.6 | 16.9 | 11.4} & \multicolumn{3}{l}{45.0 | 27.3 | 23.3 | 17.9} & \multicolumn{3}{l}{55.2 | 41.9 | 30.5 | 27.5}          \\
\textsc{MF-Product}            & \multicolumn{3}{l}{34.0 | 11.0 | 11.6 | 5.1}  & \multicolumn{3}{l}{47.3 | 19.6 | 20.1 | 10.6} & \multicolumn{3}{l}{67.4 | 38.5 | 39.3 | 26.1}          \\
\textsc{MF-Geometric}          & \multicolumn{3}{l}{6.13 | 3.1 | 0.3 | 0.1}    & \multicolumn{3}{l}{9.90 | 5.8 | 0.6 | 0.2}    & \multicolumn{3}{l}{18.5 | 12.9 | 1.8 | 0.8}            \\ \hline
\textsc{Box-Filter}             & \multicolumn{3}{l}{30.8 | 21.5 | 17.3 | 14.5} & \multicolumn{3}{l}{41.1 | 31.2 | 23.3 | 20.5} & \multicolumn{3}{l}{52.7 | 44.5 | 30.3 | 28.5}          \\
\textsc{Box-Product}            & \multicolumn{3}{l}{35.4 | 23.8 | 13.4 | 10.6} & \multicolumn{3}{l}{47.0 | 34.5 | 21.7 | 17.8} & \multicolumn{3}{l}{64.6 | 52.8 | 39.0 | 34.2}          \\
\textsc{Box-Geometric}          & \multicolumn{3}{l}{34.6 | 25.2 | 20.0 | 16.8} & \multicolumn{3}{l}{45.7 | 35.7 | 30.5 | 26.6} & \multicolumn{3}{l}{\textbf{62.6 | 53.3 | 50.1 | 46.1}} \\ \hline
\end{tabular}
}
\label{tab:generalization-spectrum-intersection-query}
\end{table}

% \begin{table}[H]
% \resizebox{\columnwidth}{!}{%
% \begin{tabular}{l|cccc|cccc|cccc}
% \hline
% \multicolumn{1}{c|}{\multirow{2}{*}{Methods}} & \multicolumn{4}{c|}{Hit Rate @10} & \multicolumn{4}{c|}{Hit Rate @20} & \multicolumn{4}{c}{Hit Rate @50} \\ \cline{2-13} 
% \multicolumn{1}{c|}{} & W & W-U & W-A & S & W & W-U & W-A & S & W & W-U & W-A & S \\ \hline
% \textsc{MF-Filter}    & 35.3 & 17.6 & 16.9 & 11.4 & 45.0 & 27.3 & 23.3 & 17.9 & 55.2 & 41.9 & 30.5 & 27.5 \\
% \textsc{MF-Product}   & 34.0 & 11.0 & 11.6 & 5.1  & 47.3 & 19.6 & 20.1 & 10.6 & 67.4 & 38.5 & 39.3 & 26.1 \\
% \textsc{MF-Geometric} & 6.13 & 3.1  & 0.3  & 0.1  & 9.90 & 5.8  & 0.6  & 0.2  & 18.5 & 12.9 & 1.8  & 0.8  \\ \hline
% \textsc{Box-Filter}    & 30.8 & 21.5 & 17.3 & 14.5 & 41.1 & 31.2 & 23.3 & 20.5 & 52.7 & 44.5 & 30.3 & 28.5 \\
% \textsc{Box-Product}   & 35.4 & 23.8 & 13.4 & 10.6 & 47.0 & 34.5 & 21.7 & 17.8 & 64.6 & 52.8 & 39.0 & 34.2 \\
% \textsc{Box-Geometric} & 34.6 & 25.2 & 20.0 & 16.8 & 45.7 & 35.7 & 30.5 & 26.6 & \textbf{62.6} & 53.3 & 50.1 & 46.1 \\ \hline
% \end{tabular}
% }
% \caption{Hit Rate Results}
% \label{tab:generalization-spectrum-intersection-query-2}
% \end{table}

\begin{figure}[ht!]
  \centering
  \includegraphics[width=\columnwidth]{pictures/weak_generaliztion.png}
  \caption{Weak Generalization Illustration}
  \label{fig:weak_generalization}
\end{figure}


The \textsc{Box-Geometric} achieves the best \textit{Generalization Spectrum Gap} for all types of queries.

\section{Error Compounding Analysis}
\label{app:error_compounding}

\begin{figure}[ht!]
    \centering
    \includegraphics[width=0.4\textwidth]{pictures/all_success.png}
    \caption{Relationships of correct answers by the three box models on $u \wedge a_1 \wedge a_2$ queries.}
    \label{fig:first-figure}
\end{figure}

\begin{figure}[ht!]
    \centering
    \includegraphics[width=0.4\textwidth]{pictures/compund_error_solved.png}
    \caption{The Geometric method subsumes the benefit of the product in compounding error.}
    \label{fig:second-figure}
\end{figure}

\begin{figure}[ht!]
    \centering
    \includegraphics[width=0.4\textwidth]{pictures/not_compunding_error.png}
    \caption{The effect is less for the non-compounding error.}
    \label{fig:third-figure}
\end{figure}

We further perform more granular analysis amongst the \textsc{Box} based methods with complex query type $U \cap A_1 \cap A_2$. As claimed in our initial hypothesis, the \textsc{Filter} method suffers from error compounding. If the target movie $m$ is in the model's prediction list for $A_1$ but not for $A_2$ or the other way round, we denote this error as \textit{compounding error}. In figure \ref{fig:second-figure}, out of the compounding errors, $34 \%$ is solved by the \textsc{Box-Geometric} method and $26 \%$ by the \textsc{Box-Product} method. However, in figure \ref{fig:third-figure}, for the error that is not due to compounding (where the model gets both $A_1$ and $A_2$ prediction wrong), only $18 \%$ are corrected by the \textsc{Box-Geometric} method and a mere $10 \%$ of them are corrected by \textsc{Box-Product}. Refer to figure \ref{fig:first-figure} \ref{fig:second-figure} \ref{fig:third-figure} for details. This demonstrates that the \textsc{Box-Geometric} significantly contributes to the correction of error compounding.


\section{{Time Efficiency analysis}}

\begin{table}[ht]
\centering
\caption{Training time (\textit{mm:ss}) for a single epoch are measured for different batch sizes with 5 negative samples on Movielens-1M dataset. Experiments are conducted on Nvidia GTX 1080Ti gpus}
\begin{tabular}{lllll}
\hline
\begin{tabular}[c]{@{}l@{}}Batch Size\end{tabular} & \textsc{MF} & \textsc{NeuMF} & \textsc{LightGCN} & \textsc{Box} \\ \hline
64                                                   & 08:37                        & 17:00                           & 70:30                            & 19:32                         \\
128                                                  & 04:32                        & 09:46                           & 38:40                              & 11:40                         \\
256                                                  & 02:29                        & 04:40                           & 20:55                              & 05:28                         \\
512                                                  & 01:18                        & 02:23                           & 10:47                              & 02:54                         \\
1024                                                 & 00:40                        & 01:20                           & 05:24                              & 01:12                         \\ \hline
\end{tabular}
\label{tab:training_time}
\end{table}

{In \Cref{tab:training_time}, we observe that the \textsc{MF}, being the simplest approach with minimal computational requirements, is consistently the fastest across all batch sizes. At the largest batch size (1024), it achieves the shortest training time of just 00:40. The \textsc{Box}-based method exhibits training times comparable to \textsc{NeuMF}. However, it is significantly faster than \textsc{LightGCN}, which relies on graph convolutional computations. The iterative message-passing operations required by \textsc{LightGCN} result in considerably higher training times, particularly at smaller batch sizes (e.g., 70:30 at a batch size of 64). As the batch size increases, the training time for \textsc{Box} embeddings becomes almost as efficient as \textsc{MF}. For instance, at a batch size of 1024, \textsc{Box} achieves a training time of 01:12, compared to 00:40 for \textsc{MF}. This demonstrates that the computational complexity of box embeddings is of the same order as \textsc{MF}, making it a scalable and efficient choice.}

{Box embeddings are generally quite fast because the computation of box intersection volumes can be parallelized over dimensions. Note that the training times above use GumbleBox embeddings, which involve log-sum-exp calculations. However, this could be improved even further at inference time by replacing these soft min and max approximations with hard operators. If such an optimized approach is desired, then training can accommodate this by regularizing temperature. For deployment in industrial set-up, we could take additional steps with Box Embeddings as outlined in \cite{box_for_search}.}



\end{document}


% This document was modified from the file originally made available by
% Pat Langley and Andrea Danyluk for ICML-2K. This version was created
% by Iain Murray in 2018, and modified by Alexandre Bouchard in
% 2019 and 2021 and by Csaba Szepesvari, Gang Niu and Sivan Sabato in 2022.
% Modified again in 2023 and 2024 by Sivan Sabato and Jonathan Scarlett.
% Previous contributors include Dan Roy, Lise Getoor and Tobias
% Scheffer, which was slightly modified from the 2010 version by
% Thorsten Joachims & Johannes Fuernkranz, slightly modified from the
% 2009 version by Kiri Wagstaff and Sam Roweis's 2008 version, which is
% slightly modified from Prasad Tadepalli's 2007 version which is a
% lightly changed version of the previous year's version by Andrew
% Moore, which was in turn edited from those of Kristian Kersting and
% Codrina Lauth. Alex Smola contributed to the algorithmic style files.
