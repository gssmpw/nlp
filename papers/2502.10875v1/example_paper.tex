%%%%%%%% ICML 2025 EXAMPLE LATEX SUBMISSION FILE %%%%%%%%%%%%%%%%%

\documentclass{article}

% Recommended, but optional, packages for figures and better typesetting:
\usepackage{microtype}
\usepackage{graphicx}
\usepackage{subfigure}
\usepackage{booktabs} % for professional tables

% hyperref makes hyperlinks in the resulting PDF.
% If your build breaks (sometimes temporarily if a hyperlink spans a page)
% please comment out the following usepackage line and replace
% \usepackage{icml2025} with \usepackage[nohyperref]{icml2025} above.
\usepackage{hyperref}


% Attempt to make hyperref and algorithmic work together better:
\newcommand{\theHalgorithm}{\arabic{algorithm}}

% Use the following line for the initial blind version submitted for review:
%\usepackage{icml2025}

% If accepted, instead use the following line for the camera-ready submission:
\usepackage[accepted]{icml2025}

% For theorems and such
\usepackage{amsmath}
\usepackage{amssymb}
\usepackage{mathtools}
\usepackage{amsthm}

% if you use cleveref..
\usepackage[capitalize,noabbrev]{cleveref}

\usepackage{hyperref}
\usepackage{url}

% Our files
\usepackage{packages/default_packages}
\usepackage{packages/mjb}
\usepackage{packages/colors}
\usepackage{packages/boxmath}
\usepackage{packages/colab}
\usepackage{packages/local_defs}

% Default packages for the document
\usepackage{subcaption}

% Added package
\usepackage{algorithm}
\usepackage{algorithmic}
\usepackage{multirow}
\usepackage{enumitem}
\usepackage{arydshln}
\usepackage{wrapfig}
\usepackage{booktabs}
\usepackage{graphicx}
\usepackage[normalem]{ulem}
\useunder{\uline}{\ul}{}

%%%%%%%%%%%%%%%%%%%%%%%%%%%%%%%%
% THEOREMS
%%%%%%%%%%%%%%%%%%%%%%%%%%%%%%%%
\theoremstyle{plain}
% \newtheorem{theorem}{Theorem}[section]
% \newtheorem{proposition}[theorem]{Proposition}
% \newtheorem{lemma}[theorem]{Lemma}
% \newtheorem{corollary}[theorem]{Corollary}
% \theoremstyle{definition}
% \newtheorem{definition}[theorem]{Definition}
% \newtheorem{assumption}[theorem]{Assumption}
% \theoremstyle{remark}
% \newtheorem{remark}[theorem]{Remark}

% Todonotes is useful during development; simply uncomment the next line
%    and comment out the line below the next line to turn off comments
%\usepackage[disable,textsize=tiny]{todonotes}
\usepackage[textsize=tiny]{todonotes}


% The \icmltitle you define below is probably too long as a header.
% Therefore, a short form for the running title is supplied here:
\icmltitlerunning{Submission and Formatting Instructions for ICML 2025}

\begin{document}

\twocolumn[
\icmltitle{A Geometric Approach to Personalized Recommendation \\ with Set-Theoretic Constraints Using Box Embeddings}

% It is OKAY to include author information, even for blind
% submissions: the style file will automatically remove it for you
% unless you've provided the [accepted] option to the icml2025
% package.

% List of affiliations: The first argument should be a (short)
% identifier you will use later to specify author affiliations
% Academic affiliations should list Department, University, City, Region, Country
% Industry affiliations should list Company, City, Region, Country

% You can specify symbols, otherwise they are numbered in order.
% Ideally, you should not use this facility. Affiliations will be numbered
% in order of appearance and this is the preferred way.
%\icmlsetsymbol{equal}{*}

\begin{icmlauthorlist}
\icmlauthor{Shib Dasgupta}{comp}
\icmlauthor{Michael Boratko}{comp}
\icmlauthor{Andrew McCallum}{comp}

\end{icmlauthorlist}

\icmlaffiliation{comp}{Manning College of Information \& Computer Sciences, UMass Amherst}
% \icmlaffiliation{comp}{Company Name, Location, Country}
% \icmlaffiliation{sch}{School of ZZZ, Institute of WWW, Location, Country}

\icmlcorrespondingauthor{Shib Dasgupta}{ssdasgupta@cs.umass.edu}
% \icmlcorrespondingauthor{Firstname2 Lastname2}{first2.last2@www.uk}

% You may provide any keywords that you
% find helpful for describing your paper; these are used to populate
% the "keywords" metadata in the PDF but will not be shown in the document
\icmlkeywords{Machine Learning, ICML}

\vskip 0.3in
]

% this must go after the closing bracket ] following \twocolumn[ ...

% This command actually creates the footnote in the first column
% listing the affiliations and the copyright notice.
% The command takes one argument, which is text to display at the start of the footnote.
% The \icmlEqualContribution command is standard text for equal contribution.
% Remove it (just {}) if you do not need this facility.

\printAffiliationsAndNotice{}  % leave blank if no need to mention equal contribution
%\printAffiliationsAndNotice{\icmlEqualContribution} % otherwise use the standard text.

\begin{abstract}
Personalized item recommendation typically suffers from data sparsity, which is most often addressed by learning vector representations of users and items via low-rank matrix factorization.
While this effectively densifies the matrix by assuming users and movies can be represented by linearly dependent latent features, it does not capture more complicated interactions.
% effectively express the notion of “similarity,” other relations between users and objects are not accurately captured.
For example, vector representations struggle with set-theoretic relationships, such as negation and intersection, \eg recommending a movie that is "comedy and action, but not romance".
In this work, we formulate the problem of personalized item recommendation as matrix completion where rows are \emph{set-theoretically dependent}.
% a novel set-theoretic framework building on box embeddings, representing both users and attributes as sets of items.
To capture this set-theoretic dependence we represent each user and attribute by a hyper-rectangle or \emph{box} (\ie a Cartesian product of intervals).
Box embeddings can intuitively be understood as trainable Venn diagrams, and thus not only inherently represent similarity (via the Jaccard index), but also naturally and faithfully support arbitrary set-theoretic relationships.
Queries involving set-theoretic constraints can be efficiently computed directly on the embedding space by performing geometric operations on the representations.
We empirically demonstrate the superiority of box embeddings over vector-based neural methods on both simple and complex item recommendation queries by up to $30\%$ overall.
\end{abstract}



\section{Introduction}
\label{sec:introduction}
Recommendation systems are a standard component of most online platforms, providing personalized suggestions for products, movies, articles, and more.
In addition to generic recommendation, these platforms often present the option for the user to search for items, either via natural language or structured queries.
While collaborative filtering methods like matrix factorization have proven successful in addressing data sparsity for unconditional generation, they often fall short when attempting to combine them with more complicated queries. 
This is not unexpected, as vector embeddings, while effectively capturing linear relationships, are ill-equipped to handle the complex set-theoretic relationships. Even advanced neural network-based approaches, which are designed to capture intricate relationships, have been shown to struggle with set-theoretic compositionally that underlie many real-world preferences. 

% Consider the common scenario where a user desires a movie that is both a "comedy" and "action," but not a "romance."
% This demonstrates a need for a recommendation model capable of handling set operations such as conjunction and negation.

% Recommending items according some logical constraints of their attributes is a key problem in many modern applications, such as e-commerce and video/music streaming platforms. These facets are invoked by simple user queries, which typically correspond to categories, tags, or attributes of the items. While some user queries are straightforward, like "comedy movies," more often they are complex, such as "comedy but not romantic comedies." 

Let us consider an example where a user named Bob wants to watch a comedy which is not a romantic comedy.
Assuming we have a prior watch history for users, standard collaborative filtering techniques (e.g. low-rank matrix factorization) would yield a learned score function $\score(m, \Bob)$ for each movie $m$.
% , however this does not incorporate Bob's search request.
If we also have movie-attribute annotations, we could form the set of comedies $C$ and set of romance movies $R$ and simply filter to those movies in $C \setminus R$, however this assumes that the movie-attribute annotations are complete, which is rarely the case.

A standard approach in a setting with sparse data is to learn a low-rank approximation for the {attribute $\times$ movie} matrix $\mathbf A$, yielding a dense matrix $\hat {\mathbf A}$. We can then form sets of movies based on this dense matrix using an (attribute-specific) threshold, \eg $\hat C \defeq \{m \mid \hat A_{\comedy, m} > \tau_\comedy\}$ and $\hat R \defeq \{m \mid \hat A_{\romance, m} > \tau_\romance\}$, and then rank movies $m \in \hat C \setminus \hat R$ according to $\score(m, \Bob)$. While this approach does allow for performing the sort of queries we are after, it suffers from three fundamental issues:

% \begin{figure}[h!]
%   \centering
%   \subfloat[Standard matrix completion assumes you are given partial information about the user $\times$ movie matrix $\mathbf U$, and potentially incomplete information about the attribute $\times$ movie matrix $\mathbf A$, and asks you to recover any unobserved entries. The task of set-theoretic matrix completion extends this to being able to predict the entries of arbitrary set-theoretic combinations of these rows.]{
%     \includegraphics[width=0.45\textwidth]{pictures/set-theoretic matrix completion.jpg}
%     \label{figure: set-theoretic matrix completion}
%   }
%   \hfill
%   \subfloat[Box embeddings represent the movies, users, and attributes as "boxes" (Cartesian products of intervals) in $\mathbb R^n$. The score for a specific movie in relation to a given query is determined by the proportion of the movie box's volume that falls within the corresponding region. During training, this membership score for a movie, w.r.t the $U$ and $A$ are optimized, creating a set-geometric representation of the matrix.]{
%     \includegraphics[width=0.45\textwidth]{pictures/box depiction.jpg}
%     \label{figure: box depiction}
%   }
%   \caption{Set-theoretic matrix completion for movies, users, and attributes, illustrating how box embeddings, trained in a set-theoretic manner, address this task.}
% \end{figure}
\begin{figure}[]
    \centering
    \includegraphics[width=0.8\columnwidth]{pictures/set-theoretic_matrix_completion.jpg}
    \caption{Standard matrix completion assumes you are given partial information about the user $\times$ movie matrix $\mathbf U$, and potentially incomplete information about the attribute $\times$ movie matrix $\mathbf A$.}
    \label{fig:set_theoretic_mc}
\end{figure}

\begin{figure}[]
    \centering
    \includegraphics[width=0.8\columnwidth]{pictures/box_depiction.jpg}
    \caption{Box embeddings represent the movies, users, and attributes as "boxes" (Cartesian products of intervals) in $\mathbb R^n$.}
    \label{fig:box_depiction}
\end{figure}


% \begin{figure}[h]
%     \centering
%     \includegraphics[width=0.8\textwidth]{ICLR 2025 Template/pictures/set-theoretic matrix completion.png} % Adjust width as necessary
%     \caption{The task of set-theoretic matrix completion depicted in the setting where users and attributes form the rows, and movies are the columns. Set-theoretic matrix completion is concerned with not simply filling in additional entries of the user $\times$ Movie matrix $\mathbf U$ or the attribute $\times$ movie matrix $\mathbf A$, but also being able to predict the entries of arbitrary set-theoretic combinations of these rows.}
%     \label{fig:side_caption_image}
% \end{figure}

\begin{enumerate}
    \item Limited user-attribute interaction:
    % separately classifying attributes and then ranking for each user does not take into account user-attribute interactions.
    Since the attribute classification is done independently from the user, any latent relationships between the user and attribute cannot be taken into account.
    \item Error compounding: Errors in the completion of attribute sets accumulate as the number of sets involved in the query increase.
    \item Mismatched inductive-bias: Our queries can be viewed as set-theoretic combinations of the rows, not linear combinations. As such, using a low-rank approximation of the matrix may be misaligned with the eventual use.
\end{enumerate}


% The recommender system has access to the ground truth of the set of movies Bob would like to watch (\textbf{Bob}), the set of comedy movies (\textbf{comedy}), and the set of romantic movies (\textbf{romance}). In this ideal scenario, the system would trivially return \textbf{Bob} $\cap$ \textbf{comedy} $\setminus$ \textbf{romance}. However, in practice, we can only construct these sets from item tags and user history, which are often incomplete and noisy. Consequently, the set operation might yield an inaccurate or empty set of items. This problem is exacerbated as the queries become more complex. (forward reference to experiment sections).

% A standard approach to mitigating the incompleteness issue is to learn representations of \Bob, \romance, and \comedy. One of the traditional yet most effective methods (cite) is to learn a low-rank approximation of the observed matrix $O$ which is the concatenation of the {User $\times$ Movie} interaction matrix $U$ and the {Tags $\times$ Movie} attribute matrix $A$. The learned representations can provide smooth score functions over all possible items for all users and attributes. In our example, we would be able to calculate $\score(\Bob, m)$, $\score(\comedy, m)$, and $\score(\romance, m)$ for all movies $m$ by calculating dot products between the vector representations for the each movie and the vector representations for \Bob, \comedy, and \romance.

% While these scores generalize to the incomplete part of the observed matrix $O$, they do not naturally allow us to compute set-theoretic queries. For example, consider how one might use these representations to address Bob's query from before. 

% This is not optimal for several reasons: the selection of the threshold is an ad-hoc process, and the prediction error for thresholding will snowball rapidly as query complexity increases (see Section ref). A better approach would be to devise a smooth score function for the entire query \textbf{Bob} $\cap$ \textbf{comedy} $\setminus$ \textbf{romance}. A common method to achieve this is by multiplying the scores corresponding to each query, e.g., $s(bob \cap comedy \cap \neg romance, m) = s(bob, m) \times s(comedy, m) \times (1 - s(romance, m))$. However, this approach ignores the interdependence between attributes and users, again resulting in suboptimal behavior for the recommender.\\

In this paper, we formulate the problem of attribute-specific recommendation as matrix completion where rows are not necessarily \emph{linear combinations} of each other but, rather, are \emph{set-theoretic combinations} of each other. More precisely, given some user $\times$ movie interaction matrix $\mathbf U$ and attribute $\times$ movie matrix $\mathbf A$, the queries we are considering are set-theoretic combinations of these rows (see \Cref{fig:set_theoretic_mc}). For example, the ground-truth data for comedies which are not romance movies which Bob likes would be the vector $x \in \{0,1\}^{|M|}$, where $x_m = 1$ if and only if $\mathbf U_{\Bob, m} = 1$ and $\mathbf A_{\comedy, m} = 1$ and $\mathbf A_{\romance, m} = 0$. Note that this is not a linear combination of the previous rows, and so while the inductive bias of low-rank factorization has proven immensely effective for collaborative filtering we should not expect it to be directly applicable in this setting.


% if the observed matrix $O$ is the concatenation of $[U; A]$, the query answering task essentially involves predicting the entries of the rows of the joint matrix $O_{q} = [U; A; U \cap A; U \cap \neg A; U \cap A \cap A; U \cap A \cap \neg A; \cdots]$. 

% Note that, the low-rank approximated vector model is capable of capturing linear dependencies between similar user or attribute rows or between movie columns. This inductive bias proves to be immensely effective for collaborative filtering. However, in our case the relationship amongst the rows of the $O_{q}$ is non-linear and strictly set-theoretic in nature, e.g., the row of \textbf{Bob} $\cap$ \textbf{comedy} is strictly an intersection between the individual rows of \textbf{Bob} and \textbf{comedy}. \\

Instead, we propose to learn representations for the users and attributes that are consistent with specific set-theoretic axioms. These representations must also be compactly parameterizable in a lower-dimensional space, differentiable with respect to some appropriate score function, and allow for efficient computation of various set operations.
% . Additionally, we need to define a measure (similar to vector dot products) to train these representations.
Box Embeddings \citep{hard_box, gumbel_box}, which are axis-parallel $n$-dimensional hyperrectangles, meet these criteria (see \Cref{fig:box_depiction}).
The volume of a box is easily calculated as the product of its side-lengths. Furthermore, box embeddings are closed under intersection (\ie the intersection of two boxes is another box). Inclusion-exclusion thus allows us to calculate the volume of arbitrary set-theoretic combinations of boxes.
% The simple axis-parallel geometry allows for the calculation of intersections of multiple boxes.
% The embedding space is closed under intersection (the intersection of two or more boxes is also a box) and the volume of a box is easily calculated as the product of its side lengths. Via inclusion-exclusion, this allows us to efficiently calculate arbitrary set-theoretic combinations of boxes.
% This ease of parameterization, along with straightforward volume and intersection calculations, makes box embeddings an excellent candidate for our purpose.


The contributions of our paper are as follows -
\begin{enumerate}
    \item We model the problem of attribute-specific query recommendation as "set-theoretic matrix completion", where attributes and users are treated as sets of items. We discuss the challenges faced by existing machine-learning approaches for this problem setup.
    \item We demonstrate the inconsistency of existing vector embedding models for this task. Additionally, we establish box embeddings as a suitable embedding method for addressing such set-theoretic problems.\mb{We don't do this, so we either need to or we need to weaken this claim.}
    \item We conduct an extensive empirical study comparing various vector and box embedding models for the task of set-theoretic query recommendation.
\end{enumerate}

Box embeddings, with their geometric set operations, significantly outperform all vector-based methods. We also evaluate score multiplication and threshold-based prediction for both vector and box embedding models, and find that performing set operations directly on the box embeddings performs best, solidifying our claim that the inductive bias of box embeddings provides the necessary generalization capabilities to address set-theoretic queries.
\section{Background}
\label{sec:background}


\subsection{Preliminaries}

{\color{red}[TODO: LLMs? in-context learning?]}

\subsection{Problem Definition}

{\color{red}[TODO: define the problem of citation intent]}



\section{Methodology}
\paragraph{Preliminaries.}
We primarily focus on the homologous model merging, in which $\boldsymbol{\theta}_i$ all come from the same base model $\boldsymbol{\theta}_{\rm{base}}$. Given $K$ tasks $\{T_1,T_2,\cdots,T_K\}$ and $K$ corresponding fine-tuned models with parameters $\{\boldsymbol{\theta}_1,\boldsymbol{\theta}_2,\cdots,\boldsymbol{\theta}_K\}$, model merging aims to combine $K$ fine-tuned models into one single model simultaneously performing on $\{T_1,T_2,\cdots,T_K\}$ without post-training~\cite{method_p1_1,method_p1_2}.
Task vector~\cite{ilharco2023editing,yang2024adamerging} is a key element in merging method which could enhances the base model‘s ability or enable the model to handle other tasks. Specifically, for task $T_i$, the task vector $\boldsymbol\tau_i\in \mathbb{R}^D$ is defined as the vector obtained by subtracting the SFT weights $\boldsymbol{\theta}_i$ from the base model weight
$\boldsymbol{\theta}_{\rm{base}}$, \emph{i.e.}, $\boldsymbol\tau_i=\boldsymbol{\theta}_i-\boldsymbol{\theta}_{\rm{base}}$. The merged model could be denoted as $\boldsymbol{\theta}_m=\boldsymbol{\theta}_{\rm{base}}+\sum_i \lambda_i\boldsymbol{\tau}_i$, which $\lambda_i$ is the scaling factor measuring the importance of task vector. For clarification, we also denote the neuron set in $\boldsymbol{\theta}_i$ as $\mathcal{N}_i$, the neuron set in $\boldsymbol{\tau}_i$ as $\mathcal{T}_i$.



\begin{algorithm}[!ht]
    \caption{LED-Merging}
    \label{alg1}
    \begin{algorithmic}[1]
        \REQUIRE  base model $\boldsymbol{\theta}_{\rm{base}}$, SFT models $\{\boldsymbol{\theta}_{i}\mid i\in [K]\}$, mask ratios \{$r_{i} \mid i\in [K]\}$, scaling factors $\{\lambda_i\mid i\in[K]\}$, location datasets $\{\mathcal{X}_{i}\mid i\in[K]\}$
        \ENSURE merged parameter $\boldsymbol{\theta}_{m}$
        \STATE $\mathcal{M}\leftarrow\phi$
        \STATE $\boldsymbol{\theta}_{m}\leftarrow \boldsymbol{\theta}_{\rm{base}}$
        \FOR{$i\in [K]$}
        \STATE $I(\boldsymbol{\theta}_i)=\mathbb{E}_{x\sim \mathcal{X}_i}|\boldsymbol{\theta}_{i}\odot \nabla_{\boldsymbol{\theta}_i}\mathcal{L}(x)|$
        \STATE $I(\boldsymbol{\theta}_{\rm{base}})=\mathbb{E}_{x\sim \mathcal{X}_i}|\boldsymbol{\theta}_{\rm{base}}\odot \nabla_{\boldsymbol{\theta}_{\rm{base}}}\mathcal{L}(x)|$
        
        \STATE calculate $\mathcal{T}^{r_i}_{i}$ following Equation \ref{vote}
        \STATE  $\mathcal{M}\leftarrow \mathcal{M}\cup\{\mathcal{T}^{r_i}_i\}$
       
        
   
        
        
        \ENDFOR  
        \FOR{$i\in [K]$}
        
        \STATE calculate $\text{Disjoint}(\mathcal{T}_i^{r_i})$ use Equation~\ref{disjoint_safety}
        \STATE $\boldsymbol{m}_i \leftarrow \boldsymbol{0}$
        \FOR{$d\in \mathcal{T}_i^{r_i}$}
        \STATE $\boldsymbol{m}_{i,d}=1$
        \ENDFOR
        \STATE $\boldsymbol{\theta}_{m}\leftarrow \boldsymbol{\theta}_{m}+\lambda_i \boldsymbol{\tau}_i\odot \boldsymbol{m}_{i}$
        \ENDFOR
    \end{algorithmic}
\end{algorithm}
    %\vspace{-5pt}
\begin{figure*}[h!]
    \centering
    \includegraphics[width=\linewidth]{figs/pipeline_v2.pdf}
    \vspace{-40mm}
    \caption{Overview of our two-stage training pipeline {\ours}.}
    \label{fig:pipeline}
\end{figure*}


\paragraph{LED-Merging: Location, Election, and Disjoint Merging}
To address the neuron misidentification and interference issues in existing model merging methods, we propose LED-Merging (Location, Election, and Disjoint Merging). Specifically, previous studies \cite{modelstock, ilharco2023editing, tiesmerging} fail to accurately identify safety-related neurons in task vectors with a single magnitude score, namely \textit{neuron misidentification}. Meanwhile, there exists an interference between safety-related and utility-related task vector neurons during the merging process, namely \textit{neuron interference}. To address neuron misidentification, we first locate important neurons both in the base and fine-tuned models and then elect neurons from the task vector considering these two scores together. Subsequently, to mitigate the interference, we introduce a disjoint step, isolating these important neurons so that they influence different base neurons. The whole process is illustrated in Figure~\ref{fig:method}. 




In the location and election step, we consider the importance score from base and fine-tuned models simultaneously to locate task-specific neurons. In this way, it is more accurate than relying on the magnitude score alone because task-specific neurons with high importance score in the fine-tuned model may not necessarily score high in the base model, and vice versa.

{\textbf{Location}}.  We first calculate importance scores for each neuron in a base/fine-tuned model. Given a location dataset $\mathcal{X}_i=\{(x,y)_k\}$, where $x$ is the question and $y$ is the answer, we calculate the importance scores for the weight $\boldsymbol{\theta}_i\in\mathbb{R}^D$ in any  layer as follows~\cite{snip,spareseGPT,sun2024a}:
\begin{equation}
    I(\boldsymbol{\theta}_i)=\mathbb{E}_{x\sim \mathcal{X}_i}[\boldsymbol{\theta}_i\odot \nabla _{\boldsymbol{\theta}_i}\mathcal{L}(x)],
    \label{location}
\end{equation}
which $\mathcal{L}(x)=-\log p(y\mid x)$ is the conditional negative log-likelihood loss. We choose the SNIP score~\cite{snip} because it balances computational efficiency and performance~\cite{cq}. Please refer to Sec.~\ref{sec:ablation} for the comparison between different location methods. After computing importance scores, we choose top-$r_i$ neurons as the important neuron subset $\mathcal{N}_{i}^{r_i}$ from $I(\boldsymbol{\theta}_i)$.
 
 % After computing locating scores, we select the neurons scoring both high in base and fine-tuned models as important neurons in task vectors. Then in the disjoint step,  with preventing  polysemantic neurons  from receiving gradient updates towards different directions,
 % we use set difference to isolate the safety   and utility-related neurons  and construct corresponding masks for merging process,

{\textbf{Election}}. A natural question is how to select important neurons in the task vector $\boldsymbol{\tau}_i$ based on $I(\boldsymbol{\theta}_{\rm{base}})$ and $I(\boldsymbol{\theta}_{i})$. The important neurons in the base model may be different from neurons in the fine-tuned model. Therefore, we introduce the following election strategy to select neurons with high scores in both base and fine-tuned models:
\begin{equation}
    \mathcal{T}_i^{r_i}=\mathcal{N}_i^{r_i}\cap \mathcal{N}_{\rm{base}}^{r_i}.
    \label{vote}
\end{equation}
\emph{Remark}. We compare different choosing methods, including scoring low or high in base or fine-tuned model in Section~\ref{sec:ablation} and find that Equation \ref{vote} achieves the best performance.





{\textbf{Disjoint}}. As important neurons from different task vectors may conflict with each other at the same position, we use the set difference to disjoint the neurons from others to prevent interference:
\begin{equation}
    \text{Disjoint}(\mathcal{T}^{r_i}_{i})=\mathcal{T}^{r_i}_{i}-\mathop{\cup}\limits_{{J}\subsetneqq [K],|J|\geq 2}\mathop{\cap}\limits_{j\in {J}}\mathcal{T}^{r_j}_{j}.
    \label{disjoint_safety}
\end{equation}

Next, we construct a mask $\boldsymbol{m}_i\in\mathbb{R}^D$ to implement disjoint in the merging process. Specifically, this mask $\boldsymbol{m}_i$ is used to select neurons from $\mathcal{T}_i$. The mask ratio is $r_i$, where $r\in(0,1]$. The mask $\boldsymbol{m}_i$ can be derived from:
\begin{equation}
    \boldsymbol{m}_{i,d}=\begin{aligned} &\left\{ \begin{array}{ll} 1, & \text{if } d\in \text{Disjoint}(\mathcal{T}_{i}^{r_i}), \\ 0, & \text{otherwise}. \end{array} \right. \end{aligned}
    \label{mask_safety}
\end{equation}


% \subsection{Merging Models with Masks}
{\textbf{Merging}}. The final
merged task vector $\boldsymbol{\tau}_m$ is as follows:
\begin{equation}
    \boldsymbol{\tau}_m= \sum_i \lambda_i\boldsymbol{\tau}_{i}\odot\boldsymbol{m}_i.
    \label{merged_task_vector}
\end{equation}
We summarize the workflow in Algorithm \ref{alg1}.



\section{Experiments}
\label{sec:experiment}

Experiments are carried out on NVIDIA RTX4090 GPUs using PyTorch 2.2.0 \cite{paszke2019pytorch} and the rotation detection tool kits: MMRotate 1.0.0 \cite{zhou2022mmrotate}. All the experiments follow the same hyper-parameters (learning rate, batch size, optimizer, etc.).

Average precision (AP) is adopted as the primary metric. All the models are configured upon ResNet50 \cite{he2016deep} and trained with AdamW \cite{loshchilov2018decoupled}.
\textbf{1) Learning rate.} Initialized at 5e-5, warm-up for 500 iterations, and divided by ten at each decay step. 
\textbf{2) Epochs.} 72 for HRSC; 12 for the others.
\textbf{3) Augmentation.} Random rotation/flip for HRSC; random flip for the others.
\textbf{4) Image size.} Split into 1,024 $\times$ 1,024 with an overlap of 200 for DOTA/FAIR1M/STAR; scaled to 800 $\times$ 800 for others.
\textbf{5) Multi-scale.} All experiments evaluated without multi-scale technique \cite{zhou2022mmrotate}. 
\textbf{6) Datasets.} Six remote sensing and one retail scene datasets, covering all datasets used by the main counterparts \cite{yu2024point2rbox, luo2024pointobb, cao2023p2rbox}:

\begin{table*}[!tb]
\fontsize{8.5pt}{10pt}\selectfont
\setlength{\tabcolsep}{0.65mm}
\setlength{\aboverulesep}{0.4ex}
\setlength{\belowrulesep}{0.4ex}
\setlength{\abovecaptionskip}{1.5mm}
\centering
\begin{tabular}{l|c|c|c|c|c|c|c|c|c|c}
\toprule
{\textbf{Methods}} & {*} & {\textbf{\,DOTA-v1.0\,}} & {\textbf{\,DOTA-v1.5\,}} & {\textbf{\,DOTA-v2.0\,}} & {\textbf{~~DIOR~~}} & {\textbf{~~HRSC~~}} & {\textbf{\,FAIR1M\,}} & {\textbf{~~STAR~~}} & {\textbf{\,SKU110K\,}} & {\textbf{~~RSAR~~}} \\
\hline
\rowcolor{gray!20} \multicolumn{11}{l}{$\blacktriangledown$ \textit{RBox-supervised OOD}} \\ \hline
RetinaNet (2017) \cite{lin2017focal} & \checkmark & 68.69 & 60.57        & 47.00 & 54.96 & 84.49   & 37.67   & 21.80 & 78.50 & 57.67  \\
GWD (2021) \cite{yang2021rethinking} & \checkmark & 71.66 & 63.27        & 48.87 & 57.60 & 86.67   & 39.11   & 25.30 & 79.16 & 57.80 \\
FCOS (2019) \cite{tian2019fcos} & \checkmark & 72.44 & 64.53        & 51.77    &  59.83  & 88.99  & 41.25   & \textbf{28.10} & 80.09 & \textbf{66.66} \\
S$^2$A-Net (2022) \cite{han2022align} & \checkmark & \textbf{75.81} & \textbf{66.53} & \textbf{52.39} & \textbf{61.41} & \textbf{90.10} & \textbf{42.44}   & 27.30 & \textbf{80.36} & 66.47 \\
\hline
\rowcolor{gray!20} \multicolumn{11}{l}{$\blacktriangledown$ \textit{HBox-supervised OOD}} \\ \hline
Sun et al. (2021) \cite{sun2021oriented} & $\times$ & 38.60 & - & - & - & - & - & - & - & - \\
KCR (2023) \cite{zhu2023knowledge} & \checkmark & - & - & - & - &  79.10  & -  & - & - & -  \\
H2RBox (2023) \cite{yang2023h2rbox} & \checkmark & 70.05 & 61.70        & 48.68    & 57.80 &  7.03  & 35.94  & 17.20 & 57.15 & 49.92    \\
H2RBox-v2 (2023) \cite{yu2023h2rboxv2} & \checkmark & 72.31 & 64.76 & 50.33 & 57.64 & \textbf{89.66} & \textbf{42.27} & \textbf{27.30} & \textbf{70.70} & \textbf{65.16} \\
AFWS (2024) \cite{lu2024afws} & \checkmark & \textbf{72.55} & \textbf{65.92} & \textbf{51.73} & \textbf{59.07} & - & 41.80 & - & - & - \\
\hline
\rowcolor{gray!20} \multicolumn{11}{l}{$\blacktriangledown$ \textit{Point-supervised OOD}} \\ \hline
P2RBox (2024) \cite{cao2023p2rbox}$^\dagger$ & $\times$ & \underline{59.04} & -        & - & - & -   & -  & -  & - & -  \\
PointSAM (2024) \cite{liu2024pointsam}$^\dagger$ & $\times$ & - & - & - & \textbf{46.20} & -   & -  & -  & - & - \\
PointOBB (2024) \cite{luo2024pointobb} & $\times$ & 30.08 & 10.66        & 5.53     &  37.31  & -   & 11.19 & 9.19  & - & 13.80    \\
Point2RBox+SK (2024) \cite{yu2024point2rbox}$^\dagger$ & \checkmark & 40.27 & 30.51        & 23.43    & 27.34 & 79.40   & 20.03 & 7.86  & 3.41 & 27.81    \\
PointOBB-v2 (2025) \cite{ren2024pointobbv2} & $\times$ & 41.68 & 30.59        & 20.64    &  39.56  & -   & 13.36 & 9.00  & 56.63 & 18.99   \\
PointOBB-v3 (2025) \cite{zhang2025pointobbv3} & $\checkmark$ & 41.20 & 31.25 & 22.82 & 37.60 & - & 11.42  & 11.31 & - & 15.84 \\
PointOBB-v3 (2025) \cite{zhang2025pointobbv3} & $\times$ & 49.24 & 33.79 & 23.52 & 40.18 & - & 18.35 & \underline{12.85} & - & 22.60 \\
\rowcolor{gray!20} Point2RBox-v2 (ours) & \checkmark & 51.00 & \underline{39.45} & \underline{27.11} & 34.70 & \underline{82.67} & \underline{25.72} & 7.80 & \underline{64.00} & \underline{28.60}
 \\
\rowcolor{gray!20} Point2RBox-v2 (ours) & $\times$ & \textbf{62.61} & \textbf{54.06}        & \textbf{38.79}   & \underline{44.45}  & \textbf{86.15}   & \textbf{34.71}  & \textbf{14.20} & \textbf{65.64} & \textbf{30.90}    \\
\bottomrule
\specialrule{0pt}{2pt}{0pt}
\multicolumn{11}{l}{$^*$Comparison tracks: \checkmark = End-to-end training and testing; $\times$ = Generating pseudo labels to train the FCOS detector (two-stage training).} \\
\multicolumn{11}{l}{$^\dagger$Using additional priors. P2RBox/PointSAM: Pre-trained SAM model; Point2RBox+SK: One-shot sketches for each class.} \\
\bottomrule
\end{tabular}
\caption{Accuracy (AP$_{50}$) comparisons on the DOTA-v1.0/1.5/2.0, DIOR, HRSC, FAIR1M, STAR, SKU110K, and RSAR datasets.}
\label{tab:exp_other}
\vspace{-4pt}
\end{table*}

\begin{itemize}
    \item \textbf{DOTA \cite{xia2018dota}.} DOTA-v1.0 has 2,806 aerial images annotated with 15 categories, while DOTA-v1.5/2.0 are the extended versions with more small objects and categories.
    
    \item \textbf{DIOR \cite{cheng2022anchor}.} It is an aerial image dataset re-annotated with RBoxes based on its original HBox version \cite{li2020object}, with a high variation in object size and high intra‐class diversity. 

    \item \textbf{HRSC \cite{liu2017hrsc}.} It contains ship instances on the sea and inshore. The train/val/test set includes 436/181/444 images.

    \item \textbf{FAIR1M \cite{sun2022fair1m}.} It has more than 1 million instances and more than 40,000 images for fine-grained object recognition in remote sensing imagery, annotated with 37 categories. The results are evaluated on FAIR1M-1.0.

    \item \textbf{STAR \cite{li2024star}.} It is extensive for scene graph generation, covering more than 210,000 objects with diverse spatial resolutions, classified into 48 fine-grained categories and precisely annotated with oriented bounding boxes. 

    \item \textbf{SKU110K \cite{pan2020dynamic}.} It focuses on the detection of densely packed retail scenes with 110,712 objects in 11,762 images. The density reaches 86 instances per image. 

    \item \textbf{RSAR \cite{zhang2025rsar}.} It is a remote sensing dataset based on Synthetic Aperture Radar (SAR) imagery with 6 categories.

\end{itemize}

\begin{table*}[!tb]
\fontsize{8.5pt}{10pt}\selectfont
\setlength{\tabcolsep}{2.08mm}
\setlength{\aboverulesep}{0.4ex}
\setlength{\belowrulesep}{0.4ex}
\setlength{\abovecaptionskip}{1.5mm}
\hspace{1pt}
\begin{minipage}[t]{0.315\linewidth}
\centering
\begin{tabular}{c|cc|cc}
\toprule
\multirow{2}{*}{$w_\text{O}$} & \multicolumn{2}{c|}{\textbf{DOTA}} & \multicolumn{2}{c}{\textbf{HRSC}} \\
                  & {E2E} & {FCOS} & {E2E} & {FCOS} \\ \midrule
3  & 48.76 & 61.62 & 81.85 & 84.36 \\
5  & 49.81 & 62.44 & 82.46 & 85.76 \\
\rowcolor{gray!20} 10 & \textbf{51.00} & \textbf{62.61} & \textbf{82.67} & \textbf{86.15} \\
30 & 45.88 & 57.83 & 81.56 & 85.61 \\
\bottomrule
\end{tabular}
\caption{Ablation with the weight of $\mathcal{L}_\text{O}$.}
\label{tab:abl_lo}
\end{minipage}
\quad
\begin{minipage}[t]{0.315\linewidth}
\centering
\begin{tabular}{c|cc|cc}
\toprule
\multirow{2}{*}{$w_\text{W}$} & \multicolumn{2}{c|}{\textbf{DOTA}} & \multicolumn{2}{c}{\textbf{HRSC}} \\
                  & {E2E} & {FCOS} & {E2E} & {FCOS} \\ \midrule
3  & 50.85 & 56.78 & 78.42 & 83.49 \\
\rowcolor{gray!20} 5  & \textbf{51.00} & \textbf{62.61} & \textbf{82.67} & \textbf{86.15} \\
10 & 49.15 & 60.54 & 30.37 & 35.13 \\
30 & 42.84 & 52.53 & 23.89 & 25.91 \\
\bottomrule
\end{tabular}
\caption{Ablation with the weight of $\mathcal{L}_\text{W}$.}
\label{tab:abl_lw}
\end{minipage}
\quad
\begin{minipage}[t]{0.315\linewidth}
\setlength{\tabcolsep}{2.04mm}
\centering
\begin{tabular}{c|cc|cc}
\toprule
\multirow{2}{*}{$w_\text{E}$} & \multicolumn{2}{c|}{\textbf{DOTA}} & \multicolumn{2}{c}{\textbf{HRSC}} \\
                  & {E2E} & {FCOS} & {E2E} & {FCOS} \\ \midrule
0.1 & 48.75 & 57.62 & 34.71 & 39.45 \\
\rowcolor{gray!20} 0.3 & 51.00 & 62.61 & \textbf{82.67} & \textbf{86.15} \\
0.5 & \textbf{51.36} & \textbf{62.63} & 76.85 & 85.22 \\
1.0 & 49.05 & 60.63 & 56.59 & 59.59 \\
\bottomrule
\end{tabular}
\caption{Ablation with the weight of $\mathcal{L}_\text{E}$.}
\label{tab:abl_le}
\end{minipage}
\vspace{-4pt}
\end{table*}

\begin{table*}[!tb]
\fontsize{8.5pt}{10pt}\selectfont
\setlength{\tabcolsep}{2.04mm}
\setlength{\aboverulesep}{0.4ex}
\setlength{\belowrulesep}{0.4ex}
\setlength{\abovecaptionskip}{1.5mm}
\hspace{1pt}
\begin{minipage}[t]{0.315\linewidth}
\centering
\begin{tabular}{c|cc|cc}
\toprule
\multirow{2}{*}{$w_\text{ss}$} & \multicolumn{2}{c|}{\textbf{DOTA}} & \multicolumn{2}{c}{\textbf{HRSC}} \\
                  & {E2E} & {FCOS} & {E2E} & {FCOS} \\ \midrule
0.1 & 49.28 & 59.66 & 73.66 & 78.92 \\
\rowcolor{gray!20} 1.0 & \textbf{51.00} & \textbf{62.61} & \textbf{82.67} & \textbf{86.15} \\
3.0 & 49.15 & 59.20 & 1.30  & 1.65 \\
\bottomrule
\end{tabular}
\caption{Ablation with the weight of $\mathcal{L}_\text{ss}$.}
\label{tab:abl_lss}
\end{minipage}
\quad
\begin{minipage}[t]{0.647\linewidth}
\setlength{\tabcolsep}{3.05mm}
\centering
\begin{tabular}{c|c|c||c|c|c}
\toprule
{R / F / S} & {\textbf{DOTA}} & {\textbf{HRSC}} & {R / F / S} & {\textbf{DOTA}} & {\textbf{HRSC}} \\
 \midrule
90\% / 10\% / 0\% & 60.42 & 85.46 & 80\% / 20\% / 0\%  & 59.46 & 84.73 \\
75\% / 0\% / 25\% & 60.79 & 86.22 & 60\% / 15\% / 25\% & 62.38 & 84.21 \\
\cellcolor{gray!20}68\% / 7\% / 25\% & \cellcolor{gray!20}\textbf{62.61} & \cellcolor{gray!20}\textbf{86.15} & 38\% / 37\% / 25\% & 45.87 & 8.56  \\
45\% / 5\% / 50\% & 60.55 & 85.34 & 40\% / 10\% / 50\% & 60.49 & 10.74 \\
\bottomrule
\end{tabular}
\caption{Ablation with the proportion of augmented views in self-supervision.}
\label{tab:abl_pro}
\end{minipage}
\vspace{-10pt}
\end{table*}

\subsection{Main Results on DOTA-v1.0}
\label{sec:experiment-main}

Table \ref{tab:exp_dota} compares Point2RBox-v2 with the state-of-the-art methods, which can be categorized into two tracks: 

\textbf{1) End-to-end training.} These methods apply the trained weakly-supervised detector directly to the test set. Without relying on priors, our approach demonstrates an improvement of 16.93\% (51.00\% vs. 34.07\%) compared to Point2RBox. Even when compared to Point2RBox+SK, which incorporates additional data-side priors (i.e. one-shot examples for each class), our method still outperforms it by 10.73\% (51.00\% vs. 40.27\%).

\textbf{2) Two-stage training.} These methods generate RBox labels on train/val sets, with which the FCOS detector is trained. In this two-stage mode, Point2RBox-v2 achieves an accuracy of 62.61\%, considerably surpassing PointOBB series. Remarkably, it even outperforms the SAM-powered method P2RBox by 3.57\% (62.61\% vs. 59.04\%).

\textbf{Class-wise analysis.} The FCOS detector trained with labels generated by Point2RBox-v2 achieves accuracy nearly equivalent to RBox-supervised FCOS across six high-density categories: SH (86.9\% vs. 87.1\%), SV (79.6\% vs. 79.8\%), LV (76.3\% vs. 79.8\%), PL (88.0\% vs. 89.1\%), ST (82.9\% vs. 84.6\%), and TC (89.1\% vs. 90.4\%). Interestingly, these six high-density categories account for 88\% of DOTA instances. By annotating these categories with points and generating RBoxes using Point2RBox-v2 while labeling the other sparse categories with RBoxes, we can significantly reduce annotation labor without sacrificing much accuracy, highlighting the valuable role our method can play.

\begin{figure*}[t!]
\setlength{\abovecaptionskip}{1.2mm}
\centering
\includegraphics[width=0.96\linewidth]{figs/case.pdf}
\caption{Qualitative analysis on failed cases and overlap cases.}
\label{fig:case}
\vspace{-6pt}
\end{figure*}

\subsection{Results on More Datasets}

The results are displayed in Table \ref{tab:exp_other}.
On more challenging DOTA-v1.5/2.0, Point2RBox-v2 presents a similar trend, 23.47\%/18.15\% higher than PointOBB-v2 in the pseudo-generation track. 
On the ship detection dataset HRSC, the gap between Point2RBox-v2 and RBox-supervised FCOS is only 2.84\% (86.15\% vs. 88.99\%).
DIOR is relatively sparse, leading to less improvement with our methods---lower than PointSAM (44.45\% vs. 46.20\%) but still higher than methods that do not use SAM. 
Our method also provides competitive performance on fine-grained datasets FAIR1M and STAR. 
In addition to remote sensing scenarios, we carry out experiments on SKU110K for densely packed retail scenes. Existing point-supervised methods struggle in this case, whereas Point2RBox-v2 achieves performance on par with HBox-supervised H2RBox (65.64\% vs. 57.15\%).

\begin{table}[!tb]
\fontsize{8.5pt}{10pt}\selectfont
\setlength{\tabcolsep}{1.78mm}
\setlength{\aboverulesep}{0.4ex}
\setlength{\belowrulesep}{0.4ex}
\setlength{\abovecaptionskip}{1.5mm}
\centering
\begin{tabular}{ccccc|cc|cc}
\toprule
\multicolumn{5}{c|}{\textbf{Modules}} & \multicolumn{2}{c|}{\textbf{DOTA}} & \multicolumn{2}{c}{\textbf{HRSC}} \\
$\mathcal{L}_\text{O}$ & $\mathcal{L}_\text{W}$ & $\mathcal{L}_\text{ss}$ & $\mathcal{L}_\text{E}$ & \textit{CP} & {E2E} & {FCOS} & {E2E} & {FCOS} \\ \midrule
\checkmark & & & & & 0.00 & 0.00 & 0.00 & 0.00 \\
\checkmark & \checkmark & & & & 41.54 & 52.98 & 17.96 & 19.64 \\
\checkmark & \checkmark & \checkmark & & & 46.64 & 54.26 & 18.10 & 22.13 \\
\checkmark & \checkmark & \checkmark & \checkmark & & 49.55 & 61.88 & 78.79 & 83.79 \\
& \checkmark & \checkmark & \checkmark & \checkmark & 48.58 & 59.56 & 20.35 & 24.76 \\
\checkmark & & \checkmark & \checkmark & \checkmark & 38.94 & 48.44 & 11.64 & 14.93 \\
\checkmark & \checkmark & \checkmark & & \checkmark & 47.08 & 55.05 & 19.58 & 21.78 \\
\rowcolor{gray!20} \checkmark & \checkmark & \checkmark & \checkmark & \checkmark & \textbf{51.00} & \textbf{62.61} & \textbf{82.67} & \textbf{86.15} \\
\bottomrule
\end{tabular}
\caption{Ablation with incremental addition of modules.}
\label{tab:abl_mod}
\vspace{-4pt}
\end{table}

\begin{table}[!tb]
\fontsize{8.5pt}{10pt}\selectfont
\setlength{\tabcolsep}{2.85mm}
\setlength{\aboverulesep}{0.4ex}
\setlength{\belowrulesep}{0.4ex}
\setlength{\abovecaptionskip}{1.5mm}
\centering
\begin{tabular}{c|c|c||c|c|c}
\toprule
16 & \cellcolor{gray!20}$K\!=\!24$ & 32 & 1.2 & \cellcolor{gray!20}$\beta\!=\!1.6$ & 2.0 \\ \midrule
50.87 & \cellcolor{gray!20}\textbf{51.00} & 48.08 & 48.14 & \cellcolor{gray!20}51.00 & \textbf{51.33} \\
\bottomrule
\end{tabular}
\caption{Ablation with $K$ and $\beta$ in edge loss on DOTA (E2E).}
\label{tab:abl_edgeparam}
\vspace{-4pt}
\end{table}

\begin{table}[!tb]
\fontsize{8.5pt}{10pt}\selectfont
\setlength{\tabcolsep}{1.75mm}
\setlength{\aboverulesep}{0.4ex}
\setlength{\belowrulesep}{0.4ex}
\setlength{\abovecaptionskip}{1.5mm}
\centering
\begin{tabular}{c|cc|cc|cc}
\toprule
\multirow{2}{*}{$\sigma$} & \multicolumn{2}{c|}{Point2RBox} & \multicolumn{2}{c|}{PointOBB-v2} & \multicolumn{2}{c}{Point2RBox-v2} \\
 & {\textbf{DOTA}} & {\textbf{HRSC}} & {\textbf{DOTA}} & {\textbf{HRSC}} & {\textbf{DOTA}} & {\textbf{HRSC}} \\ \midrule
0\%  & 40.27 & 79.40 & 44.85 & - & 62.61 & 86.15 \\
10\% & 39.60 & 78.81 & 42.30 & - & 61.58 & 85.76 \\
30\% & 38.42 & 78.28 & 38.46 & - & 60.31 & 85.71 \\
\bottomrule
\end{tabular}
\caption{Ablation with the inaccuracy in point annotations.}
\label{tab:abl_noise}
\vspace{-10pt}
\end{table}

\subsection{Ablation Studies}
\label{sec:experiment-ablation}

Tables \ref{tab:abl_lo}-\ref{tab:abl_noise} display the ablation studies on DOTA-v1.0 and HRSC. ``E2E'' denotes end-to-end training; ``FCOS'' denotes two-stage training (i.e. generating pseudo labels to train FCOS). The final values adopted are highlighted in gray.

\textbf{Weight of each loss.} Tables \ref{tab:abl_lo}-\ref{tab:abl_le} determine the weights of the proposed losses. Based on these experiments, the weights $(w_\text{O},w_\text{W},w_\text{E},w_\text{ss})$ are set to $(10, 5, 0.3, 1)$.

\textbf{Proportion of augmented views.} Table \ref{tab:abl_pro} studies the proportion between rotation, flip, and scale. The results are reported with two-stage training (FCOS). Based on the results, the proportion is set to 68\%, 7\%, and 25\%.

\textbf{Incremental addition of modules.} Table \ref{tab:abl_mod} demonstrates the constraints from Gaussian and Voronoi achieve an accuracy of 52.98\% on DOTA. Adding consistency loss and edge loss further boosts it to 54.26\% and 61.88\%, respectively, whereas the improvement from copy-paste is 0.73\%. We also demonstrate the impact of omitting each core loss.

\textbf{Edge loss parameters.} We set $K=24$ and $\beta=1.6$ as they are observed to discern the correct edges during code development. Table \ref{tab:abl_edgeparam} provides a more precise ablation.

\textbf{Annotation inaccuracy.} We offset the annotated points by a noise from the uniform distribution $\left[-\sigma H, +\sigma H \right ]$, where $H$ is the height of objects. Table \ref{tab:abl_noise} shows that the AP$_{50}$ of Point2RBox-v2 decreases by less than 3\% when noise is added to point annotations, demonstrating the robustness of the proposed learning mechanisms.

\subsection{More Discussions}
\label{sec:experiment-discussions}

The qualitative analysis on the failed/overlap cases is shown in Fig. \ref{fig:case}. \textbf{1) Failed cases.} Although our method performs well overall, it struggles with certain categories that are sparse and not constrained by other objects. \textbf{2) Overlap cases.} 
Minimizing overlap as a soft constraint during training does not entirely eliminate overlap. Once trained, the model remains robust to some overlap during inference.

% \section{Simulation Evaluation \& Results}\label{sec:results}

\subsection{Baseline Planners}

To evaluate the performance of \PlannerName, we compare it against several baseline methods. In the following section, we describe these baselines, their implementation details, and their respective advantages and limitations, particularly in the context of information gathering in large, high-dimensional search spaces. The simulation framework and vehicle parameters remain consistent across all planners, and each method is allowed to replan during testing.

\subsubsection{Monte-Carlo Tree Search}

Monte Carlo Tree Search (MCTS) can be a powerful technique for finding feasible and optimal paths in complex environments. It is a heuristic search algorithm that builds a search tree incrementally through repeated simulations. At each iteration, it selects a node to explore based on a selection policy (often the Upper Confidence Bound or UCB1 algorithm), expands the tree by adding possible actions from that node, runs a simulation from the newly added node, and updates the statistics of nodes along the path traversed during the simulation. 

The UCB1 (Upper Confidence Bound) algorithm is a technique commonly used in the context of multi-armed bandit problems and Monte Carlo Tree Search (MCTS) for balancing exploration and exploitation. It helps in selecting actions or nodes that are likely to yield high rewards while also exploring less-frequented options to gather more information about their potential rewards. 

We formulate our UCB score in the following manner, \\
\begin{equation*}
    UCB_\text{node} = \frac{I(X_{\text{node}})}{\alpha} + C \times \sqrt{\frac{\ln(N_\text{tree})}{N_\text{node}}}
\end{equation*}
%  $
% UCB_\text{node} = \frac{\overline{X_\text{node}}}{\alpha} + C \times \sqrt{\frac{\ln(N_\text{tree})}{N_\text{node}}}
% $ \\
Here $I(X_{\text{node}})$ denotes the estimated information gain from the node, $\alpha$ denotes the normalization factor which is given by $\frac{B}{v_\text{desired}}$, $B$ being the maximum planning budget and $v_\text{desired}$ being the desired speed of our UAV. $C$ denotes the exploration weight, and $N_\text{tree}$ denotes the number of visits to the tree root node while $N_\text{node}$ denotes the number of times the present node has been visited.

After selecting a candidate node, if it has been visited before, it is expanded by applying motion primitives to generate child nodes, growing the tree. Unvisited nodes skip this step. Following expansion, either the unvisited candidate node or one of its children is selected for the simulation phase, where the future values of nodes along the path are estimated to update the total potential information gain. This informs the selection policy in subsequent iterations. Once planning time is exhausted, the path with the highest information gain is returned.

% with authors goes here
\begin{figure}[t]
\centering
\includegraphics[trim={.7cm 0cm .5cm 1.4cm},clip,width=\columnwidth]{figs/5_/Results1v3.pdf}
\caption{The Monte Carlo simulation results for the planners. The plots show the average percent reduction in entropy over the course of the simulations, and the shading shows the 95\% confidence intervals. IA-TIGRIS outperforms all of the baselines.}
\label{fig:mc_results}
\end{figure}

While MCTS is probabilistically guaranteed to converge to the optimal path \cite{mcts_ref_1}, it is constrained to actions within a predefined set of motion primitives. Its reliance on random sampling to estimate the future value of nodes can result in poor approximations, particularly in environments with sparse, localized pockets of high information gain. This limitation is especially pronounced in large search areas or scenarios with large budgets constraints, where estimating future node values becomes increasingly expensive. As a result, in such scenarios, MCTS is often implemented with a finite planning horizon, which can restrict its ability to account for long-term consequences or dependencies in the environment.

% This property of MCTS, which causes unguided exploration of the environment, leads to increased convergence times on the optimal path, as a result of a lot of budget being spent in exploring information sparse areas of the map. 
% Also, the computation time of MCTS increases exponentially with the depth of the search tree. The time complexity of MCTS is given by $\mathcal{O}(\frac{T}{t_\text{iter}} \cdot |A|^d)$. Here, $T$ is the total planning time and $t_\text{iter}$ is the time taken per iteration of the planning loop. $|A|$ is the number of actions and $d$ represents the average depth of the search tree. 

% The above limitations are not inconsequential in the context of performing informative path planning in large high-dimensional search spaces. We compare MCTS with \PlannerName, in \ref{}, and empirically demonstrate its drawbacks and how \PlannerName, is able to outperform MCTS in the context of the mission parameters we examine in this work.  

\subsubsection{Greedy}

For the greedy planner, we iterated through each cell within the search bounds and calculated the reward for a given cell $i$ as $g_i = R(X_i) / d_i$ where $R(X_i)$ is given through \eqref{equ:reward} and $d_i$ represents the Euclidean distance between the current position the robot at the current time $t$ and the closest viewpoint to the cell. To compute this viewpoint, the yaw between the current pose of the robot and the intersected cell is first calculated. Using the robot's sensor configuration and this yaw, $x$ and $y$ coordinates are calculated that view the cell at the desired flight altitude. With this formulation, the planner prioritizes regions with a high ratio of entropy to distance. This can lead to locally optimal choices that contradict with paths that lead to higher information gain over the entire trajectory. 

% without authors goes here
% \begin{figure}[t]
% \centering
% \includegraphics[trim={.7cm 0cm .5cm 1.4cm},clip,width=\columnwidth]{figs/5_/Results1v3.pdf}
% \caption{The Monte Carlo simulation results for the planners. The plots show the average percent reduction in entropy over the course of the simulations, and the shading shows the 95\% confidence intervals. IA-TIGRIS outperforms all of the baselines.}
% \label{fig:mc_results}
% \end{figure}


\begin{figure*}[t]
    \centering
    \begin{subfigure}[b]{0.99\textwidth}
        \centering
        \includegraphics[trim={0cm 0.3cm 0cm 0cm},clip,width=\textwidth]{figs/5_/Fig2v1_target.png}
        % \caption{Slice by targets}
        % \vspace{.1cm}
    \end{subfigure}
    
    \begin{subfigure}[b]{0.99\textwidth}
        \centering
        \includegraphics[trim={0cm 0cm 0cm 0cm},clip,width=\textwidth]{figs/5_/Fig2v1_sigma.png}
        % \caption{Slice by sigma }
    \end{subfigure}
    \caption{A comparison of the methods based on the number of sampled prior clusters and the standard deviation of sampled prior clusters. IA-TIGRIS is most effective compared to the baselines when there is high variation in the search space. As the search space prior information becomes more evenly spread out, the performance gap between the methods tends to decrease.}
    \label{fig:targets_sigmas}
\end{figure*}

\subsubsection{Random}

The random planner operates by iteratively sampling points within the defined search bounds and calculating the minimum-cost path to observe each sampled point. This process is repeated until the available budget is fully expended. The random planner does not utilize any prior information about the environment or target distribution. Additionally, it does not optimize the sequence of actions, instead treating each sampled point independently without considering the global structure of the search problem. This simplicity allows the random planner to highlight the performance benefits of more sophisticated methods by providing a lower-bound comparison for evaluation.

\subsubsection{Coverage}

The coverage planner generates a plan that systematically covers the entire search space using a straightforward lawn-mower pattern. The spacing between each pass is set to match the width of the projected observation footprint at 20\% from the bottom, ensuring that no grid cells are missed. This spacing also maintains a distance that enables high-quality sensor measurements. However, due to the size of the search spaces considered, the coverage planner spends significant time surveying empty regions. This approach results in inefficient use of the budget, as it prioritizes full coverage with safe sensor overlap, even in areas with little or no valuable information. While simple and robust, this method highlights the tradeoff between exhaustive coverage and efficient, targeted exploration.

% \subsubsection{Branch and Bound}
% The branch and bound baseline is based on motion primitive planning. In each future step the drone has a set of motion primitives with future states and each of these future states also has a set of motion primitives. In this way, a tree can be built with multiple path candidates. The path candidate with the highest information gain will be selected and form the output. 

% By adding branch and bound, there will be an estimation of a node's upper bound information reward, using the node's current information reward, updated information map and the remaining budget. If this upper bound is already lower than the information reward of any other node in the tree, the corresponding node will be closed and not expanded in the future to accelerate the expansion of the tree. 



\subsection{Tests and Analysis}
% To evaluate the efficacy of IA-TIGRIS compared to the baseline methods, we conduct Monte Carlo testing as well as analyze how the prior and budget affect the performance of each method. In all of these test cases, there are no time-based or priority rewards and have horizon lengths set to the full budget. All tests were performed using an Intel Xeon CPU E5-2620 v4 @ 2.10GHz.
To evaluate the efficacy of IA-TIGRIS against baseline methods, we perform Monte Carlo testing and analyze the impact of the prior and budget on the performance of each method. In all test cases, rewards are calculated using \eqref{equ:reward}, and horizon lengths are set to match the full budget. The tests are conducted on an Intel Xeon CPU E5-2620 v4 @ 2.10GHz, ensuring consistent computational conditions across all evaluations.

% Random sample across which parameters.

% Quantitative ideas. Look into number and std of prior (metric for this? std of grid cell values, mediuan, mean,). 
% Uniform prior? 
% Split distinct regions, not smooth. 
% Compare to coverage and amount of time to reach specific amount. 
% Compare with different budgets. 
% Repeatability test. 
% Graph size vs time. 
% Look at coverage with different altitudes or widths. Something that shows long horizon vs not nature of things?
% Shape of search space?
% Time/budget to get x\% of all info gain. Have to do moving horizon. 
% Targets detected? 

% Key thought for results where I show time, our optimization does not optimize for time, only final value. Key thing to show across the different budgets. 

% \BM{Qualitative. Nayana idea of plot with example sampled case. Should add one here.} 



\subsubsection{Monte Carlo Testing}
Our simulated testing environment is a $5000\times5000$ m square with Gaussian-distributed prior information randomly placed throughout the search space. The number of prior clusters was sampled uniformly between $[4,20]$, with standard deviations between $[60,450]$, and maximum value between $[0.05,0.5]$. 

The results of $100$ Monte Carlo tests are shown in Fig.~\ref{fig:mc_results}. IA-TIGRIS clearly outperforms the other methods, achieving nearly a $40\%$ greater reduction in entropy than the next best method. Early in the simulation, the greedy method initially gains information more quickly, as expected, but this does not translate to better long-term performance. Since our method optimizes for total information gain, it generates paths that maximize information collection over the entire budget. MCTS performed slightly worse than the greedy approach.

The random paths slightly outperformed the coverage paths. This is likely because the lawnmower strategy requires sufficient overlap between passes to avoid missing areas, and its long straight paths often lead to redundant observations due to the UAV’s forward-facing camera. Changing the heading of the UAV is beneficial to viewing more of the search space, which may explain why random paths performed better.

We also conducted Monte Carlo tests where either the number of prior clusters or their standard deviation was held constant to analyze how variations in the information map affect planner performance. The results, shown in Fig.~\ref{fig:targets_sigmas}, include two cases: the upper figure fixes the number of priors, while the lower figure fixes their standard deviation. All other agent and simulation parameters remained unchanged.


% The first thing to note from these results is that for all tests the proportional performance gap between IA-TIGRIS and the baselines increases as the number and standard deviation of the Gaussian priors decreases. As the search space becomes more uniformly filled with entropy in the information map, the need for longer-horizon planning decreases and other simple or random approaches can perform satisfactorily given the testing budget. As the information becomes more sparsely distribution in the space, such as when the information is contained in separated pockets of areas, there is a greater need to plan longer-horizon paths that reason about the given budget.
% \BM{Could have figures here or refer to others}

Across these tests, the performance gap between IA-TIGRIS and the baselines widens as the number and standard deviation of the Gaussian priors decrease. When entropy is more uniformly distributed across the search space, simpler methods perform reasonably well within the given budget. However, when information is concentrated in sparse, distinct regions, longer-horizon planning becomes essential. In such cases, IA-TIGRIS demonstrates a significant advantage by effectively reasoning about the budget and prioritizing high-value regions.

% Show plot of first plans expected info gain versus planning time. (plans not executed)


\subsubsection{Budget Analysis}
To evaluate the impact of budget constraints on performance, we conducted additional tests beyond our initial Monte Carlo experiments, evaluating budgets of $5000$ m, $10000$ m, $30000$ m, and $60000$ m. Table~\ref{tab:budgets} summarizes the average entropy reduction across these budgets.

\definecolor{tabfirst}{rgb}{1, 0.7, 0.7} % red
\definecolor{tabsecond}{rgb}{1, 0.85, 0.7} % orange
\definecolor{tabthird}{rgb}{1, 1, 0.7} % yellow
\begin{table}[t]
    \centering
    \resizebox{\linewidth}{!}{
    \begin{tabular}{l|ccccc}
    & $5000$ m & 10000 m  & 15000 m& 30000 m& 60000 m\\ \hline

    % \hline
    IA-TIGRIS  &  \cellcolor{tabfirst}$9.41\pm1.0$ &  \cellcolor{tabfirst}$18.28\pm1.8$ & \cellcolor{tabfirst}$25.36\pm2.3$ & \cellcolor{tabfirst}$41.08\pm2.9$ & \cellcolor{tabfirst}$58.85\pm2.9$ \\
    Greedy  &  \cellcolor{tabsecond}$6.99\pm0.8$ &  \cellcolor{tabsecond}$13.10\pm1.5$ & \cellcolor{tabsecond}$17.97\pm2.0$ & \cellcolor{tabthird}$30.00\pm2.3$ & \cellcolor{tabsecond}$49.38\pm3.5$ \\
    MCTS  &  \cellcolor{tabthird}$6.06\pm0.7$ &  \cellcolor{tabthird}$11.80\pm1.1$ & \cellcolor{tabthird}$17.11\pm1.4$ & \cellcolor{tabsecond}$30.21\pm2.2$ & \cellcolor{tabthird}$48.68\pm2.7$ \\
    Random  &  $2.19\pm0.3$ & $4.29\pm0.7$ & $6.61\pm0.6$ & $17.50\pm1.2$ & $22.47\pm1.4$ \\
    Coverage  &  $1.58\pm0.3$ &  $2.82\pm0.4$ & $4.09\pm0.7$ & $12.04\pm1.9$ & $16.77\pm2.4$ \\

    \end{tabular}
    }
    \caption{Monte Carlo testing results given different budgets. The values are the average percent reduction in entropy and the 95\% confidence bounds. \mbox{IA-TIGRIS} had the best performance for all budgets.}
    \label{tab:budgets}
\end{table}
%$\uparrow$ 

IA-TIGRIS consistently achieved the highest entropy reduction across all budget constraints, with a statistically significant margin over alternative methods. Greedy generally ranked second but was slightly outperformed by MCTS at the $30000$ m budget level. Greedy and MCTS exhibited comparable performance throughout the tests, with their results closely tracking each other. Consistent with our previous findings, Random and Coverage methods yielded the lowest results.


Among the tested methods, only IA-TIGRIS and MCTS explicitly incorporate budget constraints into their planning algorithms. Notably, at lower budgets ($5000$ m and $10000$ m), these methods achieved higher entropy reduction compared to the equivalent time steps ($200$ s and $400$ s) in the $15000$ m budget scenario shown in Fig.~\ref{fig:mc_results}. This improved performance stems from IA-TIGRIS's optimization of total path reward under budget constraints, contrasting with the myopic next-best-action approach of the greedy method. The remaining methods---Greedy, Random, and Coverage---maintain consistent behavior regardless of budget constraints, as their planning strategies do not account for resource limitations.


The performance gap between IA-TIGRIS and the next-best method varied with budget size, showing margins of $34.6\%$, $39.5\%$, $41.1\%$, $36.0\%$, and $19.2\%$ in ascending budget order. This gap widened through the first three budget levels as problem complexity increased, before declining significantly at higher budgets. This performance pattern suggests that implementing a planning horizon could enhance efficiency by limiting tree search depth, enabling the planner to prioritize path quality optimization over exhaustive space exploration.


% percent improved from next best
% 34.6, 39.5, 41.1, 36.0, 19.2
% reasons, too long horizon is a larger search space, so less quality paths closer. Or larger horizon, more packing in


% with authors goes here
\begin{figure}[t] 
    \centering
    \renewcommand\arraystretch{0} % Adjust the height between rows here
    \setlength{\tabcolsep}{1pt} % Adjust the column separation here
    \begin{tabular}{c}
        \begin{tikzpicture}
            \node[anchor=south west, inner sep=0] (image) at (0,0) {
                \includegraphics[width=0.9\linewidth]{figs/5_/google_earth_prior.png}
            };
            \begin{scope}[x={(image.south east)},y={(image.north west)}]
                % \fill[OrangeRed] (0.02, 0.03) circle (2pt); 
                % \fill[OrangeRed] (0.51, 0.04) circle (2pt); 
                % \fill[OrangeRed] (0.61, 0.04) arc (0:90:2pt); 
                \fill[Orange, opacity=0.8] (0.74, 0.45) circle (3pt); % Adjust 
                \fill[Orange, opacity=0.8] (0.27, 0.42) circle (3pt); % Adjust 
                \fill[Orange, opacity=0.8] (0.39, 0.63) circle (3pt); % Adjust 
            \end{scope}
        \end{tikzpicture} \\
        % \includegraphics[width=0.9\linewidth]{figs/5_/google_earth_prior.png} \\
        \\
        \includegraphics[width=0.9\linewidth]{figs/5_/google_earth_path.png} 
    \end{tabular}
    \caption{Google Earth screenshots illustrating the mission planning process and execution. Top: Areas of high entropy targeted for search are highlighted in red, representing regions with a binary occupied/unoccupied probability of 0.2. Three points of particular interest, each assigned a 0.5 probability, are marked in orange. Bottom: The executed drone flight path (yellow) shows the optimized path for maximum information gain across the search space.} 
    \label{fig:google_earth}
\end{figure}
\begin{figure}[t]
\centering
% https://docs.google.com/presentation/d/1RjI-QqHpBRLHN60UAxzmQYs4EaWaVCOoSBkEkA39kk0/edit?usp=sharing
\includegraphics[width=\columnwidth]{figs/5_/m600_labeled.jpg}
\caption{Hexarotor system (DJI M600 Pro) with onboard compute and camera. Left image shows drone on the ground, right image shows drone in flight.}
\label{fig:m600}
\end{figure}


\section{Field Deployments}\label{sec:field}


\subsection{Hexarotor Deployment}
The first field experiment that we present uses a hexarotor drone to cover an urban area shown in Fig.~\ref{fig:fig1}.
We designed this field experiment to simulate classifying where cars are within a search area.  
Hence, we set the plan request to focus on parking lots at the field test site (Fig.~\ref{fig:google_earth}, top), with the addition of three chosen grid cells within the parking lots being marked as having a higher uncertainty. The plan request boundaries and priors were created with GPS coordinates in Google Earth, exported as kml files, and then converted into our plan request message format. 

The following sections details the hardware, autonomy, and experimental results for our hexarotor deployments.

% without the authors goes here
% \begin{figure}[t] 
%     \centering
%     \renewcommand\arraystretch{0} % Adjust the height between rows here
%     \setlength{\tabcolsep}{1pt} % Adjust the column separation here
%     \begin{tabular}{c}
%         \begin{tikzpicture}
%             \node[anchor=south west, inner sep=0] (image) at (0,0) {
%                 \includegraphics[width=0.9\linewidth]{figs/5_/google_earth_prior.png}
%             };
%             \begin{scope}[x={(image.south east)},y={(image.north west)}]
%                 % \fill[OrangeRed] (0.02, 0.03) circle (2pt); 
%                 % \fill[OrangeRed] (0.51, 0.04) circle (2pt); 
%                 % \fill[OrangeRed] (0.61, 0.04) arc (0:90:2pt); 
%                 \fill[Orange, opacity=0.8] (0.74, 0.45) circle (3pt); % Adjust 
%                 \fill[Orange, opacity=0.8] (0.27, 0.42) circle (3pt); % Adjust 
%                 \fill[Orange, opacity=0.8] (0.39, 0.63) circle (3pt); % Adjust 
%             \end{scope}
%         \end{tikzpicture} \\
%         % \includegraphics[width=0.9\linewidth]{figs/5_/google_earth_prior.png} \\
%         \\
%         \includegraphics[width=0.9\linewidth]{figs/5_/google_earth_path.png} 
%     \end{tabular}
%     \caption{Google Earth screenshots illustrating the mission planning process and execution. Top: Areas of high entropy targeted for search are highlighted in red, representing regions with a binary occupied/unoccupied probability of 0.2. Three points of particular interest, each assigned a 0.5 probability, are marked in orange. Bottom: The executed drone flight path (yellow) shows the optimized path for maximum information gain across the search space.} 
%     \label{fig:google_earth}
% \end{figure}
% \begin{figure}[t]
% \centering
% % https://docs.google.com/presentation/d/1RjI-QqHpBRLHN60UAxzmQYs4EaWaVCOoSBkEkA39kk0/edit?usp=sharing
% \includegraphics[width=\columnwidth]{figs/5_/m600_labeled.jpg}
% \caption{Hexarotor system (DJI M600 Pro) with onboard compute and camera. Left image shows drone on the ground, right image shows drone in flight.}
% \label{fig:m600}
% \end{figure}

\subsubsection{Hardware System}
The hardware consists of the DJI M600 Pro, shown in Fig.~\ref{fig:m600}, along with the physical sensing and onboard computer payload. The DJI M600 Pro contains a flight controller that handles pose estimation and position-based control. The DJI M600 Pro’s flight controller also handles teleloperation if human intervention is necessary. Beneath the drone's base, we mount a custom hardware payload.
That payload consists of an onboard computer, a Jetson Xavier, to run the autonomy software shown in Fig.~\ref{fig:functional_diagram}.
The payload also contains a downward-facing a camera for sensing the environment. The camera is a Seek S304SP thermal camera.
The camera intrinsics are used to calculate the frustum's intersection with the search map's cells in IA-TIGRIS.

% without authors goes here
\begin{figure}[t]
\centering
% https://lucid.app/lucidchart/f750ddb4-2809-4773-8361-d5fbb1ba49eb/edit?viewport_loc=-257%2C-116%2C2219%2C1140%2C0_0&invitationId=inv_56e8a3a9-e8cf-4cad-a280-48bd967ff651
\includegraphics[trim={0cm 0cm 0cm 0cm},clip,width=\columnwidth]{figs/5_/functional_diagram.jpeg}
\caption{Functional diagram of the DJI M600 Pro autonomy software.}
\label{fig:functional_diagram}
\end{figure}
\begin{figure}[b]
    \centering
    \begin{subfigure}[b]{0.48\columnwidth}
        \centering
        \includegraphics[width=1.0\linewidth]{figs/5_/field_test_altitude_over_time.png}
        \caption{}
        \label{fig:m600_altitude_over_time}
    \end{subfigure}
    \begin{subfigure}[b]{0.48\columnwidth}
        \centering
        \includegraphics[width=1.0\linewidth]{figs/5_/field_test_entropy_over_time.png}
        \caption{}
        \label{fig:m600_entropy_over_time}
    \end{subfigure}
    \caption{The results for our hexarotor field deployment. (a) Plot of flown altitude over time, showing large variation throughout the experiment. (b) Reduction in entropy percentage over time of field experiment.}
\end{figure}

\subsubsection{Autonomy System}
Fig.~\ref{fig:functional_diagram} illustrates the functional system diagram for the real world field test on the DJI M600. The user specifies the initial plan request prior to takeoff. The TIGRIS planner makes an initial plan on that plan request and sends a global path to the waypoint manager. The waypoint manager tracks the current waypoint within the plan and sends the next waypoint to the DJI software development kit, which then sends actuation commands to the motors. The position of the drone is used to calculate the distance from the drone to the ground and sends that distance parameter to the sensor model. The sensor model's true positive and false positive rate is used to calculate the per-cell entropy updates in the search map manager. The search map manager publishes the current information map, and the replanning node sends an updated plan request to the IA-TIGRIS planner every ten seconds.

The drone started at an altitude of $50$ m above the origin of the reference frame. The informed sampler in IA-TIGRIS was set to add states at altitudes of either $30$ m or $60$ m, creating a trade-off between observation area and detector accuracy. The budget was $2000$ m, the planning horizon was $600$ m, and the planning time was $10$ seconds. 

% % without authors goes here
% \begin{figure}[t]
% \centering
% % https://lucid.app/lucidchart/f750ddb4-2809-4773-8361-d5fbb1ba49eb/edit?viewport_loc=-257%2C-116%2C2219%2C1140%2C0_0&invitationId=inv_56e8a3a9-e8cf-4cad-a280-48bd967ff651
% \includegraphics[trim={0cm 0cm 0cm 0cm},clip,width=\columnwidth]{figs/5_/functional_diagram.jpeg}
% \caption{Functional diagram of the DJI M600 Pro autonomy software.}
% \label{fig:functional_diagram}
% \end{figure}
% \begin{figure}[b]
%     \centering
%     \begin{subfigure}[b]{0.48\columnwidth}
%         \centering
%         \includegraphics[width=1.0\linewidth]{figs/5_/field_test_altitude_over_time.png}
%         \caption{}
%         \label{fig:m600_altitude_over_time}
%     \end{subfigure}
%     \begin{subfigure}[b]{0.48\columnwidth}
%         \centering
%         \includegraphics[width=1.0\linewidth]{figs/5_/field_test_entropy_over_time.png}
%         \caption{}
%         \label{fig:m600_entropy_over_time}
%     \end{subfigure}
%     \caption{The results for our hexarotor field deployment. (a) Plot of flown altitude over time, showing large variation throughout the experiment. (b) Reduction in entropy percentage over time of field experiment.}
% \end{figure}

\subsubsection{Experimental Results}


The bottom image of Fig.~\ref{fig:google_earth} shows the path selected by IA-TIGRIS in the search area. The figure highlights how the planner dynamically adjusts altitudes over time to balance coverage and sensing resolution, maximizing information gain. Higher altitudes allow for broader area coverage, while lower altitudes provide more detailed observations where needed. Additionally, the planner prioritizes revisiting the three regions of higher uncertainty, recognizing the need for repeated observations reduce entropy. This adaptive strategy ensures that uncertain areas receive sufficient attention to improve the belief map. As a result, the entropy of the information map decreases to near zero by the end of the mission, as shown in Fig.~\ref{fig:m600_entropy_over_time}, indicating that the planner has effectively gathered the necessary information. This behavior demonstrates the planner’s ability to optimize sensing actions, balancing altitude selection, revisit frequency, and exploration to maximize mission success.

\begin{figure}[t]
\centering
% \includegraphics[width=2.5in]{fig1}
\includegraphics[trim={4cm 4cm 0cm 4cm},clip,width=\columnwidth]{figs/5_/TL1.jpg}
\caption{Fixed-wing platform used for autonomous flights with an onboard camera pitched at 10 degrees\cite{alarewebsite}}
\label{fig:tl1}
\end{figure}






\subsection{Fixed-wing Deployments}

Our proposed approach was extensively tested on the fixed-wing AlareTech TL-1 UAV, shown in Fig.~\ref{fig:tl1}. The UAV is equipped with an onboard camera pitched at 10 degrees, which introduces a more challenging planning problem due to the non-holonomic motion model and the camera's field of view. Over more than 20 flight hours and 100 flights running IA-TIGRIS, we validated our approach with the objective to search for objects of interest in a large search space across a variety of test scenarios, including different terrain types, varying environmental conditions, and diverse target distributions. An example mission from these tests is shown in Fig.~\ref{fig:fwd}. In this scenario, the planner was given the search bounds and a designated high-priority region. The resulting flight path prioritized revisiting the high-priority area twice, optimizing sensor use and ensuring maximum information gain. This strategy led to the successful detection of the object of interest, with its estimated position marked by the red dot in the figure. 

The map on the upper right in Fig.~\ref{fig:fwd} shows the information map after plan execution was complete. Due to the UAV's limited budget, the upper right and lower left corners of the map are not searched by the agent. The budget is instead utilized to search over the area of higher priority two times. Compared to the paths in Fig.~\ref{fig:google_earth}, we observe that the paths for the fixed wing are smoother and have a larger turning radius, demonstrating how IA-TIGRIS respects the motion constraints of the vehicle. We can also see the effect of wind on the path execution, where the flown path shown in green deviates from the planned path shown in yellow. This illustrates the importance of online planning in the cases where this deviation is large or would accumulate over the course of a longer mission and cause the expected observed area to be much different than actual observed area. 

\begin{figure}[t]
\centering
% \includegraphics[width=2.5in]{fig1}
% [trim={left bottom right top},clip]
\includegraphics[trim={3.0cm, 1.0cm, 3.0cm, 1.0cm},clip,width=\columnwidth]{figs/5_/ONRFig_v3.pdf}
\caption{An example path generated for the fixed-wing platform conducting a large-area search for an object of interest. The larger black rectangle denotes the search bounds, while the smaller black rectangle highlights a region of higher uncertainty. The red dot marks the estimated position of the detected object based on image detections. The upper-right map displays the information state after planning is complete, while the middle plot shows the percent change in entropy over mission time. The flown path illustrates a balance between allocating resources to the high-priority region and exploring other areas within the search space.}
\label{fig:fwd}
\end{figure}

% Also tested extensively on the AlareTech TL-1 (citation?) tube launched UAV seen in Fig.~\ref{fig:tl1}.

% Talk about amount of flights, hours. Platform. Compute. Show visualization fo example flight. Talk about objects of interest in a broad sense (no mention of water/ocean/land for targets). Follow similar figure format as previous section. Main thing we want to highlight is the differences introduced in plans by having a fixed-wing platform compared to a drone. Include image of Alare TL-1 somewhere.

% One big figure showing all the info we want to convey. 

% \BM{Pitch 10 degrees, onboard computer type, etc}


% \subsection{VTOL?}
% what would it bring?


\section{Discussion}
\label{sec:discussion}

% \TODO{Bryan}

Our multimodal data augmentation method is a plug-and-play method that can be applied to any future VLM. Also the T2I generation can be replaced by any future T2I model, thus the effectiveness of our method automatically improves along with the SOTA T2I model, making it future-proof.



Our main method, \textbf{Co}ntrastive Visual \textbf{D}ata \textbf{A}ugmentation (\textbf{CoDA}), is simple and easy to apply to LMMs in a variety of scenarios. Several components in the pipeline utilize existing off-the-shelf model components that can be easily swapped out for superior versions of similar models as research in their respective field progresses. Therefore, we expect the efficiency and effectiveness of \textbf{CoDA} to dramatically scale along with the advancement of relevant models. 



\pagebreak

\bibliography{citation}
\bibliographystyle{icml2025}

\pagebreak

\appendix
% \paragraph{Data-to-Text.} \citep{kukich-d2t, mckeown-d2t} is the task of converting structured data into fluent text. These structured data may correspond to tables \citep{totto}, meaning representations \citep{e2e}, relational graphs \citep{webnlg2017}, etc.
% %This complex format poses a significant challenge to LLMs pre-trained on plain text. 
% Recent approaches to data-to-text typically involve training end-to-end models with encoder-decoder architectures \citep{wiseman-etal-2017-challenges, gardent2017creating,RebuffelSSG20,RebuffelSSG20-ECIR,RebuffelRSSCG22}. Notably, using large pre-trained encoder-decoder models \citep{t5} has significantly improved performance by framing data-to-text as a text-to-text task \citep{kale-rastogi-2020-text, duong23a}. More recently, large pre-trained decoder-only models \citep{llama2} have shown strong performance and become the de facto approach for text generation, now being applied to data-to-text \citep{tablellama}. Despite these advancements, LLMs still struggle with hallucinations, and data-to-text generation is no exception.
This section reviews methods aimed at improving the faithfulness of LLMs to input contexts. We focus exclusively on approaches designed to ensure the generated content remains grounded in the provided information, excluding techniques related to factuality or external knowledge alignment.

\paragraph{Faithfulness enhancement.} Several methods have been used for improving faithfulness of text summarization. A first line of work consist in using external tools to retrieve key entities or facts form the source document and use these as weak labels during training \citep{zhang-etal-2022-improving-faithfulness}. \citet{faitful-improv} identify key entities using a Question-Answering system and modify the architecture of an encoder-decoder model to put more cross-attention weight on these entities. \citet{zhu-etal-2021-enhancing} propose to improve the faithfulness of summaries by extracting a knowledge graph from the input texts and embed it in the model cross-attention using a graph-transformer. Another line of work focuses on post-training improvements by bootstrapping model-generated outputs ranked by quality \citep{slic,brio,slic-nli}.
% \citet{zhang-etal-2022-improving-faithfulness} forces , \citet{faitful-improv} introduce a Question-Answering system enhanced encoder-decoder architecture, where the cross-attention in the decoder is directed towards key entities. \citet{zhu-etal-2021-enhancing} propose to improve the faithfulness of summaries by extracting a knowledge graph from the input texts and embed it in the model cross-attention using a graph-transformer.
Regarding data-to-text generation, \citet{RebuffelRSSCG22} propose a custom model architecture to reduce the effect of loosely aligned datasets, using token-level annotations and a multi-branch decoder model. The closest work to ours is from \citep{cao-wang-2021-cliff} which proposes a contrastive learning approach where synthetic samples are constructed using different tools like Named Entity Recognition (NER) models and back-translation.
%These approaches have been primarily designed and evaluated for text summarization. 
These approaches address specific forms of unfaithfulness and rely heavily on external tools such as NER or QA models, and are especially tailored for text summarization, while we target a more general focus. More recently, simpler methods that leverage only a pre-trained model have been proposed for summarization. \citet{cad,pmi} downweight the probabilities of tokens that are not grounded in the input context, using an auxiliary LM without access to the input context.
\citet{critic-driven} train a self-supervised classification model to detect hallucinations and guide the decoding process.  \cite{confident-decoding} propose a method to estimate the decoder's confidence by analyzing cross-attention weights, encouraging greater focus on the source during generation. Our method focuses on a decoder-only architecture and uses a single model, providing a streamlined and efficient approach specifically tailored for general conditional text generation tasks without the need for complex external tools.

\paragraph{Faithfulness evaluation.} Measuring faithfulness automatically is not straightforward. Traditional conditional text generation evaluation often relies on comparing the generated output to a reference text, typically measured using n-gram based metrics such as BLEU \citep{papineni-bleu} or ROUGE \citep{lin-2004-rouge}. However, reference-based metrics limitations are well known to correlate poorly with faithfulness \citep{fabbri-etal-2021-summeval,gabriel-etal-2021-go}. Both for summarization and data-to-text generation, new metrics evaluating the generation exclusively against the input context have been proposed, using QA models \citep{rebuffel-etal-2021-data,scialom-etal-2021-questeval} or entity-matching metrics \citep{nan-etal-2021-entity}. In this work, we evaluate primarily our models using recent NLI-related metrics \citep{alignscore, nli-d2t}, and LLM-as-a-judge, focusing on faithfulness \citep{gpt-chiang,gpt-gilardi}. For data-to-text generation, we also report the PARENT metric \citep{parent}, which computes n-gram overlap against elements of the source table cells.

%Additionally, corpora are often collected automatically, leading to divergences between the reference text and the actual input data. , since no direct comparison to the actual input source is actually performed. To address these issues, evaluation methods that take into account the input data have been proposed. \citet{parent} introduce PARENT, which computes the recall of n-gram overlap between the entities in the data and the candidate text. \citet{nli-d2t} develop an entailment metric using Natural Language Inference (NLI) models, where the generated text is compared directly to a simple verbalization of the data. The gold-standard still remains the human or human-like evaluation, conducted with powerful generalist LLMs. These metrics form the core focus of our work.

\paragraph{Preference tuning.} Recent instruction-tuned LLMs are often further refined through "human-feedback alignment" \citep{oaif}. These methods utilize human-crafted preference datasets, consisting of pairs of preferred and dispreferred texts $(\ywin, \ylose)$, typically obtained by collecting human feedback and ranking responses via voting. Recent work \citep{spin} uses the model's previous predictions in a self-play manner to iteratively improve the performance of chat-based models. Whether through an auxiliary preference model \citep{rlhf} or by directly tuning the models on the pairs \citep{dpo}, these approaches have demonstrated remarkable results in chat-based models. Our method leverages a preference framework without the need for human intervention and is specifically tailored for models trained on conditional text generation tasks.

% However, it remains unclear on what values the models are being aligned. Some works have shown that these methods can effectively alter the model's behaviour to the extent that they become useless and refuse to answer to any requests. In this work, we follow a preference fine-tuning scheme but tailored for input-aware tasks like data-to-text.


\section{Experiment Details}
\subsection{Data Splits \& Query Generation}
\label{app:data_split}

\begin{algorithm}
\caption{\textsc{Personalised Simple Query} ($u \cap a$) generation algorithm $u \cap a$}
\begin{algorithmic}[1]
    \STATE Let the set of users, attributes, and movies be $\mathcal{U}, \mathcal{A}, \mathcal{M}$
    \STATE Marginal probability of an attribute $a$ in $A$, $P(a) = \sum_{m} A_{a, m} / \sum_{a'} \sum_{m} A_{a', m}$
    \STATE Marginal probability of an user $u$ in $U$, $P(u) = \sum_{m} U_{u, m} / \sum_{u'} \sum_{m} U_{u', m}$
    \STATE Marginal probability of an movie $m$ in $U$, $P(m) = \sum_{u} U_{u, m} / \sum_{u} \sum_{m'} U_{u, m'}$
    \STATE Let $U$ be the User $\times$ Item matrix and $A$ be the Attribute $\times$ Item matrix.
    \STATE $U^{Train} \leftarrow U$, $A^{Train} \leftarrow A$
    \STATE $U^{Eval} \leftarrow \mathbf{0}$, $A^{Eval} \leftarrow \mathbf{0}$
    \STATE Set of simple personalized queries, $Q_{U \cap A} \leftarrow \phi$
    \WHILE{$|Q_{U \cap A}|$ < \textsc{Max Sample Size}}
        \STATE Sample an attribute $a$ from $\mathcal{A}$ according to $P(a)$.
        \STATE Sample a movie $m$ from for the attribute $a$, i.e., Sample from $\{m' | A_{a, m'} = 1\}$, according to $P(m)$
        \STATE Sample a user $u$ from who has rated movie $m$, i.e., Sample from  $\{u' | U_{m, u'} = 1\}$, according to $P(u)$
        \STATE $U^{Train}_{u, m} = 0$, $A^{Train}_{a, m} = 0$, $U^{Eval}_{u, m} = 1$, $A^{Eval}_{a, m} = 1$
        \STATE $Q_{U \cap A}$.\textsc{insert}($(u, a, m)$)
    \ENDWHILE
\end{algorithmic}
\label{alg:joint_sampling}
\end{algorithm}

\begin{algorithm}
\caption{\textsc{Personalised Complex Query} Generation Algorithm}
\begin{algorithmic}[1]
    \STATE Compositional Query sets $Q_{U \cap A_1 \cap A_2}$, $Q_{U \cap A_1 \cap \neg A_2}$
    \STATE Non-Trivial attribute combination set $\mathcal{A}_{\circ}$
    \FOR{each user-movie tuple in Eval set, i.e., $(u, m) \in \{(u, m) | U^{Eval}_{u, m} = 1\}$}
        \FOR{each pair of attributes $(a_1, a_2) \in \{(a_1, a_2) | A^{Eval}_{a_1, m} = 1 \text{ and } A^{Eval}_{a_2, m} = 1\}$}
            \IF{the pair is viable and non-trivial, i.e., $(a_1, a_2) \in \mathcal{A}_{\cap}$}
                \STATE $Q_{U \cap A_1 \cap A_2}$.\textsc{insert}($(u, a_1, a_2, m)$)
            \ENDIF
        \ENDFOR
        \FOR{each pair of attributes $(a_1, a_2) \in \{(a_1, a_2) | A^{Eval}_{a_1, m} = 1 \text{ and } A_{a_2, m} = 0\}$}
            \IF{the pair is viable and non-trivial, i.e., $(a_1, a_2) \in \mathcal{A}_{\setminus}$}
                \STATE $Q_{U \cap A_1 \cap \neg A_2}$.\textsc{insert}($(u, a_1, a_2, m)$)
            \ENDIF
        \ENDFOR
    \ENDFOR
\end{algorithmic}
\label{alg:complex_query}
\end{algorithm}

\subsection{Training Details}
\label{app:training_details}

\begin{table}[H]
\caption{Hyper Parameter range for all the dataset. We run 100 runs for both models and select the best model on User-Movie validation set NDCG metric}
\resizebox{\columnwidth}{!}{%

\begin{tabular}{ccccc}
\hline
Hyperparameters                         & \begin{tabular}[c]{@{}c@{}}Range\\ Box\end{tabular} & \begin{tabular}[c]{@{}c@{}}Best Value\\ Box\end{tabular} & \begin{tabular}[c]{@{}c@{}}Range\\ Vector\end{tabular} & \begin{tabular}[c]{@{}c@{}}Best Value\\ Vector\end{tabular} \\ \hline
Embedding dim                            & 64                                                  & 64                                                       & 128                                                    & 128                                                         \\
Learning Rate                            & 1e-1, 1e-2, 1e-3, 1e-4, 1e-5                        & 0.001                                                    & 1e-1, 1e-2, 1e-3, 1e-4, 1e-5                           & 0.001                                                       \\
Batch Size                               & 64, 128, 256, 512, 1024                             & 128                                                      & 64, 128, 256, 512, 1024                                & 128                                                         \\
\# Negatives                             & 1, 5, 10, 20                                        & 20                                                       & 1, 5, 10, 20                                           & 5                                                           \\
\multicolumn{1}{l}{Intersection Temp}    & 10, 2, 1, 1e-1, 1e-2, 1e-3, 1e-5                    & 2.0                                                      & -                                                      & -                                                           \\
\multicolumn{1}{l}{Volume Temp}          & 10, 5, 1, 0.1, 0.01, 0.001                          & 0.01                                                     & -                                                      & -                                                           \\
\multicolumn{1}{l}{Attribute Loss const} & 0.1, 0.3, 0.5, 0.7, 0.9                             & 0.7                                                      & 0.1, 0.3, 0.5, 0.7, 0.9                                & 0.5                                                         \\ \hline
\end{tabular}
}
\label{tab:hyperparams}
\end{table}
Hyperparameters are reported in Table \ref{tab:hyperparams}. Best parameter values are reported for Box Embeddings and \textsc{MF} method. 
\begin{figure*}[ht]
    \centering
    \includegraphics[width=0.8\textwidth]{pictures/hparam_search.png} % Adjust the width as needed
    \caption{Parallel Co-ordinate plot for different hyperparameters vs model performance. Lighter the color, better the model's performance.}
    \label{fig:generalization-spectrum}
\end{figure*}

\subsection{Model Selection}
\begin{table}[t]
    \centering
    \caption{Test NDCG on $D_{U}^\eval$ for selected models.}
    \scalebox{0.9}{
    \begin{tabular}{lllll}
        \toprule
        Dataset & \textsc{MF}   & \textsc{NeuMF} & \textsc{Lgcn} & \textsc{Box}  \\ \hline
        \addlinespace
        Last-FM & 0.51 & 0.52 & 0.56 & 0.65 \\
        NYC-R   & 0.31 & 0.33 & 0.37 & 0.39 \\
        ML-1M   & 0.51 & 0.53 & 0.55 & 0.58 \\
        ML-20M  & 0.71 & 0.70 & 0.72 & 0.73 \\ 
        \bottomrule
    \end{tabular}
    }
    \label{tab:model_selection}
\end{table}


\subsection{Set-Theoretic Generalization}
\begin{table}[H]
\caption{Hit Rate(\%)$\uparrow$ for Set-theoretic queries for dataset ML-20M. }
\resizebox{\columnwidth}{!}{%
\begin{tabular}{llllllllll}
\hline
\multicolumn{1}{c}{\multirow{2}{*}{Methods}} & \multicolumn{3}{c}{$U \cap A$} & \multicolumn{3}{c}{$U \cap A_1 \cap A_2$} & \multicolumn{3}{c}{$U \cap A_1 \cap \neg A_2$} \\ \cline{2-10} 
\multicolumn{1}{c}{}                         & h@10    & h@20    & h@50    & h@10        & h@20       & h@50       & h@10         & h@20          & h@50         \\ \hline
\addlinespace
\textsc{MF-Filter}         & 4.6      & 8.1      & 16.1     & 0.4          & 1.0         & 2.9         & 3.7             & 6.6              & 13.7             \\
\textsc{MF-Product}        & 4.1      & 7.5      & 15.6     & 3.3          & 6.6         & 16.4        & 2.7           & 5.1            & 11.4          \\
\textsc{MF-Geometric}      & 0.1      & 0.3      & 0.6      & 0.0          & 0.0         & 0.0         & 0.3           & 0.6            & 1.4           \\ \hdashline
\addlinespace
\textsc{NeuMF-Filter} & 4.6 & 8.2 &  16.1 & 1.1 & 5.6 & 6.4 & 4.9 & 7.3 & 13.9 \\
\textsc{NeuMF-product} & 4.6 & 8.2 & 16.1 & 4.1 & 8.5 & 22.1 & 4.3 & 6.9 & 12.0 \\ \hdashline
\addlinespace
\textsc{Box-Filter}         & 4.6      & 8.1      & 16.1     & 11.0         & 21.8        & 42.3        & 4.6           & 7.7            & 16.3           \\
\textsc{Box-Product}        & 4.5      & 8.2      & 16.1     & 11.1         & 21.8        & 42.5        & 4.3           & 7.1            & 15.1           \\
\textsc{Box-Geometric}      & 4.5      & 8.1      & 16.2     & 11.0         & 21.8        & 42.4        & \textbf{6.4}  & \textbf{12.8}  & \textbf{25.9} \\ \hline
\end{tabular}
}
\label{tab:set-theretic-results-ml20m}
\end{table}



\subsection{Spectrum of Weak Generalization}
\label{app:weak_generalization}

\begin{table}[H]
\caption{The spectrum of generalization for \textsc{Simple Personalized query} $U \cap A$. W: \textsc{Weakest Generalization}, W-U: \textsc{Weak Generalization-User}, W-A: \textsc{Weak Generalization-Attribute}, S: \textsc{Set Theoretic Generalization}}
\resizebox{\columnwidth}{!}{%

\begin{tabular}{llllllllll}
\hline
\multicolumn{1}{c}{\multirow{2}{*}{Methods}} & \multicolumn{3}{c}{Hit Rate @10}              & \multicolumn{3}{c}{Hit Rate @ 20}             & \multicolumn{3}{c}{Hit Rate @ 50}                      \\ \cline{2-10} 
\multicolumn{1}{c}{}                         & \multicolumn{3}{l}{W | W-U | W-A | S}         & \multicolumn{3}{l}{W | W-U | W-A | S}         & \multicolumn{3}{l}{W | W-U | W-A | S}                  \\ \hline
\textsc{MF-Filter}                         & \multicolumn{3}{l}{24.7 | 6.7 | 13.0 | 5.0}   & \multicolumn{3}{l}{36.3 | 13.3 | 20.7 | 10.2} & \multicolumn{3}{l}{54.2 | 30.1 | 33.3 | 22.3}          \\
\textsc{MF-Product}                        & \multicolumn{3}{l}{23.3 | 5.7 | 13.1 | 4.3}   & \multicolumn{3}{l}{35.0 | 10.8 | 21.4 | 8.5}  & \multicolumn{3}{l}{54.7 | 24.2 | 38.8 | 20.4}          \\
\textsc{MF-Geometric}                      & \multicolumn{3}{l}{4.9 | 0.9 | 1.8 | 0.4}     & \multicolumn{3}{l}{7.9 | 1.7 | 3.3 | 0.9}     & \multicolumn{3}{l}{15.1 | 4.5 | 7.4 | 3.0}             \\ \hline
\textsc{Box-Filter}                         & \multicolumn{3}{l}{24.1 | 13.0 | 16.4 | 11.7} & \multicolumn{3}{l}{34.5 | 22.3 | 24.6 | 19.1} & \multicolumn{3}{l}{50.5 | 40.5 | 37.6 | 32.3}          \\
\textsc{Box-Product}                        & \multicolumn{3}{l}{25.2 | 13.6 | 13.9 | 10.0} & \multicolumn{3}{l}{35.2 | 21.5 | 21.9 | 16.7} & \multicolumn{3}{l}{52.2 | 38.4 | 38.3 | 31.5}          \\
\textsc{Box-Geometric}                      & \multicolumn{3}{l}{25.4 | 14.7 | 14.8 | 11.0} & \multicolumn{3}{l}{35.6 | 23.3 | 23.5 | 18.3} & \multicolumn{3}{l}{\textbf{52.2 | 40.8 | 40.5 | 34.1}} \\ \hline
\end{tabular}
}
\label{tab:generalization-spectrum-simple-query}
\end{table}

\begin{table}[H]
\caption{The spectrum of generalization for \textsc{Complex Personalized query} $U \cap A_1 \cap \neg A_2$. W: \textsc{Weakest Generalization}, W-U: \textsc{Weak Generalization-User}, W-A: \textsc{Weak Generalization-Attribute}, S: \textsc{Set Theoretic Generalization}}
\resizebox{\columnwidth}{!}{%
\begin{tabular}{llllllllll}
\hline
\multicolumn{1}{c}{\multirow{2}{*}{Methods}} & \multicolumn{3}{c}{Hit Rate @10}              & \multicolumn{3}{c}{Hit Rate @ 20}             & \multicolumn{3}{c}{Hit Rate @ 50}                      \\ \cline{2-10} 
\multicolumn{1}{c}{}                         & \multicolumn{3}{l}{W | W-U | W-A | S}         & \multicolumn{3}{l}{W | W-U | W-A | S}         & \multicolumn{3}{l}{W | W-U | W-A | S}                  \\ \hline
\textsc{MF-Filter}                        & \multicolumn{3}{l}{25.5 | 13.0 | 12.4 | 4.7}  & \multicolumn{3}{l}{34.9 | 14.1 | 19.5 | 9.8}  & \multicolumn{3}{l}{54.7 | 29.5 | 37.1 | 22.5}          \\
\textsc{MF-Product}                       & \multicolumn{3}{l}{23.5 | 7.0 | 10.4 | 3.4}   & \multicolumn{3}{l}{34.9 | 12.8 | 18.0 | 7.3}  & \multicolumn{3}{l}{54.5 | 27.5 | 35.0 | 19.3}          \\
\textsc{MF-Geometric}                     & \multicolumn{3}{l}{5.2 | 2.0 | 1.7 | 0.5}     & \multicolumn{3}{l}{8.8 | 3.5 | 1.9 | 1.0}     & \multicolumn{3}{l}{17.4 | 8.8 | 6.5 | 2.7}             \\ \hline
\textsc{Box-Filter}                        & \multicolumn{3}{l}{24.1 | 15.3 | 15.0 | 11.4} & \multicolumn{3}{l}{35.5| 22.7 | 21.1 | 19.5}  & \multicolumn{3}{l}{\textbf{54.1 | 39.2 | 37.3 | 34.0}} \\
\textsc{Box-Product}                       & \multicolumn{3}{l}{21.1 | 13.7 | 12.0 | 8.9}  & \multicolumn{3}{l}{30.5 | 21.7 | 19.3 | 15.2} & \multicolumn{3}{l}{47.4 | 38.0 | 35.0 | 29.4}          \\
\textsc{Box-Geometric}                     & \multicolumn{3}{l}{21.1 | 13.2 | 10.8 | 8.6}  & \multicolumn{3}{l}{30.4 | 20.8 | 17.7 | 15.1} & \multicolumn{3}{l}{\textbf{47.3 | 36.6 | 33.2 | 31.0}} \\ \hline
\end{tabular}
}
\label{tab:generalization-spectrum-difference-query}
\end{table}

\begin{table}[H]
\caption{The spectrum of generalization for \textsc{Complex Personalized query} $U \cap A_1 \cap A_2$. W: \textsc{Weakest Generalization}, W-U: \textsc{Weak Generalization-User}, W-A: \textsc{Weak Generalization-Attribute}, S: \textsc{Set Theoretic Generalization}}
\resizebox{\columnwidth}{!}{%
\begin{tabular}{llllllllll}
\hline
\multicolumn{1}{c}{\multirow{2}{*}{Methods}} & \multicolumn{3}{c}{Hit Rate @10}              & \multicolumn{3}{c}{Hit Rate @ 20}             & \multicolumn{3}{c}{Hit Rate @ 50}                      \\ \cline{2-10} 
\multicolumn{1}{c}{}                         & \multicolumn{3}{l}{W | W-U | W-A | S}         & \multicolumn{3}{l}{W | W-U | W-A | S}         & \multicolumn{3}{l}{W | W-U | W-A | S}                  \\ \hline
\textsc{MF-Filter}             & \multicolumn{3}{l}{35.3 | 17.6 | 16.9 | 11.4} & \multicolumn{3}{l}{45.0 | 27.3 | 23.3 | 17.9} & \multicolumn{3}{l}{55.2 | 41.9 | 30.5 | 27.5}          \\
\textsc{MF-Product}            & \multicolumn{3}{l}{34.0 | 11.0 | 11.6 | 5.1}  & \multicolumn{3}{l}{47.3 | 19.6 | 20.1 | 10.6} & \multicolumn{3}{l}{67.4 | 38.5 | 39.3 | 26.1}          \\
\textsc{MF-Geometric}          & \multicolumn{3}{l}{6.13 | 3.1 | 0.3 | 0.1}    & \multicolumn{3}{l}{9.90 | 5.8 | 0.6 | 0.2}    & \multicolumn{3}{l}{18.5 | 12.9 | 1.8 | 0.8}            \\ \hline
\textsc{Box-Filter}             & \multicolumn{3}{l}{30.8 | 21.5 | 17.3 | 14.5} & \multicolumn{3}{l}{41.1 | 31.2 | 23.3 | 20.5} & \multicolumn{3}{l}{52.7 | 44.5 | 30.3 | 28.5}          \\
\textsc{Box-Product}            & \multicolumn{3}{l}{35.4 | 23.8 | 13.4 | 10.6} & \multicolumn{3}{l}{47.0 | 34.5 | 21.7 | 17.8} & \multicolumn{3}{l}{64.6 | 52.8 | 39.0 | 34.2}          \\
\textsc{Box-Geometric}          & \multicolumn{3}{l}{34.6 | 25.2 | 20.0 | 16.8} & \multicolumn{3}{l}{45.7 | 35.7 | 30.5 | 26.6} & \multicolumn{3}{l}{\textbf{62.6 | 53.3 | 50.1 | 46.1}} \\ \hline
\end{tabular}
}
\label{tab:generalization-spectrum-intersection-query}
\end{table}

% \begin{table}[H]
% \resizebox{\columnwidth}{!}{%
% \begin{tabular}{l|cccc|cccc|cccc}
% \hline
% \multicolumn{1}{c|}{\multirow{2}{*}{Methods}} & \multicolumn{4}{c|}{Hit Rate @10} & \multicolumn{4}{c|}{Hit Rate @20} & \multicolumn{4}{c}{Hit Rate @50} \\ \cline{2-13} 
% \multicolumn{1}{c|}{} & W & W-U & W-A & S & W & W-U & W-A & S & W & W-U & W-A & S \\ \hline
% \textsc{MF-Filter}    & 35.3 & 17.6 & 16.9 & 11.4 & 45.0 & 27.3 & 23.3 & 17.9 & 55.2 & 41.9 & 30.5 & 27.5 \\
% \textsc{MF-Product}   & 34.0 & 11.0 & 11.6 & 5.1  & 47.3 & 19.6 & 20.1 & 10.6 & 67.4 & 38.5 & 39.3 & 26.1 \\
% \textsc{MF-Geometric} & 6.13 & 3.1  & 0.3  & 0.1  & 9.90 & 5.8  & 0.6  & 0.2  & 18.5 & 12.9 & 1.8  & 0.8  \\ \hline
% \textsc{Box-Filter}    & 30.8 & 21.5 & 17.3 & 14.5 & 41.1 & 31.2 & 23.3 & 20.5 & 52.7 & 44.5 & 30.3 & 28.5 \\
% \textsc{Box-Product}   & 35.4 & 23.8 & 13.4 & 10.6 & 47.0 & 34.5 & 21.7 & 17.8 & 64.6 & 52.8 & 39.0 & 34.2 \\
% \textsc{Box-Geometric} & 34.6 & 25.2 & 20.0 & 16.8 & 45.7 & 35.7 & 30.5 & 26.6 & \textbf{62.6} & 53.3 & 50.1 & 46.1 \\ \hline
% \end{tabular}
% }
% \caption{Hit Rate Results}
% \label{tab:generalization-spectrum-intersection-query-2}
% \end{table}

\begin{figure}[ht!]
  \centering
  \includegraphics[width=\columnwidth]{pictures/weak_generaliztion.png}
  \caption{Weak Generalization Illustration}
  \label{fig:weak_generalization}
\end{figure}


The \textsc{Box-Geometric} achieves the best \textit{Generalization Spectrum Gap} for all types of queries.

\section{Error Compounding Analysis}
\label{app:error_compounding}

\begin{figure}[ht!]
    \centering
    \includegraphics[width=0.4\textwidth]{pictures/all_success.png}
    \caption{Relationships of correct answers by the three box models on $u \wedge a_1 \wedge a_2$ queries.}
    \label{fig:first-figure}
\end{figure}

\begin{figure}[ht!]
    \centering
    \includegraphics[width=0.4\textwidth]{pictures/compund_error_solved.png}
    \caption{The Geometric method subsumes the benefit of the product in compounding error.}
    \label{fig:second-figure}
\end{figure}

\begin{figure}[ht!]
    \centering
    \includegraphics[width=0.4\textwidth]{pictures/not_compunding_error.png}
    \caption{The effect is less for the non-compounding error.}
    \label{fig:third-figure}
\end{figure}

We further perform more granular analysis amongst the \textsc{Box} based methods with complex query type $U \cap A_1 \cap A_2$. As claimed in our initial hypothesis, the \textsc{Filter} method suffers from error compounding. If the target movie $m$ is in the model's prediction list for $A_1$ but not for $A_2$ or the other way round, we denote this error as \textit{compounding error}. In figure \ref{fig:second-figure}, out of the compounding errors, $34 \%$ is solved by the \textsc{Box-Geometric} method and $26 \%$ by the \textsc{Box-Product} method. However, in figure \ref{fig:third-figure}, for the error that is not due to compounding (where the model gets both $A_1$ and $A_2$ prediction wrong), only $18 \%$ are corrected by the \textsc{Box-Geometric} method and a mere $10 \%$ of them are corrected by \textsc{Box-Product}. Refer to figure \ref{fig:first-figure} \ref{fig:second-figure} \ref{fig:third-figure} for details. This demonstrates that the \textsc{Box-Geometric} significantly contributes to the correction of error compounding.


\section{{Time Efficiency analysis}}

\begin{table}[ht]
\centering
\caption{Training time (\textit{mm:ss}) for a single epoch are measured for different batch sizes with 5 negative samples on Movielens-1M dataset. Experiments are conducted on Nvidia GTX 1080Ti gpus}
\begin{tabular}{lllll}
\hline
\begin{tabular}[c]{@{}l@{}}Batch Size\end{tabular} & \textsc{MF} & \textsc{NeuMF} & \textsc{LightGCN} & \textsc{Box} \\ \hline
64                                                   & 08:37                        & 17:00                           & 70:30                            & 19:32                         \\
128                                                  & 04:32                        & 09:46                           & 38:40                              & 11:40                         \\
256                                                  & 02:29                        & 04:40                           & 20:55                              & 05:28                         \\
512                                                  & 01:18                        & 02:23                           & 10:47                              & 02:54                         \\
1024                                                 & 00:40                        & 01:20                           & 05:24                              & 01:12                         \\ \hline
\end{tabular}
\label{tab:training_time}
\end{table}

{In \Cref{tab:training_time}, we observe that the \textsc{MF}, being the simplest approach with minimal computational requirements, is consistently the fastest across all batch sizes. At the largest batch size (1024), it achieves the shortest training time of just 00:40. The \textsc{Box}-based method exhibits training times comparable to \textsc{NeuMF}. However, it is significantly faster than \textsc{LightGCN}, which relies on graph convolutional computations. The iterative message-passing operations required by \textsc{LightGCN} result in considerably higher training times, particularly at smaller batch sizes (e.g., 70:30 at a batch size of 64). As the batch size increases, the training time for \textsc{Box} embeddings becomes almost as efficient as \textsc{MF}. For instance, at a batch size of 1024, \textsc{Box} achieves a training time of 01:12, compared to 00:40 for \textsc{MF}. This demonstrates that the computational complexity of box embeddings is of the same order as \textsc{MF}, making it a scalable and efficient choice.}

{Box embeddings are generally quite fast because the computation of box intersection volumes can be parallelized over dimensions. Note that the training times above use GumbleBox embeddings, which involve log-sum-exp calculations. However, this could be improved even further at inference time by replacing these soft min and max approximations with hard operators. If such an optimized approach is desired, then training can accommodate this by regularizing temperature. For deployment in industrial set-up, we could take additional steps with Box Embeddings as outlined in \cite{box_for_search}.}



\end{document}


% This document was modified from the file originally made available by
% Pat Langley and Andrea Danyluk for ICML-2K. This version was created
% by Iain Murray in 2018, and modified by Alexandre Bouchard in
% 2019 and 2021 and by Csaba Szepesvari, Gang Niu and Sivan Sabato in 2022.
% Modified again in 2023 and 2024 by Sivan Sabato and Jonathan Scarlett.
% Previous contributors include Dan Roy, Lise Getoor and Tobias
% Scheffer, which was slightly modified from the 2010 version by
% Thorsten Joachims & Johannes Fuernkranz, slightly modified from the
% 2009 version by Kiri Wagstaff and Sam Roweis's 2008 version, which is
% slightly modified from Prasad Tadepalli's 2007 version which is a
% lightly changed version of the previous year's version by Andrew
% Moore, which was in turn edited from those of Kristian Kersting and
% Codrina Lauth. Alex Smola contributed to the algorithmic style files.
