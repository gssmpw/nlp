% \begin{table}[h]
%     \centering
% \scalebox{0.8}{
%     \begin{minipage}[t]{0.5\textwidth}

% \centering
% \caption{Compositional Query Statistics}
% \begin{tabular}{lccc}
% \toprule
% \multicolumn{1}{c}{\multirow{2}{*}{Dataset}} & \begin{tabular}[c]{@{}c@{}}Personalized\\ Simple Query\end{tabular} & \multicolumn{2}{c}{\begin{tabular}[c]{@{}c@{}}Personalized\\ Complex Query\end{tabular}} \\
% \multicolumn{1}{c}{}                         & $u \cap a$                                                          & $u \cap a_1 \cap a_2$                    & $u \cap a_1 \cap \neg a_2$                    \\ \hline
% \addlinespace % Adds space
% Last-FM                                      & 9,867                                                               & 45,142                                   & 10,814                                        \\
% NYC-R                                        & 9,482                                                               & 7,460                                    & 2,369                                         \\ 
% ML-1M                                        & 21,392                                                              & 51,299                                   & 37,769                                        \\
% ML-20M                                       & 35,368                                                              & 42,355                                   & 47,374                                        \\\bottomrule
% \end{tabular}
% \label{tab:set_queries}
% \end{minipage}
% }
% \hfill
% \scalebox{0.9}{
% \begin{minipage}[t]{0.5\textwidth}

% \caption{Test NDCG on $D_{U}^\eval$ for selected models.\\}
% \begin{tabular}{lllll}
% \toprule
% Dataset & \textsc{MF}   & \textsc{NeuMF} & \textsc{Lgcn} & \textsc{Box}  \\ \hline
% \addlinespace
% Last-FM & 0.51 & 0.52 & 0.56 & 0.65 \\
% NYC-R   & 0.31 & 0.33 & 0.37 & 0.39 \\
% ML-1M   & 0.51 & 0.53 & 0.55 & 0.58 \\
% ML-20M  & 0.71 & 0.70 & - & 0.73 \\ \bottomrule
% \end{tabular}
% \label{tab:model_selection}
% \end{minipage}
% }
% \end{table}

\begin{table}[t]
    \centering
    \caption{Compositional Query Statistics}
    \scalebox{0.9}{
    \begin{tabular}{lccc}
        \toprule
        \multicolumn{1}{c}{\multirow{2}{*}{Dataset}} & \begin{tabular}[c]{@{}c@{}}Personalized\\ Simple Query\end{tabular} & \multicolumn{2}{c}{\begin{tabular}[c]{@{}c@{}}Personalized\\ Complex Query\end{tabular}} \\
        \multicolumn{1}{c}{}                         & $u \cap a$                                                          & $u \cap a_1 \cap a_2$                    & $u \cap a_1 \cap \neg a_2$                    \\ \hline
        \addlinespace % Adds space
        Last-FM                                      & 9,867                                                               & 45,142                                   & 10,814                                        \\
        NYC-R                                        & 9,482                                                               & 7,460                                    & 2,369                                         \\ 
        ML-1M                                        & 21,392                                                              & 51,299                                   & 37,769                                        \\
        ML-20M                                       & 35,368                                                              & 42,355                                   & 47,374                                        \\
        \bottomrule
    \end{tabular}
    }
    \label{tab:set_queries}
    
\end{table}

\vspace{-5pt}
\section{Results}

% Now with those selected model we test out set-theoretic queries.
% Abletion queries. (What are these abletions? why are we doing that?)
% Qualitative analysis.
%

%We train both the baseline methods and our approach on jointly on $D_U^\trn$ and $D_A^\trn$. 
After conducting an extensive hyper-parameter search on $D_U^\textrm{eval}$, we select the top-performing model for each method based on NDCG scores (see Table \ref{tab:model_selection} in the Appendix for the model selection details). This ensures that the chosen model is optimal for set-theoretic query inference, with the following performance results.
%We train the baseline methods as well as our method on the joint data of User$\times$Movie and Attribute$\times$Movie on the training data split as discussed in Section \ref{sec:traning_details}. After an extensive hyper-parameter search for each method, we select the model that performs best on the  User$\times$Movie validation set in terms of NDCG. We choose this as the model selection criteria because the eventual task of answering queries is essentially the ranking task of user preferences.\shib{more clarity required} We report the performances of the selected model on both datasets ML-20M and ML-1M in Table \ref{tab:model_selection}
\subsection{Set-Theoretic Generalization}
\label{sec:main_results} 
We test the selected models for each method with the curated set-theoretic personalized queries (Detailed stats for the queries in Table \ref{tab:set_queries}). We report the ranking performance in terms of Hit Rates at $10$, $20$, and $50$. Please refer to \ref{tab:set_theoretic_results} for the results.
\begin{table}[]
\centering
\caption{\small Hit Rate(\%)$\uparrow$ on Set-theoretic queries for datasets Last-FM, MovieLens 1M, NYC-R.}
\resizebox{\columnwidth}{!}{%
\begin{tabular}{llllllllll}
\toprule
\multicolumn{1}{c}{\multirow{2}{*}{Methods}} & \multicolumn{3}{c}{$U \cap A$}                & \multicolumn{3}{c}{$U \cap A_1 \cap A_2$}     & \multicolumn{3}{c}{$U \cap A_1 \cap \neg A_2$} \\ \cline{2-10}
\addlinespace
\multicolumn{1}{c}{}                         & h@10          & h@20          & h@50          & h@10          & h@20          & h@50          & h@10           & h@20          & h@50          \\ \midrule
\addlinespace
\multicolumn{10}{c}{\hspace{9em} \textsc{Last-FM}}                                                                                                                                                                  \\ \midrule
\addlinespace
\textsc{MF-Filter}                                    & 14.8          & 25.1          & 37.4          & 26.8          & 46.8          & 62.8          & 15.2           & 24.4          & 35.5          \\
\textsc{MF-Product}                                   & 9.0           & 21.7          & 48.0          & 14.3          & 36.8          & 73.2          & 4.8            & 14.8          & 43.4          \\
\textsc{MF-Geometric}                                 & 6.1           & 12.2          & 29.7          & 3.4           & 7.6           & 27.5          & 1.7            & 4.8           & 15.9          \\ \hdashline
\addlinespace
\textsc{NeuMF-Filter}                                 & 13.5          & 21.9          & 32.3          & 20.0          & 19.6          & 55.7          & 11.3           & 18.8          & 28.7          \\
\textsc{NeuMF-Product}                                & 13.6          & 25.6          & 47.6          & 19.5          & 35.7          & 63.3          & 9.0            & 16.8          & 40.5          \\ \hdashline
\addlinespace
{\textsc{LGCN-Filter}}  & 20.4 & 28.5 & 39.1 & {\ul 42.4} & 54.2 & 67.4 & 15.8 & 21.5 & 27.6 \\
{\textsc{LGCN-Product}}  & 20.5 & 31.0 & 48.6 & \textbf{43.8} & {\ul 58.0} & 80.7 & 0.8 & 1.3 & 3.5 \\ \hdashline
\addlinespace
\textsc{Box-Filter}                                   & 22.9          & 31.5          & 39.0          & 32.7          & 46.5          & 55.9          & \textbf{22.0}  & 32.1          & 40.3          \\
\textsc{Box-Product}                                  & {\ul 27.9}    & {\ul 44.5}    & {\ul 68.0}    & 38.2    & 57.7    & {\ul 82.7}    & {\ul 17.8}     & {\ul 32.4}    & \textbf{60.3} \\
\textsc{Box-Geometric}                                & \textbf{28.3} & \textbf{44.8} & \textbf{68.3} & 38.8 & \textbf{58.3} & \textbf{83.1} & 17.5           & \textbf{32.5} & {\ul 60.0}    \\ \midrule
\addlinespace
\multicolumn{10}{c}{\hspace{9em} \textsc{MovieLens-1M}}                                                                                                                                                                    \\ \midrule
\addlinespace
\textsc{MF-Filter}                                    & 5.0           & 10.2          & 22.3          & 11.4          & 17.9          & 27.5          & 4.7            & 9.8           & 22.5          \\
\textsc{MF-Product}                                   & 4.3           & 8.5           & 20.4          & 5.1           & 10.6          & 26.1          & 3.4            & 7.3           & 19.3          \\
\textsc{MF-Geometric}                                 & 0.4           & 0.9           & 3.0           & 0.1           & 0.2           & 0.8           & 0.5            & 1.0           & 2.7           \\ \hdashline
\addlinespace
\textsc{NeuMF-Filter}                                 & 9.3           & 15.5          & 28.5          & 13.3          & 21.5          & 35.9          & 8.8            & 14.7          & 26.7          \\
\textsc{NeuMF-Product}                                & 10.3          & 16.8          & 31.4          & {\ul 15.3}    & {\ul 24.5}    & {\ul 43.5}    & 5.7            & 9.7           & 20.2          \\ \hdashline
\addlinespace
{\textsc{LGCN-Filter}}  & 8.2 & 12.3  & 20.9 & 11.4 & 15.6 & 24.0 & 9.9 & 13.8 & 21.9 \\
{\textsc{LGCN-Product}}  & 5.9 & 9.0 & 14.9 & 7.6 & 11.7 & 20.1 & 5.5 & 8.6 & 14.1 \\ \hdashline
\addlinespace
\textsc{Box-Filter}                                   & \textbf{11.7} & \textbf{19.1} & {\ul 32.3}    & 14.5          & 20.5          & 28.6          & \textbf{11.4}  & \textbf{19.5} & \textbf{34.0} \\
\textsc{Box-Product}                                  & 9.95          & 16.7          & 31.5          & 10.6          & 17.8          & 34.2          & {\ul 8.9}      & 15.1          & 29.4          \\
\textsc{Box-Geometric}                                & {\ul 11.0}    & {\ul 18.3}    & \textbf{34.2} & \textbf{16.9} & \textbf{26.6} & \textbf{46.1} & 8.6            & {\ul 15.2}    & {\ul 31.0}    \\ \midrule
\addlinespace
\multicolumn{10}{c}{\hspace{9em} \textsc{NYC-R}}                                                                                                                                                                    \\ \midrule
\addlinespace
\textsc{MF-Filter}                                    & 1.4           & 2.4           & 4.6           & 2.7           & 4.8          & 8.0           & 2.1            & 3.5           & 6.3           \\
\textsc{MF-Product}                                   & 1.1           & 2.9           & 8.6           & 3.7           & 8.2           & 23.3          & 8.9            & 13.1          & 17.6          \\
\textsc{MF-Geometric}                                 & 0.5           & 1.5           & 4.3           & 0.2           & 0.8           & 3.5           & 0.5            & 1.2           & 3.7           \\ \hdashline
\addlinespace
\textsc{NeuMF-Filter}                                 & 3.8           & 5.6           & 9.2           & 2.5           & 3.2           & 4.5           & 4.2            & 6.3           & 10.8          \\
\textsc{NeuMF-Product}                                & 4.6           & 7.3           & 13.7          & 6.6           & 11.2          & 20.8          & 2.7            & 5.2           & 11.2          \\ \hdashline
\addlinespace
{\textsc{LGCN-Filter}}  & 4.8 & 7.8 & 17.2 & {\ul 12.7} & 16.9 & 21.8 & 5.4 & 8.6 & 16.4 \\
{\textsc{LGCN-Product}}  & \text{5.0} & {\ul 8.7} & \textbf{18.1} & 12.1 & 17.6 & 35.1 & 4.9 & 8.0 & 13.2 \\ \hdashline
\addlinespace
\textsc{Box-Filter}                                   & 4.9           & 7.8           & 13.4          & 9.9           & 13.5          & 20.4          & 4.4            & 7.1           & 12.5          \\
\textsc{Box-Product}                                  & \textbf{5.0}  & \textbf{8.9}  & \textbf{17.9} & {\ul 10.9}    & {\ul 19.5}    & {\ul 37.3}    & {\ul 5.3}      & {\ul 9.1}     & {\ul 18.8}    \\
\textsc{Box-Geometric}                                & {\ul 4.9}     & {\ul 8.7}     & {\ul 17.6}    & \textbf{12.2} & \textbf{21.5} & \textbf{39.2} & \textbf{5.5}   & \textbf{9.2}  & \textbf{19.2} \\ \bottomrule
\end{tabular}
}
\label{tab:set_theoretic_results}
\end{table}
% Please add the following required packages to your document preamble:
% \usepackage{multirow}


% \begin{figure}[h!]
%     \centering
%     \includegraphics[width=1.0\textwidth]{Styles/pictures/u_a.png}
%     \caption{Hit Rate for $U \cap A$ type query.}
%     \label{fig:u_a}
% \end{figure}
% \begin{figure}[h!]
%     \centering
%     \includegraphics[width=1.0\textwidth]{Styles/pictures/u_a1_a2.png}
%     \caption{Hit Rate for $U \cap A_1 \cap A_2$ type query.}
%     \label{fig:u_a1_a_2}
% \end{figure}
% \begin{figure}[h!]
%     \centering
%     \includegraphics[width=1.0\textwidth]{Styles/pictures/u_a1_not_a2.png}
%     \caption{Hit Rate for $U \cap A_1 \cap \neg A_2$ type query.}
%     \label{fig:u_a1_not_a2}
% \end{figure}

The Box Embedding-based method outperforms vector-based methods by a significant margin, showing on average 30\% improvement when comparing the aggregated HR@50 performance of the best vector model (\textsc{MF-Product}/\textsc{NeuMF-Product}/\textsc{LGCN-Filter}) to the box model (\textsc{Box-Geometric}) across all the three different domains.

The $U \cap A_1 \cap A_2$ query is the most challenging, as it requires accuracy in all three individual queries. For this difficult query, \textsc{Box-Geometric} shows the largest performance gap compared to other methods. Additionally, using vector addition and subtraction as geometric proxies for intersection and difference performs significantly worse than all other vector-based methods, while geometric operations in the box embedding space outperform even other box embedding methods. This validates the set-theoretic inductive bias of box embeddings and confirms that geometric operations in this space provide valid set-theoretic operations, unlike vectors.

The \textsc{Filter} aggregation technique performs similarly to or better than other methods only for Hits@$10$. However, as $k$ increases, its performance declines across all model types (Box, MF, NeuMF) and datasets. This observation highlights the limitation of a fixed threshold filter and advocates smoother aggregation techniques like \textsc{Product} and \textsc{Geometric}.
\vspace{-4pt}
\subsection{Spectrum of Generalization}
\label{sec:spectrum_generalization}
\vspace{-2pt}
The query generation process (refer Section \ref{sec:simple_query}) ensures that for the target item $m$ corresponding to a query involving user $u$ and attribute $a$, the pair $(u, m)$ and $(a, m)$ must not be in the training set $(u,m) \notin \D_U^\trn$ and $(a,m) \notin \D_A^\trn$. The set-theoretic evaluation weakens when such pairs are added back to the training set. There are three different weakening settings applicable here, which we refer to as a spectrum -- \textsc{Weakest Generalization}
($(u,m) \in \D_U^\trn$ and $(a,m) \in \D_A^\trn$), \textsc{Weak Generalization-User}
($(u,m) \in \D_U^\eval$ and $(a,m) \notin \D_A^\trn$), \textsc{Weak Generalization-Attribute}
($(u,m) \notin \D_U^\trn$ and $(a,m) \in \D_A^\eval$). We report HitRate@50 performance on query type $U\cap A_1 \cap A_2$  for the MovieLens-1M dataset in Table \ref{tab:generalization-spectrum-gap} (More query types in Appendix - Table \ref{tab:generalization-spectrum-difference-query}, \ref{tab:generalization-spectrum-simple-query}).\\
%However, we can weaken the generalization testing criteria here. We can consider test movie $m$ where $(u,m) \in \D_U^\trn$ or $(a,m)\in \D_U^\trn$ or both.
% from the training set for either $u$ or $a$ or both.
%We report the results Hit Rate performance on these splits in the Appendix, refer to Table \ref{tab:generalization-spectrum-difference-query}, \ref{tab:generalization-spectrum-simple-query}, \ref{tab:generalization-spectrum-intersection-query-2}.\\
The weaker the generalization setting the easier it is for the models to achieve higher performance on the test set. Indeed, we observe that this is true across all the methods w.r.t each of the aggregation settings, validating the correctness of the trained models. \\
However, we are interested in observing the performance gap when we go from the weakest to the strongest set-theoretic generalization. We refer to the percentage gap \textit{Generalization Spectrum Gap} (hr(Weakest) - hr(Set-theoretic) / hr(Weakest) \%). From Table \ref{tab:generalization-spectrum-gap} we observe that the best-performing box model \textsc{Box-Geometric} achieves the best \textit{Generalization Spectrum Gap} for HR$@50$. 
%This observation is consistent across different query types and ranking metrics (Please refer to appendix \ref{app:weak_generalization}).
%One would expect the performance of the \textsc{Weakest Generalization} to be much higher than the rest of the settings. Also, the  \textsc{Weak Generalization-Attribute} or \textsc{Weak Generalization-Attribute} must produce better performance numbers than the true \textsc{Set Theoretic Generalization}. \\
%We demonstrate the same result in Figure \ref{fig:generalization-spectrum} (In Appendix \ref{app:weak_generalization}). Also, the superiority holds across all query types (Refer to Appendix \ref{app:weak_generalization}). \\
%We report more granular analysis amongst the \textsc{Box} based methods with complex query type $U \cap A_1 \cap A_2$ in \Cref{app:error_compounding}.

\begin{table}[]
\caption{\small \textit{Generalization Spectrum Gap} for \textsc{Personalized Complex Query} $U \cap A_1 \cap A_2$}
\centering
\resizebox{\columnwidth}{!}{%
\begin{tabular}{lccccc}
\toprule
\multicolumn{1}{c}{\multirow{2}{*}{Methods}} & \multicolumn{4}{c}{Hit Rate @50 $\uparrow$}                                                                                                                                                                                                      & \multirow{2}{*}{\begin{tabular}[c]{@{}c@{}} \textit{Spectrum Gap} $\downarrow$\\ \\ (W $-$ S) / W\end{tabular}} \\ \cline{2-5}
\addlinespace
\multicolumn{1}{c}{}                         & \begin{tabular}[c]{@{}c@{}}Weakest\\ (W)\end{tabular} & \begin{tabular}[c]{@{}c@{}}Weak-User\\ (W-U)\end{tabular} & \begin{tabular}[c]{@{}c@{}}Weak-Attribute\\ (W-A)\end{tabular} & \begin{tabular}[c]{@{}c@{}}Set-Theoretic\\ (S)\end{tabular} &                                                                                                                            \\ \midrule

\textsc{MF-Filter}              & 55.2                                                  & 41.9                                                      & 30.5                                                           & 27.5                                                        & 50.2\%                                                                                                                     \\
\textsc{MF-Product}             & \textbf{67.4}                                         & 38.5                                                      & 39.3                                                           & 26.1                                                        & 61.2 \%                                                                                                                    \\
\textsc{MF-Geometric}           & 18.5                                                  & 12.9                                                      & 1.8                                                            & 0.8                                                         & 95.6\%                                                                                                                     \\ \hdashline
\addlinespace
\textsc{NeuMF-Filter}           & 48.4                                                  & 33.1                                                      & 40.4                                                           & 35.9                                                        & 38.5\%                                                                                                                     \\
\textsc{NeuMF-Product}          & 67.8                                                  & 48.7                                                      & 40.6                                                           & 43.5                                                        & 35.9\%                                                                                                                     \\ \hdashline
\addlinespace
\textsc{Box-Filter}             & 52.7                                                  & 44.5                                                      & 30.3                                                           & 28.5                                                        & 45.9\%                                                                                                                     \\
\textsc{Box-Product}            & 64.6                                                  & 52.8                                                      & 39.0                                                           & 34.2                                                        & 47.1\%                                                                                                                     \\
\textsc{Box-Geometric}          & 62.6                                                  & 53.3                                                      & 50.1                                                           & \textbf{46.1}                                               & \textbf{26.4\%}                                                                                                            \\ \bottomrule
\end{tabular}
}

\label{tab:generalization-spectrum-gap}
\end{table}
% Please add the following required packages to your document preamble:
% \usepackage{multirow}
% Please add the following required packages to your document preamble:
% \usepackage{multirow}
\vspace{-5pt}
\section{Related Work}
\label{sec:related work}
\subsection{Box Embeddings} 
%If the embedding space was normalized (\eg constraining the boxes to be within a unit hypercube) this can be shown to provide a valid probability distribution. This work demonstrated this probabilistic interpretation on MovieLens, representing associating each movie with a binary random variable with marginal probability proportional to the number of users who rated the movie, and similarly defining the joint probability of two such binary random variables as proportional to the number of users who rated both.\mb{Maybe this belongs in the related work.}\\
% a directed graph such that boxes of parents contain their children with an energy function
Some of the recent works have tried to incorporate box embeddings in a recommendation systems setup.\citet{InBox, box-diverse, users-as-box} use the side-length of the box embeddings as a preference range to obtain diverse set recommendations for users, \citet{box-efficient-ranking} utilizes the axis parallel nature of the box embeddings for faster retrieval. \citet{faithful_emb, sun2020guessing, query2box} are some of the recent works that focus on logical query over knowledge bases (KB). However, in this work, we frame collaborative filtering as a set-theoretic matrix completion problem, which helps us to achieve better generalization for the composition of personalized queries.

\subsection{Set-based queries in Search and group recommendation systems.}
While set-theoretic queries are commonplace in search, popular question-answering (QA) benchmarks often do not include them. We found QUEST \citep{quest} the most closely related study, introducing a benchmark for entity-seeking queries with implicit set-based semantics. However, QUEST does not focus on explicit constraints or personalization, which are central to our work. 
% Additionally, related studies in group recommendation systems \citep{group-rec} touch on explicit constraint-based personalization, where preferences of multiple users are explicitly aggregated into a coherent recommendation.


