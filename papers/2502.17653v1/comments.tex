\usepackage{ifthen}
% \usepackage{xcolor}

\renewenvironment{comment}{}{
\PackageError{Comment Package}
{
Do not use the comments package!
%
If you want to uncomment then please use the normal
uncommenting of latex.
%
This way the commented text is highlighted and
immediately visible in the Latex source.
%
If you seek to display a comment in the pdf then
create a new command in the 'comments.tex' file
}
{extra help}
}

\newboolean{showcomments}
\setboolean{showcomments}{false}

% use \setboolean{showcomments}{false} anywhere in your document after
% importing this to disable the comments

\makeatletter
\newcommand{\mynote}[3]{%
  \ifthenelse{\boolean{showcomments}}{%
   \fbox{\bfseries\sffamily\scriptsize#1}%
   {\small\textsf{\emph{\color{#3}{#2}}}}}%
  %  {\small$\blacktriangleright$\textsf{\emph{\color{#3}{#2}}}$\blacktriangleleft$}}%
  {%
   % these two lines ensure that there is no blank space inserted
   \@bsphack
   \@esphack
  }%
}
\makeatother

% One command per author:
\definecolor{asparagus}{rgb}{0.53, 0.66, 0.42}
\newcommand{\todo}[1]{\mynote{TODO}{#1}{red}}
\newcommand{\se}[1]{\mynote{Sebastian}{#1}{brown}}
\newcommand{\jm}[1]{\mynote{Jannik}{#1}{purple}}
\newcommand{\sa}[1]{\mynote{Sara}{#1}{orange}}
\newcommand{\code}[1]{\mynote{Code Reference}{#1}{green}}

\newcommand{\out}[1]{}
