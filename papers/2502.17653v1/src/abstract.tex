\begin{abstract}
%

    %
%% The remote attestation (\ra) mechanism relies on cryptographic security protocols for verifying 
%% the security properties of remote devices. 
%% %
%% However, due to its complexity, the formalization of \ra often has some untouched formal security guarantees. 
%% %
%% This paper proposes a step towards the first machine-checked proof of \ra by reducing 
%% its security to that of a digital signature scheme. 
%% %
%% Our formal development was conducted in SSProve, a framework 
%% for modular cryptographic reasoning in \coq. 
%% %
%% We define the formal specification of digital signature and establish indistinguishability, i.e., 
%% strong existential unforgeability for digital signatures.
%% %
%% Our proof constructions demonstrate that a secure digital signature guarantees \ra security, 
%% generalizing across cryptographic primitives. 
%% %


%
% Due to its complexity, the formalization of remote attestation, which often relies on cryptographic security protocols, is unexplored. 
%
%Remote attestation, which often relies on cryptographic security protocols, is a complex 
%security mechanism in trusted computing. 
%
%%%%
%Due to its complexity, the formalization of remote attestation (\ra) 
%often has some untouched formal security guarantees. 
%
%The literature shows that this could be because the existing formal reasoning is either informal or lacks rigorous security guarantees. 
%
%This paper tries to fill the gap and proposes a step towards the first machine-checked proof of RA by reducing its 
%security to that of a digital signature scheme. 
%


  %
  % 1) What's the problem? Why does it matter?
  %

%
Remote attestation (\ra) is the foundation for 
trusted execution environments in the cloud and 
trusted device driver onboarding in operating systems.
%
However, \ra misses a rigorous mechanized definition of its
security properties in one of the strongest models, i.e.,
the semantic model. 
%
  %
  % 2) What's our new insight? / What's your new contribution?
  %
%
Such a mechanization requires 
the concept of State-Separating Proofs (SSP).
%
However, SSP was only recently implemented as a foundational framework in the \coq. 
%

%
Based on this framework, this paper presents the first mechanized formalization of the fundamental security properties of \ra. 
%
Our \coq development first defines digital signatures and
formally verifies security against forgery in the strong
existential attack model.
%
Based on these results, we define \ra and reduce the security
of \ra to the security of digital signatures.

  %
  % 3) How well does it work? / What did you find?
  %

%
Our development provides evidence that the \ra protocol
is secure against forgery.
%
Additionally, we extend our reasoning to the primitives
of \ra and reduce their security to the security of the
primitives of the digital signatures.
%
Finally, we found that proving the security of the
primitives for digital signatures was not feasible.
%
This observation contrasts textbook formalizations
and sparks a discussion on reasoning about the security
of libraries in SSP-based frameworks.
%

  % < 200 words (< 150 is even beter)
\end{abstract}
