\section{A Formal Specification for Digital Signatures}
\label{sec:sig}

%
In this section, we present our formal specification of digital
signatures and verify that it is secure against forgery. 
%
First, we present an overview of our specifications and fix their notations.
%
Afterwards, we specify the package for key generation.
%and state its assumptions for concrete implementations such as RSA. 
%
Then, we followed the specifications for digital signatures. 
%
First, we define the packages and then provide details for proof of 
game-based indistinguishability (perfect).
%

\subsection{Overview and Notations}
%
Our development has two parts: the composition of the packages
and the proof development to establish the security properties.
%

\subsubsection{Package Composition}

\begin{figure*}
	\centering
	\section{Overview}

\revision{In this section, we first explain the foundational concept of Hausdorff distance-based penetration depth algorithms, which are essential for understanding our method (Sec.~\ref{sec:preliminary}).
We then provide a brief overview of our proposed RT-based penetration depth algorithm (Sec.~\ref{subsec:algo_overview}).}



\section{Preliminaries }
\label{sec:Preliminaries}

% Before we introduce our method, we first overview the important basics of 3D dynamic human modeling with Gaussian splatting. Then, we discuss the diffusion-based 3d generation techniques, and how they can be applied to human modeling.
% \ZY{I stopp here. TBC.}
% \subsection{Dynamic human modeling with Gaussian splatting}
\subsection{3D Gaussian Splatting}
3D Gaussian splatting~\cite{kerbl3Dgaussians} is an explicit scene representation that allows high-quality real-time rendering. The given scene is represented by a set of static 3D Gaussians, which are parameterized as follows: Gaussian center $x\in {\mathbb{R}^3}$, color $c\in {\mathbb{R}^3}$, opacity $\alpha\in {\mathbb{R}}$, spatial rotation in the form of quaternion $q\in {\mathbb{R}^4}$, and scaling factor $s\in {\mathbb{R}^3}$. Given these properties, the rendering process is represented as:
\begin{equation}
  I = Splatting(x, c, s, \alpha, q, r),
  \label{eq:splattingGA}
\end{equation}
where $I$ is the rendered image, $r$ is a set of query rays crossing the scene, and $Splatting(\cdot)$ is a differentiable rendering process. We refer readers to Kerbl et al.'s paper~\cite{kerbl3Dgaussians} for the details of Gaussian splatting. 



% \ZY{I would suggest move this part to the method part.}
% GaissianAvatar is a dynamic human generation model based on Gaussian splitting. Given a sequence of RGB images, this method utilizes fitted SMPLs and sampled points on its surface to obtain a pose-dependent feature map by a pose encoder. The pose-dependent features and a geometry feature are fed in a Gaussian decoder, which is employed to establish a functional mapping from the underlying geometry of the human form to diverse attributes of 3D Gaussians on the canonical surfaces. The parameter prediction process is articulated as follows:
% \begin{equation}
%   (\Delta x,c,s)=G_{\theta}(S+P),
%   \label{eq:gaussiandecoder}
% \end{equation}
%  where $G_{\theta}$ represents the Gaussian decoder, and $(S+P)$ is the multiplication of geometry feature S and pose feature P. Instead of optimizing all attributes of Gaussian, this decoder predicts 3D positional offset $\Delta{x} \in {\mathbb{R}^3}$, color $c\in\mathbb{R}^3$, and 3D scaling factor $ s\in\mathbb{R}^3$. To enhance geometry reconstruction accuracy, the opacity $\alpha$ and 3D rotation $q$ are set to fixed values of $1$ and $(1,0,0,0)$ respectively.
 
%  To render the canonical avatar in observation space, we seamlessly combine the Linear Blend Skinning function with the Gaussian Splatting~\cite{kerbl3Dgaussians} rendering process: 
% \begin{equation}
%   I_{\theta}=Splatting(x_o,Q,d),
%   \label{eq:splatting}
% \end{equation}
% \begin{equation}
%   x_o = T_{lbs}(x_c,p,w),
%   \label{eq:LBS}
% \end{equation}
% where $I_{\theta}$ represents the final rendered image, and the canonical Gaussian position $x_c$ is the sum of the initial position $x$ and the predicted offset $\Delta x$. The LBS function $T_{lbs}$ applies the SMPL skeleton pose $p$ and blending weights $w$ to deform $x_c$ into observation space as $x_o$. $Q$ denotes the remaining attributes of the Gaussians. With the rendering process, they can now reposition these canonical 3D Gaussians into the observation space.



\subsection{Score Distillation Sampling}
Score Distillation Sampling (SDS)~\cite{poole2022dreamfusion} builds a bridge between diffusion models and 3D representations. In SDS, the noised input is denoised in one time-step, and the difference between added noise and predicted noise is considered SDS loss, expressed as:

% \begin{equation}
%   \mathcal{L}_{SDS}(I_{\Phi}) \triangleq E_{t,\epsilon}[w(t)(\epsilon_{\phi}(z_t,y,t)-\epsilon)\frac{\partial I_{\Phi}}{\partial\Phi}],
%   \label{eq:SDSObserv}
% \end{equation}
\begin{equation}
    \mathcal{L}_{\text{SDS}}(I_{\Phi}) \triangleq \mathbb{E}_{t,\epsilon} \left[ w(t) \left( \epsilon_{\phi}(z_t, y, t) - \epsilon \right) \frac{\partial I_{\Phi}}{\partial \Phi} \right],
  \label{eq:SDSObservGA}
\end{equation}
where the input $I_{\Phi}$ represents a rendered image from a 3D representation, such as 3D Gaussians, with optimizable parameters $\Phi$. $\epsilon_{\phi}$ corresponds to the predicted noise of diffusion networks, which is produced by incorporating the noise image $z_t$ as input and conditioning it with a text or image $y$ at timestep $t$. The noise image $z_t$ is derived by introducing noise $\epsilon$ into $I_{\Phi}$ at timestep $t$. The loss is weighted by the diffusion scheduler $w(t)$. 
% \vspace{-3mm}

\subsection{Overview of the RTPD Algorithm}\label{subsec:algo_overview}
Fig.~\ref{fig:Overview} presents an overview of our RTPD algorithm.
It is grounded in the Hausdorff distance-based penetration depth calculation method (Sec.~\ref{sec:preliminary}).
%, similar to that of Tang et al.~\shortcite{SIG09HIST}.
The process consists of two primary phases: penetration surface extraction and Hausdorff distance calculation.
We leverage the RTX platform's capabilities to accelerate both of these steps.

\begin{figure*}[t]
    \centering
    \includegraphics[width=0.8\textwidth]{Image/overview.pdf}
    \caption{The overview of RT-based penetration depth calculation algorithm overview}
    \label{fig:Overview}
\end{figure*}

The penetration surface extraction phase focuses on identifying the overlapped region between two objects.
\revision{The penetration surface is defined as a set of polygons from one object, where at least one of its vertices lies within the other object. 
Note that in our work, we focus on triangles rather than general polygons, as they are processed most efficiently on the RTX platform.}
To facilitate this extraction, we introduce a ray-tracing-based \revision{Point-in-Polyhedron} test (RT-PIP), significantly accelerated through the use of RT cores (Sec.~\ref{sec:RT-PIP}).
This test capitalizes on the ray-surface intersection capabilities of the RTX platform.
%
Initially, a Geometry Acceleration Structure (GAS) is generated for each object, as required by the RTX platform.
The RT-PIP module takes the GAS of one object (e.g., $GAS_{A}$) and the point set of the other object (e.g., $P_{B}$).
It outputs a set of points (e.g., $P_{\partial B}$) representing the penetration region, indicating their location inside the opposing object.
Subsequently, a penetration surface (e.g., $\partial B$) is constructed using this point set (e.g., $P_{\partial B}$) (Sec.~\ref{subsec:surfaceGen}).
%
The generated penetration surfaces (e.g., $\partial A$ and $\partial B$) are then forwarded to the next step. 

The Hausdorff distance calculation phase utilizes the ray-surface intersection test of the RTX platform (Sec.~\ref{sec:RT-Hausdorff}) to compute the Hausdorff distance between two objects.
We introduce a novel Ray-Tracing-based Hausdorff DISTance algorithm, RT-HDIST.
It begins by generating GAS for the two penetration surfaces, $P_{\partial A}$ and $P_{\partial B}$, derived from the preceding step.
RT-HDIST processes the GAS of a penetration surface (e.g., $GAS_{\partial A}$) alongside the point set of the other penetration surface (e.g., $P_{\partial B}$) to compute the penetration depth between them.
The algorithm operates bidirectionally, considering both directions ($\partial A \to \partial B$ and $\partial B \to \partial A$).
The final penetration depth between the two objects, A and B, is determined by selecting the larger value from these two directional computations.

%In the Hausdorff distance calculation step, we compute the Hausdorff distance between given two objects using a ray-surface-intersection test. (Sec.~\ref{sec:RT-Hausdorff}) Initially, we construct the GAS for both $\partial A$ and $\partial B$ to utilize the RT-core effectively. The RT-based Hausdorff distance algorithms then determine the Hausdorff distance by processing the GAS of one object (e.g. $GAS_{\partial A}$) and set of the vertices of the other (e.g. $P_{\partial B}$). Following the Hausdorff distance definition (Eq.~\ref{equation:hausdorff_definition}), we compute the Hausdorff distance to both directions ($\partial A \to \partial B$) and ($\partial B \to \partial A$). As a result, the bigger one is the final Hausdorff distance, and also it is the penetration depth between input object $A$ and $B$.


%the proposed RT-based penetration depth calculation pipeline.
%Our proposed methods adopt Tang's Hausdorff-based penetration depth methods~\cite{SIG09HIST}. The pipeline is divided into the penetration surface extraction step and the Hausdorff distance calculation between the penetration surface steps. However, since Tang's approach is not suitable for the RT platform in detail, we modified and applied it with appropriate methods.

%The penetration surface extraction step is extracting overlapped surfaces on other objects. To utilize the RT core, we use the ray-intersection-based PIP(Point-In-Polygon) algorithms instead of collision detection between two objects which Tang et al.~\cite{SIG09HIST} used. (Sec.~\ref{sec:RT-PIP})
%RT core-based PIP test uses a ray-surface intersection test. For purpose this, we generate the GAS(Geometry Acceleration Structure) for each object. RT core-based PIP test takes the GAS of one object (e.g. $GAS_{A}$) and a set of vertex of another one (e.g. $P_{B}$). Then this computes the penetrated vertex set of another one (e.g. $P_{\partial B}$). To calculate the Hausdorff distance, these vertex sets change to objects constructed by penetrated surface (e.g. $\partial B$). Finally, the two generated overlapped surface objects $\partial A$ and $\partial B$ are used in the Hausdorff distance calculation step.
	\caption{
  %
  Our formal specification of remote attestation is based on digital signatures.
  %
  The security reasoning is based on the concept of indistinguishability
  for a game pair \real, which represents the actual code,
  and \ideal, which represents the model, i.e., the semantics
  of the real code
  in the context of indistinguishability proofs.}\label{fig:overview}
\end{figure*}


\begin{figure}
  \centering
  \begin{tikzpicture}[
	font=\footnotesize,
	node distance=0.3cm
	]
	\node[] (P) at (0,0)  {
		Sig\pprot
	};
	\node[] (c) [below = 0.5cm of P] {
		\begin{minipage}{0.45\columnwidth}
			\begin{minted}[fontsize=\footnotesize,escapeinside=&&,autogobble]{coq}
def prot (m):
  #import get_pk ;;
  #import sign ;;
  #import ver_sig ;;

  pk <- get_pk tt ;;
  s <- sign m ;;
  b <- ver_sig s m ;;

  #ret (pk,s,b)
			\end{minted}
		\end{minipage}
	};
	\node[] (s) [above right = 0cm and 0cm of c.north west] {
        \begin{minipage}{0.45\columnwidth}
            \begin{minted}[fontsize=\footnotesize,escapeinside=&&,autogobble]{coq}
				  &$\{ \}$&
			\end{minted}
		\end{minipage}
	};

%%%%%%%% imports %%%%%%%
	\draw[->] 
    ([xshift=-1.8cm,yshift=-0.7cm]c.north west) -- 
    ([xshift=-0.2cm,yshift=-0.7cm]c.north west)
	    node[midway,above] { \mintinline{Coq}{get_pk} };
	
	\draw[->] 
    ([xshift=-1.8cm,yshift=-0.2cm]c.west) -- 
    ([xshift=-0.2cm,yshift=-0.2cm]c.west)
        node[midway,above] { \mintinline{Coq}{sign} };

	\draw[->] 
    ([xshift=-1.8cm,yshift=0.2cm]c.south west) -- 
    ([xshift=-0.2cm,yshift=0.2cm]c.south west)
        node[midway,above]{ \mintinline{Coq}{ver_sig} };
	
%%%%%%%% exports %%%%%%%
    \draw[->] 
    ([xshift=0.2cm,yshift=-0.2cm]c.east) -- 
    ([xshift=1.8cm,yshift=-0.2cm]c.east)
        node[midway,above]{ \mintinline{Coq}{prot} };

	% Package frame and stuff
	\draw[] (c.north west) -- (c.north east);
	\draw[] (c.north west) |- (s.north) -| (c.north east);
	\draw[] (s.north west) |- (P.north) -| (c.north east);
	\draw[] (c.north east) |- (c.south) -| (c.north west);

\end{tikzpicture}

  \caption{The protocol package for digitial signatures.}
  \label{fig:pprot}
\end{figure}
%
Figure~\ref{fig:overview} visualizes our specification using SSP-style
visualization for package composition.
%
In the figure, we accordingly highlight \real (shaded in grey) and 
\ideal (shaded in yellow) packages.
%
Our specification builds upon a key generation package \pkeygen that
establishes the probabilistic foundation.
%
The package \pkeygen abstracts over a concrete foundation to allow
for various instantiations such as RSA and ECDSA.
%
A full specification for the instantiation with RSA is in Appendix~\ref{sec:rsa}
with the details in the proof development.
%
The package \pkeygen exports a \ekeygen procedure for the packages that
define the \real and \ideal signature
primitives.
%
%Figure~\ref{fig:problem:eu-ind} shows that the composition of \prot \sa{SigProt?} with 
%packages \real and \ideal for signature
%primitves instantiates the protocol for digitial signatures.
%
%The protocol remains unchanged for remote attestation such that
%we can compose \pprot similarly with the \real and \ideal packages 
%for the remote attestation primitives.
%

\subsubsection{Proof Development}
%
Our proof development shows that our structured approach proves 
the security of signatures by establishing an equivalence between 
sEUF-CMA and indistinguishability-based security. 
%
The main idea is that if a signature scheme is secure with strong unforgeability, 
then the probability of forging a valid signature for the previously signed message is negligible. 
%
The signature verification in $\mathcal{G}_{\real}^{\Sigma}$ returns true with a valid message-signature pair. 
%
By the definition of strong unforgeability assumption, the $\mathcal{G}_{\ideal}^{\Sigma}$ must also return true, 
as any valid signature must have been generated by singing primitive. 
%
This ensures that the real and ideal signature verification processes remain indistinguishable. 
%
Contrariwise, if the real and ideal signature packages are 
indistinguishable, it implies that the ideal one must also 
accept any valid message-signature pair accepted by the real package. 
%
If an adversary could forge a valid signature, the real package would return \emph{true}, 
while the ideal package would return \emph{false}. 
%
By ensuring that the adversary can not distinguish these two cases, 
we naturally guarantee that forging a new signature is infeasible. 
%
This reasoning extends to both fresh and previously signed messages, which leads 
to the judgment that indistinguishability implies sEUF-CMA. 
%

Our proof construction establishes indistinguishability, i.e.,
strong existential unforgeability, for digital signatures based on
the protocol packages (Theorem~\ref{theo:indist-signature}). 
%
Theorem~\ref{theo:redprot} is a reduction theorem and states that
the security of remote attestation is smaller or equal to the
security of digital signatures.
%
Theorem~\ref{theo:indist-ra} then uses Theorem~\ref{theo:indist-signature}
and Theorem~\ref{theo:redprot} to verify strong existential
unforgeability of remote attestation.
%


\subsection{Key Generation}

\begin{figure}
	\centering
    \begin{subfigure}[b]{\columnwidth}
    \centering
       \begin{minted}[fontsize=\footnotesize,escapeinside=@@,autogobble]{coq}
Parameters SecKey PubKey : finType.
Parameter  @$\Sigma$.@KeyGen : @$\forall$@ s, code s (PubKey × SecKey).
		  	\end{minted}
      \caption{Parameters.}
	    \label{fig:keygen:params}
    \end{subfigure}
\\[0.3cm]
  \begin{subfigure}[b]{\columnwidth}
    \centering
    \begin{tikzpicture}[
	font=\footnotesize,
	node distance=0.3cm
	]
	\node[] (P) at (0,0)  {
		\pkeygen
	};
	\node[] (c) [below = 0.5cm of P] {
		\begin{minipage}{0.5\columnwidth}
			\begin{minted}[fontsize=\footnotesize,escapeinside=&&,autogobble]{coq}
def key_gen ():
				  
  (sk,pk) <- &$\Sigma$&.KeyGen tt ;;
  #put @&$\mathcal{SK}$& sk ;;
  #put @&$\mathcal{PK}$& pk ;;

  #ret (sk,pk)
			\end{minted}
		\end{minipage}
	};
	\node[] (s) [above right = 0cm and 0cm of c.north west] {
    \begin{minipage}{0.65\columnwidth}
		\begin{minted}[fontsize=\footnotesize,escapeinside=&&,autogobble]{coq}
      &$\mathcal{SK}$& : SecKey, &$\mathcal{PK}$& : PubKey
    \end{minted}
    \end{minipage}
	};

%%%%%%%% exports %%%%%%%
    \draw[->] 
    ([xshift=0.2cm,yshift=-0cm]c.east) -- 
    ([xshift=1.8cm,yshift=-0cm]c.east)
        node[midway,above]{ \mintinline{Coq}{key_gen} };


	% Package frame and stuff
	\draw[] (c.north west) -- (c.north east);
	\draw[] (c.north west) |- (s.north) -| (c.north east);
	\draw[] (s.north west) |- (P.north) -| (c.north east);
	\draw[] (c.north east) |- (c.south) -| (c.north west);

\end{tikzpicture}

    \caption{Package.}
    \label{fig:keygen:package}
  \end{subfigure}
  \caption{
    %
    The \pkeygen package and its parameters.
    %
  }
  \label{fig:keygen}
\end{figure}

%
Following the (textbook) definitions from Section~\ref{sec:TheoryFound},
we define the package \pkeygen in Figure~\ref{fig:keygen}.
%
We leave the definition of the secret key (\lseckey)
and the public key (\lpubkey) abstract as parameters
and just require them to be of a finite type.
%
Similarly, we parameterize the package with an algorithm
$\Sigma$.\pkeygen that implements the final key generation.
%
We do not import this algorithm because otherwise, we cannot play
the security game: game packages cannot have imports.
%
The algorithm $\Sigma$.\pkeygen nevertheless emits monadic code
because it needs to sample the keys from a distribution.
%
This differs from the textbook definition, which states that the $\Sigma$.KeyGen
is a pure function.
%
The $\Sigma$.\pkeygen is polymorph in the state \icoq{s}, i.e.,
it has no side-effects to the state other than for sampling.
%
The \egetpk procedure stores the generated secret and public
keys into its state and returns them both.
%
We are now ready to specify digital signatures.
%
We start with defining the primitives, and
afterwards, we compose the protocol games to prove perfect indistinguishability, 
i.e., strong existential unforgeability for digital signatures.
%

\subsection{Primitives}

\begin{figure}
	\centering
   \begin{subfigure}[b]{\columnwidth}
    \centering
        \begin{minted}[fontsize=\footnotesize,escapeinside=@@,autogobble]{coq}
Parameters Message Signature : finType.
Parameter  @$\Sigma$@.Sign : SecKey -> Message -> Signature.
Parameter  @$\Sigma$@.VerSig :
  PubKey -> Signature -> Message -> bool.
	\end{minted}
    \caption{Parameters}
    \label{fig:sigprim:params}
  \end{subfigure}
\\[0.3cm]
   \begin{subfigure}[b]{\columnwidth}
    \centering
        \begin{minted}[fontsize=\footnotesize,escapeinside=@@,autogobble]{coq}
Hypothesis sig_correct :
  @$\forall$@ m sk pk,
    ((sk,pk) <- @$\Sigma$@.KeyGen) ->
    @$\Sigma$@.VerSig pk (@$\Sigma$@.Sign sk m) m == true.
	\end{minted}
    \caption{Functional Correctness}
    \label{fig:sigprim:hypo}
  \end{subfigure}
\\[0.3cm]
\begin{subfigure}[b]{\columnwidth}
    \centering
    \begin{tikzpicture}[
	font=\footnotesize,
	node distance=0.3cm
	]
	\node[] (real) at (0,0)  {
		\psigprim$_{\text{\real}}$
	};
	\node[] (code-r) [below = 0.75cm of P] {
		\begin{minipage}{0.45\columnwidth}
			\begin{minted}[fontsize=\footnotesize,escapeinside=&&,autogobble]{coq}
get_pk() :=
  #import key_gen ;;

  (_,pk) <- key_gen tt ;;
  #ret pk

sign(m) :=
  sk <- #get @&$\mathcal{SK}$& ;;
  let &$\sigma$& := &$\Sigma$&.Sign sk m ;;



  #ret &$\sigma$&

ver_sig(m,s) :=
  pk <- #get @&$\mathcal{PK}$& ;;
  #ret &$\Sigma$&.VerSig pk s m
			\end{minted}
		\end{minipage}
	};
	\node[] (state-r) [above right = 0cm and 0cm of code-r.north west] {
	\begin{minipage}{0.45\columnwidth}
		\begin{minted}[fontsize=\footnotesize,escapeinside=&&,autogobble]{coq}
&$\mathcal{SK}$& : SecKey, &$\mathcal{PK}$& : PubKey
&\phantom{$\mathcal{SIG}$}&
		\end{minted}
	\end{minipage}
	};

	% Package frame and stuff
  \draw[] (code-r.north west) -- (code-r.north east);
	\draw[] (code-r.north west) |- (state-r.north) -| (code-r.north east);
	\draw[] (state-r.north west) |- (real.north) -| (code-r.north east);
	\draw[] (code-r.north east) |- (code-r.south) -| (code-r.north west);

    %%%%%%%% Ideal %%%%%%%%%%%%
    %%%%%%%%%%%%%%%%%%%%%%%%%%%

    \node[] (ideal) [right = 2.5 cm of real]  {
      \psigprim$_{\text{\ideal}}$
    };
    \node[] (code-i) [below = 0.75cm of ideal] {
    	\begin{minipage}{0.45\columnwidth}
    		\begin{minted}[fontsize=\footnotesize,escapeinside=&&,autogobble,highlightlines={10,11,12,16,17},highlightcolor=lYellow!20]{coq}
get_pk() :=
  #import key_gen ;;

  (_,pk) <- key_gen tt ;;
  #ret pk

sign(m) :=
  sk <- #get @&$\mathcal{SK}$& ;;
  let s := &$\Sigma$&.Sign sk m ;;
  S <- #get @&$\mathcal{SIG}$& ;;
  let S' := S &$\cup$& {(m,s)} ;;
  #put @&$\mathcal{SIG}$& S' ;;
  #ret s

ver_sig(m, s) :=
  S <- #get @&$\mathcal{SIG}$& ;;
  #ret ( (m,s) &$\in$& S )
    		\end{minted}
    	\end{minipage}
    };
    \node[] (state-i) [above right = 0cm and 0cm of code-i.north west] {
    	\begin{minipage}{0.45\columnwidth}
    		\begin{minted}[fontsize=\footnotesize,escapeinside=&&,autogobble,highlightlines={2},highlightcolor=lYellow!20]{coq}
&$\mathcal{SK}$&: SecKey, &$\mathcal{PK}$&: PubKey,
&$\mathcal{SIG}$&: (Message x Signature)
    		\end{minted}
    	\end{minipage}
    };

    % Package frame and stuff
    \draw[] (code-i.north west) -- (code-i.north east);
    \draw[] (code-i.north west) |- (state-i.north) -| (code-i.north east);
    \draw[] (state-i.north west) |- (ideal.north) -| (code-i.north east);
    \draw[] (code-i.north east) |- (code-i.south) -| (code-i.north west);

\end{tikzpicture}

    \caption{Packages}
    \label{fig:sigprim:packages}
  \end{subfigure}
	\caption{
    %
    The \real and \ideal Signature Primitives packages
    and their parameters and hyptohesis.
    %
    }
	\label{fig:sigprim}
\end{figure}

%
Figure~\ref{fig:sigprim} defines the abstractions and then
specifies the primitives for our digital signatures.
%

%
In line with the (textbook) definition presented in
Figure~\ref{fig:problem:ind}, our implementation of
the signature primitives abstracts over a concrete
signature scheme.
%
In Figure~\ref{fig:sigprim:params}, we abstract over concrete
implementations for messages and signatures.
%
The abstractions for signing a message and verifying a signature
complete Definition~\ref{def:sig:scheme} of a signature scheme $\Sigma$.
%
Note that both $\Sigma$\icoq{.Sign} and $\Sigma$.\icoq{VerSig}
are pure functions because they neither sample nor access any state.
%
Functional correctness (Definition~\ref{def:sig:correct}) is then
a hypothesis of our primitives for digital signatures (Figure~\ref{fig:sigprim:hypo}).
%

%
We specify the \real and \ideal packages for the primitives
of our digital signatures in Figure~\ref{fig:sigprim:packages}.
%
Our specification pays attention to accessing and updating
the state, e.g., for the set \icoq{S} of observed signatures in the
\ideal package.
%
Both packages import \ekeygen to obtain the
public key \icoq{pk}.
%
Note again that \ekeygen stores the generated keys \icoq{sk}
and \icoq{pk} into its state.
%
Respectively, the state of \pkeygen needs to be a
subset of the state of each package such that \esign and
\eversig can access the keys.
%
Apart from these implementation details,
both package definitions follow the (textbook)
specification from Figure~\ref{fig:problem:ind}.
%


\subsection{Indistinguishability}

\begin{figure}
    \begin{minted}[fontsize=\footnotesize,escapeinside=&&,autogobble]{coq}
Definition SigProt&$_{\text{\real}}$& :=
  {package SigProt &$\circledcirc$& SigPrim&$_{\text{\real}}$& &$\circ$& KeyGen}.

Definition SigProt&$_{\text{\ideal}}$& :=
  {package SigProt &$\circledcirc$& SigPrim&$_{\text{\ideal}}$& &$\circ$& KeyGen}.
    \end{minted}
   \caption{Protocol games for digital signatures.}
   \label{fig:sigprot}
\end{figure}

%We prove security against forgery for our digital signaturesbased on the generic protocol \pprot.
%In order to establish the game for this proof, we need a \real and \ideal protocol package.
%We construct both packages via package composition.
%
To prove the security against forgery for our digital signatures, 
we establish a \emph{game pair} for this security proof. 
%
We construct a \real and \ideal protocol package, as shown in Figure~\ref{fig:sigprot}. 
%
The \real protocol package
\psigprot$_{\text{\real}}$ is a composition of packages
Sig\prot, \psigprim$_{\text{\real}}$ and \pkeygen. 
%\pkeygen, \psigprim$_{\text{\real}}$ and \pprot.
%
The respective \ideal package
\psigprot$_{\text{\ideal}}$ is a composition of packages
Sig\prot, \psigprim$_{\text{\ideal}}$ and \pkeygen.
%pkeygen, \psigprim$_{\text{\ideal}}$ and \pprot.
%
We use the special notation $\circledcirc$ to denote a composition
up to renaming procedures \esign to \icoq{challenge} and
\eversig to \icoq{verify}%
\footnote{
%
In \ssprove , this is not necessary because procedure names map
to a particular natural number.
%
Mapping different procedure names to the same natural numbers
establishes the link without an auxilary package for renaming
procedures.
%
}.
%
The game that we play, and respectively, the theorem that we prove,
establishes perfect indistinguishability between the \real and 
the \ideal protocol package (Figure~\ref{fig:sigprim:packages}).

%
\begin{theorem}[%\icoq{SigProt_}$\mathcal{G}_{\text{\pindist}}^{\Sigma}$: 
  Perfect Indistinguishability of Digital Signatures]\label{theo:indist-signature}
  %
  For every attacker \A, the advantage to distinguish the packages
  \psigprot$_{\text{\real}}$ and \psigprot$_{\text{\ideal}}$ is $0$:
  %
  \begin{center}
    \begin{minipage}{0.5\columnwidth}
    \begin{minted}[fontsize=\footnotesize,escapeinside=&&,autogobble]{coq}
SigProt&$_{\text{\real}}$& &$\pindist$& SigProt&$_{\text{\ideal}}$&.
    \end{minted}
    \end{minipage}
    \end{center}
\end{theorem} 

%
\begin{figure}
\begin{center}
\begin{minipage}{0.9\columnwidth}
\begin{minted}[escapeinside=@@,fontsize=\footnotesize,autogobble,linenos]{Coq}
@$\vdash$@
{ @$\lambda$@ (s@$_1$@, s@$_2$@),
  let inv := heap_ignore @$\{\mathcal{SIG}\}$@ in
  let h   := inv @$\rtimes$@ rm_rhs @$\mathcal{SIG}$@ S in
  let h'  := set_rhs @$\mathcal{SIG}$@ (S @$\cup$@ @$\{$@(@$\Sigma$@.Sign sk m, m)@$\}$@ h in
  let h'' := h' @$\rtimes$@ rm_rhs @$\mathcal{SIG}$@ S' in
  h'' (s@$_1$@, s@$_2$@)) }
ret (pk, @$\Sigma$@.Sign sk m, @$\Sigma$@.VerSig pk (@$\Sigma$@.Sign sk m) m)
@$\approx$@
ret (pk, @$\Sigma$@.Sign sk m, (@$\Sigma$@.Sign sk m, m) @$\in$@ S')
{ @$\lambda$@ (s@$_1$@,r@$_1$@)
    (s@$_2$@,r@$_2$@), r@$_1$@ = r@$_2$@ @$\land$@ heap_ignore @$\{\mathcal{SIG}\}$@ (s@$_1$@, s@$_2$@) }
\end{minted}
\end{minipage}
\end{center}

\caption{
  %
  The final step in the proof of the Theorem \icoq{SigProt_}$\mathcal{G}_{\text{\pindist}}^{\Sigma}$ reduces it to \icoq{sig_correct}.
  %
}
\label{fig:ind:proof}
\end{figure}
%
%
\begin{IEEEproof}
%[Proof of \icoq{SigProt_}$\mathcal{G}_{\text{\pindist}}^{\Sigma}$]
The according proof (game) is a straight forward application of
\ssprove tactics that reduce the goal to the judgement in Figure~\ref{fig:ind:proof}.
%
We use the \icoq{heap_ignore} invariant that is already defined in
\ssprove to state that
\icoq{s}$_1$, the state of the \real package, and
\icoq{s}$_2$, the state of the \ideal package are equal up to
the $\mathcal{SIG}$ location.
%
The proof process extends this invariant in the precondition
with a trace of reads from state into local variables and
updates to state (Lines~4--6).
%
The commands for the \real (left) and \ideal (right) commands
are equal up to the third element of the returned triple.
%
Clearly, the trace contains the evidence (Line~5) that the
signature--message pair really is in \icoq{S'} (Line~10).
%
This allows us to reduce the indistinguishability to the equality
%
\begin{minted}[escapeinside=@@,autogobble]{Coq}
@$\Sigma$@.VerSig pk (@$\Sigma$@.Sign sk m) m = true
\end{minted}
%
which is the functional correctness Hypothesis \icoq{sig_correct}
of the signature scheme $\Sigma$.
%
\end{IEEEproof}



