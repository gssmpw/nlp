\section{A Formal Specification for Digital Signatures}
\label{sec:sig}

%
In this section, we present our formal specification of digital
signatures and verify that it is secure against forgery. 
%
First, we present an overview of our specifications and fix their notations.
%
Afterwards, we specify the package for key generation.
%and state its assumptions for concrete implementations such as RSA. 
%
Then, we followed the specifications for digital signatures. 
%
First, we define the packages and then provide details for proof of 
game-based indistinguishability (perfect).
%

\subsection{Overview and Notations}
%
Our development has two parts: the composition of the packages
and the proof development to establish the security properties.
%

\subsubsection{Package Composition}

\begin{figure*}
	\centering
	\begin{figure*}[t]
\begin{center}
\includegraphics[width=.85\linewidth]{fig_overview_v3.pdf}
\end{center}
\caption{
FastAtlas Overview: In each frame, we compute charts spanning fully or partially visible triangles (a), determine texture space bounding boxes for the visible portions of the view-space projections of each chart, and tightly pack these boxes into atlases (b, here $2K \times 2K$). We simultaneously bijectively parameterize and shade the charts into their atlas boxes, obtaining high quality texture space shading (c), and use this shading to render the shaded frames (d).}
\label{fig:overview}
\label{fig:alg_overview}
\end{figure*}

\section{Overview}
\label{sec:overview}
Our work has two core contributions: a real-time, GPU-based algorithm for tight packing of general parameterized charts into compact atlases; and a real-time TSS method that
utilizes this packing.  

\paragraph*{FastAtlas Packing.}
FastAtlas runs entirely on the GPU as a series of compute shaders. It takes the bounding boxes of parameterized charts as input, and packs them into an atlas (Fig~\ref{fig:overview}b, Sec.~\ref{sec:pack}). As such, the only input it requires are the dimensions of the bounding boxes.
Its outputs are deterministic; identical input charts are packed into identical atlases. This is critical for TSS and similar applications, as it ensures that consecutive frames taken from the same camera view have the same shading. Even minute shading differences across such frames can cause sampling jitter, leading to undesirable flicker \cite{baker2012rock}. 
While prior methods such as \cite{mueller2018shading,hladky2019tessellated,hladky2021snakebinning,Neff2022MSA} cap the dimensions of the charts that can be packed as-is for a given atlas size, and scale down all charts that exceed these dimensions, we scale all charts by the same factor, and do so only when strictly necessary to achieve packing success (Figs~\ref{fig:atlas},~\ref{fig:sas_issues}). 

\paragraph*{TSS using FastAtlas.}
Our end-to-end TSS atlas generation method combines the packing method above with a novel approach for computing seamless per-frame charts. 
We define our charts as the connected components of the visible surfaces in each frame (Fig.~\ref{fig:overview}a), and efficiently compute them using a parallel union-find algorithm (Sec.~\ref{sec:visible}). Since the boundaries of these charts coincide with the contours of the rendered surface, they are {\em invisible} to the viewer. This approach 
eliminates the artifacts caused by shading discontinuities along visible seams (Fig.~\ref{fig:seams}). 

\begin{parWithWrapFigure}
\begin{wrapfigure}{l}{.27\columnwidth}%
\includegraphics[width=\linewidth]{fig_inset_view_plane.pdf}%
\end{wrapfigure}
We bijectively parametrize the {\em visible portions} of our charts by projecting them to view space (inset). This maps a constant number of texels to each pixel in the final rendered output, evenly distributing residual undersampling error across all image pixels. While conceptually straightforward, efficiently parameterizing charts containing partially visible triangles using viewspace projection is non-trivial, as the visible portions may no longer be triangular (e.g. green triangle in the inset); applying naive projection to triangles with vertices behind the camera may produce ill-posed results. Clipping triangles before projection is both computationally expensive and significantly complicates downstream operations. We avoid explicit clipping by observing that all that is required for atlas packing is the dimensions of, potentially conservative, bounding boxes of these projected visible portions. We compute such bounding boxes without explicit chart clipping by adapting a conservative screen coverage estimator \shortcite{Blinn:CalculatingScreenCoverage} (Sec.~\ref{sec:box}). We then pack the computed boxes using FastAtlas. 
\end{parWithWrapFigure}

Finally, we shade the visible portion of each chart into its corresponding atlas bounding box (Fig~\ref{fig:overview}c). 
The resulting texture is then used during rasterization as a standard texture map (Fig. ~\ref{fig:overview}d). 
Our framework is compatible with all existing approaches for texture space shading, including forward shading, raytraced illumination, or deferred shading in texture space \cite{baker:2016}. In the examples shown, we use the standard forward shading based rendering pipeline included in the G3D Innovation Engine \cite{G3D17}, a commercial grade renderer.

	\caption{
  %
  Our formal specification of remote attestation is based on digital signatures.
  %
  The security reasoning is based on the concept of indistinguishability
  for a game pair \real, which represents the actual code,
  and \ideal, which represents the model, i.e., the semantics
  of the real code
  in the context of indistinguishability proofs.}\label{fig:overview}
\end{figure*}


\begin{figure}
  \centering
  \input{tikz/prot-pkg}
  \caption{The protocol package for digitial signatures.}
  \label{fig:pprot}
\end{figure}
%
Figure~\ref{fig:overview} visualizes our specification using SSP-style
visualization for package composition.
%
In the figure, we accordingly highlight \real (shaded in grey) and 
\ideal (shaded in yellow) packages.
%
Our specification builds upon a key generation package \pkeygen that
establishes the probabilistic foundation.
%
The package \pkeygen abstracts over a concrete foundation to allow
for various instantiations such as RSA and ECDSA.
%
A full specification for the instantiation with RSA is in Appendix~\ref{sec:rsa}
with the details in the proof development.
%
The package \pkeygen exports a \ekeygen procedure for the packages that
define the \real and \ideal signature
primitives.
%
%Figure~\ref{fig:problem:eu-ind} shows that the composition of \prot \sa{SigProt?} with 
%packages \real and \ideal for signature
%primitves instantiates the protocol for digitial signatures.
%
%The protocol remains unchanged for remote attestation such that
%we can compose \pprot similarly with the \real and \ideal packages 
%for the remote attestation primitives.
%

\subsubsection{Proof Development}
%
Our proof development shows that our structured approach proves 
the security of signatures by establishing an equivalence between 
sEUF-CMA and indistinguishability-based security. 
%
The main idea is that if a signature scheme is secure with strong unforgeability, 
then the probability of forging a valid signature for the previously signed message is negligible. 
%
The signature verification in $\mathcal{G}_{\real}^{\Sigma}$ returns true with a valid message-signature pair. 
%
By the definition of strong unforgeability assumption, the $\mathcal{G}_{\ideal}^{\Sigma}$ must also return true, 
as any valid signature must have been generated by singing primitive. 
%
This ensures that the real and ideal signature verification processes remain indistinguishable. 
%
Contrariwise, if the real and ideal signature packages are 
indistinguishable, it implies that the ideal one must also 
accept any valid message-signature pair accepted by the real package. 
%
If an adversary could forge a valid signature, the real package would return \emph{true}, 
while the ideal package would return \emph{false}. 
%
By ensuring that the adversary can not distinguish these two cases, 
we naturally guarantee that forging a new signature is infeasible. 
%
This reasoning extends to both fresh and previously signed messages, which leads 
to the judgment that indistinguishability implies sEUF-CMA. 
%

Our proof construction establishes indistinguishability, i.e.,
strong existential unforgeability, for digital signatures based on
the protocol packages (Theorem~\ref{theo:indist-signature}). 
%
Theorem~\ref{theo:redprot} is a reduction theorem and states that
the security of remote attestation is smaller or equal to the
security of digital signatures.
%
Theorem~\ref{theo:indist-ra} then uses Theorem~\ref{theo:indist-signature}
and Theorem~\ref{theo:redprot} to verify strong existential
unforgeability of remote attestation.
%


\subsection{Key Generation}

\begin{figure}
	\centering
    \begin{subfigure}[b]{\columnwidth}
    \centering
       \begin{minted}[fontsize=\footnotesize,escapeinside=@@,autogobble]{coq}
Parameters SecKey PubKey : finType.
Parameter  @$\Sigma$.@KeyGen : @$\forall$@ s, code s (PubKey × SecKey).
		  	\end{minted}
      \caption{Parameters.}
	    \label{fig:keygen:params}
    \end{subfigure}
\\[0.3cm]
  \begin{subfigure}[b]{\columnwidth}
    \centering
    \begin{tikzpicture}[
	font=\footnotesize,
	node distance=0.3cm
	]
	\node[] (P) at (0,0)  {
		\pkeygen
	};
	\node[] (c) [below = 0.5cm of P] {
		\begin{minipage}{0.5\columnwidth}
			\begin{minted}[fontsize=\footnotesize,escapeinside=&&,autogobble]{coq}
def key_gen ():
				  
  (sk,pk) <- &$\Sigma$&.KeyGen tt ;;
  #put @&$\mathcal{SK}$& sk ;;
  #put @&$\mathcal{PK}$& pk ;;

  #ret (sk,pk)
			\end{minted}
		\end{minipage}
	};
	\node[] (s) [above right = 0cm and 0cm of c.north west] {
    \begin{minipage}{0.65\columnwidth}
		\begin{minted}[fontsize=\footnotesize,escapeinside=&&,autogobble]{coq}
      &$\mathcal{SK}$& : SecKey, &$\mathcal{PK}$& : PubKey
    \end{minted}
    \end{minipage}
	};

%%%%%%%% exports %%%%%%%
    \draw[->] 
    ([xshift=0.2cm,yshift=-0cm]c.east) -- 
    ([xshift=1.8cm,yshift=-0cm]c.east)
        node[midway,above]{ \mintinline{Coq}{key_gen} };


	% Package frame and stuff
	\draw[] (c.north west) -- (c.north east);
	\draw[] (c.north west) |- (s.north) -| (c.north east);
	\draw[] (s.north west) |- (P.north) -| (c.north east);
	\draw[] (c.north east) |- (c.south) -| (c.north west);

\end{tikzpicture}

    \caption{Package.}
    \label{fig:keygen:package}
  \end{subfigure}
  \caption{
    %
    The \pkeygen package and its parameters.
    %
  }
  \label{fig:keygen}
\end{figure}

%
Following the (textbook) definitions from Section~\ref{sec:TheoryFound},
we define the package \pkeygen in Figure~\ref{fig:keygen}.
%
We leave the definition of the secret key (\lseckey)
and the public key (\lpubkey) abstract as parameters
and just require them to be of a finite type.
%
Similarly, we parameterize the package with an algorithm
$\Sigma$.\pkeygen that implements the final key generation.
%
We do not import this algorithm because otherwise, we cannot play
the security game: game packages cannot have imports.
%
The algorithm $\Sigma$.\pkeygen nevertheless emits monadic code
because it needs to sample the keys from a distribution.
%
This differs from the textbook definition, which states that the $\Sigma$.KeyGen
is a pure function.
%
The $\Sigma$.\pkeygen is polymorph in the state \icoq{s}, i.e.,
it has no side-effects to the state other than for sampling.
%
The \egetpk procedure stores the generated secret and public
keys into its state and returns them both.
%
We are now ready to specify digital signatures.
%
We start with defining the primitives, and
afterwards, we compose the protocol games to prove perfect indistinguishability, 
i.e., strong existential unforgeability for digital signatures.
%

\subsection{Primitives}

\begin{figure}
	\centering
   \begin{subfigure}[b]{\columnwidth}
    \centering
        \begin{minted}[fontsize=\footnotesize,escapeinside=@@,autogobble]{coq}
Parameters Message Signature : finType.
Parameter  @$\Sigma$@.Sign : SecKey -> Message -> Signature.
Parameter  @$\Sigma$@.VerSig :
  PubKey -> Signature -> Message -> bool.
	\end{minted}
    \caption{Parameters}
    \label{fig:sigprim:params}
  \end{subfigure}
\\[0.3cm]
   \begin{subfigure}[b]{\columnwidth}
    \centering
        \begin{minted}[fontsize=\footnotesize,escapeinside=@@,autogobble]{coq}
Hypothesis sig_correct :
  @$\forall$@ m sk pk,
    ((sk,pk) <- @$\Sigma$@.KeyGen) ->
    @$\Sigma$@.VerSig pk (@$\Sigma$@.Sign sk m) m == true.
	\end{minted}
    \caption{Functional Correctness}
    \label{fig:sigprim:hypo}
  \end{subfigure}
\\[0.3cm]
\begin{subfigure}[b]{\columnwidth}
    \centering
    \begin{tikzpicture}[
	font=\footnotesize,
	node distance=0.3cm
	]
	\node[] (real) at (0,0)  {
		\psigprim$_{\text{\real}}$
	};
	\node[] (code-r) [below = 0.75cm of P] {
		\begin{minipage}{0.45\columnwidth}
			\begin{minted}[fontsize=\footnotesize,escapeinside=&&,autogobble]{coq}
get_pk() :=
  #import key_gen ;;

  (_,pk) <- key_gen tt ;;
  #ret pk

sign(m) :=
  sk <- #get @&$\mathcal{SK}$& ;;
  let &$\sigma$& := &$\Sigma$&.Sign sk m ;;



  #ret &$\sigma$&

ver_sig(m,s) :=
  pk <- #get @&$\mathcal{PK}$& ;;
  #ret &$\Sigma$&.VerSig pk s m
			\end{minted}
		\end{minipage}
	};
	\node[] (state-r) [above right = 0cm and 0cm of code-r.north west] {
	\begin{minipage}{0.45\columnwidth}
		\begin{minted}[fontsize=\footnotesize,escapeinside=&&,autogobble]{coq}
&$\mathcal{SK}$& : SecKey, &$\mathcal{PK}$& : PubKey
&\phantom{$\mathcal{SIG}$}&
		\end{minted}
	\end{minipage}
	};

	% Package frame and stuff
  \draw[] (code-r.north west) -- (code-r.north east);
	\draw[] (code-r.north west) |- (state-r.north) -| (code-r.north east);
	\draw[] (state-r.north west) |- (real.north) -| (code-r.north east);
	\draw[] (code-r.north east) |- (code-r.south) -| (code-r.north west);

    %%%%%%%% Ideal %%%%%%%%%%%%
    %%%%%%%%%%%%%%%%%%%%%%%%%%%

    \node[] (ideal) [right = 2.5 cm of real]  {
      \psigprim$_{\text{\ideal}}$
    };
    \node[] (code-i) [below = 0.75cm of ideal] {
    	\begin{minipage}{0.45\columnwidth}
    		\begin{minted}[fontsize=\footnotesize,escapeinside=&&,autogobble,highlightlines={10,11,12,16,17},highlightcolor=lYellow!20]{coq}
get_pk() :=
  #import key_gen ;;

  (_,pk) <- key_gen tt ;;
  #ret pk

sign(m) :=
  sk <- #get @&$\mathcal{SK}$& ;;
  let s := &$\Sigma$&.Sign sk m ;;
  S <- #get @&$\mathcal{SIG}$& ;;
  let S' := S &$\cup$& {(m,s)} ;;
  #put @&$\mathcal{SIG}$& S' ;;
  #ret s

ver_sig(m, s) :=
  S <- #get @&$\mathcal{SIG}$& ;;
  #ret ( (m,s) &$\in$& S )
    		\end{minted}
    	\end{minipage}
    };
    \node[] (state-i) [above right = 0cm and 0cm of code-i.north west] {
    	\begin{minipage}{0.45\columnwidth}
    		\begin{minted}[fontsize=\footnotesize,escapeinside=&&,autogobble,highlightlines={2},highlightcolor=lYellow!20]{coq}
&$\mathcal{SK}$&: SecKey, &$\mathcal{PK}$&: PubKey,
&$\mathcal{SIG}$&: (Message x Signature)
    		\end{minted}
    	\end{minipage}
    };

    % Package frame and stuff
    \draw[] (code-i.north west) -- (code-i.north east);
    \draw[] (code-i.north west) |- (state-i.north) -| (code-i.north east);
    \draw[] (state-i.north west) |- (ideal.north) -| (code-i.north east);
    \draw[] (code-i.north east) |- (code-i.south) -| (code-i.north west);

\end{tikzpicture}

    \caption{Packages}
    \label{fig:sigprim:packages}
  \end{subfigure}
	\caption{
    %
    The \real and \ideal Signature Primitives packages
    and their parameters and hyptohesis.
    %
    }
	\label{fig:sigprim}
\end{figure}

%
Figure~\ref{fig:sigprim} defines the abstractions and then
specifies the primitives for our digital signatures.
%

%
In line with the (textbook) definition presented in
Figure~\ref{fig:problem:ind}, our implementation of
the signature primitives abstracts over a concrete
signature scheme.
%
In Figure~\ref{fig:sigprim:params}, we abstract over concrete
implementations for messages and signatures.
%
The abstractions for signing a message and verifying a signature
complete Definition~\ref{def:sig:scheme} of a signature scheme $\Sigma$.
%
Note that both $\Sigma$\icoq{.Sign} and $\Sigma$.\icoq{VerSig}
are pure functions because they neither sample nor access any state.
%
Functional correctness (Definition~\ref{def:sig:correct}) is then
a hypothesis of our primitives for digital signatures (Figure~\ref{fig:sigprim:hypo}).
%

%
We specify the \real and \ideal packages for the primitives
of our digital signatures in Figure~\ref{fig:sigprim:packages}.
%
Our specification pays attention to accessing and updating
the state, e.g., for the set \icoq{S} of observed signatures in the
\ideal package.
%
Both packages import \ekeygen to obtain the
public key \icoq{pk}.
%
Note again that \ekeygen stores the generated keys \icoq{sk}
and \icoq{pk} into its state.
%
Respectively, the state of \pkeygen needs to be a
subset of the state of each package such that \esign and
\eversig can access the keys.
%
Apart from these implementation details,
both package definitions follow the (textbook)
specification from Figure~\ref{fig:problem:ind}.
%


\subsection{Indistinguishability}

\begin{figure}
    \begin{minted}[fontsize=\footnotesize,escapeinside=&&,autogobble]{coq}
Definition SigProt&$_{\text{\real}}$& :=
  {package SigProt &$\circledcirc$& SigPrim&$_{\text{\real}}$& &$\circ$& KeyGen}.

Definition SigProt&$_{\text{\ideal}}$& :=
  {package SigProt &$\circledcirc$& SigPrim&$_{\text{\ideal}}$& &$\circ$& KeyGen}.
    \end{minted}
   \caption{Protocol games for digital signatures.}
   \label{fig:sigprot}
\end{figure}

%We prove security against forgery for our digital signaturesbased on the generic protocol \pprot.
%In order to establish the game for this proof, we need a \real and \ideal protocol package.
%We construct both packages via package composition.
%
To prove the security against forgery for our digital signatures, 
we establish a \emph{game pair} for this security proof. 
%
We construct a \real and \ideal protocol package, as shown in Figure~\ref{fig:sigprot}. 
%
The \real protocol package
\psigprot$_{\text{\real}}$ is a composition of packages
Sig\prot, \psigprim$_{\text{\real}}$ and \pkeygen. 
%\pkeygen, \psigprim$_{\text{\real}}$ and \pprot.
%
The respective \ideal package
\psigprot$_{\text{\ideal}}$ is a composition of packages
Sig\prot, \psigprim$_{\text{\ideal}}$ and \pkeygen.
%pkeygen, \psigprim$_{\text{\ideal}}$ and \pprot.
%
We use the special notation $\circledcirc$ to denote a composition
up to renaming procedures \esign to \icoq{challenge} and
\eversig to \icoq{verify}%
\footnote{
%
In \ssprove , this is not necessary because procedure names map
to a particular natural number.
%
Mapping different procedure names to the same natural numbers
establishes the link without an auxilary package for renaming
procedures.
%
}.
%
The game that we play, and respectively, the theorem that we prove,
establishes perfect indistinguishability between the \real and 
the \ideal protocol package (Figure~\ref{fig:sigprim:packages}).

%
\begin{theorem}[%\icoq{SigProt_}$\mathcal{G}_{\text{\pindist}}^{\Sigma}$: 
  Perfect Indistinguishability of Digital Signatures]\label{theo:indist-signature}
  %
  For every attacker \A, the advantage to distinguish the packages
  \psigprot$_{\text{\real}}$ and \psigprot$_{\text{\ideal}}$ is $0$:
  %
  \begin{center}
    \begin{minipage}{0.5\columnwidth}
    \begin{minted}[fontsize=\footnotesize,escapeinside=&&,autogobble]{coq}
SigProt&$_{\text{\real}}$& &$\pindist$& SigProt&$_{\text{\ideal}}$&.
    \end{minted}
    \end{minipage}
    \end{center}
\end{theorem} 

%
\begin{figure}
\begin{center}
\begin{minipage}{0.9\columnwidth}
\begin{minted}[escapeinside=@@,fontsize=\footnotesize,autogobble,linenos]{Coq}
@$\vdash$@
{ @$\lambda$@ (s@$_1$@, s@$_2$@),
  let inv := heap_ignore @$\{\mathcal{SIG}\}$@ in
  let h   := inv @$\rtimes$@ rm_rhs @$\mathcal{SIG}$@ S in
  let h'  := set_rhs @$\mathcal{SIG}$@ (S @$\cup$@ @$\{$@(@$\Sigma$@.Sign sk m, m)@$\}$@ h in
  let h'' := h' @$\rtimes$@ rm_rhs @$\mathcal{SIG}$@ S' in
  h'' (s@$_1$@, s@$_2$@)) }
ret (pk, @$\Sigma$@.Sign sk m, @$\Sigma$@.VerSig pk (@$\Sigma$@.Sign sk m) m)
@$\approx$@
ret (pk, @$\Sigma$@.Sign sk m, (@$\Sigma$@.Sign sk m, m) @$\in$@ S')
{ @$\lambda$@ (s@$_1$@,r@$_1$@)
    (s@$_2$@,r@$_2$@), r@$_1$@ = r@$_2$@ @$\land$@ heap_ignore @$\{\mathcal{SIG}\}$@ (s@$_1$@, s@$_2$@) }
\end{minted}
\end{minipage}
\end{center}

\caption{
  %
  The final step in the proof of the Theorem \icoq{SigProt_}$\mathcal{G}_{\text{\pindist}}^{\Sigma}$ reduces it to \icoq{sig_correct}.
  %
}
\label{fig:ind:proof}
\end{figure}
%
%
\begin{IEEEproof}
%[Proof of \icoq{SigProt_}$\mathcal{G}_{\text{\pindist}}^{\Sigma}$]
The according proof (game) is a straight forward application of
\ssprove tactics that reduce the goal to the judgement in Figure~\ref{fig:ind:proof}.
%
We use the \icoq{heap_ignore} invariant that is already defined in
\ssprove to state that
\icoq{s}$_1$, the state of the \real package, and
\icoq{s}$_2$, the state of the \ideal package are equal up to
the $\mathcal{SIG}$ location.
%
The proof process extends this invariant in the precondition
with a trace of reads from state into local variables and
updates to state (Lines~4--6).
%
The commands for the \real (left) and \ideal (right) commands
are equal up to the third element of the returned triple.
%
Clearly, the trace contains the evidence (Line~5) that the
signature--message pair really is in \icoq{S'} (Line~10).
%
This allows us to reduce the indistinguishability to the equality
%
\begin{minted}[escapeinside=@@,autogobble]{Coq}
@$\Sigma$@.VerSig pk (@$\Sigma$@.Sign sk m) m = true
\end{minted}
%
which is the functional correctness Hypothesis \icoq{sig_correct}
of the signature scheme $\Sigma$.
%
\end{IEEEproof}



