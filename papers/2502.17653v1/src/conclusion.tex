\section{Conclusion and future work}
\label{sec:concl}
%
This paper formally verifies the semantic security of the digital signature scheme 
and the remote attestation using the \ssprove library in the \coq.
%
To the best of our knowledge, we are the first to do so in the context of game-based proofs that rely on
the security notion of perfect indistinguishability. 
%
We demonstrated that the perfect indistinguishability definition of secure signatures 
can be applied to prove the semantic security of both signature primitives and remote atttestation.
%
Our results highlight that while the signature properties can be fully captured in a protocol setting, 
achieving the same level of formal guarantees for individual primitives remains challenging without 
access to specific protocol-level information. 
%
Our findings indicate that \ssprove is well-suited for cryptographic security models. 
%
However, the tool may require a few further extensions to capture the nuances of cryptographic primitives fully. 
%
We believe that the insights gained from this work opens avenues for future research 
to fully machine-checked the complex systems such as remote attestation. 
%
\textit{Future work}:
For our future work, we plan to explore integrating HAX toolchain\footnote{\url{https://cryspen.com/hax/}}
to transcribe Rust into \coq
into our formal verification framework to complement SSProves's game-based security proofs. 
%
HAX's algebraic reasoning and stateful computations deliver a promising avenue to model and verify 
complex state transitions in cryptographic security protocols, such as TPM-based remote attestation.
%
This unified combined framework will precisely address both structural correctness and security 
properties in complex cryptographic protocols, improving the applicability of our formal development.  
%
\subsection{Authors Role}
\textit{Sara Zain}: Formalization, Methodology, Writing-review and editing.
%
\textit{Jannik M\"ahn}: Conceptualization, Methodology, Formalization, Writing-original darft.
%
\textit{Stefan K\"opsell}: Supervision, Funding acquisition.
%
\textit{Sebastian Ertel}: Conceptualization, Methodology, Formalization, Writing-review and editing.

