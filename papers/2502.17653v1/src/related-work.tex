\section{Related Work}
\label{sec:rel}
%
Remote attestation has emerged as a critical security mechanism for verifying the integrity of remote systems.
%
Early work by Cabodi et al.~\cite{cabodi2015formal, cabodi2016secure} laid the foundation for formalizing hybrid \ra properties, 
analyzing and comparing \ra architectures using different model checkers. 
%
VRASED~\cite{vrased} extended this by introducing 
a verified hardware-software co-design for \ra, targeting low-end embedded systems 
and addressing security limitations of hybrid architectures~\cite{literature, minimalist}.
%
Hybrid RA mechanisms, such as HYDRA~\cite{hydra}, further enhanced security by 
integrating formally verified microkernels, achieving memory isolation while reducing hardware complexity.
%

Formal verification has also been applied to the industry.
%
For instance, the~\cite{sardar2021demystifying} presented the 
formal specification of one of Intel TDX's security-critical processes. 
%
They ensured secure operations in trusted execution environments against a Dolev-Yao adversary using ProVerif. 
%
%RATA~\cite{de2021toctou} further optimized hybrid \ra architectures for embedded devices, 
%demonstrating minimal hardware overhead and improved efficiency. 
%
Cryptographic advancements have also contributed to \ra development. 
%
\ra relies on digital signatures schemes to ensure that responses to challenges are authentic and cannot be forged. 
%
EasyCrypt has been employed to verify 
existential and strong unforgeability ~\cite{dupressoir2021machine, cortier2018machine}. 
%
Tamarin-based verification 
of Direct Anonymous Attestation in TPM 2.0~\cite{wesemeyer2020formal}, reinforcing the security of \ra protocols.
%
Expanding RA to post-quantum security remains an open research direction~\cite{pu2023post}.

