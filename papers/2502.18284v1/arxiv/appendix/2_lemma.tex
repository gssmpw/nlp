\section{Further Background and Auxiliary Lemmas}\label{appendix:aux_lemmas}
All the results in this section are existing results in the literature. We provide them here and prove some of them in the specific context of Sobolev spaces explicitly for the convenience of the reader.

\paragraph{More technical notions of Sobolev spaces and the Sobolev embedding theorem}
In the main text, we provide in \eqref{eq:defi_sobolev} the standard definition of Sobolev spaces $W_2^s( \calX )$ when $s \in \N$.
Actually, Sobolev spaces $W_2^s( \calX )$
can be extended to $s$ that are positive real numbers. 
Such extension could be realized through 
real interpolation spaces (see \citep[Definition 1.7]{bennett1988interpolation}),
$W_2^{s}(\calX) := [W_2^{k}(\calX), W_2^{k+1}(\calX)]_{r, 2}$ where $k \in \mathbb{N}, s \in (k, k+1), r = s-\lfloor s\rfloor$.\footnote{Strictly speaking, the definition of \eqref{eq:defi_sobolev} extended to real numbers $s$ actually corresponds to the complex interpolation space of Sobolev spaces. Fortunately, complex interpolation spaces and real interpolation spaces coincide under Hilbert spaces~\citep[Corollary C.4.2]{hytonen2016analysis}, which is precisely our setting since $p=2$.}
Actually, such interpolation relations hold for any $0 \leq s , t$ and $0 < r < 1$~\citep[Section 7.32]{adams2003sobolev},
\begin{align}\label{eq:interpolation}
    W_{2}^k(\calX) = \left[W_2^s(\calX), W_{2}^t(\calX)\right]_{r, 2}, \quad k = (1-r) s + r t .
\end{align}
A special case of the above relation is $ W_{2}^s(\calX) = \left[L_2(\calX), W_{2}^t(\calX)\right]_{s/t, 2} $. 

The Sobolev embedding theorem~\citep{adams2003sobolev}, when applied to $W_2^s(\calX)$, states that if $s>\frac{d}{2}$ (where $d$ is the dimension of $\calX)$, then $W_2^s(\calX)$ can be continuously embedded into $C^0(\calX)$, the space of continuous and bounded functions. In other words, for every equivalence class $[f] \in W_2^s(\calX)$, there exists a unique continuous and bounded representative $f \in C^0(\calX)$, and the embedding map $I$ : $W_2^s(\calX) \rightarrow C^0(\calX)$, defined by $I([f])=f$, is continuous.
This continuous embedding $I$ can be written as $W_2^s(\calX) \hookrightarrow C^0(\calX)$. 
Since every continuous linear operator is bounded, we have $\| W_2^s(\calX) \hookrightarrow C^0(\calX)\|$ bounded by a constant that only depends on $s, \calX$.

\paragraph{More technical notions of reproducing kernel Hilbert spaces (RKHSs)}
For bounded kernels, $\sup_{x\in\calX} k(x, x) \leq \kappa$, its associated RKHS $\mathcal H$ can be canonically injected into $L_2(\pi)$ using the operator $\iota_{\pi} : \calH \to L_2(\pi),\,f\mapsto f$ with its adjoint $ \iota_\pi^\ast: L_2(\pi) \rightarrow \calH$ given by $\iota_\pi^\ast f(\cdot) = \int k(x,\cdot)f(x)d\pi(x)$.
$\iota_{\pi}$ and its adjoint can be composed to form a $L_2(\pi)$ endomorphism 
$\calT_{\pi} \coloneqq \iota_{ \pi} \iota_{ \pi}^\ast$ 
called the \emph{integral operator}, and a $\mathcal H$ endomorphism  
\begin{align}\label{eq:covariance_operator}
    \Sigma_{\pi} \coloneq \iota_{ \pi}^\ast \iota_{ \pi}=\int k(\cdot, x) \otimes k(\cdot, x) d \pi(x),
\end{align}
(where $\otimes$ denotes the tensor product such that $(a\otimes b)c \coloneqq \langle b,c \rangle_{\calH} a$ for $a,b,c\in \calH$) called the \emph{covariance operator}.
Both $\Sigma_\pi$ and $\calT_{\pi}$ are compact, positive, self-adjoint, and they have the same eigenvalues $\varrho_1 \geq \cdots \varrho_i \geq \cdots \geq 0$. Please refer to Section 2 of \citet{chen2024regularized} for more details. 

\begin{lem}[Effective dimension $\calN(\lambda)$]\label{lem:dof}
Let $\calX \subset \R^d$ be a compact domain, $\pi$ be a probability measure on $\calX$ with density $p:\calX \to \R$. $k : \calX \times \calX \to \R$ is a Sobolev reproducing kernel of order $s > \frac{d}{2}$. 
$\{\varrho_m \}_{m \geq 0}$ are the eigenvalues of the integral operator $\calT_\pi$.
Define the effective dimension $\mathcal{N}:(0, \infty) \rightarrow[0, \infty)$ as $\mathcal{N}(\lambda) \coloneq \sum_{m \geq 1} \frac{\varrho_m}{\varrho_m+\lambda}$. 
If $p(x) \geq G > 0$ for any $x \in \calX$, then $ \mathcal{N}(\lambda) \leq D \lambda^{- \frac{d}{2s} }$ with constant $D$ that only depends on $G$ and $\calX$.
\end{lem}
\begin{proof}
First, we study the asymptotic behavior of the eigenvalues $\left(\varrho_m \right)_{m \geq 1}$ of the integral operator $\calT_\pi$. Theorem 15 of \cite{steinwart2009optimal} shows that the eigenvalues $\varrho_m$ share the same asymptotic decay rate as the squares of the entropy number $e_m^2\left( I_\pi \right)$ of the embedding $I_\pi: W_2^s(\calX) \rightarrow L_2(\pi)$. 
Denote $\calL_\calX$ as the Lebesgue measure on $\calX$.
Since $p(x) \geq G$ for any $x \in \calX$, we know $\frac{d \calL_\calX}{d \pi} \leq G^{-1} \text{Vol}(\calX)^{-1}$ so $\| L_2(\pi) \hookrightarrow L_2(\calX) \| \leq G^{-1} \text{Vol}(\calX)^{-1}$, and consequently we have from Equation (A.38) of \citet{steinwart2008support} that
\begin{align*}
    e_m \left( I_\pi \right) \leq e_m \left( I_{\calL_\calX} \right) \| L_2(\pi) \hookrightarrow L_2(\calX) \| \leq G^{-1} \text{Vol}(\calX)^{-1} e_m \left( I_{\calL_\calX} \right) .
\end{align*}
Moreover, \citep[Equation 4 on p. 119]{edmunds1996function} shows that the entropy number $e_m\left( I_{\calL_\calX} \right) \leq \tilde{c} m^{-s / d}$ for some constant $\tilde{c}$, so we have $e_m \left( I_\pi \right) \leq G^{-1} \text{Vol}(\calX)^{-1} \tilde{c} m^{-s / d}$ and consequently we have $\varrho_m \asymp e_m^2\left( I_\pi \right) \leq G^{-2} \text{Vol}(\calX)^{-2} \tilde{c}^2 m^{-2s / d} =: c_2 m^{-2s / d}$.

Next, we have  
\begin{align*}
    \sum_{m \geq 1} \frac{\varrho_m}{\varrho_m+\lambda} &\leq \sum_{m \geq 1} \frac{1}{1+\lambda c_2^{-1} m^{2 s / d} } \leq \int_0^{\infty} \frac{c_2}{ c_2 + \lambda  t^{2 s / d} } dt = \lambda^{- \frac{d}{2s} } \int_0^{\infty} \frac{c_2}{ c_2 + \tau^{2 s / d} } d \tau \\
    &= \lambda^{- \frac{d}{2s} } \int_0^{\infty} \frac{1 }{1 + \left( \tau c_2^{-\frac{d}{2s}} \right)^{\frac{2s}{d}} } d \tau = \lambda^{- \frac{d}{2s} } \int_0^{\infty} \frac{1}{1 + u^{\frac{2s}{d}} } {c_2}^{\frac{d}{2s}} d u
    = \lambda^{- \frac{d}{2s} } {c_2}^{\frac{d}{2s}} 
    \frac{ \frac{\pi d}{2s} }{\sin \left( \frac{\pi d}{2s} \right)} \\
    &=: D \lambda^{- \frac{d}{2s} },
\end{align*}
where $D$ is a constant that depends on the domain $\calX$ and $G$.
\end{proof}

\begin{lem}\label{lem:embedding}
Let $\calX \subset \R^d$ be a compact domain, $\pi$ be a probability measure on $\calX$ with density $p:\calX \to \R$.
$k : \calX \times \calX \to \R$ is a Sobolev reproducing kernel of order $s > \frac{d}{2}$. 
$\{ \varrho_m, e_m \}_{m \geq 0}$ are the eigenvalues and eigenfunctions of the integral operator $\calT_\pi$.
If there exists $G_0,G_1 > 0$ such that $G_0 \leq p(x) \leq G_1$ for any $x \in \calX$, then 
\begin{align}\label{eq:k_alpha}
    k_{\alpha} \coloneq \sup_{x\in\calX} \sum_{m \geq 1} \varrho_m^{\alpha } e_m^2(x) \leq M ,
\end{align}
holds for any $ \frac{d}{2s} < \alpha$.
Here, $M$ is a constant that depends on $\calX$ and $G_1, G_0$.
\end{lem}
\begin{proof}
If $t > \frac{d}{2}$, $W_2^t(\calX)$ can be continuously embedded into $L_\infty(\calX)$ the space of bounded functions \citep[Case A, Theorem 4.12]{adams2003sobolev}.
Hence, the operator $W_2^s(\calX) \hookrightarrow L_\infty(\calX) $ is a continuous linear operator between two normed vector spaces, hence a bounded operator. 
And $L_2(\pi) $ is norm equivalent to $ L_2(\calX)$ because $G_0 \leq p(x) \leq G_1$ for any $x \in \calX$. 
Notice that $k_\alpha$ defined here is exactly $\left\|k_\nu^\alpha\right\|_{\infty}$ defined in Equation 16 of \cite{fischer2020sobolev}, so we know from Theorem 9 of \citet{fischer2020sobolev} that 
\begin{align*}
     \sup_{x\in\calX} \sum_{m \geq 1} \varrho_m^{\alpha} e_i^2(x) = \left\|  \left[L_2(\pi), W_{2}^s(\calX) \right]_{\alpha, 2} \hookrightarrow L_{\infty}(\calX) \right\| .
\end{align*}
Notice that $\left[L_2(\pi), W_{2}^s(\calX) \right]_{\alpha, 2} \cong \left[L_2(\calX), W_{2}^s(\calX) \right]_{\alpha, 2} \cong W_{2}^{s \alpha}(\calX)$, and notice the fact that $W_2^{s\alpha}(\calX) \hookrightarrow L_\infty(\calX)$ for any $s\alpha > \frac{d}{2}$, the right hand side of the above equation is bounded. Therefore, we have \eqref{eq:k_alpha} holds for any $\frac{d}{2s} < \alpha$.
\end{proof}

\begin{lem}\label{lem:sobolev_algebra}
    Let $\calX \subset \R^d$ be a bounded domain with Lipschitz continuous boundary and $W_2^s(\calX)$ be a Sobolev space with $s > \frac{d}{2}$. If functions $f:\calX \to \R$ and $g:\calX \to \R$ lie in $W_2^s(\calX)$, then their product $f \cdot g$ also lies in $W_2^s(\calX)$ and satisfies $\| f \cdot g \|_s \leq \| f \|_s \| g \|_s$.
\end{lem}
\begin{proof}
This is Theorem 7.4 of \citet{behzadan2021multiplication} with $s_1=s_2=s$ and $p_1=p_2=2$. 
\end{proof}
\begin{lem}\label{lem:prob_to_expectation}
For a positive valued random variable $R$, and $c > 0$ such that $\Pb(R \leq c \tau) \geq 1 - \exp(-\tau)$ for any positive $\tau$, it holds that $\E[R^m] \leq c_o m! $ for all integers $m \geq 1$. $c_o$ is some constant that only depends on $c, m$.
\end{lem}
\begin{proof}
Notice that $R$ is essentially a sub-exponential random variable.
Since a sub-exponential random variable is equivalent to the square root of a sub-Gaussian random variable, from Proposition 2.5.2 of \citet{vershynin2018high}, we have $\E[R^m] = \E[\sqrt R ^{2m}] \leq 2 c_o \Gamma(m+1) = 2 c_o m!$.
Here $\Gamma$ denotes the gamma function and $c_o$ is some constant that only depends on $c, m$.
\end{proof}

\begin{lem}\label{lem:integral_in_hilbert}
    For a mapping $F$ from a compact domain $\calX \subset \R^d$ to a Hilbert space $H$, given a measure $\mu$ on $\calX$, if $F$ is $\mu$-Bochner integrable, then $\int F(x) d\mu(x) \in H$ and additionally $ \| \int F(x) d\mu(x)\|_H \leq \int \| F(x)\|_H d\mu(x)$.
\end{lem}
\begin{proof}
    This is Definition A.5.20 of \citet{steinwart2008support}.
\end{proof}
