% This must be in the first 5 lines to tell arXiv to use pdfLaTeX, which is strongly recommended.
\pdfoutput=1
% In particular, the hyperref package requires pdfLaTeX in order to break URLs across lines.

\documentclass[11pt]{article}

% Change "review" to "final" to generate the final (sometimes called camera-ready) version.
% Change to "preprint" to generate a non-anonymous version with page numbers.
\usepackage[preprint]{acl}

% Standard package includes
\usepackage{times}
\usepackage{latexsym}

% For proper rendering and hyphenation of words containing Latin characters (including in bib files)
\usepackage[T1]{fontenc}
% For Vietnamese characters
% \usepackage[T5]{fontenc}
% See https://www.latex-project.org/help/documentation/encguide.pdf for other character sets

% This assumes your files are encoded as UTF8
\usepackage[utf8]{inputenc}

% This is not strictly necessary, and may be commented out,
% but it will improve the layout of the manuscript,
% and will typically save some space.
\usepackage{microtype}

% This is also not strictly necessary, and may be commented out.
% However, it will improve the aesthetics of text in
% the typewriter font.
\usepackage{inconsolata}

%Including images in your LaTeX document requires adding
%additional package(s)
\usepackage{graphicx}

% Recommended, but optional, packages for figures and better typesetting:
\usepackage{microtype}
\usepackage{graphicx}
\usepackage{subfigure}
\usepackage{booktabs} % for professional tables
\usepackage{array,multirow,graphicx}
\usepackage{amsmath}
\usepackage{amsfonts}
\usepackage{amssymb}
\usepackage{listings}
\usepackage{comment}
\usepackage{tcolorbox}
\tcbuselibrary{most}
\usepackage{float}

\usepackage{enumitem}
\setlist{nolistsep}


% If the title and author information does not fit in the area allocated, uncomment the following
%
\setlength\titlebox{1.25in}
%
% and set <dim> to something 5cm or larger.

% Custom counter for RQs and experiments.
\newcounter{experimentcounter}  % Sub-counter, reset when section counter increases
\renewcommand{\theexperimentcounter}{\Roman{experimentcounter}}  % Format for sub-number (1.1, 1.2, ...)

\newcommand{\myexperimentlabel}[1]{\refstepcounter{experimentcounter}Experiment \theexperimentcounter\label{#1}}    % Print and label the current number

\newtcolorbox{longfbox}[2][]{colframe=black, colback=white, sharp corners, 
  enhanced, title=#2, fonttitle=\bfseries, #1}

\title{Understanding Silent Data Corruption in LLM Training}

% Author information can be set in various styles:
% For several authors from the same institution:
% \author{Author 1 \and ... \and Author n \\
%         Address line \\ ... \\ Address line}
% if the names do not fit well on one line use
%         Author 1 \\ {\bf Author 2} \\ ... \\ {\bf Author n} \\
% For authors from different institutions:
% \author{Author 1 \\ Address line \\  ... \\ Address line
%         \And  ... \And
%         Author n \\ Address line \\ ... \\ Address line}
% To start a separate ``row'' of authors use \AND, as in
% \author{Author 1 \\ Address line \\  ... \\ Address line
%         \AND
%         Author 2 \\ Address line \\ ... \\ Address line \And
%         Author 3 \\ Address line \\ ... \\ Address line}

% \author{Jeffrey M \\
%   Affiliation / Address line 1 \\
%   Affiliation / Address line 2 \\
%   Affiliation / Address line 3 \\
%   \texttt{email@domain} \\\And
%   Second Author \\
%   Affiliation / Address line 1 \\
%   Affiliation / Address line 2 \\
%   Affiliation / Address line 3 \\
%   \texttt{email@domain} \\}


\author{%
  Jeffrey Ma$^{1}$
  \quad Hengzhi Pei$^2$
  \quad Leonard Lausen$^2$
  \quad \textbf{George Karypis}$^2$
  \\
  $^1$Department of Computer Science, Harvard University
  \quad $^2$Amazon Web Services
  \\
  \texttt{jeffreyma@g.harvard.edu} 
  \\
  \texttt{\{philepei,lausen,gkarypis\}@amazon.com} 
}

\begin{document}
\maketitle

\begin{abstract}
As the scale of training large language models (LLMs) increases, one emergent failure is silent data corruption (SDC), where hardware produces incorrect computations without explicit failure signals. In this work, we are the first to investigate the impact of real-world SDCs on LLM training by comparing model training between healthy production nodes and unhealthy nodes exhibiting SDCs. With the help from a cloud computing platform, we access the unhealthy nodes that were swept out from production by automated fleet management. Using deterministic execution via XLA compiler and our proposed synchronization mechanisms, we isolate and analyze the impact of SDC errors on these nodes at three levels: at each submodule computation, at a single optimizer step, and at a training period. Our results reveal that the impact of SDCs on computation varies on different unhealthy nodes. Although in most cases the perturbations from SDCs on submodule computation and gradients are relatively small, SDCs can lead models to converge to different optima with different weights and even cause spikes in the training loss. Our analysis sheds light on further understanding and mitigating the impact of SDCs.
\end{abstract}



\section{Introduction}

Video generation has garnered significant attention owing to its transformative potential across a wide range of applications, such media content creation~\citep{polyak2024movie}, advertising~\citep{zhang2024virbo,bacher2021advert}, video games~\citep{yang2024playable,valevski2024diffusion, oasis2024}, and world model simulators~\citep{ha2018world, videoworldsimulators2024, agarwal2025cosmos}. Benefiting from advanced generative algorithms~\citep{goodfellow2014generative, ho2020denoising, liu2023flow, lipman2023flow}, scalable model architectures~\citep{vaswani2017attention, peebles2023scalable}, vast amounts of internet-sourced data~\citep{chen2024panda, nan2024openvid, ju2024miradata}, and ongoing expansion of computing capabilities~\citep{nvidia2022h100, nvidia2023dgxgh200, nvidia2024h200nvl}, remarkable advancements have been achieved in the field of video generation~\citep{ho2022video, ho2022imagen, singer2023makeavideo, blattmann2023align, videoworldsimulators2024, kuaishou2024klingai, yang2024cogvideox, jin2024pyramidal, polyak2024movie, kong2024hunyuanvideo, ji2024prompt}.


In this work, we present \textbf{\ours}, a family of rectified flow~\citep{lipman2023flow, liu2023flow} transformer models designed for joint image and video generation, establishing a pathway toward industry-grade performance. This report centers on four key components: data curation, model architecture design, flow formulation, and training infrastructure optimization—each rigorously refined to meet the demands of high-quality, large-scale video generation.


\begin{figure}[ht]
    \centering
    \begin{subfigure}[b]{0.82\linewidth}
        \centering
        \includegraphics[width=\linewidth]{figures/t2i_1024.pdf}
        \caption{Text-to-Image Samples}\label{fig:main-demo-t2i}
    \end{subfigure}
    \vfill
    \begin{subfigure}[b]{0.82\linewidth}
        \centering
        \includegraphics[width=\linewidth]{figures/t2v_samples.pdf}
        \caption{Text-to-Video Samples}\label{fig:main-demo-t2v}
    \end{subfigure}
\caption{\textbf{Generated samples from \ours.} Key components are highlighted in \textcolor{red}{\textbf{RED}}.}\label{fig:main-demo}
\end{figure}


First, we present a comprehensive data processing pipeline designed to construct large-scale, high-quality image and video-text datasets. The pipeline integrates multiple advanced techniques, including video and image filtering based on aesthetic scores, OCR-driven content analysis, and subjective evaluations, to ensure exceptional visual and contextual quality. Furthermore, we employ multimodal large language models~(MLLMs)~\citep{yuan2025tarsier2} to generate dense and contextually aligned captions, which are subsequently refined using an additional large language model~(LLM)~\citep{yang2024qwen2} to enhance their accuracy, fluency, and descriptive richness. As a result, we have curated a robust training dataset comprising approximately 36M video-text pairs and 160M image-text pairs, which are proven sufficient for training industry-level generative models.

Secondly, we take a pioneering step by applying rectified flow formulation~\citep{lipman2023flow} for joint image and video generation, implemented through the \ours model family, which comprises Transformer architectures with 2B and 8B parameters. At its core, the \ours framework employs a 3D joint image-video variational autoencoder (VAE) to compress image and video inputs into a shared latent space, facilitating unified representation. This shared latent space is coupled with a full-attention~\citep{vaswani2017attention} mechanism, enabling seamless joint training of image and video. This architecture delivers high-quality, coherent outputs across both images and videos, establishing a unified framework for visual generation tasks.


Furthermore, to support the training of \ours at scale, we have developed a robust infrastructure tailored for large-scale model training. Our approach incorporates advanced parallelism strategies~\citep{jacobs2023deepspeed, pytorch_fsdp} to manage memory efficiently during long-context training. Additionally, we employ ByteCheckpoint~\citep{wan2024bytecheckpoint} for high-performance checkpointing and integrate fault-tolerant mechanisms from MegaScale~\citep{jiang2024megascale} to ensure stability and scalability across large GPU clusters. These optimizations enable \ours to handle the computational and data challenges of generative modeling with exceptional efficiency and reliability.


We evaluate \ours on both text-to-image and text-to-video benchmarks to highlight its competitive advantages. For text-to-image generation, \ours-T2I demonstrates strong performance across multiple benchmarks, including T2I-CompBench~\citep{huang2023t2i-compbench}, GenEval~\citep{ghosh2024geneval}, and DPG-Bench~\citep{hu2024ella_dbgbench}, excelling in both visual quality and text-image alignment. In text-to-video benchmarks, \ours-T2V achieves state-of-the-art performance on the UCF-101~\citep{ucf101} zero-shot generation task. Additionally, \ours-T2V attains an impressive score of \textbf{84.85} on VBench~\citep{huang2024vbench}, securing the top position on the leaderboard (as of 2025-01-25) and surpassing several leading commercial text-to-video models. Qualitative results, illustrated in \Cref{fig:main-demo}, further demonstrate the superior quality of the generated media samples. These findings underscore \ours's effectiveness in multi-modal generation and its potential as a high-performing solution for both research and commercial applications.
 
\section{Background} \label{section:LLM}

% \subsection{Large Language Model (LLM)}   

Figure~\ref{fig:LLaMA_model}(a) shows that a decoder-only LLM initially processes a user prompt in the “prefill” stage and subsequently generates tokens sequentially during the “decoding” stage.
Both stages contain an input embedding layer, multiple decoder transformer blocks, an output embedding layer, and a sampling layer.
Figure~\ref{fig:LLaMA_model}(b) demonstrates that the decoder transformer blocks consist of a self attention and a feed-forward network (FFN) layer, each paired with residual connection and normalization layers. 

% Differentiate between encoder/decoder, explain why operation intensity is low, explain the different parts of a transformer block. Discuss Table II here. 

% Explain the architecture with Llama2-70B.

% \begin{table}[thb]
% \renewcommand\arraystretch{1.05}
% \centering
% % \vspace{-5mm}
%     \caption{ML Model Parameter Size and Operational Intensity}
%     \vspace{-2mm}
%     \small
%     \label{tab:ML Model Parameter Size and Operational Intensity}    
%     \scalebox{0.95}{
%         \begin{tabular}{|c|c|c|c|c|}
%             \hline
%             & Llama2 & BLOOM & BERT & ResNet \\
%             Model & (70B) & (176B) & & 152 \\
%             \hline
%             Parameter Size (GB) & 140 & 352 & 0.17 & 0.16 \\
%             \hline
%             Op Intensity (Ops/Byte) & 1 & 1 & 282 & 346 \\
%             \hline
%           \end{tabular}
%     }
% \vspace{-3mm}
% \end{table}

% {\fontsize{8pt}{11pt}\selectfont 8pt font size test Memory Requirement}

\begin{figure}[t]
    \centering
    \includegraphics[width=8cm]{Figure/LLaMA_model_new_new.pdf}
    \caption{(a) Prefill stage encodes prompt tokens in parallel. Decoding stage generates output tokens sequentially.
    (b) LLM contains N$\times$ decoder transformer blocks. 
    (c) Llama2 model architecture.}
    \label{fig:LLaMA_model}
\end{figure}

Figure~\ref{fig:LLaMA_model}(c) demonstrates the Llama2~\cite{touvron2023llama} model architecture as a representative LLM.
% The self attention layer requires three GEMVs\footnote{GEMVs in multi-head attention~\cite{attention}, narrow GEMMs in grouped-query attention~\cite{gqa}.} to generate query, key and value vectors.
In the self-attention layer, query, key and value vectors are generated by multiplying input vector to corresponding weight matrices.
These matrices are segmented into multiple heads, representing different semantic dimensions.
The query and key vectors go though Rotary Positional Embedding (RoPE) to encode the relative positional information~\cite{rope-paper}.
Within each head, the generated key and value vectors are appended to their caches.
The query vector is multiplied by the key cache to produce a score vector.
After the Softmax operation, the score vector is multiplied by the value cache to yield the output vector.
The output vectors from all heads are concatenated and multiplied by output weight matrix, resulting in a vector that undergoes residual connection and Root Mean Square layer Normalization (RMSNorm)~\cite{rmsnorm-paper}.
The residual connection adds up the input and output vectors of a layer to avoid vanishing gradient~\cite{he2016deep}.
The FFN layer begins with two parallel fully connections, followed by a Sigmoid Linear Unit (SiLU), and ends with another fully connection.

\section{Methodology}\label{sec:experimental_setup}
% \section{\ser{Experimental Setup and}{Evaluation} Methodology}\label{sec:experimental_setup}

% \emph{\color{red} \textbf{Paris:} Should write something introductory here - probably merged froom Section 3.1 below.}

% \subsection{Overview and Objectives}\label{sec:overview_objectives}
% \emph{\color{red} \textbf{Paris:} Might merge this and 3.1 with above.}

In this section, we elaborate on the evaluation methodology followed in our experimental study. We first provide an overview of the used models (Section~\ref{sec:models}), datasets (Section~\ref{sec:datasets}), and configuration parameters (Section~\ref{sec:configuration-params}). Finally, in Section~\ref{sec:technical_specifications}, we present the technical specifications of the system we used for the experimentation.  







% \begin{table}
%     \centering
%     \footnotesize
%     \begin{tabular}{lcc}
%       \hline
%       \textbf{Model}                      & \textbf{\shortstack{Context Length}}  \\
%       \hline
%       Llama 3.2 Instruct~--~1B            & 128K                       \\
%       Llama 3.2 Instruct~--~3B            & 128K                       \\
%       Llama 3.1 Instruct~--~8B            & 128K                       \\
%       Llama 3 Instruct~--~8B              & 8192                       \\
%       Mistral Nemo Instruct~--~12B        & 128K                       \\
%       Phi3 Medium Instruct~--~14B         & 128K                       \\
%       Phi3.5 Mini Instruct~--~3.8B        & 128K                       \\
%       Gemma 2 Instruct~--~2B              & 8192                       \\
%       Gemma 2 Instruct~--~9B              & 8192                       \\
%       Gemma 2 Instruct~--~27B             & 8192                       \\
%       Qwen 2 Instruct~--~7B               & 32K                        \\
%       Qwen 2.5 Instruct~--~14B            & 32K                        \\
%       \hline
%     \end{tabular}
%     \caption{Details of selected models for evaluation.}\label{tab:models}
% \end{table}



% \begin{table*}
%     \centering
%     \footnotesize
%     \begin{tabular}{llll}
%       \hline
%       \textbf{Model}                      & \textbf{Context Length}    & \textbf{Parameters}  & \textbf{HF Links} \\
%       \hline
%       Llama 3.2~--~1B Instruct            & 128K                       & 1B                   & \href{http://huggingface.co/hugging-quants/Llama-3.2-1B-Instruct-Q8_0-GGUF}{hugging-quants/Llama-3.2-1B-Instruct-Q8\_0-GGUF} \\
%       Llama 3.2~--~3B Instruct            & 128K                       & 3B                   & \href{https://huggingface.co/hugging-quants/Llama-3.2-3B-Instruct-Q8_0-GGUF}{hugging-quants/Llama-3.2-3B-Instruct-Q8\_0-GGUF} \\
%       Llama 3.1~--~8B Instruct            & 128K                       & 8B                   & \href{https://huggingface.co/lmstudio-community/Meta-Llama-3.1-8B-Instruct-GGUF}{lmstudio-community/Meta-Llama-3.1-8B-Instruct-GGUF} \\
%       Llama 3~--~8B Instruct              & 8192                       & 8B                   & \href{https://huggingface.co/lmstudio-community/Meta-Llama-3-8B-Instruct-GGUF}{lmstudio-community/Meta-Llama-3-8B-Instruct-GGUF} \\
%       Mistral Nemo~--~12B Instruct        & 128K                       & 12B                  & \href{https://huggingface.co/lmstudio-community/Mistral-Nemo-Instruct-2407-GGUF}{lmstudio-community/Mistral-Nemo-Instruct-2407-GGUF} \\
%       Phi3 Medium 128K~--~14B Instruct    & 128K                       & 14B                  & \href{https://huggingface.co/bartowski/Phi-3-medium-128k-instruct-GGUF}{bartowski/Phi-3-medium-128k-instruct-GGUF} \\
%       Phi3.5 Mini~--~Instruct             & 128K                       & 3.8B                 & \href{https://huggingface.co/bartowski/Phi-3.5-mini-instruct-GGUF}{bartowski/Phi-3.5-mini-instruct-GGUF} \\
%       Gemma 2~--~2B Instruct              & 8192                       & 2B                   & \href{https://huggingface.co/lmstudio-community/gemma-2-2b-it-GGUF}{lmstudio-community/gemma-2-2b-it-GGUF} \\
%       Gemma 2~--~9B Instruct              & 8192                       & 9B                   & \href{https://huggingface.co/bartowski/gemma-2-9b-it-GGUF}{bartowski/gemma-2-9b-it-GGUF} \\
%       Gemma 2~--~27B Instruct             & 8192                       & 27B                  & \href{https://huggingface.co/bartowski/gemma-2-27b-it-GGUF}{bartowski/gemma-2-27b-it-GGUF} \\
%       Qwen 2~--~7B Instruct               & 32K                        & 7B                   & \href{https://huggingface.co/Qwen/Qwen2-7B-Instruct}{Qwen/Qwen2-7B-Instruct} \\
%       Qwen 2.5~--~14B Instruct            & 32K                        & 14B                  & \href{https://huggingface.co/lmstudio-community/Qwen2.5-14B-Instruct-GGUF}{lmstudio-community/Qwen2.5-14B-Instruct-GGUF} \\
%       \hline
%     \end{tabular}
%     \caption{Models with Context Length, Number of Parameters, and HuggingFace Links. \emph{\color{red} \textbf{Paris}: Might remove links and make smaller -- or just mention them by name and put the detailed table in Appendix.}} \label{tab:models}
% \end{table*}


\subsection{Models}\label{sec:models}

For our experiments, we selected the Instruction-tuned versions of five prominent open-weight model families: \textit{LLaMA}~\citep{llama3}, \textit{Mistral}~\citep{mistral}, \textit{Phi}~\citep{phi3}, \textit{Gemma}~\citep{gemma2}, and \textit{Qwen}~\citep{qwen2, qwen2_5}. 
% For several models, we test different variations of training parameters, resulting to a total of $12$ different models.
% The experimental design was shaped by practical constraints and research goals. 
A key challenge was performing the experiments on commodity hardware, both due to computational limitations and to demonstrate that citation intent classification can be executed efficiently with limited resources. This influenced the selection of models, which range in size from small ($1$B parameters) to medium-sized ($32$B parameters).
% \ser{The design also sought to maximize the range of configurations explored to identify optimal performance scenarios.}{}

% \ser{The selection criteria prioritized models with parameter sizes ranging from 1B to 32B to ensure compatibility with commodity single-GPU machines.}{}
Since in-context learning inherently increases the number of tokens to each prompt (particularly in the many-shot scenario), we opted for a lower cutoff of $8,192$ tokens in the context length. 
This ensures that all selected models could process the longest prompts in our experiments without truncation.

To reduce the memory footprint and computational requirements of our evaluation, we utilize the 8-bit (Q8) quantized versions of the models. This approach significantly reduces memory usage without compromising performance. \textit{Quantization} involves converting model parameters from 16-bit floating-point precision to 8-bit integers, enabling more efficient computation while preserving expressive power and accuracy~\citep{jm3}.

% Table~\ref{tab:models} summarizes the selected models. A detailed view along with the HuggingFace links for each model is available at Appendix~\ref{sec:appendix-c-models}.
% Table~\ref{tab:models} \ser{provides an overview of}{summarizes} the selected models, \ser{including their}{listing} context length, \ser{}{and} number of parameters\ser{, and HuggingFace links for easy access}{}.

The model variations used for our evaluation were the following (a more detailed view is available in Appendix~\ref{sec:appendix-c-models}):

\begin{itemize}[nosep,topsep=1pt]
    \item Llama 3 \& 3.1 (8B), Llama 3.2 (1B, 3B)
    \item Mistral Nemo (12B)
    \item Phi 3 Medium (14B), Phi 3.5 Mini (3.8B)
    \item Gemma 2 (2B, 9B, 27B)
    \item Qwen 2 (7B), Qwen 2.5 (14B)
\end{itemize}

\subsection{Datasets}\label{sec:datasets}
For our experiments, we used two datasets:

\begin{itemize}[nosep,topsep=1pt]
    \item \textit{SciCite}~\citep{CAZ2019} consists of approximately $11,000$ citation strings annotated with three classes: \texttt{Background Information}, \texttt{Method}, and \texttt{Results Comparison}. 
    % \ser{It is widely used in the literature as a benchmark for citation intent tasks.}{}
    \item \textit{ACL-ARC}~\citep{JKH2018} contains 2,000 citation strings classified into six categories: \texttt{Background}, \texttt{Motivation}, \texttt{Uses}, \texttt{Extends}, \texttt{Compares or Contrasts}, and \texttt{Future}. 
    % \ser{It provides a more granular classification of citation intents.}{}
\end{itemize}

These two datasets are widely used in citation intent classification research. 
The SciCite dataset is larger and more diverse, while the ACL-ARC dataset offers a more granular classification scheme. 

% These characteristics allow us to evaluate model performance across a range of citation intent classification tasks.

% \emph{\color{red} \textbf{Paris:} Will add more details about the datasets from their respective papers.}

\subsection{Configuration Parameters}
\label{sec:configuration-params}
In this section, we describe the configuration parameters of the examined models, including prompting methods, system prompts, query templates, example methods, and temperature settings. 
% Details on the range of values for these parameters can be found in Table~\ref{tab:experimental-params}.

% \begin{table}[t]
%     \centering
%     \footnotesize
%     \begin{tabular}{cc}
%       \hline
%       \textbf{Parameter} & \textbf{Range of values} \\
%       \midrule
%       \multirow{2}{*}{\textbf{Prompting methods}} & Zero-shot, One-shot, \\
%                                                   & Few-shot, Many-shot \\
%       \midrule
%       \textbf{System prompts} & SP1, SP2, SP3 \\
%       \midrule
%         \textbf{Query templates} & Simple, Multiple-choice \\
%         \textbf{Example methods} & Inline, Roles \\
%         \textbf{Temperature} & $0.0$, $0.2$, $0.5$, $1.0$ \\
%       \hline
%     \end{tabular}
%     \caption{Configuration parameters. \emph{\color{red} (\textbf{Paris:} Might remove)}}\label{tab:experimental-params}
% \end{table}



\subsubsection{In-context Learning Methods}\label{sec:in_context_learning}
To evaluate model performance, we applied four prompting methods (PM) by following the In-Context Learning paradigm~\citep{BMR2020}:
% \begin{itemize} [nosep,topsep=1pt]
%     \item \textbf{Zero-shot prompting:} No examples are provided. The model infers the task solely from the system prompt.
%     \item \textbf{One-shot prompting:} One example per class is provided to guide the model.
%     \item \textbf{Few-shot prompting:} Five examples per class are included to offer additional context.
%     \item \textbf{Many-shot prompting:} Ten examples per class are provided, representing a more extensive context.
% \end{itemize}
\textit{Zero-shot} (no examples),
\textit{One-shot} (a single example per class),
\textit{Few-shot} ($5$ examples per class), and
\textit{Many-shot} ($10$ examples per class).

The aforementioned prompting methods were selected to evaluate whether model performance improves as the number of examples increases, and to identify any saturation point where additional examples yield diminishing or negative returns. 
This aligns with prior literature on in-context learning methods, where these specific configurations have been extensively studied~\citep{icl-survey}.
% \ser{In-context learning is particularly suited to citation intent classification, as it allows the models to leverage their pre-trained contextual understanding without requiring task-specific fine-tuning.}{}
Incorporating examples directly into the prompts allows models to better understand the task and the expected output format, which is especially important for citation intent classification, where the citation context is key to determining the correct label.




\subsubsection{System Prompts}\label{sec:system_prompts}
System prompts (SP) play a critical role in our experimental design, as they set the context and expectations for the model's behavior. 
They guide the model's attention on citation intent classification, providing task-specific context, 
class definitions, and guidelines to ensure outputs are aligned with the evaluation requirements.

We explored three distinct variations of system prompts. The first (SP1) was a simple, intuitive instruction, serving as a natural starting point for introducing the task.
The second prompt (SP2) adopted a structured approach inspired by the CO-STAR framework~\citep{costar}, which organizes prompts into six key components. In particular, \textit{Context} provides background information to help the model understand the scenario, while \textit{Objective} clearly defines the task to direct its focus. \textit{Style}, \textit{Tone}, and \textit{Audience} shape the model's writing style, sentiment, and intended readership, respectively. 
Finally, \textit{Response} defines the expected format to ensure clarity and downstream processing.

% \begin{itemize}[nosep,topsep=1pt]
%     \item \textbf{Context (C):} Providing background information helps the model understand the specific scenario.
%     \item \textbf{Objective (O):} Clearly defining the task directs the model’s focus.
%     \item \textbf{Style (S):} Specifying the desired writing style aligns the model’s response.
%     \item \textbf{Tone (T):} Setting the tone ensures the response resonates with the required sentiment.
%     \item \textbf{Audience (A):} Identifying the intended audience tailors the model’s output.
%     \item \textbf{Response (R):} Defining the expected output format ensures clarity and facilitates downstream processing.
% \end{itemize}
Since our task requires a single citation label as output, 
we excluded stylistic components (i.e.,~Style, Tone, and Audience),
and incorporated detailed class definitions in the Response section.
The third variation (SP3) refined the structured prompt by explicitly reiterating the expected labels in the Response section; this aims to further clarify the task, and enhance the consistency of the model's outputs. 
% \ser{Initial testing suggested that explicitly reinforcing the available classes improved performance.}{}
All the prompts can be found in Appendix~\ref{sec:appendix-a-system-prompts}.

% \emph{\color{red} \textbf{Paris:} Will add the prompts in the Appendix.}

\subsubsection{Query Templates}\label{sec:query_templates}
\emph{Query templates} (QT) define the structure in which citation sentences are presented to the model, both as examples during prompting and as queries during evaluation. 
Initially, we adopted a \textit{Simple Query} template: 
the citation sentence was followed by ``Class:'', 
either pre-filled with the correct class (in examples) or left blank (in queries). However, we observed that some models struggled to align with the expectation of responding solely with the class labels, leading to inconsistent performance and increased variance.
To address this, we introduced a \textit{Multiple-choice Query} template, where, following the citation sentence, the model is explicitly asked to identify the most likely citation intent, with the possible class labels presented as multiple-choice options. 
While this approach increases the token count per query, it demonstrates significant performance gains, as discussed in Section~\ref{sec:parameter_performance}.

% \ser{In both templates, minimal post-processing was necessary to extract the predicted class label from the model’s output and ensure alignment with the evaluation criteria.}{}

\begin{table*}[t]
    \centering
    \footnotesize
    \begin{tabular}{cclcccccc}
      \hline
      \textbf{Dataset}            &  \textbf{Rank} &   \textbf{Model}          & \textbf{Overall}    & \textbf{Zero-Shot}  & \textbf{One-Shot} & \textbf{Few-Shot}  & \textbf{Many-Shot} \\
      \hline
      \multirow{4}{*}{SciCite}  & 1             &   Qwen 2.5~--~14B         & 112                               & 20                  &  37               & 27                 & 28                 \\
                                & 2             &   Mistral Nemo~--~12B     & 22                                & 0                   &  0                & 12                 & 10                 \\
                                & 3             &   Gemma 2~--~9B           & 4                                 & 0                   &  3                & 1                  & 0                  \\
                                & 4             &   Gemma 2~--~27B          & 2                                 & 0                   &  0                & 0                  & 2                  \\
      \hline
      \multirow{2}{*}{ACL-ARC}  & 1             &   Qwen 2.5~--~14B         & 24                          & 4                   &  4                & 8                  & 8                  \\
                                & 2             &   Gemma 2~--~27B          & 4                           & 0                   &  4                & 0                  & 0                  \\
      \hline
    \end{tabular}
    \caption{Model ranking based on Best-Performing Count (Overall \& By Prompting Method).}\label{tab:best_performer_evaluation}
\end{table*}

\begin{table*}[t]
    \centering
    \footnotesize
    \begin{tabular}{cccccccccc}
        \hline
        \textbf{Dataset}            & \textbf{Rank}     & \textbf{Model}       & \textbf{PM}    & \textbf{SP}   & \textbf{QT}       & \textbf{EM}   & \textbf{T}    & \textbf{F1-Score} \\
        \hline
        \multirow{5}{*}{SciCite}    & 1                 & Qwen 2.5~--~14B       & Few-shot      & SP3           & Multiple-Choice   & Roles         & 0.5           & 78.27 \\
                                    & 2                 & Qwen 2.5~--~14B       & Few-shot      & SP3           & Multiple-Choice   & Roles         & 1.0           & 78.39 \\
                                    & 3                 & Qwen 2.5~--~14B       & Few-shot      & SP3           & Multiple-Choice   & Roles         & 0.2           & 78.34 \\
                                    & 4                 & Qwen 2.5~--~14B       & Many-shot     & SP3           & Multiple-Choice   & Roles         & 0.0           & 78.00 \\
                                    & 5                 & Qwen 2.5~--~14B       & Many-shot     & SP3           & Multiple-Choice   & Roles         & 0.5           & 77.93 \\
        \hline
        \multirow{5}{*}{ACL-ARC}    & 1                 & Qwen 2.5~--~14B       & Few-shot      & SP3           & Multiple-Choice   & Roles         & 1.0           & 60.88 \\
                                    & 2                 & Qwen 2.5~--~14B       & Few-shot      & SP3           & Multiple-Choice   & Roles         & 0.0           & 59.75 \\
                                    & 3                 & Qwen 2.5~--~14B       & Few-shot      & SP3           & Multiple-Choice   & Roles         & 0.2           & 59.66 \\
                                    & 4                 & Qwen 2.5~--~14B       & One-shot      & SP3           & Multiple-Choice   & Roles         & 0.5           & 64.87 \\
                                    & 5                 & Qwen 2.5~--~14B       & One-shot      & SP3           & Multiple-Choice   & Roles         & 1.0           & 64.32 \\
        \hline
    \end{tabular}
    \caption{Top-5 Configurations on SciCite and ACL-ARC (Q8).}
    \label{tab:top_5_configurations}
\end{table*}

\subsubsection{Example Method Variations}\label{sec:example_method_variations}

In our experiments, we explored two methods for presenting examples to the models (EM): \emph{Inline} and \emph{Roles}.

In the Inline approach, example citation sentences and their corresponding classes (as per the defined query templates) are provided directly within the system prompt. After this setup, the prompt then transitions to the evaluation phase, where the ``user'' provides a citation sentence without a corresponding class and the model, acting as the ``assistant'', predicts the correct class.

The Roles method, on the other hand, simulates a conversational exchange between the ``user'' and ``assistant''.
Each example follows a back-and-forth interaction, with the ``user'' providing a citation sentence and the ``assistant'' responding with the correct class. 
After several such examples, the interaction moves to the evaluation phase, where the ``user'' presents a new citation sentence, and the model predicts its class.

% \ser{We did not identify any immediate theoretical advantage to either method, so both were tested to evaluate their potential impact on model performance. Further discussion on the outcomes and insights derived from these methods can be found in Section~\ref{sec:parameter_performance}.}{}

\subsubsection{Temperature}\label{sec:temperature}

Temperature (T) is a hyperparameter that controls the randomness or creativity of a language model's outputs by adjusting the probability distribution of possible next tokens~\citep{jm3}. Lower temperatures (i.e., close to $0$) correspond to greedy decoding, where the model deterministically selects the most probable token at each step, while higher temperatures (closer to $1$) introduce greater variability by allowing the model to sample from a broader range of options.

For classification tasks, we aim for the model to output the most probable class label. 
Therefore, we use a temperature of $0$ as the baseline, ensuring fully deterministic predictions. 
To explore how controlled randomness affects the model's behavior and classification performance, we also evaluate higher temperatures, such as 0.2, 0.5, and 1.0.
A temperature of 0.2 introduces a small degree of randomness but still heavily favors the most probable answer, while 0.5 strikes a balance between randomness and determinism. At 1.0, randomness is maximized, allowing us to assess whether excessive stochasticity degrades performance.

% {\color{red}
% The temperature settings tested were as follows:
% \begin{itemize}[nosep,topsep=1pt]
%     \item \textbf{Temperature 0.0 (Greedy Decoding):} The model is fully-deterministic and is forced to choose the most probable answer without any randomness. We use this as our baseline to compare how flexible temperatures improve or worsen accuracy.
%     \item \textbf{Temperature 0.2:} Introduces a small degree of randomness but still heavily favors the most probable answer.
%     \item \textbf{Temperature 0.5:} Strikes a balance between randomness and determinism.
%     \item \textbf{Temperature 1.0:} Maximizes randomness. We investigate this to test if too much randomness degrades performance significantly.
% \end{itemize}
% }

% \begin{table*}[t]
%     \centering
%     \footnotesize
%     \begin{tabular}{clcccccc}
%       \hline
%       \textbf{Rank} &   \textbf{Model}          & \textbf{Best-Performing}    & \textbf{Zero-Shot}  & \textbf{One-Shot} & \textbf{Few-Shot}  & \textbf{Many-Shot} \\
%       \hline
%       1             &   Qwen 2.5~--~14B         & 112                               & 20                  &  37               & 27                 & 28                 \\
%       2             &   Mistral Nemo~--~12B     & 22                                & 0                   &  0                & 12                 & 10                 \\
%       3             &   Gemma 2~--~9B           & 4                                 & 0                   &  3                & 1                  & 0                  \\
%       4             &   Gemma 2~--~27B          & 2                                 & 0                   &  0                & 0                  & 2                  \\
%       \hline
%     \end{tabular}
%     \caption{Model ranking based on Best-Performing Count (Overall \& By Prompting Method) on SciCite.}\label{tab:best_performer_evaluation_scicite}
% \end{table*}

% \begin{table*}[t]
%     \centering
%     \footnotesize
%     \begin{tabular}{clcccccc}
%       \hline
%       \textbf{Rank} &   \textbf{Model}          & \textbf{Best-Performing}    & \textbf{Zero-Shot}  & \textbf{One-Shot} & \textbf{Few-Shot}  & \textbf{Many-Shot} \\
%       \hline
%       1             &   Qwen 2.5~--~14B         & 24                          & 4                   &  4                & 8                  & 8                  \\
%       2             &   Gemma 2~--~27B          & 4                           & 0                   &  4                & 0                  & 0                  \\
%       \hline
%     \end{tabular}
%     \caption{Model ranking based on Best-Performing Count (Overall \& By Prompting Method) on ACL-ARC.{\color{red}SER: merge these 2 tables}}\label{tab:best_performer_evaluation_aclarc}
% \end{table*}

\subsection{Technical Specifications}\label{sec:technical_specifications}

We conducted our experiments on an M1 Max Mac Studio with $64$GB of memory, chosen to demonstrate the feasibility of running inference for Citation Intent Classification on commodity hardware.

For model hosting, 
we used LM Studio\footnote{\url{https://lmstudio.ai/}} which offers an intuitive interface for testing and interacting with models hosted on HuggingFace or locally. 
It also supports a local server mode compatible with the OpenAI API, allowing interaction through an API accessible in multiple programming languages. 
A command-line interface (CLI) tool\footnote{\url{https://github.com/lmstudio-ai/lms}} is also available for managing the server without using the UI.

% For hosting the models, we used LM Studio. LM Studio provides:
% \begin{enumerate}
%     \item User Interface (UI) for testing and interacting with models hosted on HuggingFace or locally.
%     \item Local server mode that provides API compatibility with the OpenAI API and supports multiple programming languages.
%     \item Command-line interface (CLI) tool (lms) for managing the server without relying on the UI.
% \end{enumerate}

% While alternatives like Ollama or HuggingFace's transformers library exist, 
% we chose LM Studio for its ease of use and compatibility with the OpenAI API, which aligned with our workflow.

% \ser{Since we are open-sourcing our evaluation platform, \ser{and the models and experimental setups are}{with} configurable \ser{}{models and setups}, 
% we \ser{are considering adding}{plan to add} support for alternative hosting tools in the future to accommodate different user needs and computational environments.}{}

% \emph{ \color{red} \textbf{Paris:} I want to say more about the evaluation framework and technical stuff, so I might make this section smaller and add more details to an appendix called ``Evaluation Platform''.}


\section{Experiments for RQ1}
\label{sec:primitive}
\underline{\textbf{RQ1}}: \emph{What is the impact of SDCs on Transformer submodule computation outputs?}

In this section, we first introduce our experiment setups for RQ1 and analyze experimental results to understand the impact of SDCs on submodule computation. We follow the same structure for RQ2 in  Section \ref{sec:sdc_single_optimizer_step} and RQ3 in Section \ref{sec:multiple_training_steps}.

\subsection{Experiment Setups}
\label{methods:primitive_impact}
We focus on four kinds of Transformer submodule computation, namely the forward computation of a self-attention module (FWD/ATTN) and an FFN module (FWD/FFN), and the backward computation of a self-attention module (BWD/ATTN) and an FFN module (BWD/FFN). We train two models on each pair of the unhealthy node and the healthy node simultaneously and use the \textbf{computation synchronization} mechanism discussed in Section \ref{sec:research_questions}. We use a decoder-only Transformer architecture similar to the Llama3-8B configuration \cite{dubey2024llama3herdmodels} with $D=16$ decoder layers and hidden state size of $H=4096$ and use the tensor parallelism to fit a model within a node. More details can be found in Appendix \ref{appendix:primivite_investigation}.

For a submodule computation $f$ in the model, we define $f'_{i}(x_{t,j})$ as the tensor computed on TP rank $t$ of unhealthy node $i$ at the microstep $j$ and $f(x_{t,j})$ as the corresponding output on healthy node. To quantify differences between $f'_{i}(x_{t,j})$ and $f(x_{t,j})$, we define two metrics called \emph{mismatch frequency} and \emph{mismatch severity}. We calculate the mismatch frequency for submodule $f$ on unhealthy node $i$ at the microstep $j$ as follows:
\begin{equation}
    %\label{equation:mismatch_frequency}
    freq^{f}_{i,j} = \frac{\sum_{t=1}^{TP}{Mis(f'_{i}(x_{t,j}), f(x_{t,j}))}}{TP \cdot MBS \cdot L\cdot H}
\end{equation}
where $Mis(y', y)$ counts the number of mismatching elements in two tensors $y$ and $y'$. We report the aggregated mismatch frequency for each submodule computation type $F$ on unhealthy node $i$ at the microstep $j$ by averaging across decoder layers:
\begin{equation}
    \label{equation:mismatch_frequency}
    freq^{F}_{i,j} = \frac{1}{D}\sum_{f \in F}{freq^{f}_{i,j}}, F=\{f^{(1)},...,f^{(D)}\}
\end{equation} 
Mismatch severity is defined as the average over non-zero values of the element-wise relative difference. Formally, we take the maximum over all TP ranks and calculate the mismatch severity for submodule $f$ on unhealthy node $i$ at microstep $j$:
\begin{equation}
    %\label{equation:mismatch_severity}
    sev^{f}_{i,j} = \max_{0 \leq t < TP}{\left[
        {Avg_{\neq 0}}\left({
            \left|
                \frac{f'_{i}(x_{t,j}) - f(x_{t,j})  }{f(x_{t,j})}
            \right|
        }\right)
    \right]}
\end{equation}
where $Avg_{\neq 0}(x)$ computes the average value over only non-zero elements of a tensor $x$. We also calculate the mismatch severity for each type of submodule computation $F$ on unhealthy node $i$ at the microstep $j$ by taking the maximum across decoder layers as follows:
\begin{equation}
    \label{equation:mismatch_severity_max_decoder}
    sev^{F}_{i,j} = \max_{f \in F}{sev^{f}_{i,j}}, F=\{f^{(1)},...,f^{(D)}\}
\end{equation}


\subsection{Results} \label{sec:submodule_outputs_results}
Table \ref{fig:primitive_frequency_summarized} shows the mismatch frequency of submodule computation.
We observe that the impact of SDCs on submodule computation varies across different unhealthy nodes, e.g. Nodes 10 and 11 have a high mismatch frequency while Nodes 2 and 3 do not show any SDC occurrence in this setting. 

\begin{table}[t]
\begin{center}
\begin{small}
\begin{sc}
\setlength{\tabcolsep}{3pt}
\renewcommand{\arraystretch}{1.1} 
\begin{tabular}{crrrr}
\toprule
Node ID & fwd/attn & fwd/ffn & bwd/attn & bwd/ffn  \\ \midrule
Node 1  & 1.55$e$-5  & 5.06$e$-7  & 1.56$e$-04  & 2.81$e$-6   \\
Node 4  & 3.79$e$-9  & 9.20$e$-11  & 2.99$e$-9  & 2.80$e$-11  \\
Node 5  & 0  & 0  & 1.49$e$-15  & 1.25$e$-12  \\
Node 6  & 1.71$e$-9  & 1.64$e$-11  & 1.49$e$-9  & 6.02$e$-11   \\
Node 7  & 2.13$e$-6  & 1.18$e$-7  & 4.31$e$-6  & 6.73$e$-8   \\
Node 8  & 3.21$e$-9  & 1.99$e$-11   & 1.01$e$-7  & 2.21$e$-9   \\
Node 9  & 1.10$e$-5  & 5.05$e$-7   & 4.33$e$-6  & 3.86$e$-8  \\
Node 10 & 4.78$e$-3  & 1.03$e$-3    & 1.92$e$-3  & 7.98$e$-5   \\
Node 11 & 2.89$e$-2  & 2.25$e$-3  & 6.71$e$-3  & 1.08$e$-4   \\
Node 13 & 0  & 0   & 0  & 1.21$e$-10  \\
Node 14 & 6.48$e$-11  & 0   & 4.91$e$-10  & 2.99$e$-9   \\
Node 15 & 0  & 0   & 0  & 7.39$e$-15  \\
\bottomrule
\end{tabular}
\end{sc}
\end{small}
\end{center}
\vskip -0.1in
\caption{Average mismatch frequency over microsteps for Transformer submodules. The table excludes the nodes that do not show any SDC in this setting.}
\vskip -0.1in
\label{fig:primitive_frequency_summarized}
\end{table}

We further find that SDCs do not occur uniformly over time: mismatch frequency often has a large variance across steps. Figure \ref{fig:attention_forward_mismatch_frequency_over_time} shows the mismatch frequency in the forward computation of the attention module on Nodes 7 and 14. We find that spikes of mismatch frequency sometimes occur, while during the majority of training time, no mismatch occurs. In Figure \ref{fig:primitive_spike_at_beginning}, we observe a high peak of mismatch frequency at the first few steps on Nodes 10 and 11, likely due to higher overall system usage when initializing the training run. The non-uniform occurrence of SDC during training suggests that SDCs might be caused by broader, compound system-level factors.

\begin{figure}[t!]
    \centering
    \includegraphics[width=0.85\linewidth]{body/figures/attention_forward_mismatch_frequency_over_step_1.png}
    \vskip -0.15in
    \caption{Non-uniform spikes of mismatch frequency in the forward computation of the attention module over time on Node 7, 14.}
    \label{fig:attention_forward_mismatch_frequency_over_time}
\vskip -0.15in
\end{figure}

\begin{figure}[t]
    \vskip 0.1in
    \centering
    \includegraphics[width=0.85\linewidth]{body/figures/attention_forward_mismatch_frequency_over_step_3.png}
    \vskip -0.1in
    \caption{High SDC occurrence with large initial spikes in smoothed mismatch frequency for the forward computation of the attention module on Node 10, 11.}
    \label{fig:primitive_spike_at_beginning}
    \vskip -0.1in
\end{figure}

Table \ref{fig:primitive_severity_summarized}  shows the maximum mismatch severity over optimizer steps for submodule computation on different unhealthy nodes. We find that SDCs cause certain tensor values in the computation to differ by large factors. For example, on Node 9, the mismatch severity exceeds $100$, which means SDCs can cause degraded TP ranks to have very different computation results on certain microsteps.

\begin{table}[t]
\begin{center}
\begin{small}
\begin{sc}
\setlength{\tabcolsep}{3pt}
\renewcommand{\arraystretch}{1.1} 
\begin{tabular}{crrrr}
\toprule
Node ID & fwd/attn & fwd/ffn & bwd/attn & bwd/ffn \\
\midrule
Node 1 & 99 & 312 & 7392 & 3.88 \\
Node 4 & 2.95 & 1.46 & 2.39 & 5.38 \\
Node 5 & 0 & 0 & 0.0047 & 0.0908 \\
Node 6 & 1.01 & 1 & 0.162 & 0.0776 \\
Node 7 & 43.2 & 88.5 & 32.3 & 0.297 \\ 
Node 8 & 13.5 & 5.69 & 6.41 & 1.59 \\ 
Node 9 & 119 & 200 & 3.79$e$+11 & 9.63$e$+12 \\ 
Node 10 & 1120 & 262 & 12.75 & 0.648 \\
 Node 11 & 318 & 976 & 2208 & 7680 \\ 
Node 13 & 0 & 0 & 0 & 0.316 \\ 
Node 14 & 1.12 & 0 & 3.80 & 0.380 \\ 
Node 15 & 0 & 0 & 0 & 0.0176 \\ \bottomrule
\end{tabular}

\end{sc}
\end{small}
\end{center}
\vskip -0.1in
\caption{Maximum mismatch severity over microsteps for each unhealthy node. The table excludes the nodes that do not show any SDC in this setting.}
\vskip -0.1in
\label{fig:primitive_severity_summarized}

\end{table}




\begin{figure*}[ht]
    \centering
    \vskip -0.12in
    \begin{minipage}[c]{0.69\linewidth}
        \centering
        \includegraphics[width=\linewidth]{body/figures/fix_weight_only_smoothed_0.9_1.png}
    \end{minipage}
    \hfill
    \begin{minipage}[c]{0.3\linewidth} % Adjust width as needed
        \centering
        \vskip 0.2in
        \begin{tiny}
            \begin{sc}
                \begin{tabular}{cc} % Use c|c for a vertical line
                    \toprule
                    Node ID & WCNTS Ratio \\
                    \midrule
                    1 & 0.037 \\
                    2 & 0.002 \\
                    3 & 1.34$e$-15 \\
                    4 & 0.019 \\
                    5 & 0.011 \\
                    6 & 0.004 \\
                    7 & 0.008 \\
                    8 & 0.006 \\
                    9 & 0.017 \\
                    10 & 0.010 \\
                    11 & 0.051 \\
                    12 & 0.002 \\
                    13 & 0.004 \\
                    14 & 0.037 \\
                    15 & 4.16$e$-4 \\
                    \bottomrule
                \end{tabular}
            \end{sc}
        \end{tiny}

    \end{minipage}
    \vskip -0.1in
    \caption{$L_2$-norm of the gradient difference and the ground-truth gradients over steps. The left table shows Worst Case Noise-to-Signal (WCNTS) ratios for unhealthy nodes.}
    \label{fig:fix_weight_only}
    \vskip -0.1in
\end{figure*}

\section{Experiments for RQ2} \label{sec:sdc_single_optimizer_step}
\underline{\textbf{RQ2}}: \emph{What is the impact of SDCs on the gradients of model weights at a single optimizer step?}

\subsection{Experiment Setups}
\label{methods:differing_gradients}
We train same models on each pair of unhealthy and healthy nodes simultaneously and use the \textbf{parameter synchronization} mechanism discussed in Section \ref{sec:research_questions}. 
After the forward and backward pass are finished  at step $j$, we compute the $L_2$ norm of elementwise difference between the gradients of model weights on the $i$-th unhealthy node $g'_{i,j}$ and the ground-truth gradients on the healthy node $g_{j}$. After taking an optimizer step, we use parameter synchronization to overwrite the model parameters on the unhealthy node. 
We also report the \emph{worst case noise-to-signal (WCNTS) ratio} to measure how significant SDC-induced noise to gradients is:
\begin{equation}
    \label{equation:worst_case_noise_to_signal}
    WCNTS_i = \max_{j}{\frac{\|g'_{i,j} - g_{j} \|_2}{\|g_{j}\|_2}}
\end{equation}
Using the same decoder block architecture as in Section \ref{sec:primitive}, we train a $32$-layer Transformer decoder with hidden state size as $H=4096$. More details can be found in Appendix \ref{appendix:model_training}.

\subsection{Results}
\label{results:model_gradient_norm_diff}

Figure \ref{fig:fix_weight_only} shows the $L_2$ norm of gradient difference and the WCNTS ratio for the gradients over optimizer steps across unhealthy nodes. We observe that gradients computed on unhealthy nodes deviate minimally from those computed on the healthy node. Although the absolute value of $L_2$ norm of the gradient difference is large before the $100$-th step, it is still relatively small compared to the $L_2$ norm of the ground-truth gradients and continues to decrease as the norm of the ground-truth gradients decreases.
In the worst case on Node 11, the $L_2$ norm of gradient difference is $5.1\%$ of that of the ground-truth gradients, showing that the SDC-induced noise in the gradients is relatively small relative to the ground-truth gradients. 

\section{Experiments for RQ3} \label{sec:multiple_training_steps}


\underline{\textbf{RQ3}}: \emph{What is the accumulated impact of SDCs on the model quality over multiple training steps?}

\subsection{Experiment Setups}
To provide a better understanding of how different the learned representations and model decision boundaries from unhealthy nodes would be, we design experiments for both model pre-training from scratch and fine-tuning of a pre-trained model.

\textbf{\myexperimentlabel{experiment:parameter_drift}}: \emph{How different is the learned model under accumulated SDC error from the ground truth during model pre-training?} 
We pretrain same models on each pair of unhealthy node and healthy node simultaneously. We follow the same experiment setting in Section \ref{methods:differing_gradients} except we \emph{do not use parameter synchronization mechanism}. To observe how SDCs impact the learned models during training, we report the training loss and the \emph{parameter difference} which is $L_2$ norm of the element-wise difference between model parameters on healthy and unhealthy node.

\textbf{\myexperimentlabel{experiment:finetuning}}: \emph{For downstream tasks, how would SDCs affect model finetuning?} 
We want to further understand how model quality is affected by SDCs when the model is fine-tuned on a downstream task. 
In this experiment, we fine-tune \verb|Mistral-7B-v0.3| \cite{jiang2023mistral7b} on six multiple-choice question answering tasks (CosmosQA \cite{cosmos-qa}; MathQA \cite{mathqadataset}; ScienceQA \cite{lu2022learn}; OpenbookQA \cite{OpenBookQA2018}; BoolQ \cite{clark2019boolqexploringsurprisingdifficulty}; and RACE \cite{raceLai2017large}) by instruction tuning \cite{weifinetuned}. 
For each task, we use a fixed random seed for shuffling the training dataset and evaluate the test accuracy (TA) of the models fine-tuned on the healthy node and on unhealthy nodes. Using the predictions of the model fine-tuned on the healthy node as the standard, we report the disagreement percentage (DP), which is defined as the percentage of the difference in predictions on the test set. To better understand the prediction difference, we fine-tune a model on healthy node with a different random seed as a baseline to contextualize the effect of SDC error relative to the effect of data ordering. More details can be found in Appendix \ref{appendix:finetuning}.

\subsection{Results}
\label{results:model_quality}


\begin{figure}[t]
    \centering
    \includegraphics[width=0.85\linewidth]{body/figures/no_synchronization_2.png}
    \vskip -0.15in
    \caption{The curves for parameter difference, gradient norms and training loss on unhealthy nodes. Note that the loss curves on all unhealthy nodes are plotted but identical to that on the healthy node.}
    \label{fig:no_synchronization}
    \vskip -0.2in
\end{figure}

\textbf{Results for Experiment \ref{experiment:parameter_drift}.} 
Figure \ref{fig:no_synchronization} shows the parameter difference, gradient norm and training loss on unhealthy nodes during pre-training. Despite training loss on each unhealthy node nearly identical to the healthy node, model weights on unhealthy nodes incrementally drift away from those on the healthy node, suggesting that SDCs are pushing models towards different local minima. 

Note that Node 13 shows no parameter difference before step 450, which indicates that no SDC occurs during this period. It is aligned with the finding in Section \ref{sec:submodule_outputs_results} that SDCs do not occur uniformly during training. After step 450, the model on Node 13 begins to quickly drift away from the ground-truth weights. We further find that the rates at which the parameter differences increase are similar on most unhealthy nodes, although the unhealthy nodes produce SDCs with different degrees of frequency and severity as shown in Section \ref{sec:submodule_outputs_results} This suggests that the rate of parameter drift is more likely to be driven by the sharp loss surface than purely by SDCs. In other words, SDCs serve as a trigger to push the optimization trajectory onto a different path through this sharp loss landscape, leading to the divergence of the parameters. 

\begin{table}[t]

\begin{center}
\begin{tiny}
\begin{sc}
\setlength{\tabcolsep}{3pt}
\begin{tabular}{cccc}
\toprule
\multirow{2}{*}{Configuration} & CosmosQA & MATH & OpenbookQA \\
 & TA (DP) & TA (DP) & TA (DP) \\
\midrule
without fine-tuning & 56.33 (44.80) & 24.02 (82.35) & 74.70 (29.30) \\
\midrule
Healthy Node & 90.79 (-) & 37.22 (-) & 83.80 (-) \\
Healthy Node (Seed=$43$) & 89.50 (6.70) & 38.83 (56.75) & 86.30 (18.70) \\
\midrule
Unhealthy Node 1 & 90.53 (5.15) & 36.78 (42.24) & 85.00 (16.80) \\
Unhealthy Node 2 & 90.79 (0.00) & 37.22 (0.00) & 83.80 (0.00) \\
Unhealthy Node 3 & 90.77 (4.96) & 34.47 (41.84) & 83.40 (16.10) \\
Unhealthy Node 4 & 90.32 (5.59) & 36.42 (37.76) & 84.30 (16.60) \\
Unhealthy Node 5 & 90.79 (0.00) & 37.19 (34.91) & 85.10 (14.10) \\
Unhealthy Node 6 & 0.00 (100.00) & 36.92 (36.82) & 84.70 (16.30) \\
Unhealthy Node 7 & 90.32 (4.99) & 37.22 (0.00) & 83.80 (0.00) \\
Unhealthy Node 8 & 89.84 (6.23) & 38.22 (36.62) & 85.20 (15.50) \\
Unhealthy Node 9 & 89.97 (5.38) & 35.78 (37.49) & 85.00 (17.40) \\
Unhealthy Node 10 & 89.97 (4.93) & 37.05 (37.05) & 84.10 (17.20) \\
Unhealthy Node 11 & 90.61 (4.54) & 36.82 (38.53) & 87.10 (15.60) \\
Unhealthy Node 12 & 90.79 (0.00) & 37.22 (0.00) & 83.80 (0.00) \\
Unhealthy Node 13 & 90.79 (0.00) & 37.22 (0.00) & 83.80 (0.00) \\
Unhealthy Node 14 & 89.74 (6.12) & 38.63 (32.29) & 85.00 (15.90) \\
Unhealthy Node 15 & 90.77 (3.27) & 38.53 (40.87) & 83.80 (0.00) \\
\bottomrule
\end{tabular}


\end{sc}
\end{tiny}
\end{center}
\vskip -0.1in
\caption{Finetuning results for three question answering tasks on different nodes. For each task, we report the test accuracy (TA) and the disagreement percentage (DP).}
\label{tab:finetuning_results_short}

\vskip -0.2in
\end{table}

\textbf{Results for Experiment \ref{experiment:finetuning}.} 
Table \ref{tab:finetuning_results_short} shows the fine-tuning results for three of the question-answering tasks on different nodes. The full results can be found in Appendix \ref{appendix:full_finetuning_results}. We find that the models fine-tuned on most unhealthy nodes are significantly better than the base model without fine-tuning and also have similar performance compared to the models fine-tuned on the healthy node. The disagreement percentage caused by SDCs on unhealthy nodes is not larger than using a different random seed for data shuffling. Aligned with the findings in Experiment \ref{experiment:parameter_drift}, this again affirms that SDCs on unhealthy nodes push the models towards different local minima.

However, SDCs are not necessarily harmless to model fine-tuning. Figure \ref{fig:finetuning_loss_spikes} shows the training loss during fine-tuning on CosmosQA and we find that significant loss spikes can occur on some unhealthy nodes. On Node 4, Node 6 and Node 7, the loss spikes occur in the middle of fine-tuning while later the training is again stabilized, which makes the final models still have benign performance. However, on Node 6, the loss spike occurs near the end of fine-tuning, which leads to the resulting model having zero test accuracy on CosmosQA as shown in Table \ref{tab:finetuning_results_short}. It indicates that the loss spikes caused by SDCs pose a threat to the model quality. We also note that loss spikes do not occur in every fine-tuning task. For example, training loss curves for OpenbookQA on unhealthy nodes are all identical to that on the healthy node, similar to the situation in Experiment \ref{experiment:parameter_drift}. Therefore, we conclude that the impact of SDC on model training is closely related to the loss surface of the training task.

\begin{figure}[t]
    \centering
    \includegraphics[width=0.85\linewidth]{body/figures/finetune_loss_cosmos_qa.png}
    \vskip -0.15in
    \caption{Training loss curves on different nodes during fine-tuning on CosmosQA dataset. The loss curves for other unhealthy nodes that are identical to the healthy node are not shown in this figure.}
    \label{fig:finetuning_loss_spikes}
    \vskip -0.2in
\end{figure}







%% New Disucssion 
Our study reveals how heavy users integrate LLMs into their daily tasks through distinct patterns. Rather than simple tool usage, participants demonstrated sophisticated cognitive offloading strategies that transformed their decision-making processes. In our study, we observed participants delegating social and interpersonal reasoning to LLMs, suggesting ways users might leverage AI collaboration to support their social cognition processes.

Participants' mental models of LLMs directly influenced their cognitive strategies---those viewing LLMs as rational entities engaged in cognitive complementarity by leveraging LLM capabilities where they perceived personal limitations, while those viewing LLMs as average decision-makers used cognitive benchmarking, establishing baseline standards while reserving higher-order tasks for themselves.
% While delegating a broad range of decisions raised potential concerns about over-reliance and diminished critical thinking, our findings also highlight a nuanced form of human-AI collaboration where users and LLMs develop complementary relationships. Participants showed diverse usage strategies, treating LLMs as an emerging problem-solving tool and developing sophisticated prompting techniques. Most notably, participants frequently sought LLM guidance on social appropriateness and interpersonal situations. Although some users expressed concerns about potential skill degradation and a sense of unease, LLM consultations often led to a more thorough consideration of social factors and an enhanced understanding of different perspectives.

This raises questions for future research on redefining how we conceptualize and measure over-reliance on LLMs. Current metrics typically assess over-reliance through simplified quantitative measures in controlled settings, primarily focusing on users' acceptance rates of LLM outputs ~\cite{bo2024rely, kim2024rely}. However, our findings reveal more complex patterns of engagement. Participants did not blindly adopt LLM outputs, even in cases where they eventually accepted them. Instead, participants demonstrated thoughtful delegation strategies, using LLMs to validate existing decisions, automate routine tasks, or navigate unfamiliar situations. The critical concern was not users' acceptance of LLM outputs, but rather instances where users adopted LLM reasoning without exploring alternative perspectives. Future research should expand the definition of over-reliance beyond simple acceptance rates to examine how users critically engage with alternative lines of reasoning.

Another key direction for future research involves capturing diverse user contexts. Our participants valued the ability of LLMs to extract necessary contextual information when not initially provided. They appreciated that they could receive meaningful responses without extensively explaining background information, even for context-heavy topics like relationship advice. Future research should explore ways to incorporate multi-modal inputs beyond text-based interactions, allowing users to convey context through various channels. Additionally, LLMs' ability to elicit implicit user intentions without explicit prompting is crucial, as demonstrated by recent advances in reasoning-focused LLM architectures that can proactively identify and address underlying user needs.

The development of active usage patterns with LLMs appeared more prominent among younger users who had less experience managing tasks without these systems. Participants with extensive pre-LLM experience maintained clearer boundaries and showed greater awareness of system limitations. In contrast, users with less experience with LLMs demonstrated fewer reservations, viewing LLM interaction itself as a skill and actively developing their prompting strategies. Conducting design studies focused on younger generations, to better understand and support these emerging interaction patterns represents a crucial direction for future research.

\section{Limitations}
Our method's reliance on semantic embeddings introduces inherent biases present in encoder's training data. While these embeddings enable semantic consistency, they may not capture certain culturally-specific or nuanced artistic concepts. This highlights the need for more careful study on choices of the semantic embeddings and their effects on SliderSpace discovery. The current discovery process requires significant computational time ($\approx$ 2 hrs on A100), which may limit rapid experimentation and iteration. This computational overhead opens avenues for future research into training time optimizations. We also note that our method trains 4 times faster than  Concept Sliders for same number of sliders. For art style discovery, it is possible that the discovered directions are not one-to-one matched with the original artists. Further work can address discovery that nudges the directions to be aligned with real artists. 
% This can be used as an interpretability tool or an attribution tool if the discover directions are aligned with real artists. 

\bibliography{custom}

\appendix
\subsection{Lloyd-Max Algorithm}
\label{subsec:Lloyd-Max}
For a given quantization bitwidth $B$ and an operand $\bm{X}$, the Lloyd-Max algorithm finds $2^B$ quantization levels $\{\hat{x}_i\}_{i=1}^{2^B}$ such that quantizing $\bm{X}$ by rounding each scalar in $\bm{X}$ to the nearest quantization level minimizes the quantization MSE. 

The algorithm starts with an initial guess of quantization levels and then iteratively computes quantization thresholds $\{\tau_i\}_{i=1}^{2^B-1}$ and updates quantization levels $\{\hat{x}_i\}_{i=1}^{2^B}$. Specifically, at iteration $n$, thresholds are set to the midpoints of the previous iteration's levels:
\begin{align*}
    \tau_i^{(n)}=\frac{\hat{x}_i^{(n-1)}+\hat{x}_{i+1}^{(n-1)}}2 \text{ for } i=1\ldots 2^B-1
\end{align*}
Subsequently, the quantization levels are re-computed as conditional means of the data regions defined by the new thresholds:
\begin{align*}
    \hat{x}_i^{(n)}=\mathbb{E}\left[ \bm{X} \big| \bm{X}\in [\tau_{i-1}^{(n)},\tau_i^{(n)}] \right] \text{ for } i=1\ldots 2^B
\end{align*}
where to satisfy boundary conditions we have $\tau_0=-\infty$ and $\tau_{2^B}=\infty$. The algorithm iterates the above steps until convergence.

Figure \ref{fig:lm_quant} compares the quantization levels of a $7$-bit floating point (E3M3) quantizer (left) to a $7$-bit Lloyd-Max quantizer (right) when quantizing a layer of weights from the GPT3-126M model at a per-tensor granularity. As shown, the Lloyd-Max quantizer achieves substantially lower quantization MSE. Further, Table \ref{tab:FP7_vs_LM7} shows the superior perplexity achieved by Lloyd-Max quantizers for bitwidths of $7$, $6$ and $5$. The difference between the quantizers is clear at 5 bits, where per-tensor FP quantization incurs a drastic and unacceptable increase in perplexity, while Lloyd-Max quantization incurs a much smaller increase. Nevertheless, we note that even the optimal Lloyd-Max quantizer incurs a notable ($\sim 1.5$) increase in perplexity due to the coarse granularity of quantization. 

\begin{figure}[h]
  \centering
  \includegraphics[width=0.7\linewidth]{sections/figures/LM7_FP7.pdf}
  \caption{\small Quantization levels and the corresponding quantization MSE of Floating Point (left) vs Lloyd-Max (right) Quantizers for a layer of weights in the GPT3-126M model.}
  \label{fig:lm_quant}
\end{figure}

\begin{table}[h]\scriptsize
\begin{center}
\caption{\label{tab:FP7_vs_LM7} \small Comparing perplexity (lower is better) achieved by floating point quantizers and Lloyd-Max quantizers on a GPT3-126M model for the Wikitext-103 dataset.}
\begin{tabular}{c|cc|c}
\hline
 \multirow{2}{*}{\textbf{Bitwidth}} & \multicolumn{2}{|c|}{\textbf{Floating-Point Quantizer}} & \textbf{Lloyd-Max Quantizer} \\
 & Best Format & Wikitext-103 Perplexity & Wikitext-103 Perplexity \\
\hline
7 & E3M3 & 18.32 & 18.27 \\
6 & E3M2 & 19.07 & 18.51 \\
5 & E4M0 & 43.89 & 19.71 \\
\hline
\end{tabular}
\end{center}
\end{table}

\subsection{Proof of Local Optimality of LO-BCQ}
\label{subsec:lobcq_opt_proof}
For a given block $\bm{b}_j$, the quantization MSE during LO-BCQ can be empirically evaluated as $\frac{1}{L_b}\lVert \bm{b}_j- \bm{\hat{b}}_j\rVert^2_2$ where $\bm{\hat{b}}_j$ is computed from equation (\ref{eq:clustered_quantization_definition}) as $C_{f(\bm{b}_j)}(\bm{b}_j)$. Further, for a given block cluster $\mathcal{B}_i$, we compute the quantization MSE as $\frac{1}{|\mathcal{B}_{i}|}\sum_{\bm{b} \in \mathcal{B}_{i}} \frac{1}{L_b}\lVert \bm{b}- C_i^{(n)}(\bm{b})\rVert^2_2$. Therefore, at the end of iteration $n$, we evaluate the overall quantization MSE $J^{(n)}$ for a given operand $\bm{X}$ composed of $N_c$ block clusters as:
\begin{align*}
    \label{eq:mse_iter_n}
    J^{(n)} = \frac{1}{N_c} \sum_{i=1}^{N_c} \frac{1}{|\mathcal{B}_{i}^{(n)}|}\sum_{\bm{v} \in \mathcal{B}_{i}^{(n)}} \frac{1}{L_b}\lVert \bm{b}- B_i^{(n)}(\bm{b})\rVert^2_2
\end{align*}

At the end of iteration $n$, the codebooks are updated from $\mathcal{C}^{(n-1)}$ to $\mathcal{C}^{(n)}$. However, the mapping of a given vector $\bm{b}_j$ to quantizers $\mathcal{C}^{(n)}$ remains as  $f^{(n)}(\bm{b}_j)$. At the next iteration, during the vector clustering step, $f^{(n+1)}(\bm{b}_j)$ finds new mapping of $\bm{b}_j$ to updated codebooks $\mathcal{C}^{(n)}$ such that the quantization MSE over the candidate codebooks is minimized. Therefore, we obtain the following result for $\bm{b}_j$:
\begin{align*}
\frac{1}{L_b}\lVert \bm{b}_j - C_{f^{(n+1)}(\bm{b}_j)}^{(n)}(\bm{b}_j)\rVert^2_2 \le \frac{1}{L_b}\lVert \bm{b}_j - C_{f^{(n)}(\bm{b}_j)}^{(n)}(\bm{b}_j)\rVert^2_2
\end{align*}

That is, quantizing $\bm{b}_j$ at the end of the block clustering step of iteration $n+1$ results in lower quantization MSE compared to quantizing at the end of iteration $n$. Since this is true for all $\bm{b} \in \bm{X}$, we assert the following:
\begin{equation}
\begin{split}
\label{eq:mse_ineq_1}
    \tilde{J}^{(n+1)} &= \frac{1}{N_c} \sum_{i=1}^{N_c} \frac{1}{|\mathcal{B}_{i}^{(n+1)}|}\sum_{\bm{b} \in \mathcal{B}_{i}^{(n+1)}} \frac{1}{L_b}\lVert \bm{b} - C_i^{(n)}(b)\rVert^2_2 \le J^{(n)}
\end{split}
\end{equation}
where $\tilde{J}^{(n+1)}$ is the the quantization MSE after the vector clustering step at iteration $n+1$.

Next, during the codebook update step (\ref{eq:quantizers_update}) at iteration $n+1$, the per-cluster codebooks $\mathcal{C}^{(n)}$ are updated to $\mathcal{C}^{(n+1)}$ by invoking the Lloyd-Max algorithm \citep{Lloyd}. We know that for any given value distribution, the Lloyd-Max algorithm minimizes the quantization MSE. Therefore, for a given vector cluster $\mathcal{B}_i$ we obtain the following result:

\begin{equation}
    \frac{1}{|\mathcal{B}_{i}^{(n+1)}|}\sum_{\bm{b} \in \mathcal{B}_{i}^{(n+1)}} \frac{1}{L_b}\lVert \bm{b}- C_i^{(n+1)}(\bm{b})\rVert^2_2 \le \frac{1}{|\mathcal{B}_{i}^{(n+1)}|}\sum_{\bm{b} \in \mathcal{B}_{i}^{(n+1)}} \frac{1}{L_b}\lVert \bm{b}- C_i^{(n)}(\bm{b})\rVert^2_2
\end{equation}

The above equation states that quantizing the given block cluster $\mathcal{B}_i$ after updating the associated codebook from $C_i^{(n)}$ to $C_i^{(n+1)}$ results in lower quantization MSE. Since this is true for all the block clusters, we derive the following result: 
\begin{equation}
\begin{split}
\label{eq:mse_ineq_2}
     J^{(n+1)} &= \frac{1}{N_c} \sum_{i=1}^{N_c} \frac{1}{|\mathcal{B}_{i}^{(n+1)}|}\sum_{\bm{b} \in \mathcal{B}_{i}^{(n+1)}} \frac{1}{L_b}\lVert \bm{b}- C_i^{(n+1)}(\bm{b})\rVert^2_2  \le \tilde{J}^{(n+1)}   
\end{split}
\end{equation}

Following (\ref{eq:mse_ineq_1}) and (\ref{eq:mse_ineq_2}), we find that the quantization MSE is non-increasing for each iteration, that is, $J^{(1)} \ge J^{(2)} \ge J^{(3)} \ge \ldots \ge J^{(M)}$ where $M$ is the maximum number of iterations. 
%Therefore, we can say that if the algorithm converges, then it must be that it has converged to a local minimum. 
\hfill $\blacksquare$


\begin{figure}
    \begin{center}
    \includegraphics[width=0.5\textwidth]{sections//figures/mse_vs_iter.pdf}
    \end{center}
    \caption{\small NMSE vs iterations during LO-BCQ compared to other block quantization proposals}
    \label{fig:nmse_vs_iter}
\end{figure}

Figure \ref{fig:nmse_vs_iter} shows the empirical convergence of LO-BCQ across several block lengths and number of codebooks. Also, the MSE achieved by LO-BCQ is compared to baselines such as MXFP and VSQ. As shown, LO-BCQ converges to a lower MSE than the baselines. Further, we achieve better convergence for larger number of codebooks ($N_c$) and for a smaller block length ($L_b$), both of which increase the bitwidth of BCQ (see Eq \ref{eq:bitwidth_bcq}).


\subsection{Additional Accuracy Results}
%Table \ref{tab:lobcq_config} lists the various LOBCQ configurations and their corresponding bitwidths.
\begin{table}
\setlength{\tabcolsep}{4.75pt}
\begin{center}
\caption{\label{tab:lobcq_config} Various LO-BCQ configurations and their bitwidths.}
\begin{tabular}{|c||c|c|c|c||c|c||c|} 
\hline
 & \multicolumn{4}{|c||}{$L_b=8$} & \multicolumn{2}{|c||}{$L_b=4$} & $L_b=2$ \\
 \hline
 \backslashbox{$L_A$\kern-1em}{\kern-1em$N_c$} & 2 & 4 & 8 & 16 & 2 & 4 & 2 \\
 \hline
 64 & 4.25 & 4.375 & 4.5 & 4.625 & 4.375 & 4.625 & 4.625\\
 \hline
 32 & 4.375 & 4.5 & 4.625& 4.75 & 4.5 & 4.75 & 4.75 \\
 \hline
 16 & 4.625 & 4.75& 4.875 & 5 & 4.75 & 5 & 5 \\
 \hline
\end{tabular}
\end{center}
\end{table}

%\subsection{Perplexity achieved by various LO-BCQ configurations on Wikitext-103 dataset}

\begin{table} \centering
\begin{tabular}{|c||c|c|c|c||c|c||c|} 
\hline
 $L_b \rightarrow$& \multicolumn{4}{c||}{8} & \multicolumn{2}{c||}{4} & 2\\
 \hline
 \backslashbox{$L_A$\kern-1em}{\kern-1em$N_c$} & 2 & 4 & 8 & 16 & 2 & 4 & 2  \\
 %$N_c \rightarrow$ & 2 & 4 & 8 & 16 & 2 & 4 & 2 \\
 \hline
 \hline
 \multicolumn{8}{c}{GPT3-1.3B (FP32 PPL = 9.98)} \\ 
 \hline
 \hline
 64 & 10.40 & 10.23 & 10.17 & 10.15 &  10.28 & 10.18 & 10.19 \\
 \hline
 32 & 10.25 & 10.20 & 10.15 & 10.12 &  10.23 & 10.17 & 10.17 \\
 \hline
 16 & 10.22 & 10.16 & 10.10 & 10.09 &  10.21 & 10.14 & 10.16 \\
 \hline
  \hline
 \multicolumn{8}{c}{GPT3-8B (FP32 PPL = 7.38)} \\ 
 \hline
 \hline
 64 & 7.61 & 7.52 & 7.48 &  7.47 &  7.55 &  7.49 & 7.50 \\
 \hline
 32 & 7.52 & 7.50 & 7.46 &  7.45 &  7.52 &  7.48 & 7.48  \\
 \hline
 16 & 7.51 & 7.48 & 7.44 &  7.44 &  7.51 &  7.49 & 7.47  \\
 \hline
\end{tabular}
\caption{\label{tab:ppl_gpt3_abalation} Wikitext-103 perplexity across GPT3-1.3B and 8B models.}
\end{table}

\begin{table} \centering
\begin{tabular}{|c||c|c|c|c||} 
\hline
 $L_b \rightarrow$& \multicolumn{4}{c||}{8}\\
 \hline
 \backslashbox{$L_A$\kern-1em}{\kern-1em$N_c$} & 2 & 4 & 8 & 16 \\
 %$N_c \rightarrow$ & 2 & 4 & 8 & 16 & 2 & 4 & 2 \\
 \hline
 \hline
 \multicolumn{5}{|c|}{Llama2-7B (FP32 PPL = 5.06)} \\ 
 \hline
 \hline
 64 & 5.31 & 5.26 & 5.19 & 5.18  \\
 \hline
 32 & 5.23 & 5.25 & 5.18 & 5.15  \\
 \hline
 16 & 5.23 & 5.19 & 5.16 & 5.14  \\
 \hline
 \multicolumn{5}{|c|}{Nemotron4-15B (FP32 PPL = 5.87)} \\ 
 \hline
 \hline
 64  & 6.3 & 6.20 & 6.13 & 6.08  \\
 \hline
 32  & 6.24 & 6.12 & 6.07 & 6.03  \\
 \hline
 16  & 6.12 & 6.14 & 6.04 & 6.02  \\
 \hline
 \multicolumn{5}{|c|}{Nemotron4-340B (FP32 PPL = 3.48)} \\ 
 \hline
 \hline
 64 & 3.67 & 3.62 & 3.60 & 3.59 \\
 \hline
 32 & 3.63 & 3.61 & 3.59 & 3.56 \\
 \hline
 16 & 3.61 & 3.58 & 3.57 & 3.55 \\
 \hline
\end{tabular}
\caption{\label{tab:ppl_llama7B_nemo15B} Wikitext-103 perplexity compared to FP32 baseline in Llama2-7B and Nemotron4-15B, 340B models}
\end{table}

%\subsection{Perplexity achieved by various LO-BCQ configurations on MMLU dataset}


\begin{table} \centering
\begin{tabular}{|c||c|c|c|c||c|c|c|c|} 
\hline
 $L_b \rightarrow$& \multicolumn{4}{c||}{8} & \multicolumn{4}{c||}{8}\\
 \hline
 \backslashbox{$L_A$\kern-1em}{\kern-1em$N_c$} & 2 & 4 & 8 & 16 & 2 & 4 & 8 & 16  \\
 %$N_c \rightarrow$ & 2 & 4 & 8 & 16 & 2 & 4 & 2 \\
 \hline
 \hline
 \multicolumn{5}{|c|}{Llama2-7B (FP32 Accuracy = 45.8\%)} & \multicolumn{4}{|c|}{Llama2-70B (FP32 Accuracy = 69.12\%)} \\ 
 \hline
 \hline
 64 & 43.9 & 43.4 & 43.9 & 44.9 & 68.07 & 68.27 & 68.17 & 68.75 \\
 \hline
 32 & 44.5 & 43.8 & 44.9 & 44.5 & 68.37 & 68.51 & 68.35 & 68.27  \\
 \hline
 16 & 43.9 & 42.7 & 44.9 & 45 & 68.12 & 68.77 & 68.31 & 68.59  \\
 \hline
 \hline
 \multicolumn{5}{|c|}{GPT3-22B (FP32 Accuracy = 38.75\%)} & \multicolumn{4}{|c|}{Nemotron4-15B (FP32 Accuracy = 64.3\%)} \\ 
 \hline
 \hline
 64 & 36.71 & 38.85 & 38.13 & 38.92 & 63.17 & 62.36 & 63.72 & 64.09 \\
 \hline
 32 & 37.95 & 38.69 & 39.45 & 38.34 & 64.05 & 62.30 & 63.8 & 64.33  \\
 \hline
 16 & 38.88 & 38.80 & 38.31 & 38.92 & 63.22 & 63.51 & 63.93 & 64.43  \\
 \hline
\end{tabular}
\caption{\label{tab:mmlu_abalation} Accuracy on MMLU dataset across GPT3-22B, Llama2-7B, 70B and Nemotron4-15B models.}
\end{table}


%\subsection{Perplexity achieved by various LO-BCQ configurations on LM evaluation harness}

\begin{table} \centering
\begin{tabular}{|c||c|c|c|c||c|c|c|c|} 
\hline
 $L_b \rightarrow$& \multicolumn{4}{c||}{8} & \multicolumn{4}{c||}{8}\\
 \hline
 \backslashbox{$L_A$\kern-1em}{\kern-1em$N_c$} & 2 & 4 & 8 & 16 & 2 & 4 & 8 & 16  \\
 %$N_c \rightarrow$ & 2 & 4 & 8 & 16 & 2 & 4 & 2 \\
 \hline
 \hline
 \multicolumn{5}{|c|}{Race (FP32 Accuracy = 37.51\%)} & \multicolumn{4}{|c|}{Boolq (FP32 Accuracy = 64.62\%)} \\ 
 \hline
 \hline
 64 & 36.94 & 37.13 & 36.27 & 37.13 & 63.73 & 62.26 & 63.49 & 63.36 \\
 \hline
 32 & 37.03 & 36.36 & 36.08 & 37.03 & 62.54 & 63.51 & 63.49 & 63.55  \\
 \hline
 16 & 37.03 & 37.03 & 36.46 & 37.03 & 61.1 & 63.79 & 63.58 & 63.33  \\
 \hline
 \hline
 \multicolumn{5}{|c|}{Winogrande (FP32 Accuracy = 58.01\%)} & \multicolumn{4}{|c|}{Piqa (FP32 Accuracy = 74.21\%)} \\ 
 \hline
 \hline
 64 & 58.17 & 57.22 & 57.85 & 58.33 & 73.01 & 73.07 & 73.07 & 72.80 \\
 \hline
 32 & 59.12 & 58.09 & 57.85 & 58.41 & 73.01 & 73.94 & 72.74 & 73.18  \\
 \hline
 16 & 57.93 & 58.88 & 57.93 & 58.56 & 73.94 & 72.80 & 73.01 & 73.94  \\
 \hline
\end{tabular}
\caption{\label{tab:mmlu_abalation} Accuracy on LM evaluation harness tasks on GPT3-1.3B model.}
\end{table}

\begin{table} \centering
\begin{tabular}{|c||c|c|c|c||c|c|c|c|} 
\hline
 $L_b \rightarrow$& \multicolumn{4}{c||}{8} & \multicolumn{4}{c||}{8}\\
 \hline
 \backslashbox{$L_A$\kern-1em}{\kern-1em$N_c$} & 2 & 4 & 8 & 16 & 2 & 4 & 8 & 16  \\
 %$N_c \rightarrow$ & 2 & 4 & 8 & 16 & 2 & 4 & 2 \\
 \hline
 \hline
 \multicolumn{5}{|c|}{Race (FP32 Accuracy = 41.34\%)} & \multicolumn{4}{|c|}{Boolq (FP32 Accuracy = 68.32\%)} \\ 
 \hline
 \hline
 64 & 40.48 & 40.10 & 39.43 & 39.90 & 69.20 & 68.41 & 69.45 & 68.56 \\
 \hline
 32 & 39.52 & 39.52 & 40.77 & 39.62 & 68.32 & 67.43 & 68.17 & 69.30  \\
 \hline
 16 & 39.81 & 39.71 & 39.90 & 40.38 & 68.10 & 66.33 & 69.51 & 69.42  \\
 \hline
 \hline
 \multicolumn{5}{|c|}{Winogrande (FP32 Accuracy = 67.88\%)} & \multicolumn{4}{|c|}{Piqa (FP32 Accuracy = 78.78\%)} \\ 
 \hline
 \hline
 64 & 66.85 & 66.61 & 67.72 & 67.88 & 77.31 & 77.42 & 77.75 & 77.64 \\
 \hline
 32 & 67.25 & 67.72 & 67.72 & 67.00 & 77.31 & 77.04 & 77.80 & 77.37  \\
 \hline
 16 & 68.11 & 68.90 & 67.88 & 67.48 & 77.37 & 78.13 & 78.13 & 77.69  \\
 \hline
\end{tabular}
\caption{\label{tab:mmlu_abalation} Accuracy on LM evaluation harness tasks on GPT3-8B model.}
\end{table}

\begin{table} \centering
\begin{tabular}{|c||c|c|c|c||c|c|c|c|} 
\hline
 $L_b \rightarrow$& \multicolumn{4}{c||}{8} & \multicolumn{4}{c||}{8}\\
 \hline
 \backslashbox{$L_A$\kern-1em}{\kern-1em$N_c$} & 2 & 4 & 8 & 16 & 2 & 4 & 8 & 16  \\
 %$N_c \rightarrow$ & 2 & 4 & 8 & 16 & 2 & 4 & 2 \\
 \hline
 \hline
 \multicolumn{5}{|c|}{Race (FP32 Accuracy = 40.67\%)} & \multicolumn{4}{|c|}{Boolq (FP32 Accuracy = 76.54\%)} \\ 
 \hline
 \hline
 64 & 40.48 & 40.10 & 39.43 & 39.90 & 75.41 & 75.11 & 77.09 & 75.66 \\
 \hline
 32 & 39.52 & 39.52 & 40.77 & 39.62 & 76.02 & 76.02 & 75.96 & 75.35  \\
 \hline
 16 & 39.81 & 39.71 & 39.90 & 40.38 & 75.05 & 73.82 & 75.72 & 76.09  \\
 \hline
 \hline
 \multicolumn{5}{|c|}{Winogrande (FP32 Accuracy = 70.64\%)} & \multicolumn{4}{|c|}{Piqa (FP32 Accuracy = 79.16\%)} \\ 
 \hline
 \hline
 64 & 69.14 & 70.17 & 70.17 & 70.56 & 78.24 & 79.00 & 78.62 & 78.73 \\
 \hline
 32 & 70.96 & 69.69 & 71.27 & 69.30 & 78.56 & 79.49 & 79.16 & 78.89  \\
 \hline
 16 & 71.03 & 69.53 & 69.69 & 70.40 & 78.13 & 79.16 & 79.00 & 79.00  \\
 \hline
\end{tabular}
\caption{\label{tab:mmlu_abalation} Accuracy on LM evaluation harness tasks on GPT3-22B model.}
\end{table}

\begin{table} \centering
\begin{tabular}{|c||c|c|c|c||c|c|c|c|} 
\hline
 $L_b \rightarrow$& \multicolumn{4}{c||}{8} & \multicolumn{4}{c||}{8}\\
 \hline
 \backslashbox{$L_A$\kern-1em}{\kern-1em$N_c$} & 2 & 4 & 8 & 16 & 2 & 4 & 8 & 16  \\
 %$N_c \rightarrow$ & 2 & 4 & 8 & 16 & 2 & 4 & 2 \\
 \hline
 \hline
 \multicolumn{5}{|c|}{Race (FP32 Accuracy = 44.4\%)} & \multicolumn{4}{|c|}{Boolq (FP32 Accuracy = 79.29\%)} \\ 
 \hline
 \hline
 64 & 42.49 & 42.51 & 42.58 & 43.45 & 77.58 & 77.37 & 77.43 & 78.1 \\
 \hline
 32 & 43.35 & 42.49 & 43.64 & 43.73 & 77.86 & 75.32 & 77.28 & 77.86  \\
 \hline
 16 & 44.21 & 44.21 & 43.64 & 42.97 & 78.65 & 77 & 76.94 & 77.98  \\
 \hline
 \hline
 \multicolumn{5}{|c|}{Winogrande (FP32 Accuracy = 69.38\%)} & \multicolumn{4}{|c|}{Piqa (FP32 Accuracy = 78.07\%)} \\ 
 \hline
 \hline
 64 & 68.9 & 68.43 & 69.77 & 68.19 & 77.09 & 76.82 & 77.09 & 77.86 \\
 \hline
 32 & 69.38 & 68.51 & 68.82 & 68.90 & 78.07 & 76.71 & 78.07 & 77.86  \\
 \hline
 16 & 69.53 & 67.09 & 69.38 & 68.90 & 77.37 & 77.8 & 77.91 & 77.69  \\
 \hline
\end{tabular}
\caption{\label{tab:mmlu_abalation} Accuracy on LM evaluation harness tasks on Llama2-7B model.}
\end{table}

\begin{table} \centering
\begin{tabular}{|c||c|c|c|c||c|c|c|c|} 
\hline
 $L_b \rightarrow$& \multicolumn{4}{c||}{8} & \multicolumn{4}{c||}{8}\\
 \hline
 \backslashbox{$L_A$\kern-1em}{\kern-1em$N_c$} & 2 & 4 & 8 & 16 & 2 & 4 & 8 & 16  \\
 %$N_c \rightarrow$ & 2 & 4 & 8 & 16 & 2 & 4 & 2 \\
 \hline
 \hline
 \multicolumn{5}{|c|}{Race (FP32 Accuracy = 48.8\%)} & \multicolumn{4}{|c|}{Boolq (FP32 Accuracy = 85.23\%)} \\ 
 \hline
 \hline
 64 & 49.00 & 49.00 & 49.28 & 48.71 & 82.82 & 84.28 & 84.03 & 84.25 \\
 \hline
 32 & 49.57 & 48.52 & 48.33 & 49.28 & 83.85 & 84.46 & 84.31 & 84.93  \\
 \hline
 16 & 49.85 & 49.09 & 49.28 & 48.99 & 85.11 & 84.46 & 84.61 & 83.94  \\
 \hline
 \hline
 \multicolumn{5}{|c|}{Winogrande (FP32 Accuracy = 79.95\%)} & \multicolumn{4}{|c|}{Piqa (FP32 Accuracy = 81.56\%)} \\ 
 \hline
 \hline
 64 & 78.77 & 78.45 & 78.37 & 79.16 & 81.45 & 80.69 & 81.45 & 81.5 \\
 \hline
 32 & 78.45 & 79.01 & 78.69 & 80.66 & 81.56 & 80.58 & 81.18 & 81.34  \\
 \hline
 16 & 79.95 & 79.56 & 79.79 & 79.72 & 81.28 & 81.66 & 81.28 & 80.96  \\
 \hline
\end{tabular}
\caption{\label{tab:mmlu_abalation} Accuracy on LM evaluation harness tasks on Llama2-70B model.}
\end{table}

%\section{MSE Studies}
%\textcolor{red}{TODO}


\subsection{Number Formats and Quantization Method}
\label{subsec:numFormats_quantMethod}
\subsubsection{Integer Format}
An $n$-bit signed integer (INT) is typically represented with a 2s-complement format \citep{yao2022zeroquant,xiao2023smoothquant,dai2021vsq}, where the most significant bit denotes the sign.

\subsubsection{Floating Point Format}
An $n$-bit signed floating point (FP) number $x$ comprises of a 1-bit sign ($x_{\mathrm{sign}}$), $B_m$-bit mantissa ($x_{\mathrm{mant}}$) and $B_e$-bit exponent ($x_{\mathrm{exp}}$) such that $B_m+B_e=n-1$. The associated constant exponent bias ($E_{\mathrm{bias}}$) is computed as $(2^{{B_e}-1}-1)$. We denote this format as $E_{B_e}M_{B_m}$.  

\subsubsection{Quantization Scheme}
\label{subsec:quant_method}
A quantization scheme dictates how a given unquantized tensor is converted to its quantized representation. We consider FP formats for the purpose of illustration. Given an unquantized tensor $\bm{X}$ and an FP format $E_{B_e}M_{B_m}$, we first, we compute the quantization scale factor $s_X$ that maps the maximum absolute value of $\bm{X}$ to the maximum quantization level of the $E_{B_e}M_{B_m}$ format as follows:
\begin{align}
\label{eq:sf}
    s_X = \frac{\mathrm{max}(|\bm{X}|)}{\mathrm{max}(E_{B_e}M_{B_m})}
\end{align}
In the above equation, $|\cdot|$ denotes the absolute value function.

Next, we scale $\bm{X}$ by $s_X$ and quantize it to $\hat{\bm{X}}$ by rounding it to the nearest quantization level of $E_{B_e}M_{B_m}$ as:

\begin{align}
\label{eq:tensor_quant}
    \hat{\bm{X}} = \text{round-to-nearest}\left(\frac{\bm{X}}{s_X}, E_{B_e}M_{B_m}\right)
\end{align}

We perform dynamic max-scaled quantization \citep{wu2020integer}, where the scale factor $s$ for activations is dynamically computed during runtime.

\subsection{Vector Scaled Quantization}
\begin{wrapfigure}{r}{0.35\linewidth}
  \centering
  \includegraphics[width=\linewidth]{sections/figures/vsquant.jpg}
  \caption{\small Vectorwise decomposition for per-vector scaled quantization (VSQ \citep{dai2021vsq}).}
  \label{fig:vsquant}
\end{wrapfigure}
During VSQ \citep{dai2021vsq}, the operand tensors are decomposed into 1D vectors in a hardware friendly manner as shown in Figure \ref{fig:vsquant}. Since the decomposed tensors are used as operands in matrix multiplications during inference, it is beneficial to perform this decomposition along the reduction dimension of the multiplication. The vectorwise quantization is performed similar to tensorwise quantization described in Equations \ref{eq:sf} and \ref{eq:tensor_quant}, where a scale factor $s_v$ is required for each vector $\bm{v}$ that maps the maximum absolute value of that vector to the maximum quantization level. While smaller vector lengths can lead to larger accuracy gains, the associated memory and computational overheads due to the per-vector scale factors increases. To alleviate these overheads, VSQ \citep{dai2021vsq} proposed a second level quantization of the per-vector scale factors to unsigned integers, while MX \citep{rouhani2023shared} quantizes them to integer powers of 2 (denoted as $2^{INT}$).

\subsubsection{MX Format}
The MX format proposed in \citep{rouhani2023microscaling} introduces the concept of sub-block shifting. For every two scalar elements of $b$-bits each, there is a shared exponent bit. The value of this exponent bit is determined through an empirical analysis that targets minimizing quantization MSE. We note that the FP format $E_{1}M_{b}$ is strictly better than MX from an accuracy perspective since it allocates a dedicated exponent bit to each scalar as opposed to sharing it across two scalars. Therefore, we conservatively bound the accuracy of a $b+2$-bit signed MX format with that of a $E_{1}M_{b}$ format in our comparisons. For instance, we use E1M2 format as a proxy for MX4.

\begin{figure}
    \centering
    \includegraphics[width=1\linewidth]{sections//figures/BlockFormats.pdf}
    \caption{\small Comparing LO-BCQ to MX format.}
    \label{fig:block_formats}
\end{figure}

Figure \ref{fig:block_formats} compares our $4$-bit LO-BCQ block format to MX \citep{rouhani2023microscaling}. As shown, both LO-BCQ and MX decompose a given operand tensor into block arrays and each block array into blocks. Similar to MX, we find that per-block quantization ($L_b < L_A$) leads to better accuracy due to increased flexibility. While MX achieves this through per-block $1$-bit micro-scales, we associate a dedicated codebook to each block through a per-block codebook selector. Further, MX quantizes the per-block array scale-factor to E8M0 format without per-tensor scaling. In contrast during LO-BCQ, we find that per-tensor scaling combined with quantization of per-block array scale-factor to E4M3 format results in superior inference accuracy across models. 


\end{document}
