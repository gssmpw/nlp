\section{Experiments for RQ1}
\label{sec:primitive}
\underline{\textbf{RQ1}}: \emph{What is the impact of SDCs on Transformer submodule computation outputs?}

In this section, we first introduce our experiment setups for RQ1 and analyze experimental results to understand the impact of SDCs on submodule computation. We follow the same structure for RQ2 in  Section \ref{sec:sdc_single_optimizer_step} and RQ3 in Section \ref{sec:multiple_training_steps}.

\subsection{Experiment Setups}
\label{methods:primitive_impact}
We focus on four kinds of Transformer submodule computation, namely the forward computation of a self-attention module (FWD/ATTN) and an FFN module (FWD/FFN), and the backward computation of a self-attention module (BWD/ATTN) and an FFN module (BWD/FFN). We train two models on each pair of the unhealthy node and the healthy node simultaneously and use the \textbf{computation synchronization} mechanism discussed in Section \ref{sec:research_questions}. We use a decoder-only Transformer architecture similar to the Llama3-8B configuration \cite{dubey2024llama3herdmodels} with $D=16$ decoder layers and hidden state size of $H=4096$ and use the tensor parallelism to fit a model within a node. More details can be found in Appendix \ref{appendix:primivite_investigation}.

For a submodule computation $f$ in the model, we define $f'_{i}(x_{t,j})$ as the tensor computed on TP rank $t$ of unhealthy node $i$ at the microstep $j$ and $f(x_{t,j})$ as the corresponding output on healthy node. To quantify differences between $f'_{i}(x_{t,j})$ and $f(x_{t,j})$, we define two metrics called \emph{mismatch frequency} and \emph{mismatch severity}. We calculate the mismatch frequency for submodule $f$ on unhealthy node $i$ at the microstep $j$ as follows:
\begin{equation}
    %\label{equation:mismatch_frequency}
    freq^{f}_{i,j} = \frac{\sum_{t=1}^{TP}{Mis(f'_{i}(x_{t,j}), f(x_{t,j}))}}{TP \cdot MBS \cdot L\cdot H}
\end{equation}
where $Mis(y', y)$ counts the number of mismatching elements in two tensors $y$ and $y'$. We report the aggregated mismatch frequency for each submodule computation type $F$ on unhealthy node $i$ at the microstep $j$ by averaging across decoder layers:
\begin{equation}
    \label{equation:mismatch_frequency}
    freq^{F}_{i,j} = \frac{1}{D}\sum_{f \in F}{freq^{f}_{i,j}}, F=\{f^{(1)},...,f^{(D)}\}
\end{equation} 
Mismatch severity is defined as the average over non-zero values of the element-wise relative difference. Formally, we take the maximum over all TP ranks and calculate the mismatch severity for submodule $f$ on unhealthy node $i$ at microstep $j$:
\begin{equation}
    %\label{equation:mismatch_severity}
    sev^{f}_{i,j} = \max_{0 \leq t < TP}{\left[
        {Avg_{\neq 0}}\left({
            \left|
                \frac{f'_{i}(x_{t,j}) - f(x_{t,j})  }{f(x_{t,j})}
            \right|
        }\right)
    \right]}
\end{equation}
where $Avg_{\neq 0}(x)$ computes the average value over only non-zero elements of a tensor $x$. We also calculate the mismatch severity for each type of submodule computation $F$ on unhealthy node $i$ at the microstep $j$ by taking the maximum across decoder layers as follows:
\begin{equation}
    \label{equation:mismatch_severity_max_decoder}
    sev^{F}_{i,j} = \max_{f \in F}{sev^{f}_{i,j}}, F=\{f^{(1)},...,f^{(D)}\}
\end{equation}


\subsection{Results} \label{sec:submodule_outputs_results}
Table \ref{fig:primitive_frequency_summarized} shows the mismatch frequency of submodule computation.
We observe that the impact of SDCs on submodule computation varies across different unhealthy nodes, e.g. Nodes 10 and 11 have a high mismatch frequency while Nodes 2 and 3 do not show any SDC occurrence in this setting. 

\begin{table}[t]
\begin{center}
\begin{small}
\begin{sc}
\setlength{\tabcolsep}{3pt}
\renewcommand{\arraystretch}{1.1} 
\begin{tabular}{crrrr}
\toprule
Node ID & fwd/attn & fwd/ffn & bwd/attn & bwd/ffn  \\ \midrule
Node 1  & 1.55$e$-5  & 5.06$e$-7  & 1.56$e$-04  & 2.81$e$-6   \\
Node 4  & 3.79$e$-9  & 9.20$e$-11  & 2.99$e$-9  & 2.80$e$-11  \\
Node 5  & 0  & 0  & 1.49$e$-15  & 1.25$e$-12  \\
Node 6  & 1.71$e$-9  & 1.64$e$-11  & 1.49$e$-9  & 6.02$e$-11   \\
Node 7  & 2.13$e$-6  & 1.18$e$-7  & 4.31$e$-6  & 6.73$e$-8   \\
Node 8  & 3.21$e$-9  & 1.99$e$-11   & 1.01$e$-7  & 2.21$e$-9   \\
Node 9  & 1.10$e$-5  & 5.05$e$-7   & 4.33$e$-6  & 3.86$e$-8  \\
Node 10 & 4.78$e$-3  & 1.03$e$-3    & 1.92$e$-3  & 7.98$e$-5   \\
Node 11 & 2.89$e$-2  & 2.25$e$-3  & 6.71$e$-3  & 1.08$e$-4   \\
Node 13 & 0  & 0   & 0  & 1.21$e$-10  \\
Node 14 & 6.48$e$-11  & 0   & 4.91$e$-10  & 2.99$e$-9   \\
Node 15 & 0  & 0   & 0  & 7.39$e$-15  \\
\bottomrule
\end{tabular}
\end{sc}
\end{small}
\end{center}
\vskip -0.1in
\caption{Average mismatch frequency over microsteps for Transformer submodules. The table excludes the nodes that do not show any SDC in this setting.}
\vskip -0.1in
\label{fig:primitive_frequency_summarized}
\end{table}

We further find that SDCs do not occur uniformly over time: mismatch frequency often has a large variance across steps. Figure \ref{fig:attention_forward_mismatch_frequency_over_time} shows the mismatch frequency in the forward computation of the attention module on Nodes 7 and 14. We find that spikes of mismatch frequency sometimes occur, while during the majority of training time, no mismatch occurs. In Figure \ref{fig:primitive_spike_at_beginning}, we observe a high peak of mismatch frequency at the first few steps on Nodes 10 and 11, likely due to higher overall system usage when initializing the training run. The non-uniform occurrence of SDC during training suggests that SDCs might be caused by broader, compound system-level factors.

\begin{figure}[t!]
    \centering
    \includegraphics[width=0.85\linewidth]{body/figures/attention_forward_mismatch_frequency_over_step_1.png}
    \vskip -0.15in
    \caption{Non-uniform spikes of mismatch frequency in the forward computation of the attention module over time on Node 7, 14.}
    \label{fig:attention_forward_mismatch_frequency_over_time}
\vskip -0.15in
\end{figure}

\begin{figure}[t]
    \vskip 0.1in
    \centering
    \includegraphics[width=0.85\linewidth]{body/figures/attention_forward_mismatch_frequency_over_step_3.png}
    \vskip -0.1in
    \caption{High SDC occurrence with large initial spikes in smoothed mismatch frequency for the forward computation of the attention module on Node 10, 11.}
    \label{fig:primitive_spike_at_beginning}
    \vskip -0.1in
\end{figure}

Table \ref{fig:primitive_severity_summarized}  shows the maximum mismatch severity over optimizer steps for submodule computation on different unhealthy nodes. We find that SDCs cause certain tensor values in the computation to differ by large factors. For example, on Node 9, the mismatch severity exceeds $100$, which means SDCs can cause degraded TP ranks to have very different computation results on certain microsteps.

\begin{table}[t]
\begin{center}
\begin{small}
\begin{sc}
\setlength{\tabcolsep}{3pt}
\renewcommand{\arraystretch}{1.1} 
\begin{tabular}{crrrr}
\toprule
Node ID & fwd/attn & fwd/ffn & bwd/attn & bwd/ffn \\
\midrule
Node 1 & 99 & 312 & 7392 & 3.88 \\
Node 4 & 2.95 & 1.46 & 2.39 & 5.38 \\
Node 5 & 0 & 0 & 0.0047 & 0.0908 \\
Node 6 & 1.01 & 1 & 0.162 & 0.0776 \\
Node 7 & 43.2 & 88.5 & 32.3 & 0.297 \\ 
Node 8 & 13.5 & 5.69 & 6.41 & 1.59 \\ 
Node 9 & 119 & 200 & 3.79$e$+11 & 9.63$e$+12 \\ 
Node 10 & 1120 & 262 & 12.75 & 0.648 \\
 Node 11 & 318 & 976 & 2208 & 7680 \\ 
Node 13 & 0 & 0 & 0 & 0.316 \\ 
Node 14 & 1.12 & 0 & 3.80 & 0.380 \\ 
Node 15 & 0 & 0 & 0 & 0.0176 \\ \bottomrule
\end{tabular}

\end{sc}
\end{small}
\end{center}
\vskip -0.1in
\caption{Maximum mismatch severity over microsteps for each unhealthy node. The table excludes the nodes that do not show any SDC in this setting.}
\vskip -0.1in
\label{fig:primitive_severity_summarized}

\end{table}

