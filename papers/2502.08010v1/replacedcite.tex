\section{Related Work}
In this paper, we propose to apply the EaaS platform to support large-scale D2D communications between RF-powered IoE devices. Using percolation theory, we prove the feasibility of such an EaaS platform and help operators estimate critical ES density, further, the required CAPEX. Therefore, we divide the related work into: i) percolation theory for wireless networks and ii) stochastic geometry for RF-powered wireless networks.


\indent \textit{Percolation theory for wireless networks:} Continuum percolation on RGGs has been widely utilized to study the large-scale network-wide performance of wireless networks with randomly located devices____. For instance, the authors in ____ utilized continuum percolation to quantify the impact of aggregate network interference on large-scale network connectivity. End-to-end capacity in large-scale networks was characterized in ____ via percolation theory. Using the concept of information secrecy, the authors in ____ found the relative densities of legitimate devices and eavesdroppers that enable large-scale private message dissemination among legitimate devices. In cognitive radio networks, the authors in ____ characterized the relationship between the intensity of secondary devices and primary devices to allow large-scale connectivity of the secondary network. Simultaneous large-scale connectivity of both the primary and secondary devices in cognitive networks was studied in ____. For sensing/monitoring applications, the path exposure problem was characterized in ____, where the authors found the critical density of sensors/cameras to detect moving objects over arbitrary paths through a given region. In the context of cyber-security, the authors in ____ utilized continuum percolation to quantify the required spatial firewalls to thwart large-scale outbreaks of malware worms in wireless networks. By investigating the BS networks assisted by reconfigurable intelligent surfaces (RISs), authors in ____ derived the lower bound of the critical density of RISs. In ____, based on dynamic bond percolation, authors proposed an evolution model to characterize the reliable topology evolution affected by the nodes and links states. However, WET-enabled large-scale connectivity of RF-powered IoE networks is still an open problem.\\
\indent \textit{Stochastic geometry for
RF-powered wireless networks:} RF-powered wireless communication has been extensively studied at the link-level performance as summarized in____. In large-scale RF-powered networks, stochastic geometry was utilized to provide spatial averages of link-level performance metrics such as coverage probability, rate, or delay. For instance, the work in ____ characterized the energy efficiency for PPP downlink cellular networks that are powered via renewable energy sources. Outage probability and rate of downlink simultaneous wireless information and power transfer (SWIPT) were studied in ____ for cooperative non-orthogonal multiple access (NOMA) and in ____ for cell-free massive multiple-input multiple-output (MIMO) networks. On the uplink side, the work in ____ characterized coverage probability with channel inversion power control. Delay and packet throughput for grant-free uplink access of RF-powered Internet of Things (IoT) was characterized in ____. Outage probability and self-sustainability of D2D RF-powered IoT networks were characterized in ____. End-to-end outage probability for RF-powered D2D decode-and-forward relaying was studied in ____. D2D-enabled proactive caching for RF-powered wireless networks was studied in ____. For non-terrestrial networks, the authors in ____ optimized the altitude of RF-powered aerial base stations serving terrestrial devices. A similar scenario was studied in ____ for laser-powered aerial base stations. Authors in ____ combined the backscatter communication with RF-powered cognitive networks, and investigated the resource allocation for channel selection and backscatter power allocation. In ____, authors evaluated the effect on RF-powered IoT network performance of RIS density, RIS reflecting element number, and charging stage ratio. However, none of the aforementioned stochastic geometry frameworks can be utilized to study network-wide connectivity of RF-powered IoE networks in order to help EaaS providers estimate the required intensity of ESs.