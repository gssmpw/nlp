\section{Related works}
The spatial structure of underwater sound velocity field plays a decisive role in the energy distribution and propagation trajectory of acoustic signals. Therefore, real-time and accurate estimation of sound velocity distribution is of great meaningful for applications based on communication and positioning technologies such as underwater PNTC systems, and target recognition systems.

The traditional way of obtaining sound velocity distribution is usually through direct measurement using shipborne CTD or SVP equipment, which has the advantage of high accuracy ____. However, it comes with high economic costs and measurement time expenses. The development of sensor network technologies has led to the implementation of multiple underwater environmental monitoring frameworks, advancing scientific comprehension of oceanic processes. The methodological framework of underwater acoustic inversion was initially formalized in 1979 by Munk et al.____, introducing an innovative paradigm that leverages hydroacoustic propagation measurements for reconstructing SSPs. Since then, there have been three mainstream frameworks for sound velocity inversion, namely MFP ____, CS ____, and DL ____. The MFP framework mainly consists of four steps. Firstly, EOF decomposition is adopted to extract the principal component features of the regional sound velocity distribution. Then, different feature combinations are generated to form candidate SSPs. Subsequently, acoustic field distributions are computationally modeled employing an acoustic ray tracing theory. The concluding phase involves cross-validation between the modeled sound field distributions and experimentally measurements to establish the optimal inversion parameters for the SSP. To accelerate the search process of the optimal candidate SSP, Tolstoy introduced simulated annealing algorithm in the MFP, which improved the algorithm execution efficiency but only suboptimal solutions were obtained ____. Afterwards, other heuristic algorithms were combined with MFP frameworks for SSP inversion ____, but they have the same problem as ____. To further improve the efficiency of inversion algorithm execution, Bianco ____ and Choo ____ proposed SSP inversion framework based on CS, respectively. Within the context of the CS framework, the correlation between the sound field distribution and the sound velocity distribution is formulated via a matrix based approach, which eliminates the search process for matching terms. However, the matrix relationship introduces linear approximation, thus sacrificing inversion accuracy.

The theoretical frameworks of deep neural networks (DNNs) have experienced accelerated progression in contemporary research, particularly in establishing intricate nonlinear dependencies across heterogeneous data domains ____. In addition, long-term underwater environmental observations have accumulated a large amount of data for marine hydrological research, laying a data foundation for the application of deep learning underwater. To address the drawbacks of computational efficiency and accuracy loss in MFP and CS frameworks, an auto-encoding neural network model for feature mapping was introduced for SSP inversion in our prior research ____, in which the auto-encoder was created to extract deep robust features so as to reduce the impact of acoustic field measurement errors on the accuracy of sound velocity inversion. Considering the insufficient accumulation of historical sound velocity data in some ocean areas, the DL model is prone to over-fitting and reduces accuracy performance. Therefore, we proposed a meta-learning framework tailored for few-shot scenarios to enhance the rate at which the model converges ____. However,the methodologies aforementioned, which are predicated on MFP, CS, and DL, all rely on real-time measured acoustic field data, which imposes strict requirements on the arrangement of underwater observation equipment. Therefore, these methods not only face high equipment economic costs, but also have limited application scope for areas without sonar measurement systems.

% 图1 声速估计整体框架
\begin{figure*}[!htbp]
	\centering
	\includegraphics[width=0.8\linewidth]{fig01stucture.eps}
	\caption{SSP Estimation Structure based on SA-MDF-CNN.}
	\label{fig1}
\end{figure*}

Nowadays, removing the necessity for on-site underwater data measurement has emerged as a prominent research focus in the domain of SSP inversion. In accordance with this demand, both Liu et al. ____ and Lu et al. ____ put forward a Long-Short Term Memory (LSTM) neural network model for SSP forecasting, which solely requires historical SSP data. Nevertheless, owing to the insufficient temporal resolution of prior data in the majority of marine regions, this prediction method can only describe the overall trend of sound velocity changes and is difficult to obtain high-precision SSP prediction results. In fact, the SSP estimation methods, that rely on a single data modality (sound field data or historical SSP data), are susceptible to data quality issues, such as high sound field measurement noise or low sound velocity sampling time resolution. To improve the robustness of the sound velocity estimation model, scholars have proposed some multimodal data fusion methods for estimating sound velocity that combining data from different sensors and sources to obtain more comprehensive features than a single data source ____. Yu et al. ____ proposed a radial basis function (RBF) neural network for SSP estimation that mainly using historical temperature, salinity profile data and average SSP data. Nevertheless, the model is not sensitive to varieties of sound velocity, and the estimation results often approach the average SSP profile, resulting in challenges in accurately depicting the spatiotemporal variations of sound velocity. Xu et al. ____ proposed a physics-inspired SOM model for SSP estimation, which introduced remote sensing SST data. However, SOM is confined to capturing the impact of SST on the distribution of sound velocity at the surface, and lacks the ability to capture large-scale feature correlations. Therefore, the accuracy of sound velocity estimation is difficult to meet practical application requirements.

To obtain real-time and accurate estimation of sound velocity distribution without underwater on-site data measurement, we fully consider the historical sound velocity distribution patterns of different spatial coordinates, as well as the impact of real-time SST changes on the dynamic characteristics of sound velocity distribution, and propose an interpretable SA-MDF-CNN model. In this model, the local correlation of features will be captured through CNN and global correlation of features will be captured through attention mechanism.