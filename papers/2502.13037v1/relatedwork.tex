\section{Related Work~\label{sec:related_work}
}

\subsection{Power Grid Inspection Methodologies}

Traditional power grid inspections are primarily conducted manually, depending on on-site maintenance personnel or manned helicopters to visually inspect transmission lines and supporting infrastructure~\cite{lavado2024ts40k}. 
While effective, these methods are resource-intensive, expensive, and often susceptible to human error, particularly when inspecting large and intricate terrains. As transmission grids keep expanding, there is an urgent need for more efficient and scalable inspection methodologies.
%
3D semantic segmentation offers an efficient and reliable solution. By leveraging 3D point clouds captured by unmanned aerial vehicles (UAVs) equipped with LiDAR sensors, CV models can automate the identification of 3D points belonging to critical elements such as power lines, support towers, and surrounding vegetation. 
%
This segmentation allows for the detection of structural failures, potential vegetation intrusions, and areas requiring immediate inspection. This greatly reduces the dependence on manual labor. 
Automating these processes accelerates inspection workflows and mitigates risks associated with delayed maintenance, such as power outages or wildfire hazards.

% Datasets like TS40K~\cite{lavado2024ts40k} are vital for advancing this automation. TS40K offers high-density, annotated 3D point clouds of rural power grids that feature a variety of structures, including medium-voltage towers and the surrounding vegetation. 
% %
% Addressing the challenges of noisy data, class imbalance, and diverse structures in TS40K will be crucial for enabling robust, scalable automation in power grid inspection pipelines, ultimately improving the sustainability and safety of modern power systems.


\subsection{3D semantic Segmentation}

3D semantic segmentation involves dividing a 3D point cloud into distinct regions labeled with meaningful categories. This task is particularly relevant for power grid inspections, where identifying components such as power lines, support towers, and vegetation can greatly enhance maintenance efficiency and reliability. Research in this field can be categorized into four paradigms: \textit{projection-based}, \textit{discretization-based}, \textit{point-based}, and \textit{hybrid methods}.
%
Projection-based approaches~\cite{su2015multi,lawin2017deep,yang2019learning,lyu2020learning} convert 3D point clouds into 2D representations, utilizing the strengths of established convolutional neural networks (CNNs). However, this process risks losing vital geometric details, such as depth.
%
Discretization-based methods~\cite{choy20194d,zhou2018voxelnet,le2018pointgrid,meng2019vv,zhang2020polarnet} process 3D data by dividing the space into a grid of small cubes, or voxels, which maintains certain spatial structure but is computationally expensive for high-resolution data.
%
In contrast, point-based techniques~\cite{qi2017pointnet,qi2017pointnet++,li2018pointcnn,thomas2019kpconv,hu2020randla,kong2023rethinking,lai2023spherical,zhao2021point,wu2022point,wu2023ptv3,pointcept2023} directly operate on raw 3D points, preserving fine-grained geometric information and achieving state-of-the-art performance on various benchmarks.
%
Transformer-based models have recently advanced point-based segmentation by leveraging self-attention mechanisms to capture both local and global dependencies within 3D data. Architectures such as Point Transformer~\cite{zhao2021point}, Point Transformer V2~\cite{wu2022point}, and Point Transformer V3~\cite{wu2023ptv3} improve feature aggregation by dynamically weighting point relationships, allowing for better geometric reasoning. These models consistently achieve state-of-the-art results on 3D benchmarks, including TS40K, demonstrating their effectiveness in power grid inspection tasks where fine structural details are critical.
%
Hybrid methods~\cite{dai20183dmv,jaritz2019multi,tang2020searching,hou2022point,liu2023uniseg} combine various approaches to capture complementary features, enhancing overall performance.


\subsection{TS40K Dataset}
\begin{figure*}[t]
\centering
% First column: Full width figure and subfigures
\begin{minipage}[t]{0.48\textwidth} % First column
    \begin{subfigure}[b]{\textwidth}
        \includegraphics[width=\linewidth]{ts40k_teaser/raw_sample.png}
        \caption{Raw TS40K sample}
        \label{fig:teaser-raw-sample}
    \end{subfigure}
    \vfill
    \begin{subfigure}[b]{0.33\textwidth}
        \includegraphics[width=\linewidth]{ts40k_teaser/tower_radius.png}
        \caption{Tower-radius}
        \label{fig:teaser-tower-radius}
    \end{subfigure}
    \hfill
    \begin{subfigure}[b]{0.33\textwidth}
        \includegraphics[width=\linewidth]{ts40k_teaser/2_towers.png}
        \caption{Power-line}
        \label{fig:teaser-power-line}
    \end{subfigure}
    \hfill
    \begin{subfigure}[b]{0.30\textwidth}
        \includegraphics[width=\linewidth]{ts40k_teaser/no_ts.png}
        \caption{No-tower}
        \label{fig:teaser-no-ts}
    \end{subfigure}
    \vfill
    % Legend below subfigures
    \includegraphics[width=\linewidth]{ts40k_teaser/legend_classes.png}
\end{minipage}
\quad
% Second column: Two additional figures
\begin{minipage}[t]{0.48\textwidth} % Second column
    \begin{subfigure}[b]{1.0\textwidth}
        \includegraphics[width=\linewidth]{ts40k_teaser/ts40k_class_densities.png}
        \caption{Sample type density}
        \label{fig:density1}
    \end{subfigure}
    \hfill
    \begin{subfigure}[b]{1.0\textwidth}
        \includegraphics[width=\linewidth]{ts40k_teaser/ts40k_train_test_distribution.png}
        \caption{Overall train/test class density}
        \label{fig:density2}
    \end{subfigure}
\end{minipage}

\caption{The TS40K dataset is derived from raw 3D scans illustrated in Figure~\ref{fig:teaser-raw-sample} and processed into three different sample types: \textit{(1) Tower-radius} focuses on the towers that support power lines and its environment (Fig.~\ref{fig:teaser-tower-radius}). \textit{(2) Power-line} samples have power lines as their main focus in the 3D scenes (Fig.~\ref{fig:teaser-power-line}). \textit{(3) No-tower} samples represent rural terrain where the transmission system is located, excluding supporting towers but potentially including power lines (Fig.~\ref{fig:teaser-no-ts}).
%
In Figures~\ref{fig:density1} and~\ref{fig:density2}, we showcase the semantic class densities of the TS40K dataset. Figure~\ref{fig:density1} illustrates the class density for each of the sample types and Figure~\ref{fig:density2} shows the overall class density in the TS40K train and test sets.}
\label{fig:teaser}
\end{figure*}


The TS40K dataset provides a novel resource for advancing 3D scene understanding specifically related to electrical transmission systems, a domain that has been mostly neglected by existing datasets. 
%
TS40K is captured using drone-mounted LiDAR sensors and comprises high-density point clouds covering over 40,000 km of electrical transmission systems. Each scan is geo-referenced and annotated with 1 out of 5 inspection labels. 
%
The dataset offers unique characteristics, including uniform density, high spatial precision (2.5 cm), and a lack of occlusion, making it particularly suitable for the detailed analysis required in maintenance applications.
%
The 22 annotated labels are intended to streamline inspections rather than model training, so they are grouped to five semantically meaningful classes. These classes represent critical scene elements necessary for evaluating the risk of contact between the electrical system and its surroundings, specifically ground, low and medium vegetation, power line support towers, and power lines.
%
Additionally, the dataset presents three distinct sample types: tower-radius, which focuses on tower environments; power-line, which emphasizes power lines between towers; and no-tower, which captures terrain without visible transmission structures. This categorization enables tailored model evaluations based on specific inspection scenarios.

TS40K addresses the challenges of power grid inspection by replicating real-world complexities, such as class imbalance, sensor noise, and mislabeled annotations. For example, while ground and vegetation account for over 90\% of the dataset, power line support towers and power lines make up less than 2\%, presenting significant difficulties for machine learning algorithms. 
Additionally, the drone-collected data inherently includes spurious noise, particularly under adverse environmental conditions, which can obscure critical structures. 
The dataset also incorporates annotation noise, reflecting practical constraints faced by human inspectors during labeling.
%
Benchmarking on TS40K highlights significant performance gaps in current state-of-the-art segmentation models. 
Despite achieving strong results in urban datasets, methods like KPConv~\cite{thomas2019kpconv} and Point Transformer V2~\cite{wu2022point} exhibit low performance in power grid-related tasks, particularly in detecting power line support towers with 43\% Intersection over Union (IoU).
The authors attribute this performance lag to the diverse structures that compose supporting towers along with their low point-density in the dataset. Power lines on the other hand, enjoy a 94\% IoU despite having similar point-densities.
%, where IoU scores remain below the 85\% threshold deemed acceptable by industry experts. 
These results highlight the need for further research to develop robust, noise-resilient models tailored to this application domain.


\subsubsection{3D Datasets on Transmission Networks}
Several datasets exist that capture aspects of rural or forest terrain, as well as power grid elements, but each has specific limitations when applied to power grid inspections. Forest3D~\cite{trochta20173d} is centered solely on the 3D representation of tree crowns, emphasizing their instance segmentation without consideration for power grid components. 
%
GTASynth~\cite{curnis2022gtasynth}, a synthetic dataset of non-urban environments, features low-density point clouds with significant object occlusion, captured from a vehicle’s perspective that differ significantly from UAV-based grid inspections. 
Similarly, NEON~\cite{marconi2019data} focuses on airborne LiDAR data to predict tree crown dimensions but does not annotate grid infrastructure. DALES~\cite{varney2020dales}, while including high-voltage towers, is limited to urban environments and lacks the diversity found in rural grids.
%
In contrast, TS40K~\cite{lavado2024ts40k} specifically targets electrical power grids in rural areas, including medium and small-voltage towers that are more varied in shape and harder to distinguish from their surroundings.

\subsection{Power Line Detection Methods}

Power grid inspection has traditionally relied on on-site maintenance personnel and manned helicopters, where the grid is examined visually or with portable devices. These methods are labor-intensive, costly, and prone to inefficiencies, emphasizing the need for process automation. To address this, UAVs equipped with LiDAR sensors are increasingly being deployed to capture detailed 3D point cloud representations of power grid environments.

In the work of Ding et al.~\cite{ding2021electric}, simultaneous localization and mapping (SLAM) algorithms are combined with multi-sensor data to enable UAV-based patrols of electrical grids. This approach employs a multi-view-based technique for point cloud segmentation, but the reliance on 3D reconstructions derived from 2D raster maps often leads to significant information loss and reduced accuracy. 
%
Similarly, Guo et al.~\cite{guo2019research} project point clouds onto the $xy$-plane to cluster and segment power lines. However, this method overlooks critical contextual information, such as ground features and irregular terrain, and focuses primarily on incomplete segments of power lines. Alternative approaches, such as those presented by Tao et al.~\cite{tao2019study}, leverage fine-grained elevation statistics of the original point cloud in combination with $xy$-plane projections to enhance segmentation accuracy.

While these methods advance the automation of power line detection, they predominantly target high-voltage power lines and fail to account for the supporting towers, which are essential for comprehensive inspections. 
%
They also neglect other critical scene elements, such as vegetation, which poses a significant risk of contact with the grid and must be assessed to ensure operational safety.
%
Another major limitation of these approaches is the lack of publicly available datasets tailored for comprehensive power line inspections. Many methods rely on proprietary or restricted datasets, hindering reproducibility and further development.