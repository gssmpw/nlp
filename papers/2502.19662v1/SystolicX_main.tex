\documentclass[conference]{IEEEtran}
\IEEEoverridecommandlockouts
% The preceding line is only needed to identify funding in the first footnote. If that is unneeded, please comment it out.
%Template version as of 6/27/2024

\usepackage{cite}
\usepackage{amsmath,amssymb,amsfonts}
%\usepackage{algorithmic}
\usepackage{graphicx}
\usepackage{textcomp}
\usepackage{xcolor}
\def\BibTeX{{\rm B\kern-.05em{\sc i\kern-.025em b}\kern-.08em
    T\kern-.1667em\lower.7ex\hbox{E}\kern-.125emX}}
\usepackage[utf8]{inputenc} % allow utf-8 input
\usepackage[T1]{fontenc}    % use 8-bit T1 fonts
\usepackage{microtype,inconsolata}
\usepackage{times,latexsym}
\usepackage{graphicx} \graphicspath{{figures/}}
\usepackage{amsmath,amssymb,mathabx,mathtools,amsthm,nicefrac}
\usepackage[linesnumbered,ruled,vlined]{algorithm2e}
\usepackage{acronym}
\usepackage{enumitem}
\usepackage[pagebackref,breaklinks,colorlinks]{hyperref}
\usepackage{balance}
\usepackage{xspace}
\usepackage{setspace}
\usepackage[skip=3pt,font=small]{subcaption}
\usepackage[skip=3pt,font=small]{caption}
\usepackage[capitalise,noabbrev,nameinlink]{cleveref}
\usepackage{booktabs,tabularx,colortbl,multirow,multicol,array,makecell,tabularray}
\usepackage{overpic,wrapfig}
\usepackage{dblfloatfix}
\usepackage[misc]{ifsym}
\usepackage{pifont}
\usepackage{fancyvrb}

% Add a period to the end of an abbreviation unless there's one
% already, then \xspace.
\makeatletter
\DeclareRobustCommand\onedot{\futurelet\@let@token\@onedot}
\def\@onedot{\ifx\@let@token.\else.\null\fi\xspace}

\def\eg{\emph{e.g}\onedot} \def\Eg{\emph{E.g}\onedot}
\def\ie{\emph{i.e}\onedot} \def\Ie{\emph{I.e}\onedot}
\def\cf{\emph{c.f}\onedot} \def\Cf{\emph{C.f}\onedot}
\def\etc{\emph{etc}\onedot} \def\vs{\emph{vs}\onedot}
\def\wrt{w.r.t\onedot} \def\dof{d.o.f\onedot}
\def\etal{\emph{et al}\onedot}

\makeatother

\acrodef{sota}[SOTA]{State-of-the-Art}
\acrodef{method}[\textsc{PRA}]{Preference-based Robot Assistant}
\acrodef{pbp}[\textsc{PbP}]{Preference-based Planning}
\acrodef{vln}[VLN]{Vision-and-Language Navigation}
\acrodef{llm}[LLM]{Large Language Model}
\acrodef{EILEV}[EILEV]{Efficient In-context Learning on Egocentric Videos}
\acrodef{vlm}[VLM]{Vision-Language Model}
\acrodef{vivit}[ViViT]{Video Vision Transformer}
\acrodef{llava}[LLaVA]{Large Language and Vision Assistant}
\acrodef{ai}[AI]{Artificial Intelligence}
\acrodef{ik}[IK]{Inverse Kinematics}
\acrodef{ompl}[OMPL]{Open Motion Planning Library}
\acrodef{sem}[SEM]{Structural Equation Model}

% Spacing
% \medmuskip=2mu   % reduce spacing around binary operators
% \thickmuskip=3mu % reduce spacing around relational operators
\setlength{\abovedisplayskip}{3pt}
\setlength{\belowdisplayskip}{3pt}
\setlength{\abovecaptionskip}{3pt}
\setlength{\belowcaptionskip}{3pt}
% \setlength\floatsep{1\baselineskip plus 3pt minus 2pt}
% \setlength\textfloatsep{1\baselineskip plus 3pt minus 2pt}
% \setlength\dbltextfloatsep{1\baselineskip plus 3pt minus 2pt}
% \setlength\intextsep{1\baselineskip plus 3pt minus 2pt}

\newcolumntype{x}{>{\columncolor{LightCyan1}}c}
\newcolumntype{y}{>{\columncolor{MistyRose}}c}
\begin{document}

%\title{Systolic Xtreme: Selecting Low-Critical-Delay Weights for Accelerating LLM Inference}
\title{HALO: \underline{H}ardware-\underline{a}ware quantization with \underline{lo}w critical-path-delay weights for LLM acceleration}

\begin{comment}
\author{\IEEEauthorblockN{Rohan Juneja}
\IEEEauthorblockA{\textit{School of Computing} \\
\textit{National University of Singapore}\\
rohan@comp.nus.edu.sg}
\and
\IEEEauthorblockN{Shivam Aggarwal}
\IEEEauthorblockA{\textit{School of Computing} \\
\textit{National University of Singapore}\\
shivam@comp.nus.edu.sg}
\and
\IEEEauthorblockN{Safeen Huda}
\IEEEauthorblockA{\textit{Google}\\
safeen@google.com}
\and
\IEEEauthorblockN{Tulika Mitra}
\IEEEauthorblockA{\textit{School of Computing} \\
\textit{National University of Singapore}\\
tulika@comp.nus.edu.sg}
\and
\IEEEauthorblockN{Li-Shiuan Peh}
\IEEEauthorblockA{\textit{School of Computing} \\
\textit{National University of Singapore}\\
peh@comp.nus.edu.sg}
}
\end{comment}

\author{
    \IEEEauthorblockN{
        Rohan Juneja\IEEEauthorrefmark{1},
        Shivam Aggarwal\IEEEauthorrefmark{1},
        Safeen Huda\IEEEauthorrefmark{2}
        Tulika Mitra\IEEEauthorrefmark{1},
        Li-Shiuan Peh\IEEEauthorrefmark{1},
    }
    \\
    \IEEEauthorblockA{\IEEEauthorrefmark{1}School of Computing, National University of Singapore, \IEEEauthorrefmark{2}Google\\ 
    \{rohan, shivam, tulika, peh\}@comp.nus.edu.sg, 
    safeen@google.com}
}

\maketitle

\begin{comment}
\begin{abstract}
The rapid expansion of Transformer-based models has significantly outpaced advancements in hardware, leading to increased inference costs.
To address the gap between growing model sizes and hardware limitations, model quantization presents a promising solution. 
However, current quantization techniques do not account for the critical path variations in Multiply-Accumulate (MAC) operations, which differ significantly with each weight value due to the iterative shift-add architecture of MAC units.
This paper introduces Systolic Xtreme (SystolicX), a novel hardware-software co-design aimed at accelerating models on Systolic Arrays by leveraging unique frequency scaling opportunities.

SystolicX capitalizes on the relationship between weight values and their effective critical path timing at runtime. 
It proposes a Hardware-Aware Post-Training quantization framework that incorporates hardware accelerator timing and power analysis. 
This framework strategically quantizes model weights, prioritizing those with lower critical delay, thereby reducing the circuit's overall critical delay. 
This reduction enables dynamic frequency adjustment to higher clock frequencies, achieving significant performance and energy improvements.
SystolicX demonstrates the potential to deliver up to ?? times speedup and ?? power savings, all without compromising accuracy and with minimal area overhead.
\end{abstract}
\end{comment}

\begin{abstract}
%Cambrian explosion of accelerators for LLMs has fragmented the hardware landscape, highlighting the need to optimize existing platforms over developing bespoke solutions. 
Quantization is critical for realizing efficient inference of LLMs.
Traditional quantization methods are hardware-agnostic, limited to bit-width constraints, and lacking circuit-level insights, such as timing and energy characteristics of Multiply-Accumulate (MAC) units. We introduce \textit{HALO}, a versatile framework that adapts to various hardware through a Hardware-Aware Post-Training Quantization (PTQ) approach. By leveraging MAC unit properties, \textit{HALO} minimizes critical-path delays and enables dynamic frequency scaling. Deployed on LLM accelerators like TPUs and GPUs, \textit{HALO} achieves on average 270\% performance gains and 51\% energy savings, all with minimal accuracy drop.
%{\bf Peh: Why is there area overhead?}
\end{abstract}
\begin{IEEEkeywords}
LLMs, Quantization, DVFS, TPUs, GPUs
\end{IEEEkeywords}

\documentclass[../main.tex]{subfiles}
\graphicspath{{../images/}}
\makeatletter
\def\input@path{{../images/}}
\makeatother
\begin{document}
\section{Introduction}
\begin{figure}
\centering
\begin{tikzpicture}
\node[inner sep=0pt] (ws) at (0, 0) {
\includegraphics[height=.4\textwidth, trim={10cm 0 10cm 0},clip]{world_space.png}};
\node[inner sep=0pt] (cs) at (6,0) {\includegraphics[height=.4\textwidth, trim={10cm 1cm 10cm 4cm},clip]{conf_space.png}};
\end{tikzpicture}
\vspace{-5pt}
\label{fig:pbrm_intro}
\caption{\textbf{Left}: Shows world space obstacles as grey spheres. Robots start and goal configuration is colored red and green, respectively. Configurations along the computed path are colored transparent blue. \textbf{Right:} Mapped world space scenario to configuration space. Obstacle region is the grey mesh. Red spheres are collision-free regions computed by the neural SCDF. The optimized shortest path in the convex corridor is the blue curve.}
\vspace{-25pt}
\end{figure}
Motion planning is the problem of finding a collision-free trajectory that connects a given start and goal configuration. The planning takes place in the configuration space of the robot. For single body robots, like mobile robots or drones, the configuration space and the world space are usually the same. This simplifies the planning, since explicit obstacle representations are available which enables geometrical tools like separating hyperplanes, smallest distance to obstacles etc., to be used when designing motion planning algorithms. For multi-body robots like manipulators, the situation is completely different. The world space obstacles are usually mapped to non-convex regions, and to make the problem even harder, the mapping is usually not known. Forming explicit representations of the obstacle region in the configuration space is usually too expensive or intractable. Despite all of this, sampling based planners are used with great success, which mainly is due to their use of implicit representations of the obstacle region. The basic idea is to construct a graph in the configuration space that covers and connects the collision-free region. From this graph, a path can be extracted that connects a given start and goal configuration. The approach is computationally expensive, since the graph is constructed with the smallest geometrical building block available, points, which represents a collision-check. Furthermore, the extracted paths from the graph are non-smooth and jagged due to the stochastic nature of the approach. This adds an additional post-processing step to the process, where the paths are shortcutted and smoothened, before the path can be used for tracking. Clearly a lot of time is invested to form this graph and produce smooth paths. Thus, if the obstacles start to move, then all of this work is done in no use, since all points that make up this graph need to be re-verified, which is simply too time consuming to be done in real time.
\\\\
In this work, we want to address the existing drawbacks of the sampling based planners. Our main contribution is an improved motion planner where each vertex in the graph covers a collision-free region in the form of a sphere instead of a point and where the edges are formed with neighboring intersecting spheres. This representation has the advantage of instead of returning piecewise linear paths, returning a sequence of overlapping spheres, i.e. a convex corridor, that connects a given start and goal configuration, illustrated in Figure \ref{fig:pbrm_intro}. This convex corridor allows us to use convex optimization to produce smooth trajectories, instead of computationally expensive post-processing methods. The representation further allows us to estimate the coverage of the collision-free space, which gives us awareness and feedback in the offline roadmap construction phase. Finally, our representation is simple to adapt to moving obstacles, simply requery for the new radii and recheck for intersections. 
\\\\
The spherical collision-free regions are formed using a signed distance function (SDF), which is a function that returns the smallest distance from an arbitrary point to the boundary of an obstacle. As the name implies, the distance is signed, thus if the point is inside the obstacle it is negative otherwise positive. If the distance is positive, a sphere with radius equal to the distance is guaranteed to cover a collision-free region. Using an SDF in motion planning is not new, but what is novel about our approach is that we express the distance in the configuration space instead of the world space and by doing so allows us to form these convex collision-free regions. We refer to the resulting SDF as a signed configuration distance function (SCDF). Computing an SCDF analytically is non-trivial, our approach is therefore to parameterize the SCDF with a deep neural network and learn the mapping by supervised learning. Our resulting neural SCDF can compute distances for different parameter values of obstacle shapes and we also show how multiple distances can be combined, thus making our approach flexible.
\section{Related work}
Motion planning algorithms can roughly be divided into three families, grid-based, sampling based and optimization based methods. Grid-based methods (GBM) discretize the planning space from which a graph is then compiled. A standard search method is A$^\star$ \citep{a_star}, which is classified as an \textit{informed} search method, since it employs a heuristic function to speed up the search. A$^\star$ guarantees to return an optimal path at the level of discretization used. GBMs usually discretize the planning space by a regular lattice and this limits the GBMs to problems with low dimensionality due to the curse of dimensionality. Thus, GBMs are usually limited to single-body robots where the degrees of freedom (DOF) are low. To overcome the inherent scaling problem with the GBMs, stochastic methods are usually used for multi-body robots. These methods are termed as sampling-based methods (SBM) and core members within this family are the rapidly-exploring random trees (RRT) \citep{rrt} and the probabilistic roadmap (PRM) \citep{prm}. RRT grows a tree from the start configuration and explores the collision-free region in a rapid way until it is able to connect to the goal region. RRT is usually improved by bi-directional planning \citep{rrt_connect}, i.e. an additional tree is grown from the goal configuration and the trees are tested for connection after any tree has been expanded. RRT is a single-query method, thus it searches for a path from scratch each time it is queried. Contrary to this, PRM is a multi-query method, which solves for multiple queries without starting from scratch. PRM does this by creating a roadmap (graph) that covers the collision-free space as an offline step. The graph is then used to solve for multiple queries. PRMs are used in cases where the environment does not change since the extra offline step is too computationally costly and needs to be re-done if the environment is changed. In our work, we address this inherent issue by using a different roadmap representation. Our vertices in the graph cover a collision-free region in the form of spheres and we form the edges by checking for intersecting spheres. If something in the environment changes, we recompute the spheres radii and recheck the intersections, without relying on collision detection. We use a trained neural network to compute the sphere radius, therefore querying for the radius can be done fast, hence our representation enables the PRM for dynamic environments.
\\\\
In the recent decades, optimization based methods (OBM) \citep{chomp, schulman, itomp, stomp} have been introduced as an alternative to SBM for multi-body robots. Like the SBM, the OBMs scale well to higher dimensional problems and produce smoother motion. It is common to use a SDF in the optimization since it is a smooth function, thus enabling gradient-based methods. However, the standard way of expressing the SDF is in world space. The distance therefore needs to be mapped to the configuration space by the forward kinematics. This mapping makes the optimization problem a non-linear program (NLP), which is computationally expensive to solve. Recently, a different approach has been proposed. In \cite{mp_gcs} motion planning is formulated as a convex optimization problem by using the graph of convex sets framework \citep{gcs}. The underlying idea is to decompose the collision-free space into intersecting convex sets from which a convex optimization problem is formulated. In cases where an explicit representation of the obstacles in the configuration space exists, like for single-body robots, creating collision-free convex regions can be done fast \citep{iris}. For multi-body robots, this is non-trivial. Existing work does this successfully \citep{iris_nlp, iris_c} by an optimization based approach, but the methods are still too time consuming to be used in the presence of moving obstacles. Our approach is instead to use deep learning to learn an SDF expressed in the configuration space. With this, we can query for shortest distances to the collision boundary, which allows us to expand spherical regions which are collision-free. Our approach is fast and therefore enables our suggested roadmap planner to be used in dynamic environments.
\\\\
Recent research has focused on learning collision detection \citep{fk_kernel_distance, diffco, graphdistnet} by predicting the signed distance between the robot links and the surrounding obstacles in the world space. The learned SDF is used in trajectory optimization but since the distance is expressed in the world space, the problem becomes an NLP and therefore takes a long time to solve. We take a novel approach and suggest to instead express the signed distance in the configuration space. This allows us to improve the PRM at the same time as it enables convex optimization for trajectory optimization, which runs faster and is more reliable than NLP solvers. In \cite{cspf} a learned signed distance function in the configuration space is proposed similar to our approach. However, their approach is restricted to point cloud representations, while we propose to represent the obstacles as parameterized geometric shapes, e.g. spheres. Furthermore, we also show how to use our learned SCDF to improve an existing roadmap planner.
\section{Problem formulation}
A robot is located in the world space, $\W \subset \R^3 $. The unique location of the robot is given by its configuration $\q \in \C$, where $\C$ is the configuration space. The set of points covered by the robots bodies at a certain configuration is expressed as $\B(\q) \subset \W$. The robot is surrounded by $\NrObst$ obstacles $\O = \bigcup_{i=1}^{\NrObst} \O_i$, where  $\O_i \subset \W$. The representation of the obstacle in the configuration space is the set $\C\O_i = \{\q \in \C \: |\: \B(\q) \cap \O_i \neq \emptyset \}$. The obstacle space is formed as $\Co = \bigcup_{i=1}^{\NrObst} \C \O_i$. The complement is referred to as the free space, $\Cf = \C \setminus \Co$. The path planning problem is a tuple, ($\Cf$, $\qStart$, $\qGoal$), where we want to connect a query pair, consisting of a start, $\qStart$, and goal configuration, $\qGoal$, with a geometric path, $\q(s): [0, 1] \mapsto \Cf$, such that $\q(0)=\qStart$ and $\q(1)=\qGoal$, or report correctly when such a path does not exist.
\end{document}

\section{Basic Background: Supervised Learning and the PAC Model}
\label{sec:background}

At this point almost everyone has heard of machine learning (ML). Anyone likely to stumble upon this article will have also heard of its most influential special case, supervised learning, and those theoretically inclined will also be familiar with the PAC model. Nonetheless, I will set the stage by  recapping the basics.

\subsection{Basics of Supervised Learning}%Let's set the stage in any case

\emph{Supervised Learning} is the task of ``coming up'' with a function $f: \X \to \Y$ to ``explain'' or ``fit'' a sequence of input/output examples   $(x_1,y_1), \ldots, (x_n,y_n)$, with $x_i \in \X$ and $y_i \in \Y$.  Here $\X$ is a \emph{data domain} consisting of \emph{datapoints} $x \in \X$, $\Y$ is a \emph{label set} consisting of \emph{labels} $y \in \Y$, and the sequence $(x_1,y_1),\ldots,(x_n,y_n)$ is the \emph{training data} consisting of \emph{labeled examples (a.k.a. samples)}~$(x_i,y_i)$.  I~will refer to the chosen function $f$ as a \emph{predictor}, and to $n$ as the \emph{sample size}. A \emph{learning algorithm} takes as input training data, and outputs (some representation of) a predictor $f \in \Y^\X$.\footnote{Note that this describes the usual \emph{batch}, a.k.a.~\emph{offline}, setting of supervised learning. I do not discuss other paradigms such as online or active learning in this article.} 



Success in supervised learning is defined as \emph{generalization} to  future examples: For a typical \emph{test example}  $(x_{\tst},y_{\tst})$, the predicted label $y'_{\tst}=f(x_{\tst})$ should ``equal'' $y_{\tst}$, perhaps approximately. We usually assume the test example is drawn from the same  ``source'' as the training data  --- commonly, i.i.d.~from the same distribution. The quality of the prediction is quantified by $\ell(y'_{\tst},y_{\tst})$, where $\ell:~\Y~\times~\Y \to \RR_{\geq 0}$ is a \emph{loss function} chosen as part of the problem definition. Common loss functions include the 0-1 loss $\ell_{0-1}(y',y) = [y' \neq y]$ for \emph{classification} problems,\footnote{The notation $[P]$ denotes $1$ when predicate $P$ is true, and denotes $0$ when $P$ is false.} as well as the absolute loss $|y'-y|$ or squared loss $(y'-y)^2$ for \emph{regression problems} featuring $\Y  \sse \RR$.

Nontrivial generalization properties are typically only possible if one assumes something about the data.\footnote{The need for such an assumption is formalized by the  \emph{no free lunch theorems} of supervised learning \cite{wolpert_connection_1992,wolpert_lack_1996,schaffer_conservation_1994}.} The Bayesian approach to  machine learning, common in many applications, assumes some parametric form for the distribution generating the data, and postulates a prior on the parameters. This is not the approach I will take in this article. Instead, I will focus on the frequentist --- and some would say ``worst-case'' or ``adversarial'' ---  approach that is common in the computational learning theory community, embodied by the PAC model. Here we assume that the (training and test) data can be explained, perhaps approximately, by a function in some ``simple enough to learn'' class of functions $\H \sse \Y^\X$, often called the \emph{hypotheses}. Equivalently, we  seek a predictor which explains the unseen data roughly  as well as the best hypothesis $h^* \in \H$, whether or not we assume that $h^*$ itself provides a perfect explanation.



 \paragraph{Common Algorithmic Templates.} Perhaps the best known general-purpose supervised learning algorithm is \emph{empirical risk minimization (ERM)}, which chooses as its predictor a hypothesis $f \in \H$ minimizing $\frac{1}{n} \sum_{i=1}^n \ell(f(x_i),y_i)$ --- a quantity called the \emph{training error}, \emph{empirical error}, or \emph{empirical risk} of $f$. %\footnote{When multiple hypotheses minimize the empirical risk, we assume ERM breaks ties arbitrarily.}
A common template for generalizing ERM involves adding a \emph{regularization term} $\psi(f)$ to the  objective function, typically chosen to measure some notion of ``hypothesis complexity.'' An algorithm instantiating this template is known as a \emph{structural risk minimizer (SRM)}, and chooses as its predictor the hypothesis $f \in \H$ minimizing the \emph{structural risk} $\frac{1}{n} \sum_{i=1}^n \ell(f(x_i),y_i) + \psi(f)$. Other well-known algorithms, such as gradient descent and its variations,  can frequently be interpreted as approximate implementations of ERM or SRM.


\paragraph{Proper vs Improper Learning.} A learning algorithm is said to be \emph{proper} if its predictor $f$ is always chosen from the hypothesis class, i.e., $f \in \H$, otherwise it is said to be \emph{improper}. ERM  is an example of a proper learning algorithm, as are SRM algorithms of the form described above.  In the \emph{proper regime} of learning, algorithms are required to be proper. This article will be concerned with the more flexible \emph{improper regime} (a.k.a \emph{representation-independent learning}), where no such constraint is placed on the learner. In other words, all we care about is predictive power at test time, rather than any insights derived from the functional form or representation of the predictor~itself.


\subsection{The PAC Model}
A standard mathematical setup for evaluation of supervised learning algorithms, at least in the theoretical computer science community, is Valiant's \emph{Probably Approximately Correct (PAC) model} of learning (see e.g.~\cite{kearns_introduction_1994,mohri_foundations_2018}). Here, we assume there is an unknown distribution $\D$ on $\X \times \Y$ from which training and test data are  drawn.  Specifically, the labeled datapoints of the training set  $(x_1,y_1), \ldots, (x_n,y_n)$, as well as the test data  $(x_\tst,y_\tst)$, are i.i.d.~from $\D$. Often it is assumed that $\D$ lies in some class of distributions of interest. The \emph{true expected loss}, or simply \emph{loss}, of a predictor $f: \X \to \Y$ is the expected loss it incurs on draws from $\D$, written $L_\D(f) = \Ex_{(x,y) \sim \D} \ell(f(x),y)$.


There are two main ``settings'' in PAC learning. The  \emph{realizable setting} only requires that the data be perfectly explained by some hypothesis in $\H$. More generally, the \emph{agnostic setting} makes no assumption relating the data to the hypotheses, but shifts the goalposts as necessary to allow nontrivial guarantees: the expected loss at test time is evaluated only ``relative'' to that of the best hypothesis $h^* \in \H$. There are other settings which make more nuanced assumptions, such as $\D$ being of a particular parametric form or its support living in some (unknown) lower-dimensional space, etc. I will mostly discuss the realizable and agnostic settings in this article, those being the simplest and most studied from a theoretical perspective. %TODO:We will briefly discuss other settings in Section ??

The PAC model demands high probability guarantees of learners, in the worst case over distributions of interest. Consider first the realizable setting, where $\D$ is such that $\min_{h \in \H} L_{\D}(h) = 0$. A PAC learner has \emph{error} $\epsilon=\epsilon(n)$ and \emph{confidence} $\delta=\delta(n)$ if, when training data consists of $n$ i.i.d~samples from a realizable distribution $\D$, it produces a predictor $f$  satisfying $L_\D(f) \leq \epsilon$ with probability at least $1-\delta$. In the agnostic setting, where $\D$ can be arbitrary, we require $L_\D(f) - \min_{h \in \H} L_\D(h) \leq \epsilon$ with probability $1-\delta$.

In both the realizable and agnostic settings, we look for PAC learners with small $\epsilon$ and $\delta$ as a function of the sample size $n$. An equivalent perspective looks at the sample complexity $m(\epsilon,\delta)$, which is the minimum sample size which guarantees error  at most $\epsilon$ with probability at least $1-\delta$. We say a problem is \emph{PAC learnable} if its PAC sample complexity is finite whenever $\epsilon,\delta > 0$.

For most PAC learning problems, learnability and sample complexity are characterized in terms of a  ``dimension'' of the hypothesis class. Most prominently this is the \emph{VC dimension} for binary classification, the \emph{fat shattering dimension} for agnostic regression, and the \emph{DS dimension} for multiclass classification (see \cite{anthony_neural_1999,daniely_optimal_2014,brukhim_characterization_2022}). Treatment of these is beyond the scope of this article. The unfamiliar reader need not worry, however,  as dimensions will feature only tangentially in our~discussion.




%\paragraph{Learning settings: Realizable, Agnostic, etc.} In learning theory, evaluating a supervised learning algorithm requires specifying a data model and an objective. We will leave the details of the data model flexible for now, to allow for both the PAC model and the adversarial transductive model. Nonetheless we will describe two variations, which we call ``settings'', which cut across different models. The  \emph{realizable setting}  requires only that the data be perfectly explained by some hypothesis $h \in \H$ --- i.e., there exists a hypothesis which is guaranteed to suffer a loss of $0$ on training and test data. The performance of the learning algorithm is its expected loss at test time for some ``worst case'' realizable instance. More generally, the \emph{agnostic setting} makes no assumption relating the data to the hypotheses, but shifts the goalposts as necessary to allow nontrivial guarantees: the expected loss at test time is evaluated only ``relative'' to that of the best hypothesis $h^* \in \H$, again for some ``worst case'' instance. There are other settings which make more nuanced assumptions about the data, such as it is drawn from a distribution of a particular parametric form, or that it lives in some (unknown) lower-dimensional space, etc. We will mostly discuss the realizable and agnostic settings, those being the simplest and most studied from a theoretical perspective.




%%% Local Variables:
%%% mode: latex
%%% TeX-master: "learning_matching"
%%% End:

%\section{Motivation}
\label{sec:motivation}



% In LLM inference, not only does weight matter, but the memory requirements of the KV Cache are also considerable.
In this section, we first demonstrate that the emerging paradigm of group quantization demands a high level of adaptivity, which current adaptive methods lack.
We then discuss how adapting these methods to group quantization could compromise their efficiency.
Given that LLMs generate KV caches during runtime, real-time quantization capability is crucial.
These challenges lead to our proposal of a mathematical adaptive numerical type (\texttt{MANT}), which we will detail later.



\begin{figure}[t]
    \centering
    \begin{minipage}[t]{0.48\columnwidth}
      \centering
      \includegraphics[width=\columnwidth]{fig/moti_group_ppl.pdf}
      \caption{LLM accuracy with different quantization granularities. We report the perplexity (PPL) metric (lower is better).}\label{fig:moti_group_ppl} 
    \end{minipage}
    \hspace{2pt}
    \begin{minipage}[t]{0.48\columnwidth}
      \centering
      \includegraphics[width=\columnwidth]{fig/motivation_adaptive_ppl.pdf}
      \caption{Accuracy loss for \texttt{INT}, \texttt{ANT}, and Ideal (clustering algorithm K-Means) adaptive methods in group quantization. }\label{fig:moti_ppl} 
    \end{minipage}
    % \vspace*{-0.3cm}
\end{figure}




\subsection{Group Quantization Accuracy Analysis}
\label{sec:acc_analysis}

In this subsection, we begin by comparing the accuracy of traditional channel-wise quantization with group-wise quantization~\cite{shao2024omniquant,zhao2023atom,liu2024kivi,sheng2023flexgen,lin2023awq,zhao2023atom}, establishing the baseline for group-wise quantization in this study.
We then delve into the use of various adaptive data types in group quantization, emphasizing the necessity for full adaptivity.



\Fig{fig:moti_group_ppl} illustrates the perplexity when quantizing the LLaMA-7B model~\cite{touvron2023llama} with various granularities using the \texttt{INT4}-based symmetric quantization.
Channel-wise quantization significantly worsens the perplexity of the examined LLM, increasing it from 5.68 to 6.85.
Conversely, group-wise quantization mitigates this loss in perplexity with a group size of 128, corresponding to an average of 4.125 bits per element (16-bit scaling factor).
Additionally, we observe that a smaller group size of 32 offers only a slight improvement in perplexity, but the scaling factor overhead increases by $4\times$.



Given this analysis, we adopt a group size of 128 as our standard configuration for the remainder of this section.
Previous research indicates that the \texttt{INT} data type is not optimal for accuracy since tensors or channels exhibit varied distributions, leading to the proposal of various adaptive data types~\cite{guo2022ant, guo2023olive, zadeh2020gobo, zadeh2022mokey}.
We evaluate their efficacy in the context of group quantization, which falls into two main categories: data-type-based and clustering-based.



\textbf{Data-type-based adaptive methods} select data types from discrete sets based on tensor data distribution.
ANT~\cite{guo2022ant} is a representative example of the data-type-based method.
ANT packages several different data types for selection, including \texttt{INT} for the uniform distribution, \texttt{PoT} (Power of Two) for the Laplace distribution, and \texttt{flint} for the Gaussian distribution.
%ANT designed \texttt{flint} for Gaussian distributions.

\textbf{Clustering-based adaptive methods} utilize clustering algorithms to generate centroids that align with the data distribution and provide considerable adaptivity. 
Mokey~\cite{zadeh2022mokey} and GOBO~\cite{zadeh2020gobo} exemplify this approach, though they focus on tensor- or channel-wise quantization. In our study, we adapt them to group quantization through per-group clustering.

%Clustering-based methods employ clustering algorithms to generate centroids that fit the data distribution, demonstrating sufficient adaptivity.
%Mokey~\cite{zadeh2022mokey} and GOBO~\cite{zadeh2020gobo} are such presentative works, but only target tensor- or channel-wise quantization.
%In our work, we modify those works to support group quantization by performing per-group clustering.
\Fig{fig:moti_ppl} compares the accuracy of the methods described above for the LLaMA-7B model under 4-bit group-wise quantization. 
The group-wise \texttt{ANT} method outperforms the \texttt{INT} type by dynamically selecting from three data types to better match the value distribution, resulting in reduced perplexity (PPL) loss. 
Moreover, per-group clustering adjusts more effectively to the value distribution of each group, establishing itself as the accuracy-optimal and ideal adaptive method. 
This approach achieves nearly lossless 4-bit quantization, equivalent to 16 centroids per group. 
However, this ideal scenario is impractical due to the significant overhead associated with storing per-group centroids, effectively rendering it a 6-bit quantization.

\begin{figure}[t] 
    \centering 
    \includegraphics[width=1.0\linewidth]{fig/intro_cdf.pdf}  
    \caption{The cumulative distribution function (CDF) of the tensor, channel, and group, respectively. The tensor data were taken from layers 8 to 23, while the 16 channel and group data were sampled from one tensor with specific strides.}\label{fig:moti_dist} 
    %  \vspace*{-0.3cm}
\end{figure}

To illustrate the group-wise diversity in data distribution, we sampled the weights of the Q and V tensors in LLaMA-7B model. 
We normalized all sampled data to their absolute maximum values, which ranged from -1 to 1. \Fig{fig:moti_dist} displays the cumulative distribution function (CDF) for the tensor, channel, and group levels, respectively. 
We observed that the diversity at the group level is significantly higher than at the tensor level. 
In simpler terms, while different tensors exhibit similar distributions, groups can have markedly different distributions. This finding underscores the necessity for full adaptivity in group quantization to fully realize its potential.
\paragraph{Takeaway 1.} The group quantization is an emerging paradigm to accelerate LLMs, and the significant group-level diversity requires a high level of adaptivity to fully unleash its potential.

\subsection{Group Quantization Efficiency Analysis}
\label{subsec:efficiency}


In this subsection, we provide a detailed efficiency analysis for the above adaptive quantization methods.
In \Tbl{intro:dtype}, we compare OliVe~\cite{guo2023olive}, ANT~\cite{guo2022ant}, GOBO~\cite{zadeh2020gobo}, and Mokey~\cite{zadeh2022mokey} with \texttt{INT} regarding the efficiency of computation, encoding, and decoding. 
In this paper, we use the term encoding (decoding) interchangeably with quantization (dequantization).
 

Data-type-based adaptive methods such as ANT~\cite{guo2022ant} and Olive~\cite{guo2023olive} achieve computational efficiency comparable to \texttt{INT}. 
Both utilize specialized decoders that decode these data types prior to computation, resulting in high decoding efficiency. 
However, as previously demonstrated, these methods suffer from limited adaptivity in the group quantization paradigm. 
A straightforward approach to enhance adaptivity is to expand their set of data types. 
However, incorporating new data types necessitates additional decoders, escalating hardware design costs. 
Additionally, compatibility issues between new and existing data types may reduce computational efficiency. 
For instance, the \texttt{NF4} data type~\cite{dettmers2023qlora} requires an FP16 MAC unit, which is incompatible with existing \texttt{ANT} data types.


\paragraph{Takeaway 2.} Enhancing the data-type-based adaptive method for group quantization is challenging and requires a careful balance for the computation and decoding efficiency.

Clustering-based adaptive methods like GOBO~\cite{zadeh2020gobo} and Mokey~\cite{zadeh2022mokey} can sufficiently adapt to various distributions at the group level. 
However, they require codebooks for quantization and dequantization, leading to high adaptivity at the expense of encoding and computational efficiency. 
For instance, a 16-entry codebook with 8 bits per entry requires 128 bits per group, creating an inevitable trade-off between adaptivity and memory overhead. GOBO~\cite{zadeh2020gobo} employs the K-means algorithm to quantize weights and requires dequantization to \texttt{FP16} using a codebook lookup table before computation, resulting in high adaptivity but low computational efficiency. 
Conversely, Mokey~\cite{zadeh2022mokey} enhances the computation of clustering-based methods by using indices for centroid values via approximate calculations, though matrix multiplication still relies on floating-point units, increasing overhead compared to integer units. 
Furthermore, Mokey creates one \texttt{golden dictionary} for all activations and weights, akin to using a single data type in quantization, thus reducing adaptivity.


\paragraph{Takeaway 3.} Deploying the clustering-based adaptive methods under group quantization is challenging owing to the low encoding and computation efficiency. 


\begin{table}[t]
    \centering
    \small
    \renewcommand{\arraystretch}{1.2}
    \caption[]{Features of DNN accelerators with adaptive and flexible data types are summarized. Here, `Effi.' stands for efficiency, `Med.' for medium, and `LUT' for lookup table.}
  
    \resizebox{1.0\columnwidth}{!}{
      \begin{tabular}{c|cc|ccc|cc|c}
        \Xhline{1.2pt}
        \multirow{2}{*}{Architecture} & \multicolumn{2}{c|}{Encode} & \multicolumn{3}{c|}{Computation} & \multicolumn{2}{c|}{Decode} & \multirow{2}{*}{Adaptivity} \\ \cline{2-8}
        & Method & Effi. & Method & Bit & Effi. & Method & Effi. \\
        \Xhline{1.2pt}
        \texttt{INT} & Round & High & INT & 4 \& 8 & High & Calculation & High & Low \\ 
        OliVe~\cite{guo2023olive} & Search & Med. & INT & 4 \& 8 & High & Decoder & High & Med. \\ 
        ANT~\cite{guo2022ant} & Search & Med. & INT & 4 \& 8 & High & Decoder & High & Med. \\ 
        Mokey~\cite{zadeh2022mokey} & Cluster & Med. & Float & 4 \& 8 & Med. & Calculation & Med. & Low \\ 
        GOBO~\cite{zadeh2020gobo} & Cluster & Low & Float & 16 & Low & LUT & Med. & High \\ 
        \hline
        \multirow{2}{*}{\proj}  & Search  & Med.  & \multirow{2}{*}{INT} & \multirow{2}{*}{4 \& 8} & \multirow{2}{*}{High} & \multirow{2}{*}{Calculation} & \multirow{2}{*}{High} & \multirow{2}{*}{High} \\ \cline{2-3}
        &  Map &  High &  &&&\\ 
        \Xhline{1.2pt}
    \end{tabular}
    }
    \vspace*{0.1cm}
    \label{intro:dtype}
    \vspace*{-0.2cm}
  \end{table}

\subsection{Support for Real-time Quantization}
\label{sec:moti_kvcache}

The above group-wise diversity presents a challenge for both weights and KV cache.
In addition, KV cache faces challenges in real-time group-wise quantization because the KV cache is generated dynamically during LLM inference.


To facilitate low-precision computation in group-wise quantization, it is necessary to quantize K and V along the inner dimension. 
This requirement stems from the support for matrix inner product operations in most GPUs and TPUs. 
During these operations, the group-wise scaling factor can be extracted from the multiply-accumulate process. 
\Fig{fig:kv_process} depicts the computation process of K and V during the decode stage. We define the dimension used for matrix inner product operations as the inner dimension. 
The inner dimensions of the K and V caches differ; the K cache requires a transpose operation, whereas the V cache does not, complicating the situation.


In the prefill stage, K and V can easily compute the scaling factor for each group. 
During the decode stage, the newly generated K vector is concatenated along the inner dimension of the K cache, enabling immediate quantization. 
However, the newly generated V vector is associated with different groups, with only one element per group produced per iteration. This process prevents the scaling factor for the entire group from being obtained in a single iteration, posing a significant challenge for the real-time quantization of the V cache.


\begin{figure}[t] 
  \centering 
  % \includegraphics[width=1.0\linewidth]{fig/dse_kv_process.pdf}  
  \includegraphics[width=0.9\linewidth]{fig/moti_kv_dimension.pdf}  
  \caption{\small Comparison of group-wise K and V cache quantization. They have different inner dimensions due to the transposition of K (key).}

  \label{fig:kv_process}
  % \vspace*{-0.4cm}
\end{figure}


Given those challenges, we propose \proj with a mathematical encoding format that can fuse with integer computation and enhance the decoding efficiency.
In addition, this encoding format provides sufficient adaptivity for group-wise quantization.
Regarding the challenge in KV cache, \proj employs a real-time quantization engine that ensures efficient encoding and decoding for KV cache.
By addressing these challenges, \proj enables efficient low-bit group-wise quantization.


\section{FPGA-friendly Quantization Algorithm}
\label{sec:quantization_algo}



% We introduce our FPGA-friendly quantization algorithm in this  section.
% For the scattered outliers,  we propose a rotation-assisted PTQ method for Mamba to mitigate the impact of outliers.
% For SSM, we use power-of-two (PoT) quantization to reduce the large re-quant overhead.

\subsection{Rotation-assisted Linear Layer Quantization}

The rotation-assisted quantization method is first proposed in~\cite{ashkboos2024quarot}
for Transformer-based LLMs.
By multiplying the activation $X$ and weight $W$ with orthogonal matrix $Q$, i.e., $XQQ^TW$,
the result is identical with $XW$,
while the outliers in $X$ and $W$ are removed.
However, it is still unclear whether the rotation method is applicable to Mamba.
Therefore, we study the rotation equivalence in Mamba and propose a rotation-assisted method shown in Fig.~\ref{subfig: quantization_algorithm}.


% \begin{figure}[!tb]
% \centering
% \hspace*{-0.05\linewidth}  % 使用负的水平空白来左移图片
% \includegraphics[width=0.7\linewidth]{fig/rotation_PTQ_v2.pdf}
% \caption{The proposed quantization algorithm for Mamba.
% Both $Q$ and $H$ are orthogonal matrices to ensure computation correctness.
% $H$ is 
% $Q$ and $H$
% stand for multiplying the orthogonal matrix $Q$ and Hadamard matrix $H$.
% % \ml{Need to mention $H$ and $Q$ in the caption. \checkmark}
% % We illustrate the computation process within SSM
% % and label the dimension of the tensors,
% % where b is batch size, 
% % h is number of head,
% % p is head dimension,
% % n is state dimension.
% % We omit the token length dimension,
% % which is 1 in the decode stage.
% % \ml{Repetitive in Fig. 2.}
% } 
% \label{fig: quantization_algorithm}
% \end{figure}


\begin{figure}[!tb]
  \centering 
  % \hspace*{-0.5cm}
  \subfloat[]{
    \label{subfig: quantization_algorithm}
    \includegraphics[width=0.25\textwidth]{fig/rotation_PTQ_v4.pdf}
  }
  \subfloat[]{
    \label{subfig: weight_quant_error}
    \includegraphics[width=0.23\textwidth]{fig/weight_quant_error.pdf}
  }
  \caption{
(a) The proposed rotation-assisted quantization algorithm.
Both $Q$ and $H$ are Hadamard matrices to ensure computation correctness.
% $H$ is 
% $Q$ and $H$
% stand for multiplying the orthogonal matrix $Q$ and Hadamard matrix $H$.
(b) Quantization error of the output projection weight after only rotation or fusion and rotation.
% We measure the quantization error by the root squared error between the FP16 weight and quantized weight.
}
  \label{fig:fusion_cause_error}
\end{figure}

\begin{figure*}[!tb]
    \centering
    \includegraphics[width=1\linewidth]{fig/overall_new.pdf}
    \vspace{-10pt}
    \caption{Diagram of (a) the overall architecture, (b) SSMU, (c) MMU, (d) 128-point HTU, and (e) 40-point HTU.}
    \vspace{-10pt}
    \label{fig: Hardware Design}
\end{figure*}

We observe that the activations in the linear layers and SSM layer
have large number of outliers,
and the outliers of output projection layer exhibit scattered distribution
across different channels.
Rotation is helpful to remove outliers
since it amortizes large outliers with other elements.
For the input and output projection layers,
we apply rotation and remove the outliers as shown in Fig.~\ref{fig:activation_distribution}.
It is worth noting that
to rotate the activation before the output projection layer
we insert an on-line Hadamard transformation 
before it in Fig.~\ref{subfig: quantization_algorithm},
which can be efficiently performed by our customized
rotation unit in Sec.~\ref{sec:Hardware design}.
However, we find that SSM cannot be rotated since it
does not satisfy the rotation equivalence.
Specifically,
the original computation in SSM is Eq.~\ref{eq:rotate_ssm_a}.
Assuming we can rotate hidden state $h_t$ to remove the outliers,
i.e., multiply $h_t$ by Hadamard matrix $H$,
we can derive Eq.~\ref{eq:rotate_ssm_b} and Eq.~\ref{eq:rotate_ssm_c}.
However, Eq.~\ref{eq:rotate_ssm_c} cannot derive Eq.~\ref{eq:rotate_ssm_d}
because EM does not satisfy matrix associative property.
Thus we cannot derive Eq.~\ref{eq:rotate_ssm_d} from Eq.~\ref{eq:rotate_ssm_a}, i.e., SSM does not satisfy the rotation equivalence.
\begin{subequations}
\begin{align}
    & h_{t} = \bar{A} \odot h_{t-1}+\bar{B} \odot X_{t}  \label{eq:rotate_ssm_a}\\
    & h_{t}H = (\bar{A} \odot h_{t-1} + \bar{B} \odot X_{t})H \label{eq:rotate_ssm_b}\\
    & h_{t}H = (\bar{A} \odot h_{t-1})H + (\bar{B} \odot X_{t})H \label{eq:rotate_ssm_c}\\
    & h_{t}H = \bar{A} \odot (h_{t-1}H) + \bar{B} \odot (X_{t}H) \label{eq:rotate_ssm_d}
\end{align}
\end{subequations}

To reduce the computation overhead, we try to fuse rotation with neighboring operations as much
as possible. 
As shown in Fig.~\ref{subfig: quantization_algorithm}, we can fuse the first rotation
with the embedding table (i.e., \textcircled{1}),
the last rotation with the LM head (i.e., \textcircled{5}),
as well as the rotations before and after the output projection layers in each Mamba block (i.e., \textcircled{4}).
For the rotation next to the first RMSNorm operator (i.e., \textcircled{2}), to ensure
the computational invariance, we need to split the scaling factor, i.e., $D$, of the RMSNorm first,
and then, fuse it with the weights of input projection.
For the rotation next to the second RMSNorm operator (i.e., \textcircled{3}), we find whether
or not to fuse the scaling factor of the RMSNorm to the weight of output projection does not impact the computational invariance,
while fusion introduces a larger quantization error as in Fig.~\ref{subfig: weight_quant_error}. Hence, we choose not to fuse
the scaling factor of the second RMSNorm.
In our algorithm, only rotation \textcircled{3} needs to be computed online, which incurs small computation
overhead with our customized FPGA module support.

% Moreover, different from Transformer-based LLMs,
% we find that the scaling parameters of RMSNorm,
% denoted as $D=diag(\alpha)$, should apply different fusion strategies.
% RMSNorm before the input projection layer should be fused
% into the weight to ensure the computational invariance following~\cite{ashkboos2024slicegpt}.
% However, the scaling parameters of RMSNorm  before the output projection layer should not be fused into the subsequent weight.
% This is because although both fusion (Eq.~\ref{equation: in-block RMSNorm fusion}) 
% and not fusion (Eq.~\ref{equation: in-block RMSNorm w/o fusion}) ensure the computational invariance,
% fusion will introduce larger weight quantization error
% as shown in Fig.~\ref{subfig: weight_quant_error}.
% \begin{subequations}
% % \vspace{-10pt}
% \begin{align}
%     & \frac{X}{\|X \|_{2}} HH^{\top} (D W_{out})Q=\frac{X}{\|X \|_{2}} D W_{out}Q= X_{l+1}Q
%     \label{equation: in-block RMSNorm fusion}\\
%     & \frac{X}{\|X \|_{2}} D HH^{\top} W_{out}Q=\frac{X}{\|X \|_{2}} D W_{out}Q= X_{l+1}Q \label{equation: in-block RMSNorm w/o fusion}
% \end{align}
% \end{subequations}
% 
% The main reason is that the scaling parameters of the RMSNorm
% before output projection layer
% learned to have large variations.
% Multiplying it to weight, i.e., $DW_{out}$ will increase the quantization difficulty 
% since we quantize weight for each output channel as shown in Fig.~\ref{subfig: fuse_rmsnorm_to_weight}.
% It is worth noting that the fusion and rotation of weight can be performed offline, 
% which does not introduce extra overhead during inference.


\subsection{FPGA-friendly SSM Quantization}

In order to quantize SSM to reduce the heavy hardware cost by FP computations,
we leverage INT8 per-group quantization
to strike a balance between accuracy and hardware efficiency.
However, directly quantizing SSM introduces
large re-quantization overhead as shown in Fig.~\ref{fig:challenge2},
which is
because EM has larger re-quantization overhead than MM intrinsically since there is no reduction in EM.
% due to the large activation dimension in SSM.
% As shown in Fig.~\ref{fig:requant_overhead},
% EM has larger re-quantization overhead than MM intrinsically since there is no reduction in EM,
% which leads to larger output size with the same input size.
To this end,
we propose to use PoT quantization for SSM,
through which re-quantization can be implemented in bit-shifting
rather than multiplication
thus reducing the re-quantization overhead significantly.
% The re-quantization overhead is defined as:
% \begin{equation}
% \label{equation: quantization overhead}
% r=\frac{Requant \, operations}{MM \, operations}q
% \end{equation}
% where $q=(16/b)^2$, 16 is the bit precision for FP16,
% and b is the quantization bit precision.
% $r_{EM}$ is approximately $D\times$ larger than $r_{MM}$
% with the same input size

% \begin{figure}[!tb]
% \centering
% \includegraphics[width=\linewidth]{fig/requant.pdf}
% \caption{Re-quantization overhead of (a) MM and (b) EM.} 
% \label{fig:requant_overhead}
% \end{figure}
% \begin{figure}[!tb]%
%   \centering 
%   % \hspace*{-0.5cm}
%   \subfloat[]{
%     \label{subfig: weight_quant_error}
%     \includegraphics[width=0.24\textwidth]{fig/weight_quant_error.pdf}
%   }
%   \subfloat[]{
%     \label{subfig: fuse_rmsnorm_to_weight}
%     \includegraphics[width=0.24\textwidth]{fig/fuse_rmsnorm_to_weight_v1.pdf}
%   }
%   \caption{
% (a) Quantization error of the output projection weight after only rotation or fusion and rotation.
% (b) Fuse the scaling parameter of RMSNorm to weight.
% \ml{Change ``out project'' and ``in project'' to ``output projection'' and ``input projection''.}
% \ml{Figure b is not clear.}
% % We measure the quantization error by the root squared error between the FP16 weight and quantized weight.
% }
%   \label{fig:fusion_cause_error}
% \end{figure}













% To tackle the activation outliers challenge in Sec.~\ref{sec:motivation},
% we propose a rotation-based PTQ method for Mamba.
% Although the rotation-based method is recently proposed for Transformer~\cite{ashkboos2024quarot,liu2024spinquant}, 
% it is non-trivial to apply it to Mamba.
% We are the first to study it both theoretically and empirically.
% The schematic of the quantized Mamba block is shown in Fig.~\ref{fig: quantization_algorithm}.
% Through our proposed method,
% we can achieve the first accurate and fully quantized W8A8 and W4A4 Mamba.
% The key insight is that by multiplying the weight and
% activation with the orthogonal matrix,
% the output is still identical
% while the outliers in weight and activation can be removed 
% reducing quantization difficulty.

% We introduce our FPGA-friendly quantization algorithm for Mamba.
% % The key insight
% % is that by multiplying the weight and activation with the orthogonal matrix,
% % the output is still identical
% % while the outliers in weight and activation can be removed 
% % thus reducing quantization difficulty.
% Specifically, our proposed quantization algorithm consists of two steps, i.e., rotation and quantization.

% \ml{Rotation is not explained yet. What is rotation? It makes people very confusing. \checkmark}

% \ml{Right now, for the quantization algorithm, it is unclear how it considers the FPGA hardware characteristics? \checkmark}




% \subsection{Rotation}
% \label{subsec:rotation_model}

% % Leveraging the computational invariance~\cite{ashkboos2024slicegpt},
% % we post-multiplied the weight of the in project and out project layer
% % by the random orthogonal matrix $Q$.
% % The embedding and the LM head should also be multiplied by $Q$.

% \ml{If you want to mention rotation, then, you need to make it clear: 1) what is rotation, 2) why it helps with quantization, and 3) what is the difference compared to LLM?}

% Rotation should keep the network output equivalent and 
% remove the outliers in weights and activations to reduce the 
% quantization difficulty in the quantization step. 
% The computational invariance~\cite{ashkboos2024slicegpt} studies
% the RMSNorm-connected Transformer network and 
% shows that the scaling parameters of RMSNorm between-block \ml{cross-block? $\checkmark$} should be 
% absorbed into the subsequent weights to ensure the computational invariance.
% % such that the rotation of weights will not affect the network output.
% However, there are two types of RMSNorm in Mamba, i.e., 
% the between-block RMSNorm which is right before the in project layer, and the in-block RMSNorm which is right before the out project layer.
% % They have different requirements for fusion strategies.
% We find that the scaling parameters of in-block RMSNorm
% can not be fused into subsequent weights.

% \ml{Need to explain why do we want to fuse rotation?}
% \ml{This section takes too much space and yet, it is unclear how it is related to FPGA?}

% \textbf{RMSNorm Fusion}

% To keep the computation invariant, 
% the scaling parameters ($D=diag(\alpha)$) of the between-block RMSNorm
% should be fused into the weight of in project layer.
% This is because given input $X_lQ$, \ml{$Q$ is not defined} the output of the in project layer
% can be calculated as:
% \begin{equation}
% \label{equation: between-block RMSNorm}
% \frac{X_l Q}{\|X_l Q\|_{2}} Q^{\top} (D W_{in})=\frac{X_l}{\|X_l Q\|_{2}} D W_{in}=\frac{X_l}{\|X_l\|_{2}} D W_{in}
% \end{equation}
% where $\|Q\|_{2}=1$ since $Q$ is orthogonal \ml{orthogonal or orthonormal?} matrix.
% The term on the far right \ml{what is far right?} of Eq.~\ref{equation: between-block RMSNorm} is the original output of the in project layer.
% However, the scaling parameters of in-block RMSNorm should not be fused into the weight of out project layer.
% This is because although both fusion (Eq.~\ref{equation: in-block RMSNorm fusion}) 
% and not fusion (Eq.~\ref{equation: in-block RMSNorm w/o fusion}) ensure the computation invariance,
% fusion will make weights harder to quantize,
% against our intent.
% % The weight quantization error of each layer is shown in Fig.~\ref{fig:fusion_cause_error}.
% As shown in Fig.~\ref{subfig: weight_quant_error}, 
% RMSNorm fusion causes consistently larger quantization error than the case without fusion.

% \begin{subequations}
% \begin{align}
%     & \frac{X}{\|X \|_{2}} HH^{\top} (D W_{out})Q=\frac{X}{\|X \|_{2}} D W_{out}Q= X_{l+1}Q
%     \label{equation: in-block RMSNorm fusion}\\
%     & \frac{X}{\|X \|_{2}} D HH^{\top} W_{out}Q=\frac{X}{\|X \|_{2}} D W_{out}Q= X_{l+1}Q \label{equation: in-block RMSNorm w/o fusion}
% \end{align}
% \end{subequations}


% % \begin{equation}
% % \label{equation: in-block RMSNorm fusion}
% % \frac{X}{\|X \|_{2}} HH^{\top} (D W_{out})Q=\frac{X}{\|X \|_{2}} D W_{out}Q= X_{l+1}Q
% % \end{equation}

% % \begin{equation}
% % \label{equation: in-block RMSNorm w/o fusion}
% % \frac{X}{\|X \|_{2}} D HH^{\top} W_{out}Q=\frac{X}{\|X \|_{2}} D W_{out}Q= X_{l+1}Q
% % \end{equation}

% % \begin{figure}[!tb]%
% %   \centering 
% %   % \hspace*{-0.5cm}
% %   \subfloat[]{
% %     \label{subfig: weight_quant_error}
% %     \includegraphics[width=0.24\textwidth]{fig/weight_quant_error.pdf}
% %   }
% %   \subfloat[]{
% %     \label{subfig: fuse_rmsnorm_to_weight}
% %     \includegraphics[width=0.24\textwidth]{fig/fuse_rmsnorm_to_weight_v1.pdf}
% %   }
% %   \caption{
% % (a) Quantization error of the out project weight after only rotation or fusion and rotation.
% % (b) Fuse the scaling parameter of RMSNorm to weight.
% % % We measure the quantization error by the root squared error between the FP16 weight and quantized weight.
% % }
% %   \label{fig:fusion_cause_error}
% % \end{figure}

% % \begin{figure}[!tb]
% % \centering
% % \includegraphics[width=\textwidth]{fig/fuse_rmsnorm_to_weight.pdf}
% % \caption{(a) The model architecture of Mamba2~\cite{dao2024transformers}. (b) SSM runtime and memory proportion of the entire Mamba block in Mamba with different model sizes.} 
% % \label{fig:model_arch_and_ssm_propotion}
% % \end{figure}

% This is because compared with the between-block RMSNorm, 
% the scaling parameters of the in-block RMSNorm
% learned to have much larger variations.
% Multiplying it to weight, i.e., $DW_{out}$ will increase the quantization difficulty 
% since we quantize weight for each output channel as shown in Fig.~\ref{subfig: fuse_rmsnorm_to_weight}.
% It is worth noting that the fusion of between-block RMSNorm and rotation of weight can be performed offline, 
% which does not introduce extra overhead during inference.

% % Multiplying it to weight, i.e., $DW_{out}$ will introduce more outliers
% % and increase the quantization error 
% % even after being multiplied by the Hadamard matrix $H$ and random orthogonal matrix $Q$, which is shown in Fig.~\ref{fig:fusion_cause_error}.
% % Therefore, the scaling parameters of in-block RMSNorm should not be fused.
% % Fusion of the between-block RMSNorm and rotation of weight can be performed offline, 
% % which does not introduce extra overhead to inference.

% Furthermore, activations should also be rotated to remove outliers. 
% As shown in Fig.~\ref{fig: quantization_algorithm},
% since the input of the in project layer is already rotated,
% only one online Hadamard Transformation is required in our method,
% % i.e., multiply the input of the out project layer by the Hadamard matrix $H$,
% compared with four on-line Hadamard Transformations in the Transformer-based model~\cite{ashkboos2024quarot}.
% The overhead is negligible with the Fast Hadamard Transformation method~\cite{tseng2024quip}.
% Considering an activation with dimension $ \mathbb{R}^{l \times n} $, when $n$ is a power of two,
% the Hadamard Transform can be computed with Fast Walsh-Hadamard Transform (FHT)~\cite{fino1976unified}
% in $O(n \log n)$ instead of $O(n^2)$.
% When $n$ is not the power of 2,
% they factorize $n=pq$ where $p$ is the largest power of 2
% such that a known Hadamard matrix of size $q$ exists. 
% For example, for Mamba2-2.7B, we factorize 5120 into $128\times40$.
% In this way, the Hadamard matrix $H_n$ can be constructed by 
% $H_p \otimes H_q$.
% Inspired by this, 
% we propose an efficient Hadamard transformation circuit design
% in Section~\ref{sec:Hardware design}.
% % To reduce the overhead,
% % we propose an efficient Hadamard transformation circuit design
% % in Section~\ref{sec:Hardware design}
% % leveraging the Fast Hadamard Transformation method~\cite{tseng2024quip}.
% With the rotation mentioned above, 
% the computation invariance is preserved and
% outliers in weights and activations are removed
% as shown in Fig.~\ref{fig:activation_distribution}, leading to easier quantization.

% \begin{figure}[!tb]%
%   \centering 
%   \hspace*{-0.5cm}
%   % \subfloat[In Project Activation]{
%   %   \label{subfig: fusion}
%   %   \includegraphics[width=0.24\textwidth]{fig/2.8b_activation_layer10_input_sample1_v1.pdf}
%   % }
%   % \subfloat[Rotated In Project Activation]{
%   %   \label{subfig: without_fusion}
%   %   \includegraphics[width=0.24\textwidth]{fig/rotate_2.8b_activation_layer10_input_sample1.pdf}
%   % }
  
%   % \hspace*{-0.5cm}
  
%   % \subfloat[Out Project Activation]{
%   %   \label{subfig: fusion}
%   %   \includegraphics[width=0.24\textwidth]{fig/2.8b_activation_layer10_output_sample1_v1.pdf}
%   % }
%   % \subfloat[Rotated Out Project Activation]{
%   %   \label{subfig: without_fusion}
%   %   \includegraphics[width=0.24\textwidth]{fig/rotate_2.8b_activation_layer10_output_sample1.pdf}
%   % }

%   % Draw a frame as placeholder for your image
%   \fbox{\rule{0pt}{3in} \rule{0.4\textwidth}{0pt}}
  
%   \caption{
% Activation distribution in Mamba2-2.7B before and after rotation.
% }
%   \label{fig:activation_distribution}
% \end{figure}




% \textbf{Can SSM be rotated?}
% Since activation in SSM also has outliers shown in Table~\ref{tab: kurtosis},
% a natural question arises:
% can we rotate SSM to remove the outliers and reduce the quantization difficulty?
% Through the following study, the answer is negative.
% Assume that we can rotate hidden state $h_t$ to remove the outliers,
% i.e., multiply $h_t$ by Hadamard matrix $H$,
% in order to ensure Eq.~\ref{eq:rotate_ssm_a} holds,
% the right-hand side should also be multiplied by $H$ leading to Eq.~\ref{eq:rotate_ssm_b} and Eq.~\ref{eq:rotate_ssm_c}.
% However, Eq.~\ref{eq:rotate_ssm_c} cannot derive to Eq.~\ref{eq:rotate_ssm_d}
% because the element-wise multiplication does not satisfy matrix associative property.
% Only when Eq.~\ref{eq:rotate_ssm_d} holds can we rotate SSM successfully
% since we intend to remove the outliers in $h_{t-1}$ which is also $h_t$ in the last time step and $X_t$
% such that we can quantize them with low error
% and perform multiplication in low bit precision.
% Thus, the assumption does not hold
% because Eq.~\ref{eq:rotate_ssm_d} is not valid.
% Fortunately, we discover that with an INT8 dynamic per group quantization scheme in Section~\ref{subsec:quantization},
% we can obtain an almost lossless quantized SSM compared to the FP counterpart.
% \begin{subequations}
% \begin{align}
%     & h_{t} = \bar{A} \odot h_{t-1}+\bar{B} \odot X_{t}  \label{eq:rotate_ssm_a}\\
%     & h_{t}H = (\bar{A} \odot h_{t-1} + \bar{B} \odot X_{t})H \label{eq:rotate_ssm_b}\\
%     & h_{t}H = (\bar{A} \odot h_{t-1})H + (\bar{B} \odot X_{t})H \label{eq:rotate_ssm_c}\\
%     & h_{t}H = \bar{A} \odot (h_{t-1}H) + \bar{B} \odot (X_{t}H) \label{eq:rotate_ssm_d}
% \end{align}
% \end{subequations}




% \subsection{Weight and Activation Quantization}
% \label{subsec:quantization}

% After rotation, 
% both activation and weight quantization benefit from the
% outlier removal, thus leading to smaller quantization error,
% e.g., the 4-bit quantization error reduced from 19.5 to 13.1 as shown in Table~\ref{tab:quant_error}. \ml{XXX $\checkmark$}
% For weight and activation in the linear layers and conv1d layers,
% we quantize them to INT8 (W8A8) or INT4 (W4A4)
% and we set the scaling factor to dyadic number~\cite{hu2024llm},
% which is calculated as $m/2^k$,
% where m and k are integers. \ml{Why do we want to do this?}
% Therefore, when calculating the product of the scaling factors for weights and activations, 
% we can use integer multiplication and bit-shift
% instead of FP multiplication,
% which is friendly for accelerators.
% For the non-linear operations,
% we leverage the lookup table method in~\cite{guo2024hg},
% which is quite efficient for FPGA.
% In SSM shown in Fig.~\ref{fig: quantization_algorithm}, 
% $A$ and $D$ are weights which have negligible number of parameters compared to the weights of linear layers,
% others are activations, i.e., input-dependent.
% We quantize them using per group power-of-two (PoT) quantization (group size=128),
% in which the scaling factor is the power of two,
% to save the computation overhead of scaling factors.
% In particular, the group is divided along $p$ dimension,
% which corresponds to our dataflow in Sec.~\ref{sec:Hardware design}.
% Moreover, we quantize parameter $A$ to INT32 since it is important
% in controlling how the previous tokens affect the current token,
% and quantize other activations to INT8,
% the computation of which still accounts for a small portion even in W4A4 settings,
% since the number of operations of SSM only accounts for 1/12 of the Mamba block.




\subsection{Execution Phase}
\label{sec. tic execution}

\subsubsection{Intent Detection}
\coach monitors the team during task execution, identifying potential misalignments in team members' intents and computing timely interventions. This capability is enabled by \tic, a framework that has been experimentally shown to generate task-time interventions that enhance teamwork among AI agents~\cite{seo2023automated}. We extend this framework to develop an AI-enabled coaching system for teams that include human members.
During task execution, \coach can observe team members' states and actions, but their intents (a latent variable) remain unobservable. While \coach leverages a human annotator to obtain partial intent annotations during the training phase, involving a human in the loop during task execution is impractical. Therefore, to infer team members' intents, \coach frames the problem as one of Bayesian filtering. Specifically, given the learned model of team behavior $(\mathcal{H}_j \forall j = 1:n)$ and the partial $(s,a)$-trajectory of the team's task execution, \coach employs the forward-backward algorithm to infer each team member's current intent $\hat{x}$. 

\subsubsection{Intervention Generation}
\coach next uses the inferred intents to assess whether the team is aligned. If the intended plans of the team members are likely to lead to suboptimal task performance, \coach intervenes by weighing the costs and benefits of the intervention. Under the \tic framework, determining this balance requires an intervention strategy, which can be hand-crafted or learned. For \coach, we opt for a learned, value-based strategy to minimize human effort in intervention generation. Specifically,\footnote{Since this computation relies on observations, task model, and the learned model of team behavior, it requires no additional human input or domain-specific knowledge.}
 \begin{itemize}
    \item \coach first computes the expected return $(g)$ conditioned on the inferred intent: $g(\hat{x}|s) = E_{\mathcal{H}}[\sum_t \gamma^t r_t |s, \hat{x}]$.
    \item Next, \coach computes the intent values and return for a hypothetical fully aligned team as $x^* = \arg\max_{x} g(x|s)$ and $g(x^*|s) = E_{\pi, \zeta}[\sum_t \gamma^t r_t | s, x^*]$, respectively. We define the benefit of an intervention as the difference between the optimal and estimated return: $g(x^*|s) - g(\hat{x}|s)$.
    \item Finally, if the benefit of an intervention exceeds its cost $c$ by a pre-defined threshold (i.e., $g(x^*|s) - g(\hat{x}|s) > c + \delta$), then \coach prompts the team to pause, reflect on their plans, and recommends the optimal plan corresponding to $x^*$.
\end{itemize}
Choosing an appropriate cost $(c)$ and threshold $(\delta)$ for interventions is crucial, as unnecessary or incorrect interventions could impair team performance and reduce human trust in, and adoption of, \coach. 
\cref{sec. validation} outlines the approach for selecting these hyperparameters for our implementation and evaluation of \coach.

\subsubsection{Intervention Delivery}
To assist human team members, in addition to generating interventions, Socratic requires effective mechanisms for delivering these instructions. In this work, we utilize an interactive user interface for delivering interventions, as illustrated in \cref{fig. aicoach ui} and detailed in \cref{sec: data collection}. Since human team members can choose whether to accept the AI-generated recommendations, Socratic incorporates a hyperparameter $p_a$, which models the probability of a human accepting its recommendation.

%\clearpage
\section{Evaluation}
We provide three sets of insights into this section, organised as \textit{findings (F*)}. We quantitatively study the effect of the adversarial and counterfactual perturbations on the performance of informal reasoners and autoformalisation methods. Then, we dive deeper into method variants. Finally, 
we analyse the nature of formalisation errors made by the models.

\subsection{Robustness Analysis}
\paragraph{\textbf{\emph{F1: Noise perturbations have a stronger effect on formalisation methods than informal \ac{LLM} reasoners.}}}
Table~\ref{tab:distraction_k4_formalisation} shows that, on average, the accuracy of both direct and \ac{CoT} informal reasoning remains between $73\%$ and $74\%$ in the face of added noise. While the autoformalisation method performs similarly to informal reasoners on the original dataset, its performance decreases between $4\%$ and $11\%$. The accuracy drops especially with logical (L) and tautological (T) distractions, whose logical language formats trick the \ac{LLM} into formalizing the noisy clauses. On the other hand, the linguistically complex and more natural sentences of encyclopedic distractions show a minor effect, suggesting that \acp{LLM} successfully avoids formalizing the more complicated sentences.

\paragraph{\textbf{\emph{F2: All \ac{LLM}-based reasoning methods suffer a drop for counterfactual perturbations.}}} % influence .}}}
Table~\ref{tab:distraction_k4_formalisation} shows that counterfactual statements cause a significant decrease in performance for both the informal reasoners and autoformalisation methods of between $12\%$ and $13\%$ on average. 
Moreover, this observation also holds for all tested models, i.e., none are robust towards counterfactual perturbations across every evaluated dimension. Even the strongest model, GPT 4o-mini, yields a performance of 63-68\%, which is relatively close to the random performance of 50\%. The high impact of counterfactual statements (the single ``not'' inserted) could be due to the inability of \acp{LLM} to overwrite prior knowledge with explicitly stated information or memorization of the answers. We study the error sources further in §\ref{subsec:errors}.  

\noindent \paragraph{\textbf{\emph{F3: Introducing multiple noise sentences has an effect only for logical distractions.}}}
We show the impact of introducing between one and four sentences for the two top-performing autoformalisation models in Figure~\ref{fig:length_distraction}. The figure shows similar trends with and without counterfactual perturbations.
As additional logical distractions are introduced, the model performance consistently decreases. Tautological (T) distractions lead to a decline in accuracy with a single disruptive sentence, yet adding more noise does not worsen the outcome. 
The tautological corpus introduces truth constants for all sentences as a persistent unseen logical construct. Given that this leads only to a decrease for a single occurrence, we can assume that a model can consistently handle the same unseen logical construct. In contrast, the logical corpus increases the chance of adding text, requiring new, previously unseen reasoning constructs for each added sentence. The impact of encyclopedic noise remains negligible, generalising F1 to $k$ sentences. Similarly, counterfactual perturbations remain much more effective for all settings, generalising F2.

\begin{table}[!t]
\small
\setlength{\modelspacing}{2pt}
\setlength{\tabcolsep}{1.7pt} % Default value: 6pt
\setlength{\belowrulesep}{4pt}
\begin{threeparttable}
    \centering
    \begin{tabular}{cc l r rrr @{\quad} rrrr}
\toprule
\multirow{2}{*}{} & \multirow{2}{*}{} & Reasoning & \multirow{2}{*}{O} & \multicolumn{3}{c}{Distraction} & \multicolumn{4}{c}{Counterfactual} \\
 & & Format & & E& L & T & $\text{O}_C$ & $\text{E}_C$& $\text{L}_C$ & $\text{T}_C$\\
\midrule
\multirow{6}{*}{\rotatebox{90}{Gemma-2}} & \multirow{3}{*}{\rotatebox{90}{9b}}
   & Informal (direct) & \textbf{0.78} & \textbf{0.80} & \textbf{0.79} & \textbf{0.77} & 0.58 & 0.52 & 0.50 & 0.59 \\
 & & Informal (CoT) & 0.72 & 0.78 & 0.73 & 0.76 & 0.61 & \textbf{0.57} & \textbf{0.60} & \textbf{0.66} \\
 & & Formal (FOL) & 0.62 & 0.58 & 0.52 & 0.53 & \textbf{0.63} & 0.52 & 0.46 & 0.46 \\[\modelspacing]
\cmidrule{2-11}
 & \multirow{3}{*}{\rotatebox{90}{27b}} 
   & Informal (direct) & 0.71 & 0.69 & \textbf{0.66} & \textbf{0.68} & 0.59 & 0.51 & 0.54 & 0.59 \\
 & & Informal (CoT) & 0.66 & 0.65 & 0.64 & 0.63 & 0.62 & 0.58 & \textbf{0.62} & \textbf{0.64} \\
 & & Formal (FOL) & \textbf{0.74} & \textbf{0.74} & 0.61 & 0.61 & \underline{\textbf{0.72}} & \underline{\textbf{0.67}} & 0.58 & 0.51 \\[\modelspacing]
\midrule
\multirow{6}{*}{\rotatebox{90}{Mistral}} & \multirow{3}{*}{\rotatebox{90}{7B}} 
   & Informal (direct) & 0.77 & \textbf{0.77} & 0.75 & \textbf{0.79} & \textbf{0.63} & \textbf{0.54} & \textbf{0.54} & \textbf{0.66} \\
 & & Informal (CoT) & \textbf{0.79} & 0.75 & \textbf{0.77} & 0.78 & 0.55 & 0.52 & \textbf{0.54} & 0.58 \\
 & & Formal (FOL) & 0.62 & 0.58 & 0.54 & 0.57 & 0.50 & \textbf{0.54} & 0.51 & 0.52 \\[\modelspacing]
\cmidrule{2-11}
 & \multirow{3}{*}{\rotatebox{90}{Small}} 
   & Informal (direct) & \textbf{0.77} & \textbf{0.76} & \textbf{0.76} & \textbf{0.75} & 0.61 & 0.51 & 0.56 & 0.59 \\
 & & Informal (CoT) & 0.72 & 0.72 & 0.72 & 0.71 & \textbf{0.62} & \textbf{0.59} & \textbf{0.62} & \textbf{0.68} \\
 & & Formal (FOL) & 0.68 & 0.59 & 0.53 & 0.64 & 0.54 & 0.55 & 0.49 & 0.51 \\[\modelspacing]
\midrule
\multirow{6}{*}{\rotatebox{90}{Llama-3.1}} & \multirow{3}{*}{\rotatebox{90}{8B}} 
   & Informal (direct) & 0.63 & 0.61 & 0.64 & 0.66 & 0.61 & \textbf{0.62} & 0.59 & 0.61 \\
 & & Informal (CoT) & 0.73 & \textbf{0.73} & \textbf{0.71} & \textbf{0.72} & \textbf{0.62} & 0.59 & \textbf{0.61} & \textbf{0.65} \\
 & & Formal (FOL) & \textbf{0.77} & 0.71 & 0.63 & 0.52 & 0.60 & 0.58 & 0.55 & 0.52 \\[\modelspacing]
\cmidrule{2-11}
 & \multirow{3}{*}{\rotatebox{90}{70B}} 
   & Informal (direct) & 0.77 & 0.74 & 0.74 & 0.73 & 0.62 & 0.53 & 0.56 & 0.64 \\
 & & Informal (CoT) & \textbf{0.78} & \textbf{0.75} & \textbf{0.76} & \textbf{0.76} & 0.64 & 0.61 & \textbf{0.66} & \underline{\textbf{0.73}} \\
 & & Formal (FOL) & 0.74 & 0.73 & 0.71 & 0.71 & \textbf{0.66} & \textbf{0.62} & 0.59 & 0.57 \\[\modelspacing]
 \midrule
\multirow{3}{*}{\rotatebox{90}{GPT}} & \multirow{3}{*}{\rotatebox{90}{4o-mini}} 
   & Informal (direct) & 0.78 & 0.77 & 0.79 & 0.79 & 0.64 & 0.61 & 0.61 & 0.63 \\
 & & Informal (CoT) & 0.80 & 0.80 & \underline{\textbf{0.81}} & \underline{\textbf{0.82}} & \textbf{0.68} & \textbf{0.63} & \underline{\textbf{0.68}} & \textbf{0.64} \\
 & & Formal (FOL) & \underline{\textbf{0.84}} & \underline{\textbf{0.82}} & 0.73 & 0.79 & 0.63 & 0.62 & 0.57 & 0.54 \\[\modelspacing]
 \midrule
\multicolumn{2}{c}{\multirow{3}{*}{\textbf{Avg}}} 
 & Informal (direct) & 0.74 & 0.73 & 0.73 & 0.73 & 0.61 & 0.55 & 0.56 & 0.62 \\
 & & Informal (CoT) & 0.74 & 0.74 & 0.73 & 0.74 & 0.62 & 0.58 & 0.62 & 0.65 \\
  & & Formal (FOL) & 0.72 & 0.68 &	0.61 & 0.62 & 0.61 & 0.59 & 0.54 & 0.52 \\
\bottomrule
\end{tabular}
\caption{Accuracies of informal and autoformalisation-based deductive reasoners. The best overall model per dataset is underlined; the best model version is marked in bold.}
\label{tab:distraction_k4_formalisation}
\end{threeparttable}
\end{table} 

\begin{figure}[!t]
    \centering
    \scriptsize
    \begin{tikzpicture}
        \begin{axis}[name=gpt,
            title={GPT-4o-mini},
            width=0.6\linewidth,
            height=0.6\linewidth,
            xlabel={\# Noise sentences},
            ylabel={Accuracy},
            xmin=-0.1, xmax=4.1,
            ymin=0.5, ymax=0.9,
            xtick={1,2,4},
            ytick={0.55, 0.6, 0.65, 0.75, 0.8, 0.85},
            title style={yshift=-0.6em},
            legend style={at={(1,-0.15)},
	           anchor=north,legend columns=-1},
            x label style={at={(axis description cs:1,-0.05)},anchor=north},
            y label style={at={(axis description cs:-0.15,0.5)},anchor=south},
            ymajorgrids=true,
            grid style=dashed,
        ]
            \addplot[color=blue, mark=square,]
                coordinates {
                (0,0.848076939582825)(1,0.823076903820038)(2,0.826923072338104)(4,0.821153819561005)
                };
            \addplot[color=red, mark=triangle,]
                coordinates {
                (0,0.848076939582825)(1,0.817307710647583)(2,0.801923096179962)(4,0.759615361690521)
                };
            \addplot[color=green, mark=diamond,] 
                coordinates {
                (0,0.848076939582825)(1,0.767307698726654)(2,0.769230782985687)(4,0.803846180438995)
                };
            \addplot[color=blue, mark=square*] 
                coordinates {
                (0,0.627777755260468)(1,0.622222244739533)(2,0.600000023841858)(4,0.633333325386047)
                };
            \addplot[color=red, mark=triangle*,] 
                coordinates {
                (0,0.627777755260468)(1,0.611111104488373)(2,0.611111104488373)(4,0.594444453716278)
                };
            \addplot[color=green, mark=diamond*,] 
                coordinates {
                (0,0.627777755260468)(1,0.572222232818604)(2,0.538888871669769)(4,0.555555582046509)
                };
                \legend{E,L,T,$\text{E}_C$, $\text{L}_C$ , $\text{T}_C$}
        \end{axis}

        \begin{axis}[name=llama, at={($(gpt.east)+(0.1cm,0)$)},anchor=west,
            title={Llama 3.1 70b},
            width=0.6\linewidth,
            height=0.6\linewidth,
            xmin=-0.1,, xmax=4.1,
            ymin=0.5, ymax=0.9,
            xtick={1,2,4},
            ytick={0.55, 0.6, 0.65, 0.75, 0.8, 0.85},
            title style={yshift=-0.6em},
            yticklabel=\empty,
            ymajorgrids=true,
            grid style=dashed,
        ]
            \addplot[color=blue, mark=square,]
                coordinates {
                (0,0.838461518287659)(1,0.817307710647583)(2,0.805769205093384)(4,0.817307710647583)
                };
            \addplot[color=red, mark=triangle,]
                coordinates {
                (0,0.838461518287659)(1,0.819230794906616)(2,0.803846180438995)(4,0.771153867244721)
                };
            \addplot[color=green, mark=diamond,]
                coordinates {
                (0,0.838461518287659)(1,0.803846180438995)(2,0.807692289352417)(4,0.805769205093384)
                };
            \addplot[color=blue, mark=square*]
                coordinates {
                (0,0.627777755260468)(1,0.622222244739533)(2,0.577777802944183)(4,0.594444453716278)
                };
            \addplot[color=red, mark=triangle*,]
                coordinates {
                (0,0.627777755260468)(1,0.583333313465118)(2,0.561111092567444)(4,0.577777802944183)
                };
            \addplot[color=green, mark=diamond*,]
                coordinates {
                (0,0.627777755260468)(1,0.627777755260468)(2,0.566666662693024)(4,0.577777802944183)
                };
        \end{axis}
    \end{tikzpicture}
    \caption{Influence of the number of noisy sentences for FOL.}
    \label{fig:length_distraction}
\end{figure}



\subsection{Impact of Method Design}
\paragraph{\textbf{\emph{F4: \ac{CoT} prompting is most impactful when both noise and counterfactual perturbations are applied.}}}
The accuracies for the individual \acp{LLM} in Table~\ref{tab:distraction_k4_formalisation} show that the impact of \ac{CoT} is negligible for noise-only datasets (first four columns). Meanwhile, the benefit from \ac{CoT} is most pronounced in the datasets that combine noise and counterfactual perturbations.
The better-performing informal prompting strategy for a model remains stable for all types of distractions. Still, the decline in performance due to counterfactuals leads to a less consistent preference for a specific prompting style.

\paragraph{\textbf{\emph{F5: The best-performing grammar differs per model and is unstable across data versions.}}}

The evaluation of different logical forms for formal \ac{LLM}-based reasoning in Table~\ref{tab:distraction_k4_logical_form} shows the preference of some models for specific syntactic formats.
Llama 3.1 70B has a considerable improvement of $12\%$ with TPTP syntax on the original set, while Llama 3.1 8B benefits from the R-FOL syntax. However, all grammars show a declining accuracy trend and increased syntax errors for noise perturbations, where the best grammar loses its advantage over the rest. 
When comparing the grammars on the counterfactual partitions, we observe that TPTP is consistently more robust than the standard first-order logic grammar. Here, GPT 4o-mini shows a reduction from $O$ to $O_C$ of $20\%$ for FOL and only $12\%$ for the TPTP grammar. Since this does not correlate with fewer syntax errors, the formalisation in TPTP prevents semantical errors for counterfactual premises. 
A positive reading of these results, especially the minor differences between FOL and R-FOL, is that autoformalisation \acp{LLM} can adapt to the grammar syntax prescribed in the prompt without further loss in performance.

\begin{table}[!t]
\small
\setlength{\modelspacing}{2pt}
\setlength{\tabcolsep}{1.7pt} % Default value: 6pt
\setlength{\belowrulesep}{4pt}
\begin{threeparttable}
    \centering
    \begin{tabular}{cc l r rrr @{\quad} rrrr}
\toprule
\multirow{2}{*}{} & \multirow{2}{*}{} & Grammar & \multirow{2}{*}{O} & \multicolumn{3}{c}{Distraction} & \multicolumn{4}{c}{Counterfactual} \\
 & & Syntax & & E& L & T & $\text{O}_C$ & $\text{E}_C$& $\text{L}_C$ & $\text{T}_C$\\
\midrule
\multirow{6}{*}{\rotatebox{90}{Llama-3.1}} & \multirow{3}{*}{\rotatebox{90}{8B}} 
   & FOL & 0.77 & \textbf{0.71} & 0.61 & \textbf{0.53} & 0.58 & \textbf{0.55} & 0.52 & \textbf{0.56} \\
 & & R-FOL & \textbf{0.78} & 0.69 & \textbf{0.62} & \textbf{0.53} & 0.58 & \textbf{0.55} & \textbf{0.54} & 0.52 \\
 & & TPTP & 0.73 & 0.67 & 0.55 & 0.51 & \textbf{0.68} & 0.54 & 0.46 & 0.51 \\[\modelspacing]
\cmidrule{2-11}
 & \multirow{3}{*}{\rotatebox{90}{70B}} 
   & FOL & 0.76 & 0.73 & 0.71 & \textbf{0.72} & 0.67 & 0.57 & 0.63 & 0.56 \\
 & & R-FOL & 0.76 & 0.73 & 0.67 & 0.71 & 0.64 & 0.57 & 0.53 & 0.64 \\
 & & TPTP & \underline{\textbf{0.88}} & \underline{\textbf{0.84}} & \underline{\textbf{0.81}} & \textbf{0.72} & \underline{\textbf{0.81}} & \underline{\textbf{0.68}} & \underline{\textbf{0.67}} & \underline{\textbf{0.68}} \\[\modelspacing]
\midrule
\multirow{3}{*}{\rotatebox{90}{GPT}} & \multirow{3}{*}{\rotatebox{90}{4o-mini}} 
   & FOL & \textbf{0.84} & \textbf{0.82} & \textbf{0.72} & \underline{\textbf{0.78}} & 0.64 & \textbf{0.63} & \textbf{0.61} & 0.51 \\
 & & R-FOL & \textbf{0.84} & 0.77 & 0.70 & \underline{\textbf{0.78}} & \textbf{0.72} & 0.56 & 0.54 & \textbf{0.63} \\
 & & TPTP & 0.83 & \textbf{0.82} & 0.71 & 0.71 & 0.69 & \textbf{0.63} & 0.57 & 0.57 \\
\bottomrule
\end{tabular}
\caption{Accuracies of different formalisation grammars for autoformalisation.}
\label{tab:distraction_k4_logical_form}
\end{threeparttable}
\end{table} 

\paragraph{\textbf{\emph{F6: Feedback does not help \acp{LLM} self-correct to mitigate robustness issues.}}}
\autoref{tab:distraction_k4_feedback} shows the results with different error recovery mechanisms. The results indicate that no feedback strategy emerges as a winner in the different datasets. 
All feedback variants reduce syntax errors for noise perturbations, but given the lack of a consistent increase in accuracy, the corrected formalisations are most likely to contain semantic errors still. 
The type of feedback message only has a minor influence on correcting syntax errors, whereas Llama 3.1 70b and GPT 4o-mini correct slightly more syntax errors with specific error messages. This finding aligns with \cite{huang2023large}, who also found that \acp{LLM} cannot consistently self-correct their reasoning after receiving relevant feedback.

\begin{table}[!ht]
\small
\setlength{\modelspacing}{2pt}
\setlength{\tabcolsep}{1.7pt} % Default value: 6pt
\setlength{\belowrulesep}{4pt}
\begin{threeparttable}
    \centering
    \begin{tabular}{cc l r rrr @{\quad} rrrr}
\toprule
\multirow{2}{*}{} & \multirow{2}{*}{} & \multirow{2}{*}{Feedback} & \multirow{2}{*}{O} & \multicolumn{3}{c}{Distraction} & \multicolumn{4}{c}{Counterfactual} \\
 & & & & E& L & T & $\text{O}_C$ & $\text{E}_C$& $\text{L}_C$ & $\text{T}_C$\\
\midrule
\multirow{8}{*}{\rotatebox{90}{Llama-3.1}} & \multirow{4}{*}{\rotatebox{90}{8B}} 
   & No recovery & 0.77 & \textbf{0.72} & 0.62 & 0.53 & 0.59 & 0.58 & 0.56 & \textbf{0.56} \\
 & & Error type & \textbf{0.79} & 0.71 & 0.63 & \textbf{0.56} & \textbf{0.66} & 0.54 & 0.52 & 0.51 \\
 & & Error message & 0.78 & 0.71 & \textbf{0.67} & 0.55 & 0.59 & 0.53 & \underline{\textbf{0.64}} & 0.49 \\
 & & Warning & 0.74 & 0.66 & 0.58 & 0.55 & 0.55 & \textbf{0.60} & 0.49 & 0.49 \\[\modelspacing]
\cmidrule{2-11}
 & \multirow{4}{*}{\rotatebox{90}{70B}} 
   & No recovery & \textbf{0.77} & \textbf{0.72} & \textbf{0.73} & 0.71 & \textbf{0.64} & 0.59 & \textbf{0.61} & 0.56 \\
 & & Error type & 0.72 & 0.70 & 0.72 & \textbf{0.73} & 0.62 & 0.56 & 0.60 & 0.58 \\
 & & Error message & 0.71 & 0.70 & \textbf{0.73} & 0.71 & \textbf{0.64} & 0.59 & 0.54 & \underline{\textbf{0.64}} \\
 & & Warning & 0.69 & \textbf{0.72} & 0.72 & 0.72 & 0.62 & \underline{\textbf{0.65}} & \textbf{0.61} & 0.63 \\[\modelspacing]
\midrule
\multirow{4}{*}{\rotatebox{90}{GPT}} & \multirow{4}{*}{\rotatebox{90}{4o-mini}} 
   & No recovery & \underline{\textbf{0.84}} & \underline{\textbf{0.82}} & 0.73 & 0.79 & 0.64 & \textbf{0.62} & 0.56 & \textbf{0.56} \\
 & & Error type & 0.83 & 0.79 & 0.74 & 0.76 & 0.67 & 0.57 & 0.56 & \textbf{0.56} \\
 & & Error message & \underline{\textbf{0.84}} & 0.78 & \underline{\textbf{0.77}} & \underline{\textbf{0.80}} & 0.62 & 0.59 & 0.56 & \textbf{0.56} \\
 & & Warning & \underline{\textbf{0.84}} & 0.75 & 0.73 & 0.76 & \underline{\textbf{0.70}} & 0.61 & \textbf{0.61} & 0.55 \\
 \bottomrule
\end{tabular}
\caption{Accuracies of error recovery strategies.}
\label{tab:distraction_k4_feedback}
\end{threeparttable}
\end{table} 

\subsection{Error Analysis}
\label{subsec:errors}
\paragraph{\textbf{\emph{F7: Autoformalisation increases syntax errors for noise perturbations.}}}
The low performance for noise perturbations correlates with more syntax errors for all models and distraction categories (cf. execution rates in Table~\ref{tab:appendix_k4_formalisation_exec}). The three worst-performing models (both Mistral models, Gemma-2 9b) generate, at best, for $37\%$  and, at worst, for only $4\%$ of the samples, a valid logical form.
Gemma-2 9b and Llama3.1 8b produce more syntax errors than the larger counterparts, suggesting that larger models are more robust towards noise perturbations. 
The accuracy of syntactically valid samples is higher than the informal reasoning methods for most distractions (Table~\ref{tab:appendix_k4_formalisation_vacc}), motivating informal reasoning as a backup strategy for formal reasoning. The error message feedback reveals two common syntax errors: 1) errors by models with an initial low execution rate exhibit issues with the template structure, including using incorrect keywords or adding conversational phrases;
2) perturbation-related errors, the most common of which is using undefined truth constants as part of tautological distractions. 

\paragraph{\textbf{\emph{F8: Autoformalisation increases semantic errors for counterfactuals.}}}
Unlike the introduced noise, counterfactual perturbations do not lead to more syntax errors. The execution rate in Table~\ref{tab:appendix_k4_formalisation_exec} is stable or improves for counterfactuals. However, we see a drop in accuracy for the counterfactual column $\text{O}_C$ in Table~\ref{tab:distraction_k4_formalisation} and can conclude that the number of logical forms with semantic errors has to increase. This suggests that the introduced negation is not correctly formalised. Looking at the warnings generated by the feedback mechanism, for GPT 4o-mini, $161$ warning messages are generated on the unperturbed data. $54$ of these were fixed with a single iteration. Not considering predicates and individuals as part of the context is the most frequent warning across all models. 
%\clearpage
\section{Related Works}
% \subsection{Supervised Fine-Tuning}
% % 指令微调
% % 指令微调对LLM具有重要的作用,具体是什么?
% % 或者模仿Magpie的写法,这一段就纯讲作用,以及对应的工作有哪些

% % 指令微调的作用->sft技术分类->特别介绍conversation based prompt,因为我们也在用
% A series of studies find that if adjusted with annotated "instructional" data, LMs can effectively generate responses aligned with human values~\cite{sanh2022multitask, weifinetuned,ouyang2022training}. The performance of Supervised Fine-Tuning depends not only on the quality of the dataset~\cite{Zhou2023LIMALI} but also on various contextual prompting techniques, such as conversation-based prompts~\cite{sreedhar2024canttalkaboutthis, Wei2023ZeroShotIE}, chain-of-thought~\cite{Wei2022ChainOT}, and contextual calibration~\cite{Zhao2021CalibrateBU}.
% % 因为要对齐deepthink,这边强调一下conversation-based prompts
% Specifically, more models now use conversation-based prompts as the default for QA model deployment~\cite{wu2023brief,liu2024chatqa}, because they enhance the user experience by handling follow-up questions, providing clarifications, and reducing hallucinations.

% 数据合成->分为人工标注和LLM自己生成->人工标注成本高,LLM自己生成会有一些幻觉sample->我们在真实的QA下用rag来避免幻觉并且使用refiner来保证前后topic一致性以及保证数据真实性。(保证数据真实性是因为refiner前后能看到的rag的信息更广,引入更多事实数据)
\subsection{Instruction Data Synthesis}
To address the issue of limited training samples in specific domains, various works have proposed using additional data, such as manual annotation~\cite{Zhao2024WildChat1C,zheng2023lmsys} and automatic generation by LLMs~\cite{Mekala2022LeveragingQD, Wang2021TowardsZL, Wang2022SelfInstructAL, Xu2023WizardLMEL}. However, manual annotation is expensive~\cite{honovich-etal-2023-unnatural}, and iterative generation by LLMs frequently introduces the risk of hallucinations.


Our work falls into the category of automatic generation by LLMs. However, our work differs from previous approaches in two main aspects. (1) We synthesize instructions by simulating conversations closer to real-world scenarios. (2) We adopt several techniques to improve the quality of synthesized instruction. We integrate Retrieval-Augmented Generation (RAG) to mitigate hallucination in conversation-based synthesis. We apply a Conversation-based Data Refiner for filtering, ensuring topic consistency and data authenticity.
% RAG的作用->早期关注于检索器本身->现在专注于when and how ->分别举两个例子验证when and how -> 我们是在sft阶段使用rag的
\subsection{Retrieval-Augmented Generation}
Retrieval augmentation has become a standard solution to address hallucinations in LLMs by introducing external knowledge to compensate for factual shortcomings~\cite{Asai2023SelfRAGLT,ma2023query, Izacard2021UnsupervisedDI, Ram2023InContextRL}.
Early Retrieval Augmentation efforts focus primarily on the retriever itself, where both the neural retriever and generator are typically trainable Pretrained Language Models (PrLMs), such as BERT ~\cite{Devlin2019BERTPO} or BART ~\cite{Lewis2019BARTDS}. In contrast, modern Retrieval Augmentation applied to LLMs emphasizes determining when and how to retrieve relevant information~\cite{fatehkia2024t, Asai2023SelfRAGLT, Xu2024LargeLM}. For example, Self-RAG enables on-demand retrieval and generates more accurate, fact-based text through fine-grained self-reflection~\cite{Asai2023SelfRAGLT}. 
% mHyER bridges the semantic gap between learner input and practice content by generating hypothetical exercises related to the learner's input, 
% thereby improving retrieval relevance~\cite{Xu2024LargeLM}. 

Our approach uses RAG throughout the data synthesis, SFT, and inference stages. This not only improves the authenticity of the synthesized data but also helps the LLM learn how to effectively utilize the retrieved knowledge during the SFT stage. In contrast, previous research only used RAG during the inference stage, relying heavily on the LLM's ability to discern the retrieved knowledge. This can lead to insufficient utilization of relevant knowledge, especially when dealing with domain knowledge that was not included in the pretraining process.


\section*{Conclusion}
This paper aims to enhance our understanding of the computational complexity of computing various Shapley value variants. We found that for various ML models --- including decision trees, regression tree ensembles, weighted automata, and linear regression --- both local and global interventional and baseline SHAP can be computed in polynomial time under HMM modeled distributions. This extends popular algorithms, such as TreeSHAP, beyond their empirical distributional scope. We also establish strict complexity gaps between the various SHAP variants (baseline, interventional, and conditional) and prove the intractability of computing SHAP for tree ensembles and neural networks in simplified scenarios. Overall, we present SHAP as a versatile framework whose complexity depends on four key factors: \begin{inparaenum}[(i)] \item model type, \item SHAP variant, \item distribution modeling approach, \item and local vs. global explanations\end{inparaenum}. We believe this perspective provides deeper insight into the computational complexity of SHAP, paving the way for future work.




%We believe that our framework provides a more intricate understanding of SHAP computation complexity across different models, distributions, and variants, paving the way for further research.

Our work opens promising directions for future research. First, expanding our computational analysis to other SHAP-related metrics, such as asymmetric SHAP~\citep{frye20} and SAGE~\citep{covert2020understanding}, would be valuable. Additionally, we aim to explore more expressive distribution classes and relaxed assumptions beyond those in Section \ref{sec:tractable} while maintaining tractable SHAP computation. Finally, when exact computation is intractable (Section \ref{sec:intractable}), investigating the approximability of SHAP metrics through approximation and parameterized complexity theory~\citep{downey2012parameterized} is an important direction.

%Our work opens several promising avenues for future research on the computational properties of explainable AI methods, with a particular focus on SHAP. First, it would be interesting to broaden the computational analysis conducted in this work to include other popular SHAP-related metrics in the literature, such as asymmetric SHAP \cite{frye20} and SAGE \cite{covert2020understanding}. Also, in the future, we aim to explore more expressive distribution classes and relaxed distributional assumptions—extending beyond those examined in Section \ref{sec:tractable} —that still yield tractable SHAP computation. Finally, when exact computation proves intractable (Section \ref{sec:intractable}), it is worthwhile to theoretically investigate the question of the approximability of computing the SHAP metrics across various configurations, through the lens of approximation and parametrized complexity theory \cite{arora2009computational}.

%This paper aims to deepen our understanding of the computational complexity involved in obtaining different Shapley value variants. We found that for a variety of ML models, including decision trees, tree ensembles for regression, weighted automata, and linear regression models — computing both local and global interventional and baseline SHAP can be done in polynomial time when distributions are modeled by HMMs. This extends the distributional scope of popular algorithms like TreeSHAP, which is limited to empirical distributions. Additionally, we demonstrate a strict complexity gap between SHAP variants, showing that interventional and baseline SHAP can be strictly easier to compute than conditional SHAP. Despite these positive results, we uncovered intractability for various SHAP variants in neural networks and tree ensembles. Finally, we provided generalized complexity relations across SHAP variants. We believe that our framework offers a deeper understanding of the complexity involved in computing SHAP across various variants, models, distributions, as well as in both local and global computations, laying the groundwork for future research.

\clearpage
\sloppy  % Enable relaxed line breaking
\bibliographystyle{IEEEtran}
\bibliography{refs}

\end{document}
