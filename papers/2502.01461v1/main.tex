\documentclass[sigconf]{acmart}
\usepackage{multirow} 

\usepackage{algpseudocode}
\usepackage{algorithm}
\usepackage{multicol}
\usepackage{array}
\usepackage{arydshln}
\usepackage{booktabs}  
\usepackage{float}
\usepackage{color}

\newcommand{\todo}[1]{\textcolor{red}{TODO: #1}}


\begin{document}

\title{Docking-Aware Attention: Dynamic Protein Representations through Molecular Context Integration}

\author{Amitay Sicherman}
\email{amitay.s@cs.technion.ac.il}
\affiliation{%
  \institution{Technion - Israel Institute of Technology}
  \country{Israel}
}


\author{Kira Radinsky}
\email{kirar@cs.technion.ac.il}
\affiliation{
  \institution{Technion - Israel Institute of Technology}
  \country{Israel}
}


\begin{abstract}
% v1
% Protein behavior and function are inherently dynamic, adapting based on molecular interactions in their environment. However, current approaches to protein representation learning typically generate static embeddings that fail to capture this context-dependent nature. We present Docking-Aware Attention (DAA), a novel architecture that generates dynamic, context-dependent protein representations by incorporating molecular docking information into the attention mechanism. DAA combines physical interaction scores from ensemble docking predictions with learned attention patterns to focus on protein regions most relevant to specific molecular interactions. We evaluate our method on enzymatic reaction prediction, where it outperforms previous state-of-the-art methods, achieving 62.2\% accuracy versus 56.79\% on complex molecules and 55.54\% versus 49.45\% on innovative reactions. We demonstrate how DAA generates interpretable attention patterns that adapt to different molecular contexts through detailed ablation studies and visualizations. Our approach represents a general framework for context-aware protein representation with applications across computational biology. We open-source our implementation and pre-trained models to facilitate further research in this direction.


% v2
% The synthesis of molecules is a critical process in various scientific domains, ranging from drug discovery to materials science. Effective molecular synthesis often requires consideration of the broader biochemical environment, including the presence of other molecules and the enzymes or proteins that facilitate synthesis. For example, enzymes may exhibit different activities depending on their molecular context, significantly impacting the synthesis process. Current approaches to protein (enzyme) representation predominantly rely on static embeddings, which fail to account for the dynamic interactions between enzymes and their molecular surroundings.
% In this work, we propose a novel contextual protein representation that dynamically incorporates the biochemical environment into the encoding of enzymes. We empirically evaluate our algorithm in the context of molecular synthesis, demonstrating its potential to enhance generative accuracy. Our results highlight the importance of context-aware protein representations in advancing molecular synthesis research.

Computational prediction of enzymatic reactions represents a crucial challenge in sustainable chemical synthesis across various scientific domains, ranging from drug discovery to materials science and green chemistry. These syntheses rely on proteins that selectively catalyze complex molecular transformations. These protein catalysts exhibit remarkable substrate adaptability, with the same protein often catalyzing different chemical transformations depending on its molecular partners. Current approaches to protein representation in reaction prediction either ignore protein structure entirely or rely on static embeddings, failing to capture how proteins dynamically adapt their behavior to different substrates. We present Docking-Aware Attention (DAA), a novel architecture that generates dynamic, context-dependent protein representations by incorporating molecular docking information into the attention mechanism. DAA combines physical interaction scores from docking predictions with learned attention patterns to focus on protein regions most relevant to specific molecular interactions. We evaluate our method on enzymatic reaction prediction, where it outperforms previous state-of-the-art methods, achieving  62.2\% accuracy versus 56.79\% on complex molecules and 55.54\% versus 49.45\% on innovative reactions. Through detailed ablation studies and visualizations, we demonstrate how DAA generates interpretable attention patterns that adapt to different molecular contexts. Our approach represents a general framework for context-aware protein representation in biocatalysis prediction, with potential applications across enzymatic synthesis planning. We open-source our implementation and pre-trained models to facilitate further research.

\end{abstract}


\keywords{Protein Representation Learning, Molecular Docking, Biocatalysis, Deep Learning, Computational Biology}

\maketitle

\section{Introduction}


Developing effective computational approaches for enzymatic reaction prediction is a fundamental challenge in machine learning for chemistry \cite{goshisht2024machine,kreutter2021predicting}. The ability to accurately predict enzyme-catalyzed reactions has significant implications across multiple scientific domains, from drug discovery and metabolic engineering to sustainable chemistry and materials science \cite{sen2015green,fessner2015systems}. Enzymes, as protein-based catalysts, enable complex molecular transformations under mild conditions with high selectivity and efficiency. However, the same enzyme can exhibit different catalytic behaviors depending on its molecular partners, making computational modeling of these reactions particularly challenging.

This substrate-dependent adaptability of enzymes poses unique challenges for machine learning models. Current computational methods for enzymatic reaction prediction either ignore protein structure entirely or rely on enzyme commission (EC) numbers for protein representation~\cite{Probst2022,Chen2023}. While EC numbers provide useful categorical information about enzyme function, they fail to capture the nuanced ways proteins interact with different molecular partners. Even advanced deep learning approaches, whether using protein language models like ESM \cite{hayes2024simulating} or structure-based models like GearNet \cite{zhang2023protein}, face a fundamental limitation—while they effectively capture sequence or structural properties, they still produce static protein representations that fail to reflect the protein's dynamic behavior with different molecules in their environment.


Molecular docking has long been used to study protein-ligand interactions \cite{muhammed2024molecular}, providing physics-based insights into binding preferences and interaction patterns. Recent advances in deep learning have improved docking accuracy and efficiency \cite{DiffDock2023}, making it feasible to incorporate this valuable source of structural interaction information into protein representation learning. However, previous attempts to combine docking with deep learning have primarily focused on improving binding pose prediction rather than generating dynamic protein representations for reaction prediction.


We present Docking-Aware Attention (DAA), a novel architecture that addresses these limitations by generating dynamic, context-dependent protein representations. Our key insight is that molecular docking information can guide attention mechanisms to focus on protein regions most relevant to specific molecular interactions. By incorporating docking scores into the attention computation, DAA produces protein representations that adapt based on the predicted physical interactions between the protein and its molecular partners. This approach better reflects the reality of enzyme behavior, where catalytic activity depends on specific substrate interactions.

The key contributions of this work are: (i) We introduce DAA, a novel architecture that integrates physical protein-ligand interaction information into attention mechanisms to generate context-dependent protein representations for enzymatic reaction prediction; (ii) Through extensive experiments on biocatalysis prediction, we show that these context-aware protein representations substantially improve performance on challenging cases, achieving 62.2\% versus 56.79\% accuracy on complex molecules and 55.54\% versus 49.45\% on innovative reactions; (iii) We demonstrate how DAA generates interpretable attention patterns that reveal which protein regions are most relevant for specific molecular interactions, providing insights into context-dependent enzyme behavior.

To facilitate reproducibility and encourage further development in this direction, we have open-sourced our complete codebase, including model implementations, training scripts, and pre-trained models at GitHub \footnote{\url{https://anonymous.4open.science/r/DockingAwareAttention-8B8E}}.


\section{Related Works}

\subsection{Biocatalysis Prediction}
The evolution of biocatalysis prediction has been significantly shaped by advances in chemical reaction prediction, particularly in sequence-to-sequence models. Neural sequence-to-sequence approaches \cite{schwaller2019molecular} pioneered the treatment of reactions as translation tasks, with transformer architectures like ChemBERTa \cite{ChithranandaChemBERTa} and MolFormer \cite{ross2022large} further improving the capture of molecular dependencies.

The adaptation of these models to biocatalysis presented unique challenges in representing enzyme-substrate interactions. Kreutter et al. \cite{kreutter2021predicting} first demonstrated the viability of transformer models for biocatalysis by leveraging upsampled enzymatic reaction data. ECREACT \cite{Probst2022} advanced this approach by representing enzymes through EC numbers as special tokens in the sequence-to-sequence framework. However, this discrete representation limited the model's ability to capture nuanced enzyme-substrate interactions.

Our work addresses these limitations through the DAA mechanism, which creates dynamic, substrate-specific enzyme representations within the sequence-to-sequence framework, enabling more precise modeling of enzyme-substrate interactions.
% \subsection{Biocatalysis Prediction}
% The evolution of biocatalysis prediction has been significantly shaped by advances in chemical reaction prediction, particularly in the domain of sequence-to-sequence models. Understanding this progression requires examining how these approaches were adapted from chemical to enzymatic reactions.

% \subsubsection{Chemical Reaction Prediction Foundations}
% Chemical reaction prediction saw significant advancement through sequence-to-sequence architectures. Neural sequence-to-sequence models \cite{schwaller2019molecular} pioneered the treatment of reactions as translation tasks, converting reactant SMILES strings into product SMILES strings. This approach was further enhanced by transformer architectures in models like ChemBERTa \cite{ChithranandaChemBERTa} and MolFormer \cite{ross2022large}, which improved the capture of long-range molecular dependencies and reaction mechanisms.

% \subsubsection{Extension to Enzymatic Reactions}
% The adaptation of these sequence-to-sequence models to biocatalysis presented unique challenges due to the need to incorporate enzyme information. Kreutter et al. \cite{kreutter2021predicting} made the first breakthrough by demonstrating that transformer models trained on chemical reaction data could be effectively adapted to biocatalysis through strategic upsampling of enzymatic reactions, even without explicit enzyme representations. This work established that chemical reaction prediction principles could be directly extended to enzymatic systems.

% Building on this foundation, ECREACT \cite{Probst2022} introduced a more sophisticated approach by representing enzymes through their EC numbers as special tokens in the sequence-to-sequence framework. While this improved the model's ability to distinguish between different enzyme classes, the discrete EC number representation limited its capacity to capture nuanced enzyme-substrate interactions.

% Our work addresses these limitations by introducing the DAA mechanism, which creates dynamic, substrate-specific enzyme representations within the sequence-to-sequence framework. This approach enables more precise modeling of enzyme-substrate interactions while maintaining the benefits of transformer-based prediction.

\subsection{Representation Learning in Proteins}
Protein representation learning has developed along sequence-based and structure-based approaches.

\subsubsection{Sequence-based Protein Language Models}
Significant advances in protein language models have been seen in recent years, with transformer-based architectures\cite{Vaswani2017} leading the way. Notable developments include ProteinBERT \cite{brandes2021proteinbert} and ProtTrans\cite{elnaggar2021prottrans}, which adapted BERT\cite{devlin2018bert} for protein sequences, and the ESM model family \cite{hayes2024simulating, lin2023evolutionary, rives2021biological}, which demonstrated the benefits of scale in protein modeling. These models have shown remarkable success in capturing local and long-range protein interactions, establishing new benchmarks in protein property prediction tasks.

\subsubsection{Structure-based Representation Learning}
Following AlphaFold's \cite{jumper2021highly} breakthrough in protein structure prediction, structure-based representation learning has gained prominence. Models like GearNet \cite{zhang2023protein} have introduced novel architectures for capturing protein structure through graph-based approaches. Various methods have emerged to better model protein geometry and structural relationships \cite{fan2023continuous, hermosilla2020intrinsic, jing2020learning}, highlighting the importance of three-dimensional information in protein understanding.

Despite these advances, both sequence-based and structure-based approaches generate static protein representations that remain fixed regardless of molecular context, limiting their ability to capture the dynamic nature of protein-molecule interactions.

\subsection{Molecular Docking Approaches}
The evolution of molecular docking methods provides crucial context for our work. Traditional approaches relied on physics-based scoring functions and search algorithms \cite{McNutt2021,Stark2022}, facing computational challenges particularly in blind docking scenarios. Recent learning-based methods have made significant strides in addressing these limitations. EquiBind \cite{stark2022equibind} introduced keypoint-based methods for pocket-ligand alignment, while TANKBind \cite{lu2022tankbind} enabled independent predictions for multiple binding sites. DiffDock \cite{DiffDock2023} represented a paradigm shift by reformulating docking as a generative modeling problem. While these approaches have improved docking accuracy and efficiency, they primarily focus on predicting binding poses rather than using docking information to enhance protein representations. Our DAA method uniquely leverages docking predictions to create dynamic protein embeddings that adapt to different molecular interaction contexts.



%%%%%%%%%%%%%%%%%%%%%%%%%%%%%%%%%%%%%%%%%%%%%%%%%%%%%%%%%%%%%%%%
%%%%%%%%%%%%%%%%        ALGO  %%%%%%%%%%%%%%%%%%%%%%%%%%%
%%%%%%%%%%%%%%%%%%%%%%%%%%%%%%%%%%%%%%%%%%%%%%%%%%%%%%%%%%%%%%%%

\section{\thename}
\subsection{End-to-End Driving Policy}
The overall framework of \thename{} is depicted in Fig.~\ref{fig:framework}. 
\thename{} takes multi-view image sequences as input, transforms the sensor data into scene token embeddings, outputs the probabilistic distribution of actions, and samples an action to control the vehicle. 

\boldparagraph{BEV Encoder.} 
We first employ a BEV encoder~\cite{li2022bevformer} to transform multi-view image features from the perspective view to the Bird's Eye View (BEV), obtaining a feature map in the BEV space. This feature map is then used to learn instance-level map features and agent features.

\boldparagraph{Map Head.} 
Then we utilize a group of map tokens~\cite{maptrv2, liao2022maptr, lanegap} to learn the vectorized map elements of the driving scene from the BEV feature map, including lane centerlines, lane dividers, road boundaries, arrows, traffic signals, \etc.

\boldparagraph{Agent Head.} 
Besides, a group of agent tokens~\cite{jiang2022pip} is adopted to predict the motion information of other traffic participants, including location, orientation, size, speed, and multi-mode future trajectories.

\boldparagraph{Image Encoder.} 
Apart from the above instance-level map and agent tokens, we also use an individual image encoder~\cite{vit,he2016resnet} to transform the original images into image tokens. These image tokens provide dense and rich scene information for planning, complementary to the instance-level tokens.

\begin{figure}[t]
\centering
\includegraphics[width=0.98\linewidth]{fig/post-training-2.pdf} 
\caption{\textbf{Post-training.}  $N$  workers parallelly run. The generated rollout data $(s_t,a_t, r_{t+1},s_{t+1},...)$ are recorded in a rollout buffer. Rollout data and human driving demonstrations are used in RL- and IL-training steps to fine-tune the AD policy synergistically.
}
\label{fig:post-training}
\end{figure}

\boldparagraph{Action Space.} 
To accelerate the convergence of RL training, we design a decoupled discrete action representation. 
We divide the action into two independent components: lateral action and longitudinal action. 
The action space is constructed over a short $0.5$-second time horizon, during which the vehicle's motion is approximated by assuming constant linear and angular velocities. 
Under this assumption, the lateral action $a^x$ and longitudinal action $a^y$ can be directly computed based on the current linear and angular velocities.
By combining decoupling with a limited temporal scope and simplified motion model, our approach effectively reduces the dimensionality of the action space, accelerating training convergence.


\boldparagraph{Planning Head.} 
We use $E_\text{scene}$ to denote the scene representation, which consists of map tokens, agent tokens, and image tokens. We initialize a planning embedding denoted as $E_\text{plan}$. A cascaded Transformer decoder $\phi$ takes the planning embedding $E_\text{plan}$ as the query and the scene representation $E_\text{scene}$ as both key and value.

The output of the decoder $\phi$ is then combined with navigation information $E_\text{navi}$ and ego state $E_\text{state}$ to output the probabilistic distributions of the lateral action $a^x$ and the longitudinal action $a^y$:
\begin{equation}
\begin{aligned}
     \pi(a^x\mid s) = & \text{softmax}(\text{MLP}(\phi(E_\text{plan}, E_\text{scene}) \\
    & + E_\text{navi} + E_\text{state})), \\
     \pi(a^y\mid s) = & \text{softmax}(\text{MLP}(\phi(E_\text{plan}, E_\text{scene}) \\
     & + E_\text{navi} + E_\text{state})),
\label{eq:action distribution}
\end{aligned}
\end{equation}
where $E_\text{plan}$, $E_\text{navi}$, $E_\text{state}$, and the output of $\text{MLP}$ are all of the same dimension ($1 \times D$).

The planning head also outputs the value functions $V_x(s)$ and $V_y(s)$, which estimate the expected cumulative rewards for the lateral and longitudinal actions, respectively: 
\begin{equation}
\begin{aligned}
    & V_x(s) = \text{MLP}(\phi(E_\text{plan}, E_\text{scene}) + E_\text{navi} + E_\text{state}), \\
    & V_y(s) = \text{MLP}(\phi(E_\text{plan}, E_\text{scene}) + E_\text{navi} + E_\text{state}).
\end{aligned}
\end{equation}
The value functions are used in RL training (Sec.~\ref{sec:optimization}).

\subsection{Training Paradigm}
We adopt a three-stage training paradigm: perception pre-training, planning pre-training, and reinforced post-training, as shown in Fig.~\ref{fig:framework}.

\boldparagraph{Perception Pre-Training.} 
Information in the image is sparse and low-level. In the first stage,  
the map head and the agent head explicitly output map elements and agent motion information, which are supervised with ground-truth labels. Consequently,  
map tokens and agent tokens implicitly encode the corresponding high-level information.  
In this stage, we only update the parameters of the BEV encoder, the map head, and the agent head.



\boldparagraph{Planning Pre-Training.} 
In the second stage, to prevent the unstable cold start of RL training, IL is first performed to initialize the probabilistic distribution of actions based on large-scale real-world driving demonstrations from expert drivers. In this stage, we only update the parameters of the image encoder and the planning head, while the parameters of the BEV encoder, map head, and agent head are frozen. The optimization objectives of perception tasks and planning tasks may conflict with each other. However, with the training stage and parameters decoupled, such conflicts are mostly avoided.

\boldparagraph{Reinforced Post-Training.} 
In the reinforced post-training, RL and IL synergistically fine-tune the distribution. RL aims to guide the policy to be sensitive to critical risky events and adaptive to out-of-distribution situations. IL serves as the regularization term to keep the policy's behavior similar to that of humans.

We select a large amount of risky dense-traffic clips from collected driving demonstrations. For each clip, we train an independent 3DGS model that reconstructs the clip and serves as a digital driving environment.  
As shown in Fig.~\ref{fig:post-training}, we set $N$ parallel workers.  
Each worker randomly samples a 3DGS environment and begins rollout, i.e., the AD policy controls the ego vehicle to move and iteratively interacts with the 3DGS environment. After the rollout process of this 3DGS environment ends, the generated rollout data $(s_t,a_t, r_{t+1},s_{t+1},...)$ are recorded in a rollout buffer, and the worker will sample a new 3DGS environment for another round of rollout.

As for policy optimization, we iteratively perform RL-training steps and IL-training steps. For RL-training steps, we sample data from the rollout buffer and follow the Proximal Policy Optimization (PPO) framework~\cite{PPO} to update the AD policy. For IL-training steps, we use real-world driving demonstrations to update the policy. After a fixed number of training steps, the updated AD policy is sent to every worker to replace the old one, to avoid a distribution shift between data collection and optimization.
We only update the parameters of the image encoder and the planning head. The parameters of the BEV encoder, the map head, and the agent head are frozen.  
The detailed RL design is presented below.

\subsection{Interaction Mechanism between AD Policy and 3DGS Environment}
In the 3DGS environment, the ego vehicle acts according to the AD policy. Other traffic participants act according to real-world data in a log-replay manner.  
A simplified kinematic bicycle model is employed to iteratively update the ego vehicle's pose at every $\Delta t$ seconds as follows:  
\begin{equation}
\begin{aligned}
x_{t+1}^{w} & = x_{t}^w + v_t \cos \left(\psi_{t}^w\right) \Delta t, \\
y_{t+1}^{w} & = y_{t}^w + v_t \sin \left(\psi_{t}^w\right) \Delta t, \\
\psi_{t+1}^{w} & = \psi_{t}^w + \frac{v_t}{L} \tan \left(\delta_t\right) \Delta t,
\label{equation:kinematic_model}
\end{aligned}
\end{equation}  
where $x_t^{w}$ and $y_t^{w}$ denote the position of the ego vehicle relative to the world coordinate; $\psi_t^w$ is the heading angle that defines the vehicle's orientation with respect to the world $x$-coordinate; $v_t$ is the linear velocity of the ego vehicle; $\delta_t$ is the steering angle of the front wheels; and $L$ is the wheelbase, i.e., the distance between the front and rear axles.

During the rollout process, the AD policy outputs actions $(a_t^x, a_t^y)$ for a $0.5$-second time horizon at time step $t$. We derive the linear velocity $v_t$ and steering angle $\delta_t$ based on $(a_t^x, a_t^y)$.  
Based on the kinematic model in Eq.~\ref{equation:kinematic_model},  
the pose of the ego vehicle in the world coordinate system is updated from ${p}_t = (x_{t}^w, y_{t}^w, \psi_{t}^w)$ to ${p}_{t+1} = (x_{t+1}^{w}, y_{t+1}^{w}, \psi_{t+1}^{w})$.  

Based on the updated ${p}_{t+1}$, the 3DGS environment computes the new ego vehicle's state $s_{t+1}$. The updated pose ${p}_{t+1}$ and state $s_{t+1}$ serve as the input for the next iteration of the inference process.

The 3DGS environment also generates rewards $\mathcal{R}$ (Sec.~\ref{sec:reward}) according to multi-source information (including trajectories of other agents, map information, the expert trajectory of the ego vehicle, and the parameters of Gaussians), which are used to optimize the AD policy (Sec.~\ref{sec:optimization}).

\begin{figure}[t]
\centering
\includegraphics[width=1.0\linewidth]{fig/reward.pdf} 
\caption{\textbf{Example diagram of four types of reward sources.}  (1): Collision with a dynamic obstacle ahead triggers a reward $r_{\text{dc}}$. (2): Hitting a static roadside obstacle incurs a reward $r_{\text{sc}}$. (3): Moving onto the curb exceeds the positional deviation threshold $d_{\text{max}}$, triggering a reward $r_{\text{pd}}$. (4): Drifting toward the adjacent lane exceeds the heading deviation threshold $\psi_{\text{max}}$, triggering a reward $r_{\text{hd}}$.
}
\label{fig: reward source}
\end{figure}
\subsection{Reward Modeling}
\label{sec:reward}
The reward is the source of the training signal, which determines the optimization direction of RL. The reward function is designed to guide the ego vehicle's behavior by penalizing unsafe actions and encouraging alignment with the expert trajectory. It is composed of four reward components: (1) collision with dynamic obstacles, (2) collision with static obstacles, (3) positional deviation from the expert trajectory, and (4) heading deviation from the expert trajectory:
\begin{equation}
\begin{aligned}
\mathcal{R} = \{r_{\text{dc}}, r_{\text{sc}}, r_{\text{pd}}, r_{\text{hd}}  \}. 
\end{aligned}
\end{equation}

As illustrated in Fig.~\ref{fig: reward source}, these reward components are triggered under specific conditions.  
In the 3DGS environment, dynamic collision is detected if the ego vehicle's bounding box overlaps with the annotated bounding boxes of dynamic obstacles, triggering a negative reward $r_{\text{dc}}$. Similarly, static collision is identified when the ego vehicle's bounding box overlaps with the Gaussians of static obstacles, resulting in a negative reward $r_{\text{sc}}$.  
Positional deviation is measured as the Euclidean distance between the ego vehicle's current position and the closest point on the expert trajectory. A deviation beyond a predefined threshold $d_{\text{max}}$ incurs a negative reward $r_{\text{pd}}$.  
Heading deviation is calculated as the angular difference between the ego vehicle's current heading angle $ \psi_t $ and the expert trajectory's matched heading angle $\psi_{\text{expert}}$. A deviation beyond a threshold $ \psi_{\text{max}}$ results in a negative reward $r_{\text{hd}}$.

Any of these events, including dynamic collision, static collision, excessive positional deviation, or excessive heading deviation, triggers immediate episode termination. Because after such events occur, the 3DGS environment typically generates noisy sensor data, which is detrimental to RL training.

\subsection{Policy Optimization}
\label{sec:optimization}
In the closed-loop environment, the error in each single step accumulates over time. The aforementioned rewards are not only caused by the current action but also by the actions of the preceding steps.  
The rewards are propagated forward with Generalized Advantage Estimation (GAE)~\cite{gae} to optimize the action distribution of the preceding steps.

Specifically, for each time step $t$, we store the current state $s_t$, action $a_t$, reward $r_t$, and the estimate of the value $V(s_t)$.  
Based on the decoupled action space, and considering that different rewards have different correlations to lateral and longitudinal actions, the reward $r_t$ is divided into lateral reward $r_t^x$ and longitudinal reward $r_t^y$:
\begin{equation}
\begin{aligned}
r_t^x &= r_t^{\text{sc}} + r_t^{\text{pd}} + r_t^{\text{hd}}, \\
r_t^y &= r_t^{\text{dc}}.
\label{eq:reward-decouple}
\end{aligned}
\end{equation}
Similarly, the value function $V(s_t)$ is decoupled into two components: $V_x(s_t)$ for the lateral dimension and $V_y(s_t)$ for the longitudinal dimension. These value functions estimate the expected cumulative rewards for the lateral and longitudinal actions, respectively. The advantage estimates $\hat{A}_t^x$ and $\hat{A}_t^y$ are then computed as follows:
\begin{equation}
\begin{aligned}
\delta_t^x &= r_t^x + \gamma V_x(s_{t+1}) - V_x(s_t), \\
\delta_t^y &= r_t^y + \gamma V_y(s_{t+1}) - V_y(s_t), \\
\hat{A}_t^x &= \sum_{l=0}^{\infty}(\gamma \lambda)^l \delta_{t+l}^x, \\
\hat{A}_t^y &= \sum_{l=0}^{\infty}(\gamma \lambda)^l \delta_{t+l}^y,
\label{eq:advantage}
\end{aligned}
\end{equation}
where $\delta_t^x$ and $\delta_t^y$ are the temporal difference errors for the lateral and longitudinal dimensions, $\gamma$ is the discount factor, and $\lambda$ is the GAE parameter that controls the trade-off between bias and variance.

To further clarify the relationship between the advantage estimates and the reward components, we decompose $\hat{A}_t^x$ and $\hat{A}_t^y$ based on the reward decomposition in Eq.~\ref{eq:reward-decouple} and the advantage estimation in Eq.~\ref{eq:advantage}. Specifically, we derive the following decomposition:
\begin{equation}
\begin{aligned}
\hat{A}_t^x &= \hat{A}_t^{\text{sc}} + \hat{A}_t^{\text{pd}} + \hat{A}_t^{\text{hd}}, \\
\hat{A}_t^y &= \hat{A}_t^{\text{dc}},
\end{aligned}
\end{equation}
where $\hat{A}_t^{\text{sc}}$ is the advantage estimate for avoiding static collisions, $\hat{A}_t^{\text{pd}}$ is the advantage estimate for minimizing positional deviations, $\hat{A}_t^{\text{hd}}$ is the advantage estimate for minimizing heading deviations, and $\hat{A}_t^{\text{dc}}$ is the advantage estimate for avoiding dynamic collisions.

These advantage estimates are used to guide the update of the AD policy $\pi_{\theta}$, following the PPO framework~\cite{PPO}. By leveraging the decomposed advantage estimates $\hat{A}_t^x$ and $\hat{A}_t^y$, we can independently optimize the lateral and longitudinal dimensions of the policy. This is achieved by defining separate objective functions $\mathcal{L}_x^{\text{CLIP}}(\theta)$ and $\mathcal{L}_y^{\text{CLIP}}(\theta)$ for each dimension,  as follows:
\begin{equation}
\begin{aligned}
\mathcal{L}_x^{\text{PPO}}(\theta) &= \mathbb{E}_t \left[ \min \left( \rho_t^x \hat{A}_t^x, \ \text{clip}(\rho_t^x, 1-\epsilon_x, 1+\epsilon_x) \hat{A}_t^x \right) \right], \\
\mathcal{L}_y^{\text{PPO}}(\theta) &= \mathbb{E}_t \left[ \min \left( \rho_t^y \hat{A}_t^y, \ \text{clip}(\rho_t^y, 1-\epsilon_y, 1+\epsilon_y) \hat{A}_t^y \right) \right], \\
\mathcal{L}^{\text{PPO}}(\theta) &= \mathcal{L}_x^{\text{PPO}}(\theta) + \mathcal{L}_y^{\text{PPO}}(\theta),
\end{aligned}
\end{equation}
where $\rho_t^x = \frac{\pi_{\theta}(a_t^x \mid s_t)}{\pi_{\theta_{\text{old}}}(a_t^x \mid s_t)}$ is the importance sampling ratio for the lateral dimension, $\rho_t^y = \frac{\pi_{\theta}(a_t^y \mid s_t)}{\pi_{\theta_{\text{old}}}(a_t^y \mid s_t)}$ is the importance sampling ratio for the longitudinal dimension, $\epsilon_x$ and $\epsilon_y$ are small constants that control the clipping range for the lateral and longitudinal dimensions, ensuring stable policy updates.

The clipped objective function $\mathcal{L}^{\text{PPO}}(\theta)$ prevents excessively large updates to the policy parameters $\theta$, thereby maintaining training stability.

\begin{table*}[ht]
    \centering
{
\begin{tabular}{lccccccccc}
    \toprule
    RL:IL & CR$\downarrow$ & DCR$\downarrow$ & SCR$\downarrow$ & DR$\downarrow$ & PDR$\downarrow$ & HDR$\downarrow$ &ADD$\downarrow$ & Long. Jerk$\downarrow$ & Lat. Jerk$\downarrow$ \\
    \midrule
     0:1  & 0.229 & 0.211 & 0.018 & 0.066 & 0.039 & 0.027  & 0.238 & 3.928 & 0.103\\
     1:0  & 0.143 & 0.128 & 0.015 &0.080 &0.065 &0.015 &0.345 &4.204 &0.085\\
     2:1 & 0.137 & 0.125 & 0.012 & 0.059 & 0.050 & 0.009  & 0.274 & 4.538 & 0.092\\
     4:1 & 0.089 & 0.080 & 0.009 & 0.063 & 0.042 & 0.021  & 0.257 & 4.495 & 0.082 \\
     8:1 & 0.125 & 0.116 & 0.009 & 0.084 & 0.045 & 0.039  & 0.323 & 5.285 & 0.115\\
    \bottomrule
\end{tabular}
}
    \caption{\textbf{Ablation on RL-to-IL step mixing ratios in the reinforced post-training stage.}}
    \label{tab:ratio}
\end{table*}

\subsection{Auxiliary Objective}
RL usually faces the challenge of sparse rewards, which makes the convergence process unstable and slow. To speed up convergence, we introduce auxiliary objectives that provide dense guidance to the entire action distribution.

The auxiliary objectives are designed to penalize undesirable behaviors by incorporating specific reward sources, including dynamic collisions, static collisions, positional deviations, and heading deviations. These objectives are computed based on the actions \( a_t^{x, \text{old}} \) and \( a_t^{y, \text{old}} \) selected by the old AD policy \( \pi_{\theta_{\text{old}}} \) at time step \( t \). To facilitate the evaluation of these actions, we separate the probability distribution of the action into four parts:
\begin{equation}
\begin{aligned}
\Delta \pi_y^{\text{dec}} &= \sum_{a_t^y < a_t^{y, \text{old}}} \pi_\theta(a_t^y \mid s_t), \\
\Delta \pi_y^{\text{acc}} &= \sum_{a_t^y > a_t^{y, \text{old}}} \pi_\theta(a_t^y \mid s_t), \\
\Delta \pi_x^{\text{left}} &= \sum_{a_t^x < a_t^{x, \text{old}}} \pi_\theta(a_t^x \mid s_t), \\
\Delta \pi_x^{\text{right}} &= \sum_{a_t^x > a_t^{x, \text{old}}} \pi_\theta(a_t^x \mid s_t).
\end{aligned}
\end{equation}
Here, \( \Delta \pi_y^{\text{dec}} \) represents the total probability of deceleration actions, \( \Delta \pi_y^{\text{acc}} \) represents the total probability of acceleration actions, \( \Delta \pi_x^{\text{left}} \) represents the total probability of leftward steering actions, and \( \Delta \pi_x^{\text{right}} \) represents the total probability of rightward steering actions.

\boldparagraph{Dynamic Collision Auxiliary Objective.}  
The dynamic collision auxiliary objective adjusts the longitudinal control action \(a_t^y\) based on the location of potential collisions relative to the ego vehicle. If a collision is detected ahead, the policy prioritizes deceleration actions (\(a_t^y < a_t^{y, \text{old}}\)); if a collision is detected behind, it encourages acceleration actions (\(a_t^y > a_t^{y, \text{old}}\)). To formalize this behavior, we define a directional factor \(f_\text{dc}\):
\begin{equation}
\begin{aligned}
f_\text{dc} = \begin{cases} 
1 & \text{if the collision is ahead}, \\
-1 & \text{if the collision is behind}.
\end{cases} 
\end{aligned}
\end{equation}

The auxiliary objective for dynamic collision avoidance is defined as:
\begin{equation}
\begin{aligned}
\mathcal{L}_\text{dc}(\theta_y) = \mathbb{E}_t \left[ 
    \hat{A}_t^\text{dc} \cdot f_\text{dc} \cdot (\Delta \pi_y^{\text{dec}} - \Delta \pi_y^{\text{acc}})
\right],
\end{aligned}
\end{equation}
where \(\hat{A}_t^\text{dc}\) is the advantage estimate for dynamic collision avoidance.

\boldparagraph{Static Collision Auxiliary Objective.}  
The static collision auxiliary objective adjusts the steering control action $a_t^x$ based on the proximity to static obstacles. If the static obstacle is detected on the left side, the policy promotes rightward steering actions ($a_t^x > a_t^{x,\text{old}}$); if the static obstacle is detected on the right side, it promotes leftward steering actions ($a_t^x < a_t^{x,\text{old}}$). To formalize this behavior, we define a directional factor $f_\text{sc}$:  
\begin{equation}
\begin{aligned}
f_\text{sc} = \begin{cases} 
1 & \text{if static obstacle is on the left}, \\
-1 & \text{if static obstacle is on the right}.
\end{cases} 
\end{aligned}
\end{equation}

The auxiliary objective for static collision avoidance is defined as:  
\begin{equation}
\begin{aligned}
\mathcal{L}_\text{sc}(\theta_x) = \mathbb{E}_t \left[ 
    \hat{A}_t^\text{sc} \cdot f_\text{sc} \cdot (\Delta \pi_x^{\text{right}} - \Delta \pi_x^{\text{left}})
\right],
\end{aligned}
\end{equation}  
where $\hat{A}_t^\text{sc}$ is the advantage estimate for static collision avoidance.  

\boldparagraph{Positional Deviation Auxiliary Objective.}  
The positional deviation auxiliary objective adjusts the steering control action $a_t^x$ based on the ego vehicle's lateral deviation from the expert trajectory. If the ego vehicle deviates leftward, the policy promotes rightward corrections ($a_t^x > a_t^{x,\text{old}}$); if it deviates rightward, it promotes leftward corrections ($a_t^x < a_t^{x,\text{old}}$). We formalize this with a directional factor $f_\text{pd}$:  
\begin{equation}
\begin{aligned}
f_\text{pd} = \begin{cases} 
1 & \text{if ego vehicle deviates leftward}, \\
-1 & \text{if ego vehicle deviates rightward}.
\end{cases} 
\end{aligned}
\end{equation}

The auxiliary objective for positional deviation correction is:
\begin{equation}
\begin{aligned}
\mathcal{L}_\text{pd}(\theta_x) = \mathbb{E}_t \left[ 
    \hat{A}_t^\text{pd} \cdot f_\text{pd} \cdot (\Delta \pi_x^{\text{right}} - \Delta \pi_x^{\text{left}})
\right],
\end{aligned}
\end{equation}  
where $\hat{A}_t^\text{pd}$ estimates the advantage of trajectory alignment.

\boldparagraph{Heading Deviation Auxiliary Objective.}  
The heading deviation auxiliary objective adjusts the steering control action $a_t^x$ based on the angular difference between the ego vehicle’s current heading and the expert’s reference heading. If the ego vehicle deviates counterclockwise, the policy promotes clockwise corrections ($a_t^x > a_t^{x,\text{old}}$); if it deviates clockwise, it promotes counterclockwise corrections ($a_t^x < a_t^{x,\text{old}}$). To formalize this behavior, we define a directional factor $f_\text{hd}$:  
\begin{equation}
\begin{aligned}
f_\text{hd} = \begin{cases} 
1 & \text{if ego vehicle deviates clockwise}, \\
-1 & \text{if ego vehicle deviates counterclockwise}.
\end{cases} 
\end{aligned}
\end{equation}

The auxiliary objective for heading deviation correction is then defined as:  
\begin{equation}
\begin{aligned}
\mathcal{L}_\text{hd}(\theta_x) = \mathbb{E}_t \left[ 
    \hat{A}_t^\text{hd} \cdot f_\text{hd} \cdot (\Delta \pi_x^{\text{right}} - \Delta \pi_x^{\text{left}})
\right],
\end{aligned}
\end{equation}  
where $\hat{A}_t^\text{hd}$ is the advantage estimate for heading alignment.  

\begin{table*}[ht]
\begin{center}
\centering
\resizebox{0.98\textwidth}{!}{
\begin{tabular}{cccccccccccccc}
\toprule
\multirow{2}{*}{ID} & Dynamic & Static & Position & Heading & \multirow{2}{*}{CR$\downarrow$} &\multirow{2}{*}{DCR$\downarrow$} &\multirow{2}{*}{SCR$\downarrow$} &\multirow{2}{*}{DR$\downarrow$} &\multirow{2}{*}{PDR$\downarrow$} &\multirow{2}{*}{HDR$\downarrow$} &\multirow{2}{*}{ADD$\downarrow$} &\multirow{2}{*}{Long. Jerk$\downarrow$} &\multirow{2}{*}{Lat. Jerk$\downarrow$}\\
& Collision & Collision & Deviation & Deviation & & & & & & & & & \\
\midrule
1 & \cmark  &  &  &  & 0.172 & 0.154 & 0.018 & 0.092 & 0.033 & 0.059  & 0.259 & 4.211 & 0.095 \\
2 &  & \cmark & \cmark & \cmark & 0.238 & 0.217 & 0.021 & 0.090 & 0.045 & 0.045  & 0.241 & 3.937 & 0.098 \\
3 & \cmark &  & \cmark & \cmark & 0.146 & 0.128 & 0.018 & 0.060 & 0.030 & 0.030  & 0.263 & 3.729 & 0.083\\
4 & \cmark & \cmark &  & \cmark & 0.151 & 0.142 & 0.009 & 0.069 & 0.042 & 0.027 & 0.303 & 3.938 & 0.079\\
5 & \cmark & \cmark & \cmark &  & 0.166 & 0.157 & 0.009 & 0.048 & 0.036 & 0.012 & 0.243 & 3.334 & 0.067\\
6 & \cmark & \cmark & \cmark & \cmark & 0.089 & 0.080 & 0.009 & 0.063 & 0.042 & 0.021 & 0.257 & 4.495 & 0.082 \\
\bottomrule
\end{tabular}
}
\end{center}
\vspace{-2mm}
\caption{\textbf{Ablation on reward sources.} The table shows the impact of different reward components on performance.}
\label{tab:reward_ablation}
\end{table*}

\begin{table*}[ht]
\begin{center}
\centering
\resizebox{0.98\textwidth}{!}{
\begin{tabular}{ccccccccccccccc}
\toprule
\multirow{2}{*}{ID} & \multirow{2}{*}{PPO Obj.}  & Dynamic Col. & Static Col. & Position Dev. & Heading Dev. & \multirow{2}{*}{CR$\downarrow$} & \multirow{2}{*}{DCR$\downarrow$}  & \multirow{2}{*}{SCR$\downarrow$} & \multirow{2}{*}{DR$\downarrow$} & \multirow{2}{*}{PDR$\downarrow$} & \multirow{2}{*}{HDR$\downarrow$} & \multirow{2}{*}{ADD$\downarrow$} & \multirow{2}{*}{Long. Jerk$\downarrow$} & \multirow{2}{*}{Lat. Jerk$\downarrow$} \\
& & Auxiliary Obj. & Auxiliary Obj. & Auxiliary Obj. & Auxiliary Obj. & & & & & & & & & \\
\midrule
1 &\cmark&  &  &  &  & 0.249 & 0.223 & 0.026 & 0.077 & 0.047 & 0.030  & 0.266 & 4.209 & 0.104 \\
2 &\cmark& \cmark &  &  &  & 0.178 & 0.163 & 0.015 & 0.151 & 0.101 & 0.050 & 0.301 & 3.906 & 0.085 \\
3 &\cmark&  & \cmark & \cmark & \cmark & 0.137 & 0.125 & 0.012 & 0.157 & 0.145 & 0.012 & 0.296 & 3.419 & 0.071 \\
4 &\cmark& \cmark &  & \cmark & \cmark & 0.169 & 0.151 & 0.018 & 0.075 & 0.042 & 0.033 & 0.254 & 4.450 & 0.098 \\
5 &\cmark& \cmark & \cmark &  & \cmark & 0.149 & 0.134 & 0.015 & 0.063 & 0.057 & 0.006 & 0.324 & 3.980 & 0.086 \\
6 &\cmark& \cmark & \cmark & \cmark & & 0.128 & 0.119  & 0.009 & 0.066 & 0.030 & 0.036  & 0.254 & 4.102 & 0.092 \\
7 &&\cmark  &\cmark  &\cmark  &\cmark  & 0.187 &0.175  &0.012 &0.077 &0.056  &0.021  &0.309  &5.014  &0.112  \\
8 &\cmark& \cmark & \cmark & \cmark & \cmark & 0.089 & 0.080 & 0.009 & 0.063 & 0.042 & 0.021  & 0.257 & 4.495 & 0.082 \\
\bottomrule
\end{tabular}
}
\end{center}
\vspace{-2mm}
\caption{\textbf{Ablation on auxiliary objectives.} The table shows the impact of different auxiliary objectives on performance.}
\label{tab:auxiliary_ablation}
\end{table*}

\boldparagraph{Overall Auxiliary Objectives.}  
The overall auxiliary objectives are a weighted sum of the individual objectives:
\begin{equation}
\begin{aligned}
\mathcal{L}_\text{aux}(\theta) = &\lambda_1 \mathcal{L}_\text{dc}(\theta_y) + \lambda_2 \mathcal{L}_\text{sc}(\theta_x)  + \\ 
&\lambda_3 \mathcal{L}_\text{pd}(\theta_x) +\lambda_4 \mathcal{L}_\text{hd}(\theta_x),
\end{aligned}
\end{equation}
where $\lambda_1$, $\lambda_2$, $\lambda_3$, and $\lambda_4$ are weighting coefficients that balance the contributions of each auxiliary objective.

\boldparagraph{Optimization Objective.}  
The final optimization objective combines the clipped PPO objective with the auxiliary objective:
\begin{equation}
\mathcal{L}(\theta) = \mathcal{L}^{\text{PPO}}(\theta) + \mathcal{L}_\text{aux}(\theta).
\end{equation}

%%%%%%%%%%%%%%%%%%%%%%%%%%%%%%%%%%%%%%%%%%%%%%%%%%%%%%%%%%%%%%%%
%%%%%%%%%%%%%%%%  ALGO  %%%%%%%%%%%%%%%%%%%%%%%%%%%
%%%%%%%%%%%%%%%%%%%%%%%%%%%%%%%%%%%%%%%%%%%%%%%%%%%%%%%%%%%%%%%%
%%%%%%%%%%%%%%%%%%%%%%%%%%%%%%%%%%%%%%%%%%%%%%%%%%%%%%%%%%%%%%%%
%%%%%%%%%%%%%%%%        Main Figure  %%%%%%%%%%%%%%%%%%%%%%%%%%%
%%%%%%%%%%%%%%%%%%%%%%%%%%%%%%%%%%%%%%%%%%%%%%%%%%%%%%%%%%%%%%%%
\begin{figure*}[t]
    \includegraphics[width=0.85\textwidth]{figures/daa.png}
\caption{Overview of the DAA architecture. The protein sequence is processed through a pre-trained language model for per-amino acid embeddings. The DAA mechanism integrates sequence pooling, docking scores, and learned weights to create context-aware attention, which produces a final protein representation that incorporates protein-ligand interaction information}    \label{fig:overview}
\end{figure*}
%%%%%%%%%%%%%%%%%%%%%%%%%%%%%%%%%%%%%%%%%%%%%%%%%%%%%%%%%%%%%%%%
%%%%%%%%%%%%%%%%  END Main Figure  %%%%%%%%%%%%%%%%%%%%%%%%%%%
%%%%%%%%%%%%%%%%%%%%%%%%%%%%%%%%%%%%%%%%%%%%%%%%%%%%%%%%%%%%%%%%
\section{Docking-Aware Attention}
Proteins are highly dynamic molecules whose function and behavior change based on their molecular environment. For example, enzymes can catalyze different reactions depending on the substrate molecules present, and transcription factors can change their binding behavior based on the presence of different cofactors. Despite this inherent contextual nature, current protein representation learning approaches - based on sequence or structure - generate static embeddings that remain fixed regardless of the molecular context. This fundamental limitation means that the same protein receives identical representation, whether it is interacting with different substrates, cofactors, or regulatory molecules.

We present Docking-Aware Attention (DAA), a novel architecture that addresses this limitation by generating dynamic, context-dependent protein representations. Our key insight is that protein function is intimately tied to its molecular interactions, and these interactions can be predicted through molecular docking. By incorporating docking information into the attention mechanism, DAA generates protein representations that adapt based on the specific molecules in the protein's environment. This approach better reflects the reality of protein behavior, where the same protein can exhibit different properties and functions depending on its molecular context.


\subsection{Methodological Framework}
Figure \ref{fig:overview} provides a detailed architectural overview of our approach, which is formally presented in Algorithm \ref{algo:daa}. The algorithm takes as input a protein $P$ and a molecule $M$, generating a context-dependent protein representation $\mathbf{p}_M$ through the following steps:

First, we leverage the ESM protein language model to encode the protein sequence (line 2). This encoding generates per-residue embeddings $\mathbf{E}$, providing a rich foundation of protein sequence information that will be further refined through our attention mechanism (Section \ref{sec:base_encoder}).

The molecular interaction phase (lines 4-7) employs DiffDock's generative capabilities to sample $K$ distinct binding poses between the protein and molecule. For each sampled configuration, we compute interaction scores using the Lennard-Jones potential, capturing the physical basis of protein-molecule interactions. These scores are then averaged across all poses to obtain robust estimates of interaction strength (Section \ref{sec:mol_interaction}).

The interaction scores undergo adaptive smoothing (lines 8-10) through a weighted combination of local and global information, ensuring a balance between position-specific signals and overall protein context. This smoothed interaction profile guides our attention mechanism in focusing on regions most relevant to the specific molecular interaction (Section \ref{sec:mol_smooth}).

Finally, our novel attention mechanism (line 12) integrates the smoothed interaction scores with the protein embeddings to generate a context-dependent representation $\mathbf{p}_M$. This mechanism, detailed in Section \ref{sec:daa_mechanism}, learns to balance physical interaction information with sequence-based patterns, producing protein representations that dynamically adapt to different molecular contexts.


\subsection{Architectural Components}

\subsubsection{Base Protein Encoder}
\label{sec:base_encoder}
We use ESM3-6B as our base protein encoder, leveraging its state-of-the-art performance in protein representation learning. Given a protein sequence $P$, the encoder generates embeddings for each amino acid:

\begin{equation}
    \mathbf{E} = [\mathbf{e}_1, ..., \mathbf{e}_n] = \text{ESM3}(P)
\end{equation}

where $\mathbf{e}_i \in \mathbb{R}^d$ represents the embedding of the $i$-th amino acid, and $n$ is the sequence length.



\subsubsection{Molecular Interaction Module}
\label{sec:mol_interaction}
In biological systems, proteins and molecules exist as three-dimensional structures that physically interact in space. Molecular docking predicts how a molecule will position itself and bind to a protein - essentially determining the 3D coordinates where the molecule and protein will form a complex. This spatial arrangement is crucial as it determines the strength and nature of their interaction.

For predicting these protein-molecule interactions, we employ DiffDock \cite{DiffDock2023}, a state-of-the-art molecular docking method that achieves superior accuracy compared to traditional approaches. A key advantage of DiffDock is its foundation in diffusion-based generative modeling, which allows us to sample multiple possible binding configurations for each protein-molecule pair. Unlike traditional docking methods that produce a single prediction, DiffDock generates a distribution of potential binding poses:

\begin{equation}
    \{\mathbf{p}_i^{3D}, \mathbf{m}_j^{3D}\}_k = \text{DiffDock}(P, M, k), \quad k = 1,...,K
\end{equation}

Where $k$ indexes the different sampled configurations and $K$ is the number of samples. This sampling approach better reflects the dynamic nature of protein-molecule interactions and provides a more robust estimation of binding preferences.

Once we have multiple protein-molecule configurations in the same coordinate space, our next goal is to determine which amino acids are most important for this specific protein-molecule interaction. The key insight is that amino acids that interact strongly with the molecule across multiple binding poses are likely to play crucial roles in the protein's function regarding that specific molecule.

To quantify these interaction strengths, we use the Lennard-Jones potential~\cite{tee1966molecular,wang2020lennard}, a well-established approximation in molecular physics that captures both attractive and repulsive forces between particles. The potential includes a repulsive term ($r^{-12}$) that dominates at short interatomic distances, modeling strong short-range repulsion due to quantum mechanical exchange interactions and the Pauli exclusion principle. Additionally, it includes an attractive term ($r^{-6}$) that represents van der Waals interactions, which arise from induced dipole-dipole (London dispersion) forces. While the Lennard-Jones potential is an empirical approximation, it effectively captures these essential features of intermolecular interactions. For each sampled molecular configuration, we compute the system's total potential energy as the sum of all pairwise Lennard-Jones interactions:



\begin{equation}
    S_i = \frac{1}{K}\sum_{k=1}^K \sum_{j=1}^{n_m} 4\varepsilon \left[\left(\frac{\sigma}{r_{ij}^k}\right)^{12} - \left(\frac{\sigma}{r_{ij}^k}\right)^{6}\right]
\end{equation}

where:
\begin{itemize}
    \item $r_{ij}^k = \|\mathbf{p}_i^{3D} - \mathbf{m}_j^{3D}\|_2$ is the distance between amino acid $i$ and molecule atom $j$ in the $k$-th sampled configuration
    \item $\varepsilon$ determines the depth of the potential well 
    \item $\sigma$ is the distance at which the potential becomes zero
\end{itemize}

The resulting score $S_i$ provides a physics-based measure of the interaction strength between each amino acid and the molecule averaged across multiple possible binding modes. Higher absolute values of $S_i$ indicate stronger consistent interactions across different binding poses, suggesting that amino acid $i$ plays a more important role in this specific protein-molecule interaction.

These ensemble-averaged interaction scores will later bias our attention mechanism, allowing the model to focus more on amino acids that are physically relevant to the specific molecular interaction being considered. 


\subsubsection{Adaptive Score Smoothing}
\label{sec:mol_smooth}
To balance local and global information, we introduce an adaptive smoothing mechanism:

\begin{equation}
    \hat{S}_i = \beta V_i + (1-\beta) \cdot \frac{1}{n}\sum_{j=1}^n S_j
\end{equation}

The learnable parameter $\beta$ allows the model to automatically determine the optimal balance between local interaction signals and global protein context.



\subsection{Docking-Aware Attention Mechanism}
\label{sec:daa_mechanism}
Finally, we use the smoothed interaction scores to guide an attention mechanism combining amino acid embeddings into a context-dependent protein representation. Our modified attention mechanism is defined as:

\begin{equation}
    \text{Attention}(\mathbf{Q}, \mathbf{K}, \mathbf{V}, \mathbf{S}) = \text{softmax}\left(\frac{\mathbf{Q}\mathbf{K}^T + \gamma{\mathbf{S}}}{\sqrt{d}}\right)\mathbf{V}
\end{equation}

where $\mathbf{Q}, \mathbf{K}, \mathbf{V}$ are query, key, and value matrices that the model learns during training, ${\mathbf{S}}$ contains the smoothed interaction scores which vary based on the specific molecule being considered, and $\gamma$ is a parameter that learns how to balance docking information with learned attention patterns.


Unlike standard attention mechanisms that only learn from sequence patterns, our model learns during training how to effectively combine learned attention ($\mathbf{Q}\mathbf{K}^T$) with physical interaction information ($\mathbf{S}$). The trained parameters ($\mathbf{Q}, \mathbf{K}, \mathbf{V}$, and $\gamma$) become fixed, learning optimal patterns for incorporating docking information. During inference, while these parameters remain constant, the docking scores $\mathbf{S}$ vary based on each new molecule. This allows our attention mechanism to generate molecule-specific attention patterns by combining fixed learned weights with dynamic physical interaction scores - essentially creating protein representations that adapt to each molecular partner while preserving the general patterns learned during training.




%%%%%%%%%%%%%%%%%%%%%%%%%%%%%%%%%%%%%%%%%%%%%%%%%%%%%%%%%%%%%%%%
%%%%%%%%%%%%%%%%        ARC Figure  %%%%%%%%%%%%%%%%%%%%%%%%%%%
%%%%%%%%%%%%%%%%%%%%%%%%%%%%%%%%%%%%%%%%%%%%%%%%%%%%%%%%%%%%%%%%
\begin{figure*}[ht]
    \includegraphics[width=0.95\textwidth]{figures/biocat_daa.png}
    \caption{Overview of the biocatalysis generation pipeline. The model takes as input a catalyst enzyme and input molecule in SMILES format. These inputs are processed through our Docking-Aware Attention (DAA) mechanism to generate a molecule-specific protein representation. This representation is incorporated as a special token in the encoder,  which processes the input SMILES string. The decoder then predicts the output molecule's SMILES string, representing the reaction product.}
    \label{fig:biocat}
\end{figure*}

%%%%%%%%%%%%%%%%%%%%%%%%%%%%%%%%%%%%%%%%%%%%%%%%%%%%%%%%%%%%%%%%
%%%%%%%%%%%%%%%%      end  Figure  %%%%%%%%%%%%%%%%%%%%%%%%%%%
%%%%%%%%%%%%%%%%%%%%%%%%%%%%%%%%%%%%%%%%%%%%%%%%%%%%%%%%%%%%%%%%
\section{Empirical Evaluation}
\subsection{Task}
Biocatalysis prediction represents a fundamental task in computational biology, where enzymes catalyze chemical reactions under milder conditions with higher specificity compared to traditional chemical catalysis. The core task involves predicting the product molecule given an input molecule and its catalyzing enzyme.

We utilize ECREACT \cite{Probst2022}, currently the largest publicly available biocatalysis dataset, which aggregates enzymatic reactions from major biochemical databases including Rhea \cite{bansal2022rhea}, BRENDA \cite{placzek2016brenda}, PathBank \cite{wishart2020pathbank}, and MetaNetX \cite{ganter2013metanetx}. The dataset provides approximately 60,000 unique reactions, each containing input molecule, output molecule, and catalyzing enzyme information.

A significant challenge in enzymatic reaction prediction is the limited availability of training data compared to general chemical reactions. Following established practice in the field \cite{Probst2022,Chen2023,kreutter2021predicting,goshisht2024machine}, enzymatic reaction datasets are typically augmented with the USPTO dataset \cite{marco2015uspto}, which contains approximately 1 million non-enzymatic reactions. This augmentation commonly employs upsampling of non-enzymatic to enzymatic reactions. Previous works have demonstrated that this data augmentation strategy significantly improves model generalization by exposing the models to a broader range of chemical transformations, with multiple studies reporting enhanced performance across various reaction types \cite{Probst2022,Chen2023}.

\subsection{Baseline Methods}

We evaluate our approach against a diverse set of baseline methods, spanning three key categories: approaches that use no protein information, methods that rely on supervised EC number classification for discrete protein representation, and approaches that employ dense protein representations.

The first category of baselines operates without explicit protein information. The Chemical-Only baseline \cite{kreutter2021predicting} employs T5 architecture to predict reactions based solely on SMILES strings, relying on the model to learn enzyme-specific patterns purely from the reaction data. 
For methods relying on supervised enzyme classification, the EC Token method \cite{Probst2022} represents enzymes using their enzyme commission (EC) numbers as special tokens, requiring supervised enzyme classification information and limiting generalization to novel proteins that haven't been manually annotated with EC numbers.
For methods incorporating protein information through dense representations, we evaluate several approaches. ESM3 \cite{hayes2024simulating} and ProtBERT \cite{brandes2021proteinbert} utilize protein language models to generate sequence-based embeddings, while GearNet \cite{zhang2023protein} incorporates three-dimensional protein structure through graph neural networks. Additionally, ReactEmbed \cite{sicherman2025reactembed} attempts to capture protein-molecule relationships through joint embedding training. While these methods provide rich protein representations, they all generate static embeddings that remain fixed regardless of the molecular context.

This comprehensive comparison allows us to assess the effectiveness of different protein representation strategies in enzymatic reaction prediction.


\subsection{Implementation Details}
\subsubsection{Problem Formulation}
Following previous works, the task is formulated as a sequence-to-sequence translation problem \cite{schwaller2019molecular,gricourt2024artificial,kreutter2021predicting}, where molecular structures are represented using SMILES (Simplified Molecular Input Line Entry System) notation. As shown in Figure \ref{fig:biocat}, the input molecule's SMILES string is ``translated'' into the product molecule's SMILES string using transformer-based architectures.

\subsubsection{Model Architecture}
For processing molecular structures, we employ the T5 architecture consisting of 6 encoder and decoder layers, with model dimension of 512, feed-forward dimension of 2048, and 8 attention heads. All SMILES sequences are tokenized and padded to a maximum length of 200 tokens. This chemical processing backbone remains consistent across all experimental configurations.

For methods incorporating protein information (ESM3, ProtBERT, GearNet, ReactEmbed, and our DAA), the protein representation is injected as a special token at the beginning of the encoder sequence, as illustrated in Figure \ref{fig:biocat}. 

\subsubsection{Training Protocol}
All models are trained on NVIDIA A40 GPUs using the AdamW optimizer with a learning rate of 1e-5 and batch size of 256. To ensure fair comparison, we maintain identical architectural configurations and training procedures across all experimental conditions, varying only the method of protein representation between our approach and the baselines. This standardization ensures that performance differences can be attributed specifically to the protein representation strategy rather than training dynamics or model capacity.

All code and pre-trained models are available at GitHub\footnote{https://anonymous.4open.science/r/DockingAwareAttention-8B8E} to ensure reproducibility.


%%%%%%%%%%%%%%%%%%%%%%%%%%%%%%%%%%%%%%%%%%%%%%%%%%%%%%%%%%%%%%%%
%%%%%%%%%%%%%%%%        MAIN RES  %%%%%%%%%%%%%%%%%%%%%%%%%%%
%%%%%%%%%%%%%%%%%%%%%%%%%%%%%%%%%%%%%%%%%%%%%%%%%%%%%%%%%%%%%%%%
\definecolor{myblue}{RGB}{120,145,181}


\begin{table*}[]
\centering
\small
\caption{Performance comparison across different generators and benchmarks. We evaluate different configurations, with critique-revision representing an iterative process where a critic model provides feedback to guide solution improvement. Pass@1 shows the success rate, while $\Delta_\uparrow$ and $\Delta_\downarrow$ indicate the percentage of wrong solutions being correctly revised and correct solutions being revised to wrong solutions, respectively.
Results are averaged over 5 random seeds.}
\label{tab:main}
\vspace{3mm}



\begin{tabular}{lcccccccccc}
\toprule
\multirow{2}{*}{} & \multicolumn{3}{c}{\textbf{CodeContests}} & \multicolumn{3}{c}{\textbf{LiveCodeBench}} & \multicolumn{3}{c}{\textbf{MBPP+}} & \textbf{Average} \\
 & \multicolumn{1}{c}{\textbf{Pass@1}} & \multicolumn{1}{c}{\textbf{$\Delta_\uparrow$}} & \multicolumn{1}{c}{\textbf{$\Delta_\downarrow$}} & \multicolumn{1}{c}{\textbf{Pass@1}} & \multicolumn{1}{c}{\textbf{$\Delta_\uparrow$}} & \multicolumn{1}{c}{\textbf{$\Delta_\downarrow$}} & \multicolumn{1}{c}{\textbf{Pass@1}} & \multicolumn{1}{c}{\textbf{$\Delta_\uparrow$}} & \multicolumn{1}{c}{\textbf{$\Delta_\downarrow$}} & \textbf{Pass@1} \\ \midrule
\rowcolor{gray!10} \multicolumn{11}{c}{\textit{Qwen2.5-Coder as Generator}} \\
Zero-shot & 7.88 & 0.00 & 0.00 & 30.54 & 0.00 & 0.00 & 77.83 & 0.00 & 0.00  & 38.75 \\
\emph{Single-turn Critique-revision} \\
Critique w/ Qwen2.5-Coder & 8.36 & 2.30 & 1.82 & 32.14 & 2.50 & 0.89 & 77.83 & 3.49 & 3.49 & 39.45 \\
Critique w/ GPT-4o & 10.67 & \textbf{4.85} & 2.06 & 32.32 & 2.32 & \textbf{0.54} & 77.46 & \textbf{3.81} & 4.18 & 40.15 \\
\rowcolor{myblue!20} Critique w/ {\ours} & \textbf{11.76} & 4.73 & \textbf{0.85} & \textbf{33.21} & \textbf{3.39} & 0.71 & \textbf{78.84} & 2.43 & \textbf{1.43} & \textbf{41.27} \\
\emph{Multi-turn Critique-revision} \\
Critique$\times 5$ w/ Qwen2.5-Coder & 9.21 & 3.76 & 2.42 & 29.64 & 2.14 & 3.04 & 76.03 & 3.81 & 5.61 & 38.30 \\
Critique$\times 5$ w/ GPT-4o & 12.48 & 7.03 & 2.42 & 32.86 & \textbf{4.82} & 2.50 & 74.60 & \textbf{4.34} & 	\textbf{7.57} &	39.98 \\
\rowcolor{myblue!20} Critique$\times 5$ w/ {\ours} & \textbf{16.24} & \textbf{9.21} & \textbf{0.85} & \textbf{33.39} & 3.75 & \textbf{0.89} & \textbf{78.68} & 3.23 & 2.38 & \textbf{42.77} \\
\midrule
\rowcolor{gray!10} \multicolumn{11}{c}{\textit{GPT-4o as Generator}} \\
Zero-shot & 20.61 & 0.00 & 0.00 & 32.32 & 0.00 & 0.00 & 77.67 & 0.00 & 0.00 & 43.53 \\
\emph{Single-turn Critique-revision} \\
Critique w/ Qwen2.5-Coder & 20.24 & 3.52 & 3.88 & \textbf{35.36} & \textbf{3.93} & 0.89 & 76.67 & 0.85 & 1.85 & 44.09 \\
Critique w/ GPT-4o & 20.97 & 2.30 & \textbf{1.94} & 34.82 & 2.68 & \textbf{0.18} & 77.41 & \textbf{1.01} & 1.27 & 44.40 \\
\rowcolor{myblue!20} Critique w/ {\ours} & \textbf{23.03} & \textbf{4.97} & 2.55 & 33.39 & 2.14 & 1.07 & \textbf{77.83} & 0.53 & \textbf{0.37} & \textbf{44.75} \\
\emph{Multi-turn Critique-revision} \\
Critique$\times 5$ w/ Qwen2.5-Coder & 19.52 & 5.21 & 6.30 & \textbf{35.54} & \textbf{5.36} & 2.14 & 76.08 & 1.53 & 3.12 & 43.71 \\
Critique$\times 5$ w/ GPT-4o & 20.61 & 3.39 & 3.39 & 35.18 & 3.21 & \textbf{0.36} & 76.61 & \textbf{2.06} & 3.12 & 44.13 \\
\rowcolor{myblue!20} Critique$\times 5$ w/ {\ours} & \textbf{25.45} & \textbf{7.88} & \textbf{3.03} & 34.11 & 3.21 & 1.43 &  \textbf{77.94} & 0.79 & \textbf{0.53} & \textbf{45.83} \\ \bottomrule
\end{tabular}







\end{table*}

%%%%%%%%%%%%%%%%%%%%%%%%%%%%%%%%%%%%%%%%%%%%%%%%%%%%%%%%%%%%%%%%
%%%%%%%%%%%%%%%%      MAIN RES  %%%%%%%%%%%%%%%%%%%%%%%%%%%
%%%%%%%%%%%%%%%%%%%%%%%%%%%%%%%%%%%%%%%%%%%%%%%%%%%%%%%%%%%%%%%%

\subsection{Evaluation Protocol}
\label{sec:evaluation}
Following standard practice in reaction prediction literature \cite{Chen2023,kreutter2021predicting,schwaller2019molecular,gricourt2024artificial,Probst2022}, we evaluate models using Top-k accuracy metrics. In the context of chemical reaction prediction, where multiple valid products may exist for a given reaction, Top-k accuracy is particularly relevant as it captures the model's ability to propose reasonable reaction outcomes. Specifically, a prediction is considered correct if the true product appears among the model's k highest-confidence predictions.

\subsubsection{Statistical Significance Testing}
To assess the statistical significance of our results, we employ the Two-Proportion Z-Test. For this analysis, we consider each model prediction as a binary outcome (correct or incorrect), enabling direct comparison between different approaches. 
The Two-Proportion Z-Test evaluates whether the difference in success rates between two models is statistically significant by comparing their observed success rates and sample sizes.
For all reported performance improvements, we maintain a significance level of $\alpha = 0.05$.







\section{Empirical Results}
Our empirical results demonstrate the effectiveness of DAA across multiple evaluation scenarios. We analyze the performance on three key aspects: overall prediction accuracy, handling of complex molecules, and generalization to innovative reactions.
As discussed in Section \ref{sec:evaluation}, we evaluate performance using top-k accuracy.


Table \ref{tab:res-main} summarizes the results. DAA consistently outperforms all baseline approaches across all evaluation metrics. At K=1, DAA achieves 49.96\% accuracy, representing a significant improvement over both the Chemical-Only baseline (37.29\%) and methods using static protein representations such as ESM3 (44.58\%) and ProtBERT (43.49\%). Even compared to the strong ReactEmbed baseline (47.41\%), DAA shows a clear advantage with a 2.55 percentage point improvement.
The performance gap widens for higher K values, with DAA achieving 66.65\% at K=3 and 71.48\% at K=5, compared to ReactEmbed's 61.19\% and 66.25\% respectively. This consistent improvement across different K values suggests that DAA's dynamic protein representations provide valuable information for ranking potential reaction products.
Notably, the EC Tokens approach (46.01\% at K=1) performs better than sequence-based methods like ESM3 and ProtBERT, highlighting the value of enzyme classification information. However, DAA's superior performance (49.96\% at K=1) demonstrates that our context-aware representations capture more nuanced protein-substrate relationships than static EC number assignments.


\subsection{Performance on High-Complexity Reactions}
We evaluate performance on reactions involving complex molecules, defined using the Bertz Complexity Index \cite{bertz1981first}. For this analysis, we classify molecules with a Bertz Complexity Index exceeding 1500 as complex. These molecules present particular challenges for reaction prediction due to their numerous potential reaction sites and complex stereochemical considerations. When evaluating performance on these complex-molecule reactions (Table \ref{tab:res-main}), the advantages of DAA become even more pronounced. Our method achieves 41.62\% accuracy at K=1, substantially outperforming all baselines including ReactEmbed (36.90\%) and EC Tokens (34.26\%). This represents a nearly 13\% relative improvement over the next best method. The performance gap is particularly notable when compared to structure-based approaches like GearNet (28.09\% at K=1), suggesting that static structural representations alone are insufficient for handling complex molecular interactions. The significant drop in performance for all baselines on complex molecules (compared to the overall results) highlights the challenging nature of these cases, making DAA's robust performance especially valuable.

\subsection{Performance on Innovative Reactions}

Perhaps most significantly, DAA shows strong generalization capabilities when predicting innovative reactions - those involving target molecules not seen during training (Table \ref{tab:res-main}). Our method achieves 35.24\% accuracy at K=1 and 50.66\% at K=3, substantially outperforming both the Chemical-Only baseline (18.25\% and 33.75\%) and sophisticated protein representation methods like ESM3 (27.39\% and 42.35\%).
The performance gap between DAA and EC Tokens (31.42\% at K=1) is particularly noteworthy in this scenario, as it demonstrates that our dynamic representations better capture the underlying principles of enzyme-substrate interactions rather than simply memorizing known reaction patterns. This advantage in predicting novel transformations is crucial for practical applications in biocatalysis discovery and enzyme engineering.
Interestingly, structure-based methods like GearNet show relatively poor performance on innovative reactions (20.16\% at K=1), suggesting that static structural information alone may not generalize well to novel chemical transformations. The significant improvement achieved by DAA (35.24\% at K=1) validates our approach of combining physical interaction information through docking scores with learned attention patterns.

\subsection{Ablation Studies}
%%%%%%%%%%%%%%%%%%%%%%%%%%%%%%%%%%%%%%%%%%%%%%%%%%%%%%%%%%%%%%%%
%%%%%%%%%%%%%%%%        ABL RES  %%%%%%%%%%%%%%%%%%%%%%%%%%%
%%%%%%%%%%%%%%%%%%%%%%%%%%%%%%%%%%%%%%%%%%%%%%%%%%%%%%%%%%%%%%%%
\begin{table}[ht]
\caption{Ablation study results examining three key components: (1) Attention mechanism variants, comparing our full DAA approach against standard attention and docking-only alternatives, (2) Embedding model performance, comparing each model with and without DAA, and (3) Token integration strategies, evaluating different methods for incorporating protein representations into the sequence model. All results show top-k accuracy percentages.}
\label{tab:ablation-full}
\centering
\begin{tabular}{lccc}
\toprule
\textbf{Component} & \textbf{Top-1 (\%)} & \textbf{Top-3 (\%)} & \textbf{Top-5 (\%)} \\
\midrule
\multicolumn{4}{c}{Attention Mechanism} \\ \hline
Full DAA & \textbf{49.96} & \textbf{66.65} & \textbf{71.48} \\
Standard Attention & 46.55 & 61.41 & 67.47 \\
Docking-Only & 45.63 & 62.34 & 66.95 \\
\midrule
\multicolumn{4}{c}{Embedding Models} \\ \hline
\multicolumn{4}{l}{\textit{ESM3}} \\
\quad \quad Base Model  &44.58 &60.64 &65.62 \\
\quad \quad DAA Enhanced & \textbf{49.96} & \textbf{66.65} & \textbf{71.48} \\
\multicolumn{4}{l}{\textit{ProtBERT}} \\
\quad \quad Base Model &43.49 &59.52 &64.8 \\
\quad \quad DAA Enhanced &\textbf{ 45.72} & \textbf{61.68} & \textbf{67.64} \\
\multicolumn{4}{l}{\textit{GearNet}} \\
\quad \quad Base Model &43.95 &60.10 &63.84 \\
\quad \quad DAA Enhanced & \textbf{44.84} & \textbf{60.80} & \textbf{66.76} \\
\midrule
\multicolumn{4}{c}{Token Integration} \\ \hline
New Token & \textbf{49.96} & \textbf{66.65} & \textbf{71.48} \\
Concatenation & 49.51 & 62.69 & 66.77 \\
Addition & 40.23 & 57.113 & 63.49 \\
\bottomrule
\end{tabular}
\end{table}
%%%%%%%%%%%%%%%%%%%%%%%%%%%%%%%%%%%%%%%%%%%%%%%%%%%%%%%%%%%%%%%%
%%%%%%%%%%%%%%%%      ABL RES  %%%%%%%%%%%%%%%%%%%%%%%%%%%
%%%%%%%%%%%%%%%%%%%%%%%%%%%%%%%%%%%%%%%%%%%%%%%%%%%%%%%%%%%%%%%%


To thoroughly validate our approach and understand the contribution of each architectural component, we conducted three comprehensive ablation studies examining attention mechanisms, embedding model selection, and token integration strategies. Table~\ref{tab:ablation-full} presents the quantitative results across all variants, demonstrating the impact of each design choice on model performance.

\subsubsection{Attention Mechanism Analysis}
Our first ablation study evaluates different variants of the attention mechanism to understand the relative contribution of learned patterns versus physical interaction information. The results demonstrate the complementary nature of these information sources. While the standard attention mechanism achieves solid performance (67.47\% at top-5) by capturing sequence-based patterns, the docking-only approach (66.95\% at top-5) shows that physical interaction scores alone provide valuable signals for understanding protein-molecule relationships. The superior performance of our full DAA architecture (71.48\% at top-5) validates our hypothesis that combining both sources of information enables more effective modeling of protein-molecule interactions. This significant improvement over individual components suggests that learned attention patterns and physical docking information capture different aspects of protein behavior, leading to more comprehensive and accurate representations when integrated through our DAA mechanism.

\subsubsection{Impact of Embedding Models}
The second ablation study examines the influence of different protein embedding architectures on overall performance. Results show that incorporating DAA consistently improves performance across all embedding models tested. The base ESM3 model achieves 65.62\% top-5 accuracy as a baseline, improving substantially to 71.48\% with DAA enhancement. Similar improvements are observed with ProtBERT (64.8\% to 67.64\%) and GearNet (63.84\% to 66.76\%). These results demonstrate that DAA's benefits are not limited to a specific embedding architecture but rather represent a general improvement in protein representation learning. The consistent performance gains across different architectures (ranging from 2.9 to 5.9 percentage points) suggest that DAA's context-aware attention mechanism provides complementary information regardless of the base embedding model's architecture.

\subsubsection{Token Integration Strategies}
Our final ablation study investigates different methods for incorporating the protein representation vector into the sequence-to-sequence architecture. The substantial performance differences between integration strategies highlight the importance of proper protein information integration. Our new token approach achieves the best performance (71.48\% at top-5), significantly outperforming both concatenation (66.77\%) and addition (63.49\%) strategies. The poor performance of simple addition suggests that naive combination strategies can actually degrade model performance, while our new token approach allows the model to learn optimal integration patterns. The nearly 8 percentage point gap between the best and worst performing strategies emphasizes that the method of combining protein representations with molecular information significantly impacts the model's ability to leverage protein-specific context for reaction prediction.

\section{Discussion}

\subsection{Analysis of Attention Distribution Patterns}

%%%%%%%%%%%%%%%%        FIG DIST  %%%%%%%%%%%%%%%%%%%%%%%%%%%
%%%%%%%%%%%%%%%%%%%%%%%%%%%%%%%%%%%%%%%%%%%%%%%%%%%%%%%%%%%%%%%%
\begin{figure}[h]
    \includegraphics[width=0.45\textwidth]{figures/exmp3113.png}
    \caption{Attention patterns of triacylglycerol lipase (EC 3.1.1.3) across three reactions, showing: attention weights in sequence space (left), 3D structural visualization with attention intensity (center), and corresponding chemical reactions (right). The varying patterns demonstrate DAA's context-dependent adaptation.}
    \label{fig:exmp}
\end{figure}

%%%%%%%%%%%%%%%%%%%%%%%%%%%%%%%%%%%%%%%%%%%%%%%%%%%%%%%%%%%%%%%%
%%%%%%%%%%%%%%%%      FIG DIST  %%%%%%%%%%%%%%%%%%%%%%%%%%%
%%%%%%%%%%%%%%%%%%%%%%%%%%%%%%%%%%%%%%%%%%%%%%%%%%%%%%%%%%%%%%%%

Figure \ref{fig:exmp} illustrates the dynamic nature of DAA through a visualization of triacylglycerol lipase (EC 3.1.1.3) participating in three different reactions. The visualization reveals how our attention mechanism generates distinct patterns for the same enzyme when catalyzing different reactions, adapting its focus based on the specific molecular context. Each row in the figure presents a different reaction scenario, with the attention patterns visualized in both sequence space (1D attention profile) and structural space (3D protein representation), alongside the corresponding chemical transformation.

These visualizations provide compelling evidence that DAA successfully overcomes the limitations of static protein representations. By generating context-specific attention patterns, our method captures the inherent flexibility of enzyme behavior in different molecular environments. The apparent differences in attention distribution across different reactions, obvious in sequence profiles and structural visualizations, validate our core hypothesis about the importance of dynamic, interaction-aware protein representations in accurately modeling enzyme-catalyzed reactions.

%%%%%%%%%%%%%%%%%%%%%%%%%%%%%%%%%%%%%%%%%%%%%%%%%%%%%%%%%%%%%%%%



\subsection{Geometric Analysis of Protein Space}
%%%%%%%%%%%%%%%%%%%%%%%%%%%%%%%%%%%%%%%%%%%%%%%%%%%%%%%%%%%%%%%%
%%%%%%%%%%%%%%%%        FIG PCA  %%%%%%%%%%%%%%%%%%%%%%%%%%%
%%%%%%%%%%%%%%%%%%%%%%%%%%%%%%%%%%%%%%%%%%%%%%%%%%%%%%%%%%%%%%%%
\begin{figure}[h]
    \centering
    \includegraphics[width=1\linewidth]{figures/daa_protein_vectors.png}
    \caption{PCA visualization of DAA-generated protein embeddings, showing protein-specific clusters (colors) with intra-cluster variation. Each point represents a protein in a specific molecular context, demonstrating both preserved protein identity and context-dependent adaptation. Analysis covers 10 ECREACT proteins with ~50 molecular contexts each.}
    \label{fig:protein_embeddings}
\end{figure}

%%%%%%%%%%%%%%%%%%%%%%%%%%%%%%%%%%%%%%%%%%%%%%%%%%%%%%%%%%%%%%%%
%%%%%%%%%%%%%%%%      ABL PCA  %%%%%%%%%%%%%%%%%%%%%%%%%%%
%%%%%%%%%%%%%%%%%%%%%%%%%%%%%%%%%%%%%%%%%%%%%%%%%%%%%%%%%%%%%%%%



To understand the geometric properties of our learned protein representations, we analyzed the embedding space using Principal Component Analysis (PCA). We first trained PCA on all protein-substrate pairs in our dataset to learn the principal components. For visualization purposes, we then randomly selected 10 diverse proteins from ECREACT, each with approximately 50 different substrate interactions, and projected their embeddings onto the first two principal components.

Figure~\ref{fig:protein_embeddings} reveals several key properties of the learned representations: Despite their context-dependent nature, proteins maintain distinct identities by forming consistent clusters in the representation space, demonstrating DAA's ability to preserve protein-specific information while adapting to different molecular contexts. Within these clusters, we observe significant spread in the embedding positions, reflecting dynamic adaptations to different molecular contexts rather than collapsing to static, single-point representations. This intra-cluster variation illustrates how DAA generates flexible representations that respond to specific substrate interactions. Notably, while proteins generally maintain distinct clusters, we observe partial overlap between certain clusters in the embedding space, suggesting that DAA captures shared properties between different proteins when processing certain substrates. The biological basis for these inter-protein relationships merits further investigation.

These results demonstrate that DAA achieves a crucial balance: the embeddings maintain distinct protein identities while adapting to different molecular contexts. The organization of the embedding space shows both clear protein-specific clustering and meaningful variation based on molecular interactions, supporting DAA's effectiveness in generating dynamic protein representations.

\section{Conclusion}
This work introduces Docking-Aware Attention (DAA), a novel architecture that addresses a fundamental challenge in enzymatic reaction prediction: capturing the context-dependent nature of enzyme-substrate interactions. By incorporating molecular docking information into the attention mechanism, DAA generates dynamic protein representations that adapt based on specific molecular interactions. Our comprehensive evaluation demonstrates significant improvements over existing methods, particularly in challenging scenarios involving complex molecules and innovative reactions.
 
 The success of DAA in biocatalysis prediction validates our core hypothesis that protein representations should adapt to different molecular contexts. The substantial performance improvements—achieving 62.2\% accuracy versus 56.79\% baseline for complex molecules and 55.44\% versus 49.45\% for innovative reactions demonstrate the practical value of incorporating physical interaction information into reaction prediction systems. Our ablation studies further confirm that each architectural component contributes meaningfully to the final performance, with the combination of learned attention patterns and docking-based interaction scores proving particularly effective.

While our results are promising, several directions for future research emerge:  (i) \textit{Synthetic Route Planning:} Integrating DAA into retrosynthesis planning systems could enable more accurate prediction of feasible enzymatic transformations in multi-step synthesis pathways. This could lead to more efficient routes for complex molecule synthesis by better accounting for enzyme-substrate compatibility at each step; (2) \textit{Reaction Condition Optimization:} The DAA framework could be extended to predict optimal reaction conditions by modeling how enzyme-substrate interactions change under different temperature, pH, and solvent conditions. This could improve reaction yield and selectivity in practical applications;

By open-sourcing our implementation and pre-trained models, we aim to facilitate further research in computational biocatalysis and encourage the development of increasingly sophisticated approaches to enzymatic reaction prediction. DAA's success demonstrates the value of combining physical insights with modern deep learning architectures, pointing toward a promising direction for improving predictive models in chemical synthesis.

\bibliographystyle{ACM-Reference-Format}
\bibliography{ref}

\end{document}