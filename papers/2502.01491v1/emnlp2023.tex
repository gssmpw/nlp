% This must be in the first 5 lines to tell arXiv to use pdfLaTeX, which is strongly recommended.
\pdfoutput=1
% In particular, the hyperref package requires pdfLaTeX in order to break URLs across lines.

\documentclass[11pt]{article}

% Remove the "review" option to generate the final version.
\usepackage[review]{EMNLP2023}

% Standard package includes
\usepackage{times}
\usepackage{soul}
\usepackage{latexsym}

% Standard package includes
\usepackage{times}
\usepackage{latexsym}
\usepackage[algoruled,ruled,vlined,noend]{algorithm2e}
% \usepackage{algorithmicx}
% \usepackage{algorithm} % http://ctan.org/pkg/algorithms
% \usepackage{algpseudocode} % http://ctan.org/pkg/algorithmicx
\usepackage{amsmath}
\usepackage{times}
\usepackage{latexsym}
\usepackage{pgfplots}
\usepackage{graphicx}
\usepackage{enumitem}
% \usepackage{xcolor,colortbl}  
%\usepackage[usenames,dvipsnames]{color}
\usepackage{tikz}
\usepackage{diagbox}
\usepackage{tkz-tab}
\usepackage{caption}
\usepackage{booktabs}
\usepackage{latexsym}
\usepackage{todonotes}
\usepackage{amssymb}
\usepackage{amsmath}
\usepackage{subcaption}
\usepackage{natbib}
\usepackage{blindtext}
\usepackage{url}
\usepackage{color}
\DeclareMathOperator*{\argmax}{arg\,max}
\DeclareMathOperator*{\argmin}{arg\,min}
\usepackage{url}
\usepackage{hyperref}    
\usepackage{cleveref}
\SetAlFnt{\small}
\SetAlCapFnt{\small}
\SetAlCapNameFnt{\small}
\usepackage{ntheorem}
\newtheorem{definition}{Definition}
\newtheorem{assumption}{Assumption}
\usepackage{booktabs, tabularx}
\usepackage{tfrupee}  
\usepackage{caption}

\newcommand{\verna}[1]{\textcolor{purple}{#1}}

\usepackage{epigraph}

% For proper rendering and hyphenation of words containing Latin characters (including in bib files)
\usepackage[T1]{fontenc}
% For Vietnamese characters
% \usepackage[T5]{fontenc}
% See https://www.latex-project.org/help/documentation/encguide.pdf for other character sets

% This assumes your files are encoded as UTF8
\usepackage[utf8]{inputenc}

% This is not strictly necessary, and may be commented out.
% However, it will improve the layout of the manuscript,
% and will typically save some space.
\usepackage{microtype}

% This is also not strictly necessary and may be commented out.
% However, it will improve the aesthetics of text in
% the typewriter font.
\usepackage{inconsolata}


% If the title and author information does not fit in the area allocated, uncomment the following
%
%\setlength\titlebox{<dim>}
%
% and set <dim> to something 5cm or larger.

\title{On the Inheritance of Memorization in Sequence-Level Knowledge Distillation}

% Author information can be set in various styles:
% For several authors from the same institution:
% \author{Author 1 \and ... \and Author n \\
%         Address line \\ ... \\ Address line}
% if the names do not fit well on one line use
%         Author 1 \\ {\bf Author 2} \\ ... \\ {\bf Author n} \\
% For authors from different institutions:
% \author{Author 1 \\ Address line \\  ... \\ Address line
%         \And  ... \And
%         Author n \\ Address line \\ ... \\ Address line}
% To start a separate ``row'' of authors use \AND, as in
% \author{Author 1 \\ Address line \\  ... \\ Address line
%         \AND
%         Author 2 \\ Address line \\ ... \\ Address line \And
%         Author 3 \\ Address line \\ ... \\ Address line}

\author{First Author \\
  Affiliation / Address line 1 \\
  Affiliation / Address line 2 \\
  Affiliation / Address line 3 \\
  \texttt{email@domain} \\\And
  Second Author \\
  Affiliation / Address line 1 \\
  Affiliation / Address line 2 \\
  Affiliation / Address line 3 \\
  \texttt{email@domain} \\}

\begin{document}
\maketitle
\begin{abstract}

\end{abstract}

\section{Introduction}

Memorization of noisy and atypical training data creates unsystematic failure modes for neural sequence models, presenting a reliability and safety risk during their deployment in the real-world. At the same time, real-world deployments of such neural models are typically derived using knowledge distillation, wherein outputs from a larger teacher model are used to train a smaller student neural model. This is done to arrive at a better trade-off between model quality and cost, wherein the student model can often achieve teacher-like performance while being smaller and faster. The empirical efficacy of such sequence-level knowledge distillation is attributed to the `distribution simplification' hypothesis, that the student model only needs to learn the relevant parts of the teacher's distribution, rather than wasting parameters on trying the to model the entire output space \cite{kim-rush-2016-sequence}.

However, despite the widespread adoption of knowledge distillation for building sequence to sequence models such as for neural machine translation (NMT), the link between the student's memorization behavior and sequence-level knowledge distillation is unknown. There are quite a few plausible hypotheses, e.g., given the tight coupling between the teacher and student models, it is quite conceivable that the student retains the memorization behaviors of the teacher. On the other hand, the fact that knowledge distillation regularizes over the noise and atypicality observed in the natural training data distribution might create divergences in the memorization behavior between the teacher and the student models \cite{saglietti2022solvable}. Characterizing this interaction between memorization and distillation along this continuum is an important open problem in the literature since distillation is widely used to build cheaper and faster systems that are used in real-world deployments, wherein mitigating deleterious memorizations is of critical importance.

In this work, we attempt to characterize how memorization is transmitted from the teacher to the student in sequence-level knowledge distillation, with the applied goal of leveraging such a characterization into alleviating the reliability of the student models. As such, we study the following sets of research questions (RQ) in the traditional data-based sequence-level knowledge distillation setting: (RQ1) How can we statistically characterize the memorizations in the distilled student as a function of memorization behavior in the teacher? (RQ2) Does seq-level KD confer new failure memorization modes to the student? Qualitatively, do new `modes' of memorizations appear in the student that are different from the teacher or vice-versa? Does evaluating the teacher give a good estimate of memorization-based model failures that the student will display? (RQ3) Are any pre or post knowledge distillation mitigations possible to address the deleterious memorizations in the student models? Our contributions are as follows:
\begin{enumerate}
    \item We present a comprehensive characterization of \textit{memorization inheritance} in the context of data-based sequence-level knowledge distillation for Neural Machine Translation. 
    \item (\textit{Placeholder content}) We present MemFreeSeqKD, an effective teacher finetuning algorithm which could be directly plugged in the sequence-level knowledge distillation pipeline to mitigate memorizations in the student model, greatly improving student reliability while preserving its average-case performance.
\end{enumerate}

(\textit{Placeholder content}) Our experiments are conducted on 10 language pairs from different WMT datasets (WMT20, WMT22, WMT23), across a range of transformer scales. The code to reproduce our experiments is available at \hyperlink{code_for_release}{code-for-release}.

\section{Related Work}

\paragraph{Memorization in Sequence to Sequence Models} 

\paragraph{Regularization Effects of Knowledge Distillation} Most of the work in exploring the regularization effects of knowledge distillation has been done in the context of image classification models. In the context of sequence to sequence models, prior work has highlighted results on the amplification of hallucinations in a low-resource setting, however deeper characterizations on natural data remains to be done.

\section{Memorization Inheritance}
\label{sec:bibtex}

\begin{itemize}
    \item We can view knowledge distillation as a transformation applied on the full training set: the training set passes from its natural state to KD data (generated by the teacher) to the student (characterized by the student's outputs on the same data). This setting has 4 natural points of interventions: directly in the training data, in the KD data or on the teacher and the student models that we might leverage for the next section.
    \item Our goal is to characterize this transformation specifically in terms of instance-level memorization behavior of the student (we want to make claims on how the student will inherit memorization through the distillation procedure)
    \item Tracking disagreements b/w teacher, student could be useful, analyzing inputs in the neighborhood of such disagreements will also be useful 
    \item We can track what happens to input-output pairs that are extractively memorized by  the teacher, we can track what happens to counterfactually memorized instances as well (either directly or through a stability based proxy of counterfactual memorization metric)
    \item We would want to answer some questions using the collected measurements:
    \item How are the errors made by the teacher manifested in the student model?
    \item Does the student display new modes of memorization?
    \item How does memorization in the student differ across data categories: low-quality samples, outright noisy samples and rare samples?
    \item Are such memorizations calibrated in the sequence generation process?
    \item Can we characterize memorizations in the student as a function of difference in the number of teacher and student parameters? 
\end{itemize}

\textbf{Memorization Measurements:} Extractive + (approximate or exact) Counterfactual Memorization metrics + Known patterns of hallucinations (Oscillatory + Detached). Track them from the initial dataset to the student model. 

\section{Memorization Mitigation}
\label{sec:bibtex}

\begin{itemize}
    \item Are any pre or post KD mitigations possible to address the memorizations in the student? The primary baselines would be: no-mitigation and student-mitigation.
    \item Explore learning-based mitigations along the following axes: finetuning the teacher, finetuning the student. This is different from inference-time mitigations typically studied before \cite{ippolito-etal-2023-preventing}.
    \item Note that teacher and student are quite asymmetric in terms of their characteristics: teacher is bigger, quicker to learn, while student is more compact, in general harder to make it cram -- so the pre and post KD interventions definitely \textit{should} differ -- we should characterize this thororoughly -- early (pre-KD) interventions properties different to late (post-KD) interventions.
\end{itemize}

\textbf{Measurements}: Same memorization measurements applied after using the interventions, plus general quality measurements.

\clearpage

\section{Memorization inheritance}
\label{sec:memorization_inheritance}

\subsection{Experimental setup}
\label{sec:experimental_setup}
In SeqKD, teacher $\theta_T$ is trained on a corpus with source sequences $\mathcal{S}_C$ and targets $\mathcal{T}_C$, and generates translations of $\mathcal{S}_C$ with beam size $k$. The source sequences and the teacher-generated translations ($\mathcal{S}_C$ and $\mathcal{T}_T$) form the training corpus for student $\theta_S$.
We use data from the WMT20 corpus \citep{barrault-etal-2020-findings}, for five language pairs: \textsc{De}-\textsc{En} and \textsc{En}-\textsc{De} (48M), \textsc{Pl}-\textsc{En} and \textsc{En}-\textsc{Pl} (12M), and \textsc{Fr}-\textsc{De} (14M). \appendixshortcut~\ref{ap:data} provides more detail about the WMT corpora, and the validation and test data.

For all languages, a Transformer-large teacher trains for 300k steps on $\mathcal{S}_C$ and $\mathcal{T}_C$, followed by training a Transformer-base student for 100k steps on $\mathcal{S}_C$ and $\mathcal{T}_T$ ($k{=}1$).\footnote{Training duration set to equal the setup of \citet{vaswani2017attention}. We train using the Marian toolkit \citep{junczys2018marian}. For the full training setup see \appendixshortcut~\ref{ap:data} and our \href{https://github.com/vyraun/memseqkd}{\texttt{codebase}}.} 
We also train base models directly on $\mathcal{S}_C$ and $\mathcal{T}_C$, to have a baseline ($\theta_{B}$) for what the student model would have memorized when exposed directly to WMT20 data.
In \appendixshortcut~\ref{ap:hyperparams}, we furthermore experiment with varying beam size $k$ and the student's model size.

\paragraph{Model quality metrics}
We first examine the models' quality, to ascertain the SeqKD framework works as intended. 
We report the following reference-based metrics for translations generated with beam size five: \textbf{BLEU} and \textbf{chrF}, %to quantify $n$-gram and character-based overlap between translations and their references,
Translation Error Rate (\textbf{TER}), 
and the \textbf{Comet-20} and \textbf{Comet-22} metrics that use neural methods for translation quality estimation \citep{rei2020comet, rei2022comet}.
We supplement this with the reference-free metrics of \textbf{Comet-QE-20}, \textbf{Comet-QE-22}.

All metrics will be applied to WMT test data, and the reference-free methods are furthermore applied to translations of monolingual data: CommonCrawl data provided by \citet{barrault-etal-2020-findings}, and data from the Pulpo poetry corpus \citep{de2023alberti}, to examine out-of-domain performance.

\begin{figure}
    \centering
    \includegraphics[width=\columnwidth]{figures/model_quality.pdf}
    \caption{Performance of teacher, student and baseline models for four model quality metrics.}
    \label{fig:model-quality}
    \vspace{-0.2cm}
\end{figure}

\paragraph{Memorization metrics}
We quantify memorization by comparing greedily translated $\mathcal{S}_C$ to $\mathcal{T}_C$ for all models, and to $\mathcal{T}_T$, for $\theta_S$. We measure the \textbf{replication} (exact match) rate, and the \textbf{extractive memorization} \citep[ExMem,][]{raunak2022finding} rate. ExMem finds examples for which models memorized to emit the target after seeing at most $75$\% of the source, e.g., see Example~\ref{ex1}.
The ExMem rate is the percentage of extractively memorized examples out of the replicated examples. 

We also quantify the hallucination rate. \citet{raunak2021curious}---who relate memorization in NMT to hallucinations---suggested this could influence SeqKD students, but did not actually evaluate that. We firstly measure the detached \textbf{natural hallucinations} (NatHal) rate: the percentage of source sequences that map to a translation that is repeated in the model's translations for at least five times.
Secondly, we compute the \textbf{oscillatory hallucinations} (OscHal) rate: the percentage of translations with bigrams repeated at least 10 times (for which the source did not contain such a repeated bigram).\footnote{To improve the precision of the memorization metrics, some training examples are excluded (see \appendixshortcut~\ref{ap:add_results}).} 

\subsection{Results}
\paragraph{General model quality}
\figureshortcut~\ref{fig:model-quality} and \appendixshortcut~\ref{ap:add_results} provide performance differences for $\theta_T$, $\theta_S$ and $\theta_B$.
Overall, $\theta_T$ outperforms $\theta_S$, and $\theta_S$ outperforms $\theta_B$. $\theta_S$ and $\theta_B$ merely differ in the training targets, which demonstrates that our SeqKD pipeline works as intended.
The ordering of models in terms of their quality also holds for CommonCrawl and Pulpo data (see Table~\ref{tab:ap:model_quality}, \appendixshortcut~\ref{ap:add_results}).

\begin{figure}[!t]
    \centering
    \begin{subfigure}[b]{0.49\columnwidth}
        \includegraphics[height=1.43cm, right]{figures/replication_increase_barplot.pdf}
        \includegraphics[width=0.95\textwidth, right]{figures/replication_scatter.pdf}
        \caption{Replication metric}
        \label{fig:mem-metrics-replication}
    \end{subfigure}
    \begin{subfigure}[b]{0.49\columnwidth}
        \includegraphics[height=1.47cm, right]{figures/exmem_increase_barplot.pdf}
        \includegraphics[width=0.95\textwidth, right]{figures/exmem_scatter.pdf}
        \caption{ExMem rate}
        \label{fig:mem-metrics-exmem}
    \end{subfigure}
    \begin{subfigure}[b]{\columnwidth}\centering
        \includegraphics[width=0.8\textwidth]{figures/exmem_barplot_num.pdf}
        \caption{Number of ExMem examples}
        \label{fig:exmem_numbers}
    \end{subfigure}
    \caption{Memorization metrics for $\theta_T$, $\theta_S$ and $\theta_B$ and the percentual increase comparing $\theta_S$ to $\theta_B$.}
    \label{fig:mem-metrics}
\end{figure}


\paragraph{SeqKD facilitates memorization}
\figureshortcut~\ref{fig:mem-metrics} summarizes the memorization results.
If we first look at the replication rate, $\theta_S$ replicates less from WMT20 than $\theta_T$ but \textbf{more than $\theta_B$}: students' replication rate with respect to $\mathcal{T}_C$ is 3.4\%($\pm$0.9) higher than for $\theta_B$.
Students also replicate original material from $\theta_T$: the overall student replication rate for $\mathcal{T}_T$ is 35.3\%($\pm2.7$) (see Table~\ref{tab:ap:memorization}, \appendixshortcut~\ref{ap:add_results}).

For the ExMem rate with respect to $\mathcal{T}_C$ (\figureshortcut~\ref{fig:mem-metrics-exmem}), a similar pattern emerges, but with a starker difference between $\theta_S$ and $\theta_B$: students extractively memorize less from $\mathcal{T}_C$ compared to $\theta_T$, but \textbf{more compared to $\theta_B$}, with a mean increase of 57.0\%($\pm15.4$).
This is quite surprising; note that by definition, the rate reported here expresses how many of the replicated examples are extractively memorized. Students only observed 18.4\% (on average) of $\mathcal{T}_C$ through the SeqKD pipeline (the portion that $\theta_T$ replicated) and yet within that smaller pool they still memorized more than $\theta_B$, that was exposed to 100\% of the corpus.
Not only have students extractively memorized WMT20 examples (`primary ExMem'), they also show ExMem with respect to $\theta_T$ (`secondary ExMem', quantified in Table~\ref{tab:ap:memorization}, \appendixshortcut~\ref{ap:add_results}).
\figureshortcut~\ref{fig:exmem_numbers} provides the absolute numbers of ExMem examples, distinguishing primary from secondary ExMem that constitute 41\% and 59\% of all ExMem examples, respectively.
Example~\ref{ex2} demonstrates secondary ExMem: the student has memorized to hallucinate ``AmarillasLatinas.net'' from $\theta_T$'s target when merely shown the source's prefix, but $\theta_T$ has not.

\begin{figure}[!t]
    \centering
    \begin{subfigure}[b]{0.49\columnwidth}
        \includegraphics[height=1.52cm, right]{figures/oschal_increase_barplot.pdf}
        \includegraphics[width=\columnwidth]{figures/oschal_scatter.pdf}
        \caption{OscHal metric}
        \label{fig:mem-metrics-oschal}
    \end{subfigure}
    \begin{subfigure}[b]{0.48\columnwidth}
        \includegraphics[height=1.49cm, right]{figures/nathal_increase_barplot.pdf}
        \includegraphics[width=\columnwidth]{figures/nathal_scatter.pdf}
        \caption{NatHal metric}
        \label{fig:mem-metrics-nathal}
    \end{subfigure}
    \caption{Hallucination metrics for  $\theta_T$, $\theta_S$ and $\theta_B$ and the percentual increase comparing $\theta_S$ to $\theta_B$.}
    \label{fig:hal-metrics}
\end{figure}


\paragraph{SeqKD amplifies hallucinations}
It is not just memorization that is amplified through SeqKD; \figureshortcut~\ref {fig:hal-metrics} indicates that \textbf{hallucinations are also amplified}.
For the oscillatory hallucinations, both $\theta_S$ and $\theta_B$ generate more of these than $\theta_T$, and the student hallucinates more than $\theta_B$, on average (an increase of 31.0\% $\pm$ 25.7, with no increase observed for \textsc{Pl}-\textsc{En}). 
\appendixshortcut~\ref{ap:add_results} presents additional results for monolingual corpora CommonCrawl and Pulpo. Across the board, $\theta_S$ hallucinates most there, too.

For the natural hallucinations, $\theta_S$ and $\theta_B$ hallucinate less than $\theta_T$, but students still hallucinate 13.8$\pm$5.0 more than $\theta_B$.




% Entries for the entire Anthology, followed by custom entries
\bibliography{anthology,custom}
\bibliographystyle{acl_natbib}

\subsection{Lloyd-Max Algorithm}
\label{subsec:Lloyd-Max}
For a given quantization bitwidth $B$ and an operand $\bm{X}$, the Lloyd-Max algorithm finds $2^B$ quantization levels $\{\hat{x}_i\}_{i=1}^{2^B}$ such that quantizing $\bm{X}$ by rounding each scalar in $\bm{X}$ to the nearest quantization level minimizes the quantization MSE. 

The algorithm starts with an initial guess of quantization levels and then iteratively computes quantization thresholds $\{\tau_i\}_{i=1}^{2^B-1}$ and updates quantization levels $\{\hat{x}_i\}_{i=1}^{2^B}$. Specifically, at iteration $n$, thresholds are set to the midpoints of the previous iteration's levels:
\begin{align*}
    \tau_i^{(n)}=\frac{\hat{x}_i^{(n-1)}+\hat{x}_{i+1}^{(n-1)}}2 \text{ for } i=1\ldots 2^B-1
\end{align*}
Subsequently, the quantization levels are re-computed as conditional means of the data regions defined by the new thresholds:
\begin{align*}
    \hat{x}_i^{(n)}=\mathbb{E}\left[ \bm{X} \big| \bm{X}\in [\tau_{i-1}^{(n)},\tau_i^{(n)}] \right] \text{ for } i=1\ldots 2^B
\end{align*}
where to satisfy boundary conditions we have $\tau_0=-\infty$ and $\tau_{2^B}=\infty$. The algorithm iterates the above steps until convergence.

Figure \ref{fig:lm_quant} compares the quantization levels of a $7$-bit floating point (E3M3) quantizer (left) to a $7$-bit Lloyd-Max quantizer (right) when quantizing a layer of weights from the GPT3-126M model at a per-tensor granularity. As shown, the Lloyd-Max quantizer achieves substantially lower quantization MSE. Further, Table \ref{tab:FP7_vs_LM7} shows the superior perplexity achieved by Lloyd-Max quantizers for bitwidths of $7$, $6$ and $5$. The difference between the quantizers is clear at 5 bits, where per-tensor FP quantization incurs a drastic and unacceptable increase in perplexity, while Lloyd-Max quantization incurs a much smaller increase. Nevertheless, we note that even the optimal Lloyd-Max quantizer incurs a notable ($\sim 1.5$) increase in perplexity due to the coarse granularity of quantization. 

\begin{figure}[h]
  \centering
  \includegraphics[width=0.7\linewidth]{sections/figures/LM7_FP7.pdf}
  \caption{\small Quantization levels and the corresponding quantization MSE of Floating Point (left) vs Lloyd-Max (right) Quantizers for a layer of weights in the GPT3-126M model.}
  \label{fig:lm_quant}
\end{figure}

\begin{table}[h]\scriptsize
\begin{center}
\caption{\label{tab:FP7_vs_LM7} \small Comparing perplexity (lower is better) achieved by floating point quantizers and Lloyd-Max quantizers on a GPT3-126M model for the Wikitext-103 dataset.}
\begin{tabular}{c|cc|c}
\hline
 \multirow{2}{*}{\textbf{Bitwidth}} & \multicolumn{2}{|c|}{\textbf{Floating-Point Quantizer}} & \textbf{Lloyd-Max Quantizer} \\
 & Best Format & Wikitext-103 Perplexity & Wikitext-103 Perplexity \\
\hline
7 & E3M3 & 18.32 & 18.27 \\
6 & E3M2 & 19.07 & 18.51 \\
5 & E4M0 & 43.89 & 19.71 \\
\hline
\end{tabular}
\end{center}
\end{table}

\subsection{Proof of Local Optimality of LO-BCQ}
\label{subsec:lobcq_opt_proof}
For a given block $\bm{b}_j$, the quantization MSE during LO-BCQ can be empirically evaluated as $\frac{1}{L_b}\lVert \bm{b}_j- \bm{\hat{b}}_j\rVert^2_2$ where $\bm{\hat{b}}_j$ is computed from equation (\ref{eq:clustered_quantization_definition}) as $C_{f(\bm{b}_j)}(\bm{b}_j)$. Further, for a given block cluster $\mathcal{B}_i$, we compute the quantization MSE as $\frac{1}{|\mathcal{B}_{i}|}\sum_{\bm{b} \in \mathcal{B}_{i}} \frac{1}{L_b}\lVert \bm{b}- C_i^{(n)}(\bm{b})\rVert^2_2$. Therefore, at the end of iteration $n$, we evaluate the overall quantization MSE $J^{(n)}$ for a given operand $\bm{X}$ composed of $N_c$ block clusters as:
\begin{align*}
    \label{eq:mse_iter_n}
    J^{(n)} = \frac{1}{N_c} \sum_{i=1}^{N_c} \frac{1}{|\mathcal{B}_{i}^{(n)}|}\sum_{\bm{v} \in \mathcal{B}_{i}^{(n)}} \frac{1}{L_b}\lVert \bm{b}- B_i^{(n)}(\bm{b})\rVert^2_2
\end{align*}

At the end of iteration $n$, the codebooks are updated from $\mathcal{C}^{(n-1)}$ to $\mathcal{C}^{(n)}$. However, the mapping of a given vector $\bm{b}_j$ to quantizers $\mathcal{C}^{(n)}$ remains as  $f^{(n)}(\bm{b}_j)$. At the next iteration, during the vector clustering step, $f^{(n+1)}(\bm{b}_j)$ finds new mapping of $\bm{b}_j$ to updated codebooks $\mathcal{C}^{(n)}$ such that the quantization MSE over the candidate codebooks is minimized. Therefore, we obtain the following result for $\bm{b}_j$:
\begin{align*}
\frac{1}{L_b}\lVert \bm{b}_j - C_{f^{(n+1)}(\bm{b}_j)}^{(n)}(\bm{b}_j)\rVert^2_2 \le \frac{1}{L_b}\lVert \bm{b}_j - C_{f^{(n)}(\bm{b}_j)}^{(n)}(\bm{b}_j)\rVert^2_2
\end{align*}

That is, quantizing $\bm{b}_j$ at the end of the block clustering step of iteration $n+1$ results in lower quantization MSE compared to quantizing at the end of iteration $n$. Since this is true for all $\bm{b} \in \bm{X}$, we assert the following:
\begin{equation}
\begin{split}
\label{eq:mse_ineq_1}
    \tilde{J}^{(n+1)} &= \frac{1}{N_c} \sum_{i=1}^{N_c} \frac{1}{|\mathcal{B}_{i}^{(n+1)}|}\sum_{\bm{b} \in \mathcal{B}_{i}^{(n+1)}} \frac{1}{L_b}\lVert \bm{b} - C_i^{(n)}(b)\rVert^2_2 \le J^{(n)}
\end{split}
\end{equation}
where $\tilde{J}^{(n+1)}$ is the the quantization MSE after the vector clustering step at iteration $n+1$.

Next, during the codebook update step (\ref{eq:quantizers_update}) at iteration $n+1$, the per-cluster codebooks $\mathcal{C}^{(n)}$ are updated to $\mathcal{C}^{(n+1)}$ by invoking the Lloyd-Max algorithm \citep{Lloyd}. We know that for any given value distribution, the Lloyd-Max algorithm minimizes the quantization MSE. Therefore, for a given vector cluster $\mathcal{B}_i$ we obtain the following result:

\begin{equation}
    \frac{1}{|\mathcal{B}_{i}^{(n+1)}|}\sum_{\bm{b} \in \mathcal{B}_{i}^{(n+1)}} \frac{1}{L_b}\lVert \bm{b}- C_i^{(n+1)}(\bm{b})\rVert^2_2 \le \frac{1}{|\mathcal{B}_{i}^{(n+1)}|}\sum_{\bm{b} \in \mathcal{B}_{i}^{(n+1)}} \frac{1}{L_b}\lVert \bm{b}- C_i^{(n)}(\bm{b})\rVert^2_2
\end{equation}

The above equation states that quantizing the given block cluster $\mathcal{B}_i$ after updating the associated codebook from $C_i^{(n)}$ to $C_i^{(n+1)}$ results in lower quantization MSE. Since this is true for all the block clusters, we derive the following result: 
\begin{equation}
\begin{split}
\label{eq:mse_ineq_2}
     J^{(n+1)} &= \frac{1}{N_c} \sum_{i=1}^{N_c} \frac{1}{|\mathcal{B}_{i}^{(n+1)}|}\sum_{\bm{b} \in \mathcal{B}_{i}^{(n+1)}} \frac{1}{L_b}\lVert \bm{b}- C_i^{(n+1)}(\bm{b})\rVert^2_2  \le \tilde{J}^{(n+1)}   
\end{split}
\end{equation}

Following (\ref{eq:mse_ineq_1}) and (\ref{eq:mse_ineq_2}), we find that the quantization MSE is non-increasing for each iteration, that is, $J^{(1)} \ge J^{(2)} \ge J^{(3)} \ge \ldots \ge J^{(M)}$ where $M$ is the maximum number of iterations. 
%Therefore, we can say that if the algorithm converges, then it must be that it has converged to a local minimum. 
\hfill $\blacksquare$


\begin{figure}
    \begin{center}
    \includegraphics[width=0.5\textwidth]{sections//figures/mse_vs_iter.pdf}
    \end{center}
    \caption{\small NMSE vs iterations during LO-BCQ compared to other block quantization proposals}
    \label{fig:nmse_vs_iter}
\end{figure}

Figure \ref{fig:nmse_vs_iter} shows the empirical convergence of LO-BCQ across several block lengths and number of codebooks. Also, the MSE achieved by LO-BCQ is compared to baselines such as MXFP and VSQ. As shown, LO-BCQ converges to a lower MSE than the baselines. Further, we achieve better convergence for larger number of codebooks ($N_c$) and for a smaller block length ($L_b$), both of which increase the bitwidth of BCQ (see Eq \ref{eq:bitwidth_bcq}).


\subsection{Additional Accuracy Results}
%Table \ref{tab:lobcq_config} lists the various LOBCQ configurations and their corresponding bitwidths.
\begin{table}
\setlength{\tabcolsep}{4.75pt}
\begin{center}
\caption{\label{tab:lobcq_config} Various LO-BCQ configurations and their bitwidths.}
\begin{tabular}{|c||c|c|c|c||c|c||c|} 
\hline
 & \multicolumn{4}{|c||}{$L_b=8$} & \multicolumn{2}{|c||}{$L_b=4$} & $L_b=2$ \\
 \hline
 \backslashbox{$L_A$\kern-1em}{\kern-1em$N_c$} & 2 & 4 & 8 & 16 & 2 & 4 & 2 \\
 \hline
 64 & 4.25 & 4.375 & 4.5 & 4.625 & 4.375 & 4.625 & 4.625\\
 \hline
 32 & 4.375 & 4.5 & 4.625& 4.75 & 4.5 & 4.75 & 4.75 \\
 \hline
 16 & 4.625 & 4.75& 4.875 & 5 & 4.75 & 5 & 5 \\
 \hline
\end{tabular}
\end{center}
\end{table}

%\subsection{Perplexity achieved by various LO-BCQ configurations on Wikitext-103 dataset}

\begin{table} \centering
\begin{tabular}{|c||c|c|c|c||c|c||c|} 
\hline
 $L_b \rightarrow$& \multicolumn{4}{c||}{8} & \multicolumn{2}{c||}{4} & 2\\
 \hline
 \backslashbox{$L_A$\kern-1em}{\kern-1em$N_c$} & 2 & 4 & 8 & 16 & 2 & 4 & 2  \\
 %$N_c \rightarrow$ & 2 & 4 & 8 & 16 & 2 & 4 & 2 \\
 \hline
 \hline
 \multicolumn{8}{c}{GPT3-1.3B (FP32 PPL = 9.98)} \\ 
 \hline
 \hline
 64 & 10.40 & 10.23 & 10.17 & 10.15 &  10.28 & 10.18 & 10.19 \\
 \hline
 32 & 10.25 & 10.20 & 10.15 & 10.12 &  10.23 & 10.17 & 10.17 \\
 \hline
 16 & 10.22 & 10.16 & 10.10 & 10.09 &  10.21 & 10.14 & 10.16 \\
 \hline
  \hline
 \multicolumn{8}{c}{GPT3-8B (FP32 PPL = 7.38)} \\ 
 \hline
 \hline
 64 & 7.61 & 7.52 & 7.48 &  7.47 &  7.55 &  7.49 & 7.50 \\
 \hline
 32 & 7.52 & 7.50 & 7.46 &  7.45 &  7.52 &  7.48 & 7.48  \\
 \hline
 16 & 7.51 & 7.48 & 7.44 &  7.44 &  7.51 &  7.49 & 7.47  \\
 \hline
\end{tabular}
\caption{\label{tab:ppl_gpt3_abalation} Wikitext-103 perplexity across GPT3-1.3B and 8B models.}
\end{table}

\begin{table} \centering
\begin{tabular}{|c||c|c|c|c||} 
\hline
 $L_b \rightarrow$& \multicolumn{4}{c||}{8}\\
 \hline
 \backslashbox{$L_A$\kern-1em}{\kern-1em$N_c$} & 2 & 4 & 8 & 16 \\
 %$N_c \rightarrow$ & 2 & 4 & 8 & 16 & 2 & 4 & 2 \\
 \hline
 \hline
 \multicolumn{5}{|c|}{Llama2-7B (FP32 PPL = 5.06)} \\ 
 \hline
 \hline
 64 & 5.31 & 5.26 & 5.19 & 5.18  \\
 \hline
 32 & 5.23 & 5.25 & 5.18 & 5.15  \\
 \hline
 16 & 5.23 & 5.19 & 5.16 & 5.14  \\
 \hline
 \multicolumn{5}{|c|}{Nemotron4-15B (FP32 PPL = 5.87)} \\ 
 \hline
 \hline
 64  & 6.3 & 6.20 & 6.13 & 6.08  \\
 \hline
 32  & 6.24 & 6.12 & 6.07 & 6.03  \\
 \hline
 16  & 6.12 & 6.14 & 6.04 & 6.02  \\
 \hline
 \multicolumn{5}{|c|}{Nemotron4-340B (FP32 PPL = 3.48)} \\ 
 \hline
 \hline
 64 & 3.67 & 3.62 & 3.60 & 3.59 \\
 \hline
 32 & 3.63 & 3.61 & 3.59 & 3.56 \\
 \hline
 16 & 3.61 & 3.58 & 3.57 & 3.55 \\
 \hline
\end{tabular}
\caption{\label{tab:ppl_llama7B_nemo15B} Wikitext-103 perplexity compared to FP32 baseline in Llama2-7B and Nemotron4-15B, 340B models}
\end{table}

%\subsection{Perplexity achieved by various LO-BCQ configurations on MMLU dataset}


\begin{table} \centering
\begin{tabular}{|c||c|c|c|c||c|c|c|c|} 
\hline
 $L_b \rightarrow$& \multicolumn{4}{c||}{8} & \multicolumn{4}{c||}{8}\\
 \hline
 \backslashbox{$L_A$\kern-1em}{\kern-1em$N_c$} & 2 & 4 & 8 & 16 & 2 & 4 & 8 & 16  \\
 %$N_c \rightarrow$ & 2 & 4 & 8 & 16 & 2 & 4 & 2 \\
 \hline
 \hline
 \multicolumn{5}{|c|}{Llama2-7B (FP32 Accuracy = 45.8\%)} & \multicolumn{4}{|c|}{Llama2-70B (FP32 Accuracy = 69.12\%)} \\ 
 \hline
 \hline
 64 & 43.9 & 43.4 & 43.9 & 44.9 & 68.07 & 68.27 & 68.17 & 68.75 \\
 \hline
 32 & 44.5 & 43.8 & 44.9 & 44.5 & 68.37 & 68.51 & 68.35 & 68.27  \\
 \hline
 16 & 43.9 & 42.7 & 44.9 & 45 & 68.12 & 68.77 & 68.31 & 68.59  \\
 \hline
 \hline
 \multicolumn{5}{|c|}{GPT3-22B (FP32 Accuracy = 38.75\%)} & \multicolumn{4}{|c|}{Nemotron4-15B (FP32 Accuracy = 64.3\%)} \\ 
 \hline
 \hline
 64 & 36.71 & 38.85 & 38.13 & 38.92 & 63.17 & 62.36 & 63.72 & 64.09 \\
 \hline
 32 & 37.95 & 38.69 & 39.45 & 38.34 & 64.05 & 62.30 & 63.8 & 64.33  \\
 \hline
 16 & 38.88 & 38.80 & 38.31 & 38.92 & 63.22 & 63.51 & 63.93 & 64.43  \\
 \hline
\end{tabular}
\caption{\label{tab:mmlu_abalation} Accuracy on MMLU dataset across GPT3-22B, Llama2-7B, 70B and Nemotron4-15B models.}
\end{table}


%\subsection{Perplexity achieved by various LO-BCQ configurations on LM evaluation harness}

\begin{table} \centering
\begin{tabular}{|c||c|c|c|c||c|c|c|c|} 
\hline
 $L_b \rightarrow$& \multicolumn{4}{c||}{8} & \multicolumn{4}{c||}{8}\\
 \hline
 \backslashbox{$L_A$\kern-1em}{\kern-1em$N_c$} & 2 & 4 & 8 & 16 & 2 & 4 & 8 & 16  \\
 %$N_c \rightarrow$ & 2 & 4 & 8 & 16 & 2 & 4 & 2 \\
 \hline
 \hline
 \multicolumn{5}{|c|}{Race (FP32 Accuracy = 37.51\%)} & \multicolumn{4}{|c|}{Boolq (FP32 Accuracy = 64.62\%)} \\ 
 \hline
 \hline
 64 & 36.94 & 37.13 & 36.27 & 37.13 & 63.73 & 62.26 & 63.49 & 63.36 \\
 \hline
 32 & 37.03 & 36.36 & 36.08 & 37.03 & 62.54 & 63.51 & 63.49 & 63.55  \\
 \hline
 16 & 37.03 & 37.03 & 36.46 & 37.03 & 61.1 & 63.79 & 63.58 & 63.33  \\
 \hline
 \hline
 \multicolumn{5}{|c|}{Winogrande (FP32 Accuracy = 58.01\%)} & \multicolumn{4}{|c|}{Piqa (FP32 Accuracy = 74.21\%)} \\ 
 \hline
 \hline
 64 & 58.17 & 57.22 & 57.85 & 58.33 & 73.01 & 73.07 & 73.07 & 72.80 \\
 \hline
 32 & 59.12 & 58.09 & 57.85 & 58.41 & 73.01 & 73.94 & 72.74 & 73.18  \\
 \hline
 16 & 57.93 & 58.88 & 57.93 & 58.56 & 73.94 & 72.80 & 73.01 & 73.94  \\
 \hline
\end{tabular}
\caption{\label{tab:mmlu_abalation} Accuracy on LM evaluation harness tasks on GPT3-1.3B model.}
\end{table}

\begin{table} \centering
\begin{tabular}{|c||c|c|c|c||c|c|c|c|} 
\hline
 $L_b \rightarrow$& \multicolumn{4}{c||}{8} & \multicolumn{4}{c||}{8}\\
 \hline
 \backslashbox{$L_A$\kern-1em}{\kern-1em$N_c$} & 2 & 4 & 8 & 16 & 2 & 4 & 8 & 16  \\
 %$N_c \rightarrow$ & 2 & 4 & 8 & 16 & 2 & 4 & 2 \\
 \hline
 \hline
 \multicolumn{5}{|c|}{Race (FP32 Accuracy = 41.34\%)} & \multicolumn{4}{|c|}{Boolq (FP32 Accuracy = 68.32\%)} \\ 
 \hline
 \hline
 64 & 40.48 & 40.10 & 39.43 & 39.90 & 69.20 & 68.41 & 69.45 & 68.56 \\
 \hline
 32 & 39.52 & 39.52 & 40.77 & 39.62 & 68.32 & 67.43 & 68.17 & 69.30  \\
 \hline
 16 & 39.81 & 39.71 & 39.90 & 40.38 & 68.10 & 66.33 & 69.51 & 69.42  \\
 \hline
 \hline
 \multicolumn{5}{|c|}{Winogrande (FP32 Accuracy = 67.88\%)} & \multicolumn{4}{|c|}{Piqa (FP32 Accuracy = 78.78\%)} \\ 
 \hline
 \hline
 64 & 66.85 & 66.61 & 67.72 & 67.88 & 77.31 & 77.42 & 77.75 & 77.64 \\
 \hline
 32 & 67.25 & 67.72 & 67.72 & 67.00 & 77.31 & 77.04 & 77.80 & 77.37  \\
 \hline
 16 & 68.11 & 68.90 & 67.88 & 67.48 & 77.37 & 78.13 & 78.13 & 77.69  \\
 \hline
\end{tabular}
\caption{\label{tab:mmlu_abalation} Accuracy on LM evaluation harness tasks on GPT3-8B model.}
\end{table}

\begin{table} \centering
\begin{tabular}{|c||c|c|c|c||c|c|c|c|} 
\hline
 $L_b \rightarrow$& \multicolumn{4}{c||}{8} & \multicolumn{4}{c||}{8}\\
 \hline
 \backslashbox{$L_A$\kern-1em}{\kern-1em$N_c$} & 2 & 4 & 8 & 16 & 2 & 4 & 8 & 16  \\
 %$N_c \rightarrow$ & 2 & 4 & 8 & 16 & 2 & 4 & 2 \\
 \hline
 \hline
 \multicolumn{5}{|c|}{Race (FP32 Accuracy = 40.67\%)} & \multicolumn{4}{|c|}{Boolq (FP32 Accuracy = 76.54\%)} \\ 
 \hline
 \hline
 64 & 40.48 & 40.10 & 39.43 & 39.90 & 75.41 & 75.11 & 77.09 & 75.66 \\
 \hline
 32 & 39.52 & 39.52 & 40.77 & 39.62 & 76.02 & 76.02 & 75.96 & 75.35  \\
 \hline
 16 & 39.81 & 39.71 & 39.90 & 40.38 & 75.05 & 73.82 & 75.72 & 76.09  \\
 \hline
 \hline
 \multicolumn{5}{|c|}{Winogrande (FP32 Accuracy = 70.64\%)} & \multicolumn{4}{|c|}{Piqa (FP32 Accuracy = 79.16\%)} \\ 
 \hline
 \hline
 64 & 69.14 & 70.17 & 70.17 & 70.56 & 78.24 & 79.00 & 78.62 & 78.73 \\
 \hline
 32 & 70.96 & 69.69 & 71.27 & 69.30 & 78.56 & 79.49 & 79.16 & 78.89  \\
 \hline
 16 & 71.03 & 69.53 & 69.69 & 70.40 & 78.13 & 79.16 & 79.00 & 79.00  \\
 \hline
\end{tabular}
\caption{\label{tab:mmlu_abalation} Accuracy on LM evaluation harness tasks on GPT3-22B model.}
\end{table}

\begin{table} \centering
\begin{tabular}{|c||c|c|c|c||c|c|c|c|} 
\hline
 $L_b \rightarrow$& \multicolumn{4}{c||}{8} & \multicolumn{4}{c||}{8}\\
 \hline
 \backslashbox{$L_A$\kern-1em}{\kern-1em$N_c$} & 2 & 4 & 8 & 16 & 2 & 4 & 8 & 16  \\
 %$N_c \rightarrow$ & 2 & 4 & 8 & 16 & 2 & 4 & 2 \\
 \hline
 \hline
 \multicolumn{5}{|c|}{Race (FP32 Accuracy = 44.4\%)} & \multicolumn{4}{|c|}{Boolq (FP32 Accuracy = 79.29\%)} \\ 
 \hline
 \hline
 64 & 42.49 & 42.51 & 42.58 & 43.45 & 77.58 & 77.37 & 77.43 & 78.1 \\
 \hline
 32 & 43.35 & 42.49 & 43.64 & 43.73 & 77.86 & 75.32 & 77.28 & 77.86  \\
 \hline
 16 & 44.21 & 44.21 & 43.64 & 42.97 & 78.65 & 77 & 76.94 & 77.98  \\
 \hline
 \hline
 \multicolumn{5}{|c|}{Winogrande (FP32 Accuracy = 69.38\%)} & \multicolumn{4}{|c|}{Piqa (FP32 Accuracy = 78.07\%)} \\ 
 \hline
 \hline
 64 & 68.9 & 68.43 & 69.77 & 68.19 & 77.09 & 76.82 & 77.09 & 77.86 \\
 \hline
 32 & 69.38 & 68.51 & 68.82 & 68.90 & 78.07 & 76.71 & 78.07 & 77.86  \\
 \hline
 16 & 69.53 & 67.09 & 69.38 & 68.90 & 77.37 & 77.8 & 77.91 & 77.69  \\
 \hline
\end{tabular}
\caption{\label{tab:mmlu_abalation} Accuracy on LM evaluation harness tasks on Llama2-7B model.}
\end{table}

\begin{table} \centering
\begin{tabular}{|c||c|c|c|c||c|c|c|c|} 
\hline
 $L_b \rightarrow$& \multicolumn{4}{c||}{8} & \multicolumn{4}{c||}{8}\\
 \hline
 \backslashbox{$L_A$\kern-1em}{\kern-1em$N_c$} & 2 & 4 & 8 & 16 & 2 & 4 & 8 & 16  \\
 %$N_c \rightarrow$ & 2 & 4 & 8 & 16 & 2 & 4 & 2 \\
 \hline
 \hline
 \multicolumn{5}{|c|}{Race (FP32 Accuracy = 48.8\%)} & \multicolumn{4}{|c|}{Boolq (FP32 Accuracy = 85.23\%)} \\ 
 \hline
 \hline
 64 & 49.00 & 49.00 & 49.28 & 48.71 & 82.82 & 84.28 & 84.03 & 84.25 \\
 \hline
 32 & 49.57 & 48.52 & 48.33 & 49.28 & 83.85 & 84.46 & 84.31 & 84.93  \\
 \hline
 16 & 49.85 & 49.09 & 49.28 & 48.99 & 85.11 & 84.46 & 84.61 & 83.94  \\
 \hline
 \hline
 \multicolumn{5}{|c|}{Winogrande (FP32 Accuracy = 79.95\%)} & \multicolumn{4}{|c|}{Piqa (FP32 Accuracy = 81.56\%)} \\ 
 \hline
 \hline
 64 & 78.77 & 78.45 & 78.37 & 79.16 & 81.45 & 80.69 & 81.45 & 81.5 \\
 \hline
 32 & 78.45 & 79.01 & 78.69 & 80.66 & 81.56 & 80.58 & 81.18 & 81.34  \\
 \hline
 16 & 79.95 & 79.56 & 79.79 & 79.72 & 81.28 & 81.66 & 81.28 & 80.96  \\
 \hline
\end{tabular}
\caption{\label{tab:mmlu_abalation} Accuracy on LM evaluation harness tasks on Llama2-70B model.}
\end{table}

%\section{MSE Studies}
%\textcolor{red}{TODO}


\subsection{Number Formats and Quantization Method}
\label{subsec:numFormats_quantMethod}
\subsubsection{Integer Format}
An $n$-bit signed integer (INT) is typically represented with a 2s-complement format \citep{yao2022zeroquant,xiao2023smoothquant,dai2021vsq}, where the most significant bit denotes the sign.

\subsubsection{Floating Point Format}
An $n$-bit signed floating point (FP) number $x$ comprises of a 1-bit sign ($x_{\mathrm{sign}}$), $B_m$-bit mantissa ($x_{\mathrm{mant}}$) and $B_e$-bit exponent ($x_{\mathrm{exp}}$) such that $B_m+B_e=n-1$. The associated constant exponent bias ($E_{\mathrm{bias}}$) is computed as $(2^{{B_e}-1}-1)$. We denote this format as $E_{B_e}M_{B_m}$.  

\subsubsection{Quantization Scheme}
\label{subsec:quant_method}
A quantization scheme dictates how a given unquantized tensor is converted to its quantized representation. We consider FP formats for the purpose of illustration. Given an unquantized tensor $\bm{X}$ and an FP format $E_{B_e}M_{B_m}$, we first, we compute the quantization scale factor $s_X$ that maps the maximum absolute value of $\bm{X}$ to the maximum quantization level of the $E_{B_e}M_{B_m}$ format as follows:
\begin{align}
\label{eq:sf}
    s_X = \frac{\mathrm{max}(|\bm{X}|)}{\mathrm{max}(E_{B_e}M_{B_m})}
\end{align}
In the above equation, $|\cdot|$ denotes the absolute value function.

Next, we scale $\bm{X}$ by $s_X$ and quantize it to $\hat{\bm{X}}$ by rounding it to the nearest quantization level of $E_{B_e}M_{B_m}$ as:

\begin{align}
\label{eq:tensor_quant}
    \hat{\bm{X}} = \text{round-to-nearest}\left(\frac{\bm{X}}{s_X}, E_{B_e}M_{B_m}\right)
\end{align}

We perform dynamic max-scaled quantization \citep{wu2020integer}, where the scale factor $s$ for activations is dynamically computed during runtime.

\subsection{Vector Scaled Quantization}
\begin{wrapfigure}{r}{0.35\linewidth}
  \centering
  \includegraphics[width=\linewidth]{sections/figures/vsquant.jpg}
  \caption{\small Vectorwise decomposition for per-vector scaled quantization (VSQ \citep{dai2021vsq}).}
  \label{fig:vsquant}
\end{wrapfigure}
During VSQ \citep{dai2021vsq}, the operand tensors are decomposed into 1D vectors in a hardware friendly manner as shown in Figure \ref{fig:vsquant}. Since the decomposed tensors are used as operands in matrix multiplications during inference, it is beneficial to perform this decomposition along the reduction dimension of the multiplication. The vectorwise quantization is performed similar to tensorwise quantization described in Equations \ref{eq:sf} and \ref{eq:tensor_quant}, where a scale factor $s_v$ is required for each vector $\bm{v}$ that maps the maximum absolute value of that vector to the maximum quantization level. While smaller vector lengths can lead to larger accuracy gains, the associated memory and computational overheads due to the per-vector scale factors increases. To alleviate these overheads, VSQ \citep{dai2021vsq} proposed a second level quantization of the per-vector scale factors to unsigned integers, while MX \citep{rouhani2023shared} quantizes them to integer powers of 2 (denoted as $2^{INT}$).

\subsubsection{MX Format}
The MX format proposed in \citep{rouhani2023microscaling} introduces the concept of sub-block shifting. For every two scalar elements of $b$-bits each, there is a shared exponent bit. The value of this exponent bit is determined through an empirical analysis that targets minimizing quantization MSE. We note that the FP format $E_{1}M_{b}$ is strictly better than MX from an accuracy perspective since it allocates a dedicated exponent bit to each scalar as opposed to sharing it across two scalars. Therefore, we conservatively bound the accuracy of a $b+2$-bit signed MX format with that of a $E_{1}M_{b}$ format in our comparisons. For instance, we use E1M2 format as a proxy for MX4.

\begin{figure}
    \centering
    \includegraphics[width=1\linewidth]{sections//figures/BlockFormats.pdf}
    \caption{\small Comparing LO-BCQ to MX format.}
    \label{fig:block_formats}
\end{figure}

Figure \ref{fig:block_formats} compares our $4$-bit LO-BCQ block format to MX \citep{rouhani2023microscaling}. As shown, both LO-BCQ and MX decompose a given operand tensor into block arrays and each block array into blocks. Similar to MX, we find that per-block quantization ($L_b < L_A$) leads to better accuracy due to increased flexibility. While MX achieves this through per-block $1$-bit micro-scales, we associate a dedicated codebook to each block through a per-block codebook selector. Further, MX quantizes the per-block array scale-factor to E8M0 format without per-tensor scaling. In contrast during LO-BCQ, we find that per-tensor scaling combined with quantization of per-block array scale-factor to E4M3 format results in superior inference accuracy across models. 


\end{document}
