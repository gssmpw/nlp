\section{Memorization inheritance}
\label{sec:memorization_inheritance}

\subsection{Experimental setup}
\label{sec:experimental_setup}
In SeqKD, teacher $\theta_T$ is trained on a corpus with source sequences $\mathcal{S}_C$ and targets $\mathcal{T}_C$, and generates translations of $\mathcal{S}_C$ with beam size $k$. The source sequences and the teacher-generated translations ($\mathcal{S}_C$ and $\mathcal{T}_T$) form the training corpus for student $\theta_S$.
We use data from the WMT20 corpus \citep{barrault-etal-2020-findings}, for five language pairs: \textsc{De}-\textsc{En} and \textsc{En}-\textsc{De} (48M), \textsc{Pl}-\textsc{En} and \textsc{En}-\textsc{Pl} (12M), and \textsc{Fr}-\textsc{De} (14M). \appendixshortcut~\ref{ap:data} provides more detail about the WMT corpora, and the validation and test data.

For all languages, a Transformer-large teacher trains for 300k steps on $\mathcal{S}_C$ and $\mathcal{T}_C$, followed by training a Transformer-base student for 100k steps on $\mathcal{S}_C$ and $\mathcal{T}_T$ ($k{=}1$).\footnote{Training duration set to equal the setup of \citet{vaswani2017attention}. We train using the Marian toolkit \citep{junczys2018marian}. For the full training setup see \appendixshortcut~\ref{ap:data} and our \href{https://github.com/vyraun/memseqkd}{\texttt{codebase}}.} 
We also train base models directly on $\mathcal{S}_C$ and $\mathcal{T}_C$, to have a baseline ($\theta_{B}$) for what the student model would have memorized when exposed directly to WMT20 data.
In \appendixshortcut~\ref{ap:hyperparams}, we furthermore experiment with varying beam size $k$ and the student's model size.

\paragraph{Model quality metrics}
We first examine the models' quality, to ascertain the SeqKD framework works as intended. 
We report the following reference-based metrics for translations generated with beam size five: \textbf{BLEU} and \textbf{chrF}, %to quantify $n$-gram and character-based overlap between translations and their references,
Translation Error Rate (\textbf{TER}), 
and the \textbf{Comet-20} and \textbf{Comet-22} metrics that use neural methods for translation quality estimation \citep{rei2020comet, rei2022comet}.
We supplement this with the reference-free metrics of \textbf{Comet-QE-20}, \textbf{Comet-QE-22}.

All metrics will be applied to WMT test data, and the reference-free methods are furthermore applied to translations of monolingual data: CommonCrawl data provided by \citet{barrault-etal-2020-findings}, and data from the Pulpo poetry corpus \citep{de2023alberti}, to examine out-of-domain performance.

\begin{figure}
    \centering
    \includegraphics[width=\columnwidth]{figures/model_quality.pdf}
    \caption{Performance of teacher, student and baseline models for four model quality metrics.}
    \label{fig:model-quality}
    \vspace{-0.2cm}
\end{figure}

\paragraph{Memorization metrics}
We quantify memorization by comparing greedily translated $\mathcal{S}_C$ to $\mathcal{T}_C$ for all models, and to $\mathcal{T}_T$, for $\theta_S$. We measure the \textbf{replication} (exact match) rate, and the \textbf{extractive memorization} \citep[ExMem,][]{raunak2022finding} rate. ExMem finds examples for which models memorized to emit the target after seeing at most $75$\% of the source, e.g., see Example~\ref{ex1}.
The ExMem rate is the percentage of extractively memorized examples out of the replicated examples. 

We also quantify the hallucination rate. \citet{raunak2021curious}---who relate memorization in NMT to hallucinations---suggested this could influence SeqKD students, but did not actually evaluate that. We firstly measure the detached \textbf{natural hallucinations} (NatHal) rate: the percentage of source sequences that map to a translation that is repeated in the model's translations for at least five times.
Secondly, we compute the \textbf{oscillatory hallucinations} (OscHal) rate: the percentage of translations with bigrams repeated at least 10 times (for which the source did not contain such a repeated bigram).\footnote{To improve the precision of the memorization metrics, some training examples are excluded (see \appendixshortcut~\ref{ap:add_results}).} 

\subsection{Results}
\paragraph{General model quality}
\figureshortcut~\ref{fig:model-quality} and \appendixshortcut~\ref{ap:add_results} provide performance differences for $\theta_T$, $\theta_S$ and $\theta_B$.
Overall, $\theta_T$ outperforms $\theta_S$, and $\theta_S$ outperforms $\theta_B$. $\theta_S$ and $\theta_B$ merely differ in the training targets, which demonstrates that our SeqKD pipeline works as intended.
The ordering of models in terms of their quality also holds for CommonCrawl and Pulpo data (see Table~\ref{tab:ap:model_quality}, \appendixshortcut~\ref{ap:add_results}).

\begin{figure}[!t]
    \centering
    \begin{subfigure}[b]{0.49\columnwidth}
        \includegraphics[height=1.43cm, right]{figures/replication_increase_barplot.pdf}
        \includegraphics[width=0.95\textwidth, right]{figures/replication_scatter.pdf}
        \caption{Replication metric}
        \label{fig:mem-metrics-replication}
    \end{subfigure}
    \begin{subfigure}[b]{0.49\columnwidth}
        \includegraphics[height=1.47cm, right]{figures/exmem_increase_barplot.pdf}
        \includegraphics[width=0.95\textwidth, right]{figures/exmem_scatter.pdf}
        \caption{ExMem rate}
        \label{fig:mem-metrics-exmem}
    \end{subfigure}
    \begin{subfigure}[b]{\columnwidth}\centering
        \includegraphics[width=0.8\textwidth]{figures/exmem_barplot_num.pdf}
        \caption{Number of ExMem examples}
        \label{fig:exmem_numbers}
    \end{subfigure}
    \caption{Memorization metrics for $\theta_T$, $\theta_S$ and $\theta_B$ and the percentual increase comparing $\theta_S$ to $\theta_B$.}
    \label{fig:mem-metrics}
\end{figure}


\paragraph{SeqKD facilitates memorization}
\figureshortcut~\ref{fig:mem-metrics} summarizes the memorization results.
If we first look at the replication rate, $\theta_S$ replicates less from WMT20 than $\theta_T$ but \textbf{more than $\theta_B$}: students' replication rate with respect to $\mathcal{T}_C$ is 3.4\%($\pm$0.9) higher than for $\theta_B$.
Students also replicate original material from $\theta_T$: the overall student replication rate for $\mathcal{T}_T$ is 35.3\%($\pm2.7$) (see Table~\ref{tab:ap:memorization}, \appendixshortcut~\ref{ap:add_results}).

For the ExMem rate with respect to $\mathcal{T}_C$ (\figureshortcut~\ref{fig:mem-metrics-exmem}), a similar pattern emerges, but with a starker difference between $\theta_S$ and $\theta_B$: students extractively memorize less from $\mathcal{T}_C$ compared to $\theta_T$, but \textbf{more compared to $\theta_B$}, with a mean increase of 57.0\%($\pm15.4$).
This is quite surprising; note that by definition, the rate reported here expresses how many of the replicated examples are extractively memorized. Students only observed 18.4\% (on average) of $\mathcal{T}_C$ through the SeqKD pipeline (the portion that $\theta_T$ replicated) and yet within that smaller pool they still memorized more than $\theta_B$, that was exposed to 100\% of the corpus.
Not only have students extractively memorized WMT20 examples (`primary ExMem'), they also show ExMem with respect to $\theta_T$ (`secondary ExMem', quantified in Table~\ref{tab:ap:memorization}, \appendixshortcut~\ref{ap:add_results}).
\figureshortcut~\ref{fig:exmem_numbers} provides the absolute numbers of ExMem examples, distinguishing primary from secondary ExMem that constitute 41\% and 59\% of all ExMem examples, respectively.
Example~\ref{ex2} demonstrates secondary ExMem: the student has memorized to hallucinate ``AmarillasLatinas.net'' from $\theta_T$'s target when merely shown the source's prefix, but $\theta_T$ has not.

\begin{figure}[!t]
    \centering
    \begin{subfigure}[b]{0.49\columnwidth}
        \includegraphics[height=1.52cm, right]{figures/oschal_increase_barplot.pdf}
        \includegraphics[width=\columnwidth]{figures/oschal_scatter.pdf}
        \caption{OscHal metric}
        \label{fig:mem-metrics-oschal}
    \end{subfigure}
    \begin{subfigure}[b]{0.48\columnwidth}
        \includegraphics[height=1.49cm, right]{figures/nathal_increase_barplot.pdf}
        \includegraphics[width=\columnwidth]{figures/nathal_scatter.pdf}
        \caption{NatHal metric}
        \label{fig:mem-metrics-nathal}
    \end{subfigure}
    \caption{Hallucination metrics for  $\theta_T$, $\theta_S$ and $\theta_B$ and the percentual increase comparing $\theta_S$ to $\theta_B$.}
    \label{fig:hal-metrics}
\end{figure}


\paragraph{SeqKD amplifies hallucinations}
It is not just memorization that is amplified through SeqKD; \figureshortcut~\ref {fig:hal-metrics} indicates that \textbf{hallucinations are also amplified}.
For the oscillatory hallucinations, both $\theta_S$ and $\theta_B$ generate more of these than $\theta_T$, and the student hallucinates more than $\theta_B$, on average (an increase of 31.0\% $\pm$ 25.7, with no increase observed for \textsc{Pl}-\textsc{En}). 
\appendixshortcut~\ref{ap:add_results} presents additional results for monolingual corpora CommonCrawl and Pulpo. Across the board, $\theta_S$ hallucinates most there, too.

For the natural hallucinations, $\theta_S$ and $\theta_B$ hallucinate less than $\theta_T$, but students still hallucinate 13.8$\pm$5.0 more than $\theta_B$.

