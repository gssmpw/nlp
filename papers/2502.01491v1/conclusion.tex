\section{Conclusion}
\label{sec:conclusion}

SeqKD is popular for effectively training smaller NMT models, but we show that it also introduces issues like worsened ExMem and hallucinations in $\theta_S$ compared to $\theta_B$.
At the same time, the subgroup analyses showed that teachers' CM examples are not necessarily replicated by $\theta_S$, and that students exhibit amplified denoising on low-quality examples.
This highlights a paradox: through SeqKD, students memorize \textit{more} about the corpus than $\theta_B$, yet also outperform $\theta_B$ and $\theta_T$ on data where they \textit{did not} memorize.
Student improvements thus happen both by mimicking $\theta_T$, and by deviating from $\theta_T$.
Future work could suppress memorization during SeqKD, refine Adaptive-SeqKD, and adjust hyperparameters\footnote{Increasing $k$ reduces the student's OscHal rate to below that of $\theta_B$, in particular, with Adaptive-SeqKD yielding further improvements (see \appendixshortcut~\ref{ap:hyperparams}).}
to create more robust SeqKD pipelines.
We advise caution with SeqKD: students may inherit not only the teacher's strengths but also its failures, requiring careful monitoring beyond average-case performance.