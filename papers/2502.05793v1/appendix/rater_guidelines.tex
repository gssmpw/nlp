\section{RefNLI Annotation Guidelines}
\label{appendix:rater-guidelines}

The expert raters for RefNLI were presented with examples consisting of a premise and a hypothesis. For each example, they were given  instructions as follows.

You are to assign one of 4 labels to the example:

\begin{enumerate}
    \item[(a)] \textbf{ambiguous reference}: If the premise contains ambiguous reference, and it’s possible that with resolved reference, premise would actually support/contradict the claim. 
    \item[(n)] \textbf{neutral}: If the premise can’t support or contradict the claim in any possible way. e.g. No matter how you resolve the reference, the premise would still be irrelevant to the claim.
    \item[(c)] \textbf{contradiction}: If the claim is most likely false given the premise.
    \item[(e)] \textbf{entailment}: If premise fully supports the claim.
\end{enumerate}

If you find tricky cases, put yourself in the following scenario: Suppose an LLM generates the claim, you want to decide if we should, given the evidence, tell the user that that this claim is true, tell the user that it's false, or neither. 

The distinction between neutral and ambiguous is going to be difficult sometimes. See examples below for what we are after. If it’s truly unclear – feel free to skip the example.

\subsection*{Specific Guidelines}

\begin{enumerate}
    \item \textbf{Skip unclear claims or premises}: If you think the claim is difficult to understand, or there is too much ambiguity, skip the claim entirely.

    \item \textbf{Don’t label the claim by its truth value in the world}: If a claim says “The sky is blue”, and the premise says something completely different, label it as neutral. Don’t label such cases as entailment based on \textbf{just} your world knowledge.

    \item \textbf{World Knowledge is permitted}: You can assume commonly accepted world knowledge when interpreting the premise, e.g., basic geography and other commonsense knowledge are allowed. If needed, a web search is allowed when making the judgements. However, don’t make too many inferences.

    \item \textbf{Temporal considerations}: Ignore tense (e.g., past or present) in both the premise and claims. If the premise clearly indicates a time of an event, but the claim doesn’t, assume that the claim is uttered right after the event.

    \item \textbf{Personal surnames}: If only the surname of a person is mentioned in the premise, and there’s not enough evidence for in the premise for you to determine the last name is referring to the same entity as in the hypothesis, mark the example as ``ambiguous''

    \item \textbf{Neutral vs. Ambiguous Reference}: The distinction between the two can be difficult sometimes. The general rule is: if the premise can’t seem to support the claim no matter how you interpret the premise, then it's neutral.
\end{enumerate}

\subsection*{Some examples given in the instructions}

\newcommand{\instex}[3]{\noindent\textbf{Premise}: #1

\noindent\textbf{Hypothesis}: #2

\noindent\textbf{Label}: #3\vspace{1ex}}

\instex{Wales has a large region rich in coal deposits.}{The Ural Mountains contain about 48 species of economically valuable ores and economically valuable minerals.}{N; even if we didn't know whether the Ural Mountains are in Wales, the premise doesn’t mention anything about coal deposits, so there’s no way that the premise can support/contradict the claim.}

\instex{Wales has a large region rich in coal deposits.}{Famous for its coal, Newcastle is the largest coal exporting harbour in the world, exporting 159.9 million tonnes of coal in 2017.}{A: The prominent Newcastle is in New South Wales, Australia, but there happens to also be a small town named Newcastle in Wales}

\instex{The Predator made more than \$97 million worldwide.}{Up to March 2011, The Predator’s worldwide gross has reached \$172,543,519, making it the highest-grossing film in the franchise.}{E; if the premise mentions a time, and there’s no clear temporal marker in the claim – assume that the claim is made in the similar time frame as the premise.}

\instex{The Hunchback of Notre Dame is a Disney media franchise, commencing in 1996 with the release of "The Hunchback of Notre Dame".}{The Hunchback of Notre Dame has only ever been based off of a poem.}{Skip, since it's unclear in the hypothesis what ``based off of a poem'' means}
