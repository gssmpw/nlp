% This must be in the first 5 lines to tell arXiv to use pdfLaTeX, which is strongly recommended.
\pdfoutput=1
% In particular, the hyperref package requires pdfLaTeX in order to break URLs across lines.

\documentclass[11pt]{article}

% Remove the "review" option to generate the final version.
\usepackage{acl}
% \usepackage{acl}

% Standard package includes
\usepackage{times}
\usepackage{latexsym}

% For proper rendering and hyphenation of words containing Latin characters (including in bib files)
\usepackage[T1]{fontenc}
% For Vietnamese characters
% \usepackage[T5]{fontenc}
% See https://www.latex-project.org/help/documentation/encguide.pdf for other character sets

% This assumes your files are encoded as UTF8
\usepackage[utf8]{inputenc}

% This is not strictly necessary, and may be commented out,
% but it will improve the layout of the manuscript,
% and will typically save some space.
\usepackage{microtype}

\usepackage{makecell}
\usepackage{times}
\usepackage{latexsym}
\usepackage{tabularx}
\usepackage{array}
\usepackage{multirow}
\usepackage{siunitx}
\sisetup{round-mode = figures, round-precision = 4}
\usepackage{booktabs}
\usepackage[pdftex]{graphicx}
\usepackage{enumitem}
\usepackage{xspace}
\usepackage{caption}

% more packages
\usepackage{url}            % simple URL typesetting
\usepackage{amsmath,amsthm,amsfonts,amssymb,bm,stmaryrd}       % blackboard math symbols
\usepackage{nicefrac}       % compact symbols for 1/2, etc.
\usepackage{enumitem}
\usepackage[noend]{algpseudocode}
\usepackage{algorithm}
\usepackage{mathtools}
\usepackage{nccmath}
\usepackage{multirow}
\usepackage{bigdelim}
\usepackage{color, colortbl}
%\usepackage{scalerel}
\usepackage{xcolor}		% make links dark blue
\definecolor{darkblue}{rgb}{0, 0, 0.5}
\hypersetup{colorlinks=true, citecolor=darkblue, linkcolor=darkblue, urlcolor=darkblue}
\usepackage{booktabs}
\usepackage{dcolumn}
\newcolumntype{d}{D{.}{.}{-1}}
\newcolumntype{z}{D{(}{\ (}{1.1}}
\usepackage{graphicx}
\usepackage{titlesec}
\usepackage{hyperref}
\usepackage{url}
\usepackage{framed}
\usepackage{tcolorbox}
% \usepackage{subfigure}
\usepackage{subcaption}
\usepackage{graphicx}
\usepackage[normalem]{ulem}
\useunder{\uline}{\ul}{}
\usepackage{fontawesome5}
\usepackage{calc}

\usepackage{cleveref}

\crefformat{section}{\textcolor{blue}{\S#2#1#3}} % see manual of cleveref, section 8.2.1
\crefformat{subsection}{\textcolor{blue}{\S#2#1#3}}
\crefformat{subsubsection}{\textcolor{blue}{\S#2#1#3}}
\crefformat{equation}{\textcolor{orange}{Eq.~(#1)}}

% \usepackage[notes=true, done=false, later=false, ]{dtrt} % for in-line comments. Remove before submission.

% \pagenumbering{gobble}
\newcommand{\thought}[1]{{\color[rgb]{0.2,0.39,0.66}(#1)}}
\newcommand{\todo}[1]{{\color[rgb]{1.0,0.0,0.0}(#1)}}
\newcommand{\hsh}[1]{{\color{green!50!black} Henrik: #1}}
\newcommand{\st}[1]{{\color{red!50!black} Sebastian: #1}}

\newcommand{\ulm}[1]{_{\scaleto{\mathrm{#1}}{3pt}}}
\newcommand\at[2]{\left.#1\right|_{#2}}











\newtheorem{assumption}{Assumption}

\DeclareMathOperator*{\argmax}{arg\,max}
\DeclareMathOperator*{\argmin}{arg\,min}

\newcommand{\swname}[1]{\texttt{#1}}
\newcommand{\ie}{i\/.\/e\/.,\/~}
\newcommand{\eg}{e\/.\/g\/.,\/~}
\newcommand{\cf}{cf\/.\/~}

\newcommand{\fig}{Fig\/.\/~}
\newcommand{\defn}{Def\/.\/~}
\newcommand{\sect}{Sec\/.\/~}
\newcommand{\tabl}{Tab\/.\/~}
\newcommand{\algo}{Algorithm~}
\newcommand{\theo}{Theorem~}

\newcommand{\bnnl}{3 hidden layers}
\newcommand{\bnnn}{50 neurons}
\newcommand{\bnna}{tanh activations}

\newcommand{\capt}[1]{\mdseries{\emph{#1}}}

\newcommand{\videolink}{at \url{https://youtu.be/_d7AqTRjz6g}}
\newcommand{\codelink}{\url{https://github.com/wheelbot/mini-wheelbot}}

\newcommand{\fakepar}[1]{\vspace{0mm}\noindent\textbf{#1.}}

\newcommand{\needref}{\textcolor{red}{[REF]}}

\newcommand{\plotfontsize}{9pt}


% If the title and author information does not fit in the area allocated, uncomment the following
%
%\setlength\titlebox{<dim>}
%
% and set <dim> to something 5cm or larger.

% \title{Rethinking Reference Determinacy in Natural Language Inference}
\title{On Reference \textit{(In-)}Determinacy in Natural Language Inference}


\author{Sihao Chen\textsuperscript{\msftlogo}\thanks{\,~Work done during Sihao's and Chaitanya's internship at Google. Sihao was a Ph.D. student at the University of Pennsylvania at the time.  } \, \textbf{Chaitanya Malaviya}\textsuperscript{\pennlogo} \, \textbf{Alex Fabrikant}\textsuperscript{\gdmlogo} \\ \textbf{Hagai Taitelbaum}\textsuperscript{\googlogo} \, \textbf{Tal Schuster}\textsuperscript{\gdmlogo} \, \textbf{Senaka Buthpitiya}\textsuperscript{\gdmlogo} \, \textbf{Dan Roth}\textsuperscript{\pennlogo} \vspace{0.1in}
\\
\textsuperscript{\pennlogo} University of Pennsylvania \,
\textsuperscript{\gdmlogo} Google DeepMind \\
\textsuperscript{\googlogo} Google Research \, \textsuperscript{\msftlogo} Microsoft \\
}

% \author{
% \textbf{Tong Chen}\textsuperscript{$\clubsuit$}\thanks{$\ \ $Work was done during internship at Tencent AI Lab, Bellevue.} \,
% \textbf{Hongwei Wang}\textsuperscript{$\diamondsuit$} \,
% \textbf{Sihao Chen}\textsuperscript{$\heartsuit$} \,
% \textbf{Wenhao Yu\textsuperscript{$\diamondsuit$}} \\
% \textbf{Kaixin Ma}\textsuperscript{$\diamondsuit$} \,
% \textbf{Xinran Zhao}\textsuperscript{$\spadesuit$} \,
% \textbf{Hongming Zhang}\textsuperscript{$\diamondsuit$} \,
% \textbf{Dong Yu}\textsuperscript{$\diamondsuit$}
% \vspace{5pt} \\ 
% \textsuperscript{$\clubsuit$}University of Washington \,
% \textsuperscript{$\diamondsuit$}Tencent AI Lab \\
% \textsuperscript{$\heartsuit$}University of Pennsylvania \,
% \textsuperscript{$\spadesuit$}Carnegie Mellon University \,
% }


\begin{document}
\maketitle


\begin{abstract}
% Natural Language Inference (NLI) provides a general task format for evaluating the semantic relations between two pieces of text. 
% As NLI tasks, datasets and models are now used extensively in a wide variety of applications, it is important to understand how the definition of NLI could limit or influence its downstream usage.
% Natural Language Inference (NLI) provides a general task format for evaluating the semantic relations between two pieces of text, which can be useful for various applications such as fact verification and text attribution.
% As \textit{natural language inference} (NLI)
% However, existing datasets for NLI and models trained on these datasets make assumptions about the context from which the premise and hypothesis are sampled.
We revisit the \textit{reference determinacy} (RD) assumption in the task of natural language inference (NLI), i.e., the premise and hypothesis are assumed to refer to the same context when human raters annotate a label. 
While RD is a practical assumption for constructing a new NLI dataset, we observe that current NLI models---which are typically trained solely on hypothesis-premise pairs created with the RD assumption---fail in downstream applications such as fact verification, where the input premise and hypothesis may refer to different contexts. To highlight the impact of this phenomenon in real-world use cases, we introduce \datasetname, a diagnostic benchmark for identifying reference ambiguity in NLI examples.  
% . \datasetname emulates the fact verification setting, 
In \datasetname, the premise is retrieved from a knowledge source (i.e. Wikipedia) and does not necessarily refer to the same context as the hypothesis. With \datasetname, we demonstrate that finetuned NLI models and few-shot prompted LLMs both fail to recognize context mismatch, leading to $>80\%$ false contradiction and $>50\%$ entailment predictions. We discover that the existence of reference ambiguity in NLI examples can in part explain the inherent human disagreements in NLI, and provide insight into how the RD assumption impacts NLI dataset creation process.
\\\\
\faGithub \,\,\url{https://github.com/refnli-authors/refnli}
% We discuss the implications of such,   
% can We discuss the connection of this finding to human disagreement on NLI labels. We then compare and discuss a few strategies for NLI model training and LLM inference to enhance their ability to recognize reference determinacy.


% Natural Language Infernece (NLI), or recognizing textual entailment (RTE) 
% This paper discusses the effect of reference determinacy assumption empirical point of view 
% argue that the current way  NLI datasets 

% In existing general-purpose NLI datasets such as SNLI and MNLI, We observe that humans inherently disagree more on 
\end{abstract}


\section{Introduction}
\label{sec:intro}

\begin{figure*}[tb]
    \centering
    \includegraphics[width=0.848\linewidth]{figs/circuitnn.pdf} 
    \caption{Illustration of differentiable CircuitNN. CircuitNN is designed based on differentiable NAND gates. After DAS is guided by PI and PO pairs of the truth table, CircuitNN can get the precise circuit architecture logic equivalent to the truth table.}
    \label{fig:circuitnn}
\end{figure*}

% 1. Describe the importance of logic synthesis
% 2. Existing Problems
% (a) Neural Architecture Search: Unstable, Predefined Setting, etc.
% (b) Circuit Generation: Probabilistic Model, Logic Equivalence

With the rapid advancement of technology, the scale of integrated circuits (ICs) has expanded exponentially. 
This expansion has introduced significant challenges in chip manufacturing, particularly concerning power and area metrics.
A primary objective in IC design is achieving the same circuit function with fewer transistors, thereby reducing power usage and area occupancy.

Logic synthesis~\cite{hachtel2005logicsynth}, a critical step in electronic design automation (EDA), transforms behavioral-level circuit designs into optimized gate-level circuits, ultimately yielding the final IC layout. 
The primary goal of logic synthesis is to identify the physical implementation with the fewest gates for a given circuit function. 
This task constitutes a challenging NP-hard combinatorial optimization problem. 
Current logic synthesis tools~\cite{brayton2010abc, wolf2013yosys} rely on human-designed heuristics, often leading to sub-optimal outcomes.

Differentiable architecture search (DAS) techniques~\cite{liu2018darts, chu2020darts} offer novel perspectives on addressing challenges in this problem.
Circuit functions can be represented through truth tables, which map binary inputs to their corresponding outputs. 
Truth tables provide a precise representation of input-output relationships, ensuring the design of functionally equivalent circuits.
Inspired by this, researchers~\cite{deepmind2024ai4sys, wang2024tnet} have begun exploring the application of DAS to synthesize circuits directly from truth tables.
Specifically, \citet{deepmind2024ai4sys} proposed CircuitNN, a framework that learns differentiable connection structures with logic gates, enabling the automatic generation of logic circuits from truth tables.
This approach significantly reduces the complexity of traditional circuit generation. 
Building on this, \citet{wang2024tnet} introduced T-Net, a triangle-shaped variant of CircuitNN, incorporating regularization techniques to enhance the efficiency of DAS.

Despite these advancements, several challenges remain. 
The computational complexity of DAS grows quadratically with the number of gates, posing scalability issues.
Although triangle-shaped architecture~\cite{wang2024tnet} partially mitigates this problem, redundancy persists. 
%Additionally, DAS is susceptible to converging to local optima, limiting the ability to search architectures that satisfy the given truth tables~\cite{liu2018darts}. 
%Furthermore, hyperparameters (network depth and layer width) require extensive searches, introducing complexity and prolonging the synthesis process. 
Additionally, DAS is susceptible to converging to local optima~\cite{liu2018darts} and hyperparameters (network depth and layer width) require extensive searches. 
The challenges arise from the vast search space in DAS. 
% Even with predefined settings for CircuitNN, finding a configuration that meets the truth table requires extensive trial and error during the DAS process. 
Intuitively, limiting the search space through predefined parameters (network depth, gates per layer, and connection probabilities) can significantly reduce the complexity.

Recent advances~\cite{openai2023gpt4, abramson2024alphafold3, esser2024sd3, li2024mar} in conditional generative models have demonstrated remarkable performance across language, vision, and graph generation tasks. 
Motivated by these developments, we propose a novel approach to circuit generation that generates preliminary circuit structures to guide DAS in generating refined circuits matching specified truth tables. 
Firstly, we introduce CircuitVQ, a tokenizer with a discrete codebook for circuit tokenization. 
Built upon our Circuit AutoEncoder framework~\cite{hou2022graphmae,li2023maskgae,wu2025mgvga}, CircuitVQ is trained through a circuit reconstruction task. 
Specifically, the CircuitVQ encoder encodes input circuits into discrete tokens using a learnable codebook, while the decoder reconstructs the circuit adjacency matrix based on these tokens.
Subsequently, the CircuitVQ encoder serves as a circuit tokenizer for CircuitAR pretraining, which employs a masked autoregressive modeling paradigm~\cite{chang2022maskgit, li2023mage}. 
In this process, the discrete codes function as supervision signals. 
After training, CircuitAR can generate discrete tokens progressively, which can be decoded into initial circuit structures by the decoder of the CircuitVQ. 
These prior insights can guide DAS in producing refined circuits that match the target truth tables precisely.

Our key contributions can be summarized as follows:
\begin{itemize}
\item We introduce CircuitVQ, a circuit tokenizer that facilitates graph autoregressive modeling for circuit generation, based on our Circuit AutoEncoder framework;
\item Develop CircuitAR, a model trained using masked autoregressive modeling, which generates initial circuit structures conditioned on given truth tables;
\item Propose a refinement framework that integrates differentiable architecture search to produce functionally equivalent circuits guided by target truth tables;
\item Comprehensive experiments demonstrating the scalability and capability emergence of our CircuitAR and the superior performance of the proposed circuit generation approach.
\end{itemize}

% Motivation
% (a) Diffusion (Vision, Graph), Autoregressive (Language, Vision)
% (b) Circuit Generation for Predefined Setting
% (c) Neural Architecture Search for Strict Logic Equivalence

% Contribution
% (a) Circuit Tokenizer (new transformer arch, training strategy)
% (b) CircuitAR (train and gen strategies, post-ar strategy)
% (c) Extensive Evaluation including BitD (Bit Distance) for Scalability

\section{Motivation}
\label{sec:motivation}



% In LLM inference, not only does weight matter, but the memory requirements of the KV Cache are also considerable.
In this section, we first demonstrate that the emerging paradigm of group quantization demands a high level of adaptivity, which current adaptive methods lack.
We then discuss how adapting these methods to group quantization could compromise their efficiency.
Given that LLMs generate KV caches during runtime, real-time quantization capability is crucial.
These challenges lead to our proposal of a mathematical adaptive numerical type (\texttt{MANT}), which we will detail later.



\begin{figure}[t]
    \centering
    \begin{minipage}[t]{0.48\columnwidth}
      \centering
      \includegraphics[width=\columnwidth]{fig/moti_group_ppl.pdf}
      \caption{LLM accuracy with different quantization granularities. We report the perplexity (PPL) metric (lower is better).}\label{fig:moti_group_ppl} 
    \end{minipage}
    \hspace{2pt}
    \begin{minipage}[t]{0.48\columnwidth}
      \centering
      \includegraphics[width=\columnwidth]{fig/motivation_adaptive_ppl.pdf}
      \caption{Accuracy loss for \texttt{INT}, \texttt{ANT}, and Ideal (clustering algorithm K-Means) adaptive methods in group quantization. }\label{fig:moti_ppl} 
    \end{minipage}
    % \vspace*{-0.3cm}
\end{figure}




\subsection{Group Quantization Accuracy Analysis}
\label{sec:acc_analysis}

In this subsection, we begin by comparing the accuracy of traditional channel-wise quantization with group-wise quantization~\cite{shao2024omniquant,zhao2023atom,liu2024kivi,sheng2023flexgen,lin2023awq,zhao2023atom}, establishing the baseline for group-wise quantization in this study.
We then delve into the use of various adaptive data types in group quantization, emphasizing the necessity for full adaptivity.



\Fig{fig:moti_group_ppl} illustrates the perplexity when quantizing the LLaMA-7B model~\cite{touvron2023llama} with various granularities using the \texttt{INT4}-based symmetric quantization.
Channel-wise quantization significantly worsens the perplexity of the examined LLM, increasing it from 5.68 to 6.85.
Conversely, group-wise quantization mitigates this loss in perplexity with a group size of 128, corresponding to an average of 4.125 bits per element (16-bit scaling factor).
Additionally, we observe that a smaller group size of 32 offers only a slight improvement in perplexity, but the scaling factor overhead increases by $4\times$.



Given this analysis, we adopt a group size of 128 as our standard configuration for the remainder of this section.
Previous research indicates that the \texttt{INT} data type is not optimal for accuracy since tensors or channels exhibit varied distributions, leading to the proposal of various adaptive data types~\cite{guo2022ant, guo2023olive, zadeh2020gobo, zadeh2022mokey}.
We evaluate their efficacy in the context of group quantization, which falls into two main categories: data-type-based and clustering-based.



\textbf{Data-type-based adaptive methods} select data types from discrete sets based on tensor data distribution.
ANT~\cite{guo2022ant} is a representative example of the data-type-based method.
ANT packages several different data types for selection, including \texttt{INT} for the uniform distribution, \texttt{PoT} (Power of Two) for the Laplace distribution, and \texttt{flint} for the Gaussian distribution.
%ANT designed \texttt{flint} for Gaussian distributions.

\textbf{Clustering-based adaptive methods} utilize clustering algorithms to generate centroids that align with the data distribution and provide considerable adaptivity. 
Mokey~\cite{zadeh2022mokey} and GOBO~\cite{zadeh2020gobo} exemplify this approach, though they focus on tensor- or channel-wise quantization. In our study, we adapt them to group quantization through per-group clustering.

%Clustering-based methods employ clustering algorithms to generate centroids that fit the data distribution, demonstrating sufficient adaptivity.
%Mokey~\cite{zadeh2022mokey} and GOBO~\cite{zadeh2020gobo} are such presentative works, but only target tensor- or channel-wise quantization.
%In our work, we modify those works to support group quantization by performing per-group clustering.
\Fig{fig:moti_ppl} compares the accuracy of the methods described above for the LLaMA-7B model under 4-bit group-wise quantization. 
The group-wise \texttt{ANT} method outperforms the \texttt{INT} type by dynamically selecting from three data types to better match the value distribution, resulting in reduced perplexity (PPL) loss. 
Moreover, per-group clustering adjusts more effectively to the value distribution of each group, establishing itself as the accuracy-optimal and ideal adaptive method. 
This approach achieves nearly lossless 4-bit quantization, equivalent to 16 centroids per group. 
However, this ideal scenario is impractical due to the significant overhead associated with storing per-group centroids, effectively rendering it a 6-bit quantization.

\begin{figure}[t] 
    \centering 
    \includegraphics[width=1.0\linewidth]{fig/intro_cdf.pdf}  
    \caption{The cumulative distribution function (CDF) of the tensor, channel, and group, respectively. The tensor data were taken from layers 8 to 23, while the 16 channel and group data were sampled from one tensor with specific strides.}\label{fig:moti_dist} 
    %  \vspace*{-0.3cm}
\end{figure}

To illustrate the group-wise diversity in data distribution, we sampled the weights of the Q and V tensors in LLaMA-7B model. 
We normalized all sampled data to their absolute maximum values, which ranged from -1 to 1. \Fig{fig:moti_dist} displays the cumulative distribution function (CDF) for the tensor, channel, and group levels, respectively. 
We observed that the diversity at the group level is significantly higher than at the tensor level. 
In simpler terms, while different tensors exhibit similar distributions, groups can have markedly different distributions. This finding underscores the necessity for full adaptivity in group quantization to fully realize its potential.
\paragraph{Takeaway 1.} The group quantization is an emerging paradigm to accelerate LLMs, and the significant group-level diversity requires a high level of adaptivity to fully unleash its potential.

\subsection{Group Quantization Efficiency Analysis}
\label{subsec:efficiency}


In this subsection, we provide a detailed efficiency analysis for the above adaptive quantization methods.
In \Tbl{intro:dtype}, we compare OliVe~\cite{guo2023olive}, ANT~\cite{guo2022ant}, GOBO~\cite{zadeh2020gobo}, and Mokey~\cite{zadeh2022mokey} with \texttt{INT} regarding the efficiency of computation, encoding, and decoding. 
In this paper, we use the term encoding (decoding) interchangeably with quantization (dequantization).
 

Data-type-based adaptive methods such as ANT~\cite{guo2022ant} and Olive~\cite{guo2023olive} achieve computational efficiency comparable to \texttt{INT}. 
Both utilize specialized decoders that decode these data types prior to computation, resulting in high decoding efficiency. 
However, as previously demonstrated, these methods suffer from limited adaptivity in the group quantization paradigm. 
A straightforward approach to enhance adaptivity is to expand their set of data types. 
However, incorporating new data types necessitates additional decoders, escalating hardware design costs. 
Additionally, compatibility issues between new and existing data types may reduce computational efficiency. 
For instance, the \texttt{NF4} data type~\cite{dettmers2023qlora} requires an FP16 MAC unit, which is incompatible with existing \texttt{ANT} data types.


\paragraph{Takeaway 2.} Enhancing the data-type-based adaptive method for group quantization is challenging and requires a careful balance for the computation and decoding efficiency.

Clustering-based adaptive methods like GOBO~\cite{zadeh2020gobo} and Mokey~\cite{zadeh2022mokey} can sufficiently adapt to various distributions at the group level. 
However, they require codebooks for quantization and dequantization, leading to high adaptivity at the expense of encoding and computational efficiency. 
For instance, a 16-entry codebook with 8 bits per entry requires 128 bits per group, creating an inevitable trade-off between adaptivity and memory overhead. GOBO~\cite{zadeh2020gobo} employs the K-means algorithm to quantize weights and requires dequantization to \texttt{FP16} using a codebook lookup table before computation, resulting in high adaptivity but low computational efficiency. 
Conversely, Mokey~\cite{zadeh2022mokey} enhances the computation of clustering-based methods by using indices for centroid values via approximate calculations, though matrix multiplication still relies on floating-point units, increasing overhead compared to integer units. 
Furthermore, Mokey creates one \texttt{golden dictionary} for all activations and weights, akin to using a single data type in quantization, thus reducing adaptivity.


\paragraph{Takeaway 3.} Deploying the clustering-based adaptive methods under group quantization is challenging owing to the low encoding and computation efficiency. 


\begin{table}[t]
    \centering
    \small
    \renewcommand{\arraystretch}{1.2}
    \caption[]{Features of DNN accelerators with adaptive and flexible data types are summarized. Here, `Effi.' stands for efficiency, `Med.' for medium, and `LUT' for lookup table.}
  
    \resizebox{1.0\columnwidth}{!}{
      \begin{tabular}{c|cc|ccc|cc|c}
        \Xhline{1.2pt}
        \multirow{2}{*}{Architecture} & \multicolumn{2}{c|}{Encode} & \multicolumn{3}{c|}{Computation} & \multicolumn{2}{c|}{Decode} & \multirow{2}{*}{Adaptivity} \\ \cline{2-8}
        & Method & Effi. & Method & Bit & Effi. & Method & Effi. \\
        \Xhline{1.2pt}
        \texttt{INT} & Round & High & INT & 4 \& 8 & High & Calculation & High & Low \\ 
        OliVe~\cite{guo2023olive} & Search & Med. & INT & 4 \& 8 & High & Decoder & High & Med. \\ 
        ANT~\cite{guo2022ant} & Search & Med. & INT & 4 \& 8 & High & Decoder & High & Med. \\ 
        Mokey~\cite{zadeh2022mokey} & Cluster & Med. & Float & 4 \& 8 & Med. & Calculation & Med. & Low \\ 
        GOBO~\cite{zadeh2020gobo} & Cluster & Low & Float & 16 & Low & LUT & Med. & High \\ 
        \hline
        \multirow{2}{*}{\proj}  & Search  & Med.  & \multirow{2}{*}{INT} & \multirow{2}{*}{4 \& 8} & \multirow{2}{*}{High} & \multirow{2}{*}{Calculation} & \multirow{2}{*}{High} & \multirow{2}{*}{High} \\ \cline{2-3}
        &  Map &  High &  &&&\\ 
        \Xhline{1.2pt}
    \end{tabular}
    }
    \vspace*{0.1cm}
    \label{intro:dtype}
    \vspace*{-0.2cm}
  \end{table}

\subsection{Support for Real-time Quantization}
\label{sec:moti_kvcache}

The above group-wise diversity presents a challenge for both weights and KV cache.
In addition, KV cache faces challenges in real-time group-wise quantization because the KV cache is generated dynamically during LLM inference.


To facilitate low-precision computation in group-wise quantization, it is necessary to quantize K and V along the inner dimension. 
This requirement stems from the support for matrix inner product operations in most GPUs and TPUs. 
During these operations, the group-wise scaling factor can be extracted from the multiply-accumulate process. 
\Fig{fig:kv_process} depicts the computation process of K and V during the decode stage. We define the dimension used for matrix inner product operations as the inner dimension. 
The inner dimensions of the K and V caches differ; the K cache requires a transpose operation, whereas the V cache does not, complicating the situation.


In the prefill stage, K and V can easily compute the scaling factor for each group. 
During the decode stage, the newly generated K vector is concatenated along the inner dimension of the K cache, enabling immediate quantization. 
However, the newly generated V vector is associated with different groups, with only one element per group produced per iteration. This process prevents the scaling factor for the entire group from being obtained in a single iteration, posing a significant challenge for the real-time quantization of the V cache.


\begin{figure}[t] 
  \centering 
  % \includegraphics[width=1.0\linewidth]{fig/dse_kv_process.pdf}  
  \includegraphics[width=0.9\linewidth]{fig/moti_kv_dimension.pdf}  
  \caption{\small Comparison of group-wise K and V cache quantization. They have different inner dimensions due to the transposition of K (key).}

  \label{fig:kv_process}
  % \vspace*{-0.4cm}
\end{figure}


Given those challenges, we propose \proj with a mathematical encoding format that can fuse with integer computation and enhance the decoding efficiency.
In addition, this encoding format provides sufficient adaptivity for group-wise quantization.
Regarding the challenge in KV cache, \proj employs a real-time quantization engine that ensures efficient encoding and decoding for KV cache.
By addressing these challenges, \proj enables efficient low-bit group-wise quantization.


Frontier language models demonstrate a remarkable mismatch between their problem-solving capabilities and poor out-of-box verification capabilities.
These limitations have largely been attributed to the inability of current language models to self-diagnose hallucinations or enforce rigour \citep{zhang_how_2023,orgad_llms_2024,snyder_early_2024,kamoi_evaluating_2024, tyen_llms_2024, DBLP:conf/iclr/0009CMZYSZ24}.
However, our findings that models can be directed to accurately perform verifications at scale suggest that these out-of-box limitations can be addressed with standard methods like instruction tuning.
We compiled a set of challenging reasoning problems and candidate solutions to provide a benchmark for these deficits.

Each entry in this benchmark consists of a question, a correct candidate response, and an incorrect candidate response, and is manually curated from the residuals of our sampling-based search experiments (Section~\ref{section:pipeline}).
An example entry from this benchmark can be found below (see Appendix~\ref{app:examplebenchmark} for more).

\vspace{0.4cm}
\begin{tcolorbox}[title=Question from LiveBench Reasoning (Web-of-Lies Puzzle), breakable]
In this question, assume each person either always tells the truth or always lies. The person at the campground thinks their friend is lying. Mateo is at the aquarium. The person at the restaurant says the person at the hotel lies. Farid is at the movie theater. The person at the movie theater says the person at the campground lies. Ryan is at the shopping mall. The person at the cafe says the person at the campground lies. The person at the observatory says the person at the museum lies. The person at the museum says the person at the restaurant tells the truth. The person at the campground says the person at the airport tells the truth. The person at the campground tells the truth. Luna is at the restaurant. The person at the movie theater saw a firetruck. Isabella is at the observatory. The person at the amusement park says the person at the museum tells the truth. The person at the restaurant says the person at the shopping mall tells the truth. Liam is at the beach. Ayaan is at the airport. The person at the restaurant says the person at the beach tells the truth. Maya is at the amusement park. The person at the restaurant says the person at the aquarium lies. The person at the restaurant says the person at the movie theater lies. The person at the library says the person at the campground tells the truth. Elowen is at the hotel. Priya is at the library. The person at the gym says the person at the restaurant tells the truth. Kehinde is at the cafe. Anika is at the museum. Quan is at the campground. Soren is at the gym. Does the person at the museum tell the truth? Does the person at the restaurant tell the truth? Does the person at the movie theater tell the truth? Think step by step, and then put your answer in **bold** as a list of three words, yes or no (for example, **yes, no, yes**). If you don't know, guess.
\end{tcolorbox}
\definecolor{mydarkgreen}{rgb}{0.0, 0.5, 0.0} %


\begin{paracol}{2}

\switchcolumn[0]
\begin{tcolorbox}[breakable, colframe=mydarkgreen!70, colback=mydarkgreen!10, title=Correct response]
Here's how we can break down this logic puzzle:

1. **Quan is at the campground and tells the truth.** This is a given fact.

...

4. **The person at the restaurant (Luna) says the person at the hotel (Elowen) lies.**  We don't know yet if Luna is lying or telling the truth.

...

10. **Since Luna tells the truth about Farid, Luna *always* tells the truth.**  This unlocks several other pieces of information.

...

13. **The person at the cafe (Kehinde) says the person at the campground lies.** Since Quan tells the truth, Kehinde is lying.

Therefore:
* **Anika (museum):** Tells the truth.
* **Luna (restaurant):** Tells the truth.
* **Farid (movie theater):** Lies.

**yes, yes, no**
\end{tcolorbox}

\switchcolumn
\begin{tcolorbox}[breakable, colframe=red!70, colback=red!10, title=Incorrect Response Due to Hallucination]
Here's how we can solve this logic puzzle:

1. **Quan is at the campground and tells the truth.** This is given.

2. **The person at the airport (Ayaan) tells the truth.** Quan (campground) says the person at the airport tells the truth, and Quan tells the truth.

...

4. **Luna (restaurant) lies.**  The person at the restaurant says the person at the hotel (Elowen) lies. \textbf{\color{red} Later, the person at the restaurant says the person at the campground (Quan) lies.} Since Quan tells the truth, Luna (restaurant) must be lying.

...

Therefore:
* **Anika (museum): lies**
* **Luna (restaurant): lies**
* **Farid (movie theater): lies**

**Answer: no, no, no** 
\end{tcolorbox}
\end{paracol}

\noindent\begin{minipage}{\textwidth}
\captionof{figure}{Example of an entry in our verification benchmark. The question is sourced from the LiveBench Reasoning benchmark, and the two responses are generated by Gemini v1.5 Pro. The green response has the correct final answer; the red response has the wrong final answer due to hallucinating a non-existent clause.}
\label{fig:example}
\vspace{0.4cm}
\end{minipage}

\noindent
On each entry, our benchmark studies verification accuracy on two tasks:
\begin{enumerate}
    \item \textbf{Scoring task.} When given only the question and one of the responses, is the model able to discern the correctness of the response?
    \item \textbf{Comparison task.} When provided the whole tuple with the correctness labels of the responses masked and a guarantee that at least one response is correct, is the model able to discern which response is correct and which is incorrect?
\end{enumerate}

\noindent
The scoring task is also evaluated over a separate set of (question, response) pairs where the response reaches the correct final answer by coincidence but contains fatal errors and should be labeled by a reasonable verifier as being incorrect; an example can be found in Appendix~\ref{app:examplebenchmark}.
In the scoring task, models are provided only with the task description; in the comparison task, models are provided only with the task description and a suggestion to identify disagreements between responses in its reasoning.

Table~\ref{tab:benchmark} lists the baseline performances of current commercial model offerings on this benchmark.
Gemini v1.5 Pro is omitted from the benchmark as the entries in the benchmark are curated from the residuals of Gemini v1.5 Pro.
The prompts used in Table~\ref{tab:benchmark} are provided in Appendix~\ref{app:benchmarkprompts}.

As we previously observed, and has been noted in prior works \citep{tyen_llms_2024, kamoi_evaluating_2024}, verification errors are typically due to low recall.
Even the easier comparison task, models perform only marginally better---and often worse---than random chance.
In many cases, Consistency@5 performs worse than one-shot inference because Consistency simply averages out noise from an output distribution, meaning that a model biased towards producing an incorrect answer will do so with higher probability under Consistency.
Addressing these deficits in verification capabilities---which we see as low-hanging fruit for post-training---would enable not only better sampling-based search, but also other downstream applications of verification including reinforcement learning \citep[e.g.][]{o1-preview,deepseekai2025deepseekr1incentivizingreasoningcapability}, data flywheeling \citep[e.g.,][]{welleck_generating_2022}, and end-user experience (see Section~\ref{sec:related} for further discussion).


\begin{table}[htbp]
\centering
\begin{tabular}{llcccccc}
\toprule
\textbf{Model} & \textbf{Metric} & \multicolumn{3}{c}{\textbf{Scoring Accuracy}} & \multicolumn{1}{c}{\textbf{Comparison Accuracy}} \\
\cmidrule(lr){3-5} \cmidrule(lr){6-6}
 &  & \textbf{Correct} & \textbf{Wrong} & \textbf{Flawed} &  \\
\midrule
\multirow{2}{*}{GPT-4o} & Pass@1    & 76.5\%  & 31.0\% & 22.2\% & 43.2\%\\
 & Consistency@5 & 77.4\% & 30.0\% & 11.1\% & 35.4\% \\
\midrule
\multirow{2}{*}{Claude 3.5 Sonnet} & Pass@1 & 89.6\% & 22.5\% & 33.3\% & 56.1\% \\
 & Consistency@5 & 90.3\% & 17.5\% & 33.3\% & 61.2\% \\
\midrule
\multirow{2}{*}{o1-preview} & Pass@1 & 100\% & 68.8\% & 80.0\% & 84.5\% \\
 & Consistency@5 & 100\% & 79.4\% & 88.8\% & 92\% \\
\midrule
\multirow{2}{*}{Gemini 2.0 Flash} & Pass@1 & 73.5\% & 44.5\% & 60\% & 58\%  \\
 & Consistency@5 & 77.4\% & 42.5\% & 66.6\% & 58.7\% \\
\midrule
\multirow{2}{*}{Gemini 2.0 Thinking Flash} & Pass@1 & 75.4\% & 56.5\% & 53.3\%  & 80\%  \\
 & Consistency@5 & 77.4\%  & 55\% & 55.5\%  & 89.1\% \\
\midrule
\multicolumn{2}{c}{Random guessing}  & 80\% & 20\% & 20\% & 50\% \\
\bottomrule
\end{tabular}
\caption{Accuracy rates of commercial language models on our verification benchmark. For the task of response scoring (Scoring Accuracy), accuracy rates are broken down for entries that require identifying a correct response as being correct (Correct), entries that require identifying a wrong response as being wrong (Wrong), and entries that require identifying a wrong response that coincidentally reaches the correct answer as being wrong (Flawed).
GPT-4o and Claude 3.5 Sonnet only perform marginally better than random guessing across all tasks. o1-Preview performs better, but still fails to identify 20-30\% of wrong responses.
}
\label{tab:benchmark}
\end{table}

\section{Experiments: Planning outperforms Heuristics}
\label{sec:experiment}

We begin our empirical demonstrations by showcasing the effectiveness of our planning framework on both synthetic and real datasets. We focus on the simplest planning algorithm, 1-step lookaheads (Algorithm~\ref{alg:complete}), and show that even basic planning can hold great promise. 
We illustrate our framework using two uncertainty quantification modules---GPs and 
\ensembles/ \ensembleplus. 

Throughout this section, we focus on evaluating the mean squared error of 
a regression model $\model$,  and develop adaptive policies that minimize uncertainty on $g(f)$ defined in~\eqref{eqn:l2-g-f}.
When GPs provide a valid model of uncertainty, 
our experiments show that our planning framework significantly outperforms other baselines. 
We further demonstrate that our conceptual framework extends to deep learning-based uncertainty quantification methods such as  \ensembleplus while highlighting computational challenges that need to be resolved in order to scale our ideas. 
For simplicity, we assume a naive predictor, i.e., $\psi(\cdot) \equiv 0$. However, we emphasize that this problem is just as complex as if we were using a sophisticated model $\psi(.)$. The performance gap between the algorithms 
primarily depends
on the level  of uncertainty in our prior beliefs.

To evaluate the performance of our algorithm, we benchmark it against several baselines. 
%Active learning baselines use an acquisition function $\ac$ to select points that have the highest   function value: $X\opt_t \in \argmax_{X \in \xpoolj{t}} \ac({X})$ at every step $t$. These methods may also need an UQ module, which we simply use the same UQ module as in our algorithm, and it  outputs $V(X)$ that measures the the uncertainty of each point $X \in \xpoolj{t}$.
Our first set of baselines are from active learning~\citep{AggarwalKoGuHaPh14}:
\\ % \noindent\textbf{Active Learning Heuristics:} 
\textbf{(1)} 
\textsf{Uncertainty Sampling (Static):}  In this approach, we query the samples for which the model is least certain about. Specifically, we estimate the variance of the latent output $f(X)$ for each $X \in \xpool$ using the UQ module and select the top-$K$ points with the highest uncertainty. \\
\textbf{(2)} \textsf{Uncertainty Sampling (Sequential):} This is a greedy heuristic that sequentially selects the points with the highest uncertainty within a batch, while updating the posterior beliefs using pseudo labels from the current posterior state. Unlike \textsf{Uncertainty Sampling (Static)}, this method takes into account the information gained from each point within batch, and hence tries to diversify the selected points within a batch. 

 
We also compare our approach to the  \textbf{(3)} \textsf{Random Sampling}, which selects each batch uniformly at random from the pool. Additionally, we compare solving the planning problem using  \textsf{REINFORCE}-based policy gradients with   $\mathsf{Smoothed\text{-}Autodiff}$ policy gradients.\footnote{Our code repository is available at
  \url{https://github.com/namkoong-lab/adaptive-labeling}.}
%Detailed experimental setups are provided in Section \ref{sec:details-experiments}.

%We repeat all experiments with 10 random seeds.




\begin{figure}[t]
\centering
\begin{minipage}[b]{0.49\textwidth}
\centering
\includegraphics[width=\textwidth, height=5cm]{figures/original_scale/Var_of_l_2_loss.pdf}
\caption{(Synthetic data) Variance of mean squared loss evaluated through the posterior belief $\mu_t$ at each horizon $t$. This is the objective that policy gradient methods like \textsf{REINFORCE} and $\ouralgo$ optimizes. 1-step lookaheads are surprisingly effective even in long horizons.}
\label{fig:var-l2-sim}
\end{minipage}
\hfill
\begin{minipage}[b]{0.49\textwidth}
\centering \includegraphics[width=\textwidth, height=5cm]{figures/original_scale/Error_of_estimated_model_l_2_loss.pdf}
\caption{(Synthetic data) Error between MSE calculated based on collected data $\mc{D}^{0:T}$ vs. population oracle MSE over $\mc{D}_{\rm eval} \sim P_X$. Reducing uncertainty over posteriors directly leads to better OOD evaluations. 1-step lookaheads significantly outperform active learning heuristics in small horizons.}
\label{fig:mean-l2-sim}
\end{minipage}
%\caption{Simulated data for GPs}
%\label{fig:both_plots}
\end{figure}

\subsection{Planning with Gaussian processes}
\label{sec:experiment-plan-GP}
We now briefly describe the data generation process for the GP experiments,  deferring a more detailed discussion of the dataset generation to Section~\ref{sec:details-experiments}. 
We use both the synthetic data and the real data to test our methodology.
For the \emph{simulated data},  we construct a setting where the general population is distributed across \emph{51 non-overlapping clusters} while the initial labeled data $\dtrain$ just comes from one cluster. In contrast, both $\dpool \defeq (\xpool,\ypool),\deval \defeq (\xeval,\yeval)$ are generated   from all the clusters. 
We begin with a low-dimensional scenario, generating a one-dimensional regression setting using a GP. %Gaussian Process (GP).
Although the data-generating process is not known to the algorithms,  we assume that the GP hyperparameters are known to all the algorithms
to ensure fair comparisons. This can be viewed as a setting where our prior is well-specified, allowing us to isolate the effects
of different policy optimization approaches
 without any concerns about the misspecified priors. We select $10$ batches, each of size $K=5$ across $T = 10$ time horizons.

To examine the robustness of our method against the distributional assumptions made  in the simulated case, we then move to a real dataset where the correct prior is not known. We simulate selection bias from the eICU dataset~\citep{PollardJoRaCeMaBa18}, which contains real-world patient data with in-hospital mortality outcomes. 
We conduct a $k$-means clustering to generate 51 clusters and then select data from those clusters. We view this to be a credible replication of practice, as severe distribution shifts are common due to selection bias in clinical labels.  To convert the binary mortality labels into a regression setting, we train a  random forest classifier and fit a GP on predicted scores, which serves as the UQ module for all the algorithms. As before, the task is to select 10 batches, each consisting of 5 samples, across 10 time horizons.

 In Figures~\ref{fig:var-l2-sim} and~\ref{fig:mean-l2-sim}, we present results for the simulated data. 
Figure~\ref{fig:var-l2-sim} shows the variance of $\ell_2$ loss, and Figure~\ref{fig:mean-l2-sim} presents the error in the estimated $\ell_2$ loss using $\mu_t$ (relative to true $\ell_2$ loss, that is unknown to the algorithm). 
As we can see from these plots, our method one-step lookahead  gives substantial improvements  over active learning baselines and random sampling. In addition,
compared to the one-step lookahead planning approach using \textsf{REINFORCE}-based policy gradients, 
we observe that $\mathsf{Smoothed\text{-}Autodiff}$-based policy gradients provide significantly more robust performance over all horizons.

In Figures~\ref{fig:var-l2-real}~and~\ref{fig:mean-l2-real}, we observe similar findings on the eICU data. We see that planning policies (\textsf{REINFORCE} and $\mathsf{Smoothed\text{-}Autodiff}$) consistently outperform other heuristics by a large margin.  Active learning baselines perform poorly in these small-horizon batched problems and can sometimes be even worse than the random search baselines.  Overall, our results show the importance of careful planning in adaptive labeling for reliable model evaluation. 

We offer some intuition as to why one-step lookahead planning may outperform other heuristic algorithms. 
 First,  \textsf{Uncertainty sampling (Static)} while myopically selects the
 top-$K$ inputs with the highest uncertainty, it fails to consider 
the overlap in information content among the ``best” instances; see \citep{AggarwalKoGuHaPh14} for more details. 
In other words,  it might acquire points from the same region with high uncertainty while failing to induce diversity among the batch.
Although \textsf{Uncertainty Sampling (Sequential)} somewhat addresses the issue of information overlap, a significant drawback of 
this algorithm
is the disconnect between the objective we aim to optimize and the algorithm. For example, it might sample from a region with high uncertainty but very low density. 

\begin{figure}[t]
\centering
\begin{minipage}[b]{0.48\textwidth}
\centering
\includegraphics[width=\textwidth, height=5cm]{figures/original_scale/Var_of_l_2_loss_real.pdf}
\caption{(Real-world eICU data) Variance of mean squared loss evaluated through the posterior belief $\mu_t$ at each horizon $t$. Even 1-step lookaheads are extremely effective planners, and auto-differentiation-based pathwise policy gradients provide a reliable optimization algorithm based on low-variance gradient estimates.}
\label{fig:var-l2-real}
\end{minipage}
\hfill
\begin{minipage}[b]{0.48\textwidth}
\centering \includegraphics[width=\textwidth, height=5cm]{figures/original_scale/Error_of_estimated_model_l_2_loss_real.pdf}
\caption{(Real-world eICU data) Error between MSE calculated based on collected data $\mc{D}^{0:T}$ vs. population oracle MSE over $\mc{D}_{\rm eval} \sim P_X$. Reducing uncertainty over posteriors directly leads to better OOD evaluations. Our method significantly outperforms active learning-based heuristics, and random sampling.}
\label{fig:mean-l2-real}
\end{minipage}
%\caption{Real data for GPs}
\end{figure}
 
%\vspace{-1.5cm}
% \begin{wrapfigure}{r}{.32\columnwidth}
%   \vspace{-.5cm} 
%   \centering
% \includegraphics[scale=.29]{figures/Var of l2l_2 loss.pdf}
%   \vspace{-0.2cm}
%   \caption{Results of GP}
% \label{fig:var-l2-gp}
%   \vspace{-0.1cm}
% \end{wrapfigure}


% Attempts have been made  in the past to address these  drawbacks heuristically  (see \citep{AggarwalKoGuHaPh14}). We give a unified computational framework while approaching the problem in a more principled manner and solving it more optimally.




\subsection{Planning with  neural network-based uncertainty quantification methods ($\ensembleplus$)}


We now provide a proof-of-concept that shows the generalizability of our conceptual framework  to the deep learning-based UQ modules, specifically focusing on $\ensembleplus$ due to their previously observed superior performance~\citep{OsbandWenAsDwIbLuRo23}. Recall that implementing our framework with deep learning-based UQ modules  requires us to retrain the model across multiple possible random actions $\bm{a}(\theta)$ sampled from the current policy $\pi_\theta$.
This requires significant computational resources, in sharp contrast to the GPs where the posteriors are in closed form and can be readily updated and differentiated. 

Due to the computational constraints, we test $\ensembleplus$ on a toy setting to demonstrate the generalizability of our framework. We consider a setting where the general population consists of four clusters, while the initial labeled data only comes from one cluster. Again we generate data using GPs.  The task is to select a batch of 2 points in one horizon. We detail the $\ensembleplus$ architecture in Section \ref{sec:details-experiments}, and we assume prior uncertainty to be large (depends on the scaling of the prior generating functions). 
The results are summarized in the Table~\ref{tab:UQ_ensemble}.

% \begin{table}[H]
% \vspace{-10pt}
% \caption{Performance under \ensembleplus as UQ module}
%     \centering
%     \begin{tabular}{|m{3cm}|m{2.5cm}|m{2cm}|} 
%     \hline
%       Algorithm   & Variance of $\loss_2$ loss estimate & Error of $\loss_2$ loss estimate  \\ \hline Random Sampling 
%          & $1710.9 \pm 1352.1$ & $8.67\pm6.62$ 
%       \\ \hline \ouralgo & $1.30 \pm 0.68$ & $0.91\pm0.25$ \\ \hline
%     \end{tabular}
%     \label{tab:UQ_ensemble}
%     %\vspace{-10pt}
% \end{table}




\begin{table}[h]
\vspace{-10pt}
\caption{Performance under \ensembleplus as the UQ module}
\centering
\begin{tabular}{|l|l|l|}
\hline
Algorithm   & Variance of $\loss_2$ loss estimate & Error of $\loss_2$ loss estimate  \\
\hline
\textsf{Random sampling} & 7129.8 $\pm$ 1027.0 & 136.2 $\pm$ 8.28 \\ \hline
\textsf{Uncertainty sampling (Static)} & 10852 $\pm$ 0.0 & 162.156 $\pm$ 0.0 \\ \hline
\textsf{Uncertainty sampling (Sequential)} & 8585.5 $\pm$ 898.9 & 144 $\pm$ 6.93 \\ \hline
\textsf{REINFORCE} & 1697.1 $\pm$ 0.0 & 45.27 $\pm$ 0.0 \\ \hline
\ouralgo & 1697.1 $\pm$ 0.0 & 45.27 $\pm$ 0.0 \\ \hline
\end{tabular}
%\caption{Comparison of different algorithms based on variance   and   error in $\ell_2$ loss estimation with Ensemble $+$ as the UQ module. Our results demonstrate that {\ouralgo} and REINFORCE outperformthe other active learning based heuristics, confirming the benefits of our MDP formulation for the adaptive labeling problem, as also demonstrated in Section 4.\\
%\footnotesize{Experimental details: We use Gaussian Processes as our data generating process, GP parameters are the same as in Section D.3.  The task is to select a batch of 2 points along one horizon.The marginal distribution $p_X$ has 4 \textit{non-overlapping} clusters. Initial data comes from one cluster, while pool and evaluation points comes from all the clusters. We have $20$ initial labeled data points, $10$ pool points, and $252$ evaluation points.  Training procedures are similar to the one in Section D.3.} }
\label{tab:UQ_ensemble}
\end{table}



% We faced  issues in scaling up these experiments which will be our focus in the future. 





% \begin{itemize}
%     \item Posteriors should be consistent. Two dimensions: even with less training,  
%     \item the inference should be  fast enough
% \end{itemize}


% Potential research directions for uncertainty quantification

% In this section we consider a simple setting We consider a simpler setting and 


% For synthetic dataset generation, we use ...... For real datasets, we use ...... We compare our methodolgy to several baselines ()    This Section is structured as follows:
% \begin{itemize}
%     \item \textbf{GPs, square loss objective} (Section \ref{}): 
%     %the broad aim of the experiments  in this section is to isolate the performance of our methodology without any concerns for the inefficiencies induced due to a mis-specified prior or imperfect posterior inference. To accomplish this we generate synthetic datasets using GPs (detailed later). We use the well specified prior (GPs - with same hyperparameter setting) as our UQ module.   
%      As GPs provide differentaible posterior inference - any errors induced due to imperfect posterior updates are also isolated. We note that under this setting
%      \item In Section\ref{} we demonstrate why our methodology performs better than other baselines - by devising various synthetic experiments ()
%     \item  \textbf{UQ Benchmarking }(Section \ref{}): Before diving into the experiments using $\ensembleplus$ and ENNs,  we showcase our benchmarking experiments in Section \ref{}. We use real datasets We observe that ENNs perform better
%      \item \textbf{Ensemble $+$}, objective: recall, accuracy
%     \item \textbf{ENN}, objective: recall, accuracy
% \end{itemize}




% In Section {}, we test 
% \subsection{Experimental details}

% \begin{itemize}
%     \item UQ methodologies - GPs, ENNs
%     \item Objectives - Recall,  ATE
%     \item Datasets - ATE-synthetic datasets, Recall-synthetic, real datasets
%     \item Baselines - 
%     \begin{itemize}
%         \item Random sampling
%         \item Active learning - Uncertainty based sampling - In regression setting almost all of the 
%         \item Myopic greedy - Greedy Batch based sampling
%         \item Policy Gradient
%     \end{itemize}
    
% \end{itemize}

% \subsection{Experiments}
%     \begin{itemize}
%     \item GPs with square loss
%     \item Benchmarking ENN
%         \item ENNs with ATE
%         \item ENNs with Recall
%     \end{itemize}

% \subsection{Benefits over other algorithms - intuition and experiments}

%Active learning - Myopic greedy / Don't rely on the objective rather some entropy version.


%%% Local Variables:
%%% mode: latex
%%% TeX-master: "main"
%%% End:

This work identifies signal collapse as a critical bottleneck in one-shot neural network pruning. Performance loss in pruned networks is due to \textbf{signal collapse} in addition to the removal of critical parameters. We propose \textbf{REFLOW} (\textbf{Re}storing \textbf{F}low of \textbf{Low}-variance signals), a simple yet effective method that mitigates signal collapse without computationally expensive weight updates. By focusing on signal preservation, REFLOW highlights the importance of mitigating signal collapse in sparse networks and enables magnitude pruning to match or surpass state-of-the-art one-shot pruning methods such as CHITA, CBS, and WF.

REFLOW consistently achieves state-of-the-art accuracy across diverse architectures, restoring ResNeXt-101 from under 4.1\% to 78.9\% top-1 accuracy at 80\% sparsity on ImageNet. Its lightweight design makes it a practical solution for both research and deployment, delivering high-quality sparse models without the overhead of traditional approaches. These findings challenge the traditional emphasis on weight selection strategies and underscore the critical role of signal propagation for achieving high-quality sparse networks in the context of one-shot pruning.




\section{Related Work} \label{sec:related}

% \textbf{Adversarial Attack}
\textbf{Attacks on SLAM.} 
%With the rise of machine learning, 
The robustness of computer vision systems is being actively investigated. With the emergence of adversarial images in the digital domain by adding optimized noise directly to images~\cite{szegedy2013intriguing,carlini2017towards}, researchers find that such attacks also exist physically in the real world \cite{eykholt2018robust,song2018physical,zhao2019seeing}. To fill the gap between attacks in the digital and physical worlds, recent studies have demonstrated that attacks on real-world computer vision systems are practical \cite{eykholt2018robust,li2019adversarial,man2020ghostimage,sharif2016accessorize,zhao2019seeing,zhou2018invisible}. However, attacks on traditional computer vision methods such as SLAM are relatively less explored. \cite{yoshida2022adversarial} proposes an attack against the scan matching algorithm in LiDAR-based SLAM, while most SLAMs in AR/VR devices rely on different sensors like RGB/depth cameras and IMUs. \cite{ikram2022perceptual} and \cite{chen2024adversary} mislead visual SLAM by poisoning the images with special patterns, and \cite{wang2021can} causes the camera to fail using infrared light. In our work, we demonstrate attacks on Visual-Inertial SLAM (VI-SLAM) by perturbing the IMU readings, rather than cameras, and showing its impact on XR user experience. 

\textbf{Acoustic Injection Attacks.} Among various physical attacks, acoustic injection attacks are attractive due to their low cost. Son~\etal~\cite{son2015rocking} were the first to introduce acoustic attacks on MEMS gyroscopes, demonstrating how these attacks could lead to sensor denial-of-service and result in drone crashes. WALNUT~\cite{trippel2017walnut} expanded on this by developing output biasing and control attacks that enable precise manipulation of MEMS accelerometer outputs using modulated sound waves. Wang et al.~\cite{wang2017sonic} demonstrated a sonic gun, showcasing the vulnerability of various smart devices (\eg drones and self-balancing vehicles) to acoustic attacks. Tu et al. \cite{tu2018injected} designed side-swing and switching attacks to alter the outputs of MEMS gyroscopes and accelerometers. Furthermore, Ji et al. \cite{ji2021poltergeist} fool the object detectors by applying acoustic attack to the image stabilizers commonly used in modern cameras. However, none of the existing works study the relationship between the acoustic injections and SLAM outputs on recent XR devices. 

% \zijian{Do we need one session about security in AR/VR?}
% \yicheng{TODO}
%\jiasi{cite the AIVR paper (UMass Amherst?) paper is we have not already. They add IMU perturbation but w/o SLAM, iirc} \yicheng{Cited}

\textbf{XR Security and Privacy.} 
%Security and privacy concerns in XR systems have gained significant attention. 
For single-user XR systems, researchers have demonstrated various side-channel attacks to extract sensitive information (\eg keystrokes) through video feeds~\cite{ling2019know}, head movements~\cite{nair2023unique, slocum2023going}, architectural hints~\cite{zhang2023its,shang2020arspy}, power usage~\cite{li2024dangers}, and EM side-channel leakages~\cite{al2021vr}. In multi-user XR systems, Su et al.~\cite{su2024remote} use avatar motion data to infer keystrokes in shared VR environments. Slocum et al.~\cite{slocum2024doesn} reveal vulnerabilities in the shared state frameworks of multi-user AR. Similarly, Lebeck et al.~\cite{lebeck2017securing} highlight risks like deceptive virtual objects and emphasize access control for managing shared physical and virtual spaces. Ruth et al.~\cite{ruth2019secure} further propose a secure multi-user AR framework focusing on content sharing and permissions.
Chandio et al.~\cite{chandio2024stealthy} %introduced a multi-modal spatiotemporal attack that 
simultaneously manipulated visual and inertial sensors to disrupt XR pose estimation. However, their study evaluated the attack using offline datasets and assumed the attacker's capability to manipulate IMU data streams through acoustic means, without real experiments. Ours is the first to demonstrate acoustic injection attacks on recent XR devices, like the Hololens 2, in the real world.
 


\section*{Conclusion}
This paper aims to enhance our understanding of the computational complexity of computing various Shapley value variants. We found that for various ML models --- including decision trees, regression tree ensembles, weighted automata, and linear regression --- both local and global interventional and baseline SHAP can be computed in polynomial time under HMM modeled distributions. This extends popular algorithms, such as TreeSHAP, beyond their empirical distributional scope. We also establish strict complexity gaps between the various SHAP variants (baseline, interventional, and conditional) and prove the intractability of computing SHAP for tree ensembles and neural networks in simplified scenarios. Overall, we present SHAP as a versatile framework whose complexity depends on four key factors: \begin{inparaenum}[(i)] \item model type, \item SHAP variant, \item distribution modeling approach, \item and local vs. global explanations\end{inparaenum}. We believe this perspective provides deeper insight into the computational complexity of SHAP, paving the way for future work.




%We believe that our framework provides a more intricate understanding of SHAP computation complexity across different models, distributions, and variants, paving the way for further research.

Our work opens promising directions for future research. First, expanding our computational analysis to other SHAP-related metrics, such as asymmetric SHAP~\citep{frye20} and SAGE~\citep{covert2020understanding}, would be valuable. Additionally, we aim to explore more expressive distribution classes and relaxed assumptions beyond those in Section \ref{sec:tractable} while maintaining tractable SHAP computation. Finally, when exact computation is intractable (Section \ref{sec:intractable}), investigating the approximability of SHAP metrics through approximation and parameterized complexity theory~\citep{downey2012parameterized} is an important direction.

%Our work opens several promising avenues for future research on the computational properties of explainable AI methods, with a particular focus on SHAP. First, it would be interesting to broaden the computational analysis conducted in this work to include other popular SHAP-related metrics in the literature, such as asymmetric SHAP \cite{frye20} and SAGE \cite{covert2020understanding}. Also, in the future, we aim to explore more expressive distribution classes and relaxed distributional assumptions—extending beyond those examined in Section \ref{sec:tractable} —that still yield tractable SHAP computation. Finally, when exact computation proves intractable (Section \ref{sec:intractable}), it is worthwhile to theoretically investigate the question of the approximability of computing the SHAP metrics across various configurations, through the lens of approximation and parametrized complexity theory \cite{arora2009computational}.

%This paper aims to deepen our understanding of the computational complexity involved in obtaining different Shapley value variants. We found that for a variety of ML models, including decision trees, tree ensembles for regression, weighted automata, and linear regression models — computing both local and global interventional and baseline SHAP can be done in polynomial time when distributions are modeled by HMMs. This extends the distributional scope of popular algorithms like TreeSHAP, which is limited to empirical distributions. Additionally, we demonstrate a strict complexity gap between SHAP variants, showing that interventional and baseline SHAP can be strictly easier to compute than conditional SHAP. Despite these positive results, we uncovered intractability for various SHAP variants in neural networks and tree ensembles. Finally, we provided generalized complexity relations across SHAP variants. We believe that our framework offers a deeper understanding of the complexity involved in computing SHAP across various variants, models, distributions, as well as in both local and global computations, laying the groundwork for future research.
% \section{Limitations} 

In this work, we compared the effectiveness and interplay of SFT and RL-based methods, under fixed data constraints. In particular, we chose offline methods like DPO and KTO as the baseline implementation of the RL method because it eliminates the need for reward modeling or iterative finetuning. This means that the process of development is limited to collecting an offline dataset and fientuning it - making it the most fair comparable to SFT in terms of implementation effort, compute costs and annotation efforts. Since this baseline RL method shows optimal performance over SFT, we hope that this motivates future work to study more complex RL-based methods and their interplay with SFT. In addition, we used GPT4o annotation for synthetic data generation, and also for evaluating Summarization and Helpfulness, which could include potential biases inherited from the model. 

In addition, we limited the size of the model to under 10 Billion parameters, to keep the finetuning cost low enough to ignore as compared to the data annotation costs. In addition, it would be extremely compute resource intensive to run thousands of finetuning runs with larger model sizes like 70B parameters. We hope that future work would study the scaling trends of RL-based methods against different model sizes, and also study the compute-data trade-off in-depth.

% \section*{Acknowledgments}
{\textcopyright}2025 All rights reserved. The research described in this paper was carried out at the Jet Propulsion Laboratory, California Institute of Technology, under a contract with the National Aeronautics and Space Administration (80NM0018D0004).

% Entries for the entire Anthology, followed by custom entries
\bibliography{anthology,custom}

\newpage
\appendix


\section{RefNLI Annotation Guidelines}
\label{appendix:rater-guidelines}

The expert raters for RefNLI were presented with examples consisting of a premise and a hypothesis. For each example, they were given  instructions as follows.

You are to assign one of 4 labels to the example:

\begin{enumerate}
    \item[(a)] \textbf{ambiguous reference}: If the premise contains ambiguous reference, and it’s possible that with resolved reference, premise would actually support/contradict the claim. 
    \item[(n)] \textbf{neutral}: If the premise can’t support or contradict the claim in any possible way. e.g. No matter how you resolve the reference, the premise would still be irrelevant to the claim.
    \item[(c)] \textbf{contradiction}: If the claim is most likely false given the premise.
    \item[(e)] \textbf{entailment}: If premise fully supports the claim.
\end{enumerate}

If you find tricky cases, put yourself in the following scenario: Suppose an LLM generates the claim, you want to decide if we should, given the evidence, tell the user that that this claim is true, tell the user that it's false, or neither. 

The distinction between neutral and ambiguous is going to be difficult sometimes. See examples below for what we are after. If it’s truly unclear – feel free to skip the example.

\subsection*{Specific Guidelines}

\begin{enumerate}
    \item \textbf{Skip unclear claims or premises}: If you think the claim is difficult to understand, or there is too much ambiguity, skip the claim entirely.

    \item \textbf{Don’t label the claim by its truth value in the world}: If a claim says “The sky is blue”, and the premise says something completely different, label it as neutral. Don’t label such cases as entailment based on \textbf{just} your world knowledge.

    \item \textbf{World Knowledge is permitted}: You can assume commonly accepted world knowledge when interpreting the premise, e.g., basic geography and other commonsense knowledge are allowed. If needed, a web search is allowed when making the judgements. However, don’t make too many inferences.

    \item \textbf{Temporal considerations}: Ignore tense (e.g., past or present) in both the premise and claims. If the premise clearly indicates a time of an event, but the claim doesn’t, assume that the claim is uttered right after the event.

    \item \textbf{Personal surnames}: If only the surname of a person is mentioned in the premise, and there’s not enough evidence for in the premise for you to determine the last name is referring to the same entity as in the hypothesis, mark the example as ``ambiguous''

    \item \textbf{Neutral vs. Ambiguous Reference}: The distinction between the two can be difficult sometimes. The general rule is: if the premise can’t seem to support the claim no matter how you interpret the premise, then it's neutral.
\end{enumerate}

\subsection*{Some examples given in the instructions}

\newcommand{\instex}[3]{\noindent\textbf{Premise}: #1

\noindent\textbf{Hypothesis}: #2

\noindent\textbf{Label}: #3\vspace{1ex}}

\instex{Wales has a large region rich in coal deposits.}{The Ural Mountains contain about 48 species of economically valuable ores and economically valuable minerals.}{N; even if we didn't know whether the Ural Mountains are in Wales, the premise doesn’t mention anything about coal deposits, so there’s no way that the premise can support/contradict the claim.}

\instex{Wales has a large region rich in coal deposits.}{Famous for its coal, Newcastle is the largest coal exporting harbour in the world, exporting 159.9 million tonnes of coal in 2017.}{A: The prominent Newcastle is in New South Wales, Australia, but there happens to also be a small town named Newcastle in Wales}

\instex{The Predator made more than \$97 million worldwide.}{Up to March 2011, The Predator’s worldwide gross has reached \$172,543,519, making it the highest-grossing film in the franchise.}{E; if the premise mentions a time, and there’s no clear temporal marker in the claim – assume that the claim is made in the similar time frame as the premise.}

\instex{The Hunchback of Notre Dame is a Disney media franchise, commencing in 1996 with the release of "The Hunchback of Notre Dame".}{The Hunchback of Notre Dame has only ever been based off of a poem.}{Skip, since it's unclear in the hypothesis what ``based off of a poem'' means}

\section{Human Disagreements and Reference Ambiguity}
\label{app:disagree}

\autoref{fig:mnli-snli-cm} shows the confusion matrix between the majority NLI label from the ChaosNLI re-annotation vs. the original majority label from the five SNLI/MNLI annotators originally. 
\begin{figure}[t]
    \centering
    \includegraphics[width=0.7\linewidth]{figures/mnli_snli_cm.png}
    \caption{Confusion Matrices between majority label from the original annotation vs. ChaosNLI's re-annotation label for SNLI and MNLI examples from \citet{nie-etal-2020-learn}.}
    \label{fig:mnli-snli-cm}
    \vspace{-5pt}
\end{figure}




\end{document}
