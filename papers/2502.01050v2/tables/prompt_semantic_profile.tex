\begin{table*}
    \small
    \centering
    \begin{tabular}{p{14cm}}
    \toprule
    \textbf{Dataset Semantic Profiler Prompt} \\
    \midrule
    \textcolor{teal}{\textbf{Instruction:}} \\
    You are a dataset semantic analyzer. Based on the column name and sample values, classify the column into multiple semantic types. \\

    \textcolor{teal}{\textbf{Categories:}} \\
    Please group the semantic types under the following categories: \\
    - \textbf{Temporal} \\
    - \textbf{Spatial} \\
    - \textbf{Entity Type} \\
    - \textbf{Data Format} \\
    - \textbf{Domain-Specific Types} \\
    - \textbf{Function/Usage Context} \\

    \textcolor{teal}{\textbf{Template Reference:}} \\
    Following is the template: \texttt{\{TEMPLATE\}} \\

    \textcolor{teal}{\textbf{Rules:}} \\
    \begin{enumerate}
        \item The output must be a valid JSON object that can be directly loaded by \texttt{json.loads}. Example response is \texttt{\{RESPONSE\_EXAMPLE\}}.
        \item All keys from the template must be present in the response.
        \item All keys and string values must be enclosed in \textbf{double quotes}.
        \item There must be no trailing commas.
        \item Use \textbf{booleans (true/false)} and numbers without quotes.
        \item Do not include any additional information or context in the response.
        \item If you are unsure about a specific category, you can leave it as an empty string.
    \end{enumerate} \\

    \textcolor{teal}{\textbf{Dynamic Parameters:}} \\
    - Column name: \texttt{\{column\_name\}} \\
    - Sample values: \texttt{\{sample\_values\}} \\

    \bottomrule
    \end{tabular}
    \caption{Prompt for the dataset semantic profiler.}
    \label{tab:prompt_semantic_profile}
\end{table*}
