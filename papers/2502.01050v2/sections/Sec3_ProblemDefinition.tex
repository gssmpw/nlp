\section{\AutoDatasetDescriptoinGenerationAbbrev: Automating Dataset Description Generation}
\label{sec:auto_ddg}

\subsection{Problem Definition}
\label{sec:problem_definition}

% 
Given a tabular dataset $D$ with columns $C$ and rows $R$, our task is to \emph{automatically generate a descriptive summary} $S_D$ that effectively captures the key characteristics of $D$, such as column names, data types, statistical properties, and semantically enriched information.
% 
To achieve this, we design a framework $M$ that utilizes an LLM to generate the description $S_D$ based on the prompt $P$ and the context $X(D)$ derived from the dataset $D$:
$$
S_D = M(P, X(D))
$$
where $P$ represents a template or set of instructions that guides the LLM to generate the desired type of description (e.g., user-focused or search-focused). $X(D)$ represents the context generated by a Context Preparation module. Our problem involves designing effective methods for:

\noindent\textbf{Context Preparation}: Extracting and representing the dataset context $X(D)$ in a way that captures the heterogeneous characteristics of tabular data, fits within the input limitations of LLMs, and is aligned with users' information needs.

\noindent \textbf{Description Generation}: Crafting prompts $P$ that guide the LLM to generate accurate and rich descriptions tailored to specific use cases, such as enhancing human readability 
%(User-Focused Descriptions) 
or optimizing for search engines.
%(Search-Focused Descriptions).

\noindent \textbf{Quality Evaluation}: Assessing the generated descriptions using appropriate metrics that evaluate their quality and impact on dataset findability, including both intrinsic measures (e.g., coherence, readability) and extrinsic measures (e.g., impact on search performance).

By formalizing the problem in this way, we set the foundation for developing an automated system that effectively generates dataset descriptions. 
%
Below, we describe how the components of \SystemName address these challenges.
% 

