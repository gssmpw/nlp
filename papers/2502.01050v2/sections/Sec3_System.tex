\subsection{Extracting, Augmenting, and Representing Dataset Context}
\label{sec:solution_STA}

We focus on extracting and representing the context of tabular datasets to improve their interpretability and usability. 
% 
To achieve this, we consider two types of context -- data-driven and semantic~-- to create a summary of a dataset. 
% 


\myparagraph{Data-Driven Profile}
The data-driven context is derived directly from the dataset's contents, capturing structural and statistical properties that are commonly extracted by data profilers~\cite{abedjan2015profiling, naumann2014data,datamart-profiler}. 
%Our data-driven context profiles the whole dataset and captures essential statistical and structural properties, 
In our implementation, we use the Datamart Profiler~\cite{datamart-profiler}. 
% 
As illustrated in \autoref{tab:examples_datadriven_profile}, each table is profiled by analyzing all rows of each column to extract information such as attribute data types, value distributions, and uniqueness.
% 
These elements provide a global view of a dataset, summarizing its structure and content in a concise manner that can be effectively fed into LLMs for dataset description generation.
% 
Beyond statistical summaries, data-driven profiling can help domain-specific customization by extracting metadata that aligns with specific application needs. 


\begin{table}
    \small
    \centering
    % \begin{tabular}{p{7cm}}
    \begin{tabular}{p{7.7cm}}
    \toprule
    \textbf{Health Insurance Dataset} \\
    \midrule
    \textbf{Number of Rows:} 790\\
    \textbf{Number of Columns:} 6\\
    \textbf{Columns}:\\
    - Name: Year\\
    \hspace{5mm} - Data Types: Text, DateTime\\
    \hspace{5mm}  - Coverage: 2013 to 2022 \\
    \hspace{5mm} - Unique Values: 10\\
    - Name: Liabilities\\
    \hspace{5mm} - Data Types: Integer\\
    \hspace{5mm}  - Coverage: 0 to 2682301090.0 \\
    \hspace{5mm} - Unique Values: 757\\
    \hspace{7mm}...\\
    \emph{(other attributes are omitted due to space limitations)}\\
    \bottomrule
    \end{tabular}
    \caption{Example of data-driven profile.}
    \label{tab:examples_datadriven_profile}
    \vspace{-3.0em}
\end{table}


\myparagraph{Semantic Profile}
% 
The semantic profile goes beyond the dataset contents to capture contextual meaning that can enhance both human understanding and search precision, making dataset descriptions more informative.
% 
\SystemName uses LLMs to enrich the dataset context with external knowledge that complements the data-driven profile and  caters to users' information needs. This includes information about the dataset topic and usage, which improves interpretability and relevance~\cite{koesten2020everything}.
% 
%to achieve this, \SystemName employs a 
Below, we describe the \texttt{Semantic Profiler} module (Algorithm~\ref{alg:semantic_type_analyzer}) and the information it produces.


\begin{algorithm}[t]
  \caption{SemanticProfiler}
  \label{alg:semantic_type_analyzer}
\begin{algorithmic}[1]
  \STATE {\bfseries Input:} Tabular Dataset $D$, LLM model $M$, Number of Samples $sample\_size$
  \STATE {\bfseries Output:} Semantically enriched information for each column in $D$
  \STATE Initialize list $semantic\_summary$ as empty
  \FOR{\textbf{each} column $C_i$ in $D$}
      \STATE $sample\_values$ = GetSample($C_i$, $sample\_size$)
      \STATE $semantic\_info$ = Prompt($M$, $C_i$, $sample\_values$)
      \STATE Create a human-readable summary $column\_summary$ based on the semantic information $semantic\_info$
      \STATE Append $column\_summary$ to $semantic\_summary$
  \ENDFOR
  \STATE \textbf{return} $semantic\_summary$
\end{algorithmic}
\end{algorithm}

% \subsubsection{\textbf{Key Components of Semantic Profile}}

% The SP is a dynamic and extensible module that adapts to new or evolving requirements. Its key components include the following:

% \noindent {\textit{Temporal Information:}} Columns containing temporal data are identified, and their resolution (e.g., year, month, day) is determined. This facilitates an understanding of time-related aspects in the dataset, which is important for temporal analysis.

% \noindent {\textit{Spatial Information:}} Columns with spatial data are identified and categorized based on granularity (e.g., country, state, city, coordinates). This classification enhances the handling of location-based data.

% \noindent {\textit{Entity Type:}} Columns are enriched by identifying the type of entity they describe (e.g., person, organization, product), aiding in the contextualization of the dataset.

% \noindent {\textit{Domain-Specific Types:}} Columns are further classified into domain-specific categories (e.g., financial, healthcare, e-commerce) based on their content, providing tailored insights aligned with the dataset’s domain.

% \noindent {\textit{Function/Usage Context:}} The potential role or function of a column is inferred, such as whether it serves as an aggregation key or ranking metric. This adds depth to the dataset's contextual understanding.


\subsubsection{\textbf{Structure-Defined Template Prompting}}
The \texttt{Semantic Profiler(\semprof)} uses a structured prompting approach to guide the LLM. For each column, it generates a prompt that includes the column name, sample values, and data type. The LLM responds with a JSON-formatted classification of the column based on predefined semantic categories.
% 
The structure-defined template and the complete prompt for semantic enrichment analysis are available in 
% 
\subm{the extended version of this paper \cite{AutoDDG_arxiv}.}
%
\arxiv{the Appendix \Cref{tab:prompt_semantic_profile}.}
% 
Examples of the output generated using the structure-defined template are shown in \autoref{tab:examples_year_premiumWritten}.
%

% \begin{table}
%     \small
%     \centering
%     \begin{tabular}{p{7cm}}
%     \toprule
%     \textbf{Example 1:} \textit{Year} column with values like 2018, 2020, 2023 \\
%     \midrule
%     \textbf{Temporal:} \\
%     - \texttt{isTemporal}: \textbf{True} \\
%     - \texttt{resolution}: \textbf{Year} \\
%     \\
%     \textbf{Spatial:} \\
%     - \texttt{isSpatial}: \textbf{False} \\
%     - \texttt{resolution}: \textbf{} \\
%     \\
%     \textbf{Entity Type:} Temporal Entity \\
%     \textbf{Domain-Specific Types:} General \\
%     \textbf{Function/Usage Context:} Aggregation Key \\
%     \bottomrule
%     \end{tabular}
%     \caption{Example response for a column containing temporal data (e.g., Year).}
%     \label{tab:example_response}
% \end{table}
% 
% \begin{table}
%     \small
%     \centering
%     \begin{tabular}{p{7cm}}
%     \toprule
%     \textbf{Example 2:} \textit{Premium Written} column with values like \$688,221,927, \$165,693,911, \$86,114,776 \\
%     \midrule
%     \textbf{Temporal:} \\
%     - \texttt{isTemporal}: \textbf{False} \\
%     - \texttt{resolution}: \textbf{} \\
    
%     \textbf{Spatial:} \\
%     - \texttt{isSpatial}: \textbf{False} \\
%     - \texttt{resolution}: \textbf{} \\
    
%     \textbf{Entity Type:} Monetary Value \\
%     \textbf{Domain-Specific Types:} Financial \\
%     \textbf{Function/Usage Context:} Measurement \\
%     \bottomrule
%     \end{tabular}
%     \caption{Example response for a column containing monetary data (e.g., Premium Written).}
%     \label{tab:example_premium_written}
% \end{table}
% 
\begin{table}[b]
    \small
    \centering
    % \begin{tabular}{p{7cm}}
    \begin{tabular}{p{7.7cm}}
    \toprule
    \textbf{Example 1:} \textit{Year} column with values like 2018, 2020, 2023 \\
    \midrule
    \textbf{Temporal:} \\
    - \texttt{isTemporal}: \textbf{True} \\
    - \texttt{resolution}: \textbf{Year} \\
    
    \textbf{Spatial:} \\
    - \texttt{isSpatial}: \textbf{False} \\
    - \texttt{resolution}: \textbf{} \\
    
    \textbf{Entity Type:} Temporal Entity \\
    \textbf{Domain-Specific Types:} General \\
    \textbf{Function/Usage Context:} Aggregation Key \\
    \midrule
    \textbf{Example 2:} \textit{Liabilities} column with values like 137790801, 43992755, 599895 \\
    \midrule
    \textbf{Temporal:} \\
    - \texttt{isTemporal}: \textbf{False} \\
    - \texttt{resolution}: \textbf{} \\
    
    \textbf{Spatial:} \\
    - \texttt{isSpatial}: \textbf{False} \\
    - \texttt{resolution}: \textbf{} \\
    
    \textbf{Entity Type:} Monetary Value \\
    \textbf{Domain-Specific Types:} Financial \\
    \textbf{Function/Usage Context:} Measurement \\
    \bottomrule
    \end{tabular}
    \caption{Examples of semantic profile.}
    \label{tab:examples_year_premiumWritten}
    \vspace{-3.0em}
\end{table}
% 
% color text (previoius example format)
% \begin{tcolorbox}[breakable,enhanced,before upper={\parindent15pt}, colback=green!5!white, colframe=green!75!black, title=Example Outputs based on Structure-Defined Template of SEA]

% \noindent \textbf{Example 1: \textit{Year} column with values}
% \\
% \noindent \textbf{like 2018, 2020, 2023}

% \noindent \textbf{Temporal}:

% - isTemporal: True

% - resolution: Year
% \\
% \textbf{Spatial}:

% - isSpatial: False

% - resolution: 
% \\
% \textbf{Entity Type}: Temporal Entity
% \\
% \textbf{Domain-Specific Types}: General
% \\
% \textbf{Function/Usage Context}: Aggregation Key
% \\

% \noindent \textbf{Example 2: \textit{Premium Written} column with values}
% \\
% \noindent \textbf{like \$688,221,927, \$165,693,911, \$86,114,776}

% \noindent \textbf{Temporal}:

% - isTemporal: False

% - resolution: 
% \\
% \textbf{Spatial}:

% - isSpatial: False

% - resolution: 
% \\
% \textbf{Entity Type}: Monetary Value
% \\
% \textbf{Domain-Specific Types}: Financial
% \\
% \textbf{Function/Usage Context}: Measurement

% \end{tcolorbox}

The structured output is then serialized into descriptive sentences, and the summaries of all columns are concatenated to form the final output of the SP module.
% 
For example, the serialized semantic summary for example 1 is \textit{**Year**: Represents temporal entity. Contains temporal data (resolution: Year). Domain-specific type: general. Function/Usage Context: Aggregation Key.}
% 
These enriched outputs provide a detailed understanding of each column's semantic properties, enabling the generation of high-quality, contextually relevant dataset descriptions. 

\subsubsection{\textbf{Dataset Topic Generation}}
To enhance dataset understanding and usability, we incorporate dataset topics into the \semprof. 
% 
By leveraging large language models (LLMs), the process analyzes dataset metadata and samples to extract meaningful topics, which provide a high-level overview by capturing a dataset's primary theme in 2-3 words.
% 
It involves two key steps: (1) Prompt Design, where a tailored prompt is constructed using the dataset's title, original description (if available), and sample data to guide the LLM; (2) Topic Generation, where the LLM processes the prompt and generates a brief topic. \autoref{tab:dataset_topic_prompt} shows the prompt used by the Dataset Topic Generator.

\subsubsection{Discussion}

Note that the prompt design was guided by the findings
of studies of information seeking requirements for dataset search (Section~\ref{sec:related_work}). Our goal is to obtain information that is aligned with how users formulate search queries. For example, the \semprof includes information about temporal and spatial attributes, as well as their resolution, as this was a common pattern observed by~\citet{koesten2017trials}.
%
The \semprof (and corresponding prompts) can be adapted to support other information needs and domains. It can also be extended to include other useful information, such as other semantic types of interest~\cite{archetype@vldb2024,chorus@vldb2024}.


\begin{table}
    \small
    \centering
    \begin{tabular}{p{8cm}}
    \toprule
    \textbf{Dataset Topic Generation Prompt} \\
    \midrule
    % \textcolor{teal}{\textbf{Prompt Structure:}} \\
    \texttt{Using the dataset information provided, generate a concise topic in 2-3 words that best describes the dataset's primary theme:} \\
    \texttt{- Title: \{title\}} \\
    \texttt{- Original Description: \{original\_description\} (optional)} \\
    \texttt{- Dataset Sample: \{dataset\_sample\}} \\
    \texttt{- Topic (2-3 words):} \\

    \bottomrule
    \end{tabular}
    \caption{Prompt for generating concise dataset topics based on the dataset title, description, and sample data.}
    \label{tab:dataset_topic_prompt}
    \vspace{-3.0em}
\end{table}




\subsection{Prompting LLM to Derive Description}
\label{sec:solution_UFD_SFD}



\SystemName uses LLMs to generate two types of dataset descriptions: User-Focused Descriptions (UFD) and Search-Focused Descriptions (SFD). For this task, we carefully design prompts to guide the LLM toward producing descriptions optimized either for presenting to dataset search engine users or for improving search relevance.

\subsubsection{\textbf{User-Focused Description (UFD)}}
The User-Focused Description (UFD) is designed to provide a clear, concise, and accurate overview of the dataset, prioritizing human readability and presentation. This type of description works best for scenarios where the dataset needs to be communicated to users in a way that is easily understood, such as in reports, dashboards, or data catalogs aimed at human readers. Although its primary goal is readability, UFD can also be effective for search purposes, as it offers a well-structured overview containing key terms and concepts related to the dataset. The UFD is generated by prompting the LLM to describe the dataset based on dataset samples and the data-driven profile (Section~\ref{sec:solution_STA}).
% 
The prompt for generating a UFD and a typical output example are presented in \autoref{tab:prompt_example_ufd}.
%
Note that the description includes information about the temporal extent and resolution, number of records, as well as
a summary of the contents (e.g., the range of wind speed values)
and an overview statement about the dataset and what it can be used for: it contains information about wind patterns and that it is useful for environmental studies.
% 


\begin{table}
    \small
    \centering
    \begin{tabular}{p{8.1cm}}
    \toprule
    \textbf{User-Focused Description} \\
    \midrule
    % \textcolor{teal}{\textbf{Prompt Structure:}} \\
    \textcolor{teal}{\textbf{Prompt:}} 
    % \texttt{\textbf{Prompt}: 
    \texttt{Answer the question using the following information. First, consider the dataset sample: $D\_sample$. Additionally, the dataset profile is as follows: $D\_profile$. Based on this profile, please add sentence(s) to enrich the dataset description. Furthermore, the semantic profile of the dataset columns is as follows: $D\_semantic$. Based on this information, please add sentence(s) discussing the semantic profile in the description. Moreover, the dataset topic is: $D\_topic$. Based on this topic, please add sentence(s) describing what this dataset can be used for.}
    % 
    \texttt{Based on the information above and the requirements, provide a dataset description in sentences. Use only natural, readable sentences without special formatting.}\\
    % \texttt{Answer}:\\
    \midrule
    \textcolor{teal}{\textbf{Example Output:}} 
    % \texttt{\textbf{Example Output}: 
    This dataset contains wind speed and direction measurements from a specific time period in 2003. The data includes 4433 unique time stamps, with a temporal coverage of May 13 to June 12, 2003, at a resolution of minutes. The average wind speed ranges from 0 to 27.0, with a standard deviation ranging from 0 to 2.78. The average wind direction ranges from 0 to 338.0, with 16 unique values. The dataset provides a comprehensive view of wind patterns during this time period, making it suitable for environmental studies and research.
    \\
    \bottomrule
    \end{tabular}
    \caption{Prompt and Example of User-Focused Description}
    \label{tab:prompt_example_ufd}
    \vspace{-3.0em}
\end{table}


\subsubsection{\textbf{Search-Focused Description (SFD)}}


The Search-Focused Description (SFD) is optimized to enhance the findability of datasets in search engines and dataset portals. As illustrated in \autoref{fig:sfd_overview}, the process begins with a tabular dataset, which is processed by an LLM to generate an initial description and identify the dataset topic. 
% 
By including a specific topic or area related to the dataset, the LLM can focus on expanding the description with relevant terms, concepts, synonyms, and keyword variations. This helps improve search engine indexing and retrieval performance.
% 
For example, suppose a user wishes to publish a dataset on "climate data" that contains detailed measurements across various regions. The system processes the dataset to generate an initial description and identifies "climate data" as the topic. 
% 
% Alternatively, the user can explicitly provide "climate data" as the dataset topic or supply an initial description. 
% 
The system then enhances the description by including climate-related keywords such as "temperature trends," "precipitation," "regional climate analysis," and "weather patterns." This process enables the SFD to incorporate a wide range of terms related to the topic, increasing the likelihood that users searching for climate-related datasets will find the dataset in question.


\begin{table}
    \small
    \centering
    \begin{tabular}{p{8.1cm}}
    \toprule
    \textbf{Search-Focused Description} \\
    \midrule
    \textcolor{teal}{\textbf{Prompt:}} 
    % \texttt{\textbf{Prompt}: 
    \texttt{You are given a dataset about the topic $D\_topic$, with the following initial description: $D\_initial\_description$.
    % 
    Please expand the description by including the exact topic. Additionally, add as many related concepts, synonyms, and relevant terms as possible based on the initial description and the topic.
    % 
    Unlike the initial description, which is focused on presentation and readability, the expanded description is intended to be indexed at backend of a dataset search engine to improve searchability.
    % 
    Therefore, focus less on readability and more on including all relevant terms related to the topic. Make sure to include any variations of the key terms and concepts that could help improve retrieval in search results.
    % 
    Please follow the structure of following example template: $Template$}.\\
    \midrule
    \textcolor{teal}{\textbf{Example Output:}} \\
    % \texttt{\textbf{Example Output (partial)}}:\\
    \textcolor{teal}{\textit{Dataset Overview:}} This dataset contains wind speed and direction measurements ... (An overview description based on UFD.)
    \\
    \textcolor{teal}{\textit{Related Topics:}}
    - Climate/Weather Patterns
    - Renewable Energy
    - Wind Energy
    - Meteorology
    - Atmospheric Science
    - Environmental Monitoring
    - Sustainability
    - Climate Change
    - Weather Forecasting
    - Wind Turbine Performance
    - Energy Generation
    - Spatial Analysis
    \\
    \textcolor{teal}{\textit{Concepts and Synonyms:}}
    - Wind Speed/Velocity
    - Wind Direction
    - Average Wind Speed
    - Standard Deviation
    - Climate Trends
    - Renewable Energy Sources
    - Energy Generation
    - Meteorological Data
    - Atmospheric Conditions
    - Weather Patterns
    - Climate Modeling
    - Weather Forecasting
    - Spatial Analysis
    - Temporal Data
    - Climate Change Impacts
    \\
    \textcolor{teal}{\textit{Applications and Use Cases:}}
    - Analysis of wind patterns for renewable energy projects
    - Understanding climate trends and predicting wind behavior
    - Informing urban planning decisions to optimize wind energy harvesting
    - Developing weather forecasting models
    - Evaluating wind turbine performance and efficiency
    - Analyzing energy generation potential from wind resources
    - Studying the impact of climate change on wind patterns
    \\
    \textcolor{teal}{\textit{Additional Context:}}
    - This dataset can be used to address questions such as "What are the typical wind patterns in a given region?" or "How does climate change affect wind behavior?"
    - It can be integrated with other datasets, such as climate models or energy consumption data, to provide a more comprehensive understanding of the relationship between wind patterns and energy generation.
    - The dataset's relevance extends to interdisciplinary applications, such as urban planning, sustainability, and environmental science.
    \\
    (\emph{additional information omitted due to space limitations}) \\
    \bottomrule
    \end{tabular}
    \caption{Prompt and Example of Search-Focused Description} 
    \label{tab:prompt_example_sfd}
    % \vspace{-1.0em}
\end{table}


\begin{figure}
  \centering
  \includegraphics[width=0.9\linewidth]{figures/sfd_overview.png}
  \vspace{-0.3cm}
  \caption{Workflow for generating Search-Focused Descriptions (SFD). The process begins with a tabular dataset, which is processed by a Large Language Model (LLM) to generate an initial description and identify the dataset topic. These components then feed into the prompt design stage to generate the final SFD, optimized for dataset findability and tailored to the provided dataset topic.}
  \label{fig:sfd_overview}
  \vspace{-0.3cm}
\end{figure}

\autoref{tab:prompt_example_sfd} shows a typical prompt for generating an SFD, which incorporates the dataset topic $D\_topic$ and an initial description $D\_initial\_description$ as inputs, with an output structure defined by the SFD template.
% 
The result is a description that is densely packed with terms relevant to the specified topic, which significantly improves the dataset's ranking in search results and makes it easier for users to discover the dataset when searching for related topics. A complete SFD example output and the output template are in 
% 
\subm{our extended version of this paper \cite{AutoDDG_arxiv}).}
%
\arxiv{the Appendix \Cref{tab:example_sfd} and \Cref{tab:template_sfd}.}

\myparagraph{Discussion: UFDs and SFDs}
While both UFD and SFD serve important roles in dataset description, they are optimized for different purposes. The UFD is crafted with the end user in mind, making the description easy to understand and presenting a well-rounded summary of the dataset. In contrast, the SFD is more technical and keyword-driven, focusing on enhancing search engine performance rather than prioritizing human readability.
% 
The generation of both types of descriptions allows \SystemName to effectively serve both user-friendly and search-optimized needs.

There are different ways in which these descriptions can be generated. Here, we describe our initial approach that treats the SFD as an extension of the UFD.
%
In the SFD, semantic information about the dataset’s topic is reinforced to generate expanded keywords, concepts, and use cases, as illustrated in \autoref{tab:prompt_example_sfd}.
%
Indexing SFD descriptions in search engines improves dataset findability by incorporating richer, more relevant terms.
% 
In future work, we plan to explore different strategies to combine the data-driven and LLM-derived information.

