\vspace{-.3cm}
\section{Conclusions and Future Work}
\label{sec:conclusion}


Effective dataset descriptions are essential for improving findability and assisting users in assessing dataset relevance. However, many datasets lack informative descriptions, limiting their discoverability and usability in search systems. In this paper, we introduced \SystemName, an end-to-end framework for automated dataset description generation, designed to systematically address this challenge.
%
\SystemName combines data-driven profiling with LLM-powered semantic augmentation to generate high-quality descriptions that balance comprehensiveness, faithfulness, conciseness, and readability. 
%
We proposed a multi-pronged evaluation strategy for dataset descriptions that measures improvements in dataset retrieval, assesses alignment with existing descriptions, and employs LLM-based scoring for intrinsic quality evaluation. Recognizing the limitations of existing benchmarks, we introduced two new dataset search benchmarks, ECIR-DDG and NTCIR-DDG, to enable rigorous assessment.

Our experimental results demonstrate that AutoDDG significantly improves dataset retrieval performance, produces high-quality, accurate descriptions, and helps users better understand and assess datasets. Beyond its immediate impact on dataset search engines, our framework provides a scalable and systematic approach to metadata enhancement, benefiting data repositories, open data portals, and enterprise data lakes.



\myparagraph{Limitations and Future Directions}
While \SystemName demonstrates strong performance for tabular datasets, an important avenue for future work is extending its applicability to a broader range of dataset types and content structures. In particular, we aim to explore multi-modal approaches that incorporate additional data modalities, such as images, to generate richer descriptions.

Our data-driven and semantic profilers were designed to extract key information identified as critical for dataset search~\cite{koesten2017trials,papenmeier2021genuine,Sostek2024Discovering}.  Expanding their capabilities to support customization for diverse domains and use cases could further enhance their effectiveness, allowing domain-specific adaptations to better serve researchers and practitioners across different fields.
%
Additionally, while our experiments relied on low-cost LLMs, an open question remains regarding the trade-offs between efficiency and performance when using larger, more powerful models. Future studies could investigate whether more advanced models significantly improve description quality or if lightweight alternatives suffice for most applications.

A critical challenge in LLM-generated descriptions is hallucination, where the model may introduce incorrect or misleading details. Developing robust detection and mitigation strategies to ensure descriptions remain faithful to the dataset’s content and context is an important research direction.

Finally, while automatic description generation significantly improves dataset findability, the generated descriptions are necessarily incomplete. Exploring human-in-the-loop techniques, where users can enrich (e.g., add provenance and information on data collection methodology) or validate descriptions, presents an opportunity to further enhance accuracy, completeness, and trustworthiness, particularly in high-stakes domains such as healthcare and scientific research.


