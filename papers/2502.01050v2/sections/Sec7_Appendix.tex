\newpage
\appendix

% \section{Appendix}
% This appendix contains our complete prompts, examples and additional experiment results.

% \section{Examples of Generated Descriptions}
% \subsection{Example of Search Focused Description (SFD)}
% \label{app:example_sfd}
% \vspace{-0.77cm}
\begin{table*}
    \small
    \centering
    \begin{tabular}{p{14cm}}
    \toprule
    \textbf{Search Focused Description (SFD) Example} \\
    \midrule
    \textcolor{teal}{\textbf{Dataset Overview:}} \\
    - The dataset provides comprehensive financial information pertaining to various health insurance companies, specifically focusing on different types of insurers, including Health Maintenance Organizations (HMO), Managed Care Health (MCH), and Acciden \& Health (A\&H) insurance providers. It includes pivotal financial attributes of these companies such as company name, year of data collection, total assets, liabilities, and premiums written. Spanning from 2013 to 2023, the dataset reflects a breadth of financial metrics, indicating the fiscal condition of these insurers and presents premium values categorized in monetary formats. This dataset aims to furnish insights into the financial landscape of the health insurance sector.
    \\\\
    \textcolor{teal}{\textbf{Key Themes or Topics:}} \\
    - Health Insurance Finance \\
    - Health Insurance Types (HMO, MCH, A\&H) \\
    - Financial Analysis in Insurance \\
    - Risk Assessment in Health Insurance \\
    - Insurance Premium Structures \\
    - Temporal Changes in Health Insurance \\
    \\
    \textcolor{teal}{\textbf{Applications and Use Cases:}} \\
    - Supports analysis for health insurance professionals and executives in evaluating financial health. \\
    - Assists policymakers in understanding the economic implications of health insurance. \\
    - Enables researchers to investigate trends and outcomes related to health insurance provisions. \\
    - Useful for actuarial studies and financial modeling pertaining to health insurance. \\
    - Facilitates comparative studies between different types of health insurance organizations. \\
    \\
    \textcolor{teal}{\textbf{Concepts and Synonyms:}} \\
    - Health Insurance/Medical Insurance \\
    - Financial Data/Financial Metrics \\
    - Insurance Providers/Insurers \\
    - Premium Income/Premiums Written \\
    - Liability Assessment/Financial Liabilities \\
    - Asset Valuation/Total Assets \\
    - Risk Management/Insurance Risk \\
    - Health Economics/Economic Aspects of Health Insurance \\
    - HMO/Managed Care Organizations \\
    \\
    \textcolor{teal}{\textbf{Keywords and Themes:}} \\
    - Health insurance dataset \\
    - Financial health of insurers \\
    - Insurance premiums \\
    - Assets and liabilities in insurance \\
    - Temporal financial data \\
    - Insurance sector analysis \\
    - HMO, MCH, A\&H \\
    - Company financial attributes \\
    - Health insurance trends \\
    \\
    \textcolor{teal}{\textbf{Additional Context:}} \\
    - The dataset is relevant for addressing current challenges and inquiries in the health insurance sector, especially regarding financial performance metrics during economic fluctuations. \\
    - Its integration with health policy research and economic studies can provide a deeper understanding of how financial factors influence healthcare access and coverage. \\
    - It can aid in developing frameworks for better risk management and investment strategies within the health insurance market. \\
    \bottomrule
    \end{tabular}
    \caption{Search Focused Description (SFD) Example for a dataset on health insurance financials.}
    \label{tab:example_sfd}
\end{table*}


% \section{Complete Template and Prompts}
% \subsection{Prompt of Semantic Profile}
% \label{app:prompt_semantic_profile}

\begin{table*}
    \small
    \centering
    \begin{tabular}{|p{3cm}|p{11cm}|}
    \hline
    \textbf{Category} & \textbf{Details} \\
    \hline
    \textbf{Temporal} & 
    \begin{itemize}
        \item \textbf{isTemporal}: Does this column contain temporal information? Yes or No.
        \item \textbf{resolution}: If Yes, specify the resolution (Year, Month, Day, Hour, etc.).
    \end{itemize} \\
    \hline
    \textbf{Spatial} & 
    \begin{itemize}
        \item \textbf{isSpatial}: Does this column contain spatial information? Yes or No.
        \item \textbf{resolution}: If Yes, specify the resolution (Country, State, City, Coordinates, etc.).
    \end{itemize} \\
    \hline
    \textbf{Entity Type} & What kind of entity does the column describe? (e.g., Person, Location, Organization, Product). \\
    \hline
    \textbf{Domain-Specific Types} & What domain is this column from? (e.g., Financial, Healthcare, E-commerce, Climate, Demographic). \\
    \hline
    \textbf{Function/Usage Context} & How might the data be used? (e.g., Aggregation Key, Ranking/Scoring, Interaction Data, Measurement). \\
    \hline
    \end{tabular}
    \caption{Template for classifying column data into semantic types.}
    \label{tab:template}
\end{table*}


\begin{table*}
    \small
    \centering
    \begin{tabular}{p{14cm}}
    \toprule
    \textbf{Response Example for Dataset Semantic Profiler} \\
    \midrule
    \textcolor{teal}{\textbf{Response:}} \\
    \texttt{\{} \\
    \quad \texttt{"Domain-Specific Types"}: \texttt{"General"}, \\
    \quad \texttt{"Entity Type"}: \texttt{"Temporal Entity"}, \\
    \quad \texttt{"Function/Usage Context"}: \texttt{"Aggregation Key"}, \\
    \quad \texttt{"Spatial"}: \texttt{\{} \\
    \qquad \texttt{"isSpatial"}: \texttt{false}, \\
    \qquad \texttt{"resolution"}: \texttt{""} \\
    \quad \texttt{\}}, \\
    \quad \texttt{"Temporal"}: \texttt{\{} \\
    \qquad \texttt{"isTemporal"}: \texttt{true}, \\
    \qquad \texttt{"resolution"}: \texttt{"Year"} \\
    \quad \texttt{\}} \\
    \texttt{\}} \\
    \bottomrule
    \end{tabular}
    \caption{Response example showing a valid JSON output for classifying column data into semantic types.}
    \label{tab:sp_prompt_response_example}
\end{table*}

\begin{table*}
    \small
    \centering
    \begin{tabular}{p{14cm}}
    \toprule
    \textbf{Dataset Semantic Profiler Prompt} \\
    \midrule
    \textcolor{teal}{\textbf{Instruction:}} \\
    You are a dataset semantic analyzer. Based on the column name and sample values, classify the column into multiple semantic types. \\

    \textcolor{teal}{\textbf{Categories:}} \\
    Please group the semantic types under the following categories: \\
    - \textbf{Temporal} \\
    - \textbf{Spatial} \\
    - \textbf{Entity Type} \\
    - \textbf{Data Format} \\
    - \textbf{Domain-Specific Types} \\
    - \textbf{Function/Usage Context} \\

    \textcolor{teal}{\textbf{Template Reference:}} \\
    Following is the template: \texttt{\{TEMPLATE\}} \\

    \textcolor{teal}{\textbf{Rules:}} \\
    \begin{enumerate}
        \item The output must be a valid JSON object that can be directly loaded by \texttt{json.loads}. Example response is \texttt{\{RESPONSE\_EXAMPLE\}}.
        \item All keys from the template must be present in the response.
        \item All keys and string values must be enclosed in \textbf{double quotes}.
        \item There must be no trailing commas.
        \item Use \textbf{booleans (true/false)} and numbers without quotes.
        \item Do not include any additional information or context in the response.
        \item If you are unsure about a specific category, you can leave it as an empty string.
    \end{enumerate} \\

    \textcolor{teal}{\textbf{Dynamic Parameters:}} \\
    - Column name: \texttt{\{column\_name\}} \\
    - Sample values: \texttt{\{sample\_values\}} \\

    \bottomrule
    \end{tabular}
    \caption{Prompt for the dataset semantic profiler.}
    \label{tab:prompt_semantic_profile}
\end{table*}


% \subsection{Template of Search-Focused Description}
% \label{app:template_sfd}
\begin{table*}
    \small
    \centering
    \begin{tabular}{p{14cm}}
    \toprule
    \textbf{Search-Focused Description Template} \\
    \midrule
    \textcolor{teal}{\textbf{Dataset Overview:}} \\
    - Please keep the exact initial description of the dataset as shown at the beginning of the prompt. \\
    \\
    \textcolor{teal}{\textbf{Key Themes or Topics:}} \\
    - Central focus on a broad area of interest (e.g., urban planning, socio-economic factors, environmental analysis). \\
    - Data spans multiple subtopics or related areas that contribute to a holistic understanding of the primary theme. \\
    \textbf{Example:} \\
    - theme1/topic1 \\
    - theme2/topic2 \\
    - theme3/topic3 \\
    \\
    \textcolor{teal}{\textbf{Applications and Use Cases:}} \\
    - Facilitates analysis for professionals, policymakers, researchers, or stakeholders. \\
    - Useful for specific applications, such as planning, engineering, policy formulation, or statistical modeling. \\
    - Enables insights into patterns, trends, and relationships relevant to the domain. \\
    \textbf{Example:} \\
    - application1/usecase1 \\
    - application2/usecase2 \\
    - application3/usecase3 \\
    \\
    \textcolor{teal}{\textbf{Concepts and Synonyms:}} \\
    - Includes related concepts, terms, and variations to ensure comprehensive coverage of the topic. \\
    - Synonyms and alternative phrases improve searchability and retrieval effectiveness. \\
    \textbf{Example:} \\
    - concept1/synonym1 \\
    - concept2/synonym2 \\
    - concept3/synonym3 \\
    \\
    \textcolor{teal}{\textbf{Keywords and Themes:}} \\
    - Lists relevant keywords and themes for indexing, categorization, and enhancing discoverability. \\
    - Keywords reflect the dataset's content, scope, and relevance to the domain. \\
    \textbf{Example:} \\
    - keyword1 \\
    - keyword2 \\
    - keyword3 \\
    \\
    \textcolor{teal}{\textbf{Additional Context:}} \\
    - Highlights the dataset's relevance to specific challenges or questions in the domain. \\
    - May emphasize its value for interdisciplinary applications or integration with related datasets. \\
    \textbf{Example:} \\
    - context1 \\
    - context2 \\
    - context3 \\
    \bottomrule
    \end{tabular}
    \caption{Dataset description template outlining key themes, applications, concepts, keywords, and contextual relevance.}
    \label{tab:template_sfd}
\end{table*}


% \subsection{Prompt of Reference-Free Evaluation}
% \label{app:ref_free_eval}
\begin{table*}
    \small
    \centering
    \begin{tabular}{p{14cm}}
    \toprule
    \textbf{Dataset Description Evaluation Guidelines} \\
    \midrule
    \textcolor{teal}{\textbf{Task:}} 
    You will be given one tabular dataset description. Your task is to rate the description on three metrics. \\
    Please make sure you read and understand these instructions carefully. Keep this document open while reviewing, and refer to it as needed. \\
    \\
    \textcolor{teal}{\textbf{Evaluation Criteria:}} \\
    \textbf{1. Completeness (1-10):} \\
    - Evaluates how thoroughly the dataset description covers essential aspects such as the scope of data, query workloads, summary statistics, and possible tasks or applications. \\
    - A high score indicates that the description provides a comprehensive overview, including details on dataset size, structure, fields, and potential use cases. \\
    \\
    \textbf{2. Conciseness (1-10):} \\
    - Measures the efficiency of the dataset description in conveying necessary information without redundancy. \\
    - A high score indicates that the description is succinct, avoiding unnecessary details while employing semantic types (e.g., categories, entities) to streamline communication. \\
    \\
    \textbf{3. Readability (1-10):} \\
    - Evaluates the logical flow and readability of the dataset description. \\
    - A high score suggests that the description progresses logically from one section to the next, creating a coherent and integrated narrative that facilitates understanding of the dataset. \\
    \\
    \textcolor{teal}{\textbf{Evaluation Steps:}} \\
    - Read the dataset description carefully and identify the main topic and key points. \\
    - Assign a score for each criterion on a scale of 1 to 10, where 1 is the lowest and 10 is the highest, based on the Evaluation Criteria. \\
    \\
    \textcolor{teal}{\textbf{Example Evaluations:}} \\
    \textbf{Example 1:} \\
    \textbf{Description:} The dataset provides information on alcohol-impaired driving deaths and occupant deaths across various states in the United States. It includes data for 51 states, detailing the number of alcohol-impaired driving deaths and occupant deaths, with values ranging from 0 to 3723 and 0 to 10406, respectively. Each entry also contains the state abbreviation and its geographical coordinates. The dataset is structured with categorical and numerical data types, focusing on traffic safety and casualty statistics. Key attributes include state names, death counts, and location coordinates, making it a valuable resource for analyzing traffic safety trends and issues related to impaired driving. \\
    \textbf{Evaluation Form (scores ONLY):} \\
    Completeness: 7, Conciseness: 9, Readability: 9 \\
    \\
    \textbf{Example 2:} \\
    \textbf{Description:} The dataset provides a comprehensive overview of traffic safety statistics across various states in the United States, specifically focusing on alcohol-impaired driving deaths and occupant deaths. It includes data from 51 unique states, represented by their two-letter postal abbreviations, such as MA (Massachusetts), SD (South Dakota), AK (Alaska), MS (Mississippi), and ME (Maine). Each entry in the dataset captures critical information regarding the number of alcohol-impaired driving deaths and the total occupant deaths resulting from traffic incidents. \\
    The column "Alcohol-Impaired Driving Deaths" is represented as an integer, indicating the number of fatalities attributed to alcohol impairment while driving. The dataset reveals a range of values, with the highest recorded number being 2367 deaths in Mississippi, highlighting the severity of the issue in certain regions. In contrast, states like Alaska report significantly lower figures, with only 205 alcohol-impaired driving deaths. \\
    The "Occupant Deaths" column also consists of integer values, representing the total number of deaths among vehicle occupants, regardless of the cause. This data spans from 0 to 10406, with Mississippi again showing the highest number of occupant deaths at 6100, which raises concerns about overall traffic safety in the state. \\
    Additionally, the dataset includes a "Location" column that provides geographical coordinates for each state, enhancing the spatial understanding of the data. The coordinates are formatted as latitude and longitude pairs, allowing for potential mapping and geographical analysis of traffic safety trends. \\
    Overall, this dataset serves as a valuable resource for researchers, policymakers, and public safety advocates aiming to understand and address the impact of alcohol on driving safety across different states. It highlights the need for targeted interventions and policies to reduce alcohol-impaired driving incidents and improve occupant safety on the roads. \\
    \textbf{Evaluation Form (scores ONLY):} \\
    Completeness: 8, Conciseness: 7, Readability: 8 \\
    \\
    \textcolor{teal}{\textbf{Final Evaluation Form:}} \\
    Please provide scores for the given dataset description based on the Evaluation Criteria. Do not include any additional information or comments in your response. \\
    \textbf{Evaluation Form (scores ONLY):} \\
    Completeness: \_\_, Conciseness: \_\_, Readability: \_\_ \\
    \bottomrule
    \end{tabular}
    \caption{Guidelines for evaluating dataset descriptions based on completeness, conciseness, and readability.}
    \label{tab:llm_eval_comp_conc_read}
\end{table*}

\begin{table*}
    \small
    \centering
    \begin{tabular}{p{14cm}}
    \toprule
    \textbf{Dataset Description Faithfulness Evaluation Prompt} \\
    \midrule
    \textcolor{teal}{\textbf{Task Description:}} \\
    Your task is to evaluate whether the given dataset description faithfully represents the facts from the provided dataset sample, dataset profile, and semantic profile. For each sentence in the description, verify whether it accurately reflects the information from the inputs. Carefully check step-by-step for every sentence to determine if it is true or false. Use Example 1 as a reference and respond to Example 2. \\
    \\
    \textcolor{teal}{\textbf{Example 1}} \\
    \textbf{Dataset Sample:} \\
    \texttt{Time Stamp, Average Wind Speed, Standard Deviation, Average Wind Direction} \\
    \texttt{7/22/2003 1:40, 1.8, 1.66, 315} \\
    \texttt{6/18/2003 10:50, 6.0, 1.44, 225} \\
    \\
    \textbf{Dataset Profile:} \\
    - **Time Stamp:** Data is of type text, 13,576 unique values, semantic type: datetime. \\
    - **Average Wind Speed:** Data is of type float, range: 0 to 30.2. \\
    - **Standard Deviation:** Data is of type float, range: 0 to 3.05. \\
    - **Average Wind Direction:** Data is of type integer, 16 unique values, range: 0 to 338.0. \\
    \\
    \textbf{Semantic Profile:} \\
    - **Time Stamp:** Represents temporal entity, resolution: Hour. \\
    - **Average Wind Speed:** Represents measurement, domain: climate, function: measurement. \\
    - **Standard Deviation:** Represents statistical measure, domain: general, function: measurement. \\
    - **Average Wind Direction:** Represents measurement, domain: climate, function: measurement. \\
    \\
    \textbf{Data Topic:} Wind Speed Data \\
    \\
    \textbf{Generated Description:} \\
    "This dataset contains 13,576 hourly records of wind speed and direction, with wind speed ranging from 0 to 30.2 m/s and wind direction values spanning from 0 to 360°. It includes detailed hourly timestamps as temporal entities, with wind speed and direction categorized as climate measurements. This dataset is useful for renewable energy planning, meteorological research, and understanding wind patterns." \\
    \\
    \textbf{Evaluation:} \\
    - **(Sentence 1):** "This dataset contains 13,576 hourly records of wind speed and direction, with wind speed ranging from 0 to 30.2 m/s and wind direction values spanning from 0 to 360°." \\
      **Explanation:** The dataset profile specifies 13,576 records, the correct wind speed range (0–30.2 m/s), and the correct timestamp type. However, the wind direction range is incorrectly stated as 0–360° instead of 0–338°. \\
      **Verification:** F \\
    - **(Sentence 2):** "It includes detailed hourly timestamps as temporal entities, with wind speed and direction categorized as climate measurements." \\
      **Explanation:** The semantic profile confirms that the timestamps are temporal entities, and wind speed and direction are categorized as climate measurements. \\
      **Verification:** T \\
    - **(Sentence 3):** "This dataset is useful for renewable energy planning, meteorological research, and understanding wind patterns." \\
      **Explanation:** These applications are reasonable inferences based on the dataset's focus on wind speed and direction. \\
      **Verification:** T \\
    \\
    \textcolor{teal}{\textbf{Example 2}} \\
    \textbf{Dataset Sample:} \texttt{\{dataset\_sample\}} \\
    \textbf{Dataset Profile:} \texttt{\{dataset\_profile\}} \\
    \textbf{Semantic Profile:} \texttt{\{semantic\_profile\}} \\
    \textbf{Data Topic:} \texttt{\{data\_topic\}} \\
    \textbf{Generated Description:} \texttt{\{dataset\_description\}} \\
    \\
    \textbf{Evaluation:} \\
    (Sentence 1): \texttt{\{\{sent1\}\}} \\
    (Explanation): \texttt{\{\{explanation\}\}} \\
    (Verification): T/F \\
    ... \\
    (Sentence n): \texttt{\{\{sent n\}\}} \\
    (Explanation): \texttt{\{\{explanation\}\}} \\
    (Verification): T/F \\
    \bottomrule
    \end{tabular}
    \caption{Prompt for evaluating dataset description faithfulness based on dataset sample, profile, and semantic profile.}
    \label{tab:faithfulness_evaluation}
\end{table*}


% \begin{table*}
%     \small
%     \centering
%     \begin{tabular}{p{14cm}}
%     \toprule
%     \textbf{Dataset Description Faithfulness Evaluation Prompt} \\
%     \midrule
%     \textcolor{teal}{\textbf{Task Description:}} \\
%     Your task is to evaluate whether the given dataset description faithfully represents the facts from the provided dataset sample, dataset profile, and semantic profile. For each sentence in the description, verify whether it accurately reflects the information from the inputs. Carefully check step-by-step for every sentence to determine if it is true or false. Use Example 1 as a reference and respond to Example 2. \\
%     \\
%     \textcolor{teal}{\textbf{Scoring Guidelines:}} \\
%     - **High Score (8-10):** All claims are explicitly supported by the dataset inputs or are reasonable inferences that logically follow from the data. Broader usage or applications are acceptable if plausible. \\
%     - **Moderate Score (5-7):** The description is mostly faithful but includes some loosely supported or slightly overstated claims. \\
%     - **Low Score (1-4):** Several claims are unsupported, speculative, or deviate significantly from the dataset inputs. \\
%     \\
%     \textcolor{teal}{\textbf{Example 1}} \\
%     \textbf{Dataset Sample:} \\
%     \texttt{Time Stamp, Average Wind Speed, Standard Deviation, Average Wind Direction} \\
%     \texttt{7/22/2003 1:40, 1.8, 1.66, 315} \\
%     \texttt{6/18/2003 10:50, 6.0, 1.44, 225} \\
%     \\
%     \textbf{Dataset Profile:} \\
%     - **Time Stamp:** Data type: text, 13,576 unique values, semantic type: datetime. \\
%     - **Average Wind Speed:** Data type: float, range: 0 to 30.2. \\
%     - **Standard Deviation:** Data type: float, range: 0 to 3.05. \\
%     - **Average Wind Direction:** Data type: integer, 16 unique values, range: 0 to 338.0. \\
%     \\
%     \textbf{Semantic Profile:} \\
%     - **Time Stamp:** Represents temporal entity, resolution: Hour. \\
%     - **Average Wind Speed:** Represents measurement, domain: climate, function: measurement. \\
%     - **Standard Deviation:** Represents statistical measure, domain: general, function: measurement. \\
%     - **Average Wind Direction:** Represents measurement, domain: climate, function: measurement. \\
%     \\
%     \textbf{Data Topic:} Wind Speed Data \\
%     \\
%     \textbf{Generated Description:} \\
%     "This dataset contains 13,576 hourly records of wind speed and direction, with wind speed ranging from 0 to 30.2 m/s and wind direction values spanning from 0 to 338°. It includes detailed hourly timestamps as temporal entities, with wind speed and direction categorized as climate measurements. This dataset is useful for renewable energy planning, meteorological research, and understanding wind patterns." \\
%     \\
%     \textbf{Evaluation:} \\
%     - **Faithfulness Score:** \textbf{8} \\
%     - **Explanation:** Most claims in the description are supported by the dataset profile and semantic profile. Statements about "predicting weather changes" and "renewable energy policies" are reasonable inferences based on the dataset's focus on wind speed and direction. However, "advancements in technology for wind energy generation" is speculative and reduces the score slightly. \\
%     \\
%     \textcolor{teal}{\textbf{Evaluation Task:}} \\
%     After reviewing the example above, evaluate the dataset description using the provided sample, profile, and semantic profile. Provide a detailed explanation for your rating. \\
%     \\
%     \textbf{Dataset Sample:} \texttt{\{dataset\_sample\}} \\
%     \textbf{Dataset Profile:} \texttt{\{dataset\_profile\}} \\
%     \textbf{Semantic Profile:} \texttt{\{semantic\_profile\}} \\
%     \textbf{Data Topic:} \texttt{\{data\_topic\}} \\
%     \textbf{Generated Description:} \texttt{\{dataset\_description\}} \\
%     \\
%     \textbf{Evaluation:} \\
%     - **Faithfulness Score (score only):** \_\_ \\
%     - **Explanation (in sentences):** \_\_\_\_ \\
%     \bottomrule
%     \end{tabular}
%     \caption{Prompt for evaluating dataset description faithfulness based on dataset sample, profile, and semantic profile.}
%     \label{tab:faithfulness_evaluation}
% \end{table*}


% \section{Experiment}

% \subsection{SPLADE Results}
% \label{app:experiment_splade}
\begin{table*}[]
    \centering
    \begin{tabular}{l|cccc|cccc}
        \hline
         & \multicolumn{4}{c|}{\bf ECIR-DDG} & \multicolumn{4}{c}{\bf NTCIR-DDG} \\
         \bf Model & \bf NDCG@5 & \bf NDCG@10 & \bf NDCG@15 & \bf NDCG@20 & \bf NDCG@5 & \bf NDCG@10 & \bf NDCG@15 & \bf NDCG@20 \\
        \hline
         Original & 0.833 & 0.826 & 0.820 & 0.822 & 0.614 & 0.698 & 0.734 & 0.754\\
         \hline
         Header+Sample & 0.808 & 0.794 & 0.802 & 0.823 & 0.445 & 0.561 & 0.628 & 0.665\\
         MVP & 0.722 & 0.706 & 0.687 & 0.696 & 0.427 & 0.524 & 0.606 & 0.649\\
         ReasTAP & 0.799 & 0.768 & 0.758 & 0.772 & 0.526 & 0.607 & 0.659 & 0.695\\
         \hline
         AutoDDG-UFD-GPT & \bf 0.889 & \bf 0.908 & \bf 0.934 & \bf 0.943 & 0.540 & 0.638 & 0.676 & 0.708\\
         AutoDDG-UFD-Llama & 0.876 & 0.897 & 0.922 & 0.933 & 0.569 & 0.651 & 0.706 & 0.735\\
         AutoDDG-SFD-GPT & \underline{0.886} & \underline{0.907} & \underline{0.931} & \underline{0.940} & \bf 0.644 & \bf 0.712 & \bf 0.773 & \bf 0.787\\
         AutoDDG-SFD-Llama & 0.877 & 0.892 & 0.921 & 0.932 & \underline{0.615} & \underline{0.709} & \underline{0.745} & \underline{0.772}\\
      \hline
    \end{tabular}
    \caption{SPLADE: Comparison of NDCG scores across different description generation models on the ECIR-DDG and NTCIR-DDG benchmarks. Higher NDCG scores indicate better alignment between the generated descriptions and the relevance of retrieved results for keyword-based search. Bold values represent the highest scores for each metric, while underlined values indicate the second highest. 
    }
    \label{tab:experiment_splade}
\end{table*}

% \subsection{Reference-free Evaluation}
% \label{app:experiment_reference_free}
% \begin{table*}[h]
\begin{center}

  \begin{tabular}{c|cccc|cccc}
    \hline
      & \multicolumn{4}{c|}{\bf Evaluation by GPT} & \multicolumn{4}{c}{\bf Evaluation by Llama} \\
    \bf Generation Model & \bf Comp & \bf Conc & \bf Read & \bf Faith & \bf Comp & \bf Conc & \bf Read & \bf Faith\\
    % \bf Generation Model & \bf Completeness & \bf Conciseness & \bf Readability & \bf Faithfulness & \bf Completeness & \bf Conciseness & \bf Readability & \bf Faithfulness\\
    \hline
    Original & 4.07 & 6.83 & 6.04 & 0.34 & 5.02 & 7.35 & 7.14 & 0.27\\
    \hline
    Header+Sample & 3.50 & 5.39 & 4.36 & 0.80 & 3.56 & 4.0 & 4.15 & 0.65\\
    MVP & 1.52 & 3.24 & 2.41 & 0.71 & 1.57 & 3.25 & 2.84 & 0.48\\
    ReasTAP & 1.96 & 4.38 & 3.33 & 0.60 & 1.91 & 5.81 & 4.85 & 0.50\\
    \hline
    AutoDDG-UFD-GPT & 8.19 & \bf 8.97 & \bf 8.99 & \underline{0.99} & 8.57 & \bf 8.00 & \bf 8.99 & 0.90\\
    AutoDDG-UFD-Llama & 7.72& \underline{8.51} & \underline{8.625} & 0.96 & 8.43 & \underline{7.86} & \underline{8.50} & 0.92\\
    AutoDDG-SFD-GPT & \bf 8.96 & 7.35 & 7.90  & \bf 0.99 & \bf 8.98 & 5.97 & 7.00 & \underline{0.93}\\
    AutoDDG-SFD-Llama & \underline{8.95} & 6.24 & 7.19 & 0.97 & \underline{8.89} & 5.56 & 6.79 & \bf 0.95\\
  \hline
\end{tabular}
\end{center}
\caption{Evaluation of dataset description quality on reference-free metrics (Comp: Completeness, Conc: Conciseness, Read: Readability, Faith: Faithfulness) for all models, including the original description. Completeness, Conciseness, and Readability are scaled from 0 to 10, while Faithfulness is scaled from 0 to 1. Higher values indicate better performance. Bold values represent the highest scores for each metric, while underlined values represent the second highest.}
  \label{tab:eval_reference_free}
\vspace{-.3cm}
\end{table*}

% \begin{table}[h]
% \begin{center}
%   \begin{tabular}{c|cccc}
%     \hline
%     \bf Model & \bf Comp & \bf Conc & \bf Read & \bf Faith \\
%     % \bf Model & \bf Completeness & \bf Conciseness & \bf Readability & \bf Faithfulness \\
%     \hline
%     % Original & 4.07/5.02 & 6.83/7.35 & 6.04/7.14 & 0.34/0.27 \\
%     Original & 4.07 & 6.83 & 6.04 & 0.34 \\
%     \hline
%     Header+Sample & 3.50 & 5.39 & 4.36 & 0.80 \\
%     MVP & 1.52 & 3.24 & 2.41 & 0.71 \\
%     ReasTAP & 1.96 & 4.38 & 3.33 & 0.60 \\
%     \hline
%     AutoDDG-UFD-GPT & 8.19 & \bf 8.97 & \bf 8.99 & \underline{0.99} \\
%     AutoDDG-UFD-Llama & 7.72& \underline{8.51} & \underline{8.63} & 0.96 \\
%     AutoDDG-SFD-GPT & \bf 8.96 & 7.35 & 7.90  & \bf 0.99 \\
%     AutoDDG-SFD-Llama & \underline{8.95} & 6.24 & 7.19 & 0.97 \\
%   \hline
% \end{tabular}
% \end{center}
% \caption{Evaluation of dataset description quality on reference-free metrics (Comp: Completeness, Conc: Conciseness, Read: Readability, Faith: Faithfulness). Completeness, Conciseness, and Readability are scaled from 0 to 10, while Faithfulness is scaled from 0 to 1. Higher values indicate better performance. Bold values represent the highest scores for each metric, while underlined values represent the second highest.}
%   \label{tab:eval_reference_free_gpt}
% \vspace{-.77cm}
% \end{table}

% \subsection{Ablation Study on Word Limit}
% \label{app:experiment_ablation_word_limit}

% We compare the following variations of \SystemName to assess the effects of word limits on retrieval performance. 

% \noindent \textbf{\SystemName-noSP-noSFD}: Descriptions generated by \SystemName without the semantic profile (SP) and the search-focused description (SFD) expansion module. In this setting, the generated description relies solely on basic dataset sample and statistics.

% \noindent \textbf{\SystemName-noSFD}: Descriptions generated by \SystemName with the semantic profile (SP) module but without the search-focused description (SFD) expansion module. This setting emphasizes enhancing the description using semantic insights but does not optimize for search-related retrieval.

% \noindent \textbf{\SystemName}: Descriptions generated by \SystemName with both the semantic profile (SP) module and the search-focused description (SFD) expansion module. 
% % 
% % As discussed in the previous section, this configuration simulates a scenario where a user aims to publish a dataset with an expected topic $T$ and seeks to maximize its retrieval likelihood when a query related to $T$ is issued. 
% % 
% Note that when testing different word limits, the initial description provided to the SFD module adheres to the specified word limit, while the SFD expansion itself does not impose any additional word limit constraints.

% \subsubsection{Impact of Word Limit on Retrieval Performance}
% Figure~\ref{fig:ablation_different_versions} demonstrates the influence of word limits (100, 300, and 500 words) on retrieval performance across different configurations of \SystemName. Increasing the word limit improves NDCG scores consistently, as longer descriptions provide richer context for matching queries. The improvement is significant for \SystemName-noSP-noSFD and \SystemName-noSFD when increasing from 100 to 300 words, highlighting the importance of contextual detail for retrieval. However, the gap between 300 and 500 words is smaller, indicating diminishing returns as descriptions reach sufficient detail.
% % 
% The improvement is small for \SystemName with SFD. This implies that SFD is particularly effective in scenarios with shorter initial descriptions, where it compensates for the lack of detail and ensures alignment with user search intent. 

% \begin{figure*}
% 	\centering
% 	\begin{subfigure}{0.66\columnwidth} 
% 		\centering
% 		\includegraphics[width=1.0\linewidth]{figures/ndcg_basic_ufd_run.png}
% 		\caption{\SystemName-noSP-noSFD}
%         \label{fig:ablation_noSTA}
% 	\end{subfigure}
%         \begin{subfigure}{0.66\columnwidth} 
% 		\centering
% 		\includegraphics[width=1.0\linewidth]{figures/ndcg_ufd_sta_run.png}
% 		\caption{\SystemName-noSFD}
%         \label{fig:ablation_ufd}
% 	\end{subfigure}
%         \begin{subfigure}{0.66\columnwidth} 
% 		\centering
% 		\includegraphics[width=1.0\linewidth]{figures/ndcg_ufd_sta_sfd_run.png}
% 		\caption{\SystemName}
%         \label{fig:ablation_sfd}
% 	\end{subfigure}
%     \vspace{-1.0em}
% 	\caption{Comparison of NDCG scores across different word different word limits (100, 300, and 500 words) on BM25. The results demonstrate the influence of word count on retrieval effectiveness.}
% 	\label{fig:ablation_different_versions}
%     \vspace{-1.2em}
% \end{figure*}


