\section{Related Work}
\paragraph{Reasoning and Generalization}
    Over the years the study of reasoning with learning based approaches has often focused on the domain of games such as chess, shogi and Go~\citep{lai2015giraffe,silver2016alphago,silver2017mastering, silver2018general}, Poker \citep{dahl2001reinforcement,heinrich2016deep,steinberger2019pokerrl, zhao2022alphaholdem} or board games \citep{ghory2004reinforcement, szita2012reinforcement, xenou2019deep, perolat2022mastering}.
    While these mainly focus on correct play or in-distribution performance, the CLRS Algorithmic Reasoning Benchmark introduced by \citet{Velickovic2022} puts emphasis on generalizable reasoning. It consists of a diverse collection of well known algorithms collected from the textbook ``Introduction to Algorithms'' by~\citet{cormen2022clrs} providing a resource to assess algorithmic reasoning for learning based approaches. Moreover, there exists a CLRS-Text \cite{markeeva2024clrstextalgorithmicreasoninglanguage} to better assess the reasoning capabilities from a language perspective as well. 
    \citet{abbe2024generalization} provide a more theoretically supported view on the generalization performance of some classes of neural networks, trained with SGD.
    They specifically focus on Boolean functions, learned in a supervised training regime.
    

\paragraph{Graph Neural Networks} First introduced by the works of \citet{scarselli2008graph}, Graph Neural Networks have seen a recent emergence through a variety of new architecture types and applications \cite{kipf2017semisupervisedclassificationgraphconvolutional, xu2019powerfulgraphneuralnetworks, veličković2018graphattentionnetworks}. The graph-based representation underlying these models is particularly powerful as it provides a natural framework for capturing relational information and structural dependencies between entities. This has made GNNs especially interesting for tackling combinatorial optimization problems \cite{dai2018learningcombinatorialoptimizationalgorithms, cappart2022combinatorialoptimizationreasoninggraph, anycsp} and reasoning tasks that require understanding relationships between multiple elements \cite{battaglia2018relationalinductivebiasesdeep}. A key advantage of graph-based approaches is their capability to handle problems of varying sizes and complexities. One specific research direction focuses how these models generalize across different problem instances and sizes \cite{xu2021neuralnetworksextrapolatefeedforward, schwarzschild2021learnalgorithmgeneralizingeasy}, with benchmarks like CLRS \cite{Velickovic2022} providing a more systematic evaluation frameworks for assessing algorithmic reasoning capabilities with a focus on size generalization. This in turn has sparked more investigations into developing appropriate tools and architectures for such reasoning \cite{ibarz2022generalistneuralalgorithmiclearner,numeroso2023dualalgorithmicreasoning,minder2023salsaclrssparsescalablebenchmark,mahdavitowards,bohde2024markovpropertyneuralalgorithmic,müller2024principledgraphtransformers}.