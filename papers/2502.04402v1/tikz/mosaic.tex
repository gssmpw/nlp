\begin{tikzpicture}
    % Image minipage
    \node at (0, 0) {%
        \begin{minipage}{0.2\linewidth}
            \includegraphics[width=\linewidth]{pacs_sources/Images/mosaic_for_graph.png}
        \end{minipage}
    };

    % Arrow
    \draw[->, thick] (2.5, 0) -- (3, 0);

    % TikZ diagram minipage
    \node at (8.5, 0) {%
        \begin{minipage}{0.6\linewidth}
        \begin{tikzpicture}

            % Define styles for the nodes
            \tikzstyle{node_style}=[circle,draw,fill=cyan!20,minimum size=13pt,inner sep=0pt]
            \tikzstyle{special_style}=[circle,draw,fill=black!50,minimum size=13pt,inner sep=0pt]
            \tikzstyle{highlight_style}=[circle,thick, draw=black,minimum size=13.5pt,inner sep=0pt, outer sep=2pt,fill=black!50, font = \tiny]
            \tikzstyle{num_node_style_black}=[circle,draw,fill=black!50,minimum size=13pt,inner sep=0pt, font = \tiny]
            \tikzstyle{num_node_style_white}=[circle,draw,fill=red!20,minimum size=13pt,inner sep=0pt, font = \tiny]
            \tikzstyle{num_node_style_cyan}=[circle,draw,fill=cyan!20,minimum size=13pt,inner sep=0pt, font = \tiny]

            % Create the 4x4 grid of nodes
            \foreach \x in {1,2,3,4} {
                \foreach \y in {1,2,3,4} {
                    \node[node_style] (N-\x-\y) at (\x,-\y) {}; % Define a node at position (\x, -\y)
                }
            }

            % Highlight special nodes
            \node[num_node_style_black] at (1,-1) {0};
            \node[num_node_style_black] at (1,-2) {}; % Special node at (1, -2)
            \node[num_node_style_black] at (1,-3) {2}; % Special node at (2, -2)
            \node[num_node_style_black] at (2,-4) {}; % Special node at (2, -3)
            \node[num_node_style_black] at (3,-4) {};
            \node[num_node_style_black] at (4,-4) {4};
            \node[num_node_style_black] at (4,-3) {4};
            \node[num_node_style_black] at (4,-2) {};
            % \node[num_node_style_black] at (2,-3) {};
            \node[num_node_style_cyan] at (2,-2) {3};
            \node[num_node_style_cyan] at (3,-2) {5};
            \node[num_node_style_cyan] at (3,-3) {5};
            

            % Draw horizontal and vertical dashed connections between adjacent nodes
            \foreach \x in {1,2,3} { % Stop at 3 to prevent out-of-bounds connections
                \foreach \y in {1,2,3,4} {
                    \draw[] (N-\x-\y) -- (N-\the\numexpr\x+1\relax-\y); % Draw dashed line to the right
                }
            }

            \foreach \x in {1,2,3,4} {
                \foreach \y in {1,2,3} { % Stop at 3 to prevent out-of-bounds connections
                    \draw[] (N-\x-\y) -- (N-\x-\the\numexpr\y+1\relax); % Draw dashed line downward
                }
            }
            %diagonal edges
            \foreach \x in {1,2,3} {
                \foreach \y in {1,2,3} { % Stop at 3 to prevent out-of-bounds connections
                    \draw[] (N-\x-\y) -- (N-\the\numexpr\x+1\relax-\the\numexpr\y+1\relax); % Draw dashed line downward
                }
            } 
            \foreach \x in {2,3,4} {
                \foreach \y in {1,2,3} { % Stop at 3 to prevent out-of-bounds connections
                    \draw[] (N-\x-\y) -- (N-\the\numexpr\x-1\relax-\the\numexpr\y+1\relax); % Draw dashed line downward
                }
            }             
            % Highlight the specific node and connect it to the vector of numbers
            \node[highlight_style] (highlighted) at (4,-4) {4};

            % Draw the straight connecting line to the vector
            \draw[black, dotted, thick] (highlighted) to[out=0, in=180] ++(1.6,1.5) ;

            % Draw the vector of numbers
            % \node[right] at (5.5,-2.5) {%
            %     \begin{tikzpicture}[baseline]
            %         \node[anchor=base west,draw,minimum height=1em,inner sep=2pt] (vector) {%
            %             \begin{tabular}{|c|}
            %                 \hline
            %                 0 \\ 
            %                 1 \\ 
            %                 0 \\ \hline
            %                 1 \\ \hline
            %                 0 \\ \hline
            %                 1 \\ \hline
            %                 .56 \\ \hline
            %             \end{tabular}
            %         };
            %         % Add small arrows and words to the right of each component
            %         \foreach \i/\text in {0/Is state unmarked, 1/Is state marked, 2/Is state blank, 3/Has clue number, 4/Is solved, 5/Is violated, 6/Clue number normilized} {
            %             \draw[->, thin] (vector.north east) ++(0,-\i*1.62em-0.5em) -- ++(1em,0) node[right, font=\small] {\text};
            %         }
                    
            %     \end{tikzpicture}
            % };

            \node[anchor=west] (vectorNode) at (5.5,-2.5) {
                \begin{tikzpicture}[baseline]
                    % Define the matrix (vector table)
                    \matrix (vector) [matrix of nodes,
                                      nodes={draw, minimum height=1.5em, anchor=west, minimum width=1.5em, font=\scriptsize},
                                      nodes in empty cells,
                                      draw,
                                      inner sep=1pt,
                                      column sep=0pt,
                                      row sep=0pt,
                                      ] {
                            0 \\ 
                            1 \\ 
                            0 \\ 
                            1 \\ 
                            0 \\ 
                            1 \\ 
                            .56 \\ 
                    };
                    
                    % Add labels to the right of each component
                    \def\labelScale{0.8}
                    \foreach \i/\text in {1/Is state unmarked, 2/Is state marked, 3/Is state blank, 4/Has clue number, 5/Is solved, 6/Is violated, 7/Clue number normilized} {
                        % Draw an arrow from each row to its label
                        \draw[->, thin] (vector-\i-1.east) -- ++(1em, 0) 
                            node[right, font=\small, anchor=west, scale=\labelScale] {\text};
                    };
                \end{tikzpicture}
            };

        \end{tikzpicture}
        \end{minipage}
    };

\end{tikzpicture}