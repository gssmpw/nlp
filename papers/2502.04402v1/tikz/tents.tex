
% \begin{adjustbox}{width=\columnwidth,left}
% \begin{tikzpicture}
%     % Include the image at the origin
%     \node (image) at (0, 0) {
%         \includegraphics[width=0.2\columnwidth]{pacs_sources/Images/tents_for_graph.png}
%     };

%     % Draw arrow relative to the image
%     \draw[->, thick] (image.east) -- ++(1em, 0);

%     % Begin a new scope shifted relative to the image
%     \begin{scope}[shift={(image.north east)}, xshift=1em, yshift=1em]
%         % Define styles for the nodes
%         \tikzstyle{node_style}=[circle,draw,fill=blue!20,minimum size=8pt,inner sep=0pt]
%         \tikzstyle{special_style}=[circle,draw,fill=green!40,minimum size=8pt,inner sep=0pt]
%         \tikzstyle{highlight_style}=[circle,thick, draw=black,minimum size=8pt,inner sep=0pt, outer sep=1pt,fill=cyan!20]
%         \tikzstyle{meta_node_style}=[circle, draw=black,minimum size=8pt,inner sep=0pt, outer sep=1pt,fill=black!20, font=\scriptsize]

%         % Create the 4x4 grid of nodes using relative positioning
%         \foreach \x in {1,2,3,4} {
%             \foreach \y in {1,2,3,4} {
%                 \coordinate (pos\x\y) at ($ (\x*1.5em, -\y*1.5em) $);
%                 \node[node_style] (N-\x-\y) at (pos\x\y) {};
%             }
%         }

%         % Draw horizontal connections
%         \foreach \x in {1,2,3} {
%             \foreach \y in {1,2,3,4} {
%                 \draw (N-\x-\y) -- (N-\the\numexpr\x+1\relax-\y);
%             }
%         }

%         % Draw vertical connections
%         \foreach \x in {1,2,3,4} {
%             \foreach \y in {1,2,3} {
%                 \draw (N-\x-\y) -- (N-\x-\the\numexpr\y+1\relax);
%             }
%         }

%         % Add meta nodes (columns)
%         \foreach \x/\num in {1/1, 2/1, 3/0, 4/1} {
%             \coordinate (metaCol\x) at ($(pos\x1) + (0, 1.5em)$);
%             \node[meta_node_style] (N-\x-0) at (metaCol\x) {\num};
%             \draw[dotted, thick] (N-\x-0.south) -- (N-\x-1.north);
%         }

%         % Add meta nodes (rows)
%         \foreach \y/\num in {1/1, 2/1, 3/0, 4/1} {
%             \coordinate (metaRow\y) at ($(pos4\y) + (1.5em, 0)$);
%             \node[meta_node_style] (N-5-\y) at (metaRow\y) {\num};
%             \draw[dotted, thick] (N-5-\y.west) -- (N-4-\y.east);
%         }

%         \foreach \x in {1,2,3,4} {
%             \foreach \y in {2,3,4} {
%                 % Calculate control point for smooth curve
%                 \pgfmathsetmacro{\ymid}{(\y + 0)/2}
%                 \pgfmathsetmacro{\xmid}{\x - 0.15*\y}
%                 \coordinate (controlPoint) at ($ (\xmid*1.2em, -\ymid*1.2em) $);
%                 % Draw the smooth curved line
%                 \draw [dashed, dash pattern=on 2pt off 1pt] 
%                     (N-\x-0.south) .. controls (controlPoint) .. (N-\x-\y.north);
%             }
%         }


%             % \foreach \x in {1,2,3,4} {
%             %     \foreach \y in {2,3,4} {
%             %         % Calculate the mid-point
%             %         \pgfmathsetmacro{\ymid}{(\y + 0)/2}
%             %         \pgfmathsetmacro{\xmid}{\x - 0.15*\y}
%             %         % \pgfmathsetmacro{\ymid}{1}
%             %         % Draw the smooth curved line
%             %         \draw [dashed,dash pattern=on 2pt off 1pt, xshift=0cm] plot [smooth, tension=1] coordinates { (N-\x,0) (\xmid, -\ymid) (N-\x,-\y.north)};
%             %     }
%             % }
        
%         \foreach \y in {1,2,3,4} {
%             \foreach \x in {1,2,3} {
%                 % Calculate control point for smooth curve
%                 \pgfmathsetmacro{\xmid}{(\x + 5)/2}
%                 \pgfmathsetmacro{\ymid}{\y + 0.15*(5 - \x)}
%                 \coordinate (controlPoint) at ($ (\xmid*1.2em, -\ymid*1.2em) $);
%                 % Draw the smooth curved line
%                 \draw [dashed, dash pattern=on 2pt off 1pt]
%                     (N-5-\y.west) .. controls (controlPoint) .. (N-\x-\y.east);
%             }
%         }


%         % Highlight special nodes
%         \node[special_style] at (pos31) {}; % Adjust positions as needed
%         \node[special_style] at (pos13) {};
%         \node[special_style] at (pos43) {};

%         % Highlighted node
%         \node[highlight_style] (highlighted) at (pos44) {};

%         % Draw the dotted line from the highlighted node to the vector
%         \coordinate (vectorPosition) at ($(highlighted.east) + (2.5em, 2em)$);
%         \draw[black, dotted, thick] (highlighted.east) to[out=0, in=180] (vectorPosition);

%         % Vector of numbers using a TikZ matrix
%         \node[anchor=west] (vectorNode) at (vectorPosition) {
%             \begin{tikzpicture}[baseline, scale=0.9]
%                 % Define the matrix (vector table)
%                 \matrix (vector) [matrix of nodes,
%                                   nodes={draw, minimum height=1em, anchor=west, minimum width=1.5em, font=\scriptsize},
%                                   nodes in empty cells,
%                                   draw,
%                                   inner sep=1pt,
%                                   column sep=0pt,
%                                   row sep=0pt,
%                                   ] {
%                     0 \\ % Row 1
%                     1 \\ % Row 2
%                     0 \\ % Row 3
%                     0 \\ % Row 4
%                     0 \\ % Row 5
%                     0 \\ % Row 6
%                     0.0 \\ % Row 7
%                 };

%                 % Add labels to the right of each component
%                 \def\labelScale{0.8}
%                 \foreach \i/\text in {
%                     1/Is empty, 
%                     2/Is grass, 
%                     3/Is tent, 
%                     4/Is tree, 
%                     5/Is violated, 
%                     6/Is meta node, 
%                     7/Norm. num. meta node} {
%                     % Draw an arrow from each row to its label
%                     \draw[->, thin] (vector-\i-1.east) -- ++(1em, 0) 
%                         node[right, font=\small, anchor=west, scale=\labelScale] {\text};
%                 }
%             \end{tikzpicture}
%         };
%     \end{scope}
% \end{tikzpicture}
% \end{adjustbox}




%OLD



% \begin{adjustbox}{width=\columnwidth,left}
\begin{tikzpicture}
    % Image minipage
    \node at (0, 0) {%
        \begin{minipage}{0.2\linewidth}
            \includegraphics[width=\linewidth]{pacs_sources/Images/tents_for_graph.png}
        \end{minipage}
    };
    % Arrow
    \draw[->, thick] (2.5, 0) -- (3, 0);
    % TikZ diagram minipage
    \node at (8.6, 0) {%
        \begin{minipage}{0.6\linewidth}    
        \begin{tikzpicture}

            % Define styles for the nodes
            \tikzstyle{node_style}=[circle,draw,fill=blue!20,minimum size=13pt,inner sep=0pt]
            \tikzstyle{special_style}=[circle,draw,fill=green!40,minimum size=13pt,inner sep=0pt]
            \tikzstyle{highlight_style}=[circle,thick, draw=black,minimum size=13.5pt,inner sep=0pt, outer sep=2pt,fill=cyan!20]
            \tikzstyle{meta_node_style}=[circle, draw=black,minimum size=13.5pt,inner sep=0pt, outer sep=2pt,fill=black!20, font=\tiny]

            % Create the 4x4 grid of nodes
            \foreach \x in {1,2,3,4} {
                \foreach \y in {1,2,3,4} {
                    \node[node_style] (N-\x-\y) at (\x,-\y) {}; % Define a node at position (\x, -\y)
                }
            }


            % Draw horizontal and vertical dashed connections between adjacent nodes
            \foreach \x in {1,2,3} { % Stop at 3 to prevent out-of-bounds connections
                \foreach \y in {1,2,3,4} {
                    \draw[] (N-\x-\y) -- (N-\the\numexpr\x+1\relax-\y); % Draw dashed line to the right
                }
            }

            \foreach \x in {1,2,3,4} {
                \foreach \y in {1,2,3} { % Stop at 3 to prevent out-of-bounds connections
                    \draw[] (N-\x-\y) -- (N-\x-\the\numexpr\y+1\relax); % Draw dashed line downward
                }
            }

            \foreach \y in {1,2,3,4}{
                \node[meta_node_style] (N-5-\y) at (5,-\y) {};
            }
            \foreach \x in {1,2,3,4}{
                \node[meta_node_style] (N-\x-0) at (\x,0) {};
            }            
            % Connect upper meta nodes to nodes in the first row
            \foreach \x in {1,2,3,4} {
                \draw[dotted, thick] (N-\x-0) to[out=270, in=90] (N-\x-1); % Connect upper meta nodes to the first row
            }         
            % Connect right meta nodes to nodes in the corresponding column
            \foreach \y in {1,2,3,4} {
                \draw[dotted, thick] (N-5-\y) to[out=180, in=0] (N-4-\y); % Connect right meta nodes to the fourth column
            }
            % Highlight the specific node and connect it to the vector of numbers
            \node[highlight_style] (highlighted) at (4,-4) {};


            % Connect upper meta nodes to nodes in the first row with curved lines
            \foreach \x in {1,2,3,4} {
                \foreach \y in {2,3,4} {
                    % Calculate the mid-point
                    \pgfmathsetmacro{\ymid}{(\y + 0)/2}
                    \pgfmathsetmacro{\xmid}{\x - 0.15*\y}
                    % \pgfmathsetmacro{\ymid}{1}
                    % Draw the smooth curved line
                    \draw [dashed,dash pattern=on 2pt off 1pt, xshift=0cm] plot [smooth, tension=1] coordinates { (\x,0) (\xmid, -\ymid) (\x,-\y)};
                }
            }
            \foreach \y in {1,2,3,4} {
                \foreach \x in {1,2,3} {
                    % Calculate the mid-point
                    \pgfmathsetmacro{\xmid}{(\x + 5)/2}
                    \pgfmathsetmacro{\ymid}{\y + 0.15*(5-\x)}
                    % \pgfmathsetmacro{\ymid}{1}
                    % Draw the smooth curved line
                    \draw [dashed,dash pattern=on 2pt off 1pt, xshift=0cm] plot [smooth, tension=1] coordinates { (5,-\y) (\xmid, -\ymid) (\x,-\y)};
                }
            }    


            %REDRAW THE CIRCLES TO OVERLAP WITH DASHED LINES
            % Create the 4x4 grid of nodes
            \foreach \x in {1,2,3,4} {
                \foreach \y in {1,2,3,4} {
                    \node[node_style] (N-\x-\y) at (\x,-\y) {}; % Define a node at position (\x, -\y)
                }
            }
            % Highlight special nodes
            \node[special_style] at (3,-1) {}; % Special node at (1, -2)
            \node[special_style] at (1,-3) {}; % Special node at (2, -2)
            \node[special_style] at (4,-3) {}; % Special node at (2, -3)


            %numbers on metanodes
            \node[meta_node_style] at (1,0) {1};
            \node[meta_node_style] at (2,0) {1};
            \node[meta_node_style] at (3,0) {0};
            \node[meta_node_style] at (4,0) {1};
            \node[meta_node_style] at (5,-1) {1};
            \node[meta_node_style] at (5,-2) {1};
            \node[meta_node_style] at (5,-3) {0};
            \node[meta_node_style] at (5,-4) {1}; 
            
            % Draw the  connecting line to the vector
            \draw[black, dotted, thick] (highlighted) to[out=0, looseness = 0.5, in=180] ++(2.6,1.2) ;

            % Draw the vector of numbers
            % \node[right] at (6.5,-2.5) {%
            % Vector of numbers using a TikZ matrix
            \node[right] (vectorNode) at (6.5, -2.5) {
                \begin{tikzpicture}[baseline]
                    % Define the matrix (vector table)
                    \matrix (vector) [matrix of nodes,
                                      nodes={draw, minimum height=1.5em, anchor=west, minimum width=1.5em, font=\scriptsize},
                                      nodes in empty cells,
                                      draw,
                                      inner sep=1pt,
                                      column sep=0pt,
                                      row sep=0pt,
                                      ] {
                        0 \\ % Row 1
                        1 \\ % Row 2
                        0 \\ % Row 3
                        0 \\ % Row 4
                        0 \\ % Row 5
                        0 \\
                        1.1 \\ % Row 7
                    };
                    
                    % Add labels to the right of each component
                    \def\labelScale{0.8}
                    \foreach \i/\text in {
                        1/Is empty, 
                        2/Is grass, 
                        3/Is tent, 
                        4/Is tree, 
                        5/Is violated, 
                        6/Is meta node,
                        7/Norm. num. meta node} {
                        % Draw an arrow from each row to its label
                        \draw[->, thin] (vector-\i-1.east) -- ++(1em, 0) 
                            node[right, font=\small, anchor=west, scale=\labelScale] {\text};
                    }
                \end{tikzpicture}
            };

        \end{tikzpicture}

        \end{minipage}
    };      
    
\end{tikzpicture}
% \end{adjustbox}

