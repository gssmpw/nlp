
\begin{tikzpicture}
    % Image minipage
    \node at (0, 0) {%
        \begin{minipage}{0.2\linewidth}
            \includegraphics[width=\linewidth]{pacs_sources/Images/net_for_graph.png}
        \end{minipage}
    };
    % Arrow
    \draw[->, thick] (2.5, 0) -- (3, 0);
    % TikZ diagram minipage
    \node at (8.5, 0) {%
        \begin{minipage}{0.6\linewidth}
        \begin{tikzpicture}

            % Define styles for the nodes
            \tikzstyle{node_style}=[circle,draw,fill=gray!30,minimum size=13pt,inner sep=0pt]
            \tikzstyle{special_style}=[circle,draw,fill=cyan!20,minimum size=13pt,inner sep=0pt]
            \tikzstyle{highlight_style}=[circle,thick, draw=black,minimum size=13.5pt,inner sep=0pt, outer sep=2pt,fill=cyan!20]

            % Create the 4x4 grid of nodes
            \foreach \x in {1,2,3,4} {
                \foreach \y in {1,2,3,4} {
                    \node[node_style] (N-\x-\y) at (\x,-\y) {}; % Define a node at position (\x, -\y)
                }
            }

            % Highlight special nodes
            \node[special_style] at (1,-2) {}; % Special node at (1, -2)
            \node[special_style] at (2,-2) {}; % Special node at (2, -2)
            \node[special_style] at (2,-3) {}; % Special node at (2, -3)

            % Draw horizontal and vertical dashed connections between adjacent nodes
            \foreach \x in {1,2,3} { % Stop at 3 to prevent out-of-bounds connections
                \foreach \y in {1,2,3,4} {
                    \draw[] (N-\x-\y) -- (N-\the\numexpr\x+1\relax-\y); % Draw  line to the right
                }
            }

            \foreach \x in {1,2,3,4} {
                \foreach \y in {1,2,3} { % Stop at 3 to prevent out-of-bounds connections
                    \draw[] (N-\x-\y) -- (N-\x-\the\numexpr\y+1\relax); % Draw dashed line downward
                }
            }
            % Highlight the specific node and connect it to the vector of numbers
            \node[highlight_style] (highlighted) at (3,-3) {};

            % Draw the straight connecting line to the vector
            \draw[black, dotted, thick] (highlighted) to[out=0, in=180] ++(2.6,0.5) ;

            % Draw the vector of numbers
            % \node[right] at (5.5,-2.5) {%
            \node[anchor=west] (vectorNode) at (5.5, -2.5) {
                \begin{tikzpicture}[baseline]
                    % Define the matrix (vector table)
                    \matrix (vector) [matrix of nodes,
                                      nodes={draw, minimum height=1.5em, anchor=west, minimum width=1.5em, font=\scriptsize},
                                      nodes in empty cells,
                                      draw,
                                      inner sep=1pt,
                                      column sep=0pt,
                                      row sep=0pt,
                                      ] {
                        1 \\ % Row 1
                        1 \\ % Row 2
                        0 \\ % Row 3
                        1 \\ % Row 4
                        1 \\ % Row 5
                        1.1 \\ % Row 6
                    };
                    
                    % Add labels to the right of each component
                    \def\labelScale{0.8}
                    \foreach \i/\text in {
                        1/Line above, 
                        2/Line left, 
                        3/Line below, 
                        4/Line right, 
                        5/Is source, 
                        6/Connected to source} {
                        % Draw an arrow from each row to its label
                        \draw[->, thin] (vector-\i-1.east) -- ++(1em, 0) 
                            node[right, font=\small, anchor=west, scale=\labelScale] {\text};
                    }
                \end{tikzpicture}
            };

        \end{tikzpicture}
        \end{minipage}
    };

\end{tikzpicture}








% % Wrap the TikZ picture with adjustbox and align left
% \begin{adjustbox}{width=\columnwidth,left}
% \begin{tikzpicture}
%     % Shift the entire content if necessary
%     % \begin{scope}[shift={(-\xoffset, -\yoffset)}]

%         % Include the image at the origin
%         \node (image) at (0, 0) {
%             \includegraphics[width=0.2\linewidth]{pacs_sources/Images/net_for_graph.png}
%         };

%         % Draw arrow relative to the image
%         \draw[->, thick] (image.east) -- ++(1em, 0);

%         % Begin a new scope shifted relative to the image
%         \begin{scope}[shift={(image.north east)}, xshift=1em, yshift = 0.1em]
%             % Define styles for the nodes
%             \tikzstyle{node_style}=[circle,draw,fill=gray!30,minimum size=8pt,inner sep=0pt]
%             \tikzstyle{special_style}=[circle,draw,fill=cyan!20,minimum size=8pt,inner sep=0pt]
%             \tikzstyle{highlight_style}=[circle,thick, draw=black,minimum size=8pt,inner sep=0pt, outer sep=1pt,fill=cyan!20]

%             % Create the 4x4 grid of nodes using relative positioning
%             \foreach \x in {1,2,3,4} {
%                 \foreach \y in {1,2,3,4} {
%                     \coordinate (pos\x\y) at ($ (\x*1.2em, -\y*1.2em) $);
%                     \node[node_style] (N-\x-\y) at (pos\x\y) {};
%                 }
%             }

%             % Highlight special nodes
%             \node[special_style] at (pos11) {};
%             \node[special_style] at (pos21) {};
%             \node[special_style] at (pos22) {};

%             % Draw horizontal connections
%             \foreach \x in {1,2,3} {
%                 \foreach \y in {1,2,3,4} {
%                     \draw (N-\x-\y) -- (N-\the\numexpr\x+1\relax-\y);
%                 }
%             }

%             % Draw vertical connections
%             \foreach \x in {1,2,3,4} {
%                 \foreach \y in {1,2,3} {
%                     \draw (N-\x-\y) -- (N-\x-\the\numexpr\y+1\relax);
%                 }
%             }

%             % Highlighted node and vector
%             % In your preamble, add:    
            

%             % ... (rest of your LaTeX preamble)
            
%             % Define a scaling factor for the vector table
%             \def\vectorScale{1} % Adjust this value to scale the vector table as needed
            
%             % Highlighted node and vector
%             \node[highlight_style] (highlighted) at (pos33) {};
            
%             % Define the position where the vector table will be placed
%             \coordinate (vectorPosition) at ($(highlighted.east) + (2.5em, 1em)$);
            
%             % Draw the dotted line from the highlighted node to the middle left of the vector table
%             \draw[black, dotted, thick] (highlighted.east) to[out=0, in=180] (vectorPosition);
            
%             % Vector of numbers using a TikZ matrix
%             \node[anchor=west] (vectorNode) at (vectorPosition) {
%                 \begin{tikzpicture}[baseline, scale=\vectorScale]
%                     % Define the matrix (vector table)
%                     \matrix (vector) [matrix of nodes,
%                                       nodes={draw, minimum height=1em, anchor=west, minimum width=1.5em, font=\scriptsize},
%                                       nodes in empty cells,
%                                       draw,
%                                       inner sep=1pt,
%                                       column sep=0pt,
%                                       row sep=0pt,
%                                       ] {
%                         1 \\ % Row 1
%                         1 \\ % Row 2
%                         0 \\ % Row 3
%                         1 \\ % Row 4
%                         1 \\ % Row 5
%                         1.1 \\ % Row 6
%                     };
                    
%                     % Add labels to the right of each component
%                     \def\labelScale{0.8}
%                     \foreach \i/\text in {
%                         1/Line above, 
%                         2/Line left, 
%                         3/Line below, 
%                         4/Line right, 
%                         5/Is source, 
%                         6/Connected to source} {
%                         % Draw an arrow from each row to its label
%                         \draw[->, thin] (vector-\i-1.east) -- ++(1em, 0) 
%                             node[right, font=\small, anchor=west, scale=\labelScale] {\text};
%                     }
%                 \end{tikzpicture}
%             };
            
            
%         \end{scope}
%     % \end{scope}
% \end{tikzpicture}
% \end{adjustbox}
