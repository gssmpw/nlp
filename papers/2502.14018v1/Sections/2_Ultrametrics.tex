\section{Ultrametrics and Tree Representations}
\label{sec:ultrametrics}

We begin by formally introducing the data structure representing ultrametric spaces as it facilitates solving center-based clustering tasks quickly. The upcoming results operate over a generalization of the standard ultrametric:
\begin{restatable}{definition}{RelaxedUltrametric}
    \label{def:relaxed_ultrametric}
    Let $L$ be a set. Then $d: L \times L \rightarrow \mathbb{R}_{\geq 0}$ is a \emph{relaxed ultrametric} over $L$ if, for all $\ell_i, \ell_j, \ell_k \in L$, the
    following conditions are satisfied:
    \begin{align*}
        d(\ell_i, \ell_j) &= d(\ell_j, \ell_i) \geq 0 \\
        d(\ell_i, \ell_k) &\leq \max( d(\ell_i, \ell_j), d(\ell_j, \ell_k)).
    \end{align*}
\end{restatable}
Note that the standard ultrametric is a restriction that additionally requires $\dt(\ell_i, \ell_i) = 0$. Thus, not all relaxed ultrametrics are distances as $\dt(\ell_i, \ell_i) \gneq 0$ is allowed. Still, we use the word ``distance'' for readability.
%
We represent relaxed ultrametric relationships via the following data structure:

\begin{definition}
    \label{def:lca_tree}
    A \emph{lowest-common-ancestor tree} (\emph{LCA-tree}) is a rooted tree $T$ such that every node $\eta \in T$ has value $d(\eta) \geq 0$ associated with it. We write $\eta_i \preceq \eta_j$ to indicate that $\eta_j$ lies on the path from $\eta_i$ to the root and $\eta_i \lor \eta_j$ to refer to the LCA of $\eta_i$ and $\eta_j$. We say that the \emph{LCA-distance} between two leaves $\ell_i, \ell_j \in T$ is given by $d(\ell_i \lor \ell_j)$.
\end{definition}

An LCA-tree is not necessarily binary: if three or more subtrees are all equidistant, they can all be children of the same node. While similar data structures already exist for standard ultrametrics \citep{ultrametric_lca_def, memory_efficient_minimax}, the following theorem (proof in \ref{app:ultrametric_proofs}) states that it can also encode all relaxed ultrametrics:\footnote{
Note that \textit{relaxed} ultrametrics cannot be represented via shortest-path distances through a tree as is commonly done for ultrametrics \citep{ultrametrics_root_equidist} as $d(\ell_i, \ell_i) > 0$ is possible.}


\begin{restatable}{theorem}{UltrametricEquivalency}
    \label{thm:ultrametric_equivalency}
    Let $(L, d')$ be a finite relaxed ultrametric space. Then there exists LCA-tree $T$ with LCA-distance $d$ and a bijection $f:L \leftrightarrow \text{leaves}(T)$ such that, for all $\ell_i, \ell_j \in L$, $d'(\ell_i, \ell_j) = d \left( f(\ell_i) \lor f(\ell_j) \right)$.
\end{restatable}

This is visualized by the first box of Figure \ref{fig:overview}. It shows a minimum spanning tree (MST) over some data on the left. In this MST, the minimax ultrametric is given by the weight of the largest edge in the path between two nodes \cite{minimax_distance}. For example, the minimax distance between nodes $\circled{1}$ and $\circled{4}$ is $\putinsquare{5}$. The right-hand side of the first box of Figure 1 then stores these minimax distances in an LCA-tree where the LCA of nodes $\circled{1}$ and $\circled{4}$ has value $5$.

We show in Appendix \ref{app:ultrametric_proofs} that all LCA-trees are relaxed ultrametrics as long as they satisfy the following conditions:

\begin{restatable}{corollary}{LCAcorollary}
    \label{cor:ultrametric_lca}
    Let $T$ be an LCA-tree. For any leaf $\ell \in T$, let $p(\ell) = [\ell, \eta_a, \ldots, \eta_b, r(T)]$ be the path from $\ell$ to the root of the tree $r(T)$.
    Then the LCA-distances on $T$ form a relaxed ultrametric if and only if, for all $\ell \in \text{leaves}(T)$ and $\eta_i, \eta_j \in p(\ell)$, the following conditions are satisfied:
    \[\text{\emph{(1)}} \quad d(\ell) \geq 0 \quad \text{and} \quad \text{\emph{(2)}} \quad \eta_i \preceq \eta_j \implies d(\eta_i) \leq d(\eta_j).\]
\end{restatable}
%
Corollary \ref{cor:ultrametric_lca} describes a key property: if a tree's node values grow along paths from the leaves to the root, then the corresponding LCA-distances are a relaxed ultrametric. This is the natural representation of a hierarchy: since the subtrees grow in size as we go towards the root, the values corresponding to those subtrees also grow. Thus, relaxed ultrametrics and hierarchies are equivalent notions. Going forward, we assume that a relaxed ultrametric is given in its LCA-tree form, satisfying the properties in Corollary \ref{cor:ultrametric_lca}.