\section{Choosing a Partition}\label{sec:best_clustering}

Theorem \ref{thm:optimal_clusters} gives us the optimal center-based clustering solutions for all values of $k$ at once. If we simply require the solution for a specific $k$ then this is sufficient. However, what if we are instead interested in the ``best'' clustering from this hierarchy? To this end, this section discusses how known techniques for obtaining a partition can be applied to Section \ref{sec:clustering_theory}'s hierarchies in $O(n)$ time.

\subsection{Cluster Selection Criteria}

\subsubsection{Feasibility of the Elbow Method}\label{sec:elbow}

One of the most common methods for choosing a ``best'' clustering is the elbow method \citep{elbow_original}. Here, one is given a set of values of $k$, $[k_1, k_2, \ldots, k_f]$, and a set of corresponding partitions $\mathcal{P} = [\mathcal{P}_{1}, \mathcal{P}_{2}, \ldots, \mathcal{P}_{f}]$. Each partition incurs a cost $\mathcal{L} = [\mathcal{L}_{1}, \mathcal{L}_{2}, \ldots, \mathcal{L}_{f}]$ with respect to the clustering objective, where $\mathcal{L}_i = \sum_{C \in \mathcal{P}_i} \mathcal{L}(C)$. This gives us a plot of costs over the different values of $k$. Informally, the \emph{elbow method} chooses the partition $\mathcal{P}_i$ whose cost $\mathcal{L}_i$ looks to be at a sharp point in this curve.

Due to the NP-hardness of $k$-means clustering \cite{kmeans_hardness_1}, standard elbow plots can only compute approximate solutions, traditionally done sequentially for each $k$ \cite{elbow_issues}. However, the elbow method is surprisingly viable in the ultrametric setting: Theorem \ref{thm:optimal_clusters} yields optimal clusterings for all $k$ values at once, eliminating both computational overhead and approximation errors. Moreover, we show that the relaxed ultrametric's elbow plot for $(k, z)$-clustering is guaranteed to be convex:

\begin{restatable}{corollary}{ElbowPlotCor}
    \label{cor:elbow_plot}
    Let $\mathcal{P}$ and $\mathcal{L}$ correspond to the $n$ partitions and losses obtained in accordance with Theorem \ref{thm:optimal_clusters} for the $(k, z$)-clustering objective. Let $\Delta_i = \mathcal{L}_{i+1} - \mathcal{L}_i$. Then either $\Delta_{i} < \Delta_{i+1} \leq 0$ or $\Delta_{i} = \Delta_{i+1} = 0$ for all $i \in [n-1]$.
\end{restatable}

The idea here is that $\Delta_i$ represents the elbow plot's first derivative at $k=i$. Thus, Corollary \ref{cor:elbow_plot} states that the elbow plot's slope is steepest at $k=1$ and monotonically levels out to 0 as $k \rightarrow n$. We prove this in Appendix \ref{app:ultrametric_elbow}, where we also specify how we determine the index of the elbow. We depict relaxed ultrametric elbow plots in the $(k, z)$-clustering setting for various values of $z$ in Figure \ref{fig:elbow_plot}, where we see that the plots are indeed convex.

\begin{figure}[t]
    \centering
    \vspace{0.2em}
    \resizebox{0.98\linewidth}{!}{%
    \begin{tikzpicture}
        \node[inner sep=0pt] (zoomed_out) at (0, 0) {\includegraphics[width=\linewidth, trim={0.75cm, 0.50cm, 0.5cm, 0.3cm}, clip]{figures/elbow_plot_large.pdf}};

        % Draw Zoomed in Figure
        \node[inner sep=0pt] (zoomed_in) at (0.75, 0.75){\includegraphics[width=0.72\linewidth, trim={1.1cm, 0.7cm, 0.5cm, 0.3cm}, clip]{figures/elbow_plot_zoomed_in.pdf}};
        \draw[red, thick, opacity=0.55] ($(zoomed_in.north west)$) rectangle ($(zoomed_in.south east)$);

        % Draw Zoom Box
        \draw[red, very thick, dotted, opacity=0.8] (-3.72,-2.9) rectangle (-1,-2.2);
    
        % Draw connecting lines from rectangle corners to zoomed image
        \draw[red, very thick, dashed, opacity=0.6] (-3.72,-2.2) -- (-2.25,2.99);
        \draw[red, very thick, dashed, opacity=0.6] (-1,-2.9) -- (3.72,-1.485);

        \node[inner sep=0pt] (a) at (-1.52, -1.67) {\tiny \textcolor{gray}{6}};
        \node[inner sep=0pt] (a) at (-1.31, -1.67) {\tiny \textcolor{gray}{7}};
        \node[inner sep=0pt] (a) at (-0.05, -1.67) {\tiny \textcolor{gray}{14}};

        \fill[white] (2.43,-1.84) rectangle (2.68,-1.6);
        \node[inner sep=0pt] (a) at (2.55, -1.67) {\tiny \textcolor{gray}{28}};

        \fill[white] (2.98,-1.82) rectangle (3.22,-1.6);
        \node[inner sep=0pt] (a) at (3.1, -1.67) {\tiny \textcolor{gray}{31}};

        \node[inner sep=0pt] (a) at (-3.76, -3.27) {\small \textcolor{darkgray}{0}};
        \node[inner sep=0pt] (a) at (-0.65, -3.27) {\textcolor{darkgray}{40}};
        \node[inner sep=0pt] (a) at (2.475, -3.27) {\textcolor{darkgray}{80}};
        \node[inner sep=0pt] (a) at (4.05, -3.27) {\textcolor{black}{\dots}};
        \node[inner sep=0pt] (a) at (0, -3.7) {\textcolor{darkgray}{Number of clusters ($k$)}};

        \node[inner sep=0pt] (a) at (-4.3, -2.75) {\small \textcolor{darkgray}{0}};
        \node[inner sep=0pt] (a) at (-4.3, 2.95) {\textcolor{darkgray}{1}};
        \node[inner sep=0pt] (a) at (-4.5, 0) {\textcolor{darkgray}{\rotatebox{90}{Normalized cost}}};        
    \end{tikzpicture}%
    }
    \vspace*{-0.4em}
    \caption{Example elbow plots for the $z=1, 2, 3, 4\text{, and }5$ settings under the dc-dist ultrametric on the D31 synthetic dataset. Elbow locations (circles) are determined using all $k$ from $1$ to $n$.}
    \label{fig:elbow_plot}
% \end{wrapfigure}
\end{figure}


\subsubsection{Thresholding the Tree}

While Corollary \ref{cor:elbow_plot} shows that the elbow method is reasonable in the ultrametric setting, it still has two main drawbacks. \emph{First}, the choice of the elbow is arbitrary, and it is unclear whether the best technique exists \citep{elbow_issues}. \emph{Second}, the elbow method is restricted to a single partition for each value of $k$. Although the cluster hierarchy can be cut in many ways to obtain $k$ clusters, the elbow method can only produce one of these partitions for any value of $k$.

One alternative is to threshold the cluster hierarchy, as done in single-linkage clustering or DBSCAN (see \citet{hdbscan}). Specifically, let $\varepsilon$ be any user-defined threshold parameter. We now (1) label the nodes of our cluster hierarchy by their costs as discussed at the end of Section \ref{sec:clustering_theory} and (2) report all clusters whose cost is less than $\varepsilon$, both in $O(n)$ time.
We further discuss this in \Cref{app:thresholding}.


\subsubsection{Cluster Value Functions} \label{sec:clustervaluefunction}

Rather than picking clusters based on their costs, one can also assign a \emph{new} value function to clusters and choose the partition that maximizes the sum of these values. Importantly, such a partition may not be attainable by choosing a value of $k$ or thresholding the clusters' costs in the hierarchy. For example, HDBSCAN uses the \emph{stability} objective to choose the clusters from a hierarchy that persist for the largest range of threshold values. Given any cluster hierarchy obtained via Theorem \ref{thm:optimal_clusters} and any reasonable value function, one can find the partition that maximizes this value function in $O(n)$ time. We detail this in Appendix \ref{app:cluster_merging_rules}. 


\subsection{Generalizing to Noisy Settings}

For a given LCA-tree $T$ and any node $\eta \in T$, one can remove the subtree rooted at $\eta$ from $T$ without affecting Corollary \ref{cor:ultrametric_lca}.
This allows handling additional noise points consistently, e.g., \citet{k-center-q-coverage, hdbscan}.

% To be consistent with the literature treating additional noise points in the data \citep{k-center-q-coverage, hdbscan}, we point out that one can prune nodes away from an LCA-tree without compromising our theoretical results. 
% Namely, given an LCA-tree $T$ and any node $\eta \in T$, one can remove the subtree rooted at $\eta$ from $T$ without affecting Corollary \ref{cor:ultrametric_lca}. Thus, pruning away subtrees does not affect any of the tree's ultrametric properties.

For example, consider a minimum-cluster-size parameter $\mu \in \mathbb{Z}_{+}$ and a cluster hierarchy $\mathcal{H}$ obtained via Theorem \ref{thm:optimal_clusters}. Then, for some cluster $C_i$ with $|C_i| < \mu$, we can prune the hierarchy by removing the cluster trees rooted at $C_i$ in $O(n)$ time \citep{hdbscan}. We can, therefore, obtain any desired clustering over this pruned hierarchy, guaranteeing that every cluster in the returned partition will have a size greater than or equal to the minimum cluster size. 


\subsection{Integrating Multiple Partition Methods} \label{sec:integrating_multiple_partition_methods}

Lastly, we note that we can combine results of several partitions without notable impact on runtime. As Section \ref{sec:experiments} confirms, fitting an ultrametric essentially always requires longer than $O(n \log n)$ time. Given such an ultrametric, we can find hierarchies of optimal center-based clustering solutions in $\texttt{Sort}(n) \leq O(n \log n)$ time (Section \ref{sec:clustering_theory}). Furthermore, we presented several $O(n) < \texttt{Sort}(n)$-time methods for choosing a partition. As Figure \ref{fig:runtime_barplot} highlights, the time it takes to fit the density-connectivity ultrametric dwarfs the time for finding a hierarchy or partition.
%
\begin{figure}
    \centering
    \vspace*{0.2em}
    \resizebox{\linewidth}{!}{
\begin{tikzpicture}[inner sep=0]

% Cover tree      00:00.322  
% DC tree         00:09.076  
% k-means         00:00.022  
% Stability       00:00.002  
% MoE             00:00.001  
% Elbow           00:00.001  
% Eucl. k-means   00:03.386  
% Eucl. k-means’  00:03.521  
% Ward            00:10.309  
% OPTICS          00:33.109  
% DPC             11:42.910   702.910
% SCAR            00:02.133  

\tikzmath{
	\ymax = 4.75;
	\ydc = 2.75;
	\dcTime = 9.076; \hierTime = 0.022; \partTime = 0.001;
	\dcTotalTime = {floor((\dcTime + \hierTime + \partTime) * 1000) / 1000};
    \xscale = 0.7;
}

% grid
%\draw  (0,0) rectangle (11,5);
%\draw[help lines, color=gray!30, dashed] (0.1,0.1) grid (10.9,4.9);
\foreach \x in {0,...,4}{
    \draw [help lines, color=gray!70, dashed, opacity=0.4] ($(\xscale*\x,0)$) -- ($( \xscale*\x,\ymax)$);
}
\foreach \x in {5,...,11}{
    \draw [help lines, color=gray!70, dashed, opacity=0.4] ($(\xscale*\x,0)$) -- ($( \xscale*\x,3.25)$);
}

% x-axis
\node at ($(\xscale*5.5, -0.8)$) {Time in seconds};
\draw[->,thick] (-0.1,0)--($(\xscale*11.5,0)$);  
% x-ticks
\foreach \x in {0,1,2,...,11} {        
    \coordinate (X\x) at ($(\xscale*\x*1cm,0)$) {};
    \draw ($(X\x)+(0,3pt)$) -- ($(X\x)-(0,3pt)$);
    \node at ($(X\x)-(0,2.5ex)$) {\x};
}
\coordinate (dots) at ($(\xscale*11*1cm,0)$) {};
\node at ($(dots)+(0.43,-3ex)$) {...};



% y-axis
\draw[-,thick] (0,0)--(0,\ymax);

% y-ticks and bars for time < 11s
\foreach \y/\alg/\time in {
		5/SCAR/2.133,
		4/Eucl. $k$-means/3.386,
		3/{DC/hier./part.}/\dcTime,
		2/Ward/10.309} {
	\coordinate (Y\y) at ($(0,0.5cm+\y*0.75cm)$) {};
	\draw ($(Y\y)+(3pt,0)$) -- ($(Y\y)-(3pt,0)$);
	\node [anchor=east, align=right] at ($(Y\y)-(1ex,0)$) {\alg};

    % Bars
    \draw  [fill=blue, fill opacity=0.3, blue] ($(Y\y)-(0,0.25)$) rectangle ($(Y\y)+(\xscale*\time,0.25)$);
	\node at ($(Y\y)+(\xscale*\time/2,0)$) {$\time$\,s};
}

% y-ticks and bars for time > 11s
\foreach \y/\alg/\xend\time in {
		1/AMD-DBSCAN/10.9/37.055,
		0/DPC/11.3/702.910} {
	\coordinate (Y\y) at ($(0,0.5cm+\y*0.75cm)$) {};
	\draw ($(Y\y)+(3pt,0)$) -- ($(Y\y)-(3pt,0)$);
	\node [anchor=east, align=right] at ($(Y\y)-(1ex,0)$) {\alg};
	
	\tikzset{decoration={snake,amplitude=.6mm,segment length=0.44cm, post length=0mm,pre length=0mm}}

    % Bars
	\fill[blue, opacity=0.3] ($(Y\y)-(0,0.25)$) -- ($(Y\y)+(0,0.25)$) -- ($(Y\y)+(\xscale*\xend,0.25)$) decorate { -- ($(Y\y)+(\xscale*\xend,-0.25)$) } -- cycle;
	\draw[blue] ($(Y\y)-(0,0.25)$) -- ($(Y\y)+(0,0.25)$);
	\draw[blue] ($(Y\y)+(0,0.25)$) -- ($(Y\y)+(\xscale*\xend,0.25)$);
	\draw[blue] ($(Y\y)-(0,0.25)$) -- ($(Y\y)+(\xscale*\xend,-0.25)$);
	\node at ($(Y\y)+(\xscale*\xend/2,0)$) {$\time$\,s};
    \draw[->] ($(Y\y)+(\xscale*\xend-0.05,0)$) -- ($(Y\y)+(\xscale*\xend+0.2,0)$);
}



% DC tree/hier./part.
%\draw  [fill=blue, fill opacity=0.3, blue] (0,2.5) rectangle (9.076,3);  % DC tree
\draw  [fill=blue, fill opacity=0.3, blue] ($(\xscale*\dcTime,\ydc)-(0,0.25)$) rectangle ($(\xscale*\dcTime,\ydc)+(\xscale*\hierTime,0.25)$);  % hierarchy
\draw  [fill=blue, fill opacity=0.3, blue]  ($(\xscale*\dcTime+\hierTime,\ydc)-(0,0.25)$) rectangle  ($(\xscale*\dcTime+\xscale*\hierTime,\ydc)+(\xscale*\partTime,0.25)$);  % partition



% Algorithms
%\draw  [fill=blue, fill opacity=0.3, blue] (0,0.25) rectangle (2.133,0.75);  % SCAR
%\draw  [fill=blue, fill opacity=0.3, blue] (0,1) rectangle (3.521,1.5);  % $k$-Means
%\draw  [fill=blue, fill opacity=0.3, blue] (0,0.25) rectangle (10.309,0.75);  % WARD

% OPTICS
%\fill[blue, opacity=0.3] (0,1) -- (0,1.5) -- (10.4,1.5) decorate { -- (10.4,1) } -- cycle;
%\draw[blue] (0,1) -- (0,1.5);
%\draw[blue] (0,1.5) -- (10.4,1.5);
%\draw[blue] (0,1) -- (10.4,1);
%\node at (5.47, 1.25) {$33.109$\,s};





%%% Zoomed-in version
\begin{scope}[shift={(\xscale*5,4.6)},scale=0.95, every node/.append style={transform shape}, inner sep=6, local bounding box=zoomed_in]
\tikzmath{
	\ymax = 1.;
	\ydc = 0.5;
	\dcTime = 0; \hierTime = 5.5; \partTime = 0.25;
	%\dcTotalTime = {floor((\dcTime + \hierTime + \partTime) * 1000) / 1000};
}

% grid
%\draw  (0,0) rectangle (11,5);
%\draw[help lines, color=gray!30, dashed] (0.1,0.1) grid (10.9,4.9);
\foreach \x in {0,...,5}{
  \draw [help lines, color=gray!80, dashed, opacity=0.4] (\x,0) -- (\x,\ymax);
}

% x-axis
\node at (3, -0.8) {Time in \textcolor{black}{milli}seconds};
\draw[->,thick] (-0.1,0)--(6,0);
% x-ticks
\foreach \x in {4,8,...,22} {
  \coordinate (X\x) at ($(\x*0.25cm,0)$) {};
  \draw ($(X\x)+(0,3pt)$) -- ($(X\x)-(0,3pt)$);
  \node at ($(X\x)-(0,2.5ex)$) {+\x};
}
\coordinate (X0) at ($(0*0.25cm,0)$) {};
\draw ($(X0)+(0,3pt)$) -- ($(X0)-(0,3pt)$);
\node at ($(X0)-(0,2.5ex)$) {9076\,ms};


% y-axis
\draw[-,thick] (0,0)--(0,\ymax);

\tikzset{decoration={snake,amplitude=.4mm,segment length=0.35cm, post length=0mm,pre length=0mm}}

%\draw  [fill=blue, fill opacity=0.3, blue] (0,2.5) rectangle (9.076,3);  % DC tree
%\draw  [fill=blue, fill opacity=0.3, blue] ($(-0.3,\ydc)-(0,0.25)$) rectangle ($(0,\ydc)+(0,0.25)$);
\fill[blue, opacity=0.3] ($(0,\ydc-0.25)$) -- ($(0,\ydc+0.25)$) -- ($(-0.3,\ydc+0.25)$) decorate { -- ($(-0.3,\ydc-0.25)$) } -- cycle;
\draw[blue] ($(0,\ydc-0.25)$) -- ($(0,\ydc+0.25)$);
\draw[blue] ($(0,\ydc-0.25)$) -- ($(-0.3,\ydc-0.25)$);
\draw[blue] ($(0,\ydc+0.25)$) -- ($(-0.3,\ydc+0.25)$);

\draw  [fill=blue, fill opacity=0.3, blue] ($(\dcTime,\ydc)-(0,0.25)$) rectangle ($(\dcTime,\ydc)+(\hierTime,0.25)$);  % hierarchy
\node at ($(2.75,\ydc)$) {$22$\,ms};
\draw  [fill=blue, fill opacity=0.3, blue]  ($(\dcTime+\hierTime,\ydc)-(0,0.25)$) rectangle  ($(\dcTime+\hierTime,\ydc)+(\partTime,0.25)$);  % partition
\node at ($(5.63,\ydc)$) {$1$};

\draw[black,decorate,decoration={brace,amplitude=12pt}] (0.03,0.75) -- (5.497,0.75) node[midway, above,yshift=12pt,]{build hierarchy};
\draw[black,decorate,decoration={brace,amplitude=6pt}] (5.5,0.75) -- (5.75,0.75) node[midway, above,yshift=6pt,]{partition};

\end{scope}




% Draw Zoom Box
\draw[red, very thick, dotted, opacity=0.8] ($(\xscale*8.9,\ydc+0.36)$) rectangle ($(\xscale*9.25,\ydc-0.36)$);

% Draw connecting lines from rectangle corners to zoomed image
%\draw[red, thick, dashed, opacity=0.6] ($(8.8,\ydc+0.36)$) -- (4.5,3.48);
%\draw[red, thick, dashed, opacity=0.6] ($(9.3,\ydc+0.36)$) -- (11,3.48);
\draw[red, thick, dashed, opacity=0.6] ($(\xscale*8.8,\ydc+0.36)$) -- ($(zoomed_in.south west) - (0,0.02)$);
\draw[red, thick, dashed, opacity=0.6] ($(\xscale*9.3,\ydc+0.36)$) -- ($(zoomed_in.south east) - (0,0.03)$);

% Draw Zoomed in Box
%\draw[red, thick, opacity=0.55] (4.5,3.5) rectangle (11,5);
\draw[red, thick, opacity=0.55] ($(zoomed_in.south west)$) rectangle ($(zoomed_in.north east)$);




\end{tikzpicture}
}
    \caption{
        Runtimes of competitors and our framework's components on letterrec dataset. Computing clusterings on the dc-dist is faster than other density-based methods, but building the hierarchy and partitioning it, is still several orders of magnitude faster.
    }
    \label{fig:runtime_barplot}
\end{figure}%
%
Consequently, once we have fit an ultrametric, we can calculate multiple hierarchies and partitions and integrate their information in negligible time. To illustrate this, we introduce a method for choosing the number of clusters that we find to work well in practice.
Given an ultrametric, our \emph{Median-of-Elbows} (MoE) algorithm starts by fitting the $(k, z)$-cluster hierarchies for $z = 1, 2, 3, 4\text{, and }5$. Intuitively, each of these hierarchies has a different penalty for large distances. Applying the elbow method to each hierarchy gives us a set of five $k$ values, each representing its hierarchy's `best' number of clusters. 
The MoE method picks the median $k$ from this set, which is essentially the $k$ that gives stable clusterings across values of $z$. Figure \ref{fig:elbow_plot} visualizes this process on the D31 dataset. Here, MoE selects $k=14$ (see also the D31 $k$-median/MoE cell in Figure \ref{fig:exp_ablation_partitioning}).


\section{Visualizations}

\textbf{Visualizations for LMC-Synth.} We provide visualizations for the multivariate synthetic data generated by LMC-Synth. For clarity, we have limited the number of channels to 5 and the sequence length to 96. Figures 3-6 showcase the MTS data generated using LMC-Synth.

\textbf{The Forecasts of TimePFN.} We provide forecasts from \name under various data budgets and datasets, including zero-shot and full-data scenarios. Figures 7-18 display the forecasts of \name in these settings. As shown, with an increasing data budget, \name's forecasts align more closely with the ground truth.



\begin{figure*}
    \centering
    \includegraphics[width=1\linewidth]{figures/synthetic_data1.pdf}
    \caption{Examples of synthetic multivariate time time-series data generated by LMC-Synth. For the ease of understanding, we took C=5 and sequence lenght = 96. Dirichlet concentration parameter controls the diversity of variates from one another. }
    \label{fig:LMC-synth1}
\end{figure*}

\begin{figure*}
    \centering
    \includegraphics[width=1\linewidth]{figures/synthetic_data2.pdf}
    \caption{Examples of synthetic multivariate time time-series data generated by LMC-Synth. For the ease of understanding, we took C=5 and sequence lenght = 96. Dirichlet concentration parameter controls the diversity of variates from one another. }    \label{fig:LMC-synth2}
\end{figure*}

\begin{figure*}
    \centering
    \includegraphics[width=1\linewidth]{figures/synthetic_data3.pdf}
    \caption{Examples of synthetic multivariate time time-series data generated by LMC-Synth. For the ease of understanding, we took C=5 and sequence lenght = 96. Dirichlet concentration parameter controls the diversity of variates from one another. }    \label{fig:LMC-synth3}
\end{figure*}

\begin{figure*}
    \centering
    \includegraphics[width=1\linewidth]{figures/synthetic_data4.pdf}
    \caption{Examples of synthetic multivariate time time-series data generated by LMC-Synth. For the ease of understanding, we took C=5 and sequence lenght = 96. Dirichlet concentration parameter controls the diversity of variates from one another. }    \label{fig:LMC-synth4}
\end{figure*}


\begin{figure*}[htp]
\centering

% Row 1
\begin{subfigure}{0.32\textwidth}
\includegraphics[width=\linewidth]{figures/ECL/zero_shot/20.pdf}
\caption{Zero Shot}
\end{subfigure}\hfill
\begin{subfigure}{0.32\textwidth}
\includegraphics[width=\linewidth]{figures/ECL/50/20.pdf}
\caption{Budget = 50}
\end{subfigure}\hfill
\begin{subfigure}{0.32\textwidth}
\includegraphics[width=\linewidth]{figures/ECL/100/20.pdf}
\caption{Budget = 100}
\end{subfigure}

% Row 2
\begin{subfigure}{0.32\textwidth}
\includegraphics[width=\linewidth]{figures/ECL/500/20.pdf}
\caption{Few-shot with budget 500}
\end{subfigure}\hfill
\begin{subfigure}{0.32\textwidth}
\includegraphics[width=\linewidth]{figures/ECL/1000/20.pdf}
\caption{Budget = 1000}
\end{subfigure}\hfill
\begin{subfigure}{0.32\textwidth}
\includegraphics[width=\linewidth]{figures/ECL/all/20.pdf}
\caption{Budget = All}
\end{subfigure}

\caption{The forecasts of TimePFN with various data budgets on ECL dataset. }
\label{fig:ECL20}
\end{figure*}
\begin{figure*}[htp]
\centering

% Row 1
\begin{subfigure}{0.32\textwidth}
\includegraphics[width=\linewidth]{figures/ECL/zero_shot/0.pdf}
\caption{Zero Shot}
\end{subfigure}\hfill
\begin{subfigure}{0.32\textwidth}
\includegraphics[width=\linewidth]{figures/ECL/50/0.pdf}
\caption{Budget = 50}
\end{subfigure}\hfill
\begin{subfigure}{0.32\textwidth}
\includegraphics[width=\linewidth]{figures/ECL/100/0.pdf}
\caption{Budget = 100}
\end{subfigure}

% Row 2
\begin{subfigure}{0.32\textwidth}
\includegraphics[width=\linewidth]{figures/ECL/500/0.pdf}
\caption{Budget = 500}
\end{subfigure}\hfill
\begin{subfigure}{0.32\textwidth}
\includegraphics[width=\linewidth]{figures/ECL/1000/0.pdf}
\caption{Budget = 1000}
\end{subfigure}\hfill
\begin{subfigure}{0.32\textwidth}
\includegraphics[width=\linewidth]{figures/ECL/all/0.pdf}
\caption{Budget = All}
\end{subfigure}

\caption{The forecasts of TimePFN with various data budgets on ECL dataset. }

\label{fig:ECL0}
\end{figure*}

\begin{figure*}[htp]
\centering

% Row 1
\begin{subfigure}{0.32\textwidth}
\includegraphics[width=\linewidth]{figures/weather/zero_shot/280.pdf}
\caption{Zero Shot}
\end{subfigure}\hfill
\begin{subfigure}{0.32\textwidth}
\includegraphics[width=\linewidth]{figures/weather/50/280.pdf}
\caption{Budget = 50}
\end{subfigure}\hfill
\begin{subfigure}{0.32\textwidth}
\includegraphics[width=\linewidth]{figures/weather/100/280.pdf}
\caption{Budget = 100}
\end{subfigure}

% Row 2
\begin{subfigure}{0.32\textwidth}
\includegraphics[width=\linewidth]{figures/weather/500/280.pdf}
\caption{Few-shot with budget 500}
\end{subfigure}\hfill
\begin{subfigure}{0.32\textwidth}
\includegraphics[width=\linewidth]{figures/weather/1000/280.pdf}
\caption{Budget = 1000}
\end{subfigure}\hfill
\begin{subfigure}{0.32\textwidth}
\includegraphics[width=\linewidth]{figures/weather/all/280.pdf}
\caption{Budget = All}
\end{subfigure}

\caption{The forecasts of TimePFN with various data budgets on weather dataset. }
\label{fig:weather280}
\end{figure*}
\begin{figure*}[htp]
\centering

% Row 1
\begin{subfigure}{0.32\textwidth}
\includegraphics[width=\linewidth]{figures/weather/zero_shot/40.pdf}
\caption{Zero Shot}
\end{subfigure}\hfill
\begin{subfigure}{0.32\textwidth}
\includegraphics[width=\linewidth]{figures/weather/50/40.pdf}
\caption{Budget = 50}
\end{subfigure}\hfill
\begin{subfigure}{0.32\textwidth}
\includegraphics[width=\linewidth]{figures/weather/100/40.pdf}
\caption{Budget = 100}
\end{subfigure}

% Row 2
\begin{subfigure}{0.32\textwidth}
\includegraphics[width=\linewidth]{figures/weather/500/40.pdf}
\caption{Few-shot with budget 500}
\end{subfigure}\hfill
\begin{subfigure}{0.32\textwidth}
\includegraphics[width=\linewidth]{figures/weather/1000/40.pdf}
\caption{Budget = 1000}
\end{subfigure}\hfill
\begin{subfigure}{0.32\textwidth}
\includegraphics[width=\linewidth]{figures/weather/all/40.pdf}
\caption{Budget = All}
\end{subfigure}

\caption{The forecasts of TimePFN with various data budgets on weather dataset. }
\label{fig:weather40}
\end{figure*}


\begin{figure*}[htp]
\centering

% Row 1
\begin{subfigure}{0.32\textwidth}
\includegraphics[width=\linewidth]{figures/ETTh1/zero_shot/280.pdf}
\caption{Zero Shot}
\end{subfigure}\hfill
\begin{subfigure}{0.32\textwidth}
\includegraphics[width=\linewidth]{figures/ETTh1/50/280.pdf}
\caption{Budget = 50}
\end{subfigure}\hfill
\begin{subfigure}{0.32\textwidth}
\includegraphics[width=\linewidth]{figures/ETTh1/100/280.pdf}
\caption{Budget = 100}
\end{subfigure}

% Row 2
\begin{subfigure}{0.32\textwidth}
\includegraphics[width=\linewidth]{figures/ETTh1/500/280.pdf}
\caption{Few-shot with budget 500}
\end{subfigure}\hfill
\begin{subfigure}{0.32\textwidth}
\includegraphics[width=\linewidth]{figures/ETTh1/1000/280.pdf}
\caption{Budget = 1000}
\end{subfigure}\hfill
\begin{subfigure}{0.32\textwidth}
\includegraphics[width=\linewidth]{figures/ETTh1/all/280.pdf}
\caption{Budget = All}
\end{subfigure}

\caption{The forecasts of TimePFN with various data budgets on ETTh1 dataset. }
\label{fig:etth1_280}
\end{figure*}
\begin{figure*}[htp]
\centering

% Row 1
\begin{subfigure}{0.32\textwidth}
\includegraphics[width=\linewidth]{figures/ETTh1/zero_shot/340.pdf}
\caption{Zero Shot}
\end{subfigure}\hfill
\begin{subfigure}{0.32\textwidth}
\includegraphics[width=\linewidth]{figures/ETTh1/50/340.pdf}
\caption{Budget = 50}
\end{subfigure}\hfill
\begin{subfigure}{0.32\textwidth}
\includegraphics[width=\linewidth]{figures/ETTh1/100/340.pdf}
\caption{Budget = 100}
\end{subfigure}

% Row 2
\begin{subfigure}{0.32\textwidth}
\includegraphics[width=\linewidth]{figures/ETTh1/500/340.pdf}
\caption{Few-shot with budget 500}
\end{subfigure}\hfill
\begin{subfigure}{0.32\textwidth}
\includegraphics[width=\linewidth]{figures/ETTh1/1000/340.pdf}
\caption{Budget = 1000}
\end{subfigure}\hfill
\begin{subfigure}{0.32\textwidth}
\includegraphics[width=\linewidth]{figures/ETTh1/all/340.pdf}
\caption{Budget = All}
\end{subfigure}

\caption{The forecasts of TimePFN with various data budgets on ETTh1 dataset. }
\label{fig:etth1_340}
\end{figure*}

\begin{figure*}[htp]
\centering

% Row 1
\begin{subfigure}{0.32\textwidth}
\includegraphics[width=\linewidth]{figures/ETTh2/zero_shot/0.pdf}
\caption{Zero Shot}
\end{subfigure}\hfill
\begin{subfigure}{0.32\textwidth}
\includegraphics[width=\linewidth]{figures/ETTh2/50/0.pdf}
\caption{Budget = 50}
\end{subfigure}\hfill
\begin{subfigure}{0.32\textwidth}
\includegraphics[width=\linewidth]{figures/ETTh2/100/0.pdf}
\caption{Budget = 100}
\end{subfigure}

% Row 2
\begin{subfigure}{0.32\textwidth}
\includegraphics[width=\linewidth]{figures/ETTh2/500/0.pdf}
\caption{Budget = 500}
\end{subfigure}\hfill
\begin{subfigure}{0.32\textwidth}
\includegraphics[width=\linewidth]{figures/ETTh2/1000/0.pdf}
\caption{Budget = 1000}
\end{subfigure}\hfill
\begin{subfigure}{0.32\textwidth}
\includegraphics[width=\linewidth]{figures/ETTh2/all/0.pdf}
\caption{Budget = All}
\end{subfigure}

\caption{The forecasts of TimePFN with various data budgets on ETTh2 dataset. }

\label{fig:etth2_0}
\end{figure*}
\begin{figure*}[htp]
\centering

% Row 1
\begin{subfigure}{0.32\textwidth}
\includegraphics[width=\linewidth]{figures/ETTh2/zero_shot/20.pdf}
\caption{Zero Shot}
\end{subfigure}\hfill
\begin{subfigure}{0.32\textwidth}
\includegraphics[width=\linewidth]{figures/ETTh2/50/20.pdf}
\caption{Budget = 50}
\end{subfigure}\hfill
\begin{subfigure}{0.32\textwidth}
\includegraphics[width=\linewidth]{figures/ETTh2/100/20.pdf}
\caption{Budget = 100}
\end{subfigure}

% Row 2
\begin{subfigure}{0.32\textwidth}
\includegraphics[width=\linewidth]{figures/ETTh2/500/20.pdf}
\caption{Budget = 500}
\end{subfigure}\hfill
\begin{subfigure}{0.32\textwidth}
\includegraphics[width=\linewidth]{figures/ETTh2/1000/20.pdf}
\caption{Budget = 1000}
\end{subfigure}\hfill
\begin{subfigure}{0.32\textwidth}
\includegraphics[width=\linewidth]{figures/ETTh2/all/20.pdf}
\caption{Budget = All}
\end{subfigure}

\caption{The forecasts of TimePFN with various data budgets on ETTh2 dataset. }

\label{fig:etth2_20}
\end{figure*}

\begin{figure*}[htp]
\centering

% Row 1
\begin{subfigure}{0.32\textwidth}
\includegraphics[width=\linewidth]{figures/ECL/zero_shot/0.pdf}
\caption{Zero Shot}
\end{subfigure}\hfill
\begin{subfigure}{0.32\textwidth}
\includegraphics[width=\linewidth]{figures/solar/50/0.pdf}
\caption{Budget = 50}
\end{subfigure}\hfill
\begin{subfigure}{0.32\textwidth}
\includegraphics[width=\linewidth]{figures/solar/100/0.pdf}
\caption{Budget = 100}
\end{subfigure}

% Row 2
\begin{subfigure}{0.32\textwidth}
\includegraphics[width=\linewidth]{figures/solar/500/0.pdf}
\caption{Budget = 500}
\end{subfigure}\hfill
\begin{subfigure}{0.32\textwidth}
\includegraphics[width=\linewidth]{figures/solar/1000/0.pdf}
\caption{Budget = 1000}
\end{subfigure}\hfill
\begin{subfigure}{0.32\textwidth}
\includegraphics[width=\linewidth]{figures/solar/all/0.pdf}
\caption{Budget = All}
\end{subfigure}

\caption{The forecasts of TimePFN with various data budgets on Solar dataset. }

\label{fig:solar0}
\end{figure*}
\begin{figure*}[htp]
\centering

% Row 1
\begin{subfigure}{0.32\textwidth}
\includegraphics[width=\linewidth]{figures/solar/zero_shot/20.pdf}
\caption{Zero Shot}
\end{subfigure}\hfill
\begin{subfigure}{0.32\textwidth}
\includegraphics[width=\linewidth]{figures/solar/50/20.pdf}
\caption{Budget = 50}
\end{subfigure}\hfill
\begin{subfigure}{0.32\textwidth}
\includegraphics[width=\linewidth]{figures/solar/100/20.pdf}
\caption{Budget = 100}
\end{subfigure}

% Row 2
\begin{subfigure}{0.32\textwidth}
\includegraphics[width=\linewidth]{figures/solar/500/20.pdf}
\caption{Few-shot with budget 500}
\end{subfigure}\hfill
\begin{subfigure}{0.32\textwidth}
\includegraphics[width=\linewidth]{figures/solar/1000/20.pdf}
\caption{Budget = 1000}
\end{subfigure}\hfill
\begin{subfigure}{0.32\textwidth}
\includegraphics[width=\linewidth]{figures/solar/all/20.pdf}
\caption{Budget = All}
\end{subfigure}

\caption{The forecasts of TimePFN with various data budgets on Solar dataset. }
\label{fig:solar20}
\end{figure*}


\begin{figure*}[htp]
\centering

% Row 1
\begin{subfigure}{0.32\textwidth}
\includegraphics[width=\linewidth]{figures/traffic/zero_shot/0.pdf}
\caption{Zero Shot}
\end{subfigure}\hfill
\begin{subfigure}{0.32\textwidth}
\includegraphics[width=\linewidth]{figures/traffic/50/0.pdf}
\caption{Budget = 50}
\end{subfigure}\hfill
\begin{subfigure}{0.32\textwidth}
\includegraphics[width=\linewidth]{figures/traffic/100/0.pdf}
\caption{Budget = 100}
\end{subfigure}

% Row 2
\begin{subfigure}{0.32\textwidth}
\includegraphics[width=\linewidth]{figures/traffic/500/0.pdf}
\caption{Budget = 500}
\end{subfigure}\hfill
\begin{subfigure}{0.32\textwidth}
\includegraphics[width=\linewidth]{figures/traffic/1000/0.pdf}
\caption{Budget = 1000}
\end{subfigure}\hfill
\begin{subfigure}{0.32\textwidth}
\includegraphics[width=\linewidth]{figures/traffic/all/0.pdf}
\caption{Budget = All}
\end{subfigure}

\caption{The forecasts of TimePFN with various data budgets on traffic dataset. }

\label{fig:traffic0}
\end{figure*}
\begin{figure*}[htp]
\centering

% Row 1
\begin{subfigure}{0.32\textwidth}
\includegraphics[width=\linewidth]{figures/traffic/zero_shot/80.pdf}
\caption{Zero Shot}
\end{subfigure}\hfill
\begin{subfigure}{0.32\textwidth}
\includegraphics[width=\linewidth]{figures/traffic/50/80.pdf}
\caption{Budget = 50}
\end{subfigure}\hfill
\begin{subfigure}{0.32\textwidth}
\includegraphics[width=\linewidth]{figures/traffic/100/80.pdf}
\caption{Budget = 100}
\end{subfigure}

% Row 2
\begin{subfigure}{0.32\textwidth}
\includegraphics[width=\linewidth]{figures/traffic/500/80.pdf}
\caption{Budget = 500}
\end{subfigure}\hfill
\begin{subfigure}{0.32\textwidth}
\includegraphics[width=\linewidth]{figures/traffic/1000/80.pdf}
\caption{Budget = 1000}
\end{subfigure}\hfill
\begin{subfigure}{0.32\textwidth}
\includegraphics[width=\linewidth]{figures/traffic/all/80.pdf}
\caption{Budget = All}
\end{subfigure}

\caption{The forecasts of TimePFN with various data budgets on traffic dataset. }

\label{fig:traffic80}
\end{figure*}


